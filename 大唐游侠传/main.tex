\documentclass[12pt,oneside]{book}

\usepackage{mybook}
\usepackage{mybookcover}


\title{大唐游侠传}
\author{梁羽生}

\begin{document}
\bookcover{book_cover.png}

\frontmatter


\addchtoc{目录}
\setcounter{tocdepth}{2}    
\tableofcontents

\mainmatter

\chapter{第 一 章 杯酒论交甘淡薄
玉钗为聘结良缘}\label{ux7b2c-ux4e00-ux7ae0-ux676fux9152ux8bbaux4ea4ux7518ux6de1ux8584-ux7389ux9497ux4e3aux8058ux7ed3ux826fux7f18}

``恭喜恭喜,新年大吉!''

这一天正是大唐天宝七年的新年初一。

离长安六十里外的一个山村,有一家人家,主人姓史,名逸如,曾在开元二十二年中过进士,却不愿在朝为官,未到中年,便回乡隐居,乡人敬他是个饱学君子,一早便来给他拜年。他循俗与乡人互相贺喜一番,送客之后,却摇了摇头喟然微叹:``如此世道,何喜之有?''

``呜哇,呜哇!''房内传出小儿的啼声,与辟辟啪啪的``爆竿''声闹成一片,(按:唐人风俗,元旦一真竹著火爆之,称为爆竿。与后来的``爆仗''不同。来-早春诗:``新历才将半纸开,小庭犹聚爆竿灰。''即咏此也。)史逸如脸上掠过一丝笑意,忖道:``要说有喜,那就是从今天起,多添了一个婴孩,家中可以热闹一些了。''他吩咐阶前烧爆竿的书僮:``你收了供品,给我拿四盒果品,到段大爷家去,并请他过来喝两杯。''

心中颇为有点疑惑:``每年元旦,最早来拜年的必定是他,今年却何以这样迟迟不来?''

书僮应了一声,却忽地笑道:``老爷,不必去请了,你瞧,那不是段大爷来了?''

只听得有人朗声吟道:``节物风光不相待,桑田碧海须门玉,昔时金阶白玉堂,即今惟见青松在。寂寂寥寥史子居,年年岁岁一床书。幸有故人长相聚,黄鸡白酒最相知。''

史逸如哈哈道:``卢照圭的诗给你一改,倒成了即景之作了,段兄,黄鸡白酒,早已备好,待兄一醉,何以如今始来?''

史逸如所招呼的``段兄``,名唤段-璋,是个四十多岁的中年人,相貌粗豪,是个武师打扮,史逸如则是个温文儒雅的书生,从外貌来看,两人似乎不应如此熟络,但事实上这两个人却是朝夕过往的朋友。原来这个段-
璋不但通晓武艺,诗文的造诣也很不错。他本来是个外乡人,搬到这儿还不到十年,史逸如也未深知他的来历,只是敬他胸襟磊落,文武全才,两人气味相投,遂成知己。段-璋听史逸如有埋怨他的意思,一笑说道:``史兄,小弟今日来迟,有个道理。''史逸如道:``却是为何?''段-璋眉开眼笑的说道:``内人昨晚添了一个娃娃。''史逸如大喜道:``哈!

哈!那真是无独有偶了。你的是男的还是女的?''段-璋道:``是个臭小子。咦,你这么问,感情嫂夫人也一分娩了?''史逸如道:``我却是添了个不中用的女娃子。''段-璋大笑道:``哈哈,是个姑娘,那我更要加倍向你贺喜了!''史逸如微微一惊,不解其意。段圭章笑道:``史兄可曾听的长安近事么?皇上夺了他的儿媳,寿王圭的妻子杨太真做贵妃,这是天宝四年之事。杨贵妃得宠非常,至今不过三年,她的三个姐姐都被封为夫人,上月从京中传来消息,连她的从兄杨国忠也拜相了,当真是一门显贵,无与伦比。因此都中风气大改,一听到有人生女,戚友便争来贺喜,人人都说如今的世道是:不重生男重生女了。吾兄添了一个千金,岂非要加倍贺喜!''

史逸如怫然不悦,说道:``我若想求功名富贵,这十年来也不会甘心隐居乡下了。我就是因为看不惯小人当道,奸邪满朝,这才掼了乌纱的。

难道我还会学杨国忠这类卑鄙小人的行径么?''

段-璋忙道:``你我相交十载,小弟岂尚有不知吾兄的为人之理?这话不过是说说笑笑罢了。''接着叹了一口气道:``我们把都中风气当成笑话来讲,其实却足以让有心人同声一哭啊!风气日坏,国事日非,将来真不知会闹成什么样子!''

史逸如也叹气道:``笑话,笑话,简直是越来越不成话!来,来,来,我们且乐得醉个糊涂,管他闹成什么样子!''

两人对饮了几杯,史逸如满腹牢骚,取了一柄如意击桌歌道:``岑夫子,丹丘生,将进酒,杯莫停,与君歌一曲,请君为我侧耳听:钟鼓馔玉不足贵,但愿长醉不复醒。哈哈,但愿长醉不用醒。李太白这首'将进酒'真是深得我心,当世的诗人,我只佩服他与老杜而已,听说他现在长安,可惜常被皇帝留在宫中,要不然真想到长安去见他一见。''

段-璋似有所触,忽又笑道:``史兄,我说你添了千金,值得加倍贺喜,却也不是笑话,你所佩服的老杜,不是写过一首《兵车行》吗?这首诗写成之后,洛阳纸贵,传诵一时,其中便有这样几句:``信知生男恶,反是生女好,生女犹得嫁比邻,生男埋没随百草!'如今国家连年用兵,而且大乱的迹象亦已显露,生一个具小子的确是不如生一个女娃儿呢!''

史逸如满满的喝了一杯,将酒杯重重一顿,说道:``儿女的事精,我们哪还管得这么多?倒是你刚才所念的老社那几句诗引起我一个念头。''

段-璋道:``怎么?''史逸如道:``生女犹得嫁比邻,我们虽非比邻,亦是同村,难得又这样巧,两个小娃娃都是在除夕这一天生的,咱们就此结为秦晋之好,作意如何?''

段-璋大笑道:``我一听说嫂夫人添了干企,早就有这个意思了,只是不敢开口。你我是肝胆相交,如今又做了亲家,真是最好不过。恰巧我身上带有一股玉钗,就拿来作订亲之礼吧。''史逸如一看那股玉钗,不觉一怔。

只见那股玉权,晶莹温润,竟是上好的和美玉,钗头嵌的一颗明珠,宝光夺目,看来亦是价值不菲。史迪加不禁心中想道:``他怎会有这等无价之宝?''要知道段圭湾自从迁到这个村子之后,就靠教一些乡下少年习武为业,家道甚是贫寒,每每碰到艰难时节,史逸如还不时周济他,如今见他拿出玉钗为聘,目是觉得奇怪。却也不会怀疑到玉钗来路不正。

段-璋似知其意,不待他问,便即说道:``先祖曾在贞观年间,随大将军李靖远征突厥,在和田得了一对玉钗,后来论功行赏,又得太宗皇帝赏赐一对南海明珠,先祖请巧手匠人,将明珠嵌于玉钗之上,永留作传家之宝。故此小弟不论家道如何艰困,都舍不得将这对玉钗卖掉。''

史逸如道:``原来段兄乃将门之后,怪不得十八般武艺,件件精通。''对这玉钗的来历再无怀疑,但心中却又起了另一个疑团:身为将门之后,乃是光荣之事,段-璋却何以从来不讲?段-璋饮了一杯,接着说道:``小弟家无长物,只有这对玉钗是个贵重的东西,所以从不离身。这对玉钗,一支雕有龙纹,一支雕有凤纹,名为龙凤宝钗,如今我就将这支凤钗,作为给令爱的聘礼。''。

史逸如道:``吾兄将传家之宝作为聘礼,如此郑重,小弟感激不尽。''本来不敢受的,但一想将来女儿嫁到了他的家,这玉钗总是他家之物,所以他就不再推辞了。

接过玉钗一看,只见五寸来长的玉钗上,果然雕有一只展翅高飞的彩凤,具体而微,神态生动,好象是藏在玉钗之中,呼之欲出的样子,不过因为玉钗只有五寸,彩凤刻在中间,要很好眼力才能看得清楚。

史逸如喷喷称赏,段-璋道:``这支龙钗,亦请吾兄赏鉴。''史逸如看那龙钗,形式和凤钗一模一样,钗头亦是嵌着一颗明珠,只是当中雕的,却是一条张牙舞爪的金龙,雕得更为精致。

段-璋道:``目下奸人当国,乱象方萌,将来的世道如何,谁也不敢逆料。小弟将龙凤宝钗拆散,把凤钗作为聘礼,其中还含有一层意思。''

说到此处,稍稍踌躇,似有什么避忌似的、史逸如道:``什么意思,倒要请教。你我既成亲家,还有什么话不可说的?''

段-璋道:``吾兄达人,元旦佳日,当不以小弟出言不吉为忌。我想,将来你我二家,若因世乱分离,他们这对未婚夫妇,也可以各执一钗作为凭信!''

史逸如哈哈笑道:``吾兄也顾虑得太长远了!''暗自想道:``你我二家同住一村,纵然逢到世乱年荒,也定然是患难与共,岂能分散。''但见段-
璋说得甚为郑重,心中不禁隐隐感到不祥之兆,故此欢颜强笑,冲淡这沉重的气氛。一面说,一面将那股龙钗交还给段圭璋,那股凤钗,则珍重的收藏好了。

段-璋道。'小儿尚未取名,吾兄才高学广,便请代为起个名字如何?''

史逸如笑道:``我的闺女也还未曾取名呢。''门外正明着鹅毛般的雪花,庭院里几株蜡梅,却正在雪中盛开,史逸如满满的喝了一杯,便即笑道:``我最喜梅花欺霜傲雪,我的闺女,便叫做若梅把。''顿了一顿,接续说道:``仅仅欺霜傲雪,尚还不够。当今之世,好邪满道,好男儿应能上马杀贼,下马革露布才是。好,我就以这个意思,斗胆代令郎起个名字,就叫做克邪如何?''

段-璋抚掌笑道:``好,好得很!段克邪,史若梅,这两个名字,你我的节操抱负都寄托在其中了。但愿他们将来长大成人,莫忘父母对他们的期望。''

就在他们二人抚掌大笑,莫逆于心的时候,忽听得呜呜的号角声,喧哗声,杂着孩童们的尖叫声,史逸如诧异道:``咦,外面出了什么事?新年新岁,难道就有官差来拉夫征粮不成?咱们出去看看!''

史家离路边不过几十步路,两人出了大门,抬头一看,只见尘头大起,一队官军从村头疾驰而来,甲胄鲜明,人强马壮,当前一骑,挥着一面大旗,金线绣着斗大的一个``安''字,迎风飞舞,紧接着两骑,也各扯着一面大旗,上面绣的是官衔,一面是``平卢节度使'',一面是``范阳节度使''。``节度使''乃是唐朝的方面重镇,在他所管辖的地方内,军事民政,都归他一人掌管,就等如一个小王国一般,威赫无比。一人而兼有两个节度使的官衔,乃是从所未见之事。史逸如怔了一怔,心想:``原来是安禄山!''安禄山之名。在当时无人不知,史逸如却还是第一次见到,只见他是象肥猪一般的大胖子,身穿锁子黄金甲,装模作样,威风凛凛的坐在高头大马上,在前呼后拥中扬鞭喝道:``儿郎们,不必管路上那些猴崽子,踏死了就算数,快马疾驰,咱家今日要到长安给贵妃报拜年呢!''

原来去年安禄山到长安,极力巴结杨贵妃,尽管他的年岁比杨贵妃大得多,却得杨贵妃收他为养子。他得了甜头;所以今年又赶来给杨贵妃拜年,他一人兼领平卢、范阳两节度使还不满足,尚想钻营杨贵妃的门路,兼领河东节度使呢!他钻营心急,所以一路催军马疾行。

新年初一。农家之尽情欢乐,聚集在村头村尾的闲人甚多、尤其是儿童们。更象甩了绳的猴儿,到处戏耍,这时便有一群十岁左右的孩子,在大路作掷钱的游戏。

安禄山的扈从疾驰而来,挥起皮鞭,辟辟啪啪的乱打,路边的闲汉,也有几个人着了皮鞭,吓得纷纷奔逃,那还敢到路上去救护孩子。

孩子们惊得叫爹叫娘,乱成一片,但大的、机伶的急忙跑开。却还有三个年纪较小的孩子,大致是吓得软了,在大路上连爬带滚的,尚未来得及滚开,眼看就要伤在铁骑之下!

蓦地一条人影,横里掠来,疾如鹰隼,只见他用双手一抓,抓起了路当中的两个孩子,一摔便又摔出去了,说时迟,那时快,当头那骑已冲了过来,路上还有一个孩子,那人则抱起孩子,那匹高头大马离他已不到三尺之地只听得``唰''的一声,马背上的骑士一鞭挥下,那匹战马,给他一阻,人立跃起,两只包着铁掌的马蹄也向他踏下来。

就在这危险之极的一刹那,只见他抱着孩子,用脚尖一撑,身于斜飞出去,皮鞭唰的一声掠过,勾下了他的一片衣襟,却没有伤着孩子,那匹战马踏了下来,正是他刚才站立的所在,前后之间,相差不过一瞬!

史逸如只道这人是段-璋,这时方才看清楚了,却是一个乡下少年,穿着一件灰色的棉袄,土头土脑,想不到身手竟是这般矫健!

转眼间这队官军已经过去,那少年放下了孩子,说道:``孩子们受惊了,请那位叔伯送他们回家吧。''

这三个孩子的家人正巧在场,急忙跑来察看,只见路边一堆稻草堆中,爬出了两个孩子,尖声叫道:``妈妈,妈妈。''正是他刚才摔出去的那两个孩子,摔在稻草堆中,虽然受了惊吓吓,却一点没有受伤。

众人都抢上来,看顾孩子,乱哄哄中,那乡不少年却已悄悄走开,待到孩子的家人想起要向恩人道谢的时候,那乡下少年已不知所在!

史逸如在这村子里住了十几年,村子里的人个个他都认得,刚才在紧张之际,无暇辨认,这时回想这少年的面貌,方始觉出他不是本村人,史逸如大为诧异,问道:``段兄,你认得这人吗?''他怀疑自己看得不清楚,所以再问一问段-璋,听不到回答,忽地发现段圭璋已不在他的旁边!

史逸如吃了一惊,抬眼看时,只见段-璋正在前面低首疾行,他把老羊皮袄的领子翻过来,蒙着了头,好像害怕寒风,显得瑟瑟缩缩的样子。

史家离路旁不过几十步路,这时他已走到屋子外边的一棵大树底下了。

史逸如本待再大声叫他,蓦地心念一动,疑云大起,暗自想道:``段大哥平素好仗义扶危绝不是一个胆小怕事的人,刚才那几个孩子险些受到马蹄践踏,以他的本领,尽可以去救,他却不去,这已是一奇,如今又悄悄的离开,连我也未告诉一声,这是什么缘故?再者,他是个练武的人,不该如此怕冷,却为何把皮袄的领子翻起来,蒙了头显得那般瑟缩的模样?晤,莫非他是怕有外人认得他的面目么?''史逸如是个读书人,心思周密,疑云一起,便不再叫他,匆匆忙忙的也赶回家去。段-璋已进了史家的院子,待得史逸如一到,他立即把大门关上,低声问道:``官军都过去了么?史逸如说道:``都过去了。大哥,你------''段-璋道:``进会再说吧,提防隔墙有耳,漏了风声。''

史逸如满腹疑云,两人携手,进了厅堂。段-璋又小心翼翼的把门关上。史逸如忍不住问道:``段兄,你莫非是以前犯过什么事么?''

段-璋苦笑一声,斟满了一杯酒,一饮而尽、悄然的说道;''大哥可是疑心我犯了皇法?皇法我未曾犯,只是曾经犯过一个无赖少年!''

史逸如越发诧异,说道:``大哥,你不是个怕事的人,即算曾经犯过一个无赖少年,你一身武艺,又所惧何来?''

段-璋道:``说来话长,你道这无赖少年是谁?就是你刚才所见到的那个平卢节度使兼范阳节度使安禄山!''

史逸如失声叫道:''哦,安禄山!''

段-璋道:``许多年来,我从未曾告诉过你我的来历,现在可以告诉你了。我本是幽州人,迁到贵村,为的就是避开这个安禄山!''

段-璋再饮了一杯,继续说道:``先祖累积军功,做到幽州的兵马使,算得是个不大不小的武官,先父不幸早死,我继承祖父遗萌,不知天高地厚,结交了一班无所事事的少年,平B在里巷之间专管闲事,打抱不平,自命侠义,其实这班少年,有半数以上,就是无赖,为了索饮索食,和我给交罢了。其中有一个便是安禄山。哦,那时,他还未姓安。''

段-璋顿了一顿,往下说道:``安禄山是西域胡人,本姓庸,母亲是突厥人,后来再嫁胡将安延偃,他这才冒姓安氏。''史逸如笑道:``不必管他本性什么,即然大家现在都知道有个安禄山,就叫他做安禄山吧。后来你和安禄山之间发生了什么事情?''

段-璋道:``这安禄山通晓六番语言,当时在幽州做互市郎,幽州这地方汉胡杂出,附欺是在市集上专责管理汉朝商务的一种小官,碰到双方言语不通的时候空防括环。他常常从中取利,欺诈善良的商民,外表上却是个豪爽的脱路,喜欢文回回阿阿好汉。我因为他保得几路拳律,又通晓六番语言,一时不察,认为他是个人才,也就和他交上了朋友。

``渐渐我发觉他的行为不当,也曾规劝过他,他却阳奉阴违,变本加厉,有一次他伪造证券,勒索一个商民,强迫人家送闺女给他抵债,这件事给我知道,一怒之下把他重重的打了一顿。从此绝交,安禄山在市集中众目睽睽之下,被我痛骂一声,重打-顿,无颜再混下去,第二天就失了踪影,不知去向。

过了几年,忽然听说他做起了平卢军兵马使来,原来他靠着后父的援引,投到幽州节度使张友圭部下当``捉生将'',边军重用胡将,他又善于钻营,兼之也立了几次功劳,所以升迁甚速,做了兵马使之后,不到两年,又升任平卢军节度副使了。而且带兵兵回幽州驻屯。

``那时我先祖遗留的一点薄产,已经挥霍得干干净净,落魄不堪,往日所结交的一班朋友已尽都散了。我知道安禄山是个眭眦必报的小人,他做了大官之后,作威作福的事情,我也听得不少。料想他回到幽州之后,一定放不过我,而我对故乡也以无可留恋,所以我便即远离故土,辗转流离了几年,方始在贵乡落脚。却想不到今天仍然在这里碰到了他。史兄,只怕今日便是你我分手之期了。''

史逸如道:``我只道你闯了什么滔天大锅,却原来不过是少年时候,曾经打过一个无赖而已。事隔多年,安禄山也未必记得吧?''

段-璋道:``安禄山把这件事情当作平生的奇耻大辱,只怕死了也会记得。我若不走,定然身罹奇祸,我死不足借,只是怕连累了妻子亲朋!

安禄山如今气焰滔天他的淫威,你今日不是也曾亲眼见了吗?'安禄山的残暴无道,史逸如并非不知,但他却不认为事情有如此严重,他和段-
璋多年朋友。实是不舍得一旦分开。因此又劝慰道:``今天在路边的闲人甚多,安禄山在前呼后拥之下,匆匆的驰过,他未必便在人堆之中认出了你?''

段-璋道:``古人说得好,防患未然。事情总得住最坏处想。万一祸患突如其来那时我要躲也躲不及了。何况自从去年安禄山巴结上杨贵妃之后,将是必定常到长安,这儿离长安甚近,总有一天会给他发觉。''

史逸如道:``你我二人情如手足,如今又结成了儿女亲家,理该患难与共,要走,咱们两家一同走!''

段-璋面有难色,半晌说道:``吾兄高义,可佩之至。只是嫂夫人刚刚生产,这,这如何使得?''

史逸如笑道:``嫂夫人不也是刚刚生产吗?''

段-璋道:''内人略通武艺,身体强健,事到急时,要走不难。嫂夫人乃是名门闺秀,怎过得亡命生涯,受得风霜之苦?''

史逸如道:``依我之见,要走也不争在这时。想那安禄山前往长安最少也得过了元宵方回幽州。嫂夫人虽说身体强健,刚刚产后到底不宜于远行,依我之见,不如再待个十天半月,那时两家同行,岂不是好得多?''

段-璋听史逸如说得甚为有理,再想到了儿女的亲事上头,若然两家就在今日分手,虽说有龙凤宝钗为凭,他年能否相见,却还是只能听凭天命。安禄山到了长安,免不了有许多官场酬座,京中富贵繁华,他又新拜了杨贵妃做干娘,也自得大大享乐一番。即算认出了自己,要报昔日被辱之仇,大约也得等他在长安回来再经过了这个村庄的时候。

想了半晌,段-璋终于接纳了史逸如的劝告,决定在元宵前一日。两家一同远走高飞。

史逸如本来要问他认不认得那个乡下少年的,这时方有机会提起。段-璋听了之后、甚为惊诧,说道:``有这样一个人吗?当时我一见安禄山的旗号,就蒙头溜开了。原来闹哄哄的是这一桩事情。''

史逸如见段-璋神色有异,心想:``那少年的本领的确是惊人,怪不得段大哥听了也觉惊讶。''

段-璋再坐了一会,料想安禄山那队官军已过了十里之外。便向史逸如告辞,约定史逸如明日到他家相见。

段-璋走后,史逸如回到内房,着望他产后的妻子和初生的女儿,妻子甚为虚弱,精神尚未恢复;女儿则粉玉雕琢一般,生得极为可爱。史逸如怕妻子忧虑,举家远走之事,准备持她调养好了,临行之时才告诉她。

那股段-璋拿来作为聘礼的凤钗,则先拿来给妻子看了。

史逸如的妻子性卢,乃是河东大族,富贵人家,见了这股凤钗,亦是啧啧称异,忙问他是现儿来的。史逸如说道:``是段大哥的。''卢氏说道;''是那段-
璋段大哥吗?''史逸如笑道:``还有那位段大哥?''卢氏道:``咦,这倒奇了。段大哥竟有这等价值连城的宝钗。''史逸如笑道:``还有更奇的呢,段大哥也是在昨天大年除夕的晚上得了一个孩子,不过咱们是个女的,他们是个男的。''卢氏道:``有这样巧的事情!你们是好朋友,孩子又在同一天出生!夫君,我说句笑话,这两个孩子倒象是天生的一对呢。''史逸如哈哈笑道:``不是笑话,婚事已经成了。这股凤钗就是段大哥给咱们女儿的聘礼呢。你该不会嫌他贫寒吧?''卢氏想了一想,说道:``段大哥、大嫂都是百中无一的好人,段大哥且是文武全才,我看目下的世道,只怕将来难免大乱,女儿嫁到他家,比嫁到什么书香门第、官宦人家更可靠得多,只是我却有点担心-一''史逸如忙问道:``你担心什么?卢氏道:``段大哥家道贫寒,却有这等宝钗,\ldots\ldots{}''史逸如笑道:``你莫非疑心他的宝钗来路不正?卢氏摇头道:``不是这个意思、以段大哥的为人、纵使是再值钱的东西,我也不会疑心他是不义之财但从他有宝钗这件事情看来,他定非常人,若非先代曾作高官,他本身就必是荆轲聂政这流人物。而他甘心在这小村子里默默无闻,依我看来,只怕他多半是惹了什么灾祸,避难而来的!''史逸如暗暗佩服妻子的见识。心中想到:我初见这股宝钗之时。也曾暗暗疑心,却没有她这样思虑周详,一猜便破。''但他为了怕妻子产后过份担心,对段-
璋与安禄山结怨之事,还是瞒过不提。只是说道``你猜得不错,他确是将门之后。这股凤钗是他先租李靖大总管西征的时候得来的。段大哥为人好义,也许得罪过一些小人,想不至于有什么大灾大祸。''卢氏道:``但愿没有就好。''

史逸如将宝钗交给妻子收好,出外给几个本家的长辈拜年,又到村头村尾走了一转,村人都在纷纷谈论着今早的事情。痛骂安禄山的草菅人命,称赞那无名少年的本领不凡。史逸如在他们的谈话中,知道事情过后。

并没有陌生入到村子来过。放下了心。想道:要是安禄山认得他,一定会派入打听的。既然无人来过,大可不必忧虑。''

他晚上回家,因为妻子在坐蓐期中,照习俗请有产婆陪她过夜。他吃过晚饭,看了妻子一躺,便到书房歇宿那时已起将近二更,他踏入书房,点燃蜡烛,忽见一个陌生人坐在里面。史逸如骤然见着一个陌生人坐在自己的书房里面,这一惊非同小可,烛光摇曳之中,但见此人乃是个满面虬须,全身披挂的军官,这军官未持他开口,便即起立相迎抱拳笑道:``不速之客,深夜造访,冒昧之至!好在段先生乃是江湖豪士,此类事情。当已司空见惯,想不会见怪吧!''

史逸如虽是个文弱书生,但胆气素豪,虽然由于意外,大吃一惊,待到看清楚来客是个军官,心中已明白了一半,这时又听得那军官称呼自己做段先生'',事情更是完全明白,心中想道:'段大哥今早躲入我家,不问可知,这厮是把我当作段大哥了!''

史逸如定了定神,他心内虽然明白,却佯作不知表出惊诧的神情问道,``尊驾何人,此来何意,尚请示之。''

那军官望了史逸如一眼,史逸如虽说心神稍定,惊慌的神色,到底不能完全掩盖,军官心里想道:``安大帅说他精通武艺,本领非凡,却怎的是个书生模样,一见我就吓得发抖呢?莫非他是大智若愚,大勇若怯,身怀绝技,却放意装出这般模样?''

那军官坐了下来,说道:``小可在平卢节度使安大帅髦下当个骠骑将军,小姓田,名承嗣,田土的田,奉承的承,嗣位的嗣。''他一口浓浊的山东口音,似是怕史逸如听不懂似的,一边说,一边用手指蘸了茶水在书桌上划,书桌上现出了``田承嗣''三字,好像木工用凿于凿出来似的,人木三分。

这田承嗣本是江湖大盗出身,以前在黑道上可说是无人不知,他自报姓名,并显露了这手本领,用意就在要慑服``段-璋'',使``段-璋''不敢抗拒。

史逸如根本不懂武功,这时他心中已有了主意,也就不再恐惧,对田承同的装腔作势,只觉得可笑,当下淡淡说道:``原来是田将军,久仰,久仰了,有何见教,请明白说吧。''

田承回露了这手武功,见史逸如反而神色如常,毫无怯态,心道:``果然他是真人不露相,我几乎走了眼了。''越发认定史逸如便是段-璋,因为摸不清他的深浅,心里反而有些发慌,当下又显露了一手``金刚手''

的功夫,轻轻一抹,将书桌上这``田承嗣''三字抹去,强笑说道:``原来段先生早已知道小可贱名,咱们现在的身份虽有不同,但却都是在江湖上混过来的,红花绿叶,同出一源,田某决不能得罪段先生,请段先生也不要令我为难,给我一点面子,和我一道走吧!''

史逸如仍然佯作不知,淡淡说道:``田将军,这可奇了,你我素不相识。你可要我跟你去那儿啊?再说,我也没有见过三更半夜来访客的!''

田承嗣霍地起立,神色紧张。沉声说道:``段先生,你也算得是个成名人物,田某已按武林规矩,以礼相邀,难道你当真要`敬酒不吃吃罚酒'么?走与不走,一言可决!何必婆婆妈妈的推三阻四,佯作不知?这岂是英雄本色?''

史逸如笑道:``我本来就不是英雄,而且我确实是还未知道将军的来意啊,就是请客也总得有个请客的因由吧?''

田承嗣``哼''了一声,道:``这因由么?请你问咱们的节度使安大帅去!''

史逸如道;''哦,原来请客的竟是`安禄山'么?''

田承嗣道:''是呀,安大帅吩咐,无论如何,都要请你先生驾到。所以你不去也得去!''顿了一顿,又转过稍为温和的口吻说道:``段先生,你是明白人,不必细表。田某乃奉上命差遗,不得不然,请你不要再难为在下了。''原来这田承嗣对``段-璋'也有几分怯意,要不然他早就动手了。

史逸如在尽量拖延时候,这时间他已转过无数反反覆覆的念头。要是去了吧,结果如何,殊难预料。而且他半生讨厌权贵,像安禄山这种残民以逞,割据一方的土皇帝尤其是他憎恨的人。若在平时,他是宁死也不会去见安禄山的。但现在却涉及段-璋,要是不去吧,他就得说明自己的身份,让这个田承嗣明白,这是一场误会,他并不是段-
璋可是,这样一来,段-璋却就难以脱身了。

田承嗣迫到了最后关头,史逸如把心一横,暗自想道:我去还不打紧,安禄山的手下捉错了人,他纵然蛮不讲理,也未必便敢把我杀掉、段大哥去,最少也免不了一场凌辱他是一个死不辱的响当当的汉子,我说出真相,那即是害了他一条性命?''

史逸如心意已决,立即打了一个哈哈,仰天笑道;'安节度使居然知道有我这个人,还派了一位大将军来访,当真是令我受宠若惊了!这是求之不得的事情,说不定我还可以混个官儿做做,哈哈,既蒙宠召,焉有不往!''

田承嗣的心情本就像绷紧了的弓弦,随时准备动手。听他这么一说,登时松了下来,笑道:``段先生果然是明白人,听安大帅说你和他本来是老朋友,只要你肯说几句好话,你想做什么大官,都是易如反掌!段先生,我早已准备好了马,就请动身吧!''

史逸如却好整以暇的一笑说道:``这么急?我总不能说动身就动身呀!''

田承嗣面色一沉,哈声说道:``你还有什么事情?安大帅吩咐,要我在天亮之前,将尊驾`请'到长安要是再拖延时候,我可以等你,安大帅却不能闲着在那里等你!''

史逸如道:``我总得和家人道别一声吧?''

田承嗣笑道:``要不是我早已知道你的身份,我真要把你当作一个酸秀才了。大丈夫做事,岂有这样沾沾滞滞的?你去和家人道别,一时之间,那里说得请楚?万一你的婆娘哭哭啼啼,闹到天明,只怕还未能动身!

歇了一歇,又道``我看在你是武林同道的份上,丝毫没有惊扰你的家人,你又何必在这半夜三更将他们吵醒?''心里想道``这段-
璋枉有那么大的声名,却怎的简直不懂江湖规矩,也不象个江湖人物!''

其实史逸如也并不想去和妻子诀别,令妻子伤心,他这样说。乃是另有打算。而田承嗣的不肯答允,也早已在他意料之中。

他听得田承嗣井没有扰及他的亲人,先放下了一重心事,当下说道:``话更如此、但我此去,不知何时归来,总得留个字儿,免得他们疑神疑鬼,平白担忧。''

田承嗣甚不耐烦。但也只得说道:``好,你就留个字儿吧。不必涉及安节度使,胡乱找个籍D,只要让你家人知道你是平安就行了。将来你衣锦荣归,再令他们大大惊喜一番。''

史逸如笑道:``我懂得,当然不会涉及安禄山。''提起笔来,立即写了一封短札,只说出外谋事,叫妻子若遇困难,可找亲友帮忙。田承嗣在旁看他写信,不作一声。

史逸如将信笺用墨砚压住,摆在书桌当中。心里想道:``我妻子比我聪明,她明天一早,见了这封信,当会料到我是遭遇了意外,立即便会派人告诉段大哥。那时她虽然是伤心。总比现在夫妻诀别要好过一些。段大哥也定然会照料他们母女,保护她们远走高飞!''可怜史逸加虽然煞费苦心,他到底缺乏江湖经验,怎知田承嗣也早已有了安排,要不然怎能容许他写这封信?田承嗣悄声说道:``脚步放轻一些。''两人走出书房,田承嗣一个飞跃上了屋顶,见史逸如没有跟来,连忙跃下,含怒问道:``怎么,又不想走了吗?''史逸如道:``我在自己的家中,我离家也不能这样鬼鬼祟祟,要走,我得从大门走出去!''江湖中正巧有这么一条规矩,有身份的武林宗匠。纵使受人胁迫,也定然要走大门离开,才不至有失身份、田承嗣暗自骂道:``这个时候,还讲这些臭排场!''但也只得依他,从大门走出去。史逸如一看,门外已经有了三匹上了鞍的骏马。

一个黑衣军官走了上来,抱拳说道:``这位就是段先生吧?小弟薛嵩,以前也曾在幽州混过一些时日。段兄大名,如雷震耳,今日幸会。''安禄山手下,有几个得力的将领,薛嵩亦是其中之一,史逸如答礼道:``薛将军的大名,在下也是久仰的了。''薛嵩得意之极,哈哈大笑,史逸如不知他笑些什么,只听得田承嗣说道:``听说以前为了清河沟李家的事情,你们几乎要刀兵相见,有这回事么?''薛嵩道:``是呀,连时间都约好了。后来那个自称是虬髯客弟子的出头,将事情化解,我与段兄也就各走东西,始终就没有再见过面,哈,哈,说起来这是十四年前的事了。''田承嗣笑道:``以后咱们都是同僚,你们两位也可以多多亲近了!''史逸如根本就不知道什么清河沟的事情。好在他们忙着赶路,薛嵩按照江湖礼貌,叙了几句之后,立即催他上马,没有再说下去,史逸如才得免露出破绽。

田承嗣在前,薛嵩在后,他们两匹马将史逸如夹在了当中,原来这薛嵩也是江湖大盗出身,一手袁公剑法,出神入化,安禄山差遣这两个人来。乃是防备段-
璋抗命的,薛嵩刚才在外面接应,亦自准备有一场激斗,想不到田承嗣将事情办得这样顺利,他也是喜出望外。

史逸如的心情却是非常沉重,他跨上雕鞍,回头一望,心中想到:``她现在也许还在梦中,怎知己是夫妻离别?呀,不知以后还有没有夫妻重见之期?父女会面之日?女儿刚刚出世就失掉父亲,她将来长大,不知要如何悲痛?同时,心中忽又起了一层疑云,田承嗣来到他家,在他的书房里缠了他将近半个时辰,卧房在屋子内进,距离较远,妻子产后虚弱,熟睡了就不易醒来,这犹可说他家中一个书僮,一个婢女,另外还有一个请来的产婆,晚上是准备不睡觉来照料产妇和婴儿的,他们为什么都一点没有听到声息?他和田承嗣在书房里说了这么久的话,难道睡在书房后间的书僮都听不见么?可是这时已不容许他仔细思索了,田承嗣己经是放马疾驰,在前带路,他只得紧紧追随,他虽然不精于骑术,但他那匹马却是久历疆场动骏马,不必他驱策,就安安稳稳的驮着他跟着前头那匹马疾跑。

他家间长安不过六十里这三匹马都是日行数百里的骏马,不过两个时辰,便到了一处地方,前面是一座山,山下有一幢大屋,史逸如认得那就是骊山,原来这座大屋,就是安禄山在长安的府邸。

这时刚是五更时分,天还未亮,田薛二人带他从角门走入,请他先到卫士聚集的白虎堂歇息。

薛嵩得意洋洋的说道:``这位就是大名鼎鼎的段-璋以后你们多多向他请教。''

白虎堂里有十多名轮值的卫士,听说是段-璋,都``啊呀''一声,站了起来,待看清楚了史逸如的相貌,却又不禁都怔了一怔,心中均是想道:``这曾经纵横河朔,大名鼎鼎的段圭璋,却怎的竟是一个白面书生?''

这班卫士虽然觉得``段-璋''的相貌出乎意料,但段-
璋的威名,十多年前就已震惊河朔,那个敢予轻视?因此仍是纷纷上前敬礼,史逸如也大模大样的,谁向他敬礼,他都是大马金刀的坐着,淡淡的点一点头。

一个卫士问道:``段大侠见多识广,目下咱们就有一件事情,想向段大侠请教。''

史逸如摆了摆手,道:``不必多礼,说吧!''

那卫士道:``近年来有个名噪武林的妙手空空儿,段大侠可知道他的来历吗?咱们的大人想礼聘他,不知段大侠可有办法?''

史逸如冷冷说道:``什么空空儿,俺从来没有听过!''

那班卫士们大吃一惊,做声不得。要知武林中出类拔萃的人物十居八九,都是唯我独尊,目中无人。他们只道``段-璋''是看不起空空儿,所以语气才这样轻蔑。那个向他请问的卫士更是心中想道:``一山难容二虎,他投到大师的帐了,当然不愿有胜过他的人。我请他设法去找空空儿,实是失言,少不得要碰他的钉子了。但他居然敢轻视空空儿。只怕确是身怀绝技,名不虚传!''

这个卫士碰了钉子,大家都不敢作声。田承嗣微微一笑,扭转话题,问另一个卫士道:``事情办得怎么样了?''

那卫士道:``扎手得很,那个老的,武功怪异,咱们都瞧不出他的路数。还有一个小的,不知是不是他的徒弟,土头土脑的似是一个乡下少年,手底却非常狠辣、连张统领都给打伤了。''

田承嗣问道:``伤得重不重?''那卫士道:``侥幸可免于残废,但最少也得卧床三个月,田将军,我看你还是亲自出手得好。''

史逸如听他们说起那乡下少年的形貌,心中一动,想道:``莫非就是昨日在马蹄下救人的那个少年?''

田承嗣笑道:``段大哥来了,这件功劳正好让给段大哥作见面礼。段大哥,梅花针刺穴的功夫想来你定然可以解?''

史逸如未及回答,忽听得牌官高声传令道:``大帅传田二将军偕同段-璋进见!''

原来这时天色大亮,安禄山已升堂了,正是:肝胆照人真义士,不辞刀锯为良朋------

\chapter{第 二 章 无赖少年成贵显
高风义士陷囹圄}\label{ux7b2c-ux4e8c-ux7ae0-ux65e0ux8d56ux5c11ux5e74ux6210ux8d35ux663e-ux9ad8ux98ceux4e49ux58ebux9677ux56f9ux5704}

史逸如随着田薛二人,未上台阶,只听得安禄山已在堂上咯咯笑道:``小段、小段,你往日骂我无赖、泼皮,没有出息,今日如何?是你有出息还是我有出息?''

史逸如故意低下头来,默不作声,田承嗣身材高大,比他高出一个头有多,安禄山未瞧得真切,又哈哈笑道:``段-璋,你也知道害怕了么?

念在故旧之情,你给我磕头认错,我这里正缺少一个养马的厮投,就赏给你这个差事吧!''心中想道:``且待你磕头认错之后,我立即命人把你的膝盖削掉,废了你的武功,令你终生受辱。强似把你一刀两段,倒便宜了你!''安禄山正在得意非凡时,史逸如猛地抬起头,朗声说道:``区区不才,也曾中过进士,做过郎官,节度使要我做你的马夫,这与朝廷体例不合,恐怕你得先要奏请皇上准许,把我的功名革了才行吧!''想起科举制度起于唐朝,唐太宗李世民开科取士,看见士干鱼贯进入试场,曾得意笑道:``天下英雄尽人缴中矣!''他为了要笼络天下读书人,让人重视科举制度,曾立下条例,人了学的便可免除官差劳役,中了秀才的可免官刑,中了进士的,那更不用说了。安禄山吃了一惊,圆睁双眼,道:``你是什么人?怎么来到这里?''史逸如道:``我是大唐进士史逸如,怎么来的,请你问这两位将军!''

安禄山拍案骂道:``混帐,混帐!我叫你们去拿段-璋,你们怎么拿了这个人来?''

田承嗣这一惊更是非同小可,暗暗叫苦,急忙道:``我们并没有认错地方,的确是到了段家,我们说得清清楚楚,大帅请的是段-璋,这个人就跟来了!''

史逸如道:``我几时对你说过我是段-璋?你们硬要派我是段-璋,拿刀弄杖,凶神恶煞一般,我怎敢分辨。怎敢不来?你说你进的是段家,节度使可以再派人查问,我家在村中无人不知,看看究竟是史家还是段家?''

薛嵩上前禀道:``纵使我们进错了人家,白天里大帅你也看见,那个蒙着头的汉子是躲进他家的。那个汉子大帅既认得是段-璋,而又躲进他家。不用说是和他有干连的,大帅要拿段-璋,应该着落在他的身上!''

田承嗣和薛嵩是安禄山最得力的两个大将,安禄山只得给他们三分面子,小骂一顿,也就算了。回过来斥史逸如说道:``你也不是好东西,你不要自恃曾中进士,在我眼中,进士也一文不值,杀死你只当踩死一个蚂蚁!说,段-璋在哪里?''

史逸如大笑道:``你草菅人命,滥杀无辜,不必自吹自擂,我也是早已闻名的了!老实说,我要是怕死,也不会到你这来了!''

史逸如不过是个文绉绉的书生,安禄山的左右却多是杀人不眨眼的魔鬼,但史逸如此言一出,这些魔鬼,无不骇然失色!试想安禄山手绾兵符,权倾中外,凡曾有人敢在他面前如此放肆狂言,毫无忌惮。

安禄山气得七窍生烟,拍案骂道:``托、拖下去,打、打死了!''

他旁边的一员大将忽地起立说道:``元帅皙息雷霆之怒,可否听我一言?''这人就是安禄山的结拜兄弟,平卢军副节度使史思明,职位仅次于安禄山,而智谋则在安禄山之上。

安禄山道:``史兄弟有句话说?''

史思明道:``这史逸如颇有文名,而且以强项著称,听说他当年中了进士之后,曾上`治安十策',又曾弹劾当朝的宰相李林甫,因此罢官。

这种有名气的读书人,杀了恐招非议。我听说李太白曾在宫中使酒驾座,有一次酒醉之后,甚至曾叫高力土给他脱鞋,贵妃娘娘给他磨墨,这样的狂生,皇帝尚可容他,元帅,你若只想做到目前的职位,便心满意足,那么杀了他也无所谓,如其不然,何妨贷其一死,好让天下人也知道元帅是个礼贤下士之人?''

安禄山虽然祖鲁,却也是小有聪明的。他一时之气,要杀史逸如,如今听了史思明的这番话,却不由得心意一转。原来他野心勃勃,早已想篡夺李唐的江山,史思明的活,实即是暗中提醒他,要他收买人心,尤其是对于士大夫,不宜太过得罪。

安禄山心念一转,大声笑道:``好,皇帝老儿可以容得一个李太白,难道咱家就容不得你么?好,好,我看你胆量不小,也象是个有用之才,你就做我的记室(官名,相等于今之秘书)吧!至于那个段-
璋嘛,你替我将他找来,我也一样给他一名武官做做。你总该没话说了吧?''

史逸如怒极气极,大声冷笑道:``史某不才,也曾读过圣贤之书,识得忠奸之别!史某连朝廷的官都不愿做,岂能屈志降心,事你这般乱臣贼子!''

这一番恶骂,休说安禄山受不下,连史思明也吓得面都黄了,颤声叫道:``你,你,你,天下竟有你这样不识抬举的人!''

安禄山大怒骂道:``好,你们这些读书人看不起我,我就不要你们这班读书人,一样我也可以打天下!''

安禄山盛怒之下,史思明也不敢劝了。这时恰有一个卫士走进来,见此情形,不禁呆住。

安禄山喝道:``什么事?''那卫士屈下半膝,道:``禀大帅,这位段大爷的家眷已请来了!''原来田承嗣对史逸如所说的没有惊扰他的家眷,乃是假的,试想安禄山要捉拿段圭璋,如何能容得他的家人留下,让她们泄漏出去?不过,当时田薛二人,忌惮段-
璋了得,若然要用硬功,将他的家人一并捉拿,生怕引起一场激斗,互有损伤,故此满口江湖义气,将``段-璋''稳住,骗他动身。然后再由早已埋伏在他屋后的卫士,将他的家人尽数擒来。当史逸如田承嗣在书房里说话的时候,薛嵩早已用秘制的毫无气味的迷香,将他家人都迷晕了。安禄山大声笑道:``好呀,我看你还要不要妻儿?服不服我?''

笑声未停,猛听得史逸如一声大喝道:``无赖恶贼,我段大哥一点也没有说错你,朝廷用你这样的人做大将,当真令人痛心,我死为厉鬼,也不会饶过了你!''他听得妻儿被捕,一时急想,竟然不颀一切,一面痛骂一面就扑上堂来,安禄山倒吃了一惊,但不必待他吩咐,早已有卫士将史逸如挡住,可怜史逸如乃是一介书生,如何敌得住如狼似虎的卫士,被一个卫士当胸一推,一口鲜血喷了出来,登对倒在地上,晕过去了。

安禄山摇了摇头道:``读书人中,有这等硬汉,倒是少见。好,你要求死,我偏偏不让你死。待我慢慢将你折磨,看你服是不服?''

史思明也笑道:``这姓史的仗着一时气血之勇,胆大妄为,顶撞元帅,待他这股气一过,自然要想及妻儿,那时元帅再给他一点恩惠,不愁他不服。''

安禄山道:``说得是。''便即吩咐卫士,将史逸如幽禁起来。

先头那个卫士,始知捉错了人,问道:``这姓史的妻子如何发付?''

安禄山道:``罗里罗嗦,囚禁女牢里去,还用问么。''

那卫士应了一声:``是!''正待退下,安禄山忽道:``他的妻子姿色如何,唤上来看看。''

薛蒿忽地抢出来答道:``禀大帅,这妇人姿色平庸,且是刚刚产后\ldots{}

\ldots''未曾说完,安禄山已大怒斥道:``晦气,晦气,你真是一个混蛋,怎么将个产妇拿过了府邸来!''那时官场甚多忌讳,安禄山害怕产妇的血光冲犯了他的``官星'',故此勃然大怒。

那卫士被他一顿痛斥,暗叫冤枉,道:``拿是你叫我拿的,你又没有吩咐是产妇就不拿。''同时,又觉得十分奇怪\ldots\ldots 要知史逸如的妻子乃是名门闺秀,虽在产后,仍不掩其沉鱼落雁之容,这个卫士是将卢氏背上马车的人,当然看得清清楚楚,心中想道:``这妇人十分美貌,怎的薛将军说她姿色平庸?''

薛嵩见安禄山发怒,又上来禀道:``这姓史的妻子是个产妇,囚在府中,确是不便。卑将大胆向元帅求个情,便请将这个妇人交卑职处置吧。''安禄山笑道:``你要她何用?''

薛嵩道:``卑职最小的那个儿子尚未断奶,这妇人刚在产后,奶水充足,卑职想要她做个奶娘,且她知书识字,犬子将来也好跟她认几个字。''

安禄山大笑道:``薛将军你今日大发慈悲,倒也少见。好,好,你不怕晦气,就领她去吧。''

原来薛嵩是个好色之人,他故意将卢氏说得姿色平庸,将她领去,实是别有意图,心怀不轨,想持她满月之后,调养好了,便要占为已有的。

安禄山道:``这段-
璋没有拿来,咱们总是放心不下。他的踪迹既然在那村子里发现,谅他还未曾远去,田薛两位将军,今日还要辛苦你们一趟。''当即发下令箭,又添了四名得力的卫士,叫他们务必将段-
璋捉来。且说段-
璋初一那日与史逸如分手之后,回到家中,她的妻子窦氏,乃是隋末``十八路反王''之一窦建德的曾孙女儿,窦建德被李世民袭灭之后,后人仍然在绿林中做没本钱的生意,儿子、孙子,都是名震江湖的巨盗,可说得上是个``强盗世家'',但窦线娘,虽然武艺高强,却不喜欢打家劫舍的生涯,有一次她和段-
璋相遇,双方比武,不分胜负,互相爱慕,终于结成夫妇,窦线娘嫁夫之后,荆钗裙布,操持家务,尽敛锋芒,村子里相识的人都只道她是个普普通通的良家妇女,谁也不知她曾是名震江湖的女盗。因为她自幼便扎下坚实的武功,所以虽在产后,身体依然强健。

段-璋见了妻子,先把史家的亲事对她说了,窦氏亦是甚为欢喜。段-璋深知妻子是个女中豪杰,多大的风险也敢担当,接着便把碰到安禄山的事情,以及他与史逸如约定,只待过了元宵,便即两家一齐出走等等事都对她说了。

窦线娘道:``两家同走,当然是好,但却也不能不提防在元宵之前,安禄山便会派人拿你。''段-璋道:``依你之见如何?''

窦线娘道:``若在平时,安禄山帐下纵然高手如云,也未必拿得着咱们,此际。我刚刚产后,武功最多及得平日三成,又添了这个孩子,只怕大难来时,我母子俩反而成为你的累赘。''\,'段-璋道:``这是什么话?

咱们生则同生,死则同死,我还能抱怨你吗?''窦线娘微笑道:``不是这等说,我得与你同死,固然无憾,但你就不想保全咱家这点根不成,所以依我之见,依我之见\ldots\ldots{}''

段-璋说道:``咱们夫妻还有什么不好说的,依你之见怎么?说下去把!''

窦线娘道:``我说了你可不要生气。依我之见,你就让我先走一步。''段-璋道:``不等史家兄嫂吗?这,这,这怎么使得?''

窦线娘道:``不是撇下他们,我的意思是你留下来,待元宵之后,史家嫂子调养好了,你就保护他们到我家来、''段-璋双眼一睁,失声叫道:``什么,你要先回母家?''

宾线娘微笑道:``我虽在产后,对安禄山帐下的高手或者敌他不过,对沿途的小贼,我还未放在心上。因此不如让我带了孩子,到我兄长那儿暂避些时。你与史家兄嫂随后跟来,这岂非两全之计。''

段-璋佛然不悦,说道:``娘子,你当年随我出门,说过些什么话来?''窦线娘道:``当年我的叔伯兄长,要你入伙,你誓死不从,我也因此与他们决裂。出门之时,曾经说过,若非他们金盆洗手,我决不回来,决不再做强盗!''段-璋道:``那么,现在他们金盆洗手了吗?''窦线娘道:``现在是急难之时\ldots\ldots{}''段-璋截着她的话道:``一个人的志节,不该因为遇到艰难险阻,便即变移。再说,咱们在危难的时候才去投靠他们,纵使他们不加耻笑,我也是觉得没有面子!''

窦线娘知道丈夫傲骨棱棱,小事随和,碰到有关出处的大事,脾气则是十分执拗,知道劝他不转,叹口气道:``既然你不愿意,那就算了吧。''

段-璋怕妻子难过,又安慰她道:``安禄山巴结上杨贵妃,此刻正在京中享乐,未必便会来与我为难。纵然要来,也未必便在这几天,且待我想想办法。你身体虽然强健,刚刚产后,还是不要操心的好。你早些安歇吧!''

段-璋家贫,请不起服侍产妇的``稳婆'',段-璋服侍妻子过后,捡出了他以前所用的宝剑和暗器,到院子里将宝剑磨利,喟然叹道:``剑啊,剑啊,我将你弃置了十多年,今日又要用到你了!''

正自心事如潮,忽听得屋外有``嚓嚓''的声响,声音极为微细,但落在段-璋这样的大行家耳中,立即便知道是有极高明的夜行人来了!

段-璋心道:``好呀,来得好快呀!看来,我今晚只怕要大开杀戒了!''正月初一的晚上,天边只有几颗淡淡的疏星,院子里黑沉沉的,段-璋躲在墙角,一手执着宝剑,另一只手伸到暗器囊中,首先摸出两枚极毒的三棱透骨镖,想了一想,又把毒镖放回,换过两颗无毒的铁莲子。

铁莲子刚刚扣在手心,说时迟,那时快,只听得猎猎的衣裤带风之声,两条黑影已自飞过墙头,段-璋蓦地长身,一声喝道:``咄,给我躺下!''他是武学名家身份,虽然遭逢劲敌,迫得使用暗器,却也不肯毫无声息的暗中偷袭。

那料两颗莲子打出,竟如泥牛入海,无影无踪。既没有打中敌人,也没有听到落地的声因,段-璋方自一怔,他本来已听出这两人并非庸手,但还未料到他们的本领如此的高强。只听一个苍老的声音哈哈笑道:``姑爷,你的暗器功夫越发了得了!''

段-璋道:``呀,原来是三哥!''那老者笑道:``难为你还记得这门亲戚,一别十载有多,怎么连个信也不捎来?''

窦线娘有兄长五人。这个老者排行第三,名为窦令符,段-璋虽然不愿与他们同流合污,但亲戚之情总还是有的,当下便邀他们进入内堂,燃起蜡烛,只见窦令符身有血污,另外一个则是十七八岁的少年,一身灰布衣裳,从外貌看来象个农家孩子,一声不响地站在窦令符身旁,对段-
璋神情冷淡。段-璋甚为纳闷:``他深夜前来,不知所为何事?看他衣裳上的血渍,似乎是受了一点外伤。''

窦令符道:``傻孩子,一点礼貌也不懂,见了长辈,还不磕头?''

那少年只好给段-璋磕了三个响头叫了声:``姑丈。''

段-璋将他扶起。心想:``我离开他们的时候,三哥只有一个女儿,这个孩子若是他以后生的,不该有这么大。''

那少年甩了甩手,不要他扶,便站起来,手掌平伸,``当''的一声,一颗铁莲子从他指缝间跌下来,那少年冷冷说道:``姑丈,这颗铁莲子交还给你!''

段-璋大吃一惊,要知他刚才怀疑是安禄山派来捉他的高手,虽然在没有问清楚之前,不敢使用极毒暗器,但他发出这两颗铁莲子,却是运了七分内力,用的是重手法暗器打穴的功夫,窦令符能够接下不足为奇,这少年只有十七八岁年纪,却也能够硬接他的暗器,那就不能不令他大为惊诧了。

窦个符``哼''了一声,斥责那少年道:``真是个蠢才,你在江湖道上也走了两年,怎的还似个新出道的雏儿!''

那少年退过一旁,直瞅着段-璋,只听得窦令符继续说道:``以后在黑夜里切不可妄自逞能,用手来接对方的暗器,幸亏你姑丈的铁莲子没有粹过毒药,要不然,凭着你这点功力,焉能封闭穴道,毒气内侵,纵然不死,你这条臂膊也残废了。''随即在衣袖里摸出了一颗铁莲子,交还给了段-璋,一面教训那少年道:``听风辨器的本领你是早已学会的了,以后在黑夜里碰到暗器,你从暗器的破空之声,当可以听出对方的劲力,自己审度,要是能够接下的话,应该学我一样用袖子来卷,否则就该赶快避开。''

那少年道:``谢三叔的教训!''段-璋心道:``这番教训,也只说对了一半。要是碰到了绝顶的内家高手,根本就不容易听出对方的劲力。''

他一眼瞥去,只见那少年的中指淤黑,急忙掏出一包金创散来,笑道:``不经一事,不长一智,少年人吃点亏也有好处,话说回来,你我象他这般年纪的时候,只怕还没有他的本领和阅历呢!你手指痛吧?敷上一点药散就好了。''后面两句是面对那少年说的,那少年却推开了段-
璋的手,冷冷说道:``用不着,也没有碎骨头,稍微一点痛楚,就要用药,这还算得什么英雄好汉?''

窦令符笑道:``姑爷不要理他,他要充好汉,就让他受点痛吧。''

段-璋心想:``这孩子的脾气也真倔犟,难道他是因此怪了我?''这少年对段-
璋虽然冷冷淡淡,段-璋却很喜爱他,猛地心念一动:``今早在马蹄下救人的那个乡下少年莫非就是他?''正想开口问,窦令符已先问道:``我家妹子呢?''

话未说完,只听得窦线娘格格的笑声,从瓦背上跳了下来,说道:``三哥,什么好风,将你吹来了?''\,'原来窦线娘在听到了夜行人的声息之后,知道段-
璋在院子里,从正面来的敌人有他抵御,料可无妨,因此她到屋后巡视了一遍,看看有没有其他党羽,刚刚回来,就听到她哥哥的说话。

窦令符笑道:``六妹,你还没有忘记绿林中那一套伎俩,咦,你的面色怎么有些不对,是生病了吗?''

窦线娘笑而不答,段-璋笑道:``不是病,是昨天除夕晚上,刚添来一个胖娃娃。''

窦令符道:``恭喜,恭喜,可惜我这个做舅舅的没带什么见面礼了。''

那少年上前叩见窦线娘,窦线娘听他称呼自己做姑姑,有点诧异,连忙问道:``是那一位侄于,怎么我认不得呢?''

窦令符道:``六妹还记得燕山的铁寨生吗?''窦线娘说道:``哦,敢惜这位小兄弟就是铁家侄儿?小名唤作摩勒的,我记起来了,我和圭璋成亲那天,铁寨主也曾带了他的儿子来吃喜酒。''窦令符道:``那个孩子就是他了。''窦线娘说道:``嗯,日子过得真快,屈指算来,这已经是十年前的事啦,那时这位小兄弟还流着两筒鼻涕,和一群大孩子打架闹着玩,大约只有七八岁吧?想不到现在已长得这么高了,变成一位少年英雄啦!

铁寨主好吧?''那少年眼圈一红,窦令符道:``铁寨主就在你们离开之后的第二天过世,大哥收了他做义子。他学武的悟性最高,比咱们家的那些孩子都强,所以这次我什么人都不带,就带他来。摩勒,你想学梅花针的功夫,以后向你的姑姑多多请教。''

原来那燕山铁寨立名叫铁昆仑,乃是胡人,唐代的北方胡汉杂居,互通婚姻,汉胡之间的隔阂远不如后来之甚。铁昆仑的妻子便是范阳封季常老英雄的女儿,和窦家还沾有一点亲戚关系。铁昆仑的武功极高,窦氏兄弟与他们惺惺相惜,结成了生死之交,所以铁昆仑在受到仇人暗算之后,便将孩子托孤窦家。段-璋心道:``怪不得他年纪轻轻,便有如此造就。

原来他是铁昆仑的儿子。''

窦线娘问道:``三哥,你衣裳染血,这是怎么回事,是不是在路上杀了什么人来?''

窦令符哈哈笑道:``我半生杀得太多,今番却几乎给人杀了呢!''

窦线娘吃了一惊,道:``三哥碰到了什么强敌?家里出了什么事情?''她心想要不是出了事情,她的哥哥决不会万里迢迢来寻找他们。

窦令符道:``我今晚到来,正是有两件事情要请你们相助。''

段-璋道:``请说。''

窦令符道:``第一件事是请姑爷赠药。惭愧得很,我第一次吃了败仗,受了伤啦!''

段-璋不觉一怔,心道:``他只是受了一点轻微的外伤,怎么向我讨药?''心念未已,只听得``嗤''的一声,窦令符急不可待的撕下了一片衣裳,胸胛上有一点针头般大小的红点,说道:``你是大行家,可瞧得出么?''

段-璋骇然失色,道:``这是白眉针!三哥是和剑南唐家的人结了仇么?''白眉针是一种剧毒暗器,入了人体,可循着穴道,攻上心房,便即死亡。现在窦令符胸胛上的红点,距离心房不到五寸,那是很危险的了。

正是:江湖风浪重重险,那许荒村隐侠踪------

无忧书城扫校

\chapter{第 三 章 千里求援援未到
十年避祸祸难除}\label{ux7b2c-ux4e09-ux7ae0-ux5343ux91ccux6c42ux63f4ux63f4ux672aux5230-ux5341ux5e74ux907fux7978ux7978ux96beux9664}

窦令符道:``伤我这个人,我还未知道他的来历,但可以断定,他决不是唐家的人。''窦线娘问道:``三哥是给那个人暗算的吗?''窦令符道:``不是。双方光明正大的拼斗输给他的,虽然他用了这种歹毒的暗器,我也毫无话说。''窦线娘道:``这么说的确不是唐家的人了。''要知剑南唐家,虽然号称暗器第一,但若论真实的武功本领,却还不是窦氏兄弟的对手,武功到了窦令符这样的地步,除非对方出其不意的暗算他,否则明刀明论的交锋,纵有极歹毒的暗器,也断断不能伤了他的。但是段-
璋却还有些疑惑,心中想道:``这个人既然用白眉针射中了他的穴道还何须再用刀剑伤他?而且这仅仅是皮肉的轻伤,也不象高手所为,莫非他是前后受了两次伤?''只因绿林中忌讳甚多,冤仇牵连之事尤其不肯对局外人释说,段-璋既然不愿被牵连过去,所以虽有所疑,亦不愿多问,当下说道:``我家的灵芝祛毒丸虽然不是对症解药,但以三哥功力的深厚,眼了一丸,料想可以保得平安无事。''原来段-
璋的祖父在西征之时,得了一株千年灵芝,团成丸药,能解百毒,是以窦令符才向他求药。窦线娘进去取了灵芝祛毒丸给哥哥,从卧室出来,笑道:``孩子很乖,睡得正酣,我可以陪你们多坐一会。三哥,第二件事呢?''

窦令符面色一端,望着窦线娘道:``六妹,不知你念不念咱们兄妹的情谊?''窦线娘道:``三哥言重了,一母所生,同胞情谊,焉能不念?''

窦令符道:``若是你肯念兄妹情谊的话,就请你和妹夫一同回家,救救我们的性命!''窦令符知道段-
璋出身将门志行高洁,不肯与绿林中人混在一起,所以他虽然想请的是段圭璋,这番话却不直接向段-璋说。

窦令符望着他的妹妹,窦线娘却望着她的丈夫,半晌说道:``三哥,你先说说,这是怎么回事?''

窦令符道:``平阳王家的人最近与我们激斗了一场,说来惭愧,你这几个不中用的老哥哥全都败了阵啦!''

平阳王家的家世与窦家一样,是``十八路反王''之一王世充的后代,王世充被李世民袭灭之后,他的后人也成了强盗世家。王窦两家乃是世仇,明争暗斗之事无代无之,本来甚属平常,但窦线娘这次听了,却极为诧异。

原来王家到了目前这代,人才已是远远不及窦家,窦家五兄弟个个武艺高强,门人弟子数十,在武林中也都是响当当的角色。而王家只有一脉单传,当家的名唤王伯通,武功虽高,但若比起窦家五虎,却还略有逊色,既算单打独斗,窦氏兄弟任何一人也不会输给他,更不要说联手合斗了。王伯通仅有一子一女,尚未成人,门下弟子也远不及窦家之多,屡次争斗,都是窦家占胜,弄到后来,窦家的人,行踪所至,王伯通既远远避开,不敢与之争锋,所以这次窦线娘听得五位兄长全都败阵,不禁大为诧异。窦令符道:``六妹有所不知,如今黑道上的形势已与往昔大大不同,英雄辈出,我们老一辈的都给压倒了!''

窦线娘出嫁从夫,早已决心退出绿林,但对于母亲,究竟关心,连忙问道:``王伯通请来了什么厉害的人物助阵?其他几位哥哥可受了伤?''

窦令符道:``王伯通正是请来了一个极厉害的人物,名唤精精儿!''

窦线娘诧异道:''精精儿?这名字我还没有听过。''段-璋笑道:``我们在这村子里隐居了十年。真是快要变成聋子了!''

窦令符道:``近几年来,江湖上出现了两个极厉害人物,年纪轻轻,都不过二十来岁的模样,手段却狠辣无比,精精儿就是其中之一,另一个叫空空儿,我们没见过。听说比精精儿的本领还要高强得多,那简直是不可思议的了!''

窦线娘柳眉一扬道:``怎样不可思议?难道就凭精精儿一人,便能胜得五位哥哥?''

窦令符知道妹妹外柔内刚,正要激起她的同仇敌忾,叹口气道:``不要说了,窦家这次是一败涂地,连大哥都受了伤,还有四弟也中了一根白眉针!''

大哥窦令侃是湖北绿林领袖,武功之高,即段-璋也是佩服他的,起初他还不以为然,如今听说窦令侃也受了伤,方始吃惊!

窦令符道:``那天王伯通就只带了精精儿一个人来,精精儿长得又瘦又小。活像个小猴子,我们都不曾把他放在心上。他却要一个人打我们五个人,我们当然不愿自坠威名、先是二哥上去接战,不过数招,全身便全在他的剑光笼罩之下,四弟、五弟瞧见不妙,只好上去助阵,仍然给他迫得步步后退,最后我和大哥也只得加人战团,大哥仗着他那一对`天赐神牌',不惧宝剑,拚力抵住正面,我们四兄弟两翼包抄,激战了半个小时,好不容易将他困住,那知正在我们占得上风的时候,他便立即使出白眉针来了!''段-璋心道:``你们以众凌寡,本来就怪不得别人使用歹毒的暗器。''

窦令符继续说道:``若然换了别人,白眉针也未必奈何得咱们。可恨那精精儿狠辣非常,一手剑法,实在已到了出神入化的地步,就在施放白眉针的时候,剑法也丝毫不缓,紧紧迫着我们,我们若是闪避白眉针,就势必伤在他的利剑之下!两害相权取其轻,我们只好拼着毒针刺之凶,我与四弟动作慢在脚踝,大哥接连挡了他的三招杀手,结果性命虽是保全,左手的两只指头,却已被他的剑削去!尚幸二哥五弟没有受伤,就在那双方以性命相搏的刹那之间,各自还了他一剑,也让他添了两道伤,这才双方罢战。''窦线娘吁了口气,说道:''这还好,尚不至于一败涂地。''

窦令符道:``精精儿虽受伤,却只伤了一点皮肉,咱们却伤了三个人,说来也算是一败涂地了。''

窦线娘道:``四弟你伤如何?''她知道大哥本领高强,仅被削去两根指头,谅无大碍,四弟功力较弱,幸而所伤亦非要害,白眉针要升至心房,最少还要一个多月。

段-璋一算日期,窦令符中了白眉针之后,到现在也已超过了二十天,白眉针方从他的上臂循著穴道升至胸胛,心中想道:``以他的功力而论,在武林中亦已是罕见的了,普通的人,中了白眉针,最多不能活过三天,而大哥的功力,又最少比他高出一倍,但他们窦家五虎,联手合斗,却竟然给精精儿一人击败,这精精儿的本领,也确实是足以惊世骇俗的了。''

窦令符沉声说道:``六妹,你是窦家的人,你该知道咱们窦家从来不曾求过外人,好在你们也不是外人,我这次求援,还不算是出了窦家的例。''

窦线娘好生为难,一阵踌躇,眼角盯着她的丈夫,不敢回答。只听得窦令符继续说道:``当今之世,只怕只有妹丈的剑法可以与精精儿匹敌;六妹,你的本领,不是我们自己夸赞,在江湖上也是罕有伦比的了,尤其是梅花针刺穴的功夫,只有你得了爹爹的真传,无人能及。大哥的意思,要我接你们马上回家,待精精儿再来的时侯,由妹丈与他比剑,你在旁与他斗暗器,如此打法,想来可操胜算。六妹,咱们窦家就全靠你们夫妇俩了!''

窦线娘不敢作主,把眼望着丈夫,段-璋早已有几分不快,说道:``三哥,你妹子刚在产后,只怕有些不便。''

窦令符道:``那精精儿也得养好了伤。才敢再来,六妹只是在旁用暗器助阵,也不必费什么力气,最多满月之后,总可以应战了吧?''

窦线娘道:``段郎,你意下如何?''言下之意,她已是不成问题,只等丈夫的一句话了。

段-璋道:``你家里有了事情,你要回去,我不阻拦。我的武艺,已经搁下多年,那精精儿如此厉害,我自问不是他的对手!''

窦令符勃然变色,沉声说道:``你不愿去就爽爽快快说好了,你是英雄侠客,不肯从我们这门亲戚,我窦令符也不会厚着脸皮求你!''

段-璋道:``三哥,话不是这等说,我有一言奉劝,听是不听,任凭于你!

窦令符道:``说罢!''

段-璋道:``我劝你们正好趁此时机,金盆洗手!想那王伯通不过要与你们窦家争霸绿林,你们隐姓埋名,消声匿迹之后,难道他与精精儿还会赶尽杀绝?''

窦令符冷笑道:``好一个金玉良言!你不是窦家的人,但你娶了窦家的女儿,想来也该知道,窦家的家训是:宁死不辱!百余年来,从没有给人欺负上门,却缩头不出的。纵使要金盆洗手,也得先报此仇。''

段-璋心道:``若然说到报仇,你们欠下的命债大孽也不少吧,绿林中人在刀口上讨生活,胜负死伤在所不免,若然冤冤相报,杀了一个精精儿,难保就没有第二个精精儿。''但他见窦个符正在火气上头,这番话说出无异火上添油,他本来不善辞令,想说的既然不便说出,就索性闭了嘴,由得窦令符大发雷霆。

窦线娘本想劝她丈夫,只帮兄弟这次,见丈夫如此的神色,知道劝亦无用也就不敢做声。

窦令符衣袖一拂,恨恨说道:``算我上错了门,自己丢脸,告辞!''

窦线娘忙叫:``三哥,三哥,且先坐下,有话好说!''

段-璋道:``三哥定要报仇,人各有志,我也不敢再劝,这两颗灵芝祛毒九你带回给四弟吧!''

窦令符已是拂袖而起,谈谈说道:``不用了!反正医好了也还得再伤在精精儿剑下!''

窦线娘道:``这么夜深了,三哥,你要走也得明天再走吧!''

和窦令符同来的那个少年,一直在旁边冷笑,默不作声,这时却突然发活道:``住一晚不打紧,只怕姑丈做官的朋友到来。见到有绿林大盗住在你的家中,有些不便!三波,咱们还是马上离开为妙!''

段-璋怔了一怔,蓦地跳起来道:``摩勒,你说什么?''心中奇怪之极,暗自想道:``我平生也没有交过做官的朋友难道他们说的是史逸如么?史大哥却是早已辞官的了。何况他们乃是第一次到这村庄,却又如何知道?''

铁摩勒闪过一边,大声说道:``你交的好朋友,却怕我讲出来么?你不放我走,敢情是要将我缚去送给官府邀功?不错,今天在马蹄下救人的是我,冲闯了安禄山的也是我,你待怎么?''

窦令符斥责:``你义父不早教过你么,道不同,不相为谋。你多说什么?你惹了祸不打紧,我这几根老骨头也要被你连累,丧送在此了!''这几句话明里是斥责铁摩勒,其实却是针对段-璋。窦线娘吓得惊异不定,叫道:``三哥、三哥,你,你这是什么话?圭璋纵然不肯去帮你们斗那精精儿,他也不会翻脸成仇,要将你们缚去送官呀,你,你们把他当作什么人了?''

段-璋身形一晃,拦着了门口,冷静地说道:``三哥,把话说清楚了再走!''

窦令符冷冷说道:``你说得好,士各有志,不能勉强,你要到安禄山帐不图个功名官贵,也怪不得你不认我这门亲戚!但望你顾全一点江湖道义,待我们走了之后,你再去通风报讯如何?不过,你若当真要我们留下的话,我窦令符虽然不是你的对手,也绝不能束手就擒!''

窦线娘嚷道:``三哥,你说到那里去了?你不知道:安禄山正是段郎的仇人,今晚我曾和他商量避祸之计,准备逃走的啊!''

段-璋反而平静下来,说道:``二哥,这里面一定是有什么误会了。

你说说看,你怎么以为我到安禄山帐下求取功名呢?''

窦令符一听他们两人的说话,不似虚假,心中也是疑团莫释,便道:``这安禄山手下有两个得力将领,一个是田承嗣,一个是薛嵩,这两个人和你的交情如何?''

段-璋道:``我听过他们的名字,以前为了清河沟李家的事,薛嵩要约我比剑,后来虬髯客的徒弟出头,将事情化解,没有打成,一直到现在,我都没有和他们见过面了。''窦令符诧道:``你这话当真?那,那就奇怪了!''

段-璋道:``你信不过我也该相信你的妹子,你问问她,我平生几曾说过假话?''

窦线娘道:``这两个人确实与我们丝毫无涉,三哥,你怎的会把这两个人和圭璋牵在一起呢?''

窦令符道:``那么这个村头有一家人家,门前有三棵松树的,家主是个年的四十左石、白脸无须的书生,这个人难道也与你毫无关连么?''

段-璋道:``这个人是我的好朋友,他名叫史逸如。不错,这个姓史的做过官,他早在十几年前,就因弹劾奸相李林甫而被罢官的了。哈哈,你说我交了做官的朋友,莫非就是他?此人古道热肠,高风亮节,虽曾为官,却是侠义中人呢!''

窦令符道:``他既曾为官,你可知道他和安禄山有无关系?''

段-璋道:``史大哥与我十载深交,我素来知道他是痛恨安禄山的,更不要说和安禄山的牵连了。''

窦线娘插口说道:``有一件事你还未知道,史家嫂子也是昨晚得了一个女儿,我们和他已是对了儿女亲家。说起来,这姓史的也是你的亲戚呢?''

窦令符侣了捋须,沉吟半晌,说道:``这可令我越来越糊涂了。好吧,我且从头说起。''

``前几年有个朋友说在长安闹市之中,曾见过你匆匆走过,因此我猜想你大约住在长安附近,使和摩勒来找寻你们了。三天前在凤翔山道,却和安禄山帐下的八名高手遭遇,恶斗了一场。''

窦线娘问道:``你和安禄山也有仇么?''窦令符笑道:``你离开绿林不到十年,怎的连这个也不懂了。咱们窦家,就正是在安禄山管辖下的地区作强盗,要么就受他招安,要么就要与他作对,这不是很简单么?''

窦线娘笑道:``这我懂得。不过,我离家之时,安挥山还没有做书度使,我尚未知道咱们窦家正在他所管辖的地方。''

窦令符道:``我们非但不受他招安,在他兼范阳节度使那天,四弟还曾和他开过一个玩笑,偷了杨贵妃送他的一件名贵狐裘,因此他早就想收捕我们了。王伯通和安禄山帐下的田承嗣,以前是黑道上的好朋友,田承嗣投归安禄山之后,王伯通与他仍暗通声气,所以,据我猜想,这次我们在凤翔山道突遭安禄山手下的围捕,大约就是王伯通这厮通风报讯的!''

段-璋心想:``绿林中也有高下之分,我这几个舅子不屑同流合污、暗通官府,到底比王伯通胜过一筹。''

窦令符续道:``安禄山那几个卫士虽然算不上一流的高手,武功亦非凡俗,其中有一个叫做张忠志的,以前亦是黑道中人,手使一对虎头钩,最为厉害,我右臂上的伤痕,就是给他的虎头钩划破的。''

铁摩勒笑道:``三叔,你总是喜欢把敌人说得厉害了一些,若非你老人家故意卖个破绽,那姓张的如何近得你的身前?''

窦令符正色道:``摩勒,像你这样年纪,最容易犯轻敌的毛病。这个毛病不改,将来定吃大亏。须知绿林中的教训是:临敌之际,取胜第一,越快得胜越好,免至多生意外。纵使是狮子搏免,也该用全力。何况咱们不是猛狮,对方亦井非兔子呢。

``就以那天的情形来说,我身上有白眉钉的毒伤,对方合围之势已成,看得分明,他们是想拖垮咱们,若不是我故意卖个破绽,诱那张忠志上当,只怕还未必容易突围呢。像你那样强攻硬拼的打法,实在危险得很。''

教训了铁摩勒之后。窦令符回过头来说道:``我恨那张忠志以盗捕盗,同类相残,诱得他近身,立即施展霹雳掌的绝招,一拳打断他的肋骨,但他趁着我的破绽,也居然能够扎我一钩,也算得是强悍的对手了。''

窦线娘遇:``那八名卫士里面,没有田承嗣和薛嵩在内么?''

窦令符道:``田薛二人是大将身份,当然不在其中。也许是他们以为有八个人对付我个老头子,足已够了吧。''笑了一笑,又道:``幸喜他们不是怎样看得起我,要是田薛这两位将军亲自出马的话,我元气未复,远远不是他们的对手,只怕今晚已不能和你妹子相见了。''

窦线娘有点诧异,问道:``三哥,那你刚才说的\ldots\ldots{}''窦令符早知其意,立即把话接下来说道:``你是不明白我刚才何以要先提起这两个人?''那天我无缘与这两位将军相会,可是今天晚工,却见着了!''

段圭长也不禁吃了一惊,急忙问道:``今天晚上?你是在那里见着他们的?''

窦令符道:``就在这个村子里,还不到一个时辰。''窦线娘道:``这是怎么回事?''窦令符道:``你别忙,且听我按着次序说下去。''

窦令符接下去道:``过了凤翔山道,恰好在元旦这天,到了你们的村子,碰上了安禄山的大队人马,正急着要上长安,给他的贵妃娘娘拜年。

``我老头子是惊弓之鸟,不敢多惹闲事的了。赶紧在山谷口里藏起来,这小子却最初生之犊不畏虎,他却到谷口去瞧热闹。''

铁摩勒接着说道:``幸亏我出去瞧热闹,我一瞧就瞧见了姑丈把羊皮祆蒙着了头,脚不离地,步履安详,却走得甚快,一瞧就瞧出是个具有上乘武功的人。''

段-璋心中一凛,想道:``这孩子好厉客的眼光。糟糕,我一时心急,走快了两步,结果给他瞧破,他都能够瞧出我具有上乘武功,安禄山的随从高手,想来也会瞧得出的了。''

只听得铁摩勒续道:``后来就发生了安禄山的卫士马踏孩子的事,我忍不住把那几个孩子救出来。''

窦令符笑道:``幸亏他们忙着赶路,没功夫捉拿你。不过,也幸亏你瞧出了姑丈的武功,要不然我还不知道你们就住在这个村子呢!''

窦令符顿了一顿,继续说道:''摩勒一说,我就猜到是你,摩勒见你走进村头那家人家,我以为便是你们的家。''

道:``不错,我们正是在史家门口,看见了田承嗣和薛嵩。''

段-璋``啊呀''一声叫起来道:``你们有没有进去看?这史家大哥不知如何了?''

窦令符道:``我还瞧见一个年约四十,白脸无须的书生和他们在一起,谈笑甚欢,这样的情形,我还敢过去吗?''

段-璋大大吃惊,忙问:``你可听见他们说些什么?''

窦令符道:``我和摩勒躲在松树上,那时他们正在跨上马背。我只听见那薛嵩说什么,大哥一定给你官做。后来又隐隐约的听得他们提了两次,段先生,段先生,他们已经放马疾驰,话语听不情楚,似乎他们对这位`段先生'好生敬慕!''

段-璋道:``怪不得你以为那两个家伙是我的朋友,后来怎样?''

窦令符道:``还有怎样?你那位史大哥和他们走了,我也知道这不是你的家,于是到村中每一家窥探,好不容易,终于找到了你们。''顿了一顿,冷冷说道:``要不我还以为你有几分亲戚的情份,我也不敢来见你了。好吧,我听见的我都说了,不放我走,那就由不得你了!你若是要拿我去给安禄山作见面礼,就请动手吧!''

``动手''二字,刚从窦令符口中吐出,猛听得段-璋大叫一声,箭一般地射出门口。窦令符这一惊非同小可,失声叫道:``你、你、你当真-一''他只当段-
璋当真去告密,对他不利,急忙间无暇思索,也赶忙逃出段家。

他这句话未曾说完脚步刚刚跨过门槛,衣角已被窦线娘拉着,只听得窦线娘大叫道:``三哥,你好糊涂!''

窦令符道:``怎么?''实线娘道:``要是他要对你有所不利,还不会亲自动手吗?岂在这时候还去邀人,难道他不预料到你们也会马上逃走?''

窦令符的江湖经验比妹子丰富得多,窦线娘所说的道理简单明白,他当然也会想到,只因一时惊惧,时尔失态,如今一想,果然是自己的糊涂,遂停下脚步,回过头来,只见铁摩勒正在拨出一柄精光耀目的匕首,对准窦线娘的背心,原来他以为窦线娘不顾兄妹之情,要将他的``三叔''留难,故此备在必要之时,便与窦线娘拼命。

窦令符喝道:``摩勒,住手!六妹,你说,你说!你三哥的性命交付给你了!''

窦线娘笑道:``三哥,不必着慌,听我细说。''剔亮了红烛,将丈夫与安禄山结仇的经过,段史二家的关系,相约逃难的事情\ldots\ldots 一五一十,详详细细的都对窦令符讲了。

窦令符与铁摩勒这才完全明白,只听得门外鸡啼,已是五更的分,卧室内那初生的婴孩也啼哭起来,窦线娘的话刚好完毕,笑道:``我该给你喂奶了,这孩子倒乖,一睡就睡到天亮。他也该山来见舅舅了。''

窦线娘给孩子喂饱了奶,抱他出来,窦令符道:``这孩子骨格清奇,是个学武的好材料。''孩子出来,紧张的气氛冲淡了不少,但每个人心里,仍是忐忑不安。

忽听得一声长啸,段-璋的声音朗声吟道:``宝剑欲出鞘,将断佞人头,岂为报小怨,夜半刺私仇,可使寸寸折,不能绕指柔!''弹剑悲啸,宛若龙吟,大踏步走上台阶。

这时已是阳光微现,但见他须眉怒张,双眼火赤,窦线娘从未见过丈夫这等神态,吓得呆了,她尚未开口,铁摩勒却忽然地抢上前去,大声道:``我错怪了姑文!''冬、咚、冬,就给段-
璋磕了三个响头。

段-璋将铁摩勒扶了起来,仰天说道:``好,你爱憎分明,不愧英雄本色!''

窦令符也过来赔礼,段-璋却侧身避开,沉声地说道:``这个时候,还讲什么客套。三哥,我有一件事情,要重重拜托你了。''

窦令符笑道:``你我亲戚上头,怎用得上拜托二字,你刚才说不要客套,你自己却先客套了!''他见段-
璋如此的神情,情知定有非常严重之事,因此故意打个哈哈,缓和各人紧张的情绪。

段-璋指着他的孩子道:``三哥,请你照料他们母子二人,天一亮就带他们走吧!线娘,你要好好教养孩子,长大了以后将我的剑谱传给他。''

窦线娘本来就想带孩子到母家避难,并因此而与丈夫龃龉,想不到丈夫突然应允,她隐隐感到不祥之兆,颤着手儿,不敢接那剑谱。段-璋叹了口气道:``拿去吧,以后也许你我不能见面了。''

窦线娘道:``段郎,你要到那里去?''其实这对她已猜到了七八分了。

段-璋道:``我去寻史大哥去。''

龚线娘道:``你到史家看过了?到底如何?史家嫂子和她的女儿呢?''

段-璋道:``都给安禄山的爪牙绑架去了。''

窦线娘``啊呀''一声叫将起来。``真的?这真是意想不到的事!''

段-璋道:``这是意想中事,昨天我一时疏忽,避入史家,安禄山当然把史大哥当作我了。''

窦线娘道:``史大哥是个进士,他怎的不会分辨?''窦令符接着道:``我听那田承嗣说给他官做,妹丈,我看,我看,人心难测,你、你\ldots\ldots{}''

段-璋剑眉一坚,立即打断他的话道:``线娘,别人不知道史大哥的为人,难道你还不知道吗?他是为了要保全你我,已顶着我的名字去了!''

``我到了史家,屋子里鬼影都不见一个。在卧房里我嗅到有残留的迷香气味,在书房里我找到史大哥写的这封信。你拿去看吧!''

``你看,史大哥是何等苦心,他为了敷衍那田承嗣,故意和他说一些鬼话,难道你会相信他向安禄山求官?``你看史大哥是怎样信托咱们,遗书叫他的妻子找至亲好友照顾,他写这张字条的时候不便言明,这至亲好友除了咱们还有谁人?线妹,事情如此。你还不明白吗?''

窦线娘是绿林世家,对黑道上的伎俩,当然明白,恨恨说道:``这田薛二人,以前也是江湖上的成名人物,行为却这般卑劣。连妇人孺子都不放过!''

窦线娘心如刀割,她明知安禄山帐下高手如云,丈夫此去,定是凶多吉少,但事已如此,她那里还能够阻拦?而且她也是具有侠骨英风,探明大义的女子,在这关节上头若然换了是她。她也会象丈夫一样的舍生取义的。

夫妻四日相对,默默无言。过了好一会,窦线娘才用颤抖的手接过段-
璋的剑谱,低声说道:``段郎,你去吧!但愿吉人天相,你和史大哥、大嫂,都能平安回来!只、只可惜我刚在产后,不能和你同去了。''

段-璋微笑道:``你要把孩子抚养成人,这比我去拚死,还要难很多,我不能为你分劳,只有请三哥照料你了。''他极力使语调平静,但微笑之中仍然掩盖不住悲凉。

窦令符笑道:``圭璋,以你的武功,未必便不能归来,我们还等着你会对付精精儿呢!''其实这番说话,不过是慰他的妹妹而已,段-璋武功再高,闯入龙潭虎穴,双拳难敌四手,要全身而退,已极困难,何况他还要救人。''

鸡声已啼了三遍,段-璋道:``好吧,咱们都该走了。我和你们同走一程,到村头分手。''

元旦晚上,人们都睡得很迟,路上还未有行人,史家正在村头,在经过史家的时候、段圭璋忽然停下步来,说道:``让我看一下孩子。''

他在孩子的面颊上亲了一下,沉声说道:``若是我万一不能回来的话那史大哥也是不能回来的了。孩子长大了之后,你要他打听史小姐的下落------希望她还能活在人间。若是毫无音讯,也要等到三十岁之后,方能另娶。那股宝钗,你要藏好,作为凭证。''

窦钱娘含泪说道:``我会-
一告诉他的,你放心吧!''段-璋道:``十载夫妻,累你操劳不少,请受一拜!''窦线娘道:``我得到这样的英雄夫婿,不管今后如何,都是一生无憾的了!你亦请受我一拜!''

交互一揖,段-璋立即离开,他怕看妻子的泪眼,头也不回,便即上路。忽听得铁摩勒高声叫道:``姑丈,且慢!''

段-璋道:``你有何事?''钱摩勒道:``我跟你到长安去。''段-璋道:``你跟去做什么?''铁摩勒道:``想到长安开开眼界啊!''段-璋笑道:``你知道我到长安干什么?这可不是好耍的啊!''铁摩勒道:``我知道你要到安禄山府中救那性史的义士去,姑姑刚在产后,三叔的伤毒未曾痊愈,他又要赶回去应付王家的人,都不能陪你。我却闲着无事,正好和你作个伴儿!''段-璋正色道:``这是赌性命的勾当,你知道么?我不能要你同行!''铁摩勒也正色道:``姑丈,你也未免太小看我了,就只准你自己做英雄好汉么?不管你要不要我,我已是跟定你的了!''段-璋大受感动,说道:``好,你有这样的志气,我就带你同行。到了长安,你可要听我的话。''铁摩勒道:``这个当然。''窦令符本来舍不得铁摩勒,但他也知道这少年的性子极是刚强,说一不二,而且他想到这次自己前来求助,如今段-
璋有事,自已不帮帮忙,让铁摩勒去,也正好卖个人情,便即说道:``这孩子的功夫还过得去,最少也可以做个通风报讯的人。你就带他去,让他磨练磨练也好。''

段-璋道:``三哥放心,我总不能让这孩子陪我送命。到了长安,我定有处置,要是我也万一能保住性命,救得史大哥回来的话,我会到幽州去看你们,顺便跟那精精儿见见高下!''他已在心中决定,要把自己的武功心法传给铁摩勒,并且决不让他同到安禄山的府中冒险。

铁摩勒何等聪明,早也听出了这两个人的意思,心中想道:``到了长安,我自有办法,你想把我撇开,未必能行。''他眼珠一转,打定主意,却不开言。

窦令符大为欢喜,虽然段-
璋此去凶多吉少,但究竟还未完全绝望,他如今已答应了愿在事情完后,便去对付精精儿,那么只要他无恙归来,窦五二家之争,窦家是稳操胜券的了。

窦线娘听得铁摩勒同去,心中稍宽,扬手说道:''段郎,你此去见机行事,若是急切之间,不能下手,便不可强为。要人帮忙的话,可以叫摩勒捎个信来。''段-璋道:``我理会得。娘子,你也要好生保重,记着我的话,好好扶养孩儿。''他怕看眼泪,不敢回头,带了铁摩勒,便直奔长安而去。

长空离段家不过六十里路,当天便到。正是:胸中侠气未曾消,抛家暂作长安客------

无忧书城扫校

\chapter{第 四 章 敢笑荆轲非好汉
好呼南八是男儿}\label{ux7b2c-ux56db-ux7ae0-ux6562ux7b11ux8346ux8f72ux975eux597dux6c49-ux597dux547cux5357ux516bux662fux7537ux513f}

三天之后,在长安明凤门旁边的一家酒楼上,来了两个生面客人。

明凤门是唐朝皇宫的第一道大门,这座酒楼的位置在皇宫旁边,它的顾客也都是些不寻常的人物。其中有早朝归来的文武官员,因为住处距离皇宫较远,来不及回家,便到这里吃中饭的。也有些官中的宿卫,散值(即下班)之后,和同伴到这儿喝酒的,所以别的酒家晚上热闹,而这家酒家却是上午的生意最好,而顾客之中,十之八九也都是相熟的客人。

但今天来的这两个客人。却是第一次到这豪华的酒肆,应中无人相识。这两个人,一人年约四十开外,器宇轩昂,披裘佩剑,似乎是个豪客,和他同来的则是个十七、八岁的少年,打扮得也像个贵家子弟,但双眸炯炯,精光闪烁,令人一看,就知他是个精明能干的少年,远非那些徒祖先遗荫的绣花枕头可比。

酒楼上的客人虽然觉得这两个生客有点特别,但这家酒楼在长安名气很大,不时有外地豪客慕名而来,或者到此求官谋事的,所以大家虽然觉得有点特别。却也不以为意。

这两个入正是段-璋与铁摩勒。原来段-
璋到了长安之后,即借宿在一处相熟的僧舍中,寺院的主持名唤怀仁,是个高僧,段-璋的祖父在世的时候,曾经是这个寺院的大施主,怀仁和段-璋亦是方外知交,所以段-
璋选择了这间寺院作为藏身之所。但段-
璋虽然有了栖身之地,却无法知悉安禄山在长安的府邸所在,后来他打听到有这么一家酒楼,心想安禄山既是常常进宫。这家酒楼的顾客,不乏和宫廷有关系的,因此便携了铁摩勒前来饮酒,希望能探听到一些消息。为了适合这家酒楼的顾客身份,他把所带的银子都换了华贵的衣裳。

这时是近午的时分,正是酒楼上的热闹辰光,靠窗的一张桌子,有几个官儿围着轰饮,其中却有一个中年书生,只是一袭布衣,箕踞案头,言盼自如,豪气迫人!那几个官儿,却反如众星供月似的,对他甚为恭敬!

段-璋心中一凛,想道:``这人相貌清奇,气概不凡,端的是平生罕见,不知究竟是什么人物?这几个官儿,也回非凡俗,想不到官场之下竟有这班人物!''

段-璋正在注视那布衣书生,忽见那书生的眼光也向着他射来,蓦地击桌赞道:''好剑,好剑!''段-璋吃了一惊,心道:``这书生倒是个识货之人,我的剑还未出鞘,他已经知道这是把宝剑了!''那书生向他招手道:``来,来,来!金樽有酒应同醉,结客何须间姓名!你过来饮酒,宝剑借我一观。''

饶是段-
璋走遍江湖,也从未碰过这样的事情:一个素不相识的人,突然向他借宝剑观赏,这在江湖上是大大犯忌之事,可是那书生豪气迫人,似乎有一股不可抗拒的力量,令段圭璋为之倾倒,顿时间也不禁豪情勃发,忘了所应有的顾虑,应声便站了起来,走过去道:``得蒙先生邀饮,何幸如之,只怕这把剑尚不是当名剑之名,有污先生焱目!''

段-璋这把剑乃是他祖父当年跟大将军李靖西征之时,李靖赐给他祖父的家传宝剑,剑一出鞘,光芒四射,那书生弹剑笑道:``虽非干将莫邪,也算是人间神品

了。你从那里来?''段-璋含糊应道:``我从幽州来。''那书生道:``路很远啊!路途险阻,想来你若不是仗着这把宝剑,也难以走到长安了。哈,哈,我拂拭此剑,倒想起少年游侠的往事来了。''旁边一个官儿笑道:``学士豪情,至今未减。''那书生大笑道:``现在是靠着皇帝混酒食,那还有什么豪情啊?''

蓦然站了起来,手弹宝剑,朗声吟道:``金樽清酒斗十千,玉盘珍羞值万钱。停杯投箸不能食,拔剑四顾心茫然!''

吟声未毕,忽地有一个蟒袍玉带的大官从酒客丛中挤出来,走到眼前问道:``这位先生,敢情是,敢情是------''

和书生同桌的一个年老官员叫道;``啊,你不是吴司马吗?李学士,这位是湖州司马吴筠吴大人,也是咱们同道中人。''

段-璋正在惊疑不定,不知这书生是何等人物。只听得那书生哈哈大笑,随口吟诗,答那湖州司马道:``青莲居士谪仙人,酒肆逃名三十春。湖州司马何须问?金粟如来是后身!''

吴筠笑道:``我猜得不错,原来果然是青莲居士。闻名久矣,何幸今日得遇!''

段-璋又惊又喜,原来他所遇的这位书生,正是他和史逸如素来倾慕的大诗人李白。

原来这位名闻天下的大诗人,不但诗做得好,而且他通晓剑术,他嗜酒耽诗,轻财狂侠,自号青蓬居士,别人见他有飘然出世之表,又称之为``李谪仙'',他少年之时,慕游侠豪风,也曾仗剑遥游四方,登峨眉,上太行,游云梦\ldots\ldots 看尽天下名山大川,尝遍天下美酒。到了长安之后,得秘书少临贺知章的推荐和赞扬,各方重视,渐渐名传帝阕,连皇帝也知道了他的大名。这位皇帝(唐玄宗)正是中国历代皇帝中少有的``风雅''人物,通晓音乐,也懂得欣赏诗词,他爱慕李白的才华,所以对他特别破例优待,召为翰林学士,并时常邀他人宫赏花、听乐、饮酒、赋诗,但李白不爱富贵,仍然以``市衣''自豪,谈笑做公卿,结交多侠士,所以他见段-
璋相貌不凡腰悬宝剑,便脱略形骸,不拘小节邀他同饮。

段-璋又是欢喜又是伤心,心中想道:``要是史大哥在此得与他所倾慕的青篷居士斗酒论情不知该有多高兴呢!''

李白哈哈大笑,将宝剑文还段-璋,说道:``我今日得赏宝剑,结所知,如此乐事,岂可不醉!''左手携了湖州司马吴筠,右手携了段-璋,拥入席中,立即开坏痛饮,一连饮了几大盅,忽听得``啪''的一声,他将鞋子除了下来,一甩头,又把帽摔到地上,根摇晃晃的说道:``啊,醉了,醉了,当真醉了!''积头跣足,伏在桌上,果然呼呼噜噜的打起鼾来。

同桌的一个官儿惊道:``青莲学士当真醉了。要是皇上召他做诗,这却如何是好。''另一位道:``未必有这样巧的吧?''刚才与吴筠打招呼的那个老者笑道:``你们也太小觑他了,李白斗酒诗百篇,喝醉了他的诗更做得好!''

那官儿道:``李白斗酒诗百篇,妙,妙,这一句本身就是一句好诗。''同桌的一个少年笑道:``你知道这句诗是谁做的?是老杜前几天写了一首《饮中八仙歌》送给青蓬学士,饮中八仙有贺老大人,还有这位张兄\ldots\ldots{}''那老者笑说道:``也有你呢,你忘记说自己了。''那少年笑道:``我是陪衬的。''歇了一歇,又笑道:``老社写青蓬学士那几句,显好象是看到他今日这个模样似的。''吴筠问道:``那几句怎么说?''那少年朗吟道:``孪白斗酒诗百篇,长安市上酒家眠;天子呼来不上船,自称臣是酒中仙!要是皇帝今日果然召他,那就越发对景了!''

段-璋这时才利那几个人互通名姓,原来那个老者便是为李白在长安揄扬最力的秘书少监贺知章,他本人也是个著名的诗人;那美少年名叫崔宗之,姓张的那个则是以草书名闻天下的张旭,其他几个也是长安城中颇有名气的人,段-璋也胡乱捏个假名说了。

湖州司马吴筠如笑道:``饮中八仙除了李学士、贺老大人、张兄、崔兄之外,不知还有几位。杜甫的那首诗你可记得全了么?''

崔宗之道:``难得今日有此盛会,张兄就烦你大笔一挥,我把这手饮中八仙歌念给你听,你写一副草书送给吴司马,就当是咱们和他见面的礼物如何?''吴筠大喜道``张兄乃是当今草圣,老杜号称诗圣,以草圣写诗咏诗仙的名诗,直乃相得益彰,这样的礼物,更是珍同拱壁!''

张旭道:``只怕醉了写不好,教司马见笑。''崔宗之笑道:``你写草书也象李学士写诗一样,越醉了越好,何必客气。''

贺知章叫店家取了纸笔来,就在旁边一张空桌上铺好了纸,张旭选了一枝大号的狼毫笔,蘸满了墨,崔宗之念道:

'知章骑马似乘船,眼花落井水底眠。汝阳二斗始朝天,路逢曲车口流涎,恨不移封向酒泉。左相日兴费万钱,饮如长鲸吸百川,街杯乐圣称避贤。宗之潇洒美少年,举头白眼望青天,皎如玉树临风前。苏晋长斋绣伟前,醉中往往受逃禅。李白斗酒诗百篇,长安市上酒家眠,天子呼来不上船,自称臣是酒中仙。张旭三杯草圣传,脱帽露顶王公前,挥毫落纸如云烟。焦遂五斗方卓然,商谈雄辨惊四筵。

崔宗之念完大家便哄笑一场,贺知章道:``真是把咱们的醉态写得淋漓尽致!''张旭大笔挥舞,墨汁飞溅,写完了这首诗,他的面上,东黑一块,西黑一块,连胡须上也溅满了墨,旁边的人,衣裳上也是点点斑斑的墨迹,张旭哈哈大奖,挥笔笑道;``你们是醉态可掬,我却是丑态毕露了!''

贺知章道:``可借你不早些来长安,听说湖州乌程酒极佳,你就是为了乌程酒才去就湖州司马之职的,要是你在长安,老杜就应该写饮中八仙了。嗯,我忘了问你,你不在湖州任内,却上京来干什么?''

吴筠道:我是奉召进京述职的,来了五天,却尚未蒙皇上召见。''贺知章面有诧色,道:``皇上极少顾问政事,却怎的会突然召你进京述职?''沉吟半晌,忽地说道:``你可见过杨国忠没有?''吴筠道:``没有。''贺知章道:``你赶快各办一份名贵的礼物送他。''崔宗之笑道:``若是急切之间备办不来礼物,送金子更妙。我们这位宝贝相爷一见了黄澄澄的金子,就容易说话了。''

吴筠大笑道:``我为官数载,两袖清风,那来的金子?再说,我若有钱,自己不买酒吃么?为什么要送礼给杨国忠?''

贺知章道:``司马有所不知,自杨国忠专权之后,卖官晋爵,无所不为,州郡长官,若不是他的人,便陆续撤换。依我看来,召你入京述职,只怕是他的主意。他正在等着你送礼呢,谁知你却这样不懂人情世故。''笑了一笑,继续道:``要是你宦囊不便,咱们几位酒友给你凑一些如何?他大约因为你政声颇好!所以迟迟不敢换你,只是召你述职,想等你找上门来。你稍为给他一点好处,卖他一点面子,大约也就可以无事了。''

吴筠愤然说道:``小弟宁可丢了这项乌纱,也决不巴结权贵,送礼之事,再也休提。''

贺知章道:``吴兄廉洁自持,当然是好,可是你就不想想,要是湖州司马,换了一个贪鄙之人,岂不是苦了湖州百姓?我们不是劝你巴给扬国忠,而是想为湖州留一个好官。唉,现在天下的好官太少了,能留得一个就是一个。''

崔宗之道:``要是吴兄不肯送礼,还有一法,可以找李仆射给你讲讲情。他也是咱们酒友之一,杜甫`饮中八仙歌'所说的那位`左相日兴费万钱,饮如长鲸吸百川,杨杯乐圣避称贤。'就是说他。李仆射虽然豪奢,人却还是正直的。''

吴筠叹口气道:``贺老大人劝我以湖州百姓为重。此心可感,只是如此官场,实在已令我心灰意冷,再说,纵使花钱打点,我却不是个同流合污之人,这个官又能做到几时?诸兄盛情心领,这项乌纱,能不能保,听天由命吧。''

贺知章等还想再劝,忽听得楼梯声响,跑堂的弯腰曲背,道:``伺候令狐大人,令狐都尉,今天你老来得迟了。''

吴筠问道:``什么官儿,这样威风。''贺知章笑道:``大约是羽林军(即彻林军)的军官专职护卫圣上的,你别瞧他们的品级不及咱们,可比咱们阔气得多呢。这班侍卫老爷多是这家酒楼的常客,堂倌当然要巴结他们。''一个官儿道:``官中的都尉来了。不知是不是皇上要召李学士入宫?''

说话之间,只见三个军官走上楼来,当前的一个穿着羽林军的服饰。十分神气,后面两个军官,身披驼绒军装,腰围金带,脚踏蛮靴(一种长统的马靴),看这装束,便知是边军的高级将领。

那羽林军军官道:``我给你们带来两位贵客,这位是田将军,这位是薛将军,快给我们找一副雅座。''堂倌连连的应诺。还忙去收拾一副临窗的座头。

跟在令孤都尉后面那个身体有点发胖的军官,用眼光一瞥,见李白伏在桌上呼呼噜噜的打鼾,鞋子帽子都给扔在一边,远远就闻得到他那股酒气,还有一个张旭,须子上墨汁淋漓,兀自在那里手舞足蹈,要和别人斗酒,那军官皱起眉头,道:``人家都说这是长安最有名气的一家酒楼,却怎么容得这些穷酸在这里撒野。''令狐都尉不待他的话说完,急忙拉着了他,在他耳边低声说道:``打瞌睡的那个人正是皇上所宠爱的李青篷车学士。''那个军官吓了一跳,连忙禁声,脸色尴尬之极,偷偷的朝李白张旭那两张桌子望去,见那些人闹酒的闹酒,谈天的谈天,似乎并没有听到他的话,这才放心。

这时段-璋已回到了他原来的座头。铁摩勒低声说道:``这两人就是安禄山手下的田承嗣和薛嵩。''段-璋道:``沉住了气,不可闹出来。''

酒楼上有三张桌子,坐着的都是宫中的侍卫和羽林军军官,见了令狐都尉,纷纷起来招呼,那令狐都尉哈哈关道:``我给你们介绍两位好朋友,平卢军的田将军和薛将军,他们两位是安节度使的左右手。''在各路节度使中安禄山兵权最大,又是杨贵妃的干儿子,那些恃卫们和军官们对田薛二人纷纷趋奉。

段-璋听他们的言语,知道那个令狐都尉名叫今狐达,在这群军官中似乎职位最高,那些人对他都很恭敬。他们则是护送安禄山人宫的,安禄山给杨贵妃留下了,要他们到晚上才去接他。

段-璋心想:``这酒楼正对着明凤门,我今晚再来,在此守候,等这两家伙接安禄山回去之时,我暗地里跟踪他们。''铁摩勒那日在马蹄下救人,田薛二人虽然在安禄山的左右,但铁摩勒那日是个乡下少年,现在却打扮成硅家子弟的模样,田薛二人那里认得出来?何况他们的眼光都被李白的醉态吸引住了,更没有注意他们。

不过段-璋却不敢大意,生怕给他们窥出行藏,已然得到了安禄山的消息,便想离开酒楼。

正待叫堂倌过来结帐,酒楼上又来了一个客人,一进来就大声问道:``李学士可是在此喝酒么?''

这人也是个武官装束,但与田薛二人却大大不同,他着得是一身粗布军装,严冬时分,仍然穿着草鞋,但他腰挂长刀,刀鞘却是名贵的犀牛角做的,样式古拙,刀鞘上还缠有铁丝,要不是他挂着这把名贵的宝刀,那就完全象一个穷大兵了。

段-璋抬起头来,打量了这入一眼,不觉暗暗吃惊,这军官约有三十岁左右,双目炯炯有神,虬须加戟,满面风尘之极,却掩盖不住他的侠气雄风,段-璋蓦然想起了一个人来,但却不敢断定是不是他。

令狐达喝道:``你这厮是什么人?李学士是你随便见得的么?''

那军官冷笑道;``我找李学士关你什么?要你出来多事?''

薛嵩道:``你大呼小叫好设规矩,李学士正在好睡,你胆敢吵醒他么?看你这粗野的样子,李学士就不会交你这样的朋友!''薛嵩刚才认不得李白,出言无状,甚感难为情,正好趁这个机会,一来为令狐达助威,二来讨好和李白同来饮酒的那班官儿,心中想道;``这回大约不至于看错人了吧,看来这厮最多不过是个边军的小军官,谅他怎能识得了李白。''

薛嵩拦着了去路,那军官大怒道:``你狗眼看人!''平掌一推,薛嵩冷笑道:``你耍打架么?''立即施展擒拿手法来扣他的脉门,想把他一下拿着,反扭过来,在众军官面前,博个哈哈一笑。那知他没有抓着人家,却反而给那个军官一掌推开,跄跄踉踉的几乎跌倒!

令狐达大吃一惊,要知薛嵩是个有名的青州剑客,以剑术、暗器与擒拿手称为三绝,而今他竟然一交手就吃了对方的亏,而且还令令狐达也看不出那个军官是怎样闪开薛嵩的擒拿手的。

薛嵩大怒,便想拔出剑来,贺知章上前调解道:``李学士结交遍天下,薛将军敬爱李学士之情可感,这位\ldots\ldots{}''那军官道:``我姓南,东南西北的南。''贺知章继道:``这位南兄既然是李学士的相知,对薛将军的阻拦也不应见怪,李学士当真是多喝了几杯,现在已睡着了。''贺知章这番话说得婉转之极,薛嵩又知道他是个大官,只好忍住了气,不敢发作。那性南的军官游目四方,问道:``那位伏在桌上打瞌睡的人就是李学士吗?''

贺知章诧道:``不错,就是李学士。''薛嵩已冷笑道:``闹了半天,原来你是并不认识李学士的呀!''

那姓南的道:``我几时说过了我认识他,我不想谬托知己。''

贺知章道:``然则阁下找他何事?''那性南的道:``我不敢谬托知己,可是另有一位是李学士知己的人,托我稍一封信给他。''

贺知意道:``是那一位?''心想:``李白的知己朋友,说出来大约我即算不认识也总会听过名字。''那姓南的道:``是一位姓郭的朋友,这封信我得亲自交给学士,不便转托他人。''着情形是不愿说出这姓郭的名字。

贺知章心想道:``我可未曾听李白提过有姓郭的好朋友啊。''但他老于世故,别人不愿说,他也不便再问,当下说道:``李学士这觉不知要睡多少时候,可要我唤醒他么?''

那姓南的军官道:``不必,不必。我也就在这里喝酒等他醒来好了!''高声叫道:``打五斤好酒,切三斤牛肉来!''

薛嵩歪着眼睛,洋洋得意的说道:``如何,我这双眼着人还看得准吧?''言下之竟,即是说:``你看,我说李学士不会有这样的朋友,没有错吧?''那姓南的大盅大盅的喝酒,不理会他。薛诡又笑道:``这是长安最出名的一家酒楼,哈哈,却想不到有人把他当作路边酒肆了。''这是嘲笑那姓南的只知道叫路边酒肆所常卖的东西,这酒楼上有多少美味的菜式他不叫,却只叫白酒和切牛肉。

那姓南的把酒盅重重一顿,大声说道:``我吃什么东西,也要你管么?''

那酒盅是青铜做的,被他重重一顿,只听得``当''的一声,酒盅陷入桌内,与桌面相平,四座皆惊,薛嵩亦自有点气馁,但又不愿当众失了面子,退了一步,说道:``你真发横。这里不是打架的处所,有本事的,你敢与我约个地方比剑么?''口气已经软了许多。那姓南的军官冷笑道:``随你划出道儿,我一准奉陪便是。待我见过李学士之后,立刻便可赴约。''

段-璋见了这人的身手,心里想道:``这一定是他了,想不到在此地相遇。''但酒楼上人多口杂,他虽然认出了这个人,却也只得暂时忍耐,不敢立即去招呼。

田承嗣与薛嵩同来,薛嵩与那性南的发生争斗,田承嗣却躲在一边,禁若寒蝉,段-璋暗里留意,只见他的面色铁青,眼神注定那个娃南的军官,屡次手按刀柄,却始终不敢站出来,段-璋暗暗奇怪,心道:``田承嗣和这个姓南的一定有什么过节,看来只怕好戏在后头。''

薛嵩心道:``你手上功夫虽然了得。比剑我未必会输给你。''正要与那姓南的订约,贺知章等人也正要出来调解,就在这乱哄哄之际,忽听得``当、当、当''三下锣声,有人高声报道:``圣旨到!''

酒楼上肃静无哗声,有品级的官儿都站了起来,避过两边,酒店的主人急忙上前迎接道;``迎中度使大人,不知圣旨宣召那位大人。''这样的事情在这酒楼上已发生过几次,主人也知道定然是宣召李白,但仍然不能不有此一问。

唐朝的太监奉目出差的尊称``中使'',但这次率领几个小太监出来找寻李白的人,本身却不是个太监,而是二个乐工,名叫李龟年,虽是乐工,但甚得皇上宠爱,授为``拿乐御奉'',身份不比寻常,贺知章等人都认得他。

李龟年上前高声说道:``奉圣旨立宣李学士至沉香亭见驾。''他背后一个小太监,手捧冠袍、玉带和象笏,便来找寻李白。

李龟年笑道:``李学士果然又喝醉了。皇上立即便要见他,这却如何是好?贺大人也在此,帮忙我一同唤醒了他吧。''

两人正在扶起李白,李白忽地双手一推,酒气喷人,哺喃念道:``我醉欲眠君且去。''头也不抬,又倒下去睡了。贸知章和李龟年给他一推,险险跌倒。李龟年苦笑道;``这次比上次醉得更厉害了,怎么办呢?''

小太监道:``咱们抬地走吧。''李龟年道:``总得让他换过朝衣。''叫道:``店家,打一盆水来。''

贺知章官居秘书少监,也是侍从皇帝的近臣,与李龟年又稔熟,李龟年已宣读了圣旨,彼此不必再拘什么礼节,贺知章问道:``皇上这次急於宣召李学士,为了何事?''

李龟年道:``今年扬州贡来了许多种牡丹,都植于兴庆池东,沉香亭下。今日牡丹盛开,皇上命内侍设宴于亭中,同杨贵妃赏玩,命我引梨园中的一十六色子弟,各执乐器,前来承应。奏了几曲,不合上意。皇上便叫我停住,说道:``今日对妃子、赏名花,岂可复用旧乐?你即将朕所乘的玉花驰马,速往宜召李白学士前来,作一番新词庆赏!''你瞧,皇上的御马都牵来了,就等着李学士去呢,急不急煞人?''

说话之间,店主人已亲自把一盆冷水捧来,李龟年要了一条毛巾,也顾不得天寨地冻,亲自把手巾没了冷水,扭了两下,使往李白的额角敷去,又叫店家取来了四面屏风,围着李白,笑道:``幸而我熟知学土的脾气,预先到翰林院取了他的冠袍、玉带、家笏来,不出我之所料,他果然是一袭布衣,在此与诸公饮酒。''

李白等人被屏风遮住,段-璋瞧不见内里情景,过了一会,只听得李白的声音说道:``真煞风景,我还未喝够呢,做什么诗?''李龟年唧唧咕咕,似乎是在耳边低声求恳,过了片刻。又听得李白笑道:``吓,扬州的名种牡丹都盛开了,大红、深紫、淡黄、淡红、通白各色各种都全,皇上又备了凉州美酒,等我去喝,哈,这倒对了我的口味了,瞧在扬州牡丹的份上,我就去一趟吧。''楼板冬冬作响,原来当他说到各种牡丹、凉州美酒之时,禁不住手舞足蹈。随着又听得悉悉索索的声音,敢请他已是脱下布泡,换上朝衣。

再过片刻,只见李白推开屏风,走了出来兀自脚步跟跄,朦胧醉眼,酒气熏人,几个太监前呼后拥,左右扶持,走过那姓南的军官座前,李白忽然停了下来,道:``好一位壮士,咦,你、你、你\ldots\ldots{}''那姓南的道;``我给令公带了一封信来,正要见你。''话未说完,太监们早上前将他拉了开,喝道:``什么人,赶快滚开!''

李白怒道:``岂有此理,你们要赶走我的好朋友么?''双臂横伸,扶着他的那两个小太监,``扑通''一声,跌了个四脚朝天。

太监们大惊失色,旁边一个官儿好生诧异,小声问他的同伴道:``咦,刚才这人还不认得李学士呢,怎的却又忽然是他的好朋友了?''

李白推开了太监,东倒西歪。摇摇晃晃的踏上几步,指着那个姓南的军官哈哈笑道:``你不认得我?我却认得你!你。你,你,你一定是南八兄,敢知荆轲胆如鼠,好呼南八是男儿!哈,哈,哈,见了南八,谁还理会什么贵妃娘娘,来,来,来,咱们再来喝酒!''

李龟年早就上前拉着南八,对他一揖,悄声说道:``皇上等看见李学士,你帮个忙!''

李白一步跨得太阔,身躯倾倒,扶着桌子叫道:``南八南八,你怎么不来喝酒,喂,喂!你刚才说什么?有什么阔气的老公公托你带东西给我呀?哈,哈,哈,你南八怎会是给人送礼的人呀?笑话,笑话。快来说清楚了!''李白尚未醉醒,又一心放在南八身上。竟未听清楚他说些什么,将他说的``郭令公'',当成了什么阔气的老公公了。

那性南的军官大笑道:``学士果然是我辈中人,但现在楼下就有御马等着你骑进宫去,你纵然陪我吃酒,我也喝得不痛快,不如待你今晚无事,我再去与你吃个通宵!''

李白道:``好,你说得也对!待我见皇帝老儿再去见见你,的确可以吃得舒服一些!''

贸知章忙道:``李学士住在我的家中,你问城西贺家就知道了。''那姓南的道:``你老先生是贺少监,我知道。''他知道贺知章的意思,是要他让李白快走,他一想托他的说话,不是三言两语可以说得清楚,而李白又在醉中,在这样的情形下,那封信他也不方便在这个时候交出来了。

李龟年与那班大监急忙拥着李白下楼,李白那班酒友也都跟着散了。那姓南的军官摇了摇头,叹口气道:``玉门已自燃烽火,宫门沉沉醉歌舞\ldots\ldots{}''蓦地拍案叫道:``可惜了李学士!''仰着脖子,将酒盅余酒,一倾而尽,掷了一锭银子在桌子上面,便要离开。

令狐达与薛嵩忽然走了过来,令狐达陪笑说道:``南兄且慢!''

那姓南的军官剑眉一坚,朗声说道:``什么地方。是不是现在就去?除了这个姓薛的之外,你是不是也想要凑上一份?''

令狐达笑道:``南人兄,不是约你比剑。''那姓南的圆睁双眼说道:``不是约我比剑,你留我作什么?''薛嵩上来抱拳说道:``方才不知吾兄,多有冒犯,还望南兄勿怪。''

南八肚里暗暗好笑,心中想道:``想是这厮见了李白如何待我的。故此马上便变了一副脸孔!''他是个豪爽的人,虽然看不起薛嵩,但别人既来陪罪,他便也哈哈笑道:``小小一点言语角逆(冲突之意)何足介怀?薛将军既是不必要我比剑,那就请容我先走一步吧。''

令狐达道;``不打不成相识,南八兄多坐片刻何妨?''南八道``不敢高攀!''令狐达笑道:``南八兄这样说,就是还有见怪之意了。''薛嵩也道:``彼此都是武林同道,令狐都尉又是最喜爱结交朋友的,南八兄何必这样吝于赐教。''

南八心道:``这两个人的武功还过得去,却偏生这么讨厌!''只得再坐下来,谈谈说道;``两位有何指教。''

令狐达笑道:``正是有件事要请问南兄,方才南兄所提到的郭令公可是九原郡守郭子仪么?''

郭子仪后来功勋盖世,受封为汾阳王,但当时只是一个郡守,知道他的名字的人还不多。段-璋在旁边听了,也觉得有点诧异,心想:``令狐达是御林军都尉,薛嵩是安禄山手下的心爱将领。他们敬畏李学士还说得过去,因为李学士到底是皇上看重的人。但却何以对一个郡守却也象是耸然动容,这郭子仪不知是什么人物?''

南八踌躇片刻,答道:``不惜,托我捎信给李学士的就是郭郡守。两位可是认得他的么?''

原来李白与郭子仪的结识甚不寻常,有一日他在并州地界游山玩水,忽然碰着一伙军卒,执戈持棍,押着一辆囚车,车中的囚犯仪容伟岸,李白动了好奇之心,上前一问,原来此人便是郭子仪,当时是陕西节度使哥舒翰麾下的偏将,因奉军令,查视余下的兵粮,却被手下人失火把粮米烧了,罪及其主,法当处斩,当时哥舒翰出巡已在此州地界,因此军政司把他解赴军前正法。

郭子仪在囚车中诉说原由,声如洪钟,李白回马,傍着囚车而行,一头走,一头慢慢的试问他些军机、武略、剑术、兵书,郭子仪对答如流,就象碰着个知己一般。越谈越投机,越谈越高兴,神采飞扬,那里象个即将越死的囚徒,李白越听越奇,心中想道:``我平生所结交的英雄豪杰,不在少数,若说到可以足当国士之称的,似乎还只有此人!''

李白直跟着囚车走到军前,亲自过去见陇西节度使哥舒翰,申述来意,求他宽释郭子仪之罪,哥舒翰素幕李白大名,趁这机会,卖了他一个人情,许郭子仪在军前备用,将功赎罪。

别后数年,郭子仪屡建军功,渐露头角,做到了九原郡的太守,李白在长安听到了故人消息,甚为高兴。但他不愿意夸耀自己的恩德,这件事情,从未向人提过,因此即算是贸知章这样亲密的朋友,也不知道他和郭子仪的这段交情。

郭子仪也听到了李白在长安的稍息,知道他虽得皇帝宠爱,却也不过是等于皇帝的请客人一般,不会重用。而且权臣当国,心想以李白的性格,大约也不会在这样的官场混得下去。郭子仪思念及此,遂请他的一位朋友。替他带了信入京,找寻李白,想请李白到他的任所去。

这位朋友。便是李白称他为``南八兄''的这个军官,其时正在郭子仪幕下,助郭子仪守边。这人排行第八。真姓名叫做南霁云,是燕赵间一位著名的游侠,江湖上在这二十年间,先后有两位著名的游侠,十年前是段-璋,自段-
璋隐居之后,最负盛名的就是他了。他在九原,曾经以单骑击退寇边掳掠的三百羌人铁骑,所以当时民间有一句赞扬他的话道:``要如南八,方是男儿!''

此际,令狐达一再向南霁云问及郭子仪,南霁云只道他是认识郭子仪的,也就直认不讳,说出托他带信给李白的便是郭子仪。

那料令狐达问请楚之后,却皮关肉不笑的说道:``这封信李学士既然尚未取去,就请借给在下一观如何?''

此信虽然非关机密,但这要求却未免不近人情,南霁云怫然不悦,说道:``令狐大人说笑话了,别人的信,怎么好借去看?''令狐达冷冷一笑,又问道;``南八兄,你刚才说`只可惜了李学士',这句话又是什么意思?''

南霁云怒道:``你凭什么来审问我?''令狐达道:``李学士蒙皇上圣恩,派中使御马来迎,荣宠无比,你却说他可惜,恕我愚昧,实是不解其意,务请你说明白。''南霁云给他问往,解释不上来,索性放下了脸说道;``我没有功夫和你说话!''

薛嵩冷笑道:``有功夫比剑,却没功夫说话么?''令狐达做好做坏,拦在当中说道:``你将那封信交给我,咱们另找个地方说话,我仍然把你当作朋友看待。''

南霁云``哼'了一声:``我南八岂是受人威胁的,不交出来又怎么样?''

令狐达面色一变,蓦地喝道;``你替外臣奔走,勾结近臣,又心怀不满,诽谤朝廷,两罪俱发,还想逃么?''

段-璋一直冷眼旁观,刚才见令狐达过来向南霁云打拱作揖的赔罪,还只道他是个势利小人,为了李学士的缘故,故此对南霁云巴结,不料顷刻之间,他却突然翻脸。与南霁云动起手来,饶是段-
璋阅历甚丰,亦觉大大出乎意料之外。

说时迟,那时决,只见令狐达已取出了一对护手钩,一招``倒卷珠帘'',左钩横胸,右钩斜指,就向南霁云胸前划去!南霁云却未曾拔出刀来,只听得``嗤''的一声,南霁云的衣裳被他的护手约钩去了一大片,紧接着``啪''的一响,令狐达却着了一记耳光。

南霁云身手矫捷,退步、闪身、避钩、进掌、拔刀,一气呵成,左掌拍出,立即反手一刀,``当''的一声,又和薛嵩的长剑迎个正着!

火星蓬飞,薛嵩的青钢剑损了一个缺口,薛嵩号称青州剑客,剑法上实有非凡造诣,刀剑一交,立即知道对方是把宝刀,倏的变招,长剑一圈,一招``龙门鼓浪'',连环三式连袭南霁云上中下三处要害,剑光闪闪,当真就好似浪涌波翻,飞珠溅玉,耀眼生颖!令狐达的武功比薛嵩尚胜一筹,他自出道似来,还是第一次吃人一照面便打了一记耳光,怒火中烧、也立即使出杀人绝招,双钩一横一直,一招``指天划地'',前钩指到了南霁云的背后,后钩跟着刺向南霁云腿弯的关节,南霁云要是站在原地不动,背心势必给他戳个透明的窟窿,要是向前奔出,前心势必受薛嵩的一剑,要是向上跃起,那就等于凄上去给令狐达的利钧穿过腿弯了!

好个南霁云,只见他在剑光钩影之中,腾地一个倒蹬,就象背后长着眼睛一般,这一脚向后踢出,恰好踢中了令狐达的虎口,令狐达指向他腿弯的那柄护手钩,还未曾沾着他的裤管,就给他踢得脱手飞去,与此同时,他横刀一立,向前斜削出去,这一招是攻敌之所必救,薛嵩那一剑若是剑势不改,仍始向前削出的话,或者可能令他受伤,但薛嵩的一条臂膊,却先要保不住了,幸而薛嵩的招式未曾使全,忙不迭的撒剑回身,只听得南霁云哈哈大笑,已从令狐达身旁掠过!

铁摩勒看得出了神,不自觉的拍案叫道:``好功夫!''要知南霁云这两式刀脚并用,刀向前劈,脚却向后踢去,方向恰恰相反,但他却使用妙到极巅,实是非常难练的一种功夫,非但要一心二用,而且要拿捏时候,不差毫厘,铁摩勒最近曾跟窦令侃练过这种前弓后箭,解拆背腹受敌的招数,但还未曾练得成功,放此见了南霁云的前刀后腿使得如此精妙,便不自禁叫出声来。

南霁云听得喊声。如他这边望去,心中一凛:``那不是段大哥吗?''脚步自然而然的缓了一缓,就在此时,田承嗣猛地大喝一声,掀翻了一张桌子,阻着了南霁云的去路!

南霁云双眼一睁,喝道:``原来是你这个强盗,居然也做起军官来了!''田承嗣怒道:``胡说八道,我身为平卢将军,你竟敢诋毁于我!''南霁云仰天长啸,愤然说道:``官贼不分,豪强恃势,国家焉能不乱!''长啸声中,左掌拍出,把田承嗣震退两步,反手一刀,又把薛嵩的长剑荡开,令狐达喝道:``反了,反了!这厮一再诽谤朝廷,诋毁大将,乱臣贼子,人人得而诛之,乱刀把他砍了。''与令狐达交情好的几个军官,登时围了上来。

原来田承嗣在投靠安禄山之前,是个独脚大盗,有一次在并州道上,抢劫一伙客商,被南霁云遇见,仗义救人,将他砍了一刀,从此结怨。所以田承嗣刚才见南霁云过来,一时之间,不敢作声,就是为了怕地揭穿底细之故。

但薛嵩却不能不感到诧异,他在第一次和南霁云吵闹之后,太监来迎接李白之时,回到席上,就问田承嗣何以不出来帮他?田承嗣可以瞒得别人,却不敢瞒骗薛嵩和令狐达,而且他们两人也是黑道出身,便把实情讲了。令狐达听了,登时计上心头。

令狐达将南霁云罗织人罪,倒并不只是为了要替田承嗣报仇,其中实有更复杂的原因。

郭子仪当时虽然仅是官居太守,但因他善于用兵,又不肯依附安禄山,早已为安禄山所忌;而李白在朝廷里又早已为杨国忠所忌,只因李白名声太大,皇帝又正在看重他,杨国忠才无奈何罢了。另一个方面,安禄山虽然巴结上了杨贵妃,但与杨国忠利害冲突,又彼此在皇帝跟前争宠,勾心斗角,这几方面错综复杂的关系,外人不知,令狐达却是知道的。

所以当令狐达得知南霁云替郭子仪带信给李白之后,使起了一个歹毒的主意,心里想道:``不管他信里说些什么,我得了之后,便可拿来献给杨国忠,由他找了个善于书法的人,模仿郭子仪的笔迹。诬陷他们谋反,皇上或者是不会相信;但最少也可以诬陷他们内外勾结,植党营私,这也是招皇上之忌的。如此一来李白纵然不被斥退,宠信亦衰。而郭子仪则必然是被扳倒的了,我这样做,既可巴结杨国忠,又可讨好安禄山,岂非一举两得!''他本来还想拉拢南霁云,威胁利诱,双管齐下,迫他做个人证的,无奈南霁云,毫不卖他的帐,这才动起手来。

酒楼上有十几个羽林军官和大内宿卫,都是和会狐达熟识的。令狐达这么一嚷,那些人纷纷上来,将南霁云围在当中。令狐达心道:``这厮对朝廷口吐怨言,替郭于仅带信之事,也经他亲口说了出来,这一干人都可以为我作证,我就是将他杀了,也不至于有罪,而且仍然可以按照原定计划而行。''

令狐达一声令下,吩咐将南霁云乱刀砍死,登时酒楼上乱成一片,只听得叮叮当当的刀剑相交之声,乒乒乓乓的杯盆碎裂之声,轰轰隆隆的桌椅翻倒之声,怕事的酒客们尽都逃了,酒楼的人叫苦不迭,劝又劝不得,只都躲到内里去了。

南霁云大怒,一柄宝刀指东打西,指南打北,一抬脚将一张圆桌踢飞,有三个军官正朝着他冲了来,给这张圆桌一压,登对头破血流,好半天爬不起来。

但是好汉敌不过人多,令狐达的双钩、薛嵩的长剑,田承嗣的金刚掌尤其厉害,包围的圈子越缩越小,甫霁云展开全身解数,兀是冲不出去。

激战中一个大内侍卫打出了三枚透骨钉,南霁云侧身一闪,猛觉得肩头一紧,有如着了一道铁箍。

原来田承嗣就在他的侧边,他这么一闪,恰好闪到了田承嗣面前,被田承嗣一把拿着。薛嵩大喜,立即跨上一步,出剑刺他膝盖的环跳穴,令狐达双钩卷地,钩他两脚脚跟,另外还有两个军官持刀奔来,砍他两条臂膊,眼看南霁云就要被乱刀斫死。

薛嵩剑招方出,忽觉背后有金刃劈风之声,薛嵩是个使剑的行家,大吃一惊,不暇攻敌,先行自救,反手一剑,只听得''当''的一声,却是另外一个军官的长刀给来人的宝剑削断,而薛嵩却刺了个空。

薛嵩睁眼看时,却原来这个人便是刚才和李白喝酒的那个人。也即是薛嵩闻名已久,却未曾见过面的段-璋。

段-璋出剑如电,他杀入重围,长剑向薛嵩背心的``志堂穴''虚指一指,他知道薛嵩是个行家,他这一招攻敌之所必救,薛嵩必定要回剑抵御,南霁云便可以少对付一个强改了,所以他这一招不必用实,从容削了另外一个军官向他劈来的钢刀之后,这才哈哈笑道:``薛嵩,你的剑法还要再练十年!''------

51shucheng.net扫校

\chapter{第 五 章 奇闻贵妃洗儿钱
喜结英豪磨剑客}\label{ux7b2c-ux4e94-ux7ae0-ux5947ux95fbux8d35ux5983ux6d17ux513fux94b1-ux559cux7ed3ux82f1ux8c6aux78e8ux5251ux5ba2}

令狐达那里将这个少年人放在眼内。左钩住下一沉,右钩往上一带,左右盘旋,双钩霍霍,大叫一声``着!''铁摩勒的刃口已给他左手护手钩的月牙钩着,正要将他的单刀夺出手去,铁摩勒机灵之极。脚尖一挑,将地上另一只破碗踢起,破碗虽然不是什么厉害的暗器,但要是给打中了脸孔,轻则破相,重则眼睛亦可能受到伤害,令狐达迫得侧身闪开,那只破碗从他的旁边飞过来,打中了另外一个卫士的头颅,``当郎''一声,破片飞开,那个卫士固然头颅破裂,另外两个卫士也受了伤。

令狐达钩着铁摩勒单刀的是左手那柄护手钩,他这左手,刚才给南霁云踢中虎口,虽无大碍,气力却使不出来,最多只及平时的一半,铁摩勒趁他闪避之时,身子侧过一边,重心不稳,立即用力将单刀往下一沉,``咔嚓''一声,护手构上的那两齿月牙反而折了。

令狐达大怒了,右手的护手钩跟着进招,铁摩勒叫了声:``好厉害!''单刀一闪,轻灵翔动,竟然用单力使出了一招``八仙剑''的招数,令狐达不提防地突联间有此怪招。仍然当作单刀的招数来抵御,待至省觉,已来不及。``哧''一声,原来刀尖划过,在他的小臂上划开了一道三寸来长的口子。原来这几天铁库勒和段-
璋在一起,段至璋将好些精妙的剑法传了给他,还答应将来给他找一柄好剑,叫他改换兵器的。现在他碰到强敌,遂迫不及待的将剑法化到刀法上来,成了一招``怪招'',出乎意外的将令狐达刺伤了。

令狐达气得七窍生烟,他伤得不重,双钩一立,杀机随起,要把铁摩勒毙于钩下,可是薛嵩这时已被段-璋迫得连连后退,令狐达再不去帮他,薛嵩就要先毙在段-
璋的剑下,令狐达只好舍了钱摩勒,与薛嵩并力抵挡段-璋,段-璋长剑一展,把令狐达、薛嵩与其他两个大内高手,都笼罩在剑光之内。

再说田承嗣用``虎爪擒拿手''一把抓着了南霁云,正自心中大喜,方要用力将他的琵琶骨捏碎,猛觉得南霁云的肩头竟似化成了一块铁板一般,抓不进去,田承嗣大吃一惊,说时迟,那时快,南霁云陡地大喝一声,身躯一俯。用``捧角''中的``背投''绝技,将田承嗣那水牛般的身躯抛了起来,``冬''的一声巨响,楼板震裂一洞,田承嗣从洞中坠到楼下!

这时那两个手舞长刀的军官方奔到他的眼前,南霁云大喝一声,反手一刀,将第一个军官的手臂斩断后,刀背一磕,又把第二个军官拍晕,众军官惊呼道:``恶贼杀伤人啦!''除了令狐达、薛嵩和令狐达两个最要好的大内卫士之外,其他的人那里还敢上前?

段-璋叫道:``摩勒,不要找人厮杀了,走吧!''宝剑挽了一个剑花,向令狐达一指,``唰''的一声,点中了他的手腕,令狐达的护手钩第二次脱手,南霁云加上一刀,薛嵩的青钢剑也给他震得脱手飞去,南段两人奔到了临街窗口。

忽听铁摩勒大叫一声,只见一个以前未露过面的军官站在梯口,面目漆黑,身材高大,活家一个门神,铁摩勒未知他的厉害,兜头给他一刀,那军官笑道:``小娃娃,刀法不错呀!''倏地双臂一伸,左手抢过了铁摩勒的刀,右手就把铁摩勒举了起来!

段-璋这一惊非同小可,连忙转过身来,去救铁摩勒,那黑面军官将铁摩勒举了起来,盘空一舞,笑道:``你这小子胆量不小啊,饶了你吧!''忽地振臂一抛,将铁摩勒从窗口抛下街心!

话声未了,段-璋的长剑已指到了他的面前,那军官好生了得,不退反进,一招``探囊取物'',五指如钩,向段-
璋的``曲池穴''抓来,要是给他抓着,不论武功多强,这条臂膊登时就要麻木不灵,成为他的俘虏。段-璋见多识广,一见他的招数,便知是个劲敌,可是这时他已气得红了眼睛。不顾厉害,竟然拼着两败俱伤,剑锋一转,恶狠狠的削他膝盖,厉声喝道:``还我小友的命来!''

那黑面军官还真料不到他有这样拼命的打法,这一抓抓实,虽然能擒得段-璋,自己亦难免残废,敢清他还不愿真个和段-
璋排命,当下一闪闪开,笑道:``谁杀了那个小娃娃?你也不看个明白!''

就在这时,只听得铁摩勒的声音在下面叫道;``姑夫,你们还在打架吗?好好的给我揍那个黑汉子一顿!''

那黑面军官哈哈笑道:``你这娃娃不领我的人情也还罢了,怎么还要骂我!''段-璋叫道:``好,我领你这个情,咱们各不相扰!''他的第二剑本来就要刺出,这时倏然停住。令狐达急忙叫道:``这两个人乃是叛徒,尉迟都尉,你千万不可轻易的放过他们!''

原来这个黑面军官名叫尉迟北,是唐初开国功臣尉迟敬德的曾孙,兄弟二人。哥哥尉迟南任禁军统领,他则是扈从皇帝的带刀侍卫,官封龙骑都尉,职位武功均在令狐达之上,是大内三大高手之一。他家传的``空手入白刃''功夫最为厉害,当年秦王(唐太宗未即帝位之前的封号)李世民统兵伐魏(李密),在五虎谷与瓦岗军悍将单雄信相遇,李世民被单雄信追至逃魂涧,几乎被俘,幸赖尉迟敬德救驾。空手夺了单雄信所使的重达三十三斤的铁槊,天下闻名。

这尉迟北施展家传绝学,却穿不了段-
璋手中的宝剑,登时起了好胜之心,哈哈笑道:``我不管你是什么人,你这剑法,却是非得再领教几招不可!''双掌一错,一招``斜挂单鞭'',左掌猛切段-
璋的脉门,右手一抓,就要硬抢段-璋的宝剑。段-璋这时已知道铁摩勒安全无恙。打法自是不同,无须与他拼命。尉迟北的擒拿手虽然精妙绝伦,但段-
璋焉能给他抓着,但见剑光一闪,段-璋一个拗步回身,早已绕到尉迟北身后,喝声:``看剑!''唰的一剑,剑尖向着尉迟北肩后的``风府穴''点下,他出声示警,乃是为钦佩尉迟北也是一条好汉,刚才又释放了铁摩勒,所以有意对他卖个人情。尉迟北笑道:``你不必手下留情!''掌随声到,段-璋的剑尖尚未沾及他的衣裳,蓦然间给他反手一掌,就像背后长着眼睛一般,但听得``嗤''一声,段-璋的袖子已给他撕去一截,要不是段-
璋缩手得快,宝剑也要给他夺去了。

段-璋喝声:``好掌法!''一剑搠空,剑招立变,身随剑走,剑跟身转,霎时间四面八方,都是剑光人影,激战中,但听得``嗤''的一声,尉迟北喝声道:``好剑法!''原来他急于抢攻,一疏神,衣襟也给段-
璋一剑穿过。

段-璋道:``彼此两个不输亏,我还有事,请恕少陪!''砰的一掌打开窗户,立即跳下街心。尉迟北也不阻拦他,一幌身。却拦着南霁云道:``你也得留下两手!''南霁云那有心情与他纠缠,卖个破绽,容得他的手掌堪堪切到,猛地横肱一夹,反转刀背便拍下去,那知尉迟北擒拿手法实在厉害,但听得``嗤''的一声,尉迟北给他刀背拍了一下,却就在这同一时候。尉迟北一个穿掌进招,扭担了南霁云的手腕。南霁云掌握不住,宝刀脱手飞出。尉迟北叫道:``好,咱们也是两个不输亏!''

南霁云一个沉肩缩肘,忽觉对方手劲一松,南霁云趁势脱出,一个筋斗,便从段-
璋打烂了的那个窗户翻出,尉迟北一手抓去,``咔嚓''一声,抓断了一根窗格,却没有抓着他的脚跟。

原来这是用迟北有意放走他的,要知若是论到真实的功夫,他和南霁云实是各有擅长,难分高下。他刚才虽然抓住了南霁云的手腕,但要是南霁云那一刀不反转刀背拍下去的话,尉迟北的一条手臂已先给他削断,南霁云既然先对他手下留情,他本着英雄重英雄,好汉惜好汉之义,也故意虚晃一招,让南霁云从容逃走。

令狐达赶了到来,连呼可惜,还想去追,尉迟北沉声说道:``要捉拿这两个人除非把字文统领和秦都尉一并找来,否则咱们追上去也不是人家的对手,你还是坐下来和我说说吧,你说这两个人乃是叛徒,可有真凭实据么?说给我听,我好去禀告皇上,然后才好调动宇文统领和秦都尉齐来帮你的忙。''

宇文统领复性宇文,单名一个``通''字,秦都尉则是唐朝开国功臣秦琼的曾孙,名叫秦襄,这两人与尉迟北齐名,并称大内三大高手。令狐达已见识了段-
璋和南霁云的手段了,情知尉迟北所说的并非虚假,若然不是调齐三大高手,确实毫无取胜把握。只得依言坐下,细说详情。

尉迟北听了哈哈笑道:``依此说来,你也并没有拿着他们谋叛的真凭实据。郭子仪是防守边疆的得力将军,李学士又是皇上宠信的人,咱们犯不着为了巴结杨国忠就和他们作对,要是扳他们不倒,岂非未见其利,先见其害。那性南的虽有不满朝廷的语言,但并非严重,只凭他的一两句话,便想坐实他的谋反之罪,也难以说得过去。何况那姓南的是江湖上著名的游侠,交游广阔,得罪了他,他日咱们再出差在外,也有不便。依小弟之见,冤家宜解不宜结,令狐兄还是罢手算了吧!''

尉迟北深知令狐达的为人,故意用他本身的利害,劝他打消陷害人的主意。尉迟北的职位在令狐达之上,这次又是他出手相助,令狐达才得以安然无事的,何况若要调动三大高手,亦非他的能力所能办到。因此不由得令狐达不依他的说话,虽然含恨在心,却也只好罢手了。

再说南霁云跃下街心,拾起宝刀,连忙和段、铁二人逃走,他穿的是军装,背后既没人追来,在街上巡逻的官兵根本不知道在酒楼发生之事,无人拦阻他们不消片刻,他们已逃到僻静的路上。

南霁云等三人放慢了脚步,段-璋笑道:``南兄弟,一别十多年,我几乎不认得你了,要不是李学士叫出你的名字,我还不敢相认呢。''南霁云道:``段大哥,你的相貌倒没有什么改变。嫂夫人没有同来么?这位小兄弟是谁家的公子?''铁摩勒笑道:``你不认得我,我却认得你。你不是有个绰号叫做磨剑客么?今天却为什么不用宝剑而改用宝刀?嗯,你那招前刀后腿使得真好,我就不及,练了许多次,还未曾学会。''段-璋笑道:``这孩子见不得别人的本领,一见了就想学。南兄弟,你记不起他么?他就是铁昆仑铁寨主的儿子,小名唤作摩勒的那个顽童。''南霁云道:``怪不得这么了得,那年我随师父拜见窦案主的时候,他还流着两简鼻涕呢,现在已长得这么高了。''段-璋笑道:``十年人事几番新,那时,你也不过象库勒这般年纪,现在则已经是闻名天下的侠客了。令师可好么?''南霁云道:``他还是老样子,东漂西荡,替人磨镜、不过,现在是我的师弟雷万春跟随他,所以我把那把剑也送给了师弟。这把刀却是睢阳太守张巡送给我的。''铁摩勒插口道:``这几年,我也在找他老人家,可惜总是无缘相遇。''段-璋突道:``你找他老人家做什么,想跟他学磨镜的本领么?''铁摩勒眼圈一红,道:``先父遗命叫我找他老人家的。''

原来古代的镜子是用铜做的,用久了便要磨它一次,恢复光泽,所以有一种职业是专门替人磨镜的。南霁云的师父是个江湖侠隐,以磨镜作为职业,一来掩盖自己的真正身份,二来也好藉此云游四方,给文豪杰。别人不知道他的名字,都称呼他做``磨镜老人''。南霁云跟他走江湖的时候,兼替人磨镜,因此江湖上的朋友也送他一个绰号,叫做``磨剑客''。十二年前,他们两师徒曾应窦家五虎之邀,到过他们山寨作客,曾经见过段-
璋夫妇,铁昆仑有两个最好的朋友,一个是窦家五虎之首的窦令侃,另一个就是``磨镜老人''。铁昆仑曾想托孤给磨镜老人,只因磨镜老人行踪不定,不易寻觅,因此才让儿子拜窦令侃作义父。

南霁云道:``我们也曾听得铁寨主去世的消息,只因铁老死后,他的山寨已给官军挑了,窦家五虎的山寨也屡屡迁移,我们无法问讯。师父他老人家也很挂念世兄呢。幸好在这里相逢,铁兄弟你要找他老人家也不困难,我明天要到睢阳去,约好了师父在那里会面。你可以随我一道去。''

铁摩勒道:``这,这,\ldots\ldots{}''他本来想说的是:``这敢情好!''但话到口边,却变成了``这好是好,但,我、我明------

51shucheng.net扫校

\chapter{第 六 回 龙泉要断奸人首
虎贲群惊剑气寒}\label{ux7b2c-ux516d-ux56de-ux9f99ux6cc9ux8981ux65adux5978ux4ebaux9996-ux864eux8d32ux7fa4ux60caux5251ux6c14ux5bd2}

段-璋道:``好,你就在这里歇息吧。''骈指一戳,点了那卫士的麻穴和哑穴,叫他既不能说话也不能动弹,将他就安置在那假山洞里,笑道:``魏老三,对不住,委屈你了,你忍着点儿,过了两个时辰,穴道自解。''

那座房子前面有一棵松树,枝叶茂密,段-璋处置了那姓魏的卫士,便即飞身上树,从树顶俯瞰下来,先窥察屋内情景。

只见安禄山和一个身材魁悟的官儿坐在当中的胡床上,两旁有四个军官,薛嵩也在其内。段-璋心道:``这个官儿想必就是什么钦使大人了,看来倒不像是个太监。''宫廷惯例,赏赐给大臣的东西多是叫太监送去的,所以段-
璋见这个``钦使''不是太监,稍稍有点诧异,但也并不特别疑心。

只听得那钦使笑道:``安大人,你今天来的正是时候,贵妃娘娘本来正在生气的,幸亏你来了给她解闷。''安禄山问道:``贵妃娘娘为什么生气?''那钦使道:``还不是为了那李学士的几首诗。''安禄山奇道:``李白怎的招恼了贵妃娘娘?''

段-璋听他们提起李白,格外留神,只听得那钦使道:``在你入宫之前,皇上和娘娘在沉香亭赏牡丹,皇上一时高兴,宣召李学士来作诗。他正在酒楼喝得醉醺醺的,李龟年他们好不容易才将他拉来。''安禄山道:``贵妃娘娘可是恼他无礼?''那钦使道:``不是。李白的这种狂态他们是见惯了的,皇上还亲自用衣袖给他拭去涎沫呢。后来又叫贵妃娘娘亲自调羹,给他喝了醒酒汤。''安禄山摇摇头道:``这等无礼狂生,皇上和娘娘也真是太纵容他了。''那钦使道:``后来李学士醒了,皇上就叫他做诗,这位李学士也真行,立即便赋了三章清平调,安大人,这三首诗可真有意思,我念给你听。''安禄山笑道:``我是个粗人,可不懂得什么劳什子的诗。''那钦使道:``这三首诗是称赞贵妃娘娘的,很容易懂。可是惹得娘娘生气的,也正就是这三首诗。''安禄山道:``这倒奇怪了,既是称赞她的怎又惹得她生气呢?这我可要听一听了。''

那钦使念道:``李学士所赋的清平调第一章是:``云想衣裳花想容,春风拂槛露华浓。若非群玉山头见,会向瑶台月下逢。'皇上大为高兴,便命李龟年与梨园子弟,立将此诗谱出新声,着李善吹羌笛,花奴击羯鼓,贺怀智击方响(一种乐器名),郑观音拨琵琶,张野狐吹角栗,黄幡绰按拍板,一齐儿和唱起来,果然好听得很。''安禄山龇牙裂嘴地笑道:``我听你念、也觉得果然好听得很!''

那钦使笑道:``可见安大人也是个知音的人。''安禄山本来是人云亦云,得他一赞,大为高兴,问道:``第二章第三章又是说些什么?''那钦使续道:``皇上听了第一章,对李白道:``卿的新诗妙极,可惜正听得好时,却早完了。学士大才,可为我再赋两章。'那李白乘机便要皇上赐他美酒,皇上故意逼他道:``你刚刚醉醒,如何又要喝酒?朕并非吝惜,只是怕你酒醉之后,如何作诗?这酒还是等你做了诗之后再喝吧。'李白一急,便大言炎炎地道:``臣诗有云:酒渴思吞海,诗狂欲上天。吃酒醉后诗兴越高越豪。'皇上大笑道:``怪不得人家称你酒中仙。'便命内诗将西凉州进贡来的葡萄美酒,赐给他一金斗,又命以御用的端溪砚,教贵妃娘娘亲手捧着,求学士大笔。''安禄山``哼''了一声道:``简直把他捧上天了。''那钦使笑道:``他本来就自夸`诗狂欲上天'嘛!''顿了一顿,续道:``李白将一金斗的葡萄美酒喝得点滴不留,果然诗兴大发,又立即赋了两章《清平调》,第二章道:``一枝红艳露凝香,云雨巫山枉断肠,借问汉宫谁得似?可怜飞燕倚新妆。'第三章道:``名花倾国两相欢,常得君皇带笑看。解释春风无限恨,沉香亭北倚栏杆。'皇上看了,越发高兴,赞道:``此诗将花容人面,齐都写尽,妙不可言!''便叫乐工同声而歌,他自吹玉笛,又叫贵妃娘娘亲弹琵琶伴和。闹了半天,然后仍叫李龟年用御马送李白归翰林院。''

安禄山一窍不通,问道:``连皇上也称赞是好诗,那贵妃娘娘还恼什么呢?''那钦使笑道:``贵妃娘娘起初也很高兴,她退入后院,还一直吟着李白给她写的这三章《清平调》。那时高力士正在她的旁边,四顾无人,便对娘娘奏道:``老奴初意娘娘听了李白此诗,必定怨之刻骨,如今娘娘反而高兴,这可大出老奴意外!''娘娘便问他道:``有何可怨之处?'高力士道:``他说:可怜飞燕倚新妆。是把娘娘比作赵飞燕呢!'贵妃娘娘听了,勃然变色,果然将李白恨之入骨。''安禄山诧道:``这赵飞燕是个什么人?''那钦使道:``赵飞燕是汉朝汉成帝的皇后。''安禄山道:``将皇后比她,也不算辱没她了。''那钦使道:``安大人有所不知,赵飞燕是个出名的美人,体态轻盈,常恐被风吹去。皇上有一次曾对贵妃娘娘戏语道:``若你则任其吹多少。'梅妃和她争宠的时候,也曾说她是`肥婢'。贵妃娘娘焉得不怒?''安禄山笑道:``原来如此。依我看来,女人还是胖一点的更好看!''

那钦使微微一笑,笑得颇有几分诡秘,安禄山道:``怎么,我说得不对么?''那钦使小声说了几句,安禄山勃然变色,拍案骂道:``这李白当真可恶,怪不得娘娘恼他!''

原来赵飞燕曾私通宫奴燕赤凤,是汉朝出名的淫后,高力士向杨贵妃进谗,就是说李白的诗将杨贵妃比赵飞燕,实乃``暗中讥刺娘娘的私德'',杨贵妃私通安禄山,高力士这样一说,正触着她的忌讳,因此将李白恨之入骨。

那钦使笑道:``安大人无须动怒,李白触怒了贵妃娘娘,他还能在朝廷站得住么,他虽然得皇上宠爱,但总不能胜过贵妃娘娘啊!高力士也真厉害,这一下什么仇都报了。''

安禄山问道:``高力士与李白有仇?''那钦使道:``你还不知道吗?去年渤海国派使臣来呈递国书,书上番文,满朝无人能识,后来由贺知章保荐了李白,他非但能识番文,而且就用那番邦文字,写了一封回书,谴责渤海可汗的无礼,这才保全了大唐的体面。李白当时也是喝得醉醺醺的,在醉草这`吓蛮书'的时候,要杨国忠给他磨墨,高力士给他脱靴。高力士早已想找他的过失了。''

安禄山道:``好,明天我也要送一份礼给高公公。''忽地话题一转,问薛嵩道:``听说你们今天在酒楼大闹,帮姓南的那个人是什么相貌?''

薛嵩口讲指划的描述了一番,安禄山沉吟不语,那钦使却仔细地问薛嵩,与他对敌的那人用的是什么剑法,段-璋在外面偷听,听他问得居然甚是在行,暗暗诧异。

安禄山沉吟半晌,蓦地拍案说道:``我不信他有这样大胆!''话犹未了,忽听得嗤嗤两声极为强劲的暗器破空之声,一条人影箭也似的射入屋中,守卫哗然惊呼。

段-璋用暗器打穴的功夫,射出了两颗铁莲子,一取安禄山胸口的``璇玑穴'',一取那钦使耳后的``窍阴穴'',准备将他们打倒之后,立即抢出去擒获一人,作为人质。他的暗器打穴功夫百发百中,满以为即算安禄山能够避过,那``钦使大人''决计躲避不了。

哪知奇怪的事情突然发生,大大出乎他的意料之外,那个钦使竟是个身怀绝技的一流高手!

那两颗铁莲子虽然不过黄豆般大小,但经段-
璋以金刚指力弹出,劲道却是非同小可,隐隐挟着风雷之声。不料那位``钦使''大叫了一个``好''字,信手抄起一双象牙筷子,只一挟就把一颗铁莲子挟住,就像挟肉丸子一般。说时迟,那时快,第二颗铁莲子又电射而至,那钦使将筷子一甩,两颗铁莲子碰个正着,同时落地。但紧接着便是``僻啪''一声,他那双象牙筷子也当中折断,裂为四段。原来他虽然挟着了铁莲子,那双象牙筷子却经受不起这股劲力!

那钦使``噫''了一声,随即哈哈笑道:``幽州剑客果然名不虚传,今晚我可以大开眼界了!''

原来这位钦使正是大内三大高手之一的宇文通,他的职位与另外两位高手秦襄、尉迟北一样,都是官封``龙骑都尉''。但因为秦襄、尉迟北乃是开国功臣之后,虽然皇帝对待他们三人不分厚薄,他却自惭门第不如,声望不及,总是感到皇帝对那两个人亲近一些。因此,他们三人虽然并驾齐驱,但行事却甚不相同,秦襄、尉迟北不屑巴结权贵,而宇文通则在宫中奉承杨贵妃,在宫外又与安禄山结纳,双管齐下,以求巩固职位。今晚替皇帝与杨贵妃送``洗儿钱''给安禄山这个差事,便是杨贵妃替他讨的。他虽然从未见过段-璋,但他却早已探听得段-
璋与安禄山有仇,一接了这两颗铁莲子,又见了段-璋所使出的剑术,当然可以立刻断定这人便是幽州剑客段-璋了。

这时薛嵩和另外三个卫士已堵住了段-璋,就在这屋子里厮杀起来。宇文通是钦使身份,一时不便出手。

安禄山突然遇袭,随即又看出了是段-璋,这一惊自是非同小可,但到了宇文通将那两颗铁莲子接下之后,他便安定下来,心中想道:``饶你段-
璋本领再高,单身一人,总敌不过我麾下诸将,何况还有字文都尉在此!''他既然有恃无恐,便站了起来,哈哈笑道:``我道是谁,原来是老朋友来了!有话好说,何必一见面就动刀动枪?难道你就一点也不念旧时情份,居然妄想取我的性命么?''

段-璋唰唰两剑;将薛嵩迫退几步,又荡开了另一个军官的护手钩,朗声答道:``安禄山,你小人得志,毗眶必报,还何必惺惺作态?哼,你要害我也还罢了,为何将我的朋友也一同陷害?''

安禄山笑道:``那是一个误会,但错了也有错的好处,要不是错捉了你的朋友,焉有请得你的大驾到来?而且我也不想难为他,你来得正好,你就劝他一同在我这里做事吧。''段圭璋道:``哼,给你作事?''安禄山大笑道:``我身兼平卢、范阳、河东三节度使,你给我当差,难道还会辱没你么?''段-璋以更响亮的声音笑道:``在我的眼中,你以前是个无赖流氓,现在也是个无赖流氓,不过比以前作的恶事更多更多,以前只不过是欺侮善良,现在则简直是祸国殃民了。哈哈,你以为你做了什么节度使,我就看得起你了吗?''

安禄山本来要像猫儿捕捉老鼠一般,料想段-
璋已逃不出他的手掌心,先把他嘲弄一番,发泄心头的恶气,哪知反而给他毫不留情的痛骂一场,并且揭穿了他的底细不过是个无赖流氓。这一气真气得七窍生烟,登时放下了脸,厉声喝道:``不识抬举的东西,你们给我将他毙了!''

段-璋大笑道:``我既然敢到你这里来,本来就不打算活的出去。可是,你们要把我杀掉,只怕也没有那么容易!''他口中滔滔不绝地说话,手底却是毫不含糊,笑声未绝,只听得``唰''的一声,一个卫士的胸口已中了一剑,血如泉涌,急忙退出战团。

安禄山骂道:``脓包,脓包!快去多唤几个得力的人来!''薛嵩是段-
璋手下败将,心里本来害怕,但听得安禄山一骂,却不由得他不鼓勇向前。段-璋喝声:``来得好!''宝剑横空一划,一招``龙门鼓浪'',矫若游龙,剑光四射,当真有若波翻浪涌,威不可当,薛嵩吓得魂飞魄散,连忙后退,却哪里闪避得开,陡然间只觉得肩上一片沁凉,早给段-
璋的宝剑划开了一道长长的裂口。

幸而那个手持双钩的武士亦非庸手,双钩一锁,把段-
璋的攻势解开,要不然薛嵩的琵琶骨也要给宝剑割断。薛嵩这时哪里还敢恋战,拼着受主帅责骂,虚晃一剑,就想退下。

段-璋恨他是捉史逸如的凶手之一,却容不得他逃走,猛地大喝一声,右脚飞起,一个``魁星踢斗'',将欺近身前的一个卫士踢翻,宝剑一挥,又将使双钩的那个卫土迫退,剑光一展,身形急起,如箭射来,眨眼之间,已追到了薛嵩背后,眼看那明晃晃的剑尖,就要在薛嵩的后心掷个透明的窟窿!

段-璋正要跨上一步,出剑刺薛嵩的背心大穴,忽觉得背后有金刀劈风之声,来势极为劲疾;段-璋眼观四面,耳听八方,立即知道是有强敌袭到,而且这一刀也正是对准他的背心大穴。

恰如螳螂捕蝉,黄雀在后,这突然袭来的一招,正是攻敌之所必救,段-璋心中一凛:``想不到安禄山的卫士之中竟有如此人物!''无暇收拾薛嵩,巳先对付背后的敌人。

段-璋的剑术已到了炉火纯青之境,心念一动,剑招立即发出,反手一撩,身形未变,却像背后长着眼睛一般,剑尖直指那敌人的脉门,登时把他这偷袭的一招解了。

段-璋脚跟一旋,转了半个弧形,顺势一招``横云断峰'',剑势横披过去。那人似是顾忌他手中的宝剑,不敢让刀口相交,却反转刀背一磕,只听得``当''的一声,火星蓬飞,那人斜跃三步,段-璋也不禁上身一晃。

宇文通赞道:``刀法精奇,剑术更妙!两人都好!好,好!''喝彩声中,段-璋已转过身来,定睛一看,看清楚了敌人的面貌,不觉一怔!

这人正是曾经三番两次暗中替他遮掩、劝他回去的那个聂锋,真是大出段-璋意外。

使双钩的那个卫士名叫张忠志,武功与薛嵩在伯仲之间,也是安禄山手下的一名得力军官,趁这时机,双钩霍霍,卷地勾来,疾攻段-
璋的下盘。段-璋刚自一怔,一个疏神,``嗤''的一声,饶是他立即滑步闪开,裤管亦已被撕去了一幅。

聂锋大喝道:``天堂有路你不走,地狱无门你偏进来!死到临头,还敢逞凶伤人么?''听这语气,凌厉之极,但段-
璋却听出了他的话中含意,似乎还是劝他逃走的意思。段-璋心道:``他是安禄山的亲军副将,怪不得他要为安禄山出力,只是他对我却颇有惺惺相情之意,不知为了什么?''

聂锋确是有惺惺相惜之意,但在安禄山面前,他却是不敢露出些微破绽,而且刚才试了两招,他也发觉了段-
璋的本领实是在他之上,因此确是认真动手,将全身解数都施展开来,一口单刀舞得泼风也似。倒是段-
璋因为不愿伤他性命,有几招最为厉害的杀手剑招他都不敢使用,这样一来,他以一敌二,竟然渐走下风。宇文通看了片刻,心中想道:``这段圭璋剑法虽然精妙,可算得是当世一流高手,但似乎还没有武林中传说他的那样神奇。''

没多久,田承嗣和几个军官闻讯赶来,见段-璋已落在下风,大家都想抢功,一拥而上。尤其是田承嗣,为了要报日间在酒楼所受之辱,刀刀都朝着段-
璋的要害之处劈来。他知道段圭漳那口剑是把宝剑,特别挑选了一件重兵器------重达三十三斤的厚背斫山刀,段圭璋的宝剑虽然锋利,却也无法将它削断。段-璋力斗六名高手,更显得左支右绌,激战中,忽听得``当''的一声巨响,刀剑相交,田承嗣的大刀被段-
璋用巧劲带过一边,但他的宝剑也给荡开。他这一招本是一招三式,同时应付三般兵器的攻击的,剑点一歪,张忠志的双钩立即乘虚而入,喇啦一声,又撕破了他的一幅上衣,钩尖划过,即小臂上登时现出了一道伤痕。而与此同时,聂锋的单刀也正使到一招``白蛇吐信'',明晃晃的刀尖堪堪就要指到他的喉头。

段-璋一个``大弯腰、斜插柳'',身躯转了半个圆圈,倏的一剑反削出去,只听得``哎哟''一声,聂锋中了一剑,血流如注,斜跃出去,随即倒地,包围圈出了一个缺口。

段-璋这一剑本来只是想格开聂锋的单刀的,结果却令聂锋受了重伤,实是他始料之所不及。他哪知原来是聂锋有意放他逃走的,聂锋一见段-
璋出剑的姿势,已知他的剑锋削向哪边,若论两人真实的本领,聂锋仅比段-
璋稍逊一筹,他那一刀斫去,虽然一定会给段圭璋格开,但他只要向相反的方向避开,就不至于受伤,但他有意放段-璋逃走,不惜身受重伤,故意向着段-
璋剑锋所指的方向迎去,因此才被段-璋一剑戳中了他的小腹。

段-璋败里反攻的这一招本来精妙非常,剑势虚实莫测,所以聂锋虽是有意让他,旁人却看不出来。不过,段-璋是个武学的大行家,初时虽然一愕,片刻便即明白,心中想道:``我若然不死,日后定要报此人之恩。呀,只是你一番好意,我却不能接受。救不出史大哥,我还有何面目独自逃生?''

段-璋已从缺口冲出,但他却不肯夺门逃走,反而向安禄山奔来,田承嗣等人大惊,慌忙堵截。正在他们手忙脚乱之际,忽听得字文通哈哈笑道:``看了段先生这等精妙的剑法,我也有点技痒难熬了。各位暂请歇手,待我来献丑,献丑!''声到人到,双手空空,长衫飘飘,话声未了,已站在段-
璋的面前!

田承嗣等人一见字文通出手,俱都松了口气、他们知道宇文通自视极高,不待吩咐,便纷纷闪开,让出场子。段-璋见他如此声威,也不禁心中微凛:``原来这个`钦使大人',竟是一流高手。''

字文通站在段-
璋面前,紧握双拳,睥睨作态,傲然说道:``段大剑客,你刚才不是有意将我拿下的吗?现在我已站在你的面前,你怎么还不动手?''段-璋道:``你既然按照武林规矩与我单打独斗,我岂能占你的便宜,亮出兵器来吧!''

字文通大笑道:``段先生果然不愧是成名剑客,不肯贻人半点口实。不过,你可不必为我担心,你虽然有一把上好的宝剑,却也未必便能伤得了我宇文通!''

宇文通自报姓名,段-璋这才知道他是与秦襄、尉迟北齐名的大内三大高手。段-璋这一生几曾受过人如此轻视,心中怒气陡生:``你以为凭着你大内高手的名头,就可以压倒我不成?我不信你的空手入白刃的功夫,还能够在尉迟北之上?''要知若论到空手人白刃的功夫,尉迟北这一家乃是天下第一家,但段-
璋这日日间在酒楼上与尉迟北一番较量,却还稍稍占了上风,所以他才敢暗骂字文通狂妄。

当下段-璋冷冷说道:``是么?好吧,那就请你先赐高招!''他虽然气极怒极,但看在对方空手的份上,仍然不肯占先动手的便宜。

宇文通道:``好,恭敬不如从命,留神接招!''双拳一晃,立即劈面打来,段-璋一看,他既非擒拿手法,亦非最厉害的罗汉神拳招数,只不过是普普通通的北派长拳,不由得大为诧异,心道:``难道他以为凭着这套普通的拳术,就可以应付我的宝剑不成?他号称大内三大高手之一,不信他竟这般没有眼力!''

段-璋心念方动,宇文通那碗口般粗大的拳头已打了到来,段-璋横剑一削,宇文通双拳一张,忽听得``叮''的一声,火星溅起,原来宇文通并非狂妄。相反的却是极工心计。他手中藏着一对极短的判官笔,事先并不说明,由得段-
璋以为他是空拳对敌,有意激恼段圭璋并令他轻敌。待到段-
璋一剑削来,他双拳一张,暗藏的判官笔突然伸出,恰恰顶着段圭璋的剑脊。说时迟,那时快,他左笔一顶,右笔立移,趁着段-
璋剑招用老,来不及撤回之际,骤下杀手,闪电般的判官笔便向段-璋胁下的``愈气穴''点来,当真是阴毒之至,狠辣之极!

幸而段-
璋是个胆大心细的人,他虽然不知道宇文通掌中暗藏兵器,但见他只是使出一套普普通通的北派长拳,早已起了疑心,因此并不如宇文通所算,他非但没有轻敌,反而格外留神,第一招只是虚晃一招,未曾用实。

就在那电光石火的刹那之间,两人的身形都快到极点,宇文通一笔点向段-
璋胁下的愈气穴,笔尖尚未沾到他的衣裳,陡然间只见剑光一闪,段-璋的剑尖已指向他的小腹。这一招是攻敌之所必救,宇文通只得把判官笔偏斜一格,立时跳起,半攻半守,才化解了段-
璋这一凌厉的剑招。旁人看来,但见两条人影倏的分开,一个弯腰,一个跳起,却不知道就在这一招之间,两大高手都已使出了平生绝学,过了性命相搏的一招!

宇文通这时方始知道段-璋的剑法果然非同小可,刚才实是未曾使出全部本领,不觉暗暗胆寒。

说时迟,那时快,两人一分又合,段-璋挽了一个剑花,唰、唰、唰,连环三剑,疾风暴雨般的狠狠攻来,使到疾处,但见剑光,不见人影,竟似有十几口宝剑,从四面八方攻来一般,剑气纵横,剑光飘瞥,将宇文通的身形全都笼罩,旁边观战的武士,看得眼花缭乱,个个惊心。

宇文通号称大内三大高手之一,武功上确也有惊人的造诣,对于判官笔点穴,武学有云:``一寸短,一寸险!''普通的判官笔是二尺八寸,他这对判官笔只有七寸长,实是短到无可再短,因此每一招都是欺身进搏,凶险万分,不论哪一方稍稍应付不宜,都有性命立丧之虞。

段-璋一剑紧似一剑,眼看胜算可操,激战中忽听得``嚓''的一声,字文通那对判官笔陡然间暴长七寸,原来他的判官笔共有四节,每一节长度七寸,一按机括,便可以一节一节的伸出来,全长仍是与普通的判官笔一样。

高手比斗,只差毫厘,现在两人在近身肉搏之际,宇文通的判官笔暴长七寸,饶是段圭璋本领再高,也难以闪开。只听得``嚓'的一声,宇文通的判官笔已扎破了段-
璋的衣裳插入了他的小腹。旁观的武土登时彩声如雷。

可是彩声未绝,宇文通却忽地``哎哟''一声,斜跃出一丈开外,众人先闻其声,定睛看时,始见他的肩头上殷红一片!

原来段-
璋不但剑术精妙,内功亦已有了相当造诣,当宇文通的那支判官笔一扎破他的衣裳的时候,他吞胸吸腹,小腹陡然凹了三寸,判官笔的笔尖刚刚沾着他的皮肉,业已力尽,就差那么一点点劲力未到,戳不进去。段-璋的剑法何等快捷,就趁对方已是强弩之末,来不及换力进招的瞬息之间,抓着时机,剑锋一偏,削去的宇文通肩上的一片皮肉。

幸而宇文通也是个武学的大行家!一觉不妙,立刻撤笔抽身,要不然只怕琵琶骨也要给宝剑削断。

这一下突然的变化,众武士大惊失色,喝彩的声音登时止了。宇文通刚刚夸了海口,说是段-
璋的宝剑不能伤他,哪知未到三十招便当场出丑,虽然仅是皮肉的轻伤,但他是自大惯了的,在这众目睽睽之下,段-璋这一剑无异戳破了他的面皮,令得他又羞又怒。当下大怒喝道:``姓段的,我若今晚让你逃得出去,我宇文通誓不为人。''双笔横穿直插,展开了一派进手的招数,他的判官笔点穴手法独创一家,确也具有相当威力,这时两人已是如同拼命,谁也不敢轻视对方。

安禄山道:``对,还是生擒的好,你们在这里呆着作什么?还不快快上去,帮宇文都尉将这贼人缚了?''

田承嗣与张忠志这些人刚才之所以不敢去帮忙,一来是知道宇文通骄傲自大的脾气,二来他们也深知宇文通的本领,以为段-
璋的剑法虽然精妙,但在久战之后,以宇文通的本领,当可取胜无疑。哪知事情大大出乎他们的意料之外,受伤的竟然不是段-
璋而是宇文通,现在安禄山一声令下,他们再无顾忌,立即上去围攻。宇文通这时已知道不是段-璋的对手,对别人的帮忙,也就不加阻止了。

宇文通的本领和段-
璋所差有限,得了田承嗣和张忠志相助,登时扭转了劣势。只见剑气纵横,刀光如雪,双钩霍霍,笔影重重,这一场恶战,当真是惊心骇目,令得旁观的卫士,气也透不过来。

激战多时,段-璋的剑光圈子越缩越小,安禄山刚刚松了口气,陡然间,忽听得段-
璋大喝一声,剑光夭矫,宛若游龙,忽然突围而出,田承嗣的膝盖先中了一剑,跄跄踉踉的退了几步,紧接着``嚓''的一声,张忠志也给他削去了一只手指。宇文通一笔戳去,段-璋刚刚削了张忠志的手指,未及撤剑回身,捏着剑诀的手指,突然收拢,反掌向后一拍,``当嘟''声响,宇文通那枝判官笔也坠地了!

段-璋以掌拍笔这一招实是用得凶险之极,结果,宇又通那枝判官笔虽然给他拍落,但段-
璋左手手腕的寸关尺脉,给铁笔划过,也裂开了一道长长的口子。寸关尺脉受伤,这条臂膊,已是再也不能用力。

宇文通见他用这种两败俱伤的打法,暗暗吃惊,但在这一招上,他伤了段-
璋的一条臂膊,却是占了便宜。旁边一个卫士将那枝判官笔拾了起来,向他抛去,宇文通接笔在手,立即喝道:``这厮只有一只手好使用了,再凶也凶不到哪儿去了,赶快将他拿下,留心他要逃跑!''

段-璋一声长啸,冷冷说道:``好个大内高手,果然是好本领,好威风!不但是皇上跟前得力的人,而且还做了安禄山的看门狗!哼,你怕我逃走么?我踏进此门,本来就不打算活着出去了,你放心吧!''

宇文通给他一番奚落,满面通红,喝道:``我不与你斗口,看笔!''段-璋的宝剑已削了到来,登时两人又斗在一起。

这时,宇文通、段-璋张忠志、田承嗣这四个人都已或多或少的受了些伤,而以段-
璋伤得最重,其次是田承嗣,他的膝盖被削去了一片,跳跃不灵,但仍然跟着字文通他们围攻段-璋。

段-璋虽然伤了一条臂膊,但他已豁出性命,剑招越发凌厉。安禄山的手下,武功最高的是田承嗣、薛嵩、聂锋、张忠志四人,现在聂锋和薛嵩先后受了重伤:只有田、张二人助宇文通作战,其他的卫士,武功相差太远,上去了几个人,都给段-
璋刺伤,未受伤的也帮不了忙,反而碍手碍脚。宇文通气极,大声喝道:``你们去保护大帅吧,别在这儿丢人现世了。''那些卫士一哄散开,结果还只是留下了田、张二人助他。

激战中只听得``唰''的一声,田承嗣跳跃不灵,身上又中了一剑,幸而并非要害,但亦疼痛难当。宇文通趁段-
璋剑刺田承嗣的时候,一按机括,判官笔又伸长了一节,这次段圭璋早有防备,一跳避开了,但在他跳跃之时,小腿却给张忠志的利钩钩去了一片皮肉。

安禄山看得心惊胆战,生怕宇文通若然也非敌手,段-璋杀了上来,他性命难保,但``钦使大人''在这里为他抵御仇人,他又怎好意思退入后堂躲藏起来?正在心慌意乱之际,忽见薛嵩一声哈喝,带着几个卫士,推了一个人进来!

段-璋失声叫道:``史大哥!''原来给薛嵩推进来的这个人正是史逸如!只见他瘦骨支离,病容憔悴,已给折磨得不似个人形。薛嵩挺着一把长剑,顶着他的背心,大声喝道:``段-璋,你给我站住,你若是再跨上前一步,我就先把你的史大哥杀了!''

段-璋又怒又气,心痛如割,但投鼠忌器,也只好强抑怒火,停下脚步,横剑当胸,封住了宇文通攻来的双笔,向安禄山叫道:``你的仇人是我,关姓史的什么事?要杀要剐,听你的便,你把这姓史的放了!''

安绿山这才松了口气,哈哈笑道:``好,你把宝剑扔下,我可以绕这个姓史的不死。''

段-璋冷笑道:``你当我是个三岁小儿,可以任由你戏要么?要我扔下宝剑也不难,你得让我先将史大哥送出十里之外,然后再和你的人一同回来,那时我甘愿把宝剑缴给你。''

安禄山笑道:``你不相信我,你又怎能叫我相信你?先扔宝剑后放人,没有讨价还价的了!''

段-璋眼燃怒火,心里踌躇,这时宇文通、张忠志、田承嗣三人,早已占了有利的方位,三般兵器,对准了段-璋的要害。

史逸如忽道:``让我和段大哥说几句话!''安禄山道:``好,你劝他投降,我敬重你是个读书人,决不为难你,你愿做官便有官做,你不愿做官,我便立即放你,让你家人团圆。段-璋是我的老朋反,他虽然对我不敬,我也会饶恕他的,你可以不必为你的朋友担心。''

史逸如所安禄山提起他的家人,面上一阵青一阵红,又是悲愤又是伤心,他嘴唇颤动了几下,忽地双眉一坚,心意立决朗声说道:``段大哥,与其留我报仇,不如留你报仇!为了免得你被人要挟,我先走一步了!''陡然间向后一撞,薛嵩那柄长剑正对着他的后心,做梦也想不到他会借剑自杀,要缩手已来不及,史逸如这一撞用尽了浑身气力,那柄长剑从他的后心透过了前心。

这一下突如其来的变化,连安禄山和薛嵩也吓得呆了,就在这一瞬间,段-璋一声怒吼,俨如受了伤的狮子,双眼火红,挥剑便杀!

张忠志首当其冲,段-璋这一剑乃是毕生功力之所聚,张忠志如何禁受得起?但听得``咣''的一声,张忠志的一柄护手钩已给他削为两段。

宇文通一按机括,判官笔的最后一节伸了出来,段-璋一剑削断了张忠志的护手钩,立即飞身掠起,逞向安禄山扑去,本来以他的本领,要闪开宇文通这一招并不困难,但此时他怒火如焚,一心只想杀了安禄山为他的好友报仇,宇文通一笔点来,他竟浑如未觉。

宇文通这一笔正正点中他的后心,幸而习武之人骤逢袭击,虽在神智昏迷之中,也能够立时生出反应。字文通本来要点他后心的``中府穴''的,笔尖一触,忽地觉得有一股反弹的力道,笔尖滑过一边。原来就在这刹那间,段-璋已闭了全身穴道,并用``沾衣十八跌''的上乘内功,弹开了宇文通的笔尖。

可是宇文通的功力亦已到了第一流的境界,与段-
璋相差无几,他的笔尖虽然滑过一边,但顺手一拖,段-璋的背脊登时也出现了一道伤痕,他的小腿本来已受了钩伤,这一跃又用力过猛,再给宇文通的判官笔划伤了他的背心带脉,饶他功力非凡,亦是抵受不起,就在张忠志给他的猛力震倒之时,他也跟着跌倒了。

宇文通大喜,左手的判官笔立即跟着戳下,段-璋在失足跌倒之时,心里猛地想道:``大哥之仇未报,我还不能死,不能死!''也不知哪里来的气力,陡然间大喝一声,一个``鲤鱼打挺''翻起身来,正碰着宇文通那一笔向他戳下。宇文通给他那一声大喝,震得耳鼓``嗡嗡''作响,不觉呆了一呆。说时迟那时快,段-璋一招``举火撩天'',宝剑与判官笔碰个正着,宇文通大叫一声,虎口震裂,判官笔的笔尖亦已给宝剑削去。

安绿山吓得面无人色,叫道:``调,调,调弓箭手和挠钩手来!''宇文通到底是惯经阵仗的人,这时他已看出了段-
璋不过是拼着最后一股气作困兽之斗而已,立即叫道:``安大人放心,这恶贼虽凶,也挨不了多少时候了。''``咄,绕身游斗,不必和他硬碰!''

段-璋的手足、肩、背部已受伤,有如一个血人,跳跃亦已不灵,宇文通这一班人将他围着,采用了绕身游斗的战术,登时将他困在核心!但段-
璋仍然高呼酣斗,猛若怒狮!

正是:为报深仇甘拼死,气冲牛斗恨难平。

欲知后事如何,请听下回分解------

\chapter{第 七 回 落难英雄逢异丐
扶危绝技退追兵}\label{ux7b2c-ux4e03-ux56de-ux843dux96beux82f1ux96c4ux9022ux5f02ux4e10-ux6276ux5371ux7eddux6280ux9000ux8ffdux5175}

田承嗣和张忠志都是吃过段珪璋苦头的人,张忠志只剩下一柄护手钩,田承嗣的膝盖刚才被段珪璋削去了一片皮肉,痛犹未过,段珪璋高呼酣斗,他们虽然把他困在核心,兀自感到心惊胆战。薛嵩本来受伤不轻,这时也迫得和随他一道来的两个军官加入战团。薛嵩是安绿山的亲军统领,这两个军官是他的副将,武功略逊于张忠志,在安绿山帐下,是第五、第六名好手。

没多久,一队挠钩手开了到来,共是十二个人,挠钩长达一丈有余,十二个挠钩手分布四万,伸出长钩,钩段珪璋的双脚。

段珪璋大喝一声,一剑削断了两柄挠钩,但那些挠钩从四面八方伸来,削不胜削,终于给一柄挠钩勾住了腿肚。段珪璋扑通一声,坐在地上,田承嗣大喜,举刀便斫,猛听得段圭璋又是一声大喝,咔嚓声响,竟然把那柄挠钩折为两段,钩尖还嵌在肉中,另半截带着淋洒鲜血的烧钩,被他夺了过来,随着喝声,猛的向田承嗣掷去。田承嗣惊得呆了,薛嵩急忙将他一掌推开,但听得``呼''的一声,那半截挠钩从田承嗣的头顶飞过,擦破了他一片头皮,余势未衰,那名勾伤了段珪璋的挠钩手,恰好被掷回来的自己的那半截挠钩撞正胸口,登时跌了个四脚朝天!

段珪璋拔出断钩,浑身浴血,坐在地上,兀自神威凛凛,狂挥宝剑,但听得一片断金戛玉之声,震得众人的耳鼓都嗡嗡作响,又有三柄挠钩给他削断!

安禄山看得心胆俱寒,说道:``我身经百战,还未见过这样凶悍的人!''薛嵩早已退下,这时站在安禄山旁边,说道:``他已不能走动了,调弓箭手来射他,立即可以要了他的性命!''安禄山点点头道:``也只有如此了。怎么弓箭手还不来呢?''一面吩咐手下去催,一面嚷道:``宇文都尉,不必和他硬拼了,弓箭手马上就来!''

宇文通集众人之力,仍然未能把段珪璋擒下,深感面上无光。这时,先前围攻段珪璋的六个人,也只有他一人未曾退下。

段珪璋又受了两处钩伤,宇文通咬一咬牙,正要鼓勇上前,将他活捉。就在这个时候,忽听得外面嘈声大作,有人呐喊,有人奔跑。安禄山初时以为是弓箭手来到,一听那惊喊的声音,奔跑的声音,却又不似,正在惊疑不定,忽听得在门口把守的一个军官大叫道:``不好,不好!起火啦,起火啦!''

安禄山方自一惊,猛听得又有几个声音同时喊道:``捉刺客,捉刺客!''就在这时,守门的卫士忽如遇到巨浪冲击一般,发一声喊,纷纷后退,有几个来不及避开的,已给人推倒地上。

外面冲进了两个人,一个穿着军官的服饰,另一个却是十六七岁的少年。这两人冲了进来,当者披靡!安禄山第一眼瞥见是个军官,心中稍宽,喝道:``什么事情,慌慌张张的胡冲乱闯?''话犹未了,猛听得那军官大喝一声,俨如舌尖上绽了一个春雷:``安禄山,你敢害了我的段大哥,我就要你的命!''声到人到,他来不及驱散卫士,便跃了起来,呼的一声,从众卫士的头上飞过,那些挠钩手正自伸出长钩,被他凌空扑下,刀光闪处,一片断金戛玉之声,震耳欲聋,几柄挠钩,同时给他削断!那少年貌不惊人,身手却也不弱,刀斫、掌劈、脚踢,施展了全身解数,眨眼之间,把近身的卫士杀得个七零八落,还有几个挠钩手也给他踢翻了。

田承嗣失声叫道:``南霁云,你好大胆!''这两个人正是南霁云和铁摩勒!

段珪璋因为不愿连累朋友,将事情瞒着南霁云,但铁摩勒却是个机灵的孩子,早就将南霁云的地址,牢牢记在心中。他口头上答应段珪璋这一晚不出寺门,等候段珪璋回来,但段圭璋一走之后,他就偷偷去找南霁云了。

南霁云这一晚和李白有约,约好了黄昏之后在贺知章家里相会,铁摩勒找到南霁云的住所,已是将近三更,他还没有回来,铁摩勒只得在他的房间里留下字条,再到贺知章家里去找。原来他和李白喝酒畅谈,谈得高兴,忘记了时间,铁摩勒到了贺家,他们尚是酒兴未阑。李白见惯了江湖侠士的行径,铁摩勒穿着夜行衣突然闯入,他也毫不惊骇,还拉铁摩勒一同喝酒。

铁摩勒哪里还有心清喝酒,急急忙忙将事情告诉南霁云,南霁云一听,酒意全都醒了,立即向李白告辞,三步并作两步,赶来救人。可惜还是迟了一步,史逸如已经自杀身亡,段圭璋亦已受了重伤了。

田承嗣是给南霁云杀得丧了胆的,一见他来,虽然一面大呼大喊的给自己壮胆,却实是不敢和南霁云接战,一面呼喊,一面连连后退。这时,安禄山也顾不得对``钦使''的礼数,顾不得什么``大帅''的体面,紧紧捉着田承嗣的手,由他保护,慌慌张张的立刻退入后堂。

薛嵩也是给南霁云杀得丧了胆的,但他没有田承嗣的及早见机,又因伤得较重,这时还未退下,南霁云喝道:``姓薛的,酒楼上那一架打得不够痛快,再来,再来!''声到人到,抡起宝刀,倏的就劈到他的面前。薛嵩此际,即算没有受伤,也不敢硬接他这一刀,急忙虚晃一剑,转身便逃。张忠志抢来援救,斜身进钩,南霁云一招``雁阵排空'',横刀一削,张忠志的护手钩早已给段珪璋削断了一柄,但听得``咣''的一声,剩下的这柄护手钩,又给南霁云削为两段,变成了双手空空,无可抵御。南霁云见他们两人身上都染有血污,忽地将已劈出的刀势煞住,一声喝道:``我宝刀不杀受伤之人!''一个``鸳鸯双飞脚''踢出,左脚向薛嵩的背心一蹬,左脚向张忠志的腰胁一踹,薛嵩给踢翻出一丈开外,张忠志也变成个滚地葫芦。

宇文通在这混乱之中,想先把段珪璋杀了再说,他左笔刚桃开了段珪璋的宝剑,右笔正要插下,猛觉金刃劈风之声,南霁云的刀锋已戳到了他的背后。宇文通一个``盘龙绕步'',反手一招``横打金钟'',刀笔相交,火星飞溅,宇文通的判官笔是精钢所铸,给他宝刀一磕,也损了指头般粗大的一个缺口,手臂酸麻,不由得蹬、蹬、蹬在退三步。可惜段珪璋这时已不能走动,宇文通从他身边掠过,段珪璋一剑横扫,只差三寸,没有削去他的膝盖。

南霁云无暇理会宇文通,急忙将段珪璋抱了起来,叫声:``大哥!''段珪璋双眼一睁,叫道:``南兄弟,是你来了!''忽地一口瘀血喷了出来,登时晕了过去!他以寡敌众,激战了一个时辰,已是遍体鳞伤,筋疲力竭,不过全仗着口气,强力支持而已。现在,他看见了南霁云,精神一松,真气立散,饶是铁铸的人儿,亦已支持不住。

宇文通是个经验丰富的老手,见南霁云救了段珪璋,心中反而欢喜,想道:``你背了一个人,我就不怕你了!''提笔又上,双笔一分,交叉穿插,左笔横拖,虚点南霁云手少阳经脉的``中浮''\,``曲池''``少府''三穴,右笔却向段珪璋垂下的脚背`地户穴''戳下。幸而南霁云一心一意只是在保护段珪璋,对自己的安危反而置之度外,宇文通攻向他的虚招,他根本就不招架,刀锋下撤,将宇文通那一笔荡开。待到宇文通要把攻向他的那一招招数化实之时,南霁云已冲出了几步。

宇文通哪里肯舍,如影随形,急忙追上。南霁云喝道:``好狠呀你!''脚尖一点,突然跃起,宇文通双笔在他脚底穿过,说时迟,那时快,南霁云一刀便劈下来!

这一招用得凶险之极,宇文通料不到南霁云背着一个人,还居然敢跳起来用``力劈华山''的招数,不由得大吃一惊,急忙一矮身躯,避过刀锋,硬生生的将攻出去的双笔收了回来,笔尖刚好顶着刀板。只差三寸,险些就要给削去头皮。

南霁云这一劈之势刚猛之极,宇文通敌不住他的神力,只得使出``燕青十八滚''的招数,滚将出去,虽然没有刚才薛嵩那么狼狈,却也变成了个滚地葫芦。

南霁云身形未落,双脚先行踢出,砰、砰两声,又踢翻了两个卫士,大声喝道:``避我者生,挡我者死!''宝刀舞起一片银光,夺门便走。众卫士见他如此凶猛,谁敢阻拦,瞬息之间,已给他冲到门口。

这时,满天都是融融的火光,原来这是铁摩勒所点的火。铁摩勒是在强盗堆中长大的,熟谙黑道的伎俩,随身带了火种,潜入了安禄山的府邸,便在三四处地方点起火头,好趋混乱中逃走。

这一来,众卫士忙着救火,府邸里乱成一片。那一队弓箭手虽已赶了到来,但满园子人影幢幢,狂奔疾跑,弓箭手怕伤了自己人,只敢张弓,不敢放箭。

铁摩勒哈哈笑道:``今晚虽然杀不成安禄山,却也出了一口鸟气!''宇文通大怒,一笔向他点去,铁摩勒反手一刀、这一刀用的是段珪璋所教的剑术招数,甚为古怪,宇文通的武功虽然比他高出许多,也禁不住心头微凛,不敢轻敌,转过笔锋,横架金刀,斜点腰胁。铁摩勒这一刀可实可虚,一见宇文通以守为攻,立即一晃便收,斜身一跃,抓起了一个卫士,向宇文通掷去。宇文通不敢伤安禄山的手下,只好将那卫士接了过来,轻轻放下。只见铁摩勒一溜烟似的,早已穿过人丛,笑声不断,追上了南霁云去了。宇文通气得七窍生烟,穷追不舍。

哪知铁摩勒这一把火,有利却也有弊,骊山离宫的卫士,看见火光,纷纷赶来,南、铁二人刚杀出重围,迎面便碰见这群卫士。

南霁云叫道:``你们来得正好,快快帮忙救人,里面还有几个刺客未曾拿下!''他穿着军官服饰,那些卫士一时给他唬住,未敢即行动手。南霁云身法何等快疾,换了一个方向,拣个卫士较少的一方,倏的就窜了过去。

那几个卫士方自一惊,忽听得宇文通和令狐达的声音同时喝道:``这两个就是刺客!''宇文通从后面追来,令狐达在前面拦截,原来今晚正是他在离宫轮值,那些卫士就是他带领来的。

南霁云手起刀落,劈翻了两个卫士,奔上山坡,窜入树林。铁摩勒却被一个卫士追上,这卫士精于地堂刀法,抄小道绕过铁摩勒前面,忽地从斜坡上滚下来,双刀霍霍,卷地而来,削铁摩勒的双足。

铁摩勒武功虽然不弱,对敌的经验还少,不懂得应付这种地堂刀法,一时给他缠着,脱不了身。说时迟,那时快,另外两个卫士又追了到来,一个挥舞铁锤,一个使用双铜,都是沉重的兵器。

南霁云刚窜入树林,回头一望,见铁摩勒受困,一声喝道:``摩勒,这宝剑给你!''拔出段珪璋那把宝剑,反手一掷,宝剑化成了一道长虹,``唰'的一声,从那个使双锏卫士的前心穿入,透过后心。铁摩勒早有准备,飞身跳起,趁着那卫士``扑通''倒地的时候,他陡的在半空中翻了一个筋斗,头下脚上,一伸手便抓着了剑柄,将那柄宝剑拔了出来。他这几个动作一气呵成,快如闪电,使铁锤的那个卫士骤见剑光飞来,吓得心服俱寒,哪里还顾得及和他抢夺宝剑。

铁摩勒抢了宝剑,精神大振,俯冲而下,信手一挥,使地堂刀的那个家伙,正自斫来,被他宝剑一挥,双刀断为四段。铁摩勒转过剑锋一戳,又点中了使铁锤那个卫士的手腕,轰隆一声,那柄大铁锤亦已跌落,滚下斜坡。

南霁云大喝道:``令狐达,你不要命,尽管追来!''这一喝震得树叶纷落,林鸟惊飞,令狐达心惊胆战,登时如奉了圣旨一般,停了脚步,宇文通在后面叫道:``你们上呀!''

令狐达抢过一个卫士的弓箭,张弓搭箭,向南霁云射去。他犹有余悸,手指颤抖,这一箭与其说是射南霁云,不如说是为了应付宇文通才发的,箭发出去歪歪斜斜,哪能射中。

宇文通这时已经赶到,见状大怒,夺下了令狐达的弓箭,自己来射,他的功力与令狐达自是不可同日而语,强弓一拽,硬弩穿空,带着尖锐的啸声。

铁摩勒就要追上了南霁云,听得弓弦声响,他怕南霁云背了个人,闪射不便,便跳将起来,挥动宝剑,给他拨打弓箭,哪知宇文通这一箭急劲异常,结果虽然他给拨落,铁摩勒的虎口亦已震裂!

宇文通怒道:``好,你这小贼碍手碍脚,先把你杀了再说。''``嗖''的一声,第二枝箭跟着发出,逞向铁摩勒射来。铁摩勒这时已面临悬崖,前无去路,忽地大叫一声,和衣便滚下去!

南霁云大吃一惊,说时迟,那时快,宇文通第三支箭又向他射来,南霁云反手一刀,将这枝箭削断。就这样稍停一停,宇文通又已追上几步,冷笑说道:``姓南的,你还想逃吗?纵算你逃得了,这姓段的决计保全不了性命!为你设想,快快将这姓段的扔下来,我看在你是一条好汉的份上,可以网开一面。''

南霁云大怒道:``宇文通,你上来,我与你决一死战!''宇文通笑道:``我何须与你这临死的叛徒拼命!好,我善言奉劝,你不肯听,那只有陪这姓段的丧命啦!咄,看箭!''第四枚、第五枝箭连珠疾发,南霁云背着一个人,无法施展腾挪闪展的功夫,而且他不能只管自己,更紧要的还要照顾段珪璋。宇文通箭箭对准他所背的段珪璋,登时将南霁云闹得个手忙脚乱,宇文通的连珠箭一枝接着一枝,射到了第九技,这一枝是射段珪璋垂下的脚撞。南霁云弯腰拨打,宇文通乘势又是一箭,南霁云一只手要箍着段珪璋,明知这一箭射到了面前,却是无法闪避,只得将手臂一抬,用了一个``滑''字诀,箭杆贴着他的肌肉滑过,箭头铲去了他一片皮肉!

这时,南霁云亦已被迫到悬崖,弓箭手亦已纷纷赶来,要是他立即扔下段珪璋,自己或许还可以冲开一条血路。但南霁云是何等样人,这想法他连想也没有想过,就在这最危险的关头,他猛地一咬牙根,心中叫道:``段大哥,咱们要则同生,要则同死,这两条命交给天老爷啦!''心念方动,只听得宇文通的弓弦一响,一发就是三枝,南霁云猛地大叫一声,左手紧抱着段珪璋,右手的宝刀盘头一舞,步铁摩勒的后尘,也在悬崖上跳下去了。

这一着大出宇文通意外,赶到悬崖旁边一看,只见下面黑黝黝的不知有多少深。宇文通在恶斗段珪璋的时候,也曾受了两三处剑伤,虽然所伤不重,但面临悬崖,却是没有这样的胆量跳下去。心中想道:``他背着一个人跳下去,九成必死无疑!''

南霁云这样的死里求生,实在也是危险之极,幸好他有一把宝刀,利用宝刀插入峭壁,如是者接连三次,终于脚踏实地。

不过,南霁云虽然脱险,但那悬崖峭壁,尖石如刀,他滑下来的时候,也给擦伤了十几处之多,好在是他,若是换了别人,早已奄奄一息。

南霁云站稳了脚步,立即叫道:``摩勒!摩勒!''叫声未绝,只见一团黑影从茅草丛中爬出来,低低的应了一声,接着却是两声痛楚的呻吟。

南霁云知道铁摩勒是个非常倔强的少年,听得他的呻吟,不禁吃了一惊,急忙问道:``摩勒,你怎么啦?伤得很重吗?''铁摩勒咬着牙答道:``不算什么,只不过手足都脱了臼。我的段叔叔,他怎么了?''

南霁云道:``你带有火折子么?''铁摩勒道:``有!''摸了出来,擦燃火石,点起火折,递给南霁云。

火光照耀下,只见段珪璋面如金纸,遍体鳞伤,血还在不住的向外淌。南霁云心痛如绞,把段珪璋抱到山涧旁边,撕下了一幅衣衫,给他洗净了伤口,敷上了自己随身所带的金疮药。

铁摩勒跟着也爬了过来,颤声问道:``怎么样?还有得救吗?''南霁云面色沉暗,道:``血是暂时止了\ldots\ldots{}''铁摩勒迫不及待的再问道:``内伤呢?''过了半晌,南霁云低声说道:``幸好段大哥功力深湛,脉息还未断绝。咱们得给他找个大夫瞧瞧。''铁摩勒一听,霍地坐了起来,瞪大了眼睛,嚷道:``这怎么办,哪里去找大夫?''

南霁云道:``你别慌,总有办法可想。嗯,你的里衣干净吗,撕下来给我替他裹伤。''他和铁摩勒这时也已是浑身血污,只有贴身的汗衫是未沾血渍的了。

刚刚替段珪璋包扎好伤口,只见头顶上空的悬崖峭壁之间,有点点星星的火光,南霁云伏地听声,只听得有人嚷道:``我不信这三个家伙还能活命,明日再来给他们收尸也还不迟。''另一个人立即骂道:``胆小鬼,你怕跌死你么?你抓着我的腰,一个跟着一个爬下来吧!''又一个声音道:``对,食君之禄,忠君之忧,早早找到那三具尸体,也好叫咱们的大帅安心!''原来有一队卫士,正在缒绳而下!

南霁云道:``摩勒,你两条腿部伤了么?''铁摩勒道:``不,只有一边脱臼。''南霁云拉着他的手脚,给他接好脱臼,随即一剑削下一段树枝,给他当作拐杖,沉声说道:``摩勒,这是生死关头,快跑!快跑!''

南霁云背起段珪璋,铁摩勒咬牙抵痛,提了一口气,跟着南乔云跑出山谷,两人兀自不敢稍停,一口气又跑了十多里路,远远望见,路边有座孤零零的土地庙。

铁摩勒撑着那根树枝削成的拐杖,一口气飞跑了近二十里的路,实已是超出了他所能忍受的限度,南霁云听他喘气的声息越来越粗,回头一望,只见他一跷一拐的,额角上黄豆般大小的汗珠一颗一颗地滴下来。南霁云好生怜惜,凝神一听,后面并无敌骑追来,心中想道:``那些人搜遍山谷,最少也得一个时辰。''便对铁摩勒道:``小兄弟,难为你了,咱们暂且在这土地庙里歇一歇吧。''

这间土地庙想是香火冷落,檐头屋角都结着蛛网,但出乎他们的意外,在里面却有一个人!

就在土地公公的神座下面,只见一个衣衫褴楼的老汉,横伸双脚,枕着一根拐杖,睡得正沉,呼喀呼喀打着鼾,身边有个红漆葫芦,发出酒香,地上还烧有一堆火,火苗已经熄了,余烬未灭。

铁摩勒道:``看来似是一个流浪江湖的老叫化。''南霁云``唔''了一声,仔细打量,见这老汉虽然衣衫褴楼,打了许多破绽,但却洗得甚为干净,那根拐杖黑黝黝的,似乎也不是木头做的。

铁摩勒累得不堪,不管三七二十一,便坐了下来,可怜他的两条腿已是麻木不灵,一坐下来,便连移动也困难了。

南霁云踌躇了一会,只觉段珪璋的躯体渐渐僵冷,只得也坐了下来。铁摩勒道:``可惜这堆火已经熄了。''南霁云道:``待我来给他添几根柴火。''在那叫化子的身边还有几根干柴,南霁云走到他的身边,好奇心起,忍不住伸出手指,弹一弹他那根拐杖,只听得声音暗哑,非铜非铁,亦非木头,竟不知是什么东西做的!

那叫化于忽然一个翻身,霍地坐了起来,骂道:``我化子大爷正睡得舒服,好小子,你为什么吵醒我,哎、呀、呀!你、你、你是什么人?''他睡眼惺惺,骂到一半,才发现站在面前的是个血人!

南霁云赔罪道:``老大爷,我不是存心吵醒你的,我的朋友受了伤了,借这间土地庙歇歇。''那化子道:``怎么受的伤?''铁摩勒道:``碰上了强盗!''那老化子``哼:'了一声,说道:``这世道真是越来越不成话了,离长安仅有三十多里的地方,居然也有强盗伤人。''铁摩勒本来知道这话不易令人人信,但除了说是强盗之外,他还能说出什么原因?幸而那叫化只是发了几句牢骚,并未追问下去。

南霁云这时亦已是力竭精疲,百骸欲散,不过比铁摩勒稍为好一点而已,他暗地留神,只见那老叫化双眼炯炯有神,绝不类似普通乞丐。南霁云暗暗吃惊:``这老叫化不知是何等样人,要是个坏人的话,我可没有气力和他再斗了。''

那老者叫化打量了段珪璋一眼,说道:``贵友可伤得不轻啊!''南霁云道:``是啊,那些丧尽天良的强盗劈了他十几刀。''那老叫化道:``天气很冷,贵友受了重伤,恐怕会加重病况。我帮你把这堆火再燃起来吧,大家暖和一点。''南霁云见他甚为和气,稍稍放心,说道:``多谢老丈。我正想向你讨这几根柴火用用。''

那老叫化道:``彼此都是落难之人,不必客气。''顿了一顿,又笑道:``这几根柴火不够用。土地公公是应该保佑好人的,咱们不如就借他的香案一用吧,想他老人家不会见怪。''举起那根黑黝黝的拐杖,``啪''的一下,登时把那张香案打得四分五裂,铁摩勒道:``老人家你真好气力。''那老叫化笑道:``老了,不中用了,不过,这张香案,大约年纪也很大了,所以轻轻一敲,它就呜呼哀哉了!''

火堆里添了干柴,哗哗剥剥的烧起来。那老叫化道:``我这里还有半葫芦的酒,大家喝一点吧,提提神!''南霁云道:``怎好叨扰你老人家的东西?''那老叫化大笑道:``我一生都是白吃白喝人家的酒食,要是像你这样将你的,我的分得清清楚楚,我就不必干叫化子这一行啦。来,来,来,喝完了老叫化再去讨过。''南雾云只得接过他的红漆葫芦,拔了塞子,闻了一闻,他是个老于江湖的人,闻得并无刺鼻的气味,料想里面不会混有什么药物,放心喝了一口,老叫化笑道:``酒还好么?''南霁云道:``好,好!很香,很香!''其实岂上很香而已,喝下之后,不过片刻,全身便暖和起来,比十全大补的药酒更见功效,但舌尖却又尝不到半点药味,南霁云暗暗诧异,精神也恢复了几分。想道:``这老叫化倒是个有心人,我错疑他了。''

铁摩勒随着也喝了两口,连连称赞。那老叫化笑道:``你们倒是个识货的人。这是老叫化好不容易才讨来的百年老酒。让你那位受伤的朋友也喝一口吧。''南霁云这时已知道了这酒的功效,说道:``多谢老丈之赐,只是我这位朋友伤得太重,现在尚是昏迷未醒。''那老叫化道:``这容易。''捏着段珪璋的下巴,轻轻一下,就撬开了他的牙关,将葫芦中的剩酒都给他灌了下去。

那老叫化在段珪璋的背心轻轻一揉,段珪璋忽地翻了个身,``哇''的一声,一大口血狂喷出来,血色如墨,扑鼻腥臭。

铁摩勒顾不得双腿疼痛,霍地跳了起来,喝道:``你,你。你这是干吗?''原来他亦已看出这个老叫化是个异人,此际,他见那老叫化在段珪璋背心一揉,段珪璋便狂喷瘀血,一时之间,无暇思索,只道是这老叫化心怀不测,暗下毒手,是以大骂。但他刚退出一个``你''宇,便给南霁云用眼色止住了,本来是要恶骂的,却变成了一句问话的语气了。

南霁云道:``多谢老丈,他这口瘀血咯了出来,就不至有什命之忧了。''铁摩勒这才知道那老叫化志在救人,好生惭愧。

南霁云紧紧抱着段珪璋,在他耳边唤道:``大哥,醒醒,小弟在这儿,你听见我吗?''段珪璋又一口血咯了出来,猛地叫道:``史大哥,史大哥,你别走、等等我啊!''``安禄山,安禄山,你,你,你好狠啊!我段珪璋死了化鬼也要抓你!''南霁云吓得慌了,连叫:``段大哥,是我,是我,你不认得我了么?''段珪璋声音渐渐低沉,仍然断断续续地叫史大哥,骂安禄山,就像发了高烧的病人的呓语一般。

那老叫化听他骂出``安禄山''三字,跟着又报出了自己的姓名,双目陡地发出精光,脸上现出诧异的神色,指着段珪璋最后咯的那口血道:``血色已变殷红,不能再让他再咯下去了。现在应该让他酣睡一觉。''骈指如戟,轻轻点了段珪璋两处穴道,段圭湾的呓语顿时停止,便在南霁云的怀抱中,沉沉睡着了。老叫化这才吁了口气,笑道:``幸亏还剩下这半葫芦的酒给他化开了瘀血,要不然老叫化也无法救治。''

南霁云是个武学大行家,看那老叫化刚才的点穴手法,虽似轻描淡写,毫不着力,其实却是玄功暗藏,深厚之极,所以才能抓紧时机,在段珪璋瘀血化尽,新血方生之际,立即将它止住。这手点穴止血的神功,南霁云自问也有所不及。

这时南霁云哪里还有疑心,急忙说道:``多谢老前辈仁心施救,还请老前辈赐示高姓大名。''那老叫化笑道:``你不必忙着问我的姓名来历。倒是我要先问你们,你们的仇人敢情不是什么强盗,而是安禄山吧?''

铁摩勒道:``错,正是那该千刀万剐的肥猪,将我的段叔叔害成这个模样。先前我不知道老前辈是何等烊人,故此说了假话。还望老前辈恕罪。''那老叫化笑道:``你也没有说错,那安禄山虽然是三镇的节度使,其实和强盗也差不多。''

铁摩勒正要过来向他道谢,这时他已松了口气,精神支持不住,猛觉膝盖痛得有如针刺,原来是他刚才猛力跳起,扭伤了本来已经受创的关节,痛得他险些要叫出声来。那老叫化道:``小哥儿,你别动。俺老叫化除了乞食之外,还懂得几手推拿的手术,你若是信得过我,就让我替你治一治吧。''

那老叫化的推拿手术果然神妙非常,给他在手足的关节上轻轻揉了几下,再给他推血过官,铁摩勒果然痛楚立失。铁摩勒伸拳踢腿,喜哈哈地道:``你老人家真是妙手回春,灵效无比,现在我再打一架都行了!''

那老叫化却板起脸孔,正色说道:``不成!体说不能打架,连动也不能乱动。你们两人所受的伤也不轻呢,从脉象看来,你们似乎曾经从很高的地方跳下来,内脏受了震动,现在我只是治好你们的外伤,化开你们的瘀血,这内伤么,还得你们自已调治。嗯,小哥儿,你懂得吐纳的功夫么?''南霁云听他道来,有如目睹一般,暗暗惊奇,这才知道老叫化不但武功深湛,而且医术神妙。他只问铁摩勒会不会吐纳功夫,那是因为他早已看出了南霁云是个深通内功的人。

铁摩勒道:``懂得一点。''那老叫化道:``好,你们现在已经精神恢复,可以做一做吐纳的功夫了。平心静气去做,不论发生什么事情,都不要管,要做到视而不见,听而不闻的地步。好,时间无多了,你们自己练功吧。''

南霁云这才知道,这老叫化既不问他们的经过,也不肯说自己的来历,原来是要让出时间,让他们尽快恢复功力。看来他亦已预防到安禄山会有追兵。

南霁云内功深厚,做了一会吐纳的功夫,已是气机畅通,五脏六腑归回原位,就在这时,忽听得外面马嘶人语,有人说道:``这庙里有火光,咱们进去瞧瞧!''

南霁云虽然已知道那老叫化乃是异人,这时也不由得心头一震,他的功力尚未恢复,不知只这老叫化一人,能否挡得住他们?

心念未已,那一伙人已经进入庙门,果然是安禄山的追兵,而且为首的就是宇文通和令狐达!

宇文通除了邀同令狐达之外,还找了两位大内高手作伴,这两人一个叫牛千斤,一个叫龙万钧,虽然比不上宇文、尉迟,和秦襄这三大高手,却也是名列内廷卫土四大金刚中的人物,武功在令狐达之上。那山谷只有一条出口,一路追来,终于给他们发现了南、铁二人的踪迹。

宇文通一马当先,冲进庙门,忽听得一个苍老的声音骂道:``哪里来的一群王八羔子,扰得老叫化在破庙里也不得安静!''

宇文通大怒,刚要发作,忽见令狐达面如死灰,抖抖索索地说道:``小辈不知道你老的大驾驻在这儿,小辈给你老请安。''

那老叫化双眼一翻,冷冷说道:``令狐达你这小子倒抖起来啦,居然还认得我吗?''拐杖一指,接着一声喝道:``你这小子既然还认得我,应该记得我的脾气,还不快给我滚出去!''

令狐达吓得面无人色,连声应道:``是,是!''扭头便跑,宇文通怒不可遏,一把抓着了他,令狐达这才想起有个宇文通在他身边,又羞又急又惊惶,满面通红,急忙说道:``宇文大人,这位老前辈是西岳神龙皇甫先生!''

此言一出,宇文通也不禁陡然一惊。原来这个老叫化名叫皇甫嵩,喜欢游戏风尘,名列江湖七怪之一,因他是华山派的名宿,行事又有如神龙之见首不见尾,故此人称``西岳神龙''。令狐达本来是黑道出身,大约在十多年前,有一次他随师父打劫客商,他的师父心狠手辣,劫了财还想害命,碰巧遇见了皇甫嵩,他的师父挨打了三十拐杖。他那时名头未响,在黑道上只是个二流的角色,皇甫嵩责罚从宽,只打了他五拐杖。虽然如此,他挨了那五下,却足足养了半年的伤。

宇文通这时已踏进了庙门,庙中情景,一览无遗,只见南霁云和铁摩勒正在打坐,段圭璋也正躺在地上。宇文通对皇甫嵩虽然有点畏惧,但猎物就在眼前,他岂肯就此放过?心中想道:``段珪璋已是垂死的人,南霁云看来也受了重伤,这老叫化纵然了得,我和牛、龙二人联手,不信就对付不了他。何况我所听到的关于他武功的传说,都是些耳闻之言,未必就真有那么厉害?''

宇文通是一流高手,与令狐达等人自是不可同日而语,他虽然慑于``西岳神龙''的名头了却也并不怎样畏惧。当下又踏上一步,抱拳说道:``皇甫先生,咱们井水不犯河水,在下无意打扰你老,只是奉了皇命,要捉拿钦犯,不得不来,但求你老让在下交得了差。''宇文通平素目空一切,这还是他有生以来,第一次用这样客气的口物与别人说话。

皇甫嵩却不领他这个情,双眼一翻,冷笑说道:``咦,这倒奇了。老叫化虽然有时不免强讨恶化,却从未做过推倒龙床、打死太子之类的事情,怎的忽然之间变成钦犯了?''

宇文通强忍住气说道:``不是说你,我指的是这三位朋友。他们在安节度使家里放火,又杀伤了许多内廷侍卫,我身为龙骑都尉,统率宫中侍卫,不得不请这两位朋友到北街去问个明白。''

皇甫嵩搔搔头皮,说道:``这可把老叫化弄糊涂了!''宇文通愠道:``我已说得这样清楚,还有什么糊涂?''皇甫嵩道:``你瞧他们伤成这个模样,这位姓段的朋友,性命还不知能不能保得住呢!据他们说,他们是碰到了谋财害命的强盗,才给伤成这个模样的。你却说他们是钦犯,他们只是两个大人一个孩子,就敢到安禄山家中杀人放火么?哼,哼,这样的事情我不能相信,除非你把圣旨拿出来让我瞧瞧!''

宇文通怒道:``我瞧你是位武林前辈,才对你客气三分,你却和我歪缠!这案子是他们今晚刚做下来的,匆促之间,哪能请到圣旨?你瞧我的服饰,难道我这龙骑都尉,也是假的不成?''

皇甫嵩冷笑道:``难说,难说!如今的世道,就是有许多强盗冒充官府的。何况,你刚才说有圣旨,现在却又拿不出来,分明是说假话。你既说了一次假话,老叫化就不能相信你!''

宇文通气得七窍生烟,但他究竟是知道对方身份的人,正要按照江湖规矩向他挑战,随他来的那两个大内高手已沉不住气,皇甫嵩这十年来未曾在江湖上露过面,这两个人根本就不知道他的名字。

皇甫嵩话声未了,这两个人已亮出了兵器来,牛千斤使的是宣花大斧,龙万钧使的是厚背金刀,一声喝道:``凭你这老叫化也配着圣旨吗?嘿,嘿!你要圣旨,这就是圣旨!''

皇甫嵩将拐杖一横,但听得``咣咣''声响,震耳欲聋,皇甫嵩一声长啸:``这圣旨不顶事!''但见火花飞溅之中,牛千斤与龙万钧这两个水牛般粗壮的身躯,已给抛出了庙门。

宇文通这一惊非同小可,要知牛、龙二人都是著名的大力士,所练的外家功夫刚猛之极,牛千斤那柄宣花大斧重达五十六斤,龙万钧那柄厚背金刀较轻,也有四十三斤,这两件粗重的兵器斫在皇甫嵩那根拐杖上,纵使那根拐杖是铁铸的,也该断了,然而现在皇甫嵩那根拐杖却丝毫无损,反而是那柄宣花大斧和厚背金刀缺了一口,而且不过仅仅一招,牛、龙二人不但兵器毁坏。就连人也给抛出了庙门!宇文通这才知道``西岳神龙''果然是名不虚传,非但他那根拐杖是件宝物,他所显露的这手借力打力的功夫,亦已到了上乘的境界。

宇文通面色铁青,伸出手来,沉声说道:``佩服,佩服!冲着老前辈的面子,这交情我宇文通就卖给了老前辈吧!''皇甫嵩抛下拐杖,笑道:``多谢都尉大人盛情!''坦然与他握手,宇文通是点穴的大名家,双掌一按,他已使出独门点穴手法,力透指尖,中指。食指、无名指三指齐下,点中了皇甫嵩手腕的寸、关、尺三焦经脉!皇甫嵩淡淡说道:``不必客气,你请吧!''宇文通忽觉指头所触,俨如一块烧红了的烙铁一般,十指连心,痛得他禁不住``哎哟''一声,叫将出来。急忙松手,跃出庙门,走得狼狈之极,不过,比起牛、龙二人,他却又好得多了。

铁摩勒看得眉飞色舞,情不自禁地叫道:``痛快,痛快!打得好极啦!哎哟,哟!''原来他内功的根基还浅,正在气贯丹田的时候,由于心情激动的缘故,真气忽然走歪,几乎窒息。

皇甫嵩眉头一皱,责备他道:``你这娃儿怎么不听我老人家的话,叫你不要多管闲事,你偏要管!''一面责备,一面给铁摩勒施展推拿的手术,帮助他把真气纳入丹田。

这时敌人都已逃走,破庙里一片寂静,皇甫嵩用拐杖拨拨火堆,似乎是在思索什么似的,不时的望出门外,忽地自言自语道:``天都快要亮啦!''

南霁云这时已气透重关,功力即将完全恢复,他见皇甫嵩神情有异,正想和他说几句话屋甫嵩忽然又站了起来,郑重说道:``等下不论发生什么事情,你们两位都不能多管!''这话他已经说过一遍,现在再说,口气也比以前严厉得多。南霁云心中一动,想道:``他为什么要再三嘱咐?难道还会有什么意外的事情发生么?''

正是:方喜追兵才击退,一波未息一波生。

欲知后事如何?请听下回分解------

\chapter{第 八 回 为友为仇疑未释
是魔是侠事难明}\label{ux7b2c-ux516b-ux56de-ux4e3aux53cbux4e3aux4ec7ux7591ux672aux91ca-ux662fux9b54ux662fux4fa0ux4e8bux96beux660e}

南霁云心念方动,忽听得外面又传来了叮叮咣咣的马铃声响,南霁云只想到安禄山这一方面,想道:``连宇文通都已败阵而逃,他们还能派出什么能人?纵使再多来几个,也绝对不是皇甫嵩的对手。咳,上了年纪的人,大约说话就不免罗唆,我已见识过你的武功,还何劳你再三嘱咐?''

马铃声越来越近,皇甫嵩盘膝坐在地上,脸上的神情非常奇怪,好像在焦急之中又带着几分愁苦。南霁云已听出只是一人一骑,不禁大为诧异,心道:``皇甫嵩仅仅一招,就打发了宇文通,还有什么人能令他惊骇。''

南霁云正在猜疑,忽觉眼睛一亮,只见一个白衣少女走入门来!南霁云一直以为来者定然是个雄赳赳的武夫,哪知却是个美艳如花的娉婷少女,当真是大出意料之外!

那少女进入庙门,游目四顾,见有一个重伤的人躺在地上,两个浑身染血的人正在打坐,亦是好生诡异,但显然她的目标不是段-璋,只见她扫了一眼之后,眼光就转注到皇甫嵩的身上,一声喝道:``皇甫老贼,今日是你的死期到了,还不快起来领死!''

皇甫嵩抬起头来,看了那少女一眼,缓缓说道:``你是夏姑娘吗?我早预料到你要来找我的了,只是我素来与你无冤无仇,现在才是第一次见面,你为什么定要杀我?''

那少女接剑斥道:``奸邪淫恶之徒,人人得而诛之,定需要你我之间有冤仇吗?''

此言一出,南霁云虽然正在运功收息的时候,也不禁大吃一惊。要知皇甫嵩虽然有时行径怪僻,但在江湖上却是誉多于毁,即在南霁云的心目中也把他当作侠义道的人物,而这少女却骂他是奸邪淫恶之徒,南弄三几乎不敢相信自己的耳朵。

侠义道中的人物,被人骂为``奸邪淫恶'',那简直是最大的侮辱,南霁云以为皇甫嵩定要暴怒如雷,哪知又是大大出乎意料之外,只听得皇甫嵩深深说道:``对你说这样话的是什么人?''那少女道:``你管不着!你臭名远播,难道我没有耳朵吗?''皇甫嵩道:``你不说,大约我也猜得到几分。我再问你,说这话的,是不是一个你最相信他的人?''那少女怒道:``我来不是听你盘问的,哼,哼,你想套出我的话来,然后去暗杀说这话的人是不是?你别做梦啦,今天我就要你丧命在我剑下。''

皇甫嵩又问道:``要把我杀掉,这是你自己的意思,还是听别人指使的?''那少女似乎很不耐烦,斥道:``你还想花言巧语、拖延时候么?''皇甫嵩道:``不,我只是不愿做个不明不白的冤鬼罢了。你要杀我,也该让我死得甘心呀!''那少女忍着气道:``是我自己的意思怎么样?是听别人指使的又怎么样?''皇甫嵩道:``若是你自己的意思,你应该有足够的证据将我的罪恶数出来,这才能叫我心服。''

这也正是南霁云在心里想说的话,但见那少女怔了一怔,似乎她也数不出皇甫嵩有什么真凭实据的罪恶。皇甫嵩又接着说道:``若是别人要你杀我的,你就回去对那人说吧,世上有许多事情往往是难分真假的,叫他忍耐些时,自有水落石出之时,我皇甫嵩一生也许曾做过坏事,但`奸淫邪恶'这顶帽于,却绝对套不上我的头上!''

那少女怒道:``我不相信你的鬼话!我只知道你是个无恶不作的魔头!哼,哼,你这魔头居然也会怕死么?你再巧言辩解也没有用,还不快起来领死!''

皇甫嵩笑道:``我若是怕死,也不会约你到这里来了。''那少女道:``那,既然如此,为何还不动手?是不是还要等多几个帮手?''皇甫嵩道:``我平生从未要过帮手!''那少女道:``好,你有帮手也好,没有帮手也好,我只凭这口剑与你决一死生!''

皇甫嵩道:``你要杀便杀吧,我是绝不与你动手的。''那少女呆了一呆,道:``我不杀手无寸铁之人!赶快拿起你这根拐杖吧!''皇甫嵩道:``我说过不动手便不动手,要杀嘛你就杀,你若不杀我就走!''那少文显然是要照江湖规矩与他过招,然后将他杀掉的,现在皇甫嵩拒绝和她动手,倒令她一时之间失了主意。

皇甫嵩又缓缓说道:``现在我已确知你的来历,也知道要你杀我的是什么人了。我失了性命,若能平息那人的一口怨气,也是一件好事。好了,话尽于此,你再不杀我,我老叫化可要走啦!''

那少女咬了咬牙,拿起了地上那根拐杖,喝道:``起来,接拐!''皇甫嵩拿了拐杖,却又丢过一边,笑道:``己所不欲,勿施于人。我想,你也不欢喜别人强迫你做你所不愿意做的事吧!''那少女再咬了咬牙,一抖剑锋,喝道:``好,你想用撒赖的方法逃命,我偏不中你的计,我非杀你不可!''这次似是的确下了决心,但见她长剑一展,唰的一声,立即向皇甫嵩的胸膛刺去!

眼看皇甫嵩就要命丧剑下,忽见一道匹练似的白光,疾卷过来,``恍''的一声,格开了少女的长剑。

皇甫嵩叹口气道:``南大侠何必多事?''\,'南霁云却向那少女喝道:``姑娘,你杀人也得有个道理,你指斥皇甫先辈是奸邪淫恶之徒,却又说不出个所以然来,我姓南的听了先不服气。''

那少女收了氏剑,只见剑锋已损了一个缺口,少女勃然大怒,喝道:``你帮这魔头说话,料你也不是个好人!好呀,你不服气,我先把你杀了再说!''

那少女只当南霁云是皇甫嵩的党羽,下手绝不留情,但见她剑锋一颤,倏地飞起三朵剑花,竟然在一招之内,连袭南霁云三处大穴。南霁云这时也动了火,横刀疾劈,想一下就把她的长剑削断,这少女已知他手中是把宝刀,避免和他硬碰,南霁云一刀劈山,正要喝个``着''字,那少女的剑势忽然改变了方向,来得奇幻无比,南霁云也不由得吃了一惊,幸而他招数未曾使老,急忙一个盘龙绕步,回刀护身,使听得``嗤''的一声,南霁云的衣角已被她的剑锋穿过!

说时迟,那时快,那少女一剑得手,第二剑第三剑紧接而来,宛如暴风骤雨!

南霁云这时已完全恢复了功力,但在那少女凌厉的攻势下,急切之间,也只有招架的份儿。但他守得沉稳异常,那少女也攻不进去。

铁摩勒得皇甫嵩之助,真气已纳入丹田,这时功力亦已恢复了七八分,便守护在段-
璋的身边,凝神观战。但见那少女出手迅若雷霆,奇招妙着,层出不穷,铁摩勒年纪虽小,却是见过上乘剑法的人,这时看了,也不禁有点惊心:``单以剑术而论,只怕这少女的剑术也不在我的段叔叔和精精儿之下。''

南霁云展开一套游身八卦刀法,身法步法紧守着``八门''\,``五步''的方位,丝毫不乱。战到分际,他对少女的剑术路数,已渐渐有些熟悉,忽地大喝一声,刀光暴起,有如千丈洪波,溃围而出!那少女给他逼得连连后退,铁摩勒看得眉飞色舞,禁不住又失声叫道:``妙啊,妙啊!''这时,他已做完了吐纳的功夫,不怕真气再走歪了。但皇甫嵩仍然瞪了他一眼。

就在铁摩勒失声叫好的当儿,那少女的身法剑法,也突然一变,但见她衣袂飘飘,在刀光剑影之下,俨似穿花蝴蝶,和南霁云对抢攻势,当真是:一招一式,毫不放松,分寸之间,互争先手。激烈无比!

那少女见南霁云意态轩昂,武功超卓,暗暗称奇,忽地虚晃一剑,锐声问道:``你是何人?具何如此身手,为何甘心做老贼的爪牙?''

南霁云一声长啸,横刀封住门户,朗声答道:``大丈夫行不更名,坐不改姓,魏州南霁云是也!请问姑娘尊姓大名?为何要杀皇甫先生?''

那少女似乎吃了一惊,急忙问道:``你便是魏州南八么?''南霁云道:``正是在下,姑娘有何见教?''

那少女现出一派惶惑的神情,原来自段-璋销声匿迹之后,这十年来江湖上最著名的游侠便是南霁云,这少女也早已闻得他的大名,却想不到他仅是三十岁左右的中年男子。

那少女想了一想,说道:``南大侠,你少管这闲事吧!''南霁云道:``杀人是件大事,岂可当作等闲,你要杀人,须得说出个道理来,否则南某不能不管!''

那少女满面涨红,厉声说道:``南霁云你空有大侠之名,却分不清是非黑白,你当这老贼是何等样人?''南霁云道:``皇甫前辈是侠义中人,谁不知晓?你辱骂前辈,却又说不出个道理来,先就不该!''

那少女冷笑道:``皇甫老贼欺世盗名,其实却是暗中作恶的魔头,你枉称大侠,却给他骗了!''南霁云道:``你说他作恶多端,有何凭证?''那少女双眉一坚,好像本来不想说的,现在始下了决心,毅然说道:``我母亲就是证人!她说的话我不能不信!她曾亲眼看见这个老贼杀了人家的丈大,夺了人家的妻子,我骂他是奸邪淫恶之徒,难道骂错了吗?我是奉了母命来除奸的。南霁云,你素有侠义之名,今晚我不必要你助我除奸,但你最少也该袖手旁观,不应拦阻。''

南霁云大吃一惊,不由得把眼光向皇甫嵩瞥去,只见皇甫嵩在微微叹息,南霁云心头一震,暗自想道:``难道他果真做过这少女所说的坏事?''再留神看时,皇甫嵩却并没有显出些微愧怍的神色,他的叹息似乎只是一种怜悯,一种无可奈何的感伤。南霁云久历江湖,眼光何等锐利,心里不禁疑云大起,想道:``瞧这神情,皇甫嵩定是受冤枉的,但他为什么不分辩?为什么甘心让那少女所杀?看来这里面定然有更复杂的原因,皇甫嵩不愿为外人道!''

那少女见南霁云仍然横刀挡住她的去路,柳眉一竖,怒声说道:``我已说得清清楚楚,你还要拦阻我吗?''南霁云道:``我听来还有许多不明白的地方,你说皇甫前辈曾于过杀夫夺妻的恶行,那对夫妻究竟姓甚名谁?另外有何人证物证?当时的经过情形怎样?\ldots\ldots{}''那少女怒道:``这是我母亲告诉我的,我母亲说的决不会是假话,还何须什么另外的人证物证?''

南霁云心道:``看来只怕她母亲也还瞒着一些事情,未曾对她说得一清二楚。''当下将宝刀一挥,架着了少女攻过来的长剑,沉声说道:``你相信你的母亲,我却相信皇甫前辈。有我在此,你今晚想要杀人那是万万不行!依我说,你不如暂且罢手,留下姓名住址给我,待我办完一桩事情之后,至迟在三个月之内,必定登门造访,面见令堂,说个明白。''

那少女大怒道:``你既不相信我的母亲,你还见她做什么?哼,你别以为你有点声名,我母亲也还未必肯见你呢!哼,你让不让开?你再不让开,休怪我不客气了!''剑法一展,登时又是暴风骤雨般的强攻过去。

南霁云当然不肯退让,这时他对少女的剑法已略为熟悉,虽然未能取胜,却已稍稍占了上风。但在他心里,却也暗自叫了一声:``惭愧!''想道:``要是我不仗着这把宝刀,只怕当真不是她的对手。''

其实南霁云的功力也要比那少女略胜一筹,那少女强攻不下,额头已经见汗,而南霁云则仍是神色自如。那少女自知不敌,愤然说道:``你为什么拼了死命要护这个老贼?''

南霁云道:``一来我相信皇甫前辈不是坏人,二来他于我又有救命之恩,你要杀他,我焉能不管?''那少女怔了一怔,说道:``什么救命之恩?''

恰在这时,段-璋忽然又在梦中叫道:``史大哥,史大哥!我在这儿,我在这儿,你还认得我段-璋么?''

那少女忽地大叫一声,倏的向段-璋所躺的方向掠去,铁摩勒守护在段-
璋身旁,见她突如其来,大吃一惊,急忙举起宝剑便削,大声喝道:``好狠的女贼,我段叔叔已伤成这个模样,你还要侵害他么?''

那少女将长剑一引,使了一个``粘字诀'',将铁摩勒的宝剑引开,反手一招,又把南霁云的攻势解去,喝道:``且慢动手,他是谁人?''南霁云道:``幽州大使段-璋,你听过这个名宇么?''

那少女陡然一震,急忙问道:``他果然就是段-璋么,那么还有一个叫做史逸如的人呢?''

南霁云也是陡然一震,急忙问道:``姑娘,你认得史逸如的么?''那少女道:``你别问我,你只说史逸如他现在怎么样了?''

南霁云道:``史逸如么?他已被安禄山逼得自尽了!''那少女面色一沉,再问道:``那么段大伙是否在安禄山家坐受的伤?''南霁云失声叫道:``姑娘,你放情是知道他们这桩事情的?不错,段大侠正是为了要救他这位姓史的朋友,在安贼家中以寡敌众,因而受了重伤的。幸亏遇到皇甫前辈,给他急救,要不然只怕他早已没命了。''

南霁云顿了一顿,接续说道:``我们昨晚也是在安贼家中厮杀过来,叮惜我们到迟了一步,救不了史逸如\ldots\ldots{}''那少女插口道:``嗯,我明白了,也幸亏你们,所以段大侠才不至落在安贼手中,是么?''

铁摩勒嚷道:``对啦,你猜得一点不错。再告诉你吧:南大侠和我所受的伤也是这位皇甫前辈治好的,皇甫前辈还给我们打退安禄山的追兵,你怎能说他是个坏人?''

那少女现出一派迷惘的绅色,似乎对皇甫嵩的敌意已减了几分,想了一想,忽地又再问道:``那么史逸如的妻女呢?''

南霁云任了一怔,道:``我不知道。''那少女道:``胡说!你怎能不知道?''她哪里知道,段-璋根本就来曾将这件事告诉南霁云,铁摩勒拉南霁云去救段-
璋之时,虽然约略说了一些却也没有提到史逸如的妻女。

铁摩勒虽然不高兴这位少女的态度,但见她这样关心段、史二家之事,料想她也不是一个坏人,便答道:``那姓史的妻女我们没有见到,多半还是被囚在安禄山那儿,你想知道她们的消息,有胆的话,可以找安禄山问去!''

那少女被铁摩勒一激,面色陡变,忽地长剑一指,对皇甫嵩道:``看在你救段大侠的份上,今晚暂巳饶你不死,不过,以后我若是再查到你的恶行的话,我还是要和你算帐。''皇甫嵩苦笑一声,似乎想说话却又忍着不说,那少女倏地一个转身,跃出庙门,跨上马背,扬声叫道:``我叫夏凌霜,我的名字你可以说给段大侠知道。''马铃叮当,待她这几句话说完,铃声亦已渐远渐寂了。

铁库勒满腹狐疑,问道:``皇甫前辈,这姓夏的女子武功虽强,却也不见得能胜过宇文通多少,你可以轻易的打发宇文通,她绝不是你的对手,你却怎么这样怕她?''

皇甫嵩苦笑道:``叫化子受气受骂,那是很平掌的事情,算不了什么。唉,老叫化倒愿丧生在她的剑下,省得她去另外杀人。''铁摩勒听他说得奇怪,正想再问,皇甫嵩又道:``老叫化已经说得多了,这件事实是不愿再提。南大侠,你要是信得过老叫化的话,这件事请你也不必再管了。''

南霁云知他有难言之隐,心中想道:``听他说来,似是代人受过。但`奸邪淫恶'这个罪名是何等重大,若是代人受过,别样事情犹自可说,却怎能背上这个恶名?''但皇甫嵩话已至此,南霁云和铁摩勒虽然疑团塞胸,却也不便再问了。

皇甫嵩道:``天已亮了,老叫化还有旁的事情,可要先走一步了。段大侠大约再过两个时辰,就可以醒来。这里有一瓶药丸,你每天给他服食三次,每次一粒,吃完了这瓶药丸,大约他也可以恢复如初了。''

南霁云接过瓶子,瓶子里有二十粒药丸,照每天三粒来算。不出七天,段-璋便可以恢复武功。南霁云道:``老前辈再生之德,我们不知该如何报答,老前辈不知有什么话要留给段大侠么?''

皇甫嵩笑道:``老叫化时常受别人的恩惠,要说报答,哪报得了这许多?何况,你刚才救了我的一条性命,也算报答过了。''顿了一顿,忽又说道:``段大侠是个恩怨分明的人,他醒来之后,你不要说这药是老叫化给的,免得他挂在心上。''铁摩勒道:``这可不成,他若问起是谁救他性命,我们总不能不告诉他。''皇甫嵩道:``这样好了,止血疗伤的事情可以告诉他,这药丸嘛,就当作是南大侠随身携带的好了,凡是习武的人,谁都有秘制的膏丹丸散,不过效力不同罢了。若说是老叫化送的,反而不好。''南霁云见他说得甚为郑重,不禁又起了一重疑云;铁摩勒却笑道:``给他止血疗伤的也是你,他知道了,岂不是也要挂在心上吗?''皇甫嵩想了一想,说道:``好吧,那么我也向他请托一件事情,算是谁也不沾谁的恩惠。''南霁云道:``什么事情?''皇甫嵩除下了一枚铁指环,套在段-
璋的指上,说道:``拜托你们向段大侠求情,日后要是他遇见一个人,那个人带有一式一样的铁指环的话,请他看在我的份上,给那个人留点情面。''

铁摩勒心道:``这老叫化不如弄什么玄虚?''这时亦自暗暗起疑,但他是在黑道中长大的孩子,深知江湖避忌,当下不敢再问,恭恭敬敬地答道:``老前辈放心,这几句话我一定给你转达。''

皇甫嵩拿起拐杖,正要走出庙门,忽又停住,回头对南霁云道:``我几乎忘记了一件事情,上月我在涿县曾碰见你的帅父。''南霁云问道:``他老人家可有什么话说?''皇甫嵩道:``他说他本要到睢阳去的,因为有旁的事情,行期要延至下月中旬了。他和我谈起了你,说你这几年在江湖上行侠仗义的行为,他都知道,甚感欣慰。他问我认不认识你,我说名字早已知道,人还未见过面。他告诉我,你在这几天可能要到睢阳,并对我说道:``睢阳太守张巡是当今一个人物,老叫化你要是没有旁的事情,不妨到睢阳走走。我知道你素来欢喜后辈,顺便也可以见见我那个徒儿。要是见着他的话,就将这个消息告诉他。他若是在五原那边另有事情的话,就不必在睢阳等我了。哈哈,想不到我未到睢阳,却在这个破庙里和你们巧遇。''

南霁云这才想起,他们踏进这庙门的时候,皇甫嵩对他似乎特别留意,心道:``怪不得他未问我们的来历,就肯替我疗伤,敢情是师父早已将我的相貌告诉他了。''

南霁云本来正在担着一重心事:段-璋重伤未愈,铁摩勒当然要护送他前往窦家,铁摩勒虽然精明能干,武功在后辈中也是少有的人物,但究竟还是个大孩子,叫南霁云怎放心得下?现在听说师父要下月中旬才去睢阳,南霁云便也改变了主意。

皇甫嵩去后,南霁云说道:``摩勒,我不去睢阳了,陪你到窦家寨走一走吧。安顿了段大侠之后,要是你没有旁的事情,我再和你到睢阳去见我的师父。''铁摩勒大喜道:``这敢情好!不过,郭子仪不是有一封信要你带给张巡么?你护送我们,会不会误了你的事情?''南霁云道:``那封信迟一个月也不打紧,那是郭令公托我便中带去,与张太守相约,准备万一祸患起时,彼此好有个照应。其实他们二人彼此仰慕,即算没有这封信,有事之时,也必然是患难与共,同心为国的。''

铁摩勒道:``趁这天色尚未大亮,已待我去先取两件替换的衣裳。''南霁云知比要去施展神偷妙手,笑道:``你这小贼可得当心,别给人家捉住了。''铁摩勒满伸气地答道:``那是绝对不会有的事情。''

哪知铁摩勒一去就去了半个时辰,南霁云忐忑不安,心道:``莫非真应了我的话儿?''正自心焦,忽听得门外车声辘辘,南霁云一瞧,心头大石放下,原来是铁摩勒驾着一辆驴车回来了。

南霁云道:``你怎么将驴车也偷回来了?''铁摩勒道:``驴车不是偷的,是用一个金元宝换来的。''南霁云笑道:``哈,你倒阔气,随身还带有金元宝呢!''铁摩勒道:``那金元宝不是我的,是一个富户的。我到他家里偷了几件衣裳,顺手牵羊,又拿了几个金元宝,再赶到车行,天刚朦亮,我等不及将他们唤醒,扔下了一个金元宝,套了驴车便走。这头驴子不听使唤,我赶它出门时,它大声嘶叫,这一下才把那些人吵醒了。他们起初也是纷纷叫喊`捉贼',我在车上向他们扬手道:``我不是贼,我是财神。'这时他们大约已发现了那个金元宝了,于是骂声登时变作欢呼,也没有人再赶来了。''说罢哈哈大笑。笑罢,说道:``其实贼还是赋,不过,我是专偷富户,不偷穷家罢了。一锭金元宝够买十辆驴车,那班脚夫,赔了一辆驴车给车行主人,还可以发点小财。''

这时,天色已经大亮,铁摩勒早就换了干净的衣裳,南霁云在他说话的时候,也将衣裳换了。两人将段-
璋抬上驴车。这辆驴车是铁摩勒拣的车行中最好的驴车,车内铺有软垫,正好给段-璋躺着。

南霁云驱车疾走,一个时辰,已到了临潼县境,后面并无追兵,这才松了口气。南霁云是个成名的侠士,铁摩勒则是绿林世家,两人谈论江湖佚事,谈得津津有味。南霁云笑道:``你小小的年纪,就练成了这副神偷妙手,将来那还了得!只怕没有人敢再开镖行了。''

铁摩勒笑道:``我还差得远呢!你知道天下第一神偷是谁?''南霁云道:``是三手神丐车迟吗?''铁摩勒道:``不,三手神丐早已给人比下去了。现在天下第一神偷是空空儿,他曾和三手神丐打赌,三手神丐偷了宁王一枝玉萧,他却从三手神丐的手上,将那枝玉萧再偷出来,而且这还不算,他偷了再还,还了再偷,接连三次,令得三手神丐五体投地,只好让他将那枝玉萧交回宁王领赏。现在`妙手空空'这四个字,黑道上几乎是无人不知!''

南霁云道:``我也早听得空空儿的大名,但只知道他的剑法高强,可惜还未会过。''铁摩勒笑道:``你这次到我义父的家中,说不定可以碰见空空儿,就是见不着空空儿,他的师弟精精儿你是一定可以见到的。''南霁云觉得奇怪,正要问他是何原故,忽听得段-璋``哎哟''一声叫了起来。

南霁云道:``好了,他已知道疼痛了。''过了片刻,段-璋张开眼睛,``咦''了一声道:``南兄弟,怎么是你?我的史大哥呢?这是什么地方?我是在做梦么?''他重伤之后,昏迷了半夜,现在虽然开始苏醒,却显然还在混乱之中。

南霁云道:``段大哥,咱们脱脸了,这里已是临潼县的地界了。''段-璋渐渐想起了昨晚的事情,对安禄山的痛骂、和宇文通的激战、史逸如的自尽、南霁云的冲进重围\ldots\ldots 最后浮起的景象是宇文通的那枝判官笔正向他的胸前插下;而南霁云也正向着他奔来,以后就不知道了。一幕一幕的情景在他脑海中闪过,这是真的?还是一场恶梦?

驴车正在山道上奔驰,颠簸异常,段-璋突然被抛了起来,牵动伤口,感到十分疼痛,段-璋明白了,他刚才所想起的那些事情都是真的,并不是梦!

南霁云紧紧抱着他,只见他面色灰白,两眼无神,一片茫然的神色,过了片刻,忽地喃喃说道:``史大哥,你死得好惨啊!都是做兄弟的害了你!''声音低沉,并未大叫大嚷,眼中也没有滴下眼泪,但那声调、那神情,却令人心头颤震,在他说话的时候,空气都好似冷得要凝结了似的,实是比大叫大嚷、痛哭流涕更要沉痛百倍!

南霁云低声说道:``段大哥,你要保重身体,给史义士报仇要紧!''段-璋瞿然一省,耳朵边响起了史逸如临死的说话:``段大哥。与其留我报仇,不如留你报仇!我先走一步了,你为我保存身子,拼命杀出去吧。''又想起了史逸如的妻子卢氏夫人和她初生的女孩还陷身虎口,段-璋咬了咬牙,忍着了眼泪,似是向史逸如的在天之灵发誓道:``对,史大哥,我要听你的吩咐!''接着又道:``南兄弟,难为你了,为我冒这样大的危险!摩勒,你这好孩子,你虽然不听我的话,现在我也不责怪你了。''

南、铁二人见他渐渐安定下来,这才稍稍放心。段-璋试行运气,但觉四肢麻木,浑身之力,一口气怎么也提不起来,不禁叹口气道:``原来我竟然伤得这么重了!几时才报得了仇?''铁摩勒道:``姑丈,你放心,皇甫嵩老前辈说,过了七天之后,你就可以恢复如初。''段-璋怔了一怔,忽地问道:``皇甫嵩?是江湖七怪之一的西岳神龙皇甫嵩吗?''问话的语气和脸上的神情都显得有几分异样!

铁摩勒道:``正是,我们的伤都是他老人家治好的。''段-璋道:``这么说,敢情我这条命也是他救活的了?''铁摩勒道:``是呀,当时你流血不止,内伤又重,是他给你闭穴止血,然后给你推血过宫,又灌了你半葫芦的药酒。''段-璋面色铁青,过了一会,始叹口气道:``想不到我竟然胡里糊涂的受了他的救命之恩,欠下这笔人情,令我好生难受!''

铁摩勒给他的脾气吓得呆了,心里奇怪到极,一时之间,不敢说话。南霁云问道:``可有什么不对么?''段-璋道:``南兄弟,你拼死救我,我感激得很。但你我是同道中人,我受了你的恩,心里坦然,这个皇甫嵩么?我受了他的恩,将来可不知怎么好了?''

南、铁二人大吃一惊,骇然问道:``这位西岳神龙不也是侠义道吗?''段-璋道:``南兄弟,你出道比我迟了十年,难怪你不知道他的底细,在我那个时候,他也是誉多于毁的。''南霁云急忙问道:``誉多于毁?照你这么说,皇甫嵩岂不是也曾于过坏事的了?为什么我听到的却都是说他好话的呢?甚至我的师父也曾对他下这个评语,说是皇甫嵩这个人行径虽然右点怪僻,却还不失为侠义中人!''

段-璋道:``想来那是他老人家隐恶扬善的缘故。皇甫嵩这个人的确曾做过许多好事,而且是好的多过坏的,但他做的坏事,却也委实令人发指!''

南霁云面色也全都变了,道:``段大哥,你可以说几桩来听听吗?''段-璋道:``好,我先说他所做的几十年来脸炙人口的好事,他曾经劫了卢龙、许州两个节度使的赃款,用来赈济黄河灾民;他曾独力除去燕、赵五霸;他曾给崆峒、燕山两派排难解纷,消弭了武林的一场灾难\ldots\ldots{}''南霁云打断他的话道:``这些事我都已知道了,你说说他所干的恶行听听。''

段-璋道:``恶行么也有几桩伤天害理的事情,有一年有几个炼丹的修士去天山采雪莲,归途中被他劫杀,只逃出一个人。有一年他庇护一个著名的采花贼绰号叫做赛赤风的,把少林派的定一禅师打伤了,少林派本来要找他算帐的,不久就发生了他用劫来的巨款救济灾民的事情,少林派念他这件功德,才放过了他,只把赛赤凤除掉。''

说到这里,铁摩勒忽然插口道:``他可曾干过杀人之夫,夺人之妻的坏事么?''段-璋大为诧异,问道:``你怎么也知道这件事情?''

南霁云这一惊更甚,失声叫道:``当真有这样的事情?''段-璋道:``这件事直到如今还是疑案,不过,据我看来,九成是那皇甫嵩干的!''南霁云定了定神,问道:``究竟是怎么一回事?''

段-璋道:``这件事发生在二十年之前,当时有一对名闻四方的少年游侠,男的名叫夏声涛,女的名叫冷雪梅,他们联手干了许多侠义的事情,志同道合,两情悦慕,于是订下了白头之约。在他们成婚之日,热闹非常,江湖中人,不论识与不识,都纷纷前来,向他们道贺,谁不羡慕他们是一对武林罕有的佳偶?我和新郎新娘都是稔熟的朋友,当然也在贺客之中。

``岂料这对人人羡慕的新婚夫妇,就在他们洞房花烛之夜,却遭遇了意想不到的惨祸。我还记得清清楚楚,那晚我和几位也是新郎新娘的知己朋友,闹了洞房之后,兴犹未尽,聚在前厅饮酒,大家都已有了几分醉意,忽听得洞房里传出一声尖锐而凄惨的叫声,我的酒意登时醒了,顾不得礼仪,立即便冲进洞房去看,只见新郎己倒在地上,而新娘却不知去向!

``我连忙去扶起新郎,可怜他已受了重伤,一句话也说不出来,我在他耳边连问了几声:``谁是凶手,谁是凶手?'他还认得我是他的知己朋友,望了我一眼,伸出颤抖的手指,蘸了身上的血,在地上歪歪斜斜的划了几下,凶手的名字尚未写得齐全,便断了气!唉,他临死的眼光,我永远也不会忘记,那是恳求我替他复仇的眼光!

``我仔细辨认他所写的血字,第一个是`皇'字,第二个字只有两划,一横一竖,似十字而又不似卜字,'卜'宇的一横一坚是差不多长短的,而他划的这两划却是横的短,直的长,世上根本没有姓`皇'的人,个待我出声,便已有人嚷道:``凶手定然是皇甫嵩。''

南霁云颤声说道:``只凭这条线索似乎还未能说是证据确凿?''

段-璋道:``不错,有许多人也和你一样,不敢相信凶手便是皇甫嵩,他们猜疑或者这个`皇'子是指事帝派来的人呢?因为夏声涛与当时的一个内廷侍卫名叫公孙湛的有点私仇,说不定是公孙湛干的。''铁摩勒低声说道:``唔,这也有点道理。''段-璋大声道:``不,这完全没有道理!''

正是:聚讼纷纭难破案,刀光血彰事堪疑。

欲知后事如何,请听下回分解------

\chapter{第 九 回 廿年疑案情天恨
一剑惊仇侠士风}\label{ux7b2c-ux4e5d-ux56de-ux5effux5e74ux7591ux6848ux60c5ux5929ux6068-ux4e00ux5251ux60caux4ec7ux4fa0ux58ebux98ce}

段-璋接着说道:``'公孙'和`皇甫'这两个姓都是复姓,公字的笔划要比皇字简单得多,你试想夏声涛当时已是临死之际,他何必要舍`公'字不写而写`皇'字?若然公孙湛是凶手的话,他只写一个`公'字自然有人明白;而且他也不需绕个大弯,不指明`公孙'而却指他是`皇帝'的人。再者夏声涛和冷雪梅的武功都在公孙湛之上,公孙湛不可能将夏声涛杀掉并且将冷雪梅夺去。那些人替皇甫嵩辩解,不过是爱惜他的侠名,想为他开脱罢了。''

铁摩勒低下了头,他的心思正是和段-璋所说的``那些人''一样。

南霁云却仍是疑团重重,心中想道:``听段大哥的说法,皇甫嵩所干的好事很多,赈济灾民更是一件大功德;另一方面,他所干的坏事也确是令人发指。这两种极端相反的行为,依理而言,不应当发生在同一个人的身上。再者,我的师父也是个善恶分明的人,皇甫嵩若当真干过那些恶行,我师父岂能只为了`隐恶扬善'的缘故,从不向我提及,而且他还和皇甫嵩结交。''

段-璋似乎猜到他的心思,顿了一顿,又再说道:``这件事发生在二十年之前,事情过后,皇甫嵩就很少在江湖露面,偶尔也听到关于他的事情,十九是行侠仗义的事,纵然也有一两桩罪恶,但却是不算得严重的罪恶。因此,这也就是我迟迟未曾替好友报仇的原因。不过,要是给我查明确实的话,这笔帐我还是要和他算的。''

铁摩勒道:``已经有一个人为了此事要和他算帐了。''段-璋身子一震,睁大了两只眼睛问道:``谁?''铁摩勒道:``是一个二十岁左右的少女,名字叫夏凌霜。她说你也许会知道她。''

段-璋急忙问道:``相貌长得怎么样?她在什么地方与皇甫嵩遭遇?这件事是你听来的还是亲眼见的?''铁摩勒道:``就是在刚才的破庙之中。''接着便把事情的经过原原本本的告诉了段硅漳,并把她的相貌也详细的描绘了一番。

南霁云低声说道:``我不知道内里牵涉到夏大侠这件案子,不过,皇甫嵩救了我们三个人的性命,即算知道了,但在案子尚未水落石出之前,我也还是要挡住那少女的。段大哥,你可怪我么?''

段-璋摇摇头,默默不语,半晌,始在口中轻轻念道:``夏凌霜,夏凌霜\ldots\ldots{}''脸上现出一派迷惑的神情,同时脑海里现出另一个少女的影子,那是冷雪梅,铁摩勒所描划的那个少女的容貌,正是和冷雪梅一样。

原来段-璋对冷雪梅曾有过一般情慷,他和冷雪梅的结交还在夏声涛之前。可是段-
璋虽然对冷雪梅十分倾慕,冷雪梅对他却是若即若离。后来冷雪梅认识了夏声涛,两情契合,渐渐变成了她和夏声涛在一起的时候多,而和段-
璋在一起的时候少了。段-璋不久也就明白了冷雪梅爱的是夏声涛。他是个光明磊落的人,当然不会作梗,而且为了冷雪梅的缘故,把夏声涛也当作兄弟一般。

夏声涛惨死,冷雪梅失踪之后,段-璋极是伤心,直到过了十年,方始和窦线娘结婚,夫妻俩虽然思爱非常,但段-璋对冷雪梅却还是保存着一份深沉的怀念。

这时段-
璋听了铁摩勒所描绘的夏凌霜的面貌,和冷雪梅十分相似,不禁神思迷惘,往事历历,重上心头,记起了他少年时候为冷雪梅所写的两句诗:``雪冷梅花艳,凌霜独自开。''心中想道:``莫非这夏凌霜就是冷雪梅的女儿?她还记得我的诗句,是以给女儿取了这个名字?但夏声涛已经死了,何来这个姓夏的女儿?''他在百思莫解之中却又感到深心的喜悦,``要是夏凌霜当真是冷雪梅女儿的话,她岂非还在人间?''

铁摩勒道:``姑丈,皇甫嵩有一枚钦指环给你。就是现在套在你中指上这枚指环。''段圭璋如梦初醒,心中想道:``冷雪海遣这少女为她报仇,这更可以证实皇甫嵩就是当年杀害她丈夫的凶手了。不管这少女是否她的女儿,我决不能置之不理。''但为难的是:皇甫嵩对他却有救命之恩,在侠义道中又决没有把恩人杀掉之理。

段-璋摸了一下指环,问道:``皇甫嵩他有什么话说?''铁摩勒道:``他似是预知你不愿领他这个情,所以他说他要向你也求一个情,算是两无亏欠。''段-璋急忙问道:``求的是什么情?''铁摩勒道:``若是你将来碰到有一个人戴着同一式样的指环的话,他望你对这人留几分情面。''

段-璋吁了口气,道:``原来他不是为自己求情,好,这事我可以办到。待我替史大哥报仇之后,我再去找皇甫嵩,要是他杀了我,那没话说,要是我杀了他,我立即自刎,了结恩仇!''南乔云、铁摩勒相顾骤然,他们知道段-
璋的脾气,说了的话却无更改,而且又是在他心情激动之中,更不便相劝。

段-璋再问道:``那少女呢?''铁摩勒道:``她已经走了,她没有告诉我们去哪里,照我猜想,恐怕是找安禄山去了!''

段-璋吃了一惊,急忙问道:``你,你怎么知道她是去找安禄山?她,她去找安禄山干什么?''铁摩勒道:``她向我问及你那位姓史的朋友,又问及他的妻子和女儿,我告诉她姓史的已被安禄山所害,他的妻女也未曾救得出来。她听了这话,似乎很激动,她本来立誓要杀皇甫嵩的,南大侠几次劝阻她,她都不听,后来一知道了这个消息,便好像为了要做另外一件更紧要的事情似的,匆匆忙忙立即走了。所以我猜想她是要去救那史家母女。''段-璋失声叫道:``这怎么好?怎能让她一个人去独闯虎穴龙潭?''

铁摩勒被他的神气吓着,讷讷说道:``这仅是我的猜想,未必就是真的。而且那少女的剑法非常厉害,南大侠仗着宝刀,和她斗了几十个回合,也不过是打个平手。就算她真的去了,纵然救不出史家母女,她本人总可以脱身。''南霁云也道:``那少女之所以肯暂时罢手,多半还是因为她得知皇甫嵩救了你的性命,所以对他是好人坏人,一时也未能判断的缘故。段大哥你目前养伤要紧,你若是不放心那个少女,待我将你护送到窦寨主的地界之后,立即便去找她。''铁摩勒跟着说道:``是呀,待见了我义父之后,咱们还可以请他多派手下,去访查那个姓夏的女子,他在江湖上识得人多,总可以查到一点线索。何况,那少女已去了三个时辰有多,要追赶她也来不及了。''

段-璋叹口气道:``也只好如此了。''铁摩勒见他对那少女如此关心,有点奇怪;段圭璋听得夏凌霜对史逸如如此关心,也是有点奇怪:``难道她和史家也有什么关系么?要是史大哥和夏声涛夫妇也相识的话,我却怎么从未听他提过?''

夏凌霜匆匆策马而去,果然不出铁摩勒所料,为的是救史家母女。但她却不是去闯安禄山在长安的府邸,而是到安禄山手下的大将薛嵩家里救人。原来她早已知道了史家母女是被薛嵩向安禄山要了去的。至于她何以知道,以后再表。

她到达长安,已是中午时分。她扮成一个跑江湖的卖解女子,找一间容纳三教九流、不拒绝女客投宿的小客店住下,到了三更时分,便换上了夜行衣到薛家去。薛嵩的家人都在长安,他的家和安禄山的府邸也距离不远。

夏凌霜轻功超卓,比南霁云还胜两分,神不知鬼不觉的进入薛家,在薛家的客厅听到了有一男一女的谈话声音。她偷偷张望,只见男的是个军官,女的是个颜容憔悴的淡装少妇。

那军官道:``卢夫人,你赶快走吧!我已给你带来了一套男子的衣裳,趁薛将军尚未回来,你赶快换了衣装,委屈你权充我的小厮,我带你出去。你的小千金可以放在马车后厢,那马夫是我的心腹,不会泄露的。''

夏凌霜虽然和史逸如的妻子素不相识,但却知道她的母亲是河东卢氏,听那军官对她这样称呼,当然知道她是准了。她最初本来准备将那军官杀掉,然后问卢夫人道明来意,救她出去,现在突然听到那军官说出这番说话,当真是大出意外,又惊又喜,心里想道:``想不到安禄山的手下竟然也有这样的好人,我正担心那婴儿不便携带,他这个办法真是再好不过了!''卢夫人抬起头来,脸上现出一派迷惑的神情,眼光中含着深沉的忧虑,沉吟半晌,方始说道:``聂将军,多谢你的好意,但我要走就必须和丈夫一同走。''原来这个军官正是那一晚曾经暗中救护过段-
璋的聂锋。

聂锋也沉吟了半晌,然后说道:``史先生现在还在受软禁之中,帅府守卫森严,一时恐怕不易脱身,你们两母女先走,以后我再替他想法。''

卢夫人脸上的神情越发显得沉重,双眼直盯着聂锋,忽地问道:``聂将军,请你不要瞒我,我的丈夫到底怎么样了?''

聂锋讷讷说道:``他来的那天,大约是因为受了委屈,吐了几口血,现在正在调治。''

卢夫人道:``这个我早知道了。我是问他现在究竟生死如何?我听服侍我的那个小丫鬟言道,昨晚曾经有刺客要杀安禄山,闹了一晚,出了好几条人命,那刺客是不是段-璋?他救出了我的丈夫?还是他们都被安禄山捉住,一同处死了?聂将军,请你实话实说,不要瞒我!''

聂锋咬了咬牙,说道:``段大侠受了重伤,虽然没给捉住,恐亦难以活命了。至于史先生吗,他、他、他已经当场自尽了!所以,所以你必须现在立刻就走,不能再指望段大侠来救你们了!''

聂锋和在暗中偷听的夏凌霜,都以为卢夫人听到了这个恶耗,定要号陶大哭,或者当场晕倒。哪知卢夫人身子虽然陡然一震,但却并没有流出泪来。似乎这个结果早已在她意料之中。

但见她用力扶着几桌,支持着自己,呆了好一会子,忽地沉声说道:``我不走!''

这句话大出聂锋意料之外,他告诉卢夫人这个消息,本意是宁可让她悲痛一时,但必终于明白非走不可的,但她竟然拒绝逃走!

聂锋低声说道:``薛将军对你不怀好意,你,你要提防。''卢夫人道:``我知道。多谢你的好意。但我心志已决,绝无更改。除非是薛嵩将我撵出去,否则我决不离开!''

这番话不但出乎聂锋意外,夏凌霜更是大大惊奇,心中想道:``我母亲说卢夫人是极有见识的女中英杰,却怎的这样糊涂,难道是她因为受了突然的刺激,以致神智昏迷了么?''她从檐角偷窥进去,只见卢夫人虽然面色惨白,但却透露出一股坚毅的神情,似乎心中早已拿定了主意,反而觉得比刚才要镇定得多,哪里像是神智昏迷的样子?

就在这时又传来了脚步的声音,聂锋叹了口气,说道:``既然你心意已决,愿你好自为之。''

聂锋刚从角门走出,薛嵩便走了进来,说道:``卢夫人,我正想找你说话,却怕惊扰了你,原来你也未曾睡么?''

卢夫人道:``你有什么话说。''薛嵩道:``我待你好么?''卢夫人道:``薛将军,你庇护我母女二人,不让我们受安禄山的凌辱,我是感激得很的。''薛嵩眉开眼笑道:``你知道我对你的好意,那就好了。我对夫人十分仰慕,但愿夫人将这里当做自己的家里一般,安心住下来,使薛某得以时常亲近。''说着,说着,便走近了几步。

卢夫人亢声说道:``薛将军,请你记得我是朝廷命妇,你以礼相待,我可以留下,否则我唯有死在此地!''神色凛然,饶是薛嵩平素杀人不眨眼,也被她震住,有如奉了圣旨一般,急忙停了脚步,赔笑说道:``夫人哪里话来?得夫人留在寒舍,薛嵩实感荣宠无比,岂敢简慢,失了礼仪?''他搜索枯肠,说了一番文绉绉的话,听得夏凌霜暗暗好笑。

卢夫人道:``你们不让我和丈夫见面,这是什么意思?''

薛嵩道:``原来夫人想念尊夫,怪不得深夜未睡,只怕夫人不能够再和尊夫见面了。''

卢夫人道:``怎么?莫非、莫非他已经有什么三长两短了么?''夏凌霜知她是明知故问,一时之间,猜测不到她的用意。

薛嵩装出一副悲戚的神情,缓缓说道:``这消息我本来不忍告诉你,但经过我三思再想之后,觉得还是对你说了的好。这虽然是个坏消息,但夫人是个明白的人,只要你好自为之,那对你来说,就是苦尽甘来了。''

卢夫人道:``究竟怎么?''薛嵩道:``尊夫不幸,已经死了。他不肯依从大帅,昨夜又勾结刺客闹事,在混战中误触了武士的刀锋!''

卢夫人一直抑制住自己的眼泪,这时方始忍不住哭出声来。薛嵩站在一旁,见她宛如梨花带雨,泪湿罗衣,当真是又怜又爱,便轻声劝慰她道:``人死不能复生,夫人,你刚在产后,保重身子要紧。你不必担心今后的事情,一切有着我呢。要是你肯俯允的话,我想请你做我的继室,并替我训教几个小儿。尊夫之死,虽属不幸,但一了百了,却不会再牵累你们了。夫人,你要放宽心怀,就将我这儿当作你的安身立命之所吧。''

卢夫人抬起头来,抽噎说道:``将军厚义,存殁均感,继室之事,容后缓谈。现下我孤苦无依,尚望将军帮忙我料理丈夫的葬事。''

薛嵩道:``这个容易,我早已请准了安节度使,为尊夫备服成殓了,棺材亦已停在外间,只待夫人择吉安葬。''

卢夫人道:``我还有个不情之请,我与他夫妻一场,理该为他守孝,只是我现在已无家可归,不知将军可否准我在此间安设亡夫灵位,并准许我与亡夫一决?''

让别人在自己的家里治丧,这本是一件``晦气''的事情,但薛嵩为了要博取她的欢心,一切应允,立即说道:``夫人是名门淑女,朝廷命妇,我早已料到夫人要为尊夫守孝尽礼的了。不待夫人吩咐,我已经一一备办。来人!''片刻之间,果然有人将写好的牌位和香烛送来,再过一会,棺材也已搬了进来,登时将薛嵩的华贵客厅变作了灵堂。眼看又有两个小丫鬟替卢夫人拿来了孝服。

卢夫人披上了孝服,启棺哭道:``史郎,你好命苦啊!''薛嵩道:``夫人节哀。''急忙叫丫鬟拉开了她,再盖上棺盖。

卢夫人转过身来,向史逸如的灵牌磕了个头,悲声说道:``士为知己者死,女为悦己者容;史郎,你能为段大哥尽义,我岂不能为你尽节!''突然抽出一把剪刀,向面上乱划!

这一下大出薛嵩意外,卢夫人哭灵之时,围绕在她身边的是一班丫鬟,薛嵩不便近前,而且他昨晚被段-
璋的利剑刺伤了膝盖,行动也不大灵活,一时之间,竟来不及抢救,吓得呆了。

待至丫鬟抢了卢夫人手上的剪刀,她的脸上早已划了三四道伤痕,鲜血淋洒,玉貌花容,已都毁了!只听得卢夫人喊道:``史郎,我为了女儿,忍死须臾,望你九泉之下鉴谅。''

服侍卢夫人的那个小丫鬓扶着她走进后堂,薛嵩又是惋惜,又是愤怒,突然间像火山爆发似的,狠狠的瞪着那班丫鬟骂道:``你们都是死人吗?为什么不拦阻!晦气,晦气,出了这样的事情,你们还在这里做什么,都给我散了!''

薛嵩的管家低声问道:``要给卢夫人请医生吗?''薛嵩怒气未消,``啪''的打了一记耳光,骂道:``你好糊涂,还要把事情闹到外面去吗?她是你的什么人,要你这样着急?''

那管家登时省悟,要知薛嵩之所以对卢夫人奉承备至,乃是为了垂涎美色,如今卢夫人花容已毁,当然不必再巴结她了。那管家省悟之后,为了要讨好主人,连忙说道:``是,是,小的糊涂,小的糊涂!这灵堂也拆了吧?''

薛嵩把手一挥,正想说道:``连棺材也给我扔出去!''忽见聂锋走了进来,向他问道:``听说你给史进士开丧,干吗却发了这么大的脾气呀?''

聂锋是他的表弟,又是他的副手,而且武艺也比他高强,薛嵩的许多``功劳''都是倚靠了聂锋才取得的,在所有同僚之中,只有聂锋可以不用通报,直闯他的内室,而也只有聂锋的话,他最能听得进去。

薛嵩愤然说道:``我正是为这个生气,你瞧,天下竟有这样不识好坏的女人,我把她作为皇后娘娘奉养,还不怕悔气,腾出这座大厅来给她当作灵堂,她竟然一点也不领我的情,只记得她的死鬼丈夫,说什么`女为悦己者容',丈夫死了,她就把自己的颜容也毁了。哼,哼,我已算忍住了脾气了,要不然,我把她也毁了!''

聂锋笑道:``你是说卢夫人吗?她是名门淑女,熟读烈女传。圣贤书,你本来就不该动她的念头。她如今为亡夫毁容,实在是可敬可佩得很呀,你何必要发她的脾气。何况做好人就该做到底,要是你现在给她难堪,传了出去,别人一定说你为德不卒。不如仍然要为她安葬丈夫,还可以博得个好名声。''

薛嵩对卢夫人的毁容,在惋惜与愤怒之中,其实也有三分敬佩,经聂锋以好言相劝,所说的又都是堂皇正大的理由,气便慢慢消了,说道:``好吧,瞧在你替她说情的份上,我让她在这里住下去,让她教孩子念书,算作做一场好事。''

卢夫人进了自己的房间,薛家的人知道薛嵩发了脾气,无人敢来照料,只有那个以前薛嵩派来服侍的小丫鬟,替她裹好了伤,又悄悄的去找相熟的武士讨金疮药。

卢夫人倚着枕头,枕头卜绣着一对鸳鸯。她脸上的鲜血一点一点滴下来,将鸳鸯部染红了。

周围静寂之极,听不到半点声音,卢夫人想道:``想是她们都不敢来看我了,这样更好,史郎啊,你可以放心等候我了。''

门帘忽地无风自卷,并没有听到脚步的声音,却突然有一个少女走了进来,卢夫人吓了一跳,问道:``你是谁?你怎么敢来看我?''她还以为是薛府的丫鬟。

那少女低声说道:``蝶姨,你别害怕,我是来救你的,我的名字叫夏凌霜,我的母亲是你的表姐,她叫冷雪梅,你还记得她吗?''

卢夫人的小名叫做梦蝶,除了她的闺中女友和丈夫之外,别人决计不能知道;她再端详了那少女一会,活脱就像她那个多年不见的冷表姐站在床前,卢夫人再也没有疑心,又惊又喜的握着夏凌霜的手道:``你真像你的母亲,你怎么进来的?''

原来冷雪梅也是出身官宦人家,和卢夫人乃是中表之亲,她比卢夫人年长八岁,在卢夫人十一岁的时候,冷雪梅随她父亲到任所去,自此两人就不再见面,算起来已经有二十一个年头了。卢夫人小时候对这个表姐极为依恋,冷雪梅也很喜爱她的聪明。卢夫人在八九岁的时候,隐隐闻得大人闲话,说冷雪梅不务女红,却喜欢拈刀弄剑,有一次,磨着她父亲手下的一名武士比试,连那个武士也不是她的对手。卢夫人不知是真是假,有一天便问她的表姐,要表姐教她剑术。冷雪梅笑道:``你听他们乱嚼舌头,我哪里懂得什么剑术,不过有时偷看武士们练武,偷学了几个招式罢了。我的父亲是个武官,我拿刀弄剑尚自有人笑话,你是名门闺秀,学这个干吗?''卢夫人对武艺其实也是性情不近,她要表姐教她剑术,不过是闹着玩的,表姐既然不愿教她,她也便算了。

冷雪梅的父亲不久就在卢龙任内逝世,冷雪梅从此也就不知消息。卢夫人虽然忆念她,却做梦也想不到她的表姐竟是名震江湖的女侠。后来卢夫人嫁得如意即君,岁月如流,对她表姐的忆念也就渐渐淡了。

想不到隔了二十一年,而且正是在她遇难遭危、孤苦无依的时候,突然来了一个自称是冷雪梅女儿的夏凌霜!

夏凌霜替卢夫人止了血,低声说道:``你别担心,我进来没有一个人知道。你不要犹疑了,我背你出去!''

卢夫人摇了摇头,说道:``你为我冒这样大的危险,我很感激。但,我已决意不走了。''

夏凌霜焦急之极,急忙问道:``为什么?你怕我背了你不能脱险吗?我的武功虽然不算怎样高明,但这薛府里的武士我还未放在心上。''

卢夫人道:``我相信你有这个本领,小时候找已知道你的母亲是精通剑术的了,你是她的女儿,当然也是女中豪杰。嗯,说起你的母亲,我们已有二十一年没有见面了,她可好吗?''夏凌霜道:``好。''卢夫人再问道:``她什么时候结婚的我也未知道,你爹爹呢?在什么地方得意?''夏凌霜黯然道:``我出生的时候,爹爹就已死了,蝶姨,这些家务事咱们以后慢慢再说吧;我不明白你为什么不肯走?依我看来,这里绝非你可以久留之地!虽然你已毁了颜容,息了那姓薛的邪念,但你既然有亲可投,又何必寄人篱下,看人面色?''

卢夫人苦笑道:``孩子,我自有我的主意,日后你便会明白。服侍我的那个丫鬟就要回来了,咱们时候无多,我很想念你的母亲,你再告诉我一点关于你母亲的消息吧,你们是怎么知道我遭逢不幸的。''

夏凌霜道:``自从我出生之后,我母亲就和我住在玉龙山下的一个小村子里,每天督导我读书习武,没有什么特别事情可说。去年我满了十八岁生日之后,我母亲说我的剑术已经学得差不多了,叫我到江湖上见识见识,给她办一件事情,并叫我探访你的下落。今年年初三,我到了表舅家里,始知道你嫁到史家,元旦之夜,一家人莫名其妙的失踪,他们正为你着急。我再到你们所住的那条村子去查问,碰见了段-
璋段大侠的一个徒弟,说起段大侠一家也在年初二那天失踪,又说起安禄山在年初一那天从你们的村子经过,事后他到师父家中拜年,觉得师父的神色有点不对。从这些蛛丝马迹,我猜想你们两家的失踪或者会有关系,而段大侠与安禄山结怨的事情,我母亲曾对我说过。识得段大侠的人多,我便先到长安来访查地的行踪。嗯,经过的情形来不及细说,总之给我机缘凑巧,从安禄山一个武士口中查知你落在薛家。本来我昨晚就要来的了,但临时为了赴另一个约会才延到今天。''她急着要说服卢夫人和她逃走,一口气将前因后果约略讲了之后,便拉着卢夫人道:``蝶姨,你到底打的什么主意?是为了要替姨父报仇吗?即算如此,我以为你也是先逃出虎口,再和我母亲商量报仇之策为高!''

卢夫人苦笑道:``报仇二字,谈伺容易?安禄山的帅府不比这儿,他帐下武士如云,纵然你们母女剑术高超,亦难以寡敌众。再说,给丈夫报仇乃是我份内的事情,我岂能以不祥之身,连累你们母女?''夏凌霜道:``难道你留在薛嵩家里,就可以刺杀安禄山吗?''她一时情急,这两句说话冲口而出,自悔失言。卢夫人双眉一轩,沉声说道:``我虽然是个弱质文流,但有时报仇也不定需刀剑,我已立定主意,决不更移。你回去给我向你母亲问好,说我非常感激她的关心,但也请她今后不必以我为念了!''卢夫人这几句话说得斩钉截铁,虽是声音嘶哑,血污脸庞,但眉宇之间,却透出一股令人凛然的英风豪气!

夏凌霜虽然心里不以为然,但话已至此,也不好再劝了。当下问道:``蝶姨,你可还有什么话要吩咐我吗?''卢夫人道:``请你把我床边那只摇篮挪近前来,让我看看我的女儿。''

那婴孩受到震动,张开了眼睛,敢情是她这几天看惯了母亲的脸孔,骤然间见母亲换了一副丑陋的颜容,感到可怕,便``哇''的一声哭了出来。

卢夫人轻轻抚拍婴儿,低声哄她道:``小乖乖,别害怕,妈的面貌虽然变了,爱你的心还是一样。''婴儿似乎懂得母亲的心意,果然停止了啼哭。

卢夫人回过头来对夏凌霜道:``你说你曾访查段大侠的行踪,我昨日听到他的一个消息,听说他们前晚为了救我丈夫,和安禄山的武土恶斗,受了重伤,不知是生是死?你可以为我再去寻访他吗?''

夏凌霜道:``我刚想告诉你,我前晚曾遇见他,那时他刚从实禄山的帅府逃到一个破庙\ldots\ldots{}''卢夫人急忙问道:``他怎么样?''夏凌霜道:``不错,他是受了重伤,但还未死。''当下将所见的情形对卢夫人讲了。

卢夫人又惊又喜,半晌说道:``要是你今后再碰到他,烦你给我带两句话:我母女俩陷身虎穴,我虽有决心抚养女儿成人,但世事茫茫,殊难逆料,我不想误了他的儿子,要是他长大了遇有令适人家,尽可另求佳偶。''

夏凌霜证了一怔,道:``原来你们还是儿女亲家!''

外面似是有脚步声传来,卢夫人道:``你该走了!''夏凌霜叹了口气,说道:``蝶姨,你善自保重。你的话我一定替你带到。''

她飞身上屋,只见一个丫鬟带了两个军官走来,其中的一个便是想要救卢夫人的聂锋。原来他们是给卢夫人送金疮药来的。

聂锋眼利,瞥见瓦背上有个影子,吃了一惊,停下脚步说道:``夫人的内室我们不方便进去了,小红,你代我们在夫人面前请安吧。金疮药的用法你还记得吗?嗯,刘兄弟,你再给她说一遍。''

原来这个姓刘的武士乃是小红的情人,小红为卢夫人向他讨药的时候,恰巧遇着聂锋;薛嵩的家法极严,小红怕回去的时候给人盘问,若然搜出她为卢夫人带药,其罪非小。聂锋听见他们商谈,便挺身而出,与那姓刘的武士一道,送她回去。有聂锋出头,就是给薛嵩碰见,也不用怕了。

聂锋撇下了姓刘的武士和那个丫鬟,让他们多叙一会,独自走出院子,一看无人,便即飞身上屋,正在张望,忽觉微风飒然,寒气侵肤,夏凌霜的长剑已对准了他。

夏凌霜低声道:``你不要嚷,我不杀你。''聂锋这时才看清楚是个美貌的少女,惊奇之极。夏凌霜道:``聂将军,我知道你是个好人,以后还望你多多照顾卢夫人母女。''聂锋这才知道她是为救卢夫人来的。夏凌霜又道:``要是卢夫人有什么危险,请你派人送她到玉龙山的沙岗村找我的母亲,我的母亲叫冷雪梅,说起她的名字,村里的人都知道的。聂将军,以你的为人和武功,却甘心为虎作怅,我很替你可惜,倘若你将来不见容于安禄山,你也可以逃出来,我可以为你向段-
璋大侠说情,请他向江湖上的侠义道招呼一声,不把你当作敌人。''

聂锋听她说出冷雪梅的名字,这一惊更是非同小可,好半晌才定下心神,说道:``多谢女侠好意,倘有可以为卢夫人效劳之处,我一定尽力而为。还有一事相托,女侠若见了段大侠,请代我向他问安。我前晚迫不得已和他动手,还望他宽恕。''夏凌霜道:``好,只要你有心向善,段大侠决不会计较。''当下收回宝剑,身形一起,便如一缕轻烟,转眼之间出了薛家。

南霁云和铁摩勒护送段-
璋前去投奔窦家,一路无事,第四天到了平卢地界,再过二百余里,便是窦家的势力范围了。段-璋也已渐渐恢复,每餐可以进点稀饭了。南、铁二人都放下了心。这一天驴车正在山路上走,忽听得``呜''的一声,有一支响箭飞来,转眼间山坳的转角处现出两个黑衣骑士。

铁摩勒笑道:``这些瞎了眼的小贼,竟然把咱们当作肥羊,却不知是太岁头上动土!''

那两个黑衣武士远远叫道:``车上的可是段-
璋段大侠么?咱们寨主有请!''铁摩勒奇道:``奇怪,竟是请客来的。这两个人不是我义父的手下,这里也不是王伯通的地界,从来又没听说过有什么著名的绿林人物在这里安窑立柜,这两个家伙到底是哪条线上的朋友?''

段-璋揭开车帘一角,望了一眼,说道:``这两个人我都不认识,南贤弟,你上去与他们打话,给我敬辞了吧。''铁摩勒本来跃跃欲试,但南霁云已经上前,他只好留在车上保护段-璋。

南霁云问道:``请问贵寨主是哪一位?''那两个黑衣骑士道:``段大侠见了自然知道。''南霁云道:``段大侠尚在病中,我们赶着送他到他的亲戚窦家去,贵寨主既然是他的朋反,反正这里离窦家寨也不过两天的路程,就请到窦家寨去与他相会吧。''要知窦家五虎,乃是北方的绿林领袖,所以南霁云不怕实话实说,用意就是想吓退他们,免得交手。

岂知那两个黑衣骑士听了窦家的名头,神色竟是丝毫不变,一个道:``段大侠贵体违和,这个我们早知道了,正是因此,所以寨主请他就近到我们那儿疗伤养病。''另一个道:``段大侠大名,我们久已仰慕,难得今日经过,无论如何,也得请他到山寨里让兄弟们见见。''

南霁云久历江湖,一听这话,便知那个未知名的寨主不怀好意,说不定是窦家的对头,想趁段-璋重伤未愈,中途劫掳,免得他去相助窦家。而且这个寨主,绝不会与段-
璋有什么交情,要不然他也不用藏在暗中,连拜帖也不送一张来了。

南霁云沉住了气,说道:``贵寨主的好意段大侠心领了,窦家是他亲戚,他理该先去和亲戚会面。他在病中,不便和诸位相见,他已托我传话,就请你们回去上复寨主,要是贵寨主不便到窦家寨探望他,他病好之后,再来回拜如何?''

那两个黑衣骑士冷冷说道:``段大侠当真是这样说么?好吧,就算这是他的意思,我们奉了寨主之命,也得请他当面见我门寨主说去!''一声胡哨,草丛里面,乱石堆中,涌出了一群强盗,个个执着明晃晃的利刃!

南霁云面色一沉,铿锵有声,宝刀出匣,指着那两个骑士道:``你们这岂不是强人所难么?好,既然你们定要如此,我南八就替段大侠去一趟,不过你们可得先问一问我这口刀,问它肯不肯让我去!你们的人齐了没有?都请来吧!''

那两个骑士听他自报姓名,似乎吃了一惊,对望一眼,忽地哈哈笑道:``原来阁下是魏州南大侠,端的是失敬、失敬了!不过,南大侠,你这样的口气忒把人看小了,我们这些无名小卒,固然不敢与你南大侠单打独斗,但却也不是恃多为胜的下三流小贼,我已弟俩练有一套刀法,难得有此机缘,就请南大侠指教如何?要是南大侠仍认为不公平的话,就请车上那位姓铁的小兄弟也下来。''

南霁云冷冷说道:``两位既然要与南某较量,南某奉陪。你们两人齐上,我是凭这口刀,你们都上,我也是凭这口刀!''那两个骑士跳下马背,又哈哈笑道:``南大侠果然是个爽快的人,好,我兄弟俩献丑了。南大侠,你说`较量'二字,我们可当不起,我们只是向你请教,你这口宝刀锋利,还望稍稍留情。''

南霁云道:``好说,好说;两位不必太过自谦。两位既是只想与南某印证武功,那么咱们就点到划!胜败不论。''那两个骑士抽出刀来,说声:``请赐招!''南霁云忽道:``且慢!''那两个人怔了一下,只见南霁云回过头来,朗声说道:``摩勒,我与你换一把刀!''将宝刀入鞘,向铁摩勒抛去。

铁摩勒接刀愕然,段-璋躺在车中,低声说道:``摩勒,把你的腰刀换给他!''要知南霁云与段-
璋都是大侠的身份,宝刀宝剑不斩无名之辈,现在对方既非围攻,且又那样说法,南霁云当然不好再用宝刀。

铁摩勒无奈,只好将腰刀抛出,南霁云接了腰刀,说道:``两位是主,客不僭主,还是请两位先行赐招。''那两人道:``好,恭敬不如从命,那就请南大侠恕我们不客气了。''一个左手执刀,一个右手执刀,唰的一声,同时出手,左刀石指,有刀左指,合成一道弧形,把南霁云罩住,南霁云也禁不住心中一凛,他起初只当这两个人是无名之辈,哪知他们双刀合使,攻中带守,招数竟是十分老辣!

好个南霁云,就在刀光罩顶之际,蓦地一声长啸,身形骤起,举刀便劈,这一刀正从那道弧形的合缝之处劈下,但听得叮咣两声,那两柄单刀立即给他分开,那两人赞道:``好刀法!''各自身形一侧,刀走偏锋,左右夹攻,他们一个是左手刀,一个是右手刀,配合得极为纯熟,当真是攻守兼备,无懈可击!铁摩勒从车上望去,但见三道银光,忽分忽合,恍如玉龙夭矫,半空相斗!

铁摩勒蓦然省起,心道:``莫非这两个人乃是`阴阳刀'石家兄弟,怪不得他们知道我的名字。''石家兄弟,哥哥名叫石一龙,弟弟名叫石一虎,兄弟二人联手做黑道上的买卖,是西凉地方著名的独脚大盗,(他们兄弟二人如同一体,别无党羽,在黑道上的术语,叫做``独脚盗''。)因为他们兄弟一个使左手刀,一个使右手刀,哥哥性格阴沉,弟弟性格开朗,所以黑道个人称他们为``阴阳刀''。铁摩勒是大盗世家,他的父亲铁昆仑在生之时,和窦家的老大窦令侃,王家的王伯通合称``绿林三霸'',所以铁摩勒对于绿林中的成名人物,未曾见过,也曾听人说过。比南霁云要熟悉得多。

铁摩勒认出了这两人是``阴阳刀''石家兄弟,暗暗替南霁云担忧,想道:``南叔叔不知他们的来历,上了他们的当了!岂可舍宝刀不用!同时,又觉得奇怪:石家兄弟在黑道上乃是成名人物,从来都是兄弟联手,别无党羽的,怎的他们这次前来,却声称是奉了什么``寨主''之命,难道他们竟甘心屈居人下,投到什么山寨里做了头目么?

南霁云和他们越斗越烈,但见一片刀光,三条人影,时而纠作一团,时而分开三处,三个人的身法都是快到了极点,令人看得眼花撩乱,渐渐人影刀光,混成一片,竟分不出哪个是南霁云,哪个是石家兄弟了。铁摩勒年纪虽轻,却经过不少大阵仗,但这一次也看得他目眩神摇,个敢透气。

正在铁摩勒暗暗担忧的时候,忽听得南霁云一声大喝,刀光划过,登时发出了一片金铁交鸣之声,三条人影倏的分开,但见石家兄弟,面色铁青,他们手中的单刀!都只剩下半截!南霁云抱刀一揖,说道:``承让了!可以放我们的驴车走了吧?''南霁云竟以一炳寻常的朴刀,削断了石家兄弟的兵刃,不但显得刀法精奇,更足见内力深厚,这一下直把群盗吓得目瞪口呆,矫舌难下。

正是:黑道风波多险恶,单刀退敌护良朋。

欲知后事如何?请听下回分解------

\chapter{第 十 回 侠士荒山遭恶寇
神偷午夜盗婴儿}\label{ux7b2c-ux5341-ux56de-ux4fa0ux58ebux8352ux5c71ux906dux6076ux5bc7-ux795eux5077ux5348ux591cux76d7ux5a74ux513f}

乱石堆中忽地一声长啸,走出了一个人来,年纪甚轻,看来不过二十左右,书生装束,摇着一把折扇,但温文之中,却又带着几分轻佻,几分邪气。当石家兄弟拦截驴车、群盗涌现之际,并未见有这个人,似是刚刚来的、南霁云也不觉有点惊异,要知他虽在激战之中,仍然是眼观四面,耳听八方,但这个少年是什么时候来的,他却毫不知道。

这少年身形一现,群盗便发出一片欢呼。石家兄弟却是满面羞惭,丢下手上的半截朴刀,讪讪说道:``少寨主,咱俩兄弟辱命了!''那少年笑道:``南大侠岂是你们请得动的?还是待我来促驾吧!''折扇一指,面向着南霁云朗声笑道:``敝寨诚意相邀,南大侠、段大侠当真不肯赏面么?''

南霁云道:``少寨主一邀再邀,盛情可感。但段大侠尚在病中,他的妻子也正在窦家寨等待他,这些情形,刚才我也已对贵寨的两位香主说得清清楚楚了,请恕不能从命。''

那少年斜着眼睛笑道:``糟糕,我是讨了令箭来的,非得把你们三位请到不可,这怎么办呢?南大侠,请恕我说句无礼的话,尽管你们心急要走,我却是定要把你们留下的了!''

南霁云气往上冲,勃然怒道:``好吧,少寨主既有本领将我们留下,就请施展吧,废话少说了!''那少年一个笑道:``南大侠果是快人快语,好,我现在就凭这柄扇子,陪南大侠走两招!''说到一个``招''字,扇子一伸,招数便发!

这一招是铁笔点穴的招数,他把折扇合了起来,当作判官笔用,点打南霁云的``肩井穴'',手法利落,认穴奇准,确是不同凡响,南霁云心道:``怪不得这小贼骄狂,只这一招点穴的功夫,便不在宇文通之下!''

南霁云身形不动,待他扇子点到,蓦地大喝一声``撒手!''反转刀背,一刀拍下,那少年正巧在这个时候,也喝了一声``撒手!''扇子改点为粘,倏然一翻,搭着刀背,往下便按,两人的功力差不了多少,但见南霁云那柄朴刀往下略沉,随即反扬了起来,将少年的折扇荡了开去!

这一招南霁云稍占上风,但那少年的折扇没有给他拍落,也只能算打个平手。那少年笑道:``双方都没有撒手,再来,再来!''身移换步,嗖的一声,铁扇挟凤,已是绕到了南霁云背后,反手点他脑后的``风府穴''。

南霁云就似背后长着眼睛似的,反手一刀,又狠又准,刀长扇短,少年的扇头尚未触及他的背心,他的刀锋已撩到了少年的手腕,这少年急忙坠肘沉肩,慌不迭的把扇子反拨回来,``当''的一声,碰个正着,少年虎口隐隐发麻,斜窜三步,叫道:``好刀法!''

说时迟,那时快,南霁云反手一刀把敌人迫退,立即反守为攻,身形一旋,恰恰封着了那少年的退路,两人面对,南霁云一声大喝,使出一招力劈华山,朴刀斩下,隐隐挟着风雷之声、那少年也喝了一个``好''字,扇子滴溜溜一转,抵着无锋的刀板,身形蓦地向后一翻,平空跃起一丈有多!

南霁云这一刀已用了八成气力,但给那少年用了一个``卸''字诀,避重就轻,将南霁云攻来的猛力移转给全身负担,故此身形虽给冲得立足不稳,迫得跳跃起来,但那把折扇,仍然没有脱手。南霁云见他使出这等上乘的功夫,也禁不住心头一凛,想道:``江湖道上,当真是人材辈出,我若在他这般年纪,以怕还未必是他对手。''

心念末已,那少年又已向他扑来,南霁云道:``你当真要拼命么?''朴刀一起,截斩他的双足,那少年身子悬空,双足交叉踢出,铁扇又指向他的眉心``阳白穴'',这一招三式,用得狠辣非常,南霁云若不变招,纵能把他的腿骨斩碎,自己也难免受伤、第一流的高手与人比斗,除非是深仇大恨,否则断无以死相拼之理,南霁云本来就有点爱惜那少年的武功,如今又见他如此凶悍,心念一转,立即闪开,如此一来,他便反而给那少年抢了先手,迫得向后连连倒退了。

原来那少年正是要借南霁云来扬名立万。要知南霁云已是名震江湖的游侠,而他还是个初闯道的少年,若把南霁云打败,那是何等光采之事,所以他不惜连使险招。其实刚才那一招倘若南霁云不让的话,纵然受伤,但以他的内功和闭穴法应付,伤亦不会伤得很重,而那少年双足破斩,就要成为废人了。那少年承他让了这一招,过后方始想到当时的凶险,出了一身冷汗。

可是那少年立意要把南霁云打败,虽则明知这一招是对方手下留情,他却并不领南霁云这个情,一见南霁云后退,竟然如影随形,跟踪扑到,扇子一张,向南霁云面门一拨,劲风扑面,南霁云的双眼几乎睁不开来,那少年抓紧时机,立即便施杀手!

他这柄扇子是精钢打成的,扇骨上端锋利,合起来可作判官笔,张开来就可当作一柄折铁刀,但听得``嗤'的一声,扇子从南霁云手腕划过,南霁云大吼一声,右腕一翻,一掌推出,那少年蹬、蹬、蹬,连退三步,``哇''的一声,一口鲜血喷了出来。南霁云的右手手腕,也给他的扇子割开,鲜血汩汩流出。

群盗见他们的少寨主受伤,哗然大呼,纷纷涌上,那少年喝道:``都给我退开!''一个盘龙绕步,扇子倏张,又扑到了南霁云的面前,冷冷说道:``彼此挂彩,两不输亏,再来,再来!''南霁云刀交左手,道:``好!冲着你这股狠劲,南某就索性成全了你的声名吧!要是我在一百招之内不能胜你,我便甘心服输,百招之内,死伤残废,各安天命!''他以大侠的身份,定出百招,已是差不多将对方看作相等的对手了,那少年口吐鲜血之后,面色本已相当惨白听了这话,顿然光采焕发,哈哈笑道:``南大侠,我正是要你这儿句话!''

南霁云一招``横云断峰'',破解了那少年的连环点穴三式,喝道:``要是你在百招之内输了呢?''那少年知他心意。一声笑道:``最多把性命交给你,我与你比武是一回事,家父请客是另一回事,不必混在一起。喏,天色将晚,你们不必等待我和南大侠分出胜负来了,赶快先接了段大侠到寨里安顿吧!''后面这几句话是对群盗说的,群盗轰然应声,移转目标,奔向驴车!

南霁云又惊又怒,惊者是段哇璋街还未愈,如何抵挡群盗的围攻?怒者是那少年竟然如此凶悍撤泼!全不依江湖礼数、这时他已动了真气,一刀紧似一刀,毫不留情、但他左手刀的威力究竟不及右手刀,那少年在兵器上又占了便宜,一柄扇子,忽合忽张,时而作判官笔,时而作折铁刀用,缠得极紧,一时之间,南霁云竟也摆脱不开。

铁摩勒坐在驾车的座位上,提刀斩下,他用的是南霁云那把宝刀,大占便宜,但听得一片断金碎玉之声,两枝花枪、一柄单刀早已给他削断!铁摩勒大喝道:``不怕死的都来!''石龙笑道:``铁兄弟,我们看在去世的的铁老寨主的份上,不想与你为难、你也是黑道中人,你岂不知请客不到,乃是犯了绿林大忌的么?今日段大侠是主客,你们两位是陪客,你当真要敬酒不喝喝罚酒么?''

铁摩勒冷笑道:``石老大,亏你还有脸皮来和我说绿林规矩?你也算得是绿林里的一位人物,却怎的给人当起跑腿来了?这也不打紧,但你代主人送的`请帖'巳给别人退了,再要送来,也该请另一位来吧?''石家兄弟登对面色涨红,他们刚刚败在南霁云刀下,铁摩勒说他们的`请帖'已给别人退回,就是这个意思。也即是说他们已经没有资格代表主人而来请客,他们乃是在黑道上有身份的人物,给铁摩勒一顿冷嘲热讽,虽是又羞又怒,却不敢过来和他动手。

一个身材高人的强盗排众而出,朗声说道:``好,这请帖待我来下,请铁少寨主赏面!''他用的是一柄铜锤,锤重力沉,``呼''的一声,就向铁摩勒当头砸下。

铁摩勒在驴车上跳跃不灵,只好硬接他这一锤。铜锤是重兵器,宝刀虽利,决不能将它削断,铁摩勒给震得手腕酸麻,幸亏他和段-璋相处那几天,得到段-
璋传授了不少武功的上乘心法,懂得运用惜力打力的功夫,宝刀一带,那强盗的身形给他带得歪过一边,铁摩勒的刀锋划过,``嗤''的一声,将他的衣服挑穿,只差半寸,就要戳进他的琵琶骨。可惜铁摩勒尚未运用得十分纯熟,要不然这一招就可以叫他铜锤脱手,人受重伤。

那强盗大怒喝道:``好小子,你宁愿吃罚酒,我们只好不客气了!''手臂一抡,举锤冉磕,另外两个使用重兵器的强盗也攀着车辕,帮他夹攻,一个使青铜锏,一个使铁轮拔,都不是宝刀所能削断的。铁摩勒受到三般重兵器的围攻,登时险象环生,左支右绌。

段-璋忽地揭开车帘,背倚靠垫,沉声说道:``摩勒住手,他们既是冲着我来的,就让他们来见我吧!''使铜锤的那个强盗笑道:``还是段大侠是明白人,咱们是诚心请你老的。''一只手提着铜锤,另一只手就来扶他,段-璋淡淡说道:``段某平生吃软不吃硬,你这是拉客,不是请客!叫你家寨主亲自来吧!''那个强盗欺他是个病人,哪知手指刚刚触及他的手腕,段-璋蓦然把掌心一翻,反手一抓,吐出内家真力,``咔嚓''一声,将他的手腕拗断,那强盗一声惨叫,铜锤脱手飞出,打伤了两个同伴。

使青铜锏和斫山刀的那两个强盗急忙将兵器朝他劈下,段-璋虎目圆睁,喝声:``去!''双指一伸,贴着刀背轻轻一推,那柄斫山对登时反转斫来,正好和青铜锏碰个正着!

段-璋在病中用这一招,实是险到极点,若是稍差毫厘,他的手指就要先给刀锋削断了。但他用得恰到好处,只听得``当''的一声巨响,震耳欲聋,这两个强盗的兵器相交,各自给对方的猛力震倒,跌了个四脚朝天,青铜锏缺了一角,大斫刀也卷了刀锋!铁摩勤大笑道:``好啊!妙啊!''

群盗给段-璋的神威所慑,不约而同的一齐退了几步、段-璋抽出宝剑,倚着车垫,沉声喝道:``还有哪一位要来递帖?''

段-璋服了几天药,伤势虽然好了许多,到底尚未复原,如今强用真力,打发了三个强盗之后,他也感到气血翻腾,眼睛发黑,但仍然强自支持,想吓退群盗。不料那石家兄弟乃是武学行家,最初他们也慑于段-
璋的绝顶武功,随同群盗后退,但后来一听,从段-
璋的声音中听出他中气不足,伤还未愈,石一龙打了一个胡哨,群盗又聚拢来,围着驴车,石一龙自己不好意思出面,向那使青铜锏的强盗低声说了几句,那强盗大喜,站了出来,冲着段圭璋叫道:``段大侠既不赏面,请恕我们也不客气了!并肩子上,用暗青子招呼!''

一声令下,暗器齐发,飞刀、金镖、铁莲子、飞蝗石、甩手箭、流星锤\ldots\ldots 各式各样的暗器,纷如雨下,段-璋身子不能移动,只有靠着车垫,挥动宝剑防护。

铁摩勒又惊又怒,遮在段-
璋的身前,大怒骂道:``你们这些下三流的小贼,真是丢了咱们绿林好汉的脸!''那使青铜锏的强盗大笑道:``铁少寨主,你不顾行家的面子,又怎能怪得我们?你别害怕,伤了,我们给你医!''话声未了,铁摩勒已经中了两支甩手箭、一块飞蝗石,飞蝗石正打中他的额角,登时血流如注,幸而群盗志在生擒他们,未用喂毒的暗器。

段-璋道:``摩勒,你退入车厢!''铁摩勒哪里背依?正在危急之间,忽听得马铃叮当,一个少女飞骑来到,不是别人,正是那夏凌霜!

夏凌霜一眼瞥见南霁云和那少年厮杀,似乎甚感意外。``咦''了一声,那少年看见是她,面色倏变也``咦''了一声,但这时他给南霁云刀光罩住,几乎透不过气来,哪能分出心神与夏凌霜打话?夏陵霜这时已发觉了群盗围攻驴车,她本来要向南霁云耶一方驰去的,稍一踌躇,便突然拨转马头,向群盗冲来!

群盗早已有所准备,见她冲来,暗器纷纷向她射击,夏凌霜怕伤了坐骑,一个``金鲤穿波'',登时从马背上斜掠出去,身形未落,剑已出鞘,剑随身转,宛似一圈银虹,向外扩张,但听得叮叮当当之声,不绝于耳,那些暗器都已给她青霜剑荡开。群盗大惊,说时迟,那时快,他们的暗器尚未接续发出,已是被夏凌霜杀进来了。

这一来,群盗的暗器已是毫无用处,只能与她硬斗。夏凌霜步法轻灵,剑招迅捷,左边一兜,右面一绕,在群盗中穿来插去,宛如彩蝶穿花,每发一剑,便有一个强盗``哎哟''一声,兵器脱手。原来她用的是一套非常古怪的剑法,只是剑尖轻轻一点,便刺中对方的手脆,伤倒不重,但手中的兵器,却是再难掌握。使大斫刀的那个强盗大怒,抡刀向她猛劈,想把她的长剑磕飞。这人武功较高,夏凌霜一点没有点中,忽地柳腰一弯,剑锋向在斜方疾削,这强盗为了避她刚才刺腕那凌厉的一招,脚步也正好向左斜方踏出,就像凑上去碰她的剑锋似的,但听得``唰''的一声,剑锋削过,登时削去了他一片膝盖,那强盗一声惨呼,倒在地上,接连打了几个滚,滚下山坡、那些未受伤的强盗,见她的剑法如此厉害,四散奔逃。

石家兄弟早已换过兵刃,见势不妙,只好不顾身份,左右夹政。夏凌霜止在杀得兴起,信手一招``玄鸟划砂'',剑锋自左而右,横削两人手腕,哪知这两兄弟的阴阳刀法配合极妙,双刀合成一个圆弧,把夏凌霜这招化解开去,双刀倏合倏分,仍然从左右两方攻到,

段-璋道:``摩勒,你去助她一臂之力。''这时群盗已散了十之八九,纵有暗器打来。段-璋有宝剑防身,也尽可防守得了。铁摩勒挨打了半天,一口闷气正自无处发泄,听得段圭璋吩咐,立即跳下驴车,挥刀攻敌他虽然受了两三处伤,都非要害,宝刀砍出,虎虎风生。

石家兄弟本来就不是夏凌霜的对手,不过,要是铁摩勒不来的活,他们还可以支持一些时候,如今铁摩勒一来,所用的又是南霁云那柄宝刀,这两兄弟焉能抵挡;不过五招,便听得``当''的一声,石一虎手中的单刀先给铁摩勒的宝刀削断,石一龙知道今日难以讨好,拉了兄弟便跑,铁摩勒还要追上去再斫一刀,夏凌露笑劝他道:``穷寇莫追,小兄弟你就饶了他们吧!''收回长剑,眼光移转到南霁云和那少年身上。

南开云和那少年强盗正在斗到最吃紧的时候。自从夏凌霜出现之后,那少年显得非常焦躁,连使险招,南霁云久经阵仗,对敌的经验自是比那少年丰富得多,对方冒险急攻,正合他的心意,他脚踏五门八卦方位,使出一套游身断门刀法,表面看来,似乎是在步步退守,实则已是把那少年的攻势完全封住,刀锋所指,无一不是那少年的要害之处,威力暗藏,只要找到时机,立即便可以给予对方致命的一击!

待到夏凌霜将群盗驱散,那少年更是神色大变,猛地喝声:``我与你拼了!''铁扇一挥,瞬息之间,连袭南霁云七处大穴,南霁云纵声笑道:``来得好!''刀光疾闪,一口朴刀,也就在这瞬在那少年的肩头上拉开了一道五寸多长的伤口!这还幸亏是南霁云听到夏凌霜的叫声,朴刀及时收回,要不然早已砍碎了他的琵琶骨!要知南霁云恨这少年强盗太过凶狠,这一刀本来是有意将他砍成残废的!

南霁云虽然大获全胜,心里也暗叫了一声:``侥幸!''他打败这少年只用了五十一招,实在大出他的意料之外,心中想道:``倘非他心神不宁,暴躁走险,自乱章法的话,只怕在百招之内,我还未必准定能够赢他!''

那少年托的跳出圈子,满面通红,忽地抱扇一揖,叫道:``好刀法,承教了!青山绿水,后会有期!''这几句话听来是向南霁云说的,但说道``后会有期''那四个字,双眼却向夏凌霜一溜,夏凌霄嘴唇微动,似是想说什么话却没有说出来,那少年强盗已是如飞走了。夏凌霜脸上现出一派迷惘的神情!

南霁云将朴刀交还给铁摩勒,换回自己那把宝刀,然后向夏凌霜谢道:``多谢姑娘帮忙。''铁摩勒满腹疑团,问道:``夏姑娘可是认识那贼子的么?''夏凌霜的脸蛋唰的一下泛出桃红,讪讪说道:``曾经见过一面,算不得是怎样认识。''南霁云也在疑心,但见她如此,却不好再问下去。

三人回到驴车前,段-璋早已在那儿等待,一见便道:``这位可是夏姑娘么?''

夏凌霜应了一声,便恭恭敬敬的向段-
璋裣衽施礼,说道:``侄女向段伯伯请安。''段圭璋越看越觉得她像当年的白马女侠冷雪梅,又听她这样称呼,心中已无疑义,便直率问道:``令堂可是姓冷,芳名雪梅二字?''夏凌霜道了一个``是''字,随即笑道:``人人都说我似母亲,段伯伯果然看出来了。''

段-璋迟疑半晌,方再问道:``还未曾问候令尊?''夏凌霜道:``先君卢龙夏氏,名讳上声下涛,在我出生的时候,早已过世了。''

段-璋甚为纳罕,心中想道:``当年他们结婚之夕,夏声涛刚进洞房,便遭非命,却怎的生出了这个女儿?他们二人乃是光明磊落的男女侠客,若说婚前便有私情,似乎难以置信。''还有一点奇怪的是:夏凌霜在谈到她过世的父亲的时候,并没有显得特别的悲伤,要是她知道父亲当年的惨死,决不会如此冷静,见了自己的面,也决不会不央求自己给她报仇。``难道冷雪梅竟未曾告诉女儿?她已经长大了,为什么还要瞒住她呢?''段-璋越想越觉得奇怪。

夏凌霜见段-
璋神色有疑,也是有点奇怪,正想说话,段-璋又再问道:``令堂现在安居何处?''夏凌霜踌躇好久,尚未答话,段-璋道:``我和令尊令堂当年常在一起,是很要好的朋友。''夏凌霜道:``我妈也曾对我说过和段伯伯的交情,但她说她隐居多年,已不想再见以前的朋友,她托我向段伯伯问好,并请段伯伯原谅。''段-璋听了这话,大出意外,更觉惊疑,心道:``怎么雪梅连我都不愿意见了呢?难道她遭了那次惨祸,竟然万念俱灰,连丈夫的冤仇都不想报了?''

段-璋不便再问她的母亲,顿了一顿,绕个弯儿再问她道:``听说你要杀西岳神龙皇甫嵩,不知是为了何事?''夏凌霜道:``我母亲说他是个无恶不作的魔头,叫我为江湖除害。''说来说去,和她那晚答复南霁云的话大致相同,却并没有涉及自家的事。段-璋想了一想,说道:``你母亲说的不错,这皇甫嵩是个坏人,为江湖除害,这也是我辈侠义道所应为,但那皇甫嵩武功高强,你单身一人,只怕不是他的对手,若有要我效劳之处,我可以帮你的忙。只是我目前还有一件事待办,你不如和我们一道到窦家寨去,待我养好了伤,办了那件事后,再与你去找皇甫嵩如何?''

夏凌霜道:``多谢伯伯好意,只是家母吩咐,叫我最好独力除他,不必假手旁人。段伯伯,你要办的事情我也已经知道。卢夫人正有几句话要我转告于你。''

段-璋吃了一惊,道:``你那晚果然是到安禄山的府邸去了?''夏凌霜微笑道:``不,我是到薛嵩家里去,薛嵩这贼子垂涎卢夫人的美色,早已向安禄山讨了她了。''段-璋这一气非同小可,``啪''的一掌,击得车把手开了一道裂缝,骂道:``岂有此理!我不给史大哥大嫂出这口气,誓不为人!''愤火过后,又担忧道:``我那史大嫂是知书识礼的名门淑女,怎生受得了这等侮辱?''夏凌霜道:``段伯伯不用担忧,我那蝶姨早已识破薛嵩不怀好意,因此自毁颜容,虽然陷身魔窟,却可以保全名节。''当下将当晚的所见所闻,说与段、南、铁等三人知道,三人尽皆嗟叹,南霁云翘起拇指赞道:``这对夫妻高风亮节,的确令人仰慕!''

段-璋道:``夏姑娘,你刚才称呼卢夫人做什么?''夏凌霜道:``我妈是她的表姐,她闺名有个`蝶'字,所以我称呼她做蝶姨。''段-璋道:``原来你们是亲戚,这我倒还未曾知道。''歇了一歇,再问道:``这么说,你是奉了母亲之命,前来救她的了。''夏凌霜道:``不,我母亲僻处荒村,久已断绝外间消息。是她叫我寻访蝶姨,我到过你和史进士所住的那条村子,经过了许多曲折,这才探听到的。我见了她之后,确是想把她救出去,可是她不肯答应!''段-璋怔了一怔,道:``怎么,她不肯出去?''夏凌霜道:``是呀,我怎么劝也劝她不动!''铁摩勒大惑不解,喃喃说道:``这,这她可是太糊涂了!''段-璋双眉一轩,道:``我那史大嫂是女中豪杰,她下了这个决心,其中定有道理!她还有什么话要你对我说的?''

夏凌霜道:``她提到你和她两家的儿女亲事,她说她现在处境如斯,后事难料,令郎长成之后,若是另有合适人家,尽可自行婚配。''段-璋叹道:``她处境如斯,还为我的儿子着想,真是难得。不管她母女将来如何,这门亲事,我是决不更改的了!''随即又对夏凌霜说道:``要是你没有旁的事情,就和我们一道走吧。天色将晚,咱们应该起程了,免得错过宿头。''

夏凌霜踌躇片刻,眼珠一转,低声说道:``多谢伯伯好意,不过我还有一点旁的事情,反正窦家离此不过二百里,过几天我再去拜候你。''夏凌霜如此说,段-璋不便再邀,当下两家分道扬镳,段-璋目送她跨上骏马,绝尘而去,想起以前与她父母相处的日子,心中无限感伤。

南霁云驾御驴车,兼程赶路,两天之后,便到了幽州境内的飞虎山下,窦氏昆仲五人号称``窦家五虎'',这飞虎山山形险峻,又切合他们兄弟的绰号,故此他们将窦家寨建在飞虎山中。

段-璋在路上每天服食三粒药丸,至此恰好是第七天,身体果然完全复原,功力比起未受伤的时候,甚至还有少少增益,段-璋只道南霁云给他的药丸乃是磨镜老人的秘制灵丹,却不知是那西岳神龙皇甫嵩所赠。

这一行人进入山口,大寨主窦令侃早已得知消息,亲自出迎,一见面便哈哈笑道:``你这窦家娇客(古人称女婿为``娇客'')如今真变成了`稀客'了,好容易才请得你来!一去十年,也不给我们捎个信儿!''

段-璋这次来助窦家争霸绿林,本非心愿,但至此也不得不与舅兄客套几句,道歉赔罪之后,便问及那次他们窦家五虎与精精儿争斗的事情,窦令侃伸出左手笑道:``还好我的指头尚未完全削掉,不过也算得是栽到了家啦!''原来他左手的两根指头已给精精儿削去,段圭璋看了,不禁凛然。

窦令符又道:``你来得正好,王伯通与精精儿给我的期限,只有四天就到期了。线妹等你正等得心焦,还担心你在途中出事呢!''段-璋笑道:``途中的确是曾经出事,幸亏有南八兄护送,要不然只怕我想与精精比比剑,也没有机会了。''当下给两人介绍,窦令符这才知道与他同来的竟是大名鼎鼎的南霁云,当真是喜出望外,说道:``有了你们夫妇,再加上南大侠帮忙,咱们可以不必惧怕那精精儿了。''南霁云微笑道:``我是来看热闹的,算不得数。''

说话之间,不觉已来到大寨的聚义厅,窦家几兄弟和窦线娘都已聚集在那儿,段-璋历尽艰危,九死一生。虽是别来不够一月,便与妻子重逢,却已宛如隔世。窦线娘听得史逸如惨死,卢夫人母女都未曾救得出来,不禁眼泪双流。窦令侃道:``你们先帮我这个忙,待打赢了精精儿之后,咱门再一同去找那安禄山和薛嵩算帐。今日咱们家人团聚,可不许再提这些伤心事了!''

窦令符问道:``妹丈,你们在途中遇到强徒截劫,其中可有一位少年盗魁,是用折铁扇点穴的?''段-璋诧道:``你怎么知道?''

窦令符笑道:``我们在路上也碰上了,这小子好不厉害,要不是有六妹在旁,我还真不是他的对手呢!''段-璋带着既是责备又是怜惜的眼光,望了妻子一眼,意思是说:``你刚在产后,怎不顾惜身子,就与强人动手了呢?''当然他也知道在那样的情况之下,窦线娘非出手不行,但他对妻子关切的情怀,仍是禁不住自然流露。

窦令符哈哈笑道:``六妹,你丈夫如此疼你,怪不得你几乎忘记了娘家了。''回过头来对段-
璋道:``妹丈,你不用担忧,她并没有和敌人过招动手,甚至连一步也没有离开驴车,只凭着一把弹弓、就把强人都打退了!那少年盗魁也真凶悍,连中三弹,这才退下!''窦线娘的神弹绝技,在她结婚之后,从未曾对敌用过,连段-
璋也未深知,这时听了,又惊又喜。窦令侃也笑道:``爹爹当年偏心,把他最拿手的玩艺,都传给了六妹,她是窦家的凤凰,我们五只猛虎加起来,还比不上一只凤凰呢?''窦线娘噘着嘴儿道:``哥哥,你又拿我开玩笑了,你的三十六路混元牌法,我就没有学会。''窦令侃笑道:``好了,好了,再说下去,就变成了咱们兄妹互相夸赞了,岂不叫外人笑脱大牙。''南霁云道:``那少年盗魁确是了得,段嫂子令他连吃了三枚弹子,我也佩服得紧!''

众人都夸赞窦线娘的神弹绝技,窦线娘却并没有现出欢喜的神情,反而眉宇之间,似有重忧,众人都道她是故作谦虚,只有段-
璋深知妻子绝不是矫柔造作的人,也察觉到她藏有隐忧,只不知她忧的是什么事情,心里忐忑不安。

窦令符道:``你们可知道这少年盗魁是什么人?我前两天才查探出来。''段-璋道:``可是王伯通的手下?''窦令符道:``不仅是他的手下,还正是他的儿子呢!''窦令侃道:``王伯通仅有一子一女,听说从小他父亲就遣他们另投名师习艺,儿子是最近才回来的。''段-璋听了,又多一层担忧,那少年已是如此了得,他师父当然更是非常人物,这两家争斗,只怕牵连愈广,将来不知如何收拾,自己卷入了这场纠纷,也不知如何方能脱身了。

接风酒过后,段-璋夫妇回到自己的房中,窦线娘叹口气道:``璋哥,你这次来相助我的哥哥,我是感激的很,只怕,只怕我连累了你\ldots\ldots{}''段-璋道:``最初我本不想来,但现在是我自己允诺了你哥哥的,不关你的事。你我夫妻,何出此言?''窦线娘低声说道:``你且先看这一封信!''段-璋抽出信笺,上面寥寥几行,大意是说为了顾全段-
璋的声名,请窦线娘劝她丈夫不要趁这趟浑水(黑道术语,即不要卷人纠纷之意),免得两败俱伤。信后面没有署名。段-璋沉着了气问道:``这封信是怎么来的?''窦线娘道:``大约是昨晚三更时分送来的,那时我正睡得朦胧,猛听得房中声响,跳了起来,敌人的踪迹已经没了,在枕头旁边发现了这封信,你再看,反面还有宇。''段-璋反过信纸一看,果然还有两行字迹。写得十分潦草,似是临时加上去的。写的是:``取去玉钗,聊作示警,尊夫明日可到,为祸为福,幸贤伉俪善自处之。''

段-璋吃了一惊,忙问道:``你,你失去了那股玉钗么?''窦线娘道:``不是那股作为信物的龙钗,是我头上插着的一根玉钗。''段-璋吁了口气,道:``还好,要是失了那股龙钗,就对不住史大哥了。这事情,你的哥哥知道了么?''窦线娘道:``我还没有告诉他们。他们盼望你来,有如大旱之望云霓,要是他们知道此事,定然甚是为难,不知是留你好,还是不留你好了。''歇了一歇,再道:``这信上说你今日可到,我当时是半信半疑。所以,我索性等你到了,再和你商量个主意,暂时不作声张。圭璋,你看该怎么办?''

段-璋毅然说道:``咱们夫妻岂是受人威吓的人,我本来不大愿意理这种黑道上的纷争的,但有了这封信,我倒决意要在你们的窦家寨留下来,斗一斗什么精精儿、空空儿了!''

窦线娘道:``不错,我瞧这封信九成是空空儿送来的。听说他是精精儿的师兄,神偷绝技,天下无双。''段-璋道:``我也听过他的一些事迹,从这件事情看来,果然是身手不凡。但咱们也不用惧怕他,多加一点小心便是。''窦线娘有丈夫壮胆,柔声笑道:``有你在我身边,再厉害的敌人我也不会害怕了。你还没有见过孩子呢,你去瞧瞧他吧。你还记得今天是什么日子么?今天刚好是咱们孩子的满月。''

窦线娘这间房和邻房相通,窦令佩拨了两个丫鬟一个奶妈给她,为她照料婴儿,就宿在邻房。段-璋走过去看,孩子正在熟睡,窦线娘道:``这孩子骨骼还算硬朗,一个月来,丝毫没有病痛。不知他的小媳妇儿长得如何?''两夫妻想起了史家母女,不觉黯然神伤。

这一晚段-璋和他的妻子互诉别离后的种种经过,不知不觉已是五更时分,忽听得``呼''的一声,一道白光从窗口飞进来!

段-璋夫妇早有防备,就在这白光一闪之间,窦线娘的一把梅花针也撒了出去,段-璋宝剑一挥,以剑光护体,紧接着窜出窗外,掠上瓦背。

窦线娘在暗器上有极高深的造诣,尤其以梅花针刺穴和金弓神弹,堪称两项绝技,岂料这一把梅花针发出,竟然毫无声息,显然并没有一枚刺中敌人!

段-璋掠上瓦背,抬头一望,但见繁星点点,明月在天,整个山寨都好似在沉睡一般,只有前山隐约传来几声打更的梆子声响,远远近近,目力所及,哪里还能发现敌人的踪迹?

段-璋气纳丹田,运用``传音入密''的上乘内功,将声音送出去道:``有胆前来,何以无胆相见?''过了片刻,只听得远远有个声音,好像是给夜风吹来似的,``嘿、嘿、嘿!''的冷笑几声,接着说道:``何必忙在一时?''声音极为轻微,但却极为清亮,人影仍然不见,段-璋听声测远,估量这声音最少是发自三里之外!这人早已是离开山寨了!

段-璋一回头,窦线娘这时亦已掠上瓦背,正在他的背后,段-璋苦笑道:``追不上了,这人的轻功远在你我之上!''窦线娘道:``这人不只轻功超妙,你再瞧瞧!''段-璋道:``怎么?''窦线娘道:``你瞧,在瓦背上和地下可曾发现一枚金针?我那一大把梅花针竟然都给他收去了!真不知道他用的是什么手法?''

段-璋道:``既然退已无用,咱们且回房间去看,看看他又给咱们送了些什么东西来?''

但见床头的小几上,有一柄七寸来长的柳叶刀,插着一封书柬,刀柄仍自颤动。段-璋笑道:``又是留刀寄柬的把戏!他以为凭着这手玩艺就可以吓退我,那却是看错人了。''窦线娘道:``且看看他说的什么?''段-璋取起柬帖一看,只见上面写道:``先礼后兵,留刀寄柬,限你三日,速离此山。''后面又有两行小字写道:``若还视作等闲,我将取去你们二人最宝贵的东西,叫你们终身抱恨!''

段-璋大笑道:``最宝贵的东西不过是我们吃饭的家伙罢啦!以这人的武功而言,他应该是尊人物,却怎的用这种无聊的口吻来恫吓?''

窦线娘道:``是呀,我觉得奇怪的,就正是这个地方!''段-璋心念一动,已知道了妻子这说话的意思,试想以这人的本领而论,不管其他武功如何,凭着他这轻功,即算是光明正大的出来,和他们夫妇相斗,亦已立于不败之地!何以他却好像害怕自己来助窦家?一而再的想把自己吓退?

门外有急促的脚步声奔来,段-璋打开房门,只见窦令侃。窦令符、窦令策、南霁云、铁摩勒等人,不约而同来到。

段-璋把那张柬帖给窦令侃看了,窦令侃的脸色唰的一下全都变了,喃喃说道:``这一定是空空儿,这一定是空空儿!听说他是精精儿的师兄,现在果然给师弟撑腰来了!''窦令符是北方的绿林领袖,但一提起``空空儿''三字,却有如寻常人``谈虎色变''一般,可见空空儿虽仅出道几年,行踪所至,已足令武林高手闻名胆丧。

段-璋朗声大笑道:``我既然答应了大哥,死而无悔,管他是精精儿也罢,空空儿也罢,好坏也得和他们一斗,我倒要看空空儿有什么手段,能在三天之内,取去我项上的人头!''他兀自以为柬帖上所说的``最宝贵的东西'',乃是他的首级。

窦令符渐渐镇定下来,和声笑道:``圭璋,你隐居十载,豪气仍是不减当年!好,你都不怕,咱们窦家五虎又岂是怕事之人?传令下去,叫头目们在这三天之内,分班守夜,寨里塞外,小心戒备。咱们有这么多人,又有南大侠在此,空空儿何足惧哉!''话虽如此,但看他如此戒备,当真是如临深渊,如履薄冰,内心的恐惧与紧张,已是不言而喻。

窦家寨上下人等,都在严密的防备,段-璋夫妇也轮流守卫,在紧张气氛中过了三天两夜,平安无事。这一晚是最后的一晚,寨中各处灯火通明,人人都忘了睡意,即算是不需要他轮值的人,也都睁大了两只眼睛,等着发现空空儿的踪迹!

大约三更时分,大寨的西北角忽地发出一声喊道:``空空儿来了!''段-璋夫妇在房中守卫,听到这声叫喊,窦线娘拿起弹弓,便要出去。就在这时,忽又听得东北角也有人叫道:``空空儿来了!''片刻之间,四面八方,都有``空空儿来了''的告警之声。

段-璋大吃一惊,猛听得``嘿。嘿、嘿''的冷笑声,就传到了房外,正是那晚听到的笑声,段-璋大喝一声,就拔剑冲出去,就在这瞬息之间,猛又听得窦线娘大叫一声:``不好!''随即便听得婴孩``呜哇''的哭声,丫鬟奶娘纷乱的叫声,只见一条黑影,已是从后房窜出,一溜烟的往西奔去,眨眼之间,已掠过了十几间瓦面!

段-璋做梦也想不到空空儿会偷走他的孩子,这一急非同小可,施展了全副轻功,明知追不上也要去追。两人各显神通,有如追风逐电,把其他人众都抛在后面,一直追到了山边,初时段-
璋还可以看到一个黑点,不多一会,连黑点也在淡淡的月光下消失了!

窦线娘方自赶到,一见丈夫这副神情,不必再问,已知不妙。他们婚后十年,方始得子,当然是疼爱异常,两夫妻面面相觑,心乱如麻,不知说什么好,段-璋还勉强忍住,窦线娘已不禁滴下泪珠。

片刻之后,窦令侃等人亦已赶到,窦线娘``哇''的一声,哭了出来,硬咽说道:``大哥,你的外甥丢了。''窦令侃满面羞惭,只好说道:``六妹,你暂且忍住,咱们回去再从长计议。''

回到山寨,窦令侃唤齐了兄弟与段-
璋夫妇在密室之中商量,奏家威震绿林数十年,这一次在大寨严密防备之下,竟然给空空儿来去自如,如入无人之境,要拿什么东西,简直就似探囊取物一般!这样的奇耻大辱,比上一次惨败给精精儿更甚!是可忍,孰不可忍,窦家五虎个个怒发冲冠,有人主张向空空儿下战书,有人主张将王伯通的家小也掳掠来,迫他交换,议论纷纷,莫衷一是。

窦令侃道:``那空空儿神出鬼没,居无定所,到哪里去给他下战书?要是请王伯通或精精几代转,这只是惹人笑话而已!''要知武林规矩,向人挑战,战书必须送给本人,请人代转,那就是说明自己没有本事找到正主,何况还要请敌人的朋友代送战书,那就更是大大的笑话了。卖家是北方的绿林领袖,大盗世家,当然不能够这样做。

窦令策道:``这么说,只有掳掠王伯通家小这一法了。''段-璋猛地起立,高声说道:``大丈夫光明磊落,那空空儿用这等下三流的手段,咱们岂可效他所为!''

窦令侃叹了口气,说道:``这也不行,那也不行,咱们只好认栽了吧!六妹,你们夫妇俩明日下山,不必再趁这趟浑水了。我们向王伯通、精精儿低头认输,把地盘让与他们!想那空空儿劫走你们的孩子,用意也不过是想你们退出这场纷争而已,你们退出之后,他要婴儿何用,自然交还。''

段-璋心念一动,记起了明日便是精精儿与窦令侃的约会日期,当下朗声说道:``大哥此言差矣!如此一来,不但窦家声名尽丧,我段某从此也无颜在江湖立足。精精儿明日要来,我即算不是他的对手,也非得与他一战不可,若然侥幸得胜,空空儿自必要站出来,到时,我夫妇俩与他决一生死!''

窦令侃刚才那番说话,正是激将之法,如今由段-
璋自己说出来,正合他的心意,当下说道:``妹夫英名盖世,倒是我失言了!对,大丈夫宁死不辱,事已如斯,只好与他们一拼!说不定明天空空儿便要与他的师弟同来!''

正是:丈夫岂肯遭人辱?仗剑弯弓待敌来。

欲知后事如何?请听下回分解------

\chapter{第十一回 神弹宝剑逢强敌
血雨腥风起绿林}\label{ux7b2cux5341ux4e00ux56de-ux795eux5f39ux5b9dux5251ux9022ux5f3aux654c-ux8840ux96e8ux8165ux98ceux8d77ux7effux6797}

主意已定,各自回房歇息。段-璋夫妇虽然心里愁烦,但为了要应付强敌,只好暂且抛开忧虑,回到房里,便静坐运功,养足精神,准备明日的决战。

第二日一早起来,大家都怀着紧张的心情,等待王伯通和精精儿前来赴约,直等到中午时分,尚未有消息。大家正在议论纷纷,等得不耐烦的时候,忽听得呜、呜、呜的三声响箭,那是绿林中的挑战讯号,果然响箭过后,便有一个头目进来报道:``精精儿请几位寨主山前打话!''

窦家五虎执起兵器,立即便冲出去,段-璋、南霁云等人是客,跟在后头,到得山前的那一片大草场,但见草场空荡荡的,只有一个瘦削的貌似猢狲的汉子!铁摩勒对段-
璋悄声说道:``这便是精精儿!''

这次约会,是王伯通与窦令侃说好了来讨他的回复的,或战或降,就要在这次会面决定。所以这约会虽然是精精儿与王伯通联同出名,但主体还是王伯通。窦令侃见只有精精儿到来,不觉一怔,他以为王伯通已知道了自己请到了段-璋,最少也会带几个大头目前来赴会,哪知仍然是只有精精儿一人,相形之下,自己这边就显得过份紧张了!

窦令侃按下怒气,上前问道:``王寨主呢?''精精儿笑道:``你的降表写好了没有?写好了就交给我带回去,王寨主收了你的降表,自会前来!''

窦令侃勃然大怒,但他是绿林领袖的身分,盛怒之下,反而纵声笑道:``现在就说这话,不是太早了么?好,王寨王既然未来,我与他两家的事情暂且不提,这里有位朋友,先要和你算一笔帐。''

段-璋大步向前,面对着精精儿冷冷说道:``昨晚之事,是否你的师兄所为?''精精儿笑道:``什么事啊?''段-璋``哼''了一声道:``你不怕说出来丢脸么?你们若要伸量段某,段某一准奉陪,何必要劫走我刚满月的婴儿,这算是哪门子的好汉行径?''

精精儿哈哈笑道:``原来你说的是这件事呀?不错,那是我师兄所为!我师兄是爱惜你的声名,不想你身败名裂。一番好意,才屡次劝告你,谁叫你不听他的话?''

段-璋``呸''了一口道:``这样的`好意',恐怕只有不要脸的下三流人物才说得出口。好,闲话少说,叫你师兄来吧!''

精精儿沉声说道:``你再骂我的师兄,我就要对你不客气了!你莫以为你有个`大侠'的名头,我师兄却还未曾把你放在眼下呢!你要会我的师兄还早一点,先会会我这口剑吧!怎么样,是你一个人上呢?还是你们都一齐上?''这话说了,只听得唰、唰两声,段-璋和精精儿的宝剑都已拔了出来!

段-璋冷冷说道:``你们劫走的是我的孩子,与他们无关。你们师兄弟既然是冲着段某一人而来,段某敢不舍命奉陪?不管是你一人或是和你师兄同来,都由段某一人领教便是。''精精儿哈哈笑道:``好大的口气,果然不愧有大侠之称。但这孩子不只是你一个人的吧,我也还想领教领教尊夫人的神弹绝技呢!''窦线娘亢声说道:``我弹弓不打无名之辈,你赢得了我丈夫的这口剑再说!''高手比斗,争的是个面子,但窦线娘这口气在冷傲之中却实是软了几分。

精精儿一声长啸,弹剑笑道:``好,那咱们就来比划比划吧!段大侠,你是半个主人的身份,客不僭主,请赐招!''

段-璋虽然痛恨他们行事卑鄙,但为了保持大侠的身份,仍然虚晃一剑,让他半招。精精儿喝道:``好呀,你是存心看不起我么?''说时迟,那时快,长剑一起,闪电般的便向段-
璋刺来,这一剑来得凌厉之极,而且是脚踏中宫,平胸刺到。武学有云:``刀走白,剑走黑'',即是说剑势采的多是偏锋,而今精精儿第一剑就从正面攻来,不依剑术的常理,显然是存心蔑视。

段-璋大怒,身形纹丝不动,陡然间剑把一翻,一招``金鹏展翼'',斜削出去,这一招拿捏时候,恰到好处,精精儿的剑尖堪堪刺到,招数稍嫌用老,劲道已减了几分。而段-
璋则是养精蓄锐,剑招初发,正合兵法上``避其朝锐,击其暮归''的道理。观战的窦家兄弟和南霁云等人,都是武学的大行家,见段-
璋第一招就使得如此妙到毫巅,禁不住便轰然喝起彩来。

喝彩声中,但听得``嚓''的一声,火花四溅,精精儿腾身跃起,借段-璋这一剑反弹之力,来势更疾,凌空击下,迁刺段-
璋背心的``风府穴'',段-璋反剑一圈,又是``嚓''的一声,精精儿身形落地,斜窜三步,段-璋收势不住,也不由自己打了两个盘旋。

双方使的都是最上乘的剑法;虽然仅仅两招,却已曲尽攻守之妙,哪方稍有不慎,便要血染黄砂,当真是惊险绝伦,喝彩声登时都静止了。

精精儿赞道:``段大侠果然名不虚传!''段-璋却暗暗叫声``惭愧''!他通晓各派剑法,却看不出精精儿的剑术渊源。

精精儿一言甫毕,举剑又攻,这时彼此都已知道对方是个劲敌,谁都不敢再存半点轻敌之心。精精儿那柄剑黑黝黝的毫不起眼,而且刃口似乎甚钝,看来就似一片铁片一般,但以段-
璋的宝剑,他竟然硬接了几下,剑身上仍是毫无伤痕。

精精儿杀得性起,运剑如风,剑剑指向段-
璋的要害穴道,在场观战的都是武学行家,但这样精妙的剑术几曾见过?南霁云倒吸了一口凉气,心里想道:``难道他竟然得了失传的袁公剑术么?''袁公是战国时代的剑术名家,相传是一个老猿的化身,故名袁公,这当然是个神话,但由此也可知道他的剑术以轻灵矫捷见长;南霁云曾听得师父讲过,说是用剑刺穴之法,始于袁公,代远年湮,久已失传,到了本朝初年,武林怪杰虬髯客苦心钻研,重擅此技,可以在一招之内,刺敌人三处穴道,因而名震天下。但据传袁公剑法,却可以在一招之内,同时刺敌人九处大穴,因此若拿虬髯客比之古代的袁公,仍不过是小巫之与大巫。现在南霁云全神注视,见精精儿的刺穴剑术,已可以在一招之内,连袭段-
璋的七处穴道,虽未达到袁公剑术的最高境界,但比之虬髯客却胜得多了。故此以南霁云这样的大侠身份,也不禁触目惊心!

段-璋不愧是久已成名的大侠,精精儿的剑法虽然奇诡绝伦,他仍是丝毫不乱。一个攻得迅疾,有如天风海雨,迫人而来;一个守得沉稳,有如长堤卧波,不为摇动,但见他顺势破势,解招还招,当真是剑挟风雷,招招都见功力!

两人越战越紧,斗到酣处,精精儿展开凌厉异常的招数,进如猿猴窜枝,退若龙蛇疾走,起如鹰隼飞天,落若猛虎朴地,瞬息之间,四面八方,全是精精儿的剑影!但段-
璋仍是双足牢牢钉在地上,精精儿连番外击,也攻不进他周围七尺之内,斗了已将近半个时辰,段-璋兀是未曾移动一步!

虽然如此但看来段-
璋乃是处在下风,窦线娘手把弹弓,看得触目惊心,手心淌汗。精精儿的攻势有如长江大浪,一个接着一个,竟似不知疲倦似的,处此情形,人人都会想象得到:只要段-
璋的防守稍有隙罅,身上就得平添七个透明的窟窿,而且受伤之处,必然是重要的穴道方位,饶是他功力更高、也难保全性命了。

窦令侃沉声说道:``六妹,对付这样的魔头,还和他讲什么武林规矩!''话犹未了,忽见精精儿使出``俊鹃摩云''的身法,冲天而起,在半空中一个倒翻,头下脚上,向段-
璋冲来。这一招有如雷电交轰,只要双剑一触,便要优胜劣败,生死立判。窦线娘无暇思量,本能的将弹弓一曳,三颗金丸已是闪电般的向精精儿射去!

但听得一声刺耳的啸声;倏然间,满空剑光,全都收敛,窦线娘奔上前去,反手一抄,将两颗反弹回来的金丸抄在手中。睁眼望时,但见精精儿已似流星陨石般坠下山谷,他穿着一身黑色衣裳,远远望去,又似一溜黑烟,眨眼之间,便已随风而逝!

地上有几点淡淡的血渍,段-璋吁了口气,道声:``惭愧!''缓缓插剑归鞘。

原来刚才正在他们双剑相交的时候,窦线娘的三颗金丸射到,金丸沉重,窦线娘又是用尽浑身气力,弓如满月,弹似满星,劲力当然要比那晚撤出的梅花针强得多。本来以精精儿的本领,窦线娘的神弹绝技,虽然厉害,他还可以抵挡得住,但在那一瞬间,他正在与段-
璋全力相搏,可就有点难于照顾了。

饶是如此,精精儿仍然将两颗金丸反弹回去,第三颗金九正打中他的剑脊,高手比剑,相差毫厘,他的剑稍稍一震,剑尖便歪,贴肋而过,没有刺中段-璋的穴道,而段-
璋那一剑却把他伤了。

众人目睹这惊心动魄的一幕,精精儿的影子已消失了,他们还未曾透过气来。过了好一会,铁摩勒方始大叫一声:``妙呵!''接着众人才轰然喝起彩来!

窦令侃上前致贺,喜不自胜,段-璋却是没精打采,毫无胜利后应有的欢欣。要知他自从出道以来,这次还是第一次要人相助,方能打退强敌,自觉胜得并非光采,何况精精儿在受伤之后,自己仍然不能够追上他,因此心中只觉惭愧。

窦令符笑道:``妹丈这次伤了精精儿,咱们也出了口乌气!只可惜还是让地逃了。''

窦线娘叹了口气,道:``这一仗虽然打赢了,但他逃得无影无踪,却去问谁要回我的孩子?''

窦令侃道:``六妹放心,除非空空儿与王伯通甘心认输,否则他们总不能缩头不出。咱们且先回去喝庆功酒去!''

寨里的头目得知消息,早已在大厅上摆开庆功宴。筵席间窦令侃哈哈笑道:``十年不见,-璋,你的剑法越发精妙了。空空儿虽然比他的师弟高明,也定然不是你们夫妻的对手!''铁摩勒担忧道:``那空空儿几次三番对姑丈恐吓,想迫他下山,看来也是有自知之明,怕不是姑丈的对手。我就担心他不敢再来呢!''窦令侃是给段-
璋壮胆,铁摩勒却是真心为他担忧,怕空空儿不来,难以讨回孩子。段-璋摇了摇头,道:``摩勒,你岂能这样小视敌人!''话犹未了,忽听得窦令侃失声叫道:``咦,这是什么?''

众人随着他的目光注视,只见正中的横梁吊着一小匣子,窦令策扬手一柄飞刀将绳索割断,窦令侃将那个小匣子接到手中。他是黑道上的大行家,一触手便知里面并无机关、暗器,当下打开一看,里面是一张大红帖子。窦线娘坐在她哥哥的侧边,看得分明,失声叫道:``这是空空儿的拜帖!''

窦家五虎面面相觑,尽都呆了!在这白日青天,又是众目睽睽之下,空空儿将拜匣吊在他们头顶上的横梁上,竟然无人发觉!若非目睹,当真是难以相信!

过了半晌,窦令侃心神稍定,方始大声喝道:``既已前来,为何不敢露面?鬼鬼祟祟,躲躲藏藏,算哪门子好汉?''

话犹未了,只听得一阵狂笑的声音,笑声中但见一条黑影,已是疾如飞鸟般地落在筵前,朗声说道:``我早已来了,你们都是瞎了眼睛的么?''

这一瞬间,但听得咣啷啷、哗啦啦一片声响,席上诸人不约而同的都站了起来,亮出兵器。除了段-璋,南霁云二人沉得住气之外,其他的人,或多或少,都不免有些慌张,把桌子上的杯盘碗盏都碰翻了。

空空儿哈哈笑道:``怎么,我一来你们就想群殴了么?''

这几年来,空空儿名震江湖,但席上群豪,却是直到如今,方始见到他的本来面目。只见他身材不满五尺,相貌十分特别,一副``孩儿脸'',活像一个大头娃娃,说话之时,手舞足蹈,狂傲之气迫人!

段-璋越众而出,冷冷说道:``枉你有这副身手,干的却是江湖宵小所为,武功再高,又有什么可做?''

空空儿冷笑道:``你枉有大侠的名头,如不分皂白的来替绿林大盗争权夺利,这又有什么可傲?''

段-璋怔了一怔,窦令侃大怒道:``那王伯通不也是绿林大盗么?他也不见得比我好到哪里去,你又为什么充当他的打手?''

空空儿笑道:``一来我不是什么大侠,王伯通与我有交情,我就帮他;二来嘛,说到在绿林中的横行霸道,那王伯通却还逊你一筹。沙家庄的案子是你做的不是?你黑吃黑也还罢了,却为何将沙家父子斩尽杀绝?凤鸣岗劫掠药材商人的案子是你做的不是,那年流行瘟疫,你劫了药材,却用来囤积居奇,害死了多少人,你知道不?要不要我将你的所作所为一件件抖出来?要不然,为了公平起见,你说王家一件坏事,我也说你们窦家一件坏事,就让这位段大侠来评评理,你们两家准做的坏事多,如何?''

王、窦两家同是绿林``世家'',但这几十年来,窦家的势力大盛,远远压倒王家,因此若然论到所做的坏事,那当然也是窦家多了。这些坏事,在绿林中人看来,实在算不得什么,即以空空儿所举的两件事例来说,窦令侃只是对同道中的敌人斩尽杀绝,并未伤及寻常客商,那已经算是好的了。可是在段-
璋听来,却不禁出了一身冷汗。要知他当年和窦线娘结婚之后,不久便逃出窦家寨,一去十年,不肯与窦家再通音讯,便是因为他不甘随波逐流,在绿林厮混的缘故。而他对窦家的所作所为,也仅是知而不详,故此听了空空儿数说窦家的罪恶,心头不禁惶恐起来,暗自想道:``我来趁这趟浑水,当真是糊涂了!''

``砰''的一声,窦令侃拍案骂道:``干我们这一行的,哪有不伤人劫物之理?就算我用劫来的药材求些微利,那也是以性命搏来的!你这小子不懂黑道规矩,少来说话!''

窦令符也骂道:``那王家与安禄山的手下勾结,借官府之力,伤残同道,更是下流!你若是要评理的话,咱们也可以按照黑道的规矩,邀齐绿林中有头面的人物来评评!''

空空儿笑道:``我才没有那么多工夫!''

窦令侃兄弟同声喝道:``那就废话少说,照咱们绿林的现矩办事,胜者为强!''

空空儿侧目斜睨,冷冷说道:``段大侠,你不是黑道中人,你又怎么说?''

窦家兄弟和窦线娘的眼光全都望着他,段-璋踌橱片刻,缓缓说道:``绿林的纷争我不管,你夺了我的孩子,欺负到我的头上来,我是非和你一战不可!''

空空儿哈哈笑道:``我正是要你这句话!我知道你倘非与我一战,也难以在亲戚面前交代。''话声一顿,接着正容说道:``好吧,那么咱们就一言为定,你若输了给我,从今之后,就再也不许管王、窦二家的事情,我若输了给你,也是一样。比剑之后,不管胜败,我都把你的孩子送还,这个办法,总算公平合理了吧?你意如何?''

原来空空儿、王伯通之所以要追段-璋退出纷争,倒不是为了怕他一人,而是因为他相识满天下,怕他帮助窦家到底,广邀高手,那牵连就大了。

段-璋一听,正合心意,双眉一轩,立即朗声说道:``依你之言便是!请亮剑吧,咱们就在这里一决雌雄!''

空空儿道:``且慢!''转过头来,面向窦令侃说道:``我和段大侠是按武林规矩办事。你呢,咱们该按你绿林的规矩办事了吧?''

窦令侃冷冷说道:``只你一人在场,教我与谁说去?''言下之意,即是说愿意按照规矩办事,但必须王伯通才行。要知空空儿的名气虽然已经盖过了王伯通,但他与窦令侃乃是对等身份,这身份却是空空儿不能替代的。窦令佩为了保持他绿林领袖的尊严,自是非与王伯通当面打交道不可。

空空儿道:``这个容易!''忽地一声长啸,啸声未毕,只听得一个宏亮的声音从外面送进来道:``燕山王伯通拜会窦家寨主!''原来王伯通早已与空空儿约定,只待空空儿与窦令侃讲好后发出讯号,他便现身,他把时间算得很准,这时刚好到了大寨门前。

窦令侃面色微变,立即朗声说道:``打开大寨正门,请王寨主进来,休得失礼!''

片刻,只见一个年近六旬、满面红光的老者,携着一个少女,在众人注视之下,走了进来。那少女不过十六七岁的年纪,一对黑溜溜的眼睛左顾右盼,好像感到非常好玩的神气!一见空空儿便嚷道:``叔叔,你们还未曾比剑吗?''

空空儿笑道:``就等着你爹呢。怎么是你来了?你的哥哥呢?''那少女道:``我特地来瞧热闹呢!我哥哥另有客人,这眼福他只好让给我享了。''

南霁云心中一动,他已经知道了那日截劫驴车的那个黄衣少年乃是王伯通的儿子,心中想道:``那小子接什么客人,莫非是夏凌霜么?''夏凌霜那日对黄衣少年的神气颇为异样,南霁云瞧在心中,一直为此事感到不快,这时听了王伯通女儿的说话,胡乱猜疑,更觉心头烦乱,连自己也不知道是什么道理,好不容易才将这烦乱的情绪按捺下去,暗地自嘲:``他的客人是不是夏姑娘,又干你什么事了?''

王伯通道:``燕儿,你怎的这样放肆,还不快与窦家伯伯见过礼。这个小妞儿,都是我把她宠坏了,窦大哥休得见笑。''

窦令侃哈哈笑道:``咱们哥儿俩还讲这个客套吗?还是来谈谈今日的这桩交易吧。''

王伯通道:``你们不是讲好了吗?依绿林的规矩便是,我没有二话。''

窦令侃像背书似地念道:``胜者称雄,死伤不究。败者退出绿林,部属另归新主,如有不愿者,亦可自行散去,但不得再作黑道营生!''

王伯通道:``对,这些规矩,你记得非常清楚,就这样办!不过,窦大哥呀,我为你着想,可想奉劝你一句。''窦令侃道:``王大哥有何金玉良言,小弟洗耳恭听!''这两个盗魁称兄道弟,若是不知底细的人,看到他们现在的模样,哪想得到他们乃是生死世仇,而且片刻之后,就要展开你死我活的恶战!

王伯通笑道:``照这黑道的行规办事,干脆得很,只是我怕你却不免吃亏,咱们哥儿俩到底是有几十年交情的了,一旦失了对手,我也会觉得难过的啊!为你着想,不如就此金盆洗手,立下一张凭照给我如何?''

这话的意思即是劝窦令佩向他呈递降表,从此永远退出绿林,免得送命。窦令侃怒极气极,反而哈哈大笑道:``多谢王大哥的关注,小弟也正是想这样奉劝王大哥。大哥远道而来,要是在小寨里吃了亏,有什么三长两短,小弟也是难过的啊!''

因为照这规矩:``胜者称雄,死伤不究''。在双方都有人助阵的形势下,窦令侃却是占了地主之利。这话等于明说窦家将尽全力和他们一拼;而王伯通这方,连他的小女儿在内,也不过三个人。

王伯通微笑道:``既然窦兄执意不从,小弟只好奉陪了。好啦,彼此想开一点,死生由命,大家都不必难过啦!好,好,咱们且先看这一场百年难遇的比剑!''

空空儿招手道:``段大侠,他们已把话说清楚了,现在是咱们的事了。不过,刚才有一句话还未说到,久仰段夫人是女中豪杰,不知可也肯依照武林规矩,一并赐教么?''言内之意,即是向段-
璋夫妇挑战,要是他胜了的话,窦线娘也不能管她母家的事情。

段-璋眉头一皱,随即望着他的妻子,沉声说道:``也好,要是我不成了,你再来吧!''段-璋知道空空儿的本领远胜他的师弟,单凭自己这口宝剑,九成落败,他也知道自己若然落败,窦线娘断无坐视之理,因此不如把话说明了,夫妻联手合斗,更漂亮一些。窦线娘点了点头,表示同意。

空空儿道:``段大侠,刚才你和我师弟过招,起手一式,曾让我师弟半招,现在我得请你先行赐招了。''段-璋心中一凛,这才知道,在他和精精儿动手的时候,空空儿早已在旁窥伺。

``唰''的一声,段-璋宝剑出鞘,朗声说道:``请亮兵刃!''

空空儿双手空空,随身也未配戴兵刃,段-璋听他一来就提出要比剑,以为他用的是可以作腰带的软剑之类,哪知空空儿却淡淡说道:``段大侠,不必客气,这一招是由你先行出手,但请赐教便是。''

段-璋怒道:``你要凭空手对我的宝剑么?段某纵然无能,也决不能如此与你动手。''空空儿笑道:``不敢,不敢!段大侠尽管出剑。''

段-璋怒气暗生,心中想道:``我倒要瞧你拔剑的身手。''立即一招``玄乌划砂'',向空空儿当胸划去!

这一招当真是静如处子,动如脱兔,但见白光一闪,剑尖已划到胸前!纵算空空儿有软剑之类的兵刃,亦已来不及解下防御,在场的都是武学行家,见段-
璋一出手就是如此凌厉迅速的剑招,都不自禁的为空空儿捏了一把冷汗。

众人心念未已,就在电光石火的刹那间,只听得空空儿一声笑道:``礼尚往来,现在我可还招了!''笑声未了,但见他右掌一翻,一道蓝艳艳的光华,已是电射而出,``嚓''的一声,火花四溅,段-璋身形一晃,接连退了三步!

原来空空儿用的竟是一把短到出人意外的短剑,仅有七寸来长,比普通的匕首还要略短几分,这柄短剑,他早已笼在袖中。

这柄短剑蓝光湛然,锋利之极,交手一招,段-璋的宝剑非但削不断它,反而给他在剑脊上划了一道淡淡的伤痕,不由得心中大骇!

说时迟,那时快,空空儿的``还招''二字出口,段-璋立足未稳,空空儿已是如影随形地扑了过来。段-璋也真了得。身形向后一仰,``嗖''的一声,那柄短剑在他面上掠过,段-璋也即还了一招``李广射石'',挽剑刺他的手腕!

空空儿赞道:``临危不乱,果然不愧大侠之称!''一侧身,从段圭漳的剑下窜出,反手便刺他胁下的愈气穴。段-璋连遇险招,几乎透不过气来,迫得又退了三步,但他虽然连连后退,步法剑法,依然不乱!

武学有云:``一寸短,一寸险。''空空儿以匕首般的短剑进招,竞似近身肉搏一般,但见剑光飘瞥,虎虎风生,短剑所指,处处都是段-
璋的要害!旁观诸人中武功最高的南霁云也看得汗流心跳,心中想道:``要不是段大哥有这份沉着镇定的功夫,只怕早已落败了!''

段-璋斗精精儿的时候,半个时辰,未曾移动一步,如今斗空空儿,只不过十来招,却已显得只有招架的份儿,腾挪闪展,左趋右闪,兀是摆不脱那柄短剑的近身攻击,两个人就似缠在一起的,空空儿的那柄短剑,在他身前身后,身左身有,穿来插去!窦线娘见不是路,急忙发出暗器。

窦线娘的暗器功夫已到了出神入化的地步,双手齐扬,右手发出了七枚金丸,左手撤出了一把梅花针,七枚金丸袭向空空儿的七处大穴,梅花针则射向他面上的双睛,因为距离甚近,梅花针的份量极轻,与金丸一同发出,无声无息,更难防备。刚才窦线娘只用三枚金丸就打伤了精精儿,她料想空空儿的本领,纵然强过师弟一倍,至多也只能避开那七枚金丸,这一把梅花计定然可以把他的眼睛射瞎!

空空儿叫道:``好个暗器功夫!''身形一转,蓝光疾闪,但听得叮叮咣咣之声不绝于耳,接着是一片``哎哟,哎哟!''的叫声,那七枚金丸流星陨石般的飞向四方,窦令侃舞起一面金牌,将飞到他面前的金丸碰落,窦令符、窦令策在他左右,没有受伤,但他的五弟窦令湛却给金丸打中了腔骨,还有两个大头目伤得更惨,给金丸打破了头颅。

空空儿短剑一挥,笑道:``梅花针也还给你吧!''但见他的剑尖上银光灿烂,结成了一个丸形的小球,配上他那短剑本身发出的蓝色光华,更为悦目。原来那一把无影无形,逢隙即入的梅花针,竞然一支不剩,都给他吸在剑尖上,竟如磁石吸铁一般。空空儿短剑一挥,但听得哗啦声响,剑尖上的小圆球化成碎粉,有如满空飘落的雪花!

窦线娘骇然失色,只听得空空儿又叫道:``段夫人,你的暗器功夫已经见识过了,还有游身八卦刀法,亦请不吝赐教。''他口中说话,手底却是毫不放松,就在说话之间,已接连攻出了六七招凌厉之极的剑招,把段-
璋又迫退了三步!

窦线娘叫道:``好,我夫妻与你拼了!''抽出两把柳叶弯刀,一长一短,立即向空空儿攻去!

窦线娘自小得她父亲疼爱,全副本领几乎都传了给她,这游身八卦刀法,便是窦家的家传绝技之一。

但见她双刀一展,霍霍风生,刀光如练,登时将空空儿圈在当中,她随着空空儿游身疾走,当真是只见刀光,不见人影,只要空空儿稍有疏漏,她就要在他身上戳个透明的窟窿,以报爱子被抢之辱。

段-璋见妻子来援,精神陡振,宝剑一挥,剑光暴长,有如洪波溃堤,也立即反攻出去。空空儿在他夫妻夹击之下,攻势顿然受挫,只得回剑防身。不过段-
璋身受的压力虽然减轻,但心头却更为沉重,不由得暗暗叫了一声:``惭愧。''

窦令侃见他们夫妻已经稳住阵脚,正自宽心,猛听得空空儿一声长啸,陡然间,但见剑气纵横,白刃耀眼,到处都是空空儿的影子,竞似化身千百,从四面八方攻来,登时反客为主,把段-
璋夫妇圈在当中。原来空空儿聪明绝顶,他竟然在不到一往香的时刻,便把窦线娘那套刀法的精华勘破,立即反守为攻。

窦线娘的游身八卦刀法,必须以极轻灵迅捷的步法配合,然后才能按着五门八卦方位,困扰敌人。现在空空儿也按着五门八卦方位与她游斗,而他的轻功则远在窦线娘之上,因此窦线娘不论走到哪个方位,都给他堵住,他以一敌二,兀是攻多守少,段-璋在他疾风暴雨般的攻击之下,剑法也渐渐施展不开。

这时,旁观人等,除了南霁云和窦令侃之外,根本就分不出何方主攻,何方主守,但见剑气纵横,幢幢人影,聚义厅内竟似有千军万马追逐一般!人人都感到冷气沁肌,寒风扑面!

窦令侃暗自叫声``不妙'',杀机陡起,向兄弟们抛了一个眼色,忽地站了起来,朗声说道:``王寨主,咱们也凑凑热闹吧!''抡起两面金牌,不待王伯通答话,立即便是一个``雪花盖顶'',向他当头压下!与此同时,窦令符长臂一伸,也向王伯通的女儿攻去!

本来今日王、窦两家之会,窦家乃是地主,双方都有助拳的人,若然按照绿林礼节,窦家应当等到助拳的分出胜负之后,方可以下场动手;但窦令侃已看出了段-
璋夫妇败象毕露,心中一想,要是让空空儿得胜之后,再行围攻,那定然是凶多吉少,不如抓着时机,以图侥幸。要知窦家若是一战而败,便要退出绿林,甚至性命不保,窦令侃焉能心甘?因此只好不顾绿林领袖的身份,先行发难!

窦令侃自忖武功胜过王伯通,王伯通的女儿,更不在话下。只要将他们父女擒获,空空儿本领再高,也是无能为力了。

他们两兄弟同时出手,窦令侃的金牌刚要压下,忽听得窦令符一声惨呼,白光闪处,一条臂膊已给那少女齐根切下,那少女娇声笑道:``窦伯伯,侄女第一次到你家来,你却这样款待,不嫌太过份了么?礼尚往来,请恕侄女也放肆了!''声到人到,窦令侃抡起金牌一挡,只听得一片断金戛玉之声,就在这交手一招的刹那之间,那少女的短剑已在他的金牌上连刺了十七八下!

窦令侃是``窦家五虎''之首,身为绿林领袖,本领高强,自是非同小可,但吃那少女一轮急攻,虽然没有受伤,却也给追得连连后退。窦令符一声怒吼,顾不得包扎伤口,独臂抡刀,便扑上来!窦令申、窦令策、窦令湛也都亮出了兵器,形成了窦家五虎,围攻王伯通父女的场面。

那少女娇声笑道:``我陪窦家几位伯伯耍耍,爹爹,你坐着瞧热闹吧!''短剑一招``指天划地'',左刺窦令申,右削窦令湛,窦令湛刚才被金丸打伤了股骨,跳跃不灵,被那少女一剑削去了膝盖,痛上加痛,一声惨呼,仆倒地上。包围圈开了一个缺口,王伯通走了出去,大马金刀的坐在聚义厅正中,窦令侃日常所坐的那张虎皮交椅上,哈哈笑道:``真是初生之犊不畏虎,好,为爹的就瞧瞧热闹,燕儿,你可要小心了!''

段-璋见窦家五虎不顾体面,闹成了如此局面,心中暗暗叹了口气,长剑一晃,跳出圈子,叫道:``空空儿,我认输了。线娘,咱们走吧!''本来以他们夫妇联手之力,最少还可以与空空儿斗半个时辰,但处此情形,段-璋哪里还有心情恋战?

窦线娘心头大震,当真是进退两难,随夫?随兄?一时间踌躇莫决。这一边,她的五个哥哥,正临到生死的关头;那一边,她的丈夫脚步已踏出了门坎,要是自己不与他同走,十年的恩爱夫妻,今日便是永决了!

空空儿哈哈一笑,短剑归鞘,朗声说道:``承让了,三月之内,我在凉州玉树山清风观相待,贤伉俪随时可以前来,要回孩子!''

窦线娘有话在先,若然输了,从此不管母家的事,空空儿这话不啻将她提醒,窦线娘是女中豪杰,这``信义''二字,焉能不顾?这刹那间;虽然有如利箭穿心,但终于还是把两把柳叶刀收回,跄跄踉踉地出了门口,但感双睛发黑,地转天旋,不敢再看她兄弟一眼,段-璋回头一看,见她摇摇欲坠,急忙将她扶住,疾奔下山。

空空儿笑道:``王大哥,轮到我也来瞧热闹了。哈哈,好,好侄女,好剑法!我看,用不了十年,她的剑法就要追上我啦!''王伯通道:``兄弟,你太夸奖这黄毛丫头啦,你做叔叔的,还应该多加指教才是!''空空儿道:``好,就是火候还差一点,哪,这一剑应该稍慢一些,待敌人攻到,再削他的脉门;哪这一剑又稍为偏右了,喏,快,这一招应用`星海浮槎',可惜了,可惜了!''

正是:邀来妙手神机客,伏虎降龙谈笑间。

欲知后事如何?请听下回分解------

\chapter{第十二回 百年霸业随流水
一片机心起大波}\label{ux7b2cux5341ux4e8cux56de-ux767eux5e74ux9738ux4e1aux968fux6d41ux6c34-ux4e00ux7247ux673aux5fc3ux8d77ux5927ux6ce2}

空空儿与王伯通相对而坐,恣意谈论,旁若无人,面对这一场舍死忘生的恶战,意是视同儿戏一般。那少女得他从旁指点,剑招越发凌厉。

本来窦家兄弟以五敌一,足可以胜得那少女有余,虽然折了一个窦令湛,而窦令符又因上场轻敌,先被削去了一条臂膊,但剩下四人七臂和她恶斗,也仍是旗鼓相当。可是段-
璋夫妇一走之后,窦家寨人人都知道大势已去,空空儿纵然敛手旁观,已足令窦家四虎心惊胆战,更何况他还在不断地指点那少女如何应战。

窦令侃又惊又怒,一咬牙根,双牌一磕,使出了一招与敌偕亡的恶招,向那少女撞去,他身材高大,连人带牌,就似一座山似的压下来,空空儿叫道:``伏地回龙剑!'那少女应声倒地,短剑横披,但听得``咔嚓''一声,窦令侃的左脚自膝盖以下,已给她削掉,那少女一个鲤鱼打挺翻了起来,脚尖一挑,又把窦令策的单刀踢飞,矫声笑道:``爹爹,留不留活口?''王伯通还未曾答话,只听得窦令侃已在大声喝道:``王伯通,我身为历鬼亦必报仇,我岂能向你求饶!''猛然间反转金牌,朝自己的顶门一磕,登时脑浆进流,死于非命。

铁摩勒目睹义父惨死,心胆皆裂,痛不欲生,拔出佩刀,便要上去与那少女拼命,他脚步刚刚移动,忽觉手腕一麻,登时浑身酸软,动弹不得,话也说不出来,回头一看,却是南霁云紧紧地握着他的手臂,在他耳边低声说道:``摩勒,你千万不可妄动!''

王伯通沉声说道:``放虎容易捉虎难,窦家五虎反正是不服咱们王家的了,斩革除根,一个不饶!''那少女道了一声:``遵命!''又娇声笑道:``窦家伯伯,我奉了爹爹之命,今日给你们送行啦!''反手一剑,窦令策应声倒地,窦令符红了双眼,怒扑而来,那少女短剑一送,直插入他的心窝,还有一个窦令申,武功仅次于他的大哥,猛地喝道:``王伯通,我与你拼了!''不待那少女追来,便即飞身而起,抡拐向王伯通的顶门击下。那少女身手矫捷之极,拔出短剑,也跃了起来,如影随形,王伯通哈哈笑道:``窦老二,我还要多活几年呢!你先去和兄弟们相聚吧。''窦令申的铁拐刚要击下,只觉背心一凉,那少女的短剑已插入了他的背心。

南霁云见那少女如此凶狠,虽说他对王、窦两家都无好感,也禁不住大为愤怒。

聚义厅里还有十几个大头目,都是追随窦家多年、忠心耿耿的部下,这时尽皆红了眼睛,不顾死活,向那少女扑去。那少女展开凌厉无前的剑法,宛如晴艇点水,蝴蝶穿花,忽前忽后,忽左忽右,在人丛中穿来插去,每出一剑,都是刺向对方的关节要害,不过片刻,地上已是横七竖八的倒下了一堆。王伯通皱皱眉头,说道:``窦老大能令这些人为他卖命,确是不愧绿林领袖,令人叹服,他死也应该瞑目了。''

南霁云紧咬牙关,极力抑制自己,心里不停地向自己说道:``我绝不能卷入这场漩涡!''他拉着铁摩勒,趁这纷乱之中逃出。

忽地剑光一闪,那少女斥道:``往哪里走?''手起剑落,竞然是一招极狠毒的招数,向南霁云刺来,南霁云一侧身,双指贴着剑脊一推,那少女虎口发热,怔了一怔,南霁云护着铁摩勒已与她擦身而过。

那少女喝道:``你是谁?''短剑一招``白虹贯日'',再度指到了南霁云的背心,这一剑来得更其凶狠,南霁云反手一刀,只听得``嗤''的一声,紧接着``咣''的一响,南霁云的衣裳给她挑破,那少女的短剑亦已给他荡开。南霁云拔刀还招,回身旋步,这几个动作一气呵成,已经是快到了极点,但那少女出剑在先,他拔刀在后,仍然不免吃了点小小的亏。

那少女给他的宝刀一击,短剑险些脱手,亦是大吃一惊,当下一个飞身,再越过南霁云的前头,回身拦住他的去路,笑道:``想不到窦伯伯还埋伏有一个高手在此,通上名来,咱们再比划比划几招!''\,''

南霁云暗自叹惜:``小小的年纪,手段却如此狠辣,只怕将来武林中又要多了一个魔头了。''

那少女笑道:``你怎么不说话?是怕我的空空儿叔叔么?你不用谎,我不要他帮忙便是。你究竟是什么人?''

南霁云横刀当胸,朗声说道:``魏州南霁云!我是护送段大侠来的,并非窦家寨请来的帮手!我也不想理会你们两家的纠纷。只是姑娘着执意要赐教么,那南某也只有奉陪便是!''

王伯通啊呀一声叫了起来,``原来是南大侠,燕儿,不可无礼!''

那少女叫道:``刀伤我大哥的原来就是你么?爹------''似是想求父亲许她出手,王伯通只听了一个``爹''字,便沉声喝道:``燕儿,你回来,不可多事。''

王伯通站了起来,向南霁云施了一礼,说道:``日前小儿有所不知,冒犯虎威,还望恕罪。''说话和蔼,彬彬有礼,前后判若两人,南霁云好生诧异。

江湖上讲究的是个面子,有话道的是:``人敬你一尺,你敬人一丈。''因此南霁云纵然对他不满,也只得抱拳还礼道:``南某也不知是王寨主的公子,惶恐,惶恐!''顿了一顿,续道:``南某与段大侠同来,也得随他同去,不知王寨主可肯放我走么?''

王伯通笑道:``南大侠既然不是窦家的人,此事与你无关,我焉敢强留。''要知南霁云交游广阔,不在段-
璋之下,而且他的师父磨镜老人乃是武林三老之一,本领之高,人所难测,故此王伯通要给他几分面子。

南霁云道:``如此,多谢了。''拖了铁摩勒便走。王伯通忽道:``这个少年请留下来!''

南霁云吃了一惊,急忙说道:``他也不是窦家的人。''

王伯通道:``他不是铁昆仑的儿子,小名唤作摩勒的么?据说他是在窦家长大的。''南霁云道:``不错。他虽然在窦家长大,究竟不是窦家子弟,还望王寨主高抬贵手。''为了铁摩勒的缘故,南霁云第一次下气求人。

铁摩勒已经被南霁云点了哑穴,不能说话,但却是瞪着眼睛,狠狠地望着王伯通。

王伯通冷冷说道:``南大侠,你既知道他的来历,却不知道他是窦老大的义子么?这也算得是窦家的人了。''

空空儿笑道:``这小娃儿胆量倒大,你瞧,他对你怒目而视,敢情是正将你很入骨髓呢!''王伯通``哼''了一声,空空儿道:``且听他如何说?''双指一弹,随手发出一粒铁莲子,替铁摩勒解了穴道。

铁摩勒怒声喝道:``王伯通,你要是怕我报仇,就赶快把我杀了!''南霁云怕他上前拼命,紧紧握着他的手臂。

空空儿道:``王大哥,这娃儿真会说话,你若不放,反显得你惧怕于他了。''王伯通无可奈何,挥手说道:``好,你走吧!我等你来报仇便是!''南霁云急忙携了铁摩勒闯出寨门,但见漫山遍岭都是窦家寨的喽兵,这些人是不愿归顺王家,各自逃命的。南霁云拖着铁摩勒,展开陆地飞腾的轻功,一口气跑了十多里路,将喽兵抛在背后,但前面却仍然没有发现段-
璋的影子。

铁摩勒忽然停下步来,号陶大哭。南霁云知他满腔悲愤,索性计他先哭个痛快,然后再慢慢劝解道:``你义父一家都是在刀尖上讨生活的人,不是他杀人家,便是人家杀他,你要想开一点。''铁摩勒道:``话虽如此,但总不该死在王伯通那老贼父女之手。你看他今日要斩尽杀绝那般狠劲,做了绿林领袖,只怕比我义父还要凶暴得多。''南霁云叹口气道:``绿林中能称得上侠盗的又有多少?你父亲算是一个,通州的快马姚算是一个,其他的就很难说了。我劝你把今日之事当作一场噩梦,过去了就算了,你从此也不要在绿林中再混下去了。''铁摩勒道:``我义父于我有十年养育之恩,此仇我岂能不报?''南霁云知他正在气愤上头,劝也无用,便道:``你若执意报仇,那就更当爱惜身子。王伯通刚才放你,并非出于心愿,你要赶快离开这个地方才是。''

铁摩勒霍地站了起来,擦干眼泪,道:``南叔叔,你说了这许多话,只有这几句我听得进去,我是直性子的人,你不怪我吧?''南霁云暗暗叹息,心道:``似这等绿林中的冤冤相报,真不知何时始了?''当下说道:``你性情刚强,自是英雄本色,但刚则易折,而且也应该用在正当的地方。咳,这些话我知道你目前还是听不进去,待再过几年,要是咱们还能相聚的话,我再慢慢和你说吧。现在,咱们可得先找你的段叔叔去。''

走了一会,忽见前面一彪军马,打着一个绣有``王''字的大旗,王伯通的儿子,坐着一匹高头大马,得意洋洋,顾盼自豪,但他脸上青肿了一大块,好像刚刚和人打了一架似的。

原来他是带领人马来接收窦家寨的,在半路上碰到段-
璋夫妇,被窦线娘打了他一弹子,现在来到山下,又碰了南、铁二人,不觉一怔,心道:``空空儿是怎么搞的,怎的都让他们漏网了?''

前头那几个头目认得铁摩勒,纵马上来拿他,铁摩勒一声大喝,先迎了上去,南霁云急忙叫道:``不可!''说时迟,那时快,铁摩勒已握着向他刺来的长矛,将一个头目从马背上扯下,幸而南霁云叫得及时,铁摩勒一撒手,将那支长矛插下,就在那头目的颈项旁边,要不是南霁云阻止,这一下他就要把那头目钉在地上。

南霁云朗声说道:``王少寨主,你意欲何为?可是要和南某再见个高下么?''那黄衫少年望了他们一眼,忽然哈哈大笑。

铁摩勒怒道:``你狂什么?你家也不过是仗着个空空儿罢了。''那黄衫少年道:``是我爹爹放你们走的不是?''他见南、铁两人衣裳整洁,身无伤痕,要是曾和空空儿交手,决不可能这样全身而退。南霁云面上一红,道:``是又怎样?莫非你不服气,要将我们留下么?''那黄衫少年笑道:``我是败军之将,不足言勇,不过,你也不必在我的面前再逞好汉了。我爹爹既然放你下山,你就尽管走路吧!''令旗一摆,左右让开,南霁云不知怎的,自从那日之后,一直就对这少年有憎恶之感,如今听了他这番讥刺,怒气更增,刚要发作,猛地心头一跳:``我刚才还劝铁摩勒不可轻举妄动,怎的我却反而失了常态了。''当下把冲到口边的回骂咽了下去,携了铁摩勒便走。

再走了约莫十里光景,南霁云眼利,远远瞧见前面一棵树下有两个人,正是段-
璋夫妇。南霁云唤道:``大哥、大嫂,小弟和摩勒来了!''段-璋应了一声,声音苍凉之极,窦纷娘目光呆滞,默然不语,直听到铁摩勒在她面前``哇''的一声哭了出来,才好似在噩梦中醒来一般,全身抖了一下,颤声道:``怎么啦?他们,他们------''铁摩勒哭道:``我义父死了,四位叔叔也全部死了。姑姑,你,你------''窦线娘知道铁摩勒是要请她报仇,面上的肌肉抽搐了一下,沉声说道:``是空空儿下的毒手么?''铁摩勒道:``不,是王伯通那个女儿,这小丫头比空空儿还要狠毒三分。姑姑,你------''窦线娘神色如冰,冷得令人心里发抖,铁摩勒不觉噤声。

出乎意外,窦线娘并没有哭,但那神情比号陶大哭更要令人难过,过了好一会子,始听得她喃喃自语道:``我怎有面目见我的哥哥于地下?-璋、-璋------''

段-璋凄然说道:``线娘,别的事情我可以从命,只有这一件事情,我不能从命。''他们夫妻俩心意相通,段-璋知道妻子想说的是什么,而窦线娘也知道丈夫是为了守他与空空儿的信诺,决不肯为她兄弟报仇了。

窦线娘忽地抬起眼睛,说道:``大哥,我今生今世只求你一件事情了,这事情是你可以做得到的。''段-璋道:``什么?''窦线娘道:``你虽然在村子里开过武馆,却并未收过一个真正的徒弟。我要你将摩勒收做衣钵传人。摩勒,你愿意拜你姑丈为师么?''段-璋铁摩勒均是一怔,但随即两人都懂得了她的意思,铁摩勒立即跪下叩头,向段-
璋行拜师大礼。

拜师的大礼是要行三跪九叩首的,铁摩勒刚刚磕了一个响头,段-璋忽地叫声:``且慢!''将他扶起。

窦线娘道:``怎么,你不愿收他为徒?''段-璋道:``不,我这是为他打算。他应该找一个比我更高明的师父。''铁摩勒道:``姑丈,我但求学得你这手剑法,于愿已足。''段-璋苦笑道:``即算你学了我全身的本领,也还是抵敌不过空空儿,又有何用?''铁摩勒道:``但若用来对付王家父女,那却是绰有余裕的了。我想王家也总不能永远留着空空儿做他们的保镖。''

要知段-
璋夫妇已向空空儿立下誓言,从今之后,不再管王、窦二家之事,所以窦线娘要丈夫收摩勒为徒,实是指望由铁摩勒代她报仇。段-璋本意不愿再卷入漩涡,但一来为了不想妻子终生难过;二来他也是的确喜欢铁摩勒这天生的习武资质,因此踌躇再三,终于想出了两全之计。

段-璋扶起了铁摩勒,却对南霁云道:``南兄弟,我想请你将摩勒携到襄阳,拜见令师,并请你代为进言,求令师破例将他收为门下。''南霁云道:``铁寨主生前与家师交情相厚,家师也曾屡次叫我打听摩勒的下落,这事十九可以如愿。''

段-璋道:``摩勒,你我相处多时,如今分手在即,我虽然不能收你为徒,却有一件小小的礼物赠送给你,也算是我夫妻的一点心意。''说罢,将一本剑谱拿了出来,交给铁摩勒道:``这是我家传的剑谱,并附有我这二十年来学剑的心得,你拿去吧。其中重要的剑诀,我都曾经给你讲解过了,你仔细琢磨,以你的资质,学起来不会很费力的。''

铁摩勒惊道:``姑丈,这、这怎可以?我,我怎能要你的家传剑谱?''段-璋道:``这本剑谱我已熟背如流,我的儿子又还小,你先拿去,要是我的儿子能脱灾难,将来长大成人,你再交回给他也还不迟。''窦线娘也道:``傻孩子,在这个节骨眼上,你还拘泥什么名义?姑丈不肯收你为徒,是为了有更好的安排,怕乱了武林班辈。你若能够好好的用这本剑谱,不辜负你姑丈给你的这番心意,我将来还要深深的多谢你呢。''铁摩勒双眼润湿,接过剑谱,重新叩了三个响头,算是行了``半师''之礼,郑重说道:``姑姑放心,摩勒决不能辜负姑丈、姑姑的心意!''窦线娘悲惨阴沉的脸色,这时才开始有了一丝笑意。心想:``他若得了磨镜老人的内功真传,再学全了剑谱上的六十四手龙形剑法,纵然未必胜得了空空儿,也可与之一拼了。''

段-璋道:``南贤弟,摩勒今后托你照顾了。今番承你拔刀相助,长途护送,厚义深情,感激不尽。后会难期,唯望各自珍重。''四人挥泪而别。南霁云与铁摩勒一道,前往睢阳。段-璋夫妇则北走凉州,上玉树山讨回孩子。

暂且搁下段-
璋夫妇不表。只说南、铁二人,为了提防王家父子临时变卦,再发追兵,匆匆忙忙的一口气又赶了十多里路,天色将晚,腹中饥渴,恰好路旁有间茶店,南霁云道:``咱们且进去暂歇一会,吃点东西再赶路。''

这类茶店多兼卖一些酒菜,有两个大汉正在里面喝酒,店门口系着他们的两匹坐骑,铁摩勒低声说道:``这两匹黄骠马倒是不俗!''

那两个大汉听得他说话的声音,抬头一看,登时双方都是一愕,坐在上首的那个大汉,更是``啊呀''一声的叫了出来。

原来这两个大议都是安禄山手下的军官,不知何故,却换了寻常百姓的衣服。南霁云认得那个叫喊的汉子,正是安绿山帐下四大高手之一的张忠志,另一个虽然不知名字,也是那晚在安禄山府中交过手的人。

那一晚南霁云闯进安府去救段-璋,一口宝刀,杀伤了十几名武士,这两个人都是给他杀得丧了胆的,陌路相逢,大吃一惊,张忠志急忙起立说道:``南大侠,是你来了?你老人家好?''南雾云道:``没死没伤,怎么不好?你两人也好啊?''张忠志那个同伴,那晚给南霁云斫了一刀,伤口刚合,尚未痊愈,闻言甚是尴尬,却也只得拱手说道:``多承关注,彼此都好。''张忠志道:``那晚我二人是奉命而为,还望南大侠恕罪。''南霁云摆摆手道:``没什么,你们坐下来喝酒吧。''铁摩勒却瞪了他们一眼道:``喂,你们换了这身衣裳,敢情又是要偷偷摸摸的去干什么坏事?''

张忠志面色一变,连忙说道:``小哥儿取笑了。我二人是奉命去查办一件案子,故此乔装打扮。哎呀,时候不早,我们可得赶路了,夫陪,失陪,恕罪,恕罪!''铁摩勒道:``喂,什么案子?''张忠志道:``没、没什么,是乡下人两村械斗的小案子。''说话之间,已经跨上了黄骠马,南霁云道:``摩勒,不必多管闲事了,由他们去吧!''这两人如奉大赦,急忙快马加鞭,绝尘而去。

铁摩勒``哼''了一声,道:``这两人鬼鬼祟祟,支支吾吾,定然没有好事情。试想若然只是两村械斗,何劳安府的大武士出头弹压?''南霁云道:``你说得不错,这里面当然有鬼。可是咱们哪能有这些闲工夫去管他们?''

茶店主人是个年约五十左右的瘦长汉子,他听得那两个军官称呼南霁云做``南大侠'',似乎颇为留意,却也并不怎么惊诧,当下过来伺候,南霁云要了三斤汾酒,两斤卤牛肉,问道:``生意好么?''那店主人道:``托赖,托赖,这几天过路的客官很多,小店也沾光不少。''南霁云心中一动,铁摩勒已先问道:``都是些什么人?''那店主人笑道:``我瞧两位也是江湖人物,不瞒你们说,小店是只管做生意,不管客官是什么人的。这里靠近飞虎山,飞虎山的瓢把子(对山寨头目的通称),也曾在小店喝过酒呢。''

说话之间,道上又来了两骑快马,到了茶店门前,扔下一把铜钱,要了两碗热茶,在马背上匆匆喝了,便即继续赶路。铁摩勒悄声道:``这两个是线上的朋友,相貌似曾相识,却记不起他们的名字了。''要知窦家寨中,每年前来参见窦家五虎的绿林豪客甚多,铁摩勒认得的也不少,不过因为铁摩勒是个未成年的大孩子,那些豪客,除非是特别和窦家相熟,窦令侃才会叫他出来相见,所以一些普普通通的小山寨头领,却并不认得铁摩勒。

不到一柱香的时刻,陆续来了几批客人,都是挂有腰刀,乘着快马的健儿,一看就知是绿林人物,他们都像刚才那两个人一样,匆匆忙忙地喝了条便走,店主人忙着在门口招待他们。这时南霁云也起了疑心,想道:``现在已是即将入黑的时分,这些绿林好汉,匆匆忙忙地赶路,为了何事?''

其中有一个似乎神色有点犹豫不定,在茶店门前歇足的时候,用黑道上的切口向同伴说道:``面前就是两条岔路了,你看咱们该上飞虎山呢,还是去龙眠谷?''他的同伴道:``我看是去龙眠谷好些,窦老大的交椅坐不稳了,咱们若是不接王家的帖子,日后只怕有祸。''

铁摩勒勃然色变,南霁云急忙按着他道:``趋炎附势是人之常情,此时此际,你还何必生这个闲气?''

铁摩勒道:``喂,店家,你可知道龙眠谷在什么地方吗?''那店主人拖长了声音道:``龙眠谷么?你问它作甚?''铁摩勒道:``我有好朋友在那儿。''那店主人道:``哦,原来如此,龙眠谷在西边离此约二十里的地方。再往前走,就是三阳岗。''三阳岗正是那日南霁云遇着黄衣少年的地方。

铁摩勒眉头一皱,刚要说话,门外马嘶,又有两骑来到,这两个骑客却并不匆匆驰过,下了马走进店来要酒。铁摩勒睁大了眼睛,盯了他们一下,忽地离开座头,迎上前去,一把将那个大个子揪住!

那大汉吃了一惊,叫道:``啊呀,原来是铁少寨主,你,你怎么到了这儿了?''铁摩勒道:``史大叔,我正要问你呢,你却怎么也到了这儿?莫非也是要到龙眠谷去拜见新舵主么?''

这大汉名叫史彰,和窦家乃是世家,窦家寨在幽州各地的分舵事务,由他总管。另外那个人则是他的副手,名唤程通,也是窦令侃的亲信。

史彰道:``少寨王这是哪里话来?我史某岂能到龙眠谷献表投降?我正是要赶回飞虎山探听消息的。少寨主,你到了这儿,莫非。莫非大事已经不好了吗?''

铁摩勒道:``飞虎山总寨已经给王家毁了,我的义父和四位叔叔,都、都已归天了!''

史彰大惊失色,呆若木鸡,铁摩勒道:``现在不是伤心的时候,你既不愿投降王家,飞虎山你是不能再去的了,你从速派人到各处分舵传令,将兄弟们尽都遣散了吧,留得青山在,不怕没柴烧。你明白吗?''史彰道:``是,我明白少寨主的意思。''

南霁云心头微凛,想道:``摩到年纪虽小,这番安排倒是有深谋远虑,看来他还有要为窦家作东山再起的打算。咳,这么一来,绿林里只怕还要大动干戈。''

铁摩勒再问道:``王家邀各地绿林首领前往龙眠谷,这是怎么一回事?你可知道么?''

史彰道:``我也曾接到请帖,王家以前怕咱们去挑了他的大寨,因此本来是四方移动,并无定址的,最近才搬到龙眠谷来,这请帖上说他已灭了飞虎山的窦家寨,请各方豪杰,到龙眠谷来喝喜酒。当然明眼人都知道:喜酒为名,实则乃是要各处山头听他号令。''

铁摩勒``哼'了一声,满腔愤怒。想这王家的请帖是早已发出的了,可见他们搬到龙眠山来,就是为了就近指挥,要把窦家的地盘和部属全都并吞,而飞虎山窦家寨的被消灭,也早已在他们的意料之中。

这时已是夕阳西下的时分,史、程二人酒也无暇喝了,匆匆辞别。那店主人听说铁摩勒是飞虎山的少寨主,面色大变,急忙说道:``哎呀,原来发生了这样的事情。少寨主,我劝你速速远走高飞,此地离龙眠谷很近呀!''

铁摩勒冷冷说道:``你不用担心害怕,我现在就走,不会连累你的。''

就在此时,大路的东西两头,各来了一骑,在茶店门前相遇,一个是魁梧大汉,一个是面白无须的中年人,那大汉拱手道:``杜兄,你可是到龙眠谷么?''那中年人笑道:``不,我这样的无名小卒,王伯通哪能知道我,我是到韩庄去的。''

那大汉道:``杜兄,你是真人不露相,乐得自在逍遥,独往独来,无牵无碍,小弟羡慕得紧。论理小弟也该到韩庄拜寿的,只是我已经在这幽州境内安窑立柜,不能不到龙眠谷去敷衍一番。''他们两人用江湖切口谈话,铁摩勒一听便知那大汉是个山寨寨主,那个面白无须的中年人则似乎是个江湖游侠。

那中年人笑道:``如此,只好各行其是了。但盼周兄千万不要在人前提起我和韩庄主的名字,免得惹出麻烦。''那大汉道:``我理会得。''说罢,喝了一碗热茶,便即匆匆策马而去。

那中年汉子却好整以暇的系好坐骑,进店喝酒。南霁云本来就要走的,却忽然停了下来,向那中年汉子上下打量,两人对望了几眼,同声叫道:``真是巧遇了!''``南八兄,你怎的到了这儿?''``杜三哥,你怎的也到了这儿?''

南霁云道:``摩勒过来,见过这位杜叔叔,江湖上人称金剑青囊杜百英的就是他。''原来杜百英是一位江湖游侠,剑术之外,兼擅医术,人称``金剑青囊''。只是他性情闲散,不喜留名,许多行侠仗义的事情,都是暗中做的,往往飘然而来,飘然而去,人所难知。故此,在江湖上的名头远远不及南霁云响亮。南霁云在七年之前见过他一面,当时,南霁云出道未久,是以前辈之礼去谒见他的,其后叙起师门渊源,才以平辈之礼论交。

南霁云道:``我刚从飞虎山下来,这位小兄弟便是以前的燕山铁寨主、铁昆仑的儿子。''杜百英沉吟半晌道:``这里不是叙话之所,咱们且边走边谈。''抢着会了酒钱,牵着坐骑,陪南、铁二人走路。

杜百英道:``天色已晚,两位准备在何处歇足?''南霁云道:``我们是走到哪儿算那儿。''杜百英道:``南兄,你可听过韩湛的名字吗?''

南霁云吃了一惊,道:``你说的可是天下第一的,点穴名家韩老前辈?''杜百英道:``正是。今日是他的六十寿辰。''南霁云道:``怎么,他就住在附近?''杜百英道:``从这里向南走三十里便到他家,咱们不如一道去给他贺寿吧?''南霁云道:``韩老前辈和家师甚有交情,只是小弟尚未见过。''杜百英道:``他的住址只有极少数的武林朋友知道,我知道他这几年深居简出,不见闲人。不过你自然例外。他也曾和我说起过和你的师父的交情,对你亦很夸赞,所以我才敢邀你同去。''南霁云道:``如此,我理该前往给他贺寿。只不知他住的地方离龙眠谷有多远?''

杜百英道:``一处在西,一处在南,和这里的槐树庄成鼎足之势,都是三十里路的距离。南八兄,你放心,距离虽近,却也无碍。韩老前辈在此隐居,连飞虎山的窦家五虎都不知道,何况那王伯通是新近才搬来龙眠谷的,谅他更不能知晓。''南霁云道:``我不是怕了他们,只是怕给韩老前辈招惹麻烦。''杜百英笑道:``韩老前辈也不是怕沾惹麻烦的人,不过是非到不得已之时,不想去碰他们罢了。你们刚从飞虎山下来,也许他正是要见你们呢!''话中似有深意,南霁云心中一动,当下加快脚步,不过半个时辰,便到了一个靠近山边的小村庄。

这时已是炊烟四起,暮色昏瞑。杜百英找到了韩家,拉了三下门环,高声报了自己的名字,韩湛亲自开门,笑道:``百英,你来迟了!''杜百英道:``韩老前辈,我给你请来了两位稀客啦!''

南霁云放眼打量,只见那韩湛虽然年已六旬,却是神光内蕴,步履安详,绝无半点老态,长须三络,一袭青衫,看来俨似画图中的高士。南霁云急忙上前施礼,说道:``磨镜老人门下南霁云给你老人家拜寿。''韩湛怔了一怔,随即哈哈大笑,说道:``原来是南世兄,我和令师是几十年的老朋友,今日方始得见老友的爱徒,当真是意外之喜。你到这里,只当回家一般,不必拘束。哈哈,什么风把你吹来的?''铁摩勒随后也向韩湛叩头贺寿,韩湛将他扶了起来,问道:``这位小兄弟是------''南霁云道:``他是燕山铁寨主铁昆仑的公子。''韩湛道:``我和铁寨主生前也曾有几面之缘,在绿林人物中,他是我唯一钦仰的人,如此说来,都不是外人了。''

南霁云道:``铁老寨主过世之后,窦令侃将他收为义子,今日窦家寨被破,我和他一同逃了出来,幸遇杜兄,得知韩老前辈寿辰。''韩湛听了,眉心略蹙,却也并不怎样惊讶,似乎此事早已在他意料之中,说道:``你们来得合时,里面有几位朋友,刚才还正在谈论王、窦两家的事情,请进去叙话。''

韩湛做寿,只是几个最相熟的朋友知道,除了杜百英之外,只有四个贺客:青海萨氏双英,麦积石山的龙藏上人,和金鸡岭的辛寨主。前三人都是远道而来的知交,只有辛寨主是幽州境内的绿林大豪。

坐定之后,南霁云讲述空空儿和王家父女大破飞虎山的事情,众人听得连段-璋夫妇也败在空空儿剑下,相顾骇然!

韩湛叹息道:``空空儿本来是个聪明绝顶的人,这番却是做事糊涂了。''龙藏上人道:``韩兄此话怎讲?''韩湛道:``他被王家利用而不自知,还以为自己做的事情很正当,这岂不是糊涂吗?''

龙藏上人眉头一皱,似乎不大服气,想和韩湛有所争论,但他望了南、铁二人一眼,想起了铁摩勒是窦令侃的义子,便不再说话。原来他对王、奏两家都颇不满,比较起来,对窦家的恶感还更大一些,是以心中想道:``空空儿助王家争霸,最多是以暴易暴,这等绿林中的火并,本来就谈不到什么是非,也说不上什么糊涂不糊涂。''

南霁云问道:``韩老前辈敢情是和空空儿相识的么?''韩湛道:``何止相识,他小时候我还抱过他。''萨氏双英和杜百英等人都觉意外,杜百英道:``这几年来,江湖上给空空儿闹得天翻地覆,谁都不知道他的来历,想不到韩老伯却和他是世交。他的武功如此高强,不知是出自何人所授。''韩湛道:``他的师父是个当世异人,像我一样,姓名不愿为人所知,我和他也有一点点交情,请恕我为他隐瞒了。''歇了一歇又道:``可惜消息我知道得迟,空空儿又行踪无定,以至我不能事先去劝阻他。''

南霁云正想说话,忽听得门外有极轻微的声息,似是有夜行人来到,方自一怔,便听得韩湛说道:``芬儿,你回来了吗?这里几位叔伯都不是外人,进来相见吧!''

进来的是个年约十四五岁的女孩子,梳着两条小辫子,打着蝴蝶结,稚气未消,蹦蹦跳跳地进来,笑道:``爹爹,你交给我这趟差事可不好办啊,几乎给人瞧破,脱不了身。''正是:

韩家最小偏怜女,虎穴龙潭曾去来。

欲知后事如何?请听下回分解------

\chapter{第十三章 喜庆筵前来异丐
英雄会上破奸谋}\label{ux7b2cux5341ux4e09ux7ae0-ux559cux5e86ux7b75ux524dux6765ux5f02ux4e10-ux82f1ux96c4ux4f1aux4e0aux7834ux5978ux8c0b}

韩湛道:``这是小女芷芬,刚从龙眠谷回来。''南霁云吃了一惊,韩湛笑道:``你先见过各位叔伯。''韩芷芬指着铁摩勒道:``他和我年纪差不多,我也要叫他叔叔吗?''韩湛笑道:``这小妞儿就是不肯吃半点亏,也怪我未把话说清楚。好,这两位你可以叫他们做哥哥。这位是镜磨老人的大弟子南霁云,这位是燕山铁寨主的公子铁摩勒。''韩芷芬道:``南大哥,江湖上都尊称你为大侠,我是久仰的了!''转过头来又对铁摩勒道:``我也曾听人说起过你,说你是绿林中的小星君,做事是又顽皮又辣手,我也是久仰的了!''

铁摩勒本来满怀愁绪,心事重重,给那女孩子调侃了几句,弄得哭笑不得,脸蛋通红,甚是尴尬。韩湛骂道:``油嘴滑舌,没一点规矩,我看哪,天下就没有比你更顽皮的了,还不快向世兄赔礼!''那女孩子学着大人的模样,检任一礼,说道:``小女子无知,说错了话,望世兄海量包涵。''满堂大笑。

韩湛道:``你闹够了没有,来说正经的话吧,你可见看了空空儿?''韩芷芬道:``说正经的,没有见着,却见着了一个大猴子。''韩湛道:``胡说八道,哪来的大猴子?''南霁云道:``韩姑娘说的莫非是空空儿的师弟精精儿?''

韩芷芬笑道:``到底是南大哥聪明,一听便知道我说的是像猴子的人,不错,那怪模怪样的家伙正是精精儿。

``我二更时分进了龙眠谷,谷里好不热闹,那些大大小小的噗罗正在吃什么庆功酒呢!王伯通和另外四个人另在一间厢房里喝酒,与大伙隔开,围墙外边有几株愧树高出墙头,枝叶茂密,我伏在槐树上,瞧得清清楚楚。我看见空空儿不在,就没有用你所教的暗号。''

韩湛道:``除了精精儿之外,还有三个是什么模样的人?''韩芷芬道:``一个是年约二十左右的少年,长得很像王伯通,额角青肿了一大块,似是给人打伤的。''韩湛道:``唔,这是王伯通的儿子王龙客。''铁摩勒道:``他额角上的伤是给我的姑姑用弹子打的。''韩芷芬道:``你的姑姑,哦,敢情是段大侠的夫人窦线娘?这么说,王家父女与空空儿大破飞虎山的时候,你是在场的了?''韩湛道:``不要岔开,等下再叫南大哥讲给你听。你往下说吧,还有两个呢?''

韩芷芬道:``还有两个是带着外路口音的陌生人,其中一个,左臂下垂,似是受伤未愈,举不起来。''南霁云吃了一惊,道:``这两个人是安禄山帐下的武士,受伤那个,名字我不知道,不过,他左臂上那一刀却是我斫的,未受伤那个则是安禄山帐下四大高手之一的张忠志。''韩芷芬道:``怪不得我听他们老是提到什么大帅、大帅的。爹爹,你料得不错,王伯通那老狐狸果然是和安禄山有来往。''停了一停,往下续道:``我一到就瞧见王伯通向那个大猴子,哎,精精儿敬酒,说道:`今日大破飞虎山,是我生平最大的喜事,可惜你的师兄已回去了,我留也留不住,明日的盛会,缺他一人,却是一个遗憾。'

``精精儿道:`我师已就是这个脾气,他好像很爱管闲事,但事情一完了,他立即飘然远去,从不称功道劳的。'

``左臂受伤的那个陌生人道:`我们的大帅也久仰令师兄的大名,很想礼聘他,只是没有适当的人可作使者,不知阁下可代为说辞么?'

``精精儿摇头笑道:`难!难!我师兄那个脾气,怎么受得了拘束?休说是你家大帅,就是皇帝老儿只怕也请不动他。'

``那张、张什么,(南霁云插口道:``那人叫张忠志。'')说道:'王寨主,你这次是真够面子了。'王伯通笑道:'一来我和他过世的父亲有点交情,二来嘛,十多年前窦老大曾干过一件非常狠辣的、黑吃黑的事情,杀了挑阳沙庄主一家,这沙庄主是空空儿长辈亲戚,所以我和他一说要去挑飞虎山的窦家寨,他便立即答应了。'那张忠志哈哈笑道:'这也该是王寨主马到成功,以后咱们的大帅还要多多仰仗你呢。'王伯通道:'好说,好说。这是彼此有利之事,老夫要依靠你家大帅的地方更多呢。'接着又对精精几道:'如此说来,令师兄不在也好,我怕他对这件事情,不会同意。所以我也未曾告诉他。'精精几道:'王寨主放心,我自会替你善为说辞,我师兄纵不赞同,大约也不会作梗的。'王伯通马上又向精精儿敬酒,大说了一通拜托、拜托、劳驾、劳驾的说话。''

韩芷芬将夜探龙眠谷的所见所闻,一口气说到这里,方始歇下来喝茶。韩湛面色沉重,缓缓说道:``我刚才惋惜空空儿被人利用,现在各位大约明白了吧?简单的说,就是安禄山想做皇帝,一方面他拉拢各地边军的胡人将领,一方面和王伯通勾结,待王伯通成为绿林盟主之后,希望到他举事之时,这班绿林好汉也为他所用!''

龙藏上人道:``哦,原来如此!我起初还以为韩大哥偏袒窦家呢。这么说来,王伯通的确是要比窦令侃更坏了!''话说了出口,方觉失言。南霁云道:``大师的评语公允得很。可惜我段大哥还未知道这件事情。他对于这次飞虎山之行,倒是后悔得很呢。''韩湛道:``芬儿,你探听到这个消息,有用得很,后来呢?还听到他们说些什么?''

韩芷芬道:``后来嘛,我碰到了一件意想不到的事情!''韩湛道:``怎么?是给精精儿发觉你了?''

韩芷芬道:``我也不知道他发觉的是哪一个?''杜百英道:``怎么?难道还有一个这样大胆的人,敢到龙眠谷去窥探吗?''

韩芷芬已经接续说道:``我听到这里,心头一跳,树枝摇动,树叶发出轻微的沙沙声响,那精精儿好不厉害,立即听了出来,酒杯一摔,高声叫道:`外面有人!'''

韩湛奇道:``精精儿轻功卓绝,你是怎么逃脱的?可是打出了我的名号来么?''

韩芷芬笑道:``精精儿没有出来,我也未曾打出你的名号。我的运气太好,逢凶化吉,碰到了救星啦!''

韩湛道:``是哪一位武林前辈搭救你的?''在他想来,能够在龙眠谷救人的,当然是武林前辈无疑了。韩芷芬笑道:``爹爹,这次你猜错了,救星是一位美丽的姑娘,比我也大不了几岁。''韩湛道:``这可真是奇事了。那姑娘是什么人?''韩芷芬道:``爹爹,你别心急,听我慢慢道来。''她模仿说书人的口吻,慢条斯理地说道:``就在那个时候,王伯通的儿子突然摆了摆手,低声说道:`这是我的一位相熟的朋友,不用惊慌,待我请她进来便是。'\,``我正在惊奇,心道:'这小子怎么认识我的?'说时迟,那时快,他已跳出围墙,槐树下忽然现出一位美貌的姑娘,敢情她也是像我一样,早已藏在树上。

``那姑娘一见王龙客出来,便即冷冷说道:'王公子,原来你还是王少寨主,当真是失敬、失敬了!'王龙客甚是尴尬,讷讷说道:'夏姑娘,非是我对你隐瞒身份,这,这!'这时我方知道那美貌的姑娘姓夏。

``那夏姑娘不待他把话说完,便冷笑道:`你是什么身份,与我无关。我只问你,你们把我的段伯伯怎么样了?'王龙客道:`哪位是你的段伯伯?'夏姑娘道:`段大侠,段-璋!'''

南霁云心头一震,想道:``这少女不是别个,定然是夏凌霜了!呀,她果然和王伯通的儿子甚有交情!''

韩芷芬继续说道:``那王龙客似乎是怔了一怔,说道:'原来那段-
璋是你的长辈,他,他们两夫妇\ldots\ldots'那夏姑娘连忙问道:'怎么样了?'王龙客拖长了声音道:'他们打不过空空儿,逃跑了!'那夏姑娘道:'这话可真!'王龙客道:'我骗你作什么?我们可并不是胡乱杀人的强盗!'那夏姑娘道:'他们逃向何方?'王龙客道:'大约是回家了吧?'那夏姑娘道:'好,要是我找不到他们,再来和你说话!'王龙客忙着去追她,我也就趁机会溜走了。''

韩湛吁了口气说道:``如此说来,那位夏姑娘是为了段大侠而去夜探龙眠谷的,想必也是我辈中人,你为何不邀请她到这里叙叙?王伯通儿子的武功我是知道的,若然真打,你打不过他,若论轻功,他比不过你。听你说的情形,那位姑娘的轻功又要比你高明许多,王伯通的儿子定然追不上她。难道她不肯和你见面吗?''

韩芷芬道:``爹爹料得不错,那王龙客果然追不上她,我离开龙眠谷不到五里,就望见他垂头丧气的回来了。他没有发觉我,当然我也不便去惹他。后来我约莫走了五六里路,忽听得前面马铃声响,却原来是那位夏姑娘换乘了一匹白马,回头来找我。''

韩湛道:``她怎么说?''韩芷芬道:``她先问我是不是窦家的人,我说不是。她再问我是否认识段大侠,我又说不是。她便问道:'那么你到龙限谷来什么?'我心想她是个好人,不用瞒她,便直率的对她说,是奉了爹爹之命来找空空儿的,并邀请她到咱们家里暂住一宵,好大伙儿没法帮忙她找段大侠。她面色一变,不待我把话说完,便哼了一声道:'我没有这些闲功夫。'快马加鞭,立即便走,弄得我好生没趣。瞧她的神情,对那空空儿似乎也有仇。''

韩湛笑道:``她大约是有所误会了,不过,也忒性急一点。''

萨氏双英和辛寨主等人议论纷纷,他们都是在江湖上见多识广的人,却猜不到这少女的来历。铁摩勒想说话,南霁云给他打了一个眼色,铁摩勒立即会意,可是心里却暗暗纳闷,不知南霁云何以不让他透露这位夏姑娘的身世。

韩湛道:``暂且不去管这位夏姑娘,听芬儿所探听到的消息,那王伯通与安禄山暗中勾结,证据已经是很确凿的了,那么,咱们该怎么办?''

金鸡山的寨主辛天雄是个烈性的人,立即说道:``王伯通想做绿林盟主,这也还罢了,要咱们跟从他为胡儿打天下,那却是万万不能!''

萨氏双英道:``只是他这个阴谋,绿林中的众弟兄尚未知道,咱们先得揭穿他这个阴谋,弟兄们才不会让他牵着鼻子走。''

辛天雄道:``话说的是,却怎么样去揭穿他呢?''

杜百英一直在旁沉思,这时方始说道:``辛寨主,王伯通也有请帖给你的,是不是?''辛天雄道:``不错。咱家却不怕他,偏偏不去赴地的宴会。''杜百英笑道:``还是去的好。我们充作你的随从,跟你一同去。韩老前辈,你看这计策可使得么?''

韩湛道:``好是好,只是霁云、摩勒和萨家兄弟都是与王伯通瞧过相的,却怎的瞒得过他的眼睛?''

杜百英道:``老前辈不用担心,小可略懂一点变容易貌之术。''韩湛笑道:``我只知道老弟是位大国手,却原来还懂得江湖郎中这一套戏法。只是老朽年岁大了一些,充作辛老弟的随从只怕不像?''

杜百英笑道:``晚辈自有妙法叫老叔年轻二十年,只是你那把长须要剪短一些,却是有点可惜了。''接着道:``其他的人更容易改装,就是龙藏上人身材魁伟,相貌特别,又是光头,较为难办。''

韩湛道:``那么只有委屈大师替我看守这几间破屋,陪伴小女吧。''

韩芷芬噘着小嘴儿恳求道:``不,这场热闹,我也要去瞧瞧。''

杜百英道:``贤侄女,你年纪太小,就算易钗而笄,也充当不了山寨的小头目,那王伯通是个老江湖,怕会给他瞧破,我看,你不去也罢。''

韩芷芬指着铁摩勒道:``他与我年纪相差不多,他去得我怎么去不得?''

韩湛笑道:``你和他站在一定比比看,他比你高一个头呢。他充作辛寨主的随从小厮,没人怀疑,你就不行了。何况,你作男孩子打扮,也容易露出马脚。''

韩芷芬道:``不管如何,我这次是非去不可,杜叔叔,你替我想个妙法!''

杜百英沉吟半晌,道:``那末你就权当辛寨主的女儿吧,辛寨主带心爱的女儿去吃喜酒,也还可以说得过去。反正没人认识你,连装束也不必改换。''

辛天雄笑道:``这岂不折杀我了,要韩老前辈作我的随从,又要贤侄女叫我做爹爹。''

韩芷芬道:``你是占了便宜哩,还有什么不好。''龙藏上人笑道:``你们都有热闹可瞧,就只留下我一人给你们看家,可真是气闷了。''

杜百英道:``这是一时权宜之计,辛寨主也无须难为情。好吧,现在就开始吧,摩勒小兄弟充作你的随从小厮,咱们都充作你山寨里的大头目。''辛天雄道:``对,充作头目更好一些,也显得是咱们小寨对王家的尊重,阖寨头领都给他贺喜来了。只是委屈少寨主一人。''

杜百英有秘制的易容散,经过他施用手术,果然人人都换了一副面貌,韩湛脸上的皱纹也给弄平了,看起来的确像是年轻了二十年。

待到天明,这一行人等便到龙眠谷去,韩芷芬最为开心,一路上嘻嘻哈哈与人笑闹,南霁云则满怀心事,惦记着那位夏凌霜姑娘。

金鸡山的寨主辛天雄,在幽州的绿林道中,是个响当当的角色,性情强傲,窦家雄据飞虎山作绿林盟主的时候,各处山头,循例每年纳贡,只有他不肯卖帐,从无贡物,窦令侃虽然对他极为不满,但一来因有大敌当前,二来金鸡寨的实力不弱,故此也不敢向他动手。

王伯通素来知道他的为人,这次虽然发出请帖,却实是不敢指望他会亲来道贺,因此一接到辛天雄的拜帖,不由得大感意外,连忙携了儿子,亲自出来迎接。

辛天雄见过了礼,说道:``王寨主这次一举便将飞虎山的窦家寨连根拔去,真是可喜可贺。金鸡山受窦家之气,已非一日,如今得王寨主为咱们扬眉吐气,敝寨阅寨人众都是非常感激,因此小弟将率掌舵的几位弟兄,齐来给寨主贺喜。''

王伯通道:``老朽德薄能鲜,这次侥幸成功,有劳贵寨的各位当家远道而来,实是过意不去,这厢答谢。''

辛天雄道:``咱们一来是给寨主贺喜,二来是向寨主道谢,三来嘛,以后敝寨还得多多仰仗盟主的庇护呢!''接着又哈哈笑道:``王寨主这次大宴绿林豪杰,乃是百年罕遇的盛事,连小女,她还从未出过道的,也要随我来瞧瞧热闹呢!''

王伯通听他在语气之中,已承认了自己是绿林盟主,心底下自然是高兴非常,可是却也有点起疑:``金鸡山与窦家有隙,我灭了窦家,他们畏威怀德,山寨里的大头目都来给我道贺,这犹自可说。但我与辛家并非通家之好,连女儿也带来,这、这、似乎我与他还未够这个交情。难道他是为了巴结我,藉此向我表示亲热吗?以他平素的为人,又似乎不像?''

王龙客忽地踏上一步,望着铁摩勒道:``这位小当家贵姓?''辛天雄暗暗吃惊,忙道:``他是我的随从小厮,不懂规矩,少寨主别见怪。''给他胡乱捏造了一个假姓名。原来铁摩勒面对仇人,不自禁露出仇恨的眼光;给王龙客注意到了。幸而铁摩勒机伶,立即说道:``当家的,你今日带我到此,我却记起了一件旧事来了。''辛天雄道:``这里没有你说话的地方,回去再说。''王龙客道:``让他说说何妨?''铁摩勒装出惶恐的神情,李天雄道:``好,那你就说吧。''铁摩勒道:``你还记得有一次你差我到飞虎山吗?他们嫌你当家的没有送礼,迁怒到我的身上,将我打了一顿,逐出寨门。如今王家寨主待人可好得多了。因此,我想起旧事,再看今朝,真是又怒又喜!''王龙客哈哈大笑,说道:``原来如此,小兄弟,你也真是个有心人呢!''

说话之间,有两个人从里面出来,一个是精精儿,一个是王伯通的女儿。

王伯通给他们介绍道:``这位是咱们绿林道上响当当的金鸡山辛寨主。''``这位是江湖上鼎鼎大名的剑客精精儿。''精精儿神态傲岸,淡淡地说了句:``久仰了。''便不再理会辛天雄。

精精儿目光如电,环扫了众人一眼,目光停在韩湛身上,心中大吃一惊,他是个武学的大行家,这一眼已瞧出韩湛是个具有上乘内功、深藏不露的非常人物。连忙上前问道:``这位寨主贵姓大名?''

韩湛道:``韩某是金鸡山一个无足轻重的小卒。''辛天雄给他报了个假名,道:``韩大哥是金鸡山的二当家,新近才入伙的。''精精儿道:``幸会幸会!王大哥,你天大的面子,请得韩当家到来,当真是为此会生色不少!''伸出手来笑道:``我也有幸可以结交一位新朋友了!''

王伯通这一惊更甚,精精儿对金鸡山的寨主傲岸不恭,却会对他手下的一个头目表现得如此亲热客气,实是出乎常理之外,令他莫名其妙。

精精儿有意试韩湛的功夫,双掌相握,暗暗用上了小天星掌力,这小天星掌力乃是一种刚柔并用的内家真力,触及对方身体,可以令对方浑身麻软,瘫倒地上。韩湛微微一笑,说道:``多承青眼,韩某愧不敢当。''精精儿的掌力发出去,只觉对方的手掌软绵绵的,竞似毫无抵抗,却又毫无异状,这一惊非同小可,想道:``此人的内功当真是深不可测,只怕连我的师兄也未曾达到如此炉火纯青的境界。''心念末已,陡地觉得脉门一麻,原来韩湛是天下第一点穴名家,就在这双掌相握的时候,他拇指轻轻一按,虽未按正穴道位置,那股内力已达到了精精儿的脉门,冲击他的三焦经脉。

精精儿连忙放手,说道:``韩当家真好功夫,佩服!佩服!''韩湛见他禁受得起,亦是不敢小视。这时,王伯通也看出他们是在较量武功了,不禁又是惊奇,又是害怕,心道:``连金鸡山的一个头目,也有如此功夫,我这绿林盟主可不好当哪!''

王伯通的女儿蹦蹦跳跳的过来,拍掌笑道:``我可找到了伴儿啦,你是哪家姐姐?''王伯通道:``这是小女,名叫燕羽,最是爱玩,东跑西跳的,别人都管她叫小燕子。这位是辛寨主的千金,好啦,你就替我陪辛姑娘吧。''王燕羽笑道:``对,你今天请的都是大人,这位辛姐姐该算做我的客人了。辛姐姐,咱们到那边玩去。''

王家这次大宴绿林豪来,贺客盈千,龙眠谷本来是个荒谷,幸亏他们早有布置,在短短几个月里,大兴土木,不但筑了无数碉堡房屋,还兴建了一座占地数百亩的大花园,亭台楼阁,应有尽有,正好拿来作宴客的地方,园里还搭了两座戏台,演戏娱宾。宴会定在正午开始,这时尚有一个时辰,宾客们在园中或游览或看戏,或聚谈,各适其适,热闹非常。

王燕羽见韩芷芬和她年纪相若,人又长得漂亮,对她甚有好感,两人携手同行,观览园中景色。王燕羽一路上滔滔不绝和她讲大破飞虎山的事情,见韩芷芬听得好像并不怎样起劲,感到没趣,讲了一会,忽然停顿下来、问道:``你们那位韩当家武功真好,刚才他和精精儿暗中较量,你可看出来没有?''韩芷芬道:``是么,我一点也不知道。''王燕羽笑了一笑,说道:``我与你一见如故,你却何必这样谦虚,把我当作外人呢?他们刚才暗中较量,依我看来,似乎还是你们那位韩当家较胜一筹。韩当家已然如此了得,你的爹爹定然更在他之上,虎父无犬子,强将无弱兵,辛姐姐,你的技艺也一定出色当行的了!''韩芷芬淡淡说道:``我生得笨拙,虽然练过几天,哪谈得上懂什么武功,王姐姐,你别给我脸上贴金啦!''

王燕羽笑道:``我不信!''握着她的手儿,暗暗用了几分内劲,她倒是伯韩芷芬禁受不起,劲力只是一分一分的加强;韩芷芬早听过南霁云讲述王家父女大破飞虎山的事情,对王燕羽手段的狠辣,甚为不满,这时见她学精精儿的所为,又来暗中较量自己,不禁心中火起,突然施展家传的拂穴功夫,衣袖轻轻一拂,拂中了她腰胁的``愈气穴'',王燕羽``哎哟''一声,掌心往外一登,她练的是柔中带刚的绵掌功夫,这一下掌力尽吐,韩芷芬也禁不住``哎哟''一声叫了起来,接连向后退出了六七步!

王龙客这时适从旁边经过,见状大惊,急忙斥道:``妹妹,你怎么对客人无礼!''王燕羽忍痛笑道:``咱们是闹着玩的,哥哥,你却当真了!''韩芷芬也忍痛笑道:``王姐姐指点我的功夫,是我请她教的。''

王龙客皱了皱眉,道:``你们切磋功夫,本来很好。不过,等待宾客散后,再在这空园子练,不更好么?''王龙客是个细心的人,当然瞧出了她们是在暗中较量,不禁疑云大起。

要知王燕羽自幼即得异人传授,武功比她的哥哥还胜一筹,如今她和韩芷芬暗中较量,竟然讨不了便宜,这教她哥哥看了,怎不吃惊?心中想道:``辛天雄的副手和女儿都有这样高强的本领,那他以前为何不在绿林争霸,却要长期受窦家的欺压?而今又肯服服帖帖来归顺我王家?莫非其中有诈?''他暗自沉吟,自去和精精儿商议,按下不提。

王、韩二女继续在园中游玩,彼此都暗暗佩服对方的武功,不敢再试。王燕羽笑道:``辛姐姐,你这手拂穴功夫好不厉害,不知你和韩湛韩老先生是怎么个称呼?''韩芷芬吃了一惊,心道:``我父亲隐姓埋名,若非武林中的一流人物,绝不会知道他的名宇,她年纪轻轻却怎的也知道了?''好在她也是七窍玲拢的女孩子,心内吃惊,神色却丝毫不露,当下装作不解,反问王燕羽道:``这韩湛是何等人物?我只认识一个姓韩的,就是今天和我同来的这位韩叔叔,那韩湛是谁,却恕我不知了。''王燕羽道:``这韩湛么,我听师父说,他是天下第一点穴名家,所以我见了姐姐的点穴功夫如此高明,还以为姐姐是他的弟子呢。''韩芷芬道:``我这几手粗浅的功夫是我爹爹教的,今日班门弄斧,实在是贻笑大方了。姐姐,你的绵掌和闭穴功夫小妹是望尘莫及,不知令师是哪位武林前辈?''王燕羽笑道:``我师父的脾气和那位韩老先生一样,都不喜欢别人知道名字,所以我也不敢说。''韩芷芬听了,知她已在暗暗起疑,但她本来就准备今日随父亲到龙眠谷大闹一场的,故此也并不畏惧。

王燕羽带了韩芷芬走去看戏,忽见人丛中有个乞丐,王燕羽甚为诧异,叫道:``咦,你们怎么把叫化子也放进来了?还不快把他赶出去!''王家的手下人竟似谁都未曾留意,听小姐一说,大惊夫色,纷纷问道:``在哪里,在哪里?''纷乱中,转眼间已消失了那乞丐的所在,王燕羽始觉奇怪,正待去亲自找寻,她父亲已派人来叫她回去陪席。

这时已是正午时分,园中到处鸣钟击鼓,请客人席。王伯通父子、女儿和辛天雄、韩湛父女、精精儿等人一席,王燕羽坐在韩芷芬旁边,王伯通左手边是精精儿,右手边是个形容古怪的老头。南霁云、杜百英等人另一席,在首席的旁边。南霁云暗暗留心,见安禄山那两个军官就坐在相邻的一席,仍是穿着便装,他那一席上的宾客,南、社二人一个也不认识。

酒过三巡,王伯通旁边的那个老头,便站了起来,击了三下手掌,示意有话要说。

这老头儿名叫褚遂,也是绿林世家,声望仅次于窦令侃、王伯通二人,却是王伯通的好友,众人一见他站起来,便知道他要说的是什么话,果然听得他说道:``做官的有个头儿,这头儿便是皇帝;咱们做强盗的也有个头儿,这头儿便是盟主。这几十年来,一直是窦家做咱们的头儿,可是窦家只知损人利己,不顾义气,就像个无道昏君一样,相信在座诸位,都受过他家不少的气了。现在王伯通老大哥替咱们绿林除了此害,灭了飞虎山,铲了窦家寨,绿林中人人称快。不过,窦家无道是一回事,头儿还是要的。要不然,群龙元首,你争我夺,祸害就更大了。所以,正如国不可一日无君,咱们也不可一日无主!依我之见,王大哥既然替咱们除了无道之主,咱们就该请他继窦家之位,做咱们的新盟主,诸位意下如何?''

王家早已拉拢了的人,当然纷纷拥护,未曾拉拢的,慑于王家的威势,也都随声附和,看来王伯通继位已成定局。辛天雄忽然站了起来,大声叫道:``我有话说!''登时,所有喧闹的声音都静了下来!

褚遂愕然问道:``辛寨主敢情是有异议么?''辛天雄道:``我并非不赞同王寨主继位盟主,只是我尚有一事未明,要向王寨主、诸寨主领教。''

褚遂道:``不知辛大哥要问何事?''辛天雄道:``褚塞主刚才说的好,做官的有皇帝做头儿,咱们就也该拥个头儿,这才好号令一致,与官府对抗,不知小弟可有误解寨主之意?''褚遂只得说道:``正是这个意思。''辛天雄道:``好,那么今日的绿林盛会,为何却邀请了安禄山的亲信手下与会?用意究竟如何?王寨主可以向众家兄弟说说吗?''

王伯通面色大变,硬着头皮道:``哪有安禄山的人在座?是谁造的谣言?辛寨主,我看你是误信谣言了!''

话犹未了,南霁云突然起立,指着邻桌的张忠志道:``此人便是在安禄山帐下,任折冲都尉的官儿,他旁边的那一个,也是安禄山帐下的武士!''

此言一出,全场大哗,忽地有个叫化子笑嘻嘻地跑来,身法快到极点,转眼之间,便到了张忠志的席旁。王燕羽一看,正是刚才在戏台下的那个乞丐。只见他向张忠志打了个千儿,龇牙裂嘴地笑道:``盛会难逢,穷叫化讨赏来啦!先问官儿要,后向主人讨!''

席上一个胖子大怒喝道:``臭叫化,这里是什么地方,容得你胡闹么?''信手提起酒壶,朝着他的大灵盖便砸下来。绿林豪杰讲究的是大杯酒,大块肉,酒壶不是钢打便是铁制,一只酒壶足可装五斤酒,比寻常人家所用的大得多,这一下酒壶砸顶,胜如铁锤一击,实是厉害非常!

那叫化子迎面笑道:``未赏钱先赏酒么?好,谢酒!''张嘴一咬,正好咬着酒壶的尖嘴,那胖子用尽气力,酒壶竟不能向前推动分毫!说时迟,那时快,张忠志同席的另外两人亦已同时挥掌向那乞丐攻去,但听得``篷、蓬''两声,那乞丐双掌一分,将这两个人都震得摇摇晃晃;倒退几步,几乎跌倒!

褚遂叫道:``车老二,不看僧面看佛面,今天是王大哥的好日子,你有什么事过来和主人家说吧,先别动手呀!''此言一出,全场震动,有喜有惊,原来武林中有三个异丐,一个是``西岳神龙''皇甫嵩,一个是酒丐车迟,一个是疯丐卫越。三丐齐名,都有惊人的技业,褚遂称此人为``车老二'',即算不认识他的也都知道他是酒丐车迟了。王家的党羽暗暗吃惊,杜百英这班人则是暗暗欢喜。

这时已形成了那一席人围攻酒丐车迟的场面,南霁云、杜百英和萨氏双英也赶忙奔了过去。就在此时,车迟已把壶中的烧酒吸尽,张嘴一喷,漫空酒雨照头照面的向众人射来,这酒雨经他口中喷出,竟似有实质的弹子一般,饶是那班人个个武艺高强,被酒珠溅上了脸门,也觉热辣辣作痛。车迟耸肩笑道:``王、褚两位寨主,你们都瞧见了吧,是他们先动的手,怎可以单独怪我呢?''

南霁云逞向张忠志扑去,张忠志被热酒喷着,烫伤了眼睛,本来以他的武功是可以抵挡二三十招的,现在却给南霁云一个照面便抓着了手腕。另一个武士也给杜百英擒获。张忠志同席的人纷纷扑上,却给车迟和萨氏双英拦住。车迟哈哈笑道:``有好戏看啦,你们闹些什么,安心看戏不好么?''这班人本来都是王伯通与张忠志邀来的好手,却不料碰上了车迟这个煞星,只有眼睁睁的看同伴被人擒去。

南霁云与杜百英挟着人质,踏上戏台,台上的戏子早已呆住,这时见他们竟然跳上台来,发出一声喊叫,连锣鼓手都逃至小后台去了。

王伯通面色铁青,信手抓起酒壶往地上一摔,喝道:``住手!''岂知他这两字刚刚出口,韩湛伸出了一双筷子,已把壶耳挟着,说道:``王寨主有话好说,何必动气?这壶美酒,倒了它也未免可惜!''卫伯通这一摔足有几百斤力道,却给韩湛仅用一双象牙筷子,轻轻一挟,就将大酒壶挟了回来,又惊又怒、又是尴尬,这口气发不出来,只好沉声说道:``今天到龙眠谷的都是我的朋友,请朋友们给我一个面子,有什么事过了今天再说!''

韩湛笑道:``王寨主此言欠思量了,这是一件大事,趁各方朋友都在这儿,正该把事情弄清楚了,免至有损寨主名声!''辛天雄接口道:``是呀,众人正要推举你做咱们的盟主,却有官府中人混了进来,若不审个明白,众家兄弟岂不误会你与官府勾结?再说,若然这两人当真是安禄山的武士,那也就不该是你的朋友了。我们要弄清楚此事,正是为了你的好呀!''韩、辛二人一唱一和,把王伯通说得面上一阵青一阵红,虽然恼怒万分,却是做声不得。

这时,南霁云与杜百英已把那两个武士推出台前,台下站满了人,人丛中忽地有人叫道:``你们说这两人是安禄山帐下的什么将军、武士,有何证据?''此言一出,登时有人随声附和道:``是呀,焉知不是他们金鸡山的人想诬陷咱们的王大哥,得找不是金鸡山的人来作证明。有谁可以证明这两个人是安禄山的奸细?''这些人当然都是王伯通的党羽,一唱百和,声势汹汹,休说其他人等认不得张忠志与那个武士,即算认得也不敢作声。

酒丐车迟忽地在人丛中冷冷说道:``我可以证明!''他说话的声音不高,却是十分刺耳,把那一大片嘈嘈杂杂的声音都压了下去!有人喝道:``有何真凭实据?''车迟笑道:``真凭实据就在他们身上!''

南霁云得车迟提醒,在张忠志身上一搜,果然搜出了一面虎头金牌,这是安禄山派遣亲信手下出差的凭信,凭此可以调遣属下的各地官兵,绿林中有许多人认得,登时,连王伯通的党羽也不敢叫嚣了。

南霁云喝道:``你们来此是干什么的,快说!''那张忠志却是一名硬汉,南霁云用力捏他,几乎把他的腕骨捏碎,他仍然不肯开声;但他那个同伴却禁受不起,他被杜百英用分筋错骨手法一治,却忍不住``哎哟''一声,叫了出来!

杜百英喝道:``你不说,还有更厉害的让你尝尝!''那武士嘶声叫道:``好汉住手,我说成说!''

精精儿忽地把手一扬,飞出两支匕首,韩湛早就注意他的动作,立即他手中的筷子也当作暗器射出,却不料精精儿发暗器的手法十分古怪,那两支匕首飞到中途忽地拐了个弯,然后再直线飞出,正当韩湛的筷子要追上的时候,匕前已改换了方向。

匕首疾如电闪,射上台来,杜百英模剑一磕,磕落了一支匕首,但第二支匕首他却阻拦不住,只听得``嚓''的一声,那支匕首已穿过了这个正想说话的武士的喉咙,登时把他的声音打断了!

韩湛大怒喝道:``精精儿,你为什么杀人灭口?''

正在此时,戏台下忽然大乱,一片喝声,王龙客冷笑道:``辛寨主,你好大的面子,想不到飞虎山的少寨主竟然是你的随从!''

原来王龙客早就对铁摩勒起疑,暗中吩咐了几个得力的手下去摆布他。铁摩勒不知有人暗算,还想挤到台下``看戏'',迎面来了石一龙、石一虎两兄弟,铁摩勒本来也算得很机灵了,见是石家兄弟,怕给他们看破,一低头,便想从人丛中溜走。石一龙已一声喝道:``铁少寨王,往哪里走?''说时迟,那时快,突然有几个挽着水桶的小头目,向他迎头泼去。这一``招''阴损非常,要知若是动武的话,石家兄弟也未必能在数十招之内,将铁摩勒擒下,但这么一来,却立即令到铁摩勒``原形毕露'',铁摩勒被淋得全身湿透,面上的油彩和易容散都给洗净了!

王伯通这一喜非同小可,登时理直气壮地大声喝道:``你们瞧见了罢?这小子正是窦老大的干儿子铁摩勒!辛天雄带他来此,所为何事,想诸位都可以不说自明!好呀,他们想为窦家报仇,你们是已背叛了窦家的了,现在是回过头来再扶助这臭小子呢,还是愿意跟从我王伯通?''

辛天雄立即世朗声说道:``诸位别中他的诡计,别把今日之事缠到王、窦两家的纷争上,王、奏两家的纷争留到以后再说,现在要问的是:王伯通要依附安禄山,要为虎作怅,助胡人来夺华夏的江山,你们愿意跟从他吗?''

赴会的绿林群豪,听了这话,登时散了一半。可是王伯通的党羽依然很多,辛天堆的话未曾说完,已是有几个人跳上戏台,向南霁云杀上,全场大乱,人声如沸,辛天雄也没法再说下去了!

南霁云亮出宝刀,与杜百英背靠着背,抵御敌人,眨眼之间,戏台上已围上了三重人,这些人都是王伯通拉拢来的绿林大盗,个个都有看家本领,南、杜二人虽是武艺高强,急切间却也冲不出去。那张忠志趁此时机,已挣脱了南霁云的掌握,抄起兵器,也加入了战团。

台上演出了全武行,台下也展开了大厮杀。王伯通正要走开,韩湛道:``王寨主,今日之事,如何了结。你可不能走啊!''一伸手,便拿他的肩井穴。

猛然间一股劲风扑面而来,精精儿将那张桌子一掀,挡住了辛天雄,跳过来便向韩湛偷袭。这一招是攻敌之所必救,韩湛只得放开了王伯通,反掌向他拍去,精精儿手掌倏张,一道寒光电射而出,原来他掌中扣着一支精芒耀眼的匕首。

韩湛本来是想点精精儿的脉门的,这一下无异凑上去给匕首削地的手指,幸而韩湛有几十年功力。临机应变,手腕一沉,化指戳而为掌削,横掌如刀,立即削精精儿的膝盖。精精儿用个``铁板桥''的身法,向后一印,那支匕首滴溜溜的划了一道圆弧,平刺韩湛的胸口。说时迟,那时快,韩湛早已腾身跃起,一脚踢飞了精精儿那支匕首。可是精精儿的身法也快,不待韩湛身形落地,已先抢上来攻他胁下的愈气穴,韩湛喝道:``来得好!''斜身一掌,顺势再点他的脉门,只听得``嗤''的一声,精精儿从他身旁滑步而过,袖子给他撕去了一幅,可是却并没有给他点中脉门。

这几下兔起鹃落,两人都以上乘的武功相搏,当真是惊险绝伦。精精儿稍稍吃了点亏,但韩湛却也不能将他打败。就在他们交手的时间,王伯通早已避开了。

铁摩勒被他们淋得似个落汤鸡,大为恼怒,拔出刀来,便要和石家兄弟拼命,忽听得一个清脆的女孩子的声音叫道:``铁少寨主,昨日找看在空空儿叔叔给你说情的份上,让你活命,怎么天堂有路你不走,地狱无门你却偏要进来?''王伯通扬声叫道:``燕儿,和他多说做甚?斩革除根,快给我将他一剑杀了!''

仇人见面,分外眼红,铁摩勒明知不是她的对手,豁出性命,向她撞去,王燕羽眉头一皱,道:``你当真想赶着去见阎王吗?''短剑向前一送,直指铁摩勒的心胸!正是:

本是血仇深似海,谁知玉女暗倾心。

欲知后事如何?请听下回分解------

\chapter{第十四回 龙眠谷里掀风浪
玉树山头伏杀机}\label{ux7b2cux5341ux56dbux56de-ux9f99ux7720ux8c37ux91ccux6380ux98ceux6d6a-ux7389ux6811ux5c71ux5934ux4f0fux6740ux673a}

铁摩勒横刀硬劈,他拼着与敌人同归于尽,这一招是将段-
璋教他的剑法化到刀法上来,近身肉搏,凶猛无比。可惜他这套剑法还未练得十分纯熟,剑法主柔,刀法主刚,他将剑法化为刀法,刚多柔少,中路的攻势虽猛,侧翼却露出了空门。王燕羽本领比他高明得多,一见有破绽可乘,立即一个滑步回身,喝一声``着!''剑锋已戳破了他的衣裳,剑尖触及了他的肌肤。

铁摩勒胁下一片冰凉,心中方自叫道:``我命休矣!''想不到那少女突然把短剑抽了出来,悄声说道:``你的胆子果然大得可以,赶快走吧!我饶你一次!''铁摩勒呆了一呆,喝道:``谁要你饶?''猛地又是一刀斫去!

王燕羽`哼''了一声道:``你别大叫大嚷成不成?当心让我爹爹听到了!''不知怎的,她见铁摩勒勇气过人,竟然暗暗的欢喜了他。好在这时,台上台下都在高呼酣斗,王伯通忙着指挥党羽围攻辛天雄这一班人,没有留心听铁摩勒的叫喊。

铁摩勒存心与她拼命,一口气连劈了三刀,王燕羽怒道:``你这臭小子真是不知好坏!''短剑横破,也展开了进手的招数,激战中一招``玉女投梭'',欺身直进,剑光如练,这点他的脉门,想把他的朴刀打出手去。

就在这刹那间,王燕羽猛觉微风飒然,来自背后,她虽然年纪轻,经验少,但自幼得导人传授,深明上乘的武功心法,应变甚为机警,当下左手骈指如戟,贴着铁摩勒的刀背一推,先把他推开,紧接着反手一剑,又将背后袭来的兵器荡开了。回头一看,只见这个赶来救铁摩勒的人正是韩芷芬。

王燕羽笑道:``原来是辛家姐姐,好极啦,我正想再领教领教你的武功!刚才你深藏不露,现在总该抖出两手,让我开开眼界了吧!''韩芷芬骂道:``你这狠心辣手的小魔女,今日我要叫你难逃公道!''王燕羽笑道:``是么?我若当真狠心辣手,你这位好朋友早没了命啦。不信你问问他去?''铁摩勒给她气得七窍生烟,哪肯与她打话,退扑上来,便与韩芷芬联手夹击。

韩芷芬用的一对判官笔,展开家传的点穴手法,笔笔都是指向她的要害穴道,她和王燕羽的武功各有所长,难分高下,但加上了一个铁摩勒,却占了上风。

台下展开了大混战,台上也正自杀得难解难分。南、杜二人,背靠着背,刀剑联防,勇战群盗,无奈众寡悬殊,南霁云虽然大展神威,连伤了几个山寨的寨主,却兀是冲不出去。

酒丐车迟捧起一个大红葫芦,喝了满肚子酒,哈哈笑道:``这场试成真是好看煞人也,哈哈,俺老叫化也忍不着要来凑凑热闹啦!''凑近台前,张开大嘴,一股酒浪便喷了上去,登时有如来了一场暴雨,将台上的群盗冲得脚步歪斜,摇摇晃晃。尤其厉害的是,那股酒液经他运用内家真气喷出,竟似铅弹一般,打着了便火辣辣的作痛,虽然未能致人死命,却也着实难当。

群盗中最厉害的一个名叫祝三胜,使的是一支七节虬龙鞭,这时正自展开``回风扫柳''的鞭法,卷地而来,缠打南霁云的双足,忽地被一股酒浪迎面喷来,登时面前只见一片白茫茫的,眼睛被酒气一黄,睁不开来。南霁云大喝一声,手起刀落,将他劈翻,包围圈立即被冲开了一个缺口,南、杜二人,跳一下了戏台。

王伯通的副手褚遂叫道:``车老二,你我本来是井水不犯河水,你这样胡来,未免太不给主人面子啦!''车迟笑道:``你们又不请我喝酒,我为什么要卖你们的面于?再说,你是知道老叫化的脾气的,我酒痛一发,也就顾不得什么面子不面子啦!来,来,来!你不请我喝酒,我可要请你喝一点!''一张口,又把酒向褚遂喷去。褚遂大怒,一记劈空拳将酒浪冲开,和车迟打在一起。车迟因为和他是相熟的朋友,手下留情,喷他那口酒也未曾运足内劲,只是和他开开玩笑而已。不料褚遂却动了真怒,他的真实本领虽然远远不及车迟,但他却长于近身缠斗的擒拿功夫。王伯通请来的几个一流好手,这时也都拥上前去,帮褚遂合战车迟。

南霁云正要冲出去与辛天雄会合,忽地一股劲风向他扑来,却原来是王伯通的儿子王龙客到了。王龙客这时已识穿了南雾云是谁,冷笑说道:``姓南的,昨日我爹爹手下留情,让你逃下飞虎山,你今日又乔装来此打闹,算得什么英雄好汉?''南霁云喝道:``住口,你两父子甘做安禄山的鹰犬,还敢与我谈论什么是英雄好汉的行径么?''抡刀便劈,王龙客也不打话,举扇相迎。当下又是一场凶猛的厮杀!

众好汉分成几堆厮杀,其中斗得最激烈的还是韩湛与精精儿这对。精精儿早已拔出了``金精铁剑'',但韩湛只凭着一双向掌,掌劈指戳,却似手中捏着了两般兵器,掌劈之时,切、削、勾、拿,如同伸出了一柄五行剑,指戳之时,更赛似五枝判官笔同时点来!饶是精精儿矫捷非常,且又仗着宝剑,却竟然奈何不了他的一双肉掌。

精精儿出道不过数年,韩湛早已隐居,他尚未知道这个自称金鸡山的一个``小头目'',竟是天下第一点穴名家,不由得心中大骇、激战中韩湛用了一绝``拂云手'',似劈,似按,似点,似戳,掌指兼施,变幻莫测,精精儿已经闪得快极,但仍然给他的食指在小臂上划了一下,登时``玉衡''、``瑶光''、``曲池''三处穴道都是一阵酸麻,幸而精精儿的闭穴功夫也已有了相当火候,而韩湛又不是用重手法点他,因此尚不至于当场栽倒!

这时,王伯通也已指挥得力的手下,将辛天雄团在核心,他只道辛天雄乃是主谋,因此才亲自出马,决意将他生擒,立威做众。萨氏双英与辛天雄并肩作战,这三人的武功虽然不弱,但双拳难胜四手,好汉不敌人多。在重重围困之中,却是冲不出去。

韩湛眼观四面,耳听八方,见辛天雄被困核心,险象环生,当下一招``拂云手''将精精儿迫退之后,立即沉声喝道:``看在你师兄的份上,我不伤你,你还不与我滚开!''精精儿吃了一惊,道:``阁下曾姓大名?''韩湛道:``你回去问你师兄,自然知道。我没工夫与你说话!''一声长啸,立即腾身跃起,向王伯通、辛大雄那边扑去。

精精儿哪里还敢再追,心中想道:``不管他说的是真是假,他认识我的师兄,我总以不惹他为妙。''正在此时,王伯通父子都发出了呼援的叫喊;按说精精儿该去助王伯通一臂之力才对,但他对韩湛已有了几分怯意。念头转了几下,终于舍了王伯通,却去帮助他的儿子。

南霁云对王龙客憎恨已极,一刀紧似一刀,刀刀向他的要害招呼,杜百英展开青城剑法,抵挡其他敌人。战到三十余招,王龙客已抵挡不住,虚晃一招,便要抽身,南霁云大喝一声:``着!''一刀向他当头劈下。杜百英急忙叫道:``将这小贼擒住,不必杀他!''

南霁云一听便知道杜百英的意思,那是要将王伯通的儿子擒来作为人质。心中想道:``对,只怕也只有此法,方能迫令王伯通解围。''好个南霁云,心念一转,招数立变,宝刀扬空一闪,迅即从直劈而变为横斩,将王龙客的折铁扇封出外门,左臂一伸,使出``游龙探爪''的擒拿招数,迳抓王龙客的琵琶骨。

可是,高手比斗,相差只是毫厘,王龙客武功非同泛泛,南霁云这一下变招虽快,却给了王龙客脱险的机会,就在南霁云的手指将沾及他的衣裳之际,他已是一个``金鲤穿波'',倒翻出去。

南霁云大怒,使出``登云纵''的轻身功夫,也跃了起来,如影随形,跟着一刀斩下,忽地一条人影从对面撞来,疾如奔马,只听得``咣''的一声,刀剑相交,火花四溅,那人叫道:``好刀法,阁下敢情是魏州南八么?''

来的这人正是精精儿,他在这瞬息之间,一手带开了王龙客,又接了南霁云一刀,确是身手不凡。南霁云朗声说道:``不错,魏州南八,正是区区。阁下这副身手,却甘心为虎作怅,不是太可惜了么了''

精精儿笑道:``此地不是辩论之所,今日也不是辩论之时。前日在飞虎山上未曾领教,深觉遗憾,好在今日又得相逢,我先领教阁下的刀法,然后再听你的教训如何?''这时,王龙客已站稳脚步,定下心神,想起刚才那一刀之辱,又羞又怒,抢上来道:``正是,今日之事,胜者为强,何必与他多说废话!''折扇一挥,先攻上去。精精儿本来不欲以二故一,但他已知道王龙客绝不是南霁云的对手,他是王伯通卑辞重宝礼聘而来的人,刚才因有韩湛在场,他不敢去援助王伯通,已自觉得不好意思,若是如今再让王伯通的儿子遇险,那如何说得过去?

南霁云的武功与段-
璋在伯仲之间,按说也输不了精精儿多少,可是一来他已激战了半个时辰,二来王龙客也是一个劲敌,因此双方交手,还不到二十招,南霁云便已险象环生。杜百英杀退面前几个敌人,冲上来与他会合,形势稍为好转,但杜百英也已到了力竭筋疲的时候,所以仍是不能将局面扭转过来,只有招架的份儿。

正在吃惊,忽听得有人叫道:``夏姑娘来啦!''王龙客怔了一怔,定睛看时,只见夏凌霜柳眉倒坚,满面怒容,将迎接她的那个小头目一掌推开,已是挥剑杀了到来!

南霁云见夏凌霜突如其来,也是心头一震,精精儿何等厉害,一见有破绽可乘,立即便是``唰''的一剑闪电般向南霁云刺去!

夏凌霜正好赶到,青钢剑挽了一朵剑花,一招``平沙落雁'',弯腰出剑,刺精精儿的足根,两人动作都快到了极点,只见精精儿``咦''了一声,箭一般地射了出去。原来夏凌霜这一剑来得恰到好处,正是攻敌之所必救,因此饶是精精儿武艺高强,也不得不先避开她这一剑,结果是南霁云和精精儿都没有受伤。

王龙客讷讷说道:``夏姑娘,你当真要与我作对么?你,你,你听我说\ldots\ldots{}''夏凌霜斥道:``你们父子的所作所为,我现在都已经知道了,还说什么?''王龙客道:``怎么,咱们之间已经无话可说了么?''夏凌霜道:``好,我只要再问你一句话,你们是不是已把段大侠谋害了?''王龙客道:``这个么?并没有呀!''夏凌霜道:``为何我找不着他?''王龙客道:``这个么?这个------''他吞吞吐吐,欲说还休,铁摩勒已在那边叫道:``夏姑娘,段大侠还在人间,我知道他的消息,咱们冲出去再说!''夏凌霜道声:``好!''猛地向王龙客喝道:``你还不给我滚开!''反手一剑,嗤的一声,将王龙客的一条衣袖斩了下来,王龙客面色惨白,跄跄踉踉的倒退几步,摆摆手道:``让她出去。''

精精儿道:``且慢,我还要再看她两招剑法!''回身扑上,夏凌霜冷笑道:``你就看吧!''青钢剑唰的刺出,方到中途,已接连变了三个招式,精精儿施展腾挪闪展的功夫,也在这瞬息之间,攻出了四招,两人的宝剑没有碰上,但却是招招惊险,每一剑都足以致对方死命。若论剑招的迅捷,那是精精儿稍胜一筹,但若论到剑法的奇诡,那又是夏凌霜稍胜一筹了。精精儿不由得倒吸了一口凉气,心中想道:``我只道与师兄联手,便可以横行天下,哪知武林中竟有这么多高手,那姓韩的不必说了,只是这个年轻的女子,我若要胜她,只怕也得在百招开外!''

这时韩湛已把王伯通这一班人杀退,与辛天雄突出重围,精精儿已知今日难以讨好,虚晃一剑,跟着王龙客退走。

韩芷芬扬声叫道:``爹爹,就是这位夏姑娘。''韩湛道:``多承夏姑娘相助,咱们外面再叙。''

铁摩勒、韩芷芬二人被王燕羽、石家兄弟等围住,尚未能突破包围,夏凌霜走过去道:``小妹妹,那晚我错疑你了。''运剑如风,替她杀退了石家兄弟,王燕羽怒道:``我哥哥好心对你,你却将我兄妹当作仇人!''侧身一剑挡开了铁摩勒的朴刀,横掌就向她当胸劈下。这一招对铁摩勒是虚,对夏凌霜是实,当真是很辣非常.

夏凌霜喝道:``撒手。''一招``春云乍展'',剑尖上吐出碧莹莹的寒光,倏的刺到了王燕羽持剑的手腕,她也是剑掌兼施,虚实并用,正是以毒攻毒,解招还招的绝妙手法,而且她的武功较王燕羽又要胜过一筹,虽然掌击乃是虚招,但那一掌向王燕羽顶门拍下,有如奔雳骇电,声势也极是骇人。王燕羽究竟临场经验较少,一时间分不出究竟是剑实掌虚,还是剑虚掌实,说时迟,那时快,但听到``唰''的一声,陡然间只觉得手腕上好似被利针刺了一下,王燕羽吓得魂飞魄散,尖叫一声,短剑登时脱手飞出,铁摩勒一刀斫去,她早已溜进了花树丛中。低头一看,手腕上有三点红点,幸喜只是戳伤了一点点表皮。

铁摩勒叫道:``可惜,可惜!''他哪里知道夏凌霜乃是手下留情,要不然,若是剑招用实,王燕羽的一只手早已断了。

车迟笑道:``褚老大,我的朋友都要走啦,剩下我一个人打架没什么意思,我也要失陪啦!''蓦地一个转身,将两个正在问他攻击的盗魁拉着,反手一推,送到了褚遂的跟前。褚遂的大擒拿手已经发出,双手一抓,恰恰抓着这两个人,只痛得他们杀猪般似的大声叫喊,气得褚遂七窍生烟,连忙松手,那酒丐车迟早已与韩湛他们会合,杀出去了。王伯通暗通安禄山之事被揭发后,不但邀请来的贺客散了十之七八,连他的党羽也已有一半离心,还剩下的那班忠心于他的死党,见敌人如此厉害,王伯通和精精儿都不敢去追,他们也就只是虚张声势,吆喝一番。不消片刻,韩湛这一干人便已闯出了龙眠谷。

韩湛一看,后面已然没有追兵,哈哈笑道:``这一仗虽然没有获得全胜,亦已令得王伯通众叛亲离,绿林豪杰,想来也不会再受他们父子之骗了!''

车迟忽然走近夏凌霜身边,摇头晃脑的向她上上下下打量一番,喷喷赞道:``好一位美貌的姑娘;真像冷女侠当年!''他说话之际,酒意薰人,夏凌霜不太高兴,心里又在暗暗奇怪:``这臭叫化怎么知道我的来历?''

车迟解下葫芦,喝了一大口酒,说道:``我叫酒丐车迟,夏姑娘想必听得令堂说过?''夏凌霜道:``没听说过。''车运碰了一个钉子,哈哈一笑,似乎想说什么话却没说出来,只好用笑来掩饰窘态。

南霁云为了免至场面尴尬,说道:``夏姑娘,今晚多承相助,这厢道谢了。''

夏凌霜道:``你这个人怎么婆婆妈妈的,谢什么?你护送我的段叔叔,我也还未曾向你多谢呢。''南霁云也碰了她一个软钉子,但心里却是甜丝丝的,因为夏凌霜虽然是责备他,但语气之中,显然已是把他当作自己人了。

夏凌霜道:``摩勒,你刚才说到段叔叔要往凉州玉树山清虚观,为的何事?''铁摩勒在路上已把那日在飞虎山发生的事情说了一半,这时便续下去道:``是空空儿请他们夫妇去的,要将孩子交还他们。''夏凌霜道:``哦,原来如此。这么说,比起他的师弟来,空空儿倒还不算一个坏人了。''韩湛插口道:``这几年来我虽没有见过空空儿,却颇留心他的行径,他是有点任性胡为,而且因为所向无敌,在江湖上声名鹊起,也不免骄傲了些,但却未做过什么伤天害理的恶事。这回他是受了王伯通父子之骗的。''

夏凌霜听他们一再提起王伯通父子,心中感到有些难过,低下头便不再搭话,南霁云道:``夏姑娘以前是怎么认识他们的?''夏凌霜道:``这有什么奇怪,在路上碰上的。在江湖上行走,哪一天不碰见生面的人?我又不知道他们是什么绿林大盗!''南霁云再碰了一个软钉子,心里感到又酸又甜,从神情语气看来,南雾云可以猜测得到:夏凌霜以前可能对王龙客有些好感,甚至有些情意,但现在已是烟消云散了。

韩湛道:``寒舍离此已不到三十里了,夏姑娘请到合下歇歇如何?''夏凌霜道:``多谢韩老前辈好意,我早与段大侠有约,要到飞虎山看他的,因事耽搁,迟了几天,想不到便发生了这样的变故,现在既已知道了他的消息,我想赶到玉树山去会他。''说罢,一声长啸,一匹小白马从林中疾跑出来,转眼间便到她跟前停下,铁摩勒大为羡慕,说道:``这匹白马看来不起眼,却比我父亲当年那匹红鬃马还要好些!''

夏凌霜跨上白马,拱手向众人道别,南霁云忽道:``夏姑娘,我还有一句话说。''夏凌霜道:``什么?''南霁云道:``关于皇甫嵩那件案子,我回去问我的师父,或者可能知道一点端倪,最少也可以帮你再找到他。请姑娘留下个地址。''夏凌霜道:``我行踪无定,还是我去找你方便些。我见过了段叔叔后,和他一道到九原找你吧。''南霁云大为高兴,叫道:``好,我在九原郭太守府中等你!''马铃叮当,夏凌霜已经去了。铁摩勒道:``南叔叔,人家走远啦,你好像还有话未曾说尽似的!怎么又不早叫着她?现在来不及啦,咱们也该走啦!''

南霁云面上一红,道:``小鬼头,油嘴滑舌!''车返忽地问道:``皇甫嵩的案子?那位夏姑娘是不是要向皇甫嵩报仇?''铁摩勒道:``不错,但这件事情还是个疑案。皇甫嵩说不是他干的,段叔叔却又认为是他。''车返道:``慢着!慢着!她是给谁报仇?是给她的妈妈报仇么?''南霁云怔了一怔,道:``车老前辈敢情是清楚此事。她并没有说是为她妈妈报仇,只是说要奉母命给江湖除害。但据段大侠所言,当年在洞房之夜遭皇甫嵩害死的那个新郎就是她的爹爹夏声涛,而她却又似乎并不知道这件案子就与她的家庭有关,这究竟是什么一回事情?我们听了几方面的说话、,反而越弄越糊涂了!车老前辈若知真相,可以为我们一释疑团么?''

车返望了南霁云一眼,笑道:``啊,你倒是很关心这位姑娘。''接着摇了摇头,又笑道:``这话还未到说的时候。不过,我却可以替你办一件事情------''南霁云不觉又任了一怔,心道:``我有什么事情要你代办?''车迟顿了一顿,说道:``你心里未说的话我已经知道了!你放心;我一定替你做大煤,要是她不睬我这个臭叫化呢,我还有办法,我可以找小段帮我一同去说。''南霁云臊得满面通红,道:``老前辈,取笑了!''

车迟一本正经地说道:``谁说我是开玩笑的?我现在就去!老实告诉你吧,我到龙眠谷就是想等这位夏姑娘来的,可是她却好像讨厌我这个老叫化,好啦,现在我给她找到一位如意郎君,应该可以讨到她的欢喜了!''一晃身,果然拔步便走。

韩湛叫道:``车老二,你到玉树山若是见到了空空儿,就把王伯通暗通安禄山之事告诉他吧。他要是不信,你就说是我讲的。''车迟道:``我理会得!哎呀,我不能再耽搁了,再耽搁就追不上她啦!''

车迟去后,韩湛说道:``江湖三异丐,疯丐卫越嫉恶如仇,出手狠辣;西岳神龙皇甫嵩行事诡异,是正?是邪?尚难论定。只有这位酒丐车迟,虽然玩世不恭,却最是古道热肠,欢喜助人。九流三教,都是他的朋友。不过他的毛病,也就是心肠太软,若非碰到了大奸大恶,轻易不会动怒。所以在他所交的朋友之中,好人坏人都有。''南霁云道:``他刚才不肯说,不知是否有意替皇甫嵩隐恶?''韩湛道:``我看这个或者还不至于,要是皇甫嵩当真干了那件血案,疯丐卫越和他都是夏、冷二人的好友,卫越早就该与他联手将皇甫嵩干了!呀,这件血案当年轰动武林,也曾有许多侠客替夏家查究凶手,想不到如今过了二十年,还是未能破案!''

韩芷芬道:``爹爹,经过了今日龙眠谷这一场大闹,咱们只怕不能在此地安居了,不如也到玉树山去走一趟。''韩湛笑道:``我知道你是想去趁热闹。''韩芷芬道:``是呀。要是空空儿和段大侠夫妇再打起来,你也好去劝解。''韩湛道:``你若是怀着这个念头,那就准保失望。空空儿已经答应了将孩子交还他们,又怎会再打起来呢?''韩芷芬道:``你不怕他的师弟精精儿从中捣鬼么?''韩湛道:``我也曾防到这一层,但酒丐车迟已经去了,即算精精儿要去捣鬼,车返也会赶在他的前头。我已经叫车迟替我传话,空空儿不信车迟也会相信我的。''顿了一顿,再说道:``我倒是担忧他们不会放过南大侠与铁少寨主,所以我打算今晚连夜起程,送他们到睢阳去。然后再和南大侠到九原去看郭令公,将王伯通与安禄山的事情告诉他,也好让他早作准备。据我推测,空空儿可能和段大侠化敌为友,将来也到九原来的。''南、铁二人喜出望外,尤其是铁摩勒,他和韩芷芬年龄相若,相识之后,即甚为投合,正舍不得分离。

夏凌霜策马走了一程,忽听得背后有人大叫道:``夏姑娘,请等一等,俺老叫化有话要说!''夏凌霜回头一看,可不正是那酒丐车返?只见他背着大红葫芦,气喘吁吁的赶来,眨眼之间,已到马后。夏凌霜不由得大吃一惊,心中想道:``我的坐骑乃是日行千里的宝马,这老叫化居然追赶得上,轻身功夫,岂非比空空儿还要高强?''岂知车返熟识道路,他是从小径抄过来的,不过,虽然如此,他的脚程之快,亦是足以惊世骇俗的了!

车迟张嘴说话,酒气喷人,夏凌霜心里已是讨厌之极,忍着气问道:``车老前辈有何话说?''车返道:``听说你要杀那西岳神龙皇甫嵩?''夏凌霜道:``不错,他作恶多端,我是奉了母命,要为江湖除害。''车返道:``这人你杀不得。''夏凌霜道:``为何杀不得?''车迟道:``你母亲说他所做的那些坏事,没有一件曾是他亲手干的!''夏凌霜大怒,顾不得什么前辈不前辈,便即骂道:``胡说,依你的话,难道是我的母亲说谎不成?''车迟道:``你的母亲也不是说谎,这里头有误会。你母亲的仇人不是他!''夏凌霜道:``我母亲也并非与他本身有仇,但他曾害了不少人,所以我母亲定然要我杀他。我看,误会的是你。''车迟道:``不对,不对,不对\ldots\ldots{}''夏凌霜见他神色语气非常奇特,诧道:``怎么不对?''车迟叹口气道:``呀,这话跟你说不明白,你母亲现在哪儿?我和她说去!''

夏凌霜淡淡说道:``我妈不见外人,你有话就向我说。''车迟皱起眉头,似是欲说还休,夏凌霜愠道:``你不愿意跟我说,那就算了。我可要赶路啦!''提起马缰,放开马蹄便走。车迟又赶来叫道:``好,我便和你说!''夏凌霜已是极不耐烦,在马背上回头道:``你说吧,我听得见,不用大叫大嚷!''

车迟道:``皇甫嵩与那件血案毫不相关,对不住你妈的是另一个人,这个人么------''夏凌霜道:``怎么样?''车迟道:``这个人虽是行为不端,但却也不能由你将他杀掉!''夏凌霜冷笑道:``我根本就不知道你说的什么,哼,哼,皇甫嵩是好人不能杀,另一个坏人也不能杀,你的话真是好奇怪呀,哼,哼,不用说啦,我知道你与皇甫嵩都是一丘之貉!''

车迟叫道:``你再听我一句话行不行?''一掠数丈,伸手便拉她的马尾叫道:``你知道你姓什么?你不姓夏,你的爹爹也不是夏声涛!''

夏凌霜大怒,反手便是一剑,厉声骂道:``放屁,你要撒酒疯便在别处去,我不能听你的污言臭语!''这一剑居高临下,劲道十足,凌厉非常,车迟并不想与她性命相搏。只得放开双手,一个``金鲤穿波'',斜窜出去,避开她这一剑,说时迟,那时快,夏凌霜早已``唰''的一鞭,催动坐骑,绝尘而去。她这匹马乃是日行千里的宝马,夏凌霜将它放尽,当真有如追风逐电,车迟哪里还追赶得上?

夏凌霜一口气跑出了十多里,余怒未息,但心里又觉得有点奇怪,暗自想道:``他虽然酒气熏天,却非醉得胡里糊涂的模样,难道他老远赶来,是存心向我胡说八道的么?''这么一想,不觉也起了怀疑:莫非他语里有因?但随即想道:``绝无此理!人人都说我似妈妈,我怎会不是她的亲生女儿?我妈妈只有一个丈夫,我的爹爹怎会不是夏声涛?哼,不管这臭叫化是否酒醉胡说,他总是侮辱了我的母亲!''可是,虽然夏凌霜不信车迟的话,心里却因此而蒙了一层阴影。当下想道:``段大侠是我爹妈的好友,待我见了他,再把这酒丐的疯语告诉他,看他怎么说?''

段-璋和窦线娘为了急于要回孩子,日夜兼程,赶往玉树山。这日已到了山口,窦线娘认定空空儿是她母家的大仇,这次要向仇人讨回孩子,既觉气愤又觉尴尬,段-璋一路开解,几是未能消散她心头的郁气。

玉树山峭拔奇兀,山峰上的积雪亘古不化,远远望去,果然似一枝硕大无朋的晶莹玉柱,高出云霄。入山之后,山势更是越来越为险峻,触目所及,到处都是嵯峨怪石,突出雪上。从山口进去,有一条狭长的山谷,曲曲折折,望不见尽头,阴沉沉的寒气迫人。窦线娘起了怀疑,说道:``大哥,要是空空儿不怀好意,故意将咱们引进荒山,把咱们害了,也无人知晓。''段-璋道:``线妹,你也忒多疑了,那空空儿的本领远在咱们之上,若他要害咱们,何必费如许心力?''窦线娘道:``玉树山离飞虎山约莫有八百里,他劫了咱们的孩子,为何不就近收藏,却要藏在八百里外的荒山上?''段-璋对此点亦是百思不解,为了安慰妻子,只好替空空儿想出理由来解释道:``或者是他要炫耀自己的轻功,令咱们慑服,也说不定。''

空空儿那晚劫了他们的孩子,第二日下午就到飞虎山挑战,若然他真的已到玉树山打了一个来回,这脚程之快,当真是不可思议了。窦线娘摇了摇头道:``我不相信他在一日一夜之间,便能走一千多里,只怕有九成是骗咱们来的!''段-璋道:``再不然,或者这里本来就是他的老家,他信不过王伯通,所以托人将咱们的孩子送到这里收藏?''窦线娘道:``你就这样相信空空儿?''段-璋道:``已经到了这里,不相信也没办法了。反正以咱们的脚程,至多不过半日,就可以上到玉树山的主峰,那时自然可以水落石出。''窦线娘嘀咕道:``起初我不知道玉树山有这么远,越走我越怀疑,看来呀,咱们这回是白走一趟了。空空儿即使不是有心加害,也是有意将咱们戏耍的了。''

段-璋道:``线妹,事情别尽往坏处想。''话犹未了,忽听得``轰隆''一声,一块大石块从山上滚下来,段-璋还以为这是偶然,那料刚刚避过,跟着又有几块大石头滚下。窦线娘叫道:``上面有人!''只见山峰上影绰绰的现出几个人来,同声喝道:``笨蛋,谁叫你们自投罗网,进了绝地,还想活命么?''段-璋这一气非同小可,大骂道:``空空儿,我当你是一条好汉,想不到你竟是这等卑鄙无耻的小人,你站出来!''上面那些人冷笑道:``收拾你们这两个蠢家伙还用得上空空儿么?''

这时,段-璋也认定是空空儿指使的了,冷笑斥道:``用这等下三流的伎俩,藏头缩颈不敢见人,真是无耻之徒!''窦线娘道:``这等小人,不值得骂,与他们拼了就是!''

那些人高踞山头,卖线娘的弹弓打不得这么远,他们居高临下,将石块抛掷下来,那却是比窦泉娘的弹弓厉害得多了,但见石块满空乱飞,有如殒星纷落。窦线娘大怒,施展上乘轻功,腾挪闪展,片刻之间,已在峭拔的山壁上前进了十数丈,弹弓还差一点点距离,就可以打到,忽地``轰隆''一声,磨盘大的一块雪块从悬岩上坠下来,段-璋急忙伸手抓着他的妻子,窦线娘借他这一抓之力,两人携手,似荡秋千一般,斜飞出数丈之外。但听得轰轰隆隆,山呜谷应,那块巨大的雪块滚过,在坡上辗了一道沟,两夫妻被溅了满身泥土,要不是段-
璋助她一臂之力,只怕她的轻功虽好,也难免给雪块压伤。

窦线娘浑身冷汗,道声:``好险!''段-璋道:``都是我连累了你,我太过轻信人了。''窦线娘咬牙说道:``已然处此险境,咱们只有死里求生!''两夫妻在乱石袭击之下,又向前闯。

山坡上的积雪受了震动,在狂风中呼啸,炸裂,就像无数巨大的冰弹,纷纷飞来,从头顶上滚过,从身边飞过\ldots\ldots 比起石块的袭击,更是凶险百倍。段-璋为了掩护妻子,身上已被擦伤了好几处,幸而打中他的,不是巨大的雪块,要不然后果更是不堪设想。段-璋只得和妻子在一处凹进去的山坳,暂躲一躲。但这样一来,有了固定的目标,就更容易受到攻击了。山头上的那班人;将大石头纷纷向他们藏匿之处抛掷,段-璋遮着妻子。有几次险险给石头打中,幸而他的功力深湛,近身的石块,都给他以掌力震了开去,但这样不消多久,他也累得不堪了。

段-璋叹口气道:``好在现在尚未引起雪崩,不过,不过\ldots\ldots 唉,我好恨呀!难道咱们今日当真该当命绝?''要知,若是引起雪崩,山巅大量的积雪都冲泻下来,那就决非血肉之躯所能抵挡了。段-璋怕的就是积雪继续受到震动,终于会引起雪崩。窦线娘凄然笑道:``咱们做了十载恩爱夫妻,要是能够同年同月同日死,我也没有什么怨恨了。''

忽然间,石块的袭击似乎减弱了许多,段-璋道:``现在尚未绝望,咱们冲出去看,总胜于束手待毙。''两夫妻刚从山肋奔出,便听得山峰上有呼叫之声!

只见山峰上现出一个少女的影子,正在持剑追逐盗徒。段-璋又惊又喜,叫道:``是夏姑娘吗?''那少女也在扬声叫道:``是段伯伯吗?快从这边上来,咱们来个上下夹攻。''

原来夏凌霜见他们在谷中受困,她便从另一面绕过,攀上山头,与群盗展开激战。群盗与她处在同一高度的地方,不能像对付段-
璋夫妇那样用石头来抛掷她,而且因为要分出人手抵挡,对段-璋夫妇的袭击也便减弱了。

窦线娘趁此机会,疾奔上去,弹弓一拽,觑准了在夏凌霜面前的一个敌人便打,弦声响过,那名强盗应声而倒,紧接着夏凌霜``唰''的一剑,又刺伤了一个强盗。

群盗两面受攻,登时主容易势,不消片刻,段-璋夫妇已将跃上山头,盗魁叫道:``风紧,扯呼!''窦线娘施展神弹绝技,噼噼啪啪的一顿弹弓,将群盗打得头崩额裂。段-璋叫道:``打环跳穴,好歹留下一个活口。''

窦线娘再拽弹弓,三粒弹子,连珠射出,那强盗魁武功较强,横刀将射她的那颗弹子磕飞,但他左右的两个同伙,却给弹子打中手,一个打中手腕,一个正中腿弯的``环跳穴'',这``环跳穴''乃是足少阳经脉的一个重要穴道,给弹子打中,登时两腿麻软,``卜''地便倒。

那盗魁忽地一脚将这个伙伴踢下山坡,紧接着自己和衣滚下,群盗明知危险,但为了逃命,也都学他的模样,一个个和衣滚下山坡。山壁峭拔、积雪如镜,在雪面上滚下去快速非常,夏凌霜轻功虽好,也追赶不上。

突然间脚下一阵震动,雪块炸裂,声如雷鸣,段-璋叫道:``不好,是雪崩了!''幸而他们这时已登上峰顶,积雪从高处喷泻而下,越在下面,危险越大,霎眼之间,那群强盗徒已给冰雪淹没,只留下他们凄厉的叫声混杂在雪块炸裂与狂风呼啸的声音之中。

段-璋夫妇藉着高处的大石作掩蔽,幸而逃过了这场灾难,目睹这等惨酷景象,也不禁心惊肉跳。段-璋定了定神,说道:``可惜,可惜!''窦线娘道:``可惜什么?''段-璋道:``可惜未曾擒得一个活口,好迫问他的口供。''

窦线娘道:``何用迫问口供,这班人当然是空空儿的党羽了。大哥,难道你到了此时此际,还相信他吗?''段-璋默然不语,疑云却未全消,暗自想道:``这班人只是黑道上二三流的强盗,以空空儿的眼界之高,岂能看上他们?即使说他不好意思亲自出来加害于我,也该另请一些本领高强的人来,何须用这班不成材的强盗?''但若然不是空空儿指使;这班人又焉能知道他们夫妇今日要进玉树山?

这时夏凌霜亦已从一个山洞走出,向他们走来。窦线娘早就听得丈夫说过在路上与夏凌霜相遇之事,也知道了她便是当年白马女侠冷雪梅的女儿,心里暗暗喝彩:``好一个漂亮的姑娘,大哥说她非常似她的母亲,怪不得冷女侠当年能令武林倾倒!''

段-璋道:``凌霜,怎的这样巧,你也来了?今日好险,真是多亏了你啦!''夏凌霜道:``段伯伯,你受了空空儿的骗了,空空儿和那王家父子,都是和安禄山暗通声气的,他们要帮安禄山造反哪!''段-璋吃了一惊,道:``此话可真?''夏凌霜道:``我亲见亲闻,焉能有假?而且,事情也已经做出来了!''当下将那晚她到龙眠谷偷听到的谈话,和第二日群雄大闹龙眠谷的事情,一一告诉了段-璋,并道:``我就是恐怕他们加害于你,所以急急赶来。''窦线娘淡淡说道:``如何?你还相信空空儿吗?''

却不知夏凌霜那晚偷听到的谈话,只是王伯通父子与精精儿、张忠志等人密谋将来助安禄山起兵造反的一节,至于王伯通所说要暂时瞒住空空儿那一节,夏凌霜却没有听到。在她想来,空空儿和精精儿是师兄弟,空空儿当然也就是和他们一鼻孔出气的人。大闹龙眠谷之后,她和韩湛、南霁云诸人又是匆匆分手,因此也就未曾从韩湛口中得知空空儿的为人。

夏凌霜之所以想到段-璋可能在途中遭受暗算,那是因为王龙客的态度引起她的疑心的,王龙客不肯说出段-璋的去向,甚至故意骗她,说是段-
璋可能回转长安,害了她空走一遭,骑白马奔驰三百余里。在往长安时,铁摩勒已经说出他知道段-
璋的去向了,她追问王龙客,王龙客却还是吞吞吐吐,令得她又是伤心,又是愤怒。

夏凌霜却没想到,这事全是王伯通父子在暗中布置,空空儿毫不知情。要知段-
璋乃是窦家女婿,王家父子当然害怕他们夫妇将来要为窦家报仇,当时不过是碍于空空儿的面子,不得不放而已。空空儿一走之后,王伯通立即用飞鸽传书,通知凉州的分舵,叫他们派人在玉树山山口埋伏,干掉段-
璋夫妇。夏凌霜因为和王龙客曾有一段交情,知道了他的真面目之后,甚是伤心,所以她就是在段-
璋面前也不愿提起王龙客的名字,当然更不会谈到她的疑心是因为王龙客的态度而引起的了。这样一来,由夏凌霜所见所闻的事实,就更证实了空空儿的罪名,连段-
璋也不能不相信了,虽然他还有一点点怀疑,觉得以空空儿的本领,实在无须用这等卑劣的手段。

窦线娘黯然说道:``如此看来,咱们的孩子只怕是凶多吉少了。空空儿既是存心骗咱们人他的陷讲,哪还会交还咱们的孩子?''段-璋道:``事已至此,先找着了空空儿再和他理论。''窦线娘道:``这个当然,我若是要不回孩子,我也不想活了,和他拼了就是。''

夏凌霜将白马放在谷中吃草,一行三人,翻过山头,向玉树山主峰进发。一路上并无阻障,走了半天,在夕阳将下的时分,攀上了峰顶。

山顶豁然开朗,鸟飞兽走,花木葱宠,原来山顶上有许多温泉,地气比山脚还要温暖。

段-璋一看,山顶上果然有一座道观,心中燃起一线希望,急忙上前叩门叫道:``段某践约而来,请主人出现!''

哪知一连叩门几次,里面却是毫无声息。窦线娘笑道:``他做了亏心事,哪里还敢见咱们。这个时候,还和他讲什么客气,打进去就是。''

段-璋抱拳说道:``空空儿,你再不露面,请恕段某无礼了!''交代过后,张开拳头,使出金刚掌力,``砰、砰''两掌,登时将大门震开。

窦线娘提起弹弓,夏凌霜拔出长剑,护着段-璋便往里闯,里面沓无人影,夏凌霜道:``莫非他是作贼心虚,挟着尾巴逃了?''

道观没有多大,片刻之间,便已搜遍。在最后一间房子,发现一个摇篮,再仔细寻找,又找到了一些女人衣物。窦线娘哭道:``咱们的孩子给他害了。''段-璋沉吟:``他害小孩子有什么用?孩子是曾经在过这儿,可见他没有完全说谎。''正是:

慈母觅儿儿不见,案中有案费疑猜。

欲知后事如何?请听下回分解------

\chapter{第十五回 爱儿被夺仇无解
身世难明恨正长}\label{ux7b2cux5341ux4e94ux56de-ux7231ux513fux88abux593aux4ec7ux65e0ux89e3-ux8eabux4e16ux96beux660eux6068ux6b63ux957f}

窦线娘怒道:``空空儿不见,孩子也不见,即使未曾害死,也定是被他另外收藏起来了。大哥,他要了咱们的命根子,你还替他说话吗?''他们做了十年夫妻,这次还是窦线娘第一次顶撞她的丈夫。段-璋道:``我这不过是从好处着想,要是空空儿当真不还咱们的孩子,我也是要和他拼命的。''

段-璋端详了一会,又道:``看来是另有一个女子在照料婴儿,摇篮中的锦缎上还有婴儿的尿渍,似乎未曾走了多久,只不知这个女子却是空空儿的什么人?''窦线娘道:``你在这里琢磨推测有什么用,总要找到了空空儿这贼子才有办法。''

就在这时,忽听得外面有人扬声叫道:``段大侠果是信人,请恕我失迎了。''段-璋叫道:``是空空儿来了!''说时迟,那时快,窦线娘已急不可待的跑了出去。

只见空空儿双手空空,哪里有她的孩子?窦线娘大喝道:``好呀,你将我们骗上山来,却把孩子藏到哪里去了?''嗖、嗖、嗖,三颗金弹,连珠发出。

空空儿滴溜溜的转了一圈,避开三颗金弹,叫道:``且慢,且慢,我有话说!''段-璋赶了出来,说道:``线妹住手,且听他说些什么?''

空空儿道:``孩子暂时未能交还你,但请你放心,你的孩子好好的,决不会有丝毫损伤!''段-璋道:``为什么不能现在交还?''空空儿的神情显得有点尴尬,讷讷说道:``这个么这个------''窦线娘骂道:``什么这个那个的,今日不还我的孩子,决不与你干休!''

空空儿摊开双手说道:``总之,包在我的身上,定然还你的孩子就是。今天么,却是无法从命!''段-璋道:``还我,什么时候?''空空儿道:``这个,这个------我也难以说个定期。''段-璋喝道:``你吞吞吐吐的,这里面到底有个什么原故?''空空儿道:``段大侠,这次算我对你不住,你别追问啦,你若是信得过我,咱们就交个朋友,你的孩子留在一个人手上,只有好处,没有坏处!''

窦线娘怒火冲天,不由得大骂道:``谁还相信你的鬼话,你这卑鄙无耻的小人,好在我们没有给你害死,这条命我也不想要了,与其让你再用下流的手段暗害,不如现在就与你拼了吧!''

空空儿是个心高气傲的人,几曾受过这等痛骂,不禁气得浑身颤抖,戟指喝道:``你,你,你这臭婆娘敢胡乱骂我!''段-璋这时亦已是怒气暗生,见他侮辱自己的妻子,登时也爆发出来,拔剑喝道:``骂你又怎么样?你不该骂吗?''

空空儿气得哇哇大叫:``好呀,段-璋你也骂我!我怎么该骂了?''段-璋骂道:``我骂你是个不明是非、助约为虐的恶贼,我骂你是个做了恶事,却要抵赖的小人,我骂你是个卑鄙无耻的下三流小贼\ldots\ldots{}''

空空儿面色铁青,喝道:``段-璋,你给我磕头赔罪,否则休想下山!''段-璋冷笑道:``你给我磕头我也不饶你呢!不错,你的武功是远胜于我,但大丈夫死则死耳,有何惧哉?即使死在你的手上,也一样要骂!''

空空儿大怒道:``好,你既认定我是恶贼,可休怪我不留情面了,好,你再骂吧!''身形一闪,一掌便向段-璋面门掴来!

这一掌来得迅若狂飙,幸而段-璋早有准备,一个弯腰折柳,已是宝剑出鞘,向他下三路刺去,说时迟,那时快,窦线娘亦已揉身疾上,一刀向他手腕劈下。

好个空空儿,就在刀光剑影之中腾身而起,饶是段-璋应付得直,闪避得快,背脊也给他的掌缘擦了一下,辣辣作痛;空空儿这一掌本来是想打段-璋一记耳光的,幸亏段-
璋没有给他打着,要不然这更是奇耻大辱,两人的冤仇,也将终生难解!

段-璋气极怒极,叫道:``线妹,你说得不错,对付这等恶贼,只有与他拼了!''空空儿头下脚上,似兀鹰般俯冲而下,一道蓝艳艳的光华从他手心吐出,他抽出了他那柄锋利无比的匕首,人未落地,早已是一招两式,分袭段-
璋夫妇。

段-璋年轻时候游侠四方,久经阵仗,武功虽逊一筹,经验却比空空儿丰富得多,见他腾身飞起,早料他有此一着。宝剑扬空一划,剑光倏的合成一个弧形,窦线娘趁势一刀从剑底穿出,两夫妻配合得恰到好处。但听得当当两声,段氏夫妻各自退后三步,窦线娘的缅刀损了一个缺口,空空儿的衣袖却给段-
璋的剑尖穿过,不是空空儿缩手得快,险些给他划破了脉门。

这一来,双方动了真怒,都把全副本领施展出来,这一战比在飞虎山上的那一场恶战还要激烈得多!段-璋豁出了性命,展开一派进手招数,剑光挥霍,隐隐带着风雷之声,窦线娘以游龙八卦刀法绕着空空儿疾走,也是刀刀不离空空儿的要害。他们那日败给空空儿之后,曾用心推究致败之由,反复解拆了当日的招数,如今再度交锋,已是今非昔比了。

战到分际,空空儿忽地叹口气道:``贤伉俪苦苦相迫,我是无可奈何,只好舍命相陪了!''他刚才火气冲天,这几句话却说得甚是苍凉,且带着几分惋惜。

段-璋心中一动,正自想道:``难道空空儿果有苦衷,不足为外人所道。''陡然间,只见空空儿短剑盘旋,招数倏变,指东打西,指南打北,冷电精芒,续纷飞舞,剑光线绕中,四面八方都是空空儿的身影,当真是翩若惊鸿,宛若游龙。段-璋大吃一惊,迫得易攻为守,回剑防身,但听得叮叮当当之声,有如繁弦急奏,就在这瞬息之间,段-璋的宝剑已与空空儿那支匕首形的短剑接触了九下。

原来空空儿本意不想与段-璋为敌,给他激怒之余,也只是想把他们夫妇打败,迫他们赔罪而已。可是段-
璋夫妇已认定他是个狡猾奸恶的魔头,下手毫不留情,到了此际,空空儿若还不使出杀手绝招,势将自身性命难保!

空空儿用的是独门刺穴招数,在一招之内可以连袭对方九处大穴,若然给他刺中,不死也将残废。空空儿对段-
璋本有惺惺相惜之意,故此在他使出这等极其厉害的杀手招数之时,禁不住低沉叹息。

段-璋以前与精精儿恶斗之时,精精儿也曾使用匕首刺穴的毒招,可是精精儿只能在一招之内,刺对方七处穴道,段-璋还勉强可以应付,如今空空儿虽然只是在一招之内,比他的师弟多袭两处穴道,但高手比斗,相差毫厘,多要照顾两处穴道,艰难已不止一倍。何况空空儿的轻功当世无双,比起精精儿更是高出何止十倍。他以闪电般的身法展开闪电般的刺穴神招,段-璋虽是夫妻联手,也给他迫得只有招架之功,毫无反击之力。战到紧处,两夫妻都好似感到有数十支明晃晃的匕首,在他们的身前身后,身左身右,穿来插去。

夏凌霜奔上前来,高声叫道:``段婶婶,你退下去用弹弓打他!''青钢剑扬空一闪,替窦线娘接了空空儿的一招,夏凌霜的剑法以奇诡见长,论功力不及段-璋,但却要比窦线娘的八卦刀法厉害得多,空空儿噫了一声。叫道:``你的剑法是何人所授?''夏凌霜一声不响,疾进二招,每一招又分为三式,虚虚实实,变化莫测,段-璋趁势反攻,空空儿颇为惊诧。这时,已至双方性命相搏的时候,段、夏二人固然感到呼吸紧张,即空空儿亦已不能分心说话。双方只有哑斗!

窦线娘闪过一旁,一拽弹弓,嗖、嗖、嗖,三弹连发,一取空空儿上盘的``眉尖穴'',一取中盘的``风府穴'',一取下盘腿弯的``环跳穴'',窦家的神弹绝技,果然名不虚传,在这三条人影奔腾跳跃,宛若风驰电逐之中,她竟然能瞄准了空空儿,而且是三颗弹子,分打上中下三个方位,认穴不差毫厘。

空空儿托地一跳,一个鹞子翻身,衣袖挥起,已把窦线娘上中二路的弹子卷去;匕首一翻,身形不变,仍然凌空下刺,但听得``叮''的一声,第三枚弹子也给他的匕首拨开。可是窦线娘的内功也已有了相当火候,空空儿的匕首给弹子碰了一下,刀尖颤动,亦自失了准头,他这一招本来是指向夏凌霜胁下的``魂门穴''的,准头一歪,匕首贴肋而过。说时迟,那时快,段-璋``唰''的一剑,又把空空儿的衣襟削去了一幅!

空空儿大怒,衣袖一挥,将接下的两枚弹子反打出去,段-璋滑步闪开,就在这瞬息之间,但见空空儿那支匕首已化成了一道蓝光,向他前心刺到,段-璋横剑一封,夏凌霜也急忙侧身进剑,三条人影,纠作一团。窦线娘凝神注视,也只是仅能分辨人影,只好暂时停弓不发。

蓦地只听得空空儿一声长啸,三条人影霍的分开,叮咣声响,夏凌霜头上的一股玉钗已给他的匕首削断。

窦线娘急忙再发金弹,空空儿突然和身倒下,施展滚地堂的功夫,短剑贴地盘旋,化成了一团电光,削段、夏二人的双足,窦线娘的弹子全落了空,险险打伤了自己的丈夫。

段-璋长剑下刺,夏凌霜跃起来避招还招,空空儿一击不中,已自长身而起,霎时间三条人影又纠作一团。空空儿的匕首盘旋飞舞,竟然以短政长,将两柄长剑裹在,窦线娘只好又停下弹弓。

这三人倏分倏合,打得难解难分,窦线娘每每觑准了机会,但金弹一发,那边的情况又立即发生变化,她连发了十几颗弹子,仍然打不中空空儿。可是,无论如何,她的神弹绝技,仍是对空空儿的一个威胁,使得空空儿要加意提防,便不能全神对敌,如此一来,段、夏二人才堪堪和他打成平手。

这时已是西山日落,将近黄昏,双方已斗了半个时辰,正在杀得天昏地暗之时,忽听得有人大声叫道:``你们怎的打起来了?住手,住手!''

段-璋在百忙中抽眼偷瞧,只见一个衣衫褴楼的叫化,背着一个大红葫芦,正向着他们跑来。段-璋认得是酒丐车迟。

空空儿也认得酒丐车迟,他见段-
璋已回剑防身,便也停止了攻击,正想与车迟招呼,却不料窦线娘忽地又使出连珠弹的绝技,空空儿冷不及防,``卜''地一下,给弹子在额角上打个正着,血流如注!

段-璋缓了剑招,夏凌霜却趁此时机,运剑如风,连连进击,空空儿大怒,匕首一划,``叮''的一声,又把夏凌霜头上的另一股玉钗削断,段-璋挥剑来援,三个人又纠作一团。

车返温道:``夏女侠,给老叫他一个面子吧!''窦线娘一声不响,金弹接续发出。车迟捧起葫芦,咕噜噜的喝了半葫芦酒,张口一喷,一股酒浪登时似瀑布般的从空中倒泻下来,空空儿、段-璋、夏凌霜等人虽然不怕给酒浪所伤,但给他这酒液一喷,阵形却也乱了。

车迟又把酒浪向窦线娘喷去,阻止她再发弹子,窦线娘脸上给溅了几点酒珠,怒声叫道:``车老前辈,非是我不给你面子,这恶贼与我有夺子之仇,你若给他解围,我的儿子向谁去讨,你赔我么?''车迟怔了一怔,窦线娘又喝道:``你不帮我们这也罢了,若再搅局,恕我窦线娘的弹弓认不得前辈!''声出弹到,车迟捧起葫芦一挡``卜''的一声,弹子打中了葫芦,车迟叫道:``有话好说,别打,别打,打坏了我这个宝贝,老叫化没酒喝啦!''

夏凌霜也叫道:``这老叫化是他们一党,段伯伯不要理他!''段-璋心下踌躇,但这时他们已占到了上风,若然住手,只怕取胜的机会稍纵即逝,何况自己住手,夏凌霜单独一人决然应付不了空空儿,因此只好仍然挥剑猛攻,说道:``车老前辈,事情原委,请你问我内人,你清楚之后,再来劝架不迟。''

窦线娘道:``他约我们到此,却在山口理下伏兵,我夫妻二人几乎给乱石打死,到得此来,他又不肯交还我的儿子,也不知是不是已经害死了?老前辈,你评评理罢!我们该不该与他拼命?''

车迟经过山口,也曾见到几具尸体,当下不禁亦起了疑心,问道:``空空儿,你怎么说?''

空空儿喝道:``你要我说什么?''车迟道:``你当真要害他们夫妻么?''空空儿怒道:``岂有此理,我要害他们早就害了!''车迟又道:``既然你并无坏意,却为何不肯交还他们的孩子?''

空空儿正为此事内愧于心,给车迟一问,期期艾艾,答不出来。

车迟与空空儿不过是彼此认识,并无深交的朋友,在这样的情形之下,他当然是相信段圭璋,不相信空空儿。心中想道:``韩湛虽然敢为他作保,但韩湛认识他的时候,他年纪还小。他们亦已分手多年,焉知空空儿不是变坏了?''当下,疑心一起,不禁大声问道:``空空儿,你吞吞吐吐的,这究竟是怎么一回事?''

空空儿老羞成怒,也大声地说道:``车老二,你是想审问我么?我的事不用你管!''

车迟喝了口酒,冷冷说道:``老叫化生平专管闲事,韩湛韩老前辈叫我问你,你是否利欲薰心,和你的师弟精精儿走上一条路了?''其实韩湛是要车迟告诉空空儿,说明王伯通、精精儿的阴谋,问空空儿知不知道,车迟为了加重语气,这么一问,却变成了对空空儿的谴责。

空空儿和他的师弟情如手足,闻言更怒,喝道:``老叫化,你胡说什么?我师弟有何不对,给你拿了把柄了?''

车迟冷笑道:``你师弟甘心为虎作怅,难道你尚不知情?''空空儿喝道:``你说什么?''车迟又冷冷笑道:``安禄山权势遮天,收买了王伯通不奇,想不到你们师兄弟也甘心请愿作他的鹰犬!如今王伯通与安禄山勾结的阴谋,已大白于天下英雄之前,你还想抵赖么?''

空空儿证了一怔,忽地大骂道:``放屁!你含血喷人!''车迟勃然大怒,登时发作道:``空空儿,你出道不过几年,居然眼睛长到额角上啦,敢骂起我老叫化来啦!''

空空儿听了车迟的话,亦已知道事有蹊跷,但他少年气盛,性子一起,是天塌下来也不管的,车迟话未说完,他便狂笑道:``好呀,你们当我空空儿不是人,我还和你们讲什么交情,老叫化你也上吧!''

空空儿一面说话,一面与段、夏二人恶斗,本来已是险象环生,这时突然激怒,招数躁而不稳,段-璋剑走轻灵,``唰''的一剑,在他肩膊上划开了一道伤口!

空空儿大怒,陡然间展出欺身刺穴的杀手,身形一晃,旋风般的扑到段-璋跟前,匕首一场,俨似毒蛇吐信,倏的就指到了段-璋的心房要穴!

车迟飞身扑去,用葫芦一挡,只听得声如破竹,他那个视同宝贝的沉香木红漆葫芦已给空空儿一剑戳穿,葫芦中的美酒流了满地。就在窦线娘的骇叫声中,空空儿已自腾身飞起,俨如鹰隼穿林,掠波巨鸟,窦线娘的金弹竟自追他不上!

只听得他远远扬声叫道:``段-璋,你要恨我,也由得你,你的儿子,将来总会还你!老叫化,咱们后会有期,我查明之后,再来与你算帐!''说到最后一句,话声已似从山腰传来,空空儿的影子早已不见。

窦线娘走了过来,见段-
璋血流满面,大惊道:``你受伤啦?伤在哪里?''段-璋苦笑道:``没事,空空儿的匕首并未刺中我。''却原来他是给窦线娘的金弹误伤的,与空空儿刚才给窦线娘所伤的部位恰巧相同,也是打穿了额头。

窦线娘仔细一看,发觉是自己的过错,又是心痛,又是羞愧,恨恨说道:``这干刀万剐的恶贼,可惜我刚才那记弹弓,没有打瞎他的眼睛!''

段-璋却自心中想道:``空空儿刚才只要再来一下,我不死也得重伤!以他那样快捷的手法,虽有车老前辈给我一挡,但他戳破葫芦之后,还尽有机会可以伤我。莫非他使此杀手,只是仅求突围,而并非有意伤我的么?''当下说道:``线妹,反正我已侥幸逃了性命,所受的只是轻伤,你不必骂他,也不必难过了!''

车迟却未想到是空空儿手下留情,哈哈笑道:``段大使当真是宽宏大量,非常人所能企及。''接着又笑道:``段大嫂,你现在该不会再骂我老叫化了吧?''

窦线娘急忙谢过,车迟笑道:``只可惜了我这个葫芦,哈,哈,这也是我好管闲事的报应!''

段-璋夫妇都在向车迟赔礼,夏凌霜却站过一边,冷冷淡淡的毫不理睬他。车迟又笑道:``今天接连受了两个教训,爱管闲事,真是惹火烧身,不但空空儿恨我,唉,连夏姑娘现在也还生我的气!''

段-璋不明就理,对夏凌霜的态度颇觉奇怪,说道:``贤侄女,这位老前辈不是别人,正是行侠江湖、人称`酒丐'的车迟,车老前辈,你过来见个礼吧。''夏凌霜道:``我们早已见过了。哼、哼,他纵然不是空空儿一党,也是皇甫嵩一党,我才不把他当作老前辈看待呢!''

段-璋变了面色,甚是尴尬,急忙说道:``夏贤侄,你说话不可无礼。你初出江湖,或者有所不知,车老前辈与那皇甫嵩,还有一个人称`疯丐'的卫越,虽然并称``江湖三异丐',但是皇甫嵩与他们二人的行事却大不相同,皇市嵩奸恶邪僻,做过许多坏事,车、卫两位老前辈,在江湖上却是有口皆碑、嫉恶如仇的侠丐,皇甫嵩焉能与他们相比?你定是有所误会了,赶快过来赂罪吧!''

夏凌霜柳眉倒竖,仍然站着不动,似乎想说什么却碍着段-
璋的面子未曾说出,段-璋更觉奇怪,正想再问,车迟已在笑道:``段大侠,你的为人我很佩服,你这话却说得不对了!''段-璋怔了一怔,道:``怎么不对?''车迟缓缓说道:``老叫化没有你说得那么好,皇甫嵩嘛,也没有你说得那么坏!''

夏凌霜冷冷说道:``如何?你还说他不是皇甫嵩的一党?他处处都在偏袒皇甫嵩,还不许我报仇呢!''

段-璋眉头一皱,问道:``这是怎么一回事?你对车老前辈到底有何芥蒂?''

夏凌霜亦已忍不下气,愤然地说道:``岂止芥蒂,不是看在你段伯伯的份上,我现在就要替母亲雪耻报仇!''

段-璋吃了一惊,问道:``你说什么?车老前辈也是你父亲生前的朋友,他怎会与你母亲有仇?''

夏凌霜杏脸通红,墓地叫道:``他,他对我说了非常无礼的说话,辱及我的爹娘!''段圭璋睁大了眼睛望着车迟,车迟微笑道:``夏姑娘,你可以将我的话讲出来,请你段伯伯断判,究竟是否无礼?''

段-璋道:``夏贤侄,我与你父母乃是手足之交,有话对我但说无妨。''

夏凌霜冷冷说道:``他,他说我不是姓夏,我的父亲也不是夏声涛,这,这,这难道还不算辱及我的爹娘!''说到此处,登时便要拔剑。

段-璋疑心大起,要知当年夏声涛在洞房之夜便即遇害,夏凌霜此身何来,段-璋亦已是早有疑窦,听了这话,急忙按着夏凌霜,再转过头来问车迟道:``车老前辈,这件二十年未破的疑案,你一定知道内情\ldots\ldots{}''车迟拦住说道:``我和你到那边说去。''段-璋说道:``夏贤侄你暂且忍耐,此事重大,我非弄个水落石出不可。你总可以相信我吧!''夏凌霜默言无语,点了点头。段圭漳便跟着车返走出了半里之遥,找到了一个僻静的说话所在。

车返道:``这件惨案发生的时候,我不在场,但我知道你是在场的,听说就在你们闹了新房之后不久,惨案便发生了。''段-璋道:``不错,前后相差大约还不到半住香的时候,新郎就给人暗杀,新娘也给人掳走了。''车迟道:``那么,你可以相信我的说话,夏声涛决不会是这位`夏姑娘'的生身之父了?''段-璋道:``这个,------我相信。那么她生身之父究竟是谁?''车迟不答这话,却先问道:``你可有与凶手瞧过相?''段-璋道:``当时月淡星稀,我只隐约见到他的背影。''车返又道:``其他的人呢?''段-璋道:``当然是谁也没有看清凶手的面貌,要不然也不会成为疑案了。''车返道:``着啊,既然你们谁都没有见到凶手,却怎的咬定是皇甫嵩?''段-璋道:``第一,是新郎临死前写的那个`皇'字;第二,凶手的背影与皇甫嵩相似;第三,如果不是皇甫嵩,为什么冷雪梅一定要她女儿杀他?''当下,将当晚的经过情形,详细的告诉了车迟。

车迟叹口气道:``怪不得新郎新娘都疑心是皇甫嵩,唉,新郎死得冤枉,新娘更加不幸,直到现在,尚未弄清真相。''段-璋急忙问道:``然则真相究竟如何?到底谁是凶手?''车迟道:``凶手不是皇甫嵩,不过与皇甫嵩颇有关系,这凶手么,他,他------''段圭湾等待这答案已等了二十年,这时见他吞吞吐吐,大为焦急,忍不着催问道:``他,他是谁?''

车迟再叹了口气,说道:``我本来只是向冷雪梅说的,但冷雪梅不肯见我,你是他们夫妻的知交,我只好对你实说,他呀,他是\ldots\ldots{}''

刚说到这个``是''字,忽然微风飒然,从背后袭来,段-璋叫道:``有人!''说时迟,那时快,只听得车迟大叫一声``是你!''张开双手似是要保护段-璋,可是他叫声未绝,身子却忽地似木头一般倒下去了。

段-璋这一惊非同小可,但他是武学大行家,虽惊不乱,在这一瞬之间,他已知道是有人偷发暗器,宝剑亦已出鞘,脚尖一点,舞起一道剑光,护着身躯,便向那人追去。

就在这时,只听得夏凌霜也在高声叫骂,追了过来,那人倏地回头,望着夏凌霜叫了一声,似笑非笑,听起来凄凉之极,段圭湾也就在那个时候看清楚了那人的面貌,不是皇甫嵩是谁?

段-璋气怒交加,趁着皇甫嵩一怔之际,立即一剑向他刺去!

皇甫嵩横拐一迎,只听到``卡嚓''一声,皇甫嵩的拐杖给砍了一个缺口,但段-
璋也给震得虎口酸麻,禁不住连退几步,才稳了身形。说时迟,那时快,皇甫嵩早已飞身斜掠,穿入林中。

车迟倒地之后,只发出一声惨叫,便再也没有声息。段-璋放心不下,只好暂缓追敌,先回来救人。

但夏凌霜却不听呼唤,追了下去。窦线娘怕她有失,提起弹弓,随后追来,给她惊阵。

段-璋接了一招,试出皇甫嵩功力虽高,却也不如所传说之甚,心想以妻子的神弹绝技,加上夏凌霜精妙的剑术,纵使皇甫嵩反啮,她们二人也不致落败,便任凭她们追去。

段-璋弯下腰来,察看车迟的伤势,只见他面目瘀黑,嘴角沁出血丝,有一股难闻的腥臭的味道,段-璋大吃一惊,情知是凶多吉少,伸手一探,果然气息毫无,早已死了!

段-璋悲愤交集,呆了半晌,哭道:``车老前辈,你还说凶手不是他,如今你的性命也送在他的手下了。''事情非常明显,皇甫嵩早已埋伏在旁,怕车迟说出凶手的名字,所以用喂有剧毒的暗器,要把他们二人杀害,结果车迟舍命相护,牺牲了自己,却保全了段-璋。

若然他不是凶手,无须用这样狠毒的手段,但令段-
璋不解的是:车迟又为什么说凶手不是他?再者,车迟在中了暗器之后,还能叫喊,以他的功力,最少可以支持片到,在这样关键的时刻,他为什么不肯说出当年那件血案的凶手名字?若然那凶手就是皇甫嵩的话,难道车迟受了他的暗害,至死都要庇护他吗?

这种种疑团都令段-璋百思不得其解,可惜已不能将车迟起于地下而问之了。

段-璋伤痛稍过,定了一下心神,找到在皇甫嵩拐杖上削下的那片水头,木头有一股紫檀香味,段硅章藏了起来,心中想道:``皇甫嵩的拐杖是海南紫檀香木所制,武林前辈无不知道,我要将这片木头作为他行凶的证物,请几位正直的老前辈来给车迟报仇!''

过了一会,窦线娘与夏凌霜空手而回,窦线娘道:``林深树密,给那老贼跑了。啊呀!车老前辈怎么了?''段-璋道:``他已不幸去世了,咱们将他埋葬了吧。''窦线娘叫道:``怎的死得这么快?''她是便暗器的能手,上前一看,失声叫道:``这是见血封喉的毒针,皇甫嵩怎的会使这种歹毒的暗器?''

当时武林的风尚,讲究真才实学,第一流的高手,极少用喂毒的暗器,所以窦线娘发现了车迟中的是见血封喉的毒针,便觉得十分奇怪。

段-璋道:``对了,我刚才还未想到这一层,皇甫嵩是从来不用暗器的,更不要说这样喂有剧毒的暗器了,难道,难道\ldots\ldots{}''

窦线娘已知道她丈夫想说的是什么,摇摇头道:``但是刚才那个人却分明是皇甫嵩,还会是假的么?''

夏凌霜道:``我母亲说,这皇甫嵩奸恶无比,依我看来,他平时不用暗器,乃是故意自高身份,现在到了事急之时,便不择手段,连最歹毒的暗器也使用出来了。''段-璋虽然从她的语气中感到她对皇甫篙的成见太深,但那个人是皇甫嵩却是不容置辩的事实屈此也只有接受她这个解释。

段-璋道:``贤侄女,我问你一件事情,那日在骊山北面的那座土地庙中,听说你与皇甫嵩遭遇,要拔剑杀他,他端坐地上,任凭你杀,这可是真的?''

夏凌霜道:``不错,是有此事。所以当时南大侠也给他骗过,以为他是好人,因此将我拦住。现在看来,当时他的这番举动,十九是矫情做作,明知南大侠会拦阻我的。''

段-璋颇觉怀疑,沉吟说道:``当时我昏迷未醒,是他给我退了追兵,又将我救活的,这也是干真万确的事呀。现在真是连我也给弄得糊涂了,当时何以对我这样好,现在却又要暗杀我呢?''

窦线娘道:``大哥,你总是往好的方面着想。这有什么奇怪?你不是也曾说过,他当时救你,是为了向你市恩,好与你化敌为友么?现在他已知道这冤仇无法可解,又怕车迟说出真相,你已知道内清,所以当然要向你下毒手了。''

夏凌霜早已忍耐不住,听窦线娘提到,便急忙问道:``那老叫化到底对你说些什么话?''

段-璋讷讷说道:``他、他还是那一句话,说皇甫嵩不是你们的仇人。但到了最紧要的关头,他刚要说出你们仇人的真正名字时,便给皇甫嵩害死了!''

夏凌霜低声问道:``这且不必管它,我母亲本来就只是想为江湖除害,并非我们与皇甫嵩有过不去的冤仇。我要问的是、是:那老叫化可有说到与我身世相关的事。''

段-璋颇觉尴尬,半晌说道:``也还未曾谈到。不过,不过,我相信他以前对你说的,大约,大约也非全是胡说。''

夏凌霜变了面色,蹩了双眉,她心头上本来就罩有一层阴影,现在是更扩大了。她可以不相信车迟的话,但却不能不相信段-
璋的说话,她低下头来,喃喃自语道:``难道妈妈有些事情还要瞒我不成?''想了半晌,忽地又抬起头来问段-璋道:``段伯伯。你是我父亲生前的好友,你可以告诉我吗?''

但是段-
璋心里的怀疑却不便说出口,想了一想,说道:``你父亲遇害的那晚之后,我就再也没见过你的母亲。不过,据我所知,那皇甫嵩大约是你母亲的仇人,你母亲要你杀他,不单是为了给江湖除害,同时也是为自己报仇。''

夏凌霜是个聪明的女孩子,一听就知道段珠漳言犹未尽,不过,从他所透露的口风,已经可以猜想得到:自己的身世一定还有更复杂的内情。当下咬着嘴唇说道:``好,段伯伯你不肯说,我只有自个儿回家问妈妈去。''

段-璋柔声说道:``不是我不肯说,是我有许多事情还未曾弄得明白。只怕也要见了你的母亲之后,才能弄得清楚。''

窦线娘道:``我与你的母亲未曾见过面,但亦是久已仰慕地了。不知可以容我拜访她么?''

夏凌霜道:``段婶婶肯光临寒舍,我自是欢迎不暇,只是我不能作主,待我问过家母再来寻找如何?我妈的脾气有点古怪,她不愿意见外人。''有一点她还瞒着不肯说出来的是:她母亲曾郑重交代她,连住址也不要透露给段-
璋知道。

夏凌霜又道:``南大侠已经到睢阳去了,据我所知,他是要将王伯通父子与安禄山密谋作反之事告诉张巡与郭子仪的。他是准备到睢阳一转便回九原,他要我告诉你,问你愿不愿到九原会他?''

段-璋趁此下台,说道:``我正是要到九原去。你见过母亲之后,若是有事找我,可以到九原来。''

当下三人以刀剑挖土,草草的埋葬了车迟,段-璋目睹这一代丐侠埋骨荒山,心中无限伤感。

埋葬车迟之后,三人联袂下山,大家的心情都很沉重,窦线娘叹气道:``这几个月来,一件件的不如意事接踵而来,弄到如今家破人亡,真似是做着恶梦一般!''段-璋无言可慰,强笑说道:``也许是因为咱们已享了十年清福,所以天公有意要将咱们多所折磨!''

夏凌霜招回了她的小白马,一声``珍重!''跨上坐骑,挥泪而别。这一去也,正是:

狼烟遍地乱神州,重逢已是沧桑改。

欲知后事如何?请看下回分解------

\chapter{第十六回 强藩作乱囚朝使
侠士重来陷敌围}\label{ux7b2cux5341ux516dux56de-ux5f3aux85e9ux4f5cux4e71ux56daux671dux4f7f-ux4fa0ux58ebux91cdux6765ux9677ux654cux56f4}

岁月如流,星移物换,自王家父子大破飞虎山之后,转眼间便过了七年。

这七年来的变化很大,就江湖上来说,王家兴起,已替代了昔日窦家的位置。虽因龙眠谷那一闹,引致了绿林的大分裂,王伯通终于没有达到做绿林盟主的目的,但依附他的党羽也很多,在绿林中仍以他的势力最大。当年威震绿林的``窦家五虎'',已渐渐给人忘记了。

就朝廷来说,朝廷的势力日益衰微,安禄山的势力却日益扩大,他掌领范阳、平卢、河东三镇,等于在北方自成一国,与李唐政权分庭抗礼,兵精粮足,甚至还盖过了朝廷。

大唐天宝十四年九月的一天,范阳平原上有一骑健马正在飞驰,马上的骑士是一个熊腰虎背的壮健军官,此人来历非比寻常,他是大唐开国功臣秦琼之后,现封龙骑都尉,名列大内三大高手之一的秦襄。

他是奉朝廷之命,随中使冯神威,前往范阳去安抚安禄山。现在却偷偷从范阳出走,要赶回京都,向皇帝报告安禄山辖区的消息的。

本来早在七年之前,郭子仪已有密奏呈给玄宗皇帝,报告安禄山收买绿林,招兵买马,密谋造反之事。怎奈玄宗皇帝对安禄山宠信方殷,且有杨贵妃在旁替他说话,因此玄宗皇帝竟把郭子仪的奏章搁置不理,造成了安禄山的尾大不掉之势。

安禄山当时一来因为准备未曾充分,二来因为利用王伯通收买绿林的计划受了阻挠,三来因为郭子仪有密奏上朝的风声传出,安禄山也不能不有所戒惧,因此他仍然要作出赤胆忠心的模样,来哄骗玄宗皇帝,年复一年,迟迟未敢动手。

到了这一年,他自忖兵多将广,已是胜算可端,便生出一个事端,来撩拨朝廷。假借``献马''为名,上疏奏道:

``臣安禄山承乏边庭,所属地方,多产良马。臣今选得上等骏骑三千余匹,愿以贡献朝廷,臣虽不如昔日王毛仲之牧马番庶,然以此上充无厩,他年或大驾东封西讨,亦足以壮万乘观瞻。计每马一匹,用执鞍军二人,臣更遣番将二十四员部送,俟择吉日,即便起行。伏乞敕下经历地方,各该官吏预备军粮马草供应,庶不致临期缺误,谨先以表奏闻。''

此疏一上,玄宗虽然宠信安禄山,却也不免起了疑心,试想每匹马有两个``执鞍军'',三千匹便有六千人,另外有二十四员番将护送,每员番将又有跟随的军士,合计当有万人,若任它开人长安,岂能无虑?

玄宗与朝臣商议,朝臣都说安禄山居心叵测,不可轻信,若任其以精兵万人,开来京师,祸患不堪设想,请玄宗降严旨切责,破其狡谋。玄宗还不敢相信安禄山怀有异心,又怕降旨严责,反而迫反了他。后来有一个老成持重的大臣达奚玩献议玄宗以温言谕止禄山献马。玄宗如拟,遂造中使冯神威,携手诏往谕,谕云:

``览卿表献马于朝廷,具见忠悃,朕甚喜悦。但马行须冬日为便,今方秋初,正因稻将成,农秀未毕之时,且勿行动。俟至冬日,官自给夫部送来京,无烦本军跋涉之劳,特此谕知。''

冯神威受了诏书,由秦襄带领亲军护送,来至范阳。安禄山早有在长安的密探报知,十分恼怒,及闻诏到,竟不出迎。冯神威开诏宣读之时,安禄山也不跪拜接旨,却自高踞胡床,嘿嘿冷笑,听他读毕之后,便怒容满面地说道:``传闻贵妃近日于宫中,也学乘马,我意官家必爱马,我这里最有好马,故欲进献几匹。今诏书既如此,不献也罢。''冯神威见阶下陈列甲兵,不敢与他争论,只有唯唯而已。

安禄山将他们留下,对他们十分冷淡。过了几日,冯神威欲还京复命,请见安禄山,问他可有回奏表文,安禄山道:``诏书云:马行须俟冬日,至十月间,我即不献马,亦将亲诣京师,以现朝廷近政,何必复文?连你也不必急于回去,待到十月,再与我一同走罢!''

冯神威见此情形,已知安禄山必反,当下不敢多言,回到客栈之后,便密令秦襄火速回京,奏知皇上,早作准备。秦襄本领非凡,安禄山派来监视的武士拦阻不住,被他星夜逃出范阳。

秦襄心急如焚,披星戴月,催马疾驰,第二日中午时分,已离范阳城一百余里,他胯下的黄骡马是匹骏马,但亦已疲乏不堪,口吐白沫了。

秦襄正要找一处水草丰饶之处,让马儿稍歇,忽听得一声呐喊,在山脚下出来了一彪人马,齐声喝道:``此山是我开,此树是我栽,若然要经过,留下路钱来!''

秦襄大怒道:``你秦爷爷是强盗的祖宗,你等无知小丑,竟敢拦途截劫!''提起两柄金装锏,冲人贼兵阵中,挥锏便打。他这两柄金装锏乃是家传兵器,每柄重达六十四斤,当年他的祖父秦叔宝(琼)仗着这两柄金锏,曾住李世民扫平十八路烟尘。秦襄武艺不逊乃祖当年,双锏使开,登时打得贼兵狼号鬼哭!

蓦地里从贼兵中冲出两骑健马,两个长得一般相貌的中年汉子,一个使左手刀,一个使右手刀,向秦襄夹击,马来如风,刀光着电,倏然间合成了一道银虹,双刀合壁的招数凌厉之极!

秦襄心中一凛:``这不是普通的强盗!''但他武艺高强,却也傲然不惧,当下大喝一声:``来得好!''双锏霍地一分,使出秦家的``杀手锏''绝招,马不停蹄,双锏两边横磕!

来者正是王伯通麾下的``阴阳刀''石家兄弟,这两人的双刀虽然配合得非常纯熟,却怎挡得秦裹的神力,且马上的功夫也不如他,但听得咣咣两声,石一龙的单刀脱手飞出,石一虎更是不济,给他一锏打落马下。

就在此时,只听得弓弦声响,一支响箭射来,绿林规矩,用响箭乃是要对方止步的讯号,但在正式交锋之际,用响箭就是含有蔑视之意了。秦襄大怒,举锏拨落,只觉这一箭的劲道大是不凡。

说时迟,那时快,这骑马已到了他的面前,马上的骑士眉清目秀,却是个英俊的少年。此人正是王伯通的儿子王龙客。

王龙客长于点穴,他平时用的兵器是一把铁扇子,但因马上交锋,用短兵器不便,故此改用了一双特制的判官笔,一般的判官笔最长二尺八寸,他这对判官笔却长四尺有余。

王龙客飞马赶到,侧目斜睨,慢声说道:``官军中有阁下这等人物,也算是很难得了。阁下何苦为官家卖命。不如随我去做个山大王,大秤分金,小秤分银,岂不更乐得个逍遥快活!''

秦襄喝道:``小贼放屁!''金装锏以泰山压顶之势,劈头便打!王龙客在绿林中以``狠''著名,但见他如此威势,却也不敢硬接,当下施展精妙的骑术,一个``金鲤穿波'',双足勾着马鞍,钻到了马腹痛下。

秦襄双锏扫了个空,他急于赶路,无暇再取敌人性命,双足一挟,便催马疾驰。

哪知他刚刚拨转马头,尚未驰出一箭之地,猛听得``呼''的一声,只见那黄衣少年已在马背上跳起,竞然施展了``一鹤冲天''的上乘轻功,跳过他这匹马来。他凭着这俯冲的力道,抵消了秦襄的神力,双笔往下一按,秦襄挥出一锏,竟然未能将它磕飞,就在这一瞬之间,他已落到了秦襄的马上!

秦襄的金装锏每柄重达六十四斤,在马上与敌交锋,那是威力极大,近身肉搏,却不如轻兵器的灵活。王龙客落到他的马上挥笔便挑秦襄的穴道,秦襄侧身一避,``嚓''的一声,王龙客的判官笔已戳中了他的前胸,幸而他是披着软甲,又未曾点正穴道,但饶是如此,战袍亦已给笔尖戳破!

秦襄大怒,将金锏在马鞍上一搁,蓦地大喝一声:``滚开!''一伸手将王龙客的腰带抓着,将他提了起来。王龙客做梦也想不到秦襄竟敢搁下兵器,用此险招,他双笔本来要点秦襄左右``肩井穴''的,笔尖刚刚沾上,已给泰襄抓着。秦襄天生神力,有伏牛扛鼎之能,王龙客给他一把抓着,痛彻心肺,气力休想使得出来,双臂软绵绵的垂下,笔尖虽然已点到了秦襄的肩井穴,那已是一点功效也没有了。

石氏兄弟大惊,急忙催马过来救人,但见在王龙客尖叫声中,秦襄像捉着一只小鸡似的,将他提了起来,旋风一舞,喝道:``杀你这样的小贼,污我的手!''把王龙客直抛出去!秦襄那匹黄骠马久经战阵,虽然走了长途,已经疲之,但碰上了危险,却突然奋发起来,振足长嘶,将赋兵冲开,势如奔雷逐电!后面嗖嗖连声,箭如雨下,秦襄喝道:``来而不往非礼也!''放下金锏,接过了两枝冷箭,甩手射回,他以手发箭,比用弓弦的力道还要强劲,两枝箭都射个正着,登时将追到后面的两个小头目毙于箭下!其他喽兵发一声喊,勒马不敢向前。

那王龙客也真了得,在半空中一个鹞子翻身,平平稳稳地落到地上,冷笑道:``姓秦的,行你走得多远?孩儿们,暂且不必理他!''秦襄只当他显虚声恫吓,心道:``若不是赶着回京报讯,我倒要理理他们。''他快马疾驰,一口气跑了十多二十里,那匹黄骠马似乎知道已经脱险,慢了下来,累得直喘气。秦襄抚拍马颈,道:``马儿,今天亏得你了!''这时,他心中已在起疑:``我又不是押解差响的军官,这班强盗劫我作甚?呀,是了!久已风闻安禄山勾结绿林,莫非这些强盗竟是他的人?''

心念未已,忽地听得一个娇滴滴的声音叫道:``秦大人,你纵不累,马也累了,下来歇歇吧!''

只见一个容光艳丽的少女,突然从前面的林子里现出身来,长裙曳地,衣袂飘飘,步履轻盈,转眼间便来到了大路当中。她的后面,跟着一队女兵,大约有十来个人,打着一面旗号,锦旗上只有一只用金丝线绣成的燕子。这队女兵一字摆开,拦住了秦襄的去路。

秦襄愕了一愕,问道:``你们是干什么的?难道你们这些姑娘们,也是干没本钱的黑道营生么?''为首这个少女实在长得太美了,秦襄虽然知道她的来意不善,却不敢相信她竟是强盗。

那少女笑盈盈地说道:``秦大人你也忒小觑我们了,难道没本钱的生意,只有你们男子才干得了么?不过,你也不用担忧害怕,我不要你的性命,只想请你到我的山寨里去住几天。你一路奔波,也应该歇歇了。''

秦襄道:``我没有工夫与你们胡闹,快快让路。''一个女兵笑道:``你好大的面子,我们的姑娘才请你作客,你却怎的不知好歹,反而骂我们胡闹。''

秦襄实在不愿与一班女孩儿家动手,忍住了气道:``素不相识,盛情心领了。我有要事,非得赶路不可!''

那少女忽地冷笑道:``秦大人,你这么说,那是敬酒不吃要吃罚酒了。你可知道我们绿林中的规矩么?''秦襄双眼一睁,道:``怎么?''那少女道:``你不愿意做我们的客人,那我们只有把你当作羊枯看待了,拿过见面礼来!''

秦襄又怒又气,哈哈笑道:``你们也学人打劫?你可知道我刚才就从强盗堆中杀了过来?我这双锏一个打无名小卒,二不打女流之辈,我劝你们还是好生散去吧!''

那少女一声不响,从女兵手里接过一把弓箭,``嗖''的一箭就向秦襄的坐骑射来,秦襄挥锏一拨,禁不住心中一凛,这枝箭劲道之强,竞是出乎他意料之外!拨是拨落了,但这支箭余势未衰,贴着马足擦过,那匹黄骠马登时跳了起来。

秦襄怕他心爱的战马受伤,跳下马背,拍拍它道:``马儿,马儿,你在前面等着我吧。''

这匹马久经训练,振起四蹄,就向旁边的小路奔去,哪知那队女兵行动快板,陡然间伸出四柄长长的挽钩,一下子就将他的这匹黄骠马勾倒,接着就有人用鲜马索将它套住,硬生生地拉了过去!

那少女笑道:``这是一匹宝马,好生给它治伤,不可坏了。''顿了一顿,又格格笑道:``秦大人,你这匹马虽然不错,但还不够。你这两枚锏金光灿烂,沉甸甸的,敢情真是用赤金打的,怕有百来斤吧?这倒值不少银子。这样吧,再搭上这双金锏,算是我已收足了你的见面礼,便放你过去!''

秦襄禁不住怒道:``你一再胡缠,我可要不客气啦!''那少女笑道:``你现在可愿意跟我们女流之辈打了吧?好呀,只要你赢得了我手中的这把剑,我就不收你的见面礼放你过去,那匹马也还给你!''秦襄双锏一挥,``蓬''的一声,将路旁一棵树齐腰打断,说道:``姑娘,你看清楚了,我这双锏可是不好惹的,你当真要跟我单打独斗么?''那少女道:``看清楚了。树是死的,人是活的,我就不信你这双锏伤得了我。你可知道,我这把剑也是不好惹的么?''

秦襄无可奈何,说道:``好,你既口出大言,那就来吧!''

那少女慢条斯理地束紧腰身,忽地剑柄一翻,喝声:``接招''陡然间便是反手一剑,迳削秦襄手腕。

秦襄已看出了这少女武艺不凡,但却料想她不是自己的敌手,心里存在几分爱惜之念,还真怕失手打伤了她。当下双锏封出,用了一招``横架金梁'',仅仅使出了三成气力。

哪知这少女的剑招虚虚实实,奇诡非常,剑尖在金锏上一点,忽地反弹起来,一剑就刺到他胸口的``璇玑穴''。

秦襄这一惊非同小可,幸他久经阵仗,身形一仰,使出``铁板桥''的功夫,腰向后弯,只听得``唰''的一声,少女的剑在他面门掠过!

好个秦襄,趁着那少女未及换招,腰身一托,双锏便以泰山压顶之势直打下来,但他仍然不想打死这个女子,双锏是照着她的长剑压下,只想把她的兵器打出手去。

那少女叫声:``好厉害!''蓦地一个斜身滑步,使一个``卸''字决,剑脊贴着金锏,随着她这斜窜之势,将秦襄的一柄金锏引开。秦襄右手金锏磕下,打了个空,双锏失了平衡,竟然身不由己的跟着她奔出几步。

那少女一摆脱开双锏,立即便回剑还攻,秦襄见她剑法精奇,而且还居然能使用上乘的内家功夫,这时,哪里还敢再有半点轻视?秦襄双臂一振,抡起双锏,登时金光大炽,呼呼轰轰,真有排山倒海之势,风雷夹击之威!那少女格格笑道:``秦大人,你这双锏不是专打英雄好汉的么?今日蒙你以家传绝技赐教,小女子真是感到荣宠无比啦!''她一面出言挖苦,手底却是毫不放松,她的剑法走的是轻灵翔动的路子,移步变招,挥洒自如,端的是恍若行云流水,秦襄给她讥刺,面上一红,那少女指东打西,唰的一剑从他胁下穿过,险险刺中了他的愈气穴。

秦襄怒道:``好狡桧的女贼!''一招``横云断峰'',双锏平推出去,这时他已收起了怜香惜玉之心,使出了他秦家的``杀手锏'',锏影如山,每一锏都足以开碑裂石!那少女不敢硬接,一沾即退,仗着轻灵的剑法,和秦襄游斗。

秦襄双锏大开大阖,强攻猛打,一口气抢攻了数十招,可是那少女身轻如叶,她那柄剑柔如柳絮,随着锏风,飘飘晃晃,秦襄的力道虽有金刚猛扑之威,却竟然无法打脱她的兵刃。

但是秦襄用了全力,那少女却也无法再欺近他的身前。本来她这套剑法,若是到了上乘境界,足可以柔制刚,但她功力未到,秦襄神力惊人,以她现在的功力,最多只能卸开他的三成力道。因此打定了主意,想在游斗之中,等待秦襄气衰力竭。

秦襄昨夜逃出范阳,奔波百余里,先后经过了两场恶斗,纵是铁铸的身躯,也感到有些疲累了。斗到百招之后,渐渐便有点力不从心,但那少女仍然未能反守为攻。

双方正自斗到紧处,只听得后面马铃叮咣,蹄声有如潮涌,秦襄回头一看,不由得叫声:``苦也!''原来刚才给他打败的那股强盗,现在又追到来了。

王龙客跳下马背,哈哈笑道:``姓秦的,我说你逃不了,这可没有说错吧!''双笔一挺,叫道:``燕妹,这又不是比武较技,你和他多耗时候做什么?咄,你们的挠钩作什么用的,还不上前助小姐将他擒了?''

这少女正是王龙客的妹妹王燕羽,她的这队女兵,因为未得小姐吩咐,不敢上前拿人,现在给少寨主一喝,当然一拥而前,十几柄长钩,都向秦裹的双足勾去。那王龙客提起双笔,也加人了战团。这队女兵久经训练,场中人影翻腾,她们的长钩却跟定了秦襄,丝毫不乱。

秦襄大喝一声,一个``进步鸳鸯连环腿''双脚齐起,将两柄挠钩踢得飞上半空,可是第三柄挠钩却在他的脚肚上勾了一下,幸而那女兵力弱,又给秦襄的威风吓得慌了,只是勾去了一小片皮肉,随即便给秦襄一锏将她的挠钩打折。

秦襄虽勇,无奈气力不加,已是到了强弩之末,抵挡王燕羽兄妹的联手进攻,已经育点应付为难,何况还有那班挠钩手在旁窥伺,乘瑕抵隙。王龙客一笔点中,``嗤''的一声,戳破了他的衣裳,幸在他身披软甲,胸膛一挺,登时将王龙客的判官笔反弹出去,王龙客虎口受震,吃了一惊。说时迟,那时快,秦襄一锏便劈下来,他早已看出了这对兄妹,妹妹的武功要比哥哥强得多,意欲一锏先把武功较弱的王龙客打翻,便即突围而出。

哪知他的``杀手锏''虽然厉害,但因用了全力去攻击王龙客,防御方面便露出了破绽,王燕羽一见有机可乘,青钢剑疾如电闪,倏的就刺中了他的左臂,她力透剑尖,这一剑竟把秦袭的软甲都刺穿了,登时血流如注!

秦襄大吼一声,那一锏打下,已经歪过一旁,王龙客霍地一个``凤点头''避过,双笔齐挥,戳中了秦襄的肩头,秦襄虽有软甲护肩,但戳中的地方正是肩井穴所在,登时一条臂膊酸麻,发不出力。

王龙客哈哈笑道:``姓秦的,你死在眼前,还逞什么强?扔下这双锏向我磕三个响头罢,或者我还可以饶你。''王龙客刚才在部属面前,给他摔了一个筋斗,恨之刺骨,因此如今占了上风,便要将他尽情凌辱。

秦襄大怒,``呸''的一声,有如舌上绽了一个焦雷,喝道:``我虎落平阳,还是猛虎!你这狗贼,敢来欺我!''呼、呼、呼,连打三见,他气力虽不如前,但须眉怒张,神威凛凛,更为吓人!王龙客在绿林中本以凶狠著名,被他这么一喝,竟也禁不住心中打抖,不知不觉的问后连连退步。

王燕羽道:``这厮已是困兽之斗,哥哥,你何须与他拼命。''王龙客定下神来,说道:``不错,待他筋疲力竭,然后慢慢宰他!''两兄妹展开了游身缠斗的方法,加上钩手之助,竟把秦襄困在核心。秦襄的轻功比不上他们兄妹,一手一足又已受伤,登时险象环生,血染袍甲!

激战中忽听得蹄声得得,来势甚急,秦襄只当是盗徒同党,此时此际,多一个少一个已不放在他的心上,但那班强盗却纷纷呼喝起来!

只见一个少年骑士疾驰而来,大声喝道:``王家贼子,还认得我么?''马未停蹄,已是把手一扬,一支匕首,破空飞来,``咔嚓''一声,将那面飞燕旗从旗杆当中削为两段。

号旗被倒,这是绿林中最犯忌的事情,王燕羽大怒骂道:``岂有此理,你吃了狼心豹胆。胆敢在太岁头上动土!''说时迟,那时快,阴阳刀石家兄弟早已迎了上去,那少年飞身下马,傲然喝道:``滚开,唤正主儿上来!''石家兄弟欺他年轻,冷冷说道:``你过得了我们这两柄刀,再吹大气,也还不迟!''他们两个,一个使左手刀,一个使右手刀,口中说话,双刀已然攻出,使的是同一招数,截腰斩肋,但方向不同,一个攻他左半边身子,一个攻他右半边身子;只要双刀一合,就能把敌人齐腰斩断!

这本来是``阴阳刀''的一招极厉害的杀手,败在他们两兄弟这一招之下的绿林好汉不知多少。哪知话声未了,那少年唰唰两剑,出手比他们兄弟更快,双刀未合,已给他的长剑当中挑开,石一龙吃了一惊,猛地叫道:``你,你是铁、铁少寨主回来了?''那少年道:``不错,你这两个自甘下流的强盗,还在做王家的鹰犬么?''他口中说话,手底也是毫不放松,以脚跟支地,打了一个圆圈,那口长剑竞似从四面八方攻到,饶是石家兄弟见多识广,也未曾见过这样古怪的剑法,顿然间两兄弟双双中剑,连忙退下。

王燕羽赶了到来,定睛一瞧,喝道:``我道是谁?原来是铁摩勒!你不念昔日不杀之恩,还来毁我的旗号,是何道理?''

一别七年,铁摩勒已长成了一个器宇轩昂的年少英雄,王燕羽心道:``这黑小子倒是越来越漂亮了。''

铁摩勒骂道:``我与你仇深如海,岂止要倒你的旗号,哼,哼,------''王燕羽笑道:``你还要怎样?可是还要取我项上的人头么?''铁摩勒双眼一瞪,喝道:``不错!''立即使出一招``李广射石'',迳取她的心胸!

王燕羽笑道:``冤仇宜解不宜结,你又何必这样发横?''横剑一封,咣、咣两声,震得她双臂发麻,王燕羽心头一震,始知铁摩勒已是今非昔比,剑法如何,且自不说,这份功力。已经是胜过了自己了。当下不敢怠慢,与他认真斗起剑来。

秦襄去了一个强敌,虽有其他头目迅即补上,协助王龙客围攻,却怎故得住秦襄的神力,不过几个照面,秦裹一声大吼,手起锏落,便把一个头目打得头颅粉碎!

王龙客心胆皆寒,想不到他在久战之后,居然还是这般凶猛,说时迟,那时快,秦襄虎目圆睁,再一碱便朝着王龙客打去。王龙客不敢接招,侧身一闪,秦襄冲出重围,叫道:``壮士走罢!''

铁摩勒道:``你走你的,我要杀尽这班强盗再走!''

铁摩勒不肯走,秦襄本该与他合力作战,但无奈他已是伤得甚重,只有一条臂膊可以使用,久战下去,决无幸理,再想到军情紧急,不容他为了武林义气以致误了国家大事,当下只好舍了铁摩勒而去。

强盗们人呼小喝,作势堵截。王龙客撮唇一啸,唤自己那匹坐骑过来。他还待上马追赶。

秦襄笑道:``来得正!''一纵身,拦住王龙客那匹坐骑,收了金建,单臂一按,将那匹马按得四蹄伏地。秦襄跨上马背,那匹马却不肯走,秦襄道:``好呀,你敢不服我么?''反手一抓,登时在马臀上抓得鲜血淋洒,那匹马负痛狂嘶,不由得它不振蹄疾走。秦襄在马背上扬声问道:``请问英雄高姓大名?''铁摩勒应道:``飞虎山铁摩勒。''秦襄道:``我是龙骑都尉秦襄,铁少英雄救命之恩,日后自当图报!''策马直冲出去。

铁摩勒并不知道秦襄乃是秦叔宝的后人,心里暗笑:``想不到我在无意之中竟救了一个朝廷的军官。''毫不放在心上,一边答话,剑招却是越催越紧。

那班强盗仍在作势呼喝,王龙客道:``不必理这个狗官了,捉这个小贼更紧要。''其实他是怕了秦襄,不敢追他。只因当着部下面前,只好如此说法。不过,他说的也的确是心里的活。要知秦襄虽然关系重大,但铁摩勒与他王家有血海深仇,斩草未曾除根,更是心腹之患!

七年前铁摩勒随南霁云到了睢阳,便拜在磨镜老人门下,做了磨镜老人的第三个弟子。这七年来,他随着磨镜老人,学了一身本领,段-璋送他那本剑谱,他也已学得滚瓜烂熟,并在磨镜老人指点之下,悟出了许多新奇的变化。现在因为烽烟将起,他准备到九原去会见师兄,助郭子仪一臂之力。想不到在这里遇见了王家兄妹。

他只道凭着自己七年的苦学,足可以尽歼仇敌,哪知在这七年中,王燕羽的武功也是与日俱增,如今正式交手,他虽然稍占上风,可是斗了五六十招,王燕羽也还未有败象。

激战中铁摩勒使了一招``独劈华山'',竟把长剑当作大刀来使,高高举起。一剑劈下,这一招是他从段-
璋的飞龙剑法中变化出来的,有剑法的轻灵,又有刀法的雄浑,看似平平常常、却是极难抵挡,长剑一起,登时把王燕羽全身都笼罩在剑光之下。王燕羽叫道:``好狠的剑法!''闪避不开,只好横剑招架,双剑相交,咣的一声,纠作一团,竟似在半空中胶着了。

王燕羽究竟气力较弱,她的青钢剑给铁摩勒的长剑压着,震得虎口发麻,却又摆脱不开,剑身渐渐向后弯曲。

王龙客喝道:``小贼体得逞强,看扇!''拆铁扇一挥,疾点铁库勒背后的``风府穴''。这一下,铁摩勒变成了背腹受敌,不得不先解敌招,当下将剑移开,反手一招``犀牛望月'',将王龙客的折铁扇荡开。王燕羽身手何等快捷,压力一松,立却挥剑向他攻去,只听得``唰''的一声,剑尖几乎贴着铁摩勒的额角刺过。铁摩勒一矮身躯,打了一个盘旋,用了个``夜战八方''的招式,将青钢剑和折铁扇一齐迫住。

王燕羽娇声笑道:``七年不见,想不到你的剑法竟是如此高明了,当真是可喜可贺哪!对不起,我们只好兄妹二人合战你了。''铁摩勒喝道:``你们就是全部上来,我又何惧?今日不是你死,便是我亡!''

王燕羽笑道:``哥哥,这小子当真是要和咱们拼命了!''王龙客道:``那就教他早点去见阎王!''折铁扇指东打西,指南打北,招招都是指向铁摩勒的三十六道大穴。

铁摩勒虽说不惧,但那形势已是立即扭转过来。要知王龙客的武功本来不弱,他刚才与秦襄相斗,似是不堪一击,那是因为秦襄天生神力,锏重力沉,他的判官笔根本不敢与秦襄的金锏相碰的缘故。如今和铁摩勒相比,武艺虽尚不如,功力却不相上下,而且他现在改用了熟手的折铁肩,利于近身搏斗,两兄妹联起手来,当然要胜过铁摩勒了。

铁摩勒觉出不妙,心道:``段大侠与南师兄屡次告诫我不可少年气盛,自恃本领,我只道学成之后。便可立即报仇,哪知又是犯了轻敌的毛病。我已忍了七年。不争在这一日,今日敌众我寡,还是且待他日吧。''

王龙客对敌的经验其丰,见铁摩勒神情焦躁,挥剑强攻,实是走势,立即笑道:``天堂有路你不走,地狱无门你偏进来,你既自投罗网,只怕是来得去不得了!''一声吆喝,那队女兵又一齐挥动挠钩,来勾铁摩勒的双足。两兄妹一剑一扇,更是紧紧将他缠住。

正是:技成无奈沧桑改,欲报深仇岂易言。

欲知铁摩勒能否脱险,请听下回分解------

\chapter{第十七回 难分爱很情惆怅
说到恩仇意惘然}\label{ux7b2cux5341ux4e03ux56de-ux96beux5206ux7231ux5f88ux60c5ux60c6ux6005-ux8bf4ux5230ux6069ux4ec7ux610fux60d8ux7136}

铁摩勒不比秦襄,他身上没有披甲,脚上穿的只是一对麻鞋,因此受到挠钩的威胁更大。王龙客挥扇急攻,蓦然间使出杀手,一招``毒蛇吐信'',疾点他的``志堂穴'',铁摩勒的长剑给王燕羽架住,这一招除了侧身闪避之外,别无他法。

那队女兵久经训练,铁摩勒的身形方动,她们的挠钩早已伸出,正是铁摩勒所闪避的方向,这一下等于送上去挨钩,铁摩勒的腿肚、足跟、脚背登时都受了伤,一片片的皮肉被挠钩撕去,血流如注!

王龙客一声狞笑,喝道:``看你还狠?''铁扇一合,猛的就向铁摩勒天灵盖打下,铁摩勒这时正是摇摇欲倒,哪里还能抵挡?这一扇若然打实,怕不脑浆进流。

就在这电光石火的刹那之间,王燕羽忽地横剑一封,咣的一声,将她哥哥的折铁扇格开,叫道:``杀不得!''

王龙客征了一怔,问道:``怎么杀不得?''王燕羽出手点了铁摩勒的穴道,唤过侍女,将他缚了,笑道:``哥哥,你真是聪明一世,糊涂一时,你试想想,这小贼学成了武艺归来,所图何事?''王龙客道:``那当然是要向咱们报仇,并且要抢回他的飞虎山了。''王燕羽道:``看呀!他一个人哪能干得这样大事?想那窦家,将近百年的基业,正如百足之虫,死而不僵,忠心于他家的旧部,不过是畏惧咱们的声势,又没人带头,所以不敢蠢动罢了。现在铁摩勒回来,定然早有布置,说不定他和他义父的旧部,都已联络好了,咱们怎可以不问问他的口供,就把他杀了?''

王龙客笑道:``对,到底是你的心思比我周密得多,我恼他这样凶横,一时气糊涂了。''顿了一顿,又沉吟道:``但这小贼倔强得很,只怕问不出他的口供。''王燕羽道:``带他回龙眠谷会慢慢折磨他,问不出也得试试。''王龙客道:``好,我依你便是。擒他去,让爹爹处置,也好叫他老人家欢喜。''

说话之间,只见前面尘头大起,一队骑兵疾驰而来,为首的军官远远就叫道:``是王少寨主吗?''

王龙容应道:``正是。啊,张统领,你亲自来啦!''原来这个军官,正是安禄山帐下的高手,现居骑兵统领之职的张忠志。

张忠志勒住坐骑,问道:``你们没有碰见秦襄么?''王龙客满面通红,讷讷说道:``给他走了。''

原来监视朝廷使者的武士,一发现秦襄逃走,便立即用飞鸽传书,通知王伯通派人拦截,王龙客兄妹正是奉命来捉秦襄的。

张忠志道:``去了多久?''王龙客道:``已去了多时了。''王燕羽道:``本来我已快要将他拿下,不料碰到了另一伙敌人,混战中被他乘机逃去。现在我们已累得人仰马翻,要赶也赶不上了。''言下之意,若要追捕,乃可自便,恕难相助。

张忠志甚不高兴,但一来王家并非安禄山的下属,安禄山造反还要借重于他。二来他深知秦襄武艺高强,在大内三大高手之中,又以他为首,自己去追,只有送死。因此只好自打圆场,说道:``反正我们安大帅已准备就绪,指日就要进取京师,也不怕他去报告军情。安大帅连日正在召见各方将士、各路英雄,王少寨主就和卑职同回范阳如何?''

王龙客踌躇未答,王燕羽已抢着说道:``这样正好,爹爹他不方便在范阳露面,哥哥。你就去吧。这个小贼,有我押解,你尽可放心。''

王龙客只好答允,叮嘱妹妹道:``如此,你一路小心了。这小贼,我恨他不过,要杀他等我回来再杀。''当下,两兄妹各率属下,分道扬镳,王龙客随张忠志往范阳,王燕羽押解铁摩勒回龙眠谷。

王燕羽吩咐女兵,将铁摩勒反缚马上,马背上加厚锦垫,又替他扎了伤口。铁摩勒已被点了穴道,不能动弹,也不能言语,只好任凭她们摆布。

这时已是日头过午,王燕羽怕铁摩勒受到颠簸,叫女兵策马缓缓而行,到了黄昏时分,才不过走了三四十里,离龙眠谷大约还有五十里左右,她手下的兵头目前来请问,要不要赶夜路,王燕羽笑道:``你不累我也累了。又没有什么紧要的事情,不过押解一个小贼罢了,何须赶路?''女兵们正是求之不得,当下就在草原上搭起三座帐幕。王燕羽和她的贴身侍女一座,其他女兵一座,铁摩勒独自一座,这都是依照王燕羽的命令的。

铁摩勒遍体鳞伤,独自躺在帐幕里又饿又痛,正自愤火中烧,忽见帐篷开处,王燕羽笑盈盈地走了进来,剔亮了帐中的红烛,笑道:``铁少寨主,还倔强吗?''伸手解开铁摩勒的穴道。铁摩勒沉声喝道:``你要杀便杀,我铁摩勒决不受辱!''

王燕羽笑道:``谁要杀你?谁要辱你?你真是狗咬吕洞宾,不识好人心。我是来给你治伤的!''正待替他解开绷带,铁摩勒突然横肱一撞,喝道:``去你的!我,我\ldots\ldots{}''骂声忽地中断,原来这一撞正撞中她的酥胸,铁摩勒不好意思,连忙缩手,也就骂不下去了。

铁摩勒在重伤之后,且又饿得已经发软了,这一撞,当然不能造成什么伤害,王燕羽呆了一呆,满面通红,骂道:``你是一头牛么?这么蛮不讲理!是牛也知道人家对它好是不好,哼,哼,哼,你,你,你,你这冤家!''一指戳他的额角!

铁摩勒道:``我不要你这猫哭老鼠的假慈悲,你就是给我治了伤,我也不领你的情。''虽然仍是在骂,口气已经缓和了许多,也不再挣扎、打人了。

王燕羽解开绷带,叹口气道:``你这不讲理的小蛮子,我本待不管你,你却伤得这样厉害!啊呀,呀!我,我是不忍见你受苦!''

她取出金疮药轻轻替铁摩勒敷上去,凡是绿林人物,金疮药是必备之物,王家的金疮药更是灵效无比,一敷上,铁摩勒顿觉遍体沁凉,痛苦大减。他是一个年轻的小伙子,有生以来,从来未与一个女子这样靠近过,王燕羽给他敷药,肌肤相接,气息相闻,铁摩勒纵想忍着呼吸,那一缕缕幽香,仍是透入他的鼻管之中,铁摩勒迷迷糊糊的,竟似觉得十分舒服。他猛地牙根一咬,心道:``铁摩勒呀铁摩勒,你是铁铮铮的男子汉,你怎可忘了杀义父之仇!''这一发劲,他身下的木板,登时格格作响。

王燕羽皱了皱眉,道:``好端端的怎么又发脾气了?摩勒,你为何这样恨我?''铁摩勒怒道:``你这是明知故问。哼,哼,我劝你还是把我杀了的好,要不然,我有三寸气在,定要报仇!''王燕羽道:``就算是我杀了你的义父,那也不是你生身之父啊,绿林中斫斫杀杀。还不是平常得很么?''铁摩勒大怒道:``你看得平常,我却是铭心刻骨,深记此仇!''

王燕羽笑道:``好,就算你要报仇,你也总得保重自己的身子呀。你饿了一整天了,是不是?不吃点东西,哪来的气力报仇?''

铁摩勒给她弄得啼笑皆非,只见一个丫鬟走了进来,端着一碗茶水,说道:``铁少寨主,你趁热喝了吧。''

铁摩勒道:``这是什么?''王燕羽笑道:``这是毒药,你敢不敢喝?''铁摩勒道:``我怕什么!''仰着脖子,一口气就喝下去,只觉入口甘凉,喝了之后,精神陡振,原来是一碗上好的参汤。

那丫鬟笑道:``小姐,你倒真会劝人吃药!''端了空碗退下。铁摩勒道:``你别得意,不管你施什么恩惠,我们之间的怨仇,总是无法消除!''

王燕羽道:``我本来不想辩解,但你这样仇恨我,我却也不得不说几句。大破飞虎山那年,我只是十四岁。我只知道你的义父是个恃强凌弱的绿林霸王,我父亲叫我杀他,我当时并不觉得这是一件错事。''其实她现在也不认为是做错了,不过,当着铁摩勒的面,这一句却没有说出来。

铁摩勒心中一动,想道:``不错,那时候她只是个还未很懂人事的小姑娘,罪魁祸首是她的父亲,是帮王伯通为恶的空空儿!''恨意稍稍减了两分,但一转念间,却又想道:``不管她当时懂事也好,不懂事也好,她总是亲手杀了我义父的仇人,我怎么可以原谅于她?''

王燕羽聪明之极,早已从他神色之中看出他心情的变化,笑说道:``铁少寨主,你现在好了点么?''铁摩勒受伤虽重,只是皮肉之伤,这时只是气力还未使得出来,精神已恢复了四五分了。他心里也多少有点感激,口头仍是很强硬地说道:``好与不好,与你何干?我不要你献假殷勤!''

王燕羽噗嗤笑道:``谁向你献殷勤啊?你以为我想留你这臭小子当宝贝么?你知我问你这话是什么意思?''铁摩勒怔了一怔、重复她的话道:``什么意思?''

王燕羽笑道:``你好了,我就要撵你走了!''铁摩勒大出意外,叫道:``什么,你让我走?''王燕羽道:``是呀,你不是要报仇么?我不让你走,你怎能报仇?我是怕你说我怕你报仇,所以才要放你走呀!好啦,你试活动活动筋骨看看,能不能骑马?秦襄那匹黄骠马我们已给它治好伤了,这是一匹好坐骑,我可以转送给你。你要走就快走!要不然,到了龙眠谷,可就由不得我做主啦。''

铁摩勒情知她是随口捏个理由,好放自己逃走,心下踌躇,不知如何是好。只见王燕羽已把他的兵刃和背包送了过来,说道:``你的东西都在这里了,这一包肉脯,是给你在路上吃的。''

铁摩勒咬了咬牙,接了过来,说道:``你将来若是落在我的手中,我也饶你一次不死。''王燕羽笑道:``第二次就不饶了?好呀,那我可真的要小心,不可落在你的手中了。''

王燕羽牵着他的手,揭开帐幕,抬头一看,说道:``今晚月色很好,你自己知道路吗?''铁摩勒道:``不用你替我操心,哼,哼,我有言在先,你这次放我回去,可不要后悔!''

王燕羽笑道:``我本来就准备等你再来报仇,何悔之有?喂,你也不向我道别一声么?''

那丫鬟已把秦襄那匹黄骠马牵来,就在此时,忽听得呜呜呜三支响箭,掠过上空,紧接着巡夜的女兵吹起了响亮的号角。

王燕羽叫道:``不好,有敌人夜袭!''片刻之间,只见两队骑兵从东西两边冲来,采取包抄之势,杀声震天。黑夜之中,不知多寡,更不知是何方人马?

王燕羽笑道:``敌方有备而来,于我不利,叫她们各自撤退!''叫那丫鬟拿了她的令旗,下去传令。

王燕羽突然用了几分劲力,将铁摩勒的手紧紧一握,铁摩勒冷不及防,被她捏得``哎哟''一声叫将起来,大怒道:``你待怎么?''

王燕羽道:``你现在气力未曾恢复,难以抵挡敌人,在乱军交战之中,危险太大。我送佛送到西天,你随我走吧。冲了出去,我再让你一个人走。''不由分说,便把铁摩勒扶上马背,叫道:``你坐不稳可以抱着我的腰,逃难要紧!''

说话之间,双方已是展开混战,王燕羽运剑如风,接连把几个敌人刺于马下,策马直冲出去!

那匹黄骠马是匹久经训练的战马,不必鞭策,它也知道自己突围,但王燕羽不是它的主人,它似乎有意让她吃点苦头,振蹄疾走,遇到障碍,往往一跳起来,便跃了过去。

王燕羽的骑术甚精,她倒没有吃到苦头,可是铁摩勒却受不住了,他的脚背、腿肚、足跟,都是曾给挠钩勾伤了的,那匹马如此狂跑疾跃,他险险给马掼了下来,无可奈何,只好抱着王燕羽的纤腰,心里暗呼``惭愧!''

只听得敌方有人叫道:``王家的小贼不知哪里去了?却碰着这队娘儿们,真是晦气!''口气粗豪,似是不屑和这班女兵交手。

铁摩勒听这声音颇熟,一时间却想不起是谁,心念未已,对方已有许多人七嘴八舌的抢着叫道:``喏,那不是王伯通的女儿吧?你瞧,她马背上还有一个男人!''``咦,看这模样,不像是她的哥哥,这是谁呢?''``哈,哈,你瞧,这个男人还搂着她的腰,那么亲热,九成是她的野男人!''铁摩勒面上阵阵发热,只听得又有人接着叫道:``不必管他是谁,只要那女的是王伯通的女儿就行了。这女强盗比她的哥哥还要凶狠厉害,将她除掉,就等如削掉了王伯通的一条臂膊!''

先前那声音大喝道:``好,且待我上前将她一斧劈了!她手下这些臭婆娘不值得一刀,都放她们走了吧!''

说时迟,那时快,只见那个虬须大汉,手挥大斧,斜刺里一马冲来,铁摩勒猛地心头一震,原来这人正是金鸡山的寨主辛天雄。

辛天雄是北方绿林中响当当的角色,往日他雄踞金鸡山,既不依附窦家,也不依附王家,但是自从王家大破了飞虎山,铲除了窦家五虎之后,龙眠谷一会,韩湛、南霁云等人揭破了王家与安禄山勾结的阴谋,自此之后,辛天雄就一直与王家作对。这次他打听得王龙客率众出动,只道他是去做什么买卖,因此特地在他的归途设伏,进行夜袭,却不料王龙客已随张忠志去了范阳,只碰上他的妹妹王燕羽。

铁摩勒就是在龙眠谷之会的前夕,在韩湛家中与辛天雄见过一面的,时隔七年,黑夜之中,辛天雄已认不得铁摩勒了。

铁摩勒待要出声相认,心里却猛地想道:``我搂着仇人的女儿,辛叔叔是个直心眼之人,叫我如何向他解释?''

心念方动,辛天雄的快马已是冲来,一斧劈下,王燕羽冷笑道:``你这鲁莽匹夫,敢来欺我?''一个``蹬里藏身'',唰的一剑刺出,辛天雄一斧劈空,只听得``嗤''的一响,他的垫肩已给王燕羽一剑戳破!

王燕羽因为有铁摩勒抱着她的腰,这匹马又是她初次骑的,因此她的骑术剑术虽然精妙,这一剑本来可以要了辛天雄的命的,却仅仅给了他一点轻伤。

辛天雄大怒,拨转马头又是一斧劈来,这一次他领教过了王燕羽的剑法,不敢冲得太猛,仗着斧长剑短,大斧横挥,无所马颈。

辛天雄的斧重力沉,这一下王燕羽也不敢硬接。可是他不该挥斧斫马,这匹马身经百战,机警异常,一见大斧斫来,不待主人驾御,猛地就斜冲出去,反而抄到了辛天雄的马后,举蹄便踢。辛天雄的坐骑也是匹短小精悍的蒙古种良驹,但却禁不起这匹黄骠马的猛力冲击,登时被它一脚踢翻,王燕羽冷笑道:``好呀,看你还敢发横!''柳腰一弯,俯身一剑刺下。

铁摩勒搂着她的腰,当她和辛天雄恶战的时候,早已转了好几个念头。要知铁摩勒的气力虽然未曾恢复,但点穴的功夫还在,只要他在王燕羽的``愈气穴''上一按,王燕羽便得浑身瘫痪,不必铁摩勒亲自杀她,她也会被辛天雄的斧头劈死。

可是这念头一起,铁摩勒立即便感到可耻,心中想道:``大丈夫纵是报仇,也得光明磊落!她如此信任我,我岂可暗算于她。''

心念未已,辛天雄的坐骑已被踢翻,这时,王燕羽正在一剑刺下。铁摩勒心头一震,他虽然不愿暗算王燕羽,但更不愿辛天雄死于非命,百忙中无暇思索,立即使尽浑身气力,将王燕羽的腰板一扳,王燕羽这一剑刺不下去。辛天雄早已被人救走。

王燕羽怒道:``你干什么?你认识这厮?''反手就要将他抛下马背。铁摩勒定着眼睛望她,王燕羽忽地叹了口气,说道:``冤家!好,总算你还有良心,未曾乘机伤我。''

就在她说话之间,又是一骑健马如飞奔至,马上的骑士却是个刚健婀娜的女郎,铁摩勒三是心头一震,这少女不是别人,正是韩湛的女儿韩芷芬。

王燕羽叫道:``好呀,韩姐姐原来是你!咱们可得好好较量一番了。''七年之前,韩芷芬曾冒充辛天雄的女儿,参加龙眠谷之会,与王燕羽暗中较量过几手功夫。王燕羽不久就知道了她的身份,早就想找她正式比试一番,以雪被戏弄之耻。

韩芷芬笑道:``我正是为了要领教姐姐的剑法来的!''她一马冲来,马未停蹄,已在马背上挽了一个剑花,使出一招``七星伴月'',待得两匹坐骑相接,她的剑尖已绽出七点寒星,就在这一措之内,分刺王燕羽的七处大穴。

她的父亲韩湛是天下第一点穴名家,她的用剑刺穴的功夫,虽然未到炉火纯青之境,但在武林之中,也只有空空儿两师兄弟才能胜得过她;这一招使出,配合上健马冲刺的威势,王燕羽也不由得心头一凛!

但听得一片金铁交鸣之声,震得耳鼓嗡嗡作响,在这瞬息之间,双剑已接连碰击了七下。她们二人的本领本是半斤八两,各有增长,难分轩轻,但王燕羽的马背上多一个人,她处处要照顾铁摩勒,无形中等于受了牵制,这一来便不免稍稍吃亏,剑光过处,只见一缕青丝,随风飞散,王燕羽的头发被削去了一绺!

铁摩勒垂下了头,贴着王燕羽的背脊,不敢让韩芷芬瞧见。韩芷芬却忽地停手喝道:``咄,你马背的那臭小子是受了伤的不是?将他抛下来,我不想误杀受伤之人,也好让你施展本领,与我一决胜负!''原来她虽然没有眼见铁摩勒的面容,但见他不声不响,又不帮助王燕羽抗击,自然猜到他是受伤。

王燕羽一提马缰,便冲出去,韩芷芬笑道:``他是你的什么人?你怕他落在我们的手中么?我们是真正替天行道的绿林豪杰,不比你们胡乱杀人,更不会乱杀俘虏,你放心好了。反正你们也逃不了,不如将他放下,咱们可以好好比划一场,要是你胜得过我,我还可以为你向辛寨主说情,照武林中单打独斗的规矩,放你们过去。''

辛天雄的手下抛出绊马索阻道,那匹黄骡马见前路不通,登时止步,正待觅路奔逃,说时迟,那时快,韩芷芬已追了到来,笑道:``怎么样?你舍不得抛下这小子与我单独比斗一场么?''

王燕羽大怒喝道:``你罗嗦甚么?我的事不要你管!''拨转马头,反手一剑就向韩芷芬胸前刺去,这一剑来得劲道十足,韩芷芬一伙身,在马背上一剑横削出去。这时两匹马正在擦身而过,韩芷芬使这一招险到极点,但也厉害非常,她是在马背上巧使``伏地回龙剑'',倘非骑术剑术两皆精妙,这一招实在难以使得出来。

两人的剑法都迅如闪电,王燕羽一剑刺了个空,陡然间只见韩芷芬的长剑已贴着她的马身削来,除了立即缩到马前之上,她的双脚就要给剑削断。

王燕羽的骑术也真了得,就在这间不容发之际,她身形一侧,倏的就窜过一边,双足钩着另一边的马鞍,就似斜挂在马上似的,而且她的一只手还搂着铁摩勒,把铁摩勒的身子也扳平卧倒马上,避开韩芷芬的那一剑。

可是她却没想到这匹黄骠马,这时却忽然大声嘶叫,猛的跳跃起来,王燕羽只有一只脚能够使出,制它不住,登时被抛了出去!

原来这匹马甚通人性,最能护主,秦襄南征北战,就曾倚仗它脱过不少次险难,它认得王燕羽是敌人,在它被擒的时候,又曾被王燕羽女兵的挠钩所伤,因此附就不服气被王燕羽骑它,一有机会,便立即将她摔了下来。

韩芷芬大喜,飞身下马,挥剑来刺王燕羽的穴道,铁摩勒跌落地上,打了个滚,恰好滚到王燕羽的身边。他也不知哪里来的气力,忽地双臂一振,似是一时情急,忘了危险,要用手来格韩芷芬的长剑。韩芷芬怔了一怔,正觉得这人似曾相识,只听得铁摩勒已在叫道:``韩姐姐!''

韩芷芬大吃一惊,连忙缩手,失声叫道:``摩勒,怎么是你!''

王燕羽身手何等矫捷,韩芷芬的剑势一缓,她早已一个鲤鱼打挺,翻了起来,身形掠出数丈之外。

韩芷芬叫声:``不好!这女贼可要逃啦!''正要仗剑法追,铁摩勒忽地``哎哟''一声,也不知是有意还是无意,恰恰跌进她的怀中。韩芷芬这一惊非同小可,顾不得羞臊,更顾不得去追敌,连忙将他扶稳,叫道:``哎哟?摩勒,你果然是受伤了,伤得这么重呀!''

王燕羽回头一望,见他们二人已在相认,冷笑一声,挥剑便闯。她剑法精妙,武艺高强,在场诸人,除了韩芷芬外,谁也不是她的敌手,不消片刻便杀出了重围。

辛天雄用绊马索擒获了那匹黄骠马,得意扬扬的回来道:``走了王伯通的女儿,却得了这匹宝马,也算不虚此行。你也擒获了这小子么?咦,你,你,你,你不是铁,铁少寨主么?''

铁摩勒施礼道:``辛叔叔,久违了,小任正是摩勒。''

辛天雄叫道:``哈,你长得这么高了,铁老寨主算是有后了,我们大家都在惦记你呢。''顿了一顿,忽地面色一沉,问道:``摩勒,这是怎么回事,你怎的和仇人的女儿这样亲热呢?''

铁摩勒面红耳赤,有口难开,韩芷芬笑道:``辛叔叔,你怎的这样粗心,摩勒受了伤,你也未看出吗?''辛天雄道:``啊,原来你是受了伤被她们捉去的吗?''韩芷芬插口道:``可不正是,我刚刚给他解了穴道的呢!''辛天雄道:``怪不得你泥塑未雕似地坐在她的马背上,见了我也不叫一声。怎么样,伤得重么?''铁摩勒暗暗感激韩芷芬替他掩饰,说道:``还好,只是手脚受了点伤。''

辛天雄道:``韩姑娘,你家的金疮药比我的好,摩勒的伤,就麻烦你代我料理吧。咱们等会再叙。''他是首领,这时战斗已经结束,天也快将亮了。他要去点查人数,料理伤亡,安排警戒,整顿队伍,准备一待天亮,便即拔队回山。

韩芷芬拉了铁摩勒,选了一个地方,并排坐下。韩芷芬瞧了瞧他的伤势,笑道:``那位姑娘待你不错啊,她们王家的金疮药比我韩家的还好,可用不着我来操心了。''

铁摩勒好不尴尬,说道:``韩姐姐,取笑了。''韩芷芬笑道:``我说错了么?这药难道不是她给你敷的?''铁摩勒只好点头承认道:``是她敷的。''韩芷芬咳了一声,装模作样的正容说道:``现在该轮到我来问你了,这究竟是怎么回事?刚才我替你捏造谎言,现在你总应该对我说实话吧。''

铁摩勒道:``我是受伤被俘,她要押解我回龙眠谷去。''韩芷芬笑道:``可没见过对犯人这样好法,既不缚你,又不点你的穴道,却和你同乘一匹马,还让你搂着她呢!''

铁摩勒面红耳热,低声说道:``我也不知道她是何用意,我和她家仇深如海,被她捉了,本以为是活不成的了。''

韩芷芬``噗嗤''一笑,伸出中指,轻轻戳了他一下,说道:``你这傻小子,你是真不懂还是假不懂。这可辜负了人家的一番心意了。我看呀,早在七年之前,她还是个小姑娘的时候,就已经欢喜你了。那次在龙眠谷,你和她交手,她不是对你手下留情么?你还记不记得?''

铁摩勒又羞又气,大声说道:``韩姐姐,你别调侃我啦!我与她仇深如海,不管她对我如何,我这仇总是要报的!你要不信,我给你发誓!''

韩芷芬掩着他的嘴,笑道:``报不报仇,这是你的事情,我要你向我发誓做什么?快别大叫大嚷了,叫旁人听了笑话。''这话有两层意思,似是说怕别人知道了他和王伯通女儿的事情会笑话他,又似是说他要发誓这件事情是个笑话。铁摩勒想到的是前一层,心中一凛,登时不敢再说。

辛天雄走回来道:``怎么样?伤好了些么?能不能骑马?''铁摩勒道:``多谢韩姑娘的金疮药,好得多了。骑马不成问题。''辛天雄道:``好,那么就请你到我山寨里暂歇几天。有几位你认识的人也在那里呢。''这时,无色已经天亮,辛天雄下了命令,立即拔队起行。

铁摩勒本来要赶到九原会他师兄,但一想自己伤还未愈,虽然可以骑马,但在路上碰到敌人,却是难以抵敌,而且他和辛、韩等人多年不见,盛意难推,便答应了辛天雄,到他山寨去住几天。

秦襄那匹黄骠马已被擒获。有一个头目试着骑它,被它摔了下来。辛天雄笑道:``这匹马真是匹好马,就是脾气太大,不服人骑,我本来可以制伏它的,只是怕以力服它,它的心里终须不服。''

韩芷芬道:``待我试试。''走到马前,这匹马日间曾受挠钩所伤,前蹄下撕去一片皮肉,当时王燕羽的手下曾给它敷了伤处,但经过夜间一场激战,包扎马脚的绷带已甩掉了。韩芷芬重新给它换药,再裹好伤,拍一拍它的颈项,笑道:``我和你交朋友,你愿意么?''那匹马昂首嘶鸣,竟似懂得她的意思似的,轻轻的挨擦她,服服帖帖的让她骑上去。辛天雄笑道:``还是你有办法,这匹马就给了你吧。''却原来这匹马认定王燕羽是它的敌人,而韩芷芬则是把王燕羽打跑了的,所以它对韩芷芬甚有好感,倒并非完全因为她替自己治伤的缘故。

铁、韩二人并马同行,韩芷芬道:``摩勒,你饿不饿?我这里有干粮。你瞧,我多粗心,几乎忘记问你了。''摩勒暗暗感激她体贴人微,当下说道:``多谢。我还有肉脯,请你给点水我就行了。''

这肉脯正是王燕羽送给他的,铁摩勒嚼着肉脯;想起昨晚的事情,不由得一片惘然。韩芷芬道:``你想什么?''铁摩勒道:``没什么。你爹爹身体可好?当年我多蒙地照拂,正想去拜见他。''

韩芷芬道:``好。但你想见他,只怕不能如愿。他不在山寨。''铁摩勒笑道:``哦,你爹爹竟放心让你一人落草为女大王么?''韩芷芬道:``我想落草,辛叔叔也不肯要我呢。我爹爹因为要到远方访反,不便携我同行,故而将我留在山寨,托辛叔叔照顾我。''

辛天雄的马在前面,听了这话,回头笑道:``不是我照顾她,是她帮忙我呢。要不是有萨氏双英和她在山寨里,王伯通早就吞并了我的金鸡岭了。''

金鸡岭高龙眠谷约有一百五十多里,黄昏时分,大队回到山寨,山寨里的大小头目,早已出来迎接。萨氏双英与龙藏上人是以客卿的身份留在山寨的,他们和铁摩勒是旧相识,双方相见,谈起当年大闹龙眠谷之事,都是十分感慨。

众人见了那匹黄骠马都啧啧称赏,龙藏上人道:``咦,这匹马是怎么得来的?''韩芷芬道:``是王伯通女儿的坐骑,是给辛叔叔擒获的。''龙藏上人道:``不对!''韩芷芬一愕,正想问有什么不对,铁摩勒已经说道:``这本是一个军官的坐骑。那军官被他们围困,是我恰好路过,拔剑相助,他才得突围而去的。''当下将经过说了一遍,龙藏上人道:``那军官叫什么名字?''铁摩勒道:``他冲出重围时,曾报姓名,姓秦,名字我一时忘记了。''龙藏上人道:``这就对了。那军官叫做秦襄,他的祖父便是本朝的开国元勋秦叔宝。我认得他这匹坐骑。这人虽是军官,却爱结交风尘豪侠,当年我到京师化缘,就曾蒙他款待过的。''韩芷芬笑道:``如此说来,这匹马我只能暂时用它,日后还得设法将它交回原主了。''

辛天雄沉吟半晌,说道:``马倒是小事,我听说这秦襄是随朝廷的使者到范阳去的,如今安禄山却要追捕他,大局定然有变。''当下派出两路探子,一路去探范阳的军情,一路去探龙眠谷的动静。

铁摩勒留在山寨养伤,辛天雄等人为了防备王家前来报复,每日只能抽出些少时间,来看铁摩勒一两次,韩芷芬却几乎整天都陪着他,两人谈论武功,各述见闻,倒是毫不寂寞。

过了四五天,铁摩勒的伤已痊愈,受损的肌肉已复生,辛天雄所派出的两路探子亦已先后回来。安禄山果然已经起兵造反,以诛杨国忠为名,率所部步骑十五万,号称二十万大军,南下进攻长安。龙眠谷亦在忙碌备战,王伯通已发出绿林箭,命令归顺地的各处山寨起兵。

铁摩勒怕大战一起,道路断绝,伤好之后,便即辞行。辛天雄不便再留,当下设宴饯行,席间殷殷嘱托,请铁摩勒在南霁云跟前代为致意,若有所需,金鸡岭愿从差遣。

韩芷芬也与他们同席,临行之时,铁摩勒颇有惜别之感,韩芷芬却言笑自如,好像并不把这场别离当作一回事。

辛天雄送了他一匹好马,铁摩勒走了一程,不知怎的,脑子里尽是盘旋着两个少女的影子,一个是王燕羽,一个是韩芷芬。心中想道:``王燕羽对我好像依依不舍,芷芬怎的却不肯送我下山?''心念末已,忽听得马铃声响,回头一看,可不正是韩芷芬策马赶来!

正是:谁道红妆情意薄,飞骑原是为郎来。

欲知后事如何?请听下回分解------

\chapter{第十八回 客店中宵闻警报
边关千里起烽烟}\label{ux7b2cux5341ux516bux56de-ux5ba2ux5e97ux4e2dux5bb5ux95fbux8b66ux62a5-ux8fb9ux5173ux5343ux91ccux8d77ux70fdux70df}

铁摩勒又惊又喜,叫道:``芬妹。怎么你也来了?''这几天他们朝夕相处,两人之间,早已不用客套,铁摩勒比韩芷芬长三岁,所以改了称呼,不叫``韩姐姐'',而叫``芬妹''了。

韩芷芬笑道:``我不送你下山,我知道你在心里一定骂我。''铁摩勒道:``这里高山寨已远,你只一个人出来么?''要知辛天雄与王伯通作对,金鸡岭周围都在王家的势力之内,铁摩勒怕她给敌人认出是金鸡岭的人,虽然她武艺高强,但孤身遇敌,究属危险。心里想道:``你要送就该早些来送,我已经走了几十里路,你才追来,这不是开玩笑吗?''

铁摩勒正想劝她不必远送,韩芷芬忽地笑道:``摩勒,我不是来送你的,我是来和你同行的。''

铁摩勒征了一怔,道:``怎么,你要与我同行?''韩芷芬道:``是呀,我在山寨里住得厌了,正想到外面走走。怎么,你不欢喜我和你作伴么?''铁摩勒道:``你怎么可以擅离山寨?''韩芷芬道:``我又不是金鸡岭上的头目,说走就走,有何不可?''铁摩勒道:``啊呀呀,你,你,你虽是他们的客人,也不该------''韩芷芬笑道:``你放心,我已经和辛寨主说好了的,并不是不辞而行。王家忙着和安禄山图谋大事,无暇对金鸡岭报复,我走开了并无影响。你下山之后,辛寨主也在担心你一个人在路上怕有危险呢,所以我一说他就答应了。''

铁摩勒吁了口气,道:``原来如此,你怎么不早说?''韩芷芬笑道:``我是有意令你惊喜的,怎么,你不高兴与我作伴吗?''

铁摩勒笑道:``哪有不高兴的道理?我还想向你请教点穴的功夫呢?''

两人并辔同行,一路谈谈笑笑,铁摩勒的马不及她的马快,韩芷芬经常要勒住坐骑等他。但虽然如此,在这一日之间,他们也走了二百多里,黄昏时分、到了一个名叫`扶风''的小镇。

这是一个汉胡杂处的地方,男女同行,司空见惯。他们到一间客店投宿,店主人望了他们一眼,问道:``你们是夫妻吗?店里只剩下一间房子。''铁摩勒面上一红,说道:``我们是兄妹。''店主人道:``既是兄妹,那也可以将就住住。这几天南来逃难的人很多,到处都住满了。恰好今天刚有一个客人搬出,算是你们的运气。''铁摩勒没法,只好要了那间房子。他郑重嘱托主人代为照料马匹,要了几个酒菜,便和韩芷芬进房。

铁摩勒是在刀枪堆里打滚长大的,但和一个女子在晚间同处一室,却还是有生以来的第一次,进了晚餐之后,两人在烛光下相对,都不免有点异样心惰,铁摩勒低声说道:``芬妹,你早些安歇吧,这张床给你,我在地上打坐。''韩芷芬道:``你病体初愈,还是你在床上睡吧,舒服一些。''铁摩勒红着脸道:``不,我是风餐露宿惯了的,在这地上打坐满舒服。''其实他是不好意思在韩芷芬面前睡觉。韩芷芬笑道:``我也不是什么干金小姐呀。好吧!你打坐我也陪你打坐吧。''

这间房子不过了方八尺,是名副其实的斗室,除了一张双人床,一张桌子之外,剩下的地方极为有限,两人都在地上打坐,几乎是肌肤相接,气息相闻。铁摩勒但觉缕缕幽香,中人如酒,禁不住神思飘荡,忽地一个少女的影子泛上心头,那是王燕羽的影子,他也不知道为什么这个时候却会想起王燕羽来。

忽然听得外面人声喧闹,店主人高声叫道:``客人们都请出来,长官来查夜啦。''韩芷芬骂道:``讨厌,一出门就碰上这些麻烦事儿。''铁摩勒笑道:``你就忍着点吧,要是和他们闹起来,麻烦就更大了。''

客人们陆续出房,韩、铁二人也混在人难之中,未到大堂,便听得有个军官问道:``你们这里有几位女客?''店主人道:``有三个。''那军官道:``是有男人相伴的还是单身女客?''店主人道:``有一个是兄妹同来,其他两个是并无男子陪伴的,不过也非单身女客,她们是结伴同来的。''那军官``唔''了一声,又问道:``这三个女客,有没有骑着马来的?''店主人道:``只有一个是骑马来的,就是那个妹妹。''军官连忙道:``马是什么颜色?''店主人道:``好像是匹黄骠马。''那军官道:``好,你带他们到马厩去看一看。''

韩芷芬吃了一惊,心道:``难道他们是来追查秦襄这匹宝马的下落么?''铁摩勒更是吃惊,这军官的声音尖锐刺耳,甚是特别,竞似在什么地方曾听过的。

这时他们已经出到大堂,铁摩勒抬头一看,不由得当场变了面色,原来这两个军官都是他认识的,一个是安禄山的亲兵副统领聂锋,这个人也还罢了,另一个却是曾在飞虎山上,和他的段叔叔交过手的那个精精儿。铁摩勒恨得牙齿格格作响,心中想道:``幸而他的师兄空空儿没有同来。''

当年在飞虎山上,精精儿与段-
璋比剑的时候,铁摩勒只是旁观人众之一,后来大闹龙眠谷,精精儿虽也在场,却未曾和铁摩勒交过手,何况铁摩勒现在已经长大,精精儿就算当初曾有印象,如今也不认识他了。

铁摩勒心里想道:``他们又没有未卜先知的本领,怎知道芬妹今日会骑这匹黄骠马下山?不对,九成不是为匹马来的!''``可是,不为这匹马又为的什么?聂锋是安禄山帐下有数的将领,怎的会到远离范阳数百里外一个小镇来查夜?''铁摩勒心里阵阵疑云,百思不得其解。

另外两个女客是一对跑江湖的卖解女郎,都有一头长发,精精儿叫兵丁举起火把,走到她们面前,端详了一会,忽然伸出手来,拨开她们的头发,年纪长的那个媚态撩人,``噗嗤''笑道:``大人,你干什么?哎呀呀,哈,哈,哈,我最怕呵痒!''精精儿面色一沉,将她们推开,喝道:``胡说八道,谁和你们闹玩?走开,没有你们的事了!''

精精儿眼光一转,落到韩芷芬身上,怔了一怔,走过来道:``干什么的?''韩芷芬道:``和哥哥一同逃难的。''精精儿道:``好一位美貌姑娘,你是懂武艺的吗?''指一指她腰间的佩剑。韩芷芬道:``武艺虽然不懂,但兵纷马乱,带剑防身,总好一些。若有坏人,也不能教他容易欺负。''

精精儿``哼''了一声,跨上一步,忽地来捏韩芷芬的手臂,铁摩勒徒地一声大喝:``你欺侮人!''一掌就照精精儿的面门掴去!

精精儿焉能给他打中,反手一刁,立即扣着铁摩勒的脉门,冷笑道:``浑小子,你不想活啦!''双指正想扣实,铁摩勒铁腕一振,一股非常强劲的力道突然发出,精精儿权指之力禁受不起,登时松了。

说时迟,那时快,就在这闪电之间,精精儿那一只手刚沾着韩芷芬的肌肤,韩芷芬已是挥袖一拂,引开他的眼神,右手五指一拢,使出家传拂穴功夫,跃将起来,反手朝着精精儿的脑门一拂。

精精儿这一惊非同小可,他本来已看出这对``兄妹''懂得武功,却做梦也想不到他们的武功如此厉害,百忙中霍地一个``凤点头''向后跃开,饶是地闪避得快,``太阳穴''附近已给韩芷芬的手指拂中,登对脑痛如裂,眼前昏黑。

铁摩勒拔出剑来,一剑就向精精儿刺去,精精儿听得金刃劈风之声,双眼未曾睁开,已是身移步换,他的轻功还在铁摩勒之上,铁摩勒出手如风,唰、唰、唰连环三剑,都未刺中,待到第四剑攻到,极精儿亦已拔出剑来,但听得``咣''的一声,双剑相交,精精儿倒退两步,铁摩勒的长剑却已损了一个缺口。

他们两人乒乒乓乓的打将起来,登时吓得鬼哭狼号,鸡飞狗走。聂锋拔出长剑,堵住门口,扬声问道:``是这两个人吗?''精精儿叫道:``不管他们是否刺客,先拿下来再说!''言下之意,即是要聂锋帮他的忙。

聂锋未上,韩芷芬先已攻到,她将青钢剑当成判官笔使,剑尖一颤,瞬息之间,连袭精精儿七处大穴。精精儿``咦''了一声,叫道:``你这丫头也会刺穴!''使了一个``游龙绕步''的身法,避招还招,也是在一招之内,连袭韩芷芬七处大穴。精精儿轻功比她高明,功夫也较为老到,韩芷芬一剑刺空,但觉劲风飒然,精精儿的剑头已指到了她胁下的``愈气穴'',幸而铁摩勒来得及时,一招``乘龙引凤'',将精精儿的宝剑引出外门,可是双剑相交,铁摩勒的剑身又损了一个缺口。原来精精儿这剑是由玄铁合金炼成的,名为``金精铁剑'',剑刃钝而无光,看来毫不起眼,但却沉重异常,给它碰着,就似给大铁棒砸击一般。

精精儿一招将韩芷芬杀退,哈哈笑道:``你的刺穴功夫也小错了,可惜尚未到家。''他话虽如此,心头却不禁为之一凛,要知精精儿的刺穴剑术,是从袁公古剑谱中学来的,这部剑谱早已失传,直到三十年前,始由他的师父从一古墓中掘得。精精儿与空空儿同门习技,空空儿能在一招之内连袭敌人九处穴道,精精儿不及师兄,只能在一招内连袭七处大穴。他们的师父已死,精精儿以为刺穴剑法,当世除了师兄,就要数他第一。哪知韩芷芬年纪轻轻,竟然也能像他一样,在一招之内,连袭对方七处穴道,而且使出的剑法又与他的所学不同,这怎不令地惊诧,心里想道:``难道刺穴之法不止一家,除了袁公剑谱,还有别的古谱不成?这丫头现在虽不及我,但亦已练到这般境界,再过几年,还当了得?''他不知道韩芷芬乃是韩湛的女儿,韩湛是天下第一点穴名家,这刺穴之法是他自己悟出来的。

聂锋拔剑出鞘,上前助战,挽了一朵剑花,使出一招``玄鸟划砂'',斜刺铁摩勒的膝盖,铁摩勒喝道:``你也来了么?''运足气力,将长剑当最作大刀来使,一剑劈下,聂锋是安禄山帐下第一把剑术好手,却不曾见过这等看似平凡,实则威力奇大的剑法,双剑一碰,立知不妙,只听得``咣''的一声,火花四溅,这一回却是聂锋的剑身损了一个缺口,他定睛一瞧,不由得失声叫道:``是你!''

精精儿道:``聂将军,你认得他?''聂锋道:``他就是铁昆仑的儿子铁摩勒。''原来经过了飞虎山之役,空空儿对铁摩勒甚为赏识,曾叮嘱过他的师弟,若是在江湖上碰上了铁摩勒,须得手下留情。聂锋曾听得精精儿谈过此事,故此把铁摩勒的名字说出来;希望精精儿放他过去。

哪知精精儿利欲熏心,他虽然敬畏师兄,但却想已结王伯通。当下哈哈笑道:``原来你就是死鬼窦老大的干儿子铁摩勒,我师兄昔日曾饶你不死,如今我看在师兄的份上,也不要你的性命就是。快扔下兵器,免得皮肉受苦。''

铁摩勒勃然人恶,喝道:``精精儿,你给我磕三个响头吧,你给我磕了响头,或者我也会饶你。''精精儿这一气非同小可,冷笑道:``好狂妄的小贼,你练了几天功夫?''登时展开狂风骤雨般的剑法,一剑紧似一剑,剑剑指向铁摩勒的大穴。聂锋暗暗叫苦。

铁摩勒毫不畏怯,展开了从段-
璋剑谱中学来的六十四手龙形剑法与精精儿对攻。他在磨镜老人门下七年,内功上已有深湛的造诣,再配上了这套上乘剑法,与精精儿已相差无儿。只是他在兵器和轻功这两方面却要吃亏,作战的经验也还不及对方,但他却胜在有一股锐气,精精儿见他竟似全不顾性命般的强攻猛打也不得不顾忌三分。

铁摩勒不知聂锋对他存有好意,见他向精精儿说出自己的名字,只当他们都是一丘之貉,因而出手之时,对聂锋也毫不留情,聂锋一来怕精精儿起疑,二来铁摩勒的剑招既然如此狠辣,迫得他也不能不认真对付。

精精儿默运玄功,调匀气息,刚才所受的拂穴痛楚,已完全消失,剑法的威力越来越强,再加上聂锋之助,更占上风,铁摩勒的攻势不久就被阻歇,韩芷芬的刺穴剑法也渐渐施展不开。

忽听得马嘶人闹,店门外乱成一片。原来这些兵丁是精精儿到了扶风镇之后,才调来的当地兵丁,根本就谈不到有什么本领,他们奉命到马厩去将那匹黄骠马牵出来,反而给那匹马踢翻了四五个,冲了出来,现在正在大街上拦截。

韩芷芬听得黄骠马的嘶鸣,心中一动,叫道:``摩勒,走吧!''两人同样心思,忽地双剑合壁,一齐向聂锋冲过去,聂锋本就无意与他们拼命,侧身一闪,韩、铁二人登时冲出了店门。

那匹黄骠马最能护主,它本来可以自己逃走,但它却不肯逃走,在大街上东奔西窜,大声嘶叫,等待主人。兵丁们一靠近它便给它踢翻,又因奉命生擒,不敢放箭,只好作势追逐,待到马儿冲过来,他们反而要远远避开。

韩、铁二人冲出店门,那匹黄骠马立即飞跑过来,哪知精精儿的身法当真是快到了极点,``呼''的一声,竟似鹰隼飞天,倏的从韩、铁二人头顶飞过,将那匹黄骠马一按,黄骠马禁不住他的内家真力,登时倒退了十数步。这匹马久经阵仗,知道遇到了强敌,一时之间,不敢上前。

精精儿转过身来,将他们拦住,纵声笑道:``还想逃么?''韩、铁二人双剑齐出,一个刺他的肩并穴,一个用``斩马式'',将长剑当作大刀来使,横析他的双腿,两人联剑而攻,各自使出看家本领。精精儿也不敢硬接,可是他溜滑非常,仗着轻灵矫捷的身法,左右一飘,右面一闪,竟然如影随形,韩、铁二人都感到精精儿就似在他们的身边,同时向他们攻击。两人不敢分开,只好背靠着背,合力抵御。

聂锋虽然有意将他们放走,可是这个时候,精精儿已将他们绊住,聂锋自是不得不上前助战。韩、铁二人联手要胜过精精儿,多了一个聂锋,他们就只有招架的份儿了。

精精儿撮唇长啸,一个军官飞马赶到,精精儿叫道:``武大人,你不必助我,请你先降伏这匹黄骠马吧,这是宝马,不可将它伤了。''

这军官名叫武令洵,乃是安禄山手下的一个得力的将领,他认得这是秦襄的坐骑,大喜叫道:``不劳吩咐,我认得这匹马儿。它的主人就是日前从范阳逃走的秦襄,这对小贼定是与秦襄有关,不管他们是否刺客,你将他们擒了,就是大功一件。''

精精儿笑道:``聂将军,如此说来,倒是给咱们误打误撞撞上了。''聂锋知道关系重大,精精儿似乎已有点起疑,他心头一凛,只好横了心肠,全力进攻。激战中只见剑影纵横,剑光霍霍,圈子越缩越小,韩、铁二人都已在对方的剑势笼罩之下,剑招渐渐施展不开。

正在这危急万分之际,忽又听得蹄声得得,有一匹白马从街道的那一头跑过来,骑在马上的是个少女,只听得她格格笑道:``你们找错了人啦!''倏然间如箭离弦,从马背上掠出,武令洵正在追那匹黄骠马,刚好碰上了她,一照面便即给她刺中了手腕!

铁摩勒一看,大喜叫道:``夏姑娘,你来了!''这少女正是夏凌霜。

夏凌霜运剑如风,当者辟易,霎时之间,已攻到精精儿背后,精精儿反手一剑,腾身飞起,喝道:``昨晚的刺客是你!''话声未了,已是在半空中一个转身,凌空刺下,这一招宛似兀鹰扑兔,来势凶猛之极!铁摩勒使了一招``举火撩天'',恰好与夏凌霜的青钢剑同时挥出,架住了精精儿的宝剑,但听得``当''的一声,精精儿一个筋斗倒翻出去,铁摩勒与夏凌霜也各自退过一边。他们两人合力,要胜过精精儿少许,可是精精儿身法矫捷,这一招虽是稍稍吃亏,但转眼间又已翻身扑到。

精精儿笑道:``好一位标致的大姑娘,幸亏昨晚没有划伤你的花容玉貌。''他用``盘龙绕步''的身法,绕着夏凌霜打转,韩、铁二人双剑刺空,精精儿运剑防身,以闪电般的身法乘隙直进,左手一伸,骈指如戟,便来点夏凌霜穴道。

夏凌霜似乎早料到他有此一着,霍地一个``凤点头'',挥袖倒拂过来,反手便是唰的一剑,精精儿叫道:``好狠的剑法!''只听得``嗤''的一声,夏凌霜的衣袖给他撕去了一幅,但精精儿的衣襟也已给她一剑穿过,两人都未曾受伤。

夏凌霜骂道:``好贼子,我不雪此耻,誓不为人!看剑!''原来精精儿已由王伯通保荐他给安禄山,担任守护节度府之责,夏凌霜昨晚到府中行刺,给精精儿飞出一柄匕首,削去了她的一绺头发,但却没有看清她的面貌。夏凌霜逃出府门,立即跨上白马,她那匹白马也是日行千里的宝马,精精儿赶她不及,只好跟着蹄印一路追踪。夏凌霜住在这条街另一头的一间客店,听得喧闹打斗之声,才赶过来的。

夏凌霜的剑法自成一家,奇诡无比,精精儿还是第一次和她交手,欺地女流力弱,见她剑到,用了一个``压''字诀,运足内力,拍将下去。哪知夏凌霜的剑锋忽地中途一转,变了方向,从他意想不到的方位刺来。精精儿身形一晃,正要避招还招,铁摩勒亦已一剑劈下,铁摩勒的内力与他不相上下,双剑一碰,铁摩勒的长剑固然再损了一个缺口,但精精儿的宝剑亦已给他荡开、夏凌霜喝一声:``着。''剑光如练,分心疾刺,饶是精精儿闪得快极,肩头已给剑尖划破了一条伤口。

聂锋慌忙出剑相援,铁摩勒喝道:``你这厮为虎作怅,也须饶你不得!''声到人到,举剑便劈!

两人的势子都急,眼看就要碰上,哪知夏凌霜来得比他们更快,就在铁摩勒举剑劈下的那一刹那,只见寒光一闪,夏凌霜已抢在前头,一剑刺出,聂锋肩头中剑,血流如注,大叫一声,舍命飞奔。铁摩勒被夏凌霜一挤,身形歪斜,一剑劈空,连呼可惜。他哪知道夏凌霜是有意放走聂锋,将他挤开。不过她这剑剑招凌厉,而且又确是已把聂锋刺伤,所以谁也看不出来。

聂锋一走;变成了精精儿以一敌三的局面,纵使他武功再强一倍,也难以抵挡这三个人的合力围攻。不过片刻,精精儿已接连遇了好几次险招,有一次险险给韩芷芬刺中他的``璇玑穴'',又有一次,铁摩勒的剑锋几乎贴着他的额角擦过,要不是他轻功超卓,身手矫捷,随便中了一剑,便有穿心裂脑之灾。

处此情形,精精儿哪里还敢恋战?激战中,铁摩勒使出杀手,一招``独劈华山'',将长剑当成大刀来使,朝他的天灵盖劈下,精精儿喝声:``来得好!''藉他这一劈的力道,剑失在铁摩勒的剑脊上一点,倏的便腾身飞起!

夏凌霜喝道:``留下头来!''精精儿刚刚跃起,猛觉劲风扑面,头顶上空白光如练。原来夏凌霜早已料到有此一着,在铁摩勒出剑之际,她已施展``一鹤冲天''的功夫,先一步跳起来。精精儿这一跃起,无异送上去受她剑劈!

精精儿也真了得,就在这性命俄顷、死生一发之际;他竟然在空中一个转身;俨如鹰隼回翔,倏的就避了开去。可是他身子悬空,究竟不及在地上那般矫捷,避是避开了,半边头发已给夏凌霜的剑光削去。

夏凌霜也知他轻功高明,难以取他性命,这一剑本来就是只想削他的头发,目的已达,哈哈笑道:``割发代首,饶你去吧!''

精精儿身法快极,转眼间便只见一个小小的黑点,远远听得号角长呜,夏凌霜道:``这厮还不服气,想是要再调帮手前来。''铁摩勒道:``他不服气?我这口气也未出呢,只怕他不来!''夏凌霜笑道:``报仇不在一日,咱们今晚总算已把他杀得狼狈而逃了。''韩芷芬也道:``咱们还要赶往九原,不要再恋战了。''

夏凌霜跨上白马,韩芷芬道:``摩勒,你和我同乘这匹黄骠马吧。别的马儿赶不上夏姐姐的白马。''铁摩勒见她已在马上招手,只得依从,当下三人二马,离开小镇,向西疾驰。

这两匹坐骑都是日行千里的骏马,俨如棋逢对手,将遇良材,振蹄竞跑,似是有意比赛脚力一般。韩芷芬抱着铁摩勒的腰,低声笑道:``你那天是不是这个样子?''铁摩勒被她一逼,面红耳赤,但却不自禁的想起了王燕羽来。

不久,天色大明,夏凌霜勒着白马说道:``咱们可以歇歇啦,这一跑少说也跑了一百多里,精精儿轻功再好也追不上了。''

铁、夏二人多年不见,这一次意外相逢,大家都很高兴。铁摩勒首先向她打听段-
璋的消息,夏凌霜道:``他们两夫妻这几年来在江湖上到处奔跑,找寻他们失去的儿子,直到现在,还未找到。''铁摩勒道:``你可有见过他们?''夏凌霜道:``三年前见过一次。最近我听说他在范阳,但我到了范阳,却不见他。''铁摩勒恍然大悟,说道:``怪不得精精儿他们口口声声说要捉拿什么刺客,原来是你在范阳曾经去行刺安禄山。''夏凌霜笑道:``我也不全是为了行刺而去的。他起兵造反,我到了范阳,适逢其会,才动了念头,要把他除掉,却不料碰着精精儿。''

铁摩勒问道:``那西岳神龙皇甫嵩,你后来可有再碰见么?''夏凌霜面色倏变,恨声说道:``这无恶不作的大魔头,你问他干嘛?''铁摩勒道:``我已问过师父,我师父说,皇甫嵩此人虽然有时行事怪僻,但江湖上指责他做的那些恶事,我师父却不相信是他做的。''夏凌霜``哼''了一声道:``我真不明白这老贼何以竟有这样好的人缘,好几位武林老前辈竟然都替他说好话?可是我却曾亲眼见到他杀了酒丐车迟,这件事情段大侠还未曾告诉你的师父。''当下将那一年她与段-
璋夫妇同上玉树山的事情说了一遍,说到了他们合力打败了空空儿,也说到了皇甫嵩暗杀车迟的经过,听得铁摩勒诧异不已。

他们放马缓缓而行,谈了半天,到了一处三岔路口,夏凌霜再勒着马,说道:``我还未曾问你,你们是上哪儿?''铁摩勒道:``我们是要到九原去会见我的师兄,郭子仪现在正需要帮手。''

夏凌霜忽地低声说道:``你见到霁云,请告诉他我正在等他,请他这几天内来我这里一趟。若是再迟,恐怕军情紧急,他要跑不开了。''

铁摩勒观言察色,笑道:``哦,原来你们已经这样要好了,南师兄却还不肯向我透露半点风声。''

夏凌霜嗔道:``油嘴滑舌,想讨什么?我和你是说正经事情。''铁摩勒笑道:``我说的不是正经事么?男大当婚,女大当嫁\ldots\ldots。''夏凌霜抬起手来,作势欲打,却忽地停止,反过来取笑他:``韩姑娘,你听摩勒说些什么?你可会意么?''韩芷芬笑道:``夏姐姐,你可别向我开玩笑,你不知道,他已经有了意中人呢!''

铁摩勒忙道:``好,都别开玩笑了,说正经的。你叫南师兄找你,你可尚未曾将地址告诉我呢。''夏凌霜道:``我已经和他说过了的,他大约也会料到这几天内,我会在那里等他的。''铁摩勒笑道:``原来你们早已约会好了,我这才是叫做瞎操心呢!''当下,他们就在岔路分手,铁摩勒与韩芷芬迳往九原,暂且不表。

且说聂锋受伤之后,落荒而逃,跑到扶风镇郊外,忽见精精儿也赶到来,大声叫道:``聂将军,聂将军!''

聂锋只好停了脚步,问道:``可曾擒获了刺客么?''精精儿面孔铁青,道:``都逃了!''聂锋道:``这几个小辈的确是扎手得很,我中了一剑,险些穿过了琵琶骨!''\,''

精精儿道:``让我瞧瞧。''望了他伤口一眼,忽地冷冷说道:``聂将军,这个女刺客对你可是很讲交情啊!''

聂锋变了面色,说道:``你这话是什么意思?你也未免太小觑我了!难道我让那刺客杀了,才是应当的么?''

精精儿道:``岂敢,岂敢!谁不知聂将军是剑术名家,我岂敢小觑将军?我那句话其实应该这么说,你对那女刺客也很够交情。''这几句话说得非常明白,却是说聂锋有意让她刺伤,而她这一剑却也是恰到好处。

聂锋本来有点心虚,一时之间,不知是发作好,还是不发作好。精精儿诡笑道:``聂将军,咱们在剑术上还算得说是个行家,不必相瞒了。这女贼是什么人?''

聂锋道:``我不认识\ldots\ldots{}''聂锋还想为他所受的轻伤辩解,精精儿已打断他的话道:``你真的不认识?我倒知道她姓夏,就是不知道她和你有什么关系?你要这样护着她!''聂锋面色大变,愤然说道:``你含血喷人!''

精精儿笑道:``聂将军,我只是想和你交个朋友,你别多心。你不肯对我说实话,那却是不把我当作朋友看待了。''忽地迈上一步,拍一拍聂锋的肩头,聂锋正自说道:``你要我说什么实话,\ldots\ldots{}''突然被他一拍,吓了一跳,只见精精儿已从他身旁跃开。手里拿着一封信,哈哈笑道:``这是那位卢夫人写给她母亲的信是不是?现在你可以告诉我了吧?那位卢夫人是夏姑娘的什么人?你和她们又是什么关系?''

聂锋被他以迅雷不及掩耳的手法,窃去了怀中的信件,登时吓得呆了。原来这是卢夫人写给她的表姐,亦即是夏凌霜母亲的信。这信卢夫人前几天就写好了,她知道聂锋要随军出征,可能经过她表姐的家乡,托他便中带交,她却想不到就在交了信给聂锋之后的第二天晚上,夏凌霜就偷偷来看她,而且还到节度府去行刺安禄山。

精精儿目不转睛的盯着聂锋,又纵声笑道:``听说这位卢夫人以前是有名的美人,可惜她的容貌已经毁了,聂将军,你现在才充作护花使者,不是有点晚了么?哈哈,这封信,你本来应该交给那位夏姑娘,大约是因为刚才在众目睽睽之下,你不方便交给她吧?这也不必为难,我给你送去好了!''

聂锋又惊又怒,呆了半晌,叫起来道:``你别胡说八道,我只是怜惜卢夫人的遭遇,有什么私情!你要出首,我拼着把这条命交给你便是。''

精精儿笑道:``我若要出首早就出首了,老实告诉你吧,前天晚上,卢夫人将这封信交给你,我已暗中看见了。聂将军,我也爱惜你是条好汉,你别怀疑我对你存有坏心。''

聂锋道:``好,那么你要什么?''精精儿道:``我也不问你和她们有什么私情,我只是问你要她们母女的地址!怎么样?你愿不愿意交我这个朋友,也好彼此互相扶持。''要知聂锋乃是薛嵩的表弟,也很得安禄山的信任。所以精精儿一来是投鼠忌器,二来也的确想结纳他。用这件事作为要胁,好令聂锋为他所用。

聂锋在安禄山的将领之中,是个比较正直的人,可是这封信已给精精儿搜去,就等如命根子捏在他的手上,在这生死利害关头,他究竟不是圣贤,踌躇了好一会,心中想道:``我若不说,他去出首,我固然送命,卢夫人也不能保。而且夏陵箱剑术高强,她的母亲又是当年著名的女侠冷雪梅,夏凌霜的剑术还是她母亲所传授的,精精儿对她们母女,也未必便讨得了好去。''

聂锋踌躇了好一会,终于低下了头,轻声说出了冷雪梅隐居的所在,精情儿哈哈笑道:``对啦,这才够朋友!''笑声有如枭鸟夜啼,听得令人毛骨悚然,聂锋被迫做出违背良心之事,又是后悔,又是羞愧,待他抬起头时,精精儿已去得远了。

铁摩勒与韩芷芬兼程赶路,那匹黄骠马骏健非常,虽然驮着两人,仍然比寻常的马匹快了几倍。第二天中午时分,便赶到了九原,当即前往太守衙门求见,轮值的门官听说他是南霁云的师弟,殷勤接待,说道:``太守与南将军正在内校场督导诸将练习弓马,铁壮士不是外人,便请进去。''

这内校场设在太守衙门之内,是中下级军官接受检阅和练习弓马的地方,铁摩勒进去,见过郭子仪与南霁云。郭子仪见他躯体魁梧,端的是一表人材,甚为欢喜,无暇叙话,便叫他坐在身旁,看请将操练。

其时正在练习弓箭,箭靶立在场心,射者在百步之外发箭,要射中红心,非但箭要射得准,臂力最少也要开得五石强弓。郭子仪麾下的将领果是不凡,铁摩勒看了十个人射箭,有七个人俱是三箭皆中红心,有两个人中两箭,成绩最差的那个人也中了一箭。

铁摩勒忽觉其中有一人似曾相识,只是想不起来。郭子仪已对他说道:``铁壮士,你也要试试么?''

铁摩勒有意卖弄功夫,当下要了一把五石铁胎弓,施展连珠穿云箭法,三箭连发,嗖的一声,第一枝箭穿过了红心接着第二枝第三枝跟着穿过,首尾相衔,跌下地来,还排成一条直线。登时赢得了全场的彩声!要知那箭靶里外三层牛皮,厚可五寸,诸将虽然有人三箭俱中红心,但却无一箭能穿过重革的,而且穿过红心之后,还能够首尾相衔,排成一行,那更是神乎其技了。

郭子仪大喜道:``千军易得,一将难求,铁壮士前来,正是天助我也。''当下传令罢操,在内堂设宴接风。

席上免不了谈论军情,铁摩勒这才知道,安禄山已经攻陷太原,太原留守杨光翔是杨国忠的同族,当时尚未相信安禄山乃是造反,糊里糊涂竟自出城迎接,立即便给贼兵捆缚起来,解送安禄山军前杀了。他造反至今,不过半月,已经攻陷了七八处州县,所过之处,势如破竹。

铁摩勒道:``怎的就让贼势如此猖獗?''郭子仪叹口气道:``都是承平日久,朝廷的兵制坏了,猛将精兵,多聚于边塞,内地几全无武备,因此一旦变起,便竟是望风披靡。''

原来唐初的兵制为``府兵制'',分天下为十道,置军府六百三十四,关内居其半,属诸卫管辖,各有名号,而总名为``折冲府''。府兵数分上中下三等,一千二百人为上等,一千人中等,八百人为下等。民自二十岁从军,至六十岁而免,体息有时,征调有法。折冲俯都设立木契铜鱼,上下府照,朝廷若有征发,下敕书契鱼,都督郡府参验皆合,然后发遣。凡行兵则甲胄衣装皆自备,国家无养兵之费,罢兵则归散于野,将帅无握兵之权。此法近于``寓兵于农''的征兵制,本来甚好,惜乎日久弊生,有等从军之家,因杂徭之累,渐渐贫困,管理府兵的官将,又役之如奴隶,府兵便多逃亡。死亡者有司不复添补,反利其死而没其资财。于是府兵之制日坏。至李林甫为相,奏停折冲府上下鱼书,自是折冲府无兵,空设官吏而已。至天宝年间,府兵制名存实亡,各地驻军多改为募兵,其所召募之兵,十九系市井无赖子弟,不习兵事。安禄山的兵马,本来强盛,又因番人部落突厥阿布司为回纥攻破,安禄山诱降其众,所以他的部下,兵精马壮,天下莫及。

郭子仪道:``好在朝廷现在已命大将军哥舒翰屯军潼关,作为长安的屏障。哥舒翰是能征惯战之将,安禄山未必过得了这一关。另外,朝廷又已任命原来的安西节度使封常清为范阳、平卢节度使,要他驰赴东京募兵,或者可以抑阻贼兵的凶焰。''南霁云道:``那封常清是个志大才疏的人,只怕不能济事。哥舒翰虽有将才,但是胡人,只怕也未必靠得住。看来这拨乱反正的大事,还得倚靠令公。''郭子仪道:``国家大事,不能倚靠哪一个人,大家都有份儿。现在局势已然如此,我也只有尽我自己的本份便是。''

席散之后,南霁云过铁摩勒进他的私室相叙。铁摩勒笑道:``南师兄,别的事都可以缓谈,有一件是要你立刻做的。''南霁云怔了一怔,道:``什么?''铁摩勒道:``有一个人在等着你呢!''南霁云道:``怎么?你见到了夏姑娘了吗?''铁摩勒笑道:``果然一提起你便知道是她了。''当下将途中所遇之事源源本本的告诉了南霁云,笑道:``师兄,你什么时候请我吃喜酒?''南霁云红着脸道:``别胡说。''其实,他心里正在暗暗欢喜,夏凌霜之约的确是与婚事有关的。

原来在这几年间,他们二人常相过往,早已情投意合,结下鸳盟。只因夏凌霜的母亲性情孤僻,她隐居在玉龙山下的沙岗村内,二十余年来足迹未曾踏出过村庄半步,也从来未接见过外人。所以在婚约未曾定实之时,夏凌霜也不敢带南霁云去见她的母亲,直到最近,夏凌霜禀明了她的母亲,得到母亲的同意,才敢邀他到家中相见。这事是他们上次见面时说好了的,夏凌霜本来要到九原偕南霁云同往,恰巧在途中碰见铁摩勒,而她又急于回家见母,因此托铁摩勒传话。南霁云一听,便知夏凌霜的母亲已经同意,心中自是欢喜无限。

第二日一早,南霁云便向郭子仪告假,郭子仪曾经见过夏凌霜,知道她是个巾帼英雄,当下问明原委,哈哈笑道:``若得夏女侠前来,咱们还可以成立一队娘子军呢。这事于公于私,都有好处,趁现在尚未有命令要我出师,你快去快回。但愿你好事能谐,我替你在军中主持婚礼。''

铁摩勒与韩芷芬这时亦已知道了消息,向南霁云道贺,铁摩勒又怪他师兄昨晚还不肯告诉他。南霁云红着脸道:``这事要她母亲点了头才能算数。''郭子仪笑道:``南将军这等人材,夏太夫人哪有不点头之理。这不过是循例要未来的女婿见见岳母罢了。好了,南将军你有喜事在身,咱们不想耽搁你了,你去挑选一匹快马,立刻动身吧。''韩芷芬笑道:``有现成的快马,正好借给你用。就是我那匹黄骠马,不过这匹马不服生人,待我亲自牵给你骑。''

南霁云见了那匹马,喷喷称赞,韩芷芬笑道:``这匹马其实也不是我的,是龙骑都尉秦襄的。''南霁云昨晚已听得铁摩勒说知其事,笑道:``秦襄与我彼此闻名,可惜当年在京中未曾见面。待我回来之后,再备办礼物,将马送还给他,现在且先领他这个情吧。''

当下南霁云带足干粮,跨上了黄骠马,立即赶去与夏凌霜相会。玉龙山离九原八百余里,平常坐骑须得四五日,这匹黄骠马放尽脚力,第二日中午时分,便已赶到。

南霁云进了村庄,他早已问明夏凌霜,知道她家门口有三棵柳树为记,不须问人,便找到了。他牵着坐骑,到了夏家门口,心里又是欢喜,又有点腼腆,担心未来的岳母不知道会不会欢喜他。

夏家的大门紧闭,南霁云拉着门环,扣了两下,里面全无声息。南霁云踌躇片刻,只好通名叫道:``魏州南霁云求见。''叫了两声,里面仍是毫无声息。

正是:千里迢迢来践约,一场欢喜一场空。

欲知后事如何?请听下回分解------

\chapter{第十九回 践约远来人不见
传言难信事堪疑}\label{ux7b2cux5341ux4e5dux56de-ux8df5ux7ea6ux8fdcux6765ux4ebaux4e0dux89c1-ux4f20ux8a00ux96beux4fe1ux4e8bux582aux7591}

南霁云惊疑不定,心道:``纵是她母亲不肯许婚,也断无闭门不纳之理。难道有这么巧,她母女二人都外出去了?''鼓起勇气,放大了声音再叫道:``凌霜,是我,快开门!''他运用内家真气将声音送出,里面若是有人,定然听见,可是仍然无人回答。

南霁云情知不妙,这时再也顾忌不了那许多,拔出宝刀护身,施展``一鹤冲天''的轻功,立即跃上墙头,只见里面深院静,小庭空,冷冷清清,竟似无人光景。

南霁云提着宝刀,小心翼翼的一步一步进去搜查,刚踏上台阶,陡然间听得有个声音喝道:``好大的胆,白日青天,擅闯民家,干什么的?''

只见客厅里面坐着一个猴子脸的军官,不是别人,正是精精儿。

南霁云虽然料到有意外之事,却怎也想不到精精儿会在这儿。他怔了一怔,又惊又怒,正待喝问,精精儿已自发出了一声狞笑,站起来道:``我道是哪个胆大妄为的强盗,原来是你;好呀,南霁云,你也是朝廷军官,未得主人允许,白日青天,持刀进屋,你还知道有朝廷王法吗?''

南霁云怒道:``岂有此理?你简直是恶人先告状,这儿是夏姑娘的房子,你在这里干什么?夏姑娘呢?''

精精儿冷笑道:``我当然知道这儿是夏姑娘的房子。你是她的什么人,胆敢擅自闯进?''

南霁云气怒交加,但却不好意思说是夏凌霜的未婚夫。当下,强抑怒火反问他道:``你又是她的什么人?''

精精儿淡淡说道:``她是我王家兄弟的妻子,也就是我的义嫂,王家兄弟接了她们母女完婚去了。我是替她们看守房子的。哼哼,你偷偷摸摸的进来找人家的妻子,存的什么心肠?''

南霁云气得七窍生烟,骂道:``你胡说八道!看刀!''一招``跨虎登山'',进步横刀,立即劈下。

精精儿冷笑道:``你白日青天,持刀进屋,非奸即盗,我正要揪你去见官府!''说时迟,那时快,他的宝剑也早已出鞘,扬空一闪,反削南霁云的手腕。

南霁云的武功本来与精精儿在伯仲之间,但因他先动了怒火,心浮气躁,不过数招,被精精儿觑了一个破绽,唰的一剑,穿过了他的衣襟,幸而他披有软甲,退闪得快,要不然这一剑便是穿心剖腹之灾。

南霁云到底是身经百战的大侠,吃了个亏,瞿然自省,便即沉下气来,使出了一套五门八卦刀法。

这套刀法寓攻于守,沉稳非常,施展开来,泼水难进,他踏着五门八卦方位,进退之间,法度谨严,饶是精精儿身手矫捷,出剑如风,但每一招攻到,都给他随手化解,激战了三五十招,竟是无法攻破他的门户。

南霁云与精精儿的武功本来是各有擅长,难分轩轻,但在这屋子内拼斗,精精儿的轻功受到限制,未能尽展所长,而南霁云学的是正宗内功,造诣却要比精精儿稍胜一筹,加以南霁云一腔愤气,拼了性命与精精儿厮杀,当真是神威凛凛,叱咤风生,在战意上先慑伏了精精儿。

激战中南霁云运足内家功力,刀掌兼施,猛地大喝一声,横刀一摆,用了一招``铁锁拦江'',将精精儿的宝剑封出外门,立即一掌劈去。精精儿也真了得,身形微动,宝剑蓦地反弹而起,一招``金针度劫'',反挑上来。南霁云早料他有此一招,抢前一步,精精儿的剑尖在他肋旁倏然穿过,南霁云倒转刀锋,双肘一撞,突然间化为``阴阳双撞掌''的招式。这一变招古怪之极,精精儿纵是见多识广,也料不到他突然会舍刀不用,出此险招。

只听得``蓬''的一声,精精儿胸口已中了他一记肘锤,精精儿的轻功确是高明,南霁云一得手,立即便反转刀锋劈他,精精儿中了他的肘锤,竟然能在这瞬息之间,提气拔身,嗖的飞起一丈多高,攀上了屋顶的大梁。

南霁云喝道:``精精儿,你下来!''精精儿``哼''道:``你当我怕你不成?''他蹲在梁上,把手一扬,一道蓝艳艳的光华,骤然射下。南霁云知道他的毒匕首厉害,急忙把宝刀抡圆,护着全身,精精儿连发了三支匕首,都给他打落。可是南霁云在他毒匕首威胁之下,却也不敢攀上屋梁,与他决斗。

精精儿冷笑道:``你敢上来!''忽地一声长啸,双手连扬,六支匕首齐发,南霁云将宝刀舞了一个圆圈,但听得叮叮当当之声,不绝于耳,六支匕首,都给荡开,可是南霁云也被迫得连退几步。

这间客厅的两边都有个厢房,房门紧闭,南霁云这时正退到东边的厢房门口,精精儿的啸声未绝,那房门突然倒塌,向南震云压下,跟着``嗖'的一支冷箭射出,南霁云一脚踢飞门板,霍的一个``凤点头'',刚避开了那支冷箭,猛然间,西边也是轰隆一声巨响,从那边厢房里飞出一个大花瓶,南霁云脑后不长眼睛,不知是什么暗器,百忙中无暇思索。立即反手一刀。

``当嘟''一声,花瓶震裂,瓷片纷飞,南霁云给割伤了两处皮肉,虽说这不是什么厉害的暗器,但在激战之中,突遭意外,却也不禁乱了心神。

说时迟,那时快,两边厢房都已有人窜了出来。东边厢房的是薛嵩,西边厢房的是田承嗣。原来这两个人早已埋伏在厢房里面,只因精精儿素来自负,他起初以为可以独力制伏南霁云,所以没有叫这两个人出来。后来发现最多只是可以打成平手,精精儿无可奈何,这才发出暗号。

薛嵩的长剑先行攻到,南霁云大吼一声,横刀立劈,薛嵩正自使出一招``卞庄刺虎'',弯腰沉剑,刺他的膝盖,被他的宝刀一压,长剑登时弯曲,抽不起来。田承嗣用护手钩刺他的背心,南霁云头也不回,一个虎尾脚撑出,正中田承嗣的手腕,两柄护手钩都已脱手飞出。田承嗣曾是他手下败将,兵器脱手,心胆俱寒,慌忙退下。

就在此时,精精儿一声长啸,突然从屋梁上跃下,南霁云来不及结果薛嵩,手腕一抬,宝刀翻起,``当''的一声,把精精儿的``金精铁剑''格开。可是精精儿居高临下,这股冲劲大得异常,南霁云刚刚摆脱了薛嵩的攻击,步法凌乱、身形迟滞,虽然格开了他的宝剑,但精精儿同时使出的那一招擒拿手,他却没法避开,给精精儿在他的肩胛一拿,半身麻软,向前冲出两步;终于倒下地来。

精精儿连忙点了他的麻穴,哈哈笑道:``好小子,看你还凶不凶?你要见夏姑娘吗?好,我就送你去见她。''

薛嵩刚才被南霁云的猛力一震,撞到了墙壁才收得住脚步,头破血流,甚为狼狈。这时见南霁云被擒。旧仇新恨,一时间都上心头。瞪眼骂道:``好呀,姓南的,你也有今日。''提剑过来,向南霁云胸口便刺。

精精儿道:``薛将军,不可!''一伸手便扣住了薛嵩的手腕。薛嵩道:``留他作甚?''精精儿笑道:``这人大有用处,你要杀他,但怕主公却要留他呢。你杀了他,叫我如何交代?你难道不知道他是郭子仪的心腹将领么?''薛嵩翟然自省,心中虽然气愤难平,也只好罢了。

精精儿挟着南霁云走出门外,那匹黄源马还在门前,它不知道主人已是被擒,迎上前来,精精儿大喜道:``哈,原来秦襄的这匹宝马还在这儿。''他挟着南霁云,脚步一点,立即飞身上马。

这匹马甚有灵性,它见南霁云一声不响而且是被精精儿挟在胁下,知道主人遇难,登时一声长嘶,双蹄人立,跳将起来。精精儿怒道:``畜牲,你敢不服我吗?''用力一按,那匹马负痛嘶鸣,跪在地上,索性动也不动。精精儿哼了一声,取出绳索,将南露云缚在马背上,拔出宝剑,捉着那匹马,将宝剑在它面前晃了一晃,作势向南霁云刺去,骂道:``畜牲,你胆敢不听我的使唤,我先把你的主人一剑杀了,然后再把你抽筋剥皮!''这匹马被他一吓,竟似乎听得懂他的话似的,终于拱起背脊,站立起来。精精儿冷笑道:``这姓南的其实也不是你本来的主人,为什么你这畜牲愿顺从他却不顺从我?哼,哼,我非把你整治的俯首贴耳不可!今后我就是你的主人了,你知道吗?''那匹马四蹄擦地,大声嘶叫,似乎表示抗议。但是,精精儿跨上马背,它却也不敢乱跳乱跃,意图将精精儿掀下来了。

精精儿在马背上扬声说道:``这匹马的脚程比我快得多,我赶着先回去了。你们二位随后来吧。''田、薛二人都不忿他独得宝马,且又先赶回去独自邀功,可是他们的本事远不及精精儿,只有敢怒而不敢言。

南霁云被精精儿用重手法点了麻穴,动弹不得,但是神智却尚未昏迷。他学的是正宗内功,造诣已经到了第一流的境界,暗暗运气冲关,却不料精精儿的点穴手法自成一家,用的又是重手法,南霁云试了好几次,都未能解开穴道。

那玉龙山绵亘数百里,翻过此山,便是安禄山管辖的幽州境界。精精儿仗着人强马壮,贪图快捷,不走官道而走山路。快马奔驰了两个时辰,日头渐渐偏西,山路越来越险,不久来到了一处所在,那是双峰夹峙之下的一个隘口,羊肠小道陡峭险窄,像一条长蛇婉蜒在丛山峻岭之中。这匹黄骠马端的神异非凡,非但履险如夷,而且脚程也丝毫不缓。

精精儿将要驰出隘口,目光所及,忽见在隘口当道,躺着一个乞丐,那乞丐发如乱革,枕在路旁石上,半边脸孔埋在茅草丛中,身躯却横过道路,鼾声如雷,远远可闻。

精精儿喝道:``马来啦,臭叫化,快滚开去!''那叫化呼呼的睡得正沉,对他的叫声竟似未曾听见。精精儿大喝道:``你是聋子吗?要不要命?''那叫化子翻一个身,``哼''了一声,摊开了八字脚,索性睡到了山路的当中。

精精儿大怒,纵马便奔过去,心中想道:``这是你自己找死,可怪不得我!''心念未已,眼看马蹄就要踏到那叫化身上,猛听得那叫化一声喝道:``小猢狲,滚下来吧!''

就在这刹那间,黄骠马的狂奔之势突然煞住,精精儿做梦也想不到这老叫化有如此能力,冷不及防,在马背上抛了起来。说时迟,那时快,那老叫化已是长身而起,一手向他的脚踝抓来。

精精儿也真了得,身于悬空,猛地一个扭腰,在间不容发之间,避开了那老叫化的一抓,迅即俯冲而下、反手一掌,击中了那老叫化的肩头。

那老叫化骂道:``小猢狲,没人管就想造反啦。''精精儿的掌锋刚刚触着他的身体,猛觉一股大力反震过来,精精儿大吃一惊,慌忙一个筋斗倒翻出去。这老叫化用的是``沾衣十八跌''的上乘内功,幸而精精儿这一掌之力未曾用实,要不然更要大大吃了。

精精儿一个鲤鱼打挺,从地上翻了起来,他的身法已经快极,哪知脚步刚刚站稳,抬头一看,只见那老叫化又已拦在他的面前,冷冷说道:``我睡得好好的,你为何吵醒我?这也还罢了,你还居然要谋害我!哼,哼,要不是老叫化有点儿能耐,这几根老骨头早就给你踏碎啦!''

精精儿猛地想起一个人来,心头大震,想道:``莫非这老叫化就是此人。''连忙抱拳施礼,低声下气地说道:``晚辈为了赶路,一时收不住坐骑,触犯了老前辈。晚辈在这厢赔礼了。还望老前辈大度宽容,放我过去。''

那老叫化仰天打了一个哈哈,说道:``你倒说得容易,要我放你,你可得先赔我一件东西。''精精儿道:``老前辈要我赔些什么?''那老叫化道:``我正做到一个好梦,被你惊醒,梦做不成了,你可得赔我一个好梦。''精精儿忍着气道:``梦如何赔法?我马上就走,老前辈你再睡过吧。''那老叫化道:``胡说八道,我睡意已过,怎能再睡?再睡也未必有梦。有梦也未必就是好梦!''精精几道:``这我可没法了。老前辈,我再给你赔罪吧。''那老叫化道:``好,好梦你既不能赔找,那就给我磕三个响头,算作赔罪也罢。''

精精儿自大惯了,虽是对老叫化心存怯惧,却怎肯向他磕头?那老叫化又仰天打了一个哈哈,说道:``你不肯磕头么?那就将这匹马赔给我吧!''这匹黄骠马似乎也知道老叫化的厉害,受了惊吓,这时已远远的躲过一旁。

精精儿踌躇不语,那老叫化道:``怎么?舍不得马?反正你这匹马也是偷来的,送给我也不过做个顺水人情。''精精儿吃了一惊,心道:``原来他也知道这匹马的来历。''想了一下,说道:``这匹马送给老前辈不打紧,不过晚辈身居军职,现在正要押送一名犯官回去,三日之后,请老前辈到范阳的节度府来取如何?''

那老叫化双眼一睁,说道:``哈哈,瞧你不出,原来你还是安禄山手下的军官。你押的是什么人?老叫化生来爱管闲事,你说给我听听。''

精精儿暗自盘算脱身之计,讷讷说道:``这个人么?说给老前辈听也不打紧,他,他\ldots\ldots{}''他看那老叫化正在聚精会神的听他说话,忽地一柄匕首向那老叫化胸前飞去。

就在此时,南霁云忽地大声叫道:``卫老前辈,是我!我是魏州南八!''原来他暗自运气冲关,虽然尚未能够解开穴道,却已可以开声说话。

精精儿匕首掷出,立即疾如鹰隼般的向那匹黄骠马扑去,他知道这老叫化本领高强,并不指望这一柄匕首能伤得了他,但盼能暂时阻他一阻,只要自己能飞身上马,向回头路跑,那老叫化本领再高,也无可奈何他了。

精精儿轻功卓绝,那匹黄骠马正要走步奔跑,未曾发力,精精儿鼓劲一冲,疾似离弦之箭,一手抓着了马尾,正要腾身上马,猛听得那老叫化喝道:``小猢狲,想跑么?你也接接我的暗器!''

陡然间,只觉四面风生,漫天树叶,向他刮来。原来这老叫化不是别人,正是名震江湖的``疯丐''卫越。``疯丐''卫越、``酒丐''车迟与``西岳神龙''皇甫嵩并称江湖三异丐。三丐之中,卫越居长,出手也最狠辣。这一手正是他的``飞花摘叶,伤人立死''的功夫。

精精儿识得厉害,来不及跨上马背,立即腾身飞起,饶是他跃起得快,且又已闭了全身穴道,仍然给几片树叶打中,痛得他尖叫一声,在半空中打了一个筋斗,便即流星陨石般的坠下深谷。卫越``哼''道:``不是看在你死去了的师父的份上,我就要了你这小猢狲的性命。''

那匹黄源马见卫越打跑了精精儿。对他的敌意大减,它本来已在发力奔跑,这时却转过身来,向卫越摇头摆尾。卫越哈哈大笑道:``好一匹马儿!''将南霁云在马背上拉下,并替他解开了穴道。

南霁云重新施礼,谢过了卫越。卫越道:``南贤侄,你怎的落在这厮手中?''南霁云道:``这都是小侄学艺不精之故,有损师门颜面,甚是羞惭。''其实,论武功南霁云并不输于精精儿,他也并非是单打独斗而为精精儿所擒的,只因他生性爽直,输了就是输了,不愿意为自己的如何致败多加辩解。

卫越望他一眼,颇有诧异之意,他知道南霁云之失手被擒,定有内情,当下微笑说道:``胜败乃兵家常事,何足挂齿?好,这事不谈。我早就想到九原找你了,今番幸遇,我先要向你打听一个人。''

南霁云道:``不知老前辈要打听的是什么人?''卫越道:``听说你和冷雪梅的女儿很要好,是吗?''南霁云想不到他要打听的竟是自己的未婚妻子,征了一怔,说道:``不瞒前辈,小侄是和她已有了婚姻之约。''卫越哈哈笑道:``恭喜,恭喜!老叫化也算打听得对了。你可以让老叫化见见你这位未过门的妻子么?老叫化想问她一件事情。''

南霁云本来不愿多说,但卫越已然问及,他一想卫越乃是师傅的好友,说也无妨。便道:``小侄正是刚从夏家出来,我就是在夏姑娘家里碰到了这个精精儿的。''当下将经过情形说了一遍,问道:``老前辈在这里可曾见有王家的人经过吗?''

卫越道:``吓,竟然有这样的事情?你怀疑她们两母女的失踪,是被王家小贼擒去的么?冷雪梅夫妇的武功,当年与段-
璋齐名,凭着她们母女,精精儿即算邀了王家的帮手,至多也不过在打斗中占得上风,绝不至被他们擒厂。''南霁云道:``明枪易躲,暗箭难防,事情实是难以预料。精精儿怎会知道她们的地址,我就想不到其中缘故。''卫越道:``我在这里睡半天,未曾见有任何人经过。不过,若然她们两母女真的落在王家之手,老叫化拼了性命不要,和你到龙眠谷去大闹一场便是。''歇了一歇,又似自言自语地说道:``原来冷雪梅就是住此山脚下。难道传言是实,她约我在这里相会,是有点道理了?''

南霁云好生细罕,问道:``卫老前辈,你说想见复姑娘,问她一件事情,究竟是什么事情?''卫越道:``我是想问她酒丐车迟被害的事情,听说她当年与段-
璋夫妇同上玉树山。车迟的被害,是她曾经目击的!那个凶手的确是西岳神龙皇甫嵩么?''

南霁云道:``这件事她也曾对我说过,她亲自目击,凶手的的确确是皇甫嵩。据说当时车老前辈要向段大侠吐露一件秘密,话未出口,就绪皇甫嵩用毒针暗害了。我的师弟摩勒昨天到了九原,据他说段大侠亦已将这件事情告诉了我们的师父,段大侠的话和夏姑娘的话完全一样,料想是不会假了。''

卫越忽道:``南贤侄,你不忙着走吧?''南霁云道:``卫老前辈有何吩咐?''卫越道:``我与皇甫嵩订下了约会,就在今晚午夜时分,在这个山头相见。我要向他问问这件事情。你若不走,可以听听。''

南霁云本想赶回九原,再图良策。但这件事关系重大,且与夏凌霜有关,他也希望得个水落石出。心里想道:``我的假期未满,这个机会不可错过。''当下说道:``卫者前辈容许我参与这个约会,那是求之不得!''

其时已是夜幕降临,新月初上。卫越笑道:``我被精精儿扰醒清梦,还想补睡一觉。你也歇歇吧。''他靠着山石,不消一会便``呼呼啥啥''的熟睡了。南霁云心道:``订下了这样严重的约会,亏他还有心请睡觉。''

南霁云在日间那场恶斗,身上受破瓷片割伤了几处,趁这空闲的时间,便给自己裹上了金疮药,然后盘膝练功,运气疗伤。他的内功造诣甚深,不消一个时辰,已是疲劳尽去,精神恢复。

月亮将近天心,南霁云的心清也渐渐紧张,轻吉叫道:``卫老前辈,卫老前辈!''卫越翻了个身,坐起来道:``你急什么?皇甫嵩说好了是午夜时分,那就一定依时准来。''南霁云道:``你瞧头上的月亮。''卫越抬头一望,道:``还差一点点时刻。''南霁云道:``山下还未发现人影呢!''

卫越眉头一皱,登上一块岩石。向下方眺望,过了一会,月亮已到天心,交正午夜,卫越``咦''了一声,说道:``奇怪,皇甫嵩从来不是这样的人,怎的会临时失约了?''

月亮渐渐西移,约莫又过了半个时辰,仍然不见皇甫嵩的影子,卫越也有点儿烦躁了,南霁云狐疑满腹,道:``莫非他是不敢见你?''

言犹未了。忽见一条人影,如箭射来,卫越``哼''了一声,道:``这个时候才来,我先要骂他一顿!''心里好生奇怪:``皇甫嵩的轻功怎的如此高明了?''那个人的来势快得难以形容,根本就瞧不清楚他的面目。转眼之间,那个人已到了他们的面前,卫越忽地失声叫道:``怎么,是你!''南霁云定睛一瞧!这才看清楚了来的并非皇甫嵩,而是空空儿!

空空儿侧目斜睨,傲然说道:``你以为是谁?''

论起辈份,空空儿是卫越的晚辈,卫越见他用这样做岸的态度向自己说话,不禁心中有气,冷冷说道:``老叫化等的是另一个人,无须让你知道。你到此有什么事情?''

空空儿冷笑道:``你不说我也知道,你等候的人是不是皇甫嵩?''卫越怔了一怔,道:``是又怎样?''空空儿淡淡说道:``皇甫嵩说你无信无义,这样的朋友不交也罢,他不屑来见你了!''

卫越大怒道:``岂有此理,我怎么无情无义了?''空空儿道:``你听信流言,认定他是杀酒丐车返的凶手,你和他定的这个约会,实在就是想暗算他的,是也不是?但你托人传话给他,却只是说要与他叙旧,这不是骗他吗?你不顾交情,骗老朋友来上当,他骂你无信无义,难道是骂错你了?''

卫越双眼一睁,道:``这话当真是皇甫嵩说的?''空空儿举起手来,他中指上套着一枚铁指环,冷笑说道:``岂有此理,你当是我捏造的么?你认不认得这枚指环?''卫越认得这是皇甫嵩的东西,气得发抖,骂道:``若然他不是凶手,他为何不敢前来见我?却要你这小猴儿前来传活?哼,哼,在此之前,我还不大相信,如今却是不能不信了。''要知他与车迟、皇甫嵩三人并称江湖三异丐,有几十年的交情,如今皇甫嵩却叫一个晚辈来向他说出绝交的话语,怎不令他生气?

空空儿又冷笑道:``你和皇甫嵩之事与我无关,你是否无信无义,我也不管。但你倚老卖老,狂妄自大,我空空儿却不服气,你打伤了我的师弟,这事你总不能赖掉吧?''

卫越须眉怒张,骂道:``空空儿,你才是真正的狂妄,你知道你师弟做了些什么事情?不是看在你们死鬼师父的份上,我还要把他打死呢!''

卫越正要数说精精儿的罪状,空空儿已先发话道:``我的师弟纵然是做了十罪不赦的事,也轮不到你管,你懂不懂得江湖规矩?''

卫越仰天打了一个哈哈,朗声说道:``空空儿,你的眼睛长到额角去啦!休说你的师弟,连你我也要管上一管!不然,我就是对不起你死去的师父!''

空空儿道:``好,你就管吧!你伤了我的师弟,我不给你一点教训,我也是对不起我死去的师父!''他声到人到,身形一晃,倏然间就向卫越扑来!

卫越怒喝道:``狂妄小辈、我倒要看你有多大能力?''反手一掌,隐隐挟着风雷之声。空空儿给他掌力一震,身形一歪,卫越双臂箕张,倏地便向他拦腰一抱,空空儿身法快极,身形一沉一纵,猛的施展``燕子钻云''的绝顶轻功。凭空窜起三丈多高,但听得``嗤''的一声,空空儿的腰带给卫越扯断,卫越左臂一麻,肘端的``曲池穴''亦已给空空儿的手指戳中。

卫越心头一凛,想道:``怪不得他如此骄狂,这副身手果然是比精精儿高明十倍,不逊他师父当年!''连忙默运玄功,舒散气血,手臂的酸麻立时止了。只见空空儿一声冷笑,又再补上前来,说道:``卫老大,你还敢倚老卖老吗?念在你与我师父有点交情,你赔罪吧!''卫越怒极气极,喝道:``小辈如此胆大妄为,今日之事,你给我磕三个响头,我也不能将你放过!''空空儿笑道:``既是彼此都不愿放过对方,那么,咱们只有依照江湖规矩,在掌底再决雌雄了!喂,你邀来的这个帮手,怎么不一齐上来?''

空空儿指的是南霁云,南霁云忍不着发话道:``卫老前辈,请让我领教领教他的高招吧,你老在旁指点指点!''要知南霁云和空空儿是平辈,卫越则是长辈,长辈与小辈动手,胜之不武,不胜为笑。所以南霁云明知不是空空儿的对手,也要挺身而出,甘冒性命之危。

卫越面色沉暗,道:``南贤侄,这事你不用管!我宁愿拼了几根枯骨来整顿武林风气,一身荣辱,倒未放在心上!''

空空儿正是要他这句说话,他深知卫越厉害,但自信还能应付,可是若然加上南霁云,他就没有把握了。当下一声冷笑道:``卫老大,你越俎代庖,欺凌我的师弟,居然还敢口出大言,说什么整顿武林风气?''

他们两人都说得各有理由,按规矩说,卫越发现精精儿不对,该将他交给他的掌门师兄处理,卫越因为自己是长辈身份,根本就未想到这个规矩,不料空空儿竟不卖他这个帐!

当下,两人再度交锋,空空儿丝毫也不客气,拔出一柄短剑,仗着绝顶轻功,竟然欺身进迫,每出一招,都是连袭卫越的九处大穴。

卫越功力深湛,身法却没有空空儿那么矫捷,接连遇了几次险招,勃然大怒,猛然间一掌劈出,以劈空掌力,将一堆乱石打得纷纷飞起,登时便似有无数暗器,向空空儿四面八方袭来,空空儿大叫一声,脚尖一点,立即凌空飞起,短剑挥了一个圆圈,但听得一片叮当之声,乱石纷落如雨!

猛听得空空儿一声长啸,竟自在半空中一个筋斗翻转过来,头下脚上,连人带剑,化成了一道白光,向卫越疾冲而下,卫越舌绽春雷,喝了一个``去''字,在这间不容发之间,一掌拍出!

这一掌是卫越毕生功力之所聚,但听得呼的一声,空空儿已自卫越的头顶疾掠而过,再一个筋斗翻转过来,发出郁雷也似的哼声,也像刚才的精精儿那样,流星殒石般的向山谷坠下,但去势比精精儿快速得多,转瞬之间,影子已没。只听得一个声音从山谷底下传来:``好狠的老匹夫,青山不改,绿水长流,这一掌我记下了,下次还要向你领教!''那声音有些嘶哑,但仍然听得清清楚楚。

这几招兔起鹄落,端的是性命相扑,惊险绝伦,看得南霁云也不禁心惊目眩,这时方始松了口气,但当他抬头一看,却又不禁大惊起来。

只见卫越的衣裳上斑斑血渍,点点殷红,面色如灰,长须颤抖,神情竟是十分颓丧!南霁云急忙奔跑过去,将卫越扶着,问道:``卫老前辈,你,你怎么啦?''卫越叹了口气道:``老叫化第一次栽了筋斗啦。伤倒不碍事,只是我心里难过。''

原来卫越因为空空儿的剑法太狠,迫得以十成功力发出了劈空掌,但他本来无意要空空儿的性命,这一掌虽然劲力十足,但却故意打歪少许,他以为这样亦已可以将空空儿震开,哪知空空儿的功力之高,犹在他意料之上,终于两败俱伤,空空儿受掌力所震,固然受伤不浅,而卫越的肩头,也给空空儿的短剑划开了一道三寸来长的伤口。

这点伤比起空空儿所受的内伤,实在己是轻得多了,可是一来这是卫越生平第一次受到挫折;二来他已是手下留情,空空儿却未察觉,尚在骂他狠辣。要知他与空空儿的师父虽然不是深交,到底也算得是个彼此钦佩的朋友,如今他迫不得已伤了故人的徒弟,故人的徒弟又不谅解他,这怎不教他心痛。

南霁云看出了他受伤不重,见他如此说法,也体会到了他的心情,当下安慰他道:``空空儿目无长辈,狂妄自尊,老前辈对他已算是宽容的了。对这等无理可喻的狂妄之徒,不值得为他伤心、气恼。''

卫越叹道:``空空儿也还罢了,想不到皇甫嵩与我有数十年的交情而今也毁于一旦。更难过的是他这次不敢前来赴约,便证实了他是杀车老二的凶手。我们这三个老叫化本是形同手足,如今为了车老二,只怕我也要横起心去杀他了!''

南霁云心中一动,忽地说道:``刚才空空儿给前辈看的那个铁指环,那个铁指环,\ldots\ldots 嗯,有点奇怪!''卫越怔了一怔,道:``有何古怪?''南霁云道:``那个铁指环我曾经见过,是皇甫嵩的东西。''卫越道:``不错,正是因为我认得这个指环,认得是皇甫嵩之物,所以我才相信空空儿的说话。''

南霁云道:``可是皇甫嵩早已将这枚铁指环送给一个人了。''卫越连忙问道:``送给了谁?''南霁云道:``送给了段-璋。''

正是:信物难凭人事改,疑真疑幻费思量。

欲知后事如何?请听下回分解------

\chapter{第二十回 胡骑肆虐名城坠
壮士挥刀胆气豪}\label{ux7b2cux4e8cux5341ux56de-ux80e1ux9a91ux8086ux8650ux540dux57ceux5760-ux58eeux58ebux6325ux5200ux80c6ux6c14ux8c6a}

卫越甚是诧异,南霁云正想讲这件事的经过,卫越却未说道:``南贤侄,你只知其一不知其二,那铁指环本来是一对的,而且是我送给皇甫嵩的。三十年前我在回疆得到这对铁指环,据说是个土王的宫中之物,功能辟邪,后来流落在一个酋长手中,我对那酋长有恩,他送了给我,我再转送给皇甫嵩的。所以,你不能据此而说空空儿弄鬼。不过,皇甫嵩何以肯将这对铁指环拆开,送一枚给段珪璋,这却是古怪的事情。你和段珪璋相交甚厚,想必知道内里情由?''

南霁云道:``我也知道另有一枚一式一样的铁指环,但那一枚指环,似乎也不该在皇甫嵩手上。''卫越道:``这怎么讲?''

南霁云将他和段珪璋当年被安禄山的武士追捕,段珪璋受了重伤,昏迷不醒,后来在古庙中碰到皇甫嵩,皇甫嵩仗义相助,不但送药给段珪璋,而且助他们打退追兵的事说了。然后始讲到那枚指环的故事,``当时皇甫嵩知道段珪璋不轻易受人恩惠,便除了一枚铁指环,套在段珪璋的指上,那时段珪璋尚在昏迷之中,皇甫嵩就对我说:``拜托你向段大侠求情,日后要是他遇见一个人,那个人带有一式一样的铁指环的话,请他看在我的份上,给那人留点情面。''

南霁云讲完了这个故事,接续说道:``这对指环,一枚在段圭漳手上;另一枚的主人,我虽然不知道,但可以断定,皇甫嵩也早已送给另一个人了,所以我才觉得奇怪。''

卫越这时方始大感惊奇,沉吟片刻,说道:``但我却分明认得这是我当初送给皇甫嵩的指环,决不会假!空空儿从何处窃得这枚指环呢?''

南霁云道:``空空儿的神偷本领,天下无双,嗯,只怕,只怕是\ldots\ldots{}''卫越道:``你担心是段珪璋那枚铁指环给他偷了?若论空空儿的本事。这枚铁指环在谁的手中,他要偷去,也非难事。但是,我和皇甫嵩今晚的约会,只有三个人知道,除了我们两个当事人之外,还有一个就是我差遣去送信的人。''南霁云连忙问道:``那是什么人?''卫越道:``是我最信任的弟子,他决计不会向外人泄漏。除了是皇甫嵩说的,空空儿如何知道?''

两人都觉得此事疑点甚多,当真是百思莫得其解。卫越想了一会儿,说道:``我先回去问一问我的徒弟,要是问不出所以然来,我再到九原见你,帮你寻访冷雪梅母女的下落。''

南霁云碰到这种无头公案,亦自无计可施,心想:``军情紧急,也只有先回转九原再说了。''他谢过了卫越,待到天明,两人便即分手。

南霁云马快,第二日黄昏时分,便回到了九原太守府衙。因为天色已晚,他不想去惊动郭子仪,先回到自己的住所。

铁摩勒听说师兄回来,赶忙出来迎接,远远的就嚷道:``怎么,我的师嫂呢?你怎么不与她一同回来?''一抬头,这才发觉南霁云神色不对。他去时兴高采烈,如今回来,却是垂头丧气,形容枯槁,好像病了一场似的。

铁摩勒吃了一惊,问道:``师兄,这是怎么回事?''南霁云道:``此事话长,到房间里我慢慢和你说。''

铁摩勒听了事情的经过,说道:``这事定然与王家小贼有关,师兄,咱们到龙眠谷去闹他一个天翻地覆!''

南霁云苦笑道:``此地与龙眠谷相距千余里,怎能说去就去?现在军情紧急,咱们都应该听郭太守的将令,不可妄自行动。''

这一晚南霁云思潮起伏,彻夜无眠,心想以冷雪梅母女的武功,应不至于被王家的人轻易擒去;再想到夏凌霜对自己情深义重,即算落在王龙客的手中,也决不会向他屈服,这才稍稍安心。

郭子仪知道南霁云已经回来,天一亮便招他们两师兄弟进入内衙相见,郭子仪老于世故,昨晚听说他一个人没精打采的回来,已猜想到他的婚事定然有了变化,便不再问他到夏家的经过,温言笑道:``国家多难,正是男儿报国之时,家室之事,暂时搁下也罢。南将军。你回来得正是时候!''南霁云连忙问道:``可是军情又已发生了什么变化了?''

郭子仪道:``军情十分吃紧,安禄山因为他的长子被朝廷所杀,发兵猛攻,河南节度使张介然全军覆没,讨贼使封常清的大军未战即溃,望风披靡,现在已退入潼关去了。''

原来安禄山有两个儿子,长子庆宗,次子庆绪,庆绪在范阳协助他的父亲;庆宗是皇帝侄女荣义郡主的郡马,一向住在京师。

安禄山造反之后;杨国忠上奏,说他们父子常常暗通消息,若还留在朝廷,恐有心腹之患,玄宗准其所奏,传旨将安庆宗处死,妻子荣义郡主,亦赐自尽。

安禄山得知消息,大怒道:``你杀了我一个儿子,我就要踏破长安,杀尽你满朝文武!''盛怒之下,纵兵大肆屠杀,所过之处,鸡犬不留。

当时朝廷派出三路大军讨贼,一是新任范阳、平卢节度使封常清,他以所募的六万壮丁,编成新军,在河北正面拒敌;一是大将军哥舒翰,统率胡汉杂编的边军,镇守潼关,作为长安屏障;还有一路,则是河南节度使张介然,统陈留等十三郡,与封常清互为声援。安禄山先攻张介然,陈留太守郭讷开城出降,张介然全军覆没,被安禄山所擒,即行处死。那封常清是个志大才疏的人,所募的壮丁,都是市井之徒,从无训练,安禄山以铁骑冲来,官军不能抵挡,大败而走。

封常清带领残余的几千溃军,退入潼关,依附哥舒翰以求自保。

玄宗闻报震怒,即下手敕,命哥舒翰将封常清斩于军中。

南霁云听得军情如此紧急,登时热血沸腾,将儿女之情,抛之脑后,问郭子仪道:``贼势猖獗,生灵涂炭,我辈岂能坐视;不知朝廷可曾有令许令公出兵?''

郭子仪道:``我正是要和你们商议,朝廷昨日已派有中使前来宣诏。命我为朔方节度使,诏书要我`守御本土,相机出击'。依我之见,贼势正盛,若然只求自保,必为敌人所破,但若贸然出击,敌众我寡,又恐胜算难操。攻守两难,不知南将军有何良策?''

南霁云道:``张太守在睢阳早有准备,令公可以与他联兵。''郭子仪道:``睢阳太守张巡,平原太守颜真卿,这两处地方,我都早已与他们约好了,只是兵力还嫌不够。''

铁摩勒忽道:``我有一策,不知使不使得?''郭子仪道:``一人计短,二人计长。铁兄弟有何良策,但说无妨。''铁摩勒道:``若是有一支奇兵,突然插入敌后,可以事半功倍。''郭子仪道:``此计好是好,可是奇兵从何而来?若是从此地派出,又焉能通得过贼兵数千里的防区?''

铁摩勒道:``郭大人有所不知,今幽州境内有一座金鸡岭,寨主辛天雄与我交情甚厚,此人患肝义胆,是条响当当的汉子。安禄山与王伯通勾结,网罗绿林豪杰,全靠辛天雄出来揭露他们的奸谋,拉住了一班绿林同道,这才不至于全为安贼所用。他知道我来投奔大人,曾对我言道,若有所需,他愿意听从大人的差遣。只不知大人愿意收编黑道上的人物么?''

郭子仪笑道:``只要他有报国之心,论什么黑道白道?老百姓谁不愿意安居乐业,许多人流为盗寇,其实也是迫不得已的,所以我为官以来,对于盗寇,从来都是网开一面,主张用`抚',而不主张用`袭'的。绿林中既有这样的义士,他又愿意为我所用,那自是求之不得!''

铁摩勒大喜道:``如此敢请大人赐予一角文书,给他一个名义,将金鸡岭所部,编成一支义军,纵不能决胜疆场,最少也可以在敌后牵制安禄山的兵力。''

郭子仪沉吟半晌,筹思已熟,说道:``这支义军,初建之时,还得有人策划才行。南贤弟,你是个将才,就请你和铁兄弟代我去走一趟,权委那辛寨主为敌后招讨使,除了金鸡岭之外,凡有愿意改编成义军的绿林豪杰,都一律收容。但望在你的策划下能够打几场漂漂亮亮的胜仗。''

南霁云正合心愿,站起来道:``小将接令!''郭子仪立即写好文书,又将一支令箭交给了南霁云,吩咐他道:``敌后还有许多朝廷的溃军,你也可以将他们收容。我给你这支令箭,让你代传号令,便宜行事。''

南霁云郑重接过令箭,说道:``禀告令公,我此去若能编成一支义军,准备先打龙眠谷,直捣王伯通的巢穴。这样做有两个好处,既可以消灭安禄山的羽翼,又可以趁此号召绿林人物,改邪归正,弃暗投明。王伯通现在号称绿林盟主,若能一举将他打垮,归附他的人,十九可以收编过来。''

郭子仪道:``作战之事,由你全权策划,不必请示。好啦,事不宜迟,你们两师兄弟今天就去吧。我等待你们的捷音!''他携了南霁云的手,亲自将他们送出客厅,并吩咐侍从,给他们备马。

南、铁二人回到住处,整顿行装,铁摩勒笑道:``南师兄,你真该多谢我才成。你怕去不成龙眠谷,现在我已给你请得将令了。夏姑娘要是在龙眠谷的话,你这次就可以演一出勇救佳人的好戏了。''

南霁云笑道:``你别说我,你不是也可以趁此机会与韩姑娘更亲近了么?你放心,你若是要在路上和她说些情话,我决不会偷听你的。''

原来韩芷芬到了九原之后,郭子仪的夫人很喜欢她,请她人府作伴,与官眷同住,官宦之家,内外隔绝,因此铁摩勒反而不能时常和她见面了。这次郭子仪派他们师兄弟二人前往金鸡岭,说好了让韩芷芬也和他们一同回去。

铁摩勒给师兄取笑回来,不觉面红过耳,连忙说道:``师兄,这个玩笑你可不能乱开,你和夏姑娘已订了婚,我和韩姑娘只是兄妹相称。''

南霁云笑道:``这个我是过来人,我当初也是和夏姑娘兄妹相称的。''

两师兄弟正在谈笑,韩芷芬已经来到,一进来便笑道:``摩勒,你出的好主意,我在府衙里和那些夫人们作伴,正闷得发慌呢!喂,听说你们准备先打龙眠谷,是么?''

南霁云道:``正是。韩姑娘,你有何高见?''

韩芷芬笑道:``休说高见,浅见也没有。我只是有得厮杀便欢喜。王伯通那女儿尚欠我一掌,我正想去讨还呢。''

南霁云道:``好呀,这次你有机会可以和她再较量了。王家那两兄妹都不是好人。我巴望你一剑将她刺个透明窟窿。''

韩芷芬望了铁摩勒一眼,似笑非笑地说道:``这我可不敢,杀了那位王姑娘,拿什么赔给摩勒?南大哥,你不知道,那位王姑娘对摩勒可是真好呢!''铁摩勒又羞又急,叫道:``芷芬,我不是对你说过了么?不管她如何待我,她总是杀我义父的仇人!''

韩芷芬见他认起真来,笑道:``你要是没有心病,何用如此着急。好啦,不说你了。马已备好,咱们可以动身了。''

他们三骑马同出府行,轮值守卫的军官有些奇怪,问道:``南将军,你昨天才回来,今天又要走了?什么公事,这样来去匆匆?韩姑娘,你也走啦?''南霁云因为事关秘密,不愿与他多说,敷衍两句,立即策马登程。

秦襄那匹黄骠马仍由韩芷芬乘坐,南、铁二人的坐骑则是郭子仪给他们挑选的骏马,虽然比不上那匹黄源马,亦是雄健非凡,不过一个上午,便走出了百余里路。

一路上他们不免以龙眠谷作话题,说起了七年前他们大闹王家``庆功宴''之事。铁摩勒忽地似乎想起了什么,突然勒住了马。

南霁云问道:``怎么?你的马跑不动了吗?''

铁摩勒道:``不是。我是在想,我们要不要再赶回九原去?''南霁云道:``为什么?''铁摩勒道:``我想起了一件事情。''韩芷芬笑道:``甚么事情,大惊小怪的?已经走了这许多路了,还要回去?你边走边说吧,让南大哥替你参商。''

铁摩勒道:``南师兄,刚才在府衙门口,向你问话的那个人,他叫什么名字?''南霁云道:``名叫贺昆,怎么,他有甚么不对?''铁摩勒又道:``我初到九原那天,你们正在内校场操练,这个贺昆也在其中,我记得他还是三箭都中红心的,是么?''南霁云道:``不错,在校尉中他的箭法算是好的。你认得他?''

铁摩勒道:``那天我在校场中见到他,就觉得有点面熟,刚才你们提到了当年咱们大闹龙眠谷的事情,我突然想起来了,这个人我是在龙眠谷里见过的。只因当时人太多了,我一时想不起来。''

南霁云吃了一惊,道:``真的?你记得清楚,没有认错?''铁摩勒道:``绝不会错。你记得吗?那天我是冒充辛寨主的小厮,你们在园中饮宴,我却在马房里和下人们一起吃饭。他就是和我同桌吃过饭的。其他人有说有笑,只有他一声不响,所以我反而特别记得他了。你想,若然他是王伯通的人,让他留在军中,岂不可虑?''

南霁云问道:``当时和你同桌吃饭的人,都是王伯通的仆役吗?''铁摩勒道:``也有各寨主的随从,和我一样身份的人。''

南霁云沉吟半晌,说道:``自从郭令公知道安禄山有造反的迹象之后,便出榜招募勇士,广纳人材。据我所知,这个贺昆,便是第一批应募来的,他为人谨慎,也颇忠于职守。现在,我们既不能断定他是王伯通的人,又未曾拿着他甚么把柄,要是贸贸然回去告发他,那岂非小题大作了?''铁摩勒道:``咱们只是告诉郭令公一人。''南霁云道:``但是咱们这一去而复回,别人就不会起疑吗?若然他真是坏人,反而打草惊蛇了。不如这样吧,这里还是九原郡的地界,我到了前面的卫所,再写一封密信,请他们快马送回去。禀告郭令公,请他加意提防,也就是了。这些卫所和府街经常有公文来往,别人不会起疑。''

铁摩勒觉得师兄的话有理,不再坚持回去。他们马快,不过一个时辰,便到了前面的卫所,南霁云写了封信,用火漆封了口,交给卫所的军官。那人是认得南霁云的,答应当天给他送到。

离开了卫所,一行人再向前行,三天之后,就进入了安禄山管辖的地区。

路上不时碰见扶老携幼的走难的人群,当真是哀鸿遍野,触目凄凉;也不时碰见溃败的官兵和安禄山追袭的部队。幸而他们的坐骑,都是久经训练的战马,登山涉险,如履平地,一碰见军队,就绕道避开,从未生事,一路平安,到达了金鸡岭。

寨中闻报,寨主辛天雄以下,都出来迎接,韩芷芬忽见人丛有她的父亲,这一喜非同小可,急忙连蹦带跳地跑过去,叫道:``爹,你回来了?''

韩湛拉着了女儿笑道:``我早知道你这不安份的性儿,总喜欢找些事情,叫别人操心。我前天回来,听辛叔叔说你偷偷跑了,几乎把我吓了一跳。''韩芷芬噘着嘴儿道:``辛叔叔,你为什么这样说我?我上次离山,不是禀告过你的吗?''辛天雄笑道:``我和你爹开开玩笑,你这样着急做甚么?哈,你那一天呀,跨上了黄骠马,这才告诉我,那副急着要走的神情呀,我现在想起了还觉得好笑,你想,我敢不答应你吗?''

韩湛哈哈笑道:``幸亏你是和铁贤侄同走,要不然我可真不放心呢!''转过头来,和南霁云招呼之后,又拉着铁摩勒道:``铁贤侄,你长得这么高了,真是个年少英雄,令人高兴。''他一手拉着女儿,一手拉着铁摩勒,弄得铁摩勒甚感难以为情,南霁云瞧在眼里,心中想道:``他们的好事料想能谐了。但愿他们不致像我这样多受折磨。''

南霁云和众人见过,发觉山寨中除了韩湛之外,又多了几个人。``金剑青囊''杜百英和陕南著名的游侠符凌霄也都在内。南霁云与他们相交甚厚,阔别多年,当下重新施礼见过,问将起来,始知韩湛前次下山,一来是到各地访友,二来也是为了金鸡岭招揽英豪的。金鸡岭和龙眠谷距离不远,韩湛早已料到有安禄山之变,所以为山寨未雨绸缨,准备应付龙眠谷的挑衅。

辛天雄道:``目下军情紧急,怎的你们却在这个时候离开九原,郭令公也肯放你们走呢?''南霁云道:``正是要与你们共商大计,咱们进去慢慢再谈。''

群豪当日就在聚义厅里商谈,南霁云将郭子仪的委任状交给了辛天雄。提出要将金鸡岭的部属编成义军,又将自己准备先打龙眠谷的计划说了,辛大雄欣然同意,说道:``韩老前辈对龙眠谷的地形最熟,要攻取龙眠谷,他是最好的军师。''当下,经过了反复研讨,定下了一条夜袭龙眠谷之计,准备布置妥当之后,便是三天之后动手。

金鸡岭为了怕龙眠谷偷袭,本来就在龙眠谷附近设有``坐探''。龙眠谷是个葫芦形的地盘,四面高山环绕,谷中有百里方圆之地,原住有一些采药的山民与猎户,谷外边也有几个村落,王家父子占据了龙眠谷后,大兴土木,修筑武备,已把龙眠谷变成了一个硕大无朋的碉堡,但江湖大盗有一条规矩是不吃``窝边草'',王家以绿林盟主自居,当然更不会向这些村民动手。谷中原有的药农和猎户,虽然被强迫人伙,要替他们做事,但还是各守本业,不过要将采种所得的草药和打猎所获的野兽缴给山寨,每月领回一份钱粮,等如为山寨所雇一般。至于谷外边的村民,则只是要服从他们的管辖,其他并无改变。

金鸡岭的``坐探'',便是当年铁摩勒在那里吃过酒的那个茶亭主人。那个茶亭距离龙眠谷不到三十里,他在谷中有几个亲戚,故此对龙眠谷的消息颇为灵通,金鸡岭也不时派出``行探'',以走亲戚为名,打听龙眠谷的虚实,每过一个时候,便到金鸡岭回报。

第二日恰巧便有探子回来,报说王伯通父子都在谷中,而且谷中张灯结彩,四处粉饰一新,各地山寨,连日有人前来,好像要办什么喜事似的。

这消息在辛天雄听来,并不觉得什么特别,但在南霁云听来,却不免疑虑丛生,心想莫非是夏凌霜母女真的已给王家掳去而王龙客要迫夏凌霜成婚?他既盼望她们两母子是落在龙眠谷,自己可以救她们出来,又担心她们会遭意外,听了这个消息,两个晚上都没有睡过一个好觉。

南霁云的猜疑有一半对了,夏凌霜的确是已落在王龙客之手,但她的母亲却并非和她一道,下落如何,连夏凌霜也不知道。

就在金鸡岭准备向龙眠谷动手的那个晚上,王家一间布置得很雅致的房间里,有一个少女,躺在床上,她想挣扎起来,但身子却是软绵绵的,一点气力也使不出来。这个少女便是夏凌霜,她被安置在这房子里已有好几天了。

她咬了咬牙,气得眼睛发黑,那一场恐怖的遭遇,又一次在她脑海中重现出来。

那一天,她正在陪母亲闲话,心中老是在惦着南霁云,她计算日子,南霁云在这一两天内应该来了。心念未已,忽听得外间声响,她欢喜得几乎要跳起来,刚刚要去开门,那一伙人已闯了进来,大大出乎她的意料之外。

闯进她屋子里的共是四个人,第一个是精精儿,第二个是王龙客,第三个是个身形瘦长、相貌古怪的道士,只有这个人她不认识;第四个人,最出乎她的意外,那是西岳神龙皇甫嵩!

她永远也不会忘记那一刹那的情景,当皇甫嵩一出现的时候,她母亲突然尖叫一声,面色全都变了,那神情就似碰着了恶鬼、碰着了野兽一般!那叫声充满了愤怒、充满了恐惧,又似孤立无援的人,遇到危险时绝望的呼喊!她与母亲相依为命,过了二十多年,从未曾见过母亲这样愤怒的神色,听过这样恐怖的叫声!

她记得她本能的立即便跳起来,拔剑便向皇甫嵩刺去。突然,她闻到一股古怪的香味,剑招发出,一点劲道也没有,就像饮了过量的酒一般,头晕、目眩,身子软绵绵的,只想倒下床去睡觉。神智模糊中,她发觉王龙客到了她的身边,在这时候,她还隐约听得母亲叫了一声,似乎是冲着皇甫嵩喊道:``我不许你对霜儿说半句话!''接着,似乎还听到几声刀剑碰击的声音,之后,她就失去了知觉\ldots\ldots{}

待她恢复了知觉之后,已经是在这间房子里了。她发现身体并无异状,这才稍稍安心,可是气力仍然未曾恢复,只能躺在床上,一点办法也使不出来。她被安置在这房子里,已经有好几天了,王龙客也来过好几次,每次都给她骂了回去。

夏凌霜正在苦恼,忽见门帘揭处,王龙客又走了进来。

夏凌霜气得咬紧银牙,转过身去,不理睬他。却听得王龙客柔声笑道:``过了这许多天了,你的气还未消么。都是我的不好,未曾先得到你的允许,就把你带到这里来。可是,这也是由于我太喜欢你了,你应该原谅我呀。嗯,你的胸口还在感到发闷么?我一时不能给你解药,不过,我今天给你带来了一些龙诞香。可以提神醒脑,你闻一闻这香味,是不是舒服了一些?''

氤氲的香气散人帐中,夏凌霜果然觉得精神一爽,只听得王龙客又道:``夏姑娘,你要我做什么都可以,只要你说一句话呀!''

夏凌霜恼怒之极,叫道:``你别假献殷勤,装模作样啦,我宁愿你一刀把我杀掉!''王龙容笑道:``你怎的这样恼我?我请你到这里来是为了杀你吗?你放心,我宁愿自己死了也不忍伤害于你。我对你说的,句句都是出自真心。''夏凌霜转过面来,怒声说道:``好,你说得这么好,为何不让我见我的母亲?''

王龙客摇了一下折扇,柔声说道:``你母亲不在这里,可是,只要咱俩成婚之后,你自然会见着她。''夏凌霜怒道:``你好无耻,要拿这个来胁迫我么?''王龙客道:``夏姑娘,我是诚心诚意向你求婚,你可别生误会。你妈妈另有去处,她暂时不想到龙眠谷来。可是,只要咱俩一成了婚,她老人家自然要赶着来见女儿女婿的。''

夏凌霜气得粉脸通红,柳眉倒竖,``哼''了一声道:``你要迫我成婚,那是癞蛤蟆想吃天鹅肉,我夏凌霜纵使粉身碎骨也决不能嫁你!''

王龙客在她面前,本来一直是装作多情公子的模样,温柔体贴,服侍殷勤,如今听了这话,不由得面色大变,折扇狂挥,过了半晌,冷冷说道:``夏姑娘,你也不想一想,若然我真是你所说的癞蛤蟆,这块天鹅肉我早已吃到口了。你已然落在我的手中,我要怎样摆布你都可以。就因为我敬你爱你,想和你做一双你情我愿的恩爱夫妻,所以才不用强横的手段对你。夏姑娘,咱们总算也有过一段交情,你为何这样恨我?''

夏凌霜道:``我早有了未婚夫了,你又不是不知道,你明知我与南霁云订有婚约,还把我掳到这里来,这不是存心欺侮我么?你若要讲交情,快快把我放走,也许我可以少恨你一些。''

王龙客为了赢得她的芳心,本来打定主意,用水磨功夫,任凭她如何辱骂,也不发作。但如今听她提起了南霁云,王龙客这可忍不住了,只见他面色铁青,折扇``卜''的掉下地来,张开口便嚷道:``我有哪点不如这姓南的地方?他不过是郭子仪部下一个小军官,有什么出息?他只知刀来剑往,在江湖上浪得虚名,不解温柔,不懂情趣,有何值得你如此倾心?再说,我认识你也在他认识你之前,咱们也曾有过一段交情不错的日子,你移情别恋,我王龙客岂肯甘心?''

王龙客咆哮如雷,夏凌霜反而沉默下来,一面听他说,一面想起了往事。七年之前,她初出江湖,有一次她在路上碰见一队军官,那军官见她美貌,想调戏她,她正要动手,却有一个过路的少年,将那军官喝住,给她解了围,这少年便是王龙客。当时夏凌霜不知他的身份,还以为他是个仗义扶危的贵家公子,见他一表斯文,谈吐风雅,文才武艺,两皆不错,对他的确也曾暗暗倾心。

那次事情过后,两人就此缔交,结伴同行,经过一些日子。夏凌霜初出江湖,毫无经验,王龙客随时给她指点,又曾助她诛除了一个贪官,两个恶霸,夏凌霜更以为他是个少年游侠,好感日增,不过,时日无多,尚未至谈婚论嫁。不久,王龙客因为他家与窦家争霸之事,迫得离开了夏凌霜,匆匆赶回龙眠谷去。夏凌霜一直未知他的身份。

直到王龙客在乱石岗截劫段珪璋,被南霁云打败,而这件事情,又恰巧被夏凌霜碰上,从此之后,王龙客的真面目渐渐揭开。待到群雄大闹龙眠谷,王家与安禄山勾结的奸谋全被揭穿之后,夏凌霜对王龙客也就完全绝望了。

往事一幕幕的从夏凌霜脑海中翻过,这时王龙客还在她的床前指手划脚,愤愤不平,喋喋不休;夏凌霜突然仰起头来,冷冷说道:``不错,你根本不能与南霁云相比!''

王龙客怔了一怔,大声问道:``我怎么不能与他相比,我是绿林的少盟主,叱咤风云,正图霸业,他是什么东西?''

夏凌霜道:``他是行侠仗义,解困扶危,为国为民的好汉子!你勾结胡儿,残害百姓,根本就不是一个东西,又怎能与他相比?''

王龙客怒极气极,但他双眼一瞪,反而哈哈笑道:``你这真是妇人之见。你可曾读过史书么?''夏凌霜道:``我是比较你们两人的行事,这与史书何关?''

王龙客拾起扇子,摇了一摇,极力压下心头的怒火,放缓声音说道:``你不是认为我勾结胡儿乃一桩大罪么?你可知道历朝创业之君,借助外援,取得天下之事,史不绝书?你即算未读过史书,谅也当知道本朝之事,当年李渊父子与各路反王逐鹿中原,李渊就曾向突厥称臣,他派刘文静做使者,上表突厥可汗,约定`征伐所得,子女玉帛,皆可汗有之。'因而得到突厥之助,后来李渊也就成了本朝的高祖皇帝。我如今与安禄山连结,也不过是效法李渊所为,暂时借助于他而已。事成之后,我也可以将他诛灭,独占唐朝天下。哈哈,那时我就等如太宗皇帝李世民一样,是开创一代的君王了。你怎知我的抱负?你因此骂我,这岂非妇人之见么?''

王龙客能言善辩,引古证今,满以为可以将夏凌霜压服,哪知夏凌霜冷冷一笑,状更鄙夷,说道:``哎哟,真是失敬,原来你还有这样的抱负!小女子未曾熟读史书,但只知道一条道理:残害老百姓的便是十恶不赦的坏人,认贼作父的便是国人皆曰可杀的国贼!''

王龙客用尽诸般手段,软硬兼施,不料非但赢不到夏凌霜的芳心,反而招来一顿臭骂!虽然他以前也曾挨过几次骂,但却从无一次被骂得这样厉害,这样决绝,简直毫无可以转圈的余地!

王龙客面色铁青,双眼火赤,老羞成怒,蓦地跨上一步,狞笑说道:``好呀,原来我在你的眼中,竟是十恶不赦的坏人,那我还能和你说些什么,我只能用坏人的手段对付你了!哈,哈,夏姑娘呀,你这是敬酒不吃吃罚酒了!''

他站在床前,俯下腰来,双臂一伸,就要向夏凌霜搂去!

夏凌霜动弹不得,冷冷说道:``好,好威风!呸,你简直是不要脸的下流胚!''王龙客自视甚高,被她这么一骂,又是恼怒,又是羞惭,眼光相接,但觉夏凌霜的眼光中充满了鄙视、憎恨、而又冷傲的神情,王龙客禁不住心头一凛。本来夏凌霜已是毫无反抗的力量,但不知怎的,王龙客面对着她那股凛然不可侵犯的神情,却忽地心虚胆寒,双臂悬空,竟然不敢搂下!

王龙客咬了咬牙,无法下台,又舍不得离开,正在人天交战,心意踌躇的时候,忽听得一声冷笑,声音极轻,但却清清楚楚,就似有人在耳边耻笑他似的。他望了望夏凌霜,夏凌霜躺在床上,双目圆睁,向他怒视,但嘴唇却是闹得紧紧的,显然这不是夏凌霜所发出的笑声。

王龙客喝道:``谁在外处?''没人回答,但却又传来了一声冷笑,王龙客本已有些怯意,再听了这声冷笑,不由得他不放开了夏凌霜,立即便揭帘奔出。

夏凌霜松了口气,心里暗暗道声:``好险!''那两声冷笑她也听到了,她既庆幸那冷笑来得及时,同时又感到奇怪之极。

过了片刻,忽又听得有脚步声从外面走来,夏凌霜惊魂方定,不由得又吓了一跳,只道是王龙客去而复回。

一个苗条的影子一闪而进,夏凌霜定睛一看,却是王龙客的妹妹王燕羽。

虽然来的不是王龙客,但夏凌霜恨透了王家的人,对王燕羽当然亦是全无好感。她冷冷地望着王燕羽,一言不发,但见王燕羽面上却是堆着笑容,对她似是并无恶意。

王燕羽见着夏凌霜这副神情,怔了一怔,但脸上仍然挂着笑容,走上前来,对夏凌霜说道:``夏姐姐,我哥哥对你无礼,怪不得你心中气恼。小妹特来向你赔罪!''

夏凌霜冷笑道:``你哥哥刚刚被我骂得夹着尾巴逃了,你又来要什么花招?哼,哼,你们两兄妹一个做好,一个做坏,骗得过我么?''

王燕羽道:``姐姐,请勿多疑,我是诚心诚意来给姐姐赔罪,非但如此,我还想为我的哥哥赎罪!''

夏凌霜道:``吓,你要为他赎罪,如何赎法?对啦,我早已听说你是个杀人不眨眼的小魔女,你就拿出你当年刺杀窦家五虎的本事,将我一剑杀了吧,省得我活着受你们的折磨,也省得我睁开眼睛就要对着你们这班讨厌的东西。''

王燕羽变了面色,忽地两颗泪珠滴了下来,低声说道:``当年我杀了窦家五位伯伯,乃是奉父命而为,现在想来,已是后悔不及。但是窦家五位伯伯也有可死之处,不过,不应由我来杀他们就是了。姐姐,这件事情你也不能原谅我么?''

夏凌霜对窦家五虎本来亦无好感,不过是信手拈来举例罢了,听她这么郑重的辩解,倒觉得有点奇怪,当下忍不住说道:``你不必猫哭老鼠假慈悲啦,你杀了他们,后悔也好,得意也好,与我毫无关系。你干脆说吧,你哥哥差遣你来,意欲如何?不过,我可以斩钉截铁地告诉你,软的硬的,我全都不受!不论你用的是刀剑毒药,或者甜言蜜语,想我依从,那只有白费心机!''

王燕羽道:``我是他的妹妹,你不相信我,那也难怪。但是,我可并非我哥哥差遣来的,你问我意欲如何?我到此间,为的就是想助你逃走,这样,你可以相信我了吧?''

夏凌霜愕了一愕,道:``你要放我逃走?咦,这对你有什么好处?我与你也够不上这个交情!''

王燕羽道:``你一定要知道对我有什么好处,才能相信我的诚意吗?好吧,那我就告诉你。我知道你是南大侠的未婚妻子,我但求你们破镜重圆之后,你在南大侠跟前,能为我美言两句。''

夏凌霜道:``咦?这更奇怪了。你要我向他说些什么?''王燕羽脸上忽然泛起一片娇红,羞涩涩地说道:``只要你说出这件事情的经过,让南大侠明白我也并非坏得难以救药之人,那就行了。''

饶是夏凌霜心窍玲珑,一时之间,却也难明其中缘故,心里只是想道:``为什么她要求得我南大哥的好感?为什么她又是这等神情?''要不是她对南霁云素来信任,又知道他们二人向无关联,几乎会疑心其中另有隐情。

夏凌霜正在猜疑,只见王燕羽己掏出一个银瓶,盛着十瓶淡红色的液体,低声说道:``你是中了千日醉迷香散的毒,这是解药,我从哥哥那儿偷来的。''

夏凌霜半信半疑,说道:``你偷了解药给我,不怕你父兄责怪么?''王燕羽道:``你不必管我,你快些吃了解药,早早逃跑吧。要是哥哥发觉我偷他的解药,你就逃不成了!''

夏凌霜见她神情焦急,似乎恨不得自己马上就把那解药服下,反而又多了两分猜疑,冷冷说道:``这么说来,你竟然为了一个不相干的外人,和你哥哥作对了。嘿嘿,想不到你心地竟是如此善良,老虎也会念大悲咒了!''

王燕羽急道:``你要怎样才相信我?唉,你不知道,我,我是------''夏凌霜睁圆双眼问道:``你,你是为了什么?''

就在此时,忽听得有个声音叫道:``小姐,小姐!''这是王燕羽贴身丫鬟在呼唤她,声音急促,似乎出了什么事情。

王燕羽吃了一惊,将那银瓶扔到夏凌霜身边,气道:``好,你不相信我,我也没法。服不服药由你!你不是要寻死觅活么?好,你就当它是一瓶毒药吧!''

王燕羽匆匆走了,夏凌霜目送她的背影,又瞧瞧那个银瓶,王燕羽临走时那股神气,那股又是焦急、又是愤激、又是受了无限委屈的神气,一个少女似乎不可能矫揉造作得来。夏凌霜蓦地里心中想道:``她说得对,就算这是一瓶毒药,我最多也是一死而已,服了它决不会比现在半死不活的情形更坏。''她不能爬起身来,但双手还能缓缓移动,她挣扎着拿起银瓶,打开瓶塞,闻得一股芳香,登时精神一爽,终于把那半瓶药酒倒入口中。

王燕羽出来见着了那个丫鬓,急忙问道:``你可有碰见我的哥哥?''那丫鬟道:``少寨主已经走出前厅去了。听说是来了客人。''连日间都有绿林人物来到,王燕羽也不放在心上,便问道:``你大呼小叫的找我,有什么事情?''那丫鬟道:``杨总管传下老寨主的命令,叫小姐也去会客。杨总管已经找过你一趟了。''王燕羽有点诧异,心中想道:``什么重要的客人?我爹爹亲自招待,又有我的哥哥,为什么还要我也出去?''当下说道:``好,我就出去。我到过此间,你不可说给别人知道。''

王燕羽走出前厅,先在屏风后面一瞧,这一瞧不由得心头一震!

来的这两个人,可并非什么绿林人物,而是王燕羽所认识的人------名震江湖的段珪璋夫妇。

段珪璋是窦家的女婿,王家大破飞虎山,灭了窦家五虎之后,本来就准备他们夫妇要来寻仇。但是,经过了七年,他们夫妇的足迹始终未曾踏进过龙眠谷,王伯通父子,也以为他们不会来了,哪知他们却突然在今晚出现!

王燕羽恍然大悟,心道:``怪不得爹爹催我出来会客,原来是这样的客人,糟糕,要是他们动起手来,我可怎么办呢?''段珪璋与铁摩勒的关系,王燕羽是知道的,要是段珪璋果然是为了报仇而来,王燕羽就难以避免要和他们对敌了。她心头大乱,躲在屏风背后,不知如何是好?

这里,王伯通正在与段珪璋说话,他也以为段珪璋是为窦家报仇来的。王燕羽从屏风背后,偷瞧出去,只见她父亲面挟寒霜,冷冷说道:``请问段大侠,贤伉俪今晚大驾光临,是路过还是特到?''段珪璋道:``无事不登三宝殿,当然是有事才来!''

王伯通冷笑道:``君子一言,快马一鞭,请问段大侠当年在飞虎山上说过的话还记得么?''段珪璋道:``我说过些什么话了?''王伯通道:``当日我在飞虎山与窦老大评理,段大侠不是绿林中人,曾说过不管王、窦二家之事,后来贤伉俪与空空儿按武林规矩较技,段夫人也曾应允,或胜或败,只是与空空儿理论,不向王家寻仇,这话你们可是说过的么?''

段珪璋道:``一点不错,这些话都是有的。''王伯通松了口气,道:``好,既然如此,想来段大使当是个重言诺。守信义的人,我也似乎不必再多说了!''

段珪璋沉声说道:``王寨主怎的未曾动问,便一口咬定我是为了给窦家报仇而来呢?难道除了这件事情,我段珪璋就不能来么?''

王伯通愕了一愕,随即打了一个哈哈说道:``对不住,这是老夫误会了。多承段大侠把老夫当作朋友,肯到寒舍,真是何幸如之!龙儿,端上茶来。''

段珪璋冷冷说道:``且慢,这碗茶吃不吃也罢。王寨主,你还是误会了。''王伯通道:``怎么?''段珪璋道:``愚夫妇今晚前来,一非寻仇,二非访友。我怎敢高攀作王寨主的朋友呢?''

王伯通连忙问道:``那么段大侠前来,端的是为了什么?''

正是:旧仇今又添新恨,虎穴龙潭亦等闲。

欲知后事如何,请听下回分解------

\chapter{第二十一回 挑起谷中龙虎斗
可怜剑底女儿情}\label{ux7b2cux4e8cux5341ux4e00ux56de-ux6311ux8d77ux8c37ux4e2dux9f99ux864eux6597-ux53efux601cux5251ux5e95ux5973ux513fux60c5}

段-璋盯了王龙客一眼,说道:``我有一位故人的女儿,被少寨主掳到此间,敢请放回!''

王龙客怔了一怔,骂道:``胡说八道,我几曾抢了什么女子?''段-璋变了面色,手摸剑柄,便要发作,王伯通却先喝道:``龙儿,在段大侠面前,休得放肆!''随即转过身来,向段-
璋赔笑说道:``小儿一向跟在我的身边,他纵然不肖,尚不至于干出强抢民女的有失身份之事,段大侠想必是误信人言了。''

王伯通老奸巨滑,这时他已知道了段-璋是为了夏凌霜而来,心中惊疑不定,因此先用巧言搪塞,能抵赖得过最好,即算不能抵赖,也可以试探段-璋还知道些什么?

段-璋剑眉一竖,怒声说道:``段某若非知得确鉴,怎敢上你的龙眠谷来?这位姑娘名叫夏凌霜,你问问你的宝贝儿子,是否认得这位夏姑娘?''

王龙客道:``不错,我是认识这位夏姑娘,她也是我的朋友,你有何凭据,说是我把她抢了?''

王伯通帮腔道:``对呀,他们本来是朋友,不相识的人还可以抢,对相熟的朋友,怎会将她掳来?尽可以邀请呀。''

段-璋冷笑道:``不给你们凭据,谅你们还要狡辩。上月二十七日,你们在玉龙山的沙岗村掳去她们母女,本月初四,夏姑娘一人被劫到龙眠谷,当时,她中了迷药,你的儿子用一顶小轿,将她从花园右角的横门抬进,是也不是?''

段-璋说来有如目睹,王伯通父子大吃一惊,登时疑云大起,``龙眠谷中难道有了奸细不成?''

段-璋顿了一顿,朗声说道:``夏姑娘的父亲与我有八拜之交,她又是我好朋友南霁云的未婚妻子,这件事我不能不管!''

王伯通尚想抵赖,尚想问他要人证物证,王龙客却忍不住气,大声说道:``段-璋,你胡说八道,夏姑娘是我的未婚妻子,与什么姓南姓北的何干?不错,她现在是在谷中,日内我们就要成婚,你客气一些,我或者还可以请你喝杯喜酒,你再胡说八道,我只有把你轰出去了!''

段-璋冷笑道:``好呀,你这么说,好似夏姑娘愿意嫁给你的了?''王龙客傲然答道:``当然!她又不是你的女儿,她愿意嫁我,你管得着么?''窦线娘勃然大怒,骂道:``放屁,夏姑娘岂肯嫁你这个不成材的小贼!''段-璋道:``不必争辩,夏姑娘既在此地,请她出来,一问就可明白!''

王龙客骂道:``岂有此理,我的未婚妻子,岂能随便见你!''窦线娘恨不得立即闹翻动手,说道:``大哥,证据确凿,夏姑娘也在此间,还与这班强盗多说作甚?他不肯让咱们见复姑娘,咱们不会自己搜吗?''

王伯通大喝道:``王某忝为绿林盟主,请两位给些面子!''他不提``绿林盟主''这四字也还罢了,一提起来,窦线娘想起了杀兄之恨,更有如火上烧油,立即冷笑斥道:``我管你什么盟主不盟主,你胡作非为,我就要与你算帐?''

王伯通把手一挥,沉声说道:``好,与他们拼了,他们是藉事生端,分明是为了给窦家报仇来的!''嗖的一声,一枚铁蒺藜向窦线娘掷出,出手的人,是王伯通一个得力手下,此人擅打喂毒暗器,他知道窦线娘金弹厉害,故而先发制人。

窦线娘冷笑道:``什么东西,竟敢在我面前卖弄暗器,且先把你的招子废了。''话声未了,但听得弓弦疾响,那人一声惨呼,血流满面,两只眼珠果然都给窦线娘的弹子打了出来,紧接着``卜''的一声,又一名头目倒地,这个头目却是给那枚毒蒺藜打中的。原来他发暗器的劲力和准头都远不及窦线娘,窦线娘的金弹后发先至,将他的眼睛打瞎之后,这才用弓弦把那枚毒蒺藜拨开,那小头目不幸碰上,中了剧毒,不消片刻,便即七窍流血而亡。

窦线娘弹弓再拽,这一次三弹齐发,迳打王伯通的上中下三路,王伯通躲过一颗,王龙客手挥折扇,给他拨开一颗,第三颗打向他的面门,王伯通霍地一个``凤点头'',哪知窦线娘的暗器手法妙极,王伯通见金弹的来势极急,避得早了一点,不料那金弹将到,来势忽缓,王伯通抬起头来,正巧碰上,额角打裂,血流如注!王伯通大怒骂道:``给你们面子,你们反而出手伤人,今日要是让你们生出此门,我王伯通也无颜在绿林混了!''

在王伯通背后的一个胖和尚叫道:``盟主息怒,待我收拾这个泼婆娘!''抖起禅杖,疾奔出去,朝着窦线娘迎头便打,窦线娘喝道:``好,叫你这光头也吃几颗弹丸!''声出弹发,那胖和尚哈哈笑道:``米粒之珠,也放光华?你这弹子,焉能打得酒家?''禅杖泼风疾舞,当真是滴水难进,但听得噼噼啪啪一片声响,窦线娘的连珠弹尽都给他打落,碎成粉末!

段-璋一见,便知这个和尚内力雄浑,不能硬接,他怕妻子有失,猛地喝道:``撒手!''一剑便削过去。

这和尚名叫阿奢黎,乃是与安禄山同族的胡人,本来是安禄山所礼聘的``大法师'',甚得安禄山信任的。后来安禄山因与王伯通联盟,故而将他派来,名义上是``荐贤''给王伯通,由王伯通使用,实则是替他负起监视王伯通的任务。安禄山的用意王伯通当然不会不知,故而对他十分笼络,处处奉承。

阿奢黎给他们奉承惯了,只道自己当真是天下无敌,他见王伯通似乎很怕段-璋夫妇,早就心中不服,因而争着出头,满以为一顿泼风禅杖,便可以将这对夫妇打倒。

哪知段-
璋剑法精妙非常,但见剑光一闪,已攻进他禅杖防御的内圈,阿奢黎大喝一声,禅杖压下,段-璋用了个``卸''字诀,那柄宝剑竞似轻飘飘的木片一般。附着他的禅杖,阿奢黎虽是用了泰山压顶之力,却似大力士搬石头打蚂蚁一般,毫无用处,给他的宝剑附着禅杖,竟自摆脱不开。

说时迟,那时快,段-璋一声:``撒手!''宝剑便沿着禅杖,直削上去!阿奢黎大吃一惊,要是不抛开禅杖的话,五根指头,便得给他削断。他人急智生,急忙将禅杖往前一送,自己跟着一个``滚地葫芦'',伏倒地上,躲开了他这一剑。

王龙客亦已赶到,折扇一挥,替阿奢黎遮格开了段-璋的一剑。王龙客自小便在名师门下习技,功夫也是内外兼修,且又机智多变,因此,他比起段-
璋南霁云等人,虽然尚逊一筹,却不至于似阿奢黎一招落败。

阿奢黎爬起身来,吓出了一身冷汗,他的禅杖虽然幸而未曾撒手,却也狼狈非常。这时,他哪里还敢轻敌,将禅杖舞得泼风也似,与段-
璋保持一丈开外的距离,看来虽然仍是十分凶猛,其实却是只求自保而不敢攻故了。

虽然如此,但阿奢黎的禅杖打来,仍是有千斤之力,段-璋刚才是用``巧招''将他击败,现在给王龙客缠着,要是被阿奢黎的禅杖扫中一下,那仍是难以抵挡。所以段-
璋也得加意提防,不敢轻敌。幸而阿着黎给他吓破了胆,不敢向他强攻。

王伯通的两个副手从侧翼攻来,挡住窦线娘。这两个副手都是绿林中顶尖儿的角色,一个名叫褚遂,一个名叫屠龙,他们都有看家本领,武功确是非比寻常。

褚遂长于近身缠斗的小擒拿手法,刁钻古怪,一被他的手指搭上,即有扭筋断骨之灾;屠龙用的是一对日月双轮,走的却是纯然刚猛的路子,这两个人一刚一柔,配合起来,相得益彰。窦线娘被他们迫到身前,无法再用金弹退敌,只得一手持弓,一手握刀,与他们恶战。

窦线娘继承家学,有三样名震武林的绝技,第一样就是百发百中的神弹功夫,第二样是``金弓十八打'',第三样是``游身八卦刀法'',这时,她虽然不能再发弹子,但刀弓并用,和对方展开游身缠斗的功夫,却也尽可以应付。

王伯通被打穿了额角,十分愤怒,一面命令手下的四大头目都上去助战,一面又叫人进去催王燕羽来。

王燕羽早已躲在屏风后面,父亲已然下了命令,她不想被人发现,无可奈何,只好自己先走了出来,王伯通怒道:``燕儿,你怎的这个时候才来?你瞧,咱们王家已经给人欺负上门啦!''

王燕羽道:``爹爹不必焦急,谅这两个人逃不出去。调一队挠钩手来,就可以将他们生擒了!''原来王燕羽训练有一队女兵,擅长于用长钩擒敌,当日铁摩勒就是被这队挠钩手活擒的。不过,现在王燕羽贡献此计,却是想藉此拖延时候,因为她实在不愿意和段-
璋动手。

王伯通点点头道:``也好,不必你去,我自有人传令。''王燕羽没法,只好陪着她的父亲观战。

段-璋杀得性起,忽地一声长啸,连人带剑,化成了一道寒光,疾向王龙客冲去。王龙客不敢抵挡,急忙闪开。那个番僧是给段-
璋杀怕了的,连忙撤回禅杖,舞成一道圆圈,护着自身。给王龙客助战的那两个大头目,身法却没有他这么灵活,段-璋唰唰两剑,一个大头目被刺伤了肋骨,一个大头目被削去了两指,段-璋立即冲出包围,与窦线娘会合。窦线娘在褚屠二人与另外两个大头目围攻下,本来处于劣势,得到丈夫前来会合这才把劣势扭转过来。

王伯通道:``等不及挠钩手了,燕儿,你上去助你哥哥一臂之力。''王燕羽无法可施,只好拔剑出鞘,上前助阵。就在此时忽听得有人大声说道:``夏姑娘,你瞧,这是不是段大侠?老叫化可没有骗你吧!''

王龙客大吃一惊,来的这两个人不是别人,正是卫越和夏凌霜!

原来那日卫越与南霁云分手之后,回去问他那个送信的徒弟,那徒弟说确是已把信交到皇甫嵩手中,而且并无外人在旁。至于空空儿,他更是连影子也没有见过。卫越问不出所以然来,心里更增疑惑,只好先到九原,赴南霁云之约。

他来到九原,南霁云已经走了,南霁云任务是个秘密,太守府中,除了郭子仪之外,无人得知。卫越打听不到南霁云的去向,心中想道:``他曾经怀疑夏凌霜是王家劫走的,多半是到龙眠谷去了。老叫化答应帮他的忙,那就得帮忙到底。且到龙眠谷去走一遭吧。''卫越这一猜虽然没有完全猜中,却也着了几分。

卫越在九原会不到南霁云,却意外的碰见了段-
璋夫妇,原来他们两夫妇也是因为多年未见南霁云,现在军情紧急,特地赶到九原,想来助他一臂之力的。卫越碰见他们,将南霁云所遭遇的事情和他们一说,段-璋与夏家有极深厚的交情,听说冷雪梅、夏凌霜雨母女给人劫走,哪有不着急之理,于是便和卫越一道,都到龙眠谷来。

卫越是丐帮的长老,丐帮弟子遍布天下,消息特别灵通。龙眠谷中也有丐帮的弟子。卫越一到龙眠谷,便查探得那日王龙客将夏凌霜劫到谷中的详情,知道了夏凌霜确实是在王家,于是便和段-
璋夫妇定下计策,由段-璋夫妇光明正大的登门索人,卫越则在王家暗中搜查。

正巧夏凌霜眼下了解药,本身功力已经恢复,她正要出去寻王龙客算帐,便碰见卫越。这时段-璋夫妇已经在外边恶斗,他们顺理成章的当然便都出来助阵。

夏凌霜一冲出来,正是仇人见面,份外眼红,二话不说,唰的一声,便向王龙客刺去!

王龙客叫道:``夏姑娘,你------''夏凌霜斥道:``我怎么?我还没有给你害死!''只听得嗤的一声,王龙客的衣襟已给她一剑穿过!王龙客又惊又气,挥扇遮拦,夏凌霜的武功本来比他稍胜一筹,这时恨不得将他置于死地,出剑更为狠辣,招招都是杀手!王龙客挡了几招,惊慌气急之下,一个疏神,只听得``唰''的一声,王龙客又中了一剑,刚才那一剑仅是穿过衣襟,这一剑却正中胸口,幸而他立即弯腰后仰,使用``铁板桥''的功夫化解,但虽然如此,胸口亦已给剑锋划破,鲜血淋漓,沁红了衣裳!

夏凌霜柳眉倒竖,凤眼圆睁,怒声斥道:``无耻贼人,今日你罪贯满盈,还想逃命么?''话声未了,剑招续发,``唰''的一招``白虹贯日'',剑光疾吐,直指王龙客的咽喉。

眼看王龙客就要毙命在她剑下,斜刺里忽地一柄长剑插来,刚好插在他们两人当中,夏凌霜一看,却原来是王燕羽,只见她双眸泪泫,愁锁眉尖,满脸惊怕羞愧而又带着恳求的神情。夏凌霜不忍伤她,剑势稍缓,王龙客趁此时机,连忙逃走。

王伯通认得疯丐卫越,大惊叫道:``卫老大,我与你向来井水不犯河水,你何故与我为仇?''卫越哈哈笑道:``王伯通,你也知道害怕了么?不错,你做了绿林盟主这么多年,老叫化从来没有找过你的碴儿,可是你如今与安禄山兴兵作乱,荼毒生灵,老叫化可不能不管了!不过,冤有头,债有主,老叫化今日是要来插手,但你却不必担心我来杀你,杀你的另有其人!''

卫越口中说话,手底却是毫不放松,只见他一个照面,就把王伯通两个得力的头目抓了起来,笑道:``我不杀老贼,也得杀两个小贼来解解恨!''那两个头目被他抓着了琵琶骨,痛彻心肺,杀猪般的大叫饶命,卫越将他们提了起来,旋风一舞,忽地笑道:``姑念你们只是从犯,好,就饶了你们吧!''双臂一振,将那两个大头目掷出门外。那两人的琵琶骨给他捏碎,虽得保全性命,武功却已废掉,再也不能为恶了。

卫越与夏凌霜双双杀到,盗党阵脚大乱,窦线娘一声叱咤,缅刀朝着屠龙面门一晃,引开他的眼神,左手的金弓却疾的朝着褚遂拨去,这一招方是实招。褚遂仗着小擒拿手的功夫,这时正使到一招``拨云见日'',双掌成环,来扣窦线娘的手腕,哪料窦线娘将计就计,佯攻屠刚,等于卖个破绽,让他欺近身前,猛地反弓一拨,褚遂的手指正好触及她的弓弦,登时被弓弦拉断了中指,十指连心,痛得他狂呼疾退。

这时王龙客已逃得无影无踪,窦线娘眼光一瞥,发现了王燕羽,记起了杀兄之恨,立即向她奔来。夏凌霜连忙叫道:``段婶婶,这个小女贼交给我好啦!''

王伯通喝道:``好个撒拨的恶婆娘,谁给我将她擒下,重重有赏!''窦线娘大怒道:``你不来找我,我也要找你算帐哩!''心中想道:``杀我哥哥的虽是他的女儿,但罪魁祸首,却实在是这老贼!''同时,又见到夏凌霜已与王燕羽交锋,便转移了目标,迳向王伯通那边杀去!

夏凌霜感激王燕羽赠药之恩,有心相护,见窦练娘已转了方向,向王伯通杀去,便作势佯攻,欺近她的身前,低声说道:``王姑娘!你快快走了吧!''

王伯通手下见窦线娘来势凶猛,只得拼死上前,全力抵挡,窦线娘弓打刀劈,锐不可当,刹眼之间,连伤了五个头目。就要杀到王伯通跟前。

王燕羽忽地虚晃一招,抽身便退,夏凌霜只道她已听从动告,不料她飞身疾掠,却是挥剑向窦线娘杀去。

夏凌霜眉头一皱,心道:``我不能因你一人之故,便放过了王家老贼。''她足尖一点,仿如流星赶月,抢先一步,拦住了王燕羽。

王燕羽咬了咬牙,沉声说道:``夏姑娘,你迫得我没法子啦!''青钢剑扬空一闪,剑光疾吐,抖出七朵剑花,连袭夏凌霜七处穴道。要知她为了父女之情,怎忍见王伯通为窦线娘所杀?因此只得使出凌厉无前的剑法。不过她的用意仅在迫夏凌霜让开,剑招虽然凌厉,分寸之间,却拿捏得非常准确,每一招都未曾用实。

哪知夏凌霜也抱着同样心思,双剑相交,但听得一片叮咣声响,刹眼之间,两柄青钢剑已接触了七下。两人用的都是上乘剑法,本领也不相上下,夏凌霜的内力稍胜一筹,她展开了游身缠斗的剑法,就是不放王燕羽过去,王燕羽无可奈何。

卫越打得性起,大声笑道:``我再摔几个小贼玩玩,哈哈,真是有趣得紧!''他是出了名的``疯丐'',就像猫捉老鼠一般,将那些头目捉来戏要,或者打一下耳光,或者揪一把头发,戏耍够了,然后把他们一个个摔出去。

那个番僧见众人都似乎惧怕这个疯丐,大为不忿,心中想道:``将人摔倒,不过是恃着几斤气力,有何稀奇?我不信他的气力胜得过我。''他刚才败在段-
璋手下,有心挽回面子,与这疯丐较量较量。

卫越刚刚摔倒了第七个头目,忽听得呼的一声,只见一根碗口般大的禅杖向他搂头打下,卫越哈哈笑道:``好一根禅杖,好一个蛮牛。''伸手一抓,竟然凭着一双空手,将禅杖牢牢抓实,

那番僧动弹不得,大吃一惊,卫越笑道:``好,你也算得是有几分本领的了!''陡地喝道:``撒手!''使出了``隔物传功''的内家真力,那番僧忽地感到一股大力直撞胸口,果然应声撒手,连连后退!

卫越夺过了禅杖,在手中掂了一下,哈哈笑道:``份量倒是不轻,只是中看不中用,作打狗棒也嫌笨重!''笑声一收,便将禅杖往地下一插,那根禅杖登时没得无影无踪。

那番僧跄跄踉踉的连退几步,幸而未曾跌倒,见状大惊,``中原的武林人物果然厉害,这个叫化子的本领比刚才那个南蛮子还高!罢了,罢了,我还在此地作什么?''他挤开众人,夺门而走,连夜逃回范阳。

窦线娘正要杀到王伯通身前,忽听得号角大呜,脚步声呼喝声闹成一片。原来龙眠谷要办喜事,连日来到了不少绿林人物和龙眠谷属下的各处寨主,王龙客刚才逃了出去,便响起警号,召集这些人前来助战。同时,王燕羽所训练的那队挠钩手也到来了。

这班绿林人物,武功虽然亦非上乘之选,但却要比王伯通的一些小头目强得多,这班帮手一到,又把窦线娘包围起来。

那队挠钩手更其厉害,十几柄长钩,忽伸忽缩,神出鬼没,专勾敌方的双脚。卫越皱了皱眉,说道:``老叫化子可是不喜欢和娘儿们打架。''他随手将两个小头目抓到手中,当作盾牌,挠钩手不敢向他勾去。

段-璋见妻子又陷重围,陡地一声大喝。宝剑一荡一圈,与他正面对敌的是日月轮屠龙,他的日月轮本来是克制刀剑的,但却怎禁得段-
璋这精妙而又狠辣的剑法,段-璋一剑从月轮中心插进,一翻一绞,轮齿全部断了,屠龙心寒胆战,急急忙忙弃轮而逃。

那队挠钩手扇形散开,十几柄长钩都向段-璋勾来,哪知段-
璋使的是把宝剑,削铁如泥,剑光霍霍展开,登时响起了一片断金戛玉之声,十几柄挠钩断折了一半以上。段-璋喝道:``我宝剑不杀女流之辈,你们也休得助纣为虐!''

夫妻二人再次会合,不消多久,又杀开了一条血路。王伯通大为丧气,想不到铁桶般的龙眠谷竟给他们几个人闹得天翻地覆,欲待逃走,却又碍着绿林盟主的身份,要是弃众而逃,以后还有何颜面统驭部下?

王伯通正在踌躇,忽听得钟声四起,震耳欲聋,龙眠谷布防严密,各处险隘所在,都设有了望哨,安有警钟,一发现敌踪,便即鸣钟告警,如今钟声四起,那即是说敌人已不只一路,而今从四面八方窜进龙眠谷来了!王伯通这一惊非同小可,就在此时,只见一个手执红旗的头目,匆匆忙忙地跑了进来。

那头目大叫道:``赛主,不好了,敌人已杀过了龙眼岗了!''龙眼岗是龙眠谷的心腹之地,离此不过数里路程,王伯通心内吃惊,故作镇定,问道:``何方人马?人数若干?''那头目道:``黑夜之中,不知来历,到处都现敌踪,也不知多少!''

王伯通大怒骂道:``龙眠谷里里外外,有十八重防卫,敌人怎能一下子杀到了龙眼岗来?想必是敌方派了几个夜行人前来捣乱,最多也不过是零星小股,你虚张声势,造谣惑众,敢情是敌人的奸细么?''忽地拔出金刀,一刀将那报讯的头目杀掉,这小头目是王伯通的亲近人,他何尝不知道他所说的乃是实情,只因要安定人心,故此只得将他冤枉杀了。

王伯通喊道:``大家不必慌乱,边战边走,都退到外边去。与大队会合之后,再消灭敌人。''此言一出,由王伯通领先,所有盗党,都纷纷夺门奔逃。

王伯通的心腹手下仍然拼死堵住段-
璋夫妇,不让他追上王伯通。夏凌霜也紧紧缠着王燕羽,双方边打边走,混战之中,忽见有两个人飞一般的跑来,其中一人大叫道:``凌霜,凌霜!是你么?我是霁云!''

来的这两个人正是南霁云和铁摩勒。原来韩湛熟悉龙眠谷地形,有一条秘道,是王伯通也不知道的,他们分兵的路,一路从正面进攻,一路则从秘道进兵,绕过了各处险隘所在,然后再分成许多小股,从背面偷袭,拔除了王伯通设在险隘所在的关卡,里应外合,从四面八方杀来!

南、铁二人率领的一股,都是轻功有些根底的金鸡岭头目,他们从秘道插进,因此,一下子便到了龙眠谷的心腹地带,南霁云急不可待,先和铁摩勒赶了到来,正好赶上了这一场混战。

夏凌霜大喜道:``你来了!''这刹那间,她眼中只有南霁云一人,连王燕羽也不管了。南霁云道:``不只是我,金鸡岭好汉全部来了!''一双情侣,劫后重逢,当真是恍如隔世。夏凌霜与他执手相看,禁不住珠泪滴下。

王燕羽早已趁此时机跑掉,夏凌霜猛地惊醒,说道:``霁云,段大侠他们都来了,你快去帮他们厮杀!''

段-璋一声长啸,展开了``乱披风''的剑法,剑光倏的铺开,一口剑就似化成了数十百口,将近身的敌人全都裹住,叫道:``线妹,不可让那老贼跑了!''

窦线娘有丈夫替她挡住了围攻的敌人,便抽身冲了出来,远远看见王伯通在前头奔跑,她弹弓一拽,立即用连珠弹向王伯通打去!

忽听得叮叮之声,恍如繁弦急奏,窦线娘的连珠弹尚未射到王伯通身前,突然间,却不知是从哪儿飞来的暗器,将窦线娘的连珠弹全都打落!

窦线娘吃了一惊,心中想道:``想不到这老贼手下,还有如此能人!''窦线娘是暗器的大行家,听那声音,便知道对方用的是梅花针或透骨针之类的细小暗器,居然能把她的金弹碰落,而且用的也是``天女散花''的手法,每一枚都撞个正着,这人使暗器的功力和准头,最少已是与她不相上下。

窦线娘叫道:``摩勒,快来,老贼在这边!''铁摩勒正要替义父报仇,一发现了他的踪迹,立即运剑如风,赶杀过去。他气力沉雄,剑法精妙,王伯通的心腹死土抵挡段-
璋夫妇尚嫌不够,剩下的一些人,怎禁得起铁摩勒的猛斫狂冲,不消片刻便给他追上了王伯通。

铁摩勒喝道:``还我义父的命来!''长剑一挽,一招``李广射石'',势劲力急,端的似一支离弦之箭,直刺王伯通的咽喉,王伯通怒道:``小贼敢出大言!''金刀一立,刀剑相交,咣的一声,震得耳鼓嗡嗡作响。铁摩勒踏上一步,奋不顾身,又是一剑横劈过去,这一剑更是劲道十足,火花蓬飞中,王伯通抱刀急退。铁摩勒大喝一声,跑步已嫌太慢,他突然跃了起来,竟如鹰隼腾空,第三剑用的便是``饿鹰扑兔''的招数,凌空向王伯通的脑门刺下!

王伯通虽是绿林之雄,但年纪老迈,怎当得铁摩勒的神力,他连接两剑,已是双臂酸麻,无力抡刀,眼看铁摩勒如鹰扑下,心里叹口气道:``悔当初听了空空儿之言,留下了这小贼的性命!''

就在这性命俄顷之间,忽听得一声喊道:``休得伤我老父!''声到人到,比铁摩勒还快,来的正是王燕羽。

她也是凌空扑来,双剑一交,她的气力较弱,登时先跃翻了。可是铁摩勒给她一阻,王伯通又已跑开。

好个王燕羽,她在地上一个鲤鱼打挺,翻起身来又恰好拦在铁摩勒与她父亲的中间,铁摩勒正自一剑刺去,王燕羽来不及出把防御,一咬银牙,索性挺胸迎上,尖声叫道:``好狠的冤家,你就要了我的命吧!''铁摩勒心头一震,不自觉的将剑收回,幸而他的剑术已到了收发自如的境界,只差一发,险些就要穿过王燕羽的酥胸!

铁摩勒长剑一指,沉声说道:``王姑娘,一命换一命,我已还清了你的债了。你父亲欠我的债与你无关,请你快走,若还拦阻,可休怪我无情!''

铁摩勒和她说的是黑道上的规矩,当初王燕羽曾饶过他一次性命,如今铁摩勒也饶回她一次性命,故此铁摩勒说是已还清了她的债。不但如此,杀铁摩勒义父的本来是王燕羽,如今铁摩勒也把这个债算到她父亲头上,表示可以与她无关,这实在是十分宽大的了。

但王燕羽念着父女之情,岂肯放铁摩勒过去追杀她的父亲?而且铁摩勒说的话斩钉截铁,只讲江湖规矩,不顾两人情份,王燕羽听了,不由得又是伤心,又是气愤。

铁摩勒正要从她身旁掠过,王燕羽反手一剑,叫道:``冤有头,债有主,你要报仇,可先杀我!''

他们两人的剑术本来不相上下,王燕羽拼命拦截,倒教铁摩勒没了法子。他几次咬了咬牙,却依然不忍施展杀手。如此一来,反给王燕羽着着进迫,处在下风。

王燕羽和铁摩勒斗了二十余招,当然也明白是铁摩勒处处让她,心中怒火稍平,有了一点甜丝丝的感觉。

南霁云不知就里,他见铁摩勒给王燕羽迫得手忙脚乱,竟似险象环生,不由得大吃一惊,连忙施展``八步赶蝉''的身法,几个起伏,便赶了到来。

南霁云是大侠身份,不愿以多为胜,当下大叫道:``师弟,你去找那老贼报仇吧,这女贼让我来打发好了。''

铁摩勒心头一震,但觉进退两难,说时迟,那时快,南霁云已是一手将他推开,陡然大喝一声,抡刀便斩。

南霁云的功力比铁摩勒又胜一筹,王燕羽横剑遮拦,刀剑相交,咣的一声,王燕羽虎口流血,青钢剑几乎脱手飞去。南霁云心里有点奇怪,想道:``这女子剑术虽然不错,铁师弟也不弱于她,怎的敌她不住?''激战中无暇细思,南霁云一刀劈一下,跟着又是一刀,王燕羽使出了浑身本领,腾挪闪展,连避了三刀,第四刀却没法闪开,又迫得硬接了一招,登时给震得倒退七八步,剑锋也损折了。

南霁云喝道:``女贼往哪里走?''身形疾起,正想趁着王燕羽立足未稳,再补一刀,便结果她的性命,忽听得铁摩勒颤声叫道:``师兄,师兄------一''南霁云回头一望,只见铁摩勒还站在那儿,一脸惶恐的神情。

南霁云怔了一怔,正自觉得铁摩勒的行动古怪,就在此时,夏凌霜亦已向这边跑来,远远就扬声叫道:``大哥,不可、不可、不可伤了她!''连说了三个``不可'',惊慌着急之情,可想而知。

南霁云的宝刀已然劈下,听得喊声,倏然收势,距离王燕羽的天灵盖不到半寸,比铁摩勒刚才那一剑还要惊险得多。王燕羽斜跃一步,忽地低声说道:``多谢南大侠手下留情,你若是要寻人的话,可到莲花峰下断魂岩一试。''

这句没头没脑的说话,听得南霁云莫名其妙。霎眼之间,夏凌霜已到了她的面前,而王燕羽也已没人人丛,连影子都不见了。

南霁云道:``霜妹,为什么你不许我伤她?''夏凌霜道:``是她救我出来的,这事慢慢再和你说。''南霁云回头一望,只见铁摩勒满面通红,也已到了他的身旁,南霁云甚为疑惑,心里想道:``王伯通的女儿为什么肯救凌霜?她救了凌霜,铁师弟又怎能知道?''他还以为铁摩勒刚才失声惊喊,也是因为王燕羽曾救了夏凌霜,故而想他刀下留人的。

这时双方已陷入大混战之中,杀声震天,到处是刀光剑影,王伯通父女都已不知去向,南霁云挥刀冲杀,接应从外面攻进来的义军,已无暇询问究竟了。

王燕羽刚刚追上父亲,忽然听得一个清脆的声音叫道:``真是踏破铁鞋无觅处,得来全不费功失。想不到在这里又碰上了你,好呀,咱们再来比划比划!这回应该可以决个胜负了吧?''迎面一彪人马杀来,为首的正是辛天雄和韩芷芬。

辛天雄抡起斫山爷,直奔王伯通;韩芷芬则挥剑直取王燕羽。她一出手使是极为凌厉的刺穴剑法,一招之间,连袭王燕羽七处穴道。

王燕羽和她本是半斤八两,不相上下,但此时此际,一来她已厮杀了半夜,二来她要保护父亲突围,哪里还有心情恋战?

交手数招,韩芷芬笑道:``王姐姐,你怎的便怯战了?''剑光一展,蓦地一招``玉女投梭'',剑锋直指王燕羽胸口的``魂门穴'',王燕羽气力不佳,已来不及回剑防御,忽听得``铮''的一声,不知从哪里窜来了一个蒙面人,动作快到了极点,双指一弹,便把韩芷芬的长剑弹开,拉了上燕羽便跑!

王燕羽道:``你是谁?''那蒙面人一声不响,只是向前疾跑,王燕羽跟着他,只见正是向着自己父亲那边跑去。

王伯通与辛天雄拼死恶战,正到了吃紧的关头,那蒙面人如飞奔至,恰值辛天雄一斧劈下,蒙面人挥袖一卷,辛天雄臂力沉雄,这一斧劈下,少说也有六七百斤力气,却不料给这蒙面人的衣袖一卷,便把斧头裹住,竟自动弹不得。蒙面人哈哈一笑,轻轻一拂,辛天雄跌了个仰八叉,待他跳起来时,王伯通父女和那个蒙面人都已走得无踪无影了。

这时金鸡岭的各路义军亦已杀了进来,可是龙眠谷乃是王家的老巢,谷中的喽兵都是久经训练的精壮,而且人数也远较金鸡岭攻进来的义军为多,因此,虽然是黑夜被袭,仓皇应战,但仍不至于溃不成军。有好几处地方。义军反而陷入了他们的包围之中。

铁摩勒夺了一骑快马,高举火把,在谷中纵横驰骋,高声叫道:``王家勾结胡儿,为虎作怅,罪大恶极,这样的人,怎配作绿林盟主?你们都是有血气的男儿,响当当的好汉,难道甘心听这老贼驱策,为他送死么?''

有好些本来是窦家的部属,认出了铁摩勒,登时骚动起来,纷纷叫道:``啊,铁少寨主,是你回来了!''``对,铁少寨主,你的话说得对!替王家卖命,这不是绿林义气,死了也只赢得个臭名!''``好,有你铁少寨主一句话,咱们反了王家吧!''

这么一闹,有的人放下了兵器,有的人倒戈相向,登对主客势易,愿意替王家作战的十成不到三成,义军声势大壮,追奔逐北,到处扫荡。

一场恶战,出乎意料的顺利收场,待到天明,王伯通的心腹党羽都已给赶了出去,龙眠谷全被义军占领,剩下的就只是打扫战场的工作了。

辛天雄迎上了铁摩勒,执手谢道:``铁兄弟,今次攻占龙眠谷,功劳簿上,第一笔就应该写上你的功劳。只可惜让那王家老贼跑了。我本来可以一斧头斫死他的,不知是哪里钻出来的龟儿子,一下子就将他救走了。''铁摩勒谦虚了几句,问了辛天雄的经过,颇为诧异,说道:``依你说来,这蒙面人的武功实不在空空儿之下,王伯通手下有此能人,倒是出乎我意料之外。只是他为什么蒙着面不敢见人?而且只是救人,却未曾和我们厮杀呢?''辛天雄道:``谁知道他打的什么主意,总之救走王伯通的就不是好人。''韩芷芬冷冷说道:``王家老贼漏网,那是因为他有能人相助,可是在此之前,那个小女贼有几次都应该丧命的,也都给她逃过了,这才叫奇怪呢!''辛天雄道:``哦,有这样的事?她又是怎么逃过的?''韩芷芬道:``黑夜之中,我看得不十分清楚。摩勒在场,你问摩勒!''

铁摩勒满面通红,说道:``那女贼武艺高强,阻她不住,被她跑了。''辛天雄见过王燕羽的本领,知她厉害,说道:``铁贤侄已是尽力而为,只怨咱们人手不够,让他们漏网。不过,咱们总算已捣毁了他们的老巢,纵然跑了王家父女,亦已无能为患了。''

当下群雄就在龙眠谷的演武厅中聚集,重新相叙。段-璋首先向南、夏二人道贺,夏凌霜这时方有余暇,将经过向他们细说。

南霁云听得岳母尚未知下落,猛然想起了王燕羽所说的那句没头没脑的说话,便问夏凌霜道:``依你说来,王伯通的女儿倒还似乎不坏,她曾对我说道:你若是要寻人的话,可到莲花峰下断魂岩一试,莫非她所说的就是你的母亲?''夏凌霜喜道:``她当真是这样说了?晤,那就不用多问,定然是她有意向你透露他们囚禁我母亲的处所了。''

窦线娘对王家的人最为痛恨,说道:``王伯通女儿的说话你也这样相信么?提防上了敌人的当。''夏凌霜道:``段婶婶不必多虑,她苦是想害我的话,她就不会给我解药了。解药既是真的,想来这话也假不了。''当下,又把王燕羽将解药给她的时候,和她所说的话语,也原原本本的告诉了大家。段-璋夫妇越听越觉得奇怪,夏凌霜讲完之后,窦线娘问道:``南兄弟,你以前认识她的么?怎的她想你知道她是个好人?''夏凌霜代他答道:``霁云也只是那次在飞虎山上见过她,幸亏霁云所做过的事情我全都知道,要不然我可怀疑他有私情了。''南霁云想起铁摩勒刚才的神情,当王燕羽在他刀下的时候,他那惊煌的神色,心中猜到了几分。但在众人面前,他当然不方便说出来。

段-璋道:``人有向善之心,咱们就该原谅他,扶掖他,无须再揣度他何以有这念头了。现在咱们该断定的倒是她所说的是什么地方?莲花峰这个名称,好几座名山都有。''卫越正巧走来,说道:``老叫化走过的地方最多,莲花峰断魂岩,那就只是华山的莲花峰才有。''

段-璋心中一动,道:``西岳华山,唔,那岂不是皇甫嵩居住的地方?''卫越道:``华山很大,著名的山峰便有五个,据我所知,皇甫嵩却不是住在莲花峰的。''段-璋沉吟半晌,说道:``夏侄女母女被掳之时,敌方的主脑人物便是皇甫嵩,如今王伯通女儿透露的消息,她又是被囚禁在华山之上,看来十九都是与皇甫嵩有关的了!''

正是:欲解疑团何处去?莲花峰下断魂岩。

欲知后事如何?请听下回分解------

\chapter{第二十二回 胡骑已践中原地
汉帜方张细柳营}\label{ux7b2cux4e8cux5341ux4e8cux56de-ux80e1ux9a91ux5df2ux8df5ux4e2dux539fux5730-ux6c49ux5e1cux65b9ux5f20ux7ec6ux67f3ux8425}

卫越道:``你说的也有道理。好,不管是不是皇甫嵩干的,老叫化终须要查个水落石出。待这事情了结之后,老叫化就陪你们到华山去走一遭吧。''

南霁云却多了一层烦闷。他是奉了郭子仪之命,在敌后组织义军,牵制安禄山的兵力的。那华山在陕西境内潼关之西、华阴县南,距离长安也不过数百里。要是郭子仪回师保驾的话,南霁云自可抽身前往华山,现在义军方始成立,他要想抽身,却是有点为难。

辛天雄道:``大家恶战了一夜,想来都已累了。先歇歇吧,还有什么事情,以后再作商量。''

攻下了龙眠谷,义军人人兴奋,他们分班休息,就在当日办起了庆功宴来,辛天雄等人睡到日头过午,醒来的时候,正好赴宴。

除了南、铁二人有点心事之外,其他诸人无不开怀畅饮。正自高兴,忽地有中军进来报道:``山寨里有人和一个军官快马驰来,候见寨主。''辛天雄虽然接受了敌后招付使的名义,但他的手下,仍然以寨主相称。

辛天雄一怔,问道:``来的是哪位弟兄?''中军答道:``是杜先生。''

辛天雄吃了一惊,忙道:``快请,快请!''要知中军所说的``杜先生'',即是金剑青囊杜百英,他是以客卿的身份在金鸡岭留守的,如今他亲自陪伴一个军官赶来,要不是这军官的身份特别重要,那就是山寨又有了意外之事了。

只见杜百英满面风尘,匆匆赶至,在他后面的是个熊腰虎背、相貌威武的军官,辛天雄顾不得招待客人,先自问道:``可是寨中出了什么事情?''他话未说完,只听得南霁云和段圭璋已在同声叫道:``雷师弟!''``雷贤弟!''铁摩勒也慌忙站起来道:``是雷师兄么?''

杜百英道:``山寨无事,是这位雷大侠有事要见他的师兄。''原来这个军官正是磨镜老人的第二个徒弟雷万春。

雷万春在睢阳太守张巡那儿任职,铁摩勒还未曾和他见过面,当下独自另行了拜见师兄之礼。雷万春道:``你们都在这里,那好极了。南师兄、铁师弟,我正有话要和你们说。''

段-璋老于世故,猜想雷万春在军情紧急的时候赶来,定非无故,只恐他们不便在人前说话,便道:``你们师兄弟进后堂去叙叙话,雷大侠歇息过后,再来喝酒。''富万春也不客气,拱手便道:``如此,暂且少陪。''在他豪迈的神态之中,竟是显得有几分烦忧焦躁。

杜百英使了个眼色,说道:``辛大哥,你不必客气,咱们是熟朋友了,酒我自己会喝,不用你费神招呼。''辛天雄会意,知道雷万春此来,定是有要事相商,杜百英叫他不必招呼自己,那就是示意要他去招待雷万春。辛天雄笑道:``对,雷二哥初到,我做主人的可不能太简慢了,待我带路吧。''

进了密室,南霁云问道:``雷师弟,军情是否又生变化了?''雷万春沉声说道:``潼关失守,哥舒翰已经降贼,贼兵正自指向长安!''

这一惊非同小可,南霁云叫起来道:``哥舒翰是朝廷最重用的大将,身受国恩,怎的也降了安贼?''

雷万春道:``说来都是与杨国忠有关。杨国忠与哥舒翰素来不睦,哥舒翰屯军潼关,按兵不动,安贼本来无法攻破,杨国忠害怕他拥兵自雄,将对自己不利,启奉皇上,遣催哥舒翰进兵恢复陕洛。哥舒翰飞章奏道:``我兵踞险,利于坚守,况贼残虐,失众民心,势已日整,因而乘之,可以不战而自戢。要在成功,何必务速?今诸道征兵,尚多未集,请姑待之。'郭令公也曾上言:``即欲出兵,亦当先引兵北攻范阳,覆其巢穴,潼关大兵,屏障长安,惟宜固守,不宜轻出。'无奈杨国忠疑忌已深,力持进战,皇上听信他的话,连遣中使,往来不绝的催哥舒翰出战。哥舒翰无可奈何,奉了圣旨,只好引兵出关。哪知安贼已预有埋伏,引官军追到险要之处,突然数路合围,又用几百乘草车,纵火焚烧,直冲官军大营。结果潼关的二十万人马,溃不成军,逃回关西驿中的不过八千人。哥舒翰的本钱没了,一气之下,竟然就投降了安禄山,声言要借安禄山之力,杀杨国忠报仇。''

南霁云叹息道:``哥舒翰本来是个将材,可惜被杨国忠逼反了。咳,这也是朝廷久疏兵备,边疆重责,一向付诸以番人为主的边军之故。如此一来,只怕局势更难收拾了。''

雷万春道:``皇上打算逃避西蜀,由太子做兵马大元帅,郭令公做副元帅,此事尚未曾发表。我这次飞骑到来,正是奉了张、郭二公之命,要和南师兄、铁师弟商量一件事情。''南霁云道:``什么事情?''雷万春道:``这是与皇上逃难的事情有关的。''铁摩勒诧道:``皇帝老儿走难与我有何相干?''雷万春笑道:``你们两位,谁愿意做护驾将军,跟随皇上到西蜀去。这是郭令公的书信,你们请看!''

南、铁二人读了这封信,才知道事情的严重,以及雷万春此来的缘故。

原来在安绿山之乱起后,睢阳太守张巡也升任了雍丘防御使,但他责任加重了,兵力便嫌不足,兼之又缺乏粮草,因此便派出雷万春到长安向朝廷请求增兵拨粮。

雷万春到长安的时候,正值潼关失守,朝野震动,玄宗计划西迁的时候。人心惶惶,京城已陷于混乱的状态,皇帝都只顾自己逃难了,哪里还有兵可调、有粮可拨?

玄宗在承平的时候耽于逸乐,但还不是十分昏庸的皇帝,在危急的时候,还能够重用郭子仪、张巡等有才能的将领。也正因为他要倚重郭、张等人替他保住江山,作为张巡使者的雷万春才得到他的召见。

召见之时,秦襄、尉迟北二人也在一旁伺候。玄宗先讲了朝廷的困难,然后用一番好言抚慰,增兵拨粮之事,那是不用提了。非但如此,他还向张巡和郭子仪要人。因为他逃难的时候,需要有本领的心腹武士保驾,急切之间,无处可寻,他素来知道张、郭二人手下,颇有能人,而难得这两人又是忠心耿耿,他们保荐来的武士一定可靠。

当时秦襄和尉迟北向玄宗献议,本来便要把雷万春留下的,雷万春哪肯离开危险中的睢阳。最后是采取了折衷的办法,由雷万春接了圣旨,转谕郭子仪和张巡,尽速选拔可靠的武士前来长安,若是无人可选,便要调雷万春来作御前侍卫。

其时,睢阳四面都是敌兵,形势危急之极,雷万春回到睢阳,和张巡商议之后,睢阳实在是无人可调,于是雷万春再到九原,一面请郭子仪发兵援救,一面传达圣旨。

郭子仪这封信便是讲这两件事情,他的兵力虽较张巡雄厚,但是他所要防御的地区也比张巡广大得多,因此兵力也嫌不够。当下,他除了尽力抽调出一支援军之外,还想到一个计策,因为潼关失守之后,得以安全逃回后方的军队,十停不到一停,散在潼关周围的散兵游勇甚多,他计划派一个得力的将官去将这些溃军重组起来。他希望南霁云替他执行这个计划,铁摩勒则到长安听候皇帝任用。

铁摩勒读了这信,叫道:``皇帝老儿逃难,与我何干?只有他的命才值钱吗?哼,哼,我不愿去!''

南霁云道:``那么,你去潼关如何?''铁摩勒道:``这,我更不行了,我自问没有大将之材,也不耐烦和官兵打交道。''

雷万春道:``可是这两件事情定得有人去做,你不愿去长安,可不令郭、张二公为难了吗?''

铁摩勒想了一想,说道:``我知道比较起来,还是去作御前侍卫责任最轻,只是我不服气给皇帝老儿作保镖。''

南霁云笑道:``我们对皇帝老儿也并无好感,可是我只问你一句话,你恨安禄山多些,还是恨皇帝多些?''

铁摩勒道:``这怎能相比?安禄山率胡兵人寇,所到之处,奸淫掳掠,无所不为。把咱们汉人看得鸡犬不如,皇帝虽然可恼,到底还是咱们汉人,而且也尚不至于像安禄山这样凶暴。''

南霁云道:``你知道这个道理就行了,你此去不是给皇帝做私人的保镖,而是给老百姓作保镖。试想,假若是皇帝给暗杀了,这乱子岂不是更难收拾了?老百姓所受的灾难岂不是要更多更久了?所以,应当为大局着想。''

铁摩勒想了一会,说道:``师兄,你说得很有道理,好,我依你便是。''

铁摩勒虽然给他师兄说服,心中总是有点不乐。庆功宴散后,他找着了韩芷芬,两人同到梅花林里,韩芷芬笑道:``你怎的好像不大高兴的样子,是不是恼了我了?''

铁摩勒叹口气道:``我恼你作甚么?咱们只怕要暂时分手了。南师兄要我到长安去。''当下将这件事情就给韩芷芬知道。

韩芷芬听了,又是忧愁,又是欢喜。忧愁的是这一分手,不知何时方能再见;欢喜的是铁摩勒为着与自己分离而烦恼,又这样着急的来告诉自己,显然是已把她当作知心的人。

两人的手不知不觉的相握起来,韩芷芬道:``你不要难过,你去作御前侍卫,我当然不能跟着你。但是我会等待你回来的。待乱事平定之后,我想,你当然不会再做这捞什子的御前侍卫的。''

铁摩勒当然懂得她说的``等待''是什么意思,登时心里甜丝丝的,紧握住韩芷芬的手说道:``芬妹,你待我真好。''

韩芷芬忽地面色一端,说道:``还有待你更好的人呢,只怕你见了她就忘了我了!''

铁摩勒道:``唉,你怎么老是不放心?''韩芷芬满面通红,摔开了铁摩勒的手说道:``你胡说什么?我有什么放心不放心的?嗯,要不是你感激她对你好,怎的你日间将她放了?''

铁摩勒道:``你要再这么说,我可真的恼了!我只是按照江湖规矩,还清她的债罢了。她有一次可以杀我而不杀我,所以我也绕过她一次。以后倘若再有山水相逢,那就是仇人对待了。这话,我已经对你说过许多次了,怎的你还不相信我?''

韩芷芬心里还有点酸溜溜的,但她见铁摩勒着恼,不由得便软了下来,当下笑道:``我是和你闹着玩的,你怎的认起真来了。好啦,我知道你是个铁铮铮的汉子,绝不会受仇人女儿的迷惑,这好了吧?''

她这几句话实是要把铁摩勒再钉紧一步,话语中仍是透露着不放心的意思,铁摩勒自是听得出来。铁摩勒叹口气道:``你看,夏姑娘对我师兄是如何信任无猜,你要像她那样,那就好了!''

韩芷芬登时又羞得满面通红,嗔道:``你真的胡说八道,怎能将我们与他们相比?''

话犹未了,忽听得``噗嗤''一声,夏凌霜分开梅枝,走了出来,笑道:``你这两小口子,怎的在背后说起我来了?什么他们我们的,哎,说得可真亲热啊!看来,可用不着我这个媒人了!''

韩芷芬道:``夏姐姐,你也来欺负我?''夏凌霜一把拉着了她,笑道:``给你做媒,怎么是欺负你了,说正经的,你们既然是彼此相爱,趁早办了喜事吧!就和我们同一天好不好?''

铁摩勒又羞又喜,说道:``你和南师兄已定好了婚期了么?怎的不早告诉我?''夏凌霜道:``现在不是告诉你了么、?如今就看你的了!''

铁摩勒道:``嫂子,你是开玩笑了,我怎能像你们那样,无牵无挂的说成婚就成婚了。''夏凌霜大笑道:``好,好,好!这么说,你们是已经说好了要成婚的咯!差的就只是日期的问题了,是么?''

铁摩勒此言一出,方知说错了话,只见韩芷芬眼波一横,似喜还嗔,嘴唇开阔,好像是要骂他,却没有骂出来。铁摩勒羞臊得无地自容,转身便要逃跑。

忽地一声咳嗽,有个人走出来将铁摩勒拉住。这个人是段-璋。

段-璋道:``摩勒,男婚女嫁,是人生必经之事,害什么羞?夏姑娘说得不错,我们现在是和你说正经事儿。''

段-璋是铁摩勒长辈,铁摩勒只好低下了头,说道:``姑丈,你老人家有什么吩咐?''

段-璋:``夏姑娘,你已问过了他们么?''

夏凌霜笑道:``他们说的话我全都听到了,他们已是情投意合,不必再问了。''

段-璋微微一笑,说道:``摩勒,你的南师兄与夏姑娘已定好明日成婚。我们的意思,你们既是情投意合,两桩喜事就同一天办了吧!''

铁摩勒低下了头,讷讷说道:``这,这,这------''眼睛偷偷望向韩芷芬,韩芷芬面红耳赤,低声悦道:``这个,可不能由我作主。''

段-璋哈哈笑道:``我们正是受令尊之托,来作大媒的。夏姑娘是女家煤人,我算是男家的媒人又兼主婚人。''原来韩湛早已知道女儿心意,所以想在铁摩勒未去长安之前,趁早完了女儿心愿。

韩芷芬粉颈低垂,不再说话。铁摩勒却道:``多谢老伯的美意,多谢姑丈的玉成,只是,只是------''

夏凌霜笑道:``只是什么,难道你还不愿意么?''

铁摩勒是老实人,当下将心中所想直说出来道:``我只怕配韩姑娘不上,哪还有不愿意之理?只是我此次去作御前侍卫,不知何日方得归来?明日成婚,实是不宜。''

段-璋笑道:``这个我也替你们想过了。成婚之后,夫妻立即分开,那是有点不宜。但你可以先行订婚,待乱平之后,再归来迎娶。''

铁摩勒点了点头,表示同意,事情就这样说定了。

他们一对结婚,一对订婚,又正当大破龙眠谷之后,人人都是满怀高兴,喜笑颜开,人多手众,一夕之间,便把龙眠谷布置得花团锦绣,第二天便办起了喜事来。

南、夏二人经过了这场磨难,倍见恩情。美中不足的是夏凌霜的母亲不能来主持婚礼,她的安危也尚未可知。夏凌霜本想寻到母亲才结婚的,但因军情紧急,随时都可能有意外的变化,所以听从了段-
璋之劝,战乱中从权办理。

好在南霁云已奉命到渲关招集散兵游勇,可以趁此时机,到华山探个下落。段-璋夫妇和卫越诸人也说好了和他们同去了。

铁摩勒当然也很高兴,可是不知怎的,就在订婚仪式进行的时候,王燕羽的影子却突然间从他脑海中浮现出来。他自问对韩芷芬已是一心一意的了,却何以会突然想起王燕羽来,连他自己也莫名其妙。他只好自我解嘲,那大约是因为王燕羽留给他的印象太深刻了。她是杀他义父的仇人,在帐幕那夜,又曾有过一段难以忘怀的记忆。

南霁云因为有些事情需要交代,须得多留数日。铁摩勒却因``君命在身'',不能延缓,在订婚后的第二天,便即离开龙眠谷赶往长安。

辛天雄等人送出谷口,韩芷芬将秦襄那匹黄骠马牵来,说道:``你要赶路,就骑了这匹马走吧。到长安后也好还给秦襄。''段-璋、南霁云是与秦襄神交已久的朋友,当下也托铁摩勒在见到秦襄之时,替他们问好。南霁云还特别叮嘱他,叫他在皇帝跟前,不可任性使气,凡事要请教秦襄和尉迟北二人。另外,对宇文通要多加小心,着意提防。

韩芷芬走上前来,目蕴泪光,众人知趣,便与铁摩勒道别,让韩芷芬再送他一程。

他们二人刚刚订婚,便要离开,当真是临行分手,不胜依依。两人都觉得有许多话要说,但万语千言,却不知从何说起,反而默默无言。送到路口,铁摩勒道:``芬妹,你还有什么话要嘱咐我吗?''

韩芷芬深情地望着他,低声说道:``摩勒,你独自一人,须得多加保重,自己小心。''

铁摩勒强笑道:``我不是小孩子了,当会料理自己,你尽可放心!''韩芷芬道:``不单是要注意身体,事事都得小心。嗯,我不多说了,你是聪明人,一定明白我的意思,呀\ldots\ldots 只要你时时记着有我这么一个人便好。''

铁摩勒的心跳了一下,明白了她的意思,知道她仍是不放心自己。当下紧紧握住她的手道:``你放心吧,我心里只有你一个人,另外,就只记挂一件事情。''韩芷芬抬起了头,注视着他的眼睛,问道:``什么事情?''铁摩勒沉声说道:``替我的义父报仇。''

韩芷芬舒了口气,说道:``好,你走吧。不管这场战乱还得多久,我总等你回来。''

铁摩勒飞身上马,道声``珍重'',马鞭虚打一下,那黄骠马立即放开四蹄,绝尘而去。他回过头望,一刹那间,韩芷芬的影子已自模糊而终于消失,也就在这刹那间,王燕羽的影子又突然间在他脑海中闪过。

一路上避开敌兵,兼程赶路,仗着这匹骏马,来到潼关的时候,比铁摩勒原来的估计还早了两天。

可是到了潼关,立即便面临一个难题。潼关已是在安禄山之手,它在黄河岸边,要往长安,须得通过潼关,否则就只有设法在其他地方偷渡。可是在这兵荒马乱的年头,黄河上的船都逃亡了,铁摩勒来到河边,放目一望,哪里找得到一条船只?

铁摩勒沿着河边走去,走了大半个时辰,忽见河边一棵柳树之下,系有一只小舟,铁摩勒大喜,连忙走上前去,船中舟子走出船头,不待铁摩勒开口,便连连摆手说道:``我不敢在刀口上讨生活,这生意是决计不做的了,客官,你另外去找船只吧。''

铁摩勒取出一锭金子,说道:``这个时候,你叫我到哪里去找?你渡我过去,我这锭金子就给你当作船钱。''

那舟子双眼发亮,想了一会,就道:``好吧,人为财死,鸟为食亡,看在你这锭金子的份上,我拼着性命,渡你过去吧。你这匹马也要过去吗?''铁摩勒道:``这匹马是我的脚力,当然要渡。''

铁摩勒牵马上船,船舱刚好容纳得下,那舟子摸了马背一下,那黄骠马一声长嘶,举蹄便踢,幸好铁摩勒及时将它按住。那舟子道:``这马性子好烈,不过,也真是一匹好马!''铁摩勒道:``你也懂得相马?''那舟子道:``在这江边来往的军马我看得多了,可没有一匹比得上尊驾的坐骑。''

说话之间,舟子已解开了系舟的绳索,向下游划去,铁摩勒是第一次渡过黄河,抬头一望,但见浊浪滔滔,水连天野,想起了祖逖中流击揖,誓复中原的故事,不禁浩然长啸!

那舟子忽地问道:``客官,在这兵荒马乱的年头,你为什么还独自出门,而且是冒着这样大的危险偷渡?''

铁摩勒留神观察他的眼色,见他目光灼灼的注视那匹宝马,心中想道:``你若是心怀不轨,那就是自讨苦吃了。''索性坦直地告诉他道:``我是朝廷的军官,队伍失散,要赶回去归队的。怎么,你害怕了吗?''

那舟子道:``原来如此。大人一片忠心,令人可敬。莫说还有金子给我,就是没有,小人也要拼着性命,渡你过去。''

铁库勒见他神色自如,疑心顿起,想道:``河边只有他这只小船,初时他作出那等害怕的模样,现在却又是这等说法,若非真的贪财,那就是其中有诈。''他暗暗摸出一枚铜钱,扣在掌心,只待那舟子一有异动,立即就用钱骠将他制服。

那舟子的本领倒真不错,双浆使开,小舟如矢,黄昏时分,就到了对岸一处无人所在,那舟子道:``大人请上岸吧,多蒙厚赐,不必再加付船钱了。''话中有话,竟似已窥破了他掌中另扣有铜钱似的。

铁摩勒面上一红,心道:``莫非这舟子也是个风尘中的侠义人物?若然,那倒是我多疑了。''

若在平时,铁摩勒定要和他多攀谈几句,但此际他急着赶路,拱手向那舟子道谢之后,便即登程。背后还隐约听得那舟子啧啧赞道:``真是一匹宝马!''

铁摩勒趁着天黑,绕过潼关,进人了官军驻守的地区方始歇息,第二大一早,继续兼程赶路。当天晚上,便到了华阴。

华山便是在华阴县的南边,铁摩勒到了华明,不禁想起了南霁云他们计划到华山救人之事。他这次仗着马快,到了华阴,比原先的预期还早了两天,华阴离长安不过二百多里,以他这匹马的脚力,明日再兼程赶路,大约午后就可以到达长安了。因此铁摩勒也曾动过念头,想到华山一探,但经过深思熟虑之后,感到自己孤单一人,若然有失,反而误了大事,终于还是把念头打消了。

这晚,他在城中一间客店住宿。将近天亮的时分,忽听得他那匹黄骠马大声嘶叫,铁摩勒吃了一惊,慌忙赶到马厩去看,亮起火折,见那匹马好好的还在马厩之中,再往外面察看,地上并无足印,铁摩勒起了疑云,心中想道:``看来不像是有偷马贼来过,却怎的它好端端的嘶鸣起来?''

这时,东方已经发白,坐骑既然没有失去,铁摩勒也就不再查究了。当下他结了店钱,便即策马登程。

哪料走了一程,这匹宝马竟然大失常态,端起气来,越走越慢,铁摩勒大为奇怪,下马察看,只见那匹马双眼无神,口吐白沫,向着他摇头摆脑,声声嘶叫,如发悲鸣。

铁摩勒好生奇怪,心里想道:``这匹马神骏非凡,昨天还是好好的。昨晚又已吃饱了草料,今天才不过走了十多里路,怎的累坏?''

正自手足无措,对面走来了一个过路客人,到了他的眼前,忽地停下脚步,连声说道:``可惜,可惜!''铁摩勒一看,只见是个长身玉立的少年,相貌不凡,看来好似眼熟,却又想不起是在哪里曾经见过?

铁摩勒拱手说道:``兄台高姓大名,因何连呼可惜?''那少年道:``小姓展,贱名元修。我是可借你这匹马!''铁摩勒连忙问道:``怎么可惜?''展元修道:``尊驾这匹宝马是万中无一的良驹,可惜患了重病,只怕过不了今日了!''

铁摩勒大惊,忙道:``听见台之言,既然能一眼看出它患有重病,定然懂得医术,不知兄台叫能替它医治么?若蒙援手,小弟定当重报!''

那展元修双眼一翻,冷冷说道:``兄台你也未免太小觑我了,若是再提重报二字,小弟立即走开。''

铁摩勒面红耳赤,拱手赔罪道:``兄台原来是侠义中人,小弟失言,尚望恕过。请见台看在这匹马难得的份上,替它医治。''

展元修笑道:``这样说就对了。在下不懂什么侠义不侠义,只是平生爱马如命,实是不愿见这良驹死去。''

当下他就按着那匹黄骠马,在马腹上贴耳听了一会,那匹马又发出两声长嘶,还举起蹄想踢他,铁摩勒忙喝道:``他给你治病,你怎的不知好歹!''那匹马不知是听懂主人的话还是无力踢人,终于放下蹄子,服服贴贴的由他诊治。

展元修皱起双眉,说道:``它患的病很重,我也不知能不能治?姑且一试。''当下取出一管银针,管内满贮绿色的药水,在马腹上插了进去,过了一会。展元修将银针拔出,拍一拍马背道:``起来!''

说也奇怪,当真是药到病除,那匹马应声而起,可是它对展元修却似又害怕又愤怒的样子,扭头避开了他,四蹄在地上乱踢,踢得沙飞石走。

铁摩勒大喜道:``兄台真是妙手神医,小弟无以为报,只有说声多谢了。''

展元修道:``你现在多谢还嫌早了一点,你骑它走路,走出十里之外,若是仍然无事,那就是它的病已好了。若然有甚不妥,你牵它回来,我在路上等你,再给你想个办法。''

铁摩勒见那匹马精神抖擞,说道:``它已恢复了常态,想必不会再有不妥了吧?''当下再次拱手称谢,跨上马背,只见展元修却在他后面连连摇头。

果然走了不到十里,那黄骠马又口吐白泡,喘起气来,和刚才的病态一模一样、铁摩勒慌忙下马,依着那少年的吩咐,牵着黄骠马向回头路走。

走了一会,远远已看见展元修向他跑来,说道:``果然又有不妥了吧?幸亏我不敢走开。''铁摩勒心中一动,想道:``他既然早已诊断出来,何以又要我试跑十里路程,让这马多受痛苦?哎,莫非他是怕我不相信他的医术,故意显显本领,好叫我五体投地的佩服他?''

铁摩勒虽然心胸坦率,却也是个老江湖了,想到此处,反而怀疑起来。可是他转念一想,这匹马病重垂危,决不能弃它不顾,不管这少年用心如何,也只好信赖于他,把死马当活马医了。

铁摩勒心里怀疑,神色上却没有显露,他将那匹黄骠马牵到展元修的面前,说道:``兄台所料不差,它走了十里果然便走不动了。还望兄台设法救它一命。''

展元修道:``它的病已不是我所能治的了,不过,我还有个师父,他医马的本领当然比我高明十倍,\ldots\ldots 哎,我还没有请问兄台高姓大名。''

铁摩勒报了姓氏,却捏了一个假名,展元修续道:``铁兄,你若没有紧急之事,就请牵了这匹坐骑,随我同见家师如何?''

铁摩勒正是要赶往长安,可是他又实在舍不得这匹宝马,心中想道:``我已多赶了两天路程,就为这匹马再耽搁一两天,那也应当。要不然,我到了长安,如何向秦襄交代?''又想道:``此人虽是可疑,但我与他素不相识,未必他便要暗害我?何况我有一身武功,又何须惧怕于他?反正这匹马是要死的了,不如听他的话,试他一试。''

铁摩勒打定了主意,便说道:``若得尊师赐药救它,那是最好不过。就请展兄带引,同往谒见尊师吧。''

展元修再替那匹马刺了一针,那匹马略见好转,却远不如刚才的精神抖擞,而且好像对展元修更为惧怕,它挨着铁摩勒;时不时发出异样的嘶鸣。铁摩勒只当它是被银针刺体,因此才怕了展元修,也不放在心上。

走了一会,只见一座大山矗立前面。铁摩勒心中一凛,问道:``尊师是住在华山之中么?''

展元修道:``正是。他厌恶尘俗,在华山中过隐士的生涯已有十多年了。''

铁摩勒望见华山,不由得想起了``西岳神龙''皇甫嵩,又想起了王燕羽对南霁云所说的,夏凌霜的母亲可能也是被囚禁在华山的某处,不觉心意踌躇,脚步不前。

展元修道:``家师虽是住在华山,却是结庐在山谷之中,无须攀登危峰峻岭。''

展元修这么一说,铁摩勒登时放下了心上的石头,想道:``王燕羽说的所在是莲花峰下断魂岩,现在他的师父是住在山谷之中,显然是与这件事无关的了。''

铁摩勒牵着坐骑,随他走进山谷,山谷在两面山峰夹峙之下,虽是红日当头,谷中也是阴沉沉的令人感到寒意。

走了一会,只见一幢房屋,在山坡之上,依着山势修建,红墙绿瓦,气派不俗,屋前面还有花圃。一个丫鬟模样的少女,正在修剪花枝,见他们来到,忙跑出来迎接,喜孜孜地道:``少爷你回来了,这位可是请来的大夫?''展元修喝道:``好没规矩,在客人面前叫叫嚷嚷的,要你多管闲事么?快把这匹马牵到马厩里去,好生料理!''

铁摩勒疑云大起,心里想道:``听这丫鬟的称呼,这姓展的似乎是这里的少主人,屋内的主人应该是他的父亲,怎的他却说是他的师父?难道他的师父也就是他的父亲?''家学相传,以父亲兼任师父,事属寻常,但若是如此情形,为人子者决不会不称``家严''而称为``家师''的。另一样更令铁摩勒怀疑的是;自己来请他们医吗,那丫鬟却怎的反而把他当作了请来的医生?

展元修似乎已知道他起了疑心,笑道:``我师父一向和我同住,恰巧家中有人患病,家师今早叮嘱我到镇上去请医生,故而丫鬟有此误会。''

他越说铁摩勒越是疑心,问道:``这么说,兄台岂不是为了小弟之事,耽误了延医了?''

展元修道:``我师父深山隐居,不知外事,在这兵荒马乱的年头,镇上哪还请得到医生?铁兄你无须过意不去,我正有事奉商。请到里面去说。''

铁摩勒心想:``既来之,则安之。且看他有什么花样?''

展元修将他带进屋子,坐定之后,铁摩勒请见他的师父。展元修说道:``我的师父,你慢一步见也还不迟,兄台的坐骑,家师包保可以治好。只是小弟也有一件事,要请兄台相助。''

铁摩勒道:``彼此相助,份所应为,展兄请说,小弟尽力而为。''

展元修道:``那丫鬟虽是误会,但小弟也正有此意。想请铁兄给我的师妹治病。''

铁摩勒怔了一怔,说道:``我可是完全不懂医术的呀!''展元修道:``别的病铁兄也许不能医,敝师妹的病铁兄定能医治,要不然我也不会请你来了。''

铁摩勒惊疑不定:``莫非他们是黑道中人,受了敌人所伤?若然如此,金疮药我倒还有。''

展元修道:``能不能治,铁兄,你先看看再说吧!''

铁摩勒想了一想,说道:``好吧,我姑且看看,要是内伤,我就不能医了。''

展元修在前引路,经过了曲院回廊,到了那位小姐的厅房,展元修轻轻将房门推开半扇,说道:``铁兄,你悄悄走进去吧!''

铁摩勒从那半开的房门,先向里面张望了一下。一望进去,登时大吃一惊!

正是:情场无计相回避,今日冤家又聚头。

欲知后事如何?请听下回分解------

\chapter{第二十三回 情债难偿愁脉脉
相思未了恨绵绵}\label{ux7b2cux4e8cux5341ux4e09ux56de-ux60c5ux503aux96beux507fux6101ux8109ux8109-ux76f8ux601dux672aux4e86ux6068ux7ef5ux7ef5}

只见里面绣榻横陈,珠帘半卷,一个女子卧在床上,脸朝外向,星眸紧闭,带着病容,这女子正是王燕羽!

铁摩勒吃了一惊,转身便跑,忽觉劲风飒然,展元修的手指已摸上了他肩背,沉声说道:``铁兄,你不能跑!''

铁摩勒沉肩缩背,用了一招``霸王卸甲'',消去了他那一按之力,喝道:``你诱我到此,意欲何为?''

展元修如影随形,紧迫不舍,铁摩勒逃至中庭,展元修已抢快一步,堵住了门户,说道:``不错,是我诱骗铁兄,但却并无恶意,确确实实是想请你为我的师妹治病!''

铁摩勒一掌劈去,斥道:``胡说八道,你这厮分明是王伯通的党羽,想来陷害于我,哼哼,我虽然落了你们的圈套,你想要我束手就擒,那却是万万不能!''

展元修用绵掌的功夫,接连化解了铁摩勒刚猛之极的连环三掌,趁着铁摩勒换招之际,托地跳出圈子,说道:``铁兄,你已经亲眼看见她了,难道你还看不出她确是生病吗?怎的你不相信我的话?''

铁摩勒与他拆了几招,蓦地想起一人,喝道:``且慢,你是不是那日在龙眠谷救出王家老贼的那个蒙面人?''

当日那蒙面人虽然只是略施身手,但所用的都是上乘招数,所以铁摩勒的印象很深,他刚才与铁摩勒对掌,其中有一招就正是当日用过的。展元修道:``好,你既然看出我的来历,那你就更应该相信我了。''铁摩勒道:``哼,哼,你这话刚好要颠倒过来,你那日舍命救出了王伯通,还说不是他的党羽?''展元修道:``老实告诉你吧,王姑娘是我的师妹,我正是因为不愿意她跟那些强盗胡混,才把她从她父亲身边拉回来的。至于救她的父亲,那完全是为了她的缘故。并非我赞同王伯通的行为。当日,我救人的经过,你也是曾见到的了。不错,我是舍命救了他们,但我可没有伤害过你们的一个人。若然我是王伯通的党羽,辛天雄还有命吗?即是你那位韩姑娘,最少也要带点伤!''

铁摩勒想起那日他在辛天雄斧底救人,和在韩芷芬剑下拉走王燕羽的情景,心想凭他的武功这确也不是虚言,对他的敌意稍稍减了一两分,说道:``好,我姑且信你的说话,信你不是王伯通的党羽。那么,王伯通这老贼现在是不是在这儿?''

展元修道:``她父亲名利之心太重,妄想借外人之力,称王称霸,我劝不动他,只好由他去了。只留下了她的女儿在这里养病。''

铁摩勒心想:``这展元修纵使不是敌人,最少也是个是非不分的糊涂蛋,既然劝不动王伯通,为何不将他杀了?''铁摩勒是个恩怨分明、是非清楚的硬汉子,他却不想展元修是王燕羽的师兄,怎忍杀师妹的父亲,何况其中还有一段别情?铁摩勒总是要求别人都像他一样,因此往往不肯原谅人家。

展元修见铁摩勒神色不定,又钉紧一步道:``我的话已说得清清楚楚了,你当真是见死不救么?''

铁摩勒道:``你怎的歪缠不清,我不是说过了我不会治病的么?''

展元修冷冷说道:``我不是也说过了么,别人的病你不能医,我师妹的病你一定能医。只要你见一见她,说一声:是我来了。我看她的病就会好了一半!''说话的腔调,颇有点酸溜溜的味儿。

铁摩勒满面通红,在这瞬间,王燕羽和韩芷芬的影子同时在他脑中出现,他有点可怜王燕羽的痴情,同时也想起了未婚妻子临别的叮嘱,他蓦地大声说道:``你不知道你师妹是我的仇人?休说我不会治病,就是能治,我也不会救她!''

展元修道:``我知道她曾杀了你的义父,但,她不是也曾经救过你一次性命么?''铁摩勒道:``我在龙眠谷中不杀她,已经是报了她的恩了。''展元修冷笑道:``一个人的性命,也可以像债务一般,一笔一笔的计算清楚的么?''

铁摩勒的心剧烈地跳了一下,叫道:``不管你怎么说,我是非走不可!还我的马来!''

展元修道:``老实说,你的马是我弄坏了的,你不给我治病,你的马也绝好不了!''

铁摩勒固然舍不得这匹马,但却更怕见王燕羽,一怒之下,口不择言地骂道:``你这坏蛋,以后我再和你算帐。今天,我却是宁可不要此马,也决不理你歪缠!''

展元修也生了气,峭声说道:``好呀,我好心好意地请你来,你却骂人,老实说,不是看在我师妹的份上,我才不会对你这样客气!你不肯救人,今天要走,可是万万不能!''

铁摩勒道:``你不让走,我偏要走!''展元修冷笑道:``当真要走?你就试试吧!''呼的一掌,立即劈面打来,掌势既刚猛而又飘忽,与刚才大大不同!

幸亏铁摩勒早有防备,喝声:``来得好!''猛地一个翻身,双臂内圈,用了一招``斩龙手'',向对方的预项直劈下去。两人走的都是刚猛的招式,眼看就要碰上,展元修轻轻一闪,一变而为阴柔的擒拿手法,朝他的肘尖一托,五指合拢,一拂一抓,用了招``顺手牵羊'',要把铁摩勒活拿。

铁摩勒用招太猛,一时收势不住,险险就要跌进他的怀中,只听得``嗤''的一声,铁摩勒的衣袖被撕去了一幅。可是就在这间不容发之际,铁摩勒已是腾身掠起,在半空中一个转身,双臂箕张,严如饥鹰扑兔,掌势向他的顶门压下来!

展元修见他变招迅速,亦是吃了一惊,说时迟,那时快,只听得``蓬''的一声,两人四掌,已是碰个正着,铁摩勒居高临下,稍占便宜,展元修使出绵掌的功夫化解,兀自跄跄踉踉的倒退三步。

可是铁摩勒也不敢乘胜追击,原来展元修的绵掌善能以柔克刚,铁摩勒双掌似打中了一团棉花似的,不由得身向前倾,几乎立足不稳。还幸展元修的绵掌功夫,也尚未到登峰造极的境界,仅能卸开铁摩勒的掌力,未能及时反扑。

待到铁摩勒站稳脚步,展元修已是退而复上,展出了奇诡百变的招数,忽虚忽实,忽柔忽刚,或拍或接,或抓或拿,将七十二路擒拿手法混杂在``绵掌劈石''的招式之中,瞬息之间,但见四面八方都是展元修的影子!

两人的功力差不多,但铁摩勒擅长的是剑术而不是掌法,对付展元修这种变化莫测的掌法,时间稍长,便感到应付为难。好在铁摩勒曾从韩芷芬那儿学会了几招韩家的点穴手法,韩家的点穴手法神妙无比,到了危急之时,铁摩勒便突然使用出来,教展元修也不敢过份欺身进迫。打了将近半个时辰,兀自分不出胜负。不过,由于铁摩勒的点穴法未曾学全,来来去去是那几招,仅可以在危急之时作为护身之用,因此始终是他处在下风。

正在他们斗得紧张的时候,有一个人从角门走了进来,看了一会,说道:``这小子真是倔强,就似他的坐骑一样!嗯,禀少爷,那匹黄骠马已医好了,正在大发脾气,要闯出来,我已经用大石头顶着马房了。少爷,你要不要我请、请\ldots\ldots{}''

铁摩勒全神贯注的与展元修相斗,听到话声,才发现了这个人,一看,却原来就是昨日渡他过河的那个舟子。

铁摩勒恍然大悟,喝道:``原来你们乃是一伙,设下陷姘,骗我来的!''

展元修哈哈笑道:``不错,你现在才明白吗?是他通风报讯,是我将你的坐骑弄坏,这才请得你的大驾光临!你明白了也好,你试想想,我们费了如许心血,才请得阁下光临,岂能容你轻易走出此门!''

铁摩勒大怒,挥掌猛攻,展元修气定神闲的兀立不动,轻描淡写的便化解了他几招,这才转过头来笑道:``你瞧见了么,这小子虽然凶恶,料想我还有本领将他留下,你不必多事了!''

那``舟子''道:``是,是!不过,我是在想,少爷,你也实在不必费这么大气力,不如,不如\ldots\ldots{}''展元修喝道:``我叫你别管你就别管,退下!''

铁摩勒听他们的对话,那``舟子''似乎是他的仆人,要请什么人出来帮忙,展元修却不允许。铁摩勒霍然一惊,心中想道:``这是在他们家中,眼前这少年我已战他不下,要是再有帮手到来,那我可真要走不得了。哼,哼,我还和他们讲什么客气?''

展元修一掌拍下,铁摩勒忽地向后跃开,嗖的一声,拔出了佩剑,喝道:``再不让路我这把剑可从不得人了!''

展元修笑道:``你还要比试一下兵刃上的功夫么?好!主随客意,一定奉陪!大驾那是定要留的!''他随手折下了一枝树枝,迎风一抖,飓的便向铁摩勒刺去!

铁摩勒大怒,立即向树枝斩下,心中想道:``你敢藐视于我,且叫你识得厉害!''哪知展元修这枝树枝,竟似灵蛇游走,刹那间就从铁摩勒的剑底钻了出来,上刺铁摩勒的双目,铁摩勒一念轻敌,几乎吃亏。

展元修那枝树枝,挥动起来,呼呼风响,劲道十足,实在不亚于一枝长剑,可是它究竟是枝树枝,眼看就要刺中铁摩勒,却给铁摩勒用衣袖排开了。

铁摩勒轻敌之心一去,登时站稳了脚步,将长剑霍霍展开,这一来便轮到展元修吃了轻敌的亏了。他因为在掌法上占了上风,对铁摩勒的本领估计不足,哪知铁摩勒本来不长于掌法而是长于剑术,若然展元修换了一把真剑,也许还可以对付,现在用的只是一枝树枝,就不免相形见绌了。

转眼间斗了三十来招,铁摩勒一剑紧似一剑,剑招催动,如长江大河,滚滚而上。展元修只有用腾挪闪展的功夫闪避,连招架也感到为难。正在吃紧,忽听得一个苍老的声音说道:``燕儿梦里也念着的就是这小子吗?''

园门开处,一个满头白发的老婆婆走了进来。就在这时,只听得``咔嚓''一声,展元修那枝树枝已给铁摩勒一剑削断。

展元修退到那个老婆婆的身边,说道:``妈,正是这个小子!''那老婆婆厉声喝道:``给我站住!''

铁摩勒道:``对不起,我还要赶路。''正要闯出园门,忽见那老婆婆身形一晃,喝道:``乖乖的给我躺下来吧!''

铁摩勒见她年迈,且又双手空空,并无兵器,因此虽然迫于无奈,也只好一剑刺去,不过只用了三分力道,指向她的咽喉,用意是想把她吓退而已。

哪知这老婆婆却一声冷笑,厉声斥道:``你敢小觑我!''话声未了,长袖一挥,铁摩勒顿觉一股大力卷来,招数未曾用实,长剑己给她的衣袖卷去。咣啷一声,插在假山石上,火花四溅!

铁摩勒这一惊非同小可,正要闪开,那老婆婆长袖再挥,铁摩勒的身法已经快极,还是躲避不开,脚跟刚刚离地,就正好给她卷住,提了起来。那老婆婆道:``不是看在你对老年人尚有点礼貌,还要叫你多吃些苦头!''衣袖一挥一送,铁摩勒在半空接连翻了三个筋斗,摔得发昏,展元修随即将他擒住,点了他的穴道。

那老婆婆嘿嘿的冷笑几声,向铁摩勒端详了好一会子,说道:``人长得还漂亮,武功也很不错,怪不得燕儿会喜欢他。元儿,你就甘心认输了么?''

展元修道:``他的剑术是比我高明。''

那老婆婆双眼一瞪,说道:``你是真不懂还是假不懂,我说的不是武功!''

展元修低下了头,道:``燕妹喜欢他,我不认输也没法子。''

那老婆婆``哼''了一声,说道:``我当年也不欢喜你的父亲,结果还不是嫁了他了。''顿了一顿,又问道:``听说这小子的义父就是给燕儿杀掉的,你知道么?''

展元修道:``正是因为这个缘故,这小子咬牙切齿的始终把燕妹当作仇人,不肯给她医病。''

那老婆婆冷笑道:``天下竟有你们这样的两个傻小子!一个喜欢她的仇人;另一个却将他的敌人请来,给他所喜欢的人治病。哼,我劝你别打这个傻主意啦,干脆的把这小子杀了,断了她的念头,岂不一千二净。''说到此处,那老婆婆的手臂缓缓举了起来,说道:``姓铁的小子,你认命了吧!''

展元修大吃一惊,慌忙托着他母亲的手臂,颤声叫道:``不可!''

那老婆婆以眼一睁,淡淡说道:``除了杀他,你还有什么法子?''

展元修低下了头,现出了痛苦的神情,说道:``我不知道。不过,不过,我总是不想、不想让燕妹伤心。''

那老婆婆愠道:``大丈夫做事岂能畏首畏尾,哼,你简直不像是展龙飞的儿子!你父亲生前杀人如草,哪有像你这样婆婆妈妈的!''

铁摩勒心头一震,这才知道这个老婆婆乃是大魔头展龙飞的妻子,展龙飞死得早,他是被各正派的人物围攻,因而重伤致死的,那时铁摩勒还在襁褓之中。不过,他的父亲铁昆仑和他的师父磨镜老人都是参加围攻的人物之一,所以铁摩勒对他的事迹耳熟能详,并且知道他的妻子也是像他一样心狠手辣的女魔头。在展龙飞死后,他的妻子销声匿迹,经过了这许多年,江湖上从未见过她露面,大家都以为她也早已死了,哪知道还在此间;铁摩勒知道了她的来历,不禁寒意直透心头,想道:``落在这女魔头的手中,只怕是凶多吉少了!''

果然,铁摩勒心念未已,便听得展大娘一声喝道:``你走过一边,我替你了断!哼,你还要拦阻么?你懂不懂得,我杀这小子乃是为你!''

展大娘将她的儿子一把推开,手臂又举了起来。

就在这时,忽又听得一个尖锐的声音叫道:``师父,你连我也杀了吧!''只见王燕羽满面惊惶焦急的神情,颤巍巍地走来,她本来就在病中,这一来更显得花容憔悴,娇怯可怜。

展大娘道:``燕儿,你竟是这样的爱这小子吗?你也来向我求情?''

王燕羽道:``我不敢向师父求情,只是想请师父成全,将我也一同杀了!''

展大娘似乎很疼惜王燕羽,听了她这番以死要胁的``求情''说话,手臂又徐徐放下,她想了一想,忽地说道:``好,我成全你的心愿。你在一旁听着,待我来问问这个小子!''

展大娘将铁摩勒拉了起来,解开了他的穴道,阴沉沉地说道:``燕儿与你有缘,为了你,她不惜以死相救,现在就看你了,你愿不愿娶她?我今天就让你们成亲!怎么样,你到底怎么样?说呀!''

这刹那间,铁摩勒心情混乱之极,他面临着一个最难答复的难题!

形势摆在面前:要是他说一个``不''字,便将毙在这女魔头的铁掌之下。

铁摩勒并不怕死,可是,不知怎的,当他一触及王燕羽的目光,就禁不住整个身心都颤抖起来。王燕羽扶着花枝,那张娇怯可怜的脸孔正盯着他,那是充满着惶恐的、期待的、焦急的而又柔情似水的目光,铁摩勒知道,要是他说一个``不''字,只怕王燕羽也会像一朵突然遭受风雨摧残的鲜花,枯萎了的!

这几年来,铁摩勒念念不忘给义父报仇,以手刃王家父女为快。经过那次帐幕之夜,王燕羽的爱意表露无遗之后,他的仇恨大部分转移到她的父亲的身上,可是对她的恨意也还未全消,他可以不杀她,但若说到要化敌为友,却是不能想象的事!

可是,铁摩勒现在对王燕羽的目光,任他是铁石的心肠,也终于动摇了。他能够把这样爱他的人当作仇人吗?他能够让这个少女像鲜花一样的枯萎吗?不,这也是不能想象的事!

铁摩勒片刻间转了无数念头,突然,另一个少女的影子在他眼前浮现,这是韩芷芬的影子,他记起了韩芷芬临别时的叮咛嘱咐,他忆起了韩芷芬含愁责备的目光,他能够对未婚的妻子忘恩负义吗?不,这也是不能想象之事!

铁摩勒咬了咬牙,避开了王燕羽的目光,终于摇了摇头,说道:``王姑娘,我感激你的好意,我又一次欠上你的债了。只是我已经有了另外的人,她也是像你一样可爱的姑娘,我不能够抛弃她,你,你把我忘记了吧!''

王燕羽痴痴地听着,她苍白的脸上现出一丝微笑,那是因为她听到铁摩勒说她是个``可爱的姑娘'',但是这却是凄惨的笑容,因为她也从铁摩勒的话中,听出了他对韩芷芬的深情厚爱!甚至在死亡的阴影之下,韩芷芬在他心中的位置也难以动摇!

铁摩勒的话刚刚完毕,展大娘便冷冷说道:``燕儿,你听清楚了么?你愿意嫁他,他却不愿意娶你!他已经有了另外的人了!''

展元修叫道:``妈、妈、你、你、''他想说的是``你少说两句行不行?''但在母亲的积威之下,他这样顶撞的话儿在舌头上打了几个滚还不敢说出来。就在这一瞬间,只听得一声尖叫,王燕羽倒下去了!

展元修连忙跑过去将她扶住,展大娘冷冷地望了他们一眼,说道:``她是一时气昏了,你把她放下,你走过来!''

展元修道:``妈,你有什么吩咐?''展大娘道:``你把这把剑拔下来!''她指的是铁摩勒那把青钢剑,刚才在铁摩勒和她交手之时,给她拂落,正巧插在一块假山石上的。

展元修莫名其妙,拔了下来,问道:``这又不是一把宝剑,妈要它作什么?''展大娘冷冷说道:``谁希罕他这把剑?我是要他丧在自己的兵刃上。元儿,你给我将这小子一剑杀了!''

展元修吓了一跳,咣啷声响,那把剑又跌落地上。展大娘道:``真没出息,枉你是展龙飞的儿子,连杀人都没有胆量吗?''

展元修叫道:``妈,你叫我杀别的人还可以,我就是不能杀他!''

展大娘道:``你燕妹喜欢这个小子,这小子又不愿娶她。她也应该断了念头了。还留这小子何用?好,你不肯杀他,待我来杀!''

展大娘这个``杀''字刚一出口,人已走了过来,第三次举起手掌,朝着铁摩勒的脑门击下!

展元修叫道:``杀不得,杀不得!''拦在铁摩勒身前,拼命的托着他母亲的手臂!

展大娘手臂一振,将展元修摔了一个筋斗,手掌停在离铁摩勒脑门三寸之处,``哼''了一声道:``为什么杀不得?''

展元修顾不得疼痛,一个``鲤鱼打挺''翻起身来,便即说道:``妈,你不能够为你的儿子想一想么?''

展大娘诧道:``我要杀这小子,正是为你设想啊!你想要燕儿做你的妻子,是吗?''展元修道:``不错,我是有这念头。''展大娘道:``着呀!那你为什么还要留着这小子在世间碍眼?杀了他岂不正是斩草除根?''

展元修道:``你看燕妹已经这样伤心,要是杀了他,只怕燕妹病情更为恶化,那却如何是好?''

展大娘道:``这小子一点也不念她的情义,她就算一时伤心,伤心过后,也会说我杀得对的!''

展元修道:``妈,你又不是不知燕妹的脾气,宁可让她自己去杀,要是咱们杀了她喜欢的人,她这一生还会理睬我吗?''

展大娘道:``依你之见如伺?放了他?''展元修道:``放了他又怕燕妹醒来之后还要见他,或者疑心咱们害了他。''

展大娘道:``好,娘就暂时把他关起来吧!待到燕儿答应做你的妻子,我再放他!''

展元修满面通红,叫道:``妈,你不能这样做,这,这,这太令我难堪了!''

展大娘冷冷一笑,随手一拂,点了铁摩勒的昏眩穴,令他失了知觉,这才说道:``傻孩子,你以为妈当真要放这小子吗?我这不过是想燕儿嫁你。待到燕儿答应了做你的妻子,我自然有办法整治他!''

展元修打了一个寒襟,道:``妈要怎样整治他?''展大娘道:``我当着燕儿的面放他,暗地里却在他的饮食放下败血散,叫他未到长安,就要身罹重病,死在路上!''

展元修听得皮肤起栗。不错,他对铁摩勒的确是心怀妒恨,但他却是有几分傲骨的人,他不愿意用要胁的手段迫师妹嫁他,他要的是王燕羽的心,而不是王燕羽的身子。他之所以觉得``难堪'',就是因为母亲要采用这种不顾他面子的做法,可是展大娘却误会了儿子的意思。

展大娘挥了挥手,说道:``好,事情就这样定夺了。姑且让这小子多活几天!''

展元修踌躇片刻,忽地说道:``妈,我还有话说!''

展大娘道:``你还要说些什么?你不过是想要师妹做你的妻子罢了,难道你当真舍不得杀这小子么?''

展元修道:``正是我想亲手杀这小子,才解我心头之恨!妈!你将那败血散给我,待到你要放他那一天,我就用它。我要亲眼看着他在我的面前服下毒药!''

展大娘哈哈大笑说道:``这才不愧是我的儿子!好吧!败血散这就给你!你把这小子关在地牢里,我替你料理燕儿。嗯,这次的气也真够她受了,现在尚未醒来。''

展元修抱起了铁摩勒,走了几步,又回头说道:``妈,燕妹醒来,请你不要先和她说那些话。让我来说。''

展大娘说道:``燕儿是聪明人,她知道了我关了这个小子,还会不明白我的意思吗?连你也不用说。讲得太过明白,反而大家的面上都没有光彩!''

展元修听着他母亲得意的笑声,心头就像压了铅块般的沉重,想道:``怪不得江湖上的豪杰,听到我父母的名字,没有不痛骂的!他们当年所做的事情,我虽然不大知道,但看妈这次的所作所为,也就不难想象了。''

铁摩勒在黑暗中醒来,四围摸索,手指碰着了冰冷的石壁,这才知道自己已经变成了囚徒。铁摩勒大为愤怒,挥拳骂道:``你们将我骗到此间,却又为何不将我干脆杀了,哼,哼,世上的坏人我也见过不少,就没见过像你们这样卑劣的!''他越骂越气,``砰''的一拳击在墙壁上,被那反震之力震倒地上,周身骨节隐隐作痛。原来他是被展大娘用阴狠的独门手法点了穴道,还幸亏展元修一将他关进地牢,便给他解穴,要不然,若是时间较长,那就不止骨头疼痛而已,内脏还要受伤。

铁摩勒骂得力竭声嘶,无计可施,只好在地上盘膝而坐,运气调元。黑暗中也不知过了多久,忽听得头顶上有``轧轧''声响,抬头一看,只见头顶上方开了一个洞口,有一只小篮子吊下来,篮内盛满饭菜,转瞬间那洞口又关上了。

铁摩勒大叫道:``姓展的,你若还有一点男儿气概,就放我出来,与我决一死战!''外面的人回答道:``我与你无冤无仇,为何要与你拼死,你安心养息几天吧!''果然是展元修的声音。随即便听得沉重的脚步声,像是他故意要让铁摩勒知道他已经走了。

铁摩勒正自饿得发慌,小篮子内的饭菜发出香喷喷的气味,铁摩勒心道:``反正我这条命是在你们手上,就算你们放了毒药,我也乐得先吃个饱。''

铁摩勒吃饱之后,精神大大恢复,他将所遭遇的一连串事情回忆了一遍,心中想道:``这姓展的将我骗到此间,当然不是正人君子,但比起他的母亲,却要好得多了。''再想到他这样做,都是为了爱王燕羽的缘故,而王燕羽却不爱他,想到此处,他对展元修的敌意便减了几分,反而有点同情地了。

最令得铁摩勒焦急的,是他负有使命,要赶往长安,现在被关在地牢,只怕死了也无人知道,要想有人来救,那更难了。他想到闷处,自己给自己开解道:``我本来不想做皇帝的保镖,若是因此丢了差事,南大哥也不能责备我。唉,我也真傻,连生死都尚未可知,却还要想到南大哥的责备。''

黑暗中不知时日,但那小篮子是每天三次准时吊下来的,铁摩勒从送饭的次数可以算得出所过的日子。到了第三天中饭送过之后,他正在烦闷,忽地那扇石门打开了半扇,有一个人走了进来。

铁摩勒倏地跳将起来,一掌便打过去,放声骂道:``贼婆娘,你还有什么阴狠的手段。我干脆与你,与你------''``拼了''那两个字还未曾吐出口来,铁摩勒突然呆住,张大了嘴巴,做声不得,他的手指触处,温较如绵,幸而他的劲力已到了收发随心的境界,未曾把对方打伤。

只见那人晃了两晃,低声说道:``摩勒,你还是这样恨我吗?''

铁摩勒处在黑漆的地牢中,他一眼望去,只隐隐约约的辨得出是个女的,只当是那女魔头展大娘,却不料是王燕羽!

铁摩勒手足无措,呆了片刻,方始歉然说道:``是你?我还以为是你那狠毒的师父呢。''

王燕羽道:``你恨我也是应当,说起来,其实你与其恨展家的人不如恨我,你所受的灾难都是我引起来的,我又是你的仇人!''

王燕羽自动的先提出了往日的冤仇,铁摩勒的心头登时似着了火烧一般,不由得想起义父被她惨杀的情景,耳边似乎听得义父的声音说道:``摩勒,是你替我报仇的时候了!''

不错,要是铁摩勒现在动手报仇,那确是不费吹灰之力。休说王燕羽尚未曾病好,即算她已康复如常,听她那语气,大约也不会抵抗的。

可是铁摩勒怎能杀一个尚在病中的女子?他在黑暗中过得久了,眼睛渐渐习惯,这时已不止是辨认出了王燕羽面部的轮廓,还隐约看得出她那幽怨的神情。他和王燕羽面面相对,听到了她短促的呼吸,忽然,只见一颗晶莹的泪珠从她的眼角滴下来!

铁摩勒的铁石心肠都在这颗泪水中溶化了,他义父的影子也在泪水中模糊了,眼前是一个有血有肉的真人,是王燕羽俏生生的影子!

铁摩勒突然转过了头,一个字一个字地说道:``从今之后,我与你的冤仇一笔勾销,是生是死,都不恨你!''声音颤抖而又沉重,显见他的心情激动非常。

王燕羽叫道:``啊!摩勒!摩勒!''她将摩勒的名字叫了两遍,就硬咽住了,说不出话来,不知不觉的,她紧紧抓住了铁摩勒的手。

铁摩勒缓缓转过头来,可是仍然不敢面对她的目光,他想挣开,但终于还是让王燕羽将他的手紧紧握住。这刹那间,他感到了羞愧,却又得到了几分``如释重负''的轻快心情!

想起了未婚妻子的临别叮咛,他感到羞愧;但他心头上的一个``结''却解开了,在这之前,他常常为了自己与王燕羽之间的恩怨纠缠而烦恼,``要不要向她报仇?''成为了一个困惑他的问题,现在他已亲口向王燕羽答应,不再将她当作仇人,亦即是这个长期困惑他的问题,已经得到了解决了。

两人紧紧握着手儿,默然相对,彼此都感到对方跳动的心声。过了好一会子,王燕羽方始吁了口气,说道:``摩勒,你真好!尽管你不欢喜我,我还是会记得你的好处的!''

铁摩勒感到不安,轻轻的将她的手格开,说道:``王姑娘,过往的都别提了。从今之后,你忘记了我吧。嗯,我觉得你的师父虽然狠毒,你的师兄却还不算坏人。''

王燕羽道:``不错,我的师兄的确是对我很好,我已经答应了师父,愿意做他的媳妇了,你、你可以安心了吧?''

铁摩勒又喜又忧,喜者是王燕羽有了着落,忧者是从她的语气之中听得出来,她之肯答应嫁给她的师兄,并不是由于心甘情愿,而不过是仅仅要使自己``安心''!

黑暗中王燕羽看不真铁摩勒脸上的神情,但铁摩勒自己却感到了脸上一阵阵发热,他低下了头说道:``好,那我要恭喜你啦!''王燕羽道:``我却还未曾恭喜你和韩姑娘呢!''她这几句带着笑声说出,却又似笑非笑,似哭非哭,听得铁摩勒甚为难过。

铁摩勒连忙说道:``王姑娘,我多谢你来看我,咱们的话已经说得清清楚楚了,你还是回去吧,免得你的师兄多心。''

王燕羽道:``不错,我是应该回去了。我还没有将我答应婚事的事情告诉师兄呢。''她离开了铁摩勒的身边,行了两步,忽又停了下来,轻声唤道:``摩勒,摩勒!''

铁摩勒心头一震,道:``王姑娘,你请回吧!''王燕羽道:``摩勒,你也应该回去了。''

铁摩勒怔了一怔,道:``我回去哪儿?''王燕羽道:``你回到你韩姑娘那儿也好,回到你南师兄那儿也好,那是你的事情,怎么问我?''

铁摩勒吃了一惊,道:``你要放我走么?''王燕羽道:``你总不能在这地牢里过一辈子!''铁摩勒道:``你不怕你的师父责怪?''王燕羽道:``她总得给她未来的媳妇几分面子。''

铁摩勒心乱如麻,不知是领她的情好还是不领她的情好,踌躇间忽听得展大娘那尖锐的声音叫道:``燕儿,燕儿!''王燕羽忙道:``你快走吧,再迟就来不及了。''她打开了门,倏的就将铁摩勒拖了出去。

忽听得一个颤抖的声音低低的``咦''了一声,铁摩勒睁大了眼睛一看,只见展元修就站在门边,这时王燕羽还在拖着他的手,铁摩勒禁不住满面通红,尴尬之极。

展元修怔了一怔,看到了这个情形,他全都明白了,他脸上的肌肉抽搐了一下,挥挥手道:``好,你们都走吧!''

铁摩勒连忙分辨道:``只是我走,你,你不要误会了她!''展元修望了铁摩勒一眼,却不理会他,自转过头来,低声对王燕羽道:``燕妹,你也赶快走吧!那老叫化上门来啦!他,他要找你晦气!''

铁摩勒听得``老叫化''三字,心头一动,想道:``在华山上住的老叫化没有别人,敢情是西岳神龙皇甫嵩来了?''

王燕羽冷冷一笑,淡淡说道:``我早料到他会亲自登门,我做的事我自己担当,怕他怎的?''

展元修道:``料想妈也不会让你吃亏,不过妈的脾气很特别,喜怒无常,难说得很。我看你还是避开这个老叫化的好!再说,那老叫化一定是认识铁兄的,若给他发现了铁兄在这里,只怕又生枝节!''

王燕羽道:``我先送他下山,然后回来!''展元修的眼睛眨了一眨,王燕羽这话似乎颇出他意料之外,他脸上沉暗的神色也开朗了一些,说道:``也好,那么在妈的面前,我给你暂时敷衍一阵,你们走过前面院子的时候,可要特别小心!''

展大娘那尖锐的声音又在叫道:``元儿,元儿!''展元修连忙提高了声音应道:``来啦!来啦!''匆匆忙忙的便跑了进去。

王燕羽仍然拖着铁摩勒的手,走过一道回廊,便到了前面的院于,正好听得屋子里展大娘的声音在问道:``燕儿的病好了点么?怎么她不出来。''

王燕羽拉着铁摩勒,两人一同躲在一块假山石的后面,只听得展元修在回答道:``燕妹的病昨晚本来已好了些,可是今天又沉重了,她起不了床。''

这时,铁摩勒在假山石的后面渝窥进去,已经看得清清楚楚,和展大娘同在屋子里的那个人,果然是西岳神龙皇甫嵩!只是他穿着一身光鲜的衣裳,并非化子打扮,看起来没有以前所见的那么苍老。

展大娘道:``皇甫先生,小徒委实是患病卧床,没法出来。''

皇甫嵩脸儿朝外,只见他的眼珠滴溜溜地转了几下,忽地说道:``展大娘,请恕我无礼,这件事我一定要查个明白。令徒既然患病在床,我就亲自去看她吧!''

展大娘道:``这怎么敢当?''皇甫嵩道:``龙眠谷的王家大寨已经给段-
璋和南霁云这些人挑了,若是他们知道我在这里,必定会前来寻事,嘿嘿,到了那时,只怕对你老人家也有不利。我看,还是得赶快向令徒查问清楚才好。''

展大娘有点不悦,说道:``我这小徒虽然不知轻重,作事任性,但想来还不至于胳膊向外弯,帮她父亲的仇家!不过,皇甫先生既然相信不过,要亲自查问小徒,我就陪你去吧,问清楚了,也好叫你放心。''

铁摩勒听得心头一震,想道:``听这皇甫嵩的话语,竟是与王伯通这老贼同一鼻孔出气的,不但如此,他怕我的南师兄找他晦气,敢情夏姑娘的母亲也真是被他囚禁的了?''铁摩勒因为皇甫嵩以前曾救过他和段-
璋脱难,不管旁人议论如何,他对皇甫嵩却是颇有几分好感的,如今听了这番说话,那几分好感登时变为恶感,``我以前还不相信他真是坏人,谁知却是我给他的假仁假义骗了。''

心念未已,展大娘这一行人已走出台阶,展元修心惊胆战,神色上显露出来,展大娘何等厉害,``咦''了一声,问道:``元儿,你怎么啦?''展元修道:``有点不大舒服。''展大娘``哼''了一哼,停下脚步,游目四顾,忽地一声喝道:``是谁在那里躲躲藏藏的?出来!''

王燕羽知道躲避不过,应声便道:``是我!''展大娘见她和铁摩勒并肩走出,面色大变,冷冷说道:``你要和这小子离开我吗?''

展元修忙道:``妈,你不是说要放铁兄走吗?我刚才已给他饯行了,是我请燕妹送他下山的。''一边说一边向他母亲眨眨眼睛,意思似道:``在外人面前,请恕我不便直说。''

铁摩勒莫名其妙,不知展元修何以要捏造谎话,说是已给他饯行?展大娘却是心领神会,暗自想道:``哦,原来元儿已经知道燕儿答应了做他的媳妇,也给这小子服下了败血散了!''面色缓和下来,说道:``燕儿,皇甫先生有事要问你,不必你送他下山了。'''

王燕羽大喜,说道:``摩勒,你自己走吧。你的马在马厩里,你问前日送你过河的那个人要,他在园子里。''

皇甫嵩哈哈笑道:``原来王姑娘的病早已好了,可喜可贺。''眼光一转,忽地停在铁摩勒身上,问道:``这位是谁?''

铁摩勒大为诧异,他因为恼恨皇甫嵩,所以刚才出来的时候,正眼也不看他。但他却想不到皇甫嵩竟会问起他是谁来?就在这时,只听得展大娘已经回答他道:``皇甫先生不认得他吗,他就是以前`燕山王'铁昆仑的儿子铁摩勒!''

皇甫嵩作了个诧异的神情,说道:``原来你已与那磨镜的老儿和解了么?当真是意想不到!''

展大娘双眼一瞪,道:``皇甫先生,你这话从何而来?''皇甫嵩道:``你若然未曾与磨镜老人和解,怎的他的徒弟会在你的府上?''

展大娘面色倏变,叫道:``什么,这姓铁的小子是那磨镜老儿的徒弟么?''皇甫嵩哈哈一笑,立即接着她的话语说道:``我正奇怪你老人家怎会把杀夫之仇忘了,原来你还未知道这姓铁的来历,我虽然也不认得他,但江湖上谁不知道:铁昆仑的儿子铁摩勒是磨镜老人的关门弟子!''

展大娘听了这话,立即回过头来,阴沉沉地说道:``原来你是磨镜老人的高足,恕我不知,怠慢你了。你多留一会儿,等下我再亲自给你饯行!元儿,你陪着他!''

王燕羽的面色``唰''的一下变得苍白如纸,展元修也吓得嫔足颤战了。他们当然知道展大娘所说的``饯行''是什么意思,展大娘扫了他们一眼,厉声悦道:``在我的眼皮底下,你们不用再打什么主意了。姓铁的小子,你不进来,要我亲自去请你么?''

铁摩勒情知决难在展大娘与皇甫嵩的手下逃得出去,索性大大方方便走进屋来,大马金刀的坐在椅子上,看她怎样发落。

那展大娘却不理会他,自向王燕羽说道:``燕儿,你过来,皇甫先生有话问你。''

皇甫嵩冷冷的看了王燕羽一眼,说道:``我已与你的哥哥见过了,听说就在龙眠谷出事那天,我给他的那包夺魂香的解药突然不翼而飞,那位中了毒的夏姑娘也突然恢复如常,这件事可真有点奇怪!那包药藏在你哥哥的房中,别人决计不能知道!王姑娘,你是他的妹妹,你可知道是谁干的么?''

王燕羽眉毛一挺,冷笑道:``皇甫先生,你说话不必绕圈子啦,你既然怀疑了我,何不直接的说出来?不错,这事情是我干的!偷解药给夏姑娘的是我!''

皇甫嵩道:``那么,你有没有告诉那位夏姑娘,说她的母亲是我掳的?''王燕羽道:``这倒未曾!''皇甫嵩道:``真的?''王燕羽道:``我做的事我自己担当,有一句就说一句,难道我还怕你把我吃了不成?''皇甫嵩哈哈笑道:``真不愧是展大娘调教出来的好徒儿,这副倔强的脾气倒真令老夫佩服!我岂敢将你难为,只是要问个明白。那么,你可露出口风没有,比如说,将她母亲的下落告诉她?''他的话声方了,王燕羽立即答道:``有!''

皇甫嵩面色大变,况声问道:``你怎么对夏姑娘说?''王燕羽道:``我不是对夏姑娘说的,我是对她的未婚夫说的,我告诉他,他若是要找人的话,可到莲花峰断魂岩下!''皇甫嵩道:``她的未婚夫是谁?''他声音急促,似乎等待一个渴欲知道的消息,王燕羽也有点愕然,想不到他突然把紧要的事情放过一边,却盘问起夏凌霜的未婚夫来了。

王燕羽道:``夏姑娘的未婚夫就是江湖上鼎鼎大名的南大侠,南霁云!''

皇甫嵩呆了一呆,叫道:``怎么会是南霁云?哼,这南霁云不也是磨镜老人的徒弟么?''王燕羽道:``你奇怪什么?夏姑娘和南大侠相配有哪点不对?''

皇甫嵩霍然一惊,定了定神,说道:``王姑娘,我是说你!你怎么胳膊向外弯,反转过来帮你父兄的仇人,这,这可有点不对了!''

王燕羽道:``我的师父在这儿,不劳你来管教!''她知道师父的脾气,即使要将她责打,也决不容外人越俎代庖。

果然展大娘瞅了皇甫嵩一眼,便冷冷说道:``皇甫先生,你无非是怕你的仇家来捣你的老巢罢了,你我既定下守望相助之约,若是事情临头,我自不能坐视,你怕什么?你回去吧,我的家事,我会料理。''

皇甫嵩正是要她这句话,当下立即施礼说道:``多谢你老人家鼎力扶持,不过,咱们的强敌不少,风声已然泄漏出去,只怕这几天就会有人寻上门来,你老人家也该小心一些!''

展大娘道:``我知道啦,我这二十年的光阴是白过的么?但正要会会昔日的仇人,试试我的功夫,就怕不是他们上来。要你担心作甚?''

展大娘说了这番话,就不再理睬皇甫嵩,转过眼光,盯着王燕羽道:``燕儿,你做得好事,你过来!''

王燕羽见她师父面似寒露,她师父虽然凶恶,向来却也还未曾用过这样难看的面色对她。王燕羽本来在救铁摩勒的时候,就打定了主意:天塌下来也不管的了,这时在师父的威严之下,也不禁心里发毛,硬着头皮说道:``徒儿不该做的也已做了,要杀要剐,听师父的便!''

展大娘眼光一瞥,只见她的儿子也在一旁发抖,她叹了口气道:``你这两个冤家!''神情缓和了一些,对王燕羽道:``你且站过一边,待我先发落这个小子!''一个转身便到了铁摩勒的身前。

皇甫嵩说是要走却还未肯爽爽快快地走,这时他索性停下脚步,等着看展大娘如何将铁摩勒发落。

展大娘站在铁摩勒面前,阴森森的眼光紧紧地盯着他,一声不响,也不知是打什么主意。王燕羽几乎是屏息了呼吸,全神贯注的注视着她师父的动作。

皇甫嵩留意到王燕羽对铁摩勒的关心情态,恍然大悟:``我道王伯通的女儿为什么会反过来帮助仇家,原来就是为了这个小子!''

他见展大娘迟迟未肯出手,心中又是奇怪,又是着急,深怕展大娘为了爱徒之故,放走了铁摩勒。

皇甫嵩正想说几句话激怒展大娘,忽见展大娘的面色越发沉暗,突然``哼''了一声道:``元儿,你好大胆,你竟然敢欺骗你的母亲!''原来她已看出了铁摩勒气色如常,显然并未曾服下什么败血散。

展元修颤声叫道:``妈,你不是说过要为我着想,不,不杀他的吗?''展大娘大怒道:``你好没出息!''这句话包含了好几层意思,既是恼怒儿子的心肠不够硬,不够狠,又是恼怒儿子为了要讨好妻子的缘故,竟然``没出息''到要庇护妻子的情郎。

只听得``蓬''的一声,展大娘已一掌向铁摩勒的顶门拍下,王燕羽一声惨叫,扑上前去,拼命地扳着她师父的手臂!展元修略一迟疑,也扑上前去,扳他母亲的另一条臂膊。

铁摩勒早就蓄势以待,但他出尽全力,硬接了展大娘这一掌,仍是禁不住给她震得跌出一丈开外,还幸亏有王燕羽与展元修合力阻拦,展大娘的掌力未能尽发,铁摩勒虽然跌倒,却未受伤。

王燕羽叫道:``你快跑呀!''皇甫嵩忽地接着冷笑道:``王姑娘,你不用操心了,还有我呢!这小子怎跑得了?''

皇甫嵩跳出门口,拐杖一挥,就向铁摩勒打去,铁摩勒早已拔出展元修还给他的那柄佩剑,反手一剑,使出了``神龙掉尾''的杀手神招!

皇甫嵩的功力略逊于展大娘,剑杖相交,只听得``蓬''的一声,铁摩勒后退三步,却未跌倒。不但如此,他这一招``神龙掉尾''刚猛之极,竟把皇甫嵩的紫檀木杖也削去了一小块,而且震得皇甫嵩的虎口也微感酸麻。

皇甫嵩大怒,第二杖、第三杖接连打来,铁摩勒的功力究竞尚不如他,接到了第三招已是难以抵挡,眼看他又是一杖打来,铁摩勒只好使个``云里倒翻''的身法,急忙后退。

皇甫嵩正要赶上,忽地听得半空中呜呜的声响,刺耳非常,皇甫嵩大吃一惊,连忙抬起头来观看,顾不得要去杀铁摩勒了。

正是:自有奇兵天外降,伫看剑气荡魔氛。

欲知后事如何?请听下回分解------

\chapter{第二十四回 追寻狡兔翻三窟
惊见魔氛盖九天}\label{ux7b2cux4e8cux5341ux56dbux56de-ux8ffdux5bfbux72e1ux5154ux7ffbux4e09ux7a9f-ux60caux89c1ux9b54ux6c1bux76d6ux4e5dux5929}

皇甫嵩抬头一看,只见东南角的上空,有一团黑烟袅袅上升,这正是他同伴报警的讯号。原来他这次来拜会展大娘,虽然预计逗留的时间不会很久,但也怕就在这个时间之内,会有人来捣他的老巢,因此出门之时,便与同伴相约,若然发现敌踪,便立即吹起胡笳,点起烟火。他这个同伴,也是邪派中一个高手,那次皇甫嵩纠众去劫夏凌霜母女,他和精精儿都是皇甫嵩的帮手。事后精精儿要回范阳,皇甫嵩为了怕强敌来攻,故此留下这个邪派高手,与自己作伴。

铁摩勒趁着他吃惊之际,早已跑了出去,直奔后园。展大娘将儿子摔开,这时也已奔了出来。

皇甫嵩叫道:``不好了,果真是有敌人来了!''展大娘冷冷说道:``你怕什么,还有我呢!那小子呢?''

皇甫嵩定了定神,说道:``他刚刚跑了!''展大娘皱皱眉头,心道:``你怎的连个小子也管不住!''但这时她已无暇去责备皇甫蒿,她竖起耳朵一听,听出铁摩勒的脚步声,立即便冷笑道:``好在这小子还未跑出我的家门,我先把他毙了,再帮你对付敌人吧!''

铁摩勒奔至后园,那日渡他过河的那个``舟子''正在园中淋花,原来他的身份本是展家的老仆人。铁摩勒连忙叫道:``我的马呢?''

这仆人已曾得到展元修的吩咐,要把此马归还原主,但这时他见铁摩勒气急败坏的样子,不免惊疑,就在这时展大娘已经追了出来。

这仆人慌不迭的向一间矮房指了一指,铁摩勒立即会意,捧起一块大石,``轰''的一声巨响,将那马房的板门打裂,只听得一声嘶鸣,那匹黄骠马跑了出来。

展大娘怒喝道:``好小子,你还想跑吗?''说时迟,那时快,铁摩勒又已捧起一块大石,向着展大娘便掷。铁摩勒气力沉雄,将石头掷出,呼呼风响,展大娘也不敢轻敌,只得避它一避。

倏然之间,那匹黄骠马已跑到主人身前,铁摩勒大喜,急忙飞身上马,叫道:``马儿,快跑!''

展大娘身形一起,疾似离弦之箭,向那匹黄骠马射来,园门紧闭,那匹黄骠马找不到出路,看看就要给展大娘追上,忽地四蹄一曲,陡然间便跳起来,铁摩勒骑在马背,恍如腾云驾雾一般,这匹马已越过了围墙了。

展大娘与皇甫嵩跟着也越过围墙,仍然穷追不舍。可是他们的轻功虽好,却怎追得上这匹日行千里的宝马。铁摩勒快马疾驰,不消片刻,就把他们摔在后头,连影子也不见了。

铁摩勒脱险之后,却不向山下逃跑,反而向山上有黑烟升起之处,策马疾驰。要知铁摩勒年纪虽轻,却是江湖上的大行家,他听见胡笳,望见烟火,再想起皇甫嵩刚才那张皇的神色,当然也已猜想得到是有了皇甫嵩的敌人来了。

幸而他骑的是匹宝马,登山越险,如履平地,不消多久,便到了莲花峰的断魂岩下,只听得咚咚声响,似是有人用重物砸门的声音。铁摩勒遥望过去,只见人影绰绰的四五个人,其中一人已向他奔来,扬声叫道:``咦,这不是摩勒嘛?''这个人正是段-璋。

铁摩勒大喜若狂,连忙下马,走上前去,但见除了段-璋夫妇之外,还有他的师兄南霁云与夏凌霜,另外还有疯丐卫越。

他们见了铁摩勒,也都是又惊又喜,南霁云问道:``铁师弟,这是怎么回事?

铁摩勒吁了口气,笑道:``我几乎保不住性命与师兄相见呢,说来话长,先问你的,你们可是来捣那皇甫嵩的老巢的?''

南霁云道:``正是。我们已找到他的洞门了,但还未能破门而入。''

铁摩勒随着他所指的方向望去,但见石门上已有了几道裂缝,那是段-璋的宝剑划开的。

铁摩勒道:``皇甫嵩不在这里,夏伯母则确实是囚在里面。''夏凌霜急忙问道:``你怎么知道?''铁摩勒道:``我刚刚和这老贼交过手来!''

众人都吃了一惊,段-璋道:``你好大胆,怎的孤身一人,就敢来搜查?''铁摩勒道:``不是我来找他,是我误落了他们的陷阱了。姑丈,你可知道有个女魔头展大娘么?''卫越跳起来道:``什么,展大娘?那不是大魔头展龙飞的婆娘么?你碰到她了?''

段-璋道:``二十年前,各正派人物因袭他们夫妇的时候,我还年轻,未有参加。卫老前辈和你的师父却是参加围攻的主要人物。''

卫越道:``你快说,你遭遇了些什么事情?''铁摩勒简单扼要的叙述了他的遭遇,却略过了王燕羽与他的纠葛不提。卫越奇道:``这女魔头自视甚高,她为什么要诱捕一个晚辈?哦,是了,想必是她已知道了你是磨镜老人的徒弟了!''

卫越自己给自己解开了一个疑团,但另一个疑团又在心头升起,他沉吟半晌,说道:``这么说来,西岳神龙皇甫嵩当真是罪魁祸首了?唉,唉!我真是料想不到,这些坏事竟然都是他干的!''

段-璋诧道:``卫者前辈,你到了如今,尚不相信皇甫嵩是坏人么?''

卫越摸出一小块木片,说道:``我是还有点疑心,不过,摩勒既然亲眼见到他,又亲耳听到他对那女魔头所说的话,承认了冷女侠是他所囚禁的,那就不由得我不相信了。''

这一小块木片,乃是段-
璋当年在玉树山上与皇甫嵩交手之时,从皇甫嵩拐杖上削下来的。当时,段-璋是为了想邀请武林前辈,替酒丐车迟报仇,他怕别人不相信皇甫嵩会干那等坏事,因此将木片保存下来,作为证据的。这片木片,他见了卫越之后,就交给卫越,记得当时卫越接过这片木片,也曾现出过迷惘的神情。

此刻,卫越又摸出了这片木片端详,脸上又出现同样迷惘的神情。段-璋心中一动,禁不住问道:``卫老前辈,这块木头是我亲手从那老贼的拐杖上削下来的,难道还有什么不对吗?''

卫越沉吟片刻,方始说道:``难说得很。现在把我也弄得糊涂了。好在皇甫嵩既然在此,终须会有个水落石出的!''

话犹未了,只听得一声阴沉动魄的啸声,展大娘与皇甫嵩如风奔至,展大娘厉声骂道:``什么人敢到我华山撒野?''

卫越睁眼一看,正好与皇甫嵩打了一个照面,登时勃然大怒,陡地喝道:``皇甫嵩,亏你还有脸见我,今日我不杀你,就对不住地下的车老二!''

卫越身形何等快疾,就在大骂声中,纵身飞起,俨如巨鹰扑兔,一掌就朝着皇甫嵩的天灵盖打下来!

皇甫嵩面色大变,但却是一声不响,举起拐杖,便是一招``潜龙飞天'上击卫越的腕骨。

卫越一抓抓着杖头,果然发觉他的仗头缺了一块,卫越用力一送,皇甫嵩立足不稳。跄跄跟踉的直退出了七八步,有如风中之烛,摇摇欲坠!

若是卫越立即跟踪急上,一掌拍下,皇甫嵩纵然不死,也得重伤。可是,就在这一刹那间,卫越突然怔住!

你道为何?原来卫越与对方交了这招,立即便发觉两个可疑之处。第一点,他与皇甫嵩、车迟并称``江湖三异丐'',彼此的本领都差不多,卫越之所以一出手便使出极厉害的五擒掌,正是因为知道皇甫嵩了得,所以要先发制人的原故。卫越的用意,不过是想抢得先手,稍占一点上风,却怎也料想不到皇甫嵩甫接一招,便现败象!虽然这一掌也还未将他震倒,可是皇甫嵩的功力却实在不应仅至如此!

第二个疑点则出在皇甫嵩那根拐杖上,原来皇甫嵩那根拐杖是南海紫檀木做的,有一股特殊的香味。段-璋削下的那小块木片,虽然也是紫檀香木,但却不是南海所产的紫檀香木,因之香味也有点分别。卫越就是因为察觉到香味有别,故此起了疑心,疑心是段-
璋当年在玉树山看错了人。

可是现在他已经亲眼见到了皇甫嵩,而且已经面对面的拼了一招了,和他动手的人的确是皇甫嵩,那根拐杖也的确缺了一块,这证明段-
璋讲的没有错,他当年在玉树山上碰上的,暗杀了酒丐车迟的那个凶手,的确是今日所见的这个皇甫嵩!但今日所见的这个皇甫嵩,他所用的拐杖发出的香味和段-
璋所削下的那小块完全相同,却不是皇甫嵩平时所用的那根南海紫檀木所做的拐杖!

卫越发觉了这两个疑点,霎时间怔了,心中闪电般地转了几个念头:是皇甫嵩改用了兵器?或者这个人根本就是冒牌的皇甫嵩?但武林高手用惯了的兵器决无随便改换之理,何况皇甫嵩那根拐杖又是件珍奇之物?但要说是冒牌的吧?天下又怎会有如此相貌相同的人?

卫越大惑不解,一怔之后,正想再追上去细察这个人的相貌,那展大娘一声怪笑,已是到了他的身边,阴侧侧地说道:``老叫化,原来你也还没有死,还认得我这个老婆子吗?''卫越道:``今日之事与你无关,你既然保住了性命,我劝你不要强出头了!''展大娘冷笑道:``当年我也曾劝你不要强出头,你却定要恃众行凶,害死了我的丈夫,如今可怪不得我了!''话声未了,已是双掌齐发,照面打来!

卫越和她双掌相接,不由得大吃一惊,原来她的一只手掌其冷如冰,另一只手掌却如炽热的火炭,卫越虽然早识得她的厉害,却也还未想到她已练成了这等古怪的功夫!

展大娘哈哈大笑,陡地喝道:``老叫化,你还想逃么?''双掌如环,划了一个圆弧,将卫越的身形罩住。卫越怒道:``老妖妇,你当我怕你不成?''左手中指一弹,紧接着右手还了一掌,他同时使出两种武林绝学------一指禅与金刚掌的功夫,刚柔并济,功力深湛,展大娘也不由得心中一凛:``这个老叫化的功夫,也远非当年可比了!''当下双方都不敢轻敌,各出看家本领,拼个强存弱亡!

皇甫嵩给卫越震退几步,刚刚稳住身形,夏凌霜已是挥剑斩来,皇甫嵩面色大变,再向前窜出几步。南霁云恐妻子有失,亦已赶至,皇甫嵩拐杖一勾,将南霁云的刀头勾过一边,强行冲出!

段-璋一声长啸,连人带剑,化成了一道银虹,阻住了皇甫嵩的去路,说道:``南贤弟,你和夏姑娘去设法进洞救人,这老贼交给我吧!''

皇甫嵩一拐击下,段-璋将剑架住,喝道:``皇甫嵩,你今日还有何话说?''皇甫嵩一言不发,枝头一挺,迅即用了一招``神蛟出洞'',疾点段-璋腹部的愈气穴!

段-璋焉能给他点中,横剑一封,``嚓''的一声,又把他的拐杖削去了一片。但两人相较,却是皇甫嵩的功力稍胜一筹,段-璋也不由得退开一步。

窦线娘弹弓一曳,三颗金丸,连发疾发,皇甫嵩避开了两颗,第三颗金丸已是流星闪电般的打到了他的面门。

皇甫嵩反手一招,只听得``叮''的一声,那颗金丸似乎是碰到了什么坚硬的东西,发出了清脆的金石之声,竟给反弹回去!

段-璋心中一动,这才注意到皇甫嵩左手的无名指上,戴着一枚指环,和以前皇甫嵩送给他的那枚指环一式一样!

当年段-
璋为了救好友史逸如,曾单人匹马闯进安禄山在长安的别府,受了重伤,幸得南霁云救出,但安府的武士仍然穷追不舍,后来逃到了一座破庙,恰巧碰上皇甫嵩,皇甫嵩替他们打退追兵,又赠灵药救了段-
璋的性命,他留下了一枚铁指环给段-璋,并留下这样的话语:``若是日后碰到戴有同样指环的人,务请段大快手下留情。''当时段-
璋还在昏迷之中,这话是南霁云转述给他听的。

如今,段-璋见了这枚指环,心中一动,猛然省悟,喝道:``好个处心积虑的老贼,原来你当日救我性命,送我这枚指环,乃是早已算到了今日之事,要我饶你一死么?''

段-璋是个恩怨分明的人,皇甫嵩对他有救命之恩,但现在又已经证实:他就是杀害夏声涛和车迟的凶手,而且夏声涛的妻子、夏凌霜的母亲冷雪梅,现在还正被囚在他的洞中,段-璋岂能把他饶过?

段-璋虚晃一招,再退了一步,然后朗声说道:``皇甫嵩,念在你是武林前辈,又曾于我有恩,你,你自尽了吧,你若有什么未了之事,我可以替你料理!''

皇甫嵩勃然大怒,沉声喝道:``放屁!''拐杖一挥,暴风骤雨般的又向段-
璋猛攻,段-璋叫道:``皇甫嵩,你也不是无名之辈。事到如今,你还要贪生怕死吗?让你自尽,这已经是顾全了你的体面了!''皇甫嵩连声怒骂,越打越凶,段-璋为了报昔日之恩,连让他三招,险些给他打中。窦线娘怒道:``这老贼已是全无羞耻之心,你还和他客气作甚?''拔出缅刀,立即和她的丈夫联手夹攻。

皇甫嵩冷笑道:``你们连自己的儿子也保护不了,还有何面目到此逞能!''他横杖一封,将段-
璋的宝剑封出外门,杖尾起处,骤然一指,一招``毒蛇寻穴'',迳取窦线娘小腹的``血海穴''。这一招两式,又猛又狠,端的是性命相搏的杀手毒招!

窦线娘给他挑起了平生恨事,又气又怒,她缅刀一挥,只听得``咣''的一声,皇甫嵩的拐杖从她脚底扫过,而她的刀头在拐杖上一按,已借着那股猛力凌空跃起!好个窦线娘,人在半空,刀光一闪,便剁下来,这一刀恰好与丈夫的剑招配合得妙到毫颠。皇甫嵩对段-
璋心存戒惧,却想不到窦线娘功力虽然略逊丈夫,出手却比丈夫更狠。饶是皇甫嵩本领非凡,刀尖过处,但觉头皮一片沁凉,竟被削去了一丛头发。

皇甫嵩大怒,拐杖霍霍展开,登时四面八方,都是一片杖影,横扫直击,而且在杖法之中,还掺杂着点穴的手法,拐杖本来是粗重的长兵器,但他将削尖了的杖头当作判官笔使,也居然运用自如,在段-
璋大妇夹攻之下,依然有守有攻。

段-璋心中想道:``皇甫嵩号称西岳神龙,果然是名不虚传,但却也不如所传之甚。''同时又觉得有些奇怪,刚才他要皇甫嵩自尽,皇甫嵩十分愤怒,不断的出言辱骂他们夫妇,可是都无片言只字,提及当年他对自己的救命之恩,按说皇甫嵩骂他,应该骂他``忘恩负义'',最为理直气壮,但他却舍此不骂,不由得段-
璋不感到这是出乎常理之外。

但此际已到了双方性命相扑之时,段-璋虽然有些疑惑,剑招却是毫不放松。他们夫妻自第一次给空空儿打败之后,即苦心习技,精益求精,练了一套刀剑合壁的招数,在第二次与空空儿遭遇之时,已差不多可以打个平手了。现在又隔了数年,配合得更为纯熟,使将起来,刀光剑影,有如一层层的地网天罗,饶是皇甫嵩的杖影如山,也给重重裹住。而他又没有空空儿那等超卓的轻功本领,因此连突围也不可能,眼前虽尚能勉力支撑,但却显然是段-
璋夫妇占了上风,胜负无须预卜了。

另一边疯丐卫越与展大娘恶战,战况更为激烈,却是卫越稍稍不利。展大娘练成了阴阳双毒掌,左掌如寒冰,右掌如炽炭,一给她触及,不但皮肉受苦,滋味难尝,而且甚为耗损元气。幸在卫越已练成了纯厚的内家气功,真气已可以运转自如,身体任何部位给她的手掌触及,立即便可运气防御,免使寒毒与热毒攻心。

卫越的功力与展大娘不相上下,但因要耗损真气对付她的阴阳双毒掌,就难免稍稍吃亏。可是两人都差不多练成了金刚不坏的护体神功,展大娘虽是略占上风,要想取胜,却也不易。

南霁云在旁边看了一会,见段-
璋夫妇已是可以稳操胜券,而卫越与展大娘则似乎是个平手相持的局面,两边都无须自己相助。他想到洞内还有皇甫嵩的同党,只怕他的同党知道了处境不利之后,会用夏凌霜的母亲作为要胁,甚或将她伤害。因此当务之急,便是要赶紧破洞救人。

但洞门是两块坚厚的石门,刚才合他们数人之力,尚且无法攻破,现在只有南霁云夫妇与铁摩勒三人,又无宝刀宝剑,更是无计可施。

幸亏铁摩勒是绿林世家,绿林大盗也多有住在山洞中的,他对这些山洞的构造甚为在行,且又心思灵敏,想了一想,便对南霁云道:``这些山洞,必定另有出路,否则给人在一边堵死,岂不是迟早部成了瓮中之鳖吗?而且那老贼的同党刚才曾燃起烟火,作为报警的讯号,更可以断定他另有出口,而这出口必是在山洞的上方。''

南霁云道:``铁师弟言之有理,霜妹,咱们就上去搜查那另一处出口吧。铁师弟,你在洞外小心戒备,防备洞中的敌人冲出来。''

南、夏二人立即施展轻功,登上山峰,一路小心察看,并大声呼唤。只见到处山石嶙峋,并无洞穴,正在焦躁,忽听得有个声音从洞内传出来,正是夏凌霜母亲的声音,她在叫道:``霜儿,霜儿,是你来了吗?恶贼,你再走近一步,我就与你拼了!''显然她已听到了夏凌霜的呼唤,洞中的贼党正在威吓她不许出声。

夏凌霜大喜如狂,叫道:``妈,我来啦!''循声觅迹,到了那声音的来源之处,发现一块大石,孤零零的在一处,旁边寸草木生,夏凌霜道:``这里必然是出口了。''用力一推,那大石果然动了一下,显见不是与山石相连的生了根的石头。

南霁云脱下了身上的长衫,走过来帮忙夏凌霜推,大喝一声:``起!''那块大石转了几转,滚过一旁。果然露出了洞口,黑黝黝的也不知有多深。

夏凌霜便想跃下,南霁云急忙将她拉开,夏凌霜愕然道:``怎么还不下去?''南霁云道:``小心防备暗器!''他将长衫挥舞,叫夏凌霜跟在后头,然后才跳下去。

黑暗中忽见银光闪烁,幸亏南霁云早有防备,长衫一舞,风雨不透,但听得嗤嗤声响,不绝于耳,原来是在洞内暗藏的敌人撒出了一把梅花针。

夏凌霜暗叫一声:``好险!''她脚跟方定,立即使开了一招``夜战八方''的招式,剑光缭绕中只见一条黑影疾如飞鸟般的扑来,两面发出黄光的圆形武器已经打到,夏凌霜一剑削去,顿时发出鸣钟击罄之声,震耳欲聋。原来那人是个道士,用的是两面铜钹。他的双钹想夹夏凌霜的长剑,未曾夹住,却被夏凌霜一剑穿过了他的衣襟;可是夏凌霜的虎口也甚酸麻,显见那人的功力不在她之下。

说时迟,那时快,南霁云大吼一声,将长衫向敌人兜头一罩,迅即一刀劈去。那人也好生了得,霍地一个``凤点头'',双钹便反劈过来,刀钹相交,又发出了一声震耳欲聋的巨响。

夏凌霜与那人拼了一招,知道以南霁云的本领,纵不能胜,也绝不会落败,她救母心切,当下便燃起火石,进内搜查。

冷雪梅已听到外间打斗的声音,知道女儿来了,一叠声的呼唤她,夏凌霜毫不费力,便发现了她的所在。

那是在洞后面的一间房子,房内有一盏油灯,不很明亮,但已可以清清楚楚的看到她母亲的面容,只见她神情萎顿,面容憔悴,似个病人一般。

夏凌霜泪咽心头,扑上去抱着她的母亲,叫了一声:``妈!''母女泪如雨下,冷雪梅用肘支床,却是起不来。

夏凌霜曾中过皇甫嵩那``千日醉''的迷香之毒,见此情状,立即说道:``妈不必着忙,先躺下来,女儿已把解药给你带来了。''

冷雪梅道:``是那老贼将解药给你的吗?''夏凌霜道:``不是,是王伯通的女儿偷给我的。这事很有趣,待你好了,我慢慢悦给你听。''夏凌霜有点奇怪,母女劫后相逢,多少话要说,她母亲别的不问,却先问她解药的来历,而且疑心是皇甫嵩送的。夏凌霜心想:``莫非我妈被囚禁了多时,神智都糊涂了。皇甫嵩这老贼岂肯将解药给我,还用问吗?''

那解药灵验如神,冷雪梅服下之后,气力便渐渐恢复,她坐了起来,揽住了女儿道:``霜儿,得你无恙,我就放心了。外面这人是谁?''夏凌霜低下了头,说道:``是你的女婿。妈,请恕我未曾禀告于你,我已与霁云成了婚了。''

正是:相见如同隔世,可怜母女相逢。

欲知后事如何?请听下回分解------

\chapter{第二十五回 龙蛇混杂疑终释
乳燕孤飞意惘然}\label{ux7b2cux4e8cux5341ux4e94ux56de-ux9f99ux86c7ux6df7ux6742ux7591ux7ec8ux91ca-ux4e73ux71d5ux5b64ux98deux610fux60d8ux7136}

冷雪梅说道:``像霁云这样的好人,是打起灯笼火把也难以找到的。得婿如此,尚有何求?霜儿,你终身有了依托,我的担子也可以放下来了!''在黯淡的油灯光中,夏凌霜看见她母亲的脸上露出笑容,但她最后那一句话,却又似乎带点感伤的味儿,夏凌霜不由得任了一怔,随即想道:``我自幼没有父亲,母女俩相依为命,难怪她听得我的婚讯,又是欢喜又是感伤了。''

冷雪梅再问道:``外面还有些什么人?''夏凌霜道:``段伯伯夫妻和卫老前辈也都来了,段伯伯正在和那老贼动手,他们夫妻联手,也许已经把那老贼杀了。''她们母女本是握着手的,夏凌霜说话之间,忽觉她母亲的手指微微发抖,禁不住又是一惊,问道:``妈,你怎么啦?''

冷雪梅叹了口气,道:``是-璋来了,我,我\ldots\ldots 唉,我怎还、还好见他?''

夏凌霜道:``段伯伯是爹爹生前好友,妈,我不明白你为什么不愿意见他?''

冷雪梅忽地叫道:``我,我好恨啊!''夏凌霜惊道:``妈,你,你恨谁?''冷雪梅道:``我恨那皇甫老贼!他,他害了我!''夏凌霜听母亲忽将话头从段-
璋拉到皇甫嵩身上,觉得有点突兀,她呆了一呆,忽地想到了一种可怕的事情,不由得浑身颤抖。

冷雪梅蓦地跳下床来,咬牙切齿地道:``我要亲自杀那老贼!''夏凌霜赶忙扶着她,说道:``妈,我替你去杀他吧!你再歇一会儿。''冷雪梅嘴唇微微开阖,似乎有什么话要说,却终于没有说出来,只把女儿的手甩开,使跨出了房门。她现在气力已经恢复了四五分,可以走动了。

南霁云和那道士恶战,双方功力不相上下,杀得难解难分,但那道士心中有所顾虑,时间一长,不觉露出怯意,这时听得冷雪梅母女的脚步声走来,更为惊恐,虚晃一招,便想冲出洞去。

南霁云如何肯放过他,一声喝道:``妖道往哪里跑?''立即挺刀扑上,那两扇石门紧紧关闭,虽然可以从内边打开,但也要费一些时候,那道士猛然省觉:``我真是糊涂了,从正门怎能逃得出去?''说时迟,那时快,但觉刀风飒然,南霁云已是到了他的背后。

那道士使了个``凤凰展翅'',双钹向后斜飞,但因应招稍缓,双钹未合,便给南霁云一刀从中间劈进,正中他的左肩,将肩胛骨都劈得裂开了。那道士似受伤了的野兽一般,狂曝怒吼,拼了性命,将南霁云冲开两步,转过方向,向后洞奔逃。

洞中漆黑,而霁云虽是本领高强,在这洞中却不如这道士的熟悉,他一刀劈空,这道士已冲了过去,拐了个弯,身形没入黑暗之中。

这时,夏凌霜和母亲刚刚走出密室,便听得南霁云的传声叫道:``霜妹,留神!妖道向后洞逃走了。黑暗之中,防他偷袭!''

果然,这声还未了,便听得轻微的暗器破空之声,无数游丝般的光芒突然在黑暗中如火花迸现,那道士已是将一把梅花针向她们撒来。

夏凌霜一个闪身,同时拔剑,忽觉剑鞘空空,只听得她母亲厉声斥道:``龟元妖道,你是那老贼的帮凶,也须饶你不得!''声音一发,便见一道银虹飞了出去,紧接着一声骇人心魄的叫声,那道士已给长剑穿过心胸,钉在石墙之上。

就在这时,南霁云亦已赶了到来,目睹了冷雪梅掷剑毙敌的情形,不禁又惊又喜,心里想道:``我岳母当年号称白马女侠,果然名不虚传。原来这妖道竟是邪派中的有数人物龟元道人。他虽受了重伤,若非我岳母出手,要收拾他,只怕还得费一会功夫呢。''

夏凌霜见母亲掷剑杀敌,知道她的本领最少已恢复了六七成,大喜叫道:``霁云,快来见过我妈!然后咱们一同杀出去,先杀皇甫老贼,再助卫老前辈对付那女魔头!''

南霁云跪下去行了子婿之礼,冷雪梅将他扶起,说道:``雾云,今后我将女儿交给你了,你要好好看待她!''南霁云不善说话,垂手旁立,恭恭敬敬地答了一个``是''宇。夏凌霜不由得``噗嗤''一笑。冷雪梅又道:``我女儿骄纵惯了,你要容忍她一些。嗯,其实无须多说,以你的人品,我也知道你不会亏待她的。''

夏凌霜笑道:``不错,咱们一家子已经团聚,以后说话的时间长着呢。还是赶快出去帮段伯伯和卫老前辈吧。皇甫老贼也还罢了,那女魔头却是厉害得很呢!''

当下夏凌霜将剑取回,交给她的母亲,道:``妈,你没有兵器,暂且用我这把剑吧。''冷雪梅略一踌躇,便道:``唔,也好。''接过了剑,随着便走上前去,开了那扇石门。

冷雪梅吁了口气,叫道:``想不到我冷雪梅还有重见天日之时!''突然转过身来,伸指疾点,咚咚两声,南霁云和夏凌霜都给她点中了穴道,倒在地上了。

南、夏二人做梦也不会想到冷雪梅会点他们的穴道,因此毫无防备,被点倒之后,更是奇怪万分!想问原因,却又说不出话。

冷雪梅道:``我要亲手报仇,不须你们相助。一个时辰之后,穴道自解。霜儿,妈去啦!''她接连回顾三次,这才缓缓走出洞门。夏凌霜隐隐看见母亲的眼角,挂有一颗晶莹的泪珠。

夏凌霜和南霁云在地上面面相觑,两人都说不出话,两人的脸上都露出了惶惑的神情。这的确是难以理解的事,按说冷雪梅即使不要他们相助,也无须点了他们的穴道,更何况那展大娘厉害非常,多两个帮手,岂不更好?夏凌霜目送她的母亲含泪走出洞门,忽地感到莫名的恐惧,只是喊不出声。

在山洞外边,卫越和展大娘还是打得难分难解;而段-璋夫妇却已把皇甫嵩打得只有招架之功,毫无还手之力。

段-璋想起他昔日赠药之恩,不忍亲手杀他,在攻得极为猛烈之时,突然虚晃一剑,喝道:``皇甫嵩,事到如今,你还要贪生苟活吗?有骨头的,自己走吧!''那就是请他自尽,免使受辱的意思!

却不料皇甫嵩趁他攻势骤缓之际,忽地将拐杖一挥,格开了窦线娘的缅刀,仗头一翘,突然``嗤嗤''声响,射出了一蓬毒针!原来他这杖头是中空的,一按机括,毒针便射出来。他本来早已想用毒针取胜的了,只是想选择最有利的时机,出手便能置对方死命,难得段-
璋给他这个机会。

幸亏窦线娘是个使暗器的高手,在暗器的功夫上,比她丈夫要高明得多,百忙中立即将缅刀飞出,双手同时也缩到袖中,双袖一展,将那一蓬毒针都卷了去。毒针将她的半条衣袖刺得如同蜂巢,却没有伤及她的手臂。

皇甫嵩想不到窦线娘竟会用这个法子来收了他的毒针,骤不及防,缅刀过后,在他的肩上削去了一大片皮肉!

皇甫嵩大吼一声,扭头便跑,段-璋一惊之后,大怒喝道:``老贼,你不是人!''双足一点,疾似离弦之箭,一剑刺到了皇甫的后心。

皇甫嵩反手一拐,两人功力本是相当,但他肩头中了缅刀,琵琶骨亦已断了一根,如何挡得住段-
璋这全力的一击,但听得``咔嚓''一声,那根拐杖登时断为两截。段-璋正要一剑斩下,就在此时,忽听得一个声音喊道:``段大侠手下留情!''

段-璋怔了一怔,只见一条影,如飞而来,段-璋左臂疾伸,点了皇甫嵩后心的`冲枢穴'',睁眼看时,不由得大吃一惊,来的竟然又是一个``皇甫嵩'',和被他点到的这个皇甫嵩一模一样!段-璋口呆目瞪,几乎怀疑是自己眼睛花了。转眼间,那条人影已到了面前!

段-璋定了定神,正想问道:``你是谁?''忽听得疯丐卫越一声欢呼,手舞足蹈地叫道:``皇甫大哥,果然是你,哈,我早就该想到那厮是冒充你的了!''

卫越绰号疯丐,平时还不怎的,一遇到意外的欢喜或悲伤,他那疯疯癫癫的性子就发作出来。他这时大喜忘形,竟然忘了与他对敌的是什么人,就大跳大嚷起来。

那展大娘何等厉害,登时左右开弓,双掌一齐攻出,卫越大叫道:``糟糕!''只听得``蓬''的一声,竟给展大娘一掌击中,就像皮球一般,整个身子给抛上上空!

说时迟,那时快,展大娘已是捷如飞鸟,倏的就向段-
璋冲来,卖线娘急曳弹弓,嗖、嗖、嗖三弹连发,展大娘毫不躲闪,三颗弹子全都打中了她,但听得有如金属相触,发出了一片悦耳的铿锵之声,三颗金弹一碰着她的身子就反射回去了!也不知她是身上披有软甲,还是已练成了登峰造极的金钟罩功夫?窦线娘不由得大为惊骇,急忙提弓追上,劈打她的后心。

段-璋一剑斜展,刺向她胁下的``愈气穴'',这是一招以逸待劳的上乘剑法,哪知展大娘仍是笔直冲来,丝毫不避,猛地里伸手一招,手指已勾着了剑柄。段-璋临危不乱,沉腰坐马,剑身往下一压,大喝一声``着!''宝剑已经甩开,闪电般的反削过去!展大娘的功力虽然高出段-
璋许多,但她的一指之力,却还未足以夺剑。

展大娘叫道:``好剑法,但要想杀我,却是不能!''只听得叮的一声,段-璋一剑从她的胁下穿过,展大娘趁势便抓下来,要扣段-璋的脉门。

段-璋的剑招已经用老,刺她不着,正要出左掌与她硬拼,展大娘突然收势,一个转身,只听得``叮''的一声,原来是窦线娘施展``金弓十八打''的家传绝学,弓梢已将劈中她的脊骨,却给她反指一弹,弹个正着!窦线娘的功力不及丈夫,那把金弓,给她一弹,竟然震得脱手飞出。

展大娘刚要转过身去对付段-璋,忽听得皇甫嵩喝道:``展大娘,这里的事我来了结,你可以不必管了!''随着呼的一拐打下,替段-璋化解了展大娘的一招擒拿手。

展大娘瞪起眼睛喝道:``皇甫嵩,你怎么的,是老糊涂了吗?这干人要杀你的弟弟,你知道吗?你胳膊不向内弯,要帮外人杀你的弟弟吗?''

皇甫嵩恨恨说道:``我弟弟若非误交匪人,也不至于落到今日的田地!正是你害了他,吃我一杖!''

展大娘怒道:``真是个不分青红皂白的老杀材,只会关起门来欺负弟弟,俺老婆子可不惧你!''

只听得``蓬''的一声,展大娘早已飞身扑去,横掌如刀,一掌劈下,皇甫嵩也正在一拐打来,那一掌所在拐杖的中间,登时把拐杖震开!

段-璋挺剑急刺,两条人影倏地分开,展大娘曲起身子,在半空中一个倒翻,朝着段-
璋冲到,长袖如虹,疾卷下来。段-璋用了一招``横云断峰'',剑锋斜削,展大娘使出``铁袖''神功,化卷为拍,``啪''的一声,段-璋的宝剑竟给她的衣袖拍得沉下几寸,虎口发麻,宝剑也几乎掌握不住。

窦线娘急发金弹,展大娘这时方始脚踏实地,身形未稳,只得再展长袖将窦线娘的金弹卷去。说时迟,那时快,皇甫嵩又已挥杖攻来。原来展大娘刚才用肉掌硬劈他的拐杖,虽然被他震得向后倒翻,而他也被展大娘的掌力,震得倒退数步,方能稳住身形,而且衣襟也被撕去了一幅,比较起来,还是皇甫嵩吃亏稍大。

皇甫嵩成名数十年,除了吃过空空儿一次亏之外,这次乃是第二次,不由得勃然大怒,再度冲来,用尽了全力,拐杖挥出,隐隐带着风雷之声。展大娘不敢用肉掌再接,使出``流云飞袖''的阴柔功夫,两条衣袖一拂一带,化解了皇甫嵩降魔杖法的刚猛劲力,令得皇甫嵩在气怒之中,也不能不暗暗佩服。

疯丐卫越在半空中接连翻了三个筋斗,落下地来,叫道:``好厉害,幸亏我还未曾给你打伤!''他来回的走了几步,又自言自语道:``要是我们两个老叫化一齐打你,你输了一定不服气;但我若是不打你,我这口气也出不了,怎么办呢?也罢,也罢,我且先看看这场好戏。''他索性盘膝坐了下来,看到精彩的招数,就高声喝彩。原来他之所以袖手旁观,固然是为了不愿以多为胜,但另一方面,他刚才给展大娘用重手法击中一掌,虽未受伤,五脏六腑,却也受了震荡,这时也需要运气调元了。

卫越虽未出手,但展大娘在皇甫嵩与段-璋两大高手夹攻之下,还有一个窦线娘在旁边,不断用金弹向她打来,她已是有点应付为难了。

激战中皇甫嵩使到一招``龙潜深渊'',拐杖反手一点,点到了展大娘臀部的``窍阴穴''。展大娘大怒,左足一个盘旋,飞起右足,便踢皇甫嵩的拐杖。盘膝坐在地上观战的疯丐卫越忽地叫道:``刺她的血海穴!''段-璋依言出剑,果然展大娘刚好转到那个方位,一剑刺个正着,展大娘虽有闭穴的功夫,但段-
璋用的是把宝剑,剑锋削过,登时把她的胯骨也戳碎了一根,血渍染红了衣胯。原来在两个敌人之中,皇甫嵩武功较强,所以展大娘对段-璋就没有那么注意,怎知段-
璋的剑法本来已很精妙,又得了``旁观者清''的卫越从旁指点,因此她反而是先受了段-璋的剑伤。

展大娘这一气非同小可,大吼一声,向段-
璋抓下,段-璋横剑上封,却被她一指弹开,衣领被她抓着,窦线娘大惊,三弹齐发,段-璋用尽浑身气力,缩身一挣,但听得声如裂帛,整件外衣都给展大娘撕去了!皇甫嵩乘机打了她一拐。

饶是练有金钟署的功夫,这一拐也打得她疼痛非常,双睛发黑!但展大娘也端的是凶狠非常,受伤之后,狂呼猛吼,双掌盘旋飞舞,撕、抓、劈、戳,打得更为凶狠。皇甫嵩与段-
璋仍然沉着应付,窦线娘则已有点心颤手软,发出来助攻的弹子,每每失了准头。

正打到紧张之际,展大娘的吼声忽然中止,只听得远远有个声音叫道:``禀主母,少爷已经走了,他有话要奴婢代为禀告!''来的是展家那个老仆人,他看见战况激烈,不敢过来,站在对面的山峰大声叫喊。

展大娘道:``这小畜生有何话说?''她口中说话,手底毫不放松,就在这瞬息之间,仍然向皇甫嵩与段-璋二人,分别攻出了三拍。

那老仆人道:``少爷说,若是主母杀了那位铁公子,他今生就永不再见你的面了!''展大娘``哼''了一声,问道:``王姑娘呢?''那老仆人道:``王姑娘也走了,他们留有书信给你。''

场中各人都在留心听那老仆人和展大娘的对话。蓦地里忽又听得一声裂人心魄的惊呼,虽是在激战之中,皇甫嵩仍是禁不住吓了一跳,与段-
璋一样,一面发招抵御展大娘的攻击,一面不约而同的把眼光射过去。

只见那皇甫嵩的弟弟正躺在血泊之中,胸口插着一柄长剑,剑柄尚自颤动不休,在他的面前,立着一个横眉怒目、面色铁青的女子!

这个女子不是别人,正是夏凌霜的母亲,只因场中激战方酣,所以直到她挪剑杀人之后,众人方始发觉。

段-璋不禁失声叫道:``雪梅,雪梅!''他还叫得出声,皇甫嵩在这瞬间,却似完全呆了。卫越叫道:``留心!''话犹未了,展大娘已是``蓬''的一掌,击中了皇甫嵩的肩头,再一抓又将段-
璋迫退几步,要不是窦线娘金弹立即打来,只怕他们还要吃亏更大。

展大娘叫道:``皇甫华,我已尽了力了,这是你的哥哥忍心让外人杀你,怪不得我!''她扔下了这几句话,立即腾身飞起,向山下急落!

原来展大娘虽是凶狠绝伦,但在皇甫嵩与段-
璋夫妇三大高手围攻之下,她亦自知决难幸胜,何况还有一个疯丐卫越窥伺在旁,如今皇甫嵩的弟弟已死,正给她找到了一个逃跑的藉口。

可是也正由于她太要面子,分明是想逃跑,却还要扔下几句门面话来交代一番,这就令得她在受了剑伤拐伤之后,又加上了一重伤。就在她腾身飞起之际,卫越已抓起了一把石子,用``飞花摘叶''的内家阴劲向她撤去,卫越的内家功夫,已练到了飞花杀敌、摘叶伤人的境界,换上了石子,威力更是大得惊人,展大娘虽然练有金钟罩的功夫,但在受伤之后,给他所发的石子打中,也是禁受不起。但听得她一声尖叫,在半空中接连翻了几个筋斗,终于像流星殒石般的向山谷坠下。对面山峰那个老仆人,连忙大声喊叫,跑下山谷去救她。

这时段-璋、皇甫嵩等人都无暇去追那展大娘了,段-璋与冷雪梅已有二十多年未曾见面,心情激动非常,连忙向她走去。

只见冷雪梅面上已全无血色,那苍白的面容,那阴沉的神情,今得段-
璋也不禁心悸,段-璋道:``雪梅,恭喜你已亲手杀了仇人,足以告慰夏大哥在天之灵了。线妹,你来见过冷女侠。''

冷雪梅避开了他的眼光,低声说道:``多谢你助我报仇,但我已无颜再见你了。''段-璋心头一震,蓦然想起了一种可怕的事情,忙道:``雪妹,你今日已报了仇,应该欢喜才是,别再提伤心话了。''冷雪梅道:``不错,我今日的确是很高兴,尤其是见到你们夫妇。嗯,声涛、你、我三人,当年就好似兄弟妹妹一般,声涛惨死,我的命更苦,还是你最有福气。''段-璋见她又提起伤心话来,正想安尉她,只听得她又低声道:``段大哥,请你看在咱们过去的交情份上,答应我一件事情。''

段-璋道:``雪妹请说,纵是赴汤蹈火,-璋亦在所不辞。''冷雪梅缓缓说道:``事情的真相,不久你就可以明白,你是声涛生前最好的朋友,为了他的原故,我不愿意我的女儿知道真相,我要我的儿女接续夏家的香烟,请你设法替我瞒住她。我知道你是从来不说谎话的,但是为了声涛和我,你可以破例说谎吗?''段圭漳浑身发抖,颤声说道:``我愿意。你,你\ldots\ldots{}''一时间竟不知对她说些什么话好。

冷雪梅忽地将那把插在皇甫嵩弟弟身上的长剑拔了出来,仰天叫道:``夏郎,我不跟你走,就是要等今日,如今我可以见你了!''段-璋一声惊呼,扑上前去,但冷雪梅比他的动作更快,长剑已插入了自己的心房。

段-璋眼泪夺眶而出,哽咽说道:``雪妹,这都是别人害你,声涛决不会怪你的,愿你们夫妇在上天团聚。''皇甫嵩走了过来,指着他弟弟的尸体,道:``都是你害了别人,也害了自己,辜负了我的一片苦心。''跟着也嚎陶大哭起来。

疯丐卫越摇了摇头,叫道:``冷女侠死得冤枉,你的弟弟却是活该!你还为他痛哭做什么?我看你们神智都迷糊了,冷女侠的女儿女婿还在洞里呢,等下他们问起,你如何回答?你快把事情底细说给我知,你们是不惯说谎的,我却不在乎,我可以给你们编一套谎话。''

皇甫禽忍着了眼泪,在凄怆中说出这个骇人心魄的故事。

原来如今被冷雪梅杀死的,就正是他的同胞手足皇甫华,两人相貌十分相似,性情却大大不同。他们的父亲早死,皇甫华自幼顽劣,但却最为他的母亲所溺爱,母亲临死时曾郑重吩咐皇甫嵩,要他照顾弟弟。皇甫嵩深知弟弟的顽劣性成,因此对他也就管得很严,直到他十八岁的时候,还不许他出家门半步。

可是到了十八岁那年,皇甫华的武功也已有了相当造诣了,他非常羡慕闯荡江湖的无拘无束的生活,早已存了逃跑的念头。皇甫嵩又因为是丐帮中的重要人物,而且不时要到外间行依仗义,不能老是守着他的弟弟,平时他离家的时候,就叫一个老仆代负看管之责,同时每次出门,也总不忘告诫他一番。皇甫华幼时由于害怕哥哥,不敢违抗命令。在他哥哥不在家的日子,也不敢不服那老仆人的管教。但到他已经成年,武功又练好了之后,心中就不服了,十八岁那年,皇甫嵩有一次因事离家,他就做出了一件非常令他哥哥伤心的恶行。

在皇甫嵩离家的次日他便要那老仆人放他出去,那老仆人当然极力劝阻,他一怒之下,竟把这个服侍他多年的老仆人杀了。

他在江湖上浪荡了一些时候,不幸遇见了大魔头展龙飞夫妇。展龙飞见这少年武功不弱,且又年幼无知,正好作为臂助,便收服了他,导他为恶。这么一来,皇甫华性格中罪恶的一面越发得到发展,终于越陷越深,变成了一个彻头彻尾的坏人。

皇甫嵩到处寻觅,在他离家之后的第三年,将他抓了回来,痛责一顿,关在石室之中,不久便发生了各正派人物围歼展龙飞的事情,将展龙飞杀了。皇甫华幸而被他的哥哥抓回,得免波及。

好人变作坏人容易,要坏人重新变好那却困难得多。尽管皇甫嵩将展龙飞的罪恶下场作为鉴戒,殷殷的告诫他,他却不但不知感激,反而痛恨他的哥哥束缚了他的自由。不久,又得到一个机会逃了出去。

这时他已长大成人,在江湖上认识皇甫嵩的人,碰见了他都把他误认作皇甫嵩,他就索性一不做二不休,便冒了他哥哥的名头,又造了一根紫檀木拐杖,到处为非作恶,令皇甫嵩蒙受了许多不白之冤。

皇甫嵩听到了这些消息,只得暗暗叫苦,因为他若要辩白的话,那就势将把他的兄弟毁了。因此只好含冤忍垢,不敢声张,自行设法,将兄弟再抓回来。

这样一逃一抓,先后有四五次之多,每次将他抓回来的时候,皇甫嵩都曾想过要废掉他的武功,但每一次在临下手的时候,总是念及死去的母亲,不忍下手。

最后一次,发生了皇甫华暗杀夏声涛,掳走冷雪梅的事件。皇甫华用展龙飞所赠的秘制迷香,杀夫劫妻之后,将冷雪梅收藏在山洞之中,趁她昏迷未醒之际,将她奸污了。冷雪梅醒来之后,和他一场大打,双方都受了伤。皇甫华负伤逃走,冷雪梅膝盖的环跳穴中了他的梅花针,追他不上,但已认清楚了他的相貌。

事情发生后不久,皇甫嵩便把伤还未愈的弟弟再抓回来,因为这一次的祸闯得太大了,累得皇甫嵩有好几年也不敢出门。皇甫嵩待他弟弟伤愈之后,将他带到母亲灵位之前,说道:``依你的行为,我本来应该把你杀掉,看在母亲的份上,姑且再饶你一次,要是你还不知悔改,再逃出去为非作恶的话,我就把你先杀掉,然后我再自杀!我杀你总好过你给别人所杀!''跟着要他在亡母灵前,发下毒誓。

皇甫华受了这次教训,果然安份下来,在家中勤修武功,再也不提要到江湖闯荡了。皇甫嵩有几次故意试他,假装出门,躲在附近窥察他的行动,他都是规规矩矩的在家中自行习武,不敢下山。皇甫嵩暗暗欢喜,以为他的弟弟已是浪子回头,从此不敢再为非作歹了,对他的管教也就渐渐放松。

哪知全不是这回事。皇甫华之不敢逃走,固然一方面是忌惮他的哥哥,他知道他哥哥这次是动了真怒,在他的武功尚未能赶上哥哥之前,只怕自己一踏出家门,就要被哥哥抓将回来,真个说到做到,将他杀掉;但更重要的还不是害怕哥哥,而是因为在他干下了那件凶案之后,由于夏声涛是武林景仰的大侠,不但夏声涛的妻子冷雪梅要报仇,即夏声涛的朋友,识与不识,都要为他破案擒凶。他在未给他哥哥抓回家之前,各正派的人物都已侦骑四出了,幸而他是躲在荒山古寺里养伤,逃过灾难,但这个风声,他已是早已闻知了。

因此他必须骗取哥哥的相信,假作浪子回头,誓言悔改,好骗取他哥哥的武功。

皇甫嵩住在华山绝顶,极少与人往来,除了他最要好的朋友酒丐车迟之外,没人到过他的家。所以也只有车迟知道皇甫嵩有这么一个弟弟,知道这件秘密。但那时已是皇甫华表示悔改之后,他才知道的。由于皇甫嵩的央求,车迟也没有揭露这个秘密,他是个好心肠的人,像皇甫嵩一样,希望皇甫华真正能够回心向善,往事也就不必深究了。

于是者一连过了十多年,皇甫华的武功已差不多就要赶上他的哥哥,而皇甫嵩对弟弟也渐渐放心,有时离家数月,也不将他囚禁。哪知有一次,他从外面回来,又发现他的弟弟大踪了。

这一次皇甫华还并未逃出华山,原来事有凑巧,那大魔头展龙飞的妻子,选中了华山断魂谷作为她隐居之所,再度与皇甫华相遇,皇甫华是逃到了她那里求她庇护的。

皇甫嵩不久也知道了弟弟的躲藏之所,但他斗不过展大娘,又不敢声张求人相助,无可奈何,只好让他的弟弟自立门户。

皇甫华摆脱了哥哥的束缚,又在展大娘处学会使用喂毒暗器的功夫,这才大着胆子下山,其时距离夏声涛的被杀,已将近二十年。除了夏声涛最要好的几个朋友还在设法要破案擒凶之外,其他的人,对这件事情都已淡忘了。

皇甫华重现江湖之后,不久就知道冷雪梅已有了一个女儿,而他对冷雪梅也还未能忘情。

在冷雪梅那方面却是苦心孤诣,矢志报仇,但她因受了这么大的耻辱,无颜再出江湖,也不愿再见旧时的亲友,因此把复仇的希望寄托在女儿身上,她把所会的本领部传授给女儿,告诉她皇甫嵩是个无恶不作的大坏人,要她技成之后,就要杀皇甫嵩替江湖除害。

这其中的曲折与误会,夏凌霜毫无所知,而皇甫嵩则是知道的。这就是为什么那次在古庙之中,皇甫嵩不加分辩,愿意敛手让夏凌霜杀他的原因。

皇甫华下山之后不久,由于气味相投,便与精精儿深相结纳,又因为在江湖上知道他的秘密的,只有酒丐车迟一人,所以在精精儿、王伯通二人设计将段-
璋夫妇与车迟诱往玉树山时,他就追至玉树山,用毒针将车迟杀死。他本来还要下手杀害段-璋的,幸而段-璋及时发觉,又得车迟舍命相护,这才未曾受害。

皇甫华冒充地的哥哥,几乎骗过了所有的武林中的成名人物,卫越的徒弟,将卫越与皇甫嵩约会的书信错交了给他;空空儿也上了他的当,将他当作皇甫嵩,听信他一面之辞,替他赴卫越之约,与卫越大打了一场。最后他还与精精儿等人,将冷雪梅母女掳走。终于恶贯满盈,死在冷雪梅剑下。

皇甫嵩把事情的真相讲明之后,众人无不惊骇伤心。段-璋拭了眼泪,对皇甫嵩重新施礼,为过往的误会而抱歉,并多谢了他那次救命之恩。

皇甫嵩道:``过去的都过去了,现在咱们该到山洞去寻找他们了。老叫化,你的谎话编好了没有,怎的还不见他们出来?''

卫越是个江湖上的大行家,想了一想,说道:``定是冷女侠不愿他们知道真相,所以点了他们的穴道了。老叫化的谎话早已编好了,咱们走吧。''

这时已过了将近一个时辰,南霁云功力深湛,运气冲关,穴道先已解开,这时正在助夏凌霜解穴。

段-璋与皇甫嵩等一行人来到,南霁云大吃一惊,跳起来便要拔剑,段-璋道:``南贤弟,你看清楚些,这个皇甫嵩不是那个皇甫嵩!那个大坏蛋是皇甫老前辈的不肖弟弟!''南霁云呆了一呆,定睛注视,这才发现皇甫嵩身上穿的是一件缝缝补补的百袖衣,手上的拐杖也未折损,而那个``皇甫嵩''穿的却不是化子衣裳,他的那根拐杖,在南霁云未入山洞搜索之前,就已被段-
璋的宝剑削去了半段。

段-璋又道:``这次幸得皇甫前辈,赶来相助,大义灭亲,你岳母才报得了仇。''南霁云连忙道谢。

这时夏凌霜穴道已解,跳起来道:``我妈妈呢?为什么她还不来?''她已隐隐感到了凶兆,心中想道:``报了仇又打了胜仗,为什么他们的脸上却全无喜悦之情?''

段-璋道:``贤侄女,你妈是为了疼你,才不让你出去,她,她可不能再见到你了。唉,这件事,卫老前辈,还是你来对她说罢!''

南、夏二人在惊疑不定之中,只听得卫越缓缓说道:``你们也许还不知道,那皇甫华的武功虽然不算很高,但他那拐杖内藏有毒针,来无踪,去无迹,却是非常厉害,你瞧,你段婶婶那只袖子!''

窦线娘的两只袖子都刺满了毒针,这时虽然都已抖落,但那蜂窝般的针孔,还是令人触目惊心。

夏凌霜却不耐烦听他细说,她急着要知道的只是她母亲的吉凶,立即插口问道:``为什么我妈妈不能再见我们?皇甫华的毒针厉害,我早已知道了。我只要你告诉我,我的妈妈现在何处?''

卫越却慢条斯理地说道:``对啦,我想起来了,-璋对我说过,皇甫华在玉树山上,用毒针暗杀酒丐车迟的时候,你也是在场的。怪不得你早已知道他的毒针厉害了!''

夏凌霜听他尽说闲话,甚为不满,但卫越的辈份比她母亲还高一辈,她已催过一次,不便再催,心中想道:``一个人上了年纪,说话真是罗哩罗唆。''

卫越面色一端,接着说道:``你妈就因为知道了仇人的毒针厉害,所以才不让你们出去的。唉,她是亲手杀了仇人,可是她也给皇甫华的毒针刺中,终于死了!''

夏凌霜登时呆了,一口鲜血喷了出来,晕了过去。

南霁云连忙替她推血过官,铁摩勒又撕下了一幅衣衫,在冷水中浸湿,覆在她的额上。过了一会,夏凌霜醒转过来,这才能够出声痛哭。

卫越道:``夏姑娘,令堂的后事还要你办,她有遗言要我们转告你。你不要太伤心,坏了身体。''

夏凌霜哽咽问道:``我妈有什么遗言吩咐?''

卫越道:``她要你将她的骨灰与你的爹爹合葬,你爹爹当年是在德州被害的,他的坟墓我们替他建在德州城外的朱雀山下。''

夏凌霜的母亲从来没有将这件血案的真情告诉她,以前她技成之日,她母亲要她杀皇甫嵩,理由也只是因为皇甫嵩乃是无恶不作的坏人,故此要她为江湖除害,却并没有提起什么杀父之仇。南霁云从段-
璋之处虽略有所知,但以真相未明,也未曾对夏凌霜讲过。因此,夏凌霜听了卫越的话,不觉一怔,连忙问道:``我爹爹原来是给人害死的么?这是怎么回事?''

卫越接着说道:``凶手就是这个皇甫华,你爹爹是在和你妈举行第二次婚礼的当夜,就给他暗杀了的。''

此言一出,不但夏凌霜惊骇,连南霁云也吓得变了神色。卫越说道:``你们不必惊疑,夏姑娘的父亲,两次举行婚礼,新娘都是她的妈妈。事情是这样的:夏大使第一次结婚是在天山南路的一个小城,那时他们两人都在边荒之地行侠,万里同行,起居不便,因此便在小城中草草成婚,我适巧也在那个地方,参加婚礼的就只有我一个人;后来他们二人回到中原,有些朋友知道了就要他们补请喜酒,再加上我们这些喜欢热闹的朋友起哄,你的爹爹因交游太广,就索性再举行一次婚礼。''

卫越接着说道:``那时,你已经出世,过了两周岁,你父亲在回疆游历之后,回到你外公的庐龙任所,你就是在那儿出生的。你父母要在江湖游侠,携带不便,因此将你寄养在外公家里,你爹娘的第二次婚礼,你没在场,当时宾客众多,你爹爹尚未曾与知己友人畅叙别情,就给皇甫华暗杀了-璋,你那时也有参加婚礼的,想来你也不知道他们已经有了女儿吧?''

段-璋搓搓手道:``啊,原来如此,我那时当真还未知道。怪不得酒丐车迟,也曾对夏侄女的身世起疑了。''

接着卫越就将皇甫华如何与展龙飞勾结,如何屡次冒着他哥哥的名头私下华山,如何在江湖乱作非为,如何暗害夏声涛的经过,一一说了出来。除了夏凌霜的身世这一段是他伪造之外,其他的都是实情。

夏凌霜这几年来,一直为着自己的身世之谜而感到烦恼,如今才拨开云雾,豁然开朗,虽然仍有父母双亡之痛,但是比起未知``真相''之前,心情却是要较为轻松了。

卫越捏造的``真相''说得合情合理,不但解开了夏凌霜的心头之结,连南霁云也相信不疑。只有皇甫嵩老泪盈眸,伤心不已。南霁云夫妇再次向他致歉、道谢。卫越忽道:``俺老叫化又要说疯话了,南大侠,我可要为老朋友求你一件事情。''

南霁云道:``老前辈言重了,南某受惠良多,老前辈若有差遣,小辈自当效劳,怎用得上一个`求'字?''卫越似笑非笑地说道:``这件事么,也不是你一人就能`效劳'得了。''南霁云正要问他是什么事,卫越已接着说道:``时候不早,你们也应该出去,早些替你的岳母办理后事了。嗯,段嫂子,你扶夏姑娘走吧,我和南贤侄说几句正经话儿。''

夏凌霜已哭得浑身乏力,窦线娘扶着她走在后头,卫越则拉着南霁云行快了几步,低声对他说道:``南贤侄,你希望有几个儿子?''

南霁云怔了一怔,心道:``卫老前辈古道热肠,说话却怎的这样颠三倒四?''一时间不知如何回答,只听得卫越又在似笑非笑地说道:``听来似是疯话不是?但实在却是正经话儿。我是希望你最少有三个儿子。''南霁云诧道:``老前辈的意思我还是不明白。''卫越道:``大儿子接你南家的香烟,你岳父没有儿子,你的第二个儿子应该继承岳家,对不对?''南霁云本来悲伤未过,听了他的怪话也不觉有点忍俊不禁,当即问道:``那么第三个儿子呢?''卫越道:``皇甫嵩这次大义灭亲,给你们帮忙了不少。''南霁云道:``是啊,我们以前将他误作坏人。实在过意不去。但这却与老前辈所说的何关?''卫越道:``怎说无关。你不知道么,他是丐帮的长老,今生是不会再娶妻生子了,你若有第三个儿子的话,可否过继给他,以慰他的晚年。我们作化子的不讲辈份,当作是他的儿子或孙子都行。''南霁云不觉笑道:``生几个儿子,这真是老天才能作主。好吧,我若有第四个儿子的话,还可以送一个给你。''卫越笑道:``这样说,你是答应了。皇甫嵩没有亲房侄儿,所以死后想有人扫墓。我老卫却不在乎。不过,你若真肯把第四个儿子送给我。我老卫当然也是要的。''后来,南霁云果然在四年之中,生下三个儿子,这是后话,按下不表。

且说一行人走出山洞,夏凌霜见了她母亲的尸体,又哭得晕倒,卫越帮忙她把冷雪梅的尸体焚化,将骨灰装在布袋之中。也幸而夏凌霜没有仔细验看她母亲的尸体,未曾发现她是用剑自尽的。

待得夏凌霜醒转,卫越道:``南贤侄还要回到潼关附近,收编残余的官军。德州离此不过数日路程,我老叫化陪夏姑娘到德川走一遭吧。将你父母合葬之后,我再与你同回,助南贤侄一臂之力。''夏凌霜挥泪说道:``老前辈大恩大德,我真不知如何报答才好?''卫越一本正经地道:``我已与你丈夫说好了,你多生几个儿子,就算是报答了我们了。''夏凌霜听了这话,在痛哭流涕之中,也禁不住满面通红。

皇甫嵩咳了一声,说道:``我这卫大哥惯说疯言疯语,夏姑娘不必理他。''回过头来再对南霁云道:``我埋了这个孽障之后,还有一些事情料理。将来或许也会到潼关找你。''南霁云道:``得两位前辈鼎力相助,南某感激不尽。''

段-璋却叹了口气,说道:``我和夏侄女的父母,当年是生死之交,如今夏兄之仇已报,我的心事也了却一半了。只是还有史兄之冤,不知何时方雪?他的夫人,陷身贼巢,如今已有了七八年了,消息毫无,好不令人悬挂。唉,雪梅临去之前,还说在三个人之中,以我最有福气,其实我有什么福气可言?我生平最要好的两位朋友,都遭惨死,我的儿子被空空儿劫走,至今也未知下落。''

皇甫嵩道:``段大侠不必烦恼,卫大哥与我都和空空儿的师门有点渊源,听说空空儿曾受我那不肖弟弟所骗,和卫大哥还结了一段梁子。我们二人必定要找到空空儿,解开这段梁子,到时我会向他索回侄儿。''

卫越``哼''了一声,说道:``空空儿非常袒护他的师弟,只怕他是近墨者黑,早和精精儿走上一条路了。''皇甫嵩道:``空空儿我自幼就知道他,他的性情是骄傲一些,但本性还好。不过,他若然真是变得坏到不可收拾,我也不会再和他讲什么交情了。到时你我二人,以力服他,迫他交还段大侠的儿子也就是了。''

段-璋谢过了这两个异丐,又道:``小儿之事,还在其次。史家兄弟为我而死,他妻子陷身贼巢,我于心何安,现在安贼已经作反,她的处境更为可虑。我必须先探听她的消息。听说安贼正准备进攻长安,我们夫妇也准备扮作难民,若有机可乘,就偷入贼营救她出来。''

南霁云道:``摩勒,你在这里无端的耽搁了几天,只怕皇帝老儿已经抛弃京城,向西逃走了,你得赶往长安才是。''铁摩勒嘀咕道:``我倒巴不得皇帝老儿已离开长安,也省得我做这个倒霉的保镖。''南霁云正色道:``话不能这么说\ldots\ldots{}''铁摩勒笑着打断他的话道:``你的大道理我已经知道了,好,我现在就听你的话,马上赶往长安。''

当下一行人走下华山,铁摩勒牵着黄骠马与他们同走一程,在路上才有时间将他这几日的遭遇细说,不过他还是隐瞒了王燕羽对他的痴情这一段。正说话间,已走近山谷下面展大娘居住之处,只见火光融融,展大娘那几栋房子在火海之中都差不多变成瓦砾了。

正是:莲出污泥而不染,凤凰火化得新生。

欲知后事如何?请听下回分解------

\chapter{第二十六回 陌路相逢奸计泄
深宫又见逆谋生}\label{ux7b2cux4e8cux5341ux516dux56de-ux964cux8defux76f8ux9022ux5978ux8ba1ux6cc4-ux6df1ux5babux53c8ux89c1ux9006ux8c0bux751f}

卫越诧异道:``咦,这倒奇了,谁人这样大胆,竟敢放火烧这女魔头的房子?''铁摩勒道:``想必是她的儿子烧的,她的儿子虽非侠义中人,心地倒还不错,大约是已下了决心,和他的母亲决裂了。''皇甫嵩道:``若然是他烧的,那就还有一层用意,他是要使得他的母亲不能不离开这个地方。''卫越点头道:``不错,展大娘的住处已给我们发现,她的儿子是怕我们再来与他的母亲为难,又怕他的母亲自负太甚,不肯离开老巢,示人以怯,所以索性一把火将它烧了。''

段-璋道:``我对人总是喜欢朝好的方面着想,我宁可相信摩勒的猜度。不过,无论他是哪一户用意,他总是要比他的父母好得多了。''

众人一面走一面谈论,铁摩勒回头望那火光,过去几天来的经历,又在心头重现,展大娘那狰狞的面貌,王燕羽那幽怨的神情,\ldots\ldots 都似随着浓烟升起,浮现在他的眼前!他耳边又响起了王燕羽那激动的声音,那是当他在展大娘的掌下,即将毙命之时,她那动人心魄的呼叫!如今这几栋房子是烧掉了,可是王燕羽在他心中的影子却不能烧掉,想起了王燕羽,铁摩勒不自觉的有几分怅惆,但随即想道:``她的师兄对她是真情实意,当然会一生一世爱护着她,如今他们已摆脱了那个女魔头,一同逃走,我也无须为她的将来担心了。''

不久就走出了山谷,段-璋和南霁云再次叮嘱他一番,叫他到了长安,一切都得小心在意,切不可任性而为,有不懂的可以请教秦襄和尉迟北二人。诸事交代清楚,于是众人分道扬镳,铁摩勒跨上了黄骠马,迳往长安。

黄骠马脚程快疾,第二日中午时分,就已到临潼境内的骊山脚下,距离长安不过百多里了。骊山迤逦数十里,铁摩勒正沿着山边的驿道奔驰,那匹黄骠马忽然一声长嘶,似乎发现前面有什么可怕的东西,四蹄停下,不肯向前。

铁摩勒大为奇怪,心道:``这匹马在刀枪剑戟丛中尚且不惧,它却害怕何来?''铁摩勒笑着拍拍马背,说道:``马儿,马儿,你保护我已有多次了,你若有危险,我也会保护你的,不必害怕,走吧,走吧!''那匹黄骠马善解人意,在主人的命令下继续前行,但已不是似刚才那样的如飞奔跑了。看它的神气,既似有些害怕,又似有些愤怒。

走了片刻,忽见前面靠近山拗的路旁,有一堆人围在那儿,远远望去,只见他们指手划脚的似乎是在争论什么。

铁摩勒是在高山上长大的,又是自小就练习暗器的,目力极佳,那几个人围作一堆,有一个人的脸朝着他的方向,铁摩勒在马背上一眼望去,不觉心头一震:``这不是展元修吗?咦,却怎么不见王燕羽?''

铁摩勒这才明白,原来他这匹黄骠马害怕的乃是展元修,铁摩勒笑了一笑,拍拍马儿的颈项,说道:``这个人现在已经是我们的朋友了,他不会再害你了,你放大胆子,上前去吧。''

当下,铁摩勒将帽沿一压,遮着了半边面孔,双腿一夹,快马疾驰上去。这时,那些人争论的声音已隐约可闻,忽听得一个甚为熟悉的冷笑声音道:``小展,你想要人家的姑娘,却不管人家的父亲,天下哪有这等便宜的事?''

铁摩勒又是心头一凛,说话的这个人正好转过脸,活脱脱像个大猩猩,却原来正是精精儿!

只听得展元修的声音随即说道:``你别胡说八道!我与你们河水不犯井水,我展元修虽然不是什么英雄侠士,但也绝不为虎作怅!''

精精儿打了一个哈哈,嚷道:``谁不知道你想要王伯通的女儿?你既然在龙眠谷中救了他的性命,为何不帮忙到底!哈哈,为虎作怅?你骂我不打紧,但这句话岂不是连你的岳父也骂在里头了?''

铁摩勒一声叱咤,黄骠马箭一般地冲去,那些人突然见这快马飞来,都吓了一跳,精精儿双眼一翻,喝道:``好小子,原来是你!''

说时迟,那时快,铁摩勒早已翻身下马,拔剑出鞘,喝道:``精精儿,你这叛国奸贼,好大的胆子,竟敢到天子脚下的地方!你又在打什么害人的主意了?''

精精儿大笑道:``铁摩勒,我知道你就要来做御前侍卫,但你还未曾上任,就要给皇帝老儿卖命了吗?''

铁摩勒大吃一惊,郭子仪保举他做御前侍卫,这是非常秘密的事情,想不到精精儿竟已知道!

精精儿笑声一收,紧接着冷冷说道:``凭你的本领,你要给皇帝老儿卖命,只怕也未必能够!''话声未了,倏的就扑上前来,手拿一翻,一柄精芒耀目的匕首已握在掌中,向铁摩勒刺出。

铁摩勒知他匕首锋利,长剑一招``春云乍展'',避开正面,侧刺他的腰胁,精精儿又哼了一声道:``绿林世家铁昆仑的儿子来做御前侍卫,这也真是奇闻。''

精精儿一面出言讥讽,手底依然毫不放松,就在这刹那之间,他的匕首已接连攻击了七招,每一招都是指向铁摩勒的要害穴道。

铁摩勒大怒,长剑挽了一个剑花,一招``雷电交轰'',向精精儿猛劈过去,同时喝道:``我姓铁的给皇帝老儿卖命又怎么样?总胜过你给骚鞑子胡儿卖命!''

铁摩勒这一招是磨镜老人所独创的剑法,将剑法化为刀法,长剑当作大刀来使用,钢猛之中又带着三分柔劲,端的是厉害非常!

这样刚猛而又轻灵的剑势,饶是精精儿也不敢和他硬碰,可是精精儿的轻功却比铁摩勒高明得多,铁摩勒一剑劈去,只见精精儿的影子一闪,已是劈了个空。精精儿倏然间就绕到了铁摩勒的背后,冷笑道:``你这些话拿来骂我,却是骂错了人!''原来精精儿本来就不是汉人,他是西域康居族猎户的一个私生子。生下来就被抛弃深山,是山中的野人将他养大的。

冷笑声中,精精儿出手如电,匕首直指到了铁摩勒的后心,幸而铁摩勒应招也够机警,一剑掷空,立即反手撩去,'哨'的一声,碰个正着。精精儿那把匕首名为``金精短剑'',锋利非常,铁摩勒的长剑给他削了一个缺口,但终于将他这一招化解了。

铁摩勒将长剑抡圆,使出了八八六十四招龙形剑法,这套剑法的特点是招数连绵不断,使到疾处,端的有如长江大河滚滚而上,精精儿接连冲击了好几次,都未能攻破他的防御。

铁摩勒的气力比精精儿沉雄,但精精儿的身手却比铁摩勒更为矫捷而且他惯经大敌,不论在武功上和经验上都还要比铁摩勒稍胜一筹。不过铁摩勒除了气力沉雄之外,又胜在有一股锐气,正是初生之犊不畏虎,纵使是面对强过自己的敌人,他仍然是奋不顾身,攻多守少。精精儿自忖胜算可操,还不敢真的和他拼命。

精精儿那两个伙伴看了一会,忽地一齐扑上,两翼攻来,精精儿眉头一皱,正要装腔作势,叫他们退下,那两个人已先自嚷道:``我们知道你老不必帮忙,但这小子是我们当家的仇人,在龙眠谷中,他老人家险些给这小子伤了,我们是来为当家的报那一剑之仇!''

绿林规矩,寨主受辱,属下都有给他报仇的义务,加以精精儿也想早一些将铁摩勒拿下,好与展元修续谈,所以,经他们一二人这么一说,也就不再阻拦。

这两人都是王伯通的心腹勇士,一个叫做韩荆,一个叫做邓奢,韩荆使的是三节棍,邓奢使的是厚背砍山刀,都是威力很大的重兵器。他们一加入战团,精精儿登时如虎添翼。

铁摩勒对付精精儿一人,已经难以抵敌,何况再添上这两个高手。激战中,邓奢一刀砍到,铁摩勒横剑一封,将他的厚背砍山刀荡过一边,可是铁摩勒因为横剑削出,中路已露出空门。那精精儿何等很辣,一见有机可乘,立即欺身直进,匕首一送,一道蓝艳艳的光华电射而出,直指到了铁摩勒的胸口。只听得叮咣一声,铁摩勒的护身甲已给戳穿,刀锋划过胸口,皮肉也伤了少许,鲜血泪泪流出,沁红了外面的衣裳。

精精儿哈哈大笑,匕首盘旋飞舞,再向铁摩勒刺去,这一招更其厉害,竟是迳刺向铁摩勒的咽喉。

但精精儿这一招刚刚发出,猛然间便觉得背后有金刀劈风之声,精精儿武学深湛,听风辨器,便知是有高手乘虚袭击他的背心大穴。精精儿也真了得,一个盘龙绕步,身形疾起,背后刺来的这一剑已落了空,而他的匕首仍然退向铁摩勒刺去。

可是如此一来,他匕首上的劲道已减了几分,准头也歪了少许。铁摩勒一招``举火撩天'',长剑上刺,不但将他的匕首格开,剑锋还穿过了他的衣襟。

这几招迅着电光石火,精精儿站稳了脚步,这才看清楚袭击他的人竟是展元修。精精儿不禁大怒喝道:``姓展的,你怎的吃里扒外啦!''

展元修冷冷说道:``一来因为他是我的朋友,二来因为我是汉人!''他不待精精儿再说,已是如影随形,跟踪追到,又一剑向精精儿刺去。

精精儿气得哇哇大叫,但展元修的武功也极其了得,他的剑法虽不及铁摩勒的精妙,功力则在铁摩勒之上。精精儿被他们二人同时夹攻,尽管七窍生烟,也只得沉住了气应付。

韩荆、邓奢急忙过来帮手,展元修反手一剑,跟着一掌拍出,他这剑底夹掌的功夫是家传杀手,这两个人如何抵挡得起?只听得``咔啦''一声,韩荆三节棍的头一截已给他一掌劈断,邓奢更惨,虎口中了一剑,厚背砍山刀飞上了半空。

展元修喝道:``看在我师妹的份上,我不杀你们,快滚!''韩、邓二人见展元修翻了面,他们都是知道展元修的来历的,即算未曾受伤,也不敢和他对敌,何况他们又确是技不如人。当下,这两个人果然如奉圣旨,哭丧着脸,就退出了战团,并向精精儿嚷道:``大水冲倒龙王庙,自家人打自家人。小的左右为难,只有先回去向当家禀告,请恕我二人失陪啦!''

精精儿``哼''了一声,匕首向展元修一指,冷冷说道:``亏你还敢提起师妹,我看你还有甚么脸皮去见她的父亲?''

展元修喝道:``这是我的事情,不用你管!''精精儿惯会乘暇抵隙,趁他说话的当儿,那一招虚招突然化实,剑光疾吐,使出了一招``丹凤朝阳'',精金短剑指到展元修的胸口。

铁摩勒的经验不及精精儿,但比展元修却又较为丰富,他知道精精儿狠辣狡狯,早就全神贯注地盯着他,一见精精儿移步换招,立即长剑挟风,``呼''的一声,向精精儿背心刺去。

这一招是攻敌之所必救,精精儿迫得脚跟一旋,转了半个圆圈,匕首拖过,划破了展元修的袖口,``咣''的一声,又恰好挡住了铁摩勒的青钢剑,在他的剑上,再添了一道缺口。

展元修道了一声:``多谢铁兄。''剑尖一起,合成了一道圆弧,再一次使出剑中夹掌的功夫,向精精儿猛袭!

这两人同心合力,双剑齐挥,精精儿也给他们迫得喘不过气来,激战中但听得``蓬''的一声,精精儿已中了展元修的一掌,接着又给铁摩勒一剑刺中他的肩头,只差半寸,就要挑破他的琵琶软骨。

精精儿吓得冷汗沁肌,心中想道:``这姓展的小子已经横了心肠,翻面不认人了,他是展大娘的儿子,我纵然能够杀了他,展大娘这个强仇也是结不得的。''

心念未已,展、铁二人双剑又到,精精儿匕首一封,身形突然倒纵,他的轻功果然已到了炉火纯青的境界,铁摩勒的剑招先到,精精儿那炳匕首碰着了铁摩勒的青钢剑,惜了他那股猛力,去势更快,待到展元修的长剑刺来,已是连他的衣角也沾不着。

精精儿扬声叫道:``姓展的小子,今番暂且饶你,待我见了你的母亲,再和她评理去。''

展元修助铁摩勒裹好了伤口,再度向他致歉,铁摩勒笑道:``过去之事,不必提了。''向那匹黄骠马招手道:``马儿,你也不应该记恨了。不是展兄,你和我都要遭那大猩猩的毒手。''

这黄骠马甚通灵性,见展元修帮他的主人打退敌人,果然神气顿改,走过来摇头摆尾的,似乎是表示已释了前嫌。

展元修哈哈大笑,但随即面色又沉郁下来,问道:``我妈怎么啦?''铁摩勒道:``她打不过皇甫嵩和卫越两位老前辈,已经跑了。''展元修又望了铁摩勒一眼,半晌方始讷讷说道:``铁兄,你下山来,路上可曾碰见我的师妹?''

铁摩勒道:``我也正想问你王姑娘呢,我只道她是和你在一起的。''展元修面上一红,说道:``她是为了你才上断魂岩的。我,我是为了成全她的心愿,才一把火烧了老家,并叫仆人带口信给我母亲的。''铁摩勒这才明白,想是在展大娘追踪自己的时候,王燕羽也就跟着追出来,而展元修则恐怕王燕羽还不能劝阻他的母亲,因此才叫那仆人捎来口信,以终生不见母亲作要胁,阻止他的母亲向自己下毒手,然后毁家独走,避免与他们见面。

铁摩勒生怕误会更深,连忙说道:``断魂岩上,没有见到她的踪迹。既然如此,展兄,你得赶快去寻觅你的师妹。''

展元修叹了口气,说道:``铁兄,我的意思你还不明白吗?我今生今世,是不会再与师妹在一起的了。''铁摩勒呆了一呆,说道:``展兄,你和王姑娘本是天造地设的一对,你喜欢她,她也喜欢你的,你怎的说这种话?''展元修木然问道:``你怎么知道她喜欢我?''铁摩勒道:``她曾亲口对我说,她已答应了你的母亲,愿意嫁给你了。你的母亲还未告诉你吗?''

铁摩勒是个直心眼儿的汉子,他却不想:王燕羽允婚他人,却先对他言说,这是什么意思?这叫她所允婚的那个人如何受得起?

果然,展元修听了这话,神情尴尬到极,脸上一片青一片红,过了好一会,才忽地大声说道:``铁兄,我师妹属意的人是你,你要不要她是你的事。我已然明白了她的心意,尽管我喜欢她,我也不会令她讨厌我了。更明白地说,那就是我决不会再插进你们之间了。但愿你好好的看待她。''

铁摩勒不善言辞,急得青筋暴起,连连说道:``这,这从哪儿说起?找、我是\ldots\ldots{}''他想说的是:``我是已经订了婚的人了。''但一想,若然这样说法,岂非又给展元修误解他要是未曾订婚,就会对王燕羽钟情?急切之间,他实在想不出要怎样说才合适,展元修一声``失陪'',早已跨上他的坐骑,向另一个方向走了。

铁摩勒正待策马追赶,展元修忽地从马背上转过头来,大声说道:``铁兄,我也忘记告诉你一件事,你是新任了御前侍卫不是?精精儿他们要趁长安混乱,官家逃难之际,刺杀皇帝老儿,你可得小心了!''

原来展元修在路上碰见精精儿,正是精精儿从长安探听了朝廷的虚实动静回来的时候,精精儿就是因为怕高手不足,所以才想说服展元修参加他这个暗杀计划的。

铁摩勒听了这话,不觉又是一呆,尽管他本心不愿绪皇帝作保镖,但既然答应了师兄要尽忠职责,听到了这样的消息,他就不能不着急了。

铁摩勒再想,即算是追上了他,也不知说些什么话好,只得道声珍重,拨转马头,迳往长安。

赶到长安,方近黄昏,只见长安街道上已是乱成一片,人们扶老携幼,到处奔窜,更有许多流氓,趁火打劫,冲入店铺中去搬取货物,还有一些衣服华丽的王孙公子,号泣路旁,转眼之间,就给流氓推倒尘埃,剥去衣裳,洗劫一空。原来他们的家中婢仆,在大难来时,都已各自逃走,再也无人照顾他们了。种种混乱的情形,实是难以描述。后来大诗人杜甫,曾有《哀王孙》诗,其中有句云:``长安城头白头乌,夜飞延秋门上呼,又向人间啄大屋,屋底达官走避胡。金鞭断折大将死,骨肉不得同驰驱。腰下宝鱼青珊瑚,可怜王孙泣路隅,问之不肯道姓名,但道困苦乞为奴。''便是当时混乱情形的真实写照。

铁摩勒看到这一片混乱的情形,也不禁有点惊惶,心中想道:``难道皇帝老儿已经逃了?''他快马加鞭,在长街上冲开人群。疾驰而过,也顾不得什么官家规矩,便策马直到了紫禁城外面。

但是紫禁城城门紧闭,铁摩勒大声呼喊,城头上的乱箭便射下来,铁摩勒想道达来意,根本就没人出来答话。

铁库勒只得再纵马跑开,街道上碰见有几个官兵正在强抢一家人家的少女,铁摩勒激于义愤,大喝一声,飞骑追去,那几个官兵吃了一惊,有人叫道:``不好,是秦都尉来了!''原来他们认得秦襄那匹黄骠马,却未曾看清楚骑者是谁。

那几个官兵发一声喊,四散奔逃,铁摩勒心中一动,有了个主意,纵马追上一个官兵,一伸手就把他擒着,提上了马鞍,喝道:``快带我去见秦都尉,否则要你的命!''双指在他的琵琶骨一捏;痛得那个官兵杀猪般的大叫。铁摩勒双指一松,那官兵忙不迭地答应。

铁摩勒得那官兵指路,绕到了紫禁城后面的神武门,这个城门是秦襄把守的。秦襄的手下,见了这匹黄骠马,纷纷喝问,惊动了秦襄出来。

秦襄一眼认出了铁摩勒,忙叫打开城门,铁摩勒将那官兵一摔,秦襄道:``这是怎么回事?''铁摩勒道:``这厮是在街上强抢少女的,不过,我也幸遇了他,才得见你。我有郭令公的书信\ldots\ldots{}''秦襄忙道:``请到里面说话去。''一面吩咐下属将那个官兵捆了起来,按军法严办,一面带铁摩勒进入紫禁城。

那匹黄骠马重逢故主,高兴非常,摇头摆尾地走过去与他挨擦,铁摩勒道:``多谢你这匹坐骑,救了我几次性命。''秦襄笑道:``当日你救了我的性命,我也还未曾与你道谢呢。''

秦襄将铁摩勒带入私室,说道:``当日蒙受你的大恩,无缘报答,想不到今日却在这里相逢。铁壮士,你是在郭令公那儿得意吗?''铁摩勒道:``我并无官职,我的师兄南霁云在九原帮忙郭令公守城。''秦襄道:``啊,原来你的师兄就是南大侠,这真是久仰了。还有一位段-
璋段大侠你认识吗?''铁摩勒道:``他是我的长辈亲戚,我也曾跟他学过剑法,他们都托我向你问好。''秦襄更为欢喜,说道:``我与段大侠彼此闻名,我有几位江湖朋友与他也是相识的,只可惜有几次见面的机会都错过了。哈哈,如此说来,咱们更不是外人了。''

秦襄掩上了门,再问道:``你说有郭令公的书信,那是怎么一回事?''铁摩勒道:``他保举我做皇帝老儿的保镖。''秦襄怔了一怔,随即哈哈大笑,说道:``原来是荐你来作御前侍卫的。皇帝老儿这等称呼咱们可以私下说说,在别的侍卫面前,说到皇上,你可得肃立起敬,口呼万岁才对。''铁摩勒道:``原来还有这么些臭规矩,要不是郭令公和南师兄定要我来,我才不想干这差事呢。好,我记下了。''

秦襄笑道:``你来得正巧,皇上明天便要驾幸西蜀,我们方自忧愁保驾的侍卫不够,正需要你这等忠直可靠而又有本领的人。''

铁摩勒道:``啊,皇帝老儿明天就要走难了么?''秦襄道:``这是现在还不许外人知道的秘密,皇上已任命陈元礼为护驾将军,少尹崔光远做留守将军,京兆尹魏方进做置顿使,只待明天一早,车驾便要启行,随圣驾西幸的只有杨贵妃、杨国忠兄妹和几个亲信大臣以及皇子,其他王妃宫女皇室子弟等等,恐怕都不能带走呢!''他顿了一顿,又微笑道:``皇上避忌走难二字,你要说是`驾幸',否则会触霉头。''

铁摩勒皱眉笑道:``看来,我以后在和皇上说话之前,都得和你商量过了。嗯,你说皇上走难,不,驾幸西蜀是个秘密,但据我看来,外人都已知道了呢。''秦襄道:``外间的混乱情形我也知道了,可能是早就有了谣言。''铁摩勒道:``不但长安的百姓知道,连远在潼关的安禄山手下也得了风声,你可得小心,安禄山已请来了精精儿,要趁这混乱的时机行刺皇上!''

秦襄吃了一惊,问道:``你是怎么知的?''铁摩勒将精精儿邀约展元修作副手,被展元修所拒的事情告诉了秦襄。秦襄也知道展大娘的来历,听说展元修就是她的儿子,更为惊诧,说道:``原来这女魔头还在人间,精精儿和她勾结上了,这倒是一件大患。幸亏她的儿子还知道忠奸之分,不与他们同谋。''又吩咐铁摩勒道:``这件事情你不必说出去,宫中现在已是风声鹤唳了,不可再令皇上担惊,咱们暗地里小心戒备就是。''

铁摩勒问道:``现在我可以去见皇上了么?''秦襄道:``待我先给你禀明皇上,你暂且留在这里候旨吧。''铁摩勒有所不知,御前侍卫并不是容易当上的,过往的惯例,十九都是将门子弟或者是有资历的御临军军官充当,总之,那必定要是皇帝相信得过的人,才可以在皇帝身边,像铁摩勒这样由外臣保荐来的,那是个特殊的例子,对皇帝来说,他还是个生面人,当然不能让他一进宫门,便行觐见。

秦襄又问了一些关于郭子仪军事布置的情形,听说郭子仪已出兵河北,并且已派出南霁云到潼关重组义军,大为欢喜,笑道:``这几天坏消息太多了,难得有这样的好消息,可以告慰皇上。铁兄弟,你还未吃过晚饭吧?我叫人给你送酒菜进来,恕我失陪了。''

秦襄走后,铁摩勒不觉一片茫然,这生活的转变实在是太大了,他是在绿林中长大,又是在江湖上闯荡惯了的,如今进人皇宫,就像飞鸟被关进笼子里一样,想起今后处处要受拘束,心头闷闷不乐。

铁摩勒一人独自吃饭,他本来是不大会喝酒的,为了心里愁烦,也喝了一壶,颇有了几分酒意了。

过了大约一个时辰,忽听得秦襄哈哈大笑,和一个黑脸汉子走了进来,说道:``这位尉迟将军听说来了一个少年英雄,他也赶着要来见你了。尉迟兄是我最要好的朋友,你们今后,可以多多亲近。''

铁摩勒一看,认得就是以前和他交过手的尉迟北,不觉也大笑起来,说道:``尉迟将军,想不到咱们又在这儿会面,你还认得我吗?''

尉迟北怔了一怔,定睛瞧了他一会,搔头说道:``咦,铁兄弟,咱们以前在哪里见过的?我却怎么忘了?''铁摩勒笑道:``八年前在明风门外的那家酒楼上,我和你曾狠狠地打过一架,多谢你那时手下留情!''尉迟北拍手大笑道:``原来你就是那个胆大包天的小娃娃,长得这么高了。''

秦襄知道:``这真是不打不成相识了。你们是怎样会打起来的?''尉迟北道:``你还记得当年青莲学士醉倒明凤楼头,后来被召进宫赋诗的事么?那一天恰巧我也到那酒楼喝酒,青莲学士醉醺醺的被太监扶下酒楼,他似乎不大愿意离开,还在一步一回顾的唠唠叨叨的和他的一位朋友说话。他这个朋友也很特别,是个身穿粗布大衣,脚踏麻鞋的穷军官,相貌却很威武,一看就知是非常人。那一天御林军令狐达这一班人也在酒楼上,青莲学士走了之后,令狐达忽指那军官是叛逆,打了起来。安禄山手下的武士田承嗣、薛嵩等人也在场,他们都帮忙令狐达打那军官。铁兄弟和另一个中年汉子却忽然走来帮那军官。铁兄弟,你那时至多是十五岁的大娃娃吧?站起来还不及我的肩膊高,却打得真凶,一刀将令狐达伤了。我那时不明底蕴,只好将铁兄弟抓起来,摔到楼下,好不容易才停止了那场打斗。那中年汉子的剑法精妙无比,连伤了几个御林军军官和侍卫,我去劝架的时候也几乎吃了亏。却不知他是谁人。''

铁摩勒道:``他是我一个长辈亲戚,或许你也曾听过他的名字,他就是段-
璋段大侠;那个军官则是后来成为我的师兄的南霁云南大侠。我这次入京,他们也曾托我向你问好,并为那次打架的事情抱歉。''

尉迟北哈哈大笑道:``幸亏那时我心里想道,青莲学士的朋友总不至于会是坏人,所以令狐达指他们是叛逆,我是不相信的。因此虽然和他们交上了手,却还有惺惺相情之意,未曾真个将他们当叛逆来办。不过话说回来,以他们的本领,就算我用了全力,他们也仍能从容脱身的。''

铁摩勒道:``令狐达和那田、薛二人乃是好友,那次的事根本就是对我南师兄的诬蔑。''

尉迟北既然提起旧事,铁摩勒不免将那件事的来龙去脉告诉他们知道,秦襄听得安禄山陷害史逸如,段-璋、南霁云仗义救友等等事情,都不禁翘起拇指连呼``壮哉''。铁摩勒讲完了大闹安府的往事后,又道:``你们的人和安禄山有交情的似乎不少,有一个宇文通本领很高,那次也帮忙安禄山,他率众追捕我们,几乎要将我的段姑丈置于死地。''

秦襄面色一变,说道:``铁兄弟,我本来要告诉你一个好消息,现在,这个好消息却变成坏消息了。皇上封了你一个官职,但你却得在宇文通的手下做事!''

铁摩勒怔了一怔,问道:``我听得郭令公说,御前侍卫都是归你统管的,怎的现在却变成了宇文通是我的上司?''

秦襄道:``铁兄弟你有所不知,御前侍卫也是分为两种的,一种是在皇上身边的扈从,名为龙骑侍卫;一种则是随驾保护皇室的,名为散骑侍卫。除了这两种御前侍卫之外,还有一种名为宫中宿卫,那是在宫中轮值,担负晚上的守卫之责的。尉迟兄、宇文通和我都是龙骑都尉,但却各有专责,我统管龙骑侍卫,尉迟兄统管宫中宿卫,宇文通统管散骑侍卫。''

秦襄说明了各种待卫的职责之的,然后把刚才面奏皇上的情形告诉他道:``皇上见你是郭令公保举的人,本来有意授你为龙骑侍卫之职,那时宇文通和尉迟兄都在场,尉迟兄没有说话,那宇文通却启奏皇上,说是你来历未明,为了慎重起见,不可马上就安放你在皇上的身边,所以将你改任为散骑侍卫。皇上听从了他的主意,我也无法改变了。不过皇上现在封你作`散骑干牛',这个官职,在散骑侍卫之中却是最高级的。''

秦襄说了,神情有点不安,原来散骑侍卫是要比龙骑侍卫较低一级,而且不似龙骑侍卫那样接近皇上。

铁摩勒皱了皱眉,说道:``我不稀罕什么官职,皇上信不信任于我,我也不在乎。只是要在宇文通的屋檐底下低头,我却甚不甘心。''

秦襄道:``你且暂忍一时,将来立了功劳,我自会替你设法,将你调到我这儿来。不过,现在你却要立即去见宇文通报到,我可是有点为你担心。''

尉迟北道:``事隔多年,我都认不得铁兄弟了,那宇文通也未必就认得他。''

铁摩勒道:``他认得又怎么样?他曾和安禄山称兄道弟,我正要把他的底细抖出来。''

秦襄吃了一惊,说道:``铁兄弟,你切不可鲁莽从事。你要知道,安禄山在未反之前,最得皇上宠信,那时和他称兄道弟,甚至自认干儿的人,不知多少!这些人只要他现在不投降安贼,我们就不可动他,免得牵连太广,在这样混乱的时候,再迫反许多人,那就更不得了!而且若认真追究起来,贵妃娘娘就是第一个包庇安禄山的人,你那些话一说出来,可就要犯了大忌!''

铁摩勒摇了摇头,说道:``这也不可,那也不行。好吧,那我只好认命了,随那宇文通如何发放我吧!''

尉迟北大声说道:``铁兄弟不必担心,我陪你去见宇文通,要是他认得你,你直认无妨。他倘敢将你难为,我老黑就先赏他一顿鞭子!''

原来尉迟北乃唐初开国功臣尉迟敬德的曾孙,唐太宗李世民在未即帝位之前,有一次统兵伐魏(李密),在五虎谷与李密的焊将单雄信相遇,被单雄信追至断魂涧,几乎被俘,幸赖尉迟敬德救了性命。李世民因他救驾有功,踢了他一根金鞭,作为传家之宝,故此尉迟北有恃无恐。

秦襄正是要他这句说话,大喜说道:``尉迟兄,有你同往,谅那宇文通不敢将铁兄弟难为。''

宇文通本来无须在宫中轮值,但因皇帝的车驾明天便要启行,因此在这出发的前夕,不论龙骑侍卫,散骑侍卫,和宫中宿卫都已在宫中分头聚合。宇文通和他统率的散骑侍卫驻扎在延庆宫,和内苑仅是一墙之隔。

当下,尉迟北陪铁摩勒去见宇文通,秦襄也带了手下,到宫中各处巡查。

这时已是将近二更时份,月色甚为明朗。尉迟北带领铁摩勒,从神武门进去,穿过皇宫的外花园。月光之下,但见山石玲珑,奇花烂漫,异草粉垂,亭台楼阁、绣栏雕栏,在山坳树杪之间隐隐浮现。铁摩勒出身草莽,乍进皇宫,如入仙境。但铁摩勒郁闷难消,却是无心欣赏。

御花园的景色虽美,但在这走难的前夕,却似笼罩了一层愁云惨雾。铁摩勒一踏进了园中,便听得假山石下,花木丛中,处处有啼哭之声,原来都是些宫娥,自知不能蒙恩携走,故此到处哭泣,听得铁摩勒也不觉心酸。尉迟北摇了摇头,说道:``管不了这么多了,铁兄弟,走吧!''

走了片刻,将要穿出花园,忽见在一块假山石下,藏着一个宫娥,露出半边脸孔,尉迟北毫不在意,铁摩勒眼光一瞥,正好与那宫娥打个照面,却不由得大吃一惊!这``宫娥''相貌好熟,铁摩勒再瞧一眼,可不正是王燕羽是谁?

铁摩勒``啊呀''一声,方才叫得出口,王燕羽身形一起,在假山石上一点,已似箭一般的向前射出!

铁摩勒虽说本心不愿意给皇帝作保镖,但他乃是个最重言诺的人,既然答应了南霁云和秦襄要尽忠职责,便自然而然的起了警惕之心,一惊之下,猛地想道:``她是王伯通的女儿,我也不能太过相信她了。她三更半夜,偷入禁中,纵使非关行刺,我也得查个明白!''心念一动,立即向前追去。这时尉迟北亦已发觉,大声叫道:``有刺客,有刺客!''尉迟北的本领略在铁摩勒之上,轻功却有所不如,铁摩勒起步在先,转眼之间,就把尉迟北抛在背后。

铁摩勒发力一冲,距离王燕羽已只有数步,连忙叫道:``王姑娘,你到此何为?''王燕羽头也不回,只是反手向后一招,跑得更加快了!

王燕羽向他招手,那自是叫他跟随前往的意思,其实在此时此际,即算王燕羽不作如此表示,铁摩勒也非穷追不可!

王燕羽的轻功又比铁摩勒稍胜一筹,两人如风驰电逐,飞过了御花园的高墙,穿过了万寿宫前的长廊,前面有座金碧辉煌彩楼,楼中传出了兵器碰击的声音。

铁摩勒方自吃惊,就在此时,忽听得王燕羽一声长啸,停下步来,楼上随即有人扬声叫道:``王姑娘,快来!皇帝老儿就在这儿!''

铁摩勒大怒,长剑出鞘,一剑刺去,王燕羽一闪闪开,忽地低声说道:``傻小子,刺客在上面,你还不快去护驾!''

铁摩勒任了一怔,随即``啊呀''一声,赶紧舍了王燕羽,直奔彩楼。

但见有一僧一道和一个红面老人,正自攻上彩楼,和宫中的侍卫展开了恶战。侍卫虽然众多,但却是显然不敌,他们逐级争夺,负伤叫喊之声震耳欲聋,有好几个侍卫从楼阶的大理石级上直滚下来。

铁摩勒认得那红面老人乃是王伯通的副手褚遂,其他一僧一道他不认识,想来办当是安禄山或王伯通的手下无疑。铁摩勒只怕还有刺客已上了楼,一急之下,奋不顾身,立即施展``一鹤冲天''的绝技,身形向上一拨,手掌一按栏杆扶手,箭一般的便窜入楼中。楼门口布满侍卫,慌忙把刀砍他双足,铁摩勒也顾不得这许多,在他冲进去的时候,长剑已自展开夜战八方的招数,同时使出秋风扫叶的连环腿功夫,长剑磕飞了几般兵器,飞腿又踢倒了几个侍卫。

但见彩楼的正中,有一个身披龙袍的老人,他的左下边是一个珠圆玉润、宝光夺目的艳妇,右手边是一个衣饰淡雅的清丽少女,老人和艳妇都慌作一团,直打哆嗦;那少女的神情却还颇为镇定。铁摩勒心知这老人和艳妇定是玄宗皇帝和杨贵妃,只不知那少女是谁?

楼内还有许多侍卫,他们早已将皇帝和贵妃团团围住,这时猛见铁摩勒冲来,发一声喊,便有几个人上前迎敌,铁摩勒大叫道:``我不是刺客,我是来保驾的!''侍卫们哪里肯信,钢鞭钢锏长枪短戟,各种各样的兵器拼命打来!

正在斗得不可开交,陡然间忽听得一声尖锐刺耳的笑声,竟是精精儿的声音在大笑道:``皇帝老儿,你享福几十年,也该享得够了!宝座该换一个人坐坐啦!''

正是:何堪风雨飘摇际,又见深宫刺客来。

欲知后事如何?请听下回分解------

\chapter{第二十七回 妙手神偷惊帝座
多情公主慕英雄}\label{ux7b2cux4e8cux5341ux4e03ux56de-ux5999ux624bux795eux5077ux60caux5e1dux5ea7-ux591aux60c5ux516cux4e3bux6155ux82f1ux96c4}

声到人到,但见黑影飞来,疾如鹰隼,嘭嘭两声,在皇帝身前的两个卫士已给精精儿击倒。说时迟,那时快,精精儿手腕一翻,那柄精金短剑发出蓝艳艳的光华,闪电般的便向皇帝的胸口插去。铁摩勒被卫士们拦住去路,还未曾冲出重围,想去救援也来不及,不由得叫声``苦也''!

眼看玄宗皇帝就要死于非命,忽听得一声娇斥,在他身旁的那个少女突然一剑飞出,铮的一声,把精精儿的短剑格开。原来这个少女乃是玄宗皇帝的幼女长乐公主,天宝年间,玄宗曾请过女剑师公孙大娘进宫教宫女学习``剑舞'',公孙大娘的``剑舞''姿势非常美妙,当时誉遍京师,玄宗皇帝请她进宫,不过是想宫女学会一种新式的舞蹈,供他享乐而已,不料却有了个意外的收获,他的幼女长乐公主与公孙大娘甚是投缘,不但学会``剑舞'',而且还得公孙大娘传授她一些真正的剑术。玄宗因此更疼爱她,经常将她带在身边。

长乐公主用的是大内宝藏的``湛卢剑'',剑质尚在精精儿的精金短剑之上,两剑相交,``咣''的一声,精精儿的短剑竟损了一个缺口。精精儿吃了一惊,但他是个剑学的行家,立即便看出长乐公主的剑术尚未学得到家,出剑的劲道也差得远。一惊之后,迅即又是一剑刺来,哈哈笑道:``女娃儿,你这把剑给了我吧,我收你做女弟子!''

这一剑迳刺长乐公主的玉腕,长乐公主反手削出,精精儿已有了准备,不容她的宝剑碰上,短剑一引,引得她玉体倾斜,左手一伸,便用空手入白刃的功夫抢她的宝剑。

几乎就在精精儿剑刺长乐公主的同时,立在皇帝背后的一个卫士忽地大喝一声:``昏君,还想活吗?''一柄虎头钩就向皇帝的后心钩下。

这个卫士不是别人,正是与安禄山素有勾结的``龙骑千牛''令狐达,精精儿未来,他不敢发动,精精儿一来,他料想同伴之中,无人是精精儿敌手,遂放大了胆子弑君。

令狐达突然袭击,以为万无一失,哪知他的虎头钩还未曾落下,陡然间但觉一股猛力撞来,耳边响起了焦雷般的喝声:``贼子,还认得我吗?''

铁摩勒天生神力,这一撞直把令狐达像内球一般地抛了出来,碰翻了几个卫士,滚作一团。铁摩勒无暇再理会他,大喝一声,又一剑向精精儿劈去!

精精儿的手指已触到了长乐公主的玉腕,猛觉背后金刀劈风之声,不由得心头一凛:``皇帝老儿身边竟还有这般高手!莫非是秦襄来了?''他刚才一窜入楼中,便即扑向皇帝,只道在楼上和侍卫们已经展开了厮杀的乃是自己人,所以根本未曾注意。哪想得到这个被围的人,竟是自己的老对头铁摩勒。

精精儿脚跟一旋,``嗤''的一声,将长乐公主的衣袖撕去了一幅,长乐公主的身子也给他拧得像陀螺般地转了半个圆圈,几乎跌进铁摩勒的怀中,铁摩勒慌忙收剑,将她扶住。精精儿借公主作盾牌,避开了他这一剑,哈哈大笑,立即趁势反击,再一剑向铁摩勒刺来。

幸而长乐公主也有几分本领,她立足一稳,湛卢剑便已横削出去,铁摩勒及时跨出了一步,飞腿便踢精精儿的腰胯,精精儿一个变腰绕步,再闪开了铁摩勒的一招。

精精儿这才看清楚了是铁摩勒,气得哇哇大叫道:``又是你这小子,坏了我的大事!你这小强盗得了些什么封赏了,给皇帝老儿这般卖命?''

长乐公主这时也看清楚了铁摩勒的相貌,见是一个壮健的少年男子,不由臊得满面通红。但精精儿的短剑已似暴风骤雨般的攻击过来,她只得与铁摩勒并肩抵敌。

就在这时,褚遂和那一僧一道已杀进楼中,令狐达跌断了一根肋骨,也挣扎着爬了起来,大声叫道:``唐朝气数已尽,真命天子就要到来,识时务者为俊杰,你们还护着这昏君作什么?''

侍卫们见刺客接题而来,个个武艺高强,出手狠辣,而且还不知他们的党羽还有多少?有好些人心里发了毛,悄悄溜走。这一来,精精儿和令狐达他们更是气焰大盛。

铁摩勒大叫道:``尉迟将军就要来了,只有这几个小毛贼,没什么可怕的!''

精精儿大笑道:``我先给你这个小毛贼看看厉害!''短剑一指,连袭铁摩勒的七处穴道,铁摩勒追得撤剑回防,让开了一步。

哪知精精儿迫他一退,乘机便冲了出去,哈哈笑道:``小强盗,我才没工夫与你纠缠呢,宝象掸师,我将这小强盗交给你了。''

铁摩勒这才知道他是要抽身去刺杀皇帝,又惊又怒,拔步便追,但那胡僧已杀到了他的面前,铁摩勒一剑刺去,``吮''的一声与那胡僧的成刀碰个正着月B胡僧晃了一晃,铁摩勒的臂膊也震得酸麻,原来这个胡僧只是轻功较弱,内力却比精精儿还强,与铁摩勒正是半斤八两。

铁摩勒给那宝象禅师缠住,无法脱身,精精儿哈哈大笑,宝剑狂挥,当者披靡,转眼之间,已有六七名传卫中剑倒下,直给他杀到了皇帝的身边。

长乐公主仗着湛卢剑拼命抵挡,幸而还有几个忠心耿耿的龙骑侍卫也协力助她,将精精儿的凶焰暂阻遏,但那形势还是发发可危!

正在这最紧张的时刻,忽听得一声娇笑,一个少女的声音说道:``叔叔,得手了吗?哪一个是皇帝老儿?''却原来是王燕羽来了。

精精儿道:``王姑娘,你收拾这个丫头,其他的我自会料理!''

王燕羽桥笑道:``来了,来了!可是叔叔,你拣好的自己吃,这却不大公平啊!''这意思即是说她也要去刺杀皇帝,不屑于只杀一个公主。

铁库勒又惊又怒,大喝道:``王燕羽,你丧心病狂了吗?''王燕羽理也不理他,挺剑直奔玄宗。

精精儿笑道:``好吧,这件大功劳让给你也行!''他正要全力对付长乐公主,王燕羽已经来到,忽地一剑向他的背心刺下!

精精儿做梦也想不到王燕羽竟会刺他,饶是他轻功再好,武艺再强,这突如其来的一剑,也是逃避不开。

但听得精精儿一声大吼,登时跄跄踉踉的斜斜冲出几步,背上一片殷红,血似泉涌!精精儿也真厉害,迅即反手一点,自行封闭了背心的穴道,止住了流血,有一个侍卫想乘机攻他,还给他一脚踢开。

精精儿怒吼道:``好呀,你下得好毒手!窝里反了?''王燕羽笑道:``叔叔,谁叫你欺负我的师兄,我是给师兄出气!''

原来精精儿在碰见展元修之后,不久又碰到了王燕羽,精精儿愤不平地向她诉说展元修帮助铁摩勒与他作对之事,王燕羽探听了他们的行刺计划,便笑对他说:``我师兄不帮你,我来帮你。展师兄不知好歹,你不必理他。将来在师傅跟前,我再替你告状。''

王燕羽是王伯通的女儿,而这次行刺皇帝,就正是安禄山与王伯通策划的,因此精精儿当然信得过她。当下笑道:``你不是帮我,其实是帮你的父亲。''就这样,他们便带同了王燕羽进宫夜袭。哪想得到带来的不是帮手而是灾星。

精精儿听了王燕羽这话,怔了一怔,叫道:``原来如此,哼,哼,你这臭丫头为了师兄,竟连父亲也不要了么?''

王燕羽道:``这个就不必你多管了!你走不走,不走,你就看剑!''趁着精精儿立足未稳,展剑向他再刺!

褚遂大为着急,连忙叫道:``王姑娘,不可如此!有话以后好说,图谋大事要紧!''

褚遂是王燕羽父亲的好朋友,一向对王燕羽也甚为爱护,他精于擒拿手功夫,一急之下,就恃着世叔的身份,上来劝架,硬抢王燕羽的剑。

其实王燕羽说要替师兄``出气'',那只是一个借口而已,不过,由于褚遂与她家交谊深厚,她敢杀精精儿,却不敢与褚遂动手。

可是精精儿吃了大亏,几乎丧命,他却不肯就此罢手规的一下,精金短剑反手刺来,在王燕羽的肩头,拉开了一道三寸来长的伤口。幸而他要默运玄功,闭穴止血,劲力未能直透剑尖,要不然这一剑便足以刺穿王燕羽的琵琶骨!

褚遂见王燕羽受伤流血,但感进退两难,他向王燕羽脉门那一抓也就不敢再抓下去,只急得顿足大叫道:``看在我的份上,你们两位别自相残杀好不好?''

王燕羽使个``风刮落花''的身法,避开了精精儿的一招,这才对诸遂嚷道:``叔叔,什么图谋大事?你们这是给我家招来灭门大祸!而且还要毁了你们自己!你们也不想想,安禄山那胖胡猪岂能做个真命天子!''

精精儿大怒道:``你听,这才是她的真心话!我拼着受展大娘的责怪,也得替王伯通毙了她这不肖女儿!大事要紧,你也别拦阻了!''

褚遂叹了口气,说道:``王姑娘,这是你自作自受,我无法护你了!''转过了头,再次杀人重围,迳去捉拿玄宗。

在褚遂心中,以为王燕羽决不是精精儿对手,哪知精精儿所受的伤却比王燕羽要重得多,此消彼长,恰恰打成平手。

刺客这边的主力受了损伤,凶险的形势稍稍缓和,但那褚遂展开了近身肉搏的擒拿手功夫,接连摔翻了几个御前侍卫,对玄宗仍是一个很大的威胁。

那胡僧与铁摩勒杀得不可开交,双方都不能脱身。可是还有那个道士,乃是精精儿邀来的高手。使得一手``乱披风''剑法,也是厉害非常。这时楼中的侍卫或死,或伤,或逃,剩下的已经无几,都抵挡他不住。

正在吃紧,忽听得洪钟般的一声大喝:``鼠辈敢来行刺!''正是尉迟北大踏步走上楼来。

尉迟北一眼扫过去,见褚遂已迫近玄宗皇帝,立即一个踢步飞身,双掌一腿,同时发出,大声喝道:``老贼,你也瞧瞧我的擒拿手功夫!''

尉迟北的擒拿手乃是家传绝技,他的先祖尉迟恭(敬德)曾以赤手空拳,夺得瓦岗寨骁将单雄信的铁搠,威震天下。尉迟北精通此技,且又臂力沉雄,不逊乃祖当年。王伯通的副手褚逐虽然也通晓七十二路擒拿手法,与他相比,却不啻小巫之见大巫!

但听得尉迟北一声大喝,左掌用的是分筋错骨手法,抓褚遂肩上的琵琶骨,右掌用上了小天星掌力,将褚遂的双掌全部封住,这还不止,他还同时飞起了一腿,踢褚遂的膝盖。

这双掌一腿问时并发的功夫,诸遂连见也没有见过,褚遂的双掌已被对方的小天星掌力封住,肩头膝盖又同时受攻,他两害相权取其轻,只得弯腰俯首,先避开尉迟北向他琵琶骨的那一抓。

但听得``咕咚''一声,褚遂已被踢翻,尉迟北哈哈大笑,将他一把抓了起来,王燕羽忽地叫道:``尉迟将军,手下留情!''

精精儿相貌像个猢狲,尉迟北早就听人说过,所以一见便识得精精儿是谁。这时他见王燕羽力敌精精儿,却又出声代褚遂求情,不觉怔了一怔。问道:``这女娃子是谁?''喝声中,他已将褚遂舞了一个圆圈,力道将发未发!

铁摩勒答道:``她是我的朋友!''尉迟北喝声:``去!''倏的将褚遂掷下楼台!王燕羽听得褚遂在楼下``哎哟''一声大叫,知道他受伤虽然不轻,还不至于毙命,亦即是尉迟北已允她所请,手下稍稍留情了。

尉迟北再向精精儿奔去,精精儿短剑一个盘旋,避开了王燕羽的攻击,疾刺尉迟北的督脉三大穴,尉迟北展开了空手入白刃的功夫,只听得``蓬''的一声,精精儿短剑刺不中他,却先中了他的一掌。

尉迟北这一招本来是要将精精儿活擒的,见精精儿居然能够避开,仅仅中了他一掌,而且受了这样刚猛的掌力,居然还未倒下,也不由赞了一个``好''字,心中想道:``精精儿果是名不虚传。''

尉迟北却还未知道,精精儿是身负重伤来和他对敌的,身法远不及平时的敏捷。若是精精儿未伤,纵然未必胜得了尉迟北,最少也不会给他打中。

尉迟北喝道:``好呀,精精儿,你再接我一掌!''精精儿吓得魂不附体,急忙用``盘龙绕步''的身法避开他的三招,幸而那道士已及时赶至,展开了``乱披风''的剑法与尉迟北厮杀。

尉迟北哈哈笑道:``精精儿,原来你怕了我!也罢,待我先收拾了这牛鼻子再收拾你!''

精精儿气得七窍生烟,被王燕羽趁势猛攻,又中了一剑。幸而这一剑并非伤着要害,尚可支持。那道人的``乱披风''剑法使得甚好,尉迟北虽然着着抢攻,一时之间,也还未能得手。

混战的局面还在继续,但整个形势已是大大有利于侍卫这方。就在这时,又有一个人大踏步走上楼来,侍卫们欢呼道:``秦将军来了!''

秦襄一眼望去,见那番僧尚在奋勇冲杀,便向铁摩勒打了一个招呼,笑道:``铁兄弟,这秃驴你让给我吧。''

秦襄手起锏落,朝着那番僧的光头便砸,那番僧恃着内力沉雄,用了一招``横架金梁'',戒刀往上硬挡。

哪知秦襄有拔山扛鼎之能,乃是唐宫的第一条好汉,气力比尉迟北还胜三分,他这两条金装锏,每条重六十四斤,打将下来,当真有如泰山压顶。

但听得咣的一声,番僧那口戒刀,碰着金锏,刀口全都卷了,秦襄左锏又落,那番增无可躲避,翻转刀背,再接一招,这一锏力道更猛,但听得那番僧大吼一声,虎口已是震裂。秦襄笑道:``再接一锏,接得下便饶你不死。''话犹未了,第三锏也尚未曾打下来,只见那番僧晃了两晃,``咕咚''一声,便似一根木头般的直倒下去,鲜血喷了一地。原来秦襄用的是家传的``杀手锏''功夫,从未有人敢连续挡他三锏,这番僧不知厉害,与他硬拼内力,挡了两锏,五脏六腑,都已给震得反转过来,全身精力也都耗尽了。

就在这时,尉迟北已把那道人的长剑夺到手中,那道人心胆俱寒,抢到窗口,撞碎窗格横木,便跳下去,尉迟北喝道:``还想逃吗?''长剑脱手掷出,从那道人的后心穿过了前心,尸横楼下。

尉迟北哈哈笑道:``精精儿,轮到你啦!''精精儿自知必无幸理,怒声叫道:``小妖女,我死为厉鬼,也不能饶你!''精金短剑猛地往外一推,将王燕羽震退两步,铁摩勒正要上前,只见他已把短剑收回,向自己的胸口刺下。

精精儿素来自负,他是抱着宁死不辱的心情想自杀的,可是在这性命俄顷的关头,不免稍稍踌躇,剑锋尚未划破皮肉,忽听得远远传来一声啸声!

精精儿一跃而起,叫道:``师兄,快来救我!''铁摩勒大惊叫道:``是空空儿!''

空空儿来得快如闪电,顿时间,那啸声已震得众人耳鼓嗡嗡作响,秦襄和尉迟北,这时哪还顾得及去收拾精精儿?两人一听到了啸声,都不约而同的奔去救驾!

尉迟北一声大喝,使出分筋错骨手法,一手抓去,空空儿笑道:``尉迟将军,久仰了!''空空儿分明就在他的面前,说话的声音也在他的耳边,但他一手抓下,竟是空无一物,似乎那空空儿竟然不是有血有肉的真人,而是一团幻影!

尉迟北这一惊非同小可,眨眼之间,但见玄宗皇帝和杨贵妃的身前、身后、身左。身右,同时出现了无数个空空儿的影子!原来他是展开最迅捷的身法,绕着皇帝和贵妃游走,由于快到无以形容,因此旁人但见幻影重重,眼花缭乱!

秦襄高举双锏,却不敢打下。众侍卫更是目瞪口呆,谁都怕误伤了皇帝,而且由于幻影重重,谁也不知道``真正''的空空儿在哪个方位。

空空儿大笑道:``秦将军,尉迟将军,累众位担惊受怕,我实在抱歉之至,但我入了皇宫,如入宝山,绝不能空手而回,少不得要取些彩物了。''

话犹未了,只听得杨贵妃一声尖叫,空空儿的影子倏然消失,众人愕然惊顾,只见他已到了精精儿的身边。

空空儿摊开掌心一晃,掌中有一颗光泽夺目的大圆珍珠,食指中指之间,还夹着一根玉簪。

空空儿笑道:``我并不贪心,请你们看清楚了,就是这两件东西!''原来他偷去的乃是杨贵妃头上的玉簪和玄宗皇冠上的珍珠,这两件东西虽然都是价值连城的宝物,但他没有伤损皇帝的分毫,这已经是大大出乎众侍卫们的意料之外。

这刹那间,谁都噤不敢声,只怕招恼了他,偷东西事小,伤了皇帝,那就事大了。

精精儿嘶声叫道:``师兄,为何不把那昏君杀了?''

空空儿双眼一翻,``啪''的一声,忽地打了他师弟一个嘴巴,骂道:``混帐,咱们是盗亦有道,岂可给别人做咬人的凶狗?尤其安禄山那胖胡猪,我更看不起他。你不怕贬低身份,我也替你羞愧!不是见你已受了伤,我还要狠狠打你一顿。回山去吧,我罚你面壁三年!''

空空儿一手将师弟抓了起来,就像提个小鸡似的,精精儿哪敢挣扎。

空空儿眼光一扫,看见了铁摩勒,笑道:``铁兄弟,你若见到段大侠,烦你转告于他,请他放心,他的儿子很好。''

铁摩勒正要问他,空空儿挟着他的师弟,已从窗口跳出,临走之时,还在哈哈大笑,说了一声:``众位将军,少陪了!''

楼下众侍卫哗然惊呼,纷纷放箭,秦襄喊道:``万岁平安无事,刺客尽已受歼,你们不必闹了。''

忽听得有人叫道:``这里还有一个漏网的贼人呢!哼,令狐达,你人面兽心,欺君犯上,万死不饶。''

却原来是那令狐达趁着混乱的时机,偷偷溜走,不料刚出楼门,便碰见了宇文通,被宇文通一把拿着。

他和宇文通本是同谋伙伴,听了这话,大惊失色,叫道:``宇文将军,你,你\ldots\ldots{}''宇文通哪肯容他说话,迅即拨出佩刀一刀将他劈了。

尉迟北叫道:``哎哟,你简直比我还要鲁莽,怎么不留一个活口?''宇文通道:``他是我的部下,竟敢作出这等大逆不道之事,我气愤不过,一时间竟未想到要留下活口审问了。''他揩了刀上的血迹,立即便走进楼来,俯伏在皇帝跟前,叩头有如捣蒜,奏道:``臣宇文通护驾来迟,又驭下不严,有惊龙体,请陛下降罪。''

玄宗道:``你们都是朕的忠心巨子,联的心腹大将降贼的也不知多少,令狐达算得什么,宇文将军,你也不必引罪自咎了。''要知玄宗虽然沉迷酒色,却也还不是十分昏庸之主,因此在这用人之际,他不能不说这番说话笼络人心。宇文通谢了``圣恩'',站过一边。

玄宗惊魂稍定,还能保持着皇帝的尊严,杨贵妃却还在浑身打抖,这时才叫得出声:``吓死我了,吓死我了!''

玄宗又是心痛,又是怜惜,连忙叫一个宫女过来,说道:``爱妃,你进去歇歇吧。幸得平安无事,你也可以好好睡个觉了。明天还要起早赶路呢。''他本来想亲自扶杨贵妃回房安息,但他是皇帝的身份,在乱事平定之后,必须对有功之人,加以奖赏。

当下评定功劳,皇帝与众侍卫有目共睹,公认王燕羽功劳第一,她在皇帝最危险的时候,刺伤了精精儿,扭转了局势。其次是铁摩勒,他最先进来救驾,力拼精精儿,救了皇帝,又救了公主,再其次才是尉迟北与秦襄。

铁摩勒与王燕羽双双上前见驾,秦襄代为禀道:``这位少年壮士,就是郭子仪保荐来的那个人。''皇帝点了点头,说道:``你忠勇可嘉,朕已封你为`散骑千牛',现在你立了大功,自当再加升赏。你先站过一边,待朕与秦将军、宇文将军商量之后,再行定夺,给你安排。''接着便传王燕羽上来问话。

王燕羽跪倒御前,莺声呖呖的三呼``万岁'',玄宗道声:``免礼,平身。''叫她抬起头来,瞧了一眼,心里暗暗赞道:``好个标致的美人儿,活脱脱像采苹初人宫时的模样。''采苹是玄宗一个妃子的名宇,长得轻盈秀丽,最爱梅花,受封为``梅妃'',玄宗未纳杨贵妃之前,以她最为得宠。杨贵妃将她视为目中之钉,心头之刺,她擅宠专房之后,即不许玄宗再亲近梅妃,这次避难西蜀,也不许玄宗带梅妃同行,玄宗对她自是难免有所思念,故此看见王燕羽长得有几分相似梅妃,心里便先欢喜。

长乐公主道:``姐姐,你使得好剑法,这次多亏你了。''王燕羽道:``多谢公主夸奖。''长乐公主道:``你许配了人家没有?''王燕羽面上一红,想不到公主为何如此问她,答道:``民女尚未许配人家。''

长乐公主笑了一笑,说道:``那么,你今后就陪伴我如何?父皇,你赏她一个封号,叫她做我的女官吧。''原来按照唐宫规矩,在公主未出嫁之前,公主的``伴读'',以及在公主府中侍奉的女官,也必须是未婚女子。不过,长乐公主要知道她是否已婚,却还另有一层用心,以后再表。

玄宗笑道:``难得你欢喜她,朕就让她做你的女`主簿'(官名)如何?你可愿意陪伴公主么?''后面这句话是面对王燕羽说的。

王燕羽道:``多谢皇上和公主的恩典,只是民女出身草莽,不敢伺候公主。''长乐公主不懂什么叫做``出身草莽'',还在说道:``那有什么关系?''玄宗却吃了一惊,想了一想,说道:``朝廷现在是破格用人,只要有功国家,就不问他的出身。不过,你若是不愿在宫中任职,朕也可以另外赏赐你。''心里想道:``好好一个美人儿,却怎的生在强盗家里?''他虽然欢喜王燕羽,这时也不敢再说要留她在宫了。

王燕羽道:``我不敢侈求,只想皇上赏赐我一件东西。''玄宗道:``你说吧,你要什么宝贝。我大内都有。''王燕羽瞥了铁摩勒一眼,说道:``我不要珍珠宝贝,我只是想要、想要\ldots\ldots{}''铁摩勒心头卜卜地跳,只怕王燕羽要的是他。

玄宗道:``你快说吧,只要是朕拿得出的,一定给你。''王燕羽道:``我只是想求皇上赏我一面免死金牌。''玄宗诧道:``你犯了什么大罪,要朕赏你免死金牌?''王燕羽道:``这是我代我父亲求的。''玄宗道:``你父亲是谁?''王燕羽道:``我父亲是北五省绿林盟主王伯通。''玄宗大吃一惊,道:``你是王伯通的女儿!你父亲不是帮安禄山造反的吗?''

王燕羽道:``正是因此,所以才求陛下赐给免死金牌。''玄宗好生为难,想了一想,说道:``你能够劝你的父亲归顺于朕么?这样,就不只可以免他一死。朕还可以让他做个节度使。''王燕羽道:``我父亲性情刚愎,只怕劝他不转。不过他的部属都已给南大侠打得七零八落了,现在只是寄人篱下,无足为患。''玄宗道:``哪个南大侠?''王燕羽道:``郭子仪部下的骁骑将军南霁云。''

王燕羽又道:``安禄山现在也不是怎样看重他了,我劝他归顺陛下,或有困难,因为他好歹都是绿林盟主,一旦归顺朝廷,那就是犯了绿林大忌。但我当尽我所能,劝他金盆洗手,闭门封刀。''

玄宗道:``什么叫做金盆洗手,闭门封刀?''铁摩勒一时口快,代为答道:``那是绿林中的黑话,意思就是说以后再不干强盗的营生,遁迹山林,也不再理任何外事了。''

玄宗望了铁摩勒一眼,点了点头,对王燕羽道:``念你救驾有功,联也可以破例开恩,你若能劝你父亲金盆洗手,朕就赐他一面免死金牌。以后凡是朕的文武百官,捉到你的父亲,都不能擅自杀他。但倘若他还在安贼军中,阵前交锋,则格杀不论。''当下叫内侍取过一面金牌,御笔批明,交与王燕羽收执。

铁摩勒见王燕羽替她父亲取得了免死金牌,心中是又喜又忧,喜者是王燕羽今晚的表现,的确足以证明她已改邪归正;忧者是王伯通已有了御赐免死金牌,而自己现在又已受了朝廷官职,以后如何可报义父之仇?

王燕羽接过金牌,谢过了皇帝的恩典,禁不住眼中露出喜悦的神情,向铁摩勒膘了一眼。那喜悦的神情忽地消失得无影无踪,她目注着铁摩勒,却黯然说道:``多谢圣上洪恩,多谢公主好意,也多谢众位将军,民女如今走了,以后大约也不会再来和各位见面了!''话声一收,倏的便从窗口跳了出去。_

众人听了她这番说话,都觉得有点特别,只有铁摩勒心中明白,王燕羽这话实是对他说的,以后她也是不愿意再见到自己了。这话实是含有请他珍重也与他诀别的意思。不知怎的,铁摩勒忽地感到一些怅恫,目送她的背影穿窗而去,竟出了神。

宇文通忽地出声问道:``铁铮,你和这女子是相识的么?''铁摩勒因为自己本来的名字,江湖上相识者多,所以改名为``铮'',取``庸中皎皎,铁中铮铮''之义,郭子仪给他的保荐文书,用的就是这个新名字。

玄宗皇帝听宇文通这么一问,他记起来了,说道:``对啦!刚才朕记得你说过这女子是你的朋友,你们是怎么认识的?''

铁摩勒当然不敢说出真相,但他又不善于说谎,只得讷讷说道:``那是我在江湖上闯荡的时候认识的。''玄宗``哦''了一声,微笑说:``我以为你是在郭子仪的军伍出身,却原来你也是一位江湖好汉。''

秦襄和尉迟北都捏了一把冷汗,只怕玄宗再追问他的底细,长乐公主忽地插口说道:``父皇想必还记得青莲学士吧?这位鼎鼎大名的诗人,少年时候也曾是一位游侠。他的诗超脱而又豪迈,大约也是与他作过游侠有关。铁壮士,你会作诗么?''

铁摩勒笑道:``我只会舞刀弄抢,却不懂吟诗作赋。''

长乐公主道:``那你在江湖上所见所闻的趣事必然不少,将来得闲无事之时,说给我们听听,解解闷也好。''

长乐公主故意打岔,说的似是无关紧要的闲话,其实却是大有用心,她是怕父皇对铁摩勒起疑,故此持地将李白抬出来,说李白也曾在江湖闯荡过,江湖人物并没有什么可怕。

果然玄宗皇帝笑了一笑,便道:``现在可不是谈诗论词的时候了,我是宁愿多一位像他这样的壮士,胜于要青莲学士来陪伴朕了。''

宇文通忽地也插口道:``正是呀,英雄多半出风尘,铁壮士,你在江湖上的交游可也真广阔呀,那个猴子般模样的刺客的师兄,我知道他就是神偷空空儿,听来他也似与你很熟识,刚才临走的时候,不是还拜托你向一位什么段大侠问候吗?''

铁摩勒道:``好几年前,我和空空儿打过一架,我们是打了才成相识的,却并非什么熟朋友。''

秦襄也道:``空空儿所说的那位段大侠,我是知道的,他名叫段圭璋,当真是任侠仗义,像青莲学士一般的人物。现在听说也在给郭子仪效力。''

玄宗心里可是有点惊疑,空空儿的名头太响亮了,玄宗也曾听人说起过他。今晚领教了他的手段,给他取去了皇冠上的珍珠,现在还是惊魂未定。他不禁心里想道:``这姓铁的小伙子交游也真是太杂了。''他本来有意将铁摩勒封为龙骑都尉,令他随侍身边的,听了宇文通这几句话,心里便有点迟疑不定。

玄宗沉吟半晌,问秦襄和宇文通:``你们看给他个什么职位合适?''秦襄因为宇文通是铁摩勒的顶头上司,请他先说。

宇文通却也溜滑,当下奏道:``铁铮武艺高强,对江湖人物,又很熟悉,处此乱世,正宜重用。至于任他何职,臣下不敢妄参末议,还请陛下圣裁!''

宇文通这几句话表面听来,似是推重铁摩勒,其实却是特地挑起皇帝的疑心,玄宗听了,果然沉吟不决。

长乐公主忽道:``父皇,我看他忠厚老实,又是郭子仪荐来的人,定然不会差错,就着他护卫内宫眷属的车驾如何?''

玄宗这次避难西蜀,虽然不能多带妃嫔,但公主,诸王子的王妃,以及一些贴身服侍的宫娥总是要带的,玄宗预定自己与杨贵妃同一车驾,由秦襄率龙骑都尉保护,请王子的车驾由尉迟北率原来的宫中宿卫保护,宇文通的散骑侍卫则照料其他车驾。但散骑侍卫为数不多,公主们的车驾还没有指定专人保护。

玄宗心念微动,看了铁摩勒一眼,沉吟半晌,说道:``也好,铁铮,你听朕封赏!''

秦襄推了铁摩勒一把,铁摩勒这才知道要跪下来,只听得玄宗说道:``铁铮救驾有功,封为虎牙都尉,幸蜀途中,护卫公主车驾,听长乐公主调度,隶属宇文通散骑,衔加散骑副中郎将。另赐黄金百两,锦绢十匹,以奖有功。''

虎牙都尉比龙骑都尉仅低一级,他另加``散骑副中郎将''衔,即是等于宇文通的副手,但在逃难途中,则由长乐公主直接指挥,亦即是等于有两个长官,若有承平之日,依宫中体制,决无如此之例。玄宗今作如此安排,一来是为了逃难的权宜处置,二来是为了顺从女儿心意,三来也是为了看重郭子仪,唐朝天下,正要靠郭子仪支撑,所以对他荐来的人,虽然略有疑心,仍然相当信用。

铁摩勒虽得连升三级,但依然要作宇文通的副手,心中当然还有点不大乐意,但圣旨既下,也只得叩头谢恩。

宇文通心里妒忌,神色上却没有半分显露,铁摩勒谢恩之后,他第一个上前道贺。

玄宗将今晚有功之人各加封赏之后,便令侍卫散去,准备车驾,明日一早,便要启程。

下了明风楼,铁摩勒本想随秦襄回去,宇文通道:``现在已是三更,我看今晚大家都休要睡了。铁都尉,今晚散骑侍卫都已聚集在延庆宫,你与我去见见同僚,彼此也好相识。''

铁摩勒听他说得有理,只好与秦襄分手,尉迟北道:``宇文将军,这位铁兄弟是我的好朋友,你可得好好看待他。''

宇文通笑道:``铁都尉现在与我共事同朝,有如手足相依,这个还何劳吩咐?''

铁摩勒与他同行,宇文通不断用言语刺探他的来历,铁摩勒信口胡说一通,他不善说谎,当然露出了许多破绽。

走了一会,到了路灯之下,宇文通忽道:``铁都尉我越看越觉得你好生面善,咱们可是在什么地方见过的么?''

铁摩勒强笑道:``我是在江湖浪荡的无名小卒,岂能见过大人?''宇文通笑道:``如此说来,大约我与你很有缘份,所以一见如故了。''伸出手来与铁摩勒紧紧一握。

铁摩勒恼恨于他,运足了十成劲力,宇文通长于判官笔打穴,功力却稍有不如,一握之下,虎口隐隐作痛,吃了哑亏,只得哈哈笑道:``铁都尉好大的气力,有你相助,此次西行,定卜平安,我也可以减少许多忧虑了。''

到了延庆宫,散骑侍卫约有二三十人聚集在那儿,宇文通介绍他们与铁摩勒一一相见。

其中一人忽地嚷道:``铁大人,恭喜恭喜,可还记得小的么?''

铁摩勒一看,认得是郭子仪麾下的一个小校尉,名叫贺昆,八年之前,铁摩勒扮作辛天雄的随从,第一次到龙眠谷赴会时,就是这个贺昆招待他到马房中吃饭的。在九原时铁摩勒已曾对他起了疑心,也曾请南霁云将这人的底细转告郭子仪,叫郭子仪对他小心在意。

铁摩勒怔了一怔,问道:``贺昆,你也当了散骑么?''贺昆道:``我是奉了郭令公之命,来送捷报的。咱们在河北已打了两场胜仗了。我因与宇文将军旧识,故此匆匆来此与他一叙,明早就要回去。''

铁摩勒道:``原来如此,请你回去你我问候令公。''贺昆道:``一定,一定。铁大人,你已得皇上重用,令公得知,也必然欢喜的。还有南将军呢,不知他现在哪儿?''

铁摩勒道:``郭令公差遣我来给皇上当差,我与南将军离开九原之后,便分道扬镳,我可不知道他的去向。''

宇文通问道:``铁都尉,你与南霁云南将军交情很好吗?''铁摩勒见有贺昆在旁,只得如实说道:``他是我的师兄。''宇文通哈哈笑道:``原来你是南大侠的师弟,怪道如此了得!''

铁摩勒在郭子仪军中用的也是``铁铮''这个名字,他识得贺昆,至于贺昆是否亦已识破他的来历,他就不知道了。

幸而就在此时,忽听得景阳宫的大钟咣咣地敲了三下,登时四下人声鼎沸,黄门内监跑来跑去的穴叫道:``准备车驾,开启宫门!''宇文通命令散骑侍卫立即出发,在延秋宫门外等候圣驾出宫。混乱之中,那贺昆已不知在什么时候走了。

正是:景阳钟鼓惊心魄,圣驾仓皇走避胡。

欲知后事如何?请听下回分解------

\chapter{第二十八回 颠沛流离悲百姓
饥寒交迫涣军心}\label{ux7b2cux4e8cux5341ux516bux56de-ux98a0ux6c9bux6d41ux79bbux60b2ux767eux59d3-ux9965ux5bd2ux4ea4ux8febux6da3ux519bux5fc3}

铁摩勒不觉起了疑心,暗自想道:``这贺昆不过是个小小的陵尉,怎能直进宫门,与宇文通相会?再者,郭令公帐下多少能人可堪信托,这贺昆的底细,令公又已略有所知,却怎的还会差他来送捷报?嗯,看来其中有诈,怎地想个法儿使令公知道才好!''

这时,宫中早已惊动,宫人乱出,嫔妃奔窜,哭声喊声,嘈成一片!铁摩勒已无暇追寻贺昆的下落,只得随着人流,拥向延秋门。

但见无数宫娥美女,抢地呼天,攀着车辕,想要挤上车去。但每一辆车的旁边,都有卫士防护,在这关头,已顾不得借玉怜香,起初卫士们还只是把她们推开,后来高力士喊道:``谁敢强自登车的,将她们的手折了!''果然斫了几双血淋淋的粉臂,好不容易才驱散了那些官娥太监。

铁摩勒对此情景,惨不忍睹,忽听得宇文通笑道:``你在这里发呆作什么?还不快去伺候公主?''

这时宫门已经打开,数十辆车驾,纷纷拥出,铁摩勒认得有黄盖的是皇帝的车驾,长乐公主乘的是哪一辆车,却不知道。

他策吗越过几辆宫车,正想找个太监问问,忽听得身边一辆宫车,有个娇媚的声音笑道:``姐姐,你瞧瞧,这个小伙子倒长得怪俊的,以前没有见过,喂,你是新来的卫士么?''

铁摩勒抬头一看,见是两个妖艳的女人,心里正自想道:``这两个女人怎的如此肆无忌惮?简直不知羞耻。''宇文通已是纵马过来,就在马背上打躬作揖,笑道:``这是皇上新授的虎牙都尉铁铮,刚刚上任,未知宫廷礼数,两位夫人见谅。铁铮,你还不快来行礼,这位是韩国夫人,这位是虢国夫人!''

铁摩勒这才知道是杨贵妃的两个姐妹,又是感慨,又是讨厌,心想:``多少大臣都不能同行,杨家的兄弟姐妹却凭着什么功劳都得追随圣驾,还要我们伺候!''想至此处,不觉``哼''了一声,说道:``对不住两位夫人,我奉命护驾公主,请恕我不能伺候你们了。''呼的一鞭,赶马向前,头也不回。气得韩国夫人、虢国夫人面皮发黄。

宇文通追了上来,笑道:``这两位夫人的权力比公主还大得多,你不知道么?''铁摩勒板着面孔道:``我不知道,你知道你去巴结她们去!''宇文通怔了一怔,又笑道:``小伙子,脾气好大呀!不过,你也有你的道理,公主对你青眼有加,你还是专心去讨好公主更妙!''铁摩勒大怒道:``我铁某可是从不懂得逢迎谄媚的人,宇文将军,你休胡说!''宇文通面上一阵青一阵红,尴尬之极,勉强笑道:``铁都尉,我这是为了你的好啊!你不领情,那就随便你吧,我管不着!''讪讪走开,隐隐地发出了两声冷笑。铁摩勒找到了一个执事太监,那太监告诉他,前面那顶圆顶宫车,就是长乐公主的车驾,铁摩勒赶上前去,满怀委屈地禀道:``铁铮在此,听候使唤!''

长乐公主半启车帘,露出脸来微笑问道:``铁铮,你和宇文都尉是在吵架么?''铁摩勒面上一红,说道:``没什么,只因人声嘈杂,说话大声点儿。''

长乐公主笑了一笑,也没再说什么,只吩咐铁摩勒的坐骑要傍着宫车,不可离开太远。过了一会,长乐公主忽又探出头来,问铁摩勒道:``你和王伯通是相识的么?''铁摩勒变了两色,迟疑未敢答话,长乐公主笑道:``他是叛贼,你是护驾功臣,纵然相识,也没牵连,你据实说吧。''铁摩勒只得说道:``不敢欺瞒公主,那王伯通是我的仇人!''

长乐公主诧道:``这倒奇了,你和王伯通的女儿不是很要好么?她怎么是你的仇人?''铁摩勒道:``王伯通是打家劫舍的大强盗,我的家人就是给他杀掉的。至于他的女儿,则是我在闯荡江湖的时候认识的,那时我还未知道她就是仇人的女儿。后来知道了,但见她行事与父兄有别,所以不拟向她寻仇,但也说不上有什么交情。''

长乐公主道:``哦,原来如此,你倒是见事清楚,恩怨分明。一人做事一人当,王伯通与你结下的仇,本不该他的女儿担当。''

两人说了一阵闲话,长乐公主与他讨论剑法,她将公孙大娘传授给她的剑诀背给铁摩勒听,请铁摩勒指教。公孙大娘是当代数一数二的剑术大师,剑学精深尚在段圭璋之上,不过因为长乐公主火候未到,未能运用自如,所以才敌不过精精儿。铁摩勒嗜武如狂,他最初与长乐公主谈话,不过是敷衍敷衍而已,一到讨论剑法,却不由得精神勃发,与长乐公主倾谈,滔滔不绝。

长乐公主从车内抛出一颗梨儿,说道:``铁都尉,你吃颗梨儿,解解渴吧。''铁摩勒道:``谢公主赏赐。''长乐公主叹口气道:``一颗梨儿算不了什么,但只怕离了长安,再过些时,要吃它也不容易了。''铁摩勒也不禁黯然,勉强安慰公主道:``公主安心,咱们不过是暂时走难,总有回来的一天。''他一时改不了口,忘了秦襄的吩咐,又把``驾幸''说走了``走难'',幸而公主似乎也没留意。

说话之间,忽听得兵士喧哗,铁摩勒回头一看,见后面一团火光,却原来是兵士们在放火烧一座桥梁。

火光融融,惊动了玄宗,停车查问。杨国忠奏道:``这是臣下的主意,焚毁桥梁,以防追者。''玄宗叹道:``百姓各欲避贼求生,奈何绝其生路!''乃命高力士率军士速往扑灭之。杨国忠碰了一鼻子灰,做声不得。

走了一会,驾过``左藏'',这是皇家的一个库仑所在,玄宗又见有许多军役,手中各执草把在那里伺候,玄宗因又停下车驾问其缘故,杨国忠奏道:``左藏积有粮食财货颇多,一时不能载去,将来恐为贼所得,臣意欲尽焚之,无为贼守。''玄宗愀然说道:``贼来若无所得,必更苛求百姓,不如留此与之,勿重困吾民。''遂命高力士叱退军役,驱车前进。

铁摩勒见此两事,心中想道:``如此看来,这皇帝尚知爱惜子民,杨国忠却全不顾念百姓,大唐的江山,坏就坏在他们这班人手里。''却不知这正是玄宗的权术,在逃离之际,宗庙难保,自不能不笼络民心。不过话说回来,纵是权术,他到底也要比杨国忠宽厚一些,聪明一些。

逃难途中琐事,不必尽表。只说由于``圣驾''仓皇避难,所带的粮食并不充足,初时还可以就地补给,哪知``圣驾''一逃,风声四播,各地的官员百姓,都知道官家已放弃了京城,贼兵指日可到,俱先逃避。玄宗军驾所过之处,十室九空!数日之后,到了咸阳的行宫------望贤宫,行宫的留守官兵,也尽都逃了,日已晌午,随从军士,犹未进食。

幸喜咸阳郊区,还有一些百姓,护驾大将军陈元礼命令军士进村搜寻食物,百姓或献粝饭,杂以麦豆,不但军士们甘之如饴,王孙辈也争以手掬,食之须臾而尽。玄宗命以金钱重酬,百姓多痛哭失声,玄宗亦挥泪不止。

众百姓中有个白发老翁,携了一篮食物,军士纷纷向他拥去,他却推开军士,说道:``我这是要献给皇上的。''篮中所有,也不过是一些粗饭,军土道:``皇上哪里会吃你这些东西,还是给了我们吧。''那老翁大声说道:``我是要皇上知道甘苦,我还有话要奏禀皇上。''说也奇怪,那老翁衰额白发,气力却是惊人,他昂然直走,兵士们竟给他推得东倒西歪。

秦襄听得喧闹,走过来看,吃了一惊,说道:``郭老前辈,原来是你。''原来这个老翁名叫郭从瑾,少年时候也曾是一位名震江湖的侠客,中年之后,闭门隐居,传了一个徒弟,他的徒弟比他的名头更响,乃是与段圭璋、南霁云差不多齐名的金剑青囊杜百英。

秦襄认得是他,问知来意,便道:``老丈请稍待片刻,容我先行奏禀。''

玄宗听得有乡中父老来献食物,并求觐见,大为感动,说道:``寡人无道,重负百姓,流离之际,尚有父老雪中送炭,能不汗颜?''秦襄奏道:``得民者昌,民心未失,大唐之福也。''玄宗便令秦襄引郭从瑾来见。

郭从瑾道:``这是老百姓日常所吃的糙饭麦豆,请陛下尝尝,但愿他日升平,毋忘此时之苦!''玄宗哪里咽得进口,但为了笼络民心,只得假惺惺地吃了一点,赞道:``有情白水胜美酒。这篮麦饭,是父老对朕的爱戴之心,实胜于大内珍馐!''

郭从瑾涕泣进言道:``安禄山包藏祸心,已非一日,当时有赴阙若言其反者,陛下辄杀之,使得逞其奸逆,以致乘舆播迁。所以古圣王务廷访忠良,以广聪明也。犹记宋景为相,屡进直言,天下赖以安;然频岁以来,大臣皆以直言为讳,唯阿谀取容,是以阙门之外,陛下俱下得而知。草野之人,早知有今日久矣;但九重严邃,区区之心无路上达,事不至此,何由得睹天颜而诉语乎?''

这番说话听得在皇帝旁边侍立的杨国忠和高力士等辈,面色全部变了。玄宗顿足嗟叹道:``此皆朕之不明,悔已无及。多谢老丈直言。''解下玉带,温言谢遣。

铁摩勒已向秦襄问知他的来历,待郭从瑾告退,便道:``郭老前辈,我送你一程。''郭从瑾认不得他,有点诧异,秦襄道:``这位铁都尉刚从九原来,月前尚与今徒百英兄在一处。''郭从道道:``原来如此,老朽也正想投往郭令公军中。''\,''

铁、秦二人将郭从道送出五里之外,铁摩勒告诉他杜百英在金鸡岭辛天雄处,临分手时又想起一事,再拜托郭从瑾道:``郭老前辈若是见到令公,请转告他我在长安曾见到贺昆,恭贺的贺,昆仑的昆,此人与宇文通往来甚密。请令公小心。''

回来途中,秦襄听了铁摩勒细说贺昆之事,对宇文通也起了疑心,但叮嘱铁摩勒不要多言,暗中留意。

过了咸阳,逃难的生活更是越来越苦,兵士逃亡,日有所闻,不消多日,十停中便已走了三停。这日到了一个地方,名叫马嵬驿,忽然碰到了一场大风雨,打得施旗零落,人仰马翻,车篷破漏,衣甲不全,无法再往前行,只好到树林中避雨,找到了一个破庙,给皇帝贵妃王子们栖身,土兵们则只好躲在大树底下任由雨打。

这场雨一连下了数日,积水成灾,桥毁路坏,前行不得,后退不能,大队人马被困在马嵬驿。这时已是秋初时分,气候渐冷,兵士衣单,当真是饥寒交迫,苦不堪言!

从长安带来的军粮早已吃光,沿途从民间搜索来的粮食有限,要留供御驾以及杨国忠等皇亲国戚享用,士兵们只好屠杀马匹,采摘野菜充饥,过不了几天,军马屠杀殆尽,野菜也难以寻觅了。将士饥疲,都怀愤怒,怨声四起。

铁摩勒与士兵们同甘共苦,深知士兵们的怨愤,心中忧虑,难以言宣。这日幸喜雨已停了,但尚未放晴,铁摩勒上山打了两只樟子回来,晚上熬了一大锅肉汤与士卒们同喝。

他们在林中燃起野火,那锅肉汤每人分不到一小勺,士兵们聚在一起,大发牢骚,十个有九个都在痛恨杨国忠,有的还骂到了杨贵妃!杨国忠的卫士也听到了,在群情汹涌之下,他们哪敢前来干涉,只有远远避开,佯作不闻。

士兵们中有人叹道:``看来咱们已是注定了要命丧他乡,这副骸骨,不知埋在哪个荒山野地?''愤气未平,乡思又起,也不知是谁先哭出了声,顿时间呜咽之声四起,饶是铁摩勒这样的硬汉子,也不禁心酸。他既是伤心,又是忧虑,心中想道:``士气沮丧,一至如斯,若然碰到敌人,准得一败涂地!''

有个擅于吹笛子的小兵,吹起了家乡的曲调,又有一个军中的小主簿(掌管文书的官儿)用嘶哑的声音,唱起了杜甫的一首诗:``支离东北风尘际,漂泊西南天地间。三峡楼台淹日月,五溪衣服共云山。揭胡事主终无赖,词客哀时且未还。庾信平生最萧瑟,暮年诗赋动江关。''

这诗是杜甫咏怀古迹诗五首之一,说的是南北朝文人庾信的故事,他在南朝的梁亡之后,流落于西魏北周,终于老死他乡,曾作有《哀江南赋》表达乡思,充满了故国兴亡之感。杜甫此诗借古迹咏怀,以庾信自况,也是自伤飘泊的。

唐朝诗风最盛,尤其李、社二人的诗篇,当时差不多人人都能吟诵,士兵们纵使不知庾信其人其事,也略解诗中之意;纵使不解诗中之意,也听得出诗中那种愁思。``支离东北风尘际,飘泊西南天地间\ldots\ldots{}''这两句诗一唱起来,叹息声与啜泣声便此起彼落了。

铁摩勒不忍再听下去,悄悄离开,忽地在个宫女从林中闪出,说道:``铁都尉,我正在找你,公主有请!''

铁摩勒怔了一怔,道:``夜已深了,这个时候去谒见公主,怕不便吧?''那宫女道:``公主不在`行宫',她在后面的林子里等你,有紧要之事与你商量,你快去吧。''

皇家有皇家的规矩,这时虽是逃难之际,皇帝住的也是座破庙,但依然要尊称为``行宫''。在``行宫''周围的数十丈方圆之地,除了是龙骑侍卫之外,其他随从将土,都不许踏进,破庙后面的一片林子,也列为禁地。铁摩勒不是龙骑侍卫,但他宫居``虎牙都尉'',是散骑侍卫的副统领,又是皇帝特别指定地护卫公主的,所以可由公主的侍女将他引入林子。

铁摩勒听说公主有紧要之事,心头一震,他是奉命要听公主调度的,只得不避嫌疑,跟随那个宫女去见公主。

日间雨势已收,这时云开月现,下了将近十天的雨,今晚方始再现见光。铁摩勒踏进林子,月光下,只见公主衣裳淡雅,孤独一人,立在一棵老松树下,向他招手。那宫女早已悄悄地溜走了。

铁摩勒屈下半膝施礼禀道:``铁铮参见公主,不知公主何事见召?''长乐公主伸出纤纤玉手,说道:``你是我的救命恩人,不必拘礼。''便要扶他,铁摩勒着了慌,连忙站了起来,闪过一边,说道:``多谢公主厚待,但君臣之礼,不可废了。''

长乐公主秀眉微蹙,幽幽说道:``在这时候还说什么君臣之礼,你难道不可以将我当作朋友看待吗?我最不欢喜你在我面前拘拘束束的。''

铁摩勒只得与她并肩坐了下来,长乐公主道:``这些天来,你们是受尽了苦楚了。''铁摩勒道:``但得皇上和公主平安,我们受点苦算不了什么。''长乐公主叹了口气,说道:``都是我家害苦了你们,唉,在这种乱世,生在帝王之家,也真是不幸。铁铮,我倒是真羡慕你在江湖上的闯荡生涯呢!倘若我不是公主,我也想到四方走走,随你闯荡江湖,那有多自由自在呀。就不知我的本领可够得上在江湖闯荡吗?''

铁摩勒心中一跳,低头说道:``公主说笑了。''长乐公主正容说道:``我这才不是说笑呢,铁铮,你不懂我的心事的。''

铁摩勒定了定神,问道:``听说公主有什么紧要之事?\ldots\ldots{}''长乐公主打断他的话道:``你们受尽了苦楚,这还不是紧要之事吗?''铁摩勒不觉又是一怔,一时间未明其意。长乐公主叹道:``你忠心耿耿,受冷抵饥,毫无埋怨,士兵们可不见得都似你那样忍受得了吧?铁铮,我把你当作心腹之人,你也得把实情告诉于我。''

铁摩勒道:``士兵们遭受风吹雨打,且又衣食不全,少少的埋怨,那自是难免的。但他们也明白,这都是朝中出了奸臣的缘故。''铁摩勒讲得很谨慎,也没敢直指出杨国忠之名。

长乐公主叹道:``你不要瞒我了,何止少少的埋怨,那简直是怨气冲天,他们对杨国忠是恨不得食其肉而寝其皮。''

铁摩勒颇感惊奇:``公主,你已经知道了?''

长乐公主道:``今日河源军使王思礼从前方来,觐见父皇。父皇问他前方军情,他就先哭起来。他说自圣驾离京之后,士气更为不振。父皇问他:``是埋怨朕抛弃了他们吗?'王思礼说:``那倒不是。他们说,皇上以万乘之尊,离危城,幸西蜀,保国脉,图久安,那是应该的。只是有些深受皇恩的大臣,在这危难之际,却不敢挺身抗贼,只图保全一家富贵,甚至倚恃圣宠,还在作威作福,军士们却是心有不甘。只要皇上赏罚公平,有功者赏,有罪者罚,士气自能振作。'我父皇听了,当然知道他所指的是谁,黯然无话,过了好一会子,方始说道:``联知道了,卿家忠直,堪为栋梁。'即加封王思礼为河西陇有节度使,但对于他要赏罚公平的奏请,却不置一辞!''

铁摩勒道:``朝廷赏罚,我不敢妄参末议,但据我所知,即在羽林军中,也是人同此心,心同此理,都愿皇上大振乾纲,去奸佞而任贤臣。''

长乐公主道:``王思礼在我父皇跟前,还不敢说得很明白,后来他临行时,与护驾大将军陈元礼密议道:``杨国忠召乱起衅,罪大恶极,人人痛恨,除非即杀此贼,否则天下离心!'陈元礼道:``兹事体大,容我缓图。'陈元礼是碍着杨贵妃,投鼠忌器,不敢下手。他知道我得父皇宠爱,大约也还隐约知道我对杨家有点不满,暗地里来见我,将王思礼的话都告诉了我,叫我设法为国除奸。可是我又有什么办法?父皇宠爱我,更宠爱杨贵妃,我一在他跟前提起杨国忠,他就摇头叹气,不准我再说下去。如此犹疑不决,只怕大唐江山,就要断送在杨家手上。''

铁摩勒听得热血沸腾,冲口说道:``公主若有用到小人之处,小人万死不辞!''刚说到此处,忽听得那侍女在林里边一声咳嗽,公主翟然一惊,低声说道:``有人来了。你,你想个法子吧,但切不可轻举妄动。''公主扶着侍女,躲人林中,就在此时,便听得有人哈哈大笑。

铁摩勒一看,来的不是别人,正是宇文通。宇文通笑道:``铁都尉好闲情逸致,独自一人在这里赏月么?''铁摩勒道:``我是来巡查的。''宇文通道:``哦,你是来巡查的?可发现有什么可疑之人躲在林中么?我也似乎听得人声,咱们去仔细搜查一番吧!''铁摩勒忐忑不安,他问心无愧,但却怕公主受人闲话,连忙说道:``不劳宇文将军费心,我已搜查过了,并无可疑的物事。''宇文通哈哈大笑,忽地压低声音说道:``铁都尉,你是在等人吧?你真的没有发现什么?我倒见着一个影子,像是长乐公主的侍女。''铁摩勒知道他还未发现长乐公主,大着胆子道:``宇文将军体得取笑。怕是你眼花了吧?我怎么没有见着。''

铁摩勒生怕宇文通定要搜查,哪知宇文通忽地又是一阵哈哈大笑,说道:``铁都尉,既然你不是等人,那就随我去吧,有人在等着见你呢!''铁摩勒还以为他说的是公主,含嗔说道:``宇文将军,别尽管开玩笑啦,我,我\ldots\ldots{}''他想说的是:``我是奉命护卫公主,公主若要召我,自会遣内侍前来。''但他刚说得一句,宇文通便打断了他的话,正容说道:``谁和你开玩笑,相国命我请你!''

铁摩勒大吃一惊,讷讷说道:``什么?杨,杨相爷要等着见我?''宇文通大笑道:``你是受宠若惊了吧?哈哈,你这小子真好造化,快随我来!''一副亲热的神气,拉着了铁摩勒。

铁摩勒惊疑不定,蓦地把心一横,想道:``最多不过一死,我怕他杨国忠作甚?他要见我,我就正好相机把他杀了!''

杨国忠住在古庙的后座,另有门户出入,铁摩勒随着宇文通,从侧门进入,只见两廊之下,布满杨国忠的亲兵。杨国忠坐在堂上,宇文通便上前禀道:``铁都尉来了。''

杨国忠一脸奸笑,说道:``好,好,好!铁都尉,你是护驾有功之臣,我只因事忙,不然早就想见你了。兔礼,免礼,来,来,来,请到这边坐下。''

铁摩勒面对奸臣,不由得满腔怒火,便要下手除奸,忽地想起公主``不可轻举妄动''的吩咐,心道:``不错,天下人都痛恨杨国忠,但要平民愤,那最好是由皇上明正典刑,再不然也该由军士们光明正大地声讨他的罪状,将他处死,这才能消得众人的怨气。有宇文通在此,我未必便能把他杀了;即能把他杀了,民意无由上达,也还是便宜了他!''要知铁摩勒虽是热血汉子,却并非鲁莽之徒,他深思熟虑之后,便冷静下来,向杨国忠行了一个军礼,问道:``不知相爷见召,有何吩咐?''

杨国忠道:``我最赏识年轻有为之人,铁都尉,你武艺超群,又有保驾的大功,只要好自为之,定卜前途无限,目前这个职位,还是委屈了你啊!''

杨国忠皮笑肉不笑的双眼斜睨,见铁摩勒动也不动,毫无表示,不觉有点尴尬,宇文通的座位与铁摩勒相邻,连忙用肘碰了铁摩勒一下,说道:``铁都尉,相爷有意提拔你,你还不道谢?''

铁摩勒淡淡说道:``多谢相爷美意,铁铮来给皇上当差,保护圣驾,那是份所当为。蒙皇上额外加恩,封官赐爵,已是自觉非份了,哪里还能说得到委屈二字?''

杨国忠怔了一怔,随即哈哈笑道:``铁都尉,你不矜功,不夸劳,真是有古大将之风,老夫更敬重你了。但俗语说得好:人往高处走,水往低处流。你难道就当真不思上进了么?''

铁摩勒道:``无功不受禄。相爷虽是想抬举铁某,铁某和愧不敢当。''

杨国忠误解了铁摩勒之意,龇牙咧嘴地笑道:``铁都尉,只要你领会得老夫的一番好意,咱们就是一家人了,日子还长着呢,你何愁没有报答老夫的时日?''

说至此处,杨国忠忽地压低声音,问铁摩勒道:``听说军中对老夫颇有怨言,你有所闻么?''

铁摩勒这才恍然大悟,原来杨国忠叫他前来,乃是想笼络他的。与铁摩勒在一起的那班士兵痛骂杨国忠之事,想来杨国忠的侍卫也早已禀告他了。

铁摩勒佯作不知,反问道:``有这样的事情么?卑职倒未有知闻,不知他们怨些什么?''

杨国忠涨红了脸,铁摩勒推托不知,他却如何好把士兵们骂他的话转述出来?

但杨国忠毕竟是老好巨滑,想了一想,便又说道:``目下暂时受困,军士们有点牢骚,那也是难免的。老夫蒙受主恩,也难免有人妒忌。所虑者是奸人从中挑拨,煽惑军心,与老夫作对。铁都尉,你是个聪明的人,若有能为老夫尽力之处,老夫决不会忘了你的好处。''

铁摩勒道:``铁铮生性愚鲁,还是不明白相爷的意思。''杨国忠侧目斜睨,眼光从铁摩勒的身上移开,向宇文通睨了一下,宇文通连忙笑道:``铁都尉,你还当真不明白么?相爷是想要你作他的耳目,有什么人与相爷作对,你知道了就该立即禀报相爷。''

铁摩勒心头火起,想道:``原来杨国忠竟敢要我作他的走狗,哼,哼,他还未知道我是何等样人。''正要发作,却见一个校尉走上堂来。

杨国忠喝道:``我与铁都尉有要事相商,不见外客!不是早就吩咐过你们的吗?''那校尉屈膝禀道:``是李公公和回纥使者求见。''

原来这校尉所说的``李公公''即是东宫内侍李辅国,在太监之中,他的权力和地位仅次于高力士,极得玄宗之宠,所以加封他为``东宫内侍''。

杨国忠听说是李辅国亲自前来,而且还有回纥使者,不觉怔了一怔,怒气顿时平息,但仍然挥手说道:``你请李公公和两位使者暂在我的书房歇一会儿,说我就来。''

铁摩勒心里生疑:``哪里钻出来的回纥使者?这么夜深了还来求见杨国忠?''又想道:``仅这一座破庙,他们杨家倒占了半边,住不完的还拿来做什么书房,可怜许多将军们却要住在帐幕里,军士们更惨,露宿林中,还要遭受那雨打风吹之苦!''

杨国忠咳了一声,叫道:``铁都尉。''铁摩勒忍着怒气,应了一声:``在!''杨国忠打了一个哈哈,这才接下去说道:``刚才咱们说到哪儿?对啦,你提到无功不受禄的话儿。只要你为我尽力,那就是于我有功。我当然也会送你禄位。好,目前我就有一场天大的富贵要送给你,包你意想不到!''

铁摩勒半是愤怒,半是好奇,索性再逗杨国忠一逗,说道:``先谢相爷的栽培,却不知是什么富贵?''

杨国忠歪着眼睛看他,笑道:``长乐公主喜欢你,你知道吗?哈,老夫倒是知道了。只是,以你的身份,决不能当上驸马。不过,若有老夫替你们作主,托我家贵妃和皇上一说,皇上准可以破例成全你们,不问你的家世,将公主下嫁给你!哈哈,这可是你意想不到的,天大的富贵了吧。''

这是杨国忠一石二鸟之计,一来收服铁摩勒为己所用,二来拉拢长乐公主,免得她反对杨家。杨国忠以为铁摩勒听了,定必大喜过望,叩头道谢;哪知铁摩勒面色涨红,怒气勃发,立即便大声说道:``相爷,你看错人了,铁铮纵然想求富贵,也还不是这等无耻小人,藉裙带之亲,来博取功名利禄!''

这话分明是骂杨国忠靠杨贵妃而当宰相,杨国忠这一气非同小可,颤声骂道:``铁铮,你、你、你这样不受抬举!''眼看双方如箭在弦,一触即发。就在这时,忽听得两廊亲兵``哎哟哟''的叫声、跌撞声,有人大声喝道:``让开,我老黑来了,不用你们通报!''只见尉迟北提着金鞭,大踏步地走了进来,后面还有一个秦襄。

正是:富贵难移豪杰志,逢凶化吉救兵来。

欲知后事如何?请听下回分解------

\chapter{第二十九回 凄凉蜀道人少行
宛转蛾眉马前死}\label{ux7b2cux4e8cux5341ux4e5dux56de-ux51c4ux51c9ux8700ux9053ux4ebaux5c11ux884c-ux5b9bux8f6cux86feux7709ux9a6cux524dux6b7b}

杨国忠见是他们二人,不由得大吃一惊。要知杨国忠虽然是位居一人之下,万人之上,但秦襄、尉迟北二人乃是开国功臣的后代,尤其尉迟北持有太宗皇帝御赐的金鞭,且又脾气刚烈,素来不惧权贵,如今怒气冲冲的大踏步走来,杨国忠见了,怎么不心里发毛?

尉迟北一走进来,眼光一扫,便大声叫道:``哈,铁兄弟,你果然是在这儿!''他见铁摩勒安然无事,怒气减了几分,这才对杨国忠唱了一个肥诺,说道:``请恕鲁莽,未曾通禀。''

杨国忠打了一个哈哈,口不从心地说道:``得两位将军大驾同来,那是求也求不到的。下人无知,冒犯虎威,还望两位将军看在老夫面上,恕过他们。请坐,请坐,左右奉茶。''

尉迟北大笑道:``好说,好说。我老黑腹内空空,喝了你的好茶肚里更难受,这茶嘛不喝也罢。''杨国忠甚是尴尬,说道:``圣驾播迁,累两位将军受苦了。好在大雨已停,不日就可脱此苦境。''

尉迟北道:``我们受点苦倒没什么,相爷只要你没受苦就行了。''

杨国志满面通红,支吾说道:``逆贼作乱,道路难行,兵粮两缺,老夫与皇上也是甘苦共尝啊。不知两位将军前来,有何见教?''尉迟北心里骂道:````亏你厚脸皮,为何不敢说与士兵甘苦共尝?''他还想挖苦杨国忠几句,秦襄较为持重,用眼色将他止住。

秦襄道:``我正要请问相爷,不知你把铁都尉招来,可是有什么要事相商么?''杨国忠忙道:``没什么,没什么!只因他护驾有功,老夫未曾与他见过,故此请来一坐罢了。''他边说边瞅着铁摩勒,生怕铁摩勒说出些不中听的话来,当堂扫地的颜面。

好在铁摩勒没说什么,秦襄接着便道:``既是没有什么,我们倒有点事情要与铁都剧相商,请准告退!''

杨国忠心惊胆战,恨不得他们早走,当下敷衍了几句,便即送走他们。铁摩勒大步出门,冷笑一声,兀是一言不发,临行也不施礼,气得杨国忠在堂上发抖。

到了林中,铁摩勒吁了口气,方始问道:``你们怎知道我在杨国忠这儿?''尉迟北笑道:``长乐公主怕你有难,叫我们来给你保驾呀!''原来长乐公主躲在林子里,听到了字文通的说话,知道宇文通奉了杨国忠之命来``请''铁摩勒,心里大为着急,连忙遣内传唤他们二人前来,叫他们如此如此的。

尉迟北又笑道:``长乐公主生怕你给杨国忠所害,急得她坐立不宁。看来她对你倒颇有意思啊!''

铁摩勒面红耳赤,连忙说道:``尉迟大哥,这玩笑你可开不得啊!''

尉迟北大笑道:``有什么开不得,我可并没有把它当作玩笑哩!公主也是要嫁人的,她嫁给你又有什么不可以?喂,铁兄弟呀,若是第二位公主,我不敢劝你娶她,这位长乐公主,可是深明事理,文武全才的女中豪杰,你娶了她,不怕受什么皇家的腌赞气的!''

尉迟北是一片好心,铁摩勒可以对杨国忠大发脾气,对尉迟北却是不能,当下只有如实告诉他道:``大哥有所不知,小弟已是订有妻室的了。''

尉迟北甚是尴尬,忸怩笑道:``又是我老黑莽撞了,不知个罪,铁兄弟,请恕老黑失言。''秦襄问道:``铁兄弟订的是谁家!''

娘?''铁摩勒道:``就是韩老前辈韩湛的女儿。''秦襄与尉迟北一齐哈哈大笑,说道:``'原来都是熟人,这位姑娘又比公主强得多了。''

尉迟北转过话题,问铁摩勒道:``我不信杨国忠那样好心,没甚来由就请你去坐。到底是为了何事?''

铁摩勒恨恨说道:``他要我作他的爪牙。''当下将与杨国忠见面的经过说了一遍,只略去杨国忠要给他做媒的一段不提。

秦襄叹道:``杨国忠倒行逆施,天怒人怨,他尚自不知悔过,将来不知要闹出何等事情,怕只怕大唐的江山也要断送在他的手。''

铁摩勒问道:``刚才有两个回纥使者来求见杨国忠,秦大哥可知道这桩事情?''

秦襄道:``略有所闻。说起来这两个回纥使者倒不是杨国忠请来的。''原来玄宗为了贼势披猖,江山紧要,因此想借外兵平乱,这两个回纥使者便是来与玄宗商量出兵之事的。

回纥所提的出兵条件甚苛,经他收复的土地,女子玉帛要尽归于他,玄宗与陈元礼、韦见素、魏方进等几位随从文武大臣商量之后,都不敢答应,只有杨国忠力排众议,他的理由是``不要因小失大'',让回纥掳去一些女子,掠去一些财货,可以保全大唐的江山,那还是``划算''的。当时,也有一些人望风转舵,附和杨国忠的,两方争论不休,议而未决。

秦襄道:``看来是回纥使者已经打听到了这种内情,所以来走杨国忠的门路,请他们兄妹再向皇上进言,务求遂其所愿。哼哼,杨国忠大约又可以收到许多珍贵的礼物了。''

铁摩勒大怒道:``杨国忠不要老百姓,老百姓也不要他!''

秦襄忙道:``铁兄弟噤声,一切有皇上作主,咱们不可随便议论,这话若是给别人听见,只怕你要落个谋反的罪名!''

尉迟北怒道:``秦大哥,你也忒怕事了,难道咱们就任由那杨国忠胡作非为?''

秦襄苦笑道:``莫不成你还能够当真的把杨国忠打杀了么?你的金鞭吓吓他还可以,若真的打了他,只怕皇上也决不会顾念你先祖的功劳了。何况咱们身为龙骑都尉,职司仅是保护圣驾,朝廷大事,却是不能容咱们来管的。''

尉迟北恨恨说道:``'杨国忠若是有事撞在我的手上,我就拼了这条性命,偏要管他一管。''

秦襄道:``好啦,好啦,别要尽说这些愤激的说话了,还是早点去睡吧。''尉迟北发了一通脾气,也只好散了。

这一晚,铁摩勒心事如麻,却是睡不着觉。心里想道:``皇帝老子与杨国忠乃是一家人,那是决计不会将他问罪的。朝中大臣,人人都惧怕杨家的权势,连秦大哥也不敢得罪他,也就可以想见了。嗯,难道就当真没有法子除掉杨国忠。''

还有一桩心事,令得铁摩勒烦恼的,那就是长乐公主对他的日益亲近,铁摩勒本来是连想都没有想过长乐公主会对他钟情的,可是从今晚公主和他在林中的谈话,以至扬国忠的要为他做媒,以至尉迟北和他的那番说话,这就不由得铁摩勒不要好好的想一想了。``连尉迟大哥都看出来了,敢情她对我当真是有几分意思?嗯,一个王燕羽,已经是够我烦恼的了,若再招惹上公主,教我怎生摆脱得开?''

这晚铁摩勒睡得不好,第二日还是有点神思昏昏。将近中午时分,铁库勒正在帐幕里等待护军给他送饭,忽听得外面一片喧哗,铁摩勒出去一看,只见有一堆土兵围着几个人,看清楚了,却原来被围的是杨国忠的厨子。

那几个厨子抬着一只烤猪,还有其他香喷喷的菜式,士兵们正要抢那只烧猪。

那几个厨于看见铁摩勒走来,而铁摩勒穿的是军官服饰,以为得到了救兵,连忙嚷道:``大人快来救命!''哪料铁摩勒走过去道:``你把这只烤猪放下来不就完了,我敢保他们不会杀你!''

士兵们欢呼道:``对呀!我们只要这只烤猪,还不想吃你的肉呢!杨国忠少吃一顿有什么打紧,我们已是吃到草根树皮了!''正在闹得不可开交,忽地有另外一队武士冲过来,拿着皮鞭噼噼啪啪的乱打,骂道:``你们饿得发昏了,连相爷宴客的东西都敢抢!''乱鞭打下,连铁摩勒也挨了一鞭!

铁摩勒大怒,劈手夺过一个武士的皮鞭,骂道:``你们啃杨国忠吃剩的骨头,吃得脑满肠肥,就不顾士兵们的死活了么?''唰、唰、唰连环抽扫,登时把近身的几个武士打得滚地狂呼。

事情一哄起来,立即有如火山爆发,不可收拾,起初只是一小队士兵,转瞬之间,便似滚雪球一般,越滚越大,各营士兵,都骚动起来,奔跑呼叫喝骂之声,有如山崩海啸,军官们哪里还控制得住?连羽林军也卷人了漩涡,争着动手打场国忠的亲兵!

人丛中不知是谁在大叫道:``找杨国忠算帐去!''``问问他是不是要饿死咱们!''``他们杨家享尽了福,却把国家弄得这般田地,杨国忠你还好意思厚着脸皮做宰相吗?''骂声一起,四万响应,军士们拥着铁摩勒做带头人,人潮似一个个浪头,涌向杨国忠的临时住宅。杨国忠的亲兵早已抱头鼠窜,哪敢迎敌。

杨国忠昨晚留那两个回纥使者谈了一夜,这时刚刚起床,正拟大排筵席,宴请贵宾,听得鼓噪之声,心慌意乱。他的亲信卫兵进报道:``不好了,士兵哗变,由那新来的铁都尉领头,就要打进来了。请太师快去弹压!''

杨国忠定了定神,问道:``就只是那姓铁的小子吗?还有没有别位大人,陈将军呢?''亲兵遣:``陈将军不见踪迹,其他的军官也没露面。''杨国忠所问的``陈将军''即是护驾的龙虎大将军、三军统帅陈元礼,陈元礼向来与杨国忠面和心不和,故此杨国忠初时还以为是陈元礼唆使军兵叛变与他作对,如今一听,军官们除了铁摩勒外,都未参加,胆子便大了一些,一想事到此时,也只能亲自出去弹压了。于是他便在几个得力的卫兵保护下,出来与士兵们见面,同时叫那两个回纥使者,悄悄从后门溜走。

杨国忠大喝道:``铁铮,你多大的官儿,胆敢犯上作乱?''``嘿嘿,你们知不知道谋反的罪名?那是要五马分尸,九族抄轨的!姑念你们愚妄无知,受人煽惑,现在本相国法外开恩,只拿铁铮一人问罪,你们都散了吧!''

杨国忠恃着宰相的威严,把这顶``造反''的大帽子一压下来,果然有许多士兵被他吓住,便像暴风雨的前夕,暂时间静止下来,但更多的士兵印激起了更大的愤怒,酝酿着更大的风暴!

杨国忠正要指挥卫兵捉拿铁摩勒,忽听得洪钟般的一声大喝,龙骑都尉尉迟北闯了进来,大骂道:``杨国忠,你私通番使,才是谋反,却敢诬赖别人!''

铁摩勒心念一动,想道:``你说我反,我就反了吧、今日是决不能容你活了!''他抓紧机会,立即接着喊道:``你们瞧,那两个人就是回纥的使者,刚从这里出去的!''那两个回纥使者吓得没命飞奔,刚好庙后有几匹御马,这两个使者是回纥国中的著名武士,急急忙忙三拳两脚打倒了马夫,夺了马匹,从``行宫''禁地,穿过庙后那一片树林逃走了。

军士众目所视,众手所指都是向着杨国忠一人,在尉迟北揭发这件事情之前,谁也没有注意那两个回纥使者,他们逃得又快,众人也无暇去追捕他们了。但是时间虽然短促,军士们也已看清楚了那两个``番人''。有人便振臂大呼道:``杨国忠私通番使谋反,我等何不击杀反贼!''

杨国忠魂飞魄散,虽然他也提高了声音喊道:``这两个回纥使者是皇上请来的,与我无关!尉迟将军、铁都尉,你们不可诬赖好人!''但这时已是三军鼓噪,杨国忠的说话被巨雷般的呼喝声盖住,但见他的嘴唇开阖、谁也听不出他说些什么。

其实即使军士们听得清楚他的说话,亦已无济于事。要知人人对他都是久怀积愤,恨不得食其肉而寝其皮,``私通番使'',不过是杀他的一个借口而已。这时,好不容易的闹起事来,哪还有谁肯听他分辨?

有两个卫士尚不知死活,还想保护杨国忠逃走,被铁摩勒两剑劈翻,军士们蜂拥而前,兵刃乱下,登时把杨国忠砍成一团肉酱。尉迟北本来还只是想威胁杨国忠释放铁摩勒的,哪知事情的演变大出他的意外,饶是他胆气粗豪,也吓得呆了。

军士们的积债一旦债发出来,当真有如怒火融融,谁也休想压制得住。这局而不但出乎尉迟北的意外,甚至连铁摩勒也是始料不及。军士们杀了杨国念之后,转眼间又把他的儿子户部侍郎杨暄杀了,兀自不肯罢休,人人都像发了狂的大叫大嚷,要杀尽杨氏一家,连杨贵妃在内!

杨贵妃的两个姐妹韩国夫人和虢国夫人听得风声,慌忙乘车逃走,这时漫山遍野,都是乱军,哪里还逃得掉?众军士一起追去,先把韩国夫人斫死,跟着又去杀那虢国夫人。

虢国夫人死中求活,军士刚阻住她的车驾,她忽地揭开车帘,向军士们衷声求告:``你们已把我的哥哥杀了,我是女流之辈,我哥哥做的事与我无关,请你们高抬贵手,饶了我们母子俩吧!''一面哀告,一面把大把的金珠撒了下去。

虢国夫人天姿国色,比乃姐杨贵妃还胜三分,当时名诗人张祜曾有诗云:``虢国夫人承主恩,平明骑马人宫门,却嫌脂粉污颜色,淡扫蛾眉朝至尊。''这诗一面写虢国夫人是如何的得皇帝恩宠,可以平明时分骑马进人宫门;一面极力刻画她的美貌------无需靠脂粉来打扮,怕脂粉反而污损她的姿容,只是淡扫蛾眉,便足以倾国倾城了。

围着虢国夫人车驾的那些军士,对她撒下的金珠例并不放在眼内,但突然见她露出面来,却都禁不住呆了一呆,何况她又哀哀求合,像是一枝带雨的梨花,更为凄楚动人。那些军士,手中都拿着明晃晃的兵刃,却不知怎的,都不忍斩将下去,给虢国夫人驾车的家丁,连忙挥动马鞭,赶着马车逃出包围。

不过,虢国夫人也只是暂时幸免于难,她逃出马嵬驿之后。

找不到食物,饿了几天,形容憔悴,终于在逃到陈仓县的时候,仍然被县令薛景仙率吏民追捕着,将她杀了。这是题外之话,不必细表。

且说这时乱军四起,已如野火燎原,群情汹涌,难以阻歇,后面的军士见前面的军士放走了虢国夫人,都在大骂,又有人叫道:``斩革除根,这小狐狸也还罢了,杨贵妃这骚狐却是非杀不可!''此言一出,群相附和,喊声震天,此时示已无须再有人率领。

军士们已把那座暂作``行宫''的古庙重重围着,大叫大嚷,要玄宗皇帝即刻杀杨贵妃。

玄宗听得兵变,哪敢出来?忙叫龙虎大将军陈元礼出去,用好言安慰众军,令各收队。陈元礼出去道:``你等已把杨国忠杀了,为何还聚而不散,有惊圣驾?''也不知是谁作出了四句歌辞,在乱军中传开,众军士一齐唱道:``反贼虽杀,贼根犹存,不除贼根,何得安心?''陈元礼只得回去,据实奏道:``众人之意,以国忠既诛,贵妃不宜复侍至尊,伏候圣断!''

玄宗大惊失色,涕泣言道:``妃子深居官中,国忠即谋反,与她何干?朕如今已是颠沛流离,只有妃子一人在我身边,也只有她一人能解朕意,你叫朕如何舍得她去?''

陈元礼一时不敢答话。却睁起眼睛,向玄宗身边的高力土扫了一眼。这高力士是最得宠的太监,平时对杨贵妃奉承得无微不至,这时听得军士们的喧闹喊杀之声,生怕军士们把他当作贵妃一党,也要把他杀了,这时见陈元礼以目示意,心头一震,只得跪下去奏道:``贵妃诚无罪,但众将士已杀国忠,而贵妃犹在皇上左右,岂能自安?愿皇上深思之,将士安则圣躬方万安。''京兆司录韦愕也跪奏道:``众怒难犯,安危在顷刻间,皇上不舍贵妃,只恐将士要舍皇上,愿陛下割恩忍忧,以宁国家。''玄宗默然点头,尚未言语,已听得珠帘后面杨贵妃的哭声。

只听得杨贵妃哭道:``你们的话我都听见了,愿陛下保重,毋以贱妾为念。''玄宗神色惨然,挥了挥手,陈元礼诸人都不敢再说一句,悄悄的一个个溜出去。

玄宗见了贵妃,一句话也说不出口,杨贵妃还存着万一之想,呜咽说道:``三郎(玄宗排行第三),你还记得那年七月七日,夜半无人,咱们在长生殿所说的话吗?''玄宗道:``在天愿作比翼鸟,在地愿为连理枝。妃子,朕是但愿生生世世都和你作夫妇的啊,唉------''门外军士喧哗之声更甚,玄宗面色如死,眼泪已流不出来,``唉''了一声之后,再也说不下去了。杨贵妃知道已经绝望,涕泣言道:``为了陛下的江山,臣妾情愿任由陛下处置,只求乞个全尸!''玄宗也哭道:``愿仗佛力,使妃子善地受生。''回头叫道:``高力士,来!''取过一匹白绫,掷给高力土道:``你带贵妃至佛堂后面,代朕送贵妃上升仙界。''佛堂后面有一棵树,高力士奉上白绫,杨贵妃便自缢在这棵树下,死时年三十有八。后来诗人白居易有一首《长恨歌》,写杨贵妃与玄宗之事,其中一段云:``九重城阈烟尘生,千乘万骑西南行。翠华摇摇行复止,西出都门百余里。六军不发无奈何,宛转蛾眉马前死。花钿委地无人收,翠翘金雀玉搔头。君王掩面救不得,回看血泪相和流。''所咏的便是马嵬驿当日之事。

玄宗在佛堂侧边的廊下独自徘徊,众人尽都回避了,他不敢去看杨贵妃临死的情形,但又不忍离开。不久,只听得树叶籁籁的摇落声,想是为了杨贵妃临终的挣扎;不久,又听得叮的一声,想是杨贵妃头上的玉簪已掉了下来。玄宗掩面长叹,但哀痛之中,却又忽地似有轻松之感。门外的乱军大约已经知道了消息,喧哗之声已渐渐减弱了。不错,他最心爱的妃子是死了,但他本身所遭受的威协也消灭了。

玄宗但感一片茫然,也不知是悲是喜,忽地有一个人影从黑暗的角落里出来,卜通跪倒,低声说道:``陛下节哀,奴才有事禀奏\ldots\ldots{}''玄宗怒道:``滚开,任是什么事情朕也不理了。''他只道是那个太监,一看却原来是个戎装佩剑的军官。

玄宗大吃一惊,道:``你,你来这里作什么?''这时他才看清楚了是字文通,只道宇文通亦已参加了兵变,又复问道:``朕已把贵妃处死了,难道军士们还不肯饶过朕么?''宇文通道:``陛下可想为贵妃报仇么?''玄宗连连摇手,继而一想,宇文通若是意图犯上作乱,不会仍执君臣之礼,于是便又把他叫了起来,低声说道:``你有何言,小声讲吧!''

宇文通道:``这次兵变实是受人煽动的,相国贵妃本不至于死,都是此人\ldots\ldots{}''玄宗问道:``此人是谁?''字文通正要说出``此人''的名字,忽听得履声``浙浙'',龙虎将军陈元礼与长乐公主走了进来。长乐公主是来安慰父亲,陈元礼则是来请旨安抚将士的。宇文通见了公主,心头一凛,连忙把话打住,却向陈元礼解释道:``我怕有乱军闯进,故而来此保驾。''其实陈元礼并没问他,他这一解释便显得多余,反而引起了公主的疑心了。

陈元礼道:``将士们都是忠心室上的,皇上可以无忧。请皇上下安抚诏,让他们也得安心。''玄宗便即下旨,命陈元礼去晓喻众军,说是杨国忠罪有应得,皇上对此次事情只有嘉奖,决不追究,妃子杨氏,亦已军旨赐死,叫将士们各自安心散去。

御旨传出,众军还未肯信杨贵妃已死,玄宗又命高力上将杨贵妃的尸体,用绣袋覆于榻上,抬出去给军士们看,军士们这才三呼``万岁'',各自散开。

玄宗又命高力士速具棺殓,将杨贵妃草草葬于马嵬坡上。

就在此时,有两骑马自西奔来,军士们截住一问,却原来是广元太守差人来进贡荔枝的。

原来杨贵妃最喜欢吃荔枝,她是蜀州人氏;蜀中也产荔枝,不过不及岭南的甘美,所以后来她做了贵妃,``三千宠爱在一身''

之后,便不再吃蜀中的荔枝,而要岭南刺史给他设置专使,进贡岭南的荔枝。当时名诗人杜牧有诗句云:``一骑红尘妃子笑,无人知是荔枝来。''说的便是这件事。

广元太守早已接到驿书,知道玄宗与杨贵妃``驾幸''西蜀,心中想道:``贵妃在这仓皇逃难之时,岭南的荔枝是吃不到了,我让她吃到家乡之物,也好讨她欢喜。''却不料荔枝送到,正是杨贵妃下葬之时。军士们搜刮荔枝,哈哈大笑,顷刻之间,两大箩荔枝都给军士们吃得一颗不留。后来诗人张佑有诗云:``旌旗不整奈君何?南去人稀北去多。尘土已残香粉艳,荔枝犹到马嵬坡。''

诗人的吟咏不必尽述。且说玄宗见乱事已弭,洪水亦退,道路复通,虽然悲痛,亦有``不幸中之幸''之感,当下便令陈元礼约饬众军启行。哪知大乱虽然平息,却还有一点不大不小的风波,因为杨国忠原是蜀人,他的部下将吏,多在蜀中,有一部分军士便不肯西行,或请往河陇,或请往太原,或请复还京师,议论纷纷,莫衷一是。

这时道路已经复通,扶风郡守吕甫和一些地方父老也赶到了马嵬驿见驾,遮道挽留;这吕甫倒是个有胆识的官儿,攀着皇帝的车驾侃侃奏道:``至尊与太子俱往蜀中,中原百姓谁为之主?我等愿率子弟拱卫至尊,东向破贼,还保长安。''

玄宗经过了这场兵变,惊魂未定,而且安禄山的前锋已直追长安,他哪里还敢回去。心中想道:``蜀中号称天府之国,即使是偏安之局,也要比在其他地方的好,最少可以多享几年福。''但这时众议纷坛,他乃惊弓之鸟,又不敢过拂众人之意,是以只顾低眉沉吟,不即明言所向。

太子李亨是个野心勃勃的人,正想趁此机会收揽大权,好巩固他未来的皇位,当下便即奏道:``逆贼犯阙,四海分崩,不得民心,何以兴复?今父皇人蜀,倘贼兵烧绝栈道,则中原土地,拱手授贼,民心既离,岂能复合?然父皇以万乘之尊,又不能固守危城,冒不测之险;为今之计,不如由臣儿收集西北守边之兵,召郭子仪、李光弼于河北,与之并力东讨逆贼,克复二京,削平四海,然后扫除宫禁,以迎至尊。''

玄宗得太子挺身而出,愿肩重任,正合心意,立即如拟,便封太子李亨为天下兵马大元帅,郭子仪为副元帅,命他们同心讨贼。后来李亨不待父亲``驾崩'',便在灵武即天子位,是为肃宗。

这是后话,按下不表。

且说在这场大风暴之后,铁摩勒本想弃职潜逃,后来见玄宗的安抚诏书已经颁下,心中想道:``'皇帝老儿总不能失信于天下,诏书讲得明明白白,对此次事情,决不追究,而且杨贵妃亦已奉旨赐死了,我还何须恐惧。大丈夫来去当光明磊落,做事当有始有终,我既答应了师兄愿做皇帝老儿的保镖,若还中途逃走,成什么话,没说的,只好送佛送到西天吧。''

车驾启行之前,字文通忽来说道:``铁都尉,皇上命你率领数十散骑断后,保护辎重。长乐公主的车驾,不必你作扈从了。''铁摩勒正怕与长乐公主太过亲近,欣然奉旨,不疑有他。

大队人马继续西进,蜀道难行,军士马匹累坏的日有所闻,幸而粮草已有接济,军士们所愤恨的杨国忠又已杀掉,因此虽然劳苦,士气却比以前旺盛得多,全军上下,无一怨言。

一路无事,话体烦絮。这日到了广元,已人蜀境。玄宗念将上劳累,准许歇息三天。这晚,铁摩勒便与秦襄尉迟北二人喝酒畅叙,酒正酣时,忽地有一个太监匆匆来到。

尉迟北吃惊问道:``公公,何事?''那太监道:``皇上有召,命铁都尉即行见驾。''尉迟北道:``哦,原来是宣召他么?铁兄弟,反正我也没事,陪你走一遭吧。''尉迟北掌管大内宿卫,不必奉诏,亦可进宫,这时虽是在走难途中,旧规仍在,故此他敢出此言。

哪知那太监却道:``皇上只是宣召铁都尉一人,`行所'(即皇帝驻骅之所)宿卫,都已有人轮值了,尉迟将军,你自饮酒。''

尉迟北虽可自行进宫,但未奉诏却不能进去见皇帝,而且那太监的口气,又分明是不想尉迟北同行,尉迟北只好作罢,当下笑道:``既是行所无事,我也就乐得清闲了。铁兄弟,待你回来,'咱们再喝个痛快。''皇帝宣召侍卫,那也是常有之事,尉迟北不疑有他。

铁摩勒却暗暗起了疑心,``马嵬驿之变,是我首先发难的,虽然皇上有诏,对任何人都不追究,但看他在这次事变之后,即不要我作公主的扈从,分明是对我已有疑心,不似从前信任了。为何他又要单独召我进宫?哎呀,难道这是公主的主意?''

广元城是远离战火的后方,广元太守给皇帝布置的``行所'',堂皇富丽,颇有宫殿规模,远非那座破庙可比。铁摩勒随着那太监进了行所,经过一条长廊,那太监按照宫中规矩,走在前头,高声报道:``铁都尉奉召来到!''

就在此时,忽见有一个神色张皇的宫女,倚着栏杆,突然把手一场,将一团东西向铁摩勒抛过来,也幸亏铁摩勒正好与她打个照面,认得她是长乐公主的侍女,急忙将那东西接住,却是一个纸团。

铁摩勒吃了一惊,悄悄把纸团打开,刚看得清楚纸上那两个大宇,便听得站班的黄门内待一叠声的传呼道:``宣铁都尉觐见。''那太监回过头来,说道:``铁都尉你可以进去了。''这时那宫女早已闪人角门,铁摩勒定了定神,咬咬牙根,装作毫无事情发生的样子,便随着引见的黄门官,穿出回廊,走进厅堂。

只见屋子里除了玄宗之外,只有字文通一人。铁摩勒谨依君臣之礼,三呼万岁。

玄宗和颜悦色地说道:``爱卿平身。赐坐。''铁摩勒忐忑不安,谢过座位。玄宗问道:``听说日前马嵬驿之变,是你领头的,是么?''

铁摩勒心道:``来了,来了!''但他早有主意,却也不惧,便即回道:``皇上明鉴,当时群情愤激,微臣受众军推拥,实难置身事外。''玄宗道:``你的胆子倒真不小啊!''铁摩勒不卑不亢,答道:``微臣只思为皇上除奸去佞,祸福利害,从未顾及。皇上若认为不当,微臣首受刑罚,万死不辞!''

玄宗摇了摇头,说道:``爱卿误解寡人之意了。像你这样有胆识,有血性而又忠心耿耿的人,朕正是求之不得,安忍处罚?联在安抚诏中亦曾说得明白,对此次为朕除奸之人,只有嘉奖,决不追究。朕今日召你进来,就是要封赏你啊!铁铮听封!''

铁摩勒心道:``这皇帝老儿到底弄甚玄虚?''只得再跪下去,听他封赏。

玄宗说道:``朕封你为龙骑都尉,世袭罔替。另赏宫花一朵,御酒三杯。''

按当时朝廷的规例,只有中了状元的人,才可以得到皇帝赏花赐酒,所以这是莫大的荣誉。铁摩勒大觉意外,接过官花,插在襟上,再接过皇帝亲手递来的酒杯。

这刹那间,铁摩勒墓然想起了纸团内的两个大字,那两个字是:``速走!''不禁心中想道:``长乐公主向我示警,决非无因。要我速走,定是她已知道皇上有意加害于我,但现在皇上反而对我封赏,\ldots\ldots 嗯,难道这杯酒里有古怪?''

铁摩勒心念一动,不忙喝酒,先把酒在鼻端嗅了一嗅,忽地将那酒杯一摔,只听得``当啷''一声,酒杯粉碎,地上溅起了点点火星!

这是一杯毒酒!

这刹那间,铁摩勒当真是气愤填胸,又惊又怒,他做梦也想不到皇帝会用这样卑污的手段对付他,他给皇帝做保镖,也曾救过皇帝的性命,现在皇帝却要用毒酒杀他!

说时迟,那时快,只听得玄宗喝道:``铁铮目无君上,着即赐死!''宇文通已是扑了过来并指如戟,倏的就点铁摩勒胁下死穴!

铁摩勒反手一掌,正是拼着两败俱伤的打法,字文通领教过他的掌力,不敢硬拼,迅即移形换位,再点他背后的风府穴。

铁摩勒呼呼两掌,将宇文通迫退三步,大声说道:``皇帝老儿,你若说得出个道理,光明正大的将我处死,我甘受无辞!你不该言而无信,残害忠良。请恕我不能再做你的奴才了。''倏的拔出佩剑,便冲出去。

玄宗吓得直打哆嗦,待见他不是向自己杀来,这才惊魂稍定,要替杨贵妃报仇之念,又油然而生,立即喝道:``主要臣死,不得不死;父要子亡,不得不亡!你目无君上,便该处死!还要什么罪名?众侍卫,将他拿下,碎尸万段!''

宇文通不待玄宗发话,早已拔出判官笔追去,门外的侍卫也纷纷吆喝,作势拦截。

铁摩勒大喝道:``挡我者死,避我者生!''抡剑狂挥,泼风也似的真杀出去。宫中轮值的宿卫乃是尉迟北的手下,一来知道铁摩勒与他们的长官甚有交情;二来识得铁摩勒的厉害;三来,最主要的是他们也替铁摩勒抱不平,所以只是虚张声势,一触即退,待铁摩勒一个冲过去,却又立即兜截过来,反而在有意无意之间,作了字文通的障碍。

铁摩勒冲出``行所'',夺了一匹御马,快马加鞭,便向城外驰去。守城门的卫士是秦裹的部下,认得他是何人,不过也免不了要问他几句,铁摩勒伪称是奉旨出城,那个卫士便即打开城门。

就在此时,只听得字文通大叫道:``不可开门,这厮已经反了!''原来他也骑了一匹御马追来。本来是距离甚远的,只因铁摩勒在叫开城门之时,稍受阻延,如今两匹马的距离已不到百步。

那卫士``啊呀''一声,吓得定了眼睛发呆,说时迟,那时快,铁摩勒已放马直冲过去。那个卫士这才傻头傻脑地去关城门,字文通大怒道:``你疯了么?反贼已经跑了,还关城门?''快马冲到,一脚将他踢翻,衔尾疾追!

两匹马的脚力差不多,风驰电逐,转瞬间到了郊外,宇文通用判官笔的笔尖向马臀一戳,马儿负病狂奔,双方的距离拉近了几十步。

忽听得弓弦声响,字文通手挽强弓,连珠箭发,射铁摩勒的坐骑,铁摩勒挥剑拨打,但宇文通箭如雨下,铁摩勒既要保护自己,又要保护坐骑,便显得手忙脚乱,势难兼顾。

铁摩勒怒道:``来而不往非礼也!''也在暗器囊中掏了一把铁莲子撒过去,可是铁莲子的份量甚轻,不能及远,威力比起弓箭,那自是有天渊之别。虽然有几颗莲子打中了宇文通的坐骑,却未能造成伤害。

飞骑追逐,暗器交锋;宇文通追得近了,力挽强弓,嗖的一箭,洞穿马腹,铁摩勒一个筋斗,在马背上倒翻下来。宇文通哈哈大笑,叫道:``铁摩勒,你还往哪里跑?你这小贼,竟敢混入宫中,也算得是胆大包大了!哈哈,十年前给你侥幸逃脱,想不到天网恢恢,你还是撞在我的手上!''

宇文通一口喝破铁摩勒的来历,若在平时,铁摩勒定必吃惊,但在此时,他已成为皇帝所要追捕的``反贼''了,哪还有什么顾忌,立即大怒应道:``不错,我就是铁摩勒,你待怎么样?你当我怕你么?''

宇文通喝道:``好呀,你这反贼还敢抗旨拒捕么?今天可没有什么段大侠、南大侠来保护你了。''

铁摩勒听他提起旧事,怒从心起,冷笑说道:``我是反贼,你是忠臣不成?哼,哼,你当我不知你的底细么?想当年你助纣为虐,以堂堂的龙骑都尉身份,竟不惜充当安禄山的鹰犬,害了史义士一家,又想害段大侠,亏你还有胆量敢说我是反贼!''

宇文通面色陡变,大笑道:``这反贼二字是皇上封给你的,今生你也休想洗得脱了!你居然还要含血喷人,你以为皇上还会相信你的话么?''

宇文通正是为了害怕铁摩勒揭破他与安禄山勾结的底细,这才处心积虑,怂恿皇帝除掉铁摩勒的。这时他心里想道:``幸亏他这番话刚才在皇上跟前没有说出,要不然,皇上纵不相信,心中也会有个疙瘩。他如今已负上了个反贼的罪名,谅是秦襄与尉迟北也不敢维护他了,我得赶快把他杀掉灭口。''

字文通素来自负,他虽然领教过铁摩勒的掌力,但自忖在兵器上能够胜得了他。心想:``皇上必然派人随后追来,这小贼今天是必死无疑的了。但最好还是在那些人来到之前我便把他杀掉,免得他胡说八道。''

两人心中都是充满了旧仇新恨,登时在树林里交起手来。

字文通与秦襄、尉迟北二人齐名并列,号称大内三大高手,武功上确有过人的造诣,两枝判官笔展开,端的有如毒蛇吐信,笔笔指向铁摩勒的要害穴道。

铁摩勒展开了六十四手龙形剑法,剑气纵横,剑光飞舞,也端的有如玉龙夭矫,变化莫测。宇文通胜在火候较纯,经验老到;铁摩勒则胜在内力悠长,血气方刚,两人各展平生所学,打得个难解难分!

宇文通想不到十年前几乎丧命在他手下的这个毛头小子,如今竟是大非昔比,越战越勇,斗了一百来招,自己还未能占得丝毫便宜,心中不禁暗暗发毛。

忽听得马铃声响,转瞬间那匹骏马已是飞驰来到,铁摩勒失声呼道:``秦大哥,你也来要小弟的头颅么?''

原来铁摩勒``反''出行所之后,玄宗立即传令秦襄与尉迟北二人,协助字文通追捕,二人接了圣旨,大大吃惊,尚未知铁摩勒已被定了死罪,君命不可违抗,两人只好遵旨,秦襄马快,先行赶到。

字文通厉声喝道:``你是反贼,还敢与秦将军称兄道弟么?秦将军认得你,他的金锏可认不得你!''这几句话厉害之极,实乃要迫秦襄动手。

秦襄又惊又急,左右为难,若无旁人,他还可以殉情私放;(他飞骑赶来,就是打算如此的。)但现在却有个宇文通在场,那是决计不行的了。

秦襄踌躇片刻,迫得说道:``铁铮,我尚未知你犯了何罪,但既有圣旨拿你,你就不应拒捕,免得罪上加罪!你有何冤屈,见了皇上,可以再行分辨。''秦襄打算与尉迟北联同用阖家性命来保他,必要之时,还可以恳请长乐公主代为求情,因此先叫他不可抗旨拒捕。

铁摩勒悲愤交集,说道:``皇上要杀我替杨国忠、杨贵妃填命,这还有什么可分辨的?秦大哥,我知道你是奉旨拿我,我不愿令你为难,好,我就随你回去,任那昏君处置。''

铁摩勒已愿意束手受擒,可是字文通的双笔却如狂风暴雨般的袭来,莫说放下兵器,只要应招稍缓,就有性命之危!

铁摩勒大怒道:``我可以卖情面给秦大哥,却不能受你这厮欺负!''唰唰唰连劈三剑,斗得更烈!

秦襄叫道:``铁铮既愿奉旨,字文将军,你就住手吧!''宇文通道:``他口说如此,剑未扔下,即如老虎未曾拔牙,你焉知他不会反啮?''

字文通的话也并非没有道理,秦襄又想劝铁摩勒先放兵器。

但看这情形,铁摩勒与宇文通彼此互不信任,除非自己上去挥锏把铁摩勒的长剑打落,否则铁摩勒也断不敢放下兵器。

铁摩勒与宇文通本是难分上下,但秦襄一来,铁摩勒已有点心烦意乱,长剑狂挥,招数上不觉露出破绽,字文通陡地大喝一声:``着!''一笔向铁摩勒胸前的``璇玑穴''插下!

秦襄大惊,正待上前解救,忽听得``叮''的一声,宇文通的判官笔歪过一边,随即听得一个带着稚气的声音说道:``秦将军,他们打得好好的,你却从中于阻,这未免大煞风景了!''

树林中突然现出一个人来,秦襄这一惊更甚,这人身材不满五尺湘貌十分特别,一副``孩儿脸'',活像一个大头娃娃,正是那名满江湖、曾经震惊帝座的妙手神偷空空儿!

秦襄手按双锏,沉声问道:``空空儿,你到这里,意欲何为?''

空空儿笑道:``秦将军,你不必担心,你这对金锏,虽然也值得几个钱,却还未放在我的眼内,我贼瘾发作,也不会偷你的。

我是特来看打架的呀!喂,你问了我,我也要问你了,你又来这里做甚么?''

秦襄道:``我,我是奉旨来,来捉\ldots\ldots{}''他看了铁摩勒一眼,那``反贼''二字,实是不忍出口。空空儿道:``你要来捉谁呀?捉这个大个子呢,还是捉这个少年?''

秦襄道:``我们的事,你何必管?''

空空儿道:``不然。我已经说与你知,我是喜欢看打架的了。

他们打得过瘾,我也看得过瘾。他们打架,你若不管,我也不管;你若要帮那一边,我也就帮另一边,一个对一个,两个对两个,这才公平!''

秦襄给他弄得啼笑皆非,但一来他领教过空空儿的手段,也知道他的怪脾气;二来他也实是不愿去捉铁摩勒。心中想道:``也好,我找到了这个借口,正好袖手旁观。让铁贤弟得个机会逃生。''便道:'\,'空空儿,你那日曾助了我们一臂之力,抓了你的师弟回去,看在这点情分,我愿与你交个朋友,你说如何就如何吧。'空空儿大笑道:``江湖上人人都说泰将军够朋友,果然不错。

来,来,来!你放下了这对金锏,咱们都来看打架吧!''

空空儿现身之后,宇文通便变了颜色,待到空空儿说了不助任何一方,他的神色才渐渐恢复过来。可是,铁摩勒趁这机会,又已抢到了先手攻势,渐占上风。

空空儿看了一会,忽地自言自语地说道:``摩勒来作皇帝老儿的保镖,这已经算得是件奇闻,现在,他以皇帝保镖的身份,却又与护驾的都尉。他自己的上司打起来,这更是奇上加奇了。

喂,铁摩勒,你为什么和长官打架?''

铁摩勒打得正在吃紧之际,来不及答他,空空儿道:``喂,小摩勒,秦将军都愿意和我交朋友,你倒不愿意吗?我在问你呀!''

铁摩勒奋起全力,长剑一架,将宇文通迫退两步,没好气地答道:``那昏君说我是反贼,这厮要借我的头颅升官!''

秦襄听了,暗自惭愧,心想:``铁贤弟,莫非你也误会我了?''

空空儿又大声说道:``摩勒,我本来想找你的,你猜猜看,我找你作什么?''

铁摩勒心道:``空空儿,你也真是太不识趣了。这个时候我哪还有闲心情与你聊天?''

空空儿大笑道:``猜不着么?我也谅你猜不着!好,我就告诉你吧。我有心与你交个朋友,想送一件极之难得的礼物给你。

你再猜猜看,这礼物是什么?''

铁摩勒大声道:``不知道,我也不要!''

空空儿又大笑道:``你这话且慢点说,这礼物对你大有用处,你知道了非要不可!''

秦襄心中一动,问道:``到底什么礼物?你就说出来吧,别让他瞎猜了。我听着也急着想知道呢!''

空空儿道:``说出来又是一件奇闻!摩勒,你这位上司不是说你是反贼么?可是我手上有一封信,却正是这位宇文将军写给安禄山的,信中说得清清楚楚,愿意给安禄山作内应!你说这奇不奇?这封信我当礼物送你,你要不要?''

空空儿此言一出,宇文通面色登时大变,有如死灰,虚晃一招,便想夺路奔逃。铁摩勒哪能容他逃跑,脚尖一点,箭一般地又追上去,长剑指到了他的背心,宇文通只好又转身招架。

秦襄见此情形,知道空空儿所说是实,不禁心中大喜,``若是当真有这封信,铁贤弟拿到证据,回去告发,那就不难无罪,反而有功了!''他陡地精神一振,提起双锏,便要上前。

空空儿双手一拦,笑道:``秦将军,你忘记了与我的诺言么?安静下来,看他们打吧!''其实秦襄这次却是意图帮铁摩勒捉宇文通的。

不过,到了此时,铁摩勒亦已无需秦襄来帮他了。宇文通最恐惧的事情给空空儿揭了出来,而且听空空儿的口气,他又是站在铁摩勒这边的,字文通早已吓得魂魄不全,哪里还能凝神对敌?

铁摩勒大喝一声,剑招疾变,但见寒光匝地,紫电盘空,将宇文通整个身形,都笼罩在剑光之下。宇文通章法大乱,使出来已不成招数,铁摩勒``刷''的一剑刺将过去,在他的肩头上刺了一个透明的窟窿,宇文通忽地将双笔倒转过来,笔尖对准了自己的咽喉便刺。铁摩勒又是一声大喝,长剑一撩,将宇文通那一对判官笔打飞,喝道:``反贼,你想自杀,没那么便宜!''声到人到,迅即便点了字文通的穴道,他恨气未消,顺手在宇文通面上,噼噼啪啪的又打了两巴掌。

空空儿笑道:``打得好,打得好!''掏出信来,递给铁摩勒道:``这件礼物对你是大有用处了吧?''不料铁摩勒却摇了摇头,并不去接这封信。正是:只为伴君如伴虎,英雄义士已寒心。

欲知后事如何?请听下回分解------

旧雨楼扫描,虚无居士OCR,旧雨楼独家连载

\chapter{第三十回 英雄痛洒伤时泪
关塞萧条行路难}\label{ux7b2cux4e09ux5341ux56de-ux82f1ux96c4ux75dbux6d12ux4f24ux65f6ux6cea-ux5173ux585eux8427ux6761ux884cux8defux96be}

秦襄诧道:``铁贤弟,这正好可作你的护身符,你为什么不要?''铁摩勒道:``我不回去了。这封信请你拿去献给皇上,我不求什么功劳,只求抹去这`反贼'的罪名便已心满意足。''

秦襄苦笑道:``铁贤弟,在皇上跟前当差的人,谁没有受过委曲?别说这些负气的话了!''

铁摩勒正容说道:``秦大哥,我说的可不是负气话。我曾答应了郭令公和南师兄,尽忠职责,保护皇上人蜀,邀天之佑,路上虽有风波,圣驾安然无事。现在险难已过,到了蜀境,此去已是一片坦途,我的担子也可以卸下来了。想你秦大哥也不至于说我对不起朋友,对不起皇上了吧?''

秦襄低声说道:``我知道,那是皇上对不起你。''

铁摩勒道:``马克驿之变,皇上失了贵妃,即算没有字文通进谗,皇上对我,也是怀恨于心的了。我若回去,纵然这次幸免,下次也会另有其他罪名。秦大哥,你要知道刚才在行所发生的事情么?''

当下,铁摩勒将皇帝怎样骗他,说是给他加官进爵,却赐他毒酒之事说了出来,然后问秦襄道:``秦大哥,你替小弟想想,我还好回去吗?''

秦襄黯然不语,虎目蕴泪,不知是为了铁摩勒的遭遇而难过,还是为了皇帝对忠奸不分而生悲,好一会子,都说不出话来。

空空儿笑道:``这又何须难过,摩勒,皇帝老儿不赏识你,我赏识你。你本来不合适作什么侍卫的,在宫里当侍卫,就像猛禽被关在笼子里一般,那有多问呀!''

空空儿笑了一笑,又道:``我这次带礼物给你,本来是想对你有点好处的,现在也用不着了。''

铁摩勒道:``不,还是有用处的。最少也可以令到那位糊涂皇帝,明白谁才是真正的反贼。''说罢,将那封信接了过来,转交给秦襄。然后问道:``'这封信你是怎么得来的?又怎的这样巧,刚刚在这时候送到?''

空空儿道:``这是我在精精儿的身上搜出来的。字文通与安禄山的往来书信,都是他代送的,这次合该字文通倒霉,这封信他还没来得及送去,就给我揪回山了。

``我搜出了这封信,就来找你,到得广元的`行所'之时,想不到你已经出了事,我听得那皇帝老儿正下令追捕你,我则追踪字文通的马蹄痕迹,追到了这儿!''

秦襄和铁摩勒听了,不禁骇然,一面震惊于空空儿飞行绝迹的轻功;同时对空空儿的这番行事,也感到有点意外。

要知空空儿号称天下第一听神偷,一向恃强傲岸,任性胡为,黑白两道,全不买账,因此武林中人,十后八九都是咒骂他的,秦、铁二人,过去也是把他当作``妖邪''看待,想不到就是这个空空儿,两番帮了他们的大忙,不由得秦、铁二人不对他刮目相看。铁摩勒更是心中想道:``空空儿虽然行事怪僻,却原来也还有几分侠气。怪不得段大侠受了他夺子之辱,也还不肯随声附和地骂他。''

空空儿侧耳一听,笑道:``追兵已经来了,摩勒,要是你不想回去,这就该走了。''

铁摩勒道:``秦大哥,数月来多承照料,呵护周全,小弟今日拜辞了。尉迟大哥跟前,也请你代为致意。''

秦襄叹口气道:``我等三人,肝胆相交,正道是朝中有伴,却不料今日又劳燕分飞。事已如斯,铁贤弟,我也不敢强留你了。但愿你不要太计较所受的委屈,身在江湖,心存汉阙,同诛逆贼。天下太平之后,咱们还有相见之期。''

铁摩勒道:``这个不劳大哥吩咐,那昏君虽要杀我,我却是不会记这私仇的。我准备就潜回潼关敌后,助南师兄抗击贼兵。''

秦禁赞道:``铁贤弟,你不愧是个好男儿!我在蜀中等候你们的捷报。请恕我不能运送了。''当下将宇文通捆缚起来,放在马上,回首一声:``珍重。''便催马出林,那匹黄源马也似知道从此要与铁摩勒分离,长嘶不已。秦襄频频回顾,铁摩勒目送征骑,两人都不禁黯然伤别。

空空儿道:``秦襄已经出去与他们会合,追兵是不会到这儿来了。咱们还可以歇一会儿。摩勒,你不记皇帝老儿之仇,可还记着你我之间的旧恨么?''

铁摩勒正容答道:``这次,你帮我的忙,我该谢你。但你夺了段大侠的儿子,这件事,我却是怎也不能原谅你。''

空空儿笑道:'\,'刚才秦襄在这里,我的话还只说了一半。实不相瞒,我这次前来找你,除了给你送礼之外,另一半原因,却正是为了那个孩子。''

铁摩勒道:'你愿意把那孩子交还段大快了么?''

空空儿道:``那孩子不在我的手中,不由得我来作主。''铁摩勒大失所望,道:``那还有什么可说的?''

空空儿道:``不然,你还记得我当年对段大侠的诺言么?''铁摩勒道:``你说迟则十年,总之着落在你的手上,将那孩子交回。哎,现在刚好是十年了,你却又如此说法\ldots\ldots{}''空空儿截断他的话道:``我是绝不会让段大侠说我失信的,当然是有了希望才来。你听我说吧。''

空空儿续道:``收养孩子的那个人其实并无恶意,他对那孩子爱护得无微不至,当真是亲生的儿子也不过这般,而且还把一身超凡绝俗的武功也传了给他。现在,这个孩子虽然不过十岁,武功的基础已经打得非常扎实了,那个人也愿意将孩子交回他原来的父母。不过,要他的父母亲自去接他回来。''

铁摩勒问道:``这人是谁?''空空儿道:``这人是一位武林前辈,他的名字,我不敢说。''

铁摩勒听了,不禁大为奇怪,心中想道:``空空儿是个天不怕地不怕的人物,对这个人却竟是如此敬畏,连他的名字也不敢出

口,真不知是甚来头,能令空空儿如此?''又想:``虽说这人疼爱孩子,但他要了别人的孩子,十年来不许孩子的父母知道消息,这也未免太过不近人情!''

铁摩勒是个耿直的人,对这位武林前辈的行事殊不以为然,不过,这究竟是一个值得欢喜的消息。当下,铁摩勒便即问道:``如此说来,你可是为了要打听段大侠的下落而来找我的么?''

空空儿道:``正是。兵荒马乱,四海茫茫,要找一个居无定址的人太不容易,你是跟着皇帝老儿走的,找你便容易得多了。''

铁摩勒道:``段大侠的行踪我也不知,我的南师兄和皇甫前辈等人,在潼关附近编组义军,待我先去找寻他们,然后再打听段大侠的消息。''

空空儿沉吟半晌,说道:``如此辗转寻人,只怕要费许多时日,我还有点事情,要到别处去。不如这样吧,你若找到了段大侠,就请他们夫妇再到玉树山的玉泉观来,我在那里等候他们。会合之后,再一起去见那位前辈。''

铁摩勒道:``好,我一定替你把话送到。这事情了结之后,我与你的仇恨一笔勾销!''空空儿大笑道:``好小子,恩怨分明,真不愧是铁昆仑的儿子!''笑声尚在林中回旋,人影已经不见。

铁摩勒呆了片刻,心想一个人真是难以捉摸,自己曾那么样的恨过空空儿,想不到现在竟和他交上了朋友,从空空儿身上又不禁想起王燕羽来,不觉一片茫然。

铁摩勒那匹坐骑已给宇文通射死,幸而宇文通那匹坐骑只是略受轻伤,尚堪代步,铁摩勒随身带有金疮药,给它敷了伤口,便即跨马登程。

一路平安无事,但离开蜀境,回到关中的来时原路,但见荒芜的景象,比前更甚,当真是人烟稀少,十室九空,觅食也有点困难。

铁摩勒一路上猎取鸟兽,有时还要掘野菜充饥,这时已是初冬时分,鸟兽很少出来,野菜也大都枯黄了。铁摩勒为了寻觅食物,自不能专程赶路,有一顿没一顿的,常受冻馁之苦,走了一个多月,才到扶风郡境内,离长安还有三百多里。

这一日铁摩勒正骑着那匹御马在大路上走,那匹马本是匹雄健的骏马,但经过千里驰驱,途中又缺乏水草,早已形销骨立,变成了一匹瘦马,疲累不堪了。铁摩勒爱惜马力,策马缓缓而行。忽见前面尘头大起,有一彪军马驰来,前头打着一面大旗,绣着金龙,并绣有``大燕''二字。

铁摩勒初时以为是官军,待到看清旗号,方知不是。原来这``大燕''二字,乃是安禄山的``国号'',安禄山在攻陷洛阳之后,便僭号称帝,国号``大燕''。这支军队竟是安禄山的队伍。

铁摩勒大吃一惊,心中想道:``贼军在此出现,这么看来,长安是早已陷落了。''再过一会,那彪军马的距离更近,队伍前头那两个将军的面貌也看得清楚了。

铁摩勒这一惊更是非同小可,那两个伪将军不是别人,正是薛嵩和田承嗣,十年前铁摩勒在长安曾和他们交过手的。

铁摩勒慌忙离开大路,纵马向田野中奔跑,当真是``落荒而逃''!

相隔十年,薛、田二人已认不出是铁摩勒。不过,在这个兵荒马乱的时候,人烟绝迹的地方,却有一个少年骑马乱跑,当然会引起贼兵的注意。

薛嵩喝道:``你是什么人?过来,过来!''铁摩勒哪里肯听,跑得更快了。田承嗣道:``这人定是唐军探子,不必再问了!''一声令下,登时有数十骁骑,飞马来追,箭如雨下。

若在平时,铁摩勒真不会将这几十个贼兵放在心上,但此时他腹内空空,气力已使不出来,他挥剑拨打,打落了十几支箭,终于中了一箭。

贼兵追得更近,有个军官模样的人叫道:``你们看我的箭法!''拉起五石强弓,嗖的一箭,便把铁摩勒的坐骑射翻。那军官哈哈大笑,纵马上来,抛出绳索,要活捉铁摩勒。另外两个贼兵,亦已驰马赶到,成了三面包围之势。

铁摩勒提一口气,在马背上纵身飞起,喝道:``你也看我的箭法!''正有两支箭射到,铁摩勒在半空中翻了一个筋斗,接过了那两支箭,就当作甩手箭发出,登时也把贼兵的两匹马射瞎,把那两个贼兵抛下马来,他迅即一个``鹞子翻身'',又扯着了那军官抛过来的绳索。

铁摩勒虽然饿得头晕眼花,又受了伤,但他到底是具有上乘武功的人,一执着了绳索的一端,立即施展``借力反击''的功夫,但听得`勺乎''的一声,两人刚好对调了一个位置,铁摩勒落下地来,手挥绳索,却把那军官抛上了半空,摔得个发昏。

隐隐听得有人赞道:``咦,这人好俊的身手!''声音似是熟人,铁摩勒茫然四顾,想要找那说话的人,忽觉一股热血冲到喉头,登时眼睛发黑,跌倒地上,人事不知!原来他的气力、精神也都已用尽了。

也不知过了多少时候,铁摩勒悠悠醒转,视力还未完全恢复,朦朦胧胧之中但见一个戎装佩剑的人,正俯着腰看他。铁摩勒翻了个身,想跳起来,可是力不从心,``咕咚''一声,又摔倒了。铁摩勒叫道:``薛嵩反贼,你杀了我吧!''

那人忽地伸出手来,掩住了他的口,低声说道:``你别胡乱叫嚷,我不是薛将军!''

铁摩勒定睛一看,这才认出了这个人乃是聂锋。

原来出声称赞铁摩勒的那个人就是聂锋,他心肠较好,又爱惜铁摩勒的身手,因此便向薛嵩求情,救了铁摩勒的一命。聂锋是薛嵩的表弟,又是他的副手,本领比薛嵩强得多,薛嵩的``战功''大半是靠他挣来的,所以即算撇开表亲的关系不谈,他也非给聂锋的面子不可。

聂锋将铁摩勒安置在自己的帐中,给他裹好伤口,又把参场给他灌下。

当年铁摩勒在安禄山的长安府邸里也曾和聂锋交过手,事隔十年,铁摩勒已长大成人,聂锋初时也还认不出他,但越看越觉得似曾相识,待到铁摩勒醒来之后,一开口便骂薛嵩,聂锋这才识破了铁摩勒的身份。

聂锋拉过了一张毯子,给铁摩勒盖上,笑道:``你可是铁摩勒么?你好大的胆子!听说你已经给唐朝的皇帝老儿当御前侍卫去了,怎的却又单身匹马,到这儿来?''

当年段圭璋夜间安府救史逸如的时候,聂锋曾暗中庇护过他;后来他又曾想过法子,想把史逸如的妻子卢夫人救出去,这两件事情,铁摩勒都是知道的。当下也不再隐瞒,便直言说道:``不错,我就是铁摩勒。我不惯拘束,不想做皇帝老儿的侍卫了,私逃回来,想不到在这儿撞上了你们,要杀要剁,随你们便。''

聂锋笑道:``你还是当年的那副倔强脾气。我若要杀你又何必救你?不过,你可不能胡乱骂人,要是给薛将军听到了,我也就无法庇护你了。''

聂锋又道:``你既不愿给那皇帝老儿当差,那就留在我这里吧。

铁摩勒冷冷说道:``你救了我的性命,我感激你;你这样劝我,我却要骂你了!''聂锋道:``我这是一番好意,怎么反而该骂了?''铁摩勒道:``你叫我留在这里,你把我看成何等样人?我是顶天立地的大唐汉子,岂能留在反贼军中?要嘛,你就杀我;要嘛,你就放我,没有第三条路了!''

聂锋面上一阵青,一阵红,半晌说道:``大唐天子仓皇辞庙,狼狈而逃,因处一隅,偏安西蜀,亦难久存,你又无官守,却去做什么大唐的忠臣?''

铁摩勒冷笑道:``只是做官的才有守土之责么?聂将军,你看错了。皇帝老儿虽然抛弃了百姓逃难,百姓仍然是要保护自己的家园的,现在大河南北,已是民军四起,你还不知道吗?何况郭令公已兴兵于太原,太子亦督师于灵武,你们现在虽尚能肆虐于一时,亦不过回光反照而已!''

聂锋连忙摇手道:``摩勒,在这里你暂且莫谈国事,咱们只论朋情。你愿意把我当作朋友的话,就安心在这里养伤,伤好了我自有分数。''

铁摩勒翻了个身,说道:``我的伤倒没有什么,我只是为你可惜。''

聂锋睁大了眼睛,想要禁止他说话,但想了一想,却又不自禁地问道:``你为我可惜什么?''

铁摩勒道:``段大侠也曾和我谈起你,赞你是个有血性的男儿。想不到你竟然同流合污,甘心为虎作怅!''

聂锋满面通红,过了好一会子,方始叹口气道:''\,'段大侠果真这样赞过我么?这倒使我羞惭一了。摩勒,这些话请你不要再谈了,日久之后,心迹自明。''

铁摩勒试出了他的心意,也就含蓄地说道:``将军如此,我也就放心在你这里养伤了。''

正说到此处,忽听得有人走来,未曾报帐,便大声问道:``那小子可活得成么?''正是薛嵩的声音。

聂锋大吃一惊,连忙走到铁摩勒的身边,手掌在他伤口的旁边轻轻一抚,接着又在他的面上轻轻一抹,然后低声说道:``你切不可胡乱说话!''

铁摩勒最初莫名其妙,但心念一动,便即恍然大悟:``他把血污涂花了我的面,那是要叫薛嵩认不出我的本来面目。''

聂锋方才应了一声,薛嵩已拉开帐幕,走了进来。

薛嵩向铁摩勒扫了一眼,说道:``这小子可伤得不轻啊,简直象个血人!''聂锋道:``还好,受的只是外伤。他体魄强健,调养个十天半月,想必也会好了。''

薛嵩皱眉说道:``这小子武功不错,医好了他,倒是个有用之材,只不过在行军之中,却是难以伺候他啊,医药也不方便!''他横掌如刀,作了一个手势,表示不如``咔嚓''一刀,将他杀了算了。

聂锋忙道:``你猜这人是谁?说起来还是咱们的乡亲呢!''薛嵩道:``哦,是吗?说给我听,看我还记不记得?''

聂锋道:``他是我姑妈的疏堂侄子的外婆的孙子,就是那给人放牛的王老头的孙子,名叫王小黑的。你说巧不巧?''

薛嵩自小离开家乡,哪里记得这些缠七夹八的亲戚关系,不过,他有一个``好处'',对同乡还肯照顾,聂锋就利用他这个弱点,乱说一通,他也居然相信了,说道:``嗯,那可真是巧了。那就留他在军中吧,不过要拨出专人来照料他,却也还是一件麻烦的事情,就让他自生自灭吧。''

聂锋道:``小弟已想出个法子了,反正这里离长安不过两天路程,我就派人送他回去,让他在长安好生安养,痊愈之后,再来投军,那时还要请你多多照顾。''

薛嵩道:``对,你这个办法很好,就这么办!我身边正缺少有本领的人,他好了之后,可以做我的卫士!''

聂锋道:``王小黑,你还不谢过薛将军?''铁摩勒故意嘶哑着声音,含含糊糊地说了一声:``多谢,请恕小人不能起来叩头。''

薛嵩笑道:``你正在养伤,不必多礼了。哈哈,今天我还几乎把你当作唐军的探子宰了你呢!''

薛嵩说了一会闲话,兴尽告辞。聂锋抹了一把冷汗,说道:``好,幸亏你没有胡乱说话,现在你可以起来吃点稀饭了。你饿得太久,暂时只能吃点容易进口的东西。''

聂锋早已给他准备了一锅粥,还有半条蒸得烂熟的羊腿和一碗肉糜,铁摩勒也不客气,把稀饭和菜肴都吃得干干净净。他所受的伤,不过是摔倒之时,给尖利的石子割损了一些皮肉,并无大碍,吃饱之后,登时精神大振。

聂锋坐在一旁陪他,见他神色转好,大为快慰,说道:``摩勒,看来,你在明天便可以起程了。咱们相聚之时无多,我想问你一件事情。听说在皇帝老儿逃难的前夕,曾有人人宫行刺,那时,你可在场吗?''

铁摩勒道:``不错,是有这么回事,刺客便是精精儿。他是你们这边派出去的,难道你还不知?''聂锋道:``正是因为不见他回来,所以想打听一下。''铁摩勒说笑道:``他已被他的师兄揪回山去,最少在三年之内,他是不会在江湖露面了。''当下,将那次精精儿行刺的经过说给聂锋听,只隐瞒了王燕羽背叛精精儿的那一段。

聂锋又问道:``你最近可有见过夏凌霜女侠么?不知她可安好?''铁摩勒道:``她与我的南师兄已经成婚,好得很!怎么你会问起她?''聂锋道:``我以前曾在薛将军家里见过她,承蒙她还看得起我,没有把我当作坏人。''铁摩勒道:``对了,这事情她也曾对我说过,你对卢夫人暗中维护,她家已知道了。段大侠很感激你。''

聂锋色然而喜,这倒并不是因为听得夏、段二人说他好话,原来他那次被精精儿骗去了卢夫人托他转交夏家的信,生怕夏凌霜被精精儿所害,内疚于心,数年不安。所以他才特别要向铁摩勒打听这两个人的事情。但他却不知,夏凌霜虽然无事,她们母女却因此受了许多灾难,她的母亲也已死了。

也幸亏铁摩勒没有对他说起那些事情,减少了他许多顾虑,当下说道:``摩勒,你见到段大侠和夏女侠的时候,请代为致意,就说我聂某人承蒙他们当作朋友看待,将来必定有所报答他们。''

两人谈得越发投机,铁摩勒听他口气,已断定他不是甘心从贼,当下念头一动,向他说道:``我还有一件事情请你帮忙,不知你可愿意?''聂锋道:``只要我力之所及,决不推辞。''铁摩勒道:``我想见卢夫人一面,你办得到么?''

聂锋沉思一会,毅然说道:``摩勒,我可以给你设法,但我也要请你不可做出令我难为的事情。''铁摩勒道:``你放心,我只是要见她一面,决不在薛家胡闹,难道你怕我将薛家的家人残害么?''聂锋道:``你是侠义中人,我知道你不会胡乱杀人。但你亦不能将卢夫人劫走。其次,你不能在薛家露出你的身份。''铁摩勒道:``好,我都答应你。不过,若是别人来救她出去,我就管不着了。''聂锋道:``她自己愿意留在薛家,只要不是用强绑架,她是不会走的。当年我想暗中将她放走,她也不愿走呢。''

聂锋取出一面腰牌,说道:``这是我军中通行的凭证,你有了这面腰牌,路上就不会受到阻难,到了长安,也可以凭此证明你是在军中当差的。明天我设法雇一辆车送你去长安,到了长安,你可以住在我的家中,我与薛将军是比邻而居,两家有门相通的。你住下来,自有机会可以见到卢夫人。''

铁摩勒大喜拜谢,说道:``我的伤已无大碍,只须赐马一匹代步便可,不必另雇车辆了。''

聂锋道:``我再写一封信给你,交给我的管家,他会妥贴招呼你的。我家中人口无多,除了内子和小女之外,只有几个家丁,他们都是我的心腹,你可以无忧。不过,长安现在还是很乱,没事你少出门。''

铁摩勒再拜道:``我理会得,你也请放心。承你肝胆相照,道义相交,我感激不尽。''这个时候,东方已经发白,铁摩勒取过书信,藏好腰牌,便即动身。聂锋挑了一匹好马给他,亲自送他出营。

铁摩勒有了那面腰牌,不但沿途无阻,还可以充作出差的军官,在各处驿站食宿,免受了饥寒之苦。

第三日到达长安,只见大街上每隔数十步便有站岗的兵士,两旁商店都是半掩门户,街头上行人寥寥无几,道旁的沟渠还不时可以发现死人的骸骨。原来安禄山攻进长安之后,肆行杀戮,在京的宗室皇亲,无论皇子皇孙,郡主公主,驸马郡马等国戚,来不及逃走的都给剖腹剖心,文武百官,不肯降顺的,也都被一刀了结。小民枉死的,更不计其数。当时诗人韦庄有两句诗道:'内库烧为锦绣灰,天街踏碎公卿骨。''便是记录安禄山破城之后的惨象的。

铁摩勒好生感慨,``长安数代繁华,想不到今日竟变成了人间地狱,可恨那皇帝老儿,在太平时候,只顾自己寻欢觅乐,宠任奸佞,把杨国忠、安禄山都当作腹心,他宗庙被毁,乃是自食其报,不足惋惜,只是却连累了许多无辜的百姓!''

聂锋是安禄山手下有数的将军,铁摩勒取出腰牌。以回京办差事的军官身份,向站岗的士兵查问,很容易便查到了聂家的所在。

只见两座大屋毗连,一边乃是薛府,一边乃是聂府,铁摩勒心中暗喜:``我得这个藏身之所,真是最好也不过了。不但有机会可以见卢夫人,还可以等待段姑丈的消息。''段圭璋当日和他分手时,曾发过誓言,无论如何,也要将史逸如的妻女救出魔窟,故此铁摩勒料他迟早也会到长安来。

当下铁摩勒便去叩门,将那封信交给了门子,不久管家便亲自出迎,带他进去。聂锋那封信是把铁摩勒认作同乡亲戚的,他的家人当然不敢怠慢。

哪知经过了院子,正要踏上台阶的时候,忽听得一个稚嫩的声音喊道:``看镖!''

陡然间只听得铮铮两声,两枚钱镖,破空飞出,形如``人''字,一高一低,铁摩勒听风辨器,已知高飞那枚钱镖是打他胸部的``灵府穴'',低飞那枚钱嫖是打他膝盖的``环跳穴'',不由得大吃一惊,做梦也想不到会在聂家遭受暗算!

心念未已,那两枚钱镖已到,铁摩勒反手一抄,把高飞那枚钱镖接到手中,身形一仰,脚尖踢起,又把低飞那枚钱镖踢落。说时迟,那时快,铮的一声,第三枚钱镖又到,铁摩勒无可躲避,只得把接来的钱镖打出,碰个正着,两枚铜钱,同时跌落。

就在这时,只听得一个妇人斥道:``隐娘,不可无礼,这是你爹的客人!''铁摩勒抬头一看,怒气消了一大半,却原来站在台阶上发钱镖打他的人,竟是一个未成年的女孩子,流着两条辫子,一副淘气的脸孔,看来最多不过十二三岁。在她背后,有一个中年妇人,想必是她母亲。

那管家忙道:``这是我家主母,这是我家小姐,王兄,你不可见怪,我家小姐------''话犹未了,那女孩子已拍起手笑道:``叔叔,你的功夫很好呵!这一手接镖还镖真是漂亮极了,他们都比不上你!''

聂夫人呵责女儿道:``你真是越来越野了,也不看看来的是谁,就胡打一通。幸亏这位叔叔没给你打着!要不然我可要给你气死啦!''跟着对铁摩勒解释道:``这是小女隐娘,从小就欢喜拈枪弄棒的,这几天她学会了用铜钱当暗器,玩得正起劲,总是缠着家丁,要他们`接镖',哎呀,真是不好意思!''那女孩子道:``打着了也没什么,我会给他解穴的。叔叔,你不会生我的气吧?''聂夫人怒道:``你还要辩,待你爹回来,我告诉他,叫他撕了你的皮!''

铁摩勒这才明白,敢情这女孩子误将他当作家丁,拿他试``镖''来了。他小时候也是个淘气的孩子,嗜武爱玩的,非但不恼,反而替聂锋欢喜,``我在她这样年纪的时候,暗器功夫还远不如她呢!''当下便赞她道:``真是将门虎女,巾帼英雄。夫人不可怪她,暗器打穴,本来是要多练的。''

聂隐娘得意笑道:``妈,你听听人家是怎么说,不练怎么行呢?''聂夫人笑道:``你再夸奖她,她更要胡闹了,她爹爹已经把她宠坏了。你练暗器,也不该把活人当靶子呀。''聂隐娘道:``妈,这你就外行了,钱镖打穴,除了找活人`喂招',那还有什么办法?''铁摩勒道:``我倒有一个主意,叫人给你造一个木人,按照人体的穴道部位图上圆圈,叫人找着木人飞跑,你发钱镖打术人的穴道,不也是一样吗?''

聂隐娘拍着小手叫道:``这个法子真好,我怎么没有想到呢?叔叔,你一定是会家子,你陪我练武。''

铁摩勒笑道:``我是个乡下人,只懂得几手庄稼汉的把式,要我陪你练武,那就只有挨打的份儿了。''

聂隐娘撅着小嘴说道:``我不信!我的三枚钱镖都给你接了,你还说不懂,骗得了谁?''

聂夫人道:``隐娘,别胡闹。王叔叔才来,茶都未曾喝一杯,你怎么可以就歪缠客人,要人家陪你练武?简直是不懂规矩,走远一些!''跟着笑道:``都是他爹把她宠坏了,好在王叔叔不是外人,若是在别的客人面前,人家不笑话你也会怪我没有家教呢!''铁摩勒道:``这正是将门本色,她年纪轻轻,有这样的武功,人家称赞她还来不及呢,怎会笑话?''

聂隐娘给她母亲一骂,不敢再缠,但也不走开,看来不单是父亲宠她,母亲也把她娇纵惯了。所以她对母亲的话听一半不听一半,看那样子,似是还在等待铁摩勒和她练武。

聂锋的信上说铁摩勒是他的同乡王小黑,还沾着一点亲戚关系的,聂夫人不免和他叙叙乡情,并问起一些相识的人来。好在聂夫人亦是离乡日久,对乡下的事情并不清楚,铁摩勒又曾得聂锋之教,聂锋早已预料到他妻子会问起那些人,给铁摩勒准备了一套说话,铁摩勒东拉西扯,还勉强可以应付。遇到他不大清楚的,便避重就轻,拣自己知道的多说一些,含混过去。

聂夫人不过是为了礼貌关系,出来见他,并非有心盘问,谈了一会,要问的也都问了,当下便道:``在这兵荒马乱的年头,难得有乡亲来到,你在这里住下,不必客气,要当作在自己家中一般才好。房间我已给你准备好了。''

那管家正要带铁摩勒进房安歇,忽地又有一个女孩子走来,叫道:``隐娘姐姐,今天还练剑吗?''

聂隐娘道:``红线,你来得正好,这位王叔叔是新来的客人,他的武功高明得很,咱们的剑法是关在屋子里练的,没给外人看过,也不知是行还是不行。不如请王叔叔今天给咱们评一评吧!''

聂夫人道:``隐娘,你又来缠王叔叔了。你们自己练去吧。''聂隐娘道:``反正王叔叔现在已没事了。他茶也喝过了,你说他是咱们的自己人,爹不在家,我请他指点,有何不可?''

名叫红线那女孩子长得非常秀丽,年纪比聂隐娘小,看来至多十岁,铁摩勒望了她两眼,只觉她的相貌很像一个人,不觉心中一动。

铁摩勒道:``指点二字,我当不起。让我开开眼界,倒是真的。这位小姑娘是------''聂隐娘道:``她是我的薛家妹妹。红线妹妹,你也来见过王叔叔。''聂夫人补充道:``她就是隔邻薛将军的掌珠。她们一对表姐妹倒是好伴儿,天天在一起玩的。薛将军想必你已是见过的了?''铁摩勒道:``薛将军很重乡情,我这次到长安来,就是多蒙他的照顾。''

薛红线过来请了个安,说道:``我的剑法还是初练的,等会你看了可别要见笑。''她的态度比聂隐娘要文静得多,更惹人爱。铁摩勒颇感诧异,心里想道:``难道我所料想的错了?她当真是薛嵩的女儿?奇怪!薛嵩怎会生出这样的好女儿?''

铁摩勒已然答应了去看她们练刻,聂夫人也就不再拦阻了。当下,聂隐娘便带铁摩勒进人后花园,她家的练武场,就在花园之内的。两旁有兵器架,十八般兵器,-一齐全。

可是这两个女孩子并不拿起真刀真剑,而是各自在兵器架上拣出了一柄木剑来,想来这两柄木剑就是专为给她们练剑用的。场边有一桶石灰,聂隐娘将木剑在石灰中一插,反身跃出,叫道:``来吧!''

薛红线学了她的样子,木剑蘸了石灰之后,说道:``今天我不必你先让我三招了。''木剑扬空一闪,脚踏中宫,进了一招,铁摩勒一看,不觉大吃一惊。他起初只道是小孩子的玩艺,哪知薛红线使出来的竟是上乘剑法,看她中宫进剑,使的明是``白贯贯日''的招数,招数未曾使老,倏的剑锋一颠腴滑过一边,左刺肩肿,右削腰胁,变化的迅速轻灵,竟无殊武林高手。

聂隐娘的应招更怪,只见她横剑当胸,站定不动,待得薛红线的木剑已经刺到,她突然双足交叉,往下一蹲,矮了半截,薛红线的木剑几乎贴着她的头皮削过,却没有刺着她。薛红线跟着一招``红霞铺地'',木剑抖起了一个圆圈,就在她的头顶上罩下来。铁库勒正在心想:``要是当真对敌,这一招可不容易躲避。''心念未已,陡然间,只见聂隐娘单足支地,打了几个盘旋,沉剑一引,便倏的上挑,薛红线的木剑被她绞着,转了几转,她那先手攻势,已给解了。

两柄木剑一合再分,薛红线绕场游走,铁摩勒暗暗注意她的步法,竟是踏着九宫八卦方位,丝毫不乱。聂隐娘展开了攻势,俨如蝴蝶穿花,一柄木剑指东打西,指南打北,非但中规中矩,而且往往有出人意表的招数,连铁摩勒这样一位剑学行家,也料想不到的!直把铁摩勒看得眼花缭乱!正是:

长江后浪推前浪,英雄巾帼胜须眉。

欲知后事如何,请听下回分解------

旧雨楼扫描,虚无居士OCR,旧雨楼独家连载

\chapter{第三十一回 故都又见重归鹤
逋客何堪不了情}\label{ux7b2cux4e09ux5341ux4e00ux56de-ux6545ux90fdux53c8ux89c1ux91cdux5f52ux9e64-ux900bux5ba2ux4f55ux582aux4e0dux4e86ux60c5}

铁摩勒越看越觉得奇怪,不但是惊奇于她们剑法的精妙,而且,更重要的是因为看不出她们的师承。铁摩勒暗自想道:``薛嵩、聂锋我都曾经和他们较量过,薛嵩的剑法甚是平常,这且不说;聂锋的剑法虽然高明得多,但也远远比不上这两个女孩子的奇诡多变,路数也完全不同!看来她们的剑法绝不是父亲教的!''

这时,聂隐娘与薛红线已经斗了将近百招,薛红线踏着九宫八卦方位,极力抢攻,聂隐娘沉着应付,守中带攻,一剑一剑的反削回去,稳健轻灵,兼而有之,看来功力似比薛红线略胜一筹。

铁摩勒正自心想:``小的这个恐怕就要输了。''薛红线也似乎知道自己要输,突然使出个出奇制胜的险招,脚尖一点,修地身形掠起,凌空刺下。铁摩勒识得这一招是``白猿窜枝'',乃是袁公剑法中一招精妙的招数,铁摩勒曾见空空儿使过,当年他的姑丈段圭漳就是败在这一招的。但薛红线用这一招却和空空儿又不尽相同,空空儿是身形平射出去,而她则是凌空击刺,方位和剑势都有变化,不过都是妙到毫巅,真可说得上是``异曲同工''。

铁摩勒禁不住大声喝彩,就在彩声之中,只见聂隐娘双腿下弯,纤腰后仰,木剑往上一封,她用的是``铁板桥''的功夫,双足牢牢钉在地上,腰板几乎放平,薛红线的木剑在她面门刺过,只差几分。聂隐娘这一招用得更险更妙,但过后铁摩勒自己寻思,也只有这一招才能应付。

但听得``卜''的一声,聂隐娘的木剑架上去,薛红线的木剑击下来,双剑相交,薛红线的冲力较大,聂隐娘的功力较高,两炳木剑登时都脱手飞出,两个女孩子也已笑吟吟的拉着手儿站在一起。

薛红线道:``表姐,还是我输了!''这时铁摩勒方才看得清楚,薛红线的身上有七点灰点,聂隐娘身上只有三处。即是说在她们斗剑的过程中,薛红线中了对方的七剑,而聂隐娘则仅中了三剑。

聂隐娘道:``不,你已经比上次进步多了,上次我让你三招,结果也是和今天一样。你比我小两岁,过两年你会强过我的。''

薛红线道:``咱们别自己私评,还是向这位王叔叔请教吧,看看有什么使得不对的地方,要是和敌人真打的话,管不管用?''

铁摩勒笑道:``你们的剑法比我高明,这是问道于盲了。''他说的当然有点谦虚,不过也是实话,要是只论剑术,铁摩勒未必胜她们。

这两个女孩子哪里肯休,正在缠他,忽听得有人叫道:``线姑,你该回家啦!''一个装束似是保母的妇人走了进来。

这妇人的相貌甚是可怖,脸上交叉两道伤痕,额角上有几个疮疤,眼皮倒卷,裂开几条,脸上几乎没有半点血色。但虽然如此,却并不感到可憎,甚至再多看两眼之后,还感到她有一种天然风韵,远比庸脂俗粉可比。她气度雍容,举止娴静,体态苗条,虽然她头发已经花白,但可以断定:在她年轻的时候,容貌未曾毁坏之前,一定是个出自名门的美人胎子!

铁摩勒一见,禁不住心头一震,又悲又喜。想道:``这一定是卢夫人无疑了。可怜她为了保全贞节而自毁容颜,在这十年中不知曾受了多少苦难。''

果然便听得薛红线说道:``卢妈,我正玩得高兴呢,我还不想回家。''这一声``卢妈'',证实了铁摩勒的推断无差。

卢夫人柔声说道:``你已玩了半天了,你瞧你的衣裳都湿透了,是不是刚练过剑来?你肯用心练剑,我很欢喜,但出了这么多汗,就该回去换衣裳了。要是生出病来,怎么得了啊!''对薛红线的痛惜之情,溢于言表。

铁摩勒又禁不住心中一动,想道:``是了,这个薛红线一定就是她的女儿。想必是薛嵩夫妇见这孩子可爱,认了她作女儿。

却要她本来的母亲作为保母,不许她表露身份。''

薛红线揪着小嘴儿撒娇道:``卢妈,你先回去,我不会生病的,生病了也不怪你。你不知道,今天来了一位王叔叔,他的本领可高强呢,我们正要请他指点剑法呢!王叔叔,王叔叔,你佩有长剑,一定懂得剑法,也抖几手给我们瞧瞧好不好?''她像游鱼似的,从卢夫人身边溜开,又来缠铁摩勒了。

卢夫人望了铁摩勒一眼,她不知铁摩勒是谁,一时倒不好说话,想等待这位``王叔叔''帮她劝说,铁摩勒却已拔出剑来,说道:``也好,指点你们,我不敢当,咱们倒可以琢磨琢磨!''

两个女孩子拍掌叫道:``好极了,让我们看看你的剑法,那更是求之不得!''

卢夫人正自心想:``这客人真不通情。''忽听得铁摩勒弹剑歌道:``宝剑欲出鞘,将断佞人头。岂为报小怨,夜半刺私仇,可使寸寸折,不能绕指柔!''声音悲壮,大有燕赵豪侠弹剑悲歌之慨!

这几句诗正是段圭漳平日所喜欢朗吟的。当年,在他准备去刺杀安禄山的前夕,就曾经像铁摩勒如今这样,弹剑高歌。

卢夫人听了,不觉大吃一惊,定睛看着铁摩勒,忍不住两点泪滴了下来。幸而雄红线正在缠着铁摩勒,没有察觉。

这两个女孩子听得奇怪,问道:``叔叔,你可是背剑诀么?''铁摩勒胡乱点了点头,薛红线道:'你要一口气连使六招么?''原来她们初学剑术的时候,都是每学一招,便要先念一句剑诀的。薛红线听出他是共念了六句,却听不明白他是说些什么。心里在想:``这位王叔叔所念的剑诀,倒像卢妈教我念的诗句一般。''

铁摩勒道:``不错,我该套剑法县不能拆开本_地地的胜。

前面一段是六六三十六招,后面一段是四十二十八机前而具。

六把自成一节,后面是每七招自成一节。''

薛红线拍手笑道:``你的剑诀比我们的剑诀好听得多,一定是好的了,赶快练给我们瞧。''

铁摩勒道:``我是要练给你们瞧,但是小孩子也应该听大人的话,你先换衣服去,免得卢妈为你担心。''

薛红线急于要看铁摩勒的剑法,嚼着嘴儿说道:``换衣服不打紧,只是我一回家,我妈就不会让我回来了。她一定说,你今天已经玩得够了,要去明天再去吧。''

铁摩勒笑道:``那么,你就明天再来吧,反正我明天也还未走。''

淡红线道:``不成呀,要是你现在不练给我瞧,我今天晚上会睡不着。''

聂隐娘道:``我有一个办法,我只比你高一点儿,我去年的衣裳一定合你身材,你到我房里来换过一套旧衣裳吧。''

薛红线道:``好,到底是表姐你想得周到。卢妈,你在这里等着我,我看了这位叔叔的剑术就和你一道回家。''卢夫人道:``你妈等着你呢!''薛红线道:``你给我撒个谎儿,就说那个时候才找见我不就行了?园子这么大,我们倘若不在练武场上,本来你就不容易找见我们的。咱们三人一样说法,还怕骗不过吗?''卢夫人道:'\,'呀,你真淘气。好,你就去换衣裳!吧,快去快来。''

这两个女孩子走后,卢夫人露出疑惑的眼光,说道:``清恕老婆子冒昧,请问少爷,你刚才念的是什么诗句?''铁摩箭道:``我也不知,我是听得一个人常常在念,我听得多了,也跟着背熟了。''

卢夫人道:``这个人呢,他还在世上吗?''铁摩勒道:``他遭过许多灾难,您是上天怜他大仇未报,暗中保佑他,每次灾难,他都逃过了。说不定他不久就会到长安来。''卢夫人经过了这番试探,对铁摩勒已不再怀疑,连忙问道:``你是谁?你既与那人相识,又怎么会到这里来?''

铁摩勒这才说道:``实不相瞒,段门窦夫人的长兄乃是我的义父,当年我也曾随段大侠偷入长安,在安贼家中大杀了一场,可惜寡不敌众,救不了尊夫。''卢夫人吃了一惊道:``你是铁摩勒么?''铁摩勒道:``正是。夫人,你如何知道我的名字?''卢夫人道:``当日事情过后,聂锋便告诉我了。你的名字则是他后来打听到的。聂锋此人,虽然从贼,尚知是非。我也曾屡次劝说过他,料他迟早必会弃暗投明。你可是知道了他的心迹,才投到他的家中来么?''铁摩勒道:``这倒是一件巧遇,并非事前约好的。''当下便将巧遇聂锋之事,约略说了。

卢夫人道:``聂锋虽然肯庇护你,但今日城中,已是安贼天下。虎穴龙潭,究竟不是安身之所,你还是早早离开为是。''

铁摩勒道:``我来此不过一日。夫人,你身在虎穴龙潭,已经过了十年了,为何你又不想离开?''

卢夫人双眉微蹩,低声问道:``摩勒,你可是想救我出去么?''

铁摩勒道:``我心有此念,但我已答应了聂锋,不忍连累于他。我是想等待段大侠到米,由他救你出去。''

卢夫人忙道:``你快点送信给圭漳,叫他切不可轻举妄动。

现在还不是我离开薛家的时候,他若来了,对我有损无益。我也决不会随他走的。''

铁摩勒大为不解。问道:``这却是为何?''卢夫人道:``依你看来,朝廷要袭灭安贼,是易是难?''她不答复反而突然问了一句``题外''之话,铁摩勒更是不解,怔了一怔,答道:'中原沦于夷狄,安贼之势已成。要袭灭他,谈何容易?不过所幸民心都是痛恨赋人,失民者亡,安贼这江山总是坐不稳的,只是迟早而已。''

卢夫人道:````我留在贼窟,为的就是早日促使安贼败亡!以前我还只是为报私仇,现在则是兼报国仇了。你想我如何能够离开!''

卢夫人是个柔弱的女子,但说这几句话时却是英气迫人,令人血脉愤张,胸怀激动。铁摩勒正待问她,卢夫人已又说道:``不久长安必有大事发生。你听我的话快点走吧,叫圭漳也切不可来。''

铁摩勒道:``'我与段大侠也并非约好在此相会的。只是我知道他会来,所以在此等他。''

卢夫人道:``这就糟了。但愿他越迟来越好。还有,你想留在此处,就不可随便找我。我若有事要你帮忙,会叫红线送信给你。''

铁摩勒正想问她可能有什么事情发生,与及她又怎样准备报仇,那两个女孩子已经蹦蹦跳跳地走回来了。

她们一回来就嚷道:``叔叔,我们等着瞧你的剑法啦!''

铁摩勒只得应允她们,拔出剑来,笑道:``你们既然一定要看,我就只好献拙了,要是练得不对,你们也得给我指点。''她们虽是孩子,但在铁摩勒眼中,却把她们当作行家看待,认真的施展出来,一招一式,丝毫不敢含糊。

铁摩勒施展的是八八六十四手龙形剑法,这一套剑法,走的全是阳刚路数,剑势雄劲异常,使到疾处,端的是进如猿猴窜枝,退若龙蛇疾走,起如鹰隼冲天,落如猛虎扑地,夭矫变化,不可名状,不可捉摸,剑光霍霍,剑气纵横,方圆数丈之内,沙飞石走!

聂隐娘与薛红线的剑术是以柔克刚的路数,讲究的是轻灵翔动,自不苦铁摩勒这套剑法的雄悍迫人。双方路数不同,却都是上乘剑法。在铁摩勒看来,她们的剑法是美妙之极;在她们看来,铁摩勒的剑法也是好看煞人!而且她们比不得铁摩勒,铁摩勒是多见识广,她们则是除了本身所学的这套剑法之外,还没有见过其他的上乘剑法,所以更是看得目眩神迷,如痴如醉。

铁摩勒正自使到最后一招``神龙摆尾'',忽听得一个银铃般的声音喝彩道:``好剑法!''

这声音熟悉非常,铁摩勒心头一震,长剑划了一道圆弧,倏的收招,抬头看时识见一个少女已站在场边,可不正是王燕羽!

四目交投,两人相对,都感到了意外相逢的惊奇;这刹那间,双方的神情都有点尴尬,不知说些什么才好。

薛、聂二女拍手赞道:``叔叔,你的剑术真行,你听,不只是我们赞你,王姐姐也赞你了。''这两个女孩子和王燕羽很亲热,一人一边,拉着王燕羽的手便走过来,边走边说道:``这位王叔叔是新来的客人,本领好得不得了,可是就是有点不老实,他起初还推说不会,老是和我们客气呢。''

王燕羽定了定神,笑道:``大人怎像你们孩子,你们懂得一点皮毛,就到处夸口,大人就不是这样了。这不是装假,这叫做谦虚。''接着装作不认识铁摩勒的模样,大大方方的拉沃一礼,说道:``原来你是新来的客人,还未请教高姓大名。''

铁摩勒只得假戏真做,还了一礼说道:``小可姓王名小黑,是从乡下出来,投靠乡亲的。乡下人不懂礼貌,小姐,你别见怪。''

聂隐娘道:``我们这位王姐姐的武功以,本明得很呢,她常常来这儿指点我们的,你们要不要比试比试?''

卢夫人自从这两个女孩子出来之后,就一直没有与铁摩勒说过话,这时忽然插嘴说道:``这位王小姐是鲁国公讳伯通王公爷的掌珠,王公爷和薛大人、聂大人同为一殿之臣,也都是通家之好。王小姐身为公侯千金,却最是和气不过,和上下人等都不''

拘礼的。''

卢夫人这几句话实在是点明王燕羽的身份,好叫铁摩勒小心在意的。铁摩勒听了,心里想道-'原来王伯通还在长安,而且受安禄山之封,做了什么`国公'了。如此说来王燕羽还未曾劝得她的父亲金盆洗手、闭门封刀。''

王燕羽笑道:``多谢卢妈夸赞。不过她的话也有失实之处。

不错,我对人是不分上下,但也要那个人对我好,我才会对他好。''说话之时,有意无意地限了铁摩勒一眼。

这时,聂隐娘还在缠着铁摩勒与王燕羽要他们二人比试,铁摩勒听了卢夫人的话,便佯装一惊,说道:``原来是一位侯门小姐,小可只是一介乡民,如何敢与小姐比试?''

王燕羽也笑道:``你别听这两个孩子瞎说,我这几手三脚猫的功夫,和小孩子玩耍还可以,怎敢和壮士比武?''

聂隐娘见他们两人都执意不肯,好生失望,她年纪较大,不好意思再缠,但薛红线却还不肯罢休,又拉着王燕羽说道:``你不肯比试,那也罢了,你上次答应教我们的点穴功夫,现在可以教了吧?''

王燕羽道:``我今天只是走来看着你们练剑练得如何了的。

我上次不是说过了么,要学占穴。先得指头有劲,也就是要懂得怎样运用内劲才成。这要待你们的剑术练很有火候了,才能够再学点穴的。好在你们已经有了这位叔叔,你们先叫他多指点一些运劲使剑的法门吧。''卢夫人也道:``红线,你不要再缠王小姐了。你看,天也快将黑了。你再不回去,我可没法子在你妈跟前交代啦。''

王燕羽跟着说道:``对啦,你还是听卢妈的话回家去吧。我今天也还有事情,不能够和你们再磨下去啦。''

聂隐娘忙道:``王姐姐,你什么时候再来?''王燕羽道:``我要来的时候自然会来,只要是我喜欢的人,我自然会来见他的。说不定明天就来看你。''说话之时,又有意无意地脱了铁摩勒一眼。

铁摩勒心头一震,一时呆了,竟忘记给王燕羽送行。王燕羽走了两步,又回过头来似笑非笑地说道:``这个年头,只见人们从长安逃出去,少见有人到长安来。王相公,难得你这个时候却到长安来。外面乱糟糟的,你可得当心些才好啊。可惜我现在就要走了,我倒很想向你打听打听长安外面的情形呢。''

卢夫人暗暗吃惊,心道:``莫非她已看出了破绽?''聂隐娘抢着说道:``王叔叔已对我说过,他不会这样快走的。王姐姐,你明天就来吧。''铁摩勒只得和她客套几句,请她约个日期,王燕羽笑道:``我要来的时候,自然会来的。'说罢,就自己打开园门走了。

看来她是薛聂二家的常客,已到了熟不拘礼的地步。

王燕羽走后,卢夫人也带了红线回家,他们二家比邻而居,有角门相通,甚为方便,卢夫人不便再与铁摩勒说话,但她委实放心不下,``走出角门之时,故意大声说道:``快点走吧!''似是在催促孩子,但铁摩勒当然知道这话是对他说的。

铁摩勒心乱如麻,琢磨王燕羽临走时对他说的那番话,心里想道:``她已说过不愿见我的了,怎的她又说要来?还有,她要我当心,这又是什么意思?看来,这并不是寻常的嘱咐。''

聂家的老管家殷勤招待,当晚给铁摩勒备办了丰盛的接风酒,以下人的身份伺候他,铁摩勒好生过意不去,拉他坐了下来,一同喝酒,口口声声尊他``老伯'',这管家起先局促不安,但见铁摩勒甚是随和,丝毫不拿架子,喝了几杯,也就渐渐惯了。

铁摩勒瞧他已有了几分酒意,说话也渐渐多了,便问他道:``你家小姐真是将门虎女,巾帼英雄,难为她小小年纪,这套剑法也不知是怎么练出来的?聂将军南征北讨,想必在家的日子不多吧?''那块家道:``说来这倒是一件奇事,我家小姐的剑术不是她父亲教的。她三岁那年,在门前戏耍,有个尼姑路过,便进来求见夫人,夫人以为她是化缘,哪知她却说道:'这位小姑娘根骨甚好,我想收她做徒弟。'夫人当然不肯,那尼姑说道:''你不肯我也要把她带走的。'果然那天晚上,门户紧闭,小姐还是和夫人同一床睡的,半夜里却失了踪。夫人哭得死去活来。过了几天,老爷回来,听得夫人诉说,他问明了那尼姑的相貌,反而安慰她道:'这位尼姑是世外高人,求也求不到的,她肯收隐娘为徒,那是隐娘的造化,你哭什么?''

听到这里,铁摩勒连忙问道:``你可知道那尼姑的法讳?''老管家道:``我家主人没有说,但听他的口气,想必是知道这尼姑的来历的,不过我不敢打听。过了五年,小姐八岁,那尼姑方始将她送回。据说那老尼姑已将她脱胎换骨,打好了根基,可以自己练武了。这以后,那老尼姑大约每年来一次,夫人对她的态度亦已大大不同,每次到来,都接她到内室亲自款待,我虽是管家,等闲也见不到她。''

铁摩勒问道:``那么薛姑娘的剑术是否也是那老尼姑教的?''

那管家道:``我也曾听得薛姑娘叫那老尼姑做师傅,不过,薛姑娘从小在薛家长大,未听说她失过踪,也许她是跟着我家小姐叫的。我们这两家也是近几年才作邻居的。''铁摩勒道:``这两个小姑娘倒像是亲姐妹一般。''那管家道:``是呀,红线姑娘聪明伶俐,薛将军夫妇也很疼爱她的。''铁摩勒笑道:``父母当然疼爱子女,这何须说?''那管家已有了几分酒意,低声说道:``王相公,你不是外人,说给你听无防,那小姑娘不是薛将军的亲生女儿,听说她的父亲本来是唐朝的官儿,给当今皇上暗地里害了的,那时皇上还是三镇节度使,薛将军在他麾下,那小姑娘还是未满一岁的婴儿呢。薛将军见这孤女可怜,向皇上求情,将她收养下来的。哎呀,这些话本来不应该讲的,你知道了可别向外人说。''铁摩勒道:``老伯放心,我守口如瓶,绝不会泄露半点。''这管家哪里知道,铁摩勒对这原名史若梅、今名薛红线的小姑娘的身世和遭遇,比他知道得更清楚,更详细。铁摩勒看到卢夫人对薛红线的态度,早已怀疑是她的女儿,现在更是得到了证实了。

这顿饭足足吃了一个时辰,铁摩勒想要知道的薛、聂二家情形,也差不多都已打听得一清二楚,不过他为了免使卢夫人受嫌疑,却从未问过她的事情。晚饭过后,已是将近二更时分,那老管家带铁摩勒回房安歇。

铁摩勒所住的客房靠近花园,官家规矩,内外有别,客房和聂家内眷所住的内房有几道隔开,距离颇远。老管家将他当作贵客招待,怕他要人使唤,亲自来伺候他,铁摩勒住在楼上,他就住在楼下。

铁摩勒心绪不宁,哪里睡得着觉。心里在想:``卢夫人不肯离开,又不许我去找她,我该不该再住下去呢?想不到王燕羽竟是常常来这两家串门的客人,我在这儿,已经给她知道,只怕住下去会有麻烦。''铁摩勒是早已相信王燕羽不会害他了的,他倒不是怕她告密,而是怕她纠缠。``空空儿托我向段姑丈报信,段姑丈迟早会寻到这里来,我若离开这儿,更不易见得着他了。''又想:``卢夫人说日内将有大事发生,却不知是什么事?我不如多住几天,她若要人帮忙,我可以给她尽力。''

铁摩勒正在东思西想,迟疑莫决的时候,忽听得窗外``卜''的一声,那两扇窗门开了,露出一个少女的面孔,正是王燕羽在向他窥视,比他预料的来得更早!

铁摩勒吃了一惊,结结巴巴地说道:``你,你,你怎么三更半夜,到这里来?''王燕羽笑道:``你放心,没人瞧见的。那老管家已是烂醉如泥,我还不放心,又点了他的昏睡穴,不到红日高升,他是绝不会醒来的了。''\,''

铁摩勒道:``你有什么事情,明天来不行吗?哎呀,你,你不懂我的意思。''王燕羽呆了一呆,脸上忽地泛起一片晕红,嚷道:``原来你是避男女之嫌么?哼,你把我当作什么人了?我虽出身绿林,却还不是下贱的女子!''

王燕羽这么一说,铁摩勒也臊得满面通红斤好意思不开门让她进来了。王燕羽坐了下来,余怒未息,许久许久,都未说话。

铁摩勒赔罪道:``王姑娘,我是直心眼儿,不会说话,你别见怪。我只怕我们若是往来过密,给展大哥知道,可又要引起误会了。嗯,展大哥到处找你,你可知道么?''

王燕羽柳眉倒竖,说道:``我的事情,不用你管。倒是你自己的事情,你可要当心些。哼,我若不是不忍见你遭祸,我才不会来呢。你以为我是想见你吗?你放心,过了今晚,我是绝不会再来找你的了。''

铁摩勒道:``我有什么危险?难道是有人知道我到了长安,向安贼告密了么?''

王燕羽道:``安禄山现在正在大过皇帝痛,在宫里胡天胡地,什么事情也不管。但只怕还有别人,要加害于你!我先问你,你到长安来干什么?''

铁摩勒道:``来看看长安城里的群魔乱舞!''王燕羽道:``我知道你不会与我说实话,但我也猜到一二,是不是唐皇派你来行刺安禄山的?''王燕羽自负聪明,但这回她却是猜错了。

铁摩勒道:``哦,原来你是怕我自不量力,灯蛾扑火,自投罗网么?''王燕羽道:``有一个人,不知你可识得,他就是在三十年前,与我师公展飞龙齐名的火魔头------七步追魂手羊牧劳!''

此言一出,只见铁摩勒的面色陡然大变,双眼就似要喷出火来,怒声问道:``羊牧劳?这魔头居然还活在人世么?''

王燕羽也吃了一惊,说道:``敢情你是他的仇家?怪不得他屡次向我父亲打听你。''铁摩勒定了定神,连忙问道:``这魔头现在哪儿?''

王燕羽道:``他就在安禄山的身边,安禄山已礼聘他为大内总管了。前日他还和我父亲说起你。''铁摩勒道:``哦,他说什么?

是否想要我的性命?''

王燕羽道:``听他的口气,他当真是要取你性命。他说,他说\ldots\ldots 哎,总之没有好话,你可真得当心。他已经知道你离开唐王了,他也正在猜度你会到长安来呢。''原来前两日当羊牧劳与王伯通谈及铁摩勒时,正巧王燕羽也在旁边,当王伯通说到大破飞虎山的往事,羊牧劳就拍案叫道:``可惜,可惜,你杀了窦家五虎,怎的斩草却不除根,让铁昆仑那小杂种走了?''王伯通道:``当时是为了卖空空儿的面子,后悔也来不及了。这小子已跟磨镜老人学了一身武艺,事事与我作对呢!''羊牧劳道:``王见不必烦忧,这小子我也容他不得。听说他已给唐王驱逐,我怀疑这是苦肉之计。''王伯通道:``苦肉之计?难道他敢来投降咱们的皇上?''羊牧劳道:``或者不敢假意投降,但可能混人长安,图谋行刺。''王伯通道:``我的手下许多人认得他,我叫他们留心侦察便是。只是若然查到了他的行踪,还得我兄亲自出手才成。''王燕羽因为怕提起飞虎山的往事,又怕铁摩勒对她的父亲仇恨更深,故此没有详细描述他们的对话。

王燕羽正是为了怕铁摩勒去行刺安禄山,会碰上羊牧劳,这才不避嫌疑,来报消息,并劝铁摩勒离开长安的。

哪知铁摩勒听了,却是勃然大怒,拍案便骂道:``好呀,他想要我的性命,我也正想要他的性命呢!''

你道铁摩勒为何如此发怒,原来这羊牧劳乃是他的杀父仇人。

二十五年前,铁昆仑还在做燕山王的时候,有一天,他的山寨里来了一个客人,这客人便是羊牧劳。他和铁昆仑虽然相知不深,但因彼此都仰慕对方的武功,故此羊牧劳到来,铁昆仑当晚就盛筵招待。

酒至半酣,这两位武学大师不免谈论起武功来,羊牧劳道:``铁兄,你的外家功夫登峰造极,在掌力上可曾遇到过对手么?''

铁昆仑道:``老兄号称七步追魂手,在老兄面前,我就相形见细了。''言下之意,论到掌力,天下英雄,``唯使君与操耳''。

羊收劳哈哈大笑,说道:``铁兄过誉了,咱们一个是外家掌力,一个是内家掌力,只怕难分高下呢。''铁昆仑自认不如,羊牧劳却只说是``难分高下'',语气显然是比铁昆仑高做得多。

铁昆仑自认不如,这不过是谦逊之词,当时有了几分酒意,便邀羊收劳比试。哪知羊牧劳正是有心前来,要挑动他比试的。

这``比试''二字,先由铁昆仑口中说出,正合他的心意,但他还故意作态,皱着眉头说道:``咱们所学不同,原应彼此切磋,但我却有一点顾虑。铁兄,你的外家掌力至猛至刚,小弟的内家掌力,亦有几十年火候,非敢自夸,至今也还未碰过对手,倘若有所误伤,伤的是小弟,也还罢了,伤及老兄那却如何是好?''铁昆仑酒意已浓,听了这话,更不舒服,立即哈哈大笑道:``老兄尽可不用顾虑,久仰老兄七步追魂,小弟还真想试试呢。莫说误伤,即是当真给你追了魂去,我也决不怪你。''

当下两人就在筵前比试,山寨的大小头目,环立四周,屏息而观。但见铁昆仑叱咤风生,每发一掌,屋瓦随落,墙壁也似乎震动起来;羊牧劳却是气定神闲,身随掌转,每发一掌,必定移动一步,或前或后,或左或右,式式不同,招招变换,掌力发出,毫无风声,但站得稍近的人,却都感到有一股潜力迫来,不由自主的要向后退。座中的行家可以看得出来,论功力两人都已登峰造极,但羊牧劳以灵活的步法消解对方的力道,却有点取巧,因之也似乎稍稍占了一点便宜。

双方拼到了第七掌,羊牧劳一个转身,反手拍出,双掌忽地胶住,但见两人都是汗如雨下,过了半晌,铁昆仑笑道:``小弟侥幸未给追魂,咱们可以罢手了吧?''羊牧劳道:``老兄接了我的七步七掌,彼此都未受伤,是不必再强分胜负了。''

旁观的头目松了口气,都觉得这样收场,双方都有面子。哪料就在双方收掌这一瞬间,忽听得铁昆仑大叫一声,跃出了一丈开外。

羊牧劳作出了大吃一惊的样子,叫道:``铁兄,你怎么啦?伤在哪里?小弟有药。''铁昆仑一个鲤鱼打挺翻起身来,圆睁双眼喝道:``羊牧劳,你别假惺惺啦!待我伤好之后,还要领教你的真实功夫!''他虽然能够起身,但听他的声音中气不足,显然已是受了内伤。

旁观的头目明明看见两人功力悉敌,铁昆仑却忽然莫名其妙地受了重伤,再听他的口气,不由得都怀疑他是受了羊牧劳的暗算,当下便有几个忠心耿耿的部下,亮出了兵器来,向羊牧劳喝骂。

羊牧劳冷笑道:``铁兄,你怎么说?先前的话还算不算话?''

铁昆仑挥手道:``让他走,不必你们替我报仇!''

羊牧劳还故意叹了口气,说道:``铁兄,我一时失手,后悔莫及,想不到你竟把我当作仇人。我没法子,只好走了。望你早点康复,我再来请教。''

铁昆仑练有金钟罩的功夫,众头目还以为他只是受了点伤,料无大碍,哪知他当晚就寒热交作,从此一病不起,竟不能够亲自向羊收劳报那一掌之仇了。

原来他与草牧劳虽然功力悉敌,但羊牧劳练的是内家掌力,在双方同时收掌之时,铁昆仑的阳刚掌力是一撤便即收回,而羊牧劳则暗地里用上了阴劲,收掌之后,他的劲力还未消散,突然乘虚攻人,破了铁昆仑的金钟罩,且伤了他的三焦经脉。这可说是``暗算'',但却非明显的暗算,因为这是他掌力上另有奥妙之处,所以当时铁昆仑也只好怪自己过于疏忽,太过把他当作朋友看待,吃了哑亏,说不出来。

铁昆仑死后,他的部下当然要给他报仇,侦骑四出,可是草牧劳早已不知去向了。官军趁着铁昆仑之死,而几个大头目又出去追凶的时候,便乘机攻破山寨。可怜铁昆仑在燕山经营了几十年的基业,毁于一旦,而铁摩勒也成了孤儿,后来才得窦家收为义子。

攻破山寨的是幽州道行兵总管苏秉,事后铁昆仑的部下方始得知,原来这羊牧劳便是受了苏秉的重托来暗算铁昆仑的,苏秉立了此功,官升三级,不在话下。但苏秉也不过只得意了几年,后来铁摩勒的义父窦令侃亲自率领陵兵,攻人幽州,终于把苏秉杀了,算是给铁昆仑报了一半仇。这也是铁摩勒为什么将窦令侃视同生父的缘故。

羊牧劳仍是不知下落,这当然是因为铁昆仑交游广阔,他怕铁家的亲友寻仇,所以藏匿起来。窦家因为要与王家争夺绿林霸权,也无暇去寻觅他。

铁昆仑与磨镜老人交情甚厚,临死之时,曾嘱咐部属要将儿子送到磨镜老人门下学艺报仇,但又因磨镜老人行踪无定,直到过了十多年,铁摩勒与段圭湾在长安巧遇南雾云,这才由南雾云将他引人师门,这时飞虎寨亦已给王伯通灭了。

铁摩勒在磨镜老人门下八年,在第五个年头,磨镜老人有个朋友从突厥(即今新疆及青海一部)回来,据他说羊牧劳已在突厥死了,而且他还曾亲自参加羊牧劳的火丧之礼。这位朋友乃是武林七奇之一的玄空子,磨镜老人与铁摩勒都相信他决不会乱说假话,故此铁摩勒出师之后,念念不忘的只是给义父报仇,而以为父亲的仇人已死,根本无须报了。

哪知现在听王燕羽所说,羊牧劳竟还未死,而且还做了安禄山的``大内总管''!

惨痛的记忆给挑了起来,铁摩勒禁不住泪咽心酸,泪眼模糊中,现出了他父亲的影子,满面血污的愤怒神情,语语悲凉的临终嘱咐\ldots------仇恨的火焰重新从心中燃起,铁摩勒咬牙切齿地说道:``羊牧劳他在这儿?好呀,他在这儿,我就偏不离开长安!''

王燕羽吃了一惊,说道:``摩勒,我不知道你与羊牧劳有何冤仇,但我却亲眼见过他绵掌击石的功夫。那一天,他在御花园中,当着安禄山和许多武土面前炫技,十几块石头堆在一起,他说他只要打碎当中的一块石头,说罢,轻轻一掌拍下,那一堆石头纹风不动,然后他叫人将石头一块块搬开,果然周围的石头都是原状,只有当中的那块石头,一触即成粉碎!嗯,看来他这手功夫,不在我师父之下!摩勒,我不是小觑你的功夫,只怕,只怕铁摩勒是武学行家,当然知道这手绵掌击石功夫的厉害,心想:``如此看来,这魔头的内家掌力确是不容轻视,若然一掌打下,所有的石头全都碎裂,那还容易,现在他能够随心所欲,任意打碎当中的一块石头,这内家掌力,已是到了登峰造极的境界!''

但铁摩勒虽是吃惊,却仍然沉声说道:``就算他是石头,我是鸡卵,我也得碰他一碰!''

王燕羽柔声说道:``摩勒,看来你与他是有不共戴天之仇,我本不该劝你,但俗语说得好:留得青山在,不怕没柴烧。我不敢说你就比不过他,但现在长安,你是孤掌难鸣,而他却是羽翼众多。''

铁摩勒望了她一眼,见她忧急焦虑的神情现于辞色,哪里像是仇家的女儿?简直像似一个非常关心他的姐妹,心中大为感动,一时间竟说不出话来。

王燕羽又道:``摩勒,作即算是恨我也好,我却不忍见你受到任何伤害,你倘若要留在长安,我只有一件事情求你,求你不要孤身冒险,去行刺安禄山、''她的意思铁摩勒理会得到,她不敢劝铁库勒放弃报仇,但只要铁摩勒不入宫行刺,那就当然没有机会碰到羊牧劳了。

铁摩勒道:``好,我答应你。我决不单身行刺就是。天快亮了,你走吧!''

王燕羽含着幽怨的目光,凄然一笑,说道:``摩勒,你不必赶我,我也要走了。你放心,以后我再也不会单身见你。''说罢,便跳出了窗子,再不回头。铁摩勒不自禁地倚着窗儿,望着她的背影在深沉的夜色之中消失。正是:燕子穿帘来又去,可怜爱恨总难消。

欲知后事如何?请听下回分解------

旧雨楼扫描,YackerOCR,旧雨楼独家连载

\chapter{第三十二回 虎穴藏身思报国
绣闺夜话识深心}\label{ux7b2cux4e09ux5341ux4e8cux56de-ux864eux7a74ux85cfux8eabux601dux62a5ux56fd-ux7ee3ux95faux591cux8bddux8bc6ux6df1ux5fc3}

铁摩勒虽然报仇心切,但却也非鲁莽之徒,王燕羽走后,他渐渐冷静下来,仔细一想,王燕羽说的确乎有理,在这个群魔乱舞的长安,自己孤掌难呜,确足不宜露面,更不用说入宫行刺了。

心里想道:``报仇也不争在早这几天,且待姑丈到来再说。''铁摩勒在磨镜老人门下八年,以内功和剑术造诣最深,他衡量一下自己与羊牧劳的武功,估计可以接得下他的绵掌,但要想取胜,却是万难。倘得段圭樟夫妇相助,报仇或者有望另有一件令他挂心的是卢夫人,卢夫人不肯离开薛家,原因不说,只预言将有大事发生,听她的口气,似乎这件事的发生,对她也不无危险。日间言犹未尽,铁摩勒很想再找个机会去见她,但卢夫人又不许他前往薛家,铁摩勒只好等待她和红线再来。

可是此后一连几天,非但卢夫人和红线没有过来,聂隐娘也没有再来缠他练武,铁摩勒暗暗纳罕。官宦之家,内外有别,他当然也不方便退进内房去向聂隐娘打听,只好天天陪那老行家闲聊。薛嵩、聂锋仅是安禄山当作心腹的大将,这老管家对安禄山的家事倒知道得不少,据他说安禄山的次子,即现在被立为``太子''的安庆绪生来愚蠢,安禄山本来不喜欢他的,只因大儿子安庆宗在他造反的时候,还留在长安作唐室的郡马,给唐玄宗杀了(事见前书),所以才个得不立他为``太子'',他们父子二人一向不大和好。铁摩勒听过就算,并不放在心上。

大约过了五六天,这一天,聂隐娘忽然又到铁摩勒的房间来,要铁摩勒陪她到花园练剑,铁摩勒自是欣然答应。到得花园,只见薛红线已经先在那儿,一见铁摩勒,不待他问,便先说道:``王叔叔,我早就想过来的,只因卢妈病了,我舍不得离开她,功夫也丢荒几天了。''聂隐娘跟着笑道:``王叔叔,你不知道,那卢妈简直比她的亲生母亲还更疼她呢。她对卢妈也像对母亲一样孝顺。卢妈虽是乳妈,却懂得诗书,我这几天都与薛妹妹陪她,也叨光得她教我读了半部诗经呢。''铁摩勒听得卢夫人病中还能教孩子读书,料想只是小病,而看薛红线今天欢喜的神情,想必她的病亦已经好了。

这两个女孩子要铁摩勒再指点剑术,铁摩勒却有心想识她们的渊源派别,当下说道:``指教二字我不敢当,我的剑术和你们的路数不同,不如你们先把你们所学的全套练给我看,咱们才好彼此琢磨,互相增益。''薛红线道:``这样也好,但我的剑术是聂姐姐教的,我还未学会全套呢。聂姐姐你来练吧,让我也在一边学学。''

聂隐娘笑道:``红线,你怎么说起谎来了?我可要告诉卢妈去,叫她教训你一顿。''薛红线道:``我几时说谎了?''聂隐娘道:``还不是说谎吗?你的剑术不也是师父教的吗?她上次还夸赞你悟性最好呢!''薛红线道:``师父每次到来,都不过是住十天八天,我跟她学剑的日子,总共加起来还不到三个月,最初只学剑诀,招数都是你代为传授的,这套剑法到现在也确是尚未学全,怎能说我说谎?''

铁摩勒故作惊诧,说道:``哦,原来你们另有师父,我只道你们是家传的剑法呢。你们的师父是谁?''

聂隐娘沉吟片刻,说道:``叔叔,你不是外人,但我师父吩咐过我不许将她的名字胡乱对人说的。''

铁摩勒道:``那你就不必说了,只把她所教的剑法练给我看吧。''

聂隐娘在兵器架上挑了一把短剑,立了一个门户,目光直注剑锋,略一盘旋,便见剑光如练,直荡出周围丈许远近。倏然间,身形一晃,身随剑走,越展越快,但见剑光线绕,忽东忽西,忽聚忽散,当真是翩若惊鸿,宛如游龙!舞到急处,又如水银泻地,花雨缤纷,好看煞人。

铁摩勒看得暗暗奇怪,看她这套剑法与王燕羽的剑法似乎是同源异流,王燕羽的剑法比较刚健,聂隐娘的剑法则偏于阴柔,极得轻灵翔动之妙,外形虽异,但在行家眼中,却可看出是同出一源。不过,若只就剑法而论,聂隐娘这套剑法却要比王燕羽高明得多。变化的精微奥妙之处,实不在空空儿那套袁公剑法之下。

铁摩勒正自猜疑,忽见那老管家匆匆忙忙的走来,叫道:``小姐,小姐\ldots\ldots{}''聂隐娘正好将这套剑法使完,当下收剑凝身,满不高兴地问道:``什么事情,你不见我正在练剑吗?我还要请王叔叔指点呢?''

那老管家们怕说道:``外面来了一个老婆子,凶得很,她说要见什么妙慧师太,我说这里没有这个人,她说没有这个人就要见小姐,她硬闯进门,走一步就在石阶上留下一个足印,家丁们不敢拦阻她,请问小姐你是见她不见?''那老管家一面说话,一面眼睛里人铁摩勒,似乎是想请铁摩勒帮她拿个主意。

聂、薛二女都现出惊诧的神情,同声问道:``这老婆子要见妙慧师太?她可有说她是什么人吗?''老管家道:``她没有说。''聂隐娘年纪较大,想了一会,便对铁摩勒道:``她这么凶,我倒想去见见她,王叔叔,你跟在后头,要是她欺侮我,你可得帮我。''

铁摩勒笑道:``真有本领的人,是不会欺侮孩子的,你们要我同去也行,不过我是个不相干的外人,却不方便露面。不如这样吧,你去见她,我藏在屏风背后,先听听她的来意再说。''

薛红线拍掌道:``好,有你壮胆就行。聂姐姐,咱们一同去。

我不怕她凶,我才恨不得她凶呢。咱们练了这几年功夫,正好试试。''说罢在兵器架上挑了一把短剑,藏在身上,又对铁摩勒道:``王叔叔,你可不必先忙着出来,待我们真的打不过她了,你再帮忙。''看她一副跃跃欲试的神情,就像巴不得这场架打起来似的。

铁摩勒摇了摇头,笑道:``红线,一个女孩子可不该喜欢打架啊。你们应该先和和气气地问她,纵算她再凶,也不会先动手打孩子的。''

薛红线嘟着小嘴儿道:``她和气我便和气,干嘛要我们去奉承她。''

聂隐娘与薛红线手挽着手走进客厅,只见一个相貌凶恶的老婆子太马金刀地坐在当中,发乱如草,一对眼珠似金鱼般地凸出来,活像大人吓孩子时,所说的故事中的``妖婆''模样,聂、船二女虽然胆大,也不禁打了个哆喷,薛红线颤声嚷道:``你是谁?你为什么要找妙慧师太?''

那老婆子双眼一翻,直上直下地打量了薛红线一番,忽地毗牙咧嘴地笑道:``瞧你的眼神,你的姹女功也颇有点根底了,怎么,你也是妙慧的徒弟么?妙慧可真好福气,怎的一下子就找到了两个根骨上佳的徒弟,可真羡煞我了!''笑声极为难听,有如鸥鸟夜啼,听得叫人皮肤起粟。

铁摩勒躲在屏风背后,这一惊比那两个女孩子更甚,这老婆子不是别人,正是王燕羽的师父展大娘!

聂隐娘比较镇定,说道:``婆婆,你找错人家了。我家姓聂,我爹爹是带兵打仗的,家中可没有什么妙慧师太。''

展大娘碟碟笑道:``我知道你是聂锋的女儿,你爹见了我也要自称晚辈呢!你年纪轻轻,倒会说谎,你说妙慧不在这里,为什么你的妹妹又问我为什么找她?快说实话,妙慧是你们的师父不是?''

薛红线道:``我不说给你听,我师父不许我们对人说的。''

展大娘大笑道:``哦,原来妙慧还有这样的戒条。哈,小!''

娘,你不说我就试不出来吗?''笑声未了,薛红线忽觉微风飒然从身边拂过,腰间所佩的短剑已被展大娘取去。

展大娘倏的转身,并未拔剑,连着剑鞘,就向聂隐娘一剑搠去,叫道:``小丫头,小心接我这招夜叉探海!''

聂隐娘年岁较大,应变也比较机灵,在薛红线的佩剑被夺之时,她的佩剑已经亮出,正好及时招架。

展大娘先叫出剑招的名字,聂隐娘不假思索的便是一招``玉女穿梭''的还击过去,原来在她师父所授的剑法之中,这一招``玉女穿梭''正是破解展大娘那招``夜叉探海''的唯一招数,她平时早已练得十分纯熟,不过,若非展大娘预先点破,她毫无临敌经验,还不会这样快施展出来。

但听得``当''的一声,聂隐娘的短剑竟被展大娘带鞘的剑削断,展大娘哈哈笑道:``小姑娘,你们还不知道我是谁吗?''

铁摩勒早已看出展大娘乃是有心试招,这时也已看出了展大娘与聂、薛二女的师门大有渊源,但那薛红线还是个不懂事的女孩子,这时却急得叫起来道:``王叔叔,你快出来呀,我们都打不过她了!''

展大娘面色一沉,说道:``哦,原来你们还有一位王叔叔么?

他是准,我倒要会他一会。''铁摩勒在屏风背后大吃一惊。展大娘不见有人出来,便要闯进内堂搜索。

忽听得一声叫道:``师父,你怎的到了这儿?''王燕羽走了进来,正好赶上。

展大娘双目一瞪,喝道:``燕羽,你还认得师父吗?''燕羽道:``师父息怒,那天出走,是元修哥哥的主意。''

展大娘冷笑道:``好呀,原来你们早已做了一路,联起手来反对我了。我的展儿呢?你叫他来,我要问他还认不认我这娘亲?''

展大娘虽然声色俱厉,但王燕羽与她相处多年,哪会不知道她的心意,立即说道:``师父放心,元修哥哥无恙,他对你老人家也是始终孝顺的,不过他不在这儿,你想见他,还得待些时日。''

展大娘``哼''了一声道:``我才不想见他呢!''但紧跟着又问道:``他在哪儿?''

薛红线不知好歹,这时惊魂稍定,忽地打岔道:``王姐姐,这个凶婆子竟是你的师父吗?''又叫道:``王姐姐来了,王叔叔你怎么还不出来?''

展大娘道:``你和这人家很熟吗?你的师伯你见过没有?还有那个王叔叔是谁?''

王燕羽笑道:``\,'师父你这一连串问题,叫我先回答哪一个好?

嗯成还是先说元修哥哥的事吧c不过,说来话长,这里不是谈话之所,师父,请你屈驾到我家来。我爹爹也渴念着你呢!''

展大娘心意踌躇,欲走不走,王燕羽赔笑道:``师父,你老人家还在生我的气吗?''展大娘``哼''了一声,道:``我才没闲功夫和你生气呢!''王燕羽道:``那么,咱们走吧!''展大娘一拂袖子道:``且慢,你何必这样着急催我?我既到了此间,未曾打听得到你师伯的下落,怎能说走便走?''王燕羽笑道:``这个你问我好了,咱们边走边说吧。你不知道,我正有许多话要告诉你呢,见着了你,怎能不急?妙慧师伯确是不在此间,她惯例是每年冬至之后才来,大约住过了元宵便走的。现才刚是入冬,你来得早了。''展大娘心想:``此话可信,师姐虽然与我不和,但她若在此间,还不至于不出来见我。''其实展大娘也是渴欲知道儿子的消息,巴不得早点到王燕羽家中,向王燕羽仔细盘问的。现在既然知道了妙慧不在聂家,便不再踌躇,随王燕羽走了。

眼看展大娘已跨出门坎,藏在屏风背后的铁摩勒方才松了口气,忽见展大娘突然又停下脚步,问王燕羽道:``这两个小鬼头已得了你师伯的真传,她们刚才却要叫什么`王叔叔'来对付我,这`王叔叔'又是个什么样的厉害人物?''王燕羽噗嗤笑道:``这个王叔叔是个老家院,喝醉了酒挺会吹牛,又挺会骂人的,孩子们都不敢惹他,这两个顽皮的小鬼头想是要叫他出丑,所以才喊他出来。但这个酒鬼见了师父你这样凶,尽管平素惯会吹牛,这时还敢透半点大气么?恐怕早已躲到床底下去了,还会出来?''展大娘大笑道:``原来如此!''迈开大步便走,转眼之间,出了大门。

两个女孩子面面相觑,莫名其妙。聂隐娘道:``奇怪,王姐姐平日对咱们多好,今日却也帮着她的师父,骂咱们作小鬼头!王叔叔明明不是老头,又不是酒鬼,她这谎话是怎么编出来的?''

薛红线叫道:``王叔叔,你听见这些话没有?你当真是害怕得躲到床底下去了么?''铁摩勒哈哈大笑,走出来道:``王姐姐是为了你们好,你们却不知道。这个凶婆子是你们的师叔,你们胆敢对她不敬,王姐姐怕她责罚你们,所以才急急忙忙拉她走。骂你们一声小鬼头,不是已经从轻发落了吗?''聂隐娘吸着小嘴几道:``真没想到咱们有这么凶的师叔。这么说,王姐姐岂不是咱们的师姐了?她平日可从没说过。''薛红线也鼓起了腮道:``师父多疼咱们,这个师叔却一来就欺负咱们,脾气又凶人又难看,我才不想认她作师叔呢。王叔叔,你刚才为何不敢出来,教人笑话?''

铁摩勒笑道:``她到底是你们的师门长辈,我怎好和她打架?''聂隐娘年岁较长,懂事一些,也附和道:``不错,王叔叔若和她打架,打赢打输都不好。打输了固然自己吃亏,打赢了,王姐姐的面子过不去。''

这两个女孩子吱吱喳喳的谈论了一会,各自散了。铁摩勒的心上可是压上了一块石头,只怕展大娘再来,察破他的行藏,要想避开她,长安虽大,却是无处立足。而且父仇未报,就此离开,心亦不甘。

幸而过了几天,展大娘和王燕羽都未有再踏进聂家。铁摩勒猜想定是王燕羽不知用什么法儿将她绊住了。

这几天,聂隐娘和薛红线天天找他练武,他教这两个女孩子如何运劲使剑,而每天看着她们练剑,自己也得到了一些好处。

他和这两个女孩子更熟络了,只是卢夫人却一直没有露面。

这一天,他正在房中静坐,等候聂隐娘来叫他,忽听得屋外似有人马喧闹之声,不由得吃了一惊,心想:``难道是我的行藏已经泄露,安贼派兵来捕我不成?''

正自惊疑不定,忽听得聂隐娘的声音已在楼下叫道:``王叔叔,你快下来,我爹爹回来了。''铁摩勒一喜一惊,连忙下楼,与聂隐娘同去迎接。刚踏出二门,便迎着了聂锋与那管家。

聂锋刚刚回家,还无暇问那管家,只道铁摩勒养好了伤,已经走了,陡然见他挽着自己女儿的手出来,任了一怔,脱口便叫道:``铁------''一个铁字出口,方自想起铁摩勒已改了姓名,连忙转口说道:``铁骑军这次随我出征,想不到竟受了挫折,所以我这样快又回来了。王兄弟,你在这里住得惯么?''

铁摩勒见聂锋满面风尘,颇有优淬之感,心中一动,说道:``多谢这位侯管家招呼周到,比我自己的家中舒服多了。''

聂锋迟疑了一会,忽对女儿说道:``你进去告诉你妈,我要和王叔叔先叙一会。''说罢又吩咐那管家道:``'你给我拿这几包土产给夫人。若是有外客来找,你说我今天刚回家,明天才见客人。''

那管家颇为诧异,又暗自欢喜,心中想道:``幸亏我懂得巴结这王相公。老爷这次回来,竟不先进内堂会见夫人,可知他对这位王相公如何看重了。''

聂锋摒退左右,独自走进铁摩勒的客房,关上房门,便深深的叹了口气。

铁摩勒问道:``将军何事忧烦,果真是打了败仗么?''聂锋苦笑道:``幸免全军覆灭,但十停人马,也只剩下三停了。''铁摩勒道:``唐军是谁统领,如此厉害?秦襄、尉迟北二人可有出阵么?''

铁摩勒心里十分挂念这两个人,是故藉机探问。

聂锋又苦笑道:``若是败在这两人手下,倒还抢得。说来丧气,这次碰上的根本就不是正式的官军,只是乌合的民兵而已!

他们出没无常,每到夜晚,便从四面八方的袭来,天明又不见了。

我们压根儿就没有打过一场似模似样的仗,本钱便渐渐蚀光了。''

铁摩勒正容说道:``将军,这你应该欢喜才是。''聂锋诧道:``你这是什么意思?''铁摩勒道:``将军经此一败,当可明白,只是兵强马壮,仍不足恃。最紧要的还是要得民心。古语有云:顺民者昌,逆民者亡。将军明白了这个道理,化祸为福,不过转念之间耳!民气旺盛,胡儿势颓,将军若当机立断,则他年国土重光,将军也可善保禄位,这不是值得大大庆贺么?''

聂锋低下了头,沉思了一会,缓缓说道:``摩勒,现在还不是时候,暂且不谈。我想先问问你的事情,你可见过了卢夫人了?''

铁摩勒道:``初来之时,见过一面。''聂锋道:``她怎么说?''铁摩勒道:``如你所言,她不愿离开。''铁摩勒本欲把卢夫人的话告诉他的。但想了一想,仍然瞒住。

聂锋望了铁摩勒一眼,说道:``铁兄弟,你们是侠义道中人物,承蒙你和段大侠看得起我,把我当个朋友,我感激得很本来我担了天大的关系,也绝不能让你吃亏,但我不在家还好,我一回来,情形可又有点不同了。我心里担忧的,正是这件事似''

铁摩勒猜到了几分,故作不解,说道:``我还是不很明白将军的意思,既蒙将军许为肝胆之交,还望将军直言相告。''

聂锋道:``我不在家,外人个会到来。我一回来,同僚定会至此探望,问我前方的军情。你的踪迹,日子久了,恐怕难免泄露。

铁兄弟,你要见的人也已经见了,你留在长安,可还有其他事情么?''

铁摩勒心想:``原来他是怕我连累了他。''有点不悦c但转念一想,聂锋之所以暗示要他离开,也是为他着想。当下便道:''\,'将军既有为难之处,我明日告辞便是。''

正说到此处,忽听得管家在楼下禀报道:``薛将军请家主与王相公过去。''聂锋吃了一惊,低声说道:``他要见你,不去反而见疑,你镇定些,我陪你去一趟吧。''

聂、薛二家本来是打通的,当下,聂锋就领了铁摩勒从冷门过去,只见薛嵩坐在堂上,红线站在一旁。薛激一见铁摩勒,便站了起来,哈哈笑道:''\,'王小黑,我有眼不识英豪,当真是惭愧呀惭愧厂又拍拍聂锋的肩膊道:``还是你有眼力,看出他是个非常之人,保全了他的性命。''聂锋与铁摩勒都吃了一惊,但见薛嵩满怀高兴的神情,却不似含有什么恶意。

薛嵩请他们二人坐下,唤丫环倒上了茶,然后问道:``王小黑,你的剑法是跟谁学的?''铁库勒道:``是跟乡下一个教武馆的先生学的。他说我的资质可以学武,所以也照得比较用心。''薛嵩道:``如此说来,这位先生也是位遁迹山林的风尘异人了。''聂锋道:``这倒奇了,你刚刚回来,怎么就知道他的剑法了得?''薛嵩笑道:``令媛还未曾对你说吗?这些天来,王小黑天天都在指点她们的剑术呢。连隐娘和红线这两个丫头都盛赞他的剑术了得,那我就不必亲眼看到,也是可以相信的了。''铁摩勒心想:``原来如此,只是红线这一饶舌,不知要给我添上几许麻烦。''

薛红线哪知铁摩勒的心事,洋洋得意地笑道:``王叔叔,你不必回乡下老家去了。我叫爹爹给你一个官做,你就可长住这儿,和咱们作伴了。''

薛嵩道:``表弟,我正是为了此事要与你商量,王小黑是咱们的乡亲,又有一身武艺,我意欲将他提拔作我的亲兵住领,你可愿意放人么?''聂锋只得说道:``王小黑得你提拔,那是他的造化。

王小黑,你意下如何?''他以为铁摩勒必定婉辞推却的,哪知铁摩勒却立即说道:'小民何幸,得能将军栽培,那是求也求不到的。''

铁摩勒口中言谢,却并不拜跪,薛嵩心想:``到底是乡下人,不懂礼数。但这也足见他是个朴实的人,以后再慢慢教他规矩便是。''当下说道:``我已叫管家给你备好房间了。虽然两家相通,但你做了我的亲兵佐领,在我这边住较方便些。你的行李,我自会叫家丁给你拿来,你不必回去了。嗯,你还未见过夫人吧?''

铁摩勒怔了一怔,不知其意,据实答道:``我在聂将军家中,无事不敢过府,尚未曾得拜见夫人。''薛嵩道:``此后你是我的随身亲信兼充护院,就似家人一般了。你见见夫人吧。''说罢,便叫丫环去请夫人。

过了一会,只见一个华服妇人走出堂前,与薛嵩上下年纪;相貌甚是端庄,看来是个大家闺秀模样,铁摩勒心想:``薛嵩粗鄙残暴,却有这样的妻子,福气倒真不浅呢。''

当下,便上前见过,请了个安。

薛夫人已知这人是新来的护院,见他身材魁伟,器宇轩昂,心里暗暗喝彩:``他这次用人,倒是用得不错。''当下向丈夫笑道:``要不是你早就说过他的来历,我可要把他当作将门之子呢!''

薛嵩见妻子称赞铁摩勒,心里也甚欢喜,笑道:``将相本无种,男儿当自强。我的祖先也没做过官,我不是一样做到大将军么?王小黑,你好好的干,我担保你有一个锦绣前程。''铁摩勒只好又再欠身道谢。薛嵩笑道:``夫人,你称赞他相貌非凡,说来也有点奇怪,我初见他时,就觉得这人似是在哪里见过一般,心里就有点喜欢了,所以当时聂锋替他求情,我一口便答应了。''其实那时他根本未看清铁摩勒的相貌,发现似曾相识,这是后来的事。聂锋心头微凛,连忙说道:``他是咱们的乡亲,或许你小时候见过他,你自己记不得了。''薛嵩笑道:``或许如此,但这也算得是有缘的了。''铁摩勒十年之前曾在长安与薛嵩交过一次手,虽然是在混战之中,双方不过仅仅动了三招两式,但铁摩勒心上总是有着疙瘩,生怕给他看破,现在见他毫不起疑,心头大石,方始放下。

说话之间,忽有家人前来报道:``严夫人到!''薛嵩道:``是你的客人来了。她的丈夫现在正在大红大紫,难得她对你倒很有交情。''

铁摩勒见薛夫人有客,便先告退。薛红线道:``王叔叔我和你去看你的房间。''薛府管家陪铁摩勒同去,刚至回廊,一个丫环走来说道:``红线,卢妈叫你呢。她说,你应该做功课了。''薛红线伸伸舌头道:``哎呀,管得好紧。王叔叔,我只好明天见你了。''铁摩勒看她穿过回廊,从左边月牙门进去,暗暗记着方向。

那管家知道这``王小黑''是主人看重的人,对他也很巴结,闲话中告诉了铁摩勒,说那严夫人的丈夫名叫严庄,是安绿山的``大臣'',官居'\,'太子少师''之职。铁摩勒听了,也并不如何放在心上。

铁摩勒初到薛家任职,而且薛嵩又是今日回家,他以为定会有一顿接风酒的,哪知到了傍晚时分,薛嵩只是传出话来,叫管家好好招待他,并带他在家中各处,行走一遍,以便熟悉门户,兼充护院。他随那管家走了一遍,只是从外面经过,既没见着``卢妈'',也没见着薛嵩。

晚上,那管家给他单独开饭,这才告诉他道,薛嵩今晚本来准备设宴招待他的,但自那严夫人来后,薛嵩夫妇就一直在内室陪她说话,好些客人来拜候薛嵩的也都给挡驾了。听管家说,薛嵩的神色似乎有点不大愉快,晚饭也只是他们三人躲在内房里吃,连红线也没有唤进去,不知是甚原因。铁摩勒听了,暗暗纳罕。心想那严夫人是``大臣''之妻,纵然严、薛二家是通家之好,薛嵩也用不着一直陪着她呀。

晚饭过后,铁摩勒歇了一会,待到三更时分,铁摩勒换了一身黑色的夜行衣,悄悄出去。他已经熟悉了薛家的门户,又已知道了卢夫人所住的方向,不多一会,便找到了她的房间。

奇怪得很,卢夫人的房中还有灯火,碧纱窗上,映出两个女人的影子,而且还传出嘿嘿细语之声。

卢夫人的房间窗外是个庭院,庭院中有棵老梅,铁摩勒施展轻功,飞身上树,偷规进去,只见那两个人正是卢夫人和薛夫人。

铁摩勒不禁又是暗暗奇怪。

只听得薛夫人说道:``以往我每次劝他,他总是说,你们女流之辈,修得甚么国家大事?这次劝他,他虽然仍未答允,却没有再骂我了。''

卢夫人道:``听说薛将军这次出兵不利,可是真的?''

然人人道:``就是为了这个缘故。他的同僚,本来就有一些人妒忌他的,他这次打了败仗,很害怕那些人乘机落井下石。''

卢大人道:``姐姐,我在你家多年,承蒙你的厚待,在这紧要关头,我不能不直言了。姐姐,你千万要拿定主意,劝你将军及早回头,否则到了身败名裂之时,悔之晚矣!''

薛夫人道:``姐姐,我得你多年教诲,也稍知大义。即算不为身家性命打算,我也不愿见他屈身从贼,受人唾骂。只是他这人畏首畏尾,顾虑太多,我屡劝不听,却是奈何?''

卢夫人忽道:``这一篇檄文,你可见过么?''

薛夫人接过那张檄文,看了一会,轻声念其一几句道:``若有翻然来归,反戈击贼者,定邀上赏,视其立功大小,裂土分封。

咦,姐姐,你这檄文是从那里得来的?依你看,这几句话可以相信吗?''

卢大人道:``不瞒你说,这是王伯通的女儿拿来的。她是闯荡江湖的女中豪杰,前些日子,还到西蜀去了一转,揭了这张檄文回来,她也正在劝她的父亲呢!这檄文她抄了一份给我,就是有意要我给你看的。据她说,这是太子服兵马大元帅的檄文,太子上月已在灵武自即帝位,急于恢复两京,所以不惜定下重赏招降。据她说像薛将军这样的人,若然反正过去的话,最少可以做个节度使。听她的说话,似乎很可相信!''

这张檄文,铁摩勒是早就见过了的,不禁想道:``到底是卢夫人懂得说话,既喻以大义,又动以利害,这话人家自听得进去。

我劝聂锋时,就没有想到这张檄文,只一味和他讲大道理,好在聂锋本来不坏,要是换了薛嵩,我这样劝,只怕反要白送一条性命呢。''

过了一会,薛夫人说道:``好,你这张檄文给我,我拿去劝他。

他若还不依,我就拿这条老命与他拼了。''

卢大人道:``若能如此,这是国家之福,也是薛家之福。''

薛夫人忽地叹口气道:``姐姐,这许多年我们实是委屈了你。

你亲生的女儿也不能认,还委屈你做了奶妈。我实在于心有愧!''

卢夫人道:``未亡人留得余生,还计较什么名份?多年来蒙你照顾,让我母女托庇宇下,说实在的,我感激你还来不及呢!''

薛夫人道:``要是事成之后,我会对红线说明真相的。只求你让红线将我当为义母,我于愿已足。到了那时,大约他也不敢再难为你了。唉,他的脾气虽是粗暴,但也确是疼这孩子,所以才会定下那样严厉的禁条:谁泄露了风声,就要把谁打死!''

卢夫人苦笑道:``这些话以后再说吧。''刚说到此处,忽听得有脚步登楼之声,薛夫人轻轻笑道:``又有一个人要来请教你了,我避开她,让你们说话,更可方便。'卢夫人点点头道:``也好。''稍稍挪开衣柜,开了房间的另一道门,让薛夫人出去。她刚把衣柜扶正,果然便听得扣门之声。铁摩勒一看,不禁又是一怔。正是:艰难留得余生在,忍辱含羞为报仇。

欲知后事如何,请听下回分解------

旧雨楼扫描,YackerOCR,旧雨楼独家连载

\chapter{第三十三回 沐猴僭位徒贻笑
屠象逞威起杀机}\label{ux7b2cux4e09ux5341ux4e09ux56de-ux6c90ux7334ux50edux4f4dux5f92ux8d3bux7b11-ux5c60ux8c61ux901eux5a01ux8d77ux6740ux673a}

来的是个珠光宝气的贵妇人,她一面叩门,一而说道:``卢夫人,你还未睡吗?我又来打扰你了。''听这称呼,她似乎已知道卢夫人的本来身份。

卢夫人打开房门,将她迎接进去,笑道:``严夫人,你屈驾到我这下人房间,真是不敢当之至。''

铁摩勒心道:``原来是今日来的女客人,安禄山的一品大臣严庄的妻子。卢夫人怎的和她这般熟络?''

严夫人道:``姐姐,你这样说那是骂我了。你我二人的丈夫是同一科的进士,论起当年官职,我家老爷还是尊夫的下属呢。''

卢夫人道:``那是以前的事情了。当时,严大人还是大唐进士,现在他已是大燕的一品大臣了。''

严夫人眼圈一红,说道:``姐姐,我素仰你是女中诸葛,今天实是有疑难之事,要来请教你的,求你不要再讥刺我了。''

卢夫人道:``你既以姐妹之情来见我,那就恕我僭越,也称呼你一声姐姐了。姐姐,你家大人在朝中甚为得意,还有何疑难之事?''

严夫人道:``主公对太子越来越不喜欢,脾气也越来越暴躁了。不瞒姐姐,拙夫忝为大臣,也常遭主公鞭挞,连太子以储君之贵,也是隔不了三五大,就要被他鞭打一场。现在主公最宠的是段妃,段妃已生有一子,名唤庆恩,窥主公之意,似乎是想废太子而立庆恩。唉,太子与拙夫只是受辱,那还罢了,只恐还有不测之祸,性命难保。''

卢夫人沉吟半晌,叹口气道:``这等废立之事,历朝史籍,颇有记载。自古立一子废一子,那被废之子,曾有几个保得性命的?这事确是难怪尊夫过虑!''

严夫人听她这么一说,更为着慌,凄惶问道:``姐姐,既然如此,你何以教我?''卢夫人道:``这事须得从长计议,有是有个法子,只不知你敢不敢行?''说到此处,两个人已靠在一处,悄悄耳语,铁摩勒再也听不到什么了。

但见严夫人双眉紧蹩,脸上的神情甚是紧张,又似带着几分恐惧,过了一会,只听严夫人吁了口气,说道:``这事确是应该从长计议,姐姐,我今晚住在你这里了。''

铁摩勒心里想道:``原来卢夫人留在虎穴,确具有苦心。我不必再去问她了,等着瞧她所策划的事情发生吧。''

第二日,铁摩勒一早起来,薛府的管家就将一套官佐的服饰拿来,说道:``王佐领,请你换了这套衣裳,马上去见将军。''

铁摩勒暗暗纳罕,心想:``我虽受了他亲兵佐领之职,但又不是出发去打仗,在屋子里头,却要我换上这身戎装作甚?''

到得堂前,薛嵩正在那里负手徘徊,一见铁摩勒便问道:``你吃过早点没有?''铁摩勒大为奇怪,据实答道:``还未曾吃过。''

薛嵩皱了皱眉,吩咐那管家道:``你拿几个大饼来。王老弟,你在路上吃吧。时间不够了。''

铁摩勒问道:``将军要到哪里去?可是要我随行?''薛嵩道:``正是。主上今日在骊山行宫宏张盛宴,百戏杂陈,款待来朝贺的各藩邦使节,朝中文武百官都去作陪,主上听说我已回来,叫我也去凑个热闹。王小黑,你作我的卫士,也去开开眼界吧。''

这样的盛会,薛嵩刚刚回来,就得安禄山传旨叫他赴宴,本该高兴才是,但他眉头深锁,却似有隐忧,原来他因为吃了败仗,生怕有同僚乘机讲他坏话,甚或暗算他,故此虽是参加``欢乐''的宴会,也不得不提心吊胆。他要铁摩勒作他卫士,陪他同去,用意就是在预防不测的。

铁摩勒听了,大吃一惊,``要是给人认了出来,这却如何是好!''但他又想到,这个盛会,作为安禄山``大内总管''的羊牧劳也必然在场;羊牧劳害死他父亲时,他年纪还小,现在已根本记不起羊牧劳是什么模样了。因此他也想趁此机会,认识仇人的面目,同时去看看群魔乱舞的场面。

铁摩勒胆大包天,啃了几个大饼,二话不说,跟薛嵩便走。

聂锋也像薛嵩一样,受安禄山之召,要去赴宴,这时已在门前相候,他见薛嵩带铁摩勒同行,也是大吃一惊,心里暗暗叫苦。

从城中到骊山行官约有三十里路,一路车马不绝,都是被招往赴宴的新贵。铁摩勒登上骊山,经过安禄山旧时的别墅。想起当年史逸如在这里死难,自己与段圭璋、南霁云曾在这里溅血恶斗群凶,而薛嵩则正是当时的敌人之一,想不到今日却与他重来,心中不无感慨。

进人行宫,但听得处处喧闹之声,乱烘烘的哪有半点``皇家''

的尊严气象,铁摩勒暗暗好笑,``安禄山本是个市井无赖出身,想来他的文武百官也是和他差不多的胚子!''

宴会设在行宫的``御苑'',那里更是人头挤挤,好些``官员''捧着酒盅,穿来插去的东面瞧瞧热闹,西面瞧瞧热闹,见到宫女经过,就龇牙咧嘴、嘻皮笑脸地看她们。连薛嵩进来也没人注意,更不用说铁摩勒了。

铁摩勒心想:``这哪里像是个`天子'赐宴?我义父做绿林盟主的时候,每逢做了一笔大生意,也必然大宴手下的头目,和今日的情形倒是差不多。但我义父那些头目,还不似安禄山这些官儿般的丑态毕露。''

安禄山本是胡人,他所属的诸番部落头目,听说他做了皇帝,都来朝贺。安禄山有意炫耀富贵,行宫的御苑里百戏杂陈,极尽声色之娱,让他们的随从可以在御苑的各处随便闲逛,尽情享乐。安禄山自己则在园中的百花亭里,和这班诸番头目(美其名日`使臣'的)饮酒取乐,他手下有地位的将军和大臣,才有资格在亭中作陪客。

薛嵩、聂锋二人的职位是``龙虎上将军'',又是安禄山``御旨''

召他们来的,因此要去百花亭作陪客。铁摩勒是卫士,却不能进百花亭去。

园中处处陈列有酒食,可以随意取用,铁摩勒乐得自由自在,而且混在人丛之中,也可以遮掩自己百花亭中他认得一个是王伯通,至于哪个是羊牧劳,他就不知道了。

铁摩勒正在四面张望,忽听得有人叫道:``大象来了,快快闪开!''只见一群象奴,牵了四头大象,在百花亭外的那片空地上一字排开。

铁摩勒心里奇怪:``宴会之中,要这些大象来作甚?''一个醉醺醺的官儿似是发觉了他的傻态,哈哈大笑,拍拍他的肩膊道:``你不懂么?新奇的玩意儿快上演了厂'原来这些乃是官中的驯象,当初天宝年间,玄宗注意声色玩乐,每至宴酣之际,命御苑掌象的象奴,引驯象人场,以鼻擎杯,跪于御前上寿,都是平日驯练熟的。又尝教习舞马数十匹,每当奏乐之时,命掌厩的围人,牵马到庭前,那些马一闻乐声,便都昂首顿足,回翔旋转地舞将起来,却自然合着那些乐声节奏。宋人徐节孝曾有舞马诗云:``开元天子太平时,夜舞朝歌意转述。绣榻尽容麒骥足,锦衣浑盖渥洼泥。才敲昼鼓争先奋,不假金鞭势自齐。明日梨园翻旧曲,范阳戈甲满关西。''说的便是这段史事。

当年此等宴会,安禄山都得陪侍,好生艳羡,今日反叛得志,便欲照样取乐,故此叫唐宫原来的象奴将那些驯象牵来,叫他们表演,好今诸番头目惊异。

果然人们都纷纷围拢过来,安禄山叫一个太监走到场中,向众人宣言道:``圣上受天命、为天子,不但人心归附,就是那无知的物类,也莫不感格效顺。诸位请看这些大象擎杯跪献,等下还有骏马闻歌起舞!''这话说了,人人都睁大了眼睛,等着看新奇的玩意!

不料这些大象竟然不听号令,象奴喝了三遍,它们仍然僵立不动,并未跪下。象奴把酒杯先送到一个大象面前,要它擎着跪献,那大象却把鼻子一卷,将酒杯卷了过来,抛出数丈;另一头大象更糟,把递酒杯给它的那个象奴也卷翻了!登时令得安禄山左右尽皆失色,诸番头目,不懂礼仪,更忍不住掩口窃笑。

原来这几头大象,虽然都是教习熟了的驯象,但它以往每次献酒,都只是献给玄宗皇帝一人,因而早已成了习惯。如今它们见这个南面而坐的安禄山,虽然也穿着龙袍,却并非它们见惯的那个人,因此它们也就不愿做惯常的动作,甚而发了脾气了。

安禄山听得窃笑之声,又羞又恼,大骂道:``孽畜可恶,胆敢欺君,将它杀了!''象奴面面相觑,要知每头大象,都有千来斤重,要他们将大象击杀,他们哪有此力?

忽见一个身材高大的老人,走出来道:``主上息怒,这杀象的差使,交给奴婢吧。听说象鼻味道甘美,这些大象胆敢欺君,等下就叫御厨将它们的鼻子拿来佐膳。''

安禄山这才转怒为喜,拍掌笑道:``羊总管此议,妙哉!妙哉!你们都来瞧羊总管的杀象手段!''

那老人走进场中,不动声色的到一头大象身旁,那头大象以为他是来抚弄它的,虽然不很愿意,尚未发怒。那老头也并不怎样用力;果然似是抚弄一般,轻轻一掌击下,只听得轰隆一声,就像倒下了一座山,那头大象已给他一掌击毙了。登时彩声雷动,那些番邦头目不懂内功的奥妙,更是吓得目瞪口呆,好半晌才叫得出声道:``这位羊总管敢情是天上的雷神下凡么?怎的如此厉害!''

铁摩勒这时已知道了此人便是羊牧劳,也禁不住吃了一惊,``如此看来,这魔头的绵掌功夫,果然已到了最上乘的境界,看来我只怕接不了他的七步七掌。''

这时,那另外三头大象已知羊牧劳来意不善,三头大象从三面向他冲来,三条长长的象鼻就似软鞭了向他卷去。羊牧劳有意卖弄功夫,横掌如刀,一掌削下,将最凶的那头大象的鼻子削了半截,那头大象痛得呜呜大叫,遍地打滚,羊牧劳哈哈大笑。

第二头大象的鼻子卷到,羊牧劳又故意让它卷了起来,却使出了分筋错骨手法,在它鼻子的软筋上一捏,那大象空有千万斤气力,鼻子已软绵绵地失了劲道,身上的气力使不出来。

那大象给羊牧劳弄得鼻子麻痒,本能的将鼻子一缩,把羊牧劳卷到了它的面前,这一来等于凑上去受他掌击。羊牧劳对准象额,一掌拍下,登时那头大象也给他击毙了。

羊牧劳飞身一跃,跨上了另一头象背,居高临下,又一掌将它击毙。这时,那头被削了鼻子的大象正在狂性大发,冲出场来,吓得围在场边观看的官儿大呼小叫,跌跌撞撞,乱作一团。

羊牧劳双足一点,箭一般地射去,五指插下,这一插用的却是铁砂掌的硬功,但听得咔嚓一声,大象的额角上开了一个天窗,羊牧劳拔出五根鲜血淋漓的手指,哈哈大笑,这头最凶的大象,当然也没命了。

羊牧劳接连用四种不同的身法和掌法,竟然在不到一炷香的时刻,连毙四头人象,吓得诸番头目、文武百官心惊胆战,喝彩的声音也在发颤。

铁摩勒混在人丛之中,忽见两个十岁左右的孩子也挤进来,一个道:``这老头子好霸道啊!样子也凶,我看准是个恶人。''另一个道:``别再看他这副凶样了,咱们寻王叔叔去。''前面那个孩子伸直了脖子,说道:``王叔叔我没瞧见,我的爹爹和你的爹爹在亭子里面陪那个皇帝喝酒,你瞧见了没有?''

铁摩勒吃了一惊,看出了这两个扮作男装的孩子正是聂隐娘和薛红线。就在这时,只见王燕羽也挤了进来,低低的``嘘''了一声,说道:``你们怎么又不听话,到处乱跑了。赶快回那边棚子去。那亭子是进不得的!要是让你们爹爹瞧见,你们可不得了!''

有一个官儿错把王燕羽当作宫女,把这两个孩子认作小黄门(太监),仗着几分酒意,嘻皮笑脸的上来调戏她道:``别忘着走啊,今日万岁与百官同乐,咱们也乐一乐吧!''王燕羽一笑道:``你自个儿乐去吧!'卡袖一挥,就像软鞭似的在他的大肚子一拍,登时把那官儿打得矮了半截,抚着肚子雪雪呼痛,王燕羽一手携着一个孩子,挤出人丛。

旁边一个武士将那官儿扶起,说道:``你好大胆,你知道她是谁么?她是鲁国公王伯通的女儿,没把你宰了,算你运气。''

铁摩勒听官儿们的谈论,才知道那边那个棚子,是专给安禄山的妃子们和一班王公的内眷看热闹用的,胡人对男女的关防随便得多,所以他的妃子们也不怕抛头露面。但王燕羽竟敢叫聂、薛二女假扮男孩子混进来,这却颇出铁摩勒意外。

安禄山得羊牧劳给他挣回了面子,又高兴起来,接在大象献酒之后,节目本是安排骏马舞蹈的,但他怕那些``舞马''也似大象般不听号令,这节目便临时取消,另传一班乐工上来演奏。

唐宫的教访(相当于近代的剧院和音乐院合并组织)规模极大,因为唐玄宗本人就是个音乐家,懂得弹奏诸般乐器,也懂得作曲,因此他所选拔的教坊乐工,例如李暮的羌笛,贺怀智的``方响''(一种乐器名),花奴的揭鼓,张野狐的角栗,黄幡绰的拍板,雷海青和郑观音的琵琶,都是当代著名的高手。每有大宴集,先设大常雅乐,有坐部,有立部;那坐部请乐工,在堂上坐而奏技,立部诸乐工,则于堂下立而奏技,``雅乐''赛罢,继以``鼓吹''番乐,然后教访新声与府县散乐杂戏,次第毕呈。安禄山虽然不懂音乐,但他以前以杨贵妃``义子''的身份,经常陪侍,看惯了此等场面,今日做了皇帝,免不了要照样``风光''一番。

玄宗逃难西蜀,这些乐工子弟们,只有李暮、张野狐、贺怀智等人随驾西走,其余的都做了安禄山的俘虏,安禄山一声令下,便将这些人都拘唤了来。

只见教坊乐工按部分班而进,列队在百花亭下。这五部乐工,使用各种不同的乐器,本来各有所司,但安禄山却不懂这些,押班的乐宫请问他要如何演奏,他说不出个名堂,一皱眉头便骂道:``蠢材,连这个也要问吗?你叫他们将各人的绝活拿出来就是啦!''五部乐工的押班乐官面面相觑,只好挑选了各种乐器的演奏高手,给他来一支``钧天雅乐''的大合奏。

这是一个欢乐热闹的合奏,顿时间风萧龙笛,象管鸾笙,金钟玉罄,羯鼓奏筝,琵琶箜篌,方响手拍(均乐器名),吹的吹,弹的弹,鼓的鼓,敲的敲,虽然乐工情绪不佳,倒也声音铿锵,悦耳动听。安禄山大乐,掀须称快道:``朕向年陪着李三郎(按:指玄宗,因玄宗排行第三。)饮宴,也曾见过这些歌舞。只是当时乃伺候别人,未免拘束,怎比得今日这般快意。今天不足者,不得再与玉环姐妹欢聚耳!''

乐工奏毕,一个懂得音乐的突厥小王子道:``好是好了,却有不足之处。''安禄山愠道:``有哪样不足?''那王子道:``为何不听得有琵琶的音响,久闻雷海青是琵琶第一手,莫非他今日不来么?''侍立在旁的太监认得雷海青,指给安禄山看道:``来是来了,大约他刚才没有用力弹奏,所以小王子听不见。''安禄山怒道:``他敢不尽力,唤他上来,单独弹奏,给小王子听。''

铁摩勒听得太监传呼雷海青,吃了一惊,心道:``怎的他还没有逃走?''心念未已,只见一个中年乐工,已拖着琵琶,走进百花亭。

你道铁摩勒何以吃惊,原来这雷海青不是别人,正是铁摩勒二师兄雷万春的同胞兄长。他们两兄弟一母所生,性情却不大相同,雷海青性近音乐,自小投入梨园,拜名乐工为师,终于成为了国中的琵琶第一手;雷万春则自小好练武,长大之后,得磨镜老人收为徒弟,成为了一位出名的游侠。但他们二人也有一样相同之处,那就是刚直不阿的忠烈之性。

雷海青这次被迫而来,胸中本已满怀悲愤,所以在合奏``钧天雅乐''之时,他虽然手抱琵琶,却始终没有拨过一弦。这时,他被安禄山唤人百花亭,一进亭中,陡然激起忠烈之性,便高声痛哭起来,指着安禄山大骂道:``我雷海青虽是乐工,颇知忠义,怎肯侍你这反贼!''这一骂登时令得满座失惊,安禄山的左右方待擒拿,雷海青早已奋身扑去,提起琵琶,向安禄山兜头便打。

羊牧劳振臂一格,但听得``喀喇''一声,琵琶裂成片片,雷海青给震退数步,兀未跌倒。说时迟,那时快,安禄山的两个武士早已双刀齐下,砍中了他!雷海青大叫道:``今日是我殉节之日,我死之后,我兄弟雷万春自能尽忠报国,少不得手刃你这班贼徒!''骂完之后,方始倒地。后来名诗人王维有首诗道:``'万户伤心生野烟,百官何日再朝天?秋槐叶落空宫里,凝碧池头奏管弦。''写的便是当日之事。当时王维也留在长安,未及逃走,装病不仕伪朝,被安禄山软禁在普施寺中,因此他这首诗虽是为雷海青死难而作,却不敢直白地赞雷海青,而只是自写悲感之意。后来肃宗还乡,凡附逆者均分别定罪,王维和因有这首诗而得赦,那是题外之话。

铁摩勒混在人丛之中,忽逢此变,目睹雷海青被乱刀分尸,气愤填胸,一时之间,竟然控制不住自己,失声大叫起来,冲出人丛十几步,但这时雷海青已死,抢救已来不及。待到铁摩勒记起自己的``身份'',他也早已被人发现了。

王伯通最先认出铁摩勒,大吃一惊,立即叫道:``羊总管,这小子便是铁昆仑的儿子!''又向安禄山道:``主公,我听说这小子曾与段圭璋犯过你的龙驾,不知可有此事么?''

安禄山粗鄙武夫,但却也有一样长处:记性甚好。他见过的人,很久都不会忘记。这时也依稀认出了铁摩勒就是当年闹过他骊山别墅的那个少年,不禁勃然大怒,喝道:``好大胆的小子!

左右赶快将他拿下,死活不论,都有重赏!''其实不必安禄山下令,园中的武士,早已纷纷向铁摩勒扑去,羊牧劳也跃出了百花亭。

铁摩勒喝一声``去'',施展出``大摔碑手''的功夫,只一抓便把一个冲到他身前的武士,像小鸡一般的提了起来,摔到人堆里去!

御苑里百官齐集,处处都站满了人,铁摩勒故意和他们恶作剧,大展神威,接连摔了三个武士,都是向着人多的地方摔去。

这一来,真个是城门失火,殃及池鱼,许多官儿都给撞得四脚朝天,变成滚地葫芦,登时鬼哭神嚎,秩序大乱!铁摩勒便硬从人丛中闯出。

御苑里的武士虽多,但到处都是人流阻塞,而且这些人又都是朝中新贵,他们有所顾忌,不敢展开手脚;有几个好不容易才挤入人丛,接近了铁摩勒,却又不是铁摩勒的对手,反而给铁摩勒擒来,当作武器。

铁摩勒边打边走,混乱中不辨方向,竟然打近厂女棚。在女棚中的有安禄山的妃子、宫女和各王公大臣的内眷,见铁摩勒凶神恶煞般地打来,个个吓得面无人色,尖声锐叫。

羊牧劳见状大怒,不理那些官儿们的死活,施展出轻功提纵术,便从人头上踏过去,猛地大喝一声,便似空中扑下了一只兀鹰,一掌向铁摩勒击下。

铁摩勒奋起一格,双掌相交,只听得``蓬''的一声,铁摩勒跃翻地上,但羊牧劳给他一震,也要在半空中倒翻了一个筋斗,才稳得住身形。

铁摩勒一个鲤鱼打挺,又翻起身来,正好羊牧劳又已挥掌打来,铁摩勒使出十成功力,再接了一掌。这一下,双方都给对方掌力震得摇摇晃晃,铁库勒多退了两步,稍吃点亏,但却不至于跌倒了。原来羊牧劳的功力虽然胜过铁摩勒不止一筹,但因他刚才以绵掌击石的功夫,连杀回头大象,内力已消耗了不少,再与铁摩勒以全力相拼,两人已是相差无几了。第一掌他是以居高临下之势,才能把铁摩勒震翻的。到了第二掌,他虽然仍占上风,优势已经不大。

羊牧劳衣袖一挥,使出沾衣十八跌的功夫,将周围的人都震得向后直退,登时腾出了一片空地,他一个箭步冲前,第三掌再向铁摩勒打下,这一掌他也用尽了十成功力!

聂锋见铁摩勒闹出事来,这一惊非同小可,但他比较沉着,神色上还未显露出来。那薛嵩则比他惊惶更甚,他做梦也想不到,这个新任他亲兵住领的``王小黑'',竟然就是当年曾大闹安禄山府邸的那个铁摩勒,而这个铁摩勒,又还是羊总管的仇人!

王伯通见薛嵩面色有异,问道:``'敢情薛将军也认得这小子么?''安禄山笑道:``他何止认得,他还吃过这小子的亏呢。那年这小子和段圭璋来行刺我,我记得薛将军曾吃他斫了一刀。''

王伯通得意洋洋地道:``好啊,现在羊总管已赶到了。薛将军、聂将军,咱们都去助羊总管一臂之力吧,捉了这小子千刀万剐,也好替你报那一刀之仇。''

薛嵩有苦说不出来,心里只自想道:``可不知有没有人认出了他是我带来的卫士?''他怕安禄山见疑,只好站了起来,准备跟王伯通出去。就在这时,那得意洋洋的王伯通,忽然发出了一声惊叫,登时似中了``定身法''似的,僵在那儿!

你道这是什么原因?原来是他正看见他的女儿从女棚里跳出来,挺剑向羊牧劳刺去!

羊牧劳使出了十成功力,向铁摩勒一掌拍下,铁摩勒与他硬拼,虽然不致吃了大亏,但双掌却已给对方吸住,一时间竞撤不回来。

羊收劳哈哈大笑,催动掌力,加紧压下。铁摩勒的功力到底稍有不如,只觉对方的内力,像浪头般一个个打来,前浪未休,后浪又到,眼看就要支持不住。忽听得一声娇笑,竟是王燕羽的声音笑道:``羊大总管,我也来领教领教你的功夫!''

羊牧劳做梦也想不到王燕羽会突然跳出来用剑刺他,慌急中忙把掌心一登,将铁摩勒震退两步,回掌向王燕羽便斫,但还是慢了一步,王燕羽出剑如风,早已在羊牧劳的肩头戳了一下。

羊牧劳也确是了得,肩头一沉,竟把王燕羽刺来的劲道卸去了一半。王燕羽这一剑本来是想戳穿他的琵琶骨,废掉他的武功的,哪知剑尖刚刚沾肉,立即便给羊牧劳用内劲反弹开去,羊牧劳只不过给划破了少许皮肉,而王燕羽则几乎给他震倒!

羊牧劳大怒,扑过去便是一掌,骂道:``你这野丫头为什么暗算我?''

这时,铁摩勒早已拔出剑来,退而复上,唰的一剑,便刺羊牧劳的肩井穴,铁摩勒的剑术尽得段圭璋真传,而且又经过磨镜老人指点,精益求精,除了火候稍差之外,实已不在段圭湾之下。

这一剑他用的是龙形剑法中最刚猛的一招``龙飞九天'',剑尖抖起了几朵剑花,隐隐带着风雷之声!

羊牧劳识得厉害,他那一掌本来是向前打去,迫得转了方向,斜闪一步,再向铁摩勒劈出。但听得呼的一声,剑光流散,铁摩勒的剑尖给他的臂空掌力震歪,这一剑刺了个空。

王燕羽笑道:``我听说你的大号叫七步追魂手,我没见过,所以今日特来开开眼界,看你到底怎样追魂?''她口中说话,手底却是毫不放松,早已一剑刺来,恰好在铁摩勒被他震退的时候,补上了这个空位。

羊牧劳冷笑道:``好,就叫你识得厉害!''走离宫,转坎位,突然一掌向王燕羽意料不到的方位打来,王燕羽那一剑搠了个空,身形已在他掌力笼罩之内。

羊牧劳念头一动:``我打死了她,在王伯通面前可交代不过去。''改拍为按,哪知王燕羽的轻功也已将近一流境界,并且也懂得五行八卦的身法步法,不过不及羊牧劳运用得那么神妙而已。就在羊牧劳变式换招这一刹那,她已足踏``震位'',绕出``生门'',反手一剑,斜刺羊牧劳腰胁的风府穴。

铁摩勒一退复上,使出了一招``李广射石'',长剑逞刺羊牧劳的咽喉。他们二人前后夹攻,尤其铁摩勒这一剑,更是攻敌之所不得不救,羊牧劳顾不得再去擒拿王燕羽,霍地一个``凤点头'',移形换位,一招``倒打金钟'',横掌斜切铁摩勒的手腕,解开了他这一招,同时也闪开了王燕羽从后斜方刺来的一剑,可是他虽未中剑,腰带却已给王燕羽削断了。

羊牧劳大怒,展出了七步追魂的绝技,不论铁摩勒走到哪方,都给他抢先堵住。王燕羽决心要救铁摩勒,羊牧劳虽然不能分身来拦阻她,她也不肯逃走。两人或一前一后,或一左一右,合力来斗羊牧劳,他们虽然闯不出去,羊牧劳却也奈何不了他们。

铁摩勒既然无法闯到人丛中去,那些官儿们当然也远远避开,在他们周围的空地渐渐扩大,安禄山手下的那些武士去掉``障碍'',可以大踏步赶来了。

最先赶到的是安禄山的两个``龙骑都尉''------单刀张忠志和铁拐杜绶,这两人的功夫远在其他武士之上,他们不敢去惹王燕羽,不约而同的都向铁摩勒进击。张忠志挥刀斜劈铁摩勒的臂膊,杜绶则抡拐猛敲铁摩勒的膝盖。

铁摩勒当然不会惧怕他们,但他给羊牧劳紧紧迫住,一时之间,却腾不出手来应付。正在危急之际,忽听得两个娇嫩的声音同声叫道:``王叔叔,你别害怕,我来帮你。''原来是聂隐娘和薛红线这两个女孩子,这时也已从女棚中跑出来了。

她们身躯矮细,滑似游鱼,薛红线短剑一挥,刺中了张忠志的腰眼,聂隐娘更狠,她从杜绶的胯下钻过,短剑自左到右的转了一圈,将社绶的两只脚后跟都斩伤了。

杜绶大叫一声,扑通便倒,恰值羊牧劳一脚踏下,正巧踏在他的身上,登时一命呜呼。

羊牧劳怒道:``哪里来的两个野孩子?''伸开蒲扇般的大手,向下便捞,王燕羽连忙叫道:``你们不可惹这老魔头,打打那些装模作样的武士倒不妨事!'她与铁摩勒双剑齐出,双剑都指向羊牧劳的要害穴道,羊牧劳只得回掌接招,聂隐娘身子灵活,不待他再抓,先避开了。

张忠志腰眼中剑,血如泉涌,只得赶快跑出场去,找人救治。

可是其他武士,又已陆续赶来。

武士们见这两个孩子刺伤了张忠志与杜绶,都是大为奇怪,同时又不知道她们究竟是谁家的孩子,但揣想能够在这``御苑''

里出现的,父亲定是当朝显贵,说不定还是``皇家''的人,一时之间,倒还不敢动手。

薛红线叫道:``你们瞪着眼睛看我做什么?你们要伤害我的王叔叔,我就不依!''这时,正有两个武士要去夹攻铁摩勒,薛红线倏的跳起来,骑上他的肩头,倒提剑柄,在他头上一敲,薛红线虽然年纪小,气力弱,但这一敲正是人身顶门的要害部分,登时将那武士敲得发晕,晃了两晃,便跌倒了。另一个武士,也给聂隐娘在瞬息之间,接连刺中三剑,不支倒地。

薛红线跳了下来,乐得弯着腰儿笑道:``师父的剑法果然管用,这个大个子给我一打便打晕了。聂姐姐,你更不错,只一剑就刺伤了他。''

羊牧劳沉声喝道:``不管是谁家的孩子,你们将他毙了,万事有我担当。这个小子和这个野丫头却不用你们来管!''

那些武士得羊牧劳撑腰,放大了胆,刀枪剑戟纷纷刺下,薛。

聂二女身躯瘦小,在他们之间穿来插去,东刺一剑,西刺一剑,武士们反而给她们伤了好几个。可是,武士越来越多,渐渐便没有回旋的余地,聂、薛二女被困在核心,情势也渐见危险。

但来人一多,羊牧劳的身手也有点儿施展不开,王燕羽擅长的是刺穴的小巧功夫,趁着铁摩勒用刚猛的剑招迫着他的时候,忽地反手一剑,羊牧劳猛不提防,几乎给她刺中了穴道,在腰背上又添了一个伤口。羊牧劳急忙施展上乘的内功,封住伤口附近的穴道,不让鲜血流出来。

羊牧劳大怒,再用沾衣十八跌的内功,将身旁的武士震得向四边散开,双掌交错击出,又把铁摩勒与王燕羽迫转回来,不让他们杀进人丛。同时,运足了中气,大声叫道:``王伯通,你还不来管教你的女儿!''

满园子的喧闹都给羊牧劳的声音压了下去,这声音似利箭般的插进了王伯通的心房。

王伯通当然深知女儿的脾气,她执意做一件事情,那是决计劝不过来的。何况她今日做的乃是``大逆不道''的事情,即算自己亲手将她绑了,安禄山素来忌刻,也未必便肯放过他们父女。

更何况还有铁摩勒在场,哪能容许自己轻易去缚女儿,而且女儿也未必便肯任由他缚。

片刻之间,王伯通的心里已转了无数念头,饶是他惯经风浪,智计过人,这时也慌得手足无措,拿不定主意。

猛听得乒乒乓乓的碗碟破裂的声音,原来是安禄山看见王伯通的女儿竟然从女棚中跳出来,剑刺羊牧劳,也被吓得六神无主了。

他不是怕王燕羽,而是忌王伯通。王燕羽已被困住,杀不到他的身前;但王伯通却近在咫尺,要是王伯通也变了心,突然过来杀他,那岂非是个绝大的危险。他这么一想,心胆俱寒,顾不得体面,急急忙忙便从亭子后方逃走,因为匆促离席,举动慌张,将席上的杯盘磁碟,碰落了一地。

王伯通正跨出亭子,听得声响,回头一看,只见安禄山已在最亲信的几个心腹武士保护之下,仓皇而逃,有几个武士还在面向着他,作出戒备的神态,刀出鞘,弓上弦,看这情形,似乎只要他向安禄山的方向迈进一步,立刻便会有暗箭飞来。

王伯通怔了一怔,随即便明白了是安禄山对他的猜忌,他把心一横,跳出亭子,和安禄山采取相反的方向。一个原来是他的部下,现在做了安禄山卫士的人拦住他问道:``老爷子当真要去杀小姐么?''这个人是他的老家人,看着王燕羽长大的,对王燕羽一向甚为疼惜。

王伯通长叹一声,忽地将蟒袍扯下,玉带摔开,说道:``这官儿我不当了,你们好自为之,我走了!''那老部下问道:``当家的要往哪儿?''

王伯通道:``我仍然回去当山大王去!''王伯通的喽兵在盘龙谷之役,被辛天雄、南霁云的金鸡岭人马夜袭,已被十歼七八,溃不成军,余下的也被安禄山所收编,剩下他光杆儿一个。

但他得力的头目,却有很多当了安禄山的卫士,差不多占安禄山卫士总数的三分之一,这时也多在园中。如今生出了这样的变故,有些人也怕今后不能见容于安禄山,便也跟着王伯通跑,纷纷叫道:``对,还是再去占据山头,当个山大王更为自由自在!''

园子里本已乱成一片,这件意外的事情发生,乱上加乱,更是难以形容。安禄山的``禁卫军'',在``龙骑都尉''司空拔率领之下,登时布防起来,将斗场所在围得水泄不通,那自然是防备王伯通去救女儿了。

王伯通叹了口气,提高嗓子喊道:``羊总管,我管不了这个丫头,随你处置好啦!''他带领愿意跟随他的旧部,便从卫士防守薄弱的地方闯出``御苑''。安禄山的``禁卫军''见他只是弃官而逃,也就不加拦阻,并未发生战斗,便让他们走出园门。

薛嵩慌慌张张的,也想在混乱之中潜逃,聂锋一把拉着了他,低声说道:``你不要女儿了么?''薛嵩道:``反正她不是我的亲骨肉,咱们的身家性命要紧,你还不快快回去布置后事?''聂锋道:``你这一逃就逃得了么?''薛嵩道:``以后的事以后再说了,趁现在他们还没有知道,赶快回去和家人逃跑吧!''他怕聂锋多言,猛地将袖子一甩,挣脱之后,拔步便跑。聂锋摇了摇头,说道:``我的女儿可是我的亲骨肉,我不能不管!''

司空拔冲进斗场,望了一眼,大怒说道:``你们这班饭桶,这么多人,连两个小孩子也捉不到,羞也不羞?闪开,闪开,让我自己来。''原来这司空拔也是绿林出身,他听说铁摩勒乃是铁昆仑的儿子,心中先有了几分顾忌,同时他也知道羊牧劳的脾气,尽管看这情形,羊牧劳力敌二人,实在难以轻易取胜,但料想他也不愿别人前来``分功''。故此司空拔正好拣软的吃,迈步上前,抡起一柄``降魔杵'',便向聂、薛二女喝问。

司空拔是安禄山底下数一数二的好手,力大无穷,他那柄`降魔杵''长达一丈,使动起来,就是石头碰上,也会被打得粉碎。

原先困住聂、薛二女的那些武士,都怕受他误伤,不待他的吩咐,早已纷纷闪开。

司空拔接着铁杵,大声喝问道:``你们究竟是谁家的孩子,还不快说?是谁叫你们到这里胡闹的?''聂隐娘一把拉着薛红线,抢着说道:``你这样凶,我偏不告诉你。你们这许多人,欺负我的王叔叔,我们瞧不过眼,非来帮他不可!''

司空拔喝道:``你们不说,我一棍打下,你们尸骨无存!''薛红线作了一个怪脸,扁着嘴冷笑道:``他们也是这样吹牛的,你瞧,我们不是好端端还在这里?''司空拔哼了一声,陡地向她一脚踢出,意欲将她踢翻,哪知薛红线身躯灵活,像猴子般一跳便问了开去,聂隐娘趁势就一剑刺来。

司空拔慌忙缩腿,但听得``嗤''的一声,裤管已给聂隐娘的短剑划破了一道裂缝。司空拔本来只是想把她们活捉的,吃亏之后,恶念陡生,大怒喝道:``小贼种,见阎王去吧!''抡动``降魔杵'',呼的一声,就向这两个小孩子拦腰横扫!

聂隐娘脚尖一点,身轻似燕,就像``跳绳''一般,从降魔杵上面跳过,司空技手腕一翻,那碗口般粗大的降魔杵刚刚竖起,薛红线用了个``海燕掠波''的姿式,也从降魔杵上面跳过去了聂隐娘格格笑道:``我年纪太小,阎罗王说还未肯收留我呢?''

司空拔喝道:``小贼种,死在临头,还油嘴滑舌!''抡动了降魔柠,越扫越急,虎虎风生。聂、薛二女不过仗着轻功,善于问避而已,这时见他越打越猛,心里也着了慌。那降魔杵所着之处,砂飞石裂,要是一个躲闪不及,给它挨上了半点,聂、薛二女的柔肌嫩骨,怕不成为粉碎?

忽听得有人叫道:``司空都尉,我来助你!''说时迟,那时快,聂锋提着长剑,已冲了过来。薛红线失声叫道:``聂叔叔,你怎么可以帮他?''话犹未了,只听得``咚''的一声,聂锋一个肘锤,撞中了司空拔的后心,司空拔脚步一个跄踉,降魔杵砸在地上,地面凹陷,泥土飞扬,纷落如雨,几乎将薛红线淹没。薛红线冲了出来,大喜叫道:``聂叔叔,多谢你替我出气,我爹爹呢?''

要不是聂锋这么一撞,这一杵本来就要打中聂隐娘的。聂隐娘这时惊魂稍定,也在叫道:``爹爹,你再给他一剑呀!''

司空拔再提起了降魔杵,大怒喝道:``聂锋,你作反了么?''聂锋冷笑道:``你骂我的女儿是贼种,我岂肯放过你?来,来,来!

我领教你的降魔杵法!''他是大将身份,所以刚才虽是救女情急,他还不肯在背后用剑刺他,而是要和他光明正大的较量。

司空投举件一架,``当''的一声,荡开了聂锋的长剑,正要回骂,聂、薛二女可不理会什么江湖规矩,似游鱼般的钻过去便用短剑刺他。司空拔被聂锋撞正腰眼,跳跃不灵,腰胯接连中了两剑,待他踢出连环腿时,这两个小孩子又早已跑开了。

羊牧劳喝道:``好呀,原来是你的孩子广身形一晃,使出七步追魂的身法,倏然问就欺到了聂锋的身前,聂锋反手一剑,只听得``铮''的一声,剑脊已给弹厂一下。羊牧劳用的是隔物传功的内家真力,聂锋虎口破裂,青铜剑几乎脱手飞去;与此同时,司空拔的降魔杵也扫了过来。

铁摩勒飞身掠到,抡动长剑,当作大刀来使,一剑劈下,``当''

的一声,正斫在降魔杵上,但见火星蓬飞,司空拔虎口发热,禁不住连退数步,``这小子气力好大,我今番可碰到了对手I!''

羊牧劳如影随形,一个窜身,一招``游龙探爪'',又已抓到了聂锋的后心。聂隐浪尖声叫道:``休得伤我爹爹!''体看她年幼力弱,使的却是最上乘的剑法,``唰'的一剑,剑锋直指羊牧劳膝盖的``环跳穴'',羊牧劳迪前反身踢腿。说时迟,那时快,王燕羽也已一剑刺来,与聂锋联手,挡住了羊牧劳。

聂锋见女儿不知恐惧,吓得冷汗直流,慌忙叫道:``隐娘,你赶快和薛家妹子先跑出去,不可惹这魔头!''聂隐娘道:``不,爹爹不走,我也不走!''

羊牧劳大笑道:``在我掌下,谁还想逃走?''七步追魂的掌法展开,委时间四面八方都是他的影子,聂锋两父女与王燕羽都被他的掌力困住,不论走到何方,都被他迫退。而铁摩勒也被司空拔所阻,一时之间,闯不过来。

忽见一条黑影箭一般的射到场心,羊牧劳好生诧异,``卫士中怎的却有此等人物?看来竞是远在司空拔之上!''心念未已,忽见剑光一闪,那名卫士竞然向他刺来!这时,铁摩勒方始看得清楚,那卫士不是别人,正是展元修,不禁失声叫道:``展兄,怎么,你也在这儿?''

羊牧劳最初以为是聂锋的旧属,(薛嵩与聂锋,以前曾做过安禄山禁卫军的正副统领。)随着聂锋叛变的,待听得铁摩勒的呼喊,心里更是吃惊:``莫非这人是展大娘的儿子?

说时迟那时快,展元修的长剑已指到了他胸前的``大枢穴'',羊牧劳骈指如戟,身躯一矮,反戳展元修的肘尖,展元修一个移形换位,转过剑锋,剑招未出,羊牧劳已是一掌劈到。

羊牧劳与展大娘交情不浅,他知道展大娘只有一个儿子,在未问明之前,不敢使尽全力,用的是``印掌封穴''的功夫,只使出了七成气力。

哪知展元修的剑法平常,掌法却是悉得家传的奥妙,他的功力比不上羊牧劳,掌法的奇诡变幻,却在羊牧劳之上。羊牧劳的掌力刚吐,他已身随掌走,倏然间指东打西,一掌击中了羊牧劳的腰胯。

羊牧劳大叫一声,腾身起飞,他挨了这一掌,不必再问,已知他是展家的后裔,大怒喝道:``我看在你母亲的份上,意欲饶你,你却不知好歹,反而想要老夫的性命么?''声到人到,就似兀鹰扑兔一般,一掌凌空劈下!正是:邪正本来如水火,追魂魔掌绝交谊。

欲知展元修性命如何?请听下回分解------

旧雨楼扫描,YackerOCR,旧雨楼独家连载

\chapter{第三十四回 魔掌追魂难与敌
苦心为友怨何辞}\label{ux7b2cux4e09ux5341ux56dbux56de-ux9b54ux638cux8ffdux9b42ux96beux4e0eux654c-ux82e6ux5fc3ux4e3aux53cbux6028ux4f55ux8f9e}

就在羊牧劳以全力扑击展元修的时候,铁摩勒与司空拔那-
对却已经分出胜负。原来薛红线年纪虽然最小,人却十分机灵,她身躯矮细,趁着司空拔横执降魔杵,正在架着铁摩勒长剑的时候,冷不防的钻过去便是一剑,这一剑正中司空拔的后腿,司空拔立脚不牢,被铁摩勒运劲一推,降魔杵倒打回来,登时打得他头颅开花,脑浆进裂!

铁摩勒立即赶来,这一来正是时候,羊牧劳凌空击下,铁摩勒大喝一声,左掌右掌同时发出,展元修也突然一个长身,运足了十成功力,同时发掌。

羊牧劳功夫也真了得,人未落地,在半空中便先踢出一脚,他的鞋尖上镶有铁片,但听得``当''的一声,铁摩勒的长剑竟给他踢飞,可是铁摩勒那一掌却和他硬碰上了!

羊牧劳身形未稳,双掌分敌二人,铁摩勒功力和他相差无几,展元修的掌法又飘忽之极,但听得``蓬''的一声,羊牧劳单掌接不了铁摩勒的掌力,被震得摇摇晃晃,他的右掌便稍稍打歪,展元修一掌从他的掌缘擦过,``卜''的一声,趁势打去,正中他的胸口!

展元修这一掌拼了性命的,饶是羊牧劳内功深湛,也给打得他五脏翻腾,眼睛发黑,但听得他``哇''的一声,一口鲜血便喷出来,身不由己的往前冲出几步,正巧与一个赶来援救的武士撞个满怀,把那武士撞得四脚朝天。

薛红线在地上拾起了铁摩勒那柄青钢剑,叫道:``王叔叔,你的剑!''铁摩勒笑道:``红线,多谢你啦。从今之后,你不要叫我王叔叔了,我姓铁,我的真名叫摩勒。''薛红线大喜道:``原来你就是摩勒叔叔,王姑姑早就提过你的名字了。''

展元修也道:``铁兄,多谢你啦!''王燕羽笑道:``你们别再客套了,赶快趁此时机,闯出去吧。''

司空拔被杀,羊牧劳受伤,安禄山的禁卫军有一大半已经慌了,只有一小半还听指挥,在副统领洪大存率领之下掩杀过来。铁摩勒大喝一声:``挡我者死!''横剑乱劈,一马当先,便冲杀出去!聂锋也紧随着铁摩勒冲出去叫道:``弟兄们,留点香火之情,日后还好相见!''聂锋以前曾做过安禄山禁卫军的副总管,他素来对手下甚好,禁卫军听得他这么叫喊,十个人中竟有五六个跑开。

洪大存向来与聂锋不睦,大怒喝道:``聂锋,你已背叛主公,还有什么香火之情?''挺起长矛,斜刺里冲来,便向聂锋挑去。

铁摩勒怒道:``聂将军,我替你杀这为虎作伥的奸贼!''旋风也似的一个转身,抡起长剑,当作大刀来使,使出``独臂华山''的恶招,``咔嚓''一声,把洪大存那根长矛斫成两段,第二剑正待劈下,聂锋叫道:``铁兄且慢下手!''疾忙抢上,轻抒猿臂,将洪大存一把抓了过来,朗声说道:``你不念香火之情,我还念同僚之谊!''用了一个巧劲,将洪大存一抛,抛出数丈开外。洪大存手下见聂锋义气深重,登时也都散了。

余下的一班卫士,有些是王伯通的旧部,不愿与王燕羽作对,有些与聂锋素有交情,虽然被迫上前,却只是虚张声势,还有一小部分本想截击邀功的,见铁摩勒如此凶猛,也吓得踌躇不前。

一行人便从禁苑的角门杀出,薛红线回头一看,见那羊牧劳像石像般的凝立场中,双手抱拳,仰面朝天,形状甚怪,薛红线大为纳罕,说道:``聂表叔,你瞧,那老魔头的怪模样。''聂锋一看,已知羊牧劳正在默运玄功,封穴疗伤,急忙说道:``不必答他,快快随我出去。''铁摩勒心中一动,却见王燕羽摇了摇头,原来王燕羽鉴貌察色,已知铁摩勒的心意,怕他还想回去杀羊牧劳,故此摇头阻止。铁摩勒知道羊牧劳内功深厚,自己回去也未必便有把握杀他,心里想道:``不可为我一人之事,连累大家。倒不如趁他运功疗伤的时候,早早离开这龙潭虎穴。''

骊山上本来是五步一岗,十步一哨,但聂锋乃安禄山手下的大将,以前又做过``禁卫军''的副统领,站岗的都认识他,见他率众奔米,一时之间,哪想得到他是已经背叛了的?有一两个胆大的问他,他便说道:``刚才园子里发现刺客,我是迫刺客去的。你们要紧守岗位,切不可离开!''这些站岗的当然不敢拦阻,待到后面的人追来,他们早巳去得远了。

下到半山,岗``肖已疏,聂锋方才松了口气。正自踌躇向何方逃走,忽听得马蹄之声,有如暴风骤雨,回头一看,只见一彪人马,从山上冲下来,当前一骑,不是别人,正是羊牧劳。原来羊牧劳仗着玄功,封穴止血,又敷了上好的金疮药,服下了千年的老参,气血调匀,已如未受伤一般。其时安禄山也已躲进密室,不须这么多武士保护,他调拨了本事最高的八个``御前待卫'',由羊牧劳率领,乘了青海进贡来的御马,下山急迫。

转眼之间,羊牧劳率领的这彪人马已经追到,铁摩勒大怒喝道:``好,咱们再来决个死战!''

羊牧劳哈哈笑道:``你这小子,胆量倒是不小,老夫今日就成全了你吧!''把手一挥,八名侍卫部跳下了马背,从两翼包抄而来。

聂锋心头一凛,说道:``他们布的是一字长蛇阵,首尾相连,击首则尾应,击尾则首应,击中间则首尾皆应。这八个人都非庸手,更有老魔头从中策应,实是不容轻敌。铁兄弟,你不可妄动。''

聂锋这边有四个大人,两个孩子,若被对方的长蛇阵掩杀过来,大人还可抵御,小孩却是可虑。聂锋是大将之材,懂得行军布阵之道,当下便叫四个大人各占一方,结成了四方阵和对方的长蛇阵对抗,两个孩子则在方阵之中,伺隙出击。

正在两阵对圆,即将厮杀之际,忽听得有一个极为刺耳的声音说道:``羊老三,你这是捣什么鬼,你不认得我的儿子和徒弟么?''话声未了,山坳里已闪出一个人来,正是展元修的母亲展大娘!

王燕羽连忙叫道:``师父,你快来!我正要带元哥回家见你,羊叔叔却说他不该逃跑,要捉他回去呢。元哥刚才几乎受他伤了!''

原来展元修自从知道师妹对铁摩勒有情之后,本已意冷心灰,不想再见师妹了,可是一缕情丝,终难割舍;尤其当他知道了师妹居在长安之后,更是放心不下,心想:``我与她虽然做不成天妻,但也不能眼看她误人歧途。''他还以为是王燕羽贪恋荣华富贵,故此到长安来依附父亲,做安禄山所封的什么国公府的``郡主''呢。因此一念,他便也偷人长安,暗会师妹。

两师兄妹见面之后,展元修才知道师妹的苦心,她不但是想劝父亲改邪归正,而且还襄助卢夫人暗中策划,有所图谋的。结果,展元修没有劝得师妹离开,反而被师妹劝得他留下了。他改姓换名,由王燕羽荐他到``禁卫军''中当了一名小队长,要不是今日发生了这件意外之事,还没有谁知道他呢。

展大娘是那日与王燕羽相会之后,才知道儿子的消息的。但``禁卫军''军令森严,很不容易告假。展大娘是个天不怕地不怕的脾气,她探听得安禄山今日在骊山宏张盛宴,想必儿子也要在园中执役,她又恃着与羊牧劳相识,便闯了来。哪知未进离宫,先在半山撞见了羊牧劳追捕她的儿子。

展大娘听了徒弟的投诉,不禁怒道:``羊老三,你刁;看僧面看佛面,怎的欺侮起我的儿子来了?我的儿子不稀罕当刊`么禁卫军了,我现在就来接他回去,你敢不放人么?''

羊牧劳与展大娘的丈夫当年是称兄道弟、并驾齐名的两大魔头,深知展大娘的脾气,当下欲抑先扬,哈哈笑道:``展大嫂,多年不见,恭喜你真好眼力,收了这么聪明伶俐的徒儿!''展大娘怔了一怔,说道:``羊老三,我与你说我儿子的事情,你怎么扯到我的徒弟身上来了?''

羊牧劳慢条斯理地说道:``你的徒弟有编故事的天才,我是不胜佩服之至!''展大娘双眼一翻,慢道:``难道她是说谎么:``王燕羽正要砌辞分辨,展大娘瞪了她一眼,说道:``让你羊叔叔先说,你忙什么?''

羊牧劳用手一指铁摩勒,说道:``展大嫂,你刚才问我认不认得令郎,现在我也问你认不认得这个小子。''展大娘道:``他是磨镜老人的徒弟,烧变了灰,我也认得。''羊牧劳道:``既然认得,这就好说了。今日之事,都是这小子引起的。这小子刚才大闹禁苑,意图行刺皇上,我身为大内总管,怎能不理?令徒与令郎却要庇护这小子,你说我该怎么办呢?磨镜老人与你有杀夫之仇,想来你不至于忘记前仇,为了徒弟而放过这小子吧?''

展大娘认出了铁摩勒之后,早已愠怒于胸,也猜到了王燕羽对他旧情未断,这时听了羊牧劳一番说话,气得几乎炸了,登时爆发起来,大怒喝道:``都是你这小子,害得我一家人不和,好,我今日先把你毙了!''话声未了,箭一般的向铁摩勒冲来。

可是她人还未到,展元修与王燕羽已不约而同地跃出方阵,一人一边,架住了展大娘的双臂,展元修叫道:``娘,且慢动手!''展大娘怒道:``不肖的奴才!你要丢尽我的颜面吗?''展元修道:``我与铁兄已交上了朋友,娘要杀他,请先杀我!''王燕羽说道:``师父,咱们的家事,关起门来,慢慢再说。但今日我与元哥受了外人的欺负,你老人家难道反要帮忙外人,当众示弱吗?''

羊牧劳连忙说道:``大嫂,你是女中豪杰,素来果断英明,怎的今日就糊涂了?家事可以慢慢再理,目前这小子乃是你仇人的徒弟,你放过了他,以后再要找他,可就难了。不过话说回来,大嫂,要是你为了儿女之情,爱屋及乌,投鼠忌器,连带这小子你也要庇护起来,那我也没有什么好说了,你要听令徒的话,打我骂我,我都由你!''

羊牧劳这番带刺的说话,比王燕羽说的更厉害得多,尤其那``儿女女之情''四字,更为刺耳,可以解释作展大娘的溺爱儿女之情,也可解释作王燕羽与铁摩勒的``儿女之情''。若作后一解释,那就无异是说展大娘眼睁睁的看着徒弟勾引仇人,而自己还在给徒弟牵着鼻子走:

展元修道:``妈,我还记得爹爹有这么一条家训,咱们做什么恶事都可以,但却不可依附公门。这姓羊的是安禄山的鹰犬,咱们犯得上帮他的忙吗?妈,你若是要儿子的话,就请你别管这里的事丁。''

展大娘虽然凶恶,但她只有这一个儿子,她看儿子说话的神气,显然已是下了决心,要是自己当着他的面杀了铁摩勒,只怕母子俩就要一生不和!

展大娘气得面色发青,终于咬了咬牙,说道:``好,我不管这里的事,也不许你们管,你们都随我回去!''顿了一顿,再转过来对羊牧劳道:``羊老三,我不想分你的功劳,这姓铁的小于留给你吧!''

展元修还想说话,展大娘双臂平伸,一手一个,将他和王燕羽抓牢,狠声说道:``你们若然不肯随我回去,那我也就要先杀掉这小子了。''展元修没法,只好让他的母亲拖着走。

羊牧劳拱手笑道:``大嫂慢慢走,恕我不远送了。我料理了这小子,再来向你请罪。''展、王二人一走,铁摩勒这边的实力差不多减了一半,羊牧劳合八名``龙骑卫士''之力,所要对付的只是聂锋、铁摩勒与两个小孩子,那自是稳操胜算了。所以羊牧劳已无需再激展大娘来给他帮忙

展大娘拖着儿子和徒弟刚走出两步,忽见山拗里又闪出两个人来,走在前面的是个江湖郎中打扮的老头,后面跟着的是个长得很秀丽的少女。

那少女娇声笑道:``王家姐姐,真是巧呀,想不到在此时此地,竟又碰见了你!怎么,你就走了么?''接着又扬声叫道:``摩勒,你好么?你想不到我会来找你吧?你的运气倒真不错,每次遇难,总会有人帮忙!''

铁摩勒见这两人,当真是惊喜交集。原来说话的这个少女正是他的未婚妻韩芷芬,那江湖郎中打扮的老头,乃是他的岳父,天下第一点穴名家韩湛。

韩芷芬话中有刺,王燕羽听了十分难受,也便冷冷的``回敬''过去:``韩姐姐,你来得正是时候,快上去帮忙吧,要不然你的丈夫可要给人家抢走啦!''韩芷芬笑道:``你是说这姓羊的老魔头么,我倒放心得很,凭他这点能力,还抢不了我的丈夫。''展大娘正自没好气,见韩芷芬正走过来,侧目斜睨着她(其实韩芷芬这目光是射向王燕羽的);便即勃然怒道:``你是什么人,在我面前敢这样大模大样?''韩芒芳道:``我是什么人,你问你的徒弟好了''奇怪,好端端的你发什么脾气,你瞧着我不顺眼么?''展大娘``哼''了一声,捏牢了王燕羽的手臂喝问道:``快说,她是什么人?''

王燕羽未曾说话,羊牧劳已在叫道:``大嫂,你不认得这位鼎鼎大名的天下第一点穴手,韩老先生么?他和磨镜老人乃是莫逆之交,又是这位铁、铁少侠的岳丈大人。''

韩湛微笑道:``羊大总管,你给老朽脸上贴金,实是愧不敢当。不错,咱俩父女是来寻觅小婿的,小女脾气不好,且又赶路匆忙,若有礼节不周之处,还望你展大娘大度包容。''

展大娘吃了一惊,心道:``原来这个不起眼的老头竟是韩湛!他的女儿又是铁摩勒的未婚妻!''

王燕羽忽道:``元哥,咱们的事该告诉妈了。''王燕羽突如其来的插上这么一句话,展大娘不禁诧道:``什么事情?''

王燕羽脸上一片娇红,羞怯怯的低声说道:``我和元哥已经讲好了,只等你老人家替我们选一个日子。这位韩姐姐是我的好朋友!难得意外相逢,妈,你也请她来喝杯喜酒好吗?''

展元修呆了一呆,失声叫道:``羽妹,你\ldots\ldots{}''王燕羽捏着他的手,若不胜情似的娇嗔说道:``你别这么看着我好吗?怪难为情的。''展元修神迷意荡,话也就说不出来了。他做梦山想不到王燕羽会对他如此,他到长安以来,根本就没有和王燕羽谈过半句婚事,他是早已绝望的了。然而王燕羽现在却说是与他早已讲好了的。``这是骗我呢?还是我在做梦?''他看看师妹的神情,却又似是一片真情流露,虚假不来。

王燕羽这时的心情复杂之极,她说的乃是假话,但却非全是假意,原来有三个原因,第一,她知道与铁摩勒结合已是绝无可能,而韩芷芬又恰巧在这时候到来,对她冷嘲热讽,故此她急于要向韩芷芬表白。她这活实在是说给韩芷芬听的。第二,她怕师父被羊牧劳所煽动,又要枝节横生,因此就以婚事为由,转移她的注意,也可以令她快些离开此地。第三,在这几个月来,她也越来越感到师兄对她的真情,感到师兄的人品与武功都不在铁摩勒之下。为了她,他不惜留在长安,屈身在``禁卫军''中作个小卒;为了她,他与铁摩勒化敌为友,宁愿为了袒护铁摩勒而违抗母亲,这都是难能可贵的地方。因之,即使不是韩芷芬到来,她迟早也会答应做他的妻子的。

展大娘听了,果然又惊又喜,``骂''道:``原来你们早巳说好了,你这鬼丫头,怎么对我也瞒得密不透风?''

韩芷芬何等聪明,一听就知她是要向自己表白,倒有点不好意思起来,心里想道:``原来她也早已有了未婚夫了,这么说,倒

韩芷芬嫣然一笑,说道:``王姐姐,恭喜,恭喜!但只怕我不能米叨扰你的喜酒了。''

展大娘满怀高兴,同时她对韩湛也有点顾忌,当下说道:``韩老先生,咱们都是为了儿女之事,各人忙各人的去吧,请恕我也失陪了。''韩湛迈步向前,沉声向羊牧劳说道:``羊大总管,幸会,幸会!老夫今日替铁昆仑践约来了。''羊牧劳心头一凛,说道:``韩老先生,咱们似乎没有什么过节,今日我追捕令婿,山只是各为其主,不得不然。老先生若是见怪,咱们也还可以商量。''

韩湛冷冷说道:``这是两桩事情,我女婿的事情我固然要管,铁昆是我的老友,如今又是我的亲家,他人死不能复生,他与你订下的约会,说不得只好由老夫代为践约了。''羊牧劳道:``不知韩老先生要替他践什么约?''韩湛道:``羊大总管记性素来很好苎,难道反而把这样重要的约会忘怀了么?二十年前,铁昆仑与你在燕山比掌,当时你趁他撤掌收招的时候用力暗伤了他,铁昆仑曾约你二次较技,那时他尚未知道自己受伤已重,还以为伤好之后,可以再领教你的真实功夫的。哪知不久他便因伤而死,抱恨长眠了。要是我不替他践约,只怕他九泉之下,难以暝目。''韩苎芬叫道:``爹,他是在想拖延时候,你还与他多说作什?等会儿他的大队人马到来,''咱们就要大大吃亏了。''

羊牧劳的心思给韩芷芬一口道破,老羞成怒,``哼''了一声,冷笑道:``韩姑娘,你也忒把老夫看得小了。好吧,那么这两件事情就分开来办。''说到这里,稍顿一顿,便一挥手道:``你们去办公事,我来领教韩老先生的点穴功夫。''此令一下,那八名``御前待卫''组成的长蛇阵便立即向铁摩勒诸人掩杀过去。与此同时,羊牧劳与韩湛亦开始交手。

羊牧劳展出``七步迫魂''的杀手,第一步便踏正中宫,扬掌劈下。这一掌柔中带刚,袭胸插腹,好不厉害!韩湛冷笑一声,食指一弹,但听得``嗤嗤''声响,一缕劲风射了出去。他的指力已练到``隔空点穴''的境界,可以在十步之外,运暗劲伤人,那``嗤嗤''声响,便是他的指力激荡气流所致。

羊牧劳一掌劈出,忽觉虎口似被大蚂蚁叮了一口似的,大吃一惊,急忙移形换步,第二步便转过``离''方,走出``坎''位,左掌扬起,再袭韩湛的腰背。他这``七步七掌'',每走一步,便发一掌,步法奇妙,而且一掌强似一掌,韩湛也不由得心头一凛,``怪不得铁昆仑当年伤在他的掌下。''

那八名``御前侍卫''组成的长蛇阵冲杀过来,韩芷芬早已到了聂锋所布的阵中,与铁摩勒互为犄角之势,并肩御敌。那些侍卫见识过铁摩勒的功夫,都不大敢去和他硬碰,长蛇阵首尾一合,位在``蛇头''和``蛇尾''的两名卫士,不约而同的都把兵刃向韩芷芬斫去。这两名卫士一个是羊牧劳的大弟子单雄,一个是海盗出身的蒙贯,乃是八名``御前侍卫''中本领最强的两个。

哪知韩芷芬出手比铁摩勒更为狠辣,她展开家传的``刺穴''功夫,剑光一闪,只听得``唰''的一声,已刺中了蒙贯膝盖的``环跳穴'',蒙贯站立不稳,``咕咚''一声,便倒下去。单雄一拐打来,打不中韩芷芬,却把蒙贯头颅打碎了。

韩芷芬笑道:``摩勒,你真是吉人天相,遇难成祥!''笑声中一个盘龙绕步,剑光闪处,``咔嚓''声响,又把单雄的中食二指削去。单雄惨叫一声,弃拐飞逃。

铁摩勒抡起长剑,当作大刀来使,手起剑落,劈翻了一个卫士,说道:``不错,你们来得真巧,这场灾难,我大约可以躲过了。''他们一面杀敌,一面谈天,简直毫不把安禄山帐下的这八名高手放在眼内。

韩芷芬笑道:``我不是说我和爹爹,而是说那位王小姐呀,你不是幸亏得了她的帮忙吗?刚才你和她联手抗那魔头,我已经瞧见了。''铁摩勒面上一红,含糊说道:``不错,是幸亏了她,还有她的师兄,就是刚才和她在一起的那个男子。''说话之间,长剑横挥,又把一名卫士打跑。

本来这八名``御前侍卫''组成的长蛇阵若有羊牧劳居中策应,绝不至于这样容易被他们击破,只因少了一个羊牧劳,``蛇无头而不行'';更兼他们一上来就料敌错误,被韩芷芬以快刀斩乱麻之势一下子就杀伤了两个本领最强的,跟着又给铁摩勒伤了两个,``长蛇阵''总共八人,如今等于一条蛇被斩了半截,余下的哪里还敢恋战,登时一哄而散。薛红线叫道:``可惜,可惜。我还未曾发市呢,他们就都跑了。''

恰好就在这时,韩湛与羊牧劳那边亦已分出高下,原来羊牧劳接连走了六步,变换了六种步法掌法,都占不到丝毫便宜,迫不得已,把最后一招杀手拿了出来,这最后的一步一掌乃是要欺身直进,双掌齐发,拍击敌人的两边太阳穴的。这一招厉害无比,纵使敌人的武功与自己在伯仲之间,这双掌一拍,也能制敌死命。但使出这最后的绝招,也有个危险之处,因为是欺身进击,若果敌人比自己强得多,那就等于送上去挨打了。

羊牧劳在发招之前,也曾估计过这个危险,但他自恃绵掌击石的功夫已到了炉火纯青之境,所用的身法步法又奥妙无穷,心想韩湛的功力虽深,大约也不过比自己稍胜一筹而已;而且在这时候,他的后援尚未赶来,长蛇阵却已冰消瓦解,要是不行险求胜,待到铁摩勒等人一来合围,自己必将性命不保。

哪知韩湛早已胸有成竹,羊牧劳的第七步刚一踏出,韩湛也突然使出怪招,脚跟支地,一个盘旋,陡然间只见长衫飘飘,人影叠叠,羊牧劳双掌拍下,只听得``蓬''的一声,如击厚革。就在这刹那间,一缕劲风,宛如利箭,已是疾射而出,直刺羊牧劳的脑海穴。羊牧劳大叫一声,腾身飞起,他的功夫确也了得,受了重伤,居然还能辩别方向一纵身恰好落在一匹马上,双腿一夹,那是匹久经训练的御马,登时转过马头,向山上疾驰而去。

原来韩湛这一招有个名堂,叫做``旋风舞天魔指'',以``旋风舞''身法使得羊牧劳目眩神迷,双掌就不能正中他所欲击的方位,而他则可以趁羊牧劳击中他的时候,双掌无法回防,骤然使出最强劲的``天魔指'',钻人空门,点中他的要害穴道。

韩芷芬大惊,连忙过来问道:``爹,你怎么了?''韩湛笑道:``羊牧劳号称七步追魂,果然名不虚传。但侥幸我这老骨头山还禁

受得起,未曾给他追了魂去。''韩芷芬定睛看时,只见父亲的后心已有一幅衣裳破裂,现出了一个掌印。

铁摩勒这时也已走了过来,见韩湛没事,放下了心。以子婿之礼,见过了韩湛之后,笑道:``不知这老魔头性命如何?我倒有点为他担忧。''韩芷芬诧道:``你怎么为他担忧起来了?''铁摩勒道:``要是他就此死了,我岂非不能亲于报仇了吗?''韩芷芬问道:``爹,他是不是中了你的的天魔指。''韩湛道:``不错,你的功夫果然长进多了,居然看得出来。''韩芷芬又奇怪道:``咦,那他怎么还能奔马而逃?你不是说过,任何厉害的敌人,只要一给天魔指点中,就决难活命,要命毙当场的吗?''韩湛道:``天魔指练到最高深的境界,确能如此。但我的功夫却未曾练得到家,所以摩勒不必担忧,那老魔头大约还能活命。''其实并非他的功夫未练到家,而是他已想到了铁摩勒要亲手报仇的心意,所以手下稍稍留情,只令羊牧劳受到内伤,如此一来,铁摩勒要亲手报仇,就容易了。

铁摩勒问道:``爹,你老人家怎么知道我在这儿?''韩芷芬笑着插口说道:``你以为你躲在薛家就没人知道了吗?''韩湛解释道:``我们这次来京,事先曾得卫老前辈作函先容,认以了此间几位丐帮朋友。今早到薛家附近查访,经常在那里词饭的叫化子山是丐帮中的,他告诉我们,说是薛聂两位将军和一个少年天方拂晓就出门去了,我详细问了那少年的模样,料想是你。至于安禄山今日在骊山宏张盛宴,这消息我们昨天就知道了。两件事情一连起来,你们的去向当然也可猜得十之八九了。摩勒,你的胆子可真是不小啊!''

铁摩勒心中一动,连忙问道:``你们为何到薛家附近查探?''这时聂锋携了隐娘、红线,劝;已走了过来。通了姓名,见过礼后,韩湛笑道:``聂将军,你家中此刻只怕已有贵客到`了。''聂锋眉头深锁,说道:``正是呢,闹出了这样的大事,羽林军定然奉命去抄我们的家了。''韩湛道:``哦,你们闹出了什么大事?我正自不明白,聂将军你何以也与羊牧劳作对?''聂锋也说道:``原来你所指的贵客不是指安禄山的手下么?''

说话之间,只听得山上人马喧闹之声,韩湛道:``追兵已到,咱们边走边说吧。''聂锋道:``我认得一条羊肠小路,崎岖险峻,人马难越,你们跟着我来。''这一行人,连同隐娘、红线两个小孩子在内,个个轻功了得,不消半个时辰,已从小路翻过山背,聂锋方始松了口气,但随即又皱着眉头说道:``我此刻真不知该向何处去了。若是回家,只怕乃是自投罗网。嗯,韩老前辈,你刚才说有贵客会到我家,gr5是何人?''

韩湛捋着胡子道:``摩勒,你刚才不是问我何以会到薛家附近查探么?现在可以一并告诉你们了。聂将军,我所说的`贵客'便是段圭璋段大侠,他很感谢你过去对他暗中相护之恩,他今天前往薛家,一来是要见他的亲家嫂子卢夫人,二来也是想见见你呢!''铁摩勒大喜道:``原来我的段姑丈也来了么?''聂锋叹口气道:``可惜他来得太刁;凑巧了!''

铁摩勒道:``不然,我说他来得正是凑巧。他是不是和我的姑姑同来?''韩湛点了点头,铁摩勒道:``有他们夫妇二人,千军万马,也拦他们不住。要是安贼的羽林军当真已往抄你们的家,他们必然不会坐视。''聂锋道:``就不知是否刚好碰上?事发之时,薛将军已单独走了,那时我还未曾去助铁兄,他们也还未知道你是薛将军带来的。也许薛将军已先到家中,带了家人走-了。''薛红线忽地问道:``聂叔叔,我爹爹为何不理我就先跑了?我要我的爹爹。''

铁摩勒一阵心酸,忍不住道:``红线,你这个爹爹为什么不理你,你回去问卢妈就知道了。''薛红线年纪虽小,也听出这话有蹊跷,大为奇怪,问道:``卢妈今天并没有同来,难道刚习`所发生的这一些事情她会预先知道不成?为什么要去问她?再说,每一个人只有一个爹爹,你却说什么这个爹爹,那个爹爹的,这是什么意思?难道我有两个爹爹?''铁摩勒叹口气道:``红线,有许多事情你不明白的,我一时间也说不清楚。但你别心急,卢妈会一一告诉你的。总之,你只要记得卢妈是你最亲的人,你听她的话就行了。''铁摩勒本来已有点忍不住,想把她的身世告诉她`了,但一来因为``说来话长'',现在急于逃难,还不是说这些话的时候;二来她的身世也应该她的生身之母告诉她才最适合,铁摩勒不想越俎代庖。

薛红线心想:``卢妈比我妈还疼我,天天伴着我,本来就是我最亲的人,我当然听她的话,还用得着你说吗?''当下就嚷道:``那么咱们快快回家去问卢妈吧。''聂锋道:``卢妈在不在家,还未知道呢?''聂隐娘年纪较大,懂得推测事情,说道:``不错,今天咱们闯下了大祸,薛伯伯先逃走,看来怕是要赶回去报信,叫家里的人快逃,那么卢妈当然也跟着逃了。''

聂锋道:``现在就是这个问题,不知道薛嵩回过去了没有?或者是已单独逃到别个地方去了?好在咱们人多,可以分成两路。据我所知,薛嵩有一支亲军,那是他带了多年的部队,绝对听他指挥的,现在驻扎在福隆寺。他要逃必定是逃到那里,好拥兵自卫。不如这样吧:我带这两个孩子到福隆寺去找他,铁兄弟,请你和韩老前辈到我家去看看,要是真的已发生了事情,你们也好救援。''铁摩勒道:``这样也好,总有一处找着。''

聂锋想了一想又道:``我知道有小路去福隆寺,沿途的哨所不多,那一带驻军的军官又都是我和薛将军的部下,我去福隆寺不打紧,你们回去可得小心,街上现在恐怕已经戒严了。只怕也已有人认得你了。''

韩湛道:``我有办法,我给摩勒变个面貌吧。''取出随身所带的易容丹,用山水化开,涂在铁摩勒的面上,登时把他变成了个``黑张飞''模样的莽汉。铁摩勒临流自照,也不觉好笑,当下就想把军装脱下来,韩湛摇手道:``这套衣服不用换。''聂锋道:``对,你仍然以校尉的身份出现,更方便些。我以前给你的那面腰牌还在吗?''铁摩勒道:``巧得很,我正带在身上。''

聂锋笑道:``这就更妙了。我现在虽已造反,这面腰牌,想来还可通行无阻。铁兄弟,拜托你了,若是我的家人未逃,就烦你护送她们到福隆寺来。''铁摩勒道:``聂兄放心,我理会得。''

计议已定,当下便分道扬镳。铁摩勒带路,与韩湛父女回到长安街市,果然街上已布满士兵,行人绝迹。铁摩勒易容之后,相貌凶恶,又穿着军官服饰,没人敢问他,连腰牌也不用掏出来看。但跟在他后面的韩湛父女,却曾碰过几次查问,每次被查问的时候,铁摩勒就放粗了喉咙喝道:``我家里有病人,我请的大夫你敢阻迟?病人坏了,我要你填命!''那些兵士给他一喝,都是快快赔笑,连忙放行。

但到了薛、聂二家所在的这条街道,气氛便大大不同了,只见满街都是披着``锁子黄金甲''的羽林军官,铁摩勒刚踏进街口,便有军官上来喝道:``你是那个番号的军官,到这里来作什么?这两个又是什么人?''铁摩勒心想:``假作是请大夫,只怕是不行了。这里除了薛、聂二家之外,其他都是百姓人家。''他人急计生,眉头一皱,便低声说道:``我是奉了主公之命来的。主公说要留活口审问,怕要犯伤重,叫我带了御医来,她是御医的女儿,随同来照料伤犯的。''军官听他的口气,似乎是宫中的侍卫,安禄山的侍卫,这军官本来就认得不全,当下将信将疑,放不放行,一时难决,问道:``带有总管府的公文么?''铁摩勒稍稍运劲一推,沉声说道:``事情紧急,我奉了主公的口令,哪里还有功夫去备办公文?''那军官乃是羽林军中一个出名的力士,但给他轻轻一推,却已站立不稳,险险跌倒,心里想道:``看来当真是大内的高手了!''因此铁摩勒这一推,不啻证明了他的``身份'',这军官非但不发怒,反而连声诺诺,闪开-旁,让他们过去。

将近薛家之门,只见又有许多羽林军挥舞长鞭,将一群叫化子赶得东跑西窜,铁摩勒正在奇怪,只听得那些羽林军骂道:``我们在捉拿钦犯,又不是办婚丧大事,有酒肉分,你们这群化子赶来瞧热闹作甚?当心将你们的腿都打断了!''那些化子叫道:``我们都是在这条街道乞讨的,一时来不及走避,你们也用不着这样凶啊!''转眼之间,都逃进横街小巷,四散无踪。铁摩勒猛然省悟,猜想这群化子必定是丐帮中的探子无疑。

羽林军将薛、聂二家团团围着,刚才那个军官是在外面负责巡查的领队,他有心巴结铁摩勒,亲自陪他到门口,说声:``这位都尉大人领御医前来,你们让他们进去。''铁摩勒不须多费唇舌,立即便往里闯。

铁摩勒刚跨进院子,便见到好几个浑身浴血、损手折足的武士跌跌撞撞地跑出来或滚出来,他们只道铁摩勒是来增援的好手,慌慌张张地叫道:``快、快进去!那对贼夫妻好不厉害!''铁摩勒心里大喜,想道:``果然是他们了。''拔出长剑,便冲进大堂。

只听得杀声震天,白刃耀眼,段圭璋夫妇在众武土的包围中高呼酣斗,但却不见薛嵩。铁摩勒正待上前助战,忽听得有人叫道:``姓段的你还敢顽抗,我们就把薛、聂两家杀得一个不留!''

有人叫道:``段圭璋,你本是江湖上的一条好汉,为何要替薛嵩卖命?''

只见另一群武士,已把薛、聂两家十几口男女老幼,全身捆绑着,从后堂里推了出来,铁摩勒定睛看时,只见卢夫人和那个姓侯的管家都在其内。原来这些武士中有人认得段圭璋,但却不知道他是为了救卢夫人来的,只道他是与薛嵩或聂锋有甚交情,故此他们把薛、聂二夫人推到最前,在她们的背后各有一柄明晃晃的利刃指着,准备威胁段圭璋夫妇。段圭璋厉声喝道:``你们敢动她们一根毫发,我将你们杀得一个不留!''一个军官模样的人喝道:``好呀,他这样倔强,先给点颜色给他看看!开刀!''

``嗖''的一声,薛嵩妻子的一边耳朵已给快刀削了下来,痛得她杀猪般的大叫大嚷。

那些武士们``重视''的乃是薛嵩与聂锋的妻子;但铁摩勒最着紧的却是卢夫人,他一听得那一声``开刀'',生怕卢夫人也玉石俱焚,同遭毒手,连忙大喝一声``住手!''持刀在卢夫人背后的那名武士见他穿着军官的服饰,发狂的似向自己奔来,不由得怔了一怔。说时迟,那时快,只听得``当啷''一声,铁摩勒早己飞出了一颗铁莲子,将那个武士的尖刀打落。

可是如此一来,铁摩勒的目标也登时暴露,另一个武士突然抢快两步,一手抓着了卢夫人,霍的一个``凤点头''避开了跟着打来的两颗铁莲子,也是一声喝道:``住手!你敢再放暗器,我就先把这妇人毙了!''他起脚一踢,把一张桌子踢得四分五裂,碎片飞到了铁摩勒的面前。铁摩勒见他武功甚高,卢夫人又已落在他的手中,突袭救人的伎俩,只是可一而不可再,由于``投鼠忌器'',也就被他吓住,因此不敢再向前冲。原来这个武士乃是羊牧劳的三弟子,名叫尚昆,在羊牧劳的七个徒弟中,以他的武功最高,也最机智。他虽然不认得铁摩勒,也不知道卢夫人的身份,但见铁摩勒这般动作,却已看出了他是个``冒牌''的军官。心想:``敌方要费如许心力来救一个奶妈,这奶妈的身份必非寻常!''正是:救星虽是从天降,无奈灾星尚未消。

欲知后事如何?请听下回分解------

旧雨楼扫描,海之子OCR,旧雨楼独家连载

\chapter{第三十五回 十年忍辱仇终报
再度寻儿恨未消}\label{ux7b2cux4e09ux5341ux4e94ux56de-ux5341ux5e74ux5fcdux8fb1ux4ec7ux7ec8ux62a5-ux518dux5ea6ux5bfbux513fux6068ux672aux6d88}

尚昆虽然镇定,但其他看管人质,的武士,被铁摩勒这么突如其来的冲杀,却难免引起骚动,乱了阵脚,说时迟,那时快,韩湛父女也早已如飞扑至,韩湛以闪电的手法,一指点倒了伤害薛夫人的那名武士,韩芷芬则用一口飞刀插入了看管聂夫人那名武士的心胸,薛夫人只被削了一只耳朵,聂夫人则全然元损。韩湛道:``芷芬,你保护二位夫人,我去助摩勒一臂之力。''

他正想用``隔空点穴''的本领,点倒尚昆,那尚昆却是狡猾之极,他认得韩湛是天下第一点穴名家,登时退到了屋角,背靠着墙,将卢夫人牢牢抓着,遮在前面,冷冷笑道:``韩老前辈,我知道你有隔空点穴的本领,但你总不能隔物传功吧!你要是不怕毙了这妇人,你就尽管施展。''尚昆以卢夫人作挡箭牌,韩湛也无计可施。

卢夫人却是神色自如,不但不害怕,反而喜上眉梢,说道:``摩勒,你这般模样回来,想是已闹出事了。薛嵩和聂锋呢?''铁摩勒道:``聂锋父女和你的女儿都与我一道,今日已在安贼的离宫里大杀了一场,聂锋已然决意反了。看这情势,薛嵩也是不反不成,他既然不在这里,那就定是到福隆寺招集他的亲军去了。''卢夫人哈哈笑道:``好,安贼众叛亲离,死期不远了。你们等着,还有更好看的在后头呢!''尚昆喝道:``你罗哩罗唆胡说些什么,快叫他们退出去!不然就叫你先尝尝我的厉害!''卢夫人笑道:``我若怕死,也不会在薛家里做奶妈了。我虽然不能亲睹安贼覆亡,但夫仇指日可报,死亦可以无憾。''忽地提高声音叫道:``大哥、大嫂,我的女儿多劳你们照顾了!''话声未了,只听得一声惊叫,卢夫人已是血染罗衣!但这一声惊叫却不是卢夫人发出的,原来卢夫人有心效法她的丈夫,让段圭璋他们可以毫无顾忌的杀敌,竟然也用她丈夫史逸如当年自尽的法子,向后一靠,硬碰那武土的刀锋。这一声惊叫,乃是尚昆发出来的,他做梦也想不到卢夫人会有这个动作。

段圭璋一声大吼,猛狮般地冲杀过来,窦线娘更快,她人还未到,弹弓先发,尚昆失了``挡箭牌'',被窦线娘的弹丸打个正着,铁摩勒一跃而上,长剑出手,硬生生的将他``钉''在地上,从前心芽过了后心。

窦线娘抱起了卢夫人,道:``好嫂子,苦了你了。''卢夫人含泪微笑道:``重见你们,我死也死得安乐了!''窦线娘叫道:``不,你不能死!''她察看了一下卢夫人的伤口,见伤口很深,但听她的心脏还在跳动,急忙先用金疮药替她敷上。

段圭璋喝道:``挡我者死,避我者生!''一柄长剑指东打西,指南打北,杀得那群武士鬼哭神号。韩湛则以穿花绕树的身法,施展他的点穴功夫,武士们一被他点中穴道,便即不能动弹。不过片时,那群看守人质的武士都被他点倒。

房中虽然有若干好手,但他们应付段圭璋夫妇已感不易,更何况现在又添上了韩湛父女和铁摩勒三人,等如三只插翼的猛虎,一轮厮杀,武士们都已不能在屋子里立足。

可是段圭璋他们杀出了大门,却反而碰到了困难。街上满是安禄山的羽林军,在屋子里他们不可能都挤进来,现在到了街上,却不容易冲过去了。当然,假若毫无拖累的话,以段圭璋和铁摩勒他们的本领,要杀出重围,也还不太困难,但现在他们却要照顾薛嵩和聂锋的妻子,还有那些跟着他们突围的两家家人。聂锋的妻子还好,可以自己走路,薛嵩的妻子则几乎吓破了胆,要韩芷芬拖着她走。还有,窦线娘背着重伤的卢夫人,也得步步小心,不敢跳纵,怕震动了她。而且还要提防冷箭。段圭璋、铁摩勒并肩冲杀,奋战夺路,韩湛挥舞一件长衫,拨打羽林军射来的冷箭,还好是因为在混战的局面下,只有一些技艺精良的羽林军弓箭手才敢发箭,不至于乱箭射下。可是,也已有几个家人中箭伤亡。那姓侯的老管家也中了一箭,幸非要害,铁摩勒与他交情甚好,便拖着他走。

正在吃紧之际,忽见羽林军的后队阵形大乱,一大群叫化子从横街小巷里钻出来,个个手持打狗棒,碰到羽林军便打。羽林军的统带沐安大怒道:``岂有此理,叫化子也敢造反!''指挥一部分兵士便去兜截他们,一个老叫化哈哈大笑道:``安禄山这胖猪也敢造反,我们为什么不能造反?哈哈,你们这班披着老虎皮的,平日最会欺负我们,现在可要你们尝尝我们的厉害了!''沐安大怒,策马向前,居高临下,舞起长枪,一枪向那老叫化挑去,严老叫化叫道:``沐大人,你下来吧,咱们公公平平地打一场!''``呼''的一声,忽地抛出了一条绳索,套着那杆长枪,竟把沐安拉``马来。原来这个老叫化乃是京都的丐帮首领,疯丐卫越的师弟武铁樵,他的功夫虽是远远不及师兄,但要对付一个御林军的统带,却还绰绰有余。段圭璋这次人京,与丐帮早有联络,所以武铁樵一听得段圭璋在薛家出事,便立即亲自率领丐帮弟子,赶来助阵。

沐安大吃一惊,叫道:``你是什么东西,配和我打。''抛了长枪便跑。

武铁樵哈哈笑道:``大人,慢慢的走,提防摔跤。''沐安换过战马,指挥羽林军从两面包抄,这时他已知道这群叫化子个个都有武功,再也不敢轻敌,更不敢亲自出来与他们交手了。

段圭璋这边的人得丐帮来援,精神大振,奋力冲杀,不消多久,双方已经会合。但因为丐帮弟子是武铁樵在仓卒之间召集的,人数虽有四五十名,与羽林军相比较,究竟还是众寡悬殊。沐安将铁甲军调上来,个个手执盾牌,挡住去路,弓箭手就在铁甲军的后面放箭。丐帮冲杀过去,固然伤了不少铁甲军,但丐帮弟子也有好几个被箭射伤。几经艰苦,才杀出了街口,羽林军却越来越多了。

正在激战之际,忽见羽林军又起骚动,在长街另一端街口的

栏栅突然打开了,土兵们都向两边闪避,只见一骑快马,疾驰而来,骑在马上的是个面白无须的官员。薛、聂二夫人知得他是安禄山的``太子''安庆绪的太监总管李猪儿。

只听得李猪儿大叫道:``太子与丰大总管有令,令羽林军从速回宫!''带领这一支羽林军的统带是安禄山的亲信沐安,副统:带二人,都是羊牧劳的弟子,一个即是刚才死掉的尚昆,另一个,还活着的是羊牧劳的二徒弟程坚。沐安犹疑了一下,说道:``咱'们是奉了主公之命来捕反贼的,怎的太子又突然要咱们回去?咱们是该继续执行主公的命令呢?还是听太子之命?''程坚道:``薛嵩、聂锋都不在家,要捉他们也捉不到了。也许他们已带领叛军,攻打东宫,所以要咱们回去救驾。依我看来,还是听太子之命为是。''程坚是羊牧劳的徒弟,李猪儿所传的这个命令乃是``太子''与羊牧劳联合发出的,所以程坚自是主张要服从``太子''的命令。

沐安见程坚如此主张,而程坚的武功比他强,靠山又比他硬,他没了主意,只好依从,一声令下,这支羽林军后队改前队,登时撤退。

窦泉娘背着的户头人本已气息奄奄,这时忽然振作精神,向薛嵩的妻子招了开手下韩芷芬拖着她走过来,卢夫人道:``姐姐,刚才那个官儿似乎到过贵府,他是不是李猪儿。''薛嵩的妻子道:``不错,他正是李猪儿。''卢夫人道:``段大哥,你们派个人去探探消息,看是发生了什么事情?''段连障道:``嫂子,你不必操心,我们自会派人去查探。''当下与武铁樵商量,派出了两个丐帮弟子,并吩咐他们探听了消息之后,再想法买点人参,到福隆寺相会。

羽林军已退,段圭璋等人与丐帮人众从容走出,所经过的街道虽然还有许多兵士,但那些兵士呼啸成群,个个都好似慌慌张张的向皇城的方向跑。段圭璋等人手执刀剑和一大帮叫化子在一起,本来形迹极是可疑,但那些士兵却也无一人上来盘问,竟是各顾各的,两不相干。段圭璋大为奇怪,心里暗想:``难道薛、聂二人当真有那么大胆,敢率领军队去攻打皇宫?''

福隆寺在城东的白马山上,那里已是远离市中心的郊区,众人来到庙前,已将近黄昏时分,只见庙门紧闭,林子里也并没有发现土兵,但见随地都是抛弃了的破旧帐篷和一些难以搬移的重物,甚至还有一些盔甲。

薛嵩与聂锋的妻子面面相觑,那老管家道:``两位夫人先别着慌,且待老效上去叫门看看。''他受了箭伤,一跷一拐的上去叫门,过了半晌,里面有人问道:``是谁?''那管家喜道:``海哥儿,是你侯二叔呀,你听不出吗?两位夫人来了,还不快开门?''里面的人又问道:``两位夫人与谁同来,有多少人?''侯管家着了恼,叫道:``好多人,我没工夫数。你开了门自己看吧。''铁摩勒笑道:``侯老伯,你别焦躁,待我来说。''上前朗声说道:``我是聂将军的好朋友铁摩勒,和段大侠他们护送你们两家的家眷来了。''话声未了,果然那庙门便即打开。

只见一个老和尚和一个中年汉子走了出来,那中年汉子见薛夫人泪痕满面,鬓边血渍斑斑,一边耳朵已不见了,他吓了一跳,连忙跪下道:``夫人受难了,请恕小的迎接来迟。''侯管家一把揪着他道:``你还说呢,叫了半天你才开门。''那汉子道:``二叔,你别见怪。薛、聂二位将军临走时吩咐的,要问清楚了是铁相公和段大侠前来才能开门。他们担心你们已被羽林军捉去了,天幸,虽有点小灾小难,两位夫人尚还无恙。''

薛嵩的妻子跳起来道:``什么,薛将军已经走了,他为什么不等我。''这中年汉子名叫刘海,本是薛家的小厮,得薛嵩提拔,做了一名百夫长的。刘海道:``请两位夫人、段大侠、铁相公和各位大爷进去,待小的慢慢禀告吧。''他见一大群叫化子同来,也觉得很奇怪。

福隆寺地方很大,被薛嵩这支亲军占用,作为总部,里面还有未曾搬走的军粮。丐帮弟子也不客气,拿了军粮便去造饭。

段、铁二人陪着薛嵩、聂锋的妻子,听刘海细说情由。

原来薛嵩并非去攻打皇宫,而是带领亲军,到朔方郡唐皇肃宗驻躁之地投降去了。刘海说:``聂将军到来的时候,薛将军军令已下,正要拔队起行。聂将军也曾劝他在此等候夫人,薛将军说:'现在事机紧迫,探子报道朝廷已在发遣兵马,朝福隆字而来,咱们若不从速带领这支军队出走,待到大军合围之时,就要连最后这点本钱也没有了。'薛将军又说:'唐太子新近即位,自立为皇,正在募军,此去朔方郡,沿途三百里的驻军(指安禄山的军队)又多是咱们的旧部,咱们索性打起反正的旗号,至少会有半数驻军跟从咱们,到了朔方,还怕唐皇不看重咱们吗?说不定咱们也可以弄个节度使做做。'聂将军劝他不动,后来也就和他一道,随军走了。只留下小人在此,迎接夫人。''

薛嵩的妻子大哭道:``到了这样的紧急关头,他还只是顾着自己的功名富贵,连结发之妻都不要了。''段圭璋心想:``薛嵩固然是个小人,但他这次率军背叛了安禄山,总是于国家有利。''当下说道:``两位夫人不必悲伤,现有丐帮的武帮主在此,且待风波稍定,两位夫人可以改装,由丐帮护送你们到朔方与尊夫相会。''薛嵩的妻子满面着惭,拜下去道:``多谢段大侠不念旧仇,大恩大德。''段圭璋道:``过去的事还提它作甚?咱们进静室看卢夫人去口巴。''

卢夫人伤得很重,但神志仍然清醒,窦线娘在旁边服侍她。她见段圭璋进来,便问道:``薛嵩是不是走了。我的女儿呢?''段圭璋道:``薛、聂两将军已往朔方投降唐皇,若梅和隐娘也给他们带走了。''薛嵩的妻于俯伏床前终道:``姐姐,我家对不起你。''卢夫人道:``不,你家将军既已改邪归正,那就是对得起我了。我只遗憾不能见女儿一面。''段圭璋退:``大嫂,你安心养伤。''卢夫人露出微笑,说道:``咱们两亲家当真是多灾多难,好在今日还能与你相逢。怕只怕我没福份见见他们俩小口子完婚了。嗯,令郎呢?他这次没有同来吗?''段圭璋怕她更多操心,不想告诉她儿子失踪之事,说道:``在这兵慌马乱的年头,我不敢带小儿到长安来。''

卢夫人忽道:``可有官军向这里追来么?''铁摩勒道:``没有。''刘海也道:``我也正在奇怪呢,薛将军说探于已探听得朝廷(指安禄山之``朝廷'')已发遣兵马,朝福隆寺而来,但现在已有大半天了,仍未见有风吹草动。''卢夫人陡地精神一振,双目倏张,带笑说道:``好,这消息好得很!''

薛嵩的妻子怔了一怔,连忙问道:``好在哪里,我仍未明白,姐姐你是女中诸葛,请为我剖析疑团。''卢夫人道:``这很容易明白,安贼本来已经发兵,但如今未到,那当然是中途撤回去了。何以撤回?这不问可知,自是临时发生了更大的更意外的事情,亦即是比薛、聂二将军对他的背叛更严重的事情了。''段什障点点头道:``大嫂,你这看法很有道理。既然如此,你更可以安心养伤了。''

与夫人咳了几声,叶了口气,靠着床背,挣扎着半躺半坐起来,兴奋之中又似带着几分焦急,焦急着在等待什么讯息的神情。窦泉娘和薛嵩的妻子过去扶她,她忽地又张开了眼睛,面向着薛嵩的妻子说道:``姐姐,我拜托你一件事情。''薛嵩的妻子忙不迭地说道:``姐姐,你尽管吩咐便是。''

卢夫人道:``我怕见不着我的女儿了。她现在跟随薛将军到了朔方,异日你们夫妻团圆,请你向她说明她的身世来历。还有,她自小已许配给段大侠的儿子,要是薛将军给她另找婆家,你千万要设法劝阻。薛将军的脾气我是知道的,倘若你拦阻不得,就请你暗地里告诉她,叫她出走。这些事都要瞒着薛将军做的,你办得到吗?''

薛嵩的妻子现出羞愧的神情,低声说道:``姐姐,你不用担心,你会好起来的。倘若有什么三长两短,我一定照你的吩咐去做便是。我丈夫他、他抢了你的女儿,不准你们母女相认,这件事我一直抱愧于心。不过,他现在已背叛了安贼,投归唐朝,段大侠又是救了他家小的恩人,想来他也不会那样横蛮,还要做出什么对不起你和段大侠的事情。''卢夫人苦笑道:``但愿如此。''这是表示不相信薛嵩的意思,薛嵩的妻子又是羞惭,又是难过,连忙说道:``姐姐,你放心。倘若那天杀的当真蛮不讲理,纵使他杀了我,我也要对你的女儿说明真相。''窦泉娘也道:``大嫂,你女儿是我家的未过门媳妇,我们也绝不会不理她的。少则一年,迟则三载,我们亲自到朔方找薛嵩要回媳妇,咱们两家合成一家,共庆团圆。''卢夫人点点头道:``这我就放心了。''忽地她又似记起什么事情,再对薛嵩的妻子道:``我女儿头上那根风头玉钗,是段大侠给她当作聘礼的,风口中空,我已将她的身世来历,写在纸上,放在风银之中。倘若事情紧急,你来不及告诉她,或者她对你所说不信的话,你可告诉她这个秘密,叫她从风口里取出纸团。''

刚说到这里,忽听得武铁樵的声音在外面嚷道:``好,好消息来了,你快进去禀告段大侠和卢夫人!''

只见一个叫化子匆匆忙忙的奔跑进来,正是武铁樵派去打听消息的那个丐帮弟子,一进门来便大声嚷道:``喜报,喜报!安禄山已被他的儿子杀了!''

段圭璋方自一呆,忽听得卢夫人纵声长笑道:``好呀!安禄山你也有今天,史郎,你在泉下可以瞑目了。''

窦泉娘叫道:``嫂子,你、你\ldots\ldots{}''只见卢夫人脸上的笑容还未收敛,双目已经紧闭,垂下头来,窦线娘在她的鼻端------探,气息早已没了。

薛嵩的妻子失声痛哭,聂锋的妻子却向那丐帮弟子探问详情。那丐帮弟子道:听说是太子太保严庄主谋,下手的是太监李猪儿。严庄现已受封为冯诩王,总揽朝政,现在正由严庄出面,召集伪朝文武百官,善安禄山发丧,并奉新皇帝登基。呀,想到这个好消息却成了这位夫人的催命符!''他双手一摊,一包人参跌下地来,那是段圭璋叫他买来给卢夫人作``续命汤''的,街上的药铺都已关门,他费了许多气力,好不容易力才偷到-包,但现在已是用不着了。

段圭璋虎目蕴泪,呆呆地站在卢夫人床前,却哭不出来。聂锋的妻子道:``段大侠,且体悲痛,我说一件事情给你知道。安禄山之死实在是卢夫人假手于严庄将他杀的。要说主谋,卢夫人才是主谋。''铁摩勒也将那晚偷听到的秘密------严庄的妻子怎样向卢夫人请教,卢夫人怎样替她的丈夫定谋策划等等事情说了出来,直把众人听得呆了。

段圭璋仰天大笑,笑声中眼泪滚滚而下,忽地翻身拜倒,说道:``嫂子,你真是女中豪杰,愧煞我辈须眉。''这时他才哭得出来。

众人正在举哀之际,武铁樵派去打听消息的第二个丐帮弟子亦已回来,他带回来了安禄山被杀的详情,业带来了一个坏消息。羊牧劳已被新``皇帝''重用,兼任``羽林军''的统领,安禄山原来的副手史思明则掌握了兵权,仍然要称兵叛乱,抢夺唐朝的江山。

原来安禄山的``太子''安庆绪庸碌无能,得不到父亲的欢心,经常受打受骂,怕安禄山废立,因此才听从了严庄的唆使,密谋歉父。这一日安禄山在``离宫事变''之后,因为一场``盛会''被铁摩勒等人搞得一塌糊涂,回``宫''之后,又惊又气,他本有目疾,一气之下,双目全盲。安庆绪伪称探病,带了李猪儿进去,安禄山正担腹而睡,李猪儿手起刀落,一刀就剖开了他的肚皮。安禄山是个大胖子,据说被剖腹之后,肚肠流出了数斗。这也是李猪儿的幸运,安禄山勇武过人,要是他双目未盲,李猪儿绝不能将他如此轻易杀掉。

众人听了,一喜一忧。段圭璋沉吟半晌,说道:``严庄纵有弃暗投明之心,无奈军权落在他人之手,他作不得主张,看来他和安庆绪都将变成史思明的傀儡,这场叛乱还要继续下去。不过,安禄山一死,他们内部势将引起变乱,败亡之期,也当在不远了。''他顿了一顿,继续说道:``不过,那是未来的事,现在咱们倒应该提防他们派兵前来,此地还是早早离开为是。''

当下,段连库就请武铁樵前来商议,武铁樵一口答应,愿意护送薛嵩、聂锋两家家小到朔方去,薛嵩的妻子自是感激涕零,不必细表。

剩下来的就是给卢夫人安葬之事,幸喜这福隆寺乃是长安著名的大寺院,平时有些要作善事的人,施舍有许多棺材在这里,方丈广智禅师又是聂锋的好朋友,段圭璋就把安葬卢夫人之事,委托与他,等待他日太平之后,再行迁葬,与她丈夫合冢。

段圭璋夫妇给她盖棺,不禁眼泪涔涔而下,窦泉娘叹口气道:``她临死以女儿相托,现在她的女儿已有下落了,咱们的儿子却还未知落在何人之手。段、史两家的亲事真是磨难重重,咱们有没有福气要这个媳妇也还未知道呢。''

铁摩勒忽地说道:``我正有一事要禀告始丈、姑姑,两个月前,我碰见空空儿,他说十年之期已满,现在可以将表弟交还了。''

段圭璋怔了一怔,随即叫起来道:``不错,空空儿当时是曾说过这句话,他说孩子已被另一个人要去了,那人似乎是他所忌惮的前辈,但他愿意担保,至多十年,必定将咱们的孩子归还。''

窦泉娘冷笑道:``空空儿的话也信得么,你们不怕再上一次当?''她压根儿就不把空空儿的话放在心上,所以十年之约什么,早就忘记了。

段圭璋道:``你且先别发脾气,听听摩勒说说,他是怎么样遇见空空儿,又是怎么样和他谈的?''

于是铁摩勒就将当日他怎样被宇文通追捕,后来空空儿怎样突然出现,帮了他的大忙,等等情形细说一遍,最后说道:``空空儿说,请你们再上玉树山的玉皇观找他,三个月的时间内,他不会离开玉皇观。哎呀,现在已过去了将近两月,只有个多月的时间了。''

段圭璋道:``如何?空空儿若是坏人,他也不会帮助摩勒了。况且,只有这一条线索,你就是不相信他,也得去找他一次。''

窦线娘道:``好吧,若然这次还是骗局,咱们和空空儿拼命便是。''

他们夫妻争辩的时候,韩湛一直坐在旁边微笑,段圭璋觉他神情有异,问道:``韩老前辈有何高见?''韩湛笑道:``我听说空空儿为人乖僻,行事古怪,武林中有很多人赞他,也有很多人骂他,现在你们贤伉俪对空空儿的看法,也恰好是各走一边,为空空儿而引起口角,这不好笑么?其实无须争论,到玉树山看看就明白了。老夫反正没事,要是你们不嫌弃的话,我也想陪你们同去,看看空空儿到底是怎么个人?''段圭璋大喜道:``有老前辈同去,那是求之不得!线娘,你也可以放心了吧?倘若空空儿真是坏人,骗咱们上当的话,有韩老前辈在场,还怕对付不了他么?''韩湛笑道:``段大侠客气了,你们夫妻联手,还用得上老夫帮忙么?不过,不是老夫倚老卖老,大约有老夫在场,空空儿也不敢真个动手的。''

窦线娘闷声不响,心里想道:``你虽然是天下第一点穴名家,空空儿也未必便怕了你?说这个话未免太自负了。''段圭璋却在暗暗奇怪:``韩老前辈素来为人谦虚,怎的今日却会小觑空空儿,莫非其中另有缘故?''眼光一。瞥,忽见铁摩-勒也面露笑容,韩芷芬正在朝他打了一个眼色,段圭璋道:``摩勒,你可有什么话要说?''铁摩勒道:``没什么,我和芬妹都想跟去瞧瞧热闹。''其实铁摩勒却是知道那个``缘故''的,不过,他经过了这些年磨练,已比从前通晓人情世故,窦线娘既然对空空儿成见极深,因此铁摩勒也不愿意说出来了。

当下计议已定,一行五众,立即离开隆福寺。长安正在混乱之中,铁摩勒又有聂锋给他的那面腰牌,出城倒是没遇麻烦。

他们兼程赶路,这一日到了玉树山下。一计时日,从长安至此,已用了一个月零三天。还有两天,便要满空空儿的三月之约。段圭璋吁了口气道:``明天晚上,总可以到达山上的玉皇观了。''

玉树山峭拔奇兀,山势险峻,从山口进去,有一条狭长的山谷,曲曲折折,怪石嶙峋,当真是移步换景,别有洞天。窦线娘道:``圭璋,你还记得那年咱们就是在这个地方被人暗算么?''话犹未了,忽听得``呜''的一声,-枝响箭,划过长空,山坡上现出两个彪形大汉。窦线娘怒道:``好呀,果然又在旧戏重演了!''段圭璋笑道:``这回可不是暗算,咱们遇上了响马了!''

铁摩勒大笑道:``响马劫道?哈哈,你们的招子(眼睛)可不明亮了,你们知道我是谁?你们劫到贼祖宗的头上来了?''

那彪形大汉喝道:``好呀,原来你这小子也是窦家贼党,老子专杀强盗,看刀!''只听得呜呜声响,三把飞刀,排成品字,向铁摩勒飞来。铁摩勒横剑一封,``咣''的一声,把一口飞刀磕落,只觉虎口一麻。说时迟,那时快,左右两柄飞刀亦已同时飞到,铁摩勒身形贴地,一个``卧虎翻身'',滚出了数丈开外,那两口飞刀就插在他原来的位置。要是他动作稍迟,便要给飞刀钉在地上。

就在那大汉发出飞刀的时候,窦线娘也已拽弹弓,三颗金丸,闪电般的向那汉子射去。那汉子在山坡上,听得暗器破空之声,身形一缩,躲到大树后面,三颗弹丸,都嵌在树上。

窦线娘冷笑道:``窦家的人来了,你却怎么倒变作乌龟缩头了?''话犹未了,另一个汉子已在喝道:``贼婆娘休得夸口,且看谁是乌龟缩头?''双手齐扬,六口飞刀连翩飞至。

窦线娘冷笑道:``米粒之珠,也放光华?''把弹丸似流星般地射出去,她的暗器功夫已到了出神人化的地步,弹丸的份量虽较轻,但一碰上飞刀,就能把飞刀的劲力卸去,但听得叮叮咣咣之声不绝于耳,飞刀与弹丸都同时跌落,满空中银光交织,金星飞舞,蔚为奇观。

那躲在大树后面的汉子这时亦已现身出形,也是双手齐扬,同时发出六两飞刀,窦线娘的弹弓虽然发射得很快,但到底不能在瞬息之间把十二柄飞刀都打下来,有两柄飞刀没有给她的弹丸打中,在空中走了一道弧形,竟然合成了一个银色的光圈,向她的颈部削到!

窦线娘无可抵御,只得霍地一个风点头,身躯矮了半截,段圭璋身形一掠,宝剑出鞘,一招``横云断峰'',把两柄飞刀削为四段。

那大汉笑道:``原来你也变作乌龟缩头了!''窦线娘大怒,觑准他便是一弹,那大汉来不及发出飞刀,饶是他闪躲得快,腰骨也给打个正着,那大汉叫道:``风紧,扯呼!''和他的同伴一齐向山上逃跑。

窦线娘气愤难消,提起弹弓便追,段圭璋道:``咱们赶路要紧,这些小贼么,不理也罢。''窦线娘道:``你不听见他们说么?他们是冲着我窦家来的,岂可不查个水落石出。''段圭璋没法阻拦,只得与她一同追上山去。

追过了一个山坳,忽见山顶上有间屋子,似是一个寺院,韩湛忽在后面叫道:``段大侠且慢!''正是:

奇峰平地起,险难接连来。

欲知后事如何,请听下回分解------

旧雨楼扫描,海之子OCR,旧雨楼独家连载

\chapter{第三十六回 绿林血债嗟难解
魔阵妖氛化不开}\label{ux7b2cux4e09ux5341ux516dux56de-ux7effux6797ux8840ux503aux55dfux96beux89e3-ux9b54ux9635ux5996ux6c1bux5316ux4e0dux5f00}

段圭璋愕然止步,问道:``怎么?''韩湛道:``咱们误上了黑石峰了!''段圭璋这才注意到周围的山石都是黑黝黝的,十分奇特,不禁问道:``这山峰有什么古怪,上不得么?''

窦线娘正在追赶那两个汉子,她丈夫止步,她却未曾止步,就在段圭璋发问的时候,忽听得呼呼声响,突然飞出了两条铁抓,一左一右向窦线娘抓来。原来两面山坡上都埋伏有人,有两人长得一模一样,所使的武器也完全相同,乃是一条数丈长的铁索,铁索的一端装着一柄利钩,这两人能舞动数丈长的铁抓抓人,功力之高,自非泛泛之辈。

但窦线娘惯经大敌,在暗器上又有精湛的造诣,耳目灵敏,更非常人可比,她一听到铁抓荡风之声,弹弓早已发射出去。

呼的一声,右边的铁抓已到,妻绵娘施展金弓十八打的手法,举弓一拨,那条铁索夭矫如龙,一个盘旋,横扫过来,索端的利钩正好把她的金弓抓着!

就在这时,左面山坡的那个汉子发出一声尖叫,想是已被窦线娘弹丸打中,但却伤得不重,所以他那条铁抓虽然来得较慢,但仍然还朝着窦线娘抓来了!

段圭璋连忙奔一七,这条铁抓本是向窦线娘的头部抓下来,但因那人被弹丸打中,手腕颤抖,铁抓失了准头,却从窦线娘颈侧掠过。也幸亏是窦线娘的弹丸先打中了他,要不然窦线娘这时候正被另一人抓着了她的金弓,势将无可抵御。

段圭璋来得正是时候,那条铁抓一抓不中,拉回来时,段圭璋已是赶到,他所用的是一柄削铁如泥的宝剑,手起剑落,``咔嚓''一声,就把铁索上的那柄利钩削断了。

就在此。时,窦线娘却禁不住抓住她金弓那条铁索的拉扯,虎口一麻,只得撒手,那柄金弓竟被铁抓抓了去。

两条铁索同时收回,那两个人也同声骂道:``贼婆娘擅上黑石峰还胆敢伤人,想是活得不耐烦了!''

窦线娘大怒,拔出佩刀,就追上去,喝道:``管你甚么黑石峰白石峰,快把我的宝弓还来,然后磕头赔罪,要不然,你倒看看是谁要谁的命?''

那两个人不再回骂,却只是嘿嘿冷笑,他们想是走山路走惯了的,捷似猿猴,窦线娘竟然追他们不上。

可是窦线娘失了家传的宝弓,那肯罢休,仍是穷追不舍,过了一会,只见这两个汉子和先前那两个放飞刀偷袭的人,都已跑到了山上,进入山顶那间寺院去了。

窦线娘一上到山上,便见金光闪闪,耀眼生辉,原来这间寺院的建筑十分奇特,屋顶成圆锥形,而且这圆锥形的屋顶,竟是用金箔包在外面的。在荒山上竟有如此金碧辉煌的一间寺院,当真是难以思议的事情,饶是窦线娘见多识广,也不禁怔住了。

段圭璋道:``咱们已经知道了那些人是藏在这寺院里,就不必忙在一时,且先向韩老前辈请教吧。请问韩老前辈,是否知道这寺院的来历。''

这时韩湛和铁摩勒等人都已跟了上来,韩湛说道:``这是黑石峰上的金碧宫,宫中的主人是三十年前从天竺来的一位僧人,法号转轮法王。他定下禁例,这黑石峰是不许外人士来的。今日咱们误上此峰,只怕一场麻烦是难以免了。''

窦线娘问道:``这转轮法王是何等样的人物,竞敢如此骄狂?''

韩湛道:``他的武功深浅我不知道,只知道空空儿的师父藏灵于,他生前服高于顶,但对这转轮法王,在言谈之间,却也十分佩服。''

段圭璋夫妇还是第一次听得空空儿师父的名字,大为奇怪,连忙问道:``原来韩老前辈与空空儿的师父是相识的么?'''

韩湛道:``老夫西年在西北漫游,承藏灵子折节下交,我在他的玉皇观里,也曾住过不少口子,实不相瞒,空空儿还是个小娃娃的时候,我已曾见过他了。''

段圭璋道:``空空儿的师父是个道士么?''

韩湛道:``他是半路出家的,听说是夫妻不和,才戴上黄冠,做了道士,不过,我可没问过他。''

韩湛继续说道:``藏灵子和转轮法王的脾气十分怪僻,听说他们曾经是过很要好的朋友,后来却不知为了什么事情闹翻了。藏灵子在玉树山的主峰玉皇观,转轮法王这黑石峰的金碧宫,相距不过一日路程,但两家自闹翻之后,不但他们二人,即他们的门下弟子也从不往来了。转轮法王的禁例,恐怕就是为玉皇观的弟子而设的。但现在藏灵子已死了十多年,这条禁例不知是否已经取消,那我就不知道了。''

窦线娘道:``我还以为那些人是空空儿派来和我搞乱的呢,如此说来,他们却并非一路。但不管是转轮法王也好,是空空儿也好,我总不能平白受他欺侮。''

段圭璋道:``既然到此,是该问个明白,并索回宝弓。但他到底是前辈,咱们也不可鲁莽。''

段圭璋正待叩门以礼求见,那两扇门扉却已忽地打开。

只听得一个阴恻恻的声音说道:``天堂有路你不走,地狱无门你偏来!好呀,段圭璋,算你倒媚,今日又撞到老娘的手上了!''这开门出来的竟是展大娘,大出乎众人意料之外。段圭璋一惊之下,展大娘已倏的向他抓来!原来当年展大娘在华山上遭受群雄围攻,段圭璋也曾参与,在那次围攻中,展大娘曾给段圭璋刺了一剑,是以仇人见面,分外眼红,一见面便施杀手。

幸而段圭璋惯经大敌,猝逢突袭,他一个盘龙绕步,宝剑已霍地出鞘,说时迟,那时快,窦线娘亦已展开八卦游身刀法,与段圭璋刀剑相联,将展大娘挡住。

展大娘一击不中,倏的便冲出去,欺到了铁摩勒身前,喝道:``你这小贼也来了么?''声出掌发,一招``游龙探抓'',便向铁摩勒的琵琶骨抓下来!

忽听得``嗤嗤''声响,展大娘的手指堪堪就要触着铁摩勒的时候,忽觉虎口一麻,原来是韩湛以``隔空点穴''的上乘内功,向展大娘戳了一指。

韩湛笑道:``展大娘,想不到与你在此地相逢,记得你那日曾邀请我们喝令郎的喜酒,怎的今日忽而反面无情,要打起贺客来丁?''

展大娘面色沉暗,怒声说道:``你是有心讽刺我么?儿子和徒弟都不是我的了,还喝什么喜酒!''

铁摩勒好生惊异,心里想道:``难道王燕羽与展元修又闹了什么别扭了?''

展大娘还想向铁摩勒下手,但她也识得韩湛的厉害,正在踌躇,庙中又出来一人,笑嘻嘻地道:``难得诸位贵客同来,家师有请!''接着又道:``师叔息怒,他们既到了这里,如何处置,家师自会作出主张。''

这人摇着一柄折扇,婚皮笑脸,口称``贵客'',却是一副轻蔑的神情。此人不是别个,正是王伯通的儿子王龙客。

段圭璋恍然大悟,心里想道:``敢情这王龙客竟是转轮法王的门下弟子,途中伏击那些人都是他的师兄弟辈,他们是有意将我们引上黑石峰的!但他们却怎的知道我们今日会路过此地呢!''

窦线娘与王家有血海深仇,见王龙客这般神气,更为恼怒,喝了一声:``小贼!''便想弹出金丸,韩湛忙道:``打狗要看主人脸,大嫂,进了寺中见了法王再说吧。''王龙客倒并不生气,只是冷冷说道:``我奉家师之命来请你们,你们倒骂起我来了,好吧,你们尽管骂吧,否则待一会儿,只怕你们有口也难骂了。''

王龙客冷言冷语,正是存心激她发怒,他恨不得窦线娘破口大骂,甚或先行动武,然后好在师父面前派她个登门挑衅的罪名,窦线娘识穿了他的诡计,心想:``今日之事,看来难以善罢。且先容忍你这小贼片时,看你师父如何发付?''按下怒火,随王龙客进去。

到了一座大堂。大堂上摆着一张几案,后面一张檀木椅子。刚才在中途伏击那四个汉子排列两旁,倒有点像公堂审案的味儿,段圭璋这时也有点怒气了。

王龙客踏进大堂,便朗声说道:``擅闯金碧宫的来人带到,请师父登堂发落。''

段圭璋是个宁折不屈的好汉,忍不着气,冷冷说道:``咦,我以为这是佛门清静之地,谁知却误进了衙门了。''

话声未了,只见两个形貌古怪的人已走了出来。前面这人是个枯瘦的和尚,皮肤黝黑,鹰鼻黄须,双目炯炯有光,太阳穴涨鼓鼓的,一看就知内功深厚非常,后面这人活像个大猴子,却原来是精精儿!

精精儿突然在此地现身,而且随着转轮法王,众人无不诧异,尤其韩湛更觉惊奇,心中想道:``精精儿是玉皇观的人,怎么会到了金碧宫来?''

只见转轮法王双目一睁,不怒而威,便向着段圭璋说道:``你们都是些什么人?犯了我的禁例,擅上黑石峰,还胆敢在此胡言乱语?''

精精儿道:``师父不必盘问他们,这些人的来历我都知道,这婆娘是飞虎山窦家寨的女贼,这贼子是她的丈夫,其他的人都是他的同党!''

窦线娘不由得怒道:``窦家寨的人又怎么样?难道大师高年盛德,也要插手管黑道上的事么?''

转轮法王冷笑道:``好一副尖牙利齿,老衲不管你尘俗之事,只问你为何上黑石峰来?''

窦线娘道:``请你问你左右这四个弟子,问他们为何在半途偷袭我们,还抢了我家传宝弓?''

那用铁抓抓了窦线娘金弓的人,走出行列,向转轮法王躬身说道:``禀师父,飞虎山窦家寨的人作恶多端,弟子们的父兄都是给窦家五虎害了的。师父可以不理黑道之事,但他们已到此间,顺手除恶,也是一件功德。''

转轮法王道:``哦,怪不得你们四个都不愿随师父削发为僧,

原来是有父兄之仇。你们的父兄是如何被害的,说出来也好让他们死而无怨。''

那使铁抓的汉子说道:``我叫朱灵,我弟弟叫朱宝,我们的父亲是从前朱雀山的寨主朱旭。窦家自封绿林盟主,要各处山寨年年向飞虎山纳贡。有一年朱雀山的贡物不够,窦家限期要我父亲交足,否则就要灭了朱雀山的朱家寨。我父亲没法,冒险大劫幽州的府库,库银虽然劫到了手,我父亲却中了官军的箭,未回到山寨,便因伤重而死了。窦家寨乘机便吞并了朱家寨,动来的库银也都搬了去,连棺材也不给我父亲一口。我父亲若不是为了要向窦家纳贡,怎会身亡?所以穷本追源,我父亲还是死于窦家之手。''

那使飞刀的汉子接着说:``我家更惨,我父亲是幽州铜马山的寨主,窦家寨的大头领窦令侃忌我父亲在绿林有些威望,借口招开绿林英雄宴,将他诱上飞虎山囚禁起来,用酷刑将他百股拷打,迫他写了亲笔书信,将铜马山的人众都收编到他的旗下,然后将我的父亲毒杀了。''

另一个也是使飞刀的汉子说道:``我家却不是绿林中人,我哥哥是个著名的镖师,凭他的镖旗走遍大江南北,从没出过事。有一次在乎凉道上,窦家五虎齐来劫他的镖,劫了镖还不打紧,还要斩尽杀绝,我哥哥已受伤而逃,他们追出了百余里外,将我已受了伤的哥哥杀死。''

窦线娘和铁摩勒起初以为他们是捏造的,后来听他们一个个说得有名有姓,有凭有据,而且飞虎山吞并朱雀、铜马两寨的事,窦、铁二人也都是知道的,不过当时窦线娘还是个少女,而铁摩勒更是个孩子,只知其事,不知其详,做梦也想不到这两家的寨主是被窦家如此残酷的害死的。

铁摩勒听得毛骨惊然,不禁想道:``我为了义父待我之恩,无时无刻不想为他报仇,却原来我的义父也曾害过许多人命,若然似这等冤冤相报,何时得了?''

窦线娘也受到了震动,心想:``我要向王家报仇,却原来别人也要向我窦家报仇。''她想了一想,说道:``这些事纵然是我哥哥干的,与我也不相干。若说我是窦家的人,就要填命,那么这位令高足,他家把我五个哥哥都杀掉了,倘若法王果是主持公道,就请你把这姓王的弟子交给我,让我处置了他以后,我再任凭你们处置,替我窦家偿你们这几家的血债!''

转轮法王面色一沉,``哼''了一声,说道:``你这婆娘好大的胆子,竟敢对我说这样无礼的话!我金碧宫的弟子岂能是任凭外人处置的么?''

段圭璋亢声说道:``法王的弟子不能任人处置,难道我们就该由你处置么?你倘若要插手管绿林中的纠纷,就陔秉公办理。''

转轮法王老羞成怒,冷笑说道:``我才懒管你们的纠纷呢,只是你们犯了我的禁例,我却不能不问。好,你们既然擅入金碧宫,那就不必回去了。精精儿,来!''

精精儿越众而出,躬身说道:``弟子听师父吩咐。''

转轮法王冷冷说道:``金碧宫正缺少执役僧人,你把这些人的琵琶骨挑了,剃光他们的头发,每人发给他们一套僧衣。''精精儿应了一声``遵命'',却又问道:``这个婆娘呢?''转轮法王道:``金碧宫不收容尼姑,这个婆娘么,好,就只挑了她的琵琶骨,不必剃光头了。废了她的武功之后,将她送给展大娘做蝉女。''法三顿了一顿,再提高声音说道:``我这样处罚你们,已经是特别从宽,你们明白了么?倘若谁敢违抗,刑罚就更要加重,不只挑琵琶骨,还要割了你的舌头,剜掉你的眼珠,削掉你的耳朵1''

窦线娘大怒,正要发作,韩湛却忽地迎上前去,冷笑说道:``精精儿,你先来挑了老夫的琵琶骨吧!''精精儿面色一变,讷讷说道:``韩、韩老前辈,你别动怒,我、我代你求情!''韩湛厉声斥道:``谁要你求什么情,你连师父都敢违叛,与我还有什么情义可言!''

精精儿面上一阵青一阵红,原来他被师兄罚在玉皇观面壁三年,心中不服,是以逃到金碧官来,改投转轮法王。他是从师兄空空儿的口中,得知段圭璋等人就要来玉树山的消息的。朱灵、朱宝等人拦途伏击的事,都是出于他的布置。待段圭璋这班人进入金碧宫后,他料想不到韩湛也在其中,一时之间,来不及特别向法王说时韩湛的身份,法王的命令已经下了。

转轮法王的眼力何等厉害,一眼就看出了韩湛的武功最高又听他说了这样的话,便问精精儿道:``这老头儿是什么人?''

精精儿道:``他名叫韩湛,是先师的一位友人。''

转轮法王目露精光,道:``哦,原来是天下第一点穴名家韩先生,我以前也曾听藏灵子谈及。好,难得你今日也到此间,我正想问你一件事情\ldots\ldots{}''话犹未了,忽见他连人带椅,飞了起来,竟是朝着韩湛压下!

段圭璋等人都是深通武学之土,但见转轮法王露了这手超凡人圣的功夫,也都不禁大惊失色!要知身怀轻功绝技的人,从数丈之外飞身扑来,那还不足为奇,但端坐椅上,连椅子也一同飞起,这就不但要轻功高明,而且要将本身极其雄浑纯厚的内力运用得妙到毫巅!这种功夫,众人莫说见过,连听也没有听过!

说时迟,那时快,转轮法王连人带椅,已向韩湛当头压下。只听得``卜''的一声,转轮法王的椅子在空中打了一个圈圈,倏地又飞了回去,仍然落在原来的位置。

只听转轮法王微微气喘,过了片刻,打个哈哈说道:``韩先生果然名下无虚,居然点中了老衲的`璇玑穴',可是想来韩先生也该明白:倘若老衲稍存恶意的话,韩先生此时大约也不能再站在这里说话了。''说罢,拿出了一片破布,这时众人方才注意到韩湛的衣裳已被撕去了一幅,而且位置正当前心。

转轮法王将那片破布一搓,双掌一摊,那片破布已变成粉屑,洒了满地,转轮法王笑道:``韩先生,你现在应该可以回答我的问题了,我的武功比藏灵子如何?''众人这才明白,转轮法王刚才原来并非是向韩湛突袭,而只是要韩湛见识他的功夫。

韩湛不亢不卑,朗声答道:``讲到武功,法王比藏灵子大约也还差不多;但若论胸襟气度,法王就差得远了。''这样说法,其实即是说他的武功、气度,两样都及不上藏灵子。不过武功方面,较为接近而已。

转轮法王怔了一怔,随即哈哈笑道:``好,韩先生果然爽直,说的话比精精儿老实多了。''精精儿面红过耳,做声不得。

转轮法王又道:``韩先生既然是藏灵子的朋友,我看在故人份上,你的这份刑罚可以免了,你要上玉皇观,就尽管去吧,见了空空儿,可以对他说,精精儿已改投我的门下,他就不必管了。''

韩湛道:``请法王原谅,现在叫我走,我不愿走了。''转轮法王诧道:``怎么,你还要留在此地?''韩湛道:``不错,我与他们同来,要走也得与他们同走,倘若法王坚执要处罚他们,老夫也一同领罚!''

转轮法王沉声道:``韩湛,你虽是成名之辈,但要想在金碧宫中逞能,只怕还办不到吧?''韩湛道:``韩某岂敢逞能,韩某也自知要与法王相抗,无异以卵击石;但于义不能独生,倘若得在法王手下领死,那也是何幸如之!''

转轮法王冷冷说道:``哦,原来你们还要与老衲过招动手么?''段圭璋手按剑柄,朗声说道:``大丈夫死则死耳,焉能受辱?法王是武林中的泰山北斗,你既不惜以大欺小,以主凌客,那就请恕段某也要无礼了!''

转轮法王忽地摇了摇头,叹了口气,黯然说道:``藏灵故友一死,老衲即已心灰意冷,只因天下虽大,却从何处去找对手?除非是扶桑岛虬髯客还有传人,否则老衲是决不能与人过招动手的了!''言下之意,即是眼前诸人,连同韩湛在内,都不配作为他的对手。众人听了这话,都不免心中生气,但以他的武功身份,这话也的确不算``大言''。

展大娘走上前道:``这些人狂妄无礼,老婆子先就看不过眼,不劳法王动手,老婆子愿为法王效力。''

转轮法王略一沉吟,说道:``也好。展大娘,你是我金碧宫的客人;韩先生,我本来也想把你当作客人,但你既坚执要与他们一起,那么就让你与展大娘一战吧。我的刑罚不施用于你,你胜了也好,败了也好,都当作是客人之间的私斗,琵琶骨是不用挑了。''说到这里,停了一下,声音嘉转阴沉,向精精儿吩咐道:``精精儿,你率领同门,执行为师的刑罚,除了韩先生一人之外,其他人的琵琶骨你都给我挑了。我虽然没有比你先师更好的武功传给你,但我那七绝诛魔阵,却是你先师所无,你好好运用吧,谅这些人逃不出此阵。韩先生、展大娘,你们这一场是愿意押后,还是愿意移前?''

韩湛道:``韩某不必你另眼相看,你们一齐上,我们也一齐上。''

精精儿投到转轮法王门下之后,因为他的年纪比王龙客、朱灵、朱宝等人都大,且又早巳成名,因此不依入门前后来定次序,而将他作为二弟子;大弟子则是幼年就随转轮法王出家的一个和尚,名唤天德禅师,这时正随侍在法王身畔。精精儿正要请他下来,同布此阵,展大娘忽道:``这七绝诛魔阵承法王不吝传授,老婆子现在亦已略知诀窍,他们既要同上,老婆子也愿在阵中作一小卒,稍尽绵力。''原来展大娘对韩湛也有几分顾忌,只怕单打独斗,赢不了他,在法王面前失了面子,故此不惜自贬身份,愿供精精儿驱策。

精精儿一想,此阵的变化,展大娘虽然不若天德禅师之熟悉,但武功却要比天德禅师高出不知多少,有她同在此阵,更加可操胜算,便即说道:``展大娘肯予赐助,那是最好不过!''此言一出,阵势也便发动,展大娘一声长啸,一马当先,向韩湛兜头便抓!

韩湛屹立如山,待她抓到,蓦地一声喝道:``来得好!''出指如电,左点``白海穴'',右点``乳突穴'',中点``璇玑穴'',当真是飘忽之极,变化无穷,似左似有似中,叫人难以捉摸!:

就在这瞬息之间,展大娘已一掌拍下,掌风扑面,人影翻腾。但听得``嗤''的一声,倏然间两条人影业已分开,展大娘一掌从韩湛颈侧削过,相差毫厘,未曾削实,而她的衣裳,却已被韩湛戳穿了三个小洞。原来那``嗤''的一声,乃是韩湛的指力激荡气流所致,虽然同样未曾点实,但已凭着内家真力,荡气成风,戳破她的衣裳。饶是展大娘那等凶蛮,也不禁暗自心惊了。

韩湛心想,法王有言在先,绝不下场,在这金碧宫中,便以展大娘武功最高,只要将她伤了,这``七绝诛魔阵''固然可以破解,即生出金碧宫亦非全无希望。因此毫不放松,一占上风,立即追击,再度出指,反手点展大娘后心的``归藏''、``中枢''、``天柱''三大穴道。

韩湛自忖身法要比展大娘灵活快捷,这反手一点又正是他最得意的独门点穴手法,非中不可。哪知一指戳去,展大娘恰好从他侧边跨过,只觉微风飒然,精精儿又已从侧边攻来。韩湛冷笑道:``精精儿,你也要与老夫动手么?''化指为掌,运了八成功力,一掌拍出,他深知精精儿轻功极高,内功则远远不如自己,故此以己之长,攻敌之短。哪知精精儿只是向他佯攻一招,接着那两个用月牙弯刀的汉子又从两侧攻来,他们所踏的方位十分巧妙,也是一招便收,跟着又似走马灯地转过一边去了。原来这``七绝诛魔阵''按着五行生克方位,阵势展开,有如重门叠户,七人联手,泽如一体,纵使其中有人武功较弱,对方也不容易将他们各个击破。

双方甫一接触,窦线娘对王龙客最为怀恨,立即便向他攻去。窦线娘虽然失了金弓,但她还有两样家传绝技,一样是``游身八卦刀法''、一样是``穿花绕树身法''。那时阵势初展,尚未合围,窦线娘一个盘旋,便欺到了王龙客身前,``唰''的一刀,横斩腰胯,下削膝盖。王龙客也凶狠非常,铁扇一张,向窦线娘面门一扇,倏的便合起来当成点穴用,敲击窦线娘小臂的``曲池穴''。这一招也正是他的得意功夫,张扇迷惑敌人视线,便即乘机进击。哪知窦线娘早已知他狡猾,那一刀实是虚招,待王龙客合扇击来,她已绕到了五龙客背后,正要施展杀手,猛听得呼呼两声,俨如``螳螂捕蝉,黄雀在后'',两条铁抓,已从两侧攻来。

这两条铁抓矫如游龙,蓦然从半空抓下,眼看给它抓实,就是头穿脑裂之灾,忽听得``咣咣''两声,段圭璋与铁摩勒双双奔上,段圭璋一剑,将朱灵的铁抓挑开,铁摩勒则横剑当成板刀来使,一剑拍下,将朱宝的铁抓压住。

身具武功的人,临危反击,乃是本能,窦线娘并未料到丈夫会及时赶到,所以她在那双抓抓下之时,性命俄顷之际,也立即展开了``穿花绕树''的绝妙合法,趁着双抓未合,倏的就从双抓围成的弧圈中扑进,欺到了朱家兄弟的身前。喝声``好狠!''举起刀来,刀光如雪,寒气森森,吓得未灵、宋宝魂刁;附体。

这时那``七绝诛魔阵''只是阵势初展,尚未合围,而本领最高的展大娘与精精儿二人,又正在全力对讨韩湛,要是窦线娘这一刀劈下,朱家兄弟,必有一人丧命。

窦线娘与朱家兄弟迎面而立,刀光之下,只见朱家兄弟都露出了战栗的目光,不由得心头一软,想道:``他们的父兄遭我窦家所害,我岂能再杀他们?''刀锋一转,虚斫一招,便从抓下钻过,转过一旁。

不但窦线娘心软,段圭璋与铁摩勒也是同一心思,所以刚习虽急于救人,也未遽下杀手,只是将他们的兵器架住,否则朱家兄弟,焉能还有命在?

阵势瞬息即变,就在窦线娘等人不忍下手,稍一迟疑之际,精精儿与王龙客已从两翼抄来。精精儿来得尤其迅捷,短剑扬空一划,一道蓝艳艳的光华已向段圭璋的前心射到,段圭璋吞胸吸腹,脚步不移,身躯已挪后半尺,迅即``唰''的一剑还击过去。精精儿一击不中,箭一般的便从段圭璋身旁掠过,疾攻铁摩勒,铁摩勒横剑一封,咣的一声,将短剑架开,精精儿又已到了窦线娘背后。窦线娘前有王龙客,后有精精儿,幸而她也机灵之极,一听得金刀劈风之声,立即用``穿花绕树''身法,俨如蜻蜒点水,燕子掠波,从王龙客与精精儿的中间穿出,但饶是她身法如此快捷,罗裙的下摆亦已给精精儿的短剑削去了一幅。

王龙客叫道:``可惜,可惜!喂,仇人就在面前,你们还不快上,布好阵势,不用惊慌了后面这几句是对朱家兄弟说的。朱家兄弟,死里逃生,明知是敌人手下留情,因此不禁呆了一呆。王龙客的话语再度挑起了他们的仇恨,他们定了定神,辨认了门户方位,在精精儿带领之下,将阵势转动起来。眨眼之间,``七绝诛魔阵''已是合围,将段圭璋等五人围得风雨不透。

这``七绝诛魔阵''乃是转轮法王平生武学之所聚,虽由弟子主持,威力也是非同小可。精精儿将阵势催动,越转越快,当真是有如狂风巨浪一般,一个浪头未过,一个浪头又已打来。韩湛段圭璋二人犹可支持,其他三人则已有点应付不暇,尤其功力较弱的韩芷芬,更感到透不过气来。

精精儿轻功超卓,行动有如鬼魅,阵势合围之后,他一眼看出韩芷芬是对方最弱的一环,立即向展大娘打了一个眼色,叫人双双向韩湛扑去,扑到中途,一个扭身,焕然间就欺到韩芷芬身前。韩湛被展大娘绊住,急叨间竟然抽身不得。

幸亏铁摩勒与韩芷芬靠近,刻刻留神,忽见精精儿向韩芷芬偷袭,他不顾性命地大喝一声,立即和身扑上,抡剑狂劈。他这一招名为``与敌偕仁'',当真是完全拼了性命的打法,精精儿怎敢和他当真拼命,但听得``咣''的一声,接着``嗤''的一响,精精儿已从他们的身边掠过,韩芷芬头上的珠花给削去了一朵,铁摩勒肩上的衣裳也被挑开。幸亏是精精儿不敢拼命,他这一剑本来是想穿过铁摩勒的琵琶骨的,第一招未中要害,就不敢停下来再发第二招了。

铁摩勒与韩芷芬并肩而立,连忙问道:``芬妹,你没事么?''韩芷芬道:``没事。有你在旁,我一点也不害怕。''她头上珠花被削,说刁;害怕那是假的,不过,她的害怕却被欣悦的心情掩过'了:``我只道铁哥哥被王家那丫头迷住,却原来他还是真心爱我!''

韩湛猛戳三指,将展大娘逼开两步,大怒喝道:``精精儿,你敢欺侮我的女儿!''精精儿早已转过了方向,向段圭璋扑击。而那朱灵、朱宝两兄弟却依着阵势转过来,双抓向韩湛抓下,韩湛哪里将他们放在眼内,但却也不想伤害他们,当下将他们的铁抓弹开,展大娘喘息一定,又来缠斗。

韩湛与展大娘二人虽在激战之中,仍是眼观四面,耳听八方,忽听得有脚步声隐隐传来,有的沉重,有的却要极细心才听得出。两人都大为奇怪,心中均是道:``怎的会同时有六七个人敢上黑石峰来?其中有武功极高明之土,却也有好似完全不会武功的人?''

心念未已,忽听得有一个苍老的妇人声音叫道:``师妹,你看是谁来了?''展大娘大吃一惊,只见门外走进了一行人,当前的是个尼姑,正是她在长安寻访未遇的师姐妙慧神尼,在妙慧神尼背后,则是一男一女,男的是她的独子展元修,女的是她的爱徒王燕羽!

展元修叫道:``妈,你下来,不要动手了!''展大娘眼光一瞬,只见展元修形容憔悴,面如黄蜡,似是大病过后一般,而且面上还有一道刀疤。展大娘不禁大吃了一惊,连忙问道:``怎么,你受了谁的欺侮了?''可是这时阵势正转到急处,她口中说话,人却仍在阵中,手也未停下。

妙慧神尼道:``师妹,你好没来由,放下儿不理,却在这里跟人胡斗!''话声未了,倏然间便已到了阵中,那``七绝诛魔阵''门户重重,竟然拦她不住,只见她挥尘一拂,这一拂恰好从韩湛与展大娘二人之间拂下,韩湛与展大娘都感到一股极柔和的内力,将他们的身子推开。妙慧神尼化解了他们相斗的劲力,一把就将展大娘拉出阵外。

王龙客这时正依着阵势,转到铁摩勒跟着,铁摩勒横剑劈去,王龙客也正张开了铁扇,当作五行剑使,削他的手腕。那一行人已陆续进来,只听得一个声音叫道:``摩勒住手!''接着一个嘶哑的声音叫道:``龙儿!住手!''唤铁摩勒的是他的师父磨镜老人,唤王龙客的则是他的父亲王伯通。

铁摩勒又惊又喜,连忙住手,王龙客却忽地一按扇柄,``嗤''的一声,一支扇骨射了出来,原来他的扇柄安有机括,可以将扇骨当作短箭射出。距离极近,本来非中不可,幸而韩芷芬对铁摩勒也是刻刻关心,一见他停手,就立刻将他一推,但饶是如此,那支``短箭''也擦着铁摩勒的手臂射过,令他受了一点皮肉之伤。

王伯通那沉重的声音又大喝道:``不肖畜生!老子的话也不听了么?''王龙客无奈何,只好退下,一眼望过去,不由得大吃一惊。

却原来他的父亲乃是躺在担架上,让人抬进来的,抬担架这两人,一个是他父亲的结拜兄弟褚遂,另一个则是他们山寨以前的``三堂总头目''华良,都是他的叔伯辈。这两人武功本来不弱,但因抬着担架,步声沉重,故此刚才听来,似是有两人不会武功。在担架旁边的是一个麻衣阔袖的老人,满头白发,面色却极红润。

铁摩勒与师父离别多年,见他精神仍然健铄,把臂上的疼痛也忘记了,对眼前的异事暂且撇开,连忙跑过去问道:``师父,你怎么到了这儿?''

王龙客听得铁摩勒称这人为师父,这一惊非同小可,连忙也跑过去叫道:``爹,你怎么到了这儿?你,你,你落在仇人的手中'了?''他跑到距离------丈之遥,忽地想起铁摩勒已然这样厉害,他师父当然更是非同小可,虽然急于见父,却竟然踌躇起来,不敢向前行进。正是:

虽云父子关天性,利害关头顾自身。

欲知后事如何?请听下回分解------

旧雨楼扫描,海之子OCR,旧雨楼独家连载

\chapter{第三十七回 忏罪解仇宁一死
片言弭祸结新知}\label{ux7b2cux4e09ux5341ux4e03ux56de-ux5fcfux7f6aux89e3ux4ec7ux5b81ux4e00ux6b7b-ux7247ux8a00ux5f2dux7978ux7ed3ux65b0ux77e5}

王伯通斥道:``畜生,你还胡说八道,什么仇人不仇人的?要不是磨镜老人,你爹早巳活不成了!''

展大娘与王龙客已然退出,那``七绝诛魔阵''也就不攻自破。精精儿退回了师父身旁,低声说道:``师父,你老人家的禁例可还要么?''

就在此时,妙慧神尼已与磨镜老人同声说道:``蓬莱比丘尼妙慧,江湖磨镜匠卜安期谒见法王,请恕闯宫之罪!''

转轮法王面色一沉,说道:``难得贵客远来,恕我未曾迎迓,如今补礼!''忽听得``呼''的一声,转轮法王连人带椅,又飞到了空中,向妙慧神尼和磨镜老人站立之处撞来!谁都看得出,这回他是有心要与妙慧神尼和磨镜老人难为'了!

妙慧神尼手抚拂尘,向外轻轻一拂,磨镜老人合起双掌,也向外一推,同声说道:``法王不必多礼,但求免罪,已是万幸!''

忽见转轮法王那椅子在空中突然停住,原来是双方的内家真力相触,彼此相持不下,故此椅子停在空中,不能再向前移动。

但这只是瞬息间的现象,妙慧神尼的拂尘自左至右的拂了一个弧圈,法王连人带椅山在空中转了一圈,倏然间又飞了回去,仍然在原处落下。众人中只有韩湛明白其中妙处,看来法王的内力要比磨镜老人或妙慧神尼都稍胜一筹,但却输刁:他们二人的合力。

法王面色沉暗,一时间却又难以发作。王伯通忽地在担架上坐起来,说道:``禀法王,他们两位是护送我到此间来的,事前未曾禀明法王,要怪也只请怪我!''

转轮法王与王伯通的交情颇好,而且王伯通的儿子又是得他欢心的弟子,因此转轮法王更难发作,只得说道:``王寨主,你当然不算外人,禁例也不必再提了。你是怎么受伤的?快进去歇息吧。这里的事,你就不必管了!''

王伯通和道:``我多得他们千辛万苦的送到此间,现在还不是歇息的时候,请法王借这地方,让我与犬子说几句话。''

转轮法王略一沉吟,说道:``好,精精儿你在此替我陪客。朱灵、朱宝,你们也帮着招呼。''拂袖而起,声音冷得令人难受,谁也不知道他心意如何?

法王退下后,王龙客也就到了他父亲的身边。只听得他父亲气喘吁吁,似是十分痛苦,王龙客也不禁掉下泪来,说道:``爹,你的话留待伤好之后再说不行么?''

王伯通沉声说道:``不能!''他转过了头,将目光投到窦线娘身上,又道:``难得段大侠贤伉俪和几位老前辈都在这儿,我这些话更应该说了,不说出来就难以心安!''

窦线娘切齿兄仇,本是对王伯通恨之入骨的,但此时见他如此模样,不由得把仇恨的心情也减了几分。只听得王伯通说道:``龙儿,我知道你一定想明白我是怎么受伤,如何得救,又何以会来到此间?这几件事我都要对你说的,但我还得先说旁的事情。

``我这一生做了许多坏事,做了许多错事,为了称霸绿林,不惜使出许多伤天害理的手段,如今想来,实是悔恨已迟!''

他说到这里,窦线娘不由得心里想道:``我们窦家,做绿林盟主的时间比他更久,仅仅今天从朱灵、朱宝等人口中听到的,伤天害理之事也是做得不少。虽然那都是我哥哥干的,但在我未出嫁之前,我也实在难以逃脱一个帮凶的罪名。''

心念未已,王伯通已接下去说道:``我做了许多坏事,许多错事,但做得最坏最错令我最愧悔的乃是做了安禄山的帮凶!我在绿林中恃强称霸,那还只是黑道中的火并;给安禄山作鹰犬,却是对不起天下的百姓!''

段圭璋心道:``难得他有此觉悟,过去种种比如昨日死,倘若他肯重新做人,我倒该劝线妹不要报仇了。''

王伯通续道:``我做了这件错事,如今是身受其报了。你们知道是谁伤我的么?''王龙客道:``咱们的仇家很多,是蔡家么?是莫家么?是------''

王伯通道:``都不是,是安禄山派来的羊牧劳,是我的好朋友羊牧劳。''此言一出,王龙客、精精儿和铁摩勒等人都不禁骇然。

王伯通道:``除了羊牧劳这帮人之外,另外也还有两帮人,这三帮人的目的各个不同,另外那两帮人攻进了龙眠谷,但亲手将我打得重伤的则是羊牧劳!''

王龙客道:``爹,你歇一歇。''将一碗茶递到他父亲的口边,王伯通喝了之后,继续说道:``我在长安闹出的那件事情,想你已知道的了。你妹妹帮铁少寨主大闹安贼的`御苑',这件事她做得对!可恨我当时皂白不分,非但不加援手,还怪责了她。

``这件事情过后,我知道安禄山决不能再信任我,我就回转龙眠谷老家,本来你妹妹早就劝过我:金盆洗手,闭门封刀。但我这一生掌权惯了,不能做个反王,也想做个贼王。因此我在龙眠谷重立旗号,仍然想当绿林盟主。''

王龙客道:``爹,人情势利,自从那年龙眠谷之役,咱们损兵折将之后,我早就料到绿林各赛,不会再像从前那样尊奉咱家,而你在朝廷之中也难以得意了。所以我才回到师父身边。爹,你其实应该等我回来,再商大计才好。''

原来王龙客是想到金碧宫搬取救兵,拉拢他那几个师兄弟出山的。他的野心更大,不但想继承父位做绿林盟主,而且想与安禄山互相利用,趁天下混乱,培植势力,争夺江山。王伯通哪知他这个心思,还以为他也已经悔悟,怒气顿消,老怀弥慰,微笑说道:``到底是你们年轻人,比我都有见识。''只有王燕羽听得哥哥仍然称安禄山为``朝廷'',感到十分刺耳。

王伯通接续说道:``龙儿,但你的话也只说对一半,他们不再尊奉咱家,还不像是因为咱们龙眠谷的实力已减,而是因为咱们助纣为虐,失尽人心。我回到龙眠谷后,绿林中分为两帮,一帮是想`墙倒众人推',将咱们王家取而代之;另一帮则并非要来争夺霸权,而是他们摸不清楚,以为我在龙眠谷招兵买马,仍然想给安贼效力,所以要为民除害。这一帮是绿林中的`侠义道',由金鸡岭的车天雄统率;要与咱们争夺霸权的那一帮,则由洪州的李麻子统率。''王龙客听到这里,``哼''了一声,道``李麻子,他也配?''原来这李麻子名唤李天敖,论武功倒是个响当当的角色,但却只是个勇夫,不通韬略,王龙客自负文武全才,一向就不怎么看得起他。

王伯通道:``你别看不起他,在咱们失势的时候,他登高一呼,也还有不少人响应他呢。

``这两帮人不约而同,都杀进了龙眠谷。可是给我以致命伤的,还不是这两帮人,而是羊牧劳所率领来的一帮`大内高手'。

``安禄山死后,他的儿子庆绪被扶作傀儡皇帝,羊牧劳权势更重,与史思明深相结纳,死心塌地的要作逆贼的开国功臣。史思明认为我已反出长安,怕我兴风作浪,与他作对,故此要羊牧劳前来杀我。

``羊牧劳趁着我们内哄的时候,乘机突袭,首先杀进内寨。幸亏这时辛天雄这帮人已发现了他们的面目,随即又知道了我已叛了安禄山,他们反而捐弃前嫌,与我合力抵挡羊牧劳,亏得他们抵挡一阵,要不然龙眠谷早已寸草不留。

``可是李麻子那一帮,被羊牧劳说动,都和他们合伙,他们的力量,比咱们强好几倍。终于羊牧劳追上了我,他竟然不念旧情,立施杀手!''

铁摩勒忍不住握拳骂道:``好一个阴很恶毒的羊牧劳,我不杀你,誓不为人!''王龙客不知就里,好生奇怪,心中想道:``我的父亲被他打伤,为何要你报仇?''当下说道:``这都是孩儿不孝,未曾随侍身边,致有此失。爹,你不必生气,待孩儿禀明师父,前去向他问罪便是。铁少寨主,多谢你的好心啦!''他认定铁摩勒乃是惺惺作态,言语之间,显然是对铁摩勒仍存敌意。王燕羽不禁皱了眉头。

但王伯通却未曾察觉,反而哈哈笑道:``我才不生气呢,多亏羊牧劳这掌,反而把我打清醒了,叫我知道了谁是朋友,谁是敌人!多行不义必自毙,他作恶多端,自有报应。你也不必向他问罪了。''他伤口未合,一笑牵动伤口,脸上的肌肉都扭曲了,形状甚是可怖。

王龙客惊道:``爹,你怎么啦?''王伯通道:``还死不了,你听我再说后来的事。''王燕羽道:``后来的事,我已在场,我代你说罢。''王伯通喘了一会,点头说道:``也好。后来的事,你是比我知道得更清楚。''

王燕羽站了出来,首先对展大娘行了一礼,说道:``请师父原谅当日我们两人私自逃走,我们逃走的缘故,一来是不愿意跟师父来此学别人七绝诛魔阵,与江湖的侠义道作对;二来是找们已决意成婚,所以要去禀明我的父亲。''原来展大娘再度出山之后,自以为武功已经练成,可以尽歼杀夫的仇人,哪知经过两次大阵仗,第一次败在疯丐卫越和段圭璋夫妇之手;第二次在骊山脚下,又领教了韩湛点穴的功夫,始知自己连韩湛也打不过,更追论磨镜老人?因此才动了念头,要儿子、徒弟跟她上金碧宫,向转轮法王学``七绝诛魔阵'',准备学成之后,再请王龙客与他的几个师兄弟帮忙,到江湖去兴风作浪,决意复仇。哪知这个心意刚表露出来,她的儿子和王燕羽当晚就逃走了。

王燕羽接着说道:``我们离开了你老人家,立刻兼程赶往龙眠谷,来得恰是时候,那羊牧劳正将我的爹爹打翻,第二掌就要结束他的性命,元哥奋不顾身地杀上去,一剑刺伤了他的手腕。''展大娘大惊道:``元修,你也忒大胆了,你怎是羊牧劳的对手。后来怎么样?''

展元修微笑道:``妈,你不是屡次责备过我胆子小,不够狠么?但倘若只是对弱者狠,对强者怯,那还算什么大丈夫?妈,你现在当会知道了,我也是够狠的,但要看是对什么人。''\,''

展大娘怔了一怔,忽地将拐杖一顿,哈哈笑道:``好,你有这个志气,不愧是你爹的儿子!我不怪你了,快说吧,后来怎么样?''段圭璋等人心中想道:``这婆娘只知道她丈夫是个英雄,却不知儿子实在还要比父亲胜过百倍、千倍!嘿,这样说还不对,一好一坏,根本就不能相比。''王伯通却露出了一个笑容,心里想道:``展大娘也说得不错。元修这副倔强的脾气,的确是和他爹爹一模一样。更好在他学到了父亲的好处,而没有学他的坏处!找得有这个女婿,也可以心满意足了。''

展元修接下去说道:``我确实不是那羊牧劳的对手,他给我冷不防的刺了一剑,居然立即便能发招还击,我的剑尖还未拨出来,就给他打中了!他带来的那帮人也立即乱刀乱剑,向我斩下!''

展大娘明明知道儿子还活着,现在正站在她的面前,看得见,摸得着,但听到这里,也不禁失声惊呼。王燕羽笑道:``师父不必害怕,吉人自有天相,就在这个时候,救星从天而降,师伯和磨镜老人联同来了。元哥就是磨镜老人救的。''

展大娘睁大了眼睛,说不出话。只听得王燕羽接着说道:``羊牧劳一见他们,不敢动手,便逃跑了。师伯以一支拂尘,就把那些围攻元哥的所谓`大内高手'的兵器,全部拂落,磨镜老人立即施救,替我爹爹和元哥推血过宫,又用了半瓶还阳补血丹,救了他们二人的性命。师父,你还不向他道谢?''

磨镜老人笑道:``些些小事,何足挂齿?那日妙慧神尼邀我去访她的师侄,我也想化解王、窦两家之仇,并顺便打听摩勒的消息,因此同到龙眠爷来。适逢其会,便吓走了那羊牧劳,说起来根本就未出过气力。至于还阳补血丹乃是我自制的药物,更算不了什么。''

磨镜老人说来轻描淡写,展大娘听了,却心中翻滚,说不出是什么滋味。要知磨镜老人的``还阳补血丹''天下闻名,那是用十三种非常难得的药物配制的,武林中人视为起死回生的至宝,磨镜老人云游四海,费尽心力,才采齐了这十三种药物,制炼了一瓶灵丹,而今为了救她的儿子,竟然不惜用了半瓶。而这磨镜老人,且还是她丈夫生前的死对头。

王燕羽接着说道:``元哥的身子好,服了灵丹,很快就恢复了,功力也未有丝毫损失。''展元修插口笑道:``只是我脸上这道伤痕却没法消除了。妈,你看我是不是变成个丑八怪了?''

展元修硕人颀颀,颜如美玉,本是个非常英俊的少年,而今面上添了一道刀疤,不但他母亲心疼,旁人看了也觉得可惜。

展大娘未曾说话,王燕羽已笑道:``元哥,你添了这道刀疤,我觉得你更好看了。要是你没有这道刀疤,我还不一定会嫁给你呢!''说到这里,她转过脸来,对展大娘说道:``师父,请原谅我们现在才告诉你,我们已经成婚了。是我爹爹在病中给我们主持的婚礼。''

此言一出,韩芷芬心上放下了一块石头,铁摩勒更是无限欢喜,他不自觉的向王燕羽溜了一眼,只见她与展元修并肩而立,手儿相握,笑靥如花,看那神情,正是沉浸在新婚的幸福之中,根本就没有注意到铁摩勒投来的目光。

展大娘的欢喜更不用说,她忽地向磨镜老人走去,施了一礼,说道:``你杀了我的丈夫,救了我的儿子,刚好扯直,从今之后,咱们的仇冤一笔勾销。''众人愕了一愕,都欢呼起来,妙慧神尼低声笑道:``师妹,你早该如此了。''

笑声一过,王伯通肃容说道:``你妹妹已有了归宿,我担心的只有你了。我不要你学我的样子,我要你学你的妹妹,改邪归正,从今之后,也不必再在绿林中混了。,''

王龙客满肚皮不舒服,但也只得说道:``做儿子的,但凭爹爹吩咐。''

王伯通忽道:``段大侠,请你们贤伉俪过来。''窦线娘迟疑了片刻,终于和丈夫一道向他走去。

王伯通怆然说道:``我这一生罪过不少,窦女侠,我知道你一定恨透了我王家的人。我不敢求你饶恕,但我却要求你饶恕我的女儿,她那时年纪还小,是我指使她杀了你五个哥哥的,你要怪只能怪我。''

窦线娘泪咽心酸,想起了自己一家的血海深仇,但眼前的王伯通又是这般模样,她要发作也发作不起来,只好不言不语。段圭璋道:``王老前辈,过去的事还提它干嘛?''

王伯通喘了口气,说道:``不,这笔债我倘若不还,非但我心里不安,窦女侠心里的疙瘩也难以消除。但望我还债之后,王、窦一家的后人不再要互相仇视。上一代人做的错事,就让上一代的人了结好了。''

窦线娘一片茫然,一时间尚未弄清楚他的话意,段圭璋已是心中一凛,急忙叫道:``王老前辈,不可------''他``轻生''二字尚未曾说出口来,只见王伯通五指向胸口一插,登时滚落担架,原来他已自断厥阴心脉,一声惨呼,便即气绝而亡!

这意想不到的惨事突然发生,众人都不觉呆了。殿堂里静寂如死,好半晌王燕羽才哭出声来。

妙慧神尼合什赞道:``放下屠刀,脱离魔道,乃大解脱,何用哀悼!''展元修低声说道:``妙慧神尼说得不错,岳父不辞一死,解怨消仇,实在是大智大勇的行为,燕妹,你不必悲伤了。''

朱灵、朱宝和那两个使月牙刀的汉子,目睹王伯通之死,面面相觑,朱灵叹了口气,说道:``冤仇宜解不宜结,算了吧!''他走到王伯通的身边,默哀片刻,便大步走出殿堂,其他三人,一声不响,也都跟着他走了。

妙慧神尼道:``善哉,善哉!''王燕羽收了眼泪,说道:``哥哥,请你师父出来吧,咱们该替爹爹料理后事了。''

王龙客一直呆若木鸡,这时忽地圆睁双眼,大声说道:``你是爹爹的孝顺女儿,你向仇人乞怜去吧!我却不能受他临终的乱命!''衣袖一拂,摔甩了妹妹,旋风的冲了出去。王燕羽又是伤心,又是气恼,咽泪说道:``哥哥,你聪明一世,何以今日如此糊涂?''可是王龙客早巳走得无踪无影,这几句话他已是听不到了。

褚遂和王伯通是八拜之交,他从担架上扶起一条薄毡,覆盖王伯通的遗体,说道:``大哥,你好好走吧。你虽没有遗言留与我,------我亦已知道你的意思,龙眠谷中的兄弟,我会替你安置的。''

褚遂张目四顾,发觉金碧宫的弟子一个都已不在,连精精儿也不知是什么时候溜了,他是个老江湖,立即感到这情形不妙。心念未已,忽见转轮法王大踏步走出来,后面跟着的正是精精儿,精精儿朝着王伯通的尸体一指,说道:``师父,你瞧,王寨主已给他们迫死了!''

段圭璋怒道:``你胡说八道,幸亏有他的女儿在这里。''

王燕羽上前向转轮法王施了一礼,说道:``家父为了解王、窦二家之仇,舍生消怨,与他们全都无涉。请法王念在与家父生前的交谊,借个地方,给我收殓。''

精精儿冷笑道:``王姑娘,不错,你是王伯通的女儿,但你早已心向外人,甚至与你王家的敌人勾三搭四的了!有我精精儿在这儿,你想花言巧语替他们开脱,这可不成!''

韩湛斥道:``精精儿,你挑拨是非,是何居心?你想害我们,这也罢了,怎的还侮辱王姑娘?''

精精儿冷笑道:``我侮唇她?好。你瞧瞧我臂上的伤疤吧?你问问她,这是不是她刺的?''

精精儿将衣袖一卷,又道:``我再告诉你吧,她刺我这一剑的时候,正是和你现在的这位女婿同在一起。那时,你的女婿是唐皇的侍卫,我是她父亲派去的刺客,她不助她的父亲,反而当场向我偷袭,助你的女婿,把我重伤。哈,哈,你明白了吧?看来她对你的女婿,比对自己的父亲还要好上十倍、百倍!'''

王燕羽气得浑身抖颤,段圭璋朗声说道:``好,这正见得王!''娘识得大是大非,你含血喷人,于她丝毫无损!'\,''

精精儿道:``各是所是,各非所非,是非二字,难以争辩,我所说的话,却是半点不假。''他转过身来,躬身向转轮法王说道:``师父,弟子不愿与外人多费唇舌,只是想师父知道这个事实。''

转轮法王沉声说道:``我知道了。王姑娘,令尊是我的好友,我自然要替他料理后事。你愿意他埋在此地还是埋在龙眠谷?'''

王燕羽听他说的只是``料理后事'',心中一宽,说道:``我不想给法王添麻烦,还是让家父回龙眠谷吧。''

转轮法王道:``好!''唤来了两个和尚,说道:``你们将王寨王搬去火化,将他的骨殖交给王姑娘。''火葬之事,当时甚属平常,在西北一带,尤其普遍。''

王燕羽是死者的女儿,依礼当然要在场看她父亲的尸体火化,于是她和展元修一道,便跟着那两个和尚走。

褚遂、韩湛、段圭璋等人也正要跟着进去,转轮法王忽道:``你们且慢,你们迫死了王伯通,还何必猫哭老鼠假慈悲?''

王燕羽大惊,连忙停下脚步说道:``法王,我已说得明明白白了,当真不是他们迫死的!''

转轮法王沉声说道:``王姑娘,我是金碧宫的主人,在金碧宫里,诸事自有我作主张,你不必多管。你不去送你父亲归天,在此作甚?难道你当真是把这些人看得比你父亲更紧要么?''

妙慧神尼道:``王姑娘,法王这样吩咐,你就去吧。''韩湛也道:``是呀,法王聪明睿智,绝不会不讲道理,你放心走吧,不必管我们了。''

王燕羽无可奈何,只好先去看她父亲火化。转轮法王面向众人,冷冷说道:``不错,我正是要和你们讲道理。''

段圭璋道:``王寨主乃是自尽,不但他的女儿可作证明,你那几个徒弟也是亲眼见的,焉能说是我们迫死?''

转轮法王道:``王伯通已死,他是否甘心自尽,我已不能再问他了。''

段圭璋道:``他临终时说的话,你那几个徒弟也是听得清清楚楚的。精精儿,你本着良心说吧,王寨主临终时是怎么说的?''

精精儿冷冷说道:``不错,王寨主临终之时,的确是说要以一死解仇。他还请求你们不要仇视他的儿女,这正是他为了子女的缘故,才不惜自了残生的啁,还能况不是给你们迫死的吗?''

同样的一件事实,经精精儿这么加以``解释'',便显得王伯通之死,不是由于忏罪,而是为势所迫了。段圭璋不善说辞,只气得顿足道:``你这不但是污蔑生人,而且是贬低死者了。''

转轮法王沉声说道:``不是我袒护徒弟,精精儿的话实在是比你们有道理得多。但王伯通已死,他的心意已无从得知,既然你们各执一词,我也就暂且撇开这件事情不说。''

韩湛松了口气,道:``好,那么倘若法王不允我们去送王寨主归天,我们就告辞了。''

转轮法王冷冷说道:``韩先生,我已说过,看在你与我故友藏灵子的情份上,我对你可以网开一面,金碧宫的禁例不施用于你。''

韩湛听他旧话重提,大吃一惊,说道:``怎么,你还是不让我们走么,难道你当真要与小辈动手?''

转轮法王道:``韩先生,你要走尽管走,他们却不能走。你别罗嗦丁。''

磨镜老人眉头一皱,说道:``如此说来,我们擅上黑石峰,也是犯了禁例,请法王一并治罪。''

转轮法王道:``我正要和你们说,刚才我看到你们是与王伯通同来,所以未曾向你们提出我的禁例,现在王伯通已死,你们失了保人,这禁例的确也要施用于你们了。''

磨镜老人亢声说道:``好吧,法王要如何治罪,小老儿在这里恭候!''

转轮法王道:``正是因为有你与妙慧在此,我才好办。''他顿了一顿,继续说道:``韩先生知道,自藏灵子死后,天下虽大,对手难求,我是久已乎不与别人动手的了,倘若只是你一人到来,我也还不会与你较量,但如今你与妙慧同来,我却可以破例了。''言下之意,即是要磨镜老人与妙慧神尼联手,同他较量。

磨镜老人哈哈笑道:``承法王青眼有加,小老儿不胜荣幸之至,但请法王示下,敢不奉陪。''

转轮法王道:``我把话先说在头里,他们是小辈,我不屑动手,只是与你们二人较量,倘若你们胜了,你们的人,我全都让走;倘若你们败了,则都要任凭我的处置。你们可敢担负这个关系么?''

铁摩勒道:``师父,尽管应承!''磨镜老人向妙慧神尼笑道:``神尼,咱们今日可是败不得啊!他们都把性命对托给咱们了。''妙慧神尼笑道:``胜败之事,由不得咱们作主,咱们各自尽力,向法王领教便是。''

只见转轮法王把手一招,里面走出四个和尚,抬着两张香案,每张香案上插着五枝粗如儿臂的油烛,将两张香案摆在两边屋角,遥遥相对,中间的距离,约莫三丈有多。随即把蜡烛都点燃起来。

众人方在诧异:``这作什么?''只听得法王说道:``妙慧神尼,磨镜老人,咱们不比市井之徒,只知蛮打,今日各以本身功力,比比高下如何?''磨镜老人与妙慧神尼同声说道:``但凭法王吩咐。''

执役和尚在法王那边的香案下摆了一个蒲团,在磨镜老人这边的香案下摆了两个蒲团。转轮法王然后说道:``咱们各以本身功力,弄熄对方的蜡烛,烛在人在,烛灭人亡,请两位就座吧!请!''

磨镜老人刚才踏进金碧宫的时候,便与转轮法王试过一招,深知他的功力,心中想道:``倘若我和妙慧神尼联手,与他性命相搏,大约胜算可操。如今相隔数丈,各以内家真气,烛灭伤人,这胜败之数,就难预料了。''妙慧神尼也知道转轮法王所练的内功颇有怪异之处,因此也不敢托大,只好与磨镜老人联手,以二敌一。

各自端坐在蒲团之上,只听得法王一声长啸;有如裂帛,刺耳非常,功力稍弱的如诸选、王燕羽诸人,都觉得心旌摇动,似乎``灵魂''就要出窍而去,韩湛、段圭璋等人虽然禁受得起,也觉得非常之不舒服!

啸声中只见磨镜老人这边的烛光摇晃不定,原来转轮法王已练成了天竺婆罗门教的坎离气功,与佛门的狮子吼功异曲同工,可以扬声慑敌,吐气伤人。端的是厉害之极。

妙慧神尼口宣佛号,拂尘轻轻向外一拂,她的声音甚是柔和,王燕羽等人听了,如闻妙乐,心头的烦闷登时解了。展大娘又羡又妒,心中想道:``师姐沉默寡言,青灯礼佛,我只道她愚钝,谁知却练成了这等绝世神功。''

妙慧神尼座前的烛光似给一股无形的潜力扶直起来,但仍有些摇晃。磨镜老人忽地拍掌大笑,朗声吟道:``大风起兮云飞扬,安得猛士兮守四方?安得猛士兮护烛光?''前两句是汉高祖刘邦的《大风歌》,后一句是他自己加上去的,歌声雄壮豪迈,说也奇怪,''他一拍掌放歌,转轮法王面前的烛光也开始烛影摇红,忽明忽暗!

原来磨镜老人的内功居于阳刚一路,击掌放歌,正足以助长威力。转轮法王自顾不暇,只好暂时放弃攻击,转为防御。

但见转轮法王闭目垂首,状如老僧人定,香案上的烛光又复明亮如前。妙慧神尼与磨镜老人的内家真气,到了对方香案之前,竟似被一堵无形的墙壁拦住,都不由得心中一凛。

其实这并不是因为法王的内功就胜过他们二人,而是因为他们二人的内功路数不同,一个冲和,一个刚猛,二人联手,一时间尚未能彼此协调,刚柔并济。

转轮法王的武学造诣何等精深,不过片刻,就觉察到对方攻来的内力各自为战,未曾合为一股,他故意示弱,将防御的范围缩小。磨镜老人的纯阳罡气先行攻到,那张香案就似受到风浪冲击一般,摇动起来,而且格格作响,似乎不久就要震裂。

铁摩勒心中大喜,低声对韩湛说:``到底是我的师父高明。''韩湛面色沉重非常,也低声说道:``你瞧他案上的烛光!''铁摩勒定睛一看,只见他的那张香案虽然摇动,但案上的烛光却是向上燃,越燃越旺,丝毫未受影响。铁摩勒虽然不懂其中奥妙,但也想得到法王乃是用内家真气护着烛光,他不怕对方的强烈攻击,仍然闭目如前,神色不变,显见是有恃无恐,智珠在握。

铁摩勒心念未已,陡然间只见转轮法王双目倏张,啸声又起,中指向前一点,猛地喝道:``灭!''话声未了,磨镜老人香案上的一根蜡烛,烛光应声而灭!铁摩勒等人距离香案约有一丈之遥,但在这瞬间,都感到有一股劲风扑面,尖利如刀,劲风吹过,刮得肌肤隐隐作痛。

铁摩勒大吃一惊,但几乎就在同一时间,只听得磨镜老人也大喝一声``灭!''转轮法王香案上的烛光也应声灭了。而且熄掉的蜡烛不止一根,而是两根。

要知磨镜老人与妙慧神尼的武学造诣也非比寻常,正巧就在这一瞬时,两人已参悟了刚柔配合之道。而恰恰转轮法王又正全力出击,防御自然相应减弱,因此妙慧神尼与磨镜老人都同时灭掉了对方的一支烛光。

转轮法王吃了一惊,连忙双掌合什,又再转为防御。双方的内家真气互相激荡,争持不下,在两张香案的中间,形成了一股旋风,地上的泥尘随风旋转,恍如在屋中间布下了一张圆形的黄帐。

过了一会,只见转轮法王的头顶上空,升起一团白色的烟雾,额上一颗颗似黄豆般粗大的汗珠流下来,那团白色的烟雾乃是他的汗水蒸发所致。

韩湛低声说道:``法王要拼命了!''话犹未了,只听得法王大喝一声,双掌齐扬,磨镜老人这边的香案,两支烛光同时被风吹灭。

这时是法王这方占先,但磨镜老人与妙慧神尼的面上都露出喜色,他们心中明白,转轮法王实在已是将近强弩之末,虽然得逞一时,终将支持不住。

妙慧神尼念了一声:``阿弥阳佛'',拂尘往外一层,把对方的烛光也灭了一支,而磨镜老人的纯阳罡气却被对方挡住、

至此双方又打成平手,面前的烛光都只剩下两支,胜负看来即将分晓,双方都全神以赴,攻守兼备,不敢松懈。只见那股旋风,有时向磨镜老人这边移近,有时又向法王那边移近,旁人看来,仍是个功力悉敌,争持不下之局。但他们双方却都是心里有数,法王这时已用尽全力,妙慧神尼这方只要再守得片时,便可大举反攻,一举制胜。

正在双方激烈争持,面前的烛光都是忽明忽暗之际,忽见白影一晃,竟不知是什么时候,一个白衣人走了进来,无声无息的转眼间就出现在屋子当中,正当风力中心之处。

这白衣人身形一现,便双拳合抱,向周围作了一个罗圈揖,顿时间,两边香案上剩下的那四支烛光,都告消灭。

众人定睛一看,只见这人竟是个面如冠玉的少年,看来不过二十多岁,都是大为诧异。要知他趁着双方全力争持之际,乘虚而人,尸举而灭掉四支烛光,这虽有点取巧,但他处在双方内家真气激荡的中心,而居然还是神色自如,这份功力,就不在转轮法王之下。

转轮法王未曾喝问,正待缓过气来,那少年已是朗声说道:``未学后进,扶桑虬髯客第三代弟子牟沧浪奉家师之命,谒见法王。''转过身来;又向磨镜老人与妙慧神尼施礼道:``幸会两位前辈!''

转轮法王这一惊非同小可,心中想道:``他只是虬髯客的徒孙,便已有这般功力,倘若是虬髯客的衣钵传人一一他的师父严一羽到来,那我唯有拱手认输了。''

转轮法王缓了口气,定了定神,问道:``令师遣你到来,有何见教?''

牟沧浪道:``二十年前藏灵子曾到扶桑与家师相会,道及法王有意折节下交,邀他到金碧宫作客,只因家师有誓在先,不来中土,难领盛情,心中耿耿。是以今日差遣弟子前来,代表家师,向法王讨教。''

转轮法王面色大变,半晌说道:``原来严一羽叫你来伸量我的武功么?''

牟沧浪道:``法王误解家师之意了。弟子末学后进,岂敢与前辈较量?家师是因法王当年不耻下问,故此叫弟子来与法王研讨武学。''

转轮法王这才想起,当年他请藏灵子代邀严一羽前来,实是想向他请教几个武学上的难题,当时他与藏灵子尚未失和,曾同心合力探讨上乘武学,但因各人所习的内功不同,是以各有各的难题。向严一羽请教,乃是藏灵子的主张,而经转轮法王同意的。却不料严一羽有誓在先,不能前来中土。而藏灵子从扶桑归来之后,不久就与转轮法王失和。其后藏灵子武功大进,转轮法王知道这是他得严一羽的指教所致,又羡义妒,他也曾几次三番,想到扶桑岛去,但一来因为要飘洋过海,他从来不习水性,不懂驾舟;二来他比藏灵子更骄傲,藏灵子少年时候曾见过严一羽的师父虬髯客,说起来与他师门有旧,而转轮法王是个从天竺来华的僧人,只是听过虬髯客师徒的大名而已,因此他也不愿``移尊就教'',向一个陌生的大家年纪差不多的人执弟子之礼。他当年请藏灵子代为邀客,实在亦是想端住``身份'',请平辈前来``切磋'',而避免像藏灵子那样,以后辈的身份登门向前辈``请教''。

想不到严一羽自己不来,却在二十年后的今天,差遣他的弟子来了。这牟沧浪的话说得甚是谦恭,但他提起法王当年``不耻下问''之事,言下之意,却似乎是表明,他是严一羽派来,``指教''转轮法王的。

转轮法王心中着恼,心道:``你年纪轻轻,懂得多少上乘的武学,''但碍于严一羽的面子,又不愿给人说他自大自满,是以虽然气在心中,却不便发作。他想了一想,这才说道:``这么说,你来得正是合时,我的武功深浅如何,想你也知道个大概了。你倒给我说说看,我可有不到之处吗?''

牟沧浪道:``弟子本来不敢妄自谈论,不过家师心有所虑,而弟于今日所见,家师的忧虑又似乎不幸言中,是以不敢不言!''

转轮法王大吃一惊,急忙问道:``尊师所虑者何?''

牟沧浪道:``当年家师听说法王已在修习坎离气功,曾有言道,这坎离气功练成之后,威力固然极大,但只怕会有后患。依刚才的情形看来,法王的坎离气功已是为山九仞,只差一篑,大约明夷一脉尚未打通,倘依法王现在的练功途径,怕只怕一年之后,便有走火入魔之厄!''转轮法王本是端坐蒲团,听列这里,不禁离座而起!

众人见转轮法王前倨而后恭,都不禁暗暗好笑。转轮法王这时已顾不得面子,连忙合什施礼,说道:``尊师端的是学冠天人,明鉴万里。请问这走火人魔之难,可有法子避过么?''

牟沧浪先还了一礼,然后徐徐说道:``法王于武学无所不窥,想必于`三象归元'、'七宝连树'的奥义,都已是洞悉无遗的了。家师言道,欲免走火入魔,当于此二者求之。''

转轮法王听了,不禁面上一阵青,一阵红。原来这``三象归元''与``七宝连树''乃是最深奥的两种内功心法,转轮法王只知道有这两个名辞,至于具体内容,却还丝毫未知,哪里谈得到有深入的研究?不得不老着面皮说道:``不敢相瞒,这两门武学,老衲只是稍经涉猎,未曾深究。难得牟居士远来,就请在此梢住些时,容老衲得以请益如何?''

牟沧浪故意作出惶恐不安的样子,说道:``法王如此说法,岂不折杀了小辈么?这个,这个,晚辈不敢!''

转轮法王忙道:``学无前后,达者为师。牟居士,就请你看在老衲二十年前已向尊师求教的这点诚心,屈驾住下来吧!''

牟沧浪想了一想,忽地微笑说道:``家师此次差遣弟子前来,本意是想让弟子与法王研讨武学,如今法王又盛意拳拳,晚辈自是不宜过拂,理该相互琢磨,彼此增益\ldots\ldots{}''

转轮法王听他说得谦虚,心中甚喜,不待他把话说完,便连忙吩咐精精儿去收拾一间静室,请牟沧浪去住。

哪知牟沧浪顿了一顿,却拖长声音说道:``可是------''转轮法王一怔,问道:``可是什么?''

牟沧浪道:``可是晚辈到了西域之后,听说法王这里有个规则,若是外人未得法王准许,不可擅上黑石峰,晚辈此来,事前未曾请准法王,正自惶恐不安,但不知这个规矩可是真的么?''

铁摩勒口快说道:``怎么不真?我的师父和妙慧神尼,刚才还正因此而与法王比武呢!''

牟沧浪道:``哦,原来两位前辈是因此与法王比武的。磨镜老人,家师久闻侠名,曾嘱弟子到了中土,必定要谒见老人致敬;妙慧神尼,我的大师兄十六年前在长明岛曾蒙你老人家解围,敝师兄也嘱我向你问候。唉,想不到都在这里相逢,真是巧极了,却也是不巧极了!''

转轮法王忽地哈哈大笑,向磨镜老人与妙慧神尼合什作礼道:``咱们这真是不打不成相识。这条禁例,从今作废。还求两位不弃下愚,弃嫌修好,结个佛缘,随时赐教。''

要知转轮法王即算不是为了牟沧浪的缘故,他也胜不了磨镜老人与妙慧神尼,牟沧浪之来,恰巧在他将败未败之际,化解了这场恶斗,等于是给他挽回了面子,他正好藉此收篷。

这样一来,皆大欢喜。磨镜老人与妙慧神尼当然也不为已甚,齐道:``善我!''向法王还礼。

这时王燕羽已将她的父亲尸体火化,带着盛着骨灰的坛子走出来,见双方已经和好,也很欢喜。

当下,王燕羽与褚遂这一班人便向法王告辞,要将王伯通的骨灰奉回龙眠谷,展大娘为了儿子的缘故,也与他们同行。

展大娘道':``师姐,咱们姊妹多年不见,你也和我们到龙眠谷走一趟吧。''妙慧神尼道:``只是我那两个徒弟还未知道下落,放心不下。''铁摩勒道:``两位令徒可是聂隐娘和薛红线么?正好教神尼得知,薛红线真名是史若梅,她是段大侠未过门的媳妇,现在她们二人都已随薛嵩到朔方去了,将来我们都要到朔方去的。''妙慧神尼道:``既然如此,我就先走一步吧。我陪师妹到龙眠谷住几天,便去朔方。但望咱们能在朔方再见。''

铁摩勒与展元修意气相投,如今展元修又已是王燕羽的丈夫,铁摩勒更是心中欣慰,他是个直爽的人,也就不避嫌疑,单独上前,与他们夫妇殷殷道别。韩芷芬面露笑容,毫不迟疑,也跟上来与王燕羽道别。韩芷芬笑道:``王姐姐,咱们也可说是不打不成相识。就可惜没有喝上你的喜酒。''王燕羽笑道:``等你与摩勒成婚之日,我再来借花敬佛吧!''她的眼光从韩芷芬脸上溜过,瞅了铁摩勒一眼,若不经意的就携着丈夫的手走了。铁摩勒想起过去种种情事,也颇觉有点惘然,心中默默为他们祝福。

与王伯通有关的那些人都已走了,段I:璋''¨湛诸人也跟着告辞。磨镜老人得知段圭璋是要向空空儿讨还儿子,说道:``空空儿的师父当年也与我有点交情,我和伯;们一同去吧。''转轮法王送出寺外,说道:``空空儿这人眼高于顶,目无尊长,要是他恃强不还,你们回来说与我知,让我去教训他。''段圭璋道:``不敢有劳法王。还望法王念在与藏灵子的旧谊,金碧宫该与玉皇观和解才是。''正是:

宽厚待人真侠士,只求排难解纷争。

欲知段圭璋此去,能否讨回儿子,请听下回分解------

旧雨楼扫描,海之子OCR,旧雨楼独家连载

\chapter{第三十八回 喜见娇儿疑梦境
惊闻良友困危城}\label{ux7b2cux4e09ux5341ux516bux56de-ux559cux89c1ux5a07ux513fux7591ux68a6ux5883-ux60caux95fbux826fux53cbux56f0ux5371ux57ce}

黑石峰与玉树峰遥遥相对,出了金碧宫,就可以远远望见玉树峰顶的玉皇观,可是走起来却很费劲。段圭璋一行人等,都有上乘轻功,如紧脚程,但到了玉皇观前,也已是将近黄昏时分。

段圭璋满怀欢悦,上前叩门,朗声说道:``段某践约而来,请见主人。''哪知叩门几次,里面竟然毫无声息,与上次一模一样。段圭璋顿起疑云,心里想道:``莫非是空空儿等得不耐烦,已先走了?但我虽说来迟,也还没有过期呀?嗯,莫非,莫非\ldots\ldots{}''

他疑心方动,窦线娘已抢先说了出来:``我说空空儿不可靠,你看,还不是与上一次一样------又一个骗局!''

铁摩勒十分难过,说道:``空空儿怎能这样?我与他理论去!''就在窦线娘冷笑声中,他一掌震开了观门!段圭璋忙道:``你不可鲁莽。''他仍然守着客礼,进了大门,立于阶下,再一次通名禀告道:``段圭璋远道来迟,请主人恕罪,允予接见。''

话声未了,忽听得一声长笑,愤然间但见剑光一闪,一柄亮晶晶的匕首,刺到段圭璋面门。

段圭璋大吃一惊,一个``盘龙绕步'',疾忙一掌推去,只昕得``嗤''的一声,半条衣袖,已给匕首削下。

段圭璋喝道:``空空儿,你------''这``你''字刚刚出口,空空儿的短剑就划到了他的面前。

段圭璋气得七窍生烟,霍地一个``风点头'',宝剑亦已出鞘,一招``横架金粱''斜削出去,空空儿似是识得宝剑的厉害,一溜烟似的从段圭璋身旁掠过,段圭璋这才缓过口气,把未曾说完的那句话说了出来:``空空儿,你,你还是人吗?''

空空儿侧身进扪,冷冷说道:``你胜得了我,自有分晓!''话声未了,嗖、嗖、嗖,已是连发三招,当真是疾逾飘风,匕首所指,不离段圭璋要古穴道,冷电精芒,耀眼生缬,迫得段圭璋东躲西闪。

幸亏段圭璋也是惯经大敌之辈,退了几步,猛地使出一招硬碰硬的打法,宝剑抡圆,剑光暴长,疾圈过去,大声喝道:``段某自知不是你的对手,也要和你拼了!''

段圭璋深知空空儿的本领远在他上,他这一招其实是以攻为守,哪知一剑削出,空空儿竟然不敢招,架,一个筋斗便倒翻开去,同时``嘤''的一声叫了出来,那声音竟似带着几分怯惧。

段圭璋不禁大为诧异,在他使出这一招的时候,本来也估计到空空儿不会和他硬拼,但以空空儿的本领,却尽可以移形换位,从另一个方向向他攻击,他绝对料想不到空空儿竟然弄到要在地上翻滚躲避,狼狈不堪,而且还会叫出声来!

可是这只是刹那间的现象,就在段圭璋疑心方起,一怔之下,还未来得及再度进招之际,猛听得空空儿一声喝道:``你看我这招移星摘斗!''在地上一个盘旋,倏然间弓身一跃,果然便是一招``移星摘斗'',短剑直指到段圭璋的面门!

本来,在对敌之际,先说出自己所要使的招数,无异教对方如何防御,但一来由于空空儿的身法太快;二来也由于段圭璋不敢相信,哪知空空儿却真的是使出这一招,而这一招又的确是最恰当的一招。待到段圭璋心中一凛,闪身还击之时,只听得``唰''的一声,空空儿的匕首又已在他的肩头划过,挑破衣裳,只差半寸,险险就要挑了他的琵琶骨。

铁摩勒忍不住就要拔剑而起,韩湛忽地将他一按,低声说道:``事有跷蹊,你休妄动。''

空空儿一招见效,以后接连进招,一气呵成,有如流水行云,得心应手,轻灵翔动,超妙绝伦,把段圭璋迫得只有招架之功,并无还手之力。在旁人看来,段圭璋已是险象环生,但在段圭璋心中,却有个奇异的感觉,空空儿的招数虽然精妙,身法也极轻灵,但功力却似不及从前,不知他是故意留情,还是真的如此。

韩湛按得住铁摩勒,却按不住窦线娘,她早已静待时机,这时段圭璋正好又使出一招凶猛的招数,空空儿仍然不敢和他硬碰,就在两条人影倏然分开之际,窦线娘急拽弹弓,噼噼啪啪,一连串弹子打了过去,空空儿东跳西闪,弹子全部落空,可是也已显出有点手忙脚乱。

窦线娘大喜,心道:``想不到空空儿的技艺已然生疏了!''一跃而前,立即展开``金弓十八打''的家传绝技,夫妇联手,果然主客易势,占了上风,反转来把空空儿打得只有招架之功,而无还手之力!

韩湛忽地悄声说道:``你瞧这空空儿的身材似乎太矮小了。''空空儿的身材本来矮小,因此铁摩勒一直没有留意,这时听了岳父的话,留心一看,果然觉得有点奇怪,因为这个空空儿似乎比他以前所见的空空儿还要矮小几分。

铁摩勒方在疑惑,只见场中形势已是大变,原来窦线娘恨极了空空儿,她一占了上风,得理不饶人,竟然招招都是杀手。刚才是空空儿着着进迫,现在却是她咄咄迫人,空空儿东跳西闪,已显得有点慌张之态。

激战中窦线娘使出穿花绕树身法,忽地欺身进击,一招``雁落平沙'',金弓朝着空空儿的脖子,自上而下一拉,要是给她的弓弦拉实,空空儿的脖子非折断不可。

空空儿头颈一侧,叫道:``看我这招草船借箭!''匕首斜斜翘起,倏然间贴着弓弦反削过去,但听得``嗤''的一声,窦线娘的半幅衣袖也给削去了。

可是窦线娘却是拼着两败俱伤的打法,她的``金弓十八打''变化无穷,空空儿没有刺中她的皮肉,她的弓弦猛地往外一``蹦'',``啪''的一声,已``割''着了空空儿的脸皮。

段圭璋忽然惊叫道:``线妹,住手!''你道他何以如此惊惶?原来空空儿侧头发招之时,正好面向着段圭璋,窦线娘看不见,他则看得分明,空空儿的嘴巴并未张开,但却有声音发出。显然这个人并不是空空儿,真的空空儿正伏在暗处,指点他使这一招``草船借箭''。段圭璋猛地心中一动,这才不由得叫出声来!

双方动作都快如闪电,待得段圭璋出声,已经迟了。窦线娘的弓弦已划破了空空儿的脸皮,一时之间,收手不及,还要往下割去!

就在这一瞬间,窦线娘但觉眼前人影一闪,手上突然一轻,随即听得哈哈大笑的声音,窦线娘手上的金弓已给人夺去。她疾退三步,定睛看时,只见两个``空空儿''立在一起,一个空空儿手上拿着她的金弓,另一个空空儿正伸手将自己的``脸皮''撕下,原来是张根薄的人皮面具,面具被弓弦割破了,他却未有受伤,露出了本来面目,只是个稚气未消,十岁左右的孩子。

这一瞬间,段圭璋夫妻全都呆了。只听得空空儿笑道:``我没有骗你们吧?你们的孩子是不是已练成了绝世武功?''又说:``师弟,这两个人就是你的爹娘了,你还不快去拜见爹娘!''

段圭璋热泪盈眶,迎上前去,张开双臂,那孩子投进了他的怀中,说道:``爹,娘,恕孩儿认不得生身父母,刚才令你们受惊了。''窦线娘这时方始走过神来,连忙也抢上前去,将孩子揽住,说道:``好孩子,我没有伤着你吧?''空空儿笑道:``师弟,把这把金弓还给你妈妈吧!窦女侠,这回你不会再骂我了吧?''

窦线娘给他弄得啼笑皆非,有几分气恼,却也有几分感激,只好默然接过金弓,一声不响。铁摩勒道:``空空儿,你也未免太恶作剧了!''空空儿笑道:``要不如此,段大侠怎知他的儿子十年来遭遇如何,成绩怎样?再说,这场恶作剧也还不是我的主意。''

段圭璋心中一动,想起以前空空儿对他说过的话,说是另有异人收他的儿子为徒,而刚才又听得他叫自己的儿子做``师弟'',心中颇觉奇怪,暗自想道:``藏灵子早巳死了,据韩湛所云,藏灵子又并无同门兄弟,他们这师兄弟的称呼却是从何而来?''

窦线娘却无心去想这些,搂着儿子,说道:``你失踪了十年,想死了为娘的了。好孩子,难为你已练成了一身武功,明天就随爹娘回去吧。还有一个人,是你一定要见的。''段克邪现出迟疑的神气说道:``妈,这个么,孩儿还要问过师父。''窦线娘道:``啊?你另外还有师父?''她只当儿子的武功是空空儿教的,现在才知道不是。

话犹未了,忽听得一个苍老的妇人声音哈哈笑道:``克邪,你见了爹娘,还没忘记师父。不枉我收你为徒。''只见一个扶着拐杖的老妇人,已颤巍巍的向他们走来。

韩湛``啊呀''一声,连忙迎上前去,施礼说道:``归夫人,多年不见,你的精神更好了!''原来藏灵子的俗家名叫归方震,这个老妇人正是他的妻子。

归夫人道:``小韩,你也还没有什么老态呵!难得你今日也来到此间。你看,我收的这个徒弟,可比得上方震的徒弟么?''

空空儿忙道:``当然是师弟比我强得多,我像他这般年纪,还只会上树捉雀呢。''韩湛道:``你教徒弟确是比尊夫高明,这孩子现在已是强爹胜祖,再过十年,那还了得?要是方震还在,也------定向你认输的。''

归夫人又哈哈大笑,说道:``段大侠,我未得你们夫妇同意,就将这孩子留了十年,是有点不近人情,但我已将我一身的本事传了给他,想来也可以将功赎罪了。''

原来藏灵子和她本是一对很好的夫妻,只因彼此都有好强争胜的脾气,以至中道乖离,他的弟子空空儿已名满天下,归夫人一面是怀念亡夫,同时却又起了个古怪的念头,想和丈夫再``斗''一次,争一口气,自己也教出个好徒弟来。这个感情,其实也是基于她对丈夫的思念。

恰好那时空空儿将段圭璋的儿子掳来,这孩子又长得十分可爱,她一见之后,便把这孩子要了去,她怕孩子的父母不依,故此不许空空儿说明真相。

这件事情的前因后果说明之后,窦线娘只有喜出望外,哪里还敢埋怨,段圭璋道:``多谢归夫人对这孩子加惠成全,我们夫妇没齿不忘。请夫人准许我们将他领回去。''

归夫人道:``他是你们的孩子,当然应该回到父母身边。可是在他离开之前,我还要他给我办一件事。''段圭璋道:``有事弟于服其劳,请夫人吩咐他便是。''

归夫人道:``克邪,你去给我杀一个人!''

段圭璋吃了一惊,段克邪转着一双黑白的小眼珠,声音有点颤抖,问道:``师父,你要我杀什么人,我,我有点害怕!''

归夫人道:``我正是要你练练胆子。''接着说道:``听说精精儿已逃到金碧宫,求庇于转轮法王了。空空儿,你陪克邪去走一趟,将精精儿的首级取回来见我。你给克邪掠阵,我要克邪亲手杀他。''

空空儿脸色青白。归夫人道:``怎么?你不愿意?你可知道,你师父已死,你本来就应该负起这清理门户之责的。''

归夫人又道:``我知道你与精精儿自幼相处,情份太深,不忍叫你亲自下手,所以才要克邪为你代劳。但你可要小心,克邪若给精精儿伤了一根头发,回来我就问你。''

空空儿道:``要是转轮法王不依呢?''

归夫人冷笑道:``他敢?你可以对他说这是我的命令,要是他敢道半个不字,我去毁了他的金碧宫!他也应该知道,我丈夫生前处处让他,我却是不肯让人的。哼,他大约以为我早已死了,要不然他也不敢收留精精儿。''原来归夫人中年与丈夫分手,她另有住处,这回是为了交还段圭璋的儿子,才到玉皇观的。

空空儿无可奈何,只好领命,归夫人又吩咐段克邪道:``此去不比刚才,刚才你是与父母试招,你既然事前不知,我却是吩咐过你不许伤人的。这次我是要你去取精精儿的首级,你务必要狠毒心肠,下得辣手。''

段圭璋暗暗皱眉,心里想道:``这归夫人武功虽高,究竟乃是邪派。幸喜我儿天性纯良,不过自幼跟她,只怕也沾染了些邪气了。''但他心中虽然不满,却也不敢发作出来,只好眼睁睁的看着空空儿和他的孩子出去。

归夫人道:``你们走了这么多山路,肚子想必早已饿了。''吩咐观中老道备上斋饭,便邀段圭璋等人人席。

段圭璋夫妇虽然知道有空空儿陪伴,他们的孩子绝不至于吃亏,但心里仍是惴惴不安,食难下咽。归夫人却和韩湛谈笑风生,毫不在意。直到晚饭过后,她才皱起眉头道:``已过了一个时辰了,怎么还不回来?''

韩湛道:``待我去看一看如何?''归夫人道:``不必。嗯,你刚才说到的那个人是谁?他一举手而把两边的烛光全部灭了,虽说有点取巧,这份功力却也不容小视呵!''原来韩湛一直在叙述妙慧神尼、磨镜老人与转轮法王在金碧宫比武的事情,刚刚说到牟沧浪突如其来的一节。

韩湛笑道:``这个人么,说起来他的师门倒与尊夫有点渊源------''刚说到这里,归夫人忽地站了起来,一掌拍出,沉声喝道:``你是何人?来此何事?''

只觉微风飒然,那牟沧浪已进了屋子,以韩湛等人的武功,都未察觉他是何时来的。归夫人更是惊诧。她的劈空掌已用到八成功力,来人竟似毫无所觉。

牟沧浪施礼说道:``扶桑虬髯客再传弟子牟沧浪谒见归夫人。好教夫人得知,韩老先生刚才说的那个人就是晚辈。''

归夫人怔了一怔,连忙说道:``牟先生不必多礼,拙夫二十年前曾到过扶桑岛向尊师请教,你我只应以平辈论交。''

牟沧浪道:``那时我还只是三岁小童,论德论齿,小可都不敢高攀。''仍然以前辈之礼见过归夫人。归夫人见他谦抑自下,甚为好感,还了一礼,然后问道:``牟先生到此,可是奉了尊师之命,有何指教么?''

牟沧浪道:``家师差遣我到玉皇与金碧宫谒见归夫人与转轮法王两位前辈。我因路近,先到/-;碧宫,始知玉皇观与金碧宫失和,是以晚辈不揣冒昧,想来作个鲁仲连。''

归夫人道:``啊,原来你是作鲁仲连来了,可是那转轮法王私自收留了我丈夫的弟子,他不赔罪求和,我是实难遵命。''

``哦,空空儿,你回来了?''原来正在牟沧浪与归夫人说话之间,空空儿与段克邪手携着手,已从外面走进。

归夫人面色一沉,道:``精精儿的首级呢?''空空儿取出一个拜匣,说道:``请师娘恕罪,精精儿早已逃走,弟子不知他逃向何方,是以只好先回来复命。转轮法王自知理亏,写了这赔罪的拜帖,命我转呈师娘。''

归夫人有了面子,又有牟沧浪从旁劝说,气便消'了,当下说道:``既然如此,礼尚往来,你明日也拿我的贴子去回拜他吧。至才说到的那个人是谁?他一举手而把两边的烛光全部灭了,虽说有点取巧,这份功力却也不容小视呵!''原来韩湛一直在叙述妙慧神尼、磨镜老人与转轮法王在金碧宫比武的事情,刚刚说到牟沧浪突如其来的一节。

韩湛笑道:``这个人么,说起来他的师门倒与尊夫有点渊源------''刚说到这里,归夫人忽地站了起来,一掌拍出,沉声喝道:``你是何人?来此何事?''

只觉微风飒然,那牟沧浪已进了屋子,以韩湛等人的武功,都未察觉他是何时来的。归夫人更是惊诧。她的劈空掌已用到八成功力,来人竟似毫无所觉。

牟沧浪施礼说道:``扶桑虬髯客再传弟子牟沧浪谒见归夫人。好教夫人得知,韩老先生刚才说的那个人就是晚辈。''

归夫人怔了一怔,连忙说道:``牟先生不必多礼,拙夫二十年前曾到过扶桑岛向尊师请教,你我只应以平辈论交。''

牟沧浪道:``那时我还只是三岁小童,论德论齿,小可都不敢高攀。''仍然以前辈之礼见过归夫人。归夫人见他谦抑自下,甚为好感,还了一礼,然后问道:``牟先生到此,可是奉了尊师之命,有何指教么?''

牟沧浪道:``家师差遣我到玉皇与金碧宫谒见归夫人与转轮法王两位前辈。我因路近,先到/-;碧宫,始知玉皇观与金碧宫失和,是以晚辈不揣冒昧,想来作个鲁仲连。''

归夫人道:``啊,原来你是作鲁仲连来了,可是那转轮法王私自收留了我丈夫的弟子,他不赔罪求和,我是实难遵命。''

``哦,空空儿,你回来了?''原来正在牟沧浪与归夫人说话之间,空空儿与段克邪手携着手,已从外面走进。

归夫人面色一沉,道:``精精儿的首级呢?''空空儿取出一个拜匣,说道:``请师娘恕罪,精精儿早已逃走,弟子不知他逃向何方,是以只好先回来复命。转轮法王自知理亏,写了这赔罪的拜帖,命我转呈师娘。''

归夫人有了面子,又有牟沧浪从旁劝说,气便消'了,当下说道:``既然如此,礼尚往来,你明日也拿我的贴子去回拜他吧。至于精精儿我却不能让他畏罪潜逃,我限你在三年之内,将他捉回来见我。''

段克邪嘻嘻笑道:``牟大哥,你的轻功比我的师兄还要高明,我服了你了!''

牟沧浪道:``那是你师兄故意让我的。若然真个比试,在百里之内,我或许赶得上你的师兄,在百里之外,我是决比不过他白勺。''

归夫人道:``牟先生,你是长辈,他们功夫有不到之处,望你指点指点他们,不要助长他们的骄气。克邪,你应该叫牟先生做叔叔,不是大哥。''

段克邪道:``这是,这是牟大哥,嗯,牟叔叔要我这样叫他的。''他一路上叫惯了``大哥'',一时间改不过口来。

牟沧浪笑道:``我与令徒一见投缘,咱们各交各的,夫人,你不必拘执了。令徒是天生的学武资质,我结识了这位小兄弟,高兴得很呢!''

段克邪道:``这位牟大哥很好玩,他还会魔术呢!''归夫人笑道:``哦,他教会了你什么把戏?''

段克邪道:``不是耍把戏,我和他玩打手掌的游戏,他在我的掌心拍了几下,我便全身发热起来,但却舒服得很。过后,他叫我跳上一棵树上捉雀儿,那棵树很高,鸟巢在树顶,我说我一定跳不上去的,爬上去我就会。他说:你放大胆子试一试吧。我一跳,奇怪,果然跳上去了,可惜捉不到雀儿,只掏了两个雀蛋。''

归夫人又惊又喜,笑道:``克邪,还不赶快谢牟先生,他已给你打通了窍阴玄关,你这一生受益不尽。''原来若要修上乘内功,就必须打通窍阴玄关。归夫人这一派的武功虽然厉害,但所学的却不是正宗的全功心法,要打通窍阴玄关,最少得有-
卜年以上的功力。如今牟沧浪以师门秘法、无上玄功给段克邪打通了窍阴玄关,以后段克邪修习上乘内功,就可事半功倍。

段克邪哪里知道其中关系,听了师父的吩咐,依言便给牟沧浪叩头,牟沧浪哈哈笑道:``小兄弟,做哥哥的没有什么更好的见面礼给你,正自惭愧呢。过几年你长大了我再来看你。''

牟沧浪走后,众人都向段圭璋夫妇祝贺,一贺他们骨肉团圆,二贺他的儿子得此奇遇,前途无限。归夫人笑道:``这孩子的武功虽未大成,但此去江湖,差不多的也尽可应付了。''这话语即是允许段圭璋携他回去。段圭璋欢喜无限,再次向归夫人拜谢。

众人在玉皇观住宿一宵,第二天一早,便向归夫人告别。归夫人亲自送了一程,疼了孩子几回,这才挥泪而别。

段圭璋等人归心似箭,兼程赶路,不消一个月,就进了玉门关。这几个月来,他们久已不闻战汛,到了玉门关后才知道一点前方的军情。

他们听到的消息是:安禄山虽然被儿子所弑,但史思明继起,贼势仍很猖獗,目下正分兵三路,一路攻掠河北诸邵,指向灵武;一路攻打睢阳;一路留在范阳平卢境内,扫荡后方的义军。幸在郭子仪的新军已经练成,听说也已分兵两路去救灵武和睢阳了。

他们得到了这些消息,便在路上商议。铁摩勒问道:``金鸡岭是义军总寨,可不知南师兄还在金鸡岭么?''韩湛道:``我离开金鸡岭的时候,南大侠已奉郭子仪之令,回转睢阳,帮张巡守城去了。''铁摩勒心中稍宽,说道:``张巡乃当代将才,又与郭子仪互相呼应,想可无虑。''韩湛道:``我与辛寨主有约,要去金鸡岭助他一臂之力。现在看来,三路之中,其他两路都有外援,却是金鸡岭的形势最危,摩勒,你和我一道吧,先助义军突围,若是睢阳危急,再救睢阳。''铁摩勒虽然挂念师兄,但权衡缓急,而且韩湛的策划也正是兼顾两方,便依了岳父之议。韩湛又道:``段大侠,你是薛嵩、聂锋两家的救命恩人,他们既在朔方,你还是以到朔方为是。一来可以劝说他们二人出兵,二来也可了你的私事。''当下,议计已定,韩湛父女翁婿一路,便与段圭璋夫妻分手。

段圭璋心急如焚,兼程赶路,可是从玉门关到朔方,还有三千多里,路途又不好走,他们只凭着两条腿,走了将近一个月,方始踏进临淮境内。该地距离朔方六百余里,离睢阳却只是三百里左右。

时节将近中秋,天气仍很炎热,这一日他们冒着骄阳,脚步仍是不敢稍缓。他们连日奔波,窦线娘走了半天,已有点气喘,反而是段克邪这孩子精神最好,经常走在父母前头。窦线娘大为欣慰,忍不着夸奖她的儿子,段克邪笑道:``我算得什么,我的师兄才厉害呢,据说他可以日行千里。我的师父总希望我超过师兄,但看来在轻功上我是绝没办法超过他了。''

走了一程,段克邪忽地问道:``爹,这些天来,我常常听你说南大侠的故事,说当世只有他才不愧大侠二字。现在到了此地,既然离睢阳较近,为什么不先去看看他,却要这样着急赶到朝方作甚?''段圭璋心中一动,想道:``这孩子说的也有道理。''窦线娘却笑道:``孩子,你不知道,咱们赶往朔方,有一大半是为了你的缘故!''

段克邪道:``怎么是为了我的缘故?''窦线娘笑道:``我带你去会一位小朋友,她是个又聪明又漂亮的小姑娘,你见了她,一定欢喜她的。''段克邪问道:``她懂得武艺么?''奏综娘道:``她是妙慧神尼的徒弟,不但会舞刀弄剑,还会弹琴念书,懂得的东西比你还多呢。''段克邪从未有过年龄相若的朋友,听了十分高兴,但又有点担心,说道:``妈,你说她这样好那样又好,那你怎知她肯不肯和我交朋友?''窦线娘笑道:``这,你就不用担心了,她不但会和你做朋友,而且一生一世她邢不会与你分开。''段克邪莫名其妙,眨眨眼睛问道:``为什么?''段卜璋道:``孩子,你现在还小,说给你听也不懂。再过两年,你就知道她是你的什么人了。''段克邪对父亲较为畏惧,不敢冉缠问下去。但仍是高高兴兴地说道:``好,她既然也会武功,那么咱们到了朔方,就邀她一同去见南大侠,给南大侠打退那些贼人。''

段圭璋听得儿子这么说,既是高兴,又是不安,心中想道:``好几天没听到睢阳的消息了,不知南兄弟现在如何?''走了一会,路边有家卖些酒食的茶铺,段圭璋想听听消息,便叫住了儿子道:``你妈有点累了,咱们且歇一会儿。''

隔座有两个军官模样的人,段圭璋刚踏进茶铺,便听得其中有个说道:``唇亡齿寒,这点道理,咱们都懂,贺兰元帅却怎的拥兵不发?''另一个道:``还有更气人的呢,唉,大哥,咱们职位太小,说也没用,还是喝酒吧。''

段圭璋心中一动,正想过去搭话,忽听得有个客人将筷子一摔,叫道:``你们卖的是什么猪肉,好大的一股味儿,敢情是发了瘟的?''跑堂的连忙过来打拱躬揖道:``你大爷包涵点,这猪肉只是隔夜的,并不是猪瘟,味儿还不致太难闻吧!''那客人道:``还说不难闻,简直吃不下去!''瞧他的模样,似是个公子哥儿。

旁边有个客人忽地冷笑道:``隔夜的猪肉总胜过老鼠肉吧?可怜睢阳的将士现在什冬东西都没得吃了,听说连城中的老鼠和麻雀都吃光了。''

茶铺里人听他提起睢阳,都围拢过来,有人间道:``听说张巡连爱妾都杀了,给军士吃,这是真的么?''那人道:``这倒是传闻失实了,那个姬人是因见城中缺粮,自尽死的。为的是给张巡省下一份口粮。''又一个人间道:``不是听说郭令公已派了大军来救么?''那人道:``郭令公是派了一支军队来,不幸半途中伏,伤亡甚重,这支军队人数不过几千,后援未到,难以支持,只好退兵了。''众人听了,无不顿足叹气,有人问道:``郭令公与张防御使是至交好友,于公于私,他都不该坐视,为何不亲自率军来援?''那人道:``这倒怪不得郭令公,贼兵有一路攻向灵武,听说皇上一日发出七道诏书,要他全军赴援灵武,前往睢阳那支军队,还是他私自从亲军和民兵里面拨出来的。''先前那人问道:``贼兵距离灵武还远,何以轻重倒置,缓紧不辨?''那人叹口气道:``你不知道当今皇上就在灵武吗?''众人面面相觑,不敢说话。过了半晌,有人低声说道:``听说睢阳已有人来本州讨取救兵,不知贺兰元帅可肯发兵?''

忽听得有人在茶铺外面接声说道:``这事儿么你不提也罢,提起了叫人气煞!请诸位听我唱一支《挂枝儿》(曲调名),说一说怎的啮指乞师师不发。''

只见一个衣裳槛楼似是走江湖唱道情的老叫化,不知什么时候来到了茶铺外边,他说了这几句``开场白'',便敲着竹筒道:

``进明啊,你也食唐家禄否?人望你拯灾危,飞骑到此来求救,谁知你坐拥强兵空袖手,不曾见你兴师去,倒要将他勇士留!可怜那南八好男儿,他十指儿只剩九。进明啊,你厚着脸皮不顾人唾骂,任他血泪交流不听他,你眼睁睁看了他将指头儿咬;他当时乞师空咬指。我今日所说亦咬牙!元帅将军难倚靠,保家园还得咱们小百姓想办法!''

段圭璋这一惊非同小可,跳起来道:``老丈,你说的那位南八可是张巡手下的将领南霁云么?''那老人道:``不是他还是谁?可怜他空白啮指乞师,贺兰元帅不但不发兵,反而连他山不放走!''

段圭璋隔座那个军官慌忙喊道:``老叫化,你怎可肆无忌惮,在这里骂贺兰元帅!''原来这唱辞里的``进明'',正是他的长官贺兰元帅的名字。此言一出,登时整个茶馆里面的客人都骚动起来,纷纷骂道:``他坐拥强兵,见死不救,不该骂吗?''``老人家,你说得对,元帅将军难倚靠,保家园还得咱们想办法!''``对呵!有血气的男儿都往睢阳去吧!''

人声鼎沸中,忽见一条人影箭一般的飞奔出去,正是段圭璋,他宝剑一挥,所断了系马的绳子,立即飞身上马,说时迟,那时快,窦线娘与她的儿子也接踵而来,飞身上了另一马匹。

那两个军官气得暴跳如雷,大声喝骂,原来这正是他们的坐骑。段圭璋在马背上朗声说道:``对不住,反正你们不去打仗,这两匹坐骑,我们却正用得着。你们若要索回马匹,到睢阳来吧!''茶客们哄堂大笑,都道:``这壮土说得对,当兵的不打仗,还不让小民去打么?好壮士,你先走一步,咱们也会来的!''笑声中,段圭璋这对夫妻早已去得远了。

窦线娘催马追上丈夫,叫道:``圭璋,咱们这就往睢阳么?''段圭璋道:``怎么?敢情你不愿意?你不记得当年南兄弟是怎样舍了性命护送咱们么?''窦线娘道:``正是为'了要报他这大恩,所以我才问你啊,你刚才不听得那老人家说吗?据他说贺兰进明不但不发兵,反把南兄弟扣留了。那么咱们是不是应该先到城里把南兄弟救出来?''

段圭璋怔了一怔,心道:``这倒是一个难题。''要知睢阳已是危在旦夕,若去救人,倘然受了挫折的话,岂非耽误大事。但若不把南霁云先救出来,他又放心不下。

正在踌躇,不知不觉已到了一处三岔路口,有两个军官骑着马迎面而来,神色惊惶,跑得甚急,段圭璋心中一动,想道:``这条路正是从睢阳来的,莫非又有了什么紧急的军情?''

心念未已,忽听得一声马嘶,另一条路上,又出现'了一骑骏马,来得有如风驰电掣,比那两个军官的坐骑快得多!

转眼之间,那匹骏马已追上了那两个军官,只见坐在马背上的是一个身材高大、神情凶恶的老人!只听得他一声喝道:``岂有此理,你们胆敢骗我,我问你有几个脑袋?''

话声未了,两匹坐骑已是首尾相衔,那个军官喝道:``你杀了我,我也不告诉你!''反手一刀,向那老人劈去!那老人哈哈大笑,一掌拍出,但听得``咣''的一声,军官已给他打下马来,那柄月牙弯刀也飞到半空去了!

那老人马不停蹄,眨眼之间又追上了另一个军官,笑声一收,蓦地喝道:``快说实话,姓南的往哪条路走,如有半句诳言,这人就是你的榜样!''

那两匹坐骑已是并辔而行,那老人正自一抓向那军官抓下,猛听得弓弦声响,窦线娘已发出了三颗金丸,那老头好不厉害,把手一抄,把窦线娘所发的金丸全都接了。

但听得``蓬''的一声,马嘶人叫,那军官已滚下路边的稻田,原来是那老人一掌将军官的坐骑击毙了。他人未离鞍,竟然在这瞬息之间,左手接暗器,右掌毙奔马。段圭璋见他如此厉害,也不禁暗暗吃惊。

说时迟,那时快,这老人已纵马过来,冷冷说道:``原来是窦家的大小姐来了,承赐金丸,敬谢壁还!''反手将三颗金丸打出,听那锐啸破空之声,劲道比窦线娘更大。

段克邪忽道:``妈,我替你打这老贼!''陡然间从马背上飞身跃起,迳向那老人的马上扑去!窦线娘这一惊非同小可,慌忙叫道:``克儿,回来!''

段克邪身形一起,如箭离弦,哪止得住?只听得叮叮几声,他在半空中已拔出一柄短剑,将那老人打回来的三颗金丸磕落,连人带剑,化成了一道银光!

藏灵子这门的轻功冠绝武林,段克邪虽未练到他师兄空空儿那样的本领,但以他这样的年纪,已是足以惊世骇俗!

那老人赞道:``小娃儿,好俊的身手,你是空空儿的什么人?''这老人武学深湛,见多识广,段克邪的轻功一露,他已看出路数,心里不由得暗自沉吟:``我不怕得罪他的父母,但要是惹恼了空空儿,却是麻烦!''段克邪道:``你管我是谁,我只知道你是个坏人,我就要打你!''声到人到,在半空中一个筋斗,头下脚上,便即凌空刻下,剑尖直指那老人的太阳穴!那老人焉能给他刺中,中指一弹,把段克邪的短剑弹开,左臂一圈,便要把段克邪拖下来!但终是因为顾忌空空儿,未敢使出他的追魂神掌。

段克邪的短剑给他一弹,手腕隐隐作痛,也不由得心中一凛,百忙中使出师傅的轻功绝技,便借他这一弹之力,又在半空中翻了一个筋斗,但这一次却是向后倒翻。

那老人这一弹没有将他的短剑弹出手去,也是颇出意外,当下又是惊奇,又有点爱惜,他的坐骑乃是惯经战阵的良驹,不待主人指挥,便向段克邪冲去。段克邪在半空中一个筋斗翻下来,身形刚刚落地,那老人连人带马已是冲到,眼看他就要伤在马蹄之下。

猛听得一声喝道:``老贼,休得伤害我儿!''但见剑光一闪,段圭璋飞骑赶至!这老人见他剑势凌厉,不敢轻敌,拨开马头,迅即一掌劈出。

段圭璋剑尖一颤,趁势抖起了一朵剑花,一招``李广射石'',向前疾刺,这时他们的坐骑已是擦身而过,那老人一个``镫里藏身'',双足倒挂马鞍,左臂一伸,半边身子悬空,居然使出了极厉害的擒拿手法,要把段圭璋拖下马来。幸而段圭璋骑术剑术两皆精妙,左拿一拍马鞍,在马背上施展出``铁板桥''的功夫,以单臂作为支柱,整个身子在马背上腾空三尺,剑锋一转,一招``顺水推舟'',平削出去。

但听得``砰''的一声,那老人一掌击中了段圭璋的马腹,那匹马滚下斜坡,将段圭璋抛出了数丈开外!

那老人只觉头皮上一片沁凉,段圭璋这一剑刚好从他的头顶削过,一蓬乱发已是随着剑光纷落。那老人也不由得大吃一惊:``这姓段的剑法果然名不虚传,他们夫妇联手,我是决难取胜的了!''当下哈哈笑道:``姓段的,你站稳了,咱们在睢阳城下,再见个高低吧。''快马加鞭,转眼之间,走得无踪无影。

窦线娘慌忙向她丈夫奔去,段圭璋一个``鲤鱼打挺'',翻起身来,只见自己那匹坐骑已是颈折腿断,瘫作一团,不禁咋舌道:``好厉害,幸亏没有给他打着,这老贼是谁?''窦线娘道:``这老贼乃是安禄山的大内总管------七步追魂羊牧劳。''原来羊牧劳以前在黑道上混的时候,也曾到过窦家的飞虎寨,故此窦线娘认得是他。

段圭璋道:``原来是他,哎呀,不好!''窦线娘道:``怎么?''段圭璋道:``你刚才不曾听得他向那军官盘问么,敢情他就是去捉捕南兄弟的?''窦线娘道:``这里有两条路都可通睢阳,不知南兄弟走的哪条?''

忽听得呻吟之声,原来是滚落稻田的那个军官已爬了起来,嘶声叫道:``尊驾可是段大侠段圭璋么?''

段圭璋道:``不错,大侠之名,愧不敢当。足下是谁?却为何与这老魔头作对?''

那军官一看,他的同伴连人带马已倒毙路旁,忽地哀号三声,又大笑三声,哭声笑声部颤抖得很厉害,显见是受了内伤。

段圭璋怔了一怔,忙道:``你躺下来,我给你敷药。''那军官道:``你不要为我耽搁了,听我把这事情告诉你,然后赶快去与南义士会合吧。他就在前头!''段圭璋道:``你说的是南霁云?''

那军官道:``不错。我们是贺兰进明的亲军统领,奉命去追南义士的。我们怎忍害他,所以矫将令,亲自送了南义土过关。''

那军官声音微弱,继续说道:``不料在回来的路上,遇到了这个魔头,他露出绵掌碎石的功夫,迫我们说出南将军的去向。我们情知不是他的对手,只好胡乱指一条路给他,哪知他马快如风,去而复回,我们还是难逃毒手!''

段圭璋听了,肃然起敬,连忙说道:``你救了南将军,南将军他绝不忍你为他送命。''一面说话,一面掏出了金疮散来,那军官忽道:``你可知道我刚才为什么大哭三声,大笑三声?''段洼障怔了一怔,道:``正要请教。''那军官一手扫开他的药散,说道:``我是为我的兄弟丧命而号陶,为段大侠你来了而欢笑,有你到来,南将军就不至于孤掌难鸣了。南将军是从左边这条路走的,你赶快去吧。''说到一个``去''字,突然俯下头颅,向地上一块石头一撞,登时血如泉涌,随即倒在血泊之中。原来他自知伤重难治,不想耽搁段圭璋的功夫,故此不惜轻生。

段圭璋料不到他竟然如此壮烈牺牲,要拦阻已来不及,急忙问道:``你有什么身后之事,可要段某料理么?''并且将耳朵凑近他的嘴边,只听得他断断续续地说道:``只盼你转告南将军,请他多杀几个贼人!''说到最后那儿个字,段圭璋已经听得很费力,用力一抗,那军官的心脏已停止跳动了。

段圭璋虎目蕴泪,呆了片刻,向他的尸体拜了一拜,说道:``真是义士,令人感奋!可惜我连你的名字都未知道。''

窦线娘道:``咱们不可辜负了他的期望,赶快走吧!''段圭璋和那两个军官的坐骑都已给羊牧劳击毙,只剩下窦线娘这匹马。段克邪道:``爹,你和妈合乘一骑,看我能否赶上?''段圭璋知他轻功了得,说道:``也好,就让你和这匹马赛赛脚力。''

段圭璋飞身上马,问道:``刚才那老魔头向哪条路走?''窦线娘道:``他又走错了,他向中间那条路去了。''段圭璋道:``好,那么咱们快马加鞭,也许可以在他发现错误之前,赶上南兄弟。''但他们那匹马只是一匹寻常的军马,背上了两个人,虽然用力鞭打,也跑得不怎么快。段克邪施展出``八步赶蝉''的轻功,那匹马竟然赶他不上,还要段克邪放慢脚步来等它。

幸好这条小路乃是捷径,大约半个时辰,就过了临淮州界。正在催马急行之际,忽听得前面有厮杀之声!正是:

自古救兵如救火,飞骑杀敌到唯阳。

欲知后事如何?清听下回分解------

旧雨楼扫描,海之子OCR,旧雨楼独家连载Z

\chapter{第三十九回 何愁强虏侵中土
尚有将军树汉旌}\label{ux7b2cux4e09ux5341ux4e5dux56de-ux4f55ux6101ux5f3aux864fux4fb5ux4e2dux571f-ux5c1aux6709ux5c06ux519bux6811ux6c49ux65cc}

远远望去,只见有一群人在前面的山脚下厮杀,窦线娘自小练习暗器,目力极佳,吃了一惊,说道:``不好,是南兄弟被贼人包围了!王家那小贼种也在其内。''这时,双方的距离又接近了许多,段圭璋也已看得清楚,那群贼人大约有十来个,正是王龙客指挥,向南霁云猛烈攻击!

段圭璋提高了声音喊道:``南兄弟,我来了!''话犹未了,忽听得羊牧劳的声音哈哈笑道:``你来了正好,咱们可以不必等到睢阳城下再见高低了。''声音从后面传来,震得耳鼓嗡嗡作响,只闻其声,不见其人,段圭璋怔了一怔,回头一望,但见远远一个黑点,转眼之间,那黑点扩大了十倍,羊牧劳那一人一骑已出现在路上,当真是来得有如电掣风驰,迅速之极!

窦线娘笑道:``克儿,你看我把他打下马来!''在马背上一个转身,弓弦一拽,羊牧劳眼观四面,耳听八方,听得弓弦声响,便一记劈空掌发了出去,哪料窦线娘只是虚张声势,并未发出金丸。窦线娘连拉了三次弓弦,羊牧营也连劈了三掌,都不见有弹丸飞来,羊牧劳大笑道:``你弄甚么玄虚,谅你米粒之珠,岂能与日月争光?''那匹马来得甚近了。

哪知话声未了,窦豫娘第四次拉起弓弦,修然间七颗金丸,连珠发出,羊牧劳正在换掌发招,猛地浑身一震,那匹马突然将他抛了起来,原来窦线娘知道要打中羊牧劳极不容易,把那七颗金丸,有两颗却是打他那匹座骑的双眼,而且是用了后发先至的巧劲。羊牧劳武功深湛,善能听风辨器,但想不到窦线娘的弹弓如此出神人化,他``听得''那弹丸是朝着自己上身的五处穴道打来,忙于保护自己,冷不防她最后的两颗弹丸后发先至,有如迅雷不及掩耳,一下子就把他那匹黑龙驹的双眼打瞎了。这正合上了``射人先射马,擒贼先擒王''那句古话。

羊牧劳一个筋斗从马背上翻下来,窦线娘道:``圭璋,你去对付他,我去救南兄弟。''段圭璋应了一声``好'',立即便从马背上飞身掠起,人在空中,剑已出鞘,一招``鹰击长空'',便向羊牧劳凌空刺下!

羊牧劳好不厉害,他身形未稳,已是掌指兼施,用了一个以逸待劳之势,要从险中求胜!

他这一掌用的是小天星掌力,要把段圭璋的宝剑牵引过一旁,然后迅即指戳他的脉门,强夺他的宝剑。段圭璋身子悬空,双足未曾踏着实地,本来很难躲避他这以逸待劳的一击,但段圭璋乃是身经百战之人,岂能受他暗算?他在凌空下击之时,早已算准羊牧劳有这一招杀手。就在这危机瞬息之间,他也显出了卓绝非凡的本领。

只见他长剑一圈,忽地中途变招,身形一弓,双足互碰,就在半空中一个侧翻,剑招也从``鹰击长空''而变为``鱼翔浅底'',唰的一剑,抖起了一道长虹,向羊牧劳的腰胁刺去。羊牧劳喝声:``好剑法!''声出形移,方位立变,双掌交叉劈出,解开了段圭璋这招杀手。段圭璋脚尖刚刚着地,立足未稳,不敢立即进招,双方都向后退开了两步。

段克邪叫道:``爹,我来帮你!''声发人到,竟然抢在父亲的前面,短剑迳指到了羊牧劳的胸前,段圭璋忙道:``克儿,你去助你的妈吧。''段克邪道:``不,我吃了老贼的亏,非得出了这口气不可!''就在说两句话的时间,他已接连攻出了七剑,剑剑都是指向羊牧劳的要害穴道!

段克邪的功力当然不如父亲,但他的身法却比父亲更为迅速灵活,而且他已尽得师门袁公剑法的秘传,虽然还未能练到空空儿那般境界,可以在一招之内,连袭敌人九处穴道,但已可以似精精儿那样,在一招之内,刺敌人的七处穴道了。要是给他剑招刺实,即使羊牧劳有金钟罩的功夫,穴道被刺,也难免要受重伤。

羊牧劳喝道:``好狠的小娃儿!''这时他已不能再有顾忌,心想:``纵算他与空空儿乃是同门,也只能杀了他再算了。''杀机一起,立即也使出了七步追魂的绝技,脚踏五门八卦方位,掌发步移,一掌紧似一掌!

哪知段克邪聪明之极,他刚才吃过一次亏之后,已深知敌人功力高出自己不止十倍,哪里会与他硬碰,只是仗着独门轻功,与他游身缠斗。羊牧劳的掌力虽然厉害,却打不中他,才发到第三掌,段圭璋亦已飞身掠至,父子合力,与羊牧劳展开了一场恶斗。

段圭璋看了儿子的身法,稍稍放心,知道有了自己助阵,羊牧劳要想伤他的儿子,也不容易。同时心里又有点奇怪,``羊牧劳的七步追魂掌确是高明,但却也不如武林前辈所说的那样厉害!''

段圭璋有所不知,羊牧劳曾被韩湛以天魔指的绝技伤了三焦经脉,至今功力尚未完全恢复,因此在他们父子联手合斗之下,便走了下风。

斗到紧处,段克邪忽地喝一声``着!''羊牧劳听得背后金刃劈风之发,反手便是一掌。哪知就在这瞬息之间,段克邪忽地一个筋斗,从他头顶上翻过来,饶是羊牧劳身经百战,也未曾见过这等怪异的身法,而且也绝对料想不到这``小娃儿''竟然如此大胆。待到他心中一凛,收掌回来抓段克邪的时候,段克邪的短剑已刺进他的眼眶,一颗眼珠,随着剑光飞去。

羊牧劳似受伤的猛兽一般,猛地一声怒吼,双掌推出,段克邪被他的掌风一震,在半空中连翻了三个筋斗,跌落三丈之外。段圭璋怕他追上去伤害儿子,宝剑一展,化成了一道长虹,拦住了他的去路!

哪知羊牧劳却不向前进,他一掌发出,便即转身,厉声喝道:``好,这一笔帐暂且记下,羊某倘有三寸气在,誓报此仇,一颗眼珠,要换你们父子二人性命!''说到最后这句,已跑出半里之遥。原来羊牧劳尽管十分愤怒,但却绝非不自量力的鲁莽之徒,他深知受伤之后,再拼下去,只有吃更大的亏,故此扔下了几句``门面话'',便即慌忙逃命。

段圭璋惦记着儿子,当然不会去追赶敌人,他回过头来,只见段克邪已笑嘻嘻地站在他的前面,说道:``爹,我把那老贼变成了独眼龙了。''段圭璋见他未曾受伤,这才放心。说道:``克儿,你也忒大胆了。''段克邪笑道:``我不剜掉他的眼珠,怎出得这口气?''段圭璋本来还想教训他几句的,见他如此高兴,也就不忍再说了。

父子二人赶上前去,只见窦线娘弹如雨下,已把那群强盗打得七零八落,只有王龙客与阴阳刀石家兄弟还在与南霁云苦斗,但他们既要闪避弹丸,又要应付南霁云那刚猛绝伦的刀法,也已显得有点手忙脚乱。

段圭璋剑眉一竖,朗声说道:``王世兄,你还记得令尊临终的吩咐吗?岂可仍然助纣为虐!''王龙客冷冷说道:``我姓王的事情不必你姓段的多管,你走你的阳关道,我走我的独木桥,你要是看不顺眼,咱们在睢阳城下,再决个雌雄。''窦线娘大怒道:``你这小贼真是不到黄河心不死,不见棺材泪不流!''弹弓一拽,嗖、嗖、嗖三颗金丸,都对准了他的脑袋打去。

段圭璋连忙发出一记劈空掌,说道:``冤家宜解不宜结,线妹你就再饶他一次吧!''那三颗弹丸刚刚射出,被段圭璋的掌风一荡,失了准头,在王龙客的身边落下。

王龙客与石氏兄弟见他们到来,情知绝难对敌,一声呼啸,分开了三路逃走,段圭璋止住儿子,不准他去追赶,王龙客边走边喊道:``姓南的,姓段的,咱们的仇是结定了。要解此仇,今生休想!有胆的睢阳城下再见。''原来王龙客野心甚大,他一来是想在战乱中混水摸鱼,待到羽毛丰满,便割地称王,最不济也要继任绿林盟主。二来他妒忌南霁云得到了夏凌霜,故此发誓要与南霁云作对。三来他始终认定王、窦两家乃是世仇,段圭璋夫妇是他要继任绿林盟主的大碍。由于妒忌、偏见与利欲薰心,他把父亲的临终遗嘱抛诸脑后;把父亲的自杀与那番吩咐看成是被仇家所迫,不得不然。

窦线娘摇了摇头,愤然说道:``真是朽木不可雕,贼性终难改,圭璋,你也未免太厚道了。''段圭璋笑道:``今日得与南兄弟重逢,这是天大的喜事,那小贼就由他去吧。''

南霁云哈哈大笑道:``段大哥,我望你来有如大旱之望云霓,睢阳危城,正要你们相助。这位小英雄是------''段圭璋笑道:``克儿,你不是渴望见南叔叔么?还不快上去见礼。''南霁云这才知道是段圭璋的儿子,惊奇不已,说道:``当真是长江后浪推前浪,世上新人换旧人。段大哥,我看侄儿将来一定要比你还强得多!''

段圭璋一看,南霁云的左手果然缺了一个指头,南霁云笑道:``大哥,你道小弟这指头是怎样斫去的?唉------''段圭璋道:``你的事情我已经知道了。我们遇到的事情你却还未知道。南兄弟,你别心焦,贺兰进明不肯发兵这不打紧,老百姓会给你发兵!''当下将茶馆众人的议论与那两个军官壮烈牺牲等等事情都对南霁云说了,南霁云听得热泪盈眶,望空遥拜道:``两位义士为南某而死,南某若敢有违嘱咐,有如此树!''一刀劈下,将一棵树齐腰斩断。

围攻睢阳的是史思明手下的大将令狐潮,这时已进入了令狐潮的防地,幸亏南霁云熟悉地理,在前引路,翻过一座山头,抄小径直奔睢阳。

第二日中午,已到睢阳城外,他们隐藏在离城五六里外的一个土屋,只见甲帐连云,旌旗招展,人马奔腾,鼓角喧天,南霁云道:``不好,贼兵正在加紧攻城!''

段圭璋笑道:``咱们来得正是时候,好得很啊!''南霁云道:``不如由小弟先杀进城去,再领一支军队出来接应你们。''段圭璋大笑道:``南兄弟,你是响当当的汉子,段某也不是贪生畏死之人,我知道你是想保全我们,你的好意我心领了!''``唰''的一声,宝剑出鞘,先冲了下去。

贼兵见他们人少,哪里放在眼内,有个军官模样的人,骑着一匹高头大马,歪着眼睛喝道:``哪里来的?擅闯大营,还要命么?''话犹未了,忽地一个倒栽从马背上跌下来,原来给窦线娘一颗弹丸,就打碎了他的天灵盖。

段圭璋大喝道:``挡我者死,让我者生!''宝剑一挥,但听得一片断金戛玉之声,迎面挑来的几柄长矛都已给他削断!转瞬之间,南霁云亦已挥刀杀到,当真是有如两只猛虎下山,挡者辟易。

贼兵中有许多人认得南霁云,奔走骇叫道:``刁;好,是南八回来了!''要知日前南霁云曾单骑突围,杀伤敌军官兵数百,故此贼兵见他回来,先已怯了。

不消片刻,他们已冲过三座营地,忽见前面的敌人两边分开,一队骑兵从中间杀出,来得有如暴风骤雨,窦线娘一轮弹弓打去,但听得叮叮之声,不绝于耳,原来那队骑兵,连人带马,都披着厚甲,要把他们踏成肉泥。

段克邪叫道:``爹,我给你开路!''骑兵未到,他先迎了上去,只见他在马群之中,宛如蝴蝶穿花,挥剑专斩马脚,他那口短剑虽然比不上他父亲那口宝剑,也非凡品。他人既矮小,身法又极轻灵,短剑一起,便是一条马腿随剑而落,快得难以形容,那队骑兵共是三十六骑,距离段圭璋还有一箭之地,便已给他斩了十三条马腿,那些骑士跌下马来,因为身上披着重甲,想爬起来也不容易,反而做成了同伴的障碍。

杀散了这队骑兵,已到了敌人心腹之地,四面八方,密麻麻的都是枪林刀阵,到了此时,披甲的骑兵固然不能冲锋,但段圭璋等人陷入了重围,要杀出去也不容易了。

南、段二人,挥剑抡刀,正自奋力冲杀,忽听得羊牧劳的声音哈哈大笑道:``你们来得好快啊,羊某在此恭候了!''与他同来的还有敌军的副帅,以前安禄山帐下四大高手之一的张忠志。张忠志也在大声喝道:``南霁云,前日给你侥幸逃生,今日你可是自投罗网了!''

南霁云舌绽春雷,一声喝道:``今日不是你死,便是我亡!''一招``刀劈华山'',抢刀狂斩,羊牧劳一个``游龙探爪'',左掌托对方的肘尖,右掌从肘底穿出,便要施展大擒拿手法,扣南霁云的腕脉。哪知南霁云的内外功夫,都已练到炉火纯青之境,他用了一个``重身法'',双足一顿,兀立如山。羊牧劳的功力与他在伯仆之,间,这一拨竟然拔他不动,南霁云趁势一个肘锤,解开了羊牧劳的擒拿手,刀锋一转,唰唰唰一连数刀,狂风暴雨般的向羊牧劳扫去。

南霁云的``三十六式快刀''迅捷如风,沉猛如雷,羊牧劳也不由得心中微凛,他被南霁云占了先机,杀手难以施展,迫得脚踏九宫八卦方位,步步后退。

段克邪身形一起,游鱼般的从人丛中滑过,嘻嘻笑道:``老贼,你不怕再瞎一只眼睛吗?''羊牧劳怒道:``小娃儿,我要你的命!''双腿连环踢出,他掌敌南霁云,脚踢段克邪,当真是浑身上下,处处都见功夫。段克邪来得太快,收势不及,险险给他踢中,幸亏南霁云快刀斩下,向他的下盘连劈三刀,这才阻遏了羊牧劳连环腿的攻势。

要知段克邪上次之所以能伤了羊牧劳一目,全是凭着他超妙的轻功,且有父亲宝剑的助力之故,而今南霁云虽然不弱于段圭璋,但在千军万马之中,不比空旷之地,段克邪的轻功,却是难以施展,因此他对羊牧劳的威胁便大大减轻了。

段圭璋挥舞宝剑,方圆丈许之地,泼水不进,窦线娘仗着他挡住敌人,立即便杀上前,使出``金弓十八打''的家传绝技,猛攻羊牧劳。

羊牧劳力敌南、段二人,已感艰难,那禁得起又来了一只雌老虎。只听得``铮''的一声,弓弦声响,羊牧劳的衣服已被弓弦割破。南霁云大喝一声,一刀劈下,羊牧劳不敢恋战,跳出了圈子,大笑道:``南八,你要拼命,老夫恕不奉陪,反正你们是拼命也死,不拼命也死的了!''

中军是精锐所在,在羊牧劳压阵之下,段圭璋等人杀退了一重,还有一重,而且羊牧劳也并非束手旁观,若有哪方阵脚摇动,他就上去抵挡一阵。他拿定了主意,要等到南、段等人精疲力竭之时,然后一鼓尽歼。

正在杀得天昏地暗,难分难解之际,忽听得另外一方,又是杀声震天,段圭璋举目遥观,只见敌阵的``帅''旗附近,阵脚大乱,似有一支奇兵,从天而降,突然从敌军元帅的金帐里杀出来!

只见一个旗牌官快马奔来,挥着令旗叫道:``大营被袭,羊总管,元帅请你回去保驾!''羊牧劳没法,只好接令。

羊牧劳一走,压力轻了许多,但仍有张忠志在一旁指挥,敌军重重围困,突围依然不易。

南霁云道:``咱们杀过去与他们会合。''段圭璋挥舞宝剑,专

削敌人的兵器,南霁云抡刀狂劈,杀出了一条血路,远远望去,只见在``帅''旗那方冲杀出来的只是一小股健儿,最多不过十来个人,不多一会,这十多个人相继伤亡,只剩下一个老者。这老者左手提着一个人头,右手挺着一柄长矛,长矛一起,便是一个敌兵给他挑上半空,骁勇非常,当者辟易。

南霁云叫道:',咦,这不是郭老前辈吗?''话声未了,只见羊牧劳已然赶至,大声喝道:``郭老头,你又不是唐朝的命官,何苦为张巡拼命,快放下兵器,我念在昔日交情,可以饶你不死!''

那老头大喝道:``反贼不知羞耻,看矛!''挥舞长矛,向羊牧劳疾冲过去,但听得咔嚓一声,羊牧劳闪开矛头,挥臂一格,那柄长矛登时断为两截。南霁云失声惊呼,可是就在这一瞬间,那老头已和身撞去,两人距离极近,而那老者的身法又快如闪电,只听得``蓬''的一声,两人已撞个正着!羊牧劳大叫一声,竟给那个老者撞翻,跃出了数丈开外,那老者身形一晃,闷哼一声,吐出I一大口鲜血。原来老人这一撞乃是他毕生功力之所聚,但他先已受了十几处伤,故此虽然将羊牧劳撞翻,而他自己则伤得更重。

段圭璋这时也已认出了那老者是谁,拼命冲杀过去,大声叫道:``郭老前辈,段某来了!''原来这个老头乃是前辈游侠郭从瑾,他的徒弟便是差不多与南、段二人齐名的冀鲁游侠------金剑青囊杜百英。郭从瑾年过七旬,自他的徒弟出道之后,他已在江湖上销声匿迹,是以年来名头反而不如徒弟的响亮。但成名的武林老一辈人物,都知道郭从瑾是外家功夫将近登峰造极的老英雄。

羊牧劳给他撞翻,跌断了两条肋骨,他不知道郭从碰比他伤得更厉害,心中不禁大吃一惊,暗自想道:``我只道这老头儿已年迈气衰,哪知他还有廉颇之勇。''眼见南、段二人又杀了过来,羊牧劳受伤之后,不敢迎敌,借口保护元帅,退人大营。

郭从瑾浑身浴血,提着半截蛇矛,犹自神威凛凛,敌军骁将见羊牧劳尚且败在他的手下,十个之中倒有九个着了慌,不敢向前。

南、段二人双双杀到,见郭从瑾伤得如此厉害,不禁暗暗吃惊,段圭璋向南霁云递了一个眼色,南霁云将身体掩护着郭从瑾,大声说道:``郭前辈,那羊老贼业已受了重伤,反正难逃一死,我看咱们不必忙着取他的首级了,还是先杀进睢阳去吧!''万马千军,人声鼎沸,但南霁云运足了中气说的这几句话,周围的敌军却是人人听得清楚。

敌人听来,只道他们是在争论何去何从,有好几个令狐潮的心腹将官,还当真害怕他们再度杀进帅帐去取羊牧劳的首级,赶忙回去保护令狐潮。

其实郭从瑾根本就没有开过口说一句话,原来他的伤已是极为严重,只是仗着一股精神震慑敌人而已。南、段两人生怕敌军之中有能人看得出来,故此替他虚张声势。

南霁云话声方落,段圭璋已一剑劈翻了一名校尉,夺过了他的长枪,说道:``郭老前辈,这杆枪还合用吗?''郭从瑾点了点头,接过开枪,就在南、段二人掩护之下冲杀出去。他仗着几十年精纯的功夫,目下虽然将近筋疲力竭,但普通的贼兵还是禁不起他的长枪一挑。

南霁云见郭从瑾始终提着那颗首级,不肯抛弃,颇为有点奇怪,但是时亦已无暇多问。

羊牧劳受伤,敌军去了一个主脑人物,但还有个张忠志以副帅身份指挥,因此尽管他们已杀出了一条血路,但闯过一重,还有一重,眼看离城不过半里之遥,但在这半里路上,敌军少说也有数万之众,人山人海,要闯到睢阳城下,谈何容易。要知南霁云上次突围,是在黑夜,现在却是白天,白天闯阵,艰难何止十倍?

越近睢阳城,那金鼓齐鸣之声,就越为震耳,原来前头的贼军正在加紧攻城,南霁云举目遥观,城头上的动态已隐约可见。

只见城楼前面站着一员大将,正是他的师弟雷万春。南霁云又惊又喜,高声叫道:``雷贤弟,是郭老英雄与段大侠和我来了!''

就在这时,但见万箭如蝗,纷纷向城楼射去,远远望去,已可看见雷万春的衣裳已给鲜血染红,似乎不止中了一箭,但他还是兀立如山,动也不动!

南霁云距离较远,看不真切,城墙下的贼军却是大为骇异,雷万春面上连中六矢,仍是挺然兀立,威若天神,贼军中有人议论道:``莫非又是个木人?''原来就在前两天晚上,张巡因为城中缺箭,遂命军土扎了草人千余,蒙以黑衣,乘夜缒下城去,贼兵惊疑,放箭乱射,遂得箭无数。次夜仍复以草人缒下,贼都大笑,不以为意,张巡乃选壮士五百,全身衣黑,迳劫贼营,杀伤甚众。有此两役,故此如今贼兵见零万春连中六箭,仍然动也不动,遂疑心他是个假人。正在议论之际,雷万春突然把箭拔下,血流满面,舌绽春雷,大声喝道:``贼子,还你一箭!''就在随从校尉手中抢过一把五石强弓,弓如霹雳,箭若流星,一箭射去,正中贼军神箭营统领尹子奇的左目,尹子奇厉叫一声,登时坠马。雷万春将箭全都拔下,大叫道:``是谁射我的,待我一一奉还!''其实只有尹子奇射他的那箭,因为尹子奇是贼军中第一神箭手,故此箭杆上刻有名字,另外的五支箭,根本就不知是谁射的。可是那些曾经放箭射过雷万春的人,见尹子奇落马,人人都被雷万春的神威所慑,仓卒间哪里还能够细心推究,听得零万春这么一喝,竟然纷纷逃避,阵脚大乱,雷万春趁势就杀出城来。后人有诗一首赞雷万春道:``草人错认是真,真人反疑为木;笑尔草木皆兵,羡他智勇俱足!''

南、段等人拼命冲杀,里外夹攻,将挡路的贼兵杀散,待到令狐潮亲自出来督师攻城,稳下阵脚------南、段等人早已与雷万春会合,退回城中去了。

雷万春无暇问候师兄,先来照料郭从瑾,郭从瑾忽地将那颗首级一掷,说道:``南大侠,你认得这贼子吗?''南霁云一看,失声叫道:``这是郭令公手下的贺昆!''郭从瑾道:``不,他是叛贼贺昆!''接着哈哈大笑道:``我有负摩勒之托,未得及时通报郭令公,现在手刃此贼,缴回人头,我死亦可无憾了!''笑声渐转微弱,南霁云急忙上前扶他,只觉他手足如冰,已经气绝了。

原来这贺昆乃是混入郭子仪军中的奸细,南霁云与铁摩勒早在九原的时候,就发现他形迹可疑。后来铁摩勒做了玄宗皇帝的侍卫,又曾在宇文通的私室里见过他,玄宗逃难西蜀,郭从瑾在中途迎驾,铁摩勒曾托他向郭子仪禀告此事,这些经过,段圭璋都曾听得铁摩勒说过。但郭从瑾之所以杀贺昆的原因,他们却直到郭从瑾死后,几方面一说,这才明白。

原来郭从瑾受了铁摩勒之托,虽然兼程赶路,无奈处处烽烟,路途阻塞,未曾到得九原谒见郭子仪。睢阳与灵武的两路战事已起,灵武是肃宗皇帝驻跸之地,郭子仪奉了金牌宣召,亲率大军赴援;睢阳一路,则由他麾下的大将刘彦率领,只因主力放在灵武,这一路人马,半是民兵,半是郭子仪本人的护军,七拼八凑而成,不过七八千人。其时贺昆在郭子仪军中已做到``千牛卫''之职,他向郭子仪请缨,愿以所部千人,随刘彦赴援灵武,郭子仪不疑有他,允予所请。

哪知贺昆包藏祸心,与贼兵暗通消息,中途设伏,里应外合,把刘彦这支援军,打得几乎全军覆没,贺昆也就投降了敌人。

郭从瑾赶到睢阳城外,得知贺昆叛变之事,深感有负铁摩

勒之托,遂率领他在沿途组合的义军好汉三十六人,杀人令狐潮的大营,亲自取了贺昆的首级,郭从瑾与那三十六名好汉也先后牺牲。

南、段二人听了雷万春所述,嗟叹不已,段圭璋翘起大拇指说道:``古人季布千金一诺,太史公为之立传,名传后世。而今郭老英雄不惜以身殉诺,报国除奸,又比季布强得多了。''但以军情紧急,只能默哀片刻,便将郭从瑾草草掩埋,留下标记,待太平之后,再来给他立墓。

当下南霁云引领段圭璋夫妇去谒见张巡,张巡已有三日三夜目不交睫,双目深陷,发如乱草,一个堂堂的副节度使兼睢阳太守,已是形销骨立,似野人一般。段圭璋见了,又是钦佩,又是难过。

张巡已知贺兰不肯发兵之事,他反而安慰南霁云道:``老百姓说得对,元帅将军难倚靠,保家园还得百姓想办法。如今据段大侠沿途所见,老百姓已到处自组义军,给咱们发兵了。只要民心不失,就强过千百个贺兰进明!''南霁云道:``只恐远水难救近火!''张巡仰天大笑道:``一城一池的得失算不了什么,即算张巡死了,睢阳失了,民心未失,便有千百个张巡继之而起,中华锦绣江山,胡虎岂能染指,你怕什么?''这番豪言壮语,说得南、段二人大为振奋,张巡又缓缓说道:``当然,睢阳若能不失,那就更好,这就要靠大家齐心合力。现在最紧要的事是你们先去歇息,千万要养好精神,才能杀贼。''南霁云道:``你也该歇息呵!''张巡道:``我自会料理自己,现在我叫你们歇息,这是将令!''

南段二人连日奔波,又经一场大厮杀,也的确是累得很了。当下只好依从张巡之言,由南霁云去安顿段圭璋父子夫妇。

南霁云的妻子夏凌霜听说段圭璋夫妇到来,抱了两个儿子,连忙出来迎接。段圭璋见这两个孩子一般高矮,一般模样,问果然是对双胞胎。窦线娘笑道:``疯丐卫越盼你有三个儿子,你现在果然不负他之所望。''原来窦线娘见夏凌霜的肚皮隆起,她是个有经验的人,一看就知道夏凌霜最少已有五个月的身孕。

夏凌霜笑道:``这话说得早了一点,肚皮里这个还不知是男是女呢。''又道:``我真不想这个时候有孕,为了肚皮里的这个孩子,我实在难过得很。''窦线娘道:``战乱期中怀孕,是不大方便,但也用不着难过呀。''夏凌霜道:``嫂子,你不知道,霁云为了我怀有孩子,他不许我上城助战,我眼见人人奋勇杀敌,日日都有伤亡,怎不难过呢?''段圭璋笑道:``留得青山在,哪怕没柴烧?将来你把这几个孩子都造就成国家的栋梁,更胜于今日去杀几个贼人呢。''夏凌霜又道:``还有,城里现在缺乏食粮,霁云在家的时候,生怕我吃不饱,把他的门粮匀给我。他出去请救兵的那些日子,张太守又特地叫人送大米,送肉类给我,说孕妇应该吃得好一点,我知道他自己也没得吃,你说我怎能咽得下?可是退回去又不成,张太守说这是命令。我只好暗地里送给受伤的将士。''

段圭璋听了,眉头深锁,夏凌霜道:``大哥,大嫂,你们这个时候到来,只怕也要累你们挨饥受苦了。''段圭璋苦笑道:``你以为我是怕挨饥吗?我的身体总比一般兵士好得多,就是不食几天,也还挺得住。我是见兵士们个个面有菜色,不禁忧虑。要是不能早日解围,士气虽然旺盛,没东西吃,这仗也是无法打下去的。''言念及此,大家都是忧心忡仲,只盼各路民军,早日来援。

可是一连过了几天,非但援军未到,敌军倒似乎越来越多了,攻城一天比一天猛烈,幸得张巡与士兵同甘共苦,上下一心,共守危城。敌人曾先后用过云梯、火箭、战车、巨木等工具攻城,都给守城的将士破了。可是城中所有可以吃得下的东西,甚至鼠雀野菜之类,也差不多吃光了。

这一晚,段圭璋战罢归来,正在屋子里发愁,段克邪兀自兴致勃勃的和他讲日间怎样打仗的情形,忽听得一个熟悉的声音笑道:``你们父子俩果然是在这儿!''段圭璋抬头一看,只见一条影子,翩如飞鸟倏的就从檐头飞下,正是空空儿。段克邪大喜叫道:``师兄,你怎么来了?''空空儿笑道:``我来看你饿坏了没有?''正是:

烽火危城喜讯绝,不辞千里探同门。

欲知空空儿何事前来,请看下回分解------

旧雨楼扫描,海之子OCR,旧雨楼独家连载

\chapter{第四十回 名城浴血留青史
大侠捐躯表赤心}\label{ux7b2cux56dbux5341ux56de-ux540dux57ceux6d74ux8840ux7559ux9752ux53f2-ux5927ux4fa0ux6350ux8eafux8868ux8d64ux5fc3}

段克邪老老实实地说道:``这几天都吃野菜,嘴里确是淡出鸟来,但也惯了。''空空儿大笑道:``小段,也真难为了你,师兄没什么好东西送给你,送你一只烧鸡吧。这是从令狐潮的厨房里偷来的。''段克邪接过那只烧鸡,馋涎欲滴,但他还是放了下来,说道:``多谢师兄,我留待南叔叔回来,大家同吃。''

空空儿道:``段大侠,你们坐困危城,可不是办法!''段圭璋道:``依你之见如何?''空空儿道:``我沿途所见,你们敌方的援军络绎不绝,目前睢阳城下,少说也有二十万之众。你们虽然也有几路民兵赶来,但最近的一路离睢阳也还有百里之遥。令狐潮在各处险隘,都已有兵把守,最少在十天八天之内,那几路民兵,绝难通过。依我看来,你们兵微将寡,外援难至,内乏粮草,不是我说句泄气的话,这睢阳城的失陷,只怕是在旦夕之间,段大侠,你纵有天大本领,也难挽狂澜,不如趁早走了吧!''

段圭璋怫然说道:``我也知道只手难挽狂澜,但数万军民,同困危城,我岂能独自偷生?要走也只能和大伙儿一同突围而走。''空空儿道:``我早已料到你会这样回答我的了,你们是侠义道,把忠勇义侠这几个字看得比性命都重要,我也不敢劝你了。但我只想向你求一件事情,请你让我把克邪带走了吧。''段圭璋道:``这个------''他看了儿子一眼,见他已消瘦了许多,一时间踌躇难决。

空空儿道:``我对你实说了吧,我这次下山,要做四件事情。其中两事是受了师母的嘱托,一件是将精精儿捉回去,还有一件就是来探望克邪师弟。我师母很疼他,绝不愿见他在危城中遭受玉石俱焚之难,他只是一个小孩子,想来你也不愿坚执要他学你的模样,小小的年纪,就捐躯为国吧?你放心,我将他带走,百万军中,我空空儿也敢夸口来去自如,绝损不了他一根毫发!''

段克邪忽道:``师兄,你说错了!``空空儿道:``怎么?''段克邪道:``我就是要学我爹爹的榜样,这几天来,我听得人人都夸赞我的爹爹,连带还夸赞了我,我昨日杀了几个贼人,下城之后,人人都来看我,个个翘起拇指赞道:`父是英雄儿好汉!'另外有几个逃亡的军士,却被大伙儿唾骂,倘若我随你走了,他们一定会说`父是英雄儿混蛋'。哎呀,我可不愿受别人唾骂!''

段圭璋双眉一轩,哈哈笑道:``好孩子,好志气!''接着对空空儿道:``我段某岂不疼自己的孩子,但我更愿他自小就是个识大义、明是非的人。你对他的好意我终生不忘,但我也只能让他听天由命了!''

空空儿叹口气道:``既然你们心意已决,人各有志,我也不便相强了。段大侠,咱们曾做过对头,我空空儿目空天下,但你却是我最佩服的人!这大侠二字,你的确是当之无愧!''段圭璋道:``我也只是求心之所安而已。克邪,你过来给师兄磕头,多谢你师父、师兄传艺之恩。''

段圭璋此举实是含有诀别之意,段克邪不懂,空空儿却是心知,当下热泪满眶,将段克邪扶了起来,说道:``师弟,是我该向你道谢,你虽然入门最晚,尚未成年,但一出师门,便已足令本门不朽,只可惜我还未有传人,不能和你一道了。''原来空空儿因为要传他师父的衣钵,他未曾收下徒弟,就得保全自己的性命,故此有此一言。段圭璋心道:``空空儿本是个邪正之间的人物,他如今能够有陪克邪赴难的念头,已经是非常难得了。''

空空儿又道:``我这次下山,除了师母嘱托的两事之外,我自己也有两件私事,一件是劝王龙客------''段圭璋道:``对了,你和他乃是世交,当年他父亲做绿林盟主就是靠你撑腰的,他如今误人歧途,你是该劝劝他才好。''空空儿道:``我已经劝过他了,无奈他执迷不悟,我也没有办法。不过,我昨晚偷进他的营中,与他相晤,却探听到一个消息。羊牧劳的两个结义兄弟马远行与牛不耕都来了,这两个人与羊牧劳当年号称`三孽畜',武功也大致相当,要是碰上了他们,你可得稍微当心。''段圭璋笑道:``我早巳把性命豁出去了,多来几个`孽畜'又怕他何来?''

空空儿又道:``另一件事是我有件东西要送给铁摩勒,你可知道他在何处?''段圭璋道:``他在金鸡岭,但金鸡岭山正受敌人包围,也许现在他们已经突围了。''空空儿道:``我去试试看,王伯通留下的遗物中有绿林盟主的符信,当时来不及交代,这本是窦家的东西,你的娘子想来已用不着,我看还是交给铁摩勒吧。你有什么话要我对铁摩勒说么?''段圭璋道:``我只想他做个顶天立地的男儿,绿林盟主么,做不做也罢。''

空空儿道:``好,我一定给你把话带到,但愿你们能平安度过,咱们后会有期。''身形一起,疾如飞鸟,转瞬间就消失在冥冥夜色之中。

空空儿走后,段圭璋忧心如焚,空空儿已把战场形势说得很清楚,各路民军俱都被阻,城中缺粮,的确是难以等待了。段圭璋心想,``空空儿劝我走当然不对,但他的话也有些道理,困守无益,是该劝张太守突围了。''这一晚他目不交睫,只待天明就要去见张巡。

哪知刚到黎明的时分,便听得轰的一声巨响,段圭璋大吃一惊,赶忙提了宝剑出来,只见满空火蛇飞舞,轰隆轰隆之声不绝于耳。一个旗牌官挥舞着令施,一面奔跑,一面叫道:``元帅有令,军民人等,各归所部,立即突围!''

原来贼兵在五更时分,趁着防御较弱的时候,加紧攻城,用发石机攻坍了南面的城墙,火箭也纷纷射人,城中已有多处起火。幸而张巡早有部署,不但士兵,连阖城民众,都已编成队伍,突围令下,虽未能井井有条,但也不至于太过慌乱。

段圭璋一打听,知道张巡现在东门,便即吩咐儿子道:``你去接你妈与南婶婶一家人出来,到东门会合。''

段圭璋赶到东门,只见南霁云与张巡的一队护军,拱护着-辆战车,拉车的四匹马都已披上了鞍甲,正要打开城门,杀出城去。车上坐着的正是张巡。

南霁云道:``可有见到凌霜么?''段圭璋道:``我已叫克邪去接她们了。''南霁云道:``好,现在也难以顾及他们了,咱们保护元帅突围吧。''

城门打开,两军立即短兵相接,南、段二人在前开路,杀得敌人人仰马翻,厮杀声与妇孺的哀号声混成一片。张巡热泪盈刀匡,传下令道:``快分兵去保护百姓,不要只顾着我。''

混战越来越剧烈,不过多时,突围的军民已被截成了数十段,几乎陷入了人各为战的境地。张巡两翼的军队也已被冲散,只有南、段二人,和那一小队护军,都是身经百战的勇士,正自紧紧地聚在张巡车驾周围,浴血死战。

剧战中只见又是一辆战车冲了出来,所到之处,敌兵纷纷闪路,原来这辆车中坐的是夏凌霜母子,窦线娘亲自驾车,她一把弹弓,弹无虚发,段克邪在战车前面纵跃如飞,见人斩人,见马斩马。贼军见这个小孩子如此厉害,大为惊异,以为是妖星下凡,竟然不敢惹他。

张巡双眉稍展,说道:``南将军,嫂子有孕,你回到她身边去吧。''南霁云虎目蕴泪,说道:``元帅如此厚恩,南某粉身碎骨,难以图报!请恕我这次违抗将令了。''他不待张巡再说一句话,便杀进了敌军之中。

原来城中马匹差不多都已杀尽充饥,只剩下十来匹军马,分配给三部战车,张巡一部,副帅许远一部,在西门突围,还有一部,张巡临时下令,给了夏凌霜,南霁云现在才知道。

但也正因为从围城中出来的只有三部战车,遂成为贼军众矢之的,激战中忽听得贼军齐声叫道:``许远已被活擒,张巡你还往哪里跑?''张巡抬眼望去,只见许远那部战车已四轮朝天,翻倒路旁,但人头拥挤,距离太远,却看不见许远,也不知被擒之说,是真是假?张巡悲愤交集,沉声说道:``今日是我尽忠报国的时候了,宁为玉碎,不为瓦全!''夺了侍卫的一支长矛,亲自出来,运矛如风,刺杀战车前面攀辕来攻的贼军。

南霁云一轮快刀,连斩十数名敌军骁将,攻击张巡这部战车的贼军,发一声喊,暂时后退,南霁云劝道:``主帅不宜徒逞血气之勇,请张公保重,务必要突出重围!''

忽见敌军的``帅''旗高举,几十部战车冲出阵来,贼军元帅令狐潮站在当中的一辆车上,两旁侍立旗牌官挥舞帅旗,大声喊道:``元帅有令,张巡若不投降,就把他这两部破车粉碎!''贼军的战车分成两队,登时如两股怒潮,分头卷去!

张巡大怒喝道:``令狐潮,你欺负妇孺,算什么好汉,张巡在此,敢来与我决一死战么?''他目睹众寡悬殊,情知突围无望,是以不理南霁云的劝说,抱了必死之心,要把敌军的主力引来,好让夏凌霜那部战车,得有机会突围。

张巡三日三夜目不交睫,每餐又只是吃个半饱,但这一喝,仍是声如洪钟,把那战车奔驰而来的轰轰发发之声都压了下去。令狐潮本来不知道那辆车上载的张巡,这一喝果然吸引了他的注意,但见两面``帅''旗,一齐向张巡这方挥动,敌军哪一个不想争功?几十部战车,十乘中有八九乘改了方向,向张巡冲来。

雷万春大怒道:``师兄,你在这儿护卫主帅,待我毁了他这几辆车子!''他背后插有十几枝尺许长的小标枪,手上挺着一杆重达六十四斤的虎头金枪,一声大喝,不待那些战车冲到,就先杀了上去!

只见他左手一扬,两技标枪疾射而出,第一辆车前面的两匹马给他的标枪搠翻,战车也立即翻倒。雷万春连发十四技标枪,枪无虚发,连毁了贼军七部战车。可是第八部战车已到了他身前,距离太近,标枪已不济事,雷万春舌绽春雷,大喝一声:``我与你拼了!''虎头枪一挑,但听得``轰隆''一声,那辆战车,竟给他挑了出数丈开外!

雷万春连挑三辆战车,气力不继,第十一辆战车冲来,他奋力一挑,战车是挑翻了,但他也一口鲜血喷了出来,仆地不起了。

令狐潮揭起车帘,站了出来,哈哈笑道:``张巡,螳臂岂足当车?我劝你还是归顺我主吧!念在昔日同窗之谊,我不但保你身家性命,还保你官升三级,永享荣华!''张巡怒道:``令狐叛贼,你世受国恩,不思图谋,为虎作伥,助纣为虐,还敢恣口雌黄,面颜劝降!我生前不能杀你,死为厉鬼,亦必啖你之肉!''令狐潮冷笑道:``识时务者为俊杰,何况唐朝待臣下素来寡义,你又何必为他卖命?好,你倘若还是执迷不悟,我只好成全你的志愿,让你死为厉鬼了!''原来令狐潮乃是玄宗的羽林军统领令狐达之兄,令狐达因与宇文通勾结造反,举事不成,被宇文通杀之灭口,其后令狐潮就投降了安禄山。

雷万春力毁十一辆战车,贼军几曾见过这等骁勇的虎将?他虽然力竭仆地,余威仍是骇人,随后来的几部战车不觉都勒住马僵,不敢横冲直闯;令狐潮的帅旗急忙挥动,那些战车,无奈只好向前。

但也就在这个时候,南霁云亦已飞奔来到,含泪说道:``师弟,你先走一步吧!''拿过了雷万春的虎头金枪,奋力一挑,把第十二辆战车挑得飞上半空,恰巧和后一部战车相撞,两部战车,登时都成粉碎,马嘶人叫,肢体横飞,洒下了满空血雨!

雷万春的神勇,贼军已是惊为见所未见,闻所未闻,而今南霁云一枪就粉碎了两部战车,比雷万春更为厉害,后面的几十部战车,车上的``勇土''都给他吓破了胆,在那瞬间,竟然顾不得``帅''令,纷纷拨转马头,如潮退下。

令狐潮的车驾上忽然跳下一个瘦长的老者,喝道:``南霁云休得逞强,我来会你!''声到人到,转眼间就刀光罩顶,向南霁云疾劈了几刀。此人乃是羊牧劳的结义兄弟马远行。

近身恶斗,长枪不便使用,南霁云拔出宝刀,用了一招``八方风雨'',将马远行的鬼头刀荡开,蓦地又是一声大喝:``令狐贼看枪!''长枪脱手掷出,``卜'的一声,正插在令狐潮的车辕上,枪尾兀自颤动不休,令狐潮吓得魂飞魄散,慌忙缩了进去!

马远行怒喝道:``南八,你死到临头,还敢逞能?看刀!''反手一刀,搂头劈下,左掌随着刀锋穿出,五指如钩,藉着兵刃的掩护,向南霁云的琵琶骨抓来!马远行与羊牧劳、牛不耕二人齐名,他身材比南霁云高出半个头,手长脚长,居高临下,使出这刀中夹掌的凶狠恶招,果然是非同小可!

南霁云大笑道三``南某早已拼着血溅沙场,死何足惧?但我却得先宰了你这头畜牲!''霍地一个``风点头'',挥刀一架,接着呼的一拳捣出,但听得``蓬''的一声,接着``叮当''之声,不绝于耳,就在这瞬息之间,两人已是拳掌相交,双方的兵刃,也接连碰了六十下。

马远行是有名的``闪电手'',想不到南霁云的``快刀''比他更快,一片断金夏玉之声过后,只见马远行的``镔铁斫山刀''已损了三四处缺口。幸而他这口``镔铁斫山刀''刀身甚厚,还不至于削嘶。南霁云一刀紧过一刀,端的有如天风海雨,迫人而来,只见刀光,不见人影,贼军虽多,但在刀光耀眼之下,已分不出谁是南霁云,谁是马远行。但见两团刀光滚来滚去,稍为挨近,便是皮破血流,哪里插得进手。

马远行见南霁云招招都是杀手,完全是奋不顾身的拼命打法,也不禁暗暗胆寒。当下打定了主意,不求有功,但求无过,只待拖到了羊牧劳等人来到,便可以稳操胜券了。

南霁云惯经大敌,何尝不知道敌人在拖,而自己则利于速战速决。无奈他这几天,每餐只是吃个半饱,刚才又力挑两辆战车,纵是铁人,也难持久。开头数十招还是刀光霍霍,虎虎生风,渐渐便觉得力不从心,有好几招眼看可以得手的,都给马远行挡开了。

马远行也感觉到了,哈哈大笑道:``南八,我看你也是一条好汉,抛下兵刃,我饶你不死!''南霁云忍着怒气,陡然咬破舌尖,二口鲜血喷出,顿时刀光大盛,把马远行杀得只有招架之功,竟无还刀之力!原来他是用自身疼痛的刺激,把精力都集中起来,当真是以性命与敌人相搏!

激战小只听得段圭璋那边的厮杀声也是震耳欲聋,南霁云挂念张巡的安危,百忙中抽眼望去,只见张巡的车驾已陷入重围,那队护军,已是寥落可数,除了段圭璋之外,大约只剩下三四个人了!

高手比拼,哪容得心神稍乱,马远行看出有机可乘,蓦地-个``弯腰折柳'',刀锋卷地而来,迳削南霁云双足。

南霁云因为比对方矮半个头,一直都是采用仰攻的刀法,不料对方突然变招,南霁云那一刀刚好从对方头顶削过,招数使老,急切问抽不问来,眼看难逃这一刀之厄。

好个市霁云,就在这性命俄顷之间,当机立断,反而迎上前去,飞腿变踢,双方动作都快到极点,但听得``咔嚓''一声,南霁云的胸骨断了一根,接着``蓬''的一声,马远行给他踢了一个筋斗。

两个倏的分开,南霁云正想上前结果马远行的性命,哪知螳螂捕蝉,黄雀在后,乱军之中,还有一个王龙客,早就窥伺一旁,待机而动。只因他们打得难解难分,无法偷施暗算,如今好不容易得到这个机会,哪里还肯错过,王龙客用的那把折扇,扇骨乃是精钢打的,扇柄安着活括,一按机括,扇骨登时变为暗箭,嗖、嗖、嗖,接连三枝,流星闪电般的便向南霁云射去。

南霁云一足受伤,他刀背一格,磕落了一枝,翻身一闪,避开了第二枝,第三枝却躲不过,但听得``嗤''的一声,那支``暗箭'',已射人南霁云的胁下,从背后穿出来,登时血流如注!

王龙客哈哈大笑:``好呀,今日方消我心头之恨!''那马远行翻了一个筋斗,这时也已跳了起来,见南霁云恍似风中之烛,摇摇欲坠,他看出有便宜可捡,立即飞步上前,一刀向南霁云劈下!

令狐潮的手下大喜如狂,不约而同的齐声喊道:``南蛮子完啦!''就在这呐喊声中,南霁云蓦地大喝一声,恰似晴天打了一个霹雳,众人掩耳不迭,睁眼看时,只见南霁云已成了一个血人,但倒下地的却不是他而是马远行,而且马远行的头颅也已不在脖子上了!原来南霁云以毕生功力之所聚,和身扑上,作最后的一击,他中了马远行的三刀,但他却一刀便割下了马远行的首级!

呐喊声登时沉了下去,令狐潮手下身经百战的将士也有许多,却从未曾见过如此惨烈的恶战!不由得个个噤声,人人胆战!南霁云游目四顾,厉声喝道:``王龙客,你出来!王龙客躲在乱军之中哪敢应声?

夏凌霜那辆车子正在另一边疾驰而过,她听得呐喊,心头大震,推开了窦线娘便要冲出车厢,但转瞬间呐喊声便即沉寂,战场上突然静下,更是怕人。夏凌霜惊疑不定,叠声喊道:``霁云、霁云\ldots\ldots{}

南霁云吸了口气,提高声音应道:``凌霜,我没什么,你先走一步,我随后就来!''他为了要使妻子相信他未曾受伤,几乎是把残存的精力都凝聚起来,发出传音人密的内功,好教他的妻子放心!

夏凌霜哪知丈夫已是油尽灯枯,最后挣扎,她听得丈夫的声音精力充沛,只道他果然未曾受伤,心中一宽,心肠软了下来,窦线娘趁势一拉,将她拉回了车厢。

夏凌霜未曾看见丈夫,窦线娘却已瞧得清楚,她见南霁云浑身浴血,远远望去,就似一个刚从颜料缸里拖出来的,白头发到脚跟,都染得通红的人,再一望,又见她的丈夫段圭璋和张巡亦已陷在重围之中,形势岌岌可危,不由得大吃一惊。

就在这时,忽听得贼军金鼓大鸣,又一辆插着``将''旗的战车疾驰而来,窦线娘眼利,已认出那站在车上的人正是羊牧劳!

窦线娘心头大震,无暇思索,就拨转马头,要去援救丈夫。段圭璋高声叫道:``线妹,你今日要确保南弟妇母子平安,否则我永远不能恕你,赶快走吧!''

夏凌霜那对孪生孩子,被金鼓声吓得哇哇大哭,窦线娘心中如同刀绞,暗自想道:``我与圭郎一同赴死,还不打紧,但那就保不住她们母子三人!''这刹那间,她转了好几次念头,终于咬着牙根,含着眼泪,再望了丈夫一眼,便疾的一鞭,催马疾驰,向着与丈夫相反的方向逃走,可怜他们夫妻死别生离,就只能在乱军之中,远远的互相只看了一眼!

羊牧劳哈哈笑道:``釜底游魂,还要挣扎么?姓段的,明年今日,就是你的周年忌日了!''话犹未了,忽听得``轰隆''一声,他那辆车子突然倾覆,原来是段克邪不知从哪里窜出来,突然以闪电般的手法,削断了拖着他那辆车子的马腿!

羊牧劳凌空跃起,大怒喝道:``小贼,往哪里走?今日我要你父子一齐送命!''段克邪身材矮细,滑似游鱼,早已从乱军丛中钻了出来,他一面跑一面嘻嘻笑道:``老贼,你敢惹我,我叫你再瞎一只眼睛!''

转眼间,段克邪已跑到他父亲身边,段圭璋这时也正杀退了面前的敌人,见儿子到来,心中又悲又喜,他忍着眼泪,连忙说道:``克儿,你答应我要做个顶天立地的好汉的,还记得么?''

段克邪一本正经地答道:``父是英雄儿好汉。孩儿紧记不忘!''段圭璋道:``好,那你就要保护母亲,杀出阵去!''段克邪道:``爹爹,你呢?''段圭璋道:``我要在这里保护张太守,我若跑开,还算得是什么英雄呢?''段克邪道:``那么,那老贼呢?''段圭璋道:``由我来料理他,倘然我今日杀不了他,你长大了再去找他算帐。''他想说的本是``报仇''二字,但怕说得太过明白,孩子机灵,会听懂他要以身殉难的心意,是以话到口边,才把``报仇''二字改为``算帐''。

羊牧劳带着一队武士,大声吆喝,越来越近。段圭璋道:``克儿,你看你妈妈的那辆车已走得远了,你还不快迫上去?倘若你不能保护她杀出阵中,就不是好汉了!''

段克邪道:``好,爹爹,你看我的本事!爹爹,你杀了那个老贼,快些来啊!''他身形一起,恍如蝴蝶穿花,蜻蜓点水,在乱军的缝隙中直穿过去,果然万马千军,都拦他不住,转眼之间,不见踪影!

段圭璋急步走到南霁云身边,南霁云流血太多,双眼昏花,神智亦已迷糊,全仗着一股神威,兀立如山,镇慑敌人。他见一条人影向他冲来,只道又是贼军杀到,大喝一声,提刀便斫。段圭璋连忙闪过,叫道:``南兄弟,是我!我背你出去。''南霁云道:``凌霜她们呢?''段圭璋道:``弟妇那辆车子已冲出去了。'''

南霁云道:``好,那么我没有什么牵挂了。段大哥,请恕我把重担都交给你啦!''哇的一大口鲜血喷了出来,扑通''便倒!

段圭璋来不及将他抱起,羊牧劳的人马已似旋风般的卷来。羊牧劳哈哈笑道:``姓段的,今日羊某与你再决雌雄,可惜南八死了,你缺了帮手啦!''

段圭璋一弯腰,将南霁云的宝刀拿起,喝道:``段某只有一颗头颅,你们一齐来吧,看谁有本领拿去!''左刀有剑,狂冲猛斫,转眼之间,已有六七个``勇士''伤在他的刀剑之下。

羊牧劳道:``你们去活捉张巡,别在这儿碍我手脚!''那队勇土巴不得他如此吩咐,一窝蜂的都走了。段圭璋心头一震,想道:``不好,我不能中了羊牧劳调虎离山之计。''可是他要再杀回去,却给羊牧劳拦住了他的去路了!

羊牧劳大笑道:``姓段的,你没胆与老夫一战么?哈哈,你要走也容易,把你这两颗眼珠给我留下来!''

话犹未了,段圭璋蓦地大喝一声,反手便是一剑,羊牧劳一个``游龙探爪'',施展大擒拿手法扣他腕脉,段圭璋左手的宝刀已从肘底穿出,反削过来,羊牧劳使出绵掌功夫,一掌印下,段圭璋竟然不躲不闪,左刀有剑,剑刺前胸,刀削膝盖。羊牧劳大吃一惊,急忙把攻出去的一掌硬生生的撤了回来,护着前胸,蹬蹬蹬连退三步,好不容易才化解了段圭璋这一招两式!

这几招疾如暴风骤雨,双方都使出了浑身本领,每一招都足以致对方死命,但,这在段圭璋是奋不顾身,而在羊牧劳则是被迫拼命,几招过后,羊牧劳不禁胆寒。

本来羊牧劳是这样想的,他曾和段圭璋交过几次手,当然知道对方深浅,因此心中想道:``段圭璋虽然剑法精妙,但我的七步迫魂掌也尽足以应仗,最多不过半斤八两而已。而今他久战之下,已是强弩之末,何足惧战?''故此他才遣散众人,有意逞能,与段圭璋单打独斗。哪知段圭璋一抱了必死之心,竟然锐不可当,杀得他手忙脚乱!

羊牧劳正自心慌,忽听得一个阴阳怪气的声音说道:``小王,你去活捉张巡,我来会会这位段大侠。''羊牧劳大喜道:``三弟,你来得正好,你不是想要一把宝剑么?姓段的这把正是宝剑!''原来这人正是羊牧劳的把弟牛不耕,他和王龙客领了一队铁甲军冲来,本是奉命活捉张巡的,但为了觊觎段圭璋这把宝剑,他宁把活捉张巡的功劳让给王龙客了。

牛不耕用的是一柄乌金打成的``辟云锄'',黑黝黝的毫不起眼,但却沉重非常,段圭璋一剑削去,只听得``当''的一声,火花飞溅,牛不耕的``乌金锄''缺了一口,但段圭璋这把宝剑本来是削铁如泥的,而今却只不过把他的乌金锄削去了一小片,足见他的乌金锄也是一件宝物。

牛不耕试出在兵器上并不怎样吃亏,登时勇气倍增,把一百零八路辟云锄法,尽数施展出来,使辟云锄法的,武林中只他一家,段圭璋也未曾见过。

段圭璋在两大高手夹攻之下,拼死恶战,可怜他自朝至午,一路冲杀,未曾歇过片刻,他到底是血肉之躯,渐渐也感到头晕眼花,有点吃不消了。

激战中,忽听得``轰隆''一声,贼军大叫道:``好呀,张巡的破车翻了!''接着听得王龙客的声音叫道:``元帅有令,只许活捉张巡!''

段圭璋这一惊非同小可,心道:``我当口手下留情,饶了这个小贼,今日却害了张公!''百忙中抽眼望去,只见张巡的车驾果然已是四轮朝天,贼军箭如雨下,张巡的扩军伤亡殆尽,王龙客手挥折扇,正向张巡扑去!

段圭璋又悔又急,忽觉肩头热辣辣的,原来已给牛不耕的乌金锄劈了一刀,肩胛骨都裂开了。段圭璋这时已不知道什么叫做疼痛,也不知是从哪儿来的气力,蓦地里大喝一声,和身撞去,只听得``蓬''的一声,羊牧劳一掌击中他的胸膛,但段圭璋也把他撞翻了。

牛不耕一个闪身,挥锄再劈,段圭璋大喝道:``好,你要宝剑么?宝剑给你!''使出了大摔碑手法,宝剑脱手,直插进牛不耕腹中,将他钉在地上。

随着手臂一抡,左手那口宝刀,也化成了一道长虹,呼的一声,向羊牧劳掷去,羊牧劳刚自一个``鲤鱼打挺'',翻起身来,恰好碰上,被那口宝刀穿过了小腿,可惜距离较远,段圭璋又已气力不加,这一刀虽把羊牧劳重伤,还未能要了他的性命。

贼军纷纷扑来,段圭璋仰天大笑道:``段某今日死得其所,死亦无憾!南兄弟,咱们又可以相见!''不甘受辱,将全身精力凝聚,反手一拍,登时自断经脉而亡!

贼帅令狐潮乘车到来,也不禁嗟叹道:``真是两个好汉子,不愧大侠之名!''吩咐手下,将南霁云与段圭璋以礼葬之。不久,张巡也因众寡不敌,自杀不成,被贼所擒。后来,令狐潮屡次劝降,张巡总是骂不绝口,终于与许远一同就义。张巡的随从护军三十六人,或战死,或被擒,被擒的也无一人屈节。后人有诗赞曰:张巡许远同尽忠,正气浩然昭日月。从死不独南与雷,三十六人均义烈!''

窦线娘驾车疾驰,仗着一把弹弓,弹无虚发,当者披靡,冲开了一条路,虽然尚未冲出战场,离开厮杀的核心地带也已渐渐远了。

窦线娘稍稍松了口气,但远远听那金鼓震天之声,心头更为沉重,她游目四顾,丈夫当然是看不着了,儿子也未见回来。

正自心急如焚,忽听得蹄声得得,一骑健马,疾风般的追来,骑在马上的正是王龙客!

窦线娘大怒,弓弦一拽,金弹飞去,王龙客一个``镫里藏身'',弹子从他身旁擦过,没有打着。窦线娘探手入囊,想取出弹丸施展连珠弹的绝技,哪知囊里空空,这才知道暗器囊中的一百二十颗金丸,已全都用掉了!

王龙客马快如风,转瞬追上,``呼''的一声,一柄长矛掷出,穿过鞍甲,把拉车的一匹马杀了。那辆车子重心不稳,登时摇摆倾斜,幸亏四匹拉车的战马都是素经训练的,一马失蹄,其他三匹马也立即止步,车子才不至于翻倒。不过如此一来,窦线娘又陷入了包围之中。

王龙客哈哈笑道:``你们跑是跑不了的,窦线娘,你我二家的仇恨以后再行算帐,就看你识不识相了!''笑声中,突然从马背一跃而起,扑上了窦线娘这辆车子。

窦线娘手提金弓,劈面打去,王龙客伏在车顶的蓬盖上,这一打没有打着。夏凌霜跳出车厢,拔剑向车顶便刺。

王龙客叫道:``凌霜,你的丈夫已经死了,你不如跟了我吧!''夏凌霜喝道:``狗强盗,胡说八道------''话犹未了,忽听得``当''的

一声,王龙客挥刀劈下,将窦线娘的金弓削为两段!

王龙客哈哈笑道:``你不信么?你睁眼看看,这是谁的宝刀!''原来王龙客在南、段二人死后,便抢了他们的兵刃,他将段圭璋那柄宝剑献给了令狐潮,自己则拿了南霁云那把宝刀,飞马来追夏凌霜。

夏凌霜见了丈夫的宝刀,登时有如头顶打了一个焦雷,天旋地转。王龙客叫道:``你跟了我,我保你母子平安,连窦线娘我也可以饶她一命!''

夏凌霜怒极气极,一剑刺去,但她身怀六甲,一怒之下,用力过度,未刺中敌人,自己反而跌了一跤。

说时迟,那时快,王龙客已经扑进车厢,窦线娘骈指如戟,疾点他背后的``志堂穴'',这``志堂穴''是人身三十六道大穴之一,倘给点中,不死也必重伤。

可惜窦线娘血战了大半天,拉弓百余次,斩杀数十人,也早已是筋疲力竭了。点穴必须有内力相济,力透指尖,才能致人死命,如今她却是没有这个功力了。

王龙客给她一指戳中。虽未受伤,也``咕咚''一声,跌进车厢。窦线娘正要抢进去夺他的宝刀,王龙客忽地一声狞笑,复转身来,窦线娘登时吃了一惊,给吓住了。原来王龙客已把夏凌霜的一个孩子抓在手中,厉声喝道:``你再进一步,我就把这孩子扼死!凌霜,你还要不要孩子的性命?乖乖的跟了我吧!''

夏凌霜挣扎起来,忽地将佩剑抛开,叫道:``王少寨主,你饶了孩子吧,我在这里给你下跪了!''窦线娘又是伤心,又觉奇怪,因为她素来知道夏凌霜是心高气傲,决不肯向敌人乞怜的。

王龙客哈哈大笑道:``夏姑娘,你愿意顺从我了么?好,好,好!起来!起来!你我将来是要做夫妻的,夫妻只该彼此尊敬,却不宜行此大礼!''他见夏凌霜抛了佩剑,心里再无顾忌,眉开眼笑,口角春风,一面说着俏皮话儿,一面就弯腰张臂,要把夏凌霜抱起来,他抓着的那个孩子当然也就放下了。

哪知笑声未绝,忽听得``嗖''的一声,一枝袖箭射了出来,夏凌霜大骂道:``狗强盗,我不杀你,誓不为人!''

夏凌霜是趁着下跪之时,衣袖合拢,遮住了王龙客的目光,突然把袖箭放出来的,王龙客根本就没有防备,距离又近,本来非中不可。却不料王龙客正巧在这个时候,弯下腰想抱她,这一箭原是对准了王龙客的咽喉的,这么一来,就难免偏高,一箭射空,``嗖''的一声,穿过了车篷去了。

王龙客这一惊非同小可,登时怒气勃生,一咬牙根,便厉声喝道:``贼婆娘,不识抬举,我让你去和丈夫团聚吧!''一按扇柄,开动了机括,把两支扇骨,也化成了短箭射出来。他是因为已经知道夏凌霜是决不肯顺从他的了,所以凶性大发,得不到的东西,就非要毁灭不可。

夏凌霜尚未来得及起身,更谈不到躲避。就在这性命俄顷之间,忽听得窦线娘一声尖叫,夏凌霜的身体被她盖住。原来是窦线娘和身扑上,用自己的身体掩护了夏凌霜。

窦线娘的金弓早被削断,这时她是双手空空,无物抵挡,她要施展接暗器的功夫,却又因为力竭精疲,第一支``箭''接到手中,却被利簇穿过了手心,第二支``箭''就接不住,只听得``卜''的一声,从她的肩头射人,背后穿出。

王龙客大喝道:``贼婆娘,我正要送你去见你的死鬼丈夫!''提起南霁云那把宝刀,一刀便向窦线娘劈下。

忽听得一声喝道:``住手!''突然问,一条人影,快如闪电,王龙客的刀锋刚要触及窦线娘的头皮,手腕便突然一震,是段克邪捷如飞鸟的扑来,短剑一格,就把他的宝刀打落。段克邪是在百万军中,东寻西找,好不容易,才找到母亲这辆车子的,可惜他还是来迟了一步,窦线娘已受了伤了。

王龙客的武功也非泛泛,他的兵刃一脱手,立即便托着了段克邪的手肘,同时左臂横抱过来,狠狠的用尽气力,将段克邪拦腰匝实!

段克邪毕竟是个十岁刚刚出头的孩子,任凭他武功如何超卓,体力总是不及对方,这时双方缠身扭打,什么踏雪无痕的轻功,神奇奥妙的招数全都用不上上了。但听得``咕咚''一声,两人都倒在车厢里,王龙客用他粗壮的身躯,紧紧压着段克邪,大声叫道:``快来人呀!''

窦线娘欲爬起身来,上前相助,只觉骨头格格作响,登时痛彻心肺,那条手臂,竟似不属于自己了的,发不出力来。就在这时,只听得车声隆隆,一辆贼军的战车,正自向这边疾驰而来。

说时迟,那时快,夏凌霜把她丈夫那柄宝刀拾了起来,也不知是从哪儿来的气力,只一刀就把王龙客拦腰斩断!

段克邪吸了口气,幸喜未曾受伤,他一跃而起,叫道:``这辆车子来得正好,妈,你们稍等,我去去就来!''脚尖上点,即如弩箭穿空,直向对方的战车射去!

双方距离还有十余丈远,在那辆车子上的是贼军神箭营的一个小队,看见一个小孩子似飞将军的从天而降,人人惊骇之极,几乎不敢相信自己的眼睛,手颤脚战,发出的箭也都失了准头,竟没一枝射中。当然,这也是由于段克邪来得太快的缘故。

段克邪一到车上,立即以闪电般的手法,将十三名神箭手全部刺杀,勒住了马,正好停在他们原来的那辆破车旁边。

段克邪首先将两个孩子抱了过去,这才发现他母亲的肩头一片殷红,段克邪惊道:``妈,你怎么啦?''窦线娘道:``好孩子,不要顾我了,你们逃吧!''夏凌霜满眼都是泪水,俯下身躯,就要把窦线娘背起来,可是她也早已心力交疲,背不动了,终于还是段克邪把她们二人拉了上去。

有一小股贼军的骑兵策马追来,段克邪将那十三名``神箭手''的尸体一一抛出,尖声叫道:``谁不怕死的就来,这些人是你们的榜样!''那一小股骑兵见军中最精锐的神箭手尚且被这孩子尽歼,个个惊奇震骇,人人心中均是想道:``这孩子定是妖星下凡,切莫惹他!''不约而同,拨转马头,一哄而散。

这时已到了贼兵稀薄的地方,没多久就冲出了战场。夏凌霜再也支持不住,捧着丈夫的宝刀,叫了一声``南大哥'',就晕倒了。

窦线娘欲哭无泪,可是此时此际,她却必须强力支持,她半边身子已不能动弹,只有一只手还勉强可以使用。她就靠着车厢,用那只手执着马缰,策马驱车,逃出险地。

段克邪哭道:``妈,都是我不好,累你受了伤,我对不住爹爹了。''窦线娘急忙问道:``你见到了你爹么?他说些什么?''

段克邪道:``爹要我保护你平安脱险,爹要我做个顶天立地的汉子,永远永远记着他的话,嗯,妈你怎么啦?''

窦线娘道:``好孩子,蚂没什么,只不过受了点伤,总算暂时脱险了。你已经无负于你爹爹的嘱托,用不着难过了。唉,好孩子,只要你记着爹爹的说话,妈就放心了。''话声断续而又低沉,只见她面如金纸,肩头上的血泡正接连不断地冒出来。段克邪连忙撕下一幅衣衫,敷了金疮药,给她裹好伤口。他见母亲伤得如此之重,也不禁,吓慌了。

段克邪还不知道,他的金创药虽然能够止血,但对他母亲所受的伤,功效也只是仅能止血而已了。窦线娘的琵琶骨已被射穿,等于成了废人,从今之后,她的武功是再也不能使用了。

可是窦线娘伤口的疼痛比起她心上的痛苦,那就简直不算什么!她听了儿子的话语,已知丈夫决意殉国,今生今世,只怕是再也见不到丈夫了。

她四肢乏力,跟前漆黑,便似掉下了无底的深渊,不住地向下沉,向下沉!\ldots\ldots{}

她忽地一咬牙根,睁眼叫道:``不,这还不是悲伤的时候,咱们还未曾完全离开险境!南弟嫂母子也还要人照料。''可是她实在无法支持,执着的马缰也松开了。

夏凌霜刚好在这时苏醒过来,刚好听见了她这几句话。她心中本来是充满着丧夫的哀痛,整个人都还在迷迷糊糊的,突然听到了这几句话,不由得猛然惊醒,在这一刹那间,另一种更强烈的感情冲击着她,令她受到深深的感动,窦线娘用自己的性命保护了她,而窦线娘也是同样死了丈夫,(段圭璋之死,他的儿子尚未知道,但夏凌霜已从王龙客的话语中知道了。)可是窦线娘却忍受着痛苦,重伤之下,仍然为她们母子驾车。

只见窦线娘猛一咬牙把马缰重拾起来,吆喝道:``走呀,走呀!''不知是否马儿被她一催,跑得太快,她一下子又被震倒,马缰再一次脱手!

夏凌霜热泪盈眶,突然问气力长了出来,叫道:``对,这还不是悲伤的时候!好侄儿,你去照顾妈妈。''她接过了马缰,抬起了马鞭,扬空抽了一鞭,用她精良的控马技术,驾着马车,稳稳地向前奔跑!

车子上不过是两个女人,三个小孩,但却是两个丧了丈夫的女人,三个失了父亲的小孩。唉!这辆车子``载''着的悲伤,不是太过沉重了吗?

三天之后,夏凌霜回到了她在玉龙山下的老家。这个家在她们母女离开之后,交给一个奶妈看管,在战乱中幸而没有毁坏。如今夏凌霜历尽风霜,也幸而平安的回来了。可是不幸的窦线娘却病倒了!

窦线娘的病日益沉重,这一日段克邪正在床前服侍,忽觉微风飒然,回头一望,只见房中已多了一个人,正是他的师兄空空儿。

窦线娘霍地坐了起来,颤声叫道:``空空儿,你\ldots 你道她何以这样惊惶?原来空空儿手上捧着一把宝剑,正是她丈夫段圭璋的那把宝剑!空空儿面色阴沉,怆然说道:``段嫂子,尊夫这把宝剑不该落在坏人手中,所以我给你送回来,顺便来看看师弟。''

空空儿继续说道:``这是我从令狐潮手中盗回来的。嫂子,你不要太过伤心。现在郭令公的大军已直扑睢阳,李光弼的大军也已进了潼关,这场乱事指日可平,尊夫可以无恨了。''

段克邪嚷道:``什么,你是说我爹爹,我爹爹,\ldots\ldots{}''他怎也不肯相信他父亲已死,那一个``死''字到了口边,说不出来。

母子俩心意相通,窦线娘高声说道:``你爹爹是个顶天立地的汉子!不错,你今后是难以见到他了。但像你爹爹这样的人,他是、他是永远不会死的!你把你爹爹的宝剑接下来吧!''

段克邪一片茫然,对母亲的话似懂非懂,他睁着一对充满疑惑的眼睛,把这柄宝剑从空空儿手中接下。

就在这时,夏凌霜走了进来,空空儿的话,她全都听见了。窦线娘还未曾哭得出来,她的泪水已先湿了衣裳了。

窦线娘道:``霜妹,你来得正好。''她取出了一支玉钗,说道:``克儿,这是你定亲的信物。你的妻子叫史若梅,现在由薛嵩收养,改了名字叫薛红线。你长大了拿这柄玉钗去寻找她。''玉钗上雕着一条张牙舞爪的金龙,钗头还嵌着一颗夜明珠。段克邪茫然的又接过了这枝玉钗,正想问``妻子''究竟算是什么人,只听得母亲又已说道:``你若有不明白的地方,以后问你的姑姑。霜妹,我把这孩子托给你了。克儿,你把宝剑拿上前来。''

``咣''的一。声,窦线娘在宝剑上弹了下,叫道:``段郎,段郎\ldots\ldots 我,我来了。''声音突然寂灭。可怜她早已油尽灯枯,只因心中还抱着万一的希望,所以挣扎着活到如今,如今,希望已灭,她也就一瞑不视了。

接着又是``咣''的一声,玉钗从段克邪的手上掉了下来,小小的心灵充满了哀痛。正是:茫茫愁,浩浩劫,夫妻侠义兼忠烈,碧血丹心永不灭!

欲知段克邪今后如何?是否能与史若梅结成佳偶,请看续集《龙凤宝钗缘》。

(本书完)------


\backmatter
\end{document}
