\documentclass[12pt,oneside]{book}

\usepackage{mybook}
\usepackage{mybookcover}


\title{绝代双骄}
\author{古龙}

\begin{document}
\bookcover{book_cover.png}

\frontmatter


\addchtoc{目录}
\setcounter{tocdepth}{2}    
\tableofcontents

\mainmatter


\hypertarget{ux7b2cux4e00ux7ae0-ux540dux5251ux9999ux82b1}{%
\chapter{第一章
名剑香花}\label{ux7b2cux4e00ux7ae0-ux540dux5251ux9999ux82b1}}

江湖中有耳朵的人,绝无一人没有听见过``玉郎''江枫和燕南天这两人的名字;江湖中有眼睛的人,也绝无一人不想瞧瞧江枫的绝世风采和燕南天的绝代神功。

只因为任何人都知道,世上绝没有一个少女能抵挡江枫的微微一笑,也绝没有一个英雄能抵挡燕南天的轻轻一剑!任何人都相信,燕南天的剑非但能在百万军中取主帅之首级,也能将一根头发分成两根,而江枫的笑,却可令少女的心碎。

但此刻,这出生帝富世家的天下第一美男子,却穿着件粗俗的衣衫,赶着辆破旧的马车,匆匆行驶在一条久已荒废的旧道上。

此刻若有人见到他,谁也不会相信他便是那倚马斜桥、一掷千金的风流公子。

七月,夕阳如火,烈日的余威仍在。

人和马,都闷得透不过气来,但江枫手里的鞭子,仍不停鞭打着马。

马车飞驶,将道路的荒草,都辗得倒下去,就好像那些曾经为江枫着迷的少女腰肢。

突然,一声鸡啼,撕裂了天地的沉闷。

但黄昏时,旧道上哪里来的鸡啼?江枫面色变了,明锐的目光,自压在眉际上的破帽边没望过去,只见一只大公鸡站在道旁残柳的树干上,就像钉在上面似的动也不动,那雄丽的鸡冠,多彩的羽毛,在夕阳下闪动着令人眩目的金光。

公鸡的眼睛里竟也似有种恶毒的、妖异的光芒。

江枫的面色变得更苍白,突然勒住了车马。

健马长嘶,车缓缓停下,车厢中有个甜美温柔的语声问道:``什么事?''江枫微一一迟疑,苦笑道:``没有什么,只不过走错路了''拨转马头,兜了半个圈子,竟又向来路奔回,只听那公鸡又是一声长嘶却像是在对他冷笑。

江枫打马更急,路上的荒草已被辗平,车马自是走得更快了,但还未奔出四十丈,道上竟又有样东西挡住了去路。这久已荒废、久无人迹的旧道上,此刻竟突然有只巨大的肥猪横卧在路中,又有谁能猜透这只猪是哪里来的?马车方才还驶过这条路,这条路上,方才明明连半斤猪肉都没有,而此刻却有了整整一只猪。

江枫再次变色,再次勒住马车。

只见那只猪在地上翻滚着,但全身上下,却被洗得干干净净,那紧密的猪毛,在夕阳下就像是金丝织成的毯子一样。

门窗紧闭的车厢里,又传出人语道:``什么事?''江枫语塞:``我\ldots\ldots 我\ldots\ldots{}''那甜美温柔的人语轻叹着道:``你又何苦瞒我?我早已知道''江枫失声道:``你早已知道了?''我方才听见那声鸡啼,便已猜出必定是十二星相中人找上咱们了,你怕我担心,所以才瞒着我,是么?``江枫长叹一声,道:''奇怪\ldots\ldots 你我此行如此秘密,他们怎会知道?但\ldots\ldots 但你只管放心,什么事都有我来抵挡``车厢中人柔声道:''你又错了,自从那天\ldots\ldots 那天我准备和你共生共死,无论有什么危险艰难,也该由咱们俩共同承当。"``但你现在\ldots\ldots{}''``没关系,现在我觉得很好。''江枫咬了咬牙,道:``好,你还能下车走么?道路两头都已有警象,看来咱们也只有弃下车马,穿过这一片荒野\ldots\ldots{}''``为什么要弃下车马呢?他们既已盯上咱们,反正已难脱身''。倒不如就在这里等着,十二星相虽有凶名,但咱们也未必怕他们!"``我\ldots\ldots 我只是怕你\ldots\ldots{}''``你放心,我没关系。''江枫面上忽又现出温柔的笑容,轻轻道:``我能找着你,真是最幸运的事。''他在夕阳下笑着,连夕阳都似失却了颜色。

车厢人娇笑道:``幸运的该是我才对,我知道,江湖中不知道有多少女孩子在羡慕我,妒忌我,只是她们\ldots\ldots{}''语声未了,健马突然仰道惊嘶起来──暮风中方自透出新凉,这匹马却似突然出了什么惊人的警兆!一阵风吹过,猪,在地上翻了个身,远处隐隐传来鸡啼,荒草在风中摇舞,夕阳,黔淡了下来,大地竟似突然被一种不祥的气氛所笼罩,这七月夕阳下的郊野,竟突然显得说不出的凄凉、萧瑟!江枫变色道:``他们似已来了!''突然马车后有人喋喋笑道:``不错,咱们已来了!''这笑声竟也如鸡啼一般,尖锐、刺耳、短促,江枫一生之中,当真从未听过如此难听的笑声。

他大惊转身,轻叱道:``谁?!''

鸡啼般的笑声不绝,马车后已转出七八个人来。

第一个人,身长不足五尺,瘦小枯干,却穿着一身火红的衣裳,那模样正有说不出的诡秘,说不出的猥琐。

第二个人,身长却赫然在九尺开外,高大魁伟,黄衣黄冠,那满脸全无表情的横肉,看来比铁还硬。

后面踉着四个人打扮得更是奇怪,衣服是一块块五颜六色的绸锻缝成的,竟像是戏台上乞丐穿着的富贫衣。

这四人身材相貌不相同,却都是满面凶光、行动骠悍的汉子,举手投足,也是一模一样,谁也不快上一分,谁也不慢上一分。

还有个人远远跟在后面,前面七个人加起来,也末见会比这人重上儿斤,整整一匹料子,也未见能为此人做件衣服,他胖得实在已快走不动了,每走一步,就喘口气,口中不住喃喃道:``好热,热死人了。''满头汗珠,随着他颤动的肥肉不住地流下来。

江枫跃下马车,强作镇定,抱拳道:``来的可是十二星相中之司晨客与黑面君么?''红衣人格格笑道:``江公子果然好眼力,但咱们不过是一只鸡、一只猪而已,司晨客、黑面君,这些好听的名字,不过是江湖中人胡乱取的,咱们担当不起。''江枫目光闪动道:``阁下想必就是~''红衣人截口笑道:"红的是鸡冠,黄的是鸡胸,花的是鸡尾,至于后面那位,你瞧他的模样像什么,他就是什么。

江枫道:``几位不知有何见教?''

红衣鸡冠道:``闻得江公子有了新宠,咱兄弟都忍不住想来瞧瞧这位能令玉郎心动的美人儿究竟美到什么地步,再者,咱兄弟还想来向公子讨件东西''江枫暗中变色,口中却仍然沉声道:``只可惜在下此次匆匆出门,身无长物,哪有什么好东西,能入得了诸位名家法眼''.鸡冠人喋喋笑道:``江公子此刻突然将家财完全变卖,咱们虽不知为的是什么,也不想知道,但江公子以田庄换来的那袋明珠\ldots\ldots 嘿嘿,江公子也该知道咱们十二星相向来贼不空手,公子就把那袋明珠赏给咱们。''江枫突也大笑道:``好,好,原来你们倒竟也打听得如此清楚,在下也知道十二星相从来不轻易出手,出手后从不空回,但\ldots\ldots{}''鸡冠人道:``但什么,你不答应?''江枫冷笑道:``若要我答应,只有\ldots\ldots{}''语声未了,闪闪银光,已到了他胸口。

这鸡冠人好快的手法,眨眼间,手中已多了件银光闪闪的奇形兵刃,似花锄,如钢啄,闪电般击向江枫,眨眼间已攻出七招,那诡异的招式,看来正如公鸡啄米一般,沿着江枫手足少阴经俞府、神法、灵墟、步廊\ldots\ldots 等要穴,一路啄了下去。

江枫平地跃起,凌空一个翻身,堪堪避过了七啄,但这时却又有四对鸡爪镰在地上等着。

鸡枫一动,鸡尾立应,那四个花衣鸡尾人的出手之快,正也不在红衣鸡冠之下,四对鸡爪镰刀,正也是江湖罕暑的外门功夫,一个啄,四个抓,招式配合得滴水不漏,就算是一个人生着九只手,呼应得也未必如此微妙。

江枫自然不是等闲人物,但应付这五件外门兵刃,应付这从来未见的奇诡招式,已是左支右拙,大感吃力、何况还有个满脸横肉、目光闪动的黄衣鸡胸正在一旁目不转睛地瞪着他,只等着他破绽露出。

黑面君嘻嘻笑道:``哥儿们,加油,咱们可不是女人,可莫要对这小子生出怜香惜玉的心,兄弟我且先去睢瞧车子里的小美人儿。''江枫怒喝道:``站住!''他虽想冲过去,怎奈那九件兵刃却围得他风雨不透,而这时黑面君已蹒跚地走向车厢,伸手去拉门。

就在这时,车窗突然开了一线,里面伸出一只白生生的玉手,那纤柔、毫无瑶疵的手指中,却夹者只梅花。

黑色的梅花!盛夏中有梅花,已是奇事,何况是黑色的梅花?!白的手,黑的梅花,衬托出一种无法形容的、神秘的美,车厢中甜美的语声一字字缓缓道:``你们瞧瞧这是什么?''黑面君的脸,突然扭曲起来,那只正在拉门的手,也突然不会动了,鸡嘴啄、鸡爪镰,更都在半空顿住!这六个凶名震动江溺的巨盗,竟似都突然中了魔法,每个人的手、脚、面目,都似已突然被冻结。

黑面君嘎声道:``绣玉谷,移花宫\ldots{}''车厢中人道:``你的眼力倒也不错。''黑面君道:``我\ldots\ldots 小人。''

牙齿打战,竟是一句话也说不出来。

车厢人柔声道:``你们想不想死?''

``小人,不\ldots\ldots{}''"。

``不想死的还不走!''

这句话还末说完,红的、黄的、花的、黑的,全部飞也似的走了──黑面君脚步也不再蹒跚,口中也不喘气了,若非亲眼瞧见,谁也不会相信这么胖的人会有如此轻灵的身法。

江枫一步窜到车窗前,道:``你\ldots\ldots 你没事么了''车厢人笑道,``我只不过招招手而已。''江枫松了口气,叹道:``不想你竟从宫中带出了朵墨玉梅花连十二星相这样的凶人,竟也对她们如此惧怕。''车厢中人道:``由此你就可想到她们有多可怕,咱们还是快走吧,别的人来了都不要紧,但若是\ldots\ldots{}''突然间,只听``嗖嗖嗖''衣袂破风之声骤响,方才逃了的人,此刻竟又全部回来了,来的竟比去时还快。

黑面君格格笑道:``咱们险些上当了,车子里若真是移花宫中的人,方才还能活着走么?你几时听说过移花宫手下留得有活口?''车厢中人道:``我饶了你,你竟还\ldots\ldots 黑面君大喝道:''冒牌货,出来吧!"突然出手一举,那车门竟被一拳击碎!车厢里坐着的乃是个云鬓蓬乱、面带病容的妇人,却仍掩不住她的天香国色──他眼睛并不十分媚秀,鼻子并不十分挺刺。嘴唇也不十分娇小,但这些凑在一起,却教人瞧了第一眼后,目光便再也舍不得离开,尤其是她那双眼睛里所包涵的情感、了解与智慧,更是深如海水。

只是她的腹部却高高横起,原来竟已身怀六甲。

黑面君怔了一怔,突然大笑道:``原来是个大肚婆娘,居然还敢冒充移花官的\ldots\ldots{}''话末说完,那少妇身子突然飞了出来,黑面君还未弄清是怎么回事,脸上已``□□啪啪''被她掴了几个耳光。

那少妇身子又已掠回,轻笑道:``这大肚婆如何?''黑面君怒吼一声,道:``暗算偷袭,又算得了什么?''一拳击了出去,这身子虽臃肿,但这一拳击出,却是又狠、又快、又辣!那少妇面上仍带着微笑,纤手轻轻一引、一拨,也不知她用了什么手法,黑面君这一拳竟被她拨了,``砰''的一拳,竟打在自己肩头上,竞偏偏不能收住,也不能闪避,他一拳击碎车门,是何等气力,这一拳竞自己将自己打得痛吼着跃倒在地上。

鸡冠鸡尾本也跃跃欲试,但此刻却又不禁怔住了,目瞪口呆地瞧着这少妇,连手指都不敢动一动。

黑面君颤声道:``移花接玉,神鬼莫敌\ldots\ldots{}''那少妇道:``你既然知道,便也该知道我是不是冒充的。''黑面君道:``小\ldots\ldots 小人该死,该死!\ldots\ldots{}''抡起手来,正反掴了自己十几个耳括子,打得他那张脸更黑胖了。

那少妇叹了口气,道:``我要为孩子积点阴德,你们\ldots\ldots 你们快走吧。''这一次他们自然逃得更快,眨眼间便逃得踪影不见,但暮色苍茫中,远处却有条鬼魅般的人影一闪,向他们追了过去。

江柯瞧见他们去远,才又松了口气,叹道,``幸亏你还有这一手,又将他们骇佳,否则\ldots\ldots{}''突然发现那少妇面上已变了颜色,身子颤抖着,满头冷汗。滚滚而落,竟似已疼得不能忍受。

江枫大惊道:``你怎么了''那少妇道:``我\ldots\ldots 发动了胎气\ldots\ldots 只怕\ldots\ldots 只怕已\ldots\ldots 快要\ldots\ldots{}''她话还没说完,江柯已慌得乱了手脚,跺足道``这如何是好?''那少妇嘶声道:``你快将车子赶到路旁\ldots\ldots 快\ldots\ldots 快\ldots\ldots 快!''江枫手忙脚乱地将车子赶到路旁长草里,健马不住长嘶着,江枫不停地抹汗,终于一头钻进车厢里,破了的车门,被长衫挡了起来。大约数盏茶的时间,车厢中突然传出婴儿嘹亮的哭声。

过了半响,又听到江枫狂喜呼道:``两个\ldots\ldots 是双胞胎!。''又过了两盏茶时分,满头大汗,满面兴奋的江枫,一头钻出车厢,但目光所及,整个人却又被惊得呆住了!方才鼠窜而逃的黑面君、司晨客,此刻竟又站在车厢前,六只冷冰冰的目光,正眨也不眨地瞧着他!江枫想再作镇定,但面容也不禁骤然变了颜色,失声道:``你\ldots\ldots 你们又回来了?''鸡冠人诡笑道:``公子吃惊了了么''江枫大声道:``你们莫非要送死不成?!''黑面君哈哈大笑道:``送死?\ldots{}''江枫厉声道:``瞧你们并非孤陋寡闻之辈,绣玉谷,移花官的厉害,你们难道不知道?!''他平日虽然风流蕴藉,温文尔雅,但此刻却连眼睛都红了。

黑面君大笑道:``姓江的,你还装什么蒜?你知道,我也知道,移花宫的两位宫主,此刻想要的是你们两人的命,可不是我们。''汗珠,已沿着江枫那挺秀的鼻子流到嘴角,但他的嘴唇却干得发裂,他舐了舐嘴唇,纵声大笑道,``我瞧你倒真是疯了,移花官的宫主会想要我的命?\ldots\ldots 哈哈,你可知道现在车子里的人是谁?''鸡冠人冷冷道:``现在车子里的,不过是移花官的花奴、丫头,只不过是自移花宫逃出来的叛徒!''江枫身子一震,虽然想强作笑声,但再也笑不出了。

黑面君格格笑道,``江公子又吃惊了吧?江公子又怕还要问,这种事咱们又怎会知道的?嘿嘿,这可是件秘密,你可永远也猜不到''这的确是件秘密,江枫弃家而逃,为的正是要逃避移花官那二位宫主的追魂毒手!但这件秘密除了他和他妻子外,绝无别人知道,此刻这些人偏偏知道了,他们是怎会知道的?江枫想不出,也不能再想了,车厢中产妇在呻吟,婴儿在啼哭,车厢外站者的却是些杀人不眨限的恶徒!他身子突然箭一般窜了出去,只见眼前刀光一闪,黄衣鸡胸掌中一对快刀,已挡住了他去路!江枫不避反迎,咬了咬牙,自刀光中穿过去,闪电般托住黄衣人的手腕,一拧一扭,一柄刀已到了他手中。

他飞起一脚,踢向黄衣人的下腹,反手一刀,格开了鸡冠人的钢刀,身子却从鸡爪镰窜了过去,刀光直劈黑面君!这几招使得当真是又狠又准,又快又险!刀光、钢啄、鸡爪,无一件不是擦着他衣衫而过。

黑面君虽拧身避过了这一刀,但也不禁惊出了一身冷汗,抽空还击二拳,口中大喝:``留神!这小子拼上命了!''这些身经百战的恶徒,自然知道一个人若是拼起命来,任何人也难撄其锋,瞧见江枫刀光,竟不硬接,只是游斗!江枫左劈一刀,右击一招,虽然刀刀狠辣,刀刀拼命,但却刀刀落空,黑面君不住狂笑,黄衣人双刀虽只是剩下一柄,但左手刀专走偏锋,不时削来一刀,叫人难以避内,四对鸡爪镰配合无间,攻击时锐不可当,防守时密如蛛网,就只这些已足以让人魂魄!更何况还有那红衣鸡冠,身法更是快如鬼魅,红衣飘飘,倏来忽去,钢啄闪闪,所取处无一不是江枫的要穴!江枫发髻已蓬乱,吼声已嘶裂,为了他心爱人的生命,这风流公子此刻看来已如疯狂的野兽!但他纵然拼命,却也无用了,狮已入陷,虎已被困,纵然拼命,也不过只是无用的挣扎而已。

暮云四合,暮色凄迷。

这一场恶战虽然惊心动魄,却也悲惨得令人不忍卒睹,他流汗!流血!换来的不过是敌人疯狂的讪笑。

车厢中又传出人语,呻吟着呼道:``玉郎,你小心些\ldots\ldots 只要你小心些,他们绝不是你的敌手!''黑面君突然一步窜过去,一把撕开衣,狞笑道:唷,这小子福气不错,居然还是个双胞胎!``江枫嘶声呼道:''恶贼,滚开!"

他冲过去,被挡回来,又冲过去,又被挡回来,再冲过去,再被挡回来,他目毗尽裂,已裂出鲜血!那少妇紧拥着她的两个小孩子,嘶声道:``恶贼,你\ldots\ldots 你\ldots\ldots{}''黑面君格格笑过:``小美人儿,你放心,现在我不会对你怎样的,但等你好了,我却要\ldots\ldots 哈哈,哈哈\ldots\ldots 一江枫狂吼道:''恶贼,只要你敢动她\ldots\ldots{}``黑而君突然伸手在那少妇脸上摸了摸,狞笑道:''我就动她,你又能怎样?"江枫狂吼一声,刀法一乱,快刀、利爪、尖啄,立刻乘隙攻进。

他肩头、前胸、后背,立刻多了无数条血口!那少妇颤声道:``玉郎,你小心些!''黑面君大笑道:``你的玉朗就要变成玉鬼了!''江枫满身鲜血,狂吼道:``恶贼,我纵成厉鬼,也不挠你''充满忿怒的喝声,得意的笑声,悲惨的狂叫,婴儿的啼哭,混成一种令铁石人也要心碎的声音。

\hypertarget{ux7b2cux4e8cux7ae0-ux5200ux4e0bux9057ux5b64}{%
\chapter{第二章
刀下遗孤}\label{ux7b2cux4e8cux7ae0-ux5200ux4e0bux9057ux5b64}}

血!江枫脸上、身上,已无一处不是鲜血!那少妇嘶声喝道:``我和你拼了!''突然抛下孩子,向黑面君扑去,十指指向他咽喉,但黑面君抬手一挡,就将她挡了回去!黑面君大笑道:``美人儿,你方才的厉害哪里去了\ldots\ldots 女人,可怜的女人,你们为什么要生孩子\ldots\ldots{}''。

狂笑未了,那少妇突又扑了上来,黑面君再次挥掌,她却亡命似的抱住了,一口咬住他的咽喉。

黑面君痛吼了一声,鲜血已沾着她的樱唇流出来。

这是邪毒、腥臭的血,但这腥吴的血流过她齿颊,她却感觉到一阵快意,复仇的快意!黑面君痛极之下,一拳击出,那少妇便飞了出去,撞上车厢,跌倒在地,再也爬不起来了。

但仇人血的滋味,她已尝过了。

她凄然笑容,流着泪呼道:``玉郎,你走吧\ldots\ldots 走吧,不要管我,只要我死了,宫主姐妹仍然会对你好的\ldots\ldots{}''江枫狂吼道,``妹子,你死不得!''他再次冲过去,刀、爪、啄,雨点般击下,他也不管,他身中刀削、爪抓,他血肉横飞!只是他还未冲到他妻子面前,便已跌地倒下!那少妇惨呼一声,挣扎着爬过去,他也挣扎着爬过去,他们已别无所求,只要死在一起!他们的手终于握住了对方的手,但黑面君却一脚踩了下去,把两只手骨全都踩碎了!那少妇嘶声道:``你\ldots\ldots 你好狠!''黑面君狞笑道:``你现在才知道我狠么!''

江枫狂吼道:``我什么都给你\ldots\ldots 都给你,只求你能让我们死在一起!''黑面君大笑道:``你此刻再说这话,已太迟了\ldots\ldots 嘿嘿,你们方才骗我、打我时,想必开心得很,此刻我就让你们慢慢地死,让你们死也不能死在一起!''那少妇道:``为什么?\ldots\ldots 为什么?\ldots\ldots 我们和你又有何仇恨?''黑面君道:``告诉你也无妨,我如此做法,只因为我已答应了一个人,他叫我不要让你们两人死在一起。''江枫道,``谁?\ldots\ldots 这人是谁?\ldots\ldots{}''黑面君笑道:``你慢慢猜吧\ldots\ldots{}''那黄衣鸡突然过来,那赤面横肉,仍冷冰冰、死板板的。

绝无任何表情,口中冷冷道,``斩草除根,这两人的孽种也留不得!''黑面君笑道:``正是!''

黄衣人再也不答话,抬起手,一刀向车中婴儿砍下。

江枫狂吼,他妻子连声音都已发不出来。

哪知就在这时,那柄闪电般劈下的钢刀,突然``喀''一声,竟在半空中生生一断为二!黄衣人大惊之下,连退七步,喝道:``谁?!\ldots 什么人?''除了他们自己与地上垂死的人外,别无人影。

但这有炼精钢的快刀,又怎刀□空断了?鸡冠人变色道:``怎么回事?''黄衣人道:``见鬼\ldots\ldots 鬼才知道。''

突又窜了过去,用半截钢刀,再次劈下。

哪知``喀''的一声,这半截钢刀,竟又一断为二,这许多双眼睛都在留神看着,竞无一人看出刀是如何断的。

黄衣人的面色终于变了,颤声道,``莫非真的遇见鬼了?''黑面君沉吟半晌,突然道:``我来!''

轻轻一脚挑选了江枫跃落的钢刀,抓在手中,狞笑着一刀向车厢里劈下,这一刀劈得更急、更快!刀到中途,他手腕突然一抖,刀光错落\ldots\ldots 只听``当''的一声,他韧刀虽未打断,却多了个缺口!鸡冠人变色道:``果然有人暗算!''黑面君也笑不出声来了,颤声道:``这暗器我等既然不见,想必十分细小,此人能以我等瞧不见的暗器击断钢刀,这\ldots\ldots 这是何等惊人的手法,何等惊人的腕力!''黄衣人道:``世上哪有这样的人!莫非是\ldots\ldots{}''忍不住的打了个寒噤,竟不敢将那``鬼''字再说出口来。

垂死的江枫,也似惊得呆了,口中哺哺道:``她来了\ldots\ldots 必定是她来了\ldots\ldots{}''黑面君道:``谁?\ldots\ldots 莫非是燕南天?''突听一人道,``燕南天?燕南天算什么东西?''语声灵巧、活泼,仿佛带着种天真的椎气,但在这无人的荒郊里,骤然听得这种语声,却更令人吃惊。

江枫夫妇不用抬头,已知道是谁来了,两人俱都惨然变色,黑面君等人亦不禁吃了一惊,扭首望去,只见风吹长草波浪起伏,在凄迷的暮色中,不知何时,已多了条人影纤弱而苗条的女子人影!以他们的耳目,竟丝毫觉不出她是自哪里来的一阵风吹过,远在数丈的人影,忽然到了面前。

听得那天真稚气的语声,谁都会以为她必定是个豆蔻年华、稚气未脱、既美丽、又娇甜的少女。

但此刻,来到他们面前的,却是至少已有二十多岁的妇人,她身上穿的是云丝般的锦绣宫装,长裙及地,长发披肩,宛如流云,她娇靥甜美,更胜春花,她那双灵活的眼波中,非但充满了不可描述的智慧之光,也充满了稚气──不是她这种年龄该有的稚气。

无论是谁,只要瞧她一眼,便会知道这是个性格极为复杂的人,谁也休想猜着她的丝毫心事。

无论是谁,只要瞧过她一眼,就会被她这惊人的绝色所惊,但却忍不住要对她生出些怜惜之心。

这绝代的丽人,竟是个天生的残废,那流云长袖,及地长裙,也掩不了她左手与左足的畸形。黑面君瞧清了她,目中虽现出敬畏之色,但面上的惊惶,反而不如先前之甚,躬身问道:``来的可是移花宫的二宫主''宫装丽人笑道:``你认得我?''``怜星宫主的大名,天下谁不知道?!''

``想不到你口才倒不错,很会奉承人嘛。''

``不敢。''

怜星宫主眨了眨眼睛,轻笑道:``看来你倒不怕我''黑面君躬身笑道:``小人只是\ldots\ldots{}''怜星宫主笑道:``你做了这么多坏事,居然还不怕我,这倒是一件奇事,你难道不知道我立刻就要你们的命么!''黑面君面色骤然大变,但仍强笑着道:``宫主在说笑了''怜星宫主嫣然笑道:``说笑,你伤了我花奴宫主,我若让你痛痛快快地死,已是太便宜了,谁会踉你们这样的人说笑?''黑面君失声道:``但\ldots\ldots 但这是邀月宫主\ldots\ldots{}''语末说完,只听``啪啪''一阵响,他脸上已着了十几掌,情况正和他方才被江枫夫人所掴时一样,但却重得多了,十几掌掴过,他已满嘴是血,哪里还能再说得出一个字来。

怜星宫主仍站在那里,长裙飘飘神态悠然,似乎方才根本没有动过,但面上那动人的笑容却已不见,冷冷道:``我姐姐的名字,也是你叫得的么?''鸡冠、鸡胸、鸡尾也早已赅得面无人色,呆若木鸡。

鸡冠人颤声道:``但\ldots\ldots 但这的确是邀\ldots\ldots{}''这次他连``月''字和未出口,脸上也照样被掴了十几个耳光。直打得他那瘦小的身子几乎飞了出去。

怜星宫主笑道:``奇怪,难道你真的不相信我会要你的命么?唉\ldots\ldots{}''轻轻一声叹息,叹息声中,突然围着黄衣人那高大的身于一转,众人只觉眼前一花,也未瞧见她是否已出手,但黄衣人已静静地倒了下去,连一点声音都未发出。

花衣人中一个悄悄俯下身去瞧了瞧,突然嘶声惊呼道:``死了,老二死了\ldots\ldots{}''怜星宫主笑道,``现在,你总相信了吧\ldots{}''那花衣人嘶声道:``你好\ldots\ldots 好狠。''怜星宫主道:``死个人又有什么大惊小怪?你们自己杀的人,难道还不够多么?你们现在死,也蛮值得了。''鸡冠人目中已暴出凶光,突然打了个手式,剩下叁双鸡爪镰,立刻泼风般向怜星宫主卷了过去。

只听``叮咯、呼噜、哎呀\ldots\ldots{}''一一连串声响,只见那纤弱的人影在满天银光中一转。

叁个花衣人已倒下两个,剩下的一个竟急退八尺,双手已空空如也,别人是如何击倒他同伴,如何闪开他一击,又如何夺去他的兵刃,他全不知道,在方才那一刹那间,他竟似糊糊涂涂地做了一场噩梦!怜星宫主长袖一抖,五柄鸡爪镰``哗啦啦''落了一地,她手里还拿着一柄,瞧了瞧,笑道:``原来是双鸡爪子,不知道滋味如何?''微启樱口,在鸡爪镰上咬了一口,但闻``喀''的一响,这精钢所铸、江湖中闻名丧胆的外门兵刃,竟生生被她咬断。

怜星宫主摇头道:``哎呀,这鸡爪子不好吃!''``啐''的一口,轻轻将嘴里半截铁爪吐了出来,银光一闪,风声微响,剩下的一个花衣人突然惨呼一声,双手掩面,满地打滚。鲜血,不断自指缝间流出,滚了几滚,再也不会动了。

他手掌也刚刚松开,暮色中,只见他面容狰狞,血肉模糊,那半截的爪,竟将他的头骨全部击碎了!黑面君突然噗地跪了下来,颤声道:宫主饶命\ldots\ldots 饶命怜星宫主却不理他,反而瞧着那鸡冠人笑道:``你瞧我功夫如何?''鸡冠人道:``宫\ldots\ldots 宫主的武功,我\ldots\ldots 个人一辈子也没见过\ldots\ldots 小人简直连做梦都未想到世上有这样的武功。''怜星宫主道:``你怕不怕?''鸡冠人一生中当真从未想到自己会被人问出这种问小孩的话,而此刻被人问了,他竟然也只有乖乖地回答,道,``怕\ldots\ldots 怕\ldots\ldots 怕得很。''怜星宫主笑道:``既然也害怕,为何不求饶命?''鸡冠人终于噗地跪下,哭丧者脸,道,``宫主饶命\ldots\ldots{}''怜星宫主眼皮转了转,笑道,``你们要我饶命,也简单得很。只要你们一人打我一拳\ldots{}''鸡冠人道:``小人不敢\ldots\ldots{}''黑面君道:``小人天大的胆子也不敢。''怜星宫主眼睛一瞪,道:``你们不要命了吗?''鸡冠人、黑面君两人,一生中也不知被多少人问过这样的话,平时他们只觉这句话当真是问得狗而屁之,根本用不着回答,要回答也不过只是一记拳头,几声狂笑,接着刀就亮了出去。

但此刻,这同样的一句话,自怜星宫主口中问出来,两人却知道非要乖乖地回答不可了。

两人齐声道:``个人要命的。''

怜星宫主道:``若是要命,就快动手。''

两人对望一眼,终于勉强走过去。

怜星宫主笑道,``嗯,这样才是,你们只管放心打吧,打得越重越好,打得重了,我绝不回手,若是打轻了\ldots\ldots 哼!''鸡冠人暗道:``她既是如此吩咐,我何不将计就计,重重给她一啄,若是得手,岂非天幸,纵不得手,也没什么\ldots{}''黑面君暗道:``这可是你自己要的,可怪不得我,你纵有天大的本领,铁打的身子,只要不还手,我一举也可以打扁你。''两人心中突现生机,虽在暗中大喜欲狂,也面上却更是作出悉眉苦脸的模样!齐声垂首道:``是。''怜星宫主笑道:``来呀,还等什么\ldots{}''黑面君身形暴起,双拳连环击出,那虎虎的拳风,再加上他那百多斤的身子,这一击之威,甚是可观!但他双拳之势,却是灵动飘忽,变化无穷,直到最后,方自定得方向,直捣怜星宫主的胸腔!这正是他一生武功的精华,``神猪化象'',就只这一拳之威,江湖中已不知有多少人粉身碎骨。

鸡冠人身形也飞一般窜出,鸡嘴啄已化为点点银光,有如星雨般洒向怜星宫主前胸八处大穴。

这自然也是他不到性命交关时不轻易使出的煞手!``晨鸡啼屋'',据说这一招曾令``威武镖局''八大镖师同时丧生掌下!怜星宫主笑道:``嗯,果然卖力了。''笑语声中,右掌有如蝴蝶般在银雨拳风中轻轻一飘、一引,鸡冠人、黑面人突然觉得自己全力击出的一招,竞莫名其妙地失去了准头,自己的手掌,竟已似不听自己的使唤,要它往东它偏要住西,要它停,它偏偏不停,只听``呼、哧''两响,紧跟着两声惨呼。

怜星宫主仍然笑哈哈地站着,动也未动,黑面君身子却已倒下,而鸡冠人的身子竟已落入八尺外的草丛中。

草丛中呻吟两声,再无声息。

黑面君的胸膛上,却插着鸡冠人的钢啄,他咬了咬牙,反手拔出铜啄,鲜血像涌泉般流出来,颤声道:``你\ldots\ldots 你\ldots\ldots{}''怜星宫主笑道,``我可没动手伤你,唉,你们自己打自己,何必呢。''黑面君双睛怒凸,直瞪着她,嘴唇启动,仅是想说什么,却-个字也未说出──永远也说不出了。

怜星宫主叹道:``你们若不想杀我,下手轻些,也许就不会死了,我总算给了你们一个活命的机会,是么!''她问的话,永远也没有人回答了。

马,不知何时已倒在地上,车也翻了。

江枫夫妇,正挣扎着想进入车厢,抱出车厢里哭声欲裂的婴儿,两人的手,已刚刚摸着襁褓里的婴儿。

但忽然间,一只手将婴儿推开了。

那是只柔软无骨、美胜春葱的纤纤玉手,雪白的绫罗长袖,覆在手背上,但却比白绫更白。

江枫嘶声道:``给我\ldots\ldots 给我。''

那少妇颤声道:``二宫主,求求你,将孩子给我。''怜星宫主笑道:``月奴,好,想不到你竟已为江枫生出了孩子。''她虽然在笑,但那笑容却是说不出的凄惊、幽怨,而且满含怨毒。

那少妇花月奴道:``宫主,我知道对\ldots\ldots 对不起你,但\ldots\ldots 孩子可是无辜的,你饶了他们吧\ldots{}''怜星宫主目光出神地瞧着那一对婴儿,喃喃道:``孩子,可爱的孩子\ldots\ldots 若是我的多好\ldots\ldots 眼睛突然望向江枫,目光中满含怨毒、怀恨,也满含埋怨、感伤,望了半晌,幽幽瞎:''江枫,你为什么要这样做?为什么?``江枫道:''没什么,只因我爱她。``怜星宫主嘶声道:''你爱她\ldots\ldots 我姐姐哪点比不上她,你被人伤,我姐姐救你回来,百般照顾你,她一辈子也没有对人这么好过,但\ldots\ldots 但她对你却是那样好,你,你\ldots\ldots 你\ldots\ldots 竟跟她的丫头偷偷跑了。``江枫咬牙道:''好,你若要问我,就告诉你,你姐姐根本不是人,她是一团火,一块冰,一柄剑,她甚至可说是鬼,是神,但绝不是人,而她\ldots\ldots{}``目光望着他妻子,立刻变得温柔如水,缓缓接着道:''她却是人,活生生的人,她不但对我好,而且也了解我的心,世上只有她一人是爱我的心,我的灵魂,而不是爱我这张脸!``怜星宫主突然一拿掴在他脸上,道:''你说\ldots\ldots 你再说!``江枫道:''这是我心里的话,我为何不能说!``怜星宫主道:''你只知她对你好,你可知我对你怎样?你\ldots\ldots 你这张脸,你这张脸纵然完全毁了,我还是\ldots\ldots 还是\ldots\ldots"声音渐说微弱,终于再无言语。

花月奴失声道:``二宫主,原来你\ldots\ldots 你也\ldots\ldots{}''怜星宫主大声道:``我难道不能对他好了,我难道不能爱他?\ldots\ldots 是不是因为我是个残废\ldots\ldots 但残废也是人,也是女人!''她整个人竟似突然变了,在刹那之前,她还是个可以主宰别人生死的超人,高高在上,高不可攀。

而此刻,她只是个女人,一个软弱而可怜的女人。

她面上竟有了泪痕。

这在江湖传说中近乎神话般的人物,竟也流泪,江枫、花月奴望着她面上的泪痕,不禁呆住。

过了良久,花月奴黯然道:``二宫主,反正我已活不长了,他\ldots\ldots 从此就是你的了,你救救他吧,我知道唯有你还能救活他。''怜星宫主身子一颤,``他从此就是你的了\ldots\ldots{}''这句活,就像是箭一般射人她心里。

江枫突然嘶声狂笑起来,但那笑声却比世上所有痛哭还要凄厉、悲惨。他充血的目光凝注花月奴,惨笑道:``救活我?\ldots 世上还有谁能救活我?你若死了,我还能活么?\ldots\ldots 月奴,月奴,难道你直到此刻还不丁解我?''花月奴忍住了又将夺眶而出的眼泪,柔声道:``我了解你,我自然了解你,但你若也死了,孩子们又该怎么办?\ldots\ldots 孩子们又该怎么办?''她语声终于化为悲啼,紧紧捏着江枫的手,流泪道:``这是我们的罪孽,谁也无权将上一代的罪孽留给下一代去承受苦果,就算你\ldots\ldots 你也不能的,你也无权以一死来寻求解脱。''江柯的惨笑早已顿住,钢牙已将咬碎。

花月奴颤声道:``我也知道死是多么容易,而活着是多么艰苦,但求求你\ldots\ldots 求求你为了孩子,你必须活着。''江枫泪流满面,似已痴了,喃喃道:``我必须活着?\ldots\ldots 我真的必须活着?\ldots\ldots{}''花月奴道:``二宫主,无论为了什么,你都该救活他的,若是你具有一份爱他的心,你就不能眼见他死在你面前。''怜星宫主悠悠道:``是么?\ldots\ldots{}''花月奴嘶声道:``你能救活他的\ldots\ldots 你必定会救活他的''怜星宫主长长叹息了一声道:``不错,我是能救活他的\ldots\ldots{}''话未说完,也不知从哪里响起了一个人的语声,缓缓道:``错了,你不能救活他,世上再没有一个人能救活他!''这语声是那么灵动、缥缈,不可捉摸,这语声是那么冷漠、无情,令人战栗,却又是那么清柔、娇美,摄人魂魄。

世上也没有一个人听见这语声再能忘记。

大地苍穹,似乎就因为这淡淡的一句话而变得充满杀机,充满寒意,满天夕阳,也似就因这句话而失却颜色。

江枫身子有如秋叶般颤抖起来。

怜星宫主的脸,也立刻苍白得再无一丝血色。

一条白衣人影,已自漫天夕阳下来到他们面前。

她不知从何而来,也不知是如何来的。

她衣抉飘飘,宛如乘风,她白衣胜雪,长发如云,她风姿绰约,宛如仙子,但她的容貌,却无人能以描叙,只因世上再也无人敢抬头去瞧她一眼。

她身上似乎与生俱来便带来一种慑人的魔力,不可抗拒的魔力,她似乎永远高高在上,令人不可仰视!怜星宫主的头也垂下了,咬着樱唇,道``姐姐,你\ldots\ldots 你也来了。''邀月宫主悠悠道:``我来了,你可是想不到。''怜星宫主头垂得更低,道:``姐姐你是什么时候来的?''邀月宫主道:``我来的并不太早,只是已早得足以听见许多别人不愿被我听见的话。''江枫心念一闪,突然大声道:``你\ldots\ldots 你\ldots\ldots 你\ldots\ldots 原来你早已来了,那鸡冠人与黑面君敢去而复返,莫非是你叫他们回来的,那所有的秘密,莫非是你告诉他们的。''邀月宫主道:``你现在才想到,岂非已大迟了?''江枫目毗尽裂,大喝道:``你\ldots\ldots 你为何要如此做?!你为何如此狠心?!''邀月宫主道:``对狠心的人,我定要比他还狠心十倍。''花耳奴忍不住惨呼道:``大宫主,这一切都是我的错,您\ldots\ldots 您不能怪他,''邀月宫主语声突然变得刀一般冷厉,一字字道:``你\ldots\ldots 你还敢在此说话?''花月奴匍匐在地,颠声道:``我\ldots\ldots 我\ldots\ldots{}''邀月宫主缓缓道,``你很好\ldots\ldots 现在你己见着了我,现在\ldots\ldots 你已可以死了!''花月奴见她,怕得连眼泪都已不敢流下,此刻早已阖起了眼来,耳语般颤声道:``多谢宫主。''张开眼睛,瞧了瞧江枫,又瞧了瞧孩子,──她只是轻轻一瞥,也这一瞥间所包含的情感,却深于海水。

江枫心也碎了,大呼道:``月奴,你不能死\ldots\ldots 不能死\ldots\ldots{}''花月奴柔声道:``我先走了\ldots\ldots 我会等你\ldots\ldots{}''她再次阖起眼,这一次,她眼再也不会张开了。

江枫嘶声呼道:``月奴!你再等等,我陪着你\ldots\ldots{}''他也不知是从哪里来的力气,突然跃起来,向月奴仆了过去,但他身子方跃起,便已被一般劲风击倒。

邀月宫主道,``你还是静静地躺着吧。''

江枫颤声道:"我从来不求人,但现在\ldots\ldots 现在我求求你\ldots\ldots 求求你,我什么都已不要,只望能和她死在一起。

邀月宫道道:``你再也休想沾着她一根手指!''江枫瞪着她,若是目光也可杀人,她便早已死了。

若是怒火也会燃烧,大地便早已化为火窟。

但邀月宫主却只是静静地站在那里\ldots\ldots 江枫突然疯狂般大笑起来,笑声久久不绝。

怜星宫主轻叹道:``你还笑?你笑什么?''

江枫狂笑道:``你们自以为了不起!你们自以为能主宰一切,但只要我死了,便可和月奴在一起,你们能阻挡得了么?''狂笑声中,身子突然在地上滚了两滚,伏面在地,狂笑渐浙微弱,终于消寂。

怜星宫主轻呼一声,赶过去翻转他身子,只见一截刀头,已完全插入他胸膛里。

月已升起,月光已洒满大地。

怜星宫主跪在那里,石像般动也不动,只有夏夜的凉风,吹拂着她的发丝,良久良久,她终于轻轻道:``死了\ldots\ldots 他总算如愿了,而我们呢?\ldots{}''突然站起来,掠到邀月宫主面前,嘶声大呼道:``我们呢?\ldots\ldots 我们呢?他们都如愿了,我们呢?''邀月宫主似乎无动于衷,冷冷道:``住口!''

怜星宫主道:``我偏不住口,我偏要说!你这样做,究竟又得到了什么?你\ldots\ldots 你只不过使他们更相爱!使他们更恨你!''话未说完,突然``啪''的一声,脸上已被掴了一掌。

怜星宫主倒退几步,手后着脸,颤声道:``你\ldots\ldots 你;你\ldots\ldots{}''邀月宫主道:``你只知道他们恨我,你可知道我多么恨他?我恨得连心里都已滴出血来\ldots\ldots{}''突然卷起衣袖,大声:``你瞧瞧这是什么?''月光下,她晶莹的玉臂,竟满是点点血斑。

怜星宫主怔了一怔,道:``这\ldots\ldots 这是\ldots\ldots{}''邀月宫主道:``这都是我自己用针刺的,他们走了后,我\ldots\ldots 我恨\ldots\ldots 恨得只有用针刺自己,每天每夜我只有拼命折磨自己。才能减轻心里的痛苦,这些你可知道么?\ldots\ldots 你可知道么''她冷漠的语声,竟也变得激动、颤抖起来。

怜星宫主瞧着她臂上的血斑,愣了半晌,泪流满面,纵身扑入她姐姐的怀里,颤声道:``想不到\ldots\ldots 想不到,姐姐你居然也会有这么深的痛苦。''邀月宫主轻轻抱住了她肩头,仰视着天畔的新月,幽幽道:``我也是人\ldots\ldots 只可惜我也是人,便只有忍受人类的痛苦,便只有也和世人一样怀恨、嫉妒\ldots\ldots{}''月光,照着她们拥抱的娇躯,如云的柔发\ldots\ldots 此时此刻,她们已不再是叱□江湖、咸震天下的女魔头,而只是一对同病相怜、真情流露的平凡女子。

怜星宫主口中不住喃喃道:``姐姐\ldots\ldots 姐姐\ldots\ldots 我现在才知道\ldots\ldots{}''邀月宫主突然重重推开了她,道:``站好!''怜星宫主身子直被推出好几尺,才能站稳,但口中却凄然道:``二十多年来,这还是你第一次抱我,你此刻纵然推开我,我也心满意足了!''邀月宫主再也不瞧她一眼,冷冷道:``快动手!''怜星宫主道:``动手\ldots\ldots 向谁动手了!邀月宫主道:''孩子!``怜星宫主失声道:''孩子?\ldots\ldots 他们才出世,你就真要一\ldots 真要\ldots\ldots{}``邀月宫主道:''我不能留下他们的孩子!孩子若不死,我只要想到他们是江枫和那贱婢的孩子,我就会痛苦,我一辈予都会痛苦!``怜星宫主道:''但我\ldots\ldots{}``邀月宫主道:''你不愿出手?``怜星宫主道''我\ldots\ldots 我不忍,我下不了手。"

邀月宫主道:``好!我来!''她流云般长袖一飘,地上的长刀,已到了手里,银光一闪,这柄刀闪电般向那熟捶中的孩子划去。

怜星宫主突然死命地抱住了她的手,但刀尖已在那孩子的脸上划破一条血口,孩子痛哭惊醒了。

邀月宫主怒道:``你敢拦我?''

怜星宫主道:``我\ldots\ldots 我\ldots\ldots{}''邀月宫主道:``放手!你几时见过有人拦得住我!''怜星宫主突然笑道:``姐姐,我不是拦你,我只是突然想到比杀死他们更好的主意,你若杀了这两个什么都不懂的孩子,又有什么好处?他们现在根本不知道痛苦!''邀月宫主目光闪动,道:``不杀又如何?''

怜星宫主道:``你若能令这两个孩子终生痛苦,才真算的出了气,那么江枫和那贱婢纵然死了,也不能死得安稳!''邀月宫主沉默良久,终于叹道:``你且说说有什么法子能今他们终生痛苦!''怜星宫主道:``现在,世上并没有一个人知道江枫生的是双生子,是么?''邀月宫主一时间竞摸不透她这句话中有何含意,只得颔首道:``不错。''怜星宫主道:``这孩子自己也不知道,是么?''邀月宫主道:``哼!废话!''

怜星宫主道:``那自称天下第一剑客的燕南天,本是江枫的平生知友,他本已约好要在这条路上接江枫,否则江枫也不会走这条路了\ldots\ldots{}''怜星宫主微微一笑,继续说道:我们若将这两个孩子带走一个,留下一个在这里,燕南天来了,必定将留下的这孩子带走!必定会将自己一生绝技传授给这孩子,也必定会要这孩子长大了为父母复仇,是吗?我们只要在江枫身上留下个掌印,他们就必定会知道这是移花宫主下的手,那孩子长大了,复仇的对象就是移花宫,是么?``邀月宫主目中已有光芒闪动,缓缓道:''不错。``那时,我们带走的孩子也已长大了,自然也学会了一身功夫,他是移花宫中唯一的男人,若有人来向我们寻仇,他自然会挺身而出,首当其冲,他们自然不知道他们本是兄弟,世上也没人知道,这样\ldots\ldots{}''``他们弟兄间就变成不共戴天的仇人,是么?''怜星宫主拍手笑道:``正是如此,那时,弟弟要杀死哥哥复仇,哥哥自然也要杀死弟弟,他们本是同胞兄弟,智慧必定差不多,两人既然不相上下,必定勾心斗争,互相争杀,也不知要多久才能将对方杀死!''邀月宫主嘴角终于现出一丝微笑,道,``这倒有趣得很。''``这简直有趣极了,这岂非比现在杀死他们好得多!''``他们无论是谁杀死了谁,我们都要将这秘密告诉那活者的一个,那时\ldots\ldots 他面色瞧来也想必有趣得很。''怜星宫主拍手道:``那便是最有趣的时候!''

邀月宫主突又冷冷道:``但若有人先将这秘密向他们说出便无趣了。''``但世上根本无人知道此事\ldots\ldots{}''``除了你!''

``我?这主意是我想出来的,我怎会说?何况,姐姐你最知道我的脾气,如此有趣的事,我会不等着瞧么?''邀月宫主默然半晌,颔首道:``这倒不错,普天之下,只怕也只有你想得出如此古怪的主意,你既想出了这主意,只怕是不会再将秘密说出的了。''怜星宫主笑道:``这主意虽古怪,但却必定有用得很,最妙的是,他们本是孪生兄弟,但此刻有一个脸上已受伤,将来长大了。模样就必定不会相同了,那时,天下有谁能想得到这两个不死不休的仇人,竟是同胞兄弟!''那受伤的孩子一哭声竟也停住,他似乎也被刻骨的仇恨,这恶毒的计谋骇得呆住了。

他睁着一双无邪的,但却受惊的眼睛。似乎已预见来日的种种灾难,种种痛苦,似乎已预见自己一生的不幸!邀月宫主俯首瞄了他们一眼,喃喃道,``十七年\ldots\ldots 最少还要等十六年\ldots\ldots{}''

\hypertarget{ux7b2cux4e09ux7ae0-ux7b2cux4e00ux795eux5251}{%
\chapter{第三章
第一神剑}\label{ux7b2cux4e09ux7ae0-ux7b2cux4e00ux795eux5251}}

干净的石板街,简朴的房屋,淳善的人面\ldots\ldots 这是个平凡的小镇。

六月的阳光,照着这小镇唯一的长街,照着这条街上唯一酒铺的青布招牌,照着这残旧酒招上斗大的``太白居''叁个字。

酒舍里哪有什么生意,那歪戴着帽子的酒保,正伏在桌上打盹儿,不错,那边桌上是坐着位客人。但这样的客人,他却懒得招呼,两叁天来,这客人天天来喝酒,但除了最便宜的酒外,他连一文钱菜都没叫。这客人的确太穷,穷得连脚上的草鞋底都磨穿了,此刻他将脚跷在桌上,使露出鞋底两个大洞。但他却毫不在乎,他靠着墙,跷着脚,眯着眼睛,那八尺长躯,坐在这小酒店的角落中,就像是条懒睡的猛虎。

阳光,自外面斜斜地照进来,照着他两条发墨般的浓眉,照着他棱棱的颧骨,也照着他满脸青惨惨的胡渣子直发光。

他皱了皱眉头,用一只瘦骨嶙峋的大手挡住眼睛,另一只抓者柄已锈得快烂的铁剑,竟呼呼大睡起来。

这时才过正午不久,安静的小镇上,突有几匹健马急驰而过,鲜衣怒马,马行如龙,街道旁人人侧目。

几匹马到了酒铺前。竟一齐停下,几条锦衣大汉,一窝蜂挤进了那个小的酒铺,几乎将店都拆散了。

当先一条大汉腰悬宝剑,趾高气扬,就连那一脸大麻子,都似乎在一粒粒发着光,一走进酒铺,便纵声大笑着:``太白居,这破屋子、烂摊子也可叫做太白居么?''他身后一人圆圆的脸,圆圆的肚子,身上虽也挂着剑,看来却像是个布店掌柜的,接着笑道:``雷老大,你可错了,李太白的几首诗虽写得蛮不错,但却也是个没钱没势的穷小子,住在这种地方正合适\ldots{}''那雷老大仰首笑道:``可惜那李太白早死了好多年,不然咱们可请他喝两杯\ldots\ldots 喂,卖酒的,好酒好菜,快拿上来!''几杯酒下肚,几个人笑声更响了,角落那条大汉,皱着眉头,伸了个懒腰,终于坐直了,喃喃道:``臭不可闻,俗不可耐\ldots\ldots{}''突然一拍桌子,道:``快拿酒来,解解俗气。''这一声大喝,竟像是半空中打了个响雷,将那几条锦衣大双骇得几乎从桌上跳了起来。

那雷老大瞧了瞧,脸色已变了,身子已站起,但却被那个瘦小枯干、满面精悍的汉子拉住,低声道:``总镖头就要来了,咱们何必多事?''雷老大``哼''了声,终又坐下,喝了杯酒,又道:``孙老叁,老总说的可是这地方了你听错没有?''那瘦脸笑道:``错不了的,钱二哥也听见了\ldots\ldots{}''圆脸汉子截口笑道:``不错!就是这儿,老总这次来,听说要来见一位大英雄,所以要咱们先将礼物带来,在这里等着!''雷老大道:``你知道老总要见的是谁么?''

钱二微微一笑,低低说了个名字。

雷老大立刻失声道:``是他?原来是他?他也会来这里?!''钱二道:``他若不来,老总怎会来''几个人立刻老实了,笑声也小了,但酒喝得更多,嘴里也不停地在吱吱喳喳,低声谈论着。``听说那主儿掌中一口剑,是神仙给的,不但削铁如泥,而且剑光在半夜里比灯还大。''``嗯!不错,若没有这祥的宝剑,怎会在半盏茶工夫里,就把阴山那群恶鬼的脑袋都砍了下来?''说到这里,几个人情个自禁,都将膝里挂着的剑解了下来,有的还抽出来,用衣角不停地擦。

雷老大笑道:``我这口剑也算不错了,但比起人家那柄,想来还是差着点儿,否则我也能像他那样出名了!''钱二摇头道:``不然不然,你纵有那样的剑也不成,不说别的,就说人家那身轻功\ldots\ldots 嘿!北京城可算高吧,人家跺跺脚就过去了。''雷老大吐了吐舌头,道:``真的么?''

钱二道:``可不是真的,听说他天黑时还在北京城喝酒,天没亮就到了阴山,,阴山群鬼只瞧见剑光一闪,脑袋就都掉下来了\ldots\ldots 嘿!听说那剑光,简直就像是天上的闪电一样,连阴山外几百里的人都能瞧见。''角落中那穷汉,也在用衣角擦着那柄锈剑,擦两下,喝口酒,此刻突然放声大笑起来,笑道:``世上哪有那样的人!那样的剑!''雷老大脸色立刻变了,拍着桌子,怒吼道:``是谁在这里胡说八道?快给我滚过来!''那穷汉却似乎根本没有听见,还是在擦着那口锈剑,还是在喝着酒,方才那句话,似乎根本不是他说的。

雷老大再也忍不住跳了起来,向他冲过去,但却被钱二拉住,先向雷老大使了个脸色,然后自己摇摇摆摆走过去,笑道:``看来朋友你也是练剑的,所以听人说这话,就难免有些不服气,但朋友可知道咱们说的是谁么?''那穷双懒洋洋抬起头来龇牙一笑,道,``谁?''钱二道:``燕大侠,燕南天,燕神剑\ldots\ldots 哈哈,朋友你若真的是练剑的,听到这名字,就总该服气了吧!''那穷汉却眨了眨眼睛,嘻嘻笑道,``燕南天?\ldots\ldots 燕南天是谁?''钱二抚着肚子,哈哈大笑道:``你连燕大侠的名字都未听过,还算是练剑的么?''那穷汉笑道:``如此说来,你想必是认得他的了,他长得是何模样,他那柄剑\ldots\ldots{}''雷老大终于还是冲了过来,``啪''的一拍桌子,吼道:``咱们纵不认得他,但却也知道他是长得远比你这□帅得多了,他那柄剑更不知要比你这口强胜千百倍。''那穷汉大笑道:``瞧你也是个保镖的达官,怎地眼力如此不济,咱家长得虽不英俊,但这口剑么,却是\ldots\ldots{}''雷老大仰天打了个哈哈,截口道,``你这口破剑难道还是什么神物利器不成?''``咱家这口剑,正是削铁如泥的利器\ldots{}''这句话还未说完,别人已哄堂大笑起来。

又听雷老大道:``你这口剑若能削铁如泥,咱家不但要好好请你喝一顿,而且\ldots\ldots{}''那穷汉霍然长身而起,道:``好,抽出你的到来试试!''他坐在那里倒也罢了,此番一站将起来,雷老大竟不由自主被骇得倒退两步,钱二虽是胖子,但和他那雄伟的躯干一比,突然觉得自己已变成小瘦子。

只见他虽然生无余肉,也骨骼长大,双肩宽阔,一双大手垂下来,竟几乎已将垂到膝盖之下。

这时酒铺里悄然走进个面色惨白、青衣小帽的少年,瞧见这情况,倚在柜台前,忍不住嘻嘻地笑。

雷老大终于抽出了他那柄精钢长剑,终于又挺起了胸膛,大吼道:``好!就让你试试。''那穷汉道:``你只管用力砍过来就是\ldots{}''雷老大龇牙笑道:``小心些,伤了你可莫怪我。''手腕一抖,精钢剑当头劈了下来。

那穷汉左手持杯而饮,右手撩起锈剑,向上一迎,只听``当''的一声,雷老大又倒退两步,手中剑竟已只剩下半截,众人全都呆住了,几乎不相信自己的眼睛。

那穷汉子手抚锈剑,哈哈大笑道:``如何?''

雷老大张口结舌,呐呐道,好\ldots\ldots 好剑,果然好剑。

那穷汉却长叹了一声,道:``如此好剑,只可惜在我手里糟塌了\ldots{}''雷老大眼睛突然亮了起来,道:``不\ldots\ldots 不知朋友可\ldots\ldots 可有意出让?''那穷汉叹道:``虽然有意,怎奈难遇买主\ldots{}''雷老大大喜,喜动颜色道:``我\ldots\ldots 我这买主,你看如何?''那穷汉上上下下瞧了他几眼,颔首道:``看你们也有些英雄气概,也可配得上这口宝剑了,只是\ldots\ldots 你眼力既差,却不知出手如何?''雷老大喜道,``这个好说\ldots\ldots 这个好说\ldots\ldots{}''将他叁个朋友都拉在一边,叽叽咕咕商量了一阵,接着,只瞧见四个人都在掏腰包,凑银子。

那穷汉箕踞桌旁,瞧也不瞧,只是不住喝酒。

过了半晌,雷老大逡巡走过来,嗫嚅着道:``不知五百两''那穷汉眼睛一瞪,道:``多少?''

雷老大赶紧笑道:``不知一千两够不够,不瞒兄台说,咱们四个人掏空腰包,也只能凑出这么多了\ldots{}''那穷汉沉吟半晌,缓缓道:``此剑本是无价之宝,但常言说得好,红粉赠佳人,宝剑赠英雄\ldots\ldots 好,一千两卖给你也罢。''雷老大再也想不到他答应得如此痛快,生怕他又改变主意,赶紧将一大包银子双手奉上,陪笑道:``一千两全在这儿请点点。''那穷汉一手提了起来,笑道:``不用点了,错不了的\ldots\ldots 那。剑在这里,神兵利器,唯有德者佩之,你以后可要小心谦虚,否则这种神兵利器怕也会变顽铁\ldots\ldots{}''雷老大连声道:``是,是!\ldots\ldots{}''双手将剑接过,当真是大喜欲狂,如获异宝。

那穷汉从布袋里摸出锭银子,``咯''的抛在桌上,长长伸了个懒腰,打了个呵欠,笑道:``某家去了,这里的酒帐,全算我的''竟头也不回,迈开大步走了出去,那面色惨白的少年,瞧着雷老大等人一笑,也随后跟出。

这里雷老大已高兴得几乎忘了自己的生辰八字。

钱二笑道:``咱们雷老大得了这口剑,可当真是如虎添翼了,日后走江湖,还怕不是咱们雷老大的天下。''雷老大哈哈大笑道:``好说好说,这还不是各位兄弟捧场\ldots\ldots 哈哈,想来我雷老大只怕已时来运转,否则又怎能有此良缘巧遇。''钱二道,``雷老大有了这口剑,非但连燕南天都要大为失色,咱们镖局的总镖头,只怕也得让让贤了。''雷老大笑得满脸麻子都开了花,道:``日后咱家若真能如此,还能忘得了各位兄弟么?''他手里捧着那柄剑,坐也不是,站也不是,当真是含在口里怕化了,顶在头上,又怕跌下。

突听有人笑道:``各位什么事如此高兴?''

笑声中,一个短小精悍、目光如炬的锦衣汉子,大步走了进来,他身材虽瘦小,但气派却不小,举手投足间,自有一般不凡之威傲,让人一眼瞧见,便知道此人平日必定发号施令惯了。

钱二等人俱都迎上来,躬身陪笑道:``总镖头\ldots\ldots{}''几个人七嘴入舌,将方才的奇遇说了出来。

那总镖头目光闪动,笑道:``真的么?那可当真是可喜可贺之事。''雷老大也早已陪笑迎了上去,但突然觉得自己得了这口宝剑,身份已是大不同了,是以又退了回来。

此番睥睨一笑,道:``总\ldots\ldots 沈兄说的好,这不过是小弟偶然走运而已。''他变得当真不慢,居然连称呼也改了,那沈总镖头却直如未觉,瞧着他微微一笑,道:``不瞒各位,如此利器,我倒真是从未见过,不知雷兄可能让我开开眼界。''雷老大哈哈笑道,``这个容易,沈兄一试便知。''沈总镖头道:``钱兄,请借剑一用。''

接过钱二的剑,微微挽了挽袖子,微笑道:``雷兄小心了。''话犹未了,``刷''的一剑削下,雷老大也想学那穷汉的模样,左手也端起酒杯,但酒杯刚端起,剑光已削下,他哪里还顾得喝酒,慌慌张张,反手一剑撩了上去。

又听``当、当、当、砰''四声响,却竟是雷老大的那柄``宝剑''!那第一声响是双剑相击,第二声响是剑尖落地,第叁声响是酒杯摔得粉碎,第四声响却是雷老大整个人跌在地上。

这一来不但雷老大面如死灰,别的人也是目瞪口呆,一个个愣在那里,动弹不得,作声不得。

沈总镖头顺手抛了长剑,冷笑道:``这也算是宝剑么?''雷老大哭丧着脸,道:``但方才明明\ldots\ldots 明明是\ldots\ldots{}''沈总镖头冷冷道:``方才明明是你上了别人的当了。''雷老大突然跳了起来,大吼道:``我去找那□算帐\ldots\ldots{}''沈总镖头叱道:``且慢!''雷老大此刻又听话了,乖乖地停下脚步,道:``总\ldots\ldots 总镖头有何吩咐?''他又改了称呼,这沈总镖头还是直如不觉,只是冷冷问道:``方才那人是何模样?''雷老大道:``是个无赖穷汉,只不过生得高大些\ldots{}''沈总镖头沉吟半晌,突然变色道:``那人双眉可是特别浓重?骨骼特别大?一双眼睛平时永远半张半闭,仿佛有好几天未睡觉的模样。''雷老大道:``正是,总镖头莫非认得他?''

沈总镖头瞧了瞧他,又瞧了瞧钱二,突然仰天长叹了一声道:``只叹你们随我多年,不想竟还都是有眼无珠的瞎子。''雷老大哪里还敢抬起头来,只有连声道:``是\ldots\ldots 是\ldots\ldots{}''沈总镖头道:``你们可知道此人是谁么?''众人面面相觑,齐声道,``他是谁?''

沈总镖头一字字缓缓道:``他便是当今江湖第一神剑,燕南天!也就是我此番专程来拜见的人!''话未说完,雷老大已又一个跟斗栽在地上!

那面色惨白的青衣少年跟着走出,两人大步而行,走尽长街,少年方自追上去,悄声道:``是燕大爷么?''燕海天龙行虎步,头也不回口中沉声道:``你可是我江二弟差来的?''那少年道:``小人正是江二爷的书童江琴\ldots{}''燕甫天霍然回首,厉声道:``你怎地此时才来?''他双目一张,那目光当真有如夜空中击下的闪电一般,那江琴竟不由自主打丁个寒噤,垂手道:``小人\ldots\ldots 个人生怕行踪落在别人眼里,是以只敢在夜间行事,而\ldots\ldots 而小人虽从小跟着公子,轻身功夫却可怜得很。''燕南天神色大见和缓,又缓缓垂下眼,道:``你家公子令人送来书信,要我在此相候,信中却不说明原因,便知其中必有极大的隐密\ldots\ldots 这究竟是什么事?''江琴道:``我家公子不知为了什么,突然将家人全都遣散了,只留下小人,然后又令小人到这里来见大爷,请大爷由这条废道上去接他,有什么话等到当面再说,看情形\ldots\ldots 我家公子似乎在躲避着什么强仇大敌。''燕南天动容道:``哦?有这等事!他为何不早说?\ldots\ldots 唉,二弟做事总是如此糊涂,纵是强仇大敌,我兄弟难道还怕了他们!''江琴躬身道:``大爷说的是。''

``你家公子已动身多久?''

``计算时日,此刻只怕已在道上。''

``你本该早些进来才是,万一\ldots\ldots{}''突听有人大呼道:"燕大侠\ldots\ldots 燕大侠\ldots\ldots。

几个人急步奔了过来,当先一人,身法矫健,步履轻灵,自然正是那精明强悍的沈总镖头了。

燕南无微微皱眉,沉声道:``来的可是威远、镇达、宁远叁大镖局的总镖头,江湖人称飞花满天,落地无声的沈轻虹么?''沈轻虹躬身拜道:``不敢,正是小人\ldots\ldots 弟子们有眼无珠,不认得燕大侠\ldots\ldots{}''燕南天大笑道:``我听得他们竟敢说要请诗仙喝酒,便觉有气,但瞧在你家镖主面上,也不能揍他们一顿,若不取他们几文银子,怎出得了气?''沈轻虹躬身道,``是,是,原是他们该死\ldots{}''燕南天笑声突顿,道:``你可是来寻我的。''``晚辈正是专程前来拜见燕大侠。''燕南天厉声道:``你怎知我在这''``晚辈正值走投无路,幸得一位前辈的指点,说是燕大侠这两天必在此间等人,是以晚辈才赶来。''燕南天展颜笑道:``原来又是那醉鬼多口\ldots\ldots{}''转眼一望,望见了垂头丧气,站在那里,手里还提着那半截锈剑的雷老大,不禁又笑道:``想来你此刻心里还糊涂得很。''雷老大垂首道:``晚辈\ldots\ldots 这口剑\ldots\ldots 实在\ldots\ldots{}''沈轻虹叱道:``你还要丢人现眼,你莫非不知道燕大侠掌中无剑,亦胜有剑,无论什么顽铁,到了燕大侠手里,也成了削铁如泥的利器!''燕南天笑道:``你如此捧我,想必有求于我。''沈轻虹叹道:``不瞒前辈,晚辈接着一票红货,价值可说无法估计,此事本做得十分隐秘,哪知不知怎地,这风声竟走漏到十二星相''的耳里,竟令人送来星辰贴``,明言劫镖,晚辈自然不敢再走镖上路\ldots\ldots{}''燕甫天道:``你莫非是要我来为你保镖不成?''``晚辈不敢\ldots\ldots 晚辈知道前辈在此,是已将十二星相约在附近,只求前辈抽空一行,只要前辈吩咐两句,''十二星相纵有天大的胆子,想必也再不敢来打这票红货的主意\ldots{}``燕南天沉声道:''你既无力护镖,为何又要接下?"``晚辈该死,只求前辈\ldots\ldots{}''``十二星相恶名久着,若非他们行踪委实隐秘,我早已将之除去,此事我本非不愿出手助你\ldots\ldots{}''沈轻虹大喜道,``多谢前辈\ldots{}''燕南天道:``你莫谢我,我虽有心肋你,怎奈我此刻却另有急事,那是片刻也延误不得的\ldots{}''话犹未了,便待转身。

沈轻虹惶声道:``前辈留步。''

挥了挥手,钱二已送上了箱子,箱子里竟满是耀眼的黄金,沈轻虹躬身再拜,恭身道,``晚辈久已知道前辈挥手千金,是以送上\ldots\ldots{}''燕甫天仰天狂笑,厉声道:``沈轻虹,你纵将天下所有的黄金都送到我面前,也不能将我与二弟相见的时候耽误片刻\ldots\ldots{}''伸手一拍江琴肩头,喝道:``我先去了,你跟着来!''八个字说完,人已远在十丈外!沈轻虹面色立刻如土,钱二喃喃道:"这人倒当真奇怪,几十两银予,他也要骗,但别人真送上巨额黄金时,他却又不要了。

\hypertarget{ux7b2cux56dbux7ae0-ux8d64ux624bux6b7cux9b54}{%
\chapter{第四章
赤手歼魔}\label{ux7b2cux56dbux7ae0-ux8d64ux624bux6b7cux9b54}}

暮霭苍茫。

苍茫的暮色中,燕南天的身形,几乎已非肉眼所能分辨,他身形掠过时,最多也不过只能见到淡淡的灰影一闪。

旧道上荒草漫漫,迎风飞舞,既不闻人声,亦不闻马蹄,天畔新月升起,月光也不见掩去这其间的萧索之意。

燕南天身形不停,口中喃喃道:``奇怪,二弟已在道上,我怎地听不见\ldots\ldots{}''突见眼前黑影一闪,两点黑影,飞了过去,月光下瞧得清楚。

前面飞的是弱燕,后面追的却是只苍鹰。

那燕子似已飞得力竭,双翼摆动,已渐缓慢,那苍鹰雄翼拍风,眼见已将追及,燕子已难逃爪下。

燕南天喝道,``兀那恶鹰,你难道也做人间恶徒一般,欺凌弱小\ldots\ldots{}''只觉一股怒气直冲上来,身子一拧,竟箭一般向那苍鹰射了出去。

那苍鹰双翅一展,燕南天便扑了个空。

只听燕子一声哀啼。

已落入苍鹰爪下,苍鹰得志,便待一飞冲天,燕南天怒喝一声道:``好恶鹰,你逃得过燕某之手,算你有种!''喝声中,他身形再度窜起,一股劲风,先已射出,那苍鹰在空中连翻了几个跟斗,终于落了下来。

燕南天哈哈大笑,道:``二弟呀二弟,你瞧瞧我赤手落鹰的威风!''身形展动,接住了苍鹰,自鹰爪中救出了弱燕。

但燕子受伤不轻了,竟已再难飞起,燕南天喃喃道:``好燕儿,乖燕儿,忍者些,你不会死的\ldots\ldots{}''在长草间坐了下来,自怀中取出金创药,轻敷在燕子身上。

燕南天轻轻敷药,小心呵护,过了半盏茶时分,那燕子双翅已渐渐能在燕南天掌中展动。

燕南天嘴角露出笑容,道:``燕儿呀燕儿,你已耽误我不少时候,你若能飞,就快快去吧。''那燕子展动双翅,终于飞起,却在燕南天头上飞了个圈子,才投入暮色中。

燕南天大笑道:``万两黄金,不能令我耽误片刻,不想这小燕子却能拖住我了。''开怀得意的笑声中,他再次展动身形,如飞掠去。

突然间,一阵洪亮的婴儿啼哭声,远远传了过来。

燕南天大喜道:``莫非二弟已有了宝宝?''

他身形更急,掠向哭声传来处,于是,那满地的尸身,那惨绝人寰景象,便赫然呈现在他眼前!燕南天身形早已不见,甚至连那江琴都已去远了,但沈轻虹还是木立在那里,动弹不得。

钱二嗫嚅着道:``不知总镖头和那''十二星相约在何时,``沈轻虹道:''就是今日黄昏钱二变色道:``今晚?\ldots\ldots 在哪里?''``就在前面!''

``他\ldots\ldots 他们有多少人?''

``星辰贴上具名的,乃是黑面、司晨、献果、迎客、偷泉\ldots{}''``难\ldots\ldots 难道,鸡、猪、猴、狗一齐出手?''``不错!''

钱二声音早已变了,颤声道,``总镖头,咱们还是走吧,凭咱们,只\ldots\ldots 又怕\ldots\ldots{}''沈轻虹冷哼道:``你们走吧''``总镖头你\ldots\ldots{}''``镖主以义待我,沈经虹岂能无义报之,你们\ldots\ldots{}''突然顿住语声,头也不回大步走去钱二呼道:``总镖头\ldots\ldots{}''追了一步,又复驻足雷老大道:``怎么?你不去么?''钱二悄声道,``让他从容就义去吧,咱们可犯不着去送。''雷老大勃然变色,怒骂道:``畜牲\ldots\ldots 你们作畜牲,我雷啸虎可不能陪你们作畜牲。''钱二道:``好,好,我是畜牲,你是义士\ldots{}''雷啸虎:``畜牲,畜牲,我今日才算认得你们\ldots\ldots{}''一路大骂,一路追了过去。

沈轻虹缓步而行,走向暮色笼罩的荒野,他轻灵的脚步,已变得十分沉重,每走一步,脚下都似有千钧之物。

听得身后有脚步赶来,他头未回,道:``是雷啸虎么?''雷啸虎道:``总镖头,是我\ldots{}''沈轻虹叹道:``我早已知道只有你一人会来的\ldots{}''``听总镖头这句话,雷啸虎死也甘心,我雷啸虎虽然是呆子,却非无耻的畜牲,但\ldots\ldots 但总镖头,你\ldots\ldots 你这次\ldots\ldots{}''``你是奇怪我为何不多约人来么?''``正是有些奇怪\ldots{}''``十二星相,各有奇功,江湖友辈中能胜过他们的人并不多,我若约了朋友,别人为了义气虽想不来,也不能不来,但我又怎忍心令朋友们为难,送死?''雷啸虎仰天长啸道:``总镖头毕竟是总镖头,我雷啸虎纵然有总镖头这样的武功,也休想能做得上叁大镖局的总镖头,我''话犹未了,突听一声狗吠。

荒郊黄昏,有狗吠声,本非奇事,但这声狗吠却分外与众不同,这狗吠声竟似有种妖异之气。

雷啸虎耸然失色道:``莫非来\ldots\ldots{}''``了''字还未出口,满镇狗吠,已一声连着一声响了起来,眨眼之间,两人耳中除了狗吠外,已听不到别的声音。

雷啸虎平日胆子虽大,此刻手足却也不禁微微发抖,但瞧见沈轻虹神色竟未变,他也壮起胆子,强笑道:``这十二星相果然邪门\ldots\ldots{}''沈轻虹沉声道:"十二星相专喜诡异,为的却是先声夺人。

先寒敌胆,他们确实被他骇住了,便折了锐气!

雷啸虎挺起胸膛,大声道,``我不怕,谁怕谁就是孙子!''他口中虽说不怕,其实声音也有些岔了,月夜荒郊,这狗吠如哭;如狼嚎,的确摄人魂魄!沈轻虹双拳微抱,朗声道:``十二星相在哪里?洛阳沈轻虹前来拜见!''他身形虽瘦小,但此刻的声音竟自狼嗥鬼哭般时狗吠声中直穿了出去,一个字、一个字传送到远方。

苍茫的暮色中,突然跃出团黑影,骤见仿佛一人一马,却是只金丝猿猴骑在只白牙森森的大狼狗上。

这只狗,虎躯狗头,竟比平常狗大了一倍,喉中不断发出低吼,已足令人丧胆,这只金丝猿更是火眼金睛,目光中带着种说不出的妖异之气,一猴一狗,竟仿佛不是人间之物,而是来自妖魔地狱。

等这一猴一狗走过来,金丝猴``吱''的一叫,突然将只桃子送到地面前。

沈轻虹冷笑道:``好一个神犬迎客,灵猴献果,但是沈轻虹会的是十二星相中的人,却不是这些畜牲!''那金丝猿仿佛懂得人言,``吱''的又是一叫,凌空在狗背上翻了个筋斗,手中竟然又多了条白条,上面写者:``你若敢吃下去,自有人来会你。''沈轻虹冷笑道:``十二星相若是见不得人的鼠辈,沈轻虹今日也不会来了\ldots\ldots 沈轻虹信得过你们,纵是毒药,也要吃下!''他方待伸手去拿桃子,哪知雷啸虎却抢了过来,叁口两口连桃核都吞了下去,大笑道:``不要钱的桃子,不吃岂非冤枉!''只听一人阴森森笑道:``好,无怪叁远镖旗能畅行大河两岸,镖局中果然还有两个有胆子的好汉\ldots\ldots{}''八条人影,随着笑声走了出来。

沈轻虹身形已算十分瘦小,但此刻当先走出的一人,却比沈轻虹还瘦,身上穿着件金光闪闪的袍子,脸上凸颧尖腮,双目如火,笑起来嘴角几乎直裂到耳根,此人若还有叁分像人,便也七分是猴的模样。

另外六七人却全是黑衣劲装,黑巾蒙面,只露出一双闪闪的眼睛,宛如鬼眼瞅人。

沈轻虹道:``来的想必是\ldots\ldots{}''那金袍人喀咯笑道,``咱们的模样,你自然一瞧就知道,还用得着说么?''沈轻虹冷笑道:``在下只是奇怪,怎地少了黑面君与司晨客了''金猿星怪笑道,``他两人去做另一票买卖去了,有我们这几人,你还嫌不够么?''沈轻虹朗声大笑道:``沈轻虹今日反正是一个人来的,反正已没打算活者回去,能多瞧见几位十二星相的真面目,固然不错,少瞧见几个,也不觉遗憾。''金猿星狞笑道:``我知道你胆子不小,却不知道你口才竟也不错,但你辛辛苦苦爬上总镖头的宝座并不容易,死了岂非冤枉?''沈轻虹厉喝道:``沈轻虹此来并非与你逞口舌之利。''``你想打?''``正是!沈某若胜,只望各位休想再打镖货的主意\ldots\ldots{}''``败了又如何了?将镖货双手送上么?''沈轻虹哈哈大笑道:``那批红货早已由我家副总镖头双鞭宋德扬加急送上去了,沈某此来,不过是声东击西,调虎离山而已\ldots{}''金猿星抬了抬手,身后的黑狗星立刻送上个小小的檀木匣子。

金猿星打开匣子,阴森森道:``你瞧瞧这是什么!''匣子里的,竟赫然是颗人头!``双鞭''宋德扬的人头!沈轻虹面容惨变,嘶声道:``你\ldots\ldots 你竟\ldots\ldots{}''金猿星喀喀大笑道:``十二星相若是常常被骗的人,江湖中人也不会瞧见咱们那么头疼了\ldots\ldots 老实告诉你,那批红货,早已落入咱们手中,咱们此来,只不过是要你的命罢了。''突又挥了挥手,呼啸道:``上去!''一声呼啸,那金丝猿已凌空跃了起来,扑向沈轻虹,一双猿爪,闪电般直取沈虹双目!那巨大却厉吼着扑向雷啸虎,雷啸虎惊吼闪避,哪知这巨犬身子虽大,动作却出奇灵敏,一掀,一剪!雷啸虎竟再也闪避不及,生生扑倒在地,只见一排森森白牙,直往他咽喉咬了过去!雷啸虎拼命抵住狗颚,一人一狗,竟在地上翻滚起来,狗嗥不绝,雷啸虎吼声也不绝,他竟似也变成野兽!那边沈轻虹已攻出数招,但那金丝猿却是纵跃如飞,一双金光闪闪的爪子,始终不离沈轻虹双目叁寸处!金猿星怪笑道:``不想叁远镖局的大镖头们,竟连两只畜牲也打不过!''语犹未了,突见沈轻虹伸手一探,一条九尺银丝长鞭,已在手中,满天银光洒起,金丝猿立被迫退。

沈轻虹厉叱道;``哪里走!''

数十点银星,突然自那满天银光中暴射而出,小半射向那金丝猿,却有大半击向那金猴黑狗,那金丝猿虽然通灵,究竟是个畜牲,怎能避得过这大河两岸最有名的镖客所发出的杀手暗器\ldots 银星击出,这灵猿便已惨嗥倒地。

一余猿,七黑狗,八条人影,却已冲天飞起。

金狼星大喝道:``好个飞花漫天,果然有两下。''于是八条人影,全都向沈轻虹扑下,沈轻虹纵有叁头六臂,也是敌不过这八人凌空击下的一着!只见他身形就地一滚,银鞭护体,化做一团银光滚了出去,但金猿黑狗却已占得先机,他还能往哪里走?

那边巨犬已一口咬住雷啸虎的肩喉处,雷啸虎也一口咬住巨犬的咽喉,鲜血满地,一人一犬都在在血泊中,就在这时,突听一声惊天动地的怒喝声,宛如晴天霹雳,一人凌空飞坠,宛若雷神天降!众人齐被这喝声震得心魂皆落,金猿黑狗俱都住手,只见一条大汉,身长八尺,头发蓬乱,一双精光四射的虎目中,满布血丝,面上那悲愤之色,已足以令任何人心寒,那神情之威猛,更足以令任何人胆碎,但奇怪的是,这大汉身后却背着个襁褓婴儿!沈轻虹亦是满身浴血,此刻狂喜呼道:``燕大侠来了!''金猿星变色道:``莫非是燕南天!''

燕南天厉喝道:``十二星相,你们的死期到了''金猿星道:``十二星相与你无冤无仇,你为何\ldots\ldots{}''他话还没说完,燕南天已冲了过来,一条黑犬首当其冲,大惊之下,双拳齐出,急如电闪,``砰、砰''两拳,俱都打在燕南天胸膛上,但燕南天丝毫不动,那黑大双腕却已生生折断!惨呼一声尚未出口,燕南天铁掌已抓住他胸膛,他情急反噬,拼死一脚飞出。

这一脚乃是北派``无影腿''的真传,当真是来无影,去无踪,但不知怎地,这无影无踪的一脚,此刻竟被燕南天一伸手就抓住了,只听一声霹雳般大喝,那黑犬星一个人已被血淋淋撕成两半!鲜血射出,落花般沾满了燕南天的衣服。

黑狗群的眼睛红了,惊呼,怒吼,纷纷扑了上去。

这七人一个个分开来,武功还算不得是一流高手,但七人久共生死,练得有一套联手进击的武功,却是非同小可,此刻七个人虽只剩下六个,但招式发动开来,仍是配合无间,滴水不漏。

沈轻虹忍不住脱口轻呼道:``燕大侠小心了。''呼声未了,燕南天身子已冲了进去,竟有如虎入羊群一般,掌中两片尸身,化做满天血雨!六个人已倒下五个。

剩下的最后一人瞧着燕南天不备,突然,向他背后背着的那婴儿扑了过去,自是想抢得婴儿作为人质。

哪知燕南天背后似生着眼睛,虎吼道:``站住!''燕甫天手里剩下的半片尸身,已向他当头摔了下来。血雨纷飞,洒得满头满脸,他灵魂早已出窍,竟骇得忘了闪避,那半片尸身已如万钧铁锥般摔在他头上。

他整个人竟像是铁钉般被钉得短了一半!沈轻虹全身寒毛都一根根竖了起来,那金猿星虽是杀人如草芥的党徒,此刻却也被这股杀气惊得呆了。

燕南天喝道,``你还要某家动手不成?''

金猿星道:``你\ldots\ldots 你为什么?\ldots\ldots{}''燕南天怒吼道:``为什么?你可知江枫是某家的什么人?''金猿星失声道:``莫非那\ldots\ldots 那只猪已\ldots\ldots{}''燕南天:``别人都已死了,你活着又有何趣味,纳命来吧!''最后一个字说完,人已到了金猿星面前,铁掌已抓住了金猿星的胸膛。哪知金猿星竟是动也不动,也不回手。燕南天手掌一紧,五指俱都插人金猿星肉里。金猿星竟还是挺胸站在那里哼都未哼一声。燕南天道:``不想你个子虽小,倒还是条汉子,若是换了平日,某家也能饶你一命,但今日\ldots\ldots 哼,你还有何话说?''金猿星却突然仰天狂笑起来,狂笑着道:``你个子虽大,却也算不得是大丈夫。''燕南天不禁怔了一怔,喝道:``某家这一生行事,虽得天下之名,却也有不少人骂我,善恶本不两立,那也算不得什么,但你这这句话,某家倒要听听你是凭什么说出来的。''金猿星冷笑道:``是非不明,恩仇不辨,算得了大丈夫么?''燕南天怒道:``某家\ldots\ldots{}''金猿星大声截道:``你若是明辨是非之辈,便不该杀我。''燕南天道,``为何不该杀你?我二弟江枫\ldots\ldots{}''金猿星再次大声截止道:``这就对了,你若为别的事杀我,那我无活可说,但你若为江枫杀我,你便是不明是非,不辨恩仇。''燕南天怒道:``你十二星相难道未''金猿星道:``不错,十二星相确曾向江枫出手,但十二星相本是强盗,这一点你早已知道,强盗要劫人钱财,本是份内之事,既是份内之事便算不得什么深仇大恨,那前来通风报讯,要十二星相向江枫出手的,才是你真正要复仇的对象,你可知道。''``他是谁么?''他侃侃而言,居然像是理直气壮,燕南天虽是满腔怒火,此刻也不禁被他说得怔了怔。

突然大喝道:``前来通风报讯的,莫非是江琴那个畜牲?我二弟之行程,只有那小畜牲一个人知道\ldots{}''金猿星面色微变,但瞬即冷笑道:``不错,原来你非但四肢发达,头脑也不简单,江枫的确是被他视为心腹的人卖了,叁千两银子就卖了。''燕南天目□尽裂,嘶声道:``畜牲\ldots\ldots 畜牲\ldots\ldots{}''金猿星冷冷道:``那畜牲此刻在那里,你可知道?''燕南天突然一只手将金猿星整个人都提了起来,嘶声道。

``你知道他在哪里,是么?''

金猿星神色不变,缓缓道,``我若不知道,这些话就不说了\ldots{}''燕南天吼道:``他在哪里?说!''金猿星身子虽被他恳空提着,但神情却比站在地上还要笃定,瞧着燕南天微微一笑。

燕南天瞧着他那张微笑的脸,一字字缓缓道:``你若不说,我佩服你\ldots{}''他若说要把金猿星宰了,剁了,大卸八块,金猿星仍不害怕,因为金猿星明知他还未打听出江琴的下落之前,是绝不会将自己杀死的,但此时他说的是这句话,金猿星却不由自主打了个寒噤,道:``我\ldots\ldots 我说了又如何?''燕南天道:``你说了,我便挖出你一双眼睛!''沈轻虹听得几乎失声叫了出来,暗道:``这燕南天怎地如此不解人情,人家说了,他还要挖人眼睛,这样一来,金猿星想必定万万不肯说出来的了\ldots{}''哪知他心念还末转变,金猿星已长长叹了口气,道:``虽然没有眼睛,但只要能活着,也就罢了。''燕南天道:``说吧!''

金猿星道:``只要我说出了,你也未必敢去。''燕南天怒道:``普天之下,还没有燕某不敢去的地方!''金猿星眼睛半睁半闭,脸上似笑非笑,缓缓道:``那江琴不是呆子,明知我十二星相杀人不过如同睬死只蚂蚁,他拿了十二星相的银子,难道不怕脑袋搬家?他如此大胆,只因他早已有投奔之地,拿这银子,正是要用做路费。而他那投奔之地,十二星相加在一起,也不敢走近那地方半步。''燕南天厉声狂笑道:``移花宫?\ldots\ldots 某家正要去的。''金猿星道:``当今天下,也未必只有移花宫是武林禁地。''``除了移花宫还有哪里?''

``吕仑山恶人谷\ldots\ldots{}''他这六个字还只说出五个,站在一旁出神倾听的沈轻虹,便神色大变,身子也已颤抖,大声道:``燕大侠,你\ldots\ldots 你去不得!''燕南天须发皆张,日光逼视金猿星,厉声道,``你说的可是真话?''``我话已说出,信不信却由得你了。''

沈轻虹颤声道:``那恶人谷''乃是天下恶人聚集之地,那些人没有一个不是十恶不赦、满手血腥,没有一个不是被江湖中人恨之入骨,但那许多恶人聚在一起,别人纵然恨不得吃他们的肉,也没有人敢走近恶人谷一步,就连``昆仑七剑.少林四神僧、江南剑客风啸雨,都也\ldots\ldots 也不敢\ldots\ldots{}''燕南天沉声道:``燕南天既非少林神僧,也非江南剑客!''沈轻虹道:``我知道燕大侠你剑术当代无双,但那恶人谷\ldots\ldots 那谷中成千成百,也不知究竟有多少恶人\ldots\ldots{}''燕南天大喝道:``义之所在,燕某何惧赴汤蹈火。''沈轻虹大声道:``但说不定这根本是金猿星故意骗你的,他已对你恨之入骨,所以要你到那恶人谷去送\ldots\ldots 送\ldots\ldots{}''他虽未将``死''字说出口来,其实也等于说出了一样。

燕南天仰天笑道:``恶人谷纵是刀山火海,也未必能要了燕南天的命!''沈轻虹怔了一怔,苦叹一声,黯然无语。

金猿星亦自叹道:好!燕南天果然是英雄!竟连恶人谷也敢闯上一闯,你此去纵然有去无还,也必将博得天下武林佩服!``燕南天道:''你还有何话说?"

金猿星道,``没有了,拿我的眼珠去吧!''

一声惨呼,金猿星一双精光四射的火眼,已变成两个血窟窿,燕南天随手将他抛在沈轻虹面前,道:``此人交给你了!''话声未了,人已去远。

那雷啸虎横卧在血泊中,身子下压者那条巨犬,一人一犬,都已奄奄一息,连指头都不会动了。

沈轻虹瞧了瞧他,目光移向金猿星,恨声道:``你金猿星纵然一世聪明,今日却做了件笨事。''金猿星方才虽已疼得昏过去,片刻却已醒来,就像是有鬼在后面推着他似的,他竟能忍住疼,自怀中摸出一包药,塞在眼眶中,口中还能说话,颤声道:``我笨?''``燕南天虽未取你性命,但将你送到我手中,我还会饶你?\ldots\ldots 你此刻纵有灵药治伤,又有何用!''``自然有用,我死不了的!''

``还有谁能救你?''

``我自己。''

``沈某倒要瞧瞧你如何能救你自己\ldots\ldots{}''喝声中,手拿直拍金猿星天灵。

金猿星大声道:``那镖银你不想要了么?''

沈轻虹手掌立刻在空中顿住。

金猿星咬紧牙关,喀喀大笑道:``我早就算准你不敢动手杀我的,你若想要镖银,只有我能给你,除非你有这胆子去骗!''沈轻虹手掌不停颤动,几次想要击下,几次都顿住,终于长长叹息了一声,收回手掌,道:``算你赢了,''这一批镖银委实关系整个叁远镖局的命运,沈轻虹一生从不负人,又怎能负对他义重如山的叁远镖局?金猿星疯狂般笑道:``沈轻虹,如今你可知道了吧!无论谁想杀我,都没有那么容易!''夜色已深,小镇上灯火阑珊,就连那``太白居''中的酒鬼,都已踉跄着脚步,互相携扶着散步去了。

那酒保揉者发红的眼睛,正待上起店门突然间,只见一辆马车自街头走过来,拉车的却不是马,而是个人──正是那骗了人家一千两银子的大汉。

自门里透出来的昏黄灯光中望未,只见这大汉满身鲜血,满面杀气,看来有几分似恶鬼,又有儿分似天神!这酒保骇得脸都白了,方自躲回去,这大汉已拉着车到了门口,要两匹马才拖得动的大车,在他手里,竟似轻若无物。燕南天将大车靠在墙上,怀抱熟睡的婴儿大步走进店里,那店伙壮起胆子,陪笑道:``大\ldots\ldots 大爷要\ldots\ldots 要什么酒?''燕南天眼睛一瞪,喝道:``谁说我要酒?''酒保怔了怔,道:``大爷不\ldots\ldots 不要酒,要什么?''燕南天道:``米汤!''

酒保更怔住了,苦着脸道,``小店不\ldots\ldots 不卖\ldots\ldots{}''燕南天``叭''的一拍桌子,大声道:``先去煮几碗浓浓的米汤,再拿酒来。''这酒保骇得胆子都快破了,哪里还敢说"不字。

婴儿喝了米汤,睡得更沉了,燕南天喝着酒,目中神光却更惊人,那酒保连瞧也不敢瞧他一眼。

虽然不敢瞧,却偷偷数着──不到盏茶时分,燕南天已用大碗喝下了十七碗烈酒!那酒保骇得吐出了舌头,几乎缩不回去。

突见燕南天摸出两锭银子,抛在桌上,大声道:``去替我买些东西来\ldots{}''``大\ldots\ldots 大爷要买什么?''``棺材!两口上好的棺材!''

那酒保骇得几乎一个筋头跌了下去,虽张开了嘴,却过了半晌还说不出话,他几乎不相信自己的耳朵。

燕南天又一拍桌子,两锭银子突然跳了起来,竟不偏不倚跳进他怀里,燕南天喝道,``棺材,两口上好的棺材,听到了么?''``听\ldots\ldots 听\ldots\ldots 听,\ldots{}''``听到了还不快去!''那酒保见了鬼似的,转身就跑,燕南天喝下第二十八碗酒时,他已乖乖地将棺材运了回来。

燕南天红着眼睛,自车厢中将江枫和月奴尸身捧出来,放入棺材里,每件事他都是亲手做的。他不许别人再碰他二弟一根手指。然后,以赤手钉起了棺盖他将一枚枚铁钉钉科木头里,就像是钉入豆腐里似的。

那酒保眼睛更发直了,也不知今天撞见的是神是鬼?面对棺木,燕南天又连尽七碗。

他没有流泪,但那神情\ldots 却比流泪还要悲哀。

手里端着最后一碗酒,他呆呆地站着,直过了几乎有半个时辰,然后,燕南天终于缓缓道:``二弟,我要你陪着我,我要你亲眼瞧着我将你的仇人一个个杀死!''夕阳满天,照着太原大街上最大的一面招牌,招牌上叁个大金字,闪闪发者光,这叁个字是:``千里香''``千里香''可真是金字招牌,山西人个个都知道,``千里香''卖出来的香料,那是绝不会有半分掺假的。

黄昏后,``千里香''铺子里十来个伙计,正吃着饭,大街上行人熙来攘往,正是是热闹的时候。

突然一辆大车直驰而来,驶过长衔,赶车的一声吆喝,宛如霹雳,这大车已笔直闯入``千里香''店铺里。

伙计们惊怒之下,纷纷扑了过来,只见那赶车的大汉一跃而来,也不知怎地,十来个伙计但觉身子一麻,全都不能动了;眼睁睁瞧着他将一坛坛上好的香料,全都塞到两口棺材里去。片刻后那大汉便又赶着车子急驶而出,口中喝道:``半个时辰后你等便可无碍,香料银价,来日加倍奉还!''大街上的人,竟都被这大汉的神气所慑。满街人竟没有一人敢拦住这辆马车。

下午,瓜田里散发出象征着丰收的清香。

一个农家少妇。懒洋洋的坐在瓜田旁,树荫下。她半敞着衣襟,露出了那比瓜田里的瓜还要成熟的胸膛,正以比瓜汁还甜的乳什,喂着怀抱中的婴儿。凉风入怀,她似乎已要睡着了。迷迷糊糊中,她似乎觉得有双眼睛在盯着她的胸膛。农村中本也有不少轻薄的小伙子,她平日也被人瞧得不少,儿子都有了的人,哪里还会在乎这些,但此刻,她和觉得这双眼睛似是分外不同。她不由自主张开了眼,又见旁边一株树下,果然有个陌生的大汉,这大汉身躯并不甚雄壮,衣衫也不甚堂皇,面目间更带着几分憔悴之色,但不知怎地,看来却威风得很。奇怪的是这条大汉,怀里却抱者个婴儿。这少妇虽觉得有些奇怪,也不理会,又自垂下了头,只听那大汉怀抱中的婴儿,突然啼哭起来,哭声倒也洪亮。

她才做妈妈没多久,心中正充满了母性的温柔,听得这哭声,忍不住又抬起头,这一次她便发觉那大汉盯着她胸膛的那双眼睛里,并没有什么色迷述的神情,却充满恳求之意,不禁一笑,道:``这孩子的娘不在么?''那大汉摇头道:``不在''少妇沉吟半晌,道:"看来他是饿了。

那大汉点头道:``是饿了。''

少妇瞧了瞧自己怀中的婴儿,突然笑道:``把你的孩子抱过来吧,我来喂他,反正这几天我吃了两只鸡,奶水正足,咱们小妞儿也吃不了。''那大汉立刻露出喜色,赶紧道:``多谢。''将孩子抱了过去。

只见这孩子胎毛未落,出生最多也不过几天,那细皮嫩肉的小脸上,却已有了条刀痕。那少妇不禁皱眉道:``你们带孩子真该小心些,这孩子的娘也真是,竟放心把这么小的孩子交给你一个大男人''那大汉惨然道:``这孩子的娘已死了。''少扫楞了一楞,伸手抚摸者这孩子的小脸,黯然叹:``从小就没有娘的孩子,真是可怜。''那大汉仰天长长叹息了一声,垂目望向孩予,心里也正有说不出的悲哀,说不出的怜惜。

这孩子生来似乎就带着噩运,初生的第一天,就遇着那么多凶杀、死亡,他这一生的命运,似乎也注定要充满灾难,可怜他什么也不知道,此刻,他那张小脸上,反似充满了幸福的微笑。

\hypertarget{ux7b2cux4e94ux7ae0-ux6076ux4ebaux4e4bux8c37}{%
\chapter{第五章
恶人之谷}\label{ux7b2cux4e94ux7ae0-ux6076ux4ebaux4e4bux8c37}}

玉龙哈什河翻滚的河水,在六月的残阳下发着光。

到了上游,河水双分,东面的一支便是玉龙哈什河,水流处地势更见崎岖险峻,激起了奔腾的浪花。

沿着玉龙哈什河向上游走,便入了天下闻名、名侠辈出、充满了神秘传说的昆仑山区。

此刻,虽仍是夏季,残阳余犹未落,玉龙峰下,已宛如深秋,风在呼号,却也吹不开那阴森凄迷的云雾。

燕南天终于来到了玉龙烽下,人既憔悴马也疲乏,就连车轮滚动在崎呕的山路上,也似乎滚不动了,巨大的山影,沉重地压在车马上。

燕南天左手提着疆绳,右手怀抱着婴儿,一阵阵恼人的香气自车厢中传出来,刺得他几乎想吐。

婴儿却又已沉睡了,这小小的孩子,竟似也已习惯了奔波困苦。

燕南天无限怜惜地瞧着他,嘴角突然现出一丝微笑,喃喃道:``孩子,这一路上你可真是吃了不少人的奶,从中原,一路吃到这里,除了你,大概没有别的孩子能\ldots\ldots{}''说到``能''字,语声突热顿住,身子也突然凌空跃起,就在他身子离开车座的那一刹那间,只听``笃、哧、噗''十几声响,十几样长短不齐、形式各异的暗器,俱都钉入了他方才坐过的地方。

燕南天凌空翻身,左手已勒住了车马,人却藏到了马腹下,他怕的不是自己受伤,而是怀抱中的婴儿。

这一跃,一翻,一翻一勒,一藏,当真是矫如游龙,快若惊鸿,山麓阴影中,已有人忍不住失声道:``好功夫!''燕南天怒喝道:``暗箭伤人的是\ldots\ldots{}''``谁''字还未出口,那匹马突然惊嘶、一声,人立而起,马身上箭也似的喷出了十几股鲜血。

燕南天想也不想,铁掌扫出,``砰、砰''两响,套马的车轭立断,负伤的马,笔直窜了出去,燕南天跟着又是一拳击出,又是``砰''的一响,车厢生生被击破个大洞,健马长嘶未绝,他左手将婴儿自洞中送到车厢里去,又是数十点寒光,已暴雨般射向他身上!他身子也已冲天而起,又听``哧、哧、哧'',风声不绝数十点暗器,俱都自他足底扫过。

应变若有丝毫之差,自己纵不负伤,那婴儿也难免丧命,婴儿纵不丧命,大车也难免要被那匹马带得自他身上辗过」健马倒地,燕南天身形犹在空中一只剑,银光乍起,七八道剑光,有如天际长虹般,自暑影中斜飞而出,上下左右,纵横交错。

哪知他身在空中,力道竟仍末消竭,双臂一振,身子突然又向上窜起了七尺,剑光又自他脚底擦过.但闻``叮铛''龙吟之声不绝,七八柄剑收势不及,俱都撞在一起,剑光一合便分,七人个人都远远落到一旁,暮色中虽瞧不清楚,但朦胧望去,这七八人中,竟有四个是出家的道人。

燕南天双足一蹬,方自掠到车顶,竟又箭也似的窜了出去双掌如风,当头向一个蓝衫道人击下!他眼见这几个人话也不说,便下如此毒手,此刻下手自也不肯留情,这双掌击下,力道何止千钧!那道人本待举剑迎上,但心念一转,面色实然大变,身形后仰,竟不敢招架,向后倒窜而去。"燕南天剑光竟似绵绵不尽,跟着身子追去。

那人心胆皆丧,拼命一剑迎上。

只听``叮''的一声,双剑相击,两口剑本是同炉所炼,但不知怎的,那人掌中的的剑,竟已被燕南天砍成两段。

那人身子落地,就地几滚,燕南天高吭长啸,剑光如雷霆闪电,直击而下,这一剑之威,当真可惊天动地!满天银光突又飞来,接着,``呛''的一声震耳龙吟,只见叁个蓝衣道人,单足跪地,叁柄剑交叉架起,替那人挡住了燕南天的一剑,那人却已骇得晕了过去!燕南天虎立当地,须眉皆张,厉声:``接剑的四鹫?还是叁鹰?''那道人道:``四鹫,足下怎知\ldots\ldots{}''燕南天厉声笑道:``当今天下,除了昆仑七剑外,还有几人能接得住某家这一剑?''那道人道:``当今天下,除了燕南天大侠外,只怕也再无一人能令贫道兄弟叁人,同时出手抬架的一剑!''燕南天笑声突顿,喝道:``但昆仑七剑为何要向燕某下如此毒手,却令燕某不解\ldots{}''那道人苦笑道:``贫道等守在这里,本是为阻挡一个投奔恶人谷的人,贫道委实想不到燕大侠也会到这恶人谷来。''燕南天这才收回长剑,他长剑方自收回,那叁个道人掌中剑便已``当''的垂落在地,双臂似再也难以提起。

``你等要阻挡的人是谁?''

昆仑道人道,``司马烟。''

``你等怎知这恶贼要来此间''``川中八义一路将他追到这里,这叁位便是川中八义,中\emph{拇笠迨垦钇剑}士海长波,七义士海金波\ldots\ldots{}''``川中八义''在江湖中端的是赫赫有名,燕南天转目望去,只见这叁人果然风骨棱棱,气宇轩昂──虽然方自地上爬起,却无狼狈之态。

那川中八义之首杨平,国字脸,通天鼻,双眉斜飞人鬓,更是英气逼人,此刻微一抱拳,躬身道:``晚辈们直将那恶贼迫到和阗河畔,才将他追丢了,若是被他逃入恶人谷去,晚辈们实是心有不甘,是以才将四位道长请了出来,守在这里,哪知\ldots\ldots 哪知却\ldots\ldots 遇见燕大侠。''海长波苦笑道:``晚辈们方才虽已瞧出前辈形貌不同,但素知那□精于易容,晚辈们实将此人恨之人骨,是以\ldots\ldots{}''燕南天颔首道:``难怪你等出手那般狠毒,对付这恶贼,出手的确是越毒越好。''昆仑四子之首藏翼子忍不住问道:``但\ldots\ldots 但燕大侠却不知怎会来到这里。''燕南天道:``某家正是要到恶人谷去!''

昆仑四子,川中叁义齐地一怔。

藏翼子动容道:``燕大侠豪气干云,晚辈们久已深知,只是\ldots\ldots{}''恶人谷,恶人云集,古往今来,只怕从未有过那许多恶人聚在一起,更从未有一人敢孤身去面对那许多恶人,燕大侠\ldots\ldots 还望叁思。``燕南天目光火炬一般,遥注云雾凄迷的山谷,沉声道:''男儿立生于世,若能做几桩别人不敢做的事,死亦何憾!"昆仑四子对望一一眼,面上已有愧色。

杨平道:``但\ldots\ldots 据在下所知,这二十年来,在江湖中凶名最着的十大魔头,最少有四人确实已投奔谷中\ldots\ldots{}''海长波道:``只怕还不止四个\ldots\ldots 血手杜杀,笑里藏刀,小弥陀哈哈儿,不男不女屠娇娇,不吃人头李大嘴\ldots\ldots{}''燕南天皱眉道,``李大嘴?\ldots\ldots 可是那专嗜人肉的恶魔?''``\ldots 海长波道:正是那□,别人叫他不吃人头,正是说他除了人头外,什么都吃,他听了反而哈哈大笑,说他其实连人头都吃的''燕南天怒道,``如此恶徒,岂能再让他活着!''海长波道:``除了这四人外,那自命轻身功夫天下无双,从来不肯与人正面对敌,专门在暗中下毒手的阴九幽,据说也逃奔人谷\ldots{}''燕南天动穿道:``哦!半人半鬼阴九幽也在谷中么?他暗算少林俗家弟子李大元后,不是已被少林护法长老们下手除去了么?''海长波道:``不错,江湖中是有此一传说,但据深悉内幕之人言道,少林护法虽已将这半人半鬼的恶魔困在阴冥谷,底,但还是被他逃了出去,此事自然有陨少林派声威,是以少林弟子从来绝口不提。''燕南天长叹道:``昔日领袖武林的少林派,如今日渐没落,只怕就正是因为少林弟子一个个委实太爱面子。''藏翼子慨然道``要保持一派的声名不坠谈何容易?''他这话自然是有感而发──昆仑派又何尝不日渐凋零?杨平又道:``这几个无一不是极难对付的人,尤其是那不男不女屠娇矫,不但诡计多端,而且易容之术已臻化境,明明是你身畔最亲近的人,但说不定突然就变成了他的此身,此人之逃奔入谷,据说并非全因避仇,还另有原因\ldots{}''燕南天道:``无论他为了什么事逃人''恶人谷,无论他易容多么巧妙,反正某家此次入谷,乃是孤身一人,无论他扮成什么人的模样,都害不到我\ldots\ldots 哈哈,难道他能扮成出世不到半个月的婴儿不成?``杨平展颜笑道:''不错,此番燕大陕孤身入谷,他纵有通天的。

手段,只怕也是无所用其计了,但\ldots\ldots 不过\ldots\ldots{}``燕南天不等他再说话,抱拳道:''各位今日一番话,的确使燕某人获益匪浅,但无论如何,燕某人却是势在必行\ldots\ldots 燕某就此别过。``众人齐地脱口道:''燕大快,你\ldots\ldots{}``但燕南天却再也不瞧他们一眼一边挽过大车,立刻放步而''行。

"众人面面相觑,默然良久。

藏翼子终于叹道:``常听人言道燕南天武功之强强绝天下,贫道还不深信,但今日一见\ldots\ldots 唉,唉\ldots\ldots{}''杨平动容道:``他武功虽高,还不足深佩,小弟最佩服的乃是他的干云豪气,凛然大义,当真令我愧煞''。

海长波望者燕南天身段消失处,喃喃道:"但愿他此番人谷还能再出来与我等相见\ldots\ldots 山路更见崎岖,但燕南天拉着辆大车放足而行,竟仍毫不费。

力,他臂上何止有千斤之力?沉沉的暮色,凄迷的云天中,突然现出一席灯火。

那是盏青"灯制成的孔明灯,巧妙地嵌在山石间避风处在这阴冥的穷山恶谷中,碧磷磷的看来有如鬼火一般。

鬼火般的火照耀下,山石上竟刻着两行字!"入谷如登天。

来人走这边"两行字下,有只箭头,指着条曲折蜿蜒的山路,用尽目力,便可瞧出这条路正是通向四山合抱的山谷。

昆仑山山势虽险绝,但这条路却巧妙地穿过群山。

那``恶人谷''便正是群山围绕的谷底。

是以入谷的道路,非但不是向上,而且渐行向下,到后来燕南天根本已不必拉车,反倒似车在推他。

山路越来越曲折,目力难见一丈之外。

但突然间,眼前豁然开朗,四面穷山中,突然奇迹般现出了一片灯火,有如万点明星,眩人眼目。

江湖人心目中所想象的``恶人谷'',自然是说不出的阴森、黑暗,而此刻,``恶人谷''中竟是一片辉煌的灯火。

但这灯火非但未使``恶人谷''的神秘减少,反而使``恶人谷''更增加了说不出的诡异。

``恶人谷''中到底是什么情况?燕南天但觉自己的心,跳动也有些加速,这世上所有好人心中最大的秘密,此刻他立刻就要知道谜底了。

灯光下,只见,一方石牌立在道旁。

``入谷入谷,永不为奴。''

过了这石碑,道路突然平坦,在灯火下简直如镜子一般,光可鉴人,但燕南天却也知道,这平坦的道路正也是世上最最险恶的道路,他每走一步,距离危险与死亡便也近了一步。

没有门没有塔,也没有栏栅。

这``恶人谷''看来竟是个山村模样,一栋栋房屋,在灯火的照耀下,竟显得那么安静、平和。

在这安静平和的山村中,究竞藏有多少害人陷阱,多少杀人的毒手?燕南天挽着大车,已淌着汗珠,他此刻已入了``恶人谷'',随时都可能有致命的杀手向他击出!道路两旁,已有房舍,每一栋屋,建造得极精巧,紧闭的门窗中,送出明亮的灯火。

突然间,前面道路上,有人走了过来。

燕南天知道,就在这瞬息之间,便将有源源不绝的毒手,血战到来!哪知走过的两个人,竟瞧也未瞧他一眼,两人衣着都是极为华丽,竟扬长自燕南天身旁走过。

燕南天的眼睛都红了,也未瞧清他们的面容,只见道路上人已越来越多,但竟没有一个人瞧他一眼!他走入这天下武林中人视为禁地的``恶人谷'',竟和走入一繁华而平静的镇市毫无不同!燕南天脑中一片迷乱,反倒不知如何是好,他平生所遇的凶险疑难之事,何止千百,却从未有如此刻般心慌意乱!他平生所闯过的龙潭虎穴,也不知有多少,但不知怎地,无论多凶险之地,竟似乎都比不上这安静平和的``恶人谷''。

车厢中,有婴儿的啼哭声传了出来,燕南天深深吸了口气。

定下心神,便瞧见前面有扇门是开着的。

门里,似有酒菜的香味透出。

燕南天大步走了进去。

雅致的厅房中,摆着五六张雅致的桌子,有两张桌子上,坐着几人浅浅饮酒,低低谈笑。

这开着的门里,竟似个酒店的模样,只是看来比世上任何一家酒店前精致高雅得多。

燕南天抱着婴儿进去,找了张桌子坐下,只见这酒店里竟也毫无异样,饮酒的那几人,衣衫华丽,谈笑从容,哪里像是逃亡在穷山中的穷凶恶极之辈,燕南天更是奇怪,却不知越是大奸大恶之人,表面上越是瞧不出的。

若是满脸凶相,别人一见便要提防。

哪里还能做出真正的恶事?突见门启动,一个人走了出来,这人矮矮胖胖,笑脸圆圆,正是和气生财的酒店掌柜。

燕南天沉住了气,端坐不动。

这圆脸胖子已笑嘻嘻走了过来,拱手笑道:``兄台远来辛苦了。''燕南天道:``嗯。''

那圆脸胖子笑道,``叁年前闻得兄台与川中唐门结怨,在下等便已盼望兄台到来,不想兄台却害得在下一直等到今日。''燕南天道:``哦?''

这时他心里才知道这些人已将自己错认为``穿肠剑''司马烟了,但面上却丝毫不动声色。

那圆脸胖子挥了挥手,一个明眸皓齿巧笑嫣然的绿衣少女,姗姗走了过来,秋波向燕南天一瞟,万福道,``您好?''燕南天道:``哼,好。''

那圆脸胖子大笑道:``司马先生远来,没有心情与你说笑,还不快去为司马先生热酒,再去为这位小朋友来碗浓浓的米汤。''那少女娇笑道:``好可爱的孩子\ldots\ldots{}''眼皮转动,又向燕南天瞟了一眼,燕子般轻盈,娇笑道走了。

燕南天目光凝注着那圆脸胖子,暗道:"此人莫非便是笑里藏刀小弥陀\ldots\ldots 瞧他笑容如此亲切,对孩子也如此体贴,又有谁想得到他一夜之间,便将他恩师满门杀死!为的耳不过他那师。

妹骂了他一声胖猪而已!"

思念之间一那少女竟又燕子般飞来,已拿来了一盘酒莱,酒香分外清冽,菜色更是分外精美。

那圆脸胖子笑道:``兄台远来,想必饿了,快请用些酒莱,再谈正事\ldots{}''燕南天道:``嗯''他口里虽答应,但手也不抬──他若是抬手,便是要杀人,而绝不会是为着要喝酒吃菜。

那圆脸胖子笑道:``别人只道我等在此谷中,必定受罪吃苦,却不知有这许多聪明才智之士在一起,怎会吃苦,此间酒菜之精美,便是皇帝只怕也难吃到,这做菜的人是谁,只怕兄台万万想不到的\ldots{}''圆脸胖子道:``兄台可曾听说,昔日丐帮中有位天吃星,曾在半个时辰中,毒死了他本门丐帮七大长老\ldots\ldots{}''``啪''的一拍桌子,大笑道,``这当真是位大英雄呀、大豪杰,做菜的人便是他!''燕南天暗中吃惊,面上却淡淡道:``噢''那圆脸胖子突然大笑道:``司马兄果然不愧我辈好手,未弄清楚前,绝不动箸,其实司马兄你未来之前,在下等已将司马兄视为我辈兄弟一般\ldots\ldots{}''举起筷子,对每样菜都吃了一口,笑道:``喏\ldots\ldots 司马兄还不放心么?''燕南天暗中忖道:"他们既然将我认做司马烟,正是我大好机会,我得利用此良机,先将那恶贼江琴的下落打听确实,再出手也不迟。

此刻我若坚持不吃,岂非要使人怀疑,何况,他们既将我当作司马烟,就绝不会下手毒害我。``此刻他算来算去,都是吃比不吃的好,当下动起筷子,道:''好!"立刻就大吃起来。

几样菜果然做得美味绝伦,燕南天立刻就吃得干干净净──一想到吃饱也好动手,他吃得自然更快。

那圆脸胖子笑道:``天吃星手艺如何?''

``好。''

``这位小朋友的米汤想必也快来了,''``越快越好。''``等这位小朋友吃完米汤,燕大侠你就可出手了''燕南天僵然变色,道:``你\ldots\ldots 你说什么?''那圆脸胖子哈哈大笑道:``燕大侠名满天下,又生得如此异像,我哈哈儿纵是瞎子,也认得出燕大侠的,哈哈,方才我故意认错,只不过是先稳住燕大侠,否则燕大侠又怎肯放心吃天吃星以独门迷药作配料酒菜,哈哈\ldots\ldots{}''燕南天怒喝道:``好个恶贼!''飞起一脚,将整张桌子都踢得飞了出去。

那哈哈儿身子一缩,已在一丈开外,大笑道:``燕大侠还是莫要动手的好,否则药性发作得更快一哈哈,哈哈\ldots\ldots{}''燕南天只觉身子豪无异状,还怕他是危言耸听,但暗中一提气,一口真气果然懒懒地提不起。

他又惊又怒,飞扑了过去,铁掌挥出。

那哈哈儿却笑嘻嘻地站在那里,动也不动。

但燕南天铁掌还未挥出,身子便已跌落下来,四肢竟突然变得软绵绵,那千斤神力却不知到哪里去了,他耳畔只听得哈哈儿得意的笑声,那婴儿悲哀的啼哭\ldots\ldots 笑声与哭声却似乎离他越来越远\ldots\ldots 渐渐,他什么都听不到了!

\hypertarget{ux7b2cux516dux7ae0-ux6bd2ux4ebaux6bd2ux8ba1}{%
\chapter{第六章
毒人毒计}\label{ux7b2cux516dux7ae0-ux6bd2ux4ebaux6bd2ux8ba1}}

一盏灯,灯光照着燕南天的脸。

燕南天只觉得这盏灯似乎在他眼前不停地旋转,他想伸手掩住眼睛,但手脚却丝毫不能动弹。

他头疼如裂,喉咙里更似被火烧一般,他咬一咬牙用力瞪眼,瞧着这盏灯。

──灯哪里在转。

于是他瞧见灯光后的那张笑脸。

哈哈儿大笑道:``好,好,燕大侠果然醒来了,这里有几位朋友,都在等着瞧瞧天下第一神剑的风采。''燕南天也已瞧见高高矮矮的几条人影,但灯火刺着他的眼睛,根本瞧不清这几人长得是何模样。

只听哈哈儿笑道:``这几位朋友,不知道燕大侠可认得么哈哈,待在下引见引见,这位便是血手杜杀!''一个冷冰冰的声音道:``二十年前,杜某便已见过燕大侠一面,只可惜那一次在下身有要事,来不及领教燕大侠的功夫。''这人身子又瘦又长,一身雪白的长袍,双手缩在袖中,面色苍白,白得几乎如冰一般变得透明了。

燕南天忍着头疼,厉声狂笑道:``二十年前,我若不是看你才被南天大快路仲道所伤,不屑与你动手,你又怎会活到今日。''杜杀面色不变,冷冷道:``在下已活到今日,而且还要活下去,而燕大侠你却快要死了。''成如此模样\ldots\ldots"``正是不可再让燕大侠生气,人一生气,肉便酸了,此乃我苦心研究所得,各位不可不知。''哈哈儿又道:``这位便是不男不女屠娇娇\ldots\ldots{}''那娇美的语音截口笑道:``我方才还替燕大侠端过菜倒过酒,燕大侠早已认得我了,还用你来介绍什么!''燕南天心头一凛,暗道:``原来方才那绿衣少女,竟然就是不男不女,屠娇娇,这恶魔成名已有二十年,此刻扮成十六七岁的少女,不想竟还能如此神似。''杜杀的血手,李大嘴的吃人,都未能令这一代名侠吃惊,但屠娇娇这鬼神不知的易容术,当真令他变了颜色!突听一人道:``哈哈儿怎地如此噜嗦,难道要将谷中人全介绍给他不成,还不快些问话,问完了也好到阴间来与我作伴。''话声缥缥缈缈,断断续续,第一句话明明在左边说的,第二句话听来便象是在右边,别人说话纵然阴阳怪气,一口气总是有的,但此人说话却是阳气全无一既像是大病垂死,更像是死人在棺材里说出来。

就连燕南天都不禁听得寒毛直竖,暗道:``好一个半人半鬼阴九幽,真的连说话都有七分鬼气。''哈哈儿大笑道:``哈哈,阴老九做鬼也不甘寂寞,燕大侠既已来了,你还怕他不去陪你!''阴九幽道:``我等不及了!''

话声未了,燕南天突觉一只手掌从背后伸进了他的脖子,这只手简直比冰还冷,燕南天被这只手轻轻一摸,已自背脊冷到足底。

李大嘴大喝道:``阴老九,拿开你的鬼手,被你的鬼手一摸,这肉还能吃么!''阴九幽咯咯笑道:``你来动手也未尝不可,只是要快些。''``血手''杜杀突然道:``且慢,我还有话问他!''屠桥娇笑道:``问呀,又没有人拦着你\ldots{}''杜杀道,``燕南天,你此番可是为杜某才到这里来的?''燕南天道:``你还不配!''

杜杀居然也不动气,冷冷道:``杜某不配,谁配?''``江琴''``江琴?谁听说这名字?''

哈哈儿道:``哈哈,恶人谷中可没有这样的无名小卒。''燕南天切齿道:``这斯虽无名,但却比你们还要坏上十倍,只要你们将这□交出,燕某今日便放过你们!''哈哈儿大笑道:``妙极妙极,各位可听到燕大侠说的话了么?燕大侠说今日要饶了咱们,咱们还不赶紧谢谢。''话未说完,咯咯、哈哈、嘻嘻、吃吃,各式各样的笑声,全都响起,一个比一个笑得难听。

燕南天沉声道:``各位如此好笑么?''

屠娇娇吃吃笑道:``你此刻被咱们用十叁道牛索线捆住,又被杜老大点了四处穴道,你不求咱们饶你,反说要饶咱们,天下有比这更好笑的事么。''燕南天道:``哼!''

屠娇娇道:"但我也不妨告诉你,谷中的确没有江琴这个人。

你必定是被人骗了,那人想必是叫你道死的\ldots{}``哈哈儿大笑道:''可笑你居然真的听信了那人的话,哈哈!燕南天活了这么大,不想竟像个小孩子!``突的燕南天是喝一声,道:''好恶贼``这一声大喝,宛如晴空里击下个霹雳!众人耳朵都被震麻了,屠娇娇失声道:''不好,这□中气又足了起来,莫非杜老大的点穴手法,已被他方才在暗中行功破去了?``燕南天狂笑道:''你猜得不错!"

一句话未完,身子突然暴立起来,双臂振处,捆在他身上的十叁道牛筋铁线,一寸寸断落,落了满地。

阴九幽呼啸道:``不好,死鬼还魂了!''

短短七个字说完,话声已在十余丈外,此人自夸轻功第一。

逃得果然不慢,却苦了别人。

只听``咕呼''一声,哈哈儿撞倒了桌子,在地上连滚几滚,突然不见了,原来已滚入了地道。

屠桥娇呼道,``好女不跟男斗,我要脱衣裳了!''竟真的脱下件衣裳,抛向燕南天,燕南天挥掌震去衣裳,她人也不见了。

李大嘴逃得最慢,只得挺住,大笑道:``好,燕奋天,李某且来和你较量较量!''嘴里说着话,突然一闪身,到了杜杀身后,道:``不过还是杜老大的功夫好,小弟不敢和老大争锋!''其实燕南天人虽站起,真气尚未凝聚,这几人若是同心协力,齐地出手,燕南天还是难逃活命!但他算准了这些人欺软怕硬!自私自利。

若要他们齐来吃肉,那是容易得很,若是要他们齐来拼命,却是难如登天。

但见阴九幽、屠娇娇、哈哈儿、李大嘴。

果然一个个全都逃得于干净净,只留下杜杀木头般站在那里。

燕南天真气已聚,目光逼视,却仍未出手,只是厉声道:``你为何不逃?''``杜某一生对敌,从未逃过!''

``你居然敢和燕某一拼?''

``正是!''

话声未了,身形暴起,衣衫飘飘,有如一团雪花,但雪花中却闪动者两只血红的掌影!追魂血手!无论招式如何,这声势已先夺人!燕南天狂笑道:``来得好!''奋起双拳,直向那两只血手击回去!杜杀心头不禁狂喜,要知他以``血手''威震江湖,只因他手掌上戴着的,乃是一双以百毒之血淬金炼成的手套!这手套遍布芒刺,只要划破别人身上一丝肉皮,那人便再也体想活过半个时辰,当真是见血封喉,其毒绝伦!而此刻,燕南天竟以赤手来接,这岂非有如送死!``一声暴喝,一声惊呼!接着,''喀嚓``一响!燕南天双拳明明是迎着''血掌``击出哪知到了中途,不知怎地,明明不可能再变的招式,居然变了,杜杀掌力突然失了消泄之处,这感觉正如行路时突然一足踏空,心里又是惊惶,又觉飘飘忽忽!就在这里,他双腕已被捉住,''一声惊呼尚未出口,``喀嚓''声响,他右腕已被生生折断!燕南天不容他身形倒地,一把抓住他衣襟,厉声道:``谷中可有江琴其人?''杜杀疼得死去活来!咬紧牙关,嘶声道:``没有就是没有!''``我那孩子在何处?''

``不\ldots\ldots 不知道,你杀了我吧!''

``怜你也算是条硬汉,饶你一命!''

手掌一震,将杜杀抛了出去!好杜杀,果然不愧武林高手,此时此刻,犹自能稳得住,凌空一个翻身,飘落在地居然未曾跌倒。

他雪白的衣衫上已满是血花,左手捧着右手,嘶声道:``此刻你饶我,片刻后我却不会饶你!''燕南天笑道:``燕南天几时要人饶过!''杜杀跺脚道:``好!''

转身踉跄去了。

燕南天厉声喝道,``先还我的孩子来,否则燕某将此谷毁得干干净净!''喝声直上云霄,四下却寂无应声。

燕南天大怒之下,``砰''地一脚将桌上踢得粉碎,``咚''他一拳,将粉壁击穿了大洞。

他一路打了出去,桌子、椅子、墙壁、门窗,无论什么,只要他拳脚一到,立刻就变得粉碎。

方才那精致雅观的房子,立刻就变得一塌糊涂,不成模样,但``恶人谷''里的人却像已全死光了,没有一个露头的。

燕南天历喝道,``好我看你们能躲到几时!''

冲出门!身形一转,飞起一脚,旁边的一"扇门也倒了,门里有两个人,瞧见他凶神般撞进来,转身就逃。

燕南天一个箭步窜出去,一把抓住那人的后背。

那人一身武功也还不弱,但不知怎的,此刻竟丝毫也施展不出,竟乖乖地被燕南天凌空提起。

暴喝声中,反臂一抡,那人脑袋撞上墙壁,雪白的墙壁上,立刻像是画满了桃花。

另一人骇得脚都软了,虽还在逃,但未逃出两步,便``噗''地倒在地上,燕南天一把抓起。

那人突然大叫道:``且慢,我有话说。''

他还当这人要说出那孩子下落,是以立刻住手。

哪知这人却道:``我等与你有何仇恨,你要下此毒手?''``恶人谷中,俱是万恶之徒,杀光''了也不冤枉!"``不错,我万春流昔年确是恶人,但却早已改过自新你为何还要杀我?\ldots\ldots 你凭什么还要杀我广燕南天怔了半晌,喃喃道:''我为何要多杀无辜了我为何不能容人改过?恶人谷虽尽是恶人,也并非全无改过自新之辈?``手掌刚刚放松,轻叱道:''去吧!"

那人挣扎着爬起,头也不回,一拐一拐地去了。

燕面天瞧着他走出了门,长长叹息一声,喃喃道:``多杀无辜又有何用?燕南天呀燕南天,你二弟只有此一遗孤,你若不定下心神,熟思对策,你若还是如此暴躁,你二弟只怕就要绝后了,那时你纵然杀尽了恶人谷中的人,又有何用?\ldots\ldots{}''一念至此,但觉火气全消,于是他也就发现了此间的许多奇异之处。

这是间极大的房子,四面堆满各式各样的药草,占据了屋子十之五大,其余地方,放了十几具火炉,炉火俱都烧得正旺,炉子上烧着的有的是铜壶,有的是用锅,还有的是奇形怪状,说不出各自的紫铜器,每一件铜器中,都有一阵阵浓烈的药香传出。

燕南天流浪江湖多年,不仅见多识厂,而且对医药颇有研究,闲时荒山采药,也曾配制过几种独门伤药。

但此间,这屋子里的药草,无论是堆在屋角的也好,煮在壶里的也好,燕南天是多也不过认得其中二叁。

他这才吃了一惊:``原来万春流医道如此高明,幸好我未杀他,他若未改过,又怎会致力于济世活人的医术?''浓烈的药香,化做一团团蒸气,弥漫了屋子,有如迷雾一般,凭添了这屋子的神秘。

突然间,一条人影被月光投落进来,月光下,一条高瘦的黑衣人,一步步走了进来,走入了迷雾。

他脚步比猫还轻,动作比猫还灵,那一双眼睛,也比猫更狡黠,更邪异,更灵活,更明亮。

燕南天沉住了气,凝注着他,没有说话。

黑衣人居然走进了这屋子,居然站在燕南天面前,他目中闪动者狡黠的光芒,嘴角也带着狡黠的微笑。

他拱了拱手,笑道:``燕大侠,你好\ldots{}''燕南天道:``哼。''黑衣人道:``在下穿肠剑司马烟!''

``原来是你!原来你已来了!''

``燕大侠还未来,在下便已来了,但燕大侠近日的故事,在下已有耳闻,所以燕大侠一来,此间便已知道。''燕南天瞪着他,瞪了足足有半盏茶的工夫之久,突然厉声道:``你凭什么认为燕某不敢杀你?这倒有些奇怪,''司马烟笑道:``两国相争,不斩来使。''燕南天皱眉道:``你是谁的使者?''

``在下奉命而来,要请问燕大侠的一句话。''

燕南天动容道:``可是有关那孩子?''

``不错!''

燕南天一把抓住他衣襟,嘶声道:``孩子在那里了''司马烟也不答话,只是含笑瞧着燕南天的手,燕南天咬一咬牙,终于放松了手掌,司马烟这才笑道:``在下奉命来请问燕大侠,若是他们将孩子交回,又当如何?''燕南天一震,道:``这个!\ldots\ldots{}''。

``燕大侠是否可以立刻就走,永不再来,''为了孩子,我答应你!"``一言既出!\ldots{}'',"``燕某说出来的话,永无更改!''``好!燕大侠侠随我来!''

两人一先一后,走了出去,夜色正静静地笼罩着这``恶人谷'',月光下的``恶人谷'',看来更是平和、安静。

司马烟走在洒满银光的街道上,脚步更轻得没有一丝声音,他脚步不停,走到街尽头一栋孤立的小屋。

屋门半掩,有灯光透出。

司马烟道:``那孩子便在屋里,也望燕大侠抱出了孩子后,立刻原路退回,燕大侠乘来的马车,已在谷口相候。''燕南天情急如火,不等他话说完,就箭步窜了进去!屋子的中央,有张圆桌,那孩子果然就在圆桌上。

燕南天热血如沸,一步窜过去,抱起孩子,惨然道:``孩子,苦了!''一句话未说完,突然将那孩子重重摔在地上,狂吼道:``好恶贼!孩子,这哪里是什么孩子,这只是个木偶!但燕南天发觉已太迟了,满屋风声骤响,数百点银光乌芒,已四面八方,暴雨般向他射了过来!暗器风声,又尖锐,又迅急,又强劲,显然这数百点暗器,无一不是高手所发,正是必欲将燕南天置之死地!这些暗器将屋子每一个角落全都占满,当真已算准了燕南天委实再没有可以闪避之地!哪知燕南天狂啸一声,身子拔起,只听''哗啦啦``一声暴响,他身子已撞破了屋顶,飞了出去!屋子四周暗影中,惊呼不绝,十余条人影,四下飞奔逃命,燕南天狂啸声中,身形如神龙天矫,凌空而转!但听''咚、砰,噗"几响,几声惨呼,一人被他撞上屋脊一人被他抛落街心,一人被他插入屋瓦。

叁人俱都是脑袋崩裂,血浆四溅,立时毙命,但别的人还是逃了开去,眨眼间便逃得踪影不见!燕南天跃落街心,厉声狂吼道:``如此暗算,岂能奈何燕某,若是想要燕某的命,何妨出来动手!''吼声远达四山,四山回音不绝,只听``何妨出来动手\ldots\ldots 出来动手!动手!''之声良久不息。

燕南天龙行虎步,走过长街,叫骂不绝。

但``恶人谷''中却没有一个人敢探出脑袋!孤身一人的燕南天一竟骇得``恶人谷''所有恶人没有一个敢出头,这是何等威风!何等豪气!但燕南天心中却无丝毫得意,他心中有的只是焦急、痛苦、悲愤!他脚步虽轻,心情却无比沉重!谷中的灯光,不知在何时全都熄了。

虽有星光月亮,但谷中仍是黑暗得令人心胆欲裂。

突然间,一道刀光,自黑暗的屋角后直劈而下!这一刀显然也是刀法名家的手脚,无论时间、部位,俱都拿捏得准而又准,算准了一刀便可将燕甫天的脑袋劈成两半!这一刀刀势虽猛,刀风却不厉,正也算准了燕南天绝难防范!哪知看来必定猝不及防的燕南天,不知怎地,身子突然一缩,刀光堪堪自他面前劈下,竟未伤及他毫发。

``咯'',钢刀用力过猛,砍在地上,火星四射。

燕南天出手如电,已抓住了持刀人的手腕,厉喝道:``出来!我问你!''突觉手上力道一轻,那只手虽被他拉了出来,却只是血淋淋一条断臂、那人竟以左手一刀砍下了自己的右臂!好狠的人!他竟连哼都未哼一声!燕南天又惊、又急、又怒、又恨,取下钢刀,抛和断臂,随手一刀劈了出去,一扇门应手而裂。

但门里却瞧不见一条人影!燕南天有如疯狂,一间间屋子闯了过去,每间屋子里都瞧不见一条人影,他看得要疯,但疯又有何用!他铜牙几已咬碎,双目已红赤,嘶声道:``好!你们躲,我倒要看你们能躲到几时!''竟搬了张椅子出来,坐在街中央,月光,照着他身子,照着他身上的血,血一般的月光\ldots\ldots{}``恶人谷''中的若是恶鬼,燕南天便是镇鬼的凶神!突听一人大笑道:``这臭孩子又有什么了不起,你要,就还给你!''燕南天狂吼而起,扑了过去。

只见黑暗中人影一闪,一件东西被抛了出来,看来正是个襁褓中的孩子。

燕南天不由得伸手接过。

一但他指尖方自触及此物,突然厉喝道:``恶贼,还给你!一手掌一震,那包袱又笔直飞了回去,撞上墙壁,''轰``的一声,竟将那屋子炸崩了一半!这襁褓中包的竟是包火药!回声响过,四下又复静寂如死,燕南天想到自己方才若非反应灵敏,指尖触热,便将襁褓掷回,此刻岂非已被炸得粉身碎骨?他一死纵不足惜,但那孩子!\ldots\ldots 燕南天捏紧拳头,掌心已满是冷汗!毒计!''恶人谷``果然有层出不穷的毒计!纵是天大的英雄,只要稍一不慎,就难免死在此地!燕南天虽已逃过数劫,但他还能再逃几次?他精力终是有限难道真能不眠不休,和他们拼到底?突然间,他心中灵机一闪,暗道:''他们能利用这黑暗于我,我难道不能利用这黑暗来搜寻他们?"想到这里,燕南天又不觉精神一振,再不迟疑,微一纵身,也掠入黑暗里,消失不见。

这正是以牙还牙,以毒攻毒,一时间他纵然寻不着那孩子,但``恶人谷''的恶人,也再难暗算他了。

燕南天身子潜行在黑暗中,就像蛇!就像猫──就算别人有着猫的耳朵,也休想听出他的声音,就算那人有着猫一般的眼睛,也休想瞧见他身影,有这样的敌人随时会到身衅``恶人''怎不胆战心惊?只是燕南天却也找不着他们。

每间屋子,似乎都是空的,人,竟不知到哪里去了。

燕南天沉住气,一间间房子找了过去,他这才发觉这``恶人谷''里,屋子当真有不少。

夜,很静,很静!整个``恶人谷'',就像是座坟墓。

风,自山那边吹过来,已有了寒竟!突然,风中似乎有了声音,有了种奇异的声音,似乎人语。

燕南天的心一跳,屏息静气,潜行过去。

果然有极轻极轻的人语,自一栋屋子里传出来!一人道:``小屠果然有两手,竟将这孩子弄睡着了。''这人虽没有笑,却显然是哈哈儿的声音。

另一人道:``幸好有这孩子作人质,否则!\ldots\ldots{}''突听屠娇娇的语声道:``李大嘴,你要做什么?''李大嘴轻笑道:``我瞧这女的尸身细皮嫩肉,倒和昔日我那老婆相似。''屠娇娇道:``但这尸身已死了好几天了呀!''

李大嘴道:``只要保养得住,还是可以吃的。''``好,你吃了她也好,这想必就是燕南天那□的弟媳妇,你吃了她一也可替杜老大出口气。''燕南天怒火早已升到咽喉,哪里还忍耐得住,狂吼一声,闪电般掠下,一脚踢开了房门。

屋子里连声惊呼,人影四散,李大嘴喝道:``给你吃吧!''竟举起那棺材,直掷过来!棺材里香料落了一地,尸身也掉在地上。

黑暗中,只听哈哈儿狂笑道:``好,燕南天,算你狠,居然找到了咱们,但你莫忘了,孩子还在咱们手中,只要你追出来,哼哼!哈哈!哈哈!''燕南天身形已扑起,听得这语声,颓然而落,心中更是悲伤填膺,他方才一时不能忍过,又坏了大事。

月光自窗户照进来,照着地上的尸身!这是孩子的母亲,那苍白而浮肿的脸,零乱而无光的头发,被惨白的月光一映,真是说不出的恐怖凄凉。

``燕南天惨然道:''二弟,我对不起你,我!\ldots\ldots 我!\ldots\ldots 我非但不能妥为照顾你的孩子,甚至连\ldots\ldots 连你们的尸身\ldots\ldots"他语声哽咽,实难再说下去,他跺了跺脚,扶正棺材,俯身双手托起那尸身,小心的放回棺材去。

他热泪盈眶,委实不忍再瞧他弟媳的尸身一眼。

他黯然闭起眼睛,喃喃道:``但愿你从此安息。''冷月,寒棺,无边的黑暗,可怖的艳尸\ldots\ldots 这尸身竞突然自燕南天怀中跃起!只听``砰!砰!砰!砰!''四响!这``尸身''双手双脚,俱都着着实实中中了燕南天的身子!燕南天纵是天大的英雄,纵有无故的武功,无故的机智,却再也想不到有此一惊人的变化!他惊呼尚未出口,左肩``中府'',右胁``灵墟'',前胸``巨阙'',腹下``冲门''四处大穴已被击中!这一代英雄终于仰天倒了下去!那``尸身''已落地,咯咯大笑道:``燕南天呀燕南天,如今可知道我的手段!''得意的笑声中,随手在头上扯了儿扯,扯下了一堆乱发,月光,照着他的脸,那不是屠娇娇是谁!灯光,忽然亮起!哈哈儿、李大嘴、阴九幽、司马烟全都现身而出,纵然是在灯光下,这几人的模样还是和恶鬼相差不多。

哈哈儿大笑道:``燕南天,你只当方才真是你找着咱们的么?\ldots\ldots 哈哈,这不过是咱们的妙计而已,好都你自己送上门来。''李大嘴怪笑道:``燕南天,你只当方才真是咱们怕了你么?哈哈,那只不过是咱们知道你必已难逃性命,又何苦费力与你动手!''儿个人言来语去,得意的笑声,再也停不住。

燕南天叹息一声,闭起了眼睛,他知道自己此番再也难逃毒手的了!只听阴九幽道:``你们还等什么?难道还要等他再跳起来?''屠娇娇叱道:``且慢!我出力最多,要杀他,该我来动手才是。''阴九幽冷森森道:``若是早听我的,他此刻早已死了,哪里还需费这许多手脚,我瞧你们还是让我动手吧。''李大嘴大声道:``不行,你们不会杀人,一个杀不好他的肉就酸了,吃不得的,自然还是该我动手方是''几个人七嘴八舌要争着动手──能令天下第一剑客死在自己手下,自然是极大的荣耀。

\hypertarget{ux7b2cux4e03ux7ae0-ux6f0fux7f51ux4e4bux9c7c}{%
\chapter{第七章
漏网之鱼}\label{ux7b2cux4e03ux7ae0-ux6f0fux7f51ux4e4bux9c7c}}

哈哈儿看了看燕南天倒下的身体,突然大笑道:``各位也莫要争了,我有了好主意!屠娇娇道:''你又有什么好主意?``哈哈儿道:''咱们若让燕大侠痛快地死了,岂非辜负燕大侠一番美意?自然要请燕大快慢慢地享受享受死前的滋味,也不枉燕大侠结交咱们一场!阴九幽不等他说完,便已桀桀笑道:``妙极,果然妙计,我正好要他尝尝阴风搜魂手,的滋昧,保险他直到下辈子投胎还忘不了\ldots{}''屠娇娇道:``我''销魂美人功的滋味,也不比你差。``李大嘴怪叫道:''我的利骨刀难道就差了么?``眉娇娇笑道:''还是杜老大来,他的血手钻心和咱们哈哈儿的``伐髓洗脑'',这两种滋味才真是要人难以消受的。``哈哈儿道:''哈哈!既是如此,谁先动手?"

屠娇娇道:``你出的主意,你先动手吧!''

哈哈儿大笑道:``好!''

笑声中伸出手掌,向燕南天脑后轻轻抚摸过去。

夜色更深,生龙活虎般的燕南天,已被折磨得不成人形,只要是稍有心肝的人,便不忍描叙他此刻的模样。」哈哈儿道:"哈哈,我已出手六次,现在又轮到李兄了。

``李大嘴道:''不行不行,我不出手了!"

哈哈儿笑道:``若不出手,便是认输了''李大嘴怒道:``人十成已死了九成,纵然是才出世的婴儿打他一掌,他也活不了啦,你为何要我出手?''阴九幽冷冷道:``那也未必。''

李大嘴道:``好,好!既是如此,你出手吧''阴九幽道:``轮到我时我自会出手的。''李大嘴怒道:``你明知已轮不到你了,你\ldots\ldots{}''。

哈哈儿又笑道:两位也莫要争执,不妨先找咱们那万神医来鉴定鉴定,瞧瞧这燕南天是否已再也出不得一丝气力。``阴九幽冷笑道:''找谁来鉴定都无妨。"

李大嘴道:``我去找。''

片刻间他便将万春流带了回来,又见万寿流瘦小精悍,目光深沉,枯瘦的面目上绝无任何表情。

他走进来后,微微点头,便在奄奄一息的燕南天的侧身坐了下去。

又过了半个时辰,他总算才将燕南天由头到脚,仔细检查了一遍,他灵巧的手指,竟似未沾着燕南天的皮肉。

李大嘴不耐道,``此人怎样?''

万春流缓缓道:``此人肺经、脾经、心经、肾经、心包络经、叁焦经、胆经、肝经,俱已残坏,十四经脉,已毁其八,此刻还能活着已是奇迹\ldots{}''季大嘴笑道:``你瞧怎样?''阴九幽道:``他只怕错了''万春流道:``武功我虽不及你,但对医道却有自信。''阴九幽冷笑道:``自信?若非你那高明的医道,开封城一夕间也不会暴死九十七人,那些人是谁害了的?你忘了么?''万寿流冷冷道:``我杀的人虽多,但这几年来在此间救的人也不少,阁下刚来时,若非有万某在这里,只怕也活不到今日\ldots{}''阴九幽目中虽已射出火,但口中却说不出话来,他逃来此地,确是已伤重垂危,确是万春流救了他的性命。``恶人谷''的确是少不了万春流的。

哈哈儿立刻笑道,``万神医法眼鉴定,自是不会错的既是如此,你我就算不分,大家一齐动手将燕南天杀了也罢。''万寿湿沉声道:``且慢,在下正要请各位留下他的性命。''阴九幽怒道,``你\ldots\ldots 你要救他?''

万春流神色不动,缓缓道:``伤势如此沉重而不死的人,我生平还未见过,这样的人对各位完全无用,对在下却大有用处\ldots{}''李大嘴道:``有什么用处,难道你也想吃他!''万春流道:``此人身上伤残不下叁十处,正好作为我试验药草性能之用,在下若是试验成功,对各位也大有好处,''阴九幽冷笑道,``纵有用处,但你试验成功,岂非也就将燕南天救活了,等到他伤势痊愈,你就该来救咱们了。''万春流淡淡道:此人纵被救活,也势必要成残废、白痴,各位若要取他性命,还是随时都可下手,又何必急在这一时。``阴九幽哼了一声,再不说话,司马烟更从来未说话,只是望着哈哈儿,哈哈儿望着屠娇娇,屠娇娇笑道:''万神医说什么就是什么吧。``万春流冷冷道,''此人的叁十处伤,最少可试出叁十种药草之性能,这叁十种药草,说不定就有一种将来能救各位的命。``屠娇娇笑道:''万神医,你还等什么了这燕南天从头到脚,已全是你的了``万春流脸上也没有半分高兴之色,淡淡道:''多谢。"自怀中取出几粒药丸,塞入燕南天嘴里,让燕南天的唾沫将之化开,然后再慢慢流下去。

突听一阵婴儿的啼哭声传了过来。

李大嘴精神一振,笑道:``对了!还有那孩子。''哈哈儿望着阴九幽,道:``如何?''

阴九幽道:``杀!''

屠娇娇突然道:``慢着!''

李大嘴皱眉道:``你又有什么事?''

屠娇娇道:``这孩子也杀不得!''

哈哈儿笑道:``此番倒是小屠的不是了,这孩子留下也是个祸胎,倒不如斩草除根,落个干净。''屠娇娇也不答话,却反问道:``我且请教各位,咱们虽然都是恶人,但世上最凶最毒最恶的人究竟是谁,各位可知道么?''哈哈儿大笑道:``哈哈,若论天下最恶的人,自然便得算小屠了\ldots{}''屠娇娇笑道,``过奖!过奖!但\ldots\ldots{}''她还末说出下面的一句话,李大嘴已怒道:``他算是什么?会玩两手不男不女的花样,也可算是天下最恶的人?哼,他连人肉都不敢吃!''屠娇娇笑道:``他说我不是天下最恶的人,我完全同意,但能吃几斤人肉就算是天下最恶的人么?我昔年瞧见一个赶骡车的,也能吃得下几斤人肉。''李大嘴怒道:``以你说来,天下最恶的人是谁?''哈哈儿道:``哈哈!对了,阴老九!''

屠娇娇道,``阴老九的确够阴、够狠、够毒,但他的凶恶已全摆在脸上,别人一瞧就知他是恶人,已先对他提防叁分\ldots{}''哈哈儿道:``如此说来,他也不算!''屠娇娇笑道,``自然不算的,否则他能学到笑里藏刀的本事,要能在一面嘴里叫哥哥,一面在腰里掏家伙\ldots\ldots{}''哈哈儿道:``笑里藏刀\ldots\ldots 哈哈!小屠在说我了。''屠娇桥笑道:``不错!哈哈兄生得一副弥陀怫的模样,当真是谁也瞧不出他是恶人,他就算将人卖了,别人还不知是被谁卖的。''哈哈儿拍掌大笑道:``妙极妙极,我若真是天下最凶最恶之人,倒也不错,只可惜我一瞧杜老大就害怕,看来还是他比我恶得多。''哈哈儿瞧了司马烟一眼,道:``对了,还有司马兄,哈哈,穿肠毒药剑,来人如捣蒜,这句话江湖中又有谁不知道?''司马烟微微笑道:小弟在江湖虽也落有恶名,但在十大恶人面前,小弟却是麻绳穿豆腐,提也提不起起的``屠娇娇道:''是呀十大恶人中,还有五个呢?``司马烟笑道:''也以个弟看来,那五位也未必能比这五位恶多少,尤其是那位狂狮铁战,严格说来,根本就不能名列十大恶人之一"。

屠娇娇道:``狂狮若是狂起来,当真是大亲不认,见人就打,就连他的儿子,都被逼得非和他打一场不可,但真被打死,却没有半个,何况他还有不狂的时候\ldots{}''哈哈儿笑道:``狂狮不行,那迷死人不赔命的萧咪咪又如何?我瞧就算''二十四孝中的孝子若是被她迷上,也会把老子娘全部卖了的。``屠娇娇道,''萧咪咪的迷汤功夫虽到家,但真被她迷上的,也不过都是些十七十八二十来岁的毛头小伙子,她若遇见李大嘴,还不是一口将她吃了。

``李大嘴冷冷道:''半男半女的人,她自然是迷不上的。``哈哈儿赶紧道,''这也不是,那也不是,天下最凶最毒最恶的人究竟是谁,难道是大庙里的老和尚不成?``屠娇娇笑道,''咱们这些人,论凶、论毒、论恶,大家都差不多,谁也别想强过谁,所以说,到目衣为止,世上还没有一个人能算是最恶的!``李大嘴道:''哼!说了半天,原来是废话\ldots{}``屠娇娇也不理他,自管接着道:''现在虽没有,但马上就要有了\ldots{}``这句话说出来,每个人竟忍不住问道:''谁?``屠娇娇眼珠一转,缓缓道:''就是那正哭的孩子\ldots{}``这名话说出来,每个人又不禁为之一愣\ldots 李大嘴终于哈哈笑道:''你说他是天下最凶最恶的人?\ldots\ldots\ldots\ldots 哈哈嘻嘻!嘿嘿!\ldots\ldots\ldots\ldots\ldots 呸!``屠屠娇娇还是不理他,还是自管接着道:''这孩子是现在什么都不懂,咱们告诉他什么,他就听什么,嗅们若说乌鸦是白的,他也不会说不是,是么?``李大嘴道:''哼!又是废话!"

屠娇娇道:``他从小跟着咱们,眼睛瞧见的都是咱们做的事,耳朵听见的,都是咱们说的话,他长大了不但是个大坏蛋,而且是世上最大的坏蛋!你们不妨想想,他若将这恶人谷中每个人的坏花样全学会了,世上还有谁能比他更凶,更毒,更恶!''哈哈儿笑道:``这样的人,只怕连鬼见了都要害怕\ldots{}''。

屠娇娇道:``这就对了,连鬼见都怕的人,若是到了江湖中去,又当如何?''``哈哈儿拍掌大笑道:''哈哈!那不搞得天下大乱才怪。``屠娇娇缓缓道:''正是要搞得天下大乱,咱们被人逼到这里。

谁没有一肚子气,这孩子正是天赐给咱们,要他来为咱们出气的!""听到这里,就连阴九幽面上也不禁泛起一丝笑容,点着头。

道:``好主意!''

哈哈儿更是笑得前仰后合,不禁拍掌道:``哈哈!除了小屠''外,还有谁能想出这么好的主意!。

于是``恶人谷''中就多了个小孩子。

每个人都将他唤作``小鱼儿'':"因为他的确是条漏网的小鱼。

\hypertarget{ux7b2cux516bux7ae0-ux8fd1ux58a8ux8005ux9ed1}{%
\chapter{第八章
近墨者黑}\label{ux7b2cux516bux7ae0-ux8fd1ux58a8ux8005ux9ed1}}

小鱼儿渐渐长大了。

小鱼儿最最亲近的人,有杜伯伯、笑伯伯、阴叔叔、李叔叔、万叔叔,还有位叔叔,哦!不对,屠姑姑。

``小鱼儿就是跟着这些人长大的,他跟每个人过一个月──''一月是杜伯伯,二月是笑伯伯,叁月是阴叔叔\ldots\ldots 这样到了七月,就又跟着杜伯伯。

小鱼儿跟着杜伯伯时最规矩。

这位一只手断了的杜伯伯,脸上从来没有笑容。

他教小鱼儿武功时,小鱼儿只要有一招学慢了,屁股就得吃板子,小鱼儿屁股本来常常肿,但到后来肿的次数却越来越少了。

小鱼儿跟着笑伯伯时最开心。

这位笑伯伯不但自己笑,还要他跟着笑,最苦的是,小鱼儿屁股肿着时,笑伯伯也逼着他笑,``不笑不行。''小鱼儿跟着阴叔叔时最害怕。

这位阴叔叔的身上好像有股寒气,就是六月天,小鱼儿只要在他身旁,就会从心里觉得发冷。

小鱼儿跟笑伯怕一个月,连脸上的肉都笑疼了,跟着阴叔叔正好乘机休息休息。

就算心里有最开心的事,但只要一见阴叔叔,再也笑不出了,见着阴叔叔,没有人笑得出的。

小鱼儿跟着李叔叔时最难受。

这位李叔叔总是在他身上乱嗅,嗅得他全身不舒服。

小鱼儿跟着屠姑姑时最奇怪。

这位屠``姑姑''忽然是男的,忽然又变成女的,他实在弄不清这究竟是``姑姑''了还是``叔叔''了.最特别的时候,是跟着万叔叔。

这位万叔叔脸上虽也没有笑容,但却比那杜伯伯看起来和气得多了,说话也没有那么难听。

但这位万叔叔却总是喂小鱼儿吃药,还将个鱼儿整个泡在药水里,这却令小鱼儿有些受不了。

万叔叔的屋子里,还有位``药罐子''叔叔。

这位``药罐子''叔叔,像是木头人似的,坐在那里不动,每天只是吃药,吃药,不断地吃药。

他吃的药实在比小鱼儿还多几十倍,小鱼儿对他非常同情,只因为小鱼儿自己深知吃药的苦。

只是这位``药罐子''叔叔从来不诉苦──!他根本从来没有说过话,他甚至连眼睛都像是张不开似的。

此外,还有许多位叔叔伯伯,有一个会捏泥人的叔叔,小鱼儿本来很喜欢他,但有一天,突然不见了。

小鱼儿到处找他不着,他去向别人,别人也不知道,他去问屠姑姑,屠姑姑却指着李叔叔的肚子说:``在李大嘴的肚子里。''一个人怎会在李叔叔的肚子里?小鱼儿不懂。

其实李叔叔也失踪过一次。

那天李叔叔大叫大嚷道:``我憋死了,我受不了!''然后他就失踪了。

但过了半个月,他却又从谷外回来,只是回来时已满身是伤,简直差一点就没有命了。

小鱼儿五岁还不到时,有一天,杜杀将他带到屋子里,屋子里有一条狗,杜杀给他一把小刀。

小鱼儿很奇怪,忍不住问道:``刀\ldots\ldots 做什么用?''杜杀道:``刀是用来杀人的,也是杀狗的!''

小鱼儿道:``还可以用来切菜,切红烧肉,是么?''杜杀冷冷道:``这不是切菜的刀。''

小鱼儿道:``我不要这把刀,我要切菜的\ldots\ldots{}''杜杀道:``莫要多话,去将这条狗杀了!''小鱼儿道:``这狗若不听话,打它屁股好了,何必杀它?杜杀怒道:''叫你杀,你就杀!``小鱼儿简直要哭了,道:''我\ldots\ldots 不要\ldots\ldots{}``杜杀道,''你不杀?好!``突然出了屋子,''喀嚓!"一声,把门反扣起来。

小鱼儿大嚷道:``杜伯伯,让我出去\ldots\ldots 我要出去!''杜杀却在门外道:``杀了狗才准出来。''

小鱼儿叫道:``我杀不了它,我打不过\ldots\ldots{}''杜杀道:"你打不过它,就让它吃了你也罢。

小鱼儿在里面哭,在里面叫,他哭肿了眼睛,叫破了喉咙也没人理他,杜杀像是根本走开了。

小鱼儿也不哭了。

小鱼儿只有瞪着那只狗瞧,那只狗也在瞧他,这只狗虽不大,也样子却凶得很,小鱼儿实在有些害怕。

他握着刀动也不敢动,过了很久很久,他肚子``咕咕''叫了起来,那狗也``汪汪''叫了起来,他才记起还没吃过晚饭。

他饿得发慌,莫非那狗也饿得发慌。

小鱼儿道:``小狗小狗,你莫要叫,我也没有吃。''那狗却叫得更厉害,一条红舌头,不住往小鱼儿这边伸过来,小鱼儿更害怕,握紧了刀,道:``个狗小狗,我饿了不想吃你,你饿了可也不准想吃我。''那狗``汪''的一声,扑了过来。

小鱼儿大叫道,``我的肉不好吃\ldots\ldots 不好吃''杜杀插手站在门外,只听那狗吠声越来载响,越来越凄厉。

但突然间,什么声音都没有了。

又过了半晌,杜杀缓缓开了门。

只见小鱼儿手里握着刀,爬在地上,也像是只小狗似的,他满身是血,狗也满身是血,只是他还活着,狗却已死了。

小鱼儿在万春流处养了半个月的伤,才能走路,他脸上本已有条伤痕,此刻身上又添了许多。

过了几天,杜杀又将他找去,还是将他关在那屋子里,屋子里又有条狗,但却比那条大了许多。

杜杀道:``那柄刀你可带着了''小鱼儿只是点头,脸都白了,也说不出话。

杜杀道:``好!将这狗也杀了!''

小鱼儿道:``但这狗\ldots\ldots 好\ldots\ldots 好大。''

杜杀道,``你害怕么?''

小鱼儿拼命点头,道:``怕\ldots\ldots 怕的。''

杜杀怒道:``没出息!''

突又转身走了出去,``喀嚓''一声,又将门反扣上。

过了许久,门里狗又叫得厉害,叫了盎茶工夫,便又无声音,杜杀开了门,狗死了,小鱼儿还活着。

这次他虽也满身是血,但却已能站在那里,眼睛里虽有眼泪,但却咬着嘴唇,大声道:``我又杀了它,十六刀。''杜杀道:``你还怕不怕?''

小鱼儿道:``狗死了,我当然不怕了,但刚刚\ldots\ldots{}''杜杀道:``你方才怕又有何用?你害怕,我还是要你杀它,你害怕,它还是要吃你,这道理你明不明白?''小鱼儿点头道:``明白了。''

杜杀道:``你可知道你怎会受伤?''

小鱼儿垂下了头,道:``因为我害怕,所以不敢先动手。''杜杀道:``既是如此,你下次还怕不怕?''

小鱼儿捏紧拳头,道:``不怕了''。

杜杀瞧着他,嘴角又泛起一丝微笑。

这一次小鱼儿伤就好得较快了,但他的伤一好,杜杀就又将他关到那屋子里去,屋子里的狗也越来越凶,越来越大。

但小鱼儿受的伤却越来越轻,好得也越来超快。

到第六次,杜杀开了门──屋子里已不再是狗。

屋子里已是条小狼!于是小鱼儿又到床上,吃药,不断的吃药。

有一天,哈哈儿来了,小鱼儿想笑,但笑不出。

哈哈儿笑,``小鱼儿果然还躺在这里,哈哈!狼果然是不吃小鱼的。''小鱼儿道,``笑伯伯,你莫要生气好么?''

哈哈儿道:``生什么气?''

小鱼儿道:``我实在想笑的,只是\ldots\ldots 我一笑全身都疼,实在笑不出。''哈吹儿大笑道:``傻孩子,告诉你,笑伯伯我在笑的时候,身上有时也在疼的,但我身上越疼,就越笑得凶。''小鱼儿眨了眨眼睛,道,``为什么?''

哈哈儿道:``你可知道,笑不但是灵药,也是武器\ldots\ldots 最好的武器,我简直从未发现过一样比笑更好的武器''小鱼儿睁大眼睛,道:``武器\ldots\ldots 笑也能杀狼么?''哈哈儿道:``哈哈,不但能杀狼,还能杀人!''小鱼儿想了想,道:``我不懂!''

哈哈儿道:``你可知道你为什么每次都受伤?''小鱼儿道:"我不知道,我\ldots\ldots 我已不害怕了,真的已不害怕了,这大概是因为我功夫不好,不能一刀就将它杀死。

哈哈儿道:``你为什么不能一刀就将它杀死?''小鱼儿道,``因为我的功夫\ldots\ldots{}''哈哈儿笑道:``不只因为你的功夫,而且因为你没有笑,那些狗,那些狼,虽然不会说话,但也有懂事的,你一走进屋里,它们就知道你对它们没有怀好意,就在提防着你,所以纵然先下手也没有用\ldots{}''小鱼儿听得眼睛都圆了,不住点头道:``对极了!''哈哈儿大笑道:``所以下次你进屋子时,无论见着的是狼是狗,甚至是老虎都没关系,你脸上都要堆满笑,让它以你对它没有恶意,只要它不提防你,将你当作朋友,你就可一刀杀死它!这道理虽然简单,但却是最有用了!''小鱼儿道:``那么以后我就不会受伤了。''

哈哈儿道:``正是,无论是狼是狗,还是人,都不会伤害一个对他全无恶意的人的,你只要笑,不停地笑,直到你己将刀插进他身子,还是在笑,让他那临死前还不会提防你,那你就不会受伤了。''小鱼儿道:``但\ldots\ldots 但这是不是不够英雄?\ldots\ldots{}''哈哈儿大笑道:``傻孩子,它既要杀你,你就该先杀它,你既然一定要杀他,用什么手段,岂非都是一样么了?''小钍儿展颜笑道:``不错!我懂了。''

一哈哈儿大笑道:``好孩子!哈哈,这才是好孩子。''小鱼儿果然不大再受伤了。

他已杀了五条狗,四只狼,两只小山猫,一条小老虎,他身上的伤疤,数一数已有二十多条。

这时他才不过六岁。

有一天,他突然问屠娇娇:``屠姑姑,别人都说你是个非常非常聪明的人,你究竟是不是了''屠娇娇咯咯笑道:``这是谁说的?\ldots\ldots 但那人可真说对了\ldots{}''小鱼儿道:``你是不是有许多稀奇古怪的东西。''屠娇娇笑道:``你这个鬼,在转什么鬼心思?''小鱼儿眨着眼睛,道:``假如我替你出气,你肯不肯送件稀奇古怪的东西给我?''屠娇娇道,``我要你这小鬼出什么气?,小鱼儿笑道:''我看季叔叔总是惹你生气,但你却对他没法子\ldots\ldots{}``屠娇娇惊笑道:''难道你这小鬼已有法子对他?``小鱼儿点头笑道:''嗯。``屠娇娇道:''你有什么法子?``小鱼儿道:''只要屠姑姑你先给我一种药就行了``屠娇娇,小鱼儿道:''这种药他是没有的,但屠姑姑你却一定有有屠娇娇摇头笑道:``你这小鬼,简直把我都弄糊涂了,好!什么药,你说吧!''小鱼儿笑道:``臭药,越臭越好。''

屠桥桥瞧了半天,突然大笑道:``小鬼,我知道了。''小鱼儿瞪大了眼睛道:``你知道?''

屠娇娇笑道:``小鬼,你瞒得过别人,还瞒得过我?你讨厌李大嘴嗅你,就想弄包臭药藏在身上,让他嗅嗅,但你却又有些怕他,所以就绕着圈子,把我也绕进去,这样你不但有了靠山,还可以向我讨好卖乖。''小鱼儿脸有些红了,笑道:``屠姑姑果然聪明。''屠娇娇道:``你也不笨呀\ldots{}''小鱼儿道:``但我比起姑姑来\ldots\ldots{}''屠娇娇笑道:``小鱼儿!你也不想想你现在才几岁?到你有我这么样的年龄时,那还得了\ldots\ldots 可爱的孩子,总算姑姑我没有白疼你。''小鱼儿低着头,道:``那药\ldots\ldots{}''屠娇娇笑道:``药自然有的,足可臭得死人。''李大嘴再也不敢在小鱼儿身上乱嗅了──他足足吐了半个时辰,足足有一天一夜吃不下东西。

第二天,他一把抓住小鱼儿,道,``臭鱼儿,那药可是屠娇娇给你的?''小鱼儿只是嘻嘻地笑。

李大嘴狠狠道:``你不怕我吃了你。''

小鱼儿笑嘻嘻道,``臭鱼儿的肉不好吃''李大嘴笑骂道:``好!小鬼,我也不吃你,也不打你,只要你也去整那屠娇娇一次,我还有件好东西给你!''小鱼儿道:``真的?''

李大嘴道:``自然是真的''到了黄昏时,小鱼儿和屠娇娇一齐吃饭,桌上有碗红烧肉。

小鱼儿拼命将肉往屠娇娇碗里夹,笑道:``这是姑姑是喜欢吃的莱,姑姑多吃些\ldots{}''屠娇娇笑道:``小鬼,你倒会拍马屁\ldots{}''小鱼儿道:"姑姑对我好,我自然要对姑姑好。

屠娇娇道:``你怎地不吃\ldots{}''小鱼儿道:``我舍不得吃。''屠娇娇笑道:``傻孩子,有何舍不得,这又不是什么特别好的东西。''小鱼儿眨了眨眼睛,道:``但这碗肉特别好,''屠娇娇道:``为什么了''小鱼儿道:``这碗肉是我特地从李叔叔那里拿来的,听说是\ldots\ldots{}''他话未完,屠娇娇脸已白了,道:``这\ldots\ldots 这就是昨天他杀的\ldots\ldots{}''小鱼儿满脸天使般的笑容,点头道:``好像是的。''屠娇娇道:"你\ldots\ldots 你这小鬼\ldots\ldots。

话未说完,已一口吐了出来。

她也足足吐了半个时辰,也足足有一天不想吃饭。

杜杀住的地方,在``恶人谷''的边缘,他屋后便是荒山──他屋子里其实也和荒山相差无几。

就连他的卧室里,都绝无陈设,可说是``恶人谷''中最最简陋的屋子,小鱼儿每次从屠娇娇的屋子里走到他屋子里,总觉得特别不舒服,更何况他屋子里总有个吃人的野兽在等着,但小鱼儿不来却又不行。

这一天,小鱼儿又摇摇晃晃地来了。

杜杀笔直地坐在屋角,动也不动,他那一身雪白的衣衫,在阴暗的屋子里看来,就好像是雪堆成的。

小鱼儿每次来,都瞧见杜杀这样坐着,姿势从来末曾改变过,小鱼儿每次走到他面前,都不敢说话。

杜杀冷冷瞧着他,瞄了半晌,突然问道:``听说你有个小小的箱子''小鱼儿低着头,道:``嗯。''杜杀道:``听说你箱子里的东西已越来越多了''-小鱼儿道:``嗯\ldots{}''杜杀道:``有什么东西?说出来!''小鱼儿也不敢抬头,嗫嚅着道:``有\ldots\ldots 有一包很臭的药,有一根可长可短的棍子,还可打出许多钉子,还有一瓶药可以把人的骨头和肉都化成水,还有\ldots\ldots{}''杜杀冷冷截口道:``这些东西,可都是屠娇娇和李大嘴给你的。''小鱼儿道:``嗯!''

杜杀道:``听说他两人都已上过你不少次当了,你拿了屠娇娇的东西,就去害李大嘴,拿了李大嘴的,就去害屠娇娇,是么?''小鱼儿道:``嗯''杜杀道:``你不怕他们一怒之下杀了你''小鱼儿道:``我\ldots\ldots 我本来也怕的,但我后来发现,我越坏,害得他们越凶,他们就越高兴,尤其是屠姑姑,她有时根本就是故意被我害的。''杜杀又凝目瞧了他半晌,突然长身而起,道:``随我来!''还没走过那间可怕的屋子,小鱼儿已听见一阵阵吼声,令人听得忍不住要毛骨悚然的吼声。

小鱼儿失声道,``是只大老虎?''

杜杀道:``哼''将门开了一线,叱道:``快进去!''小鱼儿拔出了刀,硬着头皮,走了进去,杜杀背手站在门口。

他有种本事,可以站上四五个时辰都不动。但这一次,小鱼儿进去不久,虎吼声就没有了。

过了半晌,便听得小鱼儿轻唤道:``杜伯伯,开门!''杜杀奇道:``如此之快?''

小鱼儿道:``这还不是杜伯伯教给我的本事?''杜杀道:``哼:''将门开了一线。

突听一声虎吼,一只斑斓猛虎直扑了出来!杜杀委实做梦也未想到自那里出来的是猛虎而非小鱼儿,大惊之下,闪得慢了些,肩竟被虎爪抓破条血口。

那饿虎嗅得血腥气,性子更猛,一扑后又是一剪,变化之快。

竟比武林高手之变招还快几分,声势之猛,更非普天下任何招式与之能比拟,只见满室腥风大作,斑斓虎影流动但``血手''杜杀又是何等人物,身法虽缓不乱,拧身一跃,已掠上虎背,百忙中竟还不忘放声呼道:``小鱼儿你可受伤了?''猛虎未死,死的自然是小鱼儿了。

哪知却听小鱼儿嘻嘻笑道:``小鱼儿没有受伤,小鱼儿在这里。''杜杀不由自主回头一望,只见门梢上笑嘻嘻地坐着个梳着冲天小辫的孩子,嘴里还在嚼着半只苹果。

一时间杜杀也不知道是惊是怒,微一疏神,那猛虎乘势一掀,竟将他身子掀得滚下虎背。

小鱼儿轻呼道:``杜伯伯,小心!''

呼声中那猛虎已翻过身子,向杜杀直扑而下。

这一扑似是十拿九稳,杜杀似是再也逃不过虎爪,哪知他身子一缩,竟自虎腹下窜出,左手向上一抬,只听一声凄厉断肠的虎吼,鲜血就像是雨点般四下飞溅出来。

那猛虎左冲右撞,突然倒地,不会动了。

四面的墙,到处都染满血花,到处都被撞得一塌糊涂,杜杀站起来时,左边已成了半个血人。

原来他左手被燕南天齐腕折断后,便装上个锋利的钢钩,方才他便是以这只钢钩,洞穿了虎腹。

小鱼儿手里的半个苹果也骇掉了,手拍着胸口,吐着舌头道:``好厉害,吓死我了''杜杀木立当地,注视着他,面上既不动怒,也未生气,简直全无丝毫表情,只是冷冷道,``下来''小鱼儿两只手抱着门框,坐滑梯般滑了下来,笑嘻嘻道:``老虎虽厉害,杜伯伯更厉害\ldots{}''杜杀道:``叫你杀虎,你为何不杀?''他半边脸染着鲜血,半边脸苍白如死,在这腥风未息虎尸狼籍的屋子里,那模样教人看来委实恐怖。

但小鱼儿竟似完全不怕,眨着眼睛笑道:``杜伯伯总是要小鱼儿杀虎,小鱼儿总想瞧瞧杜伯伯杀虎的本事。''杜杀道:``你想害我?''

他左边脸上的虎血已自凝成紫色,右边脸却越来越青,地狱中的魔鬼若来和他比比,可怕的一个必定是他。

小鱼儿却笑嘻嘻地瞧着他的脸,笑道:``小鱼儿怎敢害杜伯伯,老虎是杜伯伯抓来的,杜伯伯怎会杀不了老虎\ldots\ldots 这道理小鱼儿早就懂了。''杜杀冷冷地望着他,久久没有说话,他简直已说不出话。

盛夏,在这阴瞑的昆仑山谷里,天气虽不炎热,但太阳照在人身上,仍使人觉得懒洋洋的。

正午,是阳光能照进``恶人谷''唯一的时候,幸好``恶人谷''中的人本就不喜欢阳光,太阳露面的时候越少越好,一只猫懒懒地在屋顶上晒太阳,一只苍蝇懒懒地飞过\ldots\ldots 这就是盛夏正午时,``恶人谷''中唯一在动的东西。

但就在这时,谷外却有个人飞奔而来。

他身后几百丈外都没有人,但他却似背后附者鬼似的,虽已跑得上气不接下气,仍不敢停下来歇歇,他轻功倒也不弱,只是气力十分不济,像是因为连日来奔波劳碌,又象是因为已有许久未吃饭了。

他长得倒也不难看,只是脸当中却生着个大大的鹰钩鼻子。

敌人一瞧他就觉得讨厌。

他身上衣衫本极华丽,而且显然是裁缝名手裁成的,但此刻布已变得七零八落,又脏又臭。

太阳照着他的脸,一粒粒晶亮的汗珠,沿着他那鹰钩鼻子流下来,流进他的嘴,他也似全无感觉。

直到瞧见了``恶人谷''叁个字,他才透了口气,但脚下跑得更快,笔直跑进了那条青石板的街道。

阳光照得屋顶上闪闪发光,每间屋子的门窗都是关着的,瞧不见一个人,听不到一丝声音;这人显然也大为奇怪,东瞧西望,提心吊胆地一步步走过去,又想呼唤两声,却又有些不敢。

突听左面屋檐下有人轻唤道:``喂!''

声音虽不大,但这人却当真吓了一跳,本已苍白的脸色更白了──惊弓之鸟,听见琴弦的声音都害怕的。

他扭过头望去,只见屋檐的阴影里摆着张竹椅,一个十叁四岁的少年,眯着眼斜卧在那里。

这少年赤着上身,身上横七坚八也不知有多少伤疤,他脸上有条刀疤几乎由眼角直到嘴角。

他满头黑发也未梳,只是随随便便地打了个结,他伸直了四肢,斜卧在竹椅上,像是天塌下来都不会动一动。

但不知怎地,这又懒、又顽皮、又是满身刀疤的少年,身上却似有着奇异的魅力,强烈的魅力。

尤其他那张脸,脸上虽有道刀疤,这刀疤却非但未使他难看,反使他这张脸看来更有种说不出的吸引力。

这又懒、又顽皮、又满是刀疤的少年,给人的第一个印象,竟是个美少年,绝顶的美少年。

鹰鼻汉子瞧了他一眼,竟瞧得呆住了──男人瞧他已是如此,若是女孩子瞧见他,那还得了?这少年似乎想招招手,却连手也懒得抬起,只是笑道,"你发什么呆?过来呀。

``鹰鼻汉子果然不由自主地走了过去,轻咳一声,陪笑道:''小哥你好。``少年笑道:''你认不认得我?"

鹰鼻汉子道:``不\ldots\ldots 不认得。''

少年道:"你不认得我,为何要问我好。

鹰鼻汉子怔了证,呐呐道:``这\ldots\ldots 这\ldots\ldots{}''少年哈哈笑道:告诉你,我叫小鱼儿,你呢?"``那鹰鼻汉子终于挺了挺胸,道:''在下杀虎太岁巴蜀东,小鱼儿嘻嘻笑道,``杀虎太岁\ldots\ldots 嗯,这名字不错,你杀过几只老虎呀!''巴蜀东又是一怔,道:``这\ldots\ldots 这\ldots\ldots{}''小鱼儿大笑道:``我杀过好几尺老虎,都未叫杀虎太岁,你一只老虎未杀,却叫杀虎太岁,这岂非太不公平了么?''巴蜀东愣在那里,简直哭笑不得,若非这里就是``恶人谷'',这小鱼儿若非在``恶人谷''中,他早已砍下他的脑袋。

小鱼儿道:``瞧你这祥害怕,你得罪的人,必定来头不小,武功不弱,那□竟是些什么人?你也说来听听。''巴蜀东沉吟半晌,终于道:``我得罪的人可不只一个,那其中有江甫双剑丁家兄弟,病虎常风,江北一条龙田八\ldots\ldots{}''小鱼儿笑道:``我当是谁呢,原来这些人\ldots\ldots 这些人的名字我倒也都听过,但却都也没有什么太了不起\ldots\ldots{}''巴蜀东咬了咬牙,道,``这些人纵然没什么了不起,但其中还有一个,却当真可说是人人见了,人人头疼。''小鱼儿道:``那莫非是大头鬼么?''

巴蜀东不理他,自言接道:``提起此人,在今日江湖中当其是大大有名。''小鱼儿道:``他叫什么?''

巴蜀东道:小仙女张菁。"

小鱼儿笑道:``小仙女?\ldots\ldots 听这名字,她该是个小美人儿才是,别人见了喜欢还来不及,又怎会头疼?''巴蜀东咬牙道:``这丫头长得虽不错,但心肠之狠,手段之毒,下手之辣,纵是昔年之血手杜杀,也未必比得上她!''小鱼儿道:``哦,有这样的人?''」巴蜀东牙齿咬得``吱吱''响,接道:``我五个兄弟,在一夜之间全被她杀了,虎林七太岁,到如今只剩下巴某一个。''小鱼儿笑道:"这样的人,我倒真想瞧瞧。

``巴蜀东道:''你瞄见她时,便要后悔了。"

小鱼儿道:``你再说说,你是怎么得罪他们的?''巴蜀东怒道:``你问的事怎地如此多?小鱼儿笑道:''这是规矩。``巴蜀东瞪着眼睛愣了半晌,终于笑道:''好,我说,只因我兄弟将昔年叁远镇局总镖头飞花满天,落地无声沈轻虹的寡妇和妹妹好了。"

\hypertarget{ux7b2cux4e5dux7ae0-ux9752ux51faux4e8eux84dd}{%
\chapter{第九章
青出于蓝}\label{ux7b2cux4e5dux7ae0-ux9752ux51faux4e8eux84dd}}

小鱼儿望了巴蜀东一眼道:``这也算坏事么?\ldots 一嘿,这种坏事简直只有赶骡车的粗汉才会做的。''巴蜀东怒道:``不错,这本算不得什么,但那沈轻虹昔年虽然丢了漂银,自己虽也失踪,但江湖中人对他的寡妇和妹妹却尊敬得很,所以\ldots\ldots{}''小鱼儿摇头笑道:``无论你怎样说,假如你做的只是这种见不得人的坏事,你还不够资格进恶人谷除非\ldots\ldots{}''``除非怎样?''小鱼儿笑道:``除非你先孝敬两样希奇的东西给我。''巴蜀东道:``我来得如此匆忙,哪有什么希奇之物。''小鱼儿道:``你若没有东西,就露两手成名的绝技给我瞧瞧。''巴蜀东气得脸上颜色都变了,怔了半晌,跺着脚道:``好:''他伸手一抄,便已自腰间抽出柄缅铁软刀,迎风抖得笔直,刀光闪动,``刷、刷、刷''露了叁招。

这叁招果然是他成名绝技,号称``杀虎叁绝手'',刀法果然是干净利落,又快又稳又狠!小鱼儿却摇头笑道:``这也算是绝技么\ldots\ldots 这简直和你做的事一样,完全见不得人,不敌众我看,你若想进恶人谷还得另想法子。''巴蜀东道,``还\ldots\ldots 还有什么法子?''小鱼儿眨了眨眼睛,笑道:``我看你只有跪在地上,向我磕叁个响头,喊我叁声小祖宗,然后双手将这把刀送给我。''巴蜀东道:``这也是规矩?''

小鱼儿道:``不错,这也是规矩!''

巴蜀东嘶声道:``我\ldots\ldots 我从未听过恶人谷有这样的规矩。''小鱼儿笑道,``谁说这是恶人谷的规矩?''

巴蜀东又怔住了,道:``那\ldots\ldots 那么这\ldots\ldots{}''小名儿笑嘻嘻道:``这是我的规矩。''巴蜀东气得连身子都抖了起来,突然大喝道:``好,给你!''一刀向小鱼儿砍了下去!哪知这方才手指都懒得动的小鱼儿此刻却真像是鱼似的,轻轻一动,整个人都滑了出去。

巴蜀东这一刀虽快如闪电,却劈了个空。

``喀嚓''一声,那竹椅已被他生生砍成两半。

巴蜀东大惊,又听身后有人笑道:``我在这里,你瞧不见么?''巴蜀东猛一翻身削去,哪知身后还是空空的,那笑声却从屋檐上传了下来,嘻嘻笑道:``别着急,慢慢来,我在这里。''巴蜀东气得简直快疯了,正待再扑上去。

突听一人大呼道:``那边的是巴二弟么?''

一人大步奔来,只见他和巴蜀东差不多年龄,四十出头,不到五十,但身法却比巴蜀东轻灵得多。

他身子瘦长,嘴角下垂,生得一脸凶狠之相,但右边的袖子却是空荡荡的束在腰里,右臂竟已断去。

巴蜀东瞧了两眼,大喜呼道:``闷雷刀宋叁哥,你!你果然在这里!可找死小弟了\ldots\ldots\ldots\ldots 小弟此番正是投奔叁哥来的\ldots{}''小鱼儿笑道:``原来你们两把刀是朋友\ldots{}''巴蜀东瞧见他,脸色立刻又变了,恨声道:``宋叁哥,这小鬼\ldots\ldots\ldots\ldots{}''话设说完,已被宋叁一把拉了开去,笑道:``二弟既来了,我就先带你见\ldots\ldots\ldots\ldots{}''小鱼儿嘻嘻笑道:``慢来慢来,你要带他走也可以,但叫他先赔我的椅子来再说\ldots{}''巴蜀东怒道:``你\ldots\ldots\ldots\ldots{}''一个字出口,又被宋叁截住笑道:``自然自然,椅子自然要赔的,却不知如何赔法?''小鱼儿笑道:``瞧在你面上,就叫他拿刀充数吧\ldots{}''巴蜀东怒喝道:``这把破竹椅子,也要我宝刀\ldots\ldots\ldots\ldots{}''话未说完,手中刀已被宋叁抢了去,交给小鱼儿,巴蜀东还想说话,但宋叁却拉了他就跑两人走出很远,宋叁方自叹道:``二弟你怎地一入谷就得罪了那小魔星?''巴蜀东又惊又奇,道:``叁哥为何如此怕他?''宋叁苦笑道:``岂只我怕他,这谷中谁不怕他?这几年来,这小魔星可真使人人的头都大了叁倍,谁若得得罪了他,不出三天,准要倒霉.''巴蜀东惊得目瞪口呆,道:这小鬼有如此厉害?``宋叁叹道:''二弟,不是我说,你栽在这小鬼手上,可一点也不冤,你且想想,这恶人谷中可有一个好的,他小小年纪,就能在恶人谷中称霸,他是怎么样的人,他有多厉害,你总可知道了\ldots{}``巴蜀东呐呐道:''不能相信,\ldots\ldots\ldots 小弟简直不能相信\ldots{}``忽然看及宋叁那条空空的长袖,忍不住又道:''叁哥这\ldots\ldots\ldots\ldots 这难道也是\ldots\ldots\ldots\ldots\ldots{}``宋叁苦笑道:''这虽不是他,也和他有些关系\ldots{}``他长叹一声,俯首望着断臂,接道:''这正是他入谷那日断去的,十四年,已有十四年了,燕南天那么厉害的身手,若非我当机立断,只怕已活不到今日\ldots{}``巴蜀东失声道:''燕南天?这小鬼是燕南天的\ldots\ldots\ldots{}``突然惨呼一声,噗地仆倒,背后已赫然多了个碗大的血洞,鲜血涌泉般往外流了出来\ldots 宋叁大骇转身,只见一人鬼魅般人在身后,一身惨灰色的衣服飘飘荡荡,一双黑黝黝的眼睛深不见底宋叁面色惨变,颤声道:''阴\ldots\ldots 阴公,你\ldots\ldots\ldots{}``阴九幽龀牙一笑,阴森森道:''在本谷之中,谁也不准提起小鱼儿和姓燕的事,你忘了?``宋叁道:''我\ldots\ldots 我还未来得及向他说\ldots{}``阴九幽狞笑道:''你还未来得及说,我便已宰了他,你不服是么?``宋叁身子直往后退道:''我\ldots\ldots 我\ldots\ldots{}``身子突然跳了起来,跳起两丈高,笔直摔在地上,身子虽无伤痕,但却再也不能动了!就在他方才站着的地方,此刻却站着个笑眯眯的老太婆手拄着拐杖,佝偻着身子,笑咪咪地道:''阴老九现在怎地也慈悲起来了,这□方才说这一句话,你已该将他宰了的,为何到现在还不动手?``阴九幽道:''我正要留给你\ldots{}``那老太婆笑道:''留给我?我许久没杀人,怕我手痒么?``阴九幽冷冷道:''我要瞧瞧你那销魂掌可有进步?``那老太婆咯咯笑道:''进步了又怎样?你也想销魂销魂?``她苍老的语声,突然变得柔媚入骨,这赫然正是屠娇娇的声音\ldots 屠娇娇笑道:''我问你,这两人方才说话的时候,那小鬼头在哪里?他可听见了么?``阴九幽道:''你不知道,我怎么会知道?"

突听小鱼儿的笑声远远传了过来,笑着道:``醋坛子,皱鼻子,娶个老婆生儿子,儿子儿子没鼻子\ldots\ldots\ldots{}''屠娇娇笑道:``老西又倒霉了,小鬼又找上了他\ldots{}''阴九幽道:``他既在老西那里,想必不会听到\ldots{}''突又听得一人笑道:``两位在这里说话,却有一男一女,一人一鬼两个加在一起,竟变成了四个,你说奇怪不奇怪\ldots{}''屠娇娇头也不回,笑道:``李大嘴,这里有两个死人,还堵不住你的嘴么?''李大嘴笑道:``死在你两人手下的,我还没胃口哩\ldots{}''阴九幽道:``你倒可是也要去杜老大处?''李大嘴道:``正是要去的,哈哈儿突然要咱们聚在一起,不知又要搞什么鬼?''叁个人一起走向杜杀居处,但彼此间却都走得远远的,谁也不愿意接近到另外那人身体一丈之内\ldots 杜杀还是坐在角落里,动也不动\ldots 人都已来齐了,哈哈儿道:``哈哈,哈哈,咱们许久未曾如此热闹了\ldots{}''阴九幽冷冷道:``我最恨的就是热闹,你将我找来,若没话说,我\ldots\ldots\ldots\ldots{}''哈哈儿赶紧拱手,截口笑道:``莫骇我,我胆子小\ldots{}''屠娇娇道:``你找咱们来,莫非为了那小鱼儿\ldots{}''哈哈儿道:``哈哈,还是小屠聪明\ldots{}''阴九幽道:``为了那小鬼,为那小鬼有什么好谈的,你们一个教他杀人,一个教他害人,一个教他哭,一个教他笑,\ldots\ldots\ldots 好了\ldots 他现在不是全学会了么\ldots{}''哈哈儿道:``就因为全学会了,所以我才请各位来\ldots{}''李大嘴道:``为啥?哈哈儿叹了口气,道:''我受不了啦\ldots{}``屠娇娇笑道:''哈哈儿居然也会叹息,想来是真的受不了啦\ldots{}``李大嘴苦着脸道:''谁受得了谁是孩子\ldots{}``哈哈儿道:''如今这位小太爷要来就来,要走就走,要吃就吃,要喝就喝谁也不敢惹他,惹了他就倒霉,恶人谷可真受够了,这几个月来至少叁十个人向我诉苦,每人至少诉过八次\ldots"``穿肠剑''司马烟叹道:``这小鬼委实越来越厉害了,如今他和我说话,我至少要想上六七次才也回答,否则就要上当\ldots{}''李大嘴苦笑道:``你还好,我简直瞧见他就怕,若有哪一天他不来找我,我哪天真是走了运了,哪天我才能好好睡一天觉,否则我睡觉时都得提防着他\ldots{}''哈哈儿道:``咱们害人,多少还有个目的,这小鬼害人却只是为了好玩\ldots{}''屠娇娇道:``咱们本来不就正希望他如此么\ldots?''哈哈儿道:``咱们本来希望他害的是别人呀,哪知这小鬼竟是六亲不认,见人就害,这其中恐怕只有小屠舒服些\ldots{}''屠娇娇道:``我舒服?我舒服个屁,我那几手,这小鬼简直全学会了,而且简直学得比我自己还道地\ldots{}''哈哈儿道:``杜老大怎样?''杜杀道:``嗯\ldots{}''屠娇娇笑道:``嗯是什么意思?''杜杀默然半晌,终于缓缓道:``此刻若将他与我关在一个屋子里那活着出来的人,必定是他\ldots{}''屠娇娇叹了口气道:``好了,现在好了,恶人谷都已受不了他,何况别人,现在只怕已是请他出去的时候\ldots\ldots\ldots\ldots{}''李大嘴赶紧截口道:``是极是极,他害咱们已害够了,何况别人,现在只怕咱们联手还能制他,等到一日,若是咱们加起来也制不住他时,就完蛋了\ldots{}''阴九幽道:``要送他走,越快越好\ldots{}''杜杀道:``就是今朝!''哈哈儿道:``哈哈,江湖中的各位朋友们,黑道的朋友们\ldots 白道的朋友们,山上的朋友们,水里的朋友们,你们受罪的日子已到了\ldots{}''李大嘴以手加额,笑道:``这小鬼一走,我老李一个月不吃人肉\ldots{}''黄昏后,恶人谷才渐渐有了生气\ldots 小鱼儿左逛逛,右逛逛,终于逛到万春流之处,万春流将七种药放在瓦罐里熬,此刻正在观察着药色的变化,瞧见小鱼儿进来,将垂下的眼皮一抬,道:``今日有何收获?''小鱼儿笑道:``弄了把缅刀,倒也不错\ldots{}''万春流道:``刀在哪里?''小鱼儿道:``送给醋坛子老西了\ldots{}''万春流以筷子搅动着药汤,浓浓的水雾,使他的脸看仿佛有些神秘,他道:``你那小箱子呢?''小鱼儿笑道:``小箱子早就丢了,里面的东西都送了人\ldots{}''万春流道:``你辛苦弄来,为何要送人?''小鱼儿笑道:``这些一拿来玩玩倒蛮好的,但若要保留它,可就费神了,又怕它丢,又怕它被偷,又怕它被抢,你说多麻烦\ldots{}''万春流道:``好\ldots{}''小鱼儿笑道:``但若将这些东西送人,这些麻烦就全是人家的了,听说世上有些人专门喜爱聚宝钱财,却又舍不得花\ldots 这些想必都是呆子\ldots{}'':万春流道:``若没有这些呆子,怎显得你我之快乐\ldots{}''突然站了起来,道:``拿起这药罐,随我来\ldots{}''这间药香弥漫的大屋子后面,有一排叁部小房子,这叁间屋子里既没门,也没窗户\ldots 这就是万春流的病房\ldots{}``万春流在这些病房中时,谁也不会前来打扰,因为他们其中任何一人,自已都有睡到这病房中来的可能\ldots 没有灯光的病房,正如万春流的面容一般,显得十分神秘,角落中的小床上,盘膝端坐着一条人影,动也不动,像是亘古以来,他就是这样坐在那里的这正是别人口中所说的药罐子\ldots 一入病房万春流立刻紧紧关起了门,这病人病房就立刻变成一个单独的世界,似乎变得和恶人谷全无关系\ldots 小鱼儿神情也立刻变了,拉着万春流的手,轻声道:''燕伯伯的病,可有起色?``万春流神秘而冷漠的面容竟也变得充满焦虑与关切,长长地叹息了一声,黯然摇头道:''这五年来,竟无丝毫变化,我已几乎将所有的药都试遍了,我我累得很\ldots{}``沉重地坐在椅子上,似是再也不愿站起\ldots 小鱼儿呆呆地出了半天神,突然道:''我今天听见有人提起燕伯伯的名字\ldots{}``万春流动容道:''哦,什么人?``小鱼儿道:''死人!说话的人已死了\ldots{}``万春流一把抓住小鱼儿的肩头,沉声道:''可有人知道你听到了他们的话?``小鱼儿笑道:''怎么会有人知道?我听了这话,立刻远远地溜了,溜到醋坛子那里去,故意大声骂了他一顿,所以我就将那柄刀送给了他\ldots{}``万春流缓缓放松了手,默然垂首,:''不容易,真不容易你虽是小小年纪,但五年来,你竟能将这秘密保守得如此严密\ldots{}``他抬头瞧了小鱼儿一眼,苦笑道:''这秘密若是泄露出去,我们叁个人,都休想再活半个时辰,你你要特别小心,莫把别人都当做呆子\ldots{}``小鱼儿点头道:''我知道\ldots 万叔叔冒了生命的危险来救燕伯伯我我难道不感激,别人就算砍下我脑袋,我也不会说一个字的\ldots{}``说着说着,他眼圈竟已红了\ldots 万春流叹息着道:''说实话,我本不敢相信你的,哪知你虽然-生长在这环境里,却还没有失去良心,还是个好孩子。``小鱼儿展颜笑道:''个鱼儿坏起来可也真够坏的,只是,那却要看对付什么人,而且,自从我知道燕伯伯和我的关系后,我就变得更\ldots\ldots 更乖了广万春流竟也展颜一笑,道:``但五年前那天晚上,你突然跑来对我说,你已知道药罐子''叔叔是什么人,你已知道这秘密时。

``我可当真吓了一跳。''

小鱼儿垂头笑道,``对不起''万春流默然半晌,笑着又皱眉道:``你再想想,对你说出这秘密的人,究竟是谁?''小鱼儿想了想道:"那天晚上,我是睡在杜杀外面的屋子里。

半夜里,我突然觉得身子竟似被人抱了起来\ldots\ldots{}``那时你未叫喊?''小鱼儿道:"我喊也喊不出,何况,那时我还以为是杜杀又不知在用什么花样对付我了,根本没想到是别人。

-万春流叹道,``的确是想不到的\ldots{}''小鱼儿道:``我只觉得那人身法快得简直骇人,我躺在他怀里,就像是腾云驾雾似的,片刻间,就远远离开了恶人谷,万春流道:''那时你真的不怕?``小鱼儿道,''老虎我都不怕,怎会怕人``万春流喃喃道:''你以后就会知道人有时比起老虎可怕得多``小鱼儿道:''那人将我放到地上,就问我:``你姓什么?我说''不知道。``那人就骂我简直和畜牧一样,连姓什么都不知道.''万春流道``然后,他就告诉你你姓江\ldots{}''小鱼儿道:``嗯,他还说我爹爹叫江枫,是被移花官中的人害死的,他叫我千万莫忘了这仇恨,长大一定要找移花宫的人复仇。''万春流道,``他真的没有提起江琴这名字?''小鱼儿道:``没有。''

万春流道:奇怪,你燕伯伯到恶人谷来,本为的是要找个叫``江琴''的人,为的也正是要替你爹爹报仇。``小鱼儿眨了眨眼睛,道:''也许江琴也是我仇人之一."``嗯\ldots\ldots{}''然后,他又告诉我,有关燕伯伯的事,我想问他究竟是谁,哪知他却像是一阵风似的,突然就消失了``万春流叹道:''我知道\ldots\ldots 我知道\ldots\ldots{}``小鱼儿道:''那天晚上很黑,我只瞧见他穿着一件黑袍子,头上也戴着个黑布罩,两只眼睛,又亮又大又怕人\ldots\ldots 这双眼睛我到现在还忘不了。``万春流道:''以后你再见到这双眼睛还能认得么?``小鱼儿道,''一定认得的。"

万春流道:``这双眼睛不是谷中的人?''

小鱼儿道:``绝不是,谷中无论是谁的眼睛,都没有这双眼睛那么亮,屠娇娇的眼睛虽也亮,但和他一比,简直就是睁眼瞎子。''万春流叹道:``此人竟能在恶人谷中来去自如而他又知道这许多秘密,唉!他究竟是谁,实在叫人猜不透。''小鱼儿道:``想必是个武功很高的人。''\ldots 万寿流道:``那是自然,江湖中能随意进出口恶人谷的人,除了你燕伯伯外,我简直想不出还有几个?''小鱼儿道:``一个都没有了么?''

万春流道:``还有的就是移花宫中的大小两位官主,但这人既然要你找移花宫中的人报仇,又怎会是这两位宫主?''小鱼儿突然拍手道:``对了,我想起来了''万春流赶紧追问道:``你想起了什么?''小鱼儿道:``那人是女的''万春流动容道:``女的?''小鱼儿道:``嗯,她虽然蒙着脸,而且故意将说话的声音扮得很粗,但看她的举动,却必定是个女的\ldots{}''万春流道:``什么举动?''小鱼儿道:``比如\ldots\ldots 她头上虽然戴着布罩,但在无意中却还不时去摸头发,还有,她虽然将我抱在怀里,但总是不让我碰到她的胸\ldots\ldots{}''万春流道:``她是女的,可就更难猜了,江湖中女子除了邀月、怜星两人外,我简直再也想不出有一人能在恶人谷中来去自如。''小鱼儿道:``但总是有个人的,第一,这人认得我爹,也认得燕伯伯,这人对我爹爹死的原因知道得很清楚''万春流:``想必如此!''小鱼儿道:``第叁,这人不但知道我家的仇恨,而且,还很关心。第四,这人的武功很高。第五,这人必定和移花官,有些过不去。第六,这人的眼睛又大又亮,和别人的眼睛简直完全不同\ldots\ldots{}''万春流叹道:``不想你小个年纪,分析事情,已有如此清楚,,小鱼儿道:''但\ldots\ldots 但我要去找她,第一先得走出这恶人谷,我\ldots\ldots 我什么时候才能走出去呢?他们什么时候才会放我走?``万春流长叹道:''这就难说了,但愿\ldots\ldots{}``突听外面有人大呼道:''万神医,小鱼儿可是在这里么?``万春流变色道:''屠娇娇来找你了,快出去!"

\hypertarget{ux7b2cux5341ux7ae0-ux8c37ux5916ux98ceux5149}{%
\chapter{第十章
谷外风光}\label{ux7b2cux5341ux7ae0-ux8c37ux5916ux98ceux5149}}

一离开这屋子,两人就又变了。

万春流又回复成那冷漠而不动情感的``神医'',小鱼儿回复成那精灵古怪的顽皮小孩。

屠娇娇斜倚着门,娇笑道,``你们一老一少在干什么?''小鱼儿扮了个鬼脸,笑道:``我们正在商量怎么害你\ldots{}''屠娇娇笑道:``哎呀,你这小鬼,你们若商量着害人,也该商量如何才能做出一种最臭的药来,臭死李大嘴才是,怎么能害我。''小鱼儿笑嘻嘻道,``李叔叔太容易上当了,害他也没意思。''屠娇娇笑道:``哎呀,你听,这小鬼好大的口气,小心李大嘴吃了你。''小鱼儿道:``屠姑姑来找我,究竟为的什么事?''屠娇娇道:``你笑伯伯弄了几样菜,李大嘴弄了两坛酒,我\ldots\ldots 我烧了好大一锅笋烧肉,大家今天晚上要请你宵夜。''小鱼儿眨了眨眼睛,道,``为什么?''

屠娇娇道:``你吃过就知道了。''

小鱼儿摇头笑道:``屠姑姑若不说出原因,这顿饭我可不敢吃,否则我吃过,说不定立刻上吐下泻,叁天起不了床。''屠娇娇笑骂过:``小鬼,好大的疑心病。''

小鱼儿笑道:这可是跟屠姑姑你学的。"

屠娇娇道:``好,我告诉你,大家请你吃宵夜,只是为了要替你送行''小鱼儿这真吓了一跳,失声道:``送行\ldots\ldots 替我送行。''屠娇娇笑道:``小鬼,这次你可想不到了吧?''小鱼儿道:``为\ldots\ldots 为什么要替我送行?''

屠娇娇道:``只因为你今天晚上就要走了。''

小鱼儿张大了嘴,瞪大了眼睛,道:``我\ldots\ldots 我今天晚上就要走?我要到哪里去?''屠娇娇道:``外面呀,外面的世界那么大,你难过不想去瞧瞧么?''小鱼儿摸着脑袋,道:``我\ldots\ldots 我\ldots\ldots{}''屠娇娇咯咯笑道:``何况,你年纪也不小了,也该出去找个老婆了\ldots\ldots 唉,像你这样的小鬼,出去后真不知要迷死多少女孩子。''他拉起小鱼儿的手,又笑道:``万神医,你难道不来为小鱼儿送行么?''万春流木立当地,默然良久冷冷道:``请恕在下不将大好时间浪费在此等事上两位请走吧。''转过身子,大步走了进去。

屠娇娇轻呼道:``这人一脑门子里,除了他那些破树皮、烂草根外,就什么都没有了,就算他亲爹要走,他都不会送行的。''两坛酒一个时辰里就光了。

李大嘴的脸越喝越红,杜杀的脸超喝越青,哈哈儿越喝笑声越大,屠娇娇越喝越像女人。

只有小鱼儿,一杯又一杯地喝着,却是面不改色。

哈哈儿道:``哈哈,这小鱼儿的酒量真不错,喝起酒来,简直就像喝水\ldots{}''小鱼儿笑道:``老是喝水,我可喝不下这么多。''阴九幽冷笑道:``喝酒又非什么好事,有何值得夸耀之处。''屠娇娇笑道:``鬼自然是不喝酒的,但人,人却得喝两杯\ldots\ldots 小鱼儿呀个鱼儿,你可知道,除了一样事外,别的坏事你可都学全了''李大嘴怒道:``什么坏事!这全都是好事!一个人活在世上,若不学会这些好事,可真是等于白活了一辈子。''他说的得意,就想喝酒,但才端起酒杯,``叮''的,整只酒杯突然粉碎,阴九幽冷冷道:``酒是不能再喝了!''李大嘴怒道:``为什么?你凭什么打碎我的酒杯?''阴九幽道:``再喝,小鱼儿就走不成了''李大田狠狠瞪着他,瞪了半天,突然飞起一脚,将酒坛踢带飞出去,咬着牙道:``总有一天,我要灌几坛酒到你肚子里,让你做鬼也得成个醉鬼。''小鱼儿笑嘻嘻地望着他们,笑嘻嘻道:``各位叔叔们这么急者要赶我走,为什么?''屠娇娇道:``小鬼,疑心病,谁急着要赶你走?''小鱼儿笑道:``你们不说,我也知道的。''

屠娇娇道:``你知道?好,你说来听听。''

小鱼儿道``因为小鱼儿越变越坏了,已坏得令各位叔叔伯伯都头痛了,都吃不消了,所以赶紧要送瘟神似的把我送走,好去害别人。''屠娇娇格格笑道:``无论如何,你最后一句话算是说对了的。''小鱼儿道:``你们要我走可以,要我去害别人可以,但这都是为了你们自己,我又有什么好处?你们总得也让我得些好处才行。''哈哈儿道:``哈哈,问得好,你能问出这句话来,也不枉咱们教了你这么多年\ldots\ldots 若没有好处的事,我亲爹叫我做,我也不做的,何况叔叔伯伯?''小鱼儿拍掌笑道:``对了,笑伯伯的话,正说进我心里去了。''李大嘴道:``你放心,我们自然都有东西送给你。''小鱼儿笑嘻嘻道:``那却要先拿来让我瞧瞧,东西好不好,我欢喜不欢喜,否则,我就要赖在这里不走了\ldots{}''屠娇娇道:``小鬼,算你厉害,杜老大,就拿给他瞧吧。''杜杀提着的包袱里,有一套藏青色的锦衣,一件腥红的斗篷,一顶绣着条金鱼的帽子,一双柔软的皮靴。

小鱼儿道:``还有什么?''

屠娇娇笑道:``还有\ldots\ldots 你瞧瞧?''

她打开另一个包袱,包袱里竟是一大叠金叶子,世上能一次瞧见这么多金子的人,只怕没几个。

小鱼儿却皱眉道:``这算什么好东西,饿了既不能拿它当饭吃,渴了也不能拿它当水喝,带在身上又重\ldots\ldots 这东西我不要。''屠娇娇笑骂道:``小笨蛋,这东西虽不好,但只要有它,你随便要买什么东西都可以,世上不知有多少人为了它打得头破血流,你还不要?!''小鱼儿摇头道:``我不要,我又不是那种呆子''李大嘴两根指头夹了一小块金叶子,笑道:``你可知道,就只这一小块,就可以买你身上穿的这种衣服至少叁套,普通人家就可以吃两年。''哈哈儿道:``你不是喜欢马么!就只这一个块,就可以买一匹上好的藏马,这东西若不好,世上就没有好东西了。''小鱼儿叹了口气,道:``你们既将它说得这么好\ldots\ldots 好吧,我就马马虎虎收下来也罢,但除了这些还有什么?''屠娇娇道,``哎哟,小鬼,你还想要?你的心倒是真黑,你也不想想,我们的好东西,这些年来早已被你刮光了,哪里还有什么!''小鱼儿歪着头,想了想,提起包袱,站起来就走。

李大嘴道:``喂喂,你干什么?''

小鱼儿道:``干什么?\ldots\ldots 走呀。''

李大嘴道:``你说走就走?''

小鱼儿道:``还等什么?酒也不准喝了,东西也没有了\ldots\ldots{}''李大嘴道:``你要到哪里去?''小鱼儿道:``出了谷,我就一直往东南走,走到哪里算哪里。''李大嘴道:``你想干什么?''

小鱼儿道:``什么也不干,遇见顺眼的,我就跟他喝两杯,遇见不顺眼的,我就害他一害,让他哭笑不得。''杜杀突然道:``你\ldots\ldots 还回不回来?''

小鱼儿嘻嘻笑道:"我将外面的人都害光了,就快回来了,回来再害你们。

哈哈儿道:``哈哈,妙极妙极,你若真的将外面的人都害得痛哭流涕,咱们欢迎你回来,情愿被你害也没关系。''小鱼儿援了搓手,港:``再见,我很快就会回来的。''竟真的走了,头也不回地走了。

小鱼儿穿着新衣,提者包袱,走过那条街,新皮靴在地上定得``壳壳''作响,在深夜里传得分外远。

他一路大叫大嚷道,``各位,小鱼儿这就走了,各位从此可出安心睡觉了。''两边的屋子,有的开了窗,有的开了门,一个个脑袋伸了出来,眼睛都睁得圆圆的瞧着小鱼儿。

小鱼儿道:``我做了这么大的好事,你们还不赶紧拍掌欢送我\ldots\ldots 你们若不拍掌,我可就留下来不走了。''他话未说完,大家已一齐鼓起掌来小鱼儿哈哈大笑,只有在走过万春流门口时,他笑声顿了顿,瞧了万春流一流\ldots\ldots 只瞧了一眼,没有说话。

万春流也没有说话,有些事是用不着说出来的。

小鱼儿终于走出了``恶人谷''!星光满天,天高得很,虽然是夏夜,但在这藏边的阴山穷谷中,晚风中仍带者刺骨的寒意。

小鱼儿围起了斗篷,仰视着满天星光,呆呆地出了会儿神,如此星辰,他以后虽然还会时常瞧见,但却不是站在这里瞧了。

他立刻要走到一个陌生的天地中,他怕?他不怕的!他心里只是觉得有种很奇怪的滋味,也说不上是什么滋味。

但是他没有回头,他笔直走了出去。

黄昏,山色已被染成深碧。

雾渐渐落下山腰穹苍灰黯,苍苍茫茫,笼罩着这片一望无际的大草原,风吹草低,风中有羊嗥、牛啸、马嘶混合成一种苍凉的声韵,然后,羊群、牛群、马群,排山倒海般合围而来。

这是幅美丽而雄壮的图画!这是支哀宛而苍凉的恋歌。

黑的牛,黄的马,白的羊,浩浩荡荡奔驰在蓝山绿草间,正如十万大军长驱挺进!小鱼儿远远地瞧着,脸上闪动着兴奋的光,眸子里也闪着光,这是何等伟大的景象!这是何等伟大的天地!由薄暮,至黄昏,由黄昏,至黑夜,他就那样呆呆地站在那里,他的心胸己似突然开阔了许多。

兽群终于远去,远处却传来了歌声,歌声是那么高亢而清越,但小鱼儿却听不出唱的究竟是什么。

他又听出歌曲的起端总是``阿位\ldots\ldots{}''他自然不知道这两个字的意思就是游牧回民所信奉的神祗。他只是朝歌声传来处走了过去。

星光在草原上升起,月色使草浪看来有如碧海的清波,小鱼儿也不知奔行了多久,才瞧见几顶白色的帐篷点缀在这无际的草原中,点点灯光与星光相映,看来是那么渺小,却又是那么富有诗意。

小鱼儿脚步更紧,大步奔了过去。

帐篷前,有营火,藏女们正在唱歌。

她们穿着鲜艳的彩衣,长袍大袖,她们的柔发结束无数根细小的长辫,流水般垂在双肩。

她们的身子娇小,满身缀着环佩,焕发着珠光宝气的金银色彩,她们的头上,都戴着顶小巧而鲜艳的呢帽。

小鱼儿瞧得呆了,痴痴地走过去,走到她们面前。

藏女们瞧见了他,竟齐歇下了歌声,拥了过来,吃吃地笑着,摸着他的衣服,说些他听不懂的话。

藏女们本就天真、多情而爽朗。

小鱼儿忍不住笑道:``你们说的什么?''

一个辫子最长、眼睛最大、笑起来最甜的少女甜笑着道:``我们说的是藏语,你\ldots\ldots 你是汉人?''小鱼儿眨了眨眼睛,道:``大概是吧,''你叫什么名字?``大眼睛抿着嘴娇笑道:''我的名字用汉语来说,是叫做桃花,因为,他们许多人都说我的脸\ldots\ldots 我的脸像桃花。"这时帐篷中又走出许多男人,个个都瞪大着眼睛,瞧着小鱼儿,他们的身子虽不高大但却都结实得很。

小鱼儿道:``我要走了。''

桃花道:``你莫要怕,他们虽瞪着眼睛,却没有恶意。''小鱼儿道:``我不是怕,我只是要走了。''

桃花大眼睛转动着,咬着嘴唇,轻道,``你不要走,明天\ldots\ldots 明天早上,会有很多像你一样的汉人到这里来的,那一定热闹得很,好玩得狠。''小鱼儿道:``很多人\ldots 我这一路上简直没有见过十个人。''桃花道:``真的,我不骗你。''

小鱼儿道:``那么,今天晚上\ldots\ldots{}''桃花垂首笑道:"今天晚上,你就睡在我帐篷里,我陪你说话,她比小鱼儿还高些,风吹起她的发辫,吹到小鱼儿脸上,她的眼睛亮如星光。

这一夜,小鱼儿睡得舒服得狠,他平日虽然惊醒,但这一夜却故意睡得很沉,故意不被任何声音吵醒。

他醒来时,桃花已不在了,却留了瓶羊奶在枕旁。

小鱼儿喝了羊乳,穿好衣服,走出去,便瞧见两丈外已多了一圈帐蓬,这边的人已全部走到那边。

他远远就瞧见林花站在一群藏人和汉人的中间,甜甜地笑着,吱吱喳喳像小鸟般说着话。

她的小辫子随着她的头动来动去,她的脸在阳光下看来更像是桃花,怕的只是世上没有这么美的桃花。

她每说几句话,就有个藏人和一个汉人走出来,握一握手,显然是做成了一笔交易,每做成一笔交易,她的笑也就更甜。

小鱼儿走过去,也没有叫她,只是四下逛着,只见每座帐篷门口,都摆着珍奇的玩物,奇巧的首饰。

一些胖胖瘦瘦、高高矮矮的汉人,就守在这些摊子旁,另一些胖胖瘦瘦、高高矮矮的藏人,指手划脚地向他们买东西。

小鱼儿瞧得很有趣,他觉得这些人都愚蠢得很,他忽然发现世上愚蠢的人远比聪明的人多得多。

一个又高又瘦的人,牵着匹健壮的小马走了过来,雪白的马鬃在风中飞舞着,吸引了小鱼儿的目光。

小鱼儿忍不住走过去,问道:``这匹马卖不卖?''那瘦子上下瞧了他两眼,道:``你要买?叫你家的大人来吧。''小鱼儿笑道:``何必还要叫大人,有银子的就是大人\ldots{}''那瘦子笑了,道,``你有银子?''小鱼儿拍了拍腰,道:``银子不多,金子却不少。''那瘦子嘴笑得更大了,眼睛死死盯着他腰带上系着的包袱,手摸着那匹幼马的柔毛,知道:``这马可是匹好马,价钱可以高些。''小鱼儿笑道:``随便什么价钱,你只管说吧''那瘦子眼睛闪着光,吱唔着道:``这匹马要一百\ldots\ldots 至少要一百九十两银子。''小鱼儿想了想,摇头道,``这价钱不对。''

那瘦子脸上的笑立刻不见了,沉着脸道:``怎么不对?你更知道,这是匹宝马,这最少\ldots\ldots{}''」小鱼儿笑道:``这既然是匹宝马,所以至少该值叁百八十两银子,一百九十两简直太少了,简直少得不像话\ldots{}''那瘦子楞住了,突又怒道:``你在开玩笑?''小鱼儿笑道:``金子是从来不开玩笑的\ldots\ldots 一两金子是六十两银子,叁百八十两合金子六两叁钱叁分叁,这块金叶子大概是七两,喏,拿去。''那瘦子这才真的愣住了,迷迷糊糊地接过金子,迷迷糊糊地递过马□,若不是手抓得紧,连金子都要掉到地上。

小鱼儿笑嘻嘻地牵着马,逛来逛去。

他发现这些人不但愚蠢的比聪明的多,丑的也比俊的多,只有个白衣少年,模样和这些人全都不同。

这少年远远地站在一边,似是不屑与别人为伍。

他负着手,白色的轻衣,在风中飘动着,就像是昆仑山头的白雪,他的眼睛,就像是昨夜草原上的星光。

小鱼儿的大眼睛不觉多瞧了他两眼,他的大眼睛也在瞪着小鱼儿朝他笑笑,他却连眼睛都没有眨一眨,小鱼儿朝他皱了皱鼻子,伸了伸舌头,做了鬼脸,他却将头转过去,再也不瞧小鱼儿一眼。

小鱼儿喃喃道:``你神气什么,你不睬我,我难道还要睬你!''他故意特声音说得很大,故意要让那少年听见。

那少年却偏偏听不见。

小鱼儿就走过去,走到离他最近的一个摊子上,摊子上的膺品首饰,也在闪着光,像是只等着别人来上当\ldots 小鱼儿拈起朵珠花,眼睛瞧着少年,小声道:``这卖不卖?''答话的却不是那少年,而是个戴着高帽子的矮胖子,笑得满脸肥肉都像是长草般起了波浪。

他嘻嘻笑道:``小少爷眼光真不错,这种上好的珍珠,市面上可真不多。''他眼睛也瞧着小鱼儿腰里的包袱,他方才已瞧见小鱼儿买马的情况。

小鱼儿道:``多少?''

那胖子道,``四\ldots\ldots 五\ldots\ldots 七十两。''

小鱼儿叫道:``七十两?''

那胖子吓了一跳,道:``七\ldots\ldots 七十两不多吧。''小鱼儿道:``但这珠子是假的呀。''

那胖子道:``假的,谁说是假的,这\ldots\ldots 简直\ldots\ldots 是侮辱我。''他不笑的时候,那张脸就像是堆死肉。

小鱼儿嘻嘻笑道:``我从两岁的时候,就开始用珍珠当弹子打,这珍珠是真是假,我只要用鼻子嗅嗅也知道的。''那胖子暗中几乎气破了肚子:``这小子怎地突然变得精明起来了?''脸上却作出一副受了委屈的摸样。道:``那\ldots\ldots 那么就六十两\ldots\ldots{}''小鱼儿大笑道:``你又错了,真的珍珠,只要从海里捞就有了,假的珍珠却要费许多工夫去做,而且做得这么像,那本该比真的贵才是。''那胖子怔住了,结结巴巴,道:``这\ldots\ldots 那\ldots\ldots 嗯!''小鱼儿道:``真的要六十两,假的最少要一百四十两,合金子二两多\ldots\ldots{}''他就希望那少年瞧他一眼,朝他笑笑。

谁知那少年非但不瞧他,还走开了。

小鱼儿赶紧将金子往地上一抛,道:``这里是叁两。''他也不瞧瞧胖子那张吃惊得象是被人揍了一拳的脸,赶紧去追,但那少年却已不知到哪里去了。

\hypertarget{ux7b2cux5341ux4e00ux7ae0-ux5f04ux5de7ux6210ux62d9}{%
\chapter{第十一章
弄巧成拙}\label{ux7b2cux5341ux4e00ux7ae0-ux5f04ux5de7ux6210ux62d9}}

小鱼儿觉得有些失望,正咬着嘴唇发呆,突然一只手伸过来,拉着他就跑,那柔软而温暖的小手,正是桃花。

她拉着小鱼儿,小鱼儿拉着她,一路跑回她的帐蓬里,她的脸更红,轻轻喘着气,轻轻跺着脚,娇嗔道:``你\ldots\ldots 你这小呆子,''要买东西,也不来找我,却去上人家的当,这匹马连八十两都不值,这珍珠\ldots\ldots{}``小鱼儿道,''这珍珠最多只值十两。"

桃花怔了怔,道:``你\ldots\ldots 你\ldots\ldots 你知道?''

小鱼儿笑道:``我这样聪明的人,还会不知道?''桃花道:``你知道了还要上当?''

小鱼儿眨眨眼睛,笑道:``上当有时就是占便宜。''桃花瞪着眼睛瞧着他,像是在瞧什么稀奋古怪的怪物似的。

她实在一辈子也没瞧见过这么奇怪的孩子。

小鱼儿将珠花插上她的鬓角,笑道:``好姐姐,莫要生气了,你瞧,你戴上这珠花多美,就像是个公主,只可惜,这里却没有配得上公主的王子。''桃花``噗哧''一笑,道,``你不就是个傻王子么!''小鱼儿又眨眨眼睛,道:``你说我傻\ldots\ldots 过一会儿你就知道我不傻了,你就会知道,方才要我上当的人,立刻就要上我更大的当了\ldots{}''桃花忍不住轻叹道:``你真是奇怪的孩子,你说话,总是要人听不懂,你做事,也总是叫人猜不透。''小鱼儿还未说话,帐篷外突有一阵人声传了过来。

一个嘶哑的语声嚷道:``方才买马的那位小少爷可在帐篷里?''小鱼儿做了个鬼脸,轻笑道:``上当的送上门来了。''他突然将桃花推到被窝里,道:``乖乖地躺着,莫要动,莫要说话。''``桃花一肚子狐疑,怎肯不说话,但话还未说出口时,小鱼儿却已用被子蒙住了她的头,大声道:''我在这里,你们进来吧。"进来的是少有十个人,领头的正是那卖马的瘦子。十个人手里都捧着个大大小小的包袱,那卖珠花的胖子手里捧着的包袱最大,压得他整个人都似已变成圆的。

小鱼儿故意皱眉道:``你们干什么?这么多东西\ldots{}'',那瘦子躬身笑道,``常言说得好,货要卖识家,这些人听说小少爷是识货的,却要将好货色送来让少爷您礁瞧\ldots{}''小鱼儿嘻中笑道:``你们不是要来让我上当吧\ldots{}''那瘦子赶紧道:``焉有此理,焉有此理\ldots\ldots 各位还不快将包袱打开,让这位少爷瞧瞧。''话还没说完,包袱已一齐打开了。这些包袱里好东西果然不少,有珍宝、首饰,还有珍贵的皮毛、鹿角、麝香\ldots\ldots 这些简直就是他们刚从藏人手里买来的\ldots 小鱼儿笑道:``这些东西都不错,我都想买。''.十个人一齐喜笑颜开,笑得连嘴都合不拢来,齐声道:``少爷一齐买下最好,小鱼儿道:''好,全给我包起来吧!"几个人七手八脚,将十个包袱变成了一个,包袱已比小鱼儿的人还大了,普通的人简直搬不动。

那胖子终于忍不住道:``但\ldots\ldots 但货款\ldots\ldots{}''

小鱼儿笑道:``你要银子?这还不容易,多少银子,随你们说吧。''几个人立刻七嘴八舌将自己货物的价钱说了出来,每样东西都说得比实在价钱最少要多七八倍。

桃花在被里听得已忍不住跳了起来,却被小鱼儿一只手按住了她的头,她连动也不能动。

只听小鱼儿笑道:``加起来一共多少?''。

那瘦子算得最快,道:``一共六千六百两。''小鱼儿摇头道:``这价钱不对。''那胖子和瘦子都已听过这句话了,都知道这位小少爷有把价钱再加一倍的脾气,别人自然也早已听说过这种``好脾气''、``好习惯''。

大家赶紧一齐陪笑道:``是,这价钱不对,少爷您说价钱吧。''小鱼儿道:``我说?你们只怕\ldots\ldots{}''。

几个人又一齐抢着道:``小人们绝对没有异议。''小鱼儿笑嘻嘻道:``既是如此\ldots\ldots 好,我说,这些东西加起来,我一共给你们\ldots\ldots{}''他又打开包袱,大家的眼睛又直了。

只见他两只手指夹下一小块金叶子,笑道:``我一共就给你们一两吧。''几个人一齐呆住了,那瘦子结结巴巴,强笑道,``少爷你\ldots\ldots 你在开玩笑?小鱼儿脸一板,道,''我早已说过,你们既要我说价钱,而且声明绝无异议,此刻要想反悔,已来不及了。"他将那个块金子在地上一抛,举起包袱就走,这包袱虽比他人还大,但他举在手上却毫不费力。

桃花这才忍不住笑出来,悄悄探出了头,只见那几个人呆了呆,一齐怒喝着追了出去。

几个人一齐大骂道:``小骗子,还咱们东西来。''又听得小鱼儿道:``谁是骗子!你们才是骗子。''接着,便是一连窜``哎哟、呀\ldots\ldots 救命\ldots\ldots{}''之声,还有一连串``砰砰咯咚''好象重物坠地的声音。

桃花忍了半晌,终于忍不住站了起来,跑出去一瞧,只见那些人已没有一个是站着的。

这十来条大汉竟被小鱼儿打得七零八落,有的被打肿了脸。

有的摔断了腿,一个个躺在地上,到现在还爬不起。

桃花也不觉惊得呆了,她知道这些敢到关外来做买卖的江湖客,非但力气都不小,手底下也都有两下子!

她实在想不到那奇怪的孩子竟有这么大的本事。

她呆了半晌,才转头去瞧\ldots\ldots 阳光,照着柔软的草地,那奇怪的孩子和那匹小白马,却已都不见了。

小白马驮着包袱,个鱼儿牵着白马,一人一马直跑出四五里地,小鱼儿一想起那些人物模样,还忍不住要笑。

已将正午了,太阳已越来越热,小鱼儿虽还不觉得怎样,但那匹马却已经有些吃不消了\ldots\ldots{}

大草原上瞧不见人烟,也没有遮萌的地方。

小鱼儿眼珠子转了转,突然将包袱打开,拿了羚羊的角,瞧了瞧,笑了笑,远远抛了出去\ldots 他一路走,一路抛,竟将那一包价值千金的珍贵之物,笑嘻嘻地随手抛了,就像是丢草纸似的。

至最后包袱里剩下的已不多,小鱼儿索性将它们又包成一包,远远地抛入长草之间,这才拍手笑道:``痛快呀痛快!\ldots{}''突然远处有人娇唤道,``小鱼儿\ldots\ldots 江鱼\ldots\ldots 莫要走,等等我!''一匹马飞驰而来,马上人衣服闪着光,十几条又黑又亮的小辫子,在风中飞扬,那张脸正红得有如桃花。

小鱼儿拍手笑呼道:``好骑术\ldots\ldots 好漂亮!''

马弛到近前,桃花已站到马上,突然一个筋斗翻下来,小鱼儿刚吓了一跳,桃花已站在他面前。她咬着嘴唇,跺着脚,大眼睛里水汪汪的,似乎刚哭过,又似乎刚要哭,她喘息着娇嗔道:``你\ldots\ldots 你不说一声就走?你\ldots\ldots{}''小鱼儿笑道:``我惹了麻烦,再不走就要连累你了。''桃花跺脚道:``那\ldots\ldots 那你为什么要骗别人?''小鱼儿道:``他们骗我,我为什么不可以骗他们?''桃花又怔住了,转着大眼睛,道:``东西呢?''小鱼儿道:``全都丢了。''

桃花吃惊道:``丢了?你\ldots\ldots 你为什么?''

小鱼儿笑道:``让那些东西坐马,我却在这么大太阳下走路,我岂非也变成了呆子了,我自然要把它们丢光。''桃花睁大了眼睛,道:``但\ldots\ldots 但那些东西都值钱得很,你不在乎?''小鱼儿笑道:``这又有什么关系?我自然不在乎,反正天下值钱的东西又不止这些,办要我想要,我随时都可以要得到的。''桃花道:``你\ldots\ldots 你简直是个小疯子。''

小鱼儿哈哈大笑,过了半晌,又道:``我将这些东西抛在地上,总有人会拾到的,他们若是好人,拾着这些东西一定开心得要死,我只要想想他们拾着这些东西时的脸,也觉得很开心了,那总比自己还要花心思带着它们走好得多。''桃花道:``他们若是坏人呢?小鱼儿道:''这些东西若被坏人拾着,一定会因为分赃不均而打起来,打得你死我话,头破血流,其中若有人独吞,甚至还会将别人都打死!``桃花失声道:''这样你也开心么?"

小鱼儿道:``我为什么不开心?我简直太开心了。''桃花睁大眼睛,道:``你\ldots\ldots 你简直是个小坏蛋。''小鱼儿道:``还有,这些东西若被懒骨头拾着,一定什么事都不想做了,整天都要去草丛里找了,四处去找\ldots\ldots 一直找到饿死为止。''他咯咯笑着,接道:``你瞧,我只不过是抛了这些东西出去,却显然不知要把多少人一生的生命都改变了,这岂非天下最好玩的事?''桃花整个人像是木头人似的呆住,呆了半晌,轻叹一声,道:``你简直是个小魔王。''小鱼儿道:"``,你方才骂我是呆子,现在又骂我是疯子、坏蛋、魔王,我既是如此,你为什么还要来追我?''。

桃花的头垂了下去,道:"我\ldots\ldots 我只是\ldots\ldots 只是来问问你。

为什么\ldots\ldots 为什么连招呼都不打一个,就这样走了。``小鱼儿道,''既然反正是要走的,还打什么招呼?打个招呼又有什么用?\ldots\ldots 假如打个招呼能令你忘了我,我打个招呼也无妨,只可惜你总是忘不了我的。``挑花霍然抬起头,大声道:''你怎知我忘不了你?``小鱼儿笑嘻嘻道,''只要见过我的人,都忘不了我"桃花瞪着眼睛瞧他,不知怎地,泪珠竟已流上面颊。

小鱼儿道:``你哭什么,反正我年纪太小,也不能做你的丈夫,何况,你生得这么漂亮,也不怕找不着丈夫的。''挑花嘶声道:``你\ldots\ldots 你简直是个\ldots\ldots 是个\ldots\ldots{}''她实在再也找不出一个名词来形容这个``小怪物'',狠狠跺了跺脚,突然飞身上马,拼命地打着马屁股,飞驰而去。

小鱼儿摇头叹道:``女人\ldots\ldots 唉,原来女人都有些神经病\ldots{}''他抚摸着那个白马柔软的鬃毛,喃喃道:``马儿呀马儿,你若也和我一样聪明,就千万莫要接近女人,更莫要被女人骑」否则你就要倒霉了,女人生气时,就要将你当出气筒\ldots\ldots 唉,那匹马的屁股,''应怕已要被桃花打肿了\ldots"他骑上马、往前走,突然瞧见一个人挡住了他的去路。

阳光下,只见这人雪白的衣衫,发亮的眼睛,虽然满面怒容,但看起来却一点也不可怕,反觉可爱得很。

小鱼儿认得他正是那``很神气''的白衣少年,不禁笑过道,原:来你到这里来了,站在这里晒太阳么?``白衣少年冷冷道:''正在等你!"

小鱼儿笑了,道:``等我?你方才不理我,现在却\ldots{}''白衣少年叱道:``少废话,拿来!''。

小鱼儿奇怪道:``拿来,拿什么?''。

白衣少年道:``你骗走的东西。''\ldots 小鱼儿又笑了,道,"哦,原来你是说那些东西,早知道你要。

我就留给你了,但现在\ldots\ldots 唉,全都被我丢了\ldots"。

白衣少年怒道:``丢了?哼,你想骗谁?!\ldots 小鱼儿道:''我为何要骗你?那些废物我留着又有什么用?``他又笑一笑道:''喂,你知不知道,你生气的时候,脸红红的,漂亮得很,简直就像是个女孩子\ldots\ldots 我真的认识个女孩子生气时脸也是红红的,也很漂亮,看来倒和你做是天生的一对.要不要我介绍给你?\ldots{}``那白衣少年脸更红了,想作出凶狠的佯子,却偏偏作不出来,只有用那双大眼角瞪着小鱼儿,厉声道:''你若真的将那些东西丢丁,就得赔。``小鱼儿道:''你真要我赔?"

白衣少年道:``当然要赔!''

小鱼儿道,``你真是为追东西来的?''

白衣少年大声首:``当然!''

小鱼儿道:``只怕未必吧,那些笨蛋是死是活,你都不会放在心上,何况不过被骗了些东西,这本是他们罪有应得,你\ldots\ldots 你只怕不是来追东西,而是来追我的。''白衣少年红着脸喝道,``不错,我就是来追你的,我瞧你小小年纪就已这么坏了,若是长大了那还得了!''小鱼儿摸了摸头,笑道:``你要杀我?''

白衣少年道,``哼,杀了你本也不冤,只是\ldots\ldots 你年纪还小,还未必不可救药,若肯拜我为师,我好好管教管教你,也许还可成器\ldots{}''小鱼儿瞧着他,突然大笑起来,弯着腰笑道:``你想收我做徒弟?''。

白衣少年怒道:``这有什么好笑?''

小鱼儿笑道:``有你这样漂亮的小伙子做师父,倒也不错,只是,你能教我什么?你哪点比我强?我做\ldots\ldots 你做我的徒弟倒差不多。''白衣少年冷笑道:``你想不想学武功?''。

小鱼儿笑道:``你以为你武功比我强?''

白衣少年怒道:``你可知道我乃川中第一高手!''小鱼儿缓缓道:``你若真是高手!就不会逃到这里来了,是么?你既不是来做生意,也不是来玩的,却到了关外,想必是要逃避别人的追踪,是么?''白衣少年面色立刻变了,小鱼儿这句话,正说中了他的心事,他眼中真的射出了凶光,喝道:``你究竟是什么人?究竟是何来历?''小鱼儿笑道:``你莫管我是什么人,也莫管我是何来历,你若认为你的武功高,不妨和我比,谁输了谁就做徒弟。''白衣少年冷笑道:``好,我正要瞧瞧你的武功是何人传授?''小鱼儿笑道:``谁输了谁做徒弟,这可是你自己答应的,不准赖\ldots\ldots{}''话犹未了,身子突然自马上飞起,凌空踢了两脚,直取那少年双目。

白衣少年倒未想到小鱼儿出手竟是如此迅急,倒真吃了一惊,但这少年非但武功真的不弱,与人交手的经验,竟也似丰富得很,惊慌之中,居能不退反进一身子一偏,已到了小鱼儿背后。

头也不回,反手一掌挥出,这一掌不但掌势迅急,而且姿势优美。

认穴之准,更似背后也生着眼睛。

小鱼儿本想一招就抢得先机,哪知先机却被人占了,突然双足一收,凌空翻了筋斗,落在五尺之外,笑道:"等等再打。

白衣少年只得停下进击之势,道:``等什么?''小鱼儿道:``你真能瞧出我武功是何人传授?''白衣少年冷笑道:``十招之内。''

小鱼儿摇着头笑道:``我不信''

他脸上笑容笑得正甜,双拳却已击出,他笑容虽和善,出手却狠辣,这正是他从哈哈儿那里学来的法子。

那白衣少年果然上了当了,虽然未被这两拳击中,但方才占得的先机已失,竟被小鱼儿一轮抢攻逼退数步。

小鱼儿嘻嘻笑道:``我看你还是\ldots\ldots{}''

一句话未说完,这少年突然欺身扑了进来,竟拼着挨小鱼儿两拳,一个肘拳走向个鱼儿胸膛,用的竟是存心和小鱼儿同归于尽的抬式!这次是小鱼儿吃了一惊了,他可不想挨这一举,反甩手,大仰身,身子``嗖''的倒窜了出去。

但那少年哪肯放松,如影随形,跟了过去,双拳如雨点般密密击下,用的竟全是拼命的招式。小鱼儿两只手忽拳忽掌,他的招式忽而狠快,忽而诡谲,忽而刚烈,忽而阴柔,忽而不刚不柔,不软不硬。他正是已将杜杀武功之狠辣,阴九幽之诡谲,李大嘴之刚烈,屠娇娇之阴柔,以及哈哈儿之变化集于一身。这样的武动,在江湖中本已少有敌手,谁知这少年的拳法简直有如狂风暴雨一般,竟打得小鱼儿喘不过气来。但这少年心里也正在暗暗吃惊,他实在也想不到这孩子武功的变化竟有如此之多,他实在瞧不出是何门路。

突听小鱼儿大声道:"喂,住手。!

白衣少年道:``好,我住手!''

``我住手''叁个字说出来时,他己攻出六拳。

小鱼儿左避右闪,乘隙还了叁拳,大叫道:``这样也算住手么?''白衣少年冷笑道:``这次我不上你的当了。''

小鱼儿边打边嚷,道,``但十招已过去了,早已过去了,你可瞧出我的武功门路,你若瞧不出就快住手听我说\ldots\ldots{}''白衣少年的拳势不由得一缓,小鱼儿已乘机退出数尺,笑嘻嘻道:``你瞧出了么?白衣少年只得也停住了手,冷笑道:''自然瞧不出,你武功简直没有门路\ldots{}``小鱼儿大笑道,''不是没有门路,只是门路大多,瞧得你眼都花了。``白衣少年道:''门路众多?是哪些门路?"

小鱼儿道:``告诉你,我武功是从五个人学来的,这五个人的武功又不知包括了多少门路,每个人的武功都是又复杂、又奇怪\ldots\ldots{}''。

白衣少年道:``中土武林名家武功路数,可说绝无一家我不知道,也绝无一家与你的武功路数相同,你那五个师父只怕是卖膏药,练把式的吧。''小鱼儿笑道:"练把式的\ldots\ldots 嘿嘿,这五人的名字说出来,不。

吓你一跳才怪,只是这五人归隐时你只拍还在穿开裆裤,你自然不知道。``白衣少年怒道:''此等旁门左道,又怎能与我的武功相比!``小鱼儿道,''你的武功\ldots\ldots 喂,倒也不错,但你瞧你这种文文静静、秀秀气气的模样,实在猜不透你竟会学那种疯子般不要命招式。``。''白衣少年道:``哼,你知道什么?我这疯狂一百零八打,在当今武林各门各派的掌法中,纵不能列第一也可算第二。''

\hypertarget{ux7b2cux5341ux4e8cux7ae0-ux610fux5916ux98ceux6ce2}{%
\chapter{第十二章
意外风波}\label{ux7b2cux5341ux4e8cux7ae0-ux610fux5916ux98ceux6ce2}}

小鱼儿拍掌大笑道:``疯狂一百零八打,哈哈,果然是疯子才会使的拳法,只可惜这么漂亮的人,却学这种疯子的拳法,真教人看着难受。''白衣少年道:``看起来虽难受,用出来更教别人难受。''小鱼儿笑道:``我可不难受,我也不要学\ldots\ldots{}''``学''字出口,人已扑了上去,``呼呼''就是两掌\ldots\ldots{}

这一次白衣少年却已学乖了,早已在暗中防范,小鱼儿这两掌攻来,他早已击出两拳,封住了小鱼儿的掌路。

这一次小鱼儿也学乖了,绝不用他硬接硬封,只是展动身形,左一拳,右一拳,围着他打转,和他游斗。

但这``疯在一百零八打''威力实是惊人,这种``疯狂''的武功,委实比杜杀之狠辣,阴九幽之诡谲,李大嘴之刚烈,屠娇娇之阴柔都要厉害得多,果然打得小鱼儿非常难受!

小鱼儿又接了数十招,突又喝道:``住手,你这拳法果然不错,我愿意学了。''白衣少年身子一转,转出五尺,胸膛微微起伏,也有些喘息,心想:这小鱼儿可真是有点不好斗。

小鱼儿笑道:``怪不得别人常说,好好的人绝不能和疯子打架,因为他绝对打不过疯子的,如今我才知道这话果然不错。''白衣少年道:``如今你可知道厉害了么?''

小鱼儿道,``只可惜你不是疯子,否则你使出的这套拳法,一定更要厉害\ldots\ldots 怕只怕你将这套拳法用久了,也会变得有些疯味了\ldots{}''。

白衣少年皱眉道,``你既要拜我为师,怎地如此无礼?''小鱼儿笑道,``我只说要学这套拳法,可没说要拜你为师。师父一样也可以向徒弟学拳的,你说是不是?''白衣少年怒道:``你还想打么?''

小鱼儿大笑道:``不能打了,不能打了,你只要再一出手,立刻就要七窍流血而死,我好心告诉你,你可莫要不信''白衣少年怒极之下,反倒不觉笑了,道:``你这小鬼满嘴鬼话,也想来骇我\ldots{}''小鱼儿道,``骇你?我可不是骇你,你可知道武林中有种绝传的秘密,叫六步阴风掌。这就是说,无论是谁,只要在七步内被这种掌风击中,除非他站着不动,否则他走不出七步,嘿嘿,就要送终。''白衣少年道:``鬼话,世上哪有这种拳法。''

他嘴里虽在说``鬼话'',脚却又有些发软,再也不敢动了。

小鱼儿瞧着他的嘴,笑道:``这种掌法绝传已有百年,你自然不知道,但我却在无意中得到绝世奇缘,学会了这种掌法,而且\ldots\ldots{}''白衣少年冷笑道,``而且还打了我一掌,是么?''他虽然故意要作出不信的样子,但此刻无论是谁,也不能在教他再走七步了,七步阴风掌名字已够吓人!

小鱼儿拍手笑道:``这次你说对了,不过,我只打了一掌,轻轻的一掌,只要你拜我为师,我还可将你救活。'':白衣少年冷笑道:``你若以为几句话就可将我吓倒,你就大错而特错了\ldots{}''小鱼儿道:``你不信?好,你且摸摸你左面第叁根肋骨下是不是有些发疼,这就是中了七步阴风掌的征象。''白衣少年道:``哼\ldots\ldots{}''

他嘴里虽在``哼哼哈哈'',手却不觉已向左面第叁根胁骨下摸了去,脸上也已不觉变了颜色!

小鱼儿垂头瞧着脚下的影子,道:``怎么样,疼吧?''白衣少年指尖已有些拌,口中却大声道,"自然痒的,任何人这地方都是最容易觉得痒的\ldots。

小鱼儿道:``但这不是普通的疼,是特别的痒,就好像被针刺,被火烧一样,疼得热辣辣的,疼得叫人咧嘴!''他目光自地上抬起,瞪着白衣少年的手,缓缓道``你再摸,不是这里,再往左一点\ldots\ldots 再往下一点\ldots\ldots{}''白衣少年的手指,不知不觉已随着他的话在动了。

小鱼儿突然叫道:``对了,就是这里,用力往下按!''白衣少年手指不知不觉用力一按\ldots\ldots。

他身子突然一阵麻木,``噗''地听从,再也不会动了!

小鱼儿拍掌大笑道:``饶你精似鬼,也要喝我的洗脚水,如今你终于上了我的当了吧,你可知道是怎么上的当?''白衣少年狠狠瞪住他,眼里虽冒火,嘴里却说不出话。

小鱼儿道:``告诉你,世上根本没有七步阴风掌,我自然也不会,但世上却真有另一神秘的武功,叫做点血截脉''他跑过去将那匹已骇得远远跑开的小白马拉回来,白衣少年眼睛瞪得更大,似是已等不及地想听了。

小鱼儿缓缓道:``这点血虽是一字之差,而且音也近似,但手法却大不相同,点穴是死的,点血却是活的。''他随手点了那少年身上``地门''、``气血囊''两处穴道,口中笑道:``这是点穴,你''期门与气血囊两处穴,永远都在这个部位,绝不会动,所以点穴是死的\ldots{}``说着话,他又在那少年肋下拍了两掌,接道:''点血却是要截断你的血脉,你的血脉不能流通,身于自然不能动自然要倒下去,你的血脉整天都在不但地流动着,点血就是要恰巧点在你血脉流动时前面那一点,才能恰巧将你的血脉截断血在流动,这一点自然也时时刻刻都不同,所以点血是活的,你懂得我的意思了么。``白衣少年已听得入神,不觉应声进:''懂了。``小鱼儿笑道:''但这闭血点穴为时不能太久,否则被点的人就要死了,方才我已解开你闭住的血,所以你现在才能说话。``白衣少年虽然生气,却忍不住道:''方才你瞧着地上的影子,可是在计算时辰,计算我血脉该流在何处?然后再叫我用力按下去!``小鱼儿拍掌大笑道:''对了,举一反叁,孺子可教也\ldots{}``白衣少年咬了咬牙,又道:''你虽然会一点点血的皮毛,但会的却不多,而且根本就点不着我,所以,你就骗我,让我自己动手。``小鱼儿大笑道:''对极对极,一点也不错,因为教我``点血''的那人,医道虽高明已极,武功却不行已极,他虽对人休各部都了如指掌,虽能算得出人体血脉流动的系统,却也不知道该用什么手法去点,所以我也只有请你代劳了\ldots{}``他歇了口气,接道:''因为你还在随时准备动手,所以真气仍在掌指间流动,我一叫你用力,你真气就不觉自指间透出,这自也因为我叫你点的不是穴道,甚至根本不在穴道附近,所以,你就根本未去留意。``白衣少年恨声道:''诡计伤人,又算得什么!``小鱼儿道:''诡计?你可知道要多大的学问能使得出这样的诡计。第一,我要先让人时时刻刻都防备着我,这样体内真气才不会自指学问撤出。第二,我要先编出七步阴风掌这样个怕人的名字,让你不得不含糊\ldots{}``白衣少年不由得叹了口气,进:''这两样已够了。小鱼儿道:``不够,我至少还得略懂点血术的门径,还要算准血脉恰巧正流动在你穴道附近,让你全不提防。''他挺起胸膛,大声道:``这简直是武功与智慧的结晶,我武功若不高,怎能教你提防,我智慧若不高,又怎能教你不提防,你先提防而后不提防,可见你怎样都不如我,你拜我这样的人为师,总算不冤吧。''白衣少年怒喝道:``拜你为师,你\ldots\ldots 你做梦?''小鱼儿道:``你未动手前明明已说好的,如今怎能反悔。''白衣少年涨红了脸,道:``你杀了我吧!''

小鱼儿笑道:``我何必杀你,你若要食言反悔,我就切下你的鼻子,挖去你的眼睛,割下你的舌头,把你\ldots\ldots{}''白衣少年大喝道:``我死都不怕,还怕这些?''小鱼儿眨了眨眼睛,道:``你真的不怕?''。

白衣少年这:``哼!''

小鱼儿眼珠子一转,嘻嘻笑道:``好!你既不怕,我就换个法子。''白衣少年大叫道,``我什么都不怕\ldots{}''

.小鱼儿道:``我把你吊在树上,脱下你的裤子打屁股,你怕不怕?''他知道有些人纵然刀斧加身,也不会皱眉头,但若要脱下他的裤子打屁股,他却是万万受不了的。

白衣少年脸色果然变了,-阵青,一阵红,青的时候青得像生铁,红的时候红得像猪血。

小鱼儿大笑道,``你终于还是怕了吧,快叫师父。''白衣少年身子发抖,嘶声道:``你\ldots\ldots 你这恶魔\ldots\ldots{}''小鱼儿道:``你不叫我师父反叫我恶魔\ldots\ldots 好。''弯下腰,就要去拉那少年的腰带。

白衣少年突然大叫了起来,叫道:``师父!师父\ldots\ldots{}''两声``师父''叫出,眼泪已流了满脸。:小鱼儿立刻就为他擦干了,柔声道:``你哭什么,有我这样个师父也不错呀,何况,你现已叫了我师父,哭也没用了\ldots\ldots 呀,你还哭,再哭我又要打屁股了。''白衣少年拼命咬着嘴唇,不让眼泪流下。

小鱼儿笑道:``这样才乖,对了,你得先告诉我,叫什么名字?''白衣少年道:``铁\ldots\ldots 铁心男!''

小鱼儿眨着眼笑道,``兰花的兰?''

白衣少年大声道:``自然是男儿的男\ldots 小鱼儿大笑道:''铁心的男儿,好,好名字,男儿的心,本该像铁一样硬,不想你模样虽生得有些像女孩子,名字却取得似乎刚强。``铁心男突然抬起目光,道:''你!``小鱼儿道:''我人虽比你刚强,名字却没你刚强,我叫江鱼\ldots\ldots 你知不知道,有人说江里的鱼很好吃,你吃过没有?``铁心男咬了咬嘴唇,道:''我\ldots\ldots 我很想吃\ldots{}``他很想吃的,倒不是远在江里的鱼,而是近在眼前的这条小鱼儿'',他真恨得咬``鱼儿''一口,咬下他一块肉来。

小鱼儿笑嘻嘻地瞧着他,突然伸出手,伸到他嘴边,笑道:``你想吃,就吃吧。''铁心男呆住了,道:``你\ldots\ldots 你\ldots\ldots{}''小鱼儿大笑道:你不是想吃我的肉么?\ldots\ldots 告诉你,无论你心里在想什么,都瞒不过我的,我一猜就猜出。"铁心男叹了口气──除了叹气,他还能怎样?

小鱼儿道:``你今年几岁了?''铁心男道,``总比你大两岁\ldots{}''小鱼儿笑道:``就算你比我大两岁,但学无长幼,能者为师,这\ldots\ldots{}''突然间,远处有人嘶声大呼道:``小鱼儿!江鱼!休莫要走!不能走!''一匹马飞驰而来,马上人的衣服仍闪着光,小辨子也仍在飞扬,但马到近前,她却几乎是滚下来的。

她的脸也不再像桃花,简直苍白得像是死人,她的眼睛仍是发亮的,但却充满了惊慌与恐惧!

她一把拖住小鱼儿,喘着气道:``阿拉,真主,感谢你\ldots\ldots 他还在这里。''小鱼儿道:``阿桃?是什么事将你又''拉``来了?''桃花道:``求求你,莫要再笑我,你打我骂我都可以,但你\ldots\ldots 你\ldots\ldots 一定要跟我走!''说到第二句话时口他眼泪已流了满脸。

小鱼儿叹道:``唉,又多了泪人儿,真要命。''他用衣袖擦了擦挑花脸上的眼泪,道:``你要是再哭,哭肿了眼睛,就不该叫桃花,要叫桃子。''桃花``噗嗤''一笑,小鱼儿拍手道:``又哭又笑,猫儿撒尿\ldots\ldots{}''一句话未说完,桃花却又哭了起来,拉过小鱼儿的衣袖,``嗤的擦了一把鼻涕,边哭边道,''方才我被你气走,越想越气,骑者马兜了个圈子,刚想回去,但远远就瞧见家里出了事了。``小鱼儿笑道:''什么事,新衣服被人弄上鼻涕了么?``桃花根本没听见他说什么,''嗤``的又擤了把鼻涕,道:''我远远就听见帐篷圈子里传来男人的惊呼,女人的哭声,就连马也在乱叫乱跳,乱成一团,其中还夹着皮鞭子吧哒吧哒在抽人的声音,还有个破锣嗓子在大吼:谁也不准动,排成一排,小心老子宰了你!\ldots\ldots{}``小鱼儿道:''你嗓子再哭哑些,就学得更像了。``桃花道:''我本想冲过去,但想了想,又下了马,伏下身子,在草丛里爬了过去,幸好草很长一我爬到近前,便瞧见那一团帐筐四周,不知何时已被一堆人围上了,这些人一个个拿着大刀,又拿着鞭子,凶眉横眼,骑在马上,不像强盗才怪。``小鱼儿道:''哎呀,强盗来了,有意思。"

桃花道:``这些强盗将我的族人和那些做生意的汉客全都赶牛赶羊般赶成一团,我瞧见他们的鞭子抽在我的族人身上,我的心都碎了。''。

小鱼儿道:``草原上的强盗原来这么凶。''。

桃花道:``草原上虽有强盗,但却不是这些人。''小鱼儿笑道:``你怎知不是?草原上的强盗你认得?''桃花道:"草原上的强盗虽是汉人,但为了方便,也都是穿着牧人的衣服,但这些强盗的打扮,我一看就知道是从关内来的。

他们骑的也不是咱们的藏马,而是川马,藏马的腿长,川马的腿短,我一瞧就能分出来。``小鱼儿不再笑了,皱眉道:''这些人不远千里自关内赶来,自然不是为着要抢你们的贷物牛羊,关内的有钱人,总比关外多\ldots\ldots{}``桃花道,''他们不是要抢东西,而是要抢人\ldots{}``小鱼儿道:''抢人?抢谁?抢你?桃花咬着嘴唇,道:``汉家的女孩子,也总比我们漂亮得多\ldots\ldots 他们要抢的,也是个汉客,他们一路自关内将他追到这里,而且他们的探子还瞧见这人在我们的帐蓬里,所以,他们就逼着我的族人要人!''。

小鱼儿道:``你的族人可给了他们?''

桃花道:``我的族人根本不知道他们要的是谁,他们自己在帐篷里找,也没有找着,于是他们就一定说是我的族人藏起了他,还要限半个时辰内将他交出来,否则\ldots\ldots 否则他们就要凌辱我们的姐妹,打死我们的兄弟。''她说到此刻,又忍不住放声大哭起来。

她扑到小鱼儿身上,大哭道:``所以我来求你回去救救他们,我知道你很有本事\ldots\ldots{}''小鱼儿沉吟道:``你可知他们要的那人是谁?!''桃花道:``我\ldots\ldots 我本来还以为他们要的人是你,后来才听见,他们要的,是一个姓铁的小子,你\ldots\ldots 你可知道他是谁?''小鱼儿眼珠子一转,笑道:``姓铁的\ldots\ldots 我没听见过,我铁心男一直瞪着眼睛在听他们的话,此刻忽然大叫道:''我就姓铁,我就是他们要找的人!"桃花一惊,两只大眼睛瞪着铁心男,再也不转了。

小鱼儿摸了摸头,苦笑道:``呆子,你为何要承认?''铁心男也不理他,大声道,``那些强盗中可有女子?''桃花呐呐道:``没\ldots\ldots 没有。''她实在想不到那些强盗要找的竟是个这么漂亮、这么秀气的小伙子,竟呆在那里,眼泪也不流了。

铁心男已大声道:``好,他们既要找我,我跟你去!''桃花道:``你去了不行!不行!''。

铁心男道:``只有我去,才能救你的族人,为何不行?''桃花垂下头,幽幽道:``像你这样的人,去了岂非等于羊人虎口,我怎忍看你前去送死?你\ldots\ldots 你\ldots\ldots 你还是快逃吧\ldots 铁心男冷笑道:''你以为我怕他们?\ldots 哼!像他们这种蠢材,一一百个加在一起,也抵不过我一根小指头。"\ldots。

桃花道:``你不怕他们,为何要从关内逃到这里来?''铁心男呆了呆,道:``我\ldots\ldots 我\ldots\ldots{}''

桃花忽然抬起头,道:``莫非你怕的只是个女人,是以一听他们全是男的,你就不怕了。''铁心男脸红了,大声道:``这些事不用你管。''小鱼儿却拍拿笑道:``原来你不怕男人,只怕女人,哈哈,这毛病倒和我差不多,我委实也是一见了女人就头疼。''铁心男叫道:``放过我\ldots\ldots 我去!''\ldots 小鱼儿道:"你若去死了,我岂非连徒弟也没了。

铁中男道:``我担保一定回来。''小鱼儿歪看头想了想,笑道:``桃花,你看我的这徒弟是不是英雄?''桃花痴痴地瞧着铁心男,合掌道:``阿拉保佑你。''小鱼儿大笑道:``英雄救美人,这可是佳话一段,我江鱼可不能煞风景\ldots\ldots 好,你去吧\ldots{}''手掌拍了两下,铁心男一跃而起。

桃花道:``你\ldots\ldots{}''小鱼儿笑道:``你有了一个英雄还不够么?我\ldots\ldots 我在这里等你们''桃花跺了跺脚,道:``不愿救人的人,将来也没有人救你\ldots\ldots{}''她再也不瞧小鱼儿一眼,一跃上马,道:``铁\ldots\ldots 你也上马来呀。''铁心男却瞧了瞧小鱼儿,道:``我\ldots\ldots 你\ldots\ldots{}''终于什么话也没说,飞身上马,飞驰而去。"

小鱼儿瞧着那渐去渐远的蹄尘,喃喃笑道:``多情的姑娘,情总是不专的,这话可一点儿也不错,铁心男这下子被他缠住了,却不知要几时才能脱身。''他轻轻拍着那个白马的头,道:"马儿马儿,咱们也去瞧瞧热闹好么,但你瞧见漂亮的小母马时,可要走远点,咱们年纪还小。

若被女人缠着,可就一辈子不能翻身了。"

桃花打马飞驰,长长的秀发被风吹起,吹到铁心男的脸上。

铁心男却似毫无感觉,动也不动。

桃花又觉他呼吸的热气吹在脖子里,全身都像是发软了,她小手拼命抓紧绳,回眸道:``你坐得稳么?''铁心男道,``嗯。''

桃花道:``你若是坐不稳,最好抱住我免得跌下马去。''铁心男道:"嗯\ldots 居然毫不推辞,真的抱住了她。

桃花都软了,突然道,``只要你救了我的族人,我\ldots\ldots 我什么事都答应你。''铁心男道:``嗯。''

桃花眸子立刻又发出了光,马打得更急,这段路本不短,但桃花却觉得仿佛一下子就到了。

他们已可瞧见那黄色的帐篷,已可听见声声惊呼。

桃花道:``我们是不是就这样冲进去?''

话未说完,突见一条白色的人影,自身后直飞了出去,本来坐在马背上的铁心男,已站在十丈外。

桃花又惊又喜,赶紧勒住了马。

只见铁心男笔直地站在那里,雪白的衣衫虽然染了灰尘,但在阳光下,看来仍是那么干净,那么潇洒。

这正是每个女孩子梦寐中盼望的情人。

桃花心里飘飘荡荡,几乎将什么事都忘了。

但惊呼叱骂声仍不断传来,铁心男已在厉声喝道:``铁心男在这里!谁要来找我?''惊呼叱骂声突然一齐消寂。

风吹长草,铁心男衣袂飘飘。

帐蓬里突然有人嘎声狂笑道:``好,姓铁的,算你还有种,总算没叫我李家兄弟白等。''铁心男冷笑道:"我早已猜中是你们\ldots\ldots 你们要找的是我。

``还耽在那里作什么,随我来吧!''他转过身子,缓步而行。

帐篷那边呼啸之声大起,十余匹健马,一起奔了过来凄厉的呼啸夹杂着震耳的蹄声,委实叫人胆战心惊。但铁心男仍是慢慢地走着,连眼睛都没有眨一眨。

桃花远远地瞧着,心里又忧又喜,喜的是铁家的儿郎果然是出色的芙雄,忧的是他文质彬彬的模样,只怕不是这些野强盗的对手。十余铁骑瞬即将铁心男包围往了,铁心男连眼皮都不抬,马上的汉子手里虽拿着长鞭大刀,竟偏偏不敢出手。直走出数十丈外,铁心男才停住脚,冷笑道:``好了,你们干什么找我,说吧''迎面一匹马上坐着的虬髡独眼大双厉声道:``我兄弟先得问问你,那东西可是在你身上。''铁心男笑道:``不错,是在我身上,但就凭你们兄弟这几块料,可还不配动它,你们若认为我到关外是躲你们你们就错了。''那眼限大汉怒吼道:``放屈!''突然一提绳,迎头飞弛而来。

长鞭迎风一抖,``吧''的带着尖锐的破风声,毒蛇般抽了下来``铁心男叱道:''下来!"

手一扬,不知怎地,已提着了鞭梢,乘势一抖,独眼大汉百来斤重的身子,竞被他凌空抖起,摔在两丈外。铁心男身子一抡,马群惊嘶着退了开去,突然刀光闪动,两匹马自后面偷袭而来,.鬼头刀直砍铁心男的脖子。铁心男头也不回,身予轻铰一缩,两把鬼头刀呼啸着从他面前砍了过去,他长鞭扬起,鞭梢轻轻在这两人肋下一点,这两条大汉就滚下马来,一人被马蹄踢中,惨呼着滚出几丈,自己手中的刀将自己左脸整个削去了半边;另一人右脚还套在马蹬里,急切中挣它不脱,竟被惊马直拖了出去。

他举手投足,眨眼间便打发了叁个人,真是轻而易举,不费吹灰之力,别的人可全都吓得呆住了。

铁心男微声笑道:``李家兄弟的马上刀鞭动夫,原来也不过如此,别人想动我怀里的东西,还有话说,不知你们竟也不量量自己的斤两,也想插一脚''

笑声未了,突听身后一人冷冷道:李家兄弟不配动你怀里的东西,毛家兄弟配不配?

这语声有气无力,像是远远自风中飘来,简直教人听不清,但越是听不清,就越是留意去听,一听之下,就好像有无数个瞧不见的小毛虫钻进自己耳朵里,简直恨不得将自己耳朵割下来。

铁心男脸色立刻变了,失声道:``峨嵋山上叁根毛\ldots 一。一身后另一个人怪笑着接道:''人鬼见了都难逃\ldots\ldots 嘻嘻,这句话原来你也听过,这声音却是又尖又细,宛如踩着鸡脖子,刺得人耳朵发麻。"铁心男一寸一寸地转过身子,这才瞧见身后一匹大马,特制的大马鞍上,一排坐着叁个人!

第一个骤看似是五大岁的小孩子,仔细一看,这``孩子''竟已生出了胡须,胡须又白又细,却又仿佛猴毛。他不但嘴角生着毛,就连眼睛上、额角头、手背、脖子\ldots\ldots 凡是压在衣服外面的地方,都生着层毛。他面上五官倒也不缺什么,但生的地方和完全不对,左眼高,右眼低,嘴巴歪到脖子里,鼻子像是朝上的。这简直不像个人,纵然是人,也仿佛老天爷造时,造坏了模子,一生气就索性想把他揉成稀泥,却又不小心被他溜进了他妈的肚子,铁心男瞧着他,虽在光天化日之下,全身也不禁起了寒栗。

他也在瞧着铁心男喀喀笑道:``嚼心蛀肺毛毛虫这名字你总听说过吧,那就是我,你最好莫要多瞧,多瞧两眼,就会肚子疼的!''铁心男要想不去听他说话,却又偏偏忍不住去听,听完了又觉得直要恶心,赶紧去瞧第二个人。这第二人模样也未必比那``毛毛虫''好看多少,但身子和比``毛毛虫''整整大了一倍,脖子和比``毛毛上''长了"叁倍,那又细又长的脖子上,一个头却是又尖又小,简直和脖子一般粗细,满头乱发刺猥般竖起,一张嘴却像是椎子,上面足足可挂五六只油瓶。

铁心男拼命咬着牙,道:``休你是毛公鸡?''

这人咧嘴一笑,露出排锯子般的牙齿,道:``你莫要咬着牙,无论谁见着我,牙齿也要发痒的\ldots{}''铁心男恨不得赶紧掩住耳朵──这人哪里是在说话,这简直像是在杀鸡,杀鸡的声音都比他柔和得多。

他实在不想再瞧那第叁个人了,却又忍不住去瞧,他想,这第叁个人总要好看些的──世上还有比他们更难看的人么?他不瞧倒罢了,这一瞧之下──唉,老天,前面那两个多少还有些人形,这第叁个简直连人形都没有了。

这第叁个人简直是个猩猩。``毛公鸡''的身子要比``毛毛虫,大上一倍这''猩猩``的身子却要比''毛毛虫``整整大上四倍。''毛公鸡``脖子又细又长,这''猩猩``却根本没有脖子,一颗方方正正的头,简直就是直接从肩膀上长出来的,''毛毛虫``身上的毛又白又细这''猩猩"身上的毛又黑又粗,连鼻子嘴巴都分不出了,只能瞧出一双野兽般的的发光的眼睛。

这双眼睛正瞧着铁心男,道:``毛猩猩!''

远处草丛中的小鱼儿,也瞧见这叁个人了他实在忍不住要笑。他实在想不通他们妈妈是怎么将这叁人生出来的,能生出这样叁兄弟来的女人,那模样他更不敢想象。但他却不知这兄弟叁人正是近十年来最狠毒的角色,江湖中人瞧见他们,莫说笑,简直连哭都哭不出。

\hypertarget{ux7b2cux5341ux4e09ux7ae0-ux4ed9ux5973ux60e9ux51f6}{%
\chapter{第十三章
仙女惩凶}\label{ux7b2cux5341ux4e09ux7ae0-ux4ed9ux5973ux60e9ux51f6}}

小鱼儿在暗中已瞧了许久,他瞧见李家兄弟在前面追铁心男,这毛家兄弟就在后面跟着李家兄弟。他们坐的那匹马又高又大,但走的步子却是又轻又快,一路在后面跟着李家兄弟,李家兄弟竟没人知道。

现在,李家兄弟自然知道了,这些看来威风凛凛的大汉,一瞧见这叁个怪物,身子竟像是弹琵琶般抖了起来。

小鱼儿不禁暗中奇怪:"这叁个怪物找的又不是他们,他们怕什么?难道这些怪物竟是六亲不认,见人就杀的么?

只见李家兄弟一面发抖,一面就想溜,这兄弟十余人的马上功夫果然都不错,身子未动,马已在后退。

毛毛虫突然笑道:``奇怪呀奇怪,姓铁的还未溜,姓李的却想溜了。''诸李中一人赶紧抱拳笑道:``我兄弟不敢与前辈争功,这姓铁的身上东西,我兄弟也不想分了,是以\ldots\ldots 我兄弟先走一步''毛公鸡咯咯笑道,``你们一瞧见我们兄弟就走,难道是嫌咱们难看么?''那大汉脸色已黄了,牙齿打战道:``不!不\ldots\ldots 不敢。''毛公鸡道:``既然不敢,为何还要走。''毛毛虫笑道:``老二这就错了,腿又不是生在他们身上的,他们的腿可没有动呀,动的只不过是马腿而已。''毛公鸡道:``如此说来,不是他们不听话,是马不听话。''那大汉赶紧道:``不\ldots\ldots 不错,是\ldots\ldots 是马\ldots\ldots{}''韦公鸡道:"这些与其该死\ldots 死一字刚说出,那毛猩猩已跃了下来。

他身子虽是方的,两条手臂却是又粗又长,几乎要拖到地上,他身子看来虽笨,行动倒一点也不笨。

又见他身子一晃,已到了第一匹马前,一拳往马头上击去,那匹马连哼都未哼,就倒在地上,马头竟被他一拳打得稀烂。

小鱼儿也不禁骇了一跳:``这家伙好大力气。''一念转过,已又有叁匹马的头被他打烂了。

群马惊嘶,毛猩猩大步赶过去,就像是砍瓜切菜,十几匹马眨眼间就再也瞧不见一个好好的马脑袋。李家兄弟一个个跃下马来,一个个面无人色,其中一人突然狂呼着往后就逃,简直已被吓疯了。

韦公鸡道:"还有不听话的。!

语声中突然飞起,头前脚后,一支箭似的射了出去,``砰''的一声,公鸡般的脑袋已撞上了那大汉的后背。那大汉逃的不慢。

只听身后风响,连回头都来不及回头,已被撞着,一根脊椎骨断成十几截。他身子竟不是倒下去的,简直就像是面人儿似的瘫下去,毛公鸡的手却已捉着他的身子,喝道:"老大,分菜给你!

那大汉竟被抛了出来,飞过众人头顶。

毛毛虫笑道:``刚出笼的馒头来了。''

眼见那大汉身子飞来,突然伸出猴爪般的小手,往那大汉胸口一抓,他人不过是较轻掏了掏。那大汉身子还是照样往前飞。

便却有鲜血涌了出来,又飞了叁丈,才跌在地上,地上多了一串鲜血,他胸口也多了一个大洞。

再瞧毛毛虫手上已是血淋淋的,掌心一颗鲜红的人心,似是还在微微跳动,毛毛虫笑道:``各位谁要吃这慢头,好香好热的馒头,还烫手哩。''李家兄弟脸如死灰,铁心男脸色也变了。

毛毛虫大笑道:``你们既然无福消受,可就便宜我了''竟张口咬了下去,一口就咬了一半,嚼得吱吱作响,顺着嘴角直淌鲜血。

李家兄弟身子发软,简直已站不住了,铁心男不由自主掩住了嘴,否则就得当场吐了出来。就连小鱼儿,也不禁直犯恶心。李大嘴虽然也是吃人的,但吃得到底``文明''得多,还讲究细切慢烹,煎炒蒸煮,吃相也文质彬彬的,并不吓人。像毛毛虫这样的吃法,小鱼儿简直没瞧过,简直也瞧不起,他觉得这人,简直太野蛮,简直太不懂得享受。就算要吃人,最少也该学学李大嘴那样的吃法才是\ldots 但毛猩猩的气力实在不小,毛公鸡的身法实在不错,这毛毛虫手上的功夫,也实在令人吃惊。

这点小鱼儿还是承认的,尤其是毛毛虫,伸伸手一掏就能将人心掏出来,这出手之快且不去说它,部位认得之准,竟不会掏错地方,这份眼明手快,当真连小鱼儿也不得不佩服他索性沉住了气,瞧个明白。

只见毛毛虫片刻间已将一颗心吃得干干净净,甚至连嘴角的血都舐干净了,拍了拍手,笑道:"秋风将近,进补及时,人心最补,大家不可不知,你们瞧,我刚吃完了,精神可不就来了」他的精神果然来了,不但说话的声音已响亮得多,就连眼睛也亮得多,脸上也冒出了红光。

铁心男突然冷冷笑道:``你们这是向我示威?''毛毛虫笑道:``你胸口里也藏着这个馒头,你若不想被我吃掉,就赶紧把那东西拿过来吧,免得我多花力气动手,费了力气就又想吃馒头。''铁心男道:``你想也休想!''

身子突然倒翻而出,叁十六着,是走为上策。

哪知那毛猩猩突然已挡住了他的去路,两条手臂一伸,加起手足有两丈,铁心思竟窜不过。

毛猩猩咧嘴一笑,道:``好漂亮的小脑袋,打坏了真可惜''他一共只说了十叁个字,铁心男却已攻出十四招!铁心男固然是快,他说的也委实慢得不像人话。

这十四招击出去,从第一拳开始便未落空,只听``砰、砰、砰\ldots\ldots{}''之声不绝于耳,毛猩猩肩头胸口肚子已挨了十四拳之多,着着实实的十四拳,可没有半分虚假。

但毛猩猩却当他是假的,非但身子动也不动,嘴里还是照样说话,铁心男这十四拳竟像打鼓为他话声助威一样。十四拳击过,铁心男嘴唇已发白,那第十五拳,委实再也打不出手,竟似已呆在地上。

毛猩猩透了口气,道:``完了么?''

铁心男咬咬牙,道:``完了''

毛猩猩道:``好,轮到我了!''

``呼''的一拳,直击而出。他的拳头铁心男可受不了,身子一伏,突然自他肋下穿出。

乘势在他脚下轻轻一勾,反手又添了一掌。

毛猩捏身子已推重山倒玉柱地扑面跌在地上。

铁心男却不敢回头瞧他狼狈的模样,身形不停地前窜,突见地上钻出个毛毛的东西,竟是毛公鸡的脑袋。

他再回头去瞧,毛猩猩已从地上弹了起来,正咧着大嘴望着他笑,左面却伸过来一只长满白毛的小爪子,道:``拿来!''这兄弟叁人竟有两下子,小鱼儿瞧见他们的身法,就知道铁心男逃是绝对逃不了的,打,也打不过。

他叹了口气,暗暗道:``看来只好我出手了,师父虽然未必帮着徒弟打架,但徒弟身上若有好东西时,做师父的可不能让他被别人抢走。''只见铁心男已被围在中央,他磨了磨拳头,就要出手,但就在这时,突听一阵铃声远远传了过来。接着,他便瞧见了一个火红的影子,像是火。这团火竟是一人一马,火红的马,火红的衣服,人马本来极远,但来得好快,简直像是在飞!

铃声传来,李家兄弟、毛家兄弟、铁心男已全都一惊,再瞧见这火红的人马,十几人竟似一起吓呆了。

只听一个又娇又脆的声音喝道:``一共十九个,谁也不准走!''人马已火云般飞到眼前,马上人红衣如火,手里挥动着根火红的鞭子,鞭子雨点般落下,眨限间李家兄弟已被抽得倒在地上打滚,那鞭子就像毒蛇,就像火,但李家兄弟眼见这鞭子抽下来。

非但不取逃,不敢招架,竟连惨呼都不敢呼出声来,只是咬着牙直哼哼。火红的人马兜着圈子,李家兄弟在地上直滚。

小鱼儿不禁暗中鼓掌道:``好鞭法,打得好,不想铁心男竟有这样的朋友,看来用不着我出手了。''他未瞧见这其中脸色变得最惨的,就是铁心男,他目光委实己被这马上的人吸住了,且也没空去瞧别人。

毛家兄弟实在太丑,这人却实在太美,毛家兄弟丑得不像人,这人美得也不像人,简直像是仙子。

她的衣服红如火,她的面靥上也带着胭脂的红润,她的鞭子若是地狱中的毒蛇,她的睛睛就是天上的明星。她的鞭子飞舞。

她的眼波流动。

小鱼儿暗叹道:``只要能被她瞧两眼,挨几鞭子也没关系,但她这鞭子却未免太毒了,别人说过越美的人越狠心,这话果然不错。''他瞧见李家兄弟身子本来还在打滚,嘴里本来还在哼哼,到后来却连滚也滚不动了,哼也哼不出。但这红衣少文手里的鞭子还是不停,她瞪着眼睛,咬者牙,嫣红的面庞上,没有半分笑容,竟冷得怕人。

铁心男突然大喝道:``他们和你有什么仇恨,你要下如此毒手?''那红衣少女冷笑道:``天下的恶人,都和我仇深如海。''铁心男嘶声道:``你\ldots\ldots 你住手!''

红衣少女道:``你要我住手,我偏要打!偏要打!'',又抽了十儿鞭子,她却霍然住手,兜转马头,面对着毛家兄弟,她的眼睛发着光,冷笑道:``很好,你们没有走,很聪明,但我也没有忘记你们''毛毛虫咯咯笑道:``姑娘叫咱们留下,咱们自然这命\ldots{}''红衣少女道:``你可知这我为什么未用鞭子对付你们?''毛毛虫笑道,``不知道。''

红衣少女道:``挨鞭子的人能活,不挨鞭子的就得死!''毛毛虫道:``姑娘可知道咱们为什么不走?''

红衣少女道:``你敢走么?''

毛毛虫怪笑道:``咱们不走因别人怕你,我兄弟却不怕你!''叁人像是早已商量好,此刻突然同时飞起。毛公鸡一头撞向那少女的腰,毛猩猩一拳击向马头,毛毛虫一双猴爪,闪电般直抓她的眼睛。这兄弟叁人不但出手迅急,配合佳妙,而且所攻的部位,更是上、中、下叁路全都照顾得周周到到。小鱼儿实在想不出她怎能挡得住这叁招,她就算能保住头,也保不住腰,就算能保住腰,也保不住马。

只听这少女冷冷叱道:``找死!''

接着,又是轻轻一声呼啸,那匹胭脂马竟突然人立而起,一双腿,直住毛猩猩头上咂了下去。

毛猩猩纵能受得了人的拳头,却也受不了这马腿,拼命一躲,肩头还是被踢中,踢得满地打滚!小鱼儿瞧得几乎要拍起手来,他虽已猜出这少女武功必定厉害,却未料到连她坐下的马也有两下子。再瞧毛毛虫与毛公鸡,两人躺了了下来,毛毛虫一双手已齐腕折断,毛公鸡的脑袋却分成了两半。小鱼儿眼睛虽然快,但毕竟只有一双眼睛,瞧得这边便顾不了那边,他竟末瞧出这少女是如何出手的!

他简直瞧得连眼睛都发直了,脖子里直冒凉气,这少女连马鞍都未下,已打发了这叁个怪物,这是什么样的本事!

草原昼短,日已西沉。

夕阳,照着这少女嫣红的脸,照着她嫣红的面颊,也照着这些``死尸''──一个骑着红马的美丽小姑娘,慢慢走在满地死尸间,风吹草长,夕阳将暮,这\ldots\ldots 这又又是幅什么样的图画了铁心男站在那里,像是丝毫也没有想逃的念头,只是瞪大了眼睛瞧着她,脸色和躺在地上的人也差不了多少。

穿红衣的小姑娘终于将马兜到他面前,小鱼儿虽瞧不见她的脸,却猜想她此时一定笑了,她不笑已是那么美,笑的时侯模样更不知有多可爱了,又可惜自己瞧不见,他又想,这小姑娘又怕也对铁心男很有意思,所以才会将和铁心男作对的人都打在地上。

哪知这小姑娘却冷笑道:``好,铁心男,算你有本事,竟能一直逃到这里,能从我手里逃得这么远的人,除了你,还没有第二个。但现在你可再也逃不了啦。''铁心男道,``所以我根本没有逃。''

红衣姑娘道:``你很聪明,你果然比这些人都聪明得多,但你若是真聪明,就快些将那东西交出来,免得我费事。''小鱼儿越听越不对了,他这才知道这个姑娘虽然出手救了铁心男,却是黄鼠狼给鸡拜年,没存好心。

.他眼珠子一转,自怀中摸出件东西,悄悄爬了出去,风吹草长,不住作响,恰巧掩饰了他的声音。

只听红衣姑娘道:``你拿不拿来?''

铁心男道:``什么东西,我根本不知道\ldots{}''

红衣姑娘大怒道:``我从来没有对别人这样好好说过话,你\ldots\ldots 你\ldots\ldots 你还要装蒜?''鞭子突然飞起,一鞭子抽了过去。

``啪''的,鞭子抽在铁心男身上,用的力却不重,铁心男动也不动地挨着,神色不变,淡淡道:``你打死我,我也不知道是什么东西。''红衣姑娘喝道:``好,你这是逼我动手,你可和我一动手就不会停手,你难道不知道我的脾气?你难道\ldots\ldots{}''她的气越来越大,全未觉察小鱼儿已爬到她的马后,将手里的东西迎风一晃,便有一股火焰飘了出来,立刻燃着了马屈股和马尾巴,这胭脂马虽然神骏,但完全是畜牲,世上哪有不怕火烧的畜牲,当下惊嘶一声,直窜了出去。

红衣姑娘一句话没说完,马已将她带到十丈外,她要是跃下马来,小鱼儿和铁心男还是逃不了。怎奈她对这匹马爱逾生命。

怎舍得丢下,这自然是小鱼儿早已算准了的,否则他又怎会使出这一着!

那火烧得好厉害,烧得马疯了似的向前跑。

红衣姑娘惊呼道:``樱桃,莫要怕,樱桃\ldots\ldots 站住!''她跳下马虽容易,但要勒住这匹受惊的马,可不简单,何况她简直根本舍不得使力勒马。这``樱桃''腿力也实在真快,眨眼间便跑得不见了。

小鱼儿自然也早已拉着铁心男的手,向另一方向飞逃而去。

那小白马远远瞧见,居然认得他,也跟着他跑。也不知跑了多远,小鱼儿不敢停住脚,铁心男更不敢停住脚,两人脸已发青,汗珠已和黄豆差不多大。

天色已暗了,这一趟直跑了不少里路,莫说小鱼儿,就连铁心男一生也没有一口气跑得这么远过。跑着跑着,只见前面有个破破烂烂的小木屋,小鱼儿也不管里面有人没人,一头就冲了进去。

一冲进去,两入可忍不住全躺下了,喘气的声音,简直比牛还粗,小鱼儿就在铁心男怀里,铁心男心跳的声音像是在打鼓。

幸好这屋里果然没人,只见蜘蛛网不少,显然已有许久无人居住,两人冲进来时,自然沾得满头满脸。小鱼儿刚想去弄掉它,哪知铁心男一喘过气来,突然用力一推,几乎将他推得远远滚了出去。

小鱼儿瞪起眼睛道:``我救了你命,你就这样谢我?''铁心男脸红了红,道,``对\ldots\ldots 对不起,谢谢你。''小鱼儿笑道:``对不起,行个礼,放个屁,臭死你\ldots\ldots{}''铁心男竟真的放了屁,小鱼儿早已笑得满地打滚。铁心男脸更红得像茄子似的,恨不得一头钻进地里。

小鱼儿爬了起来,笑道:"放屁有什么要紧,人在害怕时,不撒尿就算好了,放屁又算得什么,你怎么像个大姑娘似的,动不动就红脸?

铁心男道:``我\ldots\ldots 我\ldots\ldots{}''、他说话的声音简直像是蚊子叫,连他自己都听不清。

小鱼儿道:"莫说你害怕,就连我\ldots\ldots 连我这天不怕地不怕的人都怕了她,还有谁不怕她\ldots\ldots 喂!你可知道她叫什么名字?

铁心男道:``她姓张,别人都叫她小仙女张菁,小鱼儿拍掌道:''呀,这名字我听过\ldots\ldots{}``他突然想起自己出谷那天下午,逃入''恶人谷``的那''病虎"常风,就在他面前提起过这名字。

那常风的确也是怕得她要死,但小鱼儿那时候未想到这人人闻名丧胆的角色,竟是个无锡泥娃娃般的小姑娘。,小鱼儿想到她,骑着小红马,穿着红衣裳,闯荡江湖,走过的地方,人人都向她磕头\ldots\ldots 小鱼儿不觉想得出神了。``过了半晌,铁心男轻轻道:''你能将我从她手里救出来,可真不容易,但\ldots 但她必定恨你入骨,你以后可要小心。``小鱼儿笑道:''我不怕,她根本没瞧见我,不认得我,何况\ldots\ldots 就算真的打起来,我也未必一定会输给她\ldots 铁心男笑道,"你打不过她的,他的武功也不知是谁传授的,出道才不过一年多,最少已有五六十个武林高手栽在她手里。

小鱼儿道:``那些一装一篓的高手算人才?''。

铁心男道:``但其中却也有不少功夫是真硬的,譬如\ldots\ldots{}''小鱼儿大声道:``这些且不去管它,你且将那东西拿来给我瞧瞧\ldots{}''铁心男身子微微一震,道,``什\ldots\ldots 什么东西?''小鱼儿道:``就是他们不要命地来抢的东西,也就是你宁可不要命他不肯给他们的东西,你自然知道是什么的。''铁心男道:``我,我不知道\ldots{}''

小鱼儿一把拉住了他的衣襟,大声道:``我救了你的性命,要你拿那东西给我瞧瞧,你都不肯,你这人还有良心么,何况我只不过想瞧瞧,又不要你的。''铁心男道:``你\ldots\ldots 你放手,我告诉你。''

铁心男叹了口气,道:``但这是件秘密,你可不能告诉别人''小鱼儿道:"我会去告诉谁?呆子,你才是我最喜欢的人呀。

别人害你,我不要命地救你,我怎会会告诉别人!``铁心男脸又一红,但立刻抬起头来,轻声道,''那东西不在我这里。"小鱼儿瞪着眼睛瞧了他半天,突然大笑起来。

铁心男道:``你笑什么?''。

小鱼儿道:``那东西既不在你身上,他们为什么要追你,你为什么要逃?''铁心男叹道:``因那东西是我的一个最亲近的人拿去的我怕别人去害他,所以就故意装成东西在我身上的模样,好教别人都来追我,他就可以平安了。'',:..\ldots\ldots 小鱼儿呆了呆,道:``原来这是蝉脱壳、调虎离山之计。想不到你竟是个肯舍己为人的好人。''、铁心男垂首道:``我虽不是好人,但那人是我哥哥。''小鱼儿道:``哦,原来如此,但那究竟是什么东西,你总可以告诉我吧\ldots{}''铁心男头垂得更低,道:``那是张藏宝的秘图\ldots{}''小鱼儿笑道:``原来是这种东西,早知道是这种东西,我连瞧都不要瞧了,我若要宝贝,简直到处都有,何必那么费事\ldots{}''他站起来,转了一圈,小鱼儿走到门口,笑道:``这外面还有井。''铁心男道:``这破柜子里还有几只破碗,我去打些水来给你喝。''小鱼儿眨了眨眼睛,道:``你不会逃吧!''

铁心男道:``我为什么要逃。''

小鱼儿大笑道:``我知道你不会逃的。''

佚心男果然没有逃,却提着个木桶走了进来。他脸上的傲气已全不见了,突然变得十分温柔,竟真的打水、洗碗,做了些男人不愿做的事,而且做得很仔细。

小鱼儿瞧着他,觉得有趣得很,突然一阵马蹄传来,两人俱都一惊,面无人色,幸好小鱼儿眼尖,已瞧见是匹白马。

那个白马居然也一路追着他们来了。

小鱼儿又惊又喜,□着迎了出去,抚着小白马的头道:``马儿马儿你真乖,明天请你吃白莱,对了,我也该给你取个名字,别人红马叫樱桃,你就叫白菜吧。''``他向屋子里瞟了一眼,屋子里很黑,过了半晌,铁心男端着两碗水出来,满面笑容,道,''我已尝了尝,这水是甜的。``小鱼儿道:''我们喝水,马儿呢,它跑累了让它先喝吧。``铁心男赶紧道''不行不行,这\ldots\ldots 我只洗了两个干净碗,叫它拿桶喝吧。"将一只碗放到井边,一只碗交给小鱼儿,飞也似的跑了回去。

他跑得可真快,等他跑回来的时候,小鱼儿还站在那里没动哩,铁心男眨了眨眼睛,笑道``你喝呀,水真是甜的!''小鱼儿笑道;``我怕这井水有毒。''

铁心男咯咯笑道``不一\ldots 不会的,水里有毒,我已经被毒死了,我刚才已经喝了一碗,现在我再喝一碗。''他拿起进边的碗,一口气喝了下去。

小鱼儿笑着道;``你先喝了,我就放心了。''

他喝了一碗,又是一碗,简直比马喝得还多。

天色更暗了,星,已在草原上升起。

小鱼儿面色突然大笑,道:``不.\ldots.不好!是的,我的头怎么发晕了。话未说完,真的倒了下去,大呼道:''毒,井水里一定有毒``铁心男突然后退两步,冷冷笑道:''你放心,水里没有毒的,只不过是迷药,你在这里好好睡上一觉,明天早上,就可以走路小鱼儿呻吟着道:``你。\ldots 你为什么要下迷药。''铁心男道:``只因我要去个地方,不能被你缠着。''小鱼儿道:``你\ldots{}''你。``。''

他越来越不行了,连话已说不清。

铁心男笑道:``你这孩子,虽然还算聪明,但..他边说边走,说到这里,脚下突然一滑,几乎跌倒,他面色也立刻变了,再走两步,竟真的扑地跌倒,倒在水桶旁,竟似连爬都没有力气爬起来颤声道:''这\ldots。这是怎么回事?``小鱼儿道''莫非你在自己碗里也下了迷药?"

铁心男道``不\ldots。不会的,我\ldots\ldots 我明明\ldots{}''

小鱼儿突然大笑起来,大笑过后一跃而起。

铁心男大骇道``你\ldots。你莫非\ldots。''

小鱼儿拍掌大笑道``你这孩子,虽然还算聪明,但和我比起来可就差多了你在屋子里下迷药,以为我瞧不见?嘿嘿,告诉你,我这双眼睛是药水泡大的,就算半夜里,也可以在地上找出根绣花针的。铁心男面色如土,道:''原来你,你换了碗。``小鱼儿笑道''不错,我换了碗,你却瞧不见,老实告诉你,这种把戏,我在两岁时就会玩了,把我带大的那些人,都是天下迷药的祖宗。``铁心男连眼睛都张不开了,但却拼命大声道''你\ldots\ldots 你想把\ldots\ldots 我怎样\ldots\ldots{}``小鱼儿道:''我也不想把你怎么样,只是.你说的话,我全不相信,我先要将你从头到脚仔细搜一搜,看看究竟存有什么东西。``他话未说完,铁心男苍白的脸,又像是火一般的红了起来,颤声道:''求求你\ldots。求``\ldots 求你,不''。不要``\ldots 他不但声音颤抖,竟连身子也颤抖起来,他的一双手,死命地抓紧衣襟,死也不肯放松,他口中不断呻吟着道:''求求你\ldots\ldots{}

不\ldots\ldots 求求你``\ldots\ldots{}''

但声音越来越弱,终于没有声音了,手也终于松开。小鱼儿站在那里,笑嘻嘻地瞧着他,直等他再也不会动了,小鱼儿才在他身旁蹲了下来,把他的手拉开,他越是求,小鱼儿越想搜。这时,一阵风吹来,吹来了一条人影。

这人影来得竟一丝声音也没有,幽灵般站在小鱼儿身后,朦陇的星光下,依稀可看出她身上衣裳是红的。小鱼儿竟似完全没有察觉!

\hypertarget{ux7b2cux5341ux56dbux7ae0-ux5029ux5973ux51faux73b0}{%
\chapter{第十四章
倩女出现}\label{ux7b2cux5341ux56dbux7ae0-ux5029ux5973ux51faux73b0}}

红衣的人影,在星光下看来是那么窈窕,那么可爱。

她缓缓抬起了手,姿势也是这么轻柔而美丽,就像是多情的仙子,在星光下向世人散播着欢乐和幸福。

但这只手带来的却只有死亡这只手刹那间就要取小鱼儿的性命。

小鱼儿还是好像完全不知道,但口中却突然喃喃道``这人真奇怪,怎么躺在这里睡觉,叫也叫不醒\ldots。喂,喂!这位大哥,你醒醒呀,在这里睡觉要着凉的。''那只本要拍下的手,突然停住不动\ldots。

小鱼儿还在自言自语道``这怎么办呢?''我既然见着了,就不能不管,唉,谁叫我瞧见这口井,谁叫我要来喝水,我也只好自认倒霉了。``红衣人影突然道''你不认得此人?"

小鱼儿就像被针戳着屁股似的跳了起来,转了个身,瞪着大眼睛瞧着这条人影,又像是见了鬼似的。其实,星光下,水桶里剩下的半桶水,就像是面镜子,早已告诉了小鱼儿来的这人就是``小仙女''。但小鱼儿却装得真像,他瞪着眼睛怔了半天,才嗫嚅着道:``小\ldots 小姑娘,你是几时来的?''他话末说完,小仙女已一个耳光打了过去,他想躲,却像是躲不开,直被打得滚倒在地。

小仙女``张菁冷冷道''你这小鬼也敢叫我小姑娘?``小鱼儿捂着嘴,哭丧着脸从地上爬起,惨兮兮地道''是\ldots\ldots 大姑娘,我\ldots\ldots"话未说完,另外半边脸又挨了一个耳括子。

小仙女厉声道``大姑娘也不是你叫的。''

小鱼儿道``是,姑姑\ldots\ldots 阿姨\ldots\ldots 我不敢了。''小仙女道:``哼,这样还差不多。''

这话虽然还是冷冰冰的,但在她说来已是和气多了。她简直想不到自己会这样和气,也不知怎地,瞧见小鱼儿这样的孩子,竟连她的心都硬不起来。

小鱼儿眨着眼睛,突然又道``阿姨,你也莫要生气,我有个叔叔,说人若生气,肉会变酸,不\ldots.''不``人若生气,就会变老,变丑的,阿姨你这么美,若是万一真的变老变丑了岂非要教人难受得很。''他眨着大眼睛说着,小仙女居然听了下去。她瞧着小鱼儿的脸,不禁觉得这孩子真是奇怪得狠。

她竟不由自主脱口道``我真的很美么?一句话出口突然觉得自己实在太和气了,反手又是一个耳光掴了出去,瞪圆了那双美丽的眼睛,厉声道:''就算美也不要你说。``小鱼儿暗暗好笑,他已觉出这一掌已轻得多但口中却哭兮兮道''是,阿姨虽然美,但我却不说了。``小仙女道:''你这小鬼,怎会到这里来的?"

小鱼儿道``我跟着几位叔叔来做生意今天我大叔买了匹小马,叫我骑着玩,哪知这匹马虽小,却厉害得狠,竟发疯般一阵跑,我拉也拉不住,就糊里糊涂地被这鬼马弄到这里,也不知这里究竟是什么地方。''他眼睛眨边不眨,想也不想,大篇谎话就顺理成章地从嘴里流出来,简直比真的还叫人相信。

小仙女点头道;``不错,无论多柔顺的马,一旦疯狂起来,真是谁也拉不住的莫说你这么个小孩子了。''她自然是身受其痛,所以对这``小鬼''的遭遇不觉有些同情.却不知她``痛''的正是面前这小鬼``小鱼儿暗中几乎笑断了肠子,口中却连连道:''是呀,我被这疯马折腾了一天,好容易等它跑不动了,瞧见这里有口井,刚想喝口水,哪知却瞧见这个睡虫。``小仙女瞧了铁心男两眼,冷笑道''哼!你以为他是真的睡着了么?``小鱼儿失声道''不是睡着,难道是死了?"

小仙女道:小鬼,告诉你,他是中了别人迷药\ldots\ldots 奇怪,他怎会被人迷倒的?\ldots。也好,我正可搜搜那东西在哪里?"她对小鱼儿已全无疑心,竟也喃喃自语起来,小鱼儿瞧着她捏铁心男的身子\ldots 心里直着急,却也没法子。

哪知她搜了一遍,却什么也没搜着,小鱼儿更奇怪,想不到那东西。竟真的不在铁心男身上,那么,我说要搜他时,他为什么急得要命?

突听小仙女失声道``不好,那东西莫非已被迷倒他的人先换走了?那会是什么人?''小鬼快提桶水来,泼醒他,我要问他的话。``小鱼儿赶紧笑道:''是,莫说一桶,十桶我也提得动。``但他却像是一桶也提不动的样子,一面打水一面喘气,好容易打满了一捅,喘着,喃喃道:''这鬼捅怎么这样重,我\ldots.脚下突然一个踉跄,身子也噗地跌倒,水桶也直飞了出去,一满桶水溅在小仙女身上。

小仙女大骂道``你这笨猪,你你要死。''

小鱼儿脸都骇白了,连滚带爬站起来,脱下衣服.笨手笨脚地去擦小仙女身上的水,嘴里连声道:``阿姨.姑姑\ldots 我不是故意的,我该死''小仙女恨声道:``瞧你长得还像个人哪知你却是个笨猪、死猪你要不把我身上弄干净,我不宰了你才怪。''她跺着脚,抖着衣服,小鱼儿手忙脚乱,跪在地上替她擦,她越说越气,刚想把这``小笨猪''一脚踢出去。哪知她脚还未抬起,膝上``阴陵穴''突然一麻,半边身子立刻不能动了小仙女大惊喝道``小鬼,你\ldots{}''小鱼儿道:``对不起,我不是故意的.对不起\ldots.''对不起``口中说话,手也没闲着,竟自她''宗鼻``、''梁邱``、''伏免``、''鹊灵``等穴道一路点了上去,竟几乎将她''足阳明经"上所有的穴道全都点了遍。

小仙女哪里还会不跌倒。

她年纪虽小,但厉害的角色却已会过不少,其中也颇有几个出名的坏蛋,她做梦也想不到这小鬼竟比所有的坏蛋加起来还坏十倍.竟连她都瞧不出,竟连她都栽了,她气得全身发抖,却又偏偏无可奈何。

小鱼儿这才笑嘻嘻站起来,故意瞪大眼睛道:``哎呀,你生病了么?着凉了么?怎会跌倒了?\ldots\ldots 唉,不想你竟如此娇弱,才沾沾冷水就病了。''小仙女眼睛已冒出火来,颤声道:``好,你很好,我竟瞧不出你有这么好!''小鱼儿笑道:``对不起,我实在不是故意的,这桶水我本来是要送给你那匹马喝的,我烧了它的屁股,心里实在过意不去,只可惜它想来被你送去治伤去了,我只好将这桶水转送给你反正你们俩姐妹谁受都一样。''小仙女嘶声道:``原来樱桃就是被你\ldots\ldots 你这小鬼烧伤的。''小鱼儿大笑道:``火烧樱桃,水淹仙女,确这笨猪还不算太笨吧\ldots\ldots 告诉你,永远莫要将别人瞧得太笨也永远不要占人家的便宜要别人叫你阿姨,一个小孩子若总是想占别人的便宜,就一定会倒霉的。''他也不管小仙女气得发疯,笑嘻嘻地抱起了铁心男的身子,放到那匹小白马的背上,像是要走了。

小仙女拼命咬着牙,拼命忍信,她毕竟算聪明,知道这``眼前亏''若能不吃时,总是不吃的好。

哪知小鱼儿突又回过头,瞧着她笑道``对了,还有,你方才打了我叁巴掌,我可不能不还给你,瞧在你是个女人份上,我不加利息就是。''小仙女惊呼道``你\ldots\ldots 你敢?''

小鱼儿笑道``我不敢\ldots\ldots 我不敢\ldots。''

随手就是一个大耳光掴了过去,直打得小仙女脸都红了,她一辈子几曾吃过这样的亏,嘶声呼道``你\ldots 你,好你记着''小鱼儿笑道``你放心,我什么事都忘不了的,你第一个耳光打得我好重,所以我也不能打轻,但第二个就会打轻些了。''第二个耳光掴下小仙女虽除拼命忍住,但眼泪已不禁流了出来,她从生出来到今天,哪有人碰过她一根手指。

她流泪的眼睛,狠狠瞪着小鱼儿,道:``好,我永远也不会忘记你永远永远''小鱼儿笑道``我知道你永远也不会忘记我的,女人对第一个打她的男人,总是忘不了的,能被你这样的女人常常记在心上,我也开心得很\ldots{}''他大笑着接道``侗我这第叁个巴掌还是不能留着''。只是,你第叁下却又实在打得我很轻,我也实在不忍打重了,你说该怎么办呢``小仙女大吼道''你\ldots\ldots 你去死吧"

小鱼几眨了眨眼睛,笑道:好,就这样吧,这样就算互相抵过.谁也不欠谁了。"眼睛瞧着小仙女的眼睛,缓缓俯下了头。

小仙女连心都颤抖了起来,道``你\ldots 你想怎么样?''小鱼儿笑道``你用手打我,我用嘴打你,一定比你手打得还轻。''小仙女大惊叫道``你这恶贼你\ldots{}''

``敢''字还未说出.小鱼儿已轻轻托住了她的下巴,在她那柔软的小嘴上,轻轻亲了亲。

小仙女突然不叫了,整个人都似已呆住,整个人都似已麻小鱼儿却突然叹道``你也最多不过十五六岁,怎么能做我的阿姨,做我的老婆还差不多\ldots.你这么香的嘴,我一天亲十次都不会嫌多。''小仙女瞪着眼睛,一字字道``你若敢再动我一动,我一定要杀死你\ldots 一定要杀死你\ldots\ldots{}''小鱼儿大笑道:``你放心,我再也不会动你了,像你这么凶的女人.送给我我都不要,若有人真的娶了你这雌老虎,那才是真倒了穷霉。''小仙女突然嘶声大叫道``你杀了我吧你最好杀了我否则我一定要你死在我手里,我要让你慢慢的死,一寸寸的死!''小鱼儿哈哈大笑,转身拉过了马。

小仙女大叫道``你为何不杀我?为何不杀我?总有一天,你要后梅的,我发誓,你一定要后悔的。''小鱼儿却已笑着扬长而去,连瞧都不再瞧她一眼。小仙女望着他走远,终于忍不住放声痛哭起来。

只听远远传来小鱼儿的歌声:``小仙女,惨兮兮,掉眼泪,流鼻涕,小鱼儿听见了,拍手笑嘻嘻\ldots{}''小鱼儿一面走,一面唱。他突然发觉自己歌喉还不错,唱得简直比小仙女的哭还好听。直到小仙女的哭声听不见了唱得也没了精神摸摸脸.叹了口气摸摸嘴,又忍不住笑了起来。

那母老虎下手可真不轻,他的脸到现在还疼,但她的嘴却又真香,那甜甜的香气此刻似乎还留在他嘴边。他突然大笑着向前跑,跑得小白马又开始喘了气,他突又停住了脚,在星空下下来,他委实累了。草原上的星空,是那么辽阔,那么灿烂,风吹着他的脸,他糊里糊涂地想着,竟糊里糊涂地睡着了。

他梦见小仙女躺在他怀里,对他说``每天只准你亲我一百次,一次也不能多,一次也不能少。''但他刚要去亲时小仙女却又跳了起来,打他的耳光\ldots 不对,真的有人在打他耳光,莫非小仙女又追来了?!他一惊醒,却瞧见了铁心男,打他的竟是铁心男,方才那桶水,也有些溅到他脸上,他竟提前醒来了。

星光下,铁心男苍白的脸,满是怒容,一双美丽的大眼睛,正狠狠地瞪着小鱼儿,咬着牙道:``小鬼,你也有睡着的时候,你也有落在我手里的时候。''小鱼儿想跳起来,身子已不能动了,他竟也被人点了穴道。

但他却似全不生气,也不着急,反面笑嘻嘻道``我正在做着好梦,你把我吵醒了,你可得赔,我方才正在要亲别人一百次,你就得让我亲一百次。''铁心男身子突然一阵震颤,失声道:``方才你将我怎么样小鱼儿笑道;''也没有怎么样,只不过把你的身子搜了一遍从头到脚,仔仔细细搜了一遍,一寸地方都没有漏。"铁心男身子更抖得像是在打摆子,脸也红得在星光下也能辨出那红色,竟站在那里,说不出话来。

小鱼儿眨着眼睛,叹道:"但你为什么不早告诉我你是女人?

否则我也就不搜你了唉,你要知道,我年纪虽小,毕竟也是个男人呀,怎忍得住``铁心男大叫道''住口!住口再说我就杀你"

小鱼儿笑道:我既已做了,说不说又有什么两样"铁心男咬着牙,眼泪又已在眼圈里打转。

小鱼儿扮着鬼脸道:``看来,你只有嫁给我了,我也只有娶个年纪大的老婆\ldots{}''``唉,等到我叁十岁时,你已是老太婆了。''铁心男突然自靴筒里拔出匕首,颤声道:``你\ldots 你还有什么遗言留下来,快说吧。''小鱼儿瞪大了眼睛,失声道``你要杀我?!你就算还要嫁给别人,也没关系呀,我保证绝不反对,你又何必定要杀我?''铁心男咬着牙道``你若无话说,我就动手了''

她突然转过头,颤声接着道``但你也可放心,我绝不嫁给别人''小鱼儿听得几乎要笑出来,却又实在笑不出,非但笑不出,倒差点要哭,老天,她竟真的相信了。

``唉女人,女人\ldots 你究竟是聪明还是笨?''

小鱼儿苦笑着道:``求求你,嫁给别人,你爱谁就嫁给谁,嫁给谁都没关系只要不嫁给我就好了,我实在受不了。''铁心男嘶声道``这,这就是你要说的话么?\ldots\ldots{}''手里紧握着的匕首,竟真的往小鱼儿的胸膛刺了下去。

小鱼儿大叫道``慢着,慢着,我还有话说''

铁心男跺脚道``快说快说''

小鱼儿叹道``我还有句话,要你转告天下的男人,叫他们千万不要救别人的命,尤其不要救女人的命.他若瞧见有别人要杀女人,千万莫烧那人的马屁股,要烧的也只能烧自己的马屁股,走得越远越好越快越好。''铁心男道``不错,你是救了我性命但''\ldots 但我\ldots。

突然坐到地上,放声痛哭起来,痛哭着道``我怎么办呢?\ldots\ldots 怎么办呢?''小鱼儿柔声道:``你不要烦恼还是杀了我吧,与其比你烦恼,倒不如让我死了算了,我能死在你手上,也很开心了。''他嘴里说着,眼睛却一直偷偷瞪着铁心男,铁心男果然越哭越伤心,小鱼儿心里却越来越得意:对付女人的法子,我总算知道了你只要能打动她的心,她就会像马一样乖乖地被你骑着,你要她往东,她就往东,要她往西,她就往西。"哪知他正在得意时,铁心男却已痛哭着一跃而起,发了狂似的向前跑,也不知要跑到哪里去。

小鱼儿这才真的吃惊,大呼道:``嗯,你不能抛下我走呀,若是有狼来了,老虎来了怎么办?若是小仙女来了怎么办?你可知道,我方才又救了你?''一─"他叫得虽响,铁心男却已听不见了。

风,虽仍是那么柔和,星空虽也是同样的那么灿烂,那么辽阔,但躺在下面的小鱼儿,却一点也不舒服了。他真是一肚子恼火口中喃喃叹道;``江鱼呀江鱼,这怪谁?这还不是怪你自己.谁叫你要惹上女人?狼来吃了你,小仙女来宰了你你也活该。''那小白马已走了过来,在他身旁不住轻嘶。

小鱼儿道``小白菜,我说的话不错吧,下次你若见到有人要用绳子勒死女人,你就赶紧替他架板凳,你若见到有人要用刀杀女人,你就赶紧替他磨刀。''那小白马一声轻嘶突然跑了开去。

小鱼儿苦笑道``好个小白菜,原来你也是不可靠的,你竟也抛下了我,唉,想来你大概也是匹母马\ldots\ldots{}''但他已突然发现小白菜跑去的地方,竟动也不动地站着一个人,星光下,这人身上那雪白的衣裳,比马还白、铁心男竟也回来了。小鱼儿又惊又喜,却忍住不出声,只见小白马跑到她身旁,轻嘶着,她身子终于移动,一步步走了过来\ldots 风吹着她的衣服,她的体态是那么轻盈。

小鱼儿暗叹道``我真是瞎子,竟直到现在才猜到她是女人,我\ldots{}''我第一眼该已瞧出来的,男人哪有这样走路的?"铁心男已走到他身边。小鱼儿却闭起眼睛,故意不理她。

只听铁心男幽幽道:``你并没有真的欺负我。''小鱼儿再也忍不住,笑道``你现在才知道么?''铁心男道``但。\ldots 但你还是欺负了我,所以你。\ldots 你''\ldots\ldots{}``小鱼儿道''看在老天的份上,把你真正要说的话快些说出来吧。``铁心男垂下了头,沉着脸道''你愿不愿意陪我去一个地方?``小鱼儿道''我自然愿意,但你先得解开我的穴道,我才能走呀你\ldots\ldots 你总不能,背着我抱着我走吧。``铁心男脸更红了,却忍不住''噗嗤"一笑,果然俯下身子,轻轻拖着小鱼儿,虽然还在为他解着穴道,却也像是不忍下重手。

小鱼儿苦笑道``你方才打我时,下手那么重,此刻解我的穴道,下手却又这么轻了,唉,老天.唉.女人\ldots。''总算站了起来。

铁心男却背转了脸,轻轻道``我以前不要你跟我,此刻又要你陪着我,只因我想来想去,知道你\ldots{}''你还是对我很好的。``小鱼儿道''你以前不知道?"

铁心男道``我\ldots\ldots 我以前不让你去,只因那地方太秘密\ldots。''小鱼儿道:``你要去的地方究竟是在哪里?''

铁心男缓缓道``那地方在昆仑山中,是''\ldots\ldots"

小鱼儿失声道``恶人谷?!你要去的地方莫非竟是恶人谷?''铁心男霍然回首,睁大了眼睛,道``你怎么知道?''小鱼儿打着自己的头,喃喃道:``老天\ldots\ldots 老天,这位大姑娘在问我怎会知道恶人谷?我若不知道恶人谷.世上人怕再也没有人知道了。''铁心男眼睛瞪得更大,道``为什么?''

小鱼儿道``你且莫问我为什么?看在老天份上,先告诉我你为什么要去恶人谷吧?看你的模样,实在不像是要去恶人谷的人。铁心男道:''我。\ldots 我只是去找个人``小鱼儿道''找谁?"

铁心男道``告诉你,你也不会知道。''

小鱼儿大笑道``我不会知道?''\ldots\ldots 恶人谷上上下下,大大小小,有谁我不知道的?``铁心男吃惊道''你\ldots"

小鱼儿大声道:``我\ldots\ldots 我就是在恶人谷长大的。''铁心男脸色变了,道:``我不信\ldots。我简直不能相信。''小鱼儿大笑道:``你不信?我且问你,除了恶人谷那种地方,还有什么地方能养大一个像我这样的人?''铁心男呆了许久嫣然一笑,道:``的确没有别的地方了,我本该早巳想到的。''小鱼儿道``现在你总可告诉我,找的是谁了吧?''铁心男又垂下了头,默然半响,缓缓道``我找的人也姓铁他是个很有名的人。''小鱼儿道``莫非是十大恶人中的狂狮铁战?''铁心男霍然抬头,失声道:``你认得他?他果真在那里?''小鱼儿笑道:``幸好你遇着我,否则你就要白走一趟了,是什么人告诉你狂狮铁战在恶人谷的?你真该打那人的屁股。''铁心男骑在马上,小鱼儿拉着马,铁心男没有说话.小鱼儿也没有说话,那小白马自然更不会说话了。

夜,很静,很冷,回头望夫,仍可望见那千里无际的大草原,静静地沐浴在星光下,草浪起伏如海浪。他们终于已走出了草原,这平静但又雄奇壮丽,单调却又变化迷人的大草原,已在小鱼儿心中留下永生不能磨灭的印象\ldots 但小鱼儿却没有回头,没有再去瞧一眼一─过去的,既已过去了,就让它过去吧。留恋?不!绝不铁心男的脸,在星光下看来更苍白得可怕,她的确很美,小鱼儿自从知道她是女人后,就发现她实在比别的女人都美,也发现她比自己想象中脆弱得多,自从知道那消息,她非但没有说话简直连动都不能动了,若不是还有这匹小白马,她简直连一步都不能走。

小鱼儿不禁在暗中摇头叹息``女人\ldots\ldots 女人究竟是经不起打击的,最美的女人和最丑的都是一样。''他暗中摇头,嘴里并没有说,他懒得再说。铁心男却突然说她长长的睫毛,覆盖着朦胧的眼波,她眼睛并没有去瞧小鱼儿,只是梦呓、轻语着道``你已有许久未曾说话了。''小鱼儿道:``你不说话,我为何要说话?''

铁心男道``但\ldots。你难道没有话问我''

小鱼儿道:``我为何要问你我什么不知道铁心男道:''你知道什么?``小鱼儿懒洋洋地一笑,道''被人逼得没路可走了终于想到去投靠你的父亲,虽然你本来对他并没有多大的好感甚至在很小的时候便已离开了他甚至是在很小的时候便已被他抛弃了,但他,毕竟是你的亲人。``铁心男朦胧的眼波突然亮了瞪着小鱼儿,道:''我的父亲?

谁是我的父亲?"

小鱼儿道``狂狮铁战。''铁心男失声道``谁\ldots\ldots 谁说的。''小鱼儿打了哈欠,道``我说的!''``唉,女人,我知道女人明明被人说中了心事,也是万万不肯承认的,所以,你承不承认都没关系。''铁心男瞪着小鱼儿,好像是从来都没有见过他似的一─这孩子简直不是人,是妖怪,是人中的精灵。

她呆了半晌终于又道:你\ldots\ldots 你还知道什么?``小鱼儿道''我还知道你的名字并不是男人的男,而是兰花的兰,铁心兰\ldots 这才像是你的名字,是么?"

\hypertarget{ux7b2cux5341ux4e94ux7ae0-ux6709ux60caux65e0ux9669}{%
\chapter{第十五章
有惊无险}\label{ux7b2cux5341ux4e94ux7ae0-ux6709ux60caux65e0ux9669}}

铁心男道:不\ldots\ldots 不\ldots\ldots 唉,不错,兰花的兰。``小鱼儿一笑道''我知道你现在心里很彷徨,也不知要到哪里去,也不知该怎么办,所以,我不说话,让你静静想一想。``铁心兰苦笑道:''你究竟有多少岁?\ldots\ldots 我有时真害怕,不知道你究竟是个真正的孩子,还是个\ldots\ldots 是个\ldots\ldots{}``小鱼儿道''妖怪?"

铁心兰轻轻叹息一声,道``有时真忍不住要以为你是精灵变幻而成的,否则你为什么总是能猜中别人心里的事?''小鱼儿正色道``因为我比世上所有的人都聪明得多。''铁心兰幽幽道:``也许你真的是\ldots\ldots{}''

小鱼儿道;``好,现在你想通了么''

铁心兰道``想通什么?''

小鱼儿道:``你可想通你究竟该怎么办?到哪里去?''铁心兰又垂下了头,道``我\ldots 我\ldots。''

小鱼儿道``你可要快些想,我不能总是陪着你。''铁心兰霍然抬头,脸更白得像张纸,失声道:``你\ldots 你不能?''小鱼儿道``自然不能。''

铁心兰道``但\ldots\ldots 但本来\ldots{}''

小鱼儿道``不错本来我想和你结伴,到处去闯闯,但现在你既然是个女人我计划就要变了.我也不能再要你做徒弟了。''铁心兰颤声道``但你你\ldots{}''

小鱼儿道:``我和你非亲非故,两个人在一起到处跑算什么?何况,我还有许多事要做怎么能被个女人缠着。''铁心兰像是突然挨了鞭子,整个人都呆住,整个人都颤抖了起来.也不知过了多久,终于凄然一笑道``不错,我和你非亲非故,你\ldots\ldots 你走吧。''小鱼儿道:``那么你\ldots\ldots{}''

铁心兰努力挺直身子,冷笑道``我自然有我去的地方,用不着你关心。''小鱼儿道``好,你现在只怕还不能走路,这匹马,就送给你吧。''铁心兰拼命咬着嘴唇,道:``谢谢,但\ldots 但我也用不着你的马我什么都用不着你的,你\ldots\ldots 你\ldots\ldots{}''跃下马,立刻转过了头。只因她死也不愿小鱼儿瞧见她泪流满面。小鱼儿也装作没有瞧见,牵过了马,笑道``你用不着也好,我本也有些舍不得这匹马我若和它分别倒真还有些难受。''铁心兰颤声道:``我\ldots\ldots 我\ldots\ldots{}''

她本想说;``我难道还不如这匹马?你和我分别难道没有一点难受?''但她没有说出来,显然她心已碎了。

小鱼儿道:``好,我走了,但愿你多多保重。''铁心兰没有回头,只听到他上马,打马,马蹄刚去──他竟就真的这样走了,铁心兰终于忍不住嘶声呼道:我自然会保重的,我用不着你假情假意地来关照我,我\ldots 但愿死也不要再见你!"终予扑倒地上,放声大哭起来。

小鱼儿并没有听到这哭声──无论如何,他至少装作没有听见,他只是拍马的头,喃喃道、``小白菜,你瞧我可是个聪明人,这么容易就将个女人打发走了,你要知道,女人可不是好打发的。''他骑着马,头也不回地往前走走了许久,突又喃喃道``小白菜,你猜她会到哪里去,你猜不着吧?一一告诉你,我也猜不着,咱们在这里等等,偷偷瞧瞧好么?''小白菜自然不会答对的,虽然它也未必赞成小鱼儿却已下了马,喃喃道``能瞧瞧女孩子的秘密,总不是件坏事,何况\ldots。咱们也没有什么事急着去做,等等也没关系,是么?''小白菜自然也不会揭穿他,这不过是自己在替自己解释的有时候马的确要比人可爱得多,至少它不会揭破别人的秘密!也不会出卖你。

星群渐渐落下,夜已将尽。

铁心兰还没有来,难道她不走这条路但这是唯一的路呀,莫非她迷了路?莫非她又\ldots\ldots{}

小鱼儿突然上马,大声道:``走\ldots\ldots 小白菜.咱们再瞧瞧去,瞧瞧她究竟要搞什么鬼你要知道,我可不是关心她,我是什么人都不关心的。''他话未说完,马早巳走了,走的可比来时要快得多,片刻间又到了那地方,小鱼儿远远便瞧见了铁心兰。

铁心兰竟还卧倒在那里,也不哭了,但也不动。

小鱼儿从马上就飞身掠过去,大声道:``喂,这里可不是睡觉的地方。''铁心兰身子一震,挣扎着爬起,大声道``走!走!谁要你回来的,你回来干什么?''夜色中,只见她苍白的面色,竟已像是红得发紫了,那娇俏的嘴唇不住颤抖着,每说一个字,都要花不少力气。

小鱼儿以失声道:``你病了。''

铁心兰冷笑道``我病了也用不着你管你\ldots\ldots 你和我非亲非故你为什么要管我?''她身子虽已站起但却摇摇欲倒小鱼儿道``我现在就偏偏又要管你了。''突然飞快地伸出手,一探她的额角,她额角竟烫得像是火。

铁心兰拼命拦开他的手,颤声道``我不要你碰我。''小鱼儿道``我偏要碰你。''突然飞快地抱起了她。

铁心兰大叫道``你敢碰我\ldots\ldots\ldots\ldots 你放手,你滚。''她一面挣扎一面叫,但挣扎既挣不脱,叫也没力气,她拳头打在小鱼儿身上,也是软绵绵的。

小鱼儿道你已病得要死了,再不乖乖的听话,我\ldots 我就又要脱下你的裤子打屁股了,你信不信?``铁心兰嘶声叫道''你\ldots\ldots 你\ldots"

突然埋头在小鱼儿怀里,又放声痛哭起来。

铁心兰真的病了,而且病得很重。

到了海晏,小鱼儿就找了家最好的客栈,最好的屋子,这屋子本已有人住着,但他拿出块金子,大声道:``你搬走,金子就给你。''他一共只说了八个字,那人已走得比马都快─金子虽然不会说话,但却比任何人说几百句都有用得多。

焦急、失望、险难、打击、伤心,再加上草原夜里的风寒,竟使得铁心兰在高热中晕迷了一天多。

她醒来的时候,小鱼儿正在煎药,她挣扎着想爬起,小鱼儿却将她按下去。

她只呻吟着道``你\ldots\ldots 你为什么\ldots\ldots{}''

小鱼儿却大声道:``不准开口。''

她瞧见小鱼儿眼圈已陷了下去.好像是为了照顾她已有许多夜没睡了她眼泪不禁又流下面颊。

小鱼儿将药碗端过来,道``不准哭,吃药,这是最好的药方,最好的药,你吃下去后,立刻就会好了,若像小孩子似的好哭,就又要打屁股了。''铁心兰道:``这\ldots\ldots 这是谁开的药方?''

小鱼儿板着脸道:``我。''

铁心兰道``原来你还会看病,你难道什么都会''小鱼儿道:``不准开口,吃药。''

铁心兰轻轻一笑,虽在病中,笑得仍是那么妩媚。

她嫣然笑道``你不准我开口,我怎么吃药呀?''小鱼儿也笑了,他突然发现女孩子有时也是很可爱的,尤其是她在对你很温柔地笑着的时候:黄昏,铁心兰又睡了。

小鱼儿踱到檐下,喃喃道``江鱼呀江鱼,你切莫忘记,女孩子这样对你笑的时候,就是想害你,就是想弄条绳子套住你的头,她对你越温柔,你就越危险,只要一个不小心,你这一辈子就算完了。''那白马正在那边马棚嚼着草。小鱼儿走过去,抚着它的头,道``小白菜,你放心,别人纵会上当,但我却不会上当的,等她病一好,我立刻就走''突听一阵急遽的马蹄声,停在客栈外,这客栈麻雀虽小,五脏俱全,外面还附带家酒铺。

小鱼儿听得这蹄声来得这么急,忍不住想出去瞧瞧。

远远就瞧见四五条大汉冲进店来,一言不发,寻了张桌子坐下.店家也不敢问,立刻摆上了酒,但这些人却呆了似的坐在那里,动也不动。他们的衣着鲜明,腰佩长剑,气派看来倒也不小,但一张张脸却都是又红又肿,竟像是被人打了几十个耳括子。过了半晌.又有两个人走进来,这两人更惨,非但脸是肿的,而且耳朵也像是不见了一只.血淋淋地包着布。

先来的五个人瞧见这两个人,眼睛都瞪圆了,后来的瞧见先来的,脚一缩,就想往后退,却已来不及。

小鱼儿瞧得有趣,索性躲在外面,瞧个仔细。

这两批人莫非是冤家路窄,仇人见面,说不定立刻就要动起手来.小鱼儿可不愿进去淌这趟浑水!哪知这两批人却全没有动手的意思,只是先来的瞪着后来的,后来的瞪着先来的,像是在斗公鸦!

先来的五人中有个麻面大汉,脸上已肿得几乎连满脸的金钱麻子都辨不清了,他瞧着瞧着,突然大笑道``镖银入安西,太平送到底\ldots。安西镖局的大镖师岂不是从来不丢东西的么,怎地连自己耳朵都丢!,这倒是奇案。''他这一笑,脸就疼得要命,但却又实在忍不住要笑,到后来只是咧着嘴,也分不出是哭是还是笑,后来的两人连眼睛都气红了,左面一条脸带刀瘤的大汉.突也冷笑道:``若是被人打肿了脸,还是莫要笑的好,笑起来疼得狠的.''麻面大汉一拍桌子,大声道``你说什么?''

刀疤大汉冷冷笑道:``大哥莫说二哥,大家都是差不多。''麻面大汉跳了起来,就要冲过去刀疤大汉也冷笑着站起身子,小鱼儿暗道``这下可总算要打起来了。''哪知两人还未动手,手已被身旁的人拉住。

拉住麻面大汉手的,是个颔下胡子已不短的老者年纪看来最大脸上也被打得最轻,此刻摇手强笑道``安西镖局和定远镖局.平日虽然难免互相争生意.抢买卖,但那也不过只是生意买卖而已,大家究竟还都是从中原来的江湖兄弟,千万不可真的动起手来,伤了兄弟间的和气。''拉住刀疤大汉的一条瘦长汉子,也强笑道``欧阳大哥说的不错咱们这些人被总局派到这种穷地方来,已是倒了霉了,大家都是失意人,又何必再呕这闲气。''那老者欧阳叹道;``何况,咱们今曰这跟斗,还像是栽在同一人的手上,大家中该同仇敌忾才是,怎么能窝里翻,却让别人笑那瘦长汉子失声道''各位莫非也是被她\ldots。

老者欧阳苦笑道``不是她是谁?除了她,还有谁会莫名其妙地下如此毒手,唉咱们兄弟今天可真算栽了。''他说了这句话,七个人全都长叹着坐了下去。

这七人脸上虽已肿得瞧不出什么表情,但一双双圆睁的眼睛里,却充满了怀恨怨毒之意。

那麻面大汉又一拍桌子,恨声道``若真是为着什么,咱们被那丫头欺负,那倒也罢了,只恨什么事也不为,那丫头就出手了!''老者欧阳长叹道``江湖之中,本是弱肉强食,不是我长他人志气,咱们武功实在连人家十成中的─成都赶不上,纵然受气,也只得认了。''那瘦长汉子突然笑道;``但瞧那丫头的模样,也像是在别处受了欺负,非但眼睛红红的,像是痛哭了场,就连她那匹宝贝马都不见了,只怪咱们倒霉恰巧撞在她火头上她就将一脑子气都出在咱们身上了。''麻面大汉拍掌笑道``徐老大说的不错,那丫头想必是遇上了比她更厉害的,也说不定遇着个漂亮的小伙子,非但人被骗去了,就连马也被人骗走了。''几个人一起大笑起来,虽然一面笑,一面疼得龇牙咧嘴但还是笑得极为开心,像是总算出了口气。

听到这里,小鱼儿早已猜出这些人必定是遇着小仙女了,小仙女打耳光的手段,他是早巳领教过的!但小仙女这次出手,可比打他时重得多,她在那井边想必受了一夜活罪,这口气正好出在这群倒霉蛋身上。小鱼儿越想越好笑,但突然间,外面七个人全都顿住了笑声,龇牙的龇牙,咧嘴的咧嘴,歪鼻子的歪鼻子,所有奇形怪状的模样,全都像中了魔般冻结在脸上,双双眼睛瞪着门口,头上往外直冒冷汗。

``小仙女''张菁已站在门口,一字字道``我叫你们去找人,谁叫你们来喝酒''小鱼儿一颗心已跳出腔来,但却沉着气,一步步往后退,他自然知道小仙女要他们找的人,就是他自已。幸好这时已入夜,屋子里已点上灯,院子里就更暗,小鱼儿沿着墙角退,一直退到那马棚。

他不但人不能被小仙女瞧见,就是马也不能被她瞧见,该死的是,这匹马偏偏是白的,白得刺眼。马槽旁地是湿的,小鱼儿抓起两把湿泥,就往马身上涂,马张嘴要叫,小鱼儿就塞了把稻草在它嘴里,拍着它的头,轻轻道``小白菜.白菜兄你此刻可千万不能叫出来,谁叫你皮肤生得这么白,简直比铁心兰还要白得他说完了,白马已变成花马小鱼儿自己瞧瞧都觉得好笑,他将手上的泥都擦在马尾上,悄悄退回屋子,这屋子里没点灯,但铁心兰却已醒了,两只眼睛就像是灯一样瞪着瞧见小鱼儿进来,突然一把抓住了他,嘶声道;''我的靴子呢?``小鱼几道''靴子?就是那双破靴子?``铁心兰喘息着道;''就\ldots。就是那双小鱼儿道``那双靴子底都已磨穿,我已抛到阴沟里去了。''铁心兰身子一颤颤声道``你\ldots\ldots 你抛了''

小鱼儿笑道:``那双破靴子,叫化子穿都嫌太破,你可借什么?紧张什么我已替你买了双新的,比那双好十倍''铁心兰挣扎着往床下跳,颤声道``你抛到哪里?快带我去找!你一一─你这死人,你可知道我那靴子.靴子里藏着\ldots\ldots{}''小鱼儿眼睛眨眨,道:``藏着什么?''铁心兰道``就是那东西\ldots\ldots 我为了它几乎将命都送了,但你却将它抛到阴沟里,我\ldots\ldots 我不如死了算了。''小鱼儿道:``那东西?那东西莫非不在你身上么?''铁心兰眼眶里已满是眼泪,道:``那是我骗你的。''小鱼儿叹道``谁要你骗我,这一来你可是自己害自己,我把那破靴子随手─抛,根本不知道抛在哪里。''铁心兰当场倒在床上,不能动了,口中喃喃道``好\ldots 很好,什么都完了。''小鱼儿微微笑道:``那东西也只不过是张破纸而已丢了也没什么了不起,你又何苦如此着急,急坏了身子可不是好玩的。他话未说完,铁心兰已一骨碌爬起来,瞪着他道''你。\ldots 你怎知道那─一那是张纸?``小鱼儿笑道''你若说的就是那张纸.我已从靴子里拿出来过纸不但已破了,还是臭臭的,有股臭咸鱼的味道。``铁心兰整个人都扑到他身上捶着他的胸.又笑又叫,道:''你这死人\ldots\ldots 你放意让我着急。``小鱼儿笑道''谁叫你骗我\ldots{}``我早巳猜出那东西是在你靴子里的\ldots。你居然想得出把那么重要的东西藏在靴子里,可真是个鬼灵精。''铁心兰道``你才是鬼灵精,什么事都瞒不过你,你\ldots\ldots 你方才真骇死我了。''小鱼儿道"但东西还是落在我的手里,你不着急?

铁心兰垂下了头,道:``在你手里.我还着急什么?''小鱼儿道``你不怕我不还给你?''

铁心兰道``我不怕。''

小鱼儿道``好,我就不还你。''

铁心兰柔声道:``那,我就送给你。''

小鱼儿瞪起眼睛道:``但\ldots\ldots 但你本来死也不肯将这东西给别人的。''铁心兰道``你\ldots 你和别人不同。''

也不知怎地,小鱼几突然觉得心里甜了起来,全身飘飘然,就好像一跤跌进成堆的棉花糖里。

但他立刻告诉自已``江鱼,小心些,这糖里有毒的。''他立刻想把铁心兰往外推.不知怎地,却推不下手。

铁心兰悠悠道``方才你到哪里去了?''

小鱼儿道:``外面\ldots\ldots 我还瞧见一个人。''铁心兰道:``谁?''小鱼儿道"这人你认得的\ldots\ldots 我不幸也认得\ldots\ldots{}

铁心兰耸然道``小\ldots\ldots 小仙女?''

小鱼儿笑道``对了,就是她。''

铁心兰颤声道``她在哪里?''

小鱼儿道``你打开窗子只怕就可见到。铁心兰手脚都凉了,道''她\ldots{}``她就在外面,你却还有心在这里和我开玩笑?''小鱼儿道``她就在我面前,我也是照样开玩笑。''铁心兰咬着嘴唇,道``你这人\ldots{}''现在,我们该怎么办呢?``小鱼儿道:''现在,叁十六着.走为上着,咱们\ldots{}``话犹未了,突听外面远处有人厉声喝道:''叫你开门你就得开门,大爷们是干什么的.你管不着``接着,''砰"的一声,像是有扇门被撞开了!

小鱼儿叹道``好啦.走也走不了啦''

铁心兰面色如土,颤声道:``看样子小仙女已找了人一间间屋子查过来了,她想必已听说咱们落脚在这附近,但现在他们还未查到这里,咱们赶紧从窗子里逃,还来得及。''她一把拉住小鱼儿的手,就想往窗外逃。

小鱼儿却摇头道;``不行,咱们现在若从窗里逃走,他们就必会猜出是咱们了,那时小仙女追踪而来,咱们也是逃不远的。''铁心兰掌心已满是冷汗,道:``那''``那怎么办?''小鱼儿微微笑道``不怕,我自有法子。''

这时远处又传来女子尖锐的呼声,叫道:``出去\ldots{}''快出去,你们这群强盗怎地也不敲门就闯进来了!\ldots{}``小鱼儿笑道:''这女子莫非正在洗澡。"

他竟似一点也不着急.一面嘻嘻笑着,一面从怀里掏出个已陈旧得褪了颜色的绣花小布袋。

铁心兰道:``这是什么?''

小鱼儿道``这是宝贝\ldots\ldots 是我从一个姓屠的人那里偷来的。''说话间他已自袋里取出一叠薄薄的、软软的、粘粘的,像是豆腐皮,又像是人皮般的东西。

铁心兰眼睛瞪圆了.突然失声道``这莫非就是人皮面具?''小鱼儿笑道``总算你还识货?''

他从那一叠中仔细选出了两张,道;"你先脱下外面的衣服,随便塞在哪里\ldots 再把我这斗篷,反着被在身上.\ldots 好,现在把脸伸过来\ldots\ldots{}

铁心兰只觉脸上一凉,全身都起了鸡皮疙瘩,等她张开眼来,小鱼儿的脸已完全变了模样。

他竟己满脸都是皱纹,只差没有胡子。

铁心兰忍不住轻笑道:``真像是活见鬼,你''\ldots 你竟已变成个小老头了。``小鱼儿道''小老头正好配小老太婆。"

这时脚步声、人语声己渐渐近了。小鱼儿仍是不慌不忙,先从袋子里掏出一撮胡子粘在他自己嘴上,又取出瓶银粉,往铁心兰和他自己头发上洒两个人头发立刻变为花白的,然后,小鱼儿又取出几只粗细不同的笔,也不知画了些什么,就往铁心兰脸上画。

人语声、脚步声越来越近,好像已到他们门口。铁心兰手脚冰冷,四肢已簌簌的发抖。

小鱼儿的手仍是那么稳,口中还不住悄声道:莫怕,莫怕,我这易容改扮的功夫,虽还并不十分到家,但唬唬他们已足够有余了"现在,脚步声真的已到他们门口。

小鱼儿闪电般收拾好东西,扶着铁心兰,道;``走.咱们从大门出去。''铁心兰骇然道``大\ldots\ldots 大门?''

她连声音都急哑了但小鱼儿却己不慌不忙地打开了门.只见方才那几条脸被打肿的大汉,恰巧正走到他们门外,小仙女那窈窕的红衣人影,就在这几人身后。

小鱼儿却连头也不抬,连声道``大爷们让让路,我这老婆子也不知吃错了什么,突然得了重病,再不快去瞧大夫,就要送终了。''他语声竟突然变得又哑又苍老,活像是个着急的老头子,铁心兰身子不住发抖,也正像是个生病的老太婆。

那群大汉非但立刻闪开了路,还闪得远远的,生怕被这老太婆传染,那麻面大汉连鼻子都掩住,皱眉道:``六月天突然发病,八成是打摆子,否则怎会冷得发抖.''小鱼几一面叹着气,慢吞吞地从他们中间走了过去,铁心兰简直要晕了,恨不得立刻插翅而逃,她真不懂小鱼儿怎地如此沉得住气。好不容易走过小仙女身旁,走到院子里,小仙女瞪大了眼睛瞧着他们,也像是丝毫没有怀疑。

哪知他们还未走出几步,``呛□''一声,小仙女突然自一条大汉腰畔抽出了柄快刀.一刀向小鱼儿脑袋上砍下,口中喝道:``你想骗得了我?''铁心兰骇得魂都飞了,但小鱼儿却似毫未觉察,直到那柄刀已到了他头上,立刻就可以将他脑袋切成两半,他还是动也不动,还是一步步慢吞吞走着。那柄刀居然在距离他头发不及半寸处顿住。

就连那些大汉们都不禁叹了口气,暗暗道:``这丫头疑心病好重,连这个糟老头子都不肯放过。''小鱼儿像是什么事都不知道,居然还走到马棚里,牵出了那匹也``易容''过的马,喃喃道``马儿马儿,老太婆虽病了.我可也不能丢下你。''铁心兰急得跟睛都花了,汗已湿透衣服──小鱼儿居然还要牵这匹马她真恨不得狠狠捏他几把。

现在,小鱼儿和铁心兰已站在大街上,铁心兰真不知道自已是怎么走出来的,这简直像做梦,一场恶梦。

她糊里糊涂的被小鱼儿扶上了马,小鱼儿拉着马居然还在慢吞吞的走,铁心兰忍不住道:``老天,求求你,走快些好么?''小鱼儿道``千万不能走快,他们或许还在后面瞧,走快就露馅了,你瞧夜色这么美骑在马上慢慢逛,多么富有诗情画意。''他居然还有心情欣赏夜色,铁心兰长长叹了口气,真不知是该哭还是该笑但长街终于还是走完了。

眼前是一片郊野,灯火已落在他们身后很远。

铁心兰这才长长松了口气,苦笑道:``你这人\ldots\ldots 我真猜不出你的心究竟是什么做的?''小鱼儿道;``心?\ldots\ldots 我这人什么都有,就是没有心\ldots{}''铁心兰咬着嘴唇,带笑瞪着他,道:``方才那把刀若是砍下,你就连头也没有了。''小鱼儿笑道``我早就知她那把刀只不过是试试我的,她若真瞧破了我,真要动手,又怎会再去拨别人的刀?''铁心兰叹道:``不错\ldots\ldots 你在那种时候居然还能想到这种关节,你真是个怪人\ldots\ldots 你难道从来不知道害怕?''小鱼儿大笑道``你以为我不害怕?\ldots\ldots 老实告诉你,我也怕得要死,世上只有疯子白痴才会完全不害怕的。''铁心兰嫣然一笑.道``咱们现在到哪里去?''

小鱼儿道``到哪里去都没关系了,反正再也没有人能认得出你,只是.你的病''铁心兰笑道``我方才被他们一骇,孩出一身冷汗,病倒像是好了,手脚也像是有了力气,你说怪不怪''小鱼儿道``你已能走了?''

铁心兰道:``能.不信我下马走给你看看。''

小鱼儿道``好,你下马走吧\ldots。我也要走了。''铁心兰身子一震,失声道``你。你你说什么?''小鱼儿道``我们不是早巳分手了么?只因为你有病,我才照顾你,现在你病好了,我们自然还是各走各的路。''铁心兰面色惨变.变得比方才听到小仙女来了时更苍白,更可怕,她身子竟又开始发抖泪珠已夺眶而出,嘶声道``你''\ldots\ldots{}

``你难道真的\ldots\ldots 真的\ldots\ldots{}''

小鱼儿道``自然是真的,你将那东西送给了我,我也救了你一命,咱们可算两相抵过谁也不欠谁了''铁心兰泪流满面,咬牙道``你难道真的没有心,你\ldots。你的心莫非已被狗吃了。''小鱼儿笑道:``这次你猜对了。''

铁心兰突然扬起手,狠狠给了小鱼儿一个耳掴子。

小鱼儿动也不动,瞧着她,淡淡道:``幸好我的心已被狗吃了,我真该谢谢那条狗,否则男人的心若被女人捏在手里,倒真不如被狗吃了算了。''铁心兰已痛哭得自马背上扑倒在地,放声痛哭道``你不是人不是人\ldots 一你根本不是人''小鱼儿拉起了她.笑道:再见吧\ldots\ldots 无论我是不是人,至少不是会被女人眼泪打动的呆子,我\ldots\ldots{}``突听一人冷冷道:''不错,你不是呆子,你聪明得很只可惜太聪明了些!"

\hypertarget{ux7b2cux5341ux516dux7ae0-ux5f04ux5de7ux53cdux62d9}{%
\chapter{第十六章
弄巧反拙}\label{ux7b2cux5341ux516dux7ae0-ux5f04ux5de7ux53cdux62d9}}

这语声冷而美,赫然竟是小仙女的声音。

铁心兰哭声立刻顿住,小鱼儿身子虽也一震,但却绝不回头去瞧一眼,口中立刻叹息道``孩子的妈,你哭什么,又死不了的,快去找大夫吧.再迟人家只怕就要关起门来睡大觉了。''只听小仙女冷笑道``你说完了么?不错,你装得很像,你此刻真该去找大夫了,只可借世上所有的大夫都已救不了你。''小鱼儿站在那里,像是突然被钉子钉在地上,动也不能动,铁心兰也是那样伏在地上,连头都未始起。

小仙女道``你还有什么话说?''

小鱼儿突然转过头,突然大笑道``很好,终于被你瞧破了\ldots 但你是如何瞧出来的?可否说来听听。''小仙女冷笑道``我砍下那一刀时风声连聋子都听得出,你若真是个糟老头子,早已骇得扑倒在地,又怎会还是若无其事地往前走?''小鱼儿歪着头想了想,长叹道:``不错,原来你也是个聪明人.聪明得出乎我意料之外。''小仙女道``你现在才知道,不嫌太迟了么?''

小鱼儿笑道:``但你也莫要神气,我总算还是骗过你一阵子,你发觉得才真的是太迟了,我若不是身旁有个累赘,早已不知走到哪里去了,还会等着被你追上!''小仙女居然没有动怒,冷笑道:``你既然那么聪明,此刻就该还能再想出个法子逃走\ldots。你若想不出,可见你的脑袋还是没有用,不如割下来也罢。''小鱼儿笑嘻嘻道"我何必再想什么法子,你以为我真的打不过你?我先前只不过是懒得和你动手罢了,常言道,好男不与女斗,我。\ldots 他话未说完,小仙女的手掌已到了他面前。这一掌招式倒也平常,但却奇快,简直快得不可思议,若非眼见,谁也想不到世上竟有人出手如此迅急。小鱼儿口中说话,眼睛虽一直盯着她,防备着她,但这一掌击来,他竟然还是躲不开。

他身子全力一拧.脸上还是被那春葱般的指尖刮着一些,脸上立刻多了叁道红印,火辣辣的发疼。

小仙女第二掌又跟着发出。

小鱼儿大嚷道:``住手,好男不跟女斗,住手!''他大叫大嚷,小仙女却似全未听见,她实在恨透这坏小子了,铁青着脸,瞬息间已击出了二叁十掌小鱼儿看来看去,也看不出她招式有什么奇妙之处,她一掌击来,小鱼儿明明觉得自己可以从容化解,但到她一掌真的击来时,小鱼儿却不知躲得多么狼狈,他连变了十几种身法,连掏心窝的本事都使了出来,但却竟然无法还手击出一掌他一招还未击出,小仙女的第二招已跟着攻来,他好容易再躲过这一掌,再想还手,小仙女第叁招又来了,他简直只有挨打的份儿。

铁心兰已忍不住抬起头来,眼睛也已瞧直了。

她根本瞧不清小仙女的身法、招式,她只瞧见一条红衣人影,那两只白生生的手掌,竟已化为一条白线。这条白线在红影中窜来窜去,又好像一条鞭子,小鱼儿就被这条鞭子打得到处乱跑,他跑到哪里鞭子就追到哪里,铁心兰委实也瞧不出这掌法有什么特别奇妙之处但却一辈子没有瞧见过这么快的掌法,小仙女的这双手像是附着什么妖魔精灵,否则怎会有如此快的出手,小鱼儿只觉她像生着十几只手似的.刚躲过这一只另一只已来了.他简直连气都不能喘。到后来小鱼儿眼前已全都是她那白生生的、兰花般的掌影,他连头都晕了突又放声大呼道:``住手,住手,你已中了我的毒,你''\ldots\ldots"他又想重施故伎,怎奈小仙女却全不听他这一套,铁心兰也急得变了颜色,但身子还是软软的,却又无法助他出手。

小鱼儿满头大汗,叫道``你还不相信?!你可知我这毒药有多厉害。''小仙女冷笑道``在我手下,天下可说绝无一人还能抽出手来施毒.何况是你这小鬼,你又想骗我?你简直是做梦''小鱼儿大叫道:``我不骗你,我。\ldots{}''

突然``吧''的一声,他脸上已着了一掌,身子竟被打得直飞了出去.远远落在一文外,在地上直滚。

铁心兰失声惊呼,道;``小鱼儿你\ldots\ldots 你\ldots\ldots{}''哪知小鱼儿不等她话说完,一个翻身又跳了起来,擦了擦从嘴角淌下来的鲜血,笑嘻嘻道``你放心她打不死我的,只要她打不死我,我总能打倒她。''小仙女冷笑道"好,我倒要看看你骨头有多硬。她话末说完,身子又冲了过去,又攻出七掌,她掌式既不奇诡,也不算狠辣,但却实在太快,快得令对方简直不能喘息,不能还手。

别人若不还手,又怎能胜她。

小鱼儿咬着牙,发下狠.无论如何,也得还她两拳。他看准小仙女掌法中有个破绽,拼命一招击出哪知等到他这一招击出时,小仙女手掌已将那破绽补上,他一招还只击出一半,肚子上已挨了一拳铁心兰惊呼道:``不好''呼声中小鱼儿又被打得飞了出去,满地乱滚。

铁心兰颤声道``算了吧,求求你\ldots\ldots 你打不过她,她实在太快了''哪知小鱼儿还是站起来。

他虽然疼得龇牙咧嘴,还是笑道:``就因为她太快,所以打不死我\ldots。出手太快,就不会太重,这道理你难道不明白。''小仙女面色也变了,她委实也未想到这小于竟然变得如此有种,居然还能站起来,她知道自已出手并不轻,若是换了别人,挨了这叁下,纵然不死,也丢了半条命,但这小于非但能站起来,竟反而也出手反击来了。

小仙女咬了咬嘴唇,道``好,算你骨头硬,我倒要瞧瞧你的骨头有多硬''她出手越来越快,小鱼几却越打越慢。

但是他躺下去,又爬起来,躺下去,又爬起来!\ldots 小鱼儿第七次爬起来,却又跌下去,他还是挣扎着要爬起。

小仙女瞧着他,脸上的表情很奇怪,也不知是愤怒?是痛恨?还是已有些可怜,有些不忍。

她口中只是冷冷道:``你只要服输,我就饶了你''小鱼儿道:``放屁谁要你饶我.''要你求我烧你\ldots\ldots 我要扒下你的衣裳,把你吊在树上,狠狠地抽你\ldots\ldots"他摇摇摆摆,才站直身子,小仙女已冲过去,飞起一脚,将他踢得连滚几滚。

铁心兰已闭起眼睛,不忍去瞧了,她的心已碎,肠己断,她自己也不知道为何对这可恨的冤家如此关心。

小鱼儿伏在地上,不住喘息,终于不能动了。小仙女胸膛已有些起伏,她喘息着道``小鬼小坏种小流氓你还能站起来么?你还能再打么?''小鱼儿双手抓着地上的草,身子慢慢向上爬,颤声道``你才是坏种流氓你\ldots。你还是强盗\ldots{}''小仙女大怒叫道``你还敢骂我''

她又冲上去,脚又将小鱼儿踢了几个滚。

铁心兰嘶声道:``你\ldots。你。''``你好狠,人家已躺在地上,你还要动手''小仙女根声道``谁叫这小鬼骂我!''

小鱼儿道``我骂你,我偏要骂你,你见财起意、你无恶不作、你杀人如草芥、你一\ldots 你是见鬼的小仙女,你简直是个母夜叉。''他声音己越来越弱,但还是骂不绝口。

小仙女气得身子发抖,脚踩在他胸膛上,道:"好,你骂,你骂\ldots\ldots 我叫你永远再也骂不出,我本不想杀你,这是你逼我的,她咬着牙,一掌方待击下,铁心兰失声惊呼,也挣扎着要爬过去,滚过去,哪知就在此刻─小鱼儿他也不知道是哪里来的力气,竟将小仙女纤巧窈窕的身子一抡,抡了想来,接着飞起脚踢在小仙女腰眼上小仙女再也想不到这垂死的人还能出手,脚一麻,身子被抡起,头一晕.腰上挨了一脚,接着就摔在地上。

小鱼儿也扑倒下去,压在她身上,两只手片刻不停,把可以摸得到的穴道,不管叁七二十一全点了。

铁心兰又惊又喜,颤声道;``鱼儿,你\ldots。你这是怎么回事?''小鱼儿喘息着笑道``我早就告诉过你,她打不死我的\ldots\ldots 我这身于是被药水泡大的,别人吃奶的时候我就己开始吃药..''莫说是她,就算是出手比她再重十倍的人,也休想将我打得真个爬不起来``铁心兰道''但你\ldots\ldots 你方才\ldots\ldots"

小鱼儿大笑道:``我方才只是故意装出来骗她的,好教她不防备,然后再故意骂她,让她生气,她气晕了头,我就笑歪了嘴。''铁心兰终于破涕为笑,但还是有些不放心,道:``你真的没事么?''小鱼儿站起来,笑道``我这一身钢筋铁骨,凭她那两只又白又嫩的小手能伤得了我?她拳头打在我身上,简直好像在弹棉花似的。''但这棉花却委实弹的不轻,他嘴虽说得硬,但身子一动,就到处发疼,全身骨头却像是被打散了。

他狠狠瞧着小仙女,道:``现在你还有什么话说?''小仙女闭着眼睛,眼泪已一连串流下来。

小鱼儿大笑道:``你哭也没有用的,我说过要还你几拳,就是要还你几拳,一拳也不会少\ldots 一''说着说着,他一拳打了出去他一连打了四拳,打得可真不轻。小仙女闭着眼,咬着牙,哼也不哼。

小鱼儿道:``你求我饶你,我就少打几拳。小仙女突然大叫道:''你这恶贼你打死我吧"小鱼儿一个耳光打过去,打得她住了嘴。

铁心兰忍不住道``你就饶了她吧。''

小鱼儿道``饶她,我为什么要饶她她方才为何不饶我,我说过要扒下她的衣服,将她吊在树上\ldots。''小仙女嘶声呼道``你敢你若真的,我\ldots。我死了也不饶你''小鱼儿笑嘻嘻道``你活着我尚不怕,何况死的。''他一把抓起小仙女的头发,将她整个人抓起来,正正反反,先打了她四个耳括子,笑道``这是本钱,先还你,还要再加利小仙女泪流满面道''你\ldots\ldots 你好狠\ldots\ldots{}``小鱼儿道:''我狠?你自己难道不狠?..\ldots 你只知别人对你出手狠,难道就忘了你对别人出手时,岂非还要比这狠得多。"他越说越气,一把就撕开了小仙女的衣服。

小仙女整个软玉般的肩头都露了出来,她嘶声大骂道:``你这恶狗,恶魔。\ldots{}''她简直将心里想得出的什么话都骂了出来。

小鱼儿笑嘻嘻地听着,摇头道``你若骂得好,我听听也没关系,还觉有趣,但你实在不会骂人,骂人的技术你一点也不懂我只有请你住嘴了。''他竟从地上抓把烂泥,要往小仙女嘴里塞。

小仙亥现在真的怕了,终于痛哭着道:``求求你\ldots\ldots 饶了我吧''\ldots\ldots 饶了我吧``\ldots\ldots{}''小鱼儿大笑道``好,你终于求我饶你了,你莫要忘记。''小仙女哭得肠子都断了,她毕竟是个女孩子.她毕竟年纪还小,她第一次尝到被人欺负的滋昧。

小鱼儿大笑着将她摔在地上,道:``好,我饶了你.''他再也不瞧小仙女一眼.转过身子,扶起铁心兰,撮口而哨,叫道``小白菜\ldots\ldots 小白菜\ldots\ldots{}''那匹小白马竟真和他有缘竟真的跑了回来。

小鱼儿笑道:``白菜兄这次辛苦了你,背我们两人一程吧,到了前面.我一定好好请你吃一顿还得喝两杯。''他扶着铁心兰上了马,自己也上了马这匹马虽然小,气力却不小,轻嘶一声,轻快地向前就跑。

小鱼儿大笑道``小仙女,再见了\ldots 嗯,还是莫要再见的好。''他竟然就这样扬长而去,留下动也不能动的小仙女,躺在地上,小仙女的哭声,他像是完全没有听到。

两人挤在马背上,靠得紧紧的,铁心兰只觉身子又轻又软,像是靠在云堆里,既不愿动,也不愿说话。小仙女的哭声.终于听不见了。铁心兰终于轻叹一声,道:``你真是张菁的克星。''小鱼儿笑道"``她遇见我,算她倒霉。''

铁心兰默然半晌,悠悠道:``我真没想到,你真的打起来时,竟那么狠,那么不怕死\ldots\ldots{}''小鱼儿大笑道``我也许是个坏蛋,但却绝不是懦种,别人想要我干什么都容易,但谁也休想叫我求饶''铁心兰媚然一笑,柔声道``不错,你就算坏,但也坏得是个男子汉''星光月色都很亮,银子般的月光,将他们的影子照在地上,他们两人的影子,几乎已变成了一个。

又过了半晌,铁心兰突然道;``你可知道小仙女张菁为什么要抢我那张藏宝图?''小鱼儿道``还不是见财起意。铁心兰道''那你就错了,她手段虽然毒辣,却不是个坏人\ldots\ldots{}

小鱼儿笑道``她难道是个好人?\ldots\ldots 好人要杀你,坏人却救了你,这岂非怪事''铁心兰道:``我跟你说正经的,她要抢我的藏珍图只因为她母亲和这批藏珍的主人有很密切的关系。''小鱼儿道"哦!\ldots 她已经这么凶了她母亲岂非更是个母夜叉\ldots\ldots{}

铁心兰笑道:``她母亲非但不是个母夜叉,还是昔日江湖中一位大大有名的美人,只要看见过她的男人,没有一个不被她迷得要死要活的。''小鱼儿笑道:``这样的人,我倒愿瞧瞧。''

铁心兰咬着嘴唇,道``只可借你迟生了几年,她现在已经老了,但江湖中老一辈的人听到玉娘子张叁娘这名宇,心还会直跳。''小鱼儿笑道:``你为什么不说只可惜她早生几年,见不着我..那么,小仙女的父亲又是个何许人物?''铁心兰道``这''。这我却不清楚。"

小鱼儿大笑道``不错,有名美人的子女,的确有许多是找不到父亲的,只因为可能是她父亲的人太多了。''铁心兰``噗嗤''一笑,道:``你少缺德,那''玉娘子虽然美得如玉,但也冷得象玉,江湖中追求她的人也不知有多少,但她瞧得上的却只有一个``小鱼儿眨了眨眼睛,道,谁有如此艳福?''

铁心兰道:``就是那藏宝的主人,名叫燕南天:''小鱼儿身子微微一震,失声道``燕南天?!''

铁心兰道``你也听过这名字?''

小鱼儿道:``我\ldots\ldots 我好像所见过,却已记不清了。''铁心兰道``你若听见过这名字,就不该忘记,他本是昔日江湖中最最有名的剑客,他的剑法,至今还没有一个人能比得上。''小鱼儿道``哦''

铁心兰悠悠道``他生得虽不英俊,但却是江湖中最有男子气概的男予汉,只可惜我也迟生了几年,见不着他。''小鱼儿笑道``你可要我帮你找他?''

铁心兰叹道``你已找不着他,任何人都找不着他,江湖传言,十几年前,他不知为了什么闯人恶人谷,从此就没有再出来,他虽然剑法无故,但遇着那许多恶人,只怕\ldots{}''还是难逃毒手,小鱼儿默然半晌,道:``噢\ldots\ldots{}''铁心兰道``这藏珍图,据说就是他入谷之前留下的,他似乎也自知入谷之后必死,所以便将他生前搜集的古玩珍宝,以及他无敌天下的剑谱,全都藏在一个隐秘之处,若没有这藏珍图谁也找不到。''小鱼儿缓缓点头道``珍宝虽不足令人动心,但这剑谱却的确令人眼红,谁得了这剑谱,谁就可无敌于天下,那就难怪有这许多人要来抢了。''铁心兰道``但小仙女却非为这剑谱,而是为了要安慰她的母亲''\ldots\ldots{}``她方待回头,但眼光溜过地上,整个身子突然一震,失声道:''你\ldots。你瞧,这\ldots 一这是\ldots.``小鱼儿笑道:''我早就瞧见了,地上的影子,已多了一个。"地上的影子,竟猛然真的多了一个,多出来的影子,就站在小鱼儿身后的马屁股上。

但马还是照样往前跑像是全无知觉。小鱼儿虽沉得住气,铁心兰却慌了抱着小鱼儿的手,拼命一勒马.邢匹马长嘶而起,铁心兰却跃下马去只听一人冷冷道``你怕什么,我若要取你们性命,早巳出手小鱼儿笑道''我若害怕.早已跳下马了。``那声音咯咯笑道''不错,你这人很有意思,我早就瞧出你很有意思,想交交你这朋友所以才跟着来的。"这语声又尖又亮,说话人的嗓子,就像是金铁铸成的,这语声虽然冰冰冷冷,但却又似带着稚气。

铁心兰惊惶爬起,抬眼望去,只见一个身材瘦小的黑衣人,轻飘飘站在马股上.活像是粘在上面的纸人。他不但全身被一件闪闪发光的紧身衣服紧紧裹住,一张脸也蒙着漆黑的面具,只剩下一双黑白分明的眸子,黑的地方如漆,白的地方如雪,这双眼睛在夜色中眨一眨的,也说不出有多么诡异可怖。

铁心兰耸然动容.失声道``你莫非就是黑蜘蛛''那黑衣人怪笑道``不错,你居然认得我。''

铁心兰道``你\ldots\ldots 你怎会来到这里?''

黑蜘蛛道``我本也是为你来的,但瞧见这小伙子,觉得很有趣.可真比那藏珍图有趣多了,我想交这朋友,只好放弃那藏珍图。''小鱼儿大笑道:``想不到居然会有人将我瞧得比这藏珍图还重,这种朋友我也要交的\ldots{}''只是,黑蜘蛛,这又算什么名字?``黑蜘蛛冷冷道:''你连黑蜘蛛这名字都未听过,简直是孤陋寡闻当今天下不知我的名宇的,还能在江湖中混么?``小鱼儿道:你什么时候跟上我们的?''

黑蜘蛛道``你将白马涂成花马时,我就瞧见了。''小鱼儿道``奇怪,我竟不知道。''

黑蜘蛛冷笑道``我若存心要跟上一个人,就算跟上一辈子,那人也不会知道。我若不愿被人瞧见,当今天下,又有谁能够瞧见我的影子。''小鱼儿纵身下马来,瞧着他摇来摇去的身子,笑道``你年纪虽小,口气可真不小。''黑蜘蛛怒道:``谁说我年纪小''

小鱼儿道:``我听你说话,难道还听不出?''

黑蜘蛛眨着眼睛瞧了他半晌,格格笑道``我年纪纵然小,也大得可以做铁心兰的叔叔伯伯了,只是我既想交你这朋友,也不愿以老卖老,你就叫我大哥吧!''

\hypertarget{ux7b2cux5341ux4e03ux7ae0-ux78a7ux86c7ux795eux541b}{%
\chapter{第十七章
碧蛇神君}\label{ux7b2cux5341ux4e03ux7ae0-ux78a7ux86c7ux795eux541b}}

小鱼儿笑道:``大哥?\ldots\ldots 你个子比我还小,该叫我大哥才对''黑蜘蛛眼睛一瞪,怒道``江湖中人求我要叫我一声大哥的人也不知有多少,但却被我一个个踢回去了,我要你叫我,你还不愿意。''铁心兰已站了起来,不住向小鱼儿使眼色。

小鱼儿却似没有瞧见,还是笑道``很好!\ldots 黑老弟,你的本事不小\ldots{}''黑蜘蛛怒道:``你叫我什么?''

小鱼儿道:``黑老弟,咱们喝两杯去如何?''

黑蜘蛛格格笑道``你可知你现在已将大祸临头除了我外,没有人能帮你,你着叫我一声大哥,不知有多少好处。''铁心兰已急得要跺脚,直恨不得捏小鱼儿的脖子,要他叫``大哥'',但小鱼儿却还是笑嘻嘻道``黑老弟,我有什么大祸临头,你且说来听听。''黑蜘蛛瞪着眼睛瞧住他,瞧了半晌,突然冷笑道:我本来想帮你个忙的,但你既然要在我面前充老大,我也就犯不着再管你的事了。"说话间,手突然一扬,月光下只见他袖管中仿佛有条闪闪发光的银丝,笔直飞了出去.小鱼儿还想仔细瞧瞧这是什么哪知他眼睛才眨了眨。黑蜘蛛的手一抖,人已跟着飞了出去,就像是箭一般接着,他人就不见了,那银丝也不见了。

小鱼儿也不禁怔了征,叹道:``难怪他口气这么大,轻功果然有两下子。''铁心兰叹道``岂只有两下子,他这手独门轻功,神蛛凌空银丝渡虚.在江湖中简直没有第二个人能比得上。''小鱼儿道``这种功夫有什么巧妙?''

铁心兰道``他袖中所藏的,据说真是南海千年神蛾所结的丝,又坚又韧,刀剑难伤,他将这蛛丝藏在一个特制的机簧筒中,手一扬,蛛丝就飞了出去,最远据说要达一二十丈,而蛛丝顶端的银针,无论钉住什么东西,他人立刻就能跟着到哪里,当真要说是来去飘忽,快如鬼魅。''小鱼儿笑道:``这小子非但人古怪得有趣,所练的功夫也古怪得有趣,却不知他年纪究竞是大是小?为什么如此喜欢充老。''铁心兰道:``江湖中没有一个人瞧见过他的脸,更没有人知道他年纪,人知他最恨别人说他小谁要犯了他这毛病,马上就要倒霉。''小鱼儿道;``我怎么还没有倒霉?''

铁心兰展颜笑道``这倒是怪事,他倒真像是和你有缘,否则,就凭你叫他那几声老弟,他人怕已经要割下你的舌头了。''笑着笑着,突又长长叹息了一声,皱眉道:``但这人从来不说假话,他说咱们立刻就将有大祸临头,只怕\ldots。只怕也不会是说假。''小鱼儿笑道``哪有什么大祸临头?你别听他鬼话。''他语声越说越小,说到最后一字,已几乎听不出了,他的眼睛,也已紧紧盯在马屁股上,不知瞧见什么。

铁心兰发觉,刚想去瞧。

但小鱼儿却拖着她上了马道``明们快走吧?''

铁心兰道``你\ldots\ldots 你瞧见什么?''

小鱼儿道``没有什么\ldots\ldots 哈哈哪有什么?''

铁心兰垂下了头,默然半晌幽幽道我知道你一打哈哈,说的就不是真话。``小鱼儿征了征,大笑道;''不想我这毛病竟被你瞧出来了..``我这毛病是从小被一个人传染的,竟一直到现在还改不过来。''铁心兰自然不知道传染这毛病给他的就是从来不说真话的``哈哈儿'',她也不想问,只是急着道:``那么,你究竟瞧见了什么?''小鱼儿道``也没什么了不起的东西,你不瞧也罢。''铁心兰笑道;``我知道你不让我瞧,是怕我着急,但我若不瞧,就会更着急\ldots\ldots{}''小鱼儿苦笑摇头道``唉女人\ldots\ldots 女人,你要瞧,就瞧瞧吧。''马股上,不知何时,竟被人印上一条绿色的小蛇。

这条小蛇是以碧磷印上去的,在月光下闪着丑恶的绿光,光芒闪动,这条蛇也像是在蠕动,那铲形的蛇头,更像是随时都会跳出来噬人。小鱼儿虽然明知它不是活的,但不知怎的,却越瞧越觉得恶心全身上下,像是都起了鸡皮疙瘩。

铁心兰更早巳面色大变,道``蛇\ldots{}''碧磷蛇\ldots\ldots 青海之灵,食鹿神君``小鱼儿眨着眼睛,笑道''你说什么?``铁心兰苍白着脸颤声道:''你不懂的\ldots\ldots 不懂的\ldots{}``小鱼儿道''一条小蛇就算是真的,也没什么可怕?``铁心兰道;''真的不可怕这假的才可怕"

小鱼儿失笑道``不怕真的怕假的为什么?''

铁心兰深深吸了曰气,道``这碧磷蛇就是那青海之灵食鹿神君的标志,标志所在,他人就不远了,他人既不远,祸事就真的要来了。''小鱼儿皱眉道:``这食鹿神君又是什么玩意儿?''铁心兰道:``你可听过''十二星相这名字?"

小鱼儿目光闪动,道;``好橡听过,又好像没有。''铁心兰叹道:``这十二星相乃是近二十年,江湖中最残酷、最狠毒的一批强盗,他们平日极少下手,但若瞧见值得下手的东西,被他们瞧中的人便再也休想跑得了,叁十年来,据说十二星相只有一次失手''小鱼儿道``这条蛇自然就是十二星相中的人。''铁心兰道``不错,这食鹿神君正是十二星相中最阴毒、最狡猾的一人,他的老窝就在青海\ldots\ldots 唉我本该早巳想到他要向我下手的。''小鱼儿道``为什么你早就该想到?''

铁心兰道``十二星相唯一失手的一次,据说就是栽在燕南天手上,他们若知道燕南天有藏剑谱留下,又怎肯放过''小鱼儿眨着眼睛笑道``不想你年纪虽小,知道的事却不少。''铁心兰幽幽道``我很小的时候,就出来闯荡江湖,知道的江湖秘闻,自然比别人多些,你将来在江湖走动,便会知道的''小鱼儿笑道:知道的越多,就害怕的越多,倒不如索性什么都不知道,无论遇着什么人,都可以不管叁七二十一先和他拼了再说。``铁心兰笑道''但我们现任既然知道了,又该怎么办呢``小鱼儿道''咱们此刻既拼不过他,自然唯有走。``铁心兰喃喃道:''走?\ldots\ldots 能走得了么?\ldots。``两人一骑,策马狂奔,两人惧是满头大汗,都已将面具取了下来,小鱼儿轻轻道''小白菜,辛苦你了,抱歉抱歉!``。''只见前面有个小小的山村,此刻虽然只不过曙色初露,但这山村的屋顶上,却已是袅袅起了炊姻。

青灰色的炊烟,在乳白色的苍穹下袅娜四散,就像是一幅绝美的图画。但任何丹青妙手也休想描绘得出。

这里已迫近青海、四川的边境,汉人已多。

只见一个身穿青布短褂的老汉,站在一家门曰.嘴里刁着管旱烟,瞧着天色,喃喃道``看来今天又是个好天气,该把棉被拿出来晒晒了。''小鱼儿翻身下马,走过去唱了个喏,笑道``老丈可有什么吃喝的赏给我兄妹一些。''那老者上下瞧了他几眼,又瞧了瞧马上的铁心兰,呵呵笑道:``小官人说话真客气,只要不嫌老汉家里茶饭粗陋,就快请进来。''一面说着话,面已含笑揖客。

小鱼儿笑着谢过,扶铁心兰下马,悄声道:``不想这里的乡下人倒好客得很。''铁心兰笑道``瞧见你这么可爱的孩子,话又说得这么甜,无论你要什么,只怕没人能狠得下心拒绝你。''说到这里,脸突然一红,垂下了头。

小鱼儿瞧着她嫣红的脸。笑道:``只怕别人是瞧在你这病美人的面子,他虽是个老头子。但却没有瞎眼。''铁心兰嫣然一笑,扶着他的肩走了进去。

只见那老汉已擦干净了桌子,摆上了四副碗筷,笑道:``两位稍坐,老汉去瞧瞧老婆子饭可煮好了没有。''他人走进去,饭香就一阵阵传了出来,小鱼儿肚子叽哩咕噜直叫。眼睛睁得大大的瞪着厨房的门,厨房里碗勺叮当直响。

一个白发苍苍的老婆子终于走了出来,一手棒着一大碗热气腾腾的糙米饭,上面还摆着一块威肉几条咸菜。

她蹒跚着将饭送到桌上,弯腰笑道:``两位小客人先用吧,莫客气,饭凉了就不好吃了。''小鱼儿笑道``既是如此,我兄妹就不客气。''

他还没等到这老婆子定出门,巳拿起了碗筷.就要往嘴里扒饭.突听``当''的一声,铁心兰刚端起了碗,立刻又松下了手,笑道"真烫\ldots\ldots{}

小鱼儿目光一闪,突然出手如风,用筷子在铁心兰手上一敲,铁心兰筷子落地瞪大了眼睛道``你这是干什么?''小鱼儿也不说话,却将那碗饭倒在桌上,又干又硬的糙米饭撒了一桌子,却有条小小的青蛇从饭粒中蠕动着钻了出来。

铁心兰失声惊呼,道:``蛇\ldots\ldots 十二星相''

小鱼儿已飞身冲进了厨房.铁心兰跟着冲进去,只见方才那老汉抑天倒在地上,一张脸已变成黑的还有个老婆子倒在灶旁,脸也是又黑又青,但头发却也是黑的,看得出不是方才送饭进去的那老婆子。

那白发苍苍的老婆子已不见了!

铁心兰颤声道:"好狠\ldots\ldots 好毒,。唉,好险\ldots\ldots{}

小鱼儿咬着牙恨声道``这些人看来竟比我还坏十倍,竟连这老人家都不肯放过。''铁心兰道``我\ldots\ldots 我早就知道咱们跑不了的''

小鱼儿取出块金子,抛在地上,又用块焦柴,在墙上写了十个大宇``厚殓两人,否则必追你''突听门外马嘶,小鱼儿立刻冲出去,一条小蛇已沿着马腿在往上爬,小鱼儿撕下条衣襟,将蛇掸在地上,踩得稀烂,摸着马鬃道``小白菜,莫要怕,这些恶人害不死你的也休想害得死我。''拉着铁心兰上马,打马飞奔而去。

那白马似也知道凶险,跑得更是卖力,眨眼间便穿过那小小的村庄,铁心兰身子还在发抖,不住喃喃道``好险!\ldots。好险,咱们只要吃进一粒饭就活不到现在了。''小鱼儿大笑道,"但咱们现在还是好好的活着!

铁心兰道``你\ldots。你是怎么发觉的''

小鱼儿道``你端起饭碗,还烫得不能留手,那老婆子却安安稳稳从厨房里一路捧出来,这双手没有练过毒砂掌一类的功夫才怪。''铁心兰叹道;``真是什么事都逃不过你这双眼睛。''突见前面路上,一块绿草如茵,仔细一瞧,这块草竟不住蠕动赫然是百余条青色的小蛇。铁心兰失声惊呼,小鱼儿已调转马头,往旁边一条岔路冲了过去,这条路虽然窄小,但两旁竟有林荫夹道,小鱼儿一路上从未见过如此干净幽美的道路,心里方自有些惊疑,突然一条蛇自树上倒挂下来这条蛇虽仍是碧绿色,但却不小.绿油油的蛇身,粗细儿臂,赫然正挂在铁心兰的眼前。

白马惊呼人立,铁心兰吓得魂都飞了。

小鱼儿喝道:``莫慌,捉蛇打狗的本事我最在行''喝声中出手如电,捏住蛇的七寸,往树上摔了过去,这一抓一摔果然是迅急美妙,蛇果然已被摔晕。

铁心兰这才松式口气,道``幸好你不是女人,女人可都是怕蛇的。''小鱼儿道``你那柄匕首拿来。''

铁心兰递过匕道,道"小心些杀,莫要被蛇血溅在身上\ldots\ldots{}

小鱼儿道``哼!\ldots{}''

只见他铁青着脸,突然一刀往自己手臂上割下!

铁心兰吃惊道:``你\ldots\ldots 你这是\ldots\ldots{}''

一句话未说完,已像是被人扼住咽喉,再也说不出一个字,甚至连呼吸都已呼吸不出。

自小鱼儿臂上刀口流出来的血,竟是黑的。

小鱼儿脸色惨白,嘶声道:``我终于还是上当了''缓缓摊开手掌,掌心凝结着几滴血珠,竟是黑的冉瞧那条蛇虽已晕死但蛇身却仍笔直,七寸处隐隐竟似有光芒阔动,铁心兰变色道``原\ldots\ldots 原来这条蛇早巳死了,那恶魔竟在蛇身里藏着一柄软剑,剑上有剧毒,你一捏蛇身,里面的剑锋就割伤了你''

小鱼儿慢笑道:你真聪明,真是天才儿童。"

铁心兰道``幸\ldots\ldots 幸好你\ldots\ldots 你发觉得早,已将毒血放出.只怕已没,没事了吧。''小鱼儿道,``没汉事了''。``半个时辰后,什么都没了!''铁心兰身子一震,从马上跌了下去,颤声道``你\ldots\ldots 你胡说''小鱼儿道``这毒是没有救的,我若不放血此刻已要去见那老头子了纵然放了务,也拖不过半个时辰''铁心兰扑到他身上,泪流满面,道``这毒有救的,你根本不知道\ldots。''小鱼儿大笑道``我从小就在使毒的大名家群中打滚我若不知道,天下还有谁知道''他居然还像是得意得很,居然还笑得出来,铁心兰叫道,``既然如此称就该能配解药。''小鱼儿道``我自然能配解药。''

铁心兰大喜道:你\ldots\ldots 你原来又在吓我!"

小鱼儿缓缓道``但这解药却要叁个月才配得好!''铁心兰笑容还未绽开,又已软软地跌倒,流泪道``你现在还有心情开玩笑,你\ldots\ldots 你\ldots\ldots 你叫我怎么办呢?''流泪变为抽泣,抽泣变为痛哭,痛哭捶地道:``你简直不是人你竟对自己的生命都要开玩笑,却不管别人心里如何,我恨死你\ldots\ldots 恨死你了''小鱼儿也不理她,部从怀里掏出了张发黄的羊皮纸,拿在手里挥来挥去,口中大声呼道``小臭蛇,你瞧见么,这就是那藏珍图,你想不想要?''他喊了两遍,树梢果然传下来一声又尖又细,又滑又腻教人听得全身都要起鸡皮疙瘩的冷笑。

一人冷笑着道``这迟早是我的,我并不着急。''只见这人穿着条碧绿的紧身衣,藏在树叶中,当真教人难以发觉,他又长又瘦的身子,弯弯曲曲地藏在枝桠间,全身都是没有骨头,那双又细又小的眼睛瞪着小鱼儿,活脱脱的就像是条蛇,毒蛇!

铁心兰抬头瞧了一眼,全身都不觉发麻,就像是有条冰凉的蛇钻进了她衣服,沿着她背脊在爬。

小鱼儿却大笑道``这真已迟早是你的么?''

那碧蛇神君阴恻恻地笑道:``你若乘早双手奉上,本座只怕还会救你的命。''小鱼儿大笑道:``是,是,我很相信\ldots\ldots{}''

铁心兰嘶声道``你就给他吧,反正。\ldots,反正咱们已用不着碧蛇神君道''还是这女子聪明\ldots\ldots{}

小鱼儿哈哈笑道:``是,是,她聪明,我却很笨''突然将那张羊皮纸塞入大笑的嘴里,大嚼起来。

碧蛇神君身在树上一滑一闪,便``嗖''的窜了下来,从马上一把抓住小鱼儿,厉声怒喝道``吐出来''小鱼儿也不招架闪避,任凭他拖下马,却乘机将那图纸吞了下去,张开嘴笑道:``吐不出来了。''碧蛇神君怒喝道:``你这是找死!''

小鱼儿嘻嘻笑道:``这藏珍图世上只有一张,也只有我一人,将它看熟了,你让我死,一辈子都休想瞧那藏珍图一眼。''碧蛇神君征了怔,手掌不由得渐渐放松。

小鱼儿悠悠道:``我若是你,此刻就该将解药拿出来了,只要我活着,说不定还会将那藏珍图画出来,死人的手是不会动的。''碧蛇神君狠狠瞧着他,一张几乎已只有皮包着骨头的脸上,突然泛起了残酷的狞笑,狞笑着道:``你只当本座真的要被你这小鬼要挟住了么?''小鱼儿仰起了头,笑嘻嘻道:假的么?"

碧蛇神君一字字道``那羊皮纸又轻又韧,你纵然吞下去也还是好好地在你肚子里,本座只要剖开你的肚子,还怕拿不到?''小鱼儿脸上虽在笑着,心里却不禁透出一股寒意。

铁心兰嘶声大呼道``你不能这么做\ldots\ldots 你不能卜\ldots/碧蛇神君咯咯笑道''谁说不能?你瞧着吧"他手一抖,已自腰畔拔出碧光闪闪的软剑,迎风抖得笔直。

小鱼儿虽然智计百出,此刻却也想不出法子,铁心兰拼命扑过去,怎奈大病未愈碧蛇神君反手一掌就将她打得滚倒在地,狞笑道:``捉蛇打狗你最在行开膛剖腹却是我最在行的,但你只管放心,我这一剑刺下绝不会要你的命。''小鱼儿虽已满头大汗,却仍笑道``多谢多谢!。''碧蛇神君道:``我就算将你肚子刻开,将那羊皮纸拿了出来.你还未必死的''``。我要叫你慢慢地死''小鱼儿笑道``但你动手时却要小心些,我今天早上吃了条蛇祖宗在肚子里,还未消化,你切莫不小心伤了你的祖宗.''碧蛇神君怒道今小鬼临死还耍贫嘴"

他一剑刺下,突听``当''的一声,掌中剑竟被震开!

原来小鱼儿已悄悄将那条``死蛇''拿在乎里,用死蛇身子里的剑,挡了他一剑,接着又是一剑刺出碧蛇神君轻轻一闪,狞笑道``你妄动力气,毒性发作更快,死得更早。''口中说话,掌中剑连续击出,小鱼儿挡了四剑,手臂发软,竟再也举不起来铁心兰已晕了过去,小鱼儿心也凉了。

碧蛇神君嘶声笑道:``小鬼你还有什么花样?''他掌中剑抵住了小鱼儿的胸膛,一分分往下刺。

小鱼儿胸膛已见血,放声狂笑道:``剖肚子乃人生一大快事也,不想我江鱼竟在无意中得之!\ldots 笑声未了,突听''当,当,当"叁声,碧蛇神君右掌中剑不知怎地,竟突然断成四段,段段落在地上!

碧蛇神君凌空翻身,紧紧贴在树上小眼睛四下乱闪,嘶声道:什么人``一个甜美的女子声音道''我是什么人,你会不知道?"这语声竟赫然又像是小仙女的声音。小鱼儿绝处逢生,方才欢喜,听见这语声,又如一捅冷水当头淋下,落在小仙女手里可未必比落在碧蛇神君手里好多少。

碧蛇神君面色煞时苍白,道:``你\ldots 姑娘你\ldots{}''那语声缓缓道:``你纵不知道我是谁,总该知道这条路是通向什么地方的,你有多大的胆子,竟敢在这里撤野!''小鱼儿本已垂头丧气,此刻又几乎拍起掌来!

\hypertarget{ux7b2cux5341ux516bux7ae0-ux6155ux5bb9ux4e5dux59b9}{%
\chapter{第十八章
慕容九妹}\label{ux7b2cux5341ux516bux7ae0-ux6155ux5bb9ux4e5dux59b9}}

这不是小仙女.她的语声,听来虽和小仙女也有七分相似但小仙女说话不会这么慢的,小鱼儿从未听过小仙女慢慢的说过一句话。

只见一条绿衣少女,手挽花篮,肩着花锄,款款自树后走出,她的体态是那么轻盈,像是一阵风就能将她吹倒,她的柳眉轻轻大大的眼睛充满了忧郁,容貌虽非绝美,但却楚楚动人,我见犹怜。

她身后还跟着个浓眉大眼的少年,个子虽然又高又大,却是满面稚气,毕恭毕敬地跟在她身后,连头都不敢抬起。这男女两人一个就像是弱不禁风的闺阁千金,一个又像是循规蹈矩,一步路也不敢走错的世家少年。

但碧蛇神君瞧见这两人,却像是被人在脖子上砍了一刀,头立刻垂了下去,强笑着道:原来是九姑娘。``缘衣少女淡淡道:''很好,你还未忘记我,但你莫非忘了这是什么地方,居然要在这里开膛剖腹,你的胆子也未免太大了吧\ldots\ldots{}

她神色并非冷酷,只是一种淡淡的轻蔑与冷漠,她并非要对别人不好,只是对任何人都不关心。世上无论多重要的人物,在她眼中似乎都不值得一顾。

小鱼儿实在猜不出这少女身份,她看来本该是皇族贵胄千金公主,却又偏偏只不过是个草野女子,她年纪轻轻,本该对世上一切都抱着美丽的幻想与希望,但她却偏偏似乎已看破一切,所以对任何事都这么冷淡。

只见碧蛇神君头垂得更低,颤声道:``小人以为这里还未到禁区,所以\ldots\ldots{}''绿衣少女道``现在你知道了么?''

碧蛇神君道``现在知道了。''

绿衣少女道``既已知道,你总该知道怎么办吧。''碧蛇神君惨笑道``是,小人知道。''

突见剑光一闪,他竟将自己的左手齐腕斩断就连小鱼儿都不禁为之动容,但这绿衣少女``九姑娘''却仍是那么淡漠,只是轻轻挥了挥,道:``好,你现在可以走了。''话未说完,碧蛇神君竟飞也似的逃走。

突听铁心兰放声大呼道``你不能放他走\ldots\ldots 不能放他走。''她不知何时已醒来,此刻挣扎着要站起,却又跌倒。

绿衣少女瞧了她一眼,道:``为什么?''

铁心兰指着小鱼儿,道:``他已中了剧毒,只有碧蛇神君的解药,否则他\ldots\ldots 他\ldots\ldots 他只怕活不过今天了''绿衣少女淡谈道;``他的死活,与我又有何干?''铁心兰身子一震,又扑倒在地.那少年突然笑道``九姐,咱们救救他吧。''缘衣少女道``你若要救他们,你只管救,我不管。''转过身子款步而去,再也不回头瞧任何人一眼。

那少中瞧了瞧躺在地上的铁心兰,垂头道:``对不起\ldots。''突也大步赶了上去,跟着她走了。

铁心兰颤声呼道:``姑娘\ldots\ldots 求求你\ldots\ldots 你\ldots。''小鱼儿大眼睛转来转去,突然大笑道;``咱们也走吧,何必求她。''铁心兰道``但你。你\ldots\ldots{}''

小鱼儿大声道;``我死就死,活就活,有什么关系?她小小年纪.又怎能救得了咱们你逼她相救.岂非令她为难。''他用力挟起铁心兰才走了两步。突听那少女冷冷道``站住''小鱼儿嘴角泛起一丝微笑,但口中却大声道:``为何要我站住,我若死在这里,岂非玷污这条干净的道路。''他头也不回,还是往前走。

人影一闪,绿衣少女已挡住了他的去路,冷冷道``你已死不了啦\ldots─但你莫以为我不知道你这是在激我,要我救你,只是为了要你知道世上没有慕容姐妹办不到的事。''小鱼儿冷笑道``我可没有激你,也并未要你救我,我自己高兴死就死,高兴活就活,用不着别人操心。''九姑娘淡淡道:``我既已要救你,现在你想死都已不能死小鱼儿眨了眨眼睛,道;''这可是你自已心甘情愿要做的,我既未求你你纵然救活了我,我也不会感激你的。``九始娘不答话,转过身子,道;''随我来。"

道路尽头,竟是座庄院。

这庄院依山而建,占地并不广,气派也不大,但每一片瓦,每间房子,都建筑得小巧玲珑别具匠心,看来别有一番风味。走进去便是个小小的院子,小小的厅房,虽然瞧不见一个仆役,但每寸地方都打扫得干干净净,一尘不染。小鱼儿走到这里,已不住的喘气,似将跌倒,那少年悄悄出手,在后面扶着他,小鱼儿感激的一笑道:``谢谢你,你叫什么名字?''那少年脸红了红道``顾人玉。''

小鱼儿道``你不姓幕容?''顾人玉红着脸道:``我是她们的表弟。''小鱼儿笑道``你这人例真不错,只是太老实了些,倒像是个女孩子,怎地还没说话,脸就先红了起来。''顾人玉吃吃道``我,我\ldots\ldots 我\ldots\ldots{}''

他若非生得又高又大,浓眉大眼绝不会是个男子,小鱼儿真要以为他又是个女扮男装的。

九姑娘脚步不停,穿过厅房,穿过回廊,诺大的庭院,到处都不闻人声,更瞧不见一个人影。

最后,她走到小园中两叁间雅轩门前,方自战住了脚,道:``进去。''说完了这句话,竟又转身走了。

顾人玉道:``请\ldots 请进,这就是我住的屋子。''铁心兰竟也笑了笑,接道:``这里恐怕只有这间屋子是男人能住住。''小鱼儿笑道``哦\ldots\ldots 这里除了你,莫非全是女子?''顾人玉瞪大了眼睛,道:``你难道没有听过慕容九姐妹的名字.''铁心兰本己连眼睛都己图起,此刻突失声道``莫非就是江湖人称的人间九秀?''她一说话,顾人玉脸又红了,轻声道:``不\ldots 不错。''小鱼儿瞧着铁心兰笑道``原来你又知道,你且说说这九姐妹又有什么厉害?''铁心兰轻轻叹了口气,道``这九姐妹不但轻功、暗器可称天下一绝,而且每个人都是秀外慧中,只要是别人会的事,她们姐妹就没有不会的,所以天下的名门世家,没有一家不想娶个幕容家的女儿回去做媳妇。''小鱼儿眨了眨眼睛,笑道``她们嫁了么?''

铁心兰道``据说除了最小的九妹外,另外八姐妹嫁的不是武林世家的公子,就是声名显赫的少年英雄\ldots\ldots{}''小鱼儿大笑道``这就难怪江湖中人要怕她们,别人纵然惹得起她们九姐妹却也惹不起她们这八个有本事的丈夫。''他此刻脸上已泛起黑气,说话时一口气也常常提不上来但他居然还是旁若无人,大声谈笑,竟又一拍顾人玉肩头,笑道:``常言说得好,近水楼台先得月,你只管紧紧盯住她吧,这主意一点也不错,哈哈,一点也不错!''顾人玉脸更红得像火,垂下了头,偷偷瞧了铁心兰一眼,道:``这\ldots\ldots 这是家母的意思,小弟我''\ldots 哪知慕容九妓娘突然走了进来,冷笑道:``这本是舅妈的意思,你本不愿来这里受气的,是么?''顾人玉简直恨不得找个地缝钻下去,吃吃道``我\ldots 我不是这意思。''慕容九妹冷冷道``顾少爷,这里可没有人请你来,也没有人留着你,舅母虽当你是宝贝,别人可不稀罕你。''她再也不瞧顾人玉一眼,``当''的,将一个小小的黑色玉瓶,抛在小鱼儿面前的桌子上,冷冷道``一半内服,一半外敷,叁个时辰内,你这条命就算捡回来了,就快走吧。''转过身子,就往外小鱼儿嘻嘻一笑,道:``我可没有求你救我,也没有要娶你做媳妇,你用不着对我这么神气,别人虽当你是宝贝,我可不稀罕''慕容九妹霍然回身,冷冷的瞪着他。

小鱼儿却若无其事,拔开瓶塞,``咕''的一声,将半瓶药咽了下去,舐了舐嘴唇,啧啧道:``这药怎地酸得像醋。''接着又把另半瓶药敷在伤口─一他究竞是聪明人,嘴里虽说着风凉话,手里却赶紧将药先用了再说。

慕容九妹狠狠瞪着他,冷漠的目光中,突然像是要冒出火来,她眨也不眨瞪了半晌,一字字道``我虽然救了你,一样还是可以杀你''小鱼儿吐了吐舌头,笑道:``你不会的,你看来虽狠,心却还是不错。''也不知怎地,慕容九妹苍白的面颊竟红了红.但瞬间厉声喝道:``出去,现在就出去,永远莫要被我再瞧见,否则我.\ldots 我就先割下你的舌头,挖出你的眼睛,再杀了你''顾人玉已吓呆了,他一生从未见到冷冷淡淡的九姑娘,发这么大的脾气,更未想到她会说出这么狠的话来!

小鱼儿却仍是笑嘻嘻的,道``我自然要走的!但我走了后,你可莫要再求我回来。''慕容九妹气得身子发抖,道``你\ldots\ldots 你这.。''突听外面一人遥遥呼道``慕容九妹,你在哪里?\ldots\ldots 小姐姐来瞧你了。''这呼声来得好快,一句话说完,便饭已由大门外来到小园里,慕容九妹咬了咬嘴唇,轻盈的身子,流云般飘了出去。

小鱼儿听到那呼声整个人都呆住了,再也笑不出来。

铁心兰也变了颜色,道:``莫非是\ldots\ldots 是小仙女张菁。''顾人玉道:``不\ldots。不错,她和九姐是好朋友。''小鱼儿噗地坐到椅上,苦笑道:``这世界怎地如此小!\ldots.''只听小仙女与慕容九妹在园中寒暄的语声渐渐走进。铁心兰听得手足冰凉,悄声道``咱们怎\ldots\ldots 怎么办?''小鱼儿坐在椅子上,长叹道``打又不能打,逃也不能逃,我也什么法子都没有了。''话末说完小仙女已冲了进来,失声道``果然是你这小鬼在这里''小鱼儿笑嘻嘻道``许久不见,你好吗?''

慕容九妹皱眉道``菁姐,你认得他?''

小仙女恨声道``认得,我自然认得,但''..。但他怎会在这里?``慕容九妹淡淡道:''他在外面受了伤,我\ldots\ldots{}``小鱼儿突然大声道''你莫要问了,我和慕容家丝毫没有关系,此刻又受了伤,你若要杀我,只管杀吧,既不必怕伤别人的面子,也不必怕我还手``小仙女冷笑道''你还手又怎样?"

小鱼儿大笑道``我若能还手,你就又要躺着不能动了''小仙女反手一个耳光掴过去,怒道``你再说?''小鱼儿动也不动,反而笑道``我不说了,我还有什么可说的,你两次落在我手上,只怪我看你可怜,两次都饶了你,今日就算死在你手上,也是活该。''他说的当真是大仁大义,动人已极,至于小仙女是如何会落在他手上的,他自然一字不提。

慕容九妹终于忍不住问道:``菁姐,你真的两次?\ldots。''小仙女气得全身发抖,却偏偏说不出一句辩驳的话来.慕容九妹瞧见她这模样,面上神情突然变得甚是古怪。

小鱼儿瞧在眼里,失声道:``慕容姑娘,你就让她杀了我吧,我虽然是在你家里被她杀的,但我也知道你看不起她,我绝不怪你。''小仙女己气极了不怒反笑,道:``你以为我不敢杀你''小鱼儿道:你自然敢的,大名鼎鼎的小仙女张菁,一辈子怕过什么人来?何况是我这根本不能还手的人``小仙女忽喝一声并指如剑.向小鱼儿额角太阳穴直点过去,小鱼儿根本不能闪避,铁心兰心胆俱裂哪知就在这时,人影一闪,慕容九妹突然已挡在小鱼儿面前,小仙女的手指已触及她娇怯怯的身子,方自硬生生收往,怒道:九妹.你难道要帮外人''慕容九妹淡谈道``若是在别的地方,你将他是打是杀,我全不管,但在这里菁姐你总该给小妹个面子。''小仙女道:``我杀了他再向你赔罪。''

慕容九妹道;``这庄院自从盖成以后,就没有杀人流血的事,菁姐你一定非想被这个例?你难道不能等等?''小仙女跺脚道``你\ldots\ldots 你不知道这小鬼有多可恶''慕容九妹道``纵然可恶,也等他走出去再\ldots\ldots{}''小仙女大喝道:``我等不及了''

她身形连闪七次,想冲过去但慕容九妹娇怯怯的身子,却总是如影随形,挡住了她的路。

其实慕容九妹要真是让她动手,她也未必会真个杀了小鱼儿,但慕容九妹越是拦阻于她,她反而越是愤恨,竟真的要将小鱼儿杀了才甘心,只见她纤指连续向慕容九妹攻出了七招慕容九妹身子飘飘闪动,冷冷道:``菁姐,这是你先向小妹出手助,可怪不了我。''小仙女手上不停,冷笑道``我若要做一件事时,世上没有一个人能拦得住我,我也不行\ldots\ldots 你只管将慕容家那些小针小箭使出来吧''\ldots{}``话犹未了,突听身后一人喝道''用不着,看招!``一股拳风击过来.竟是雄深沉厚,无与伦比小仙女一伏身''嗖``的窜了出来,大喝道:''好呀,顾小妹你也敢向我动手了。``小鱼儿暗笑道''原来他外号叫做顾小妹,这倒真的是名符其实,只是他人虽老实,武功却端的扎实,究竟不傀为武林世家的后人,看来就算这自命不见的小仙女,也未必能胜得了他。

他却不知顾人玉正因为人老实,是以武功才能练得扎实.``玉面神拳''顾人玉这七字,在江湖中也是赫赫有名的小仙女瞪着眼睛,叉着腰,喝道:``你们还客气什么,来呀''小鱼儿也在心里说``是呀,还客气什么,赶紧打吧。''谁知顾人玉却站在那里动也不动,低着头道:``只要张姑娘不向九姐出手,小弟又怎敢向张姑娘出手。''小仙女冷笑道``原来顾家神拳的传人,竟是个没出息的小子,你除了向你的九姐讨好之外,难道什么都不会?''顾人玉站在那里,连一句话都不说了。

小仙女气得跺脚,道;``好,慕容九妹,你来吧,你那宝贝七巧囊中,究竟有什么玩意儿也只管一齐使出来。''慕容九妹冷冷道;"只要你不在这里杀人,我又怎会和你动手小仙女瞧瞧她,又瞧瞧顾人玉,两个人一个堵着窗子,一个堵着门,竟硬是和小仙女泡上了。

小鱼儿笑嘻嘻道``你瞧也没用,反正你是闯不进来的,原来大名鼎鼎的小仙女,也有被人拦住的时候。''小仙女眼珠子一转,突也笑道``你希望我和他们打得落花流水,你才好在旁边瞧热闹,是不是?''小鱼儿大笑道``你不敢打就走吧,又何必找个梯子下台阶。''小仙女道``我正要走了,你若能在这地方躲上一辈子,我算服你,否则你只要踏出这大门一步,我就要你的命。''转身问慕容九妹一笑,道:``除非你嫁给他一辈子守着他,否则他总是要死在我手上的,我又何苦现在和你动手,教别人听见,反说我欺负你。''她倒退叁步,身形已在银铃般的笑声中飞掠而去,这位姑娘居然真的说走就走,倒也是小鱼儿想不到的事。

他瞪着眼睛,呆了半晌,苦笑道``女人\ldots。女人\ldots{}''唉,女人的心思,变起来真是吓得死人\ldots\ldots{}``慕容九妹轻轻叹息了一声,道''此人心思变化,当真无人能以猜测,性格也教人捉摸不定,唉!当今天下,只怕也唯有她才配做我的对手``\ldots\ldots{}''小鱼儿眨了眨眼睛,道``如此说来,天下英雄,只有你和她两人了。''慕容九妹道:``正是。''

小鱼儿道:``那么,谁是江湖第一?''

蹈容九妹沉吟道:``她行事精灵古怪,脾气变化无常,连我都猜不透。''小鱼几道:``你呢?慕容九妹玲冷道''我并末插足江湖。``小鱼儿道:''你若插足江湖,她就得变为第二了,是么?``慕容九妹道:''哼。"

小鱼几一本正经,点头道``不错,你确是天下第一。\ldots{}''慕容九妹扬了扬眉淡淡一笑,小鱼儿却又接着说道``你这自我陶醉的本事,的确可算是天下第一。''慕容九妹心情立刻又变了.小鱼儿终于忍不住大笑起来,笑得前仰后合,抚着肚子笑道``我本来以为只有男人才会自我陶醉,哪知女人自我陶醉起来,比男人还要厉害得多,何不走出去瞧瞧,就该知通江湖中比你强的人也不知有多少,但你若只要关起门来称第一,我也没法子。''慕容九妹道``你\ldots\ldots 你\ldots\ldots{}''

小鱼儿笑道``你虽然两次救我性命,但那都是你自己愿意的,我可没有求你,我既不领你的情,自然也不必说好听的话拍你的马屁。''慕容九妹道``好\ldots\ldots 很好。''

她虽然拼命想作出冷淡从容、若无其事的样子,却偏偏作不出,偏偏忍不住气得全身发抖。她确也是个冷漠寡情,不易动怒的人,但不知怎地,小鱼儿随便叁两句话,就能把她气得发疯。

顾人玉走了过来,呐呐道:``她总算对你不错你又何苦如此气她。''小鱼儿笑嘻嘻瞧着她,道:``我就是喜欢故意逗她生气,她生气的时候,岂非比平时那副冷冰冰的样子好看得多。''顾人玉忍不住地转头瞧了瞧,只见葛容九妹苍白冷漠的面颇微现晕红.早就比平时更增妩媚.他瞧了两眼,不觉已瞧得痴了.连连摇头道``不错,不错,果然漂亮多了。''慕容九妹眼睛一瞪,道:``你\ldots。你也敢在我面前说这样的话,你当我是什么?''顾人玉骇得赶紧低下了头,道``不\ldots\ldots 不\ldots\ldots 不漂亮,你生起气来丑得很。''铁心兰虽然满腔心事,一言未发,到此刻也不禁``噗嗤''笑出声来,小鱼儿更早已笑弯了腰。

只见两个垂髻少女,穿林而来,远远便娇笑唤道``九姑娘\ldots\ldots 九姑娘\ldots\ldots{}''慕容九妹正是满肚子气没处发作怒道``喊什么?我又不是聋子。''那少女也骇得赶紧一齐垂下了头,道``是\ldots。九站娘。''四只眼睛偷偷一瞟小鱼儿,又赶紧垂下头接着道``屋子已经整理好了,姑娘你是不是现在\ldots\ldots{}''慕容九妹道:``自然现在就去瞧,每天都如此,还问什么?''那两个少女从来未见过她们的九姑娘这样说话,垂头说了声``是'',头也不抬,一溜烟走了。

慕容九妹冷冷道:顾少爷若是没事,就请在这里看着他们,否则我也不敢留你。``顾人玉道:''小弟没事,没事,没事\ldots\ldots"

他一连说了五六句``没事'',慕容九妹早巳走出了门外,小鱼儿向铁心兰挤了挤眼睛,也跟着走了出去。

顾人玉失魂落魄地瞧着慕容九妹,铁心兰也呆呆地瞧着小鱼儿,顾人玉不由自主叹了口气,铁心兰也不由自主叹了口气,道,``你对她真好\ldots\ldots 也许太好了。''她嘴里在说顾人玉的事,心里想的却是小鱼儿的事,顾人玉为什么会对慕容九妹这么的好,而小鱼儿\ldots\ldots 她柔肠百折,想来想去,顾人玉说了句什么话,她完全没有听到,过了半晌,幽幽道:你是不是很喜欢她。``顾人玉茫然道:''我\ldots\ldots 我不知道。"

铁心兰轻轻一笑,道``你不知道?''

顾人玉叹道``别人都觉得我应该喜欢她,我自己也觉得应该喜欢她,但\ldots\ldots 但我\ldots\ldots 我是不是喜欢她,我也不知道,我只知道我是怕她的。''铁心兰嫣然一笑,道:``你真是好人。''

顾人玉瞧了她一眼,垂首道``你\ldots。你也是个好人。''慕密九妹走到园中,突然回过头.冷冷道``你跟来干什么?''小鱼儿笑嘻嘻道:``我本不想因来的,但我若不跟着你,小仙女若是乘机来将我杀了,我生死虽没有什么要紧,你的面子岂非难看。''慕容九妹瞪了他半晌,再不说话,又往前走。小鱼儿踉跄地跟在她身后,不住喘着气,柔声道``我走不动了你拉着我的手好吗?''慕容九妹根本不理他,走得更快。

小鱼儿道``好我就累死算了,我死了之后,你把我的尸体送给小仙女,她以后就必定不会找伤的麻烦了。''慕容九妹虽末回头,但脚步却果然已放缓。

小鱼儿道``有些女孩子,平时看来虽比男人强,但真的见着男人,可就没用了!喂,你可瞧见过不敢拉女人手的男人么?''慕容九妹终于忍不住冷冷笑道:``不敢?哼,我只是\ldots。''小鱼儿:``你只是不愿,是么?哈哈,世上没有一个人会承认自已是不敢的,这不愿,两字,正是不敢的最好托词。''慕容九妹突地转手,拉起了他的手,于是急行。

小鱼儿不由自主的跟着她跑,嘴里还笑嘻嘻道:``你的手真小,大概还没有我一半大\ldots\ldots{}''他嘴里不停在说.眼珠子也不停在转,只见花园之侧,一道浅阶曲廊,沿着山坡婉蜒而下。曲廊之旁,便是一间间精致的屋子,每一间建筑的形式都不一样,每一间的窗子颜色也不一样。小鱼儿数了数,这样的屋子一共有九间,想来就是慕容九妹妹的闺房第一间的窗纸是浅黄色的,慕容九妹推门走了进去,屋子里的窗樱、桌布、被褥\ldots\ldots 也都是浅黄色的,简简单单几样东西却自有一种优雅之意。

慕容九妹走了进去,把每样东西都仔细瞧了一遍,瞧瞧上面可有灰尘,小鱼儿却在瞧着她,道``这是你大姐的闺房.你大姐可是就要回来了。''``不回来就可以任它脏么?''

小鱼儿笑道:``不错,虽然不回来,也要将每样东西保持干干净净,看来你们姐妹间果然是情意深厚。''他突然不再说尖酸刻薄的话了,慕容九妹一时间倒摸不到他的用意,哼了一声,也不答话。

小鱼儿道:``你大姐想必是位优雅娴静、温柔美丽的女人,唉,这样的女人,世上已不多了,却不知她的夫婿可配得上她。''慕容九妹终于回头瞧了他一眼,道:``世上自然没有能配得上我大姐的人,但若有一人能勉强配得上她,那就是我大姐夫小鱼儿道:''他武功如何``塞容九妹冷冷道;''你总该细道,美玉剑客这名字。``她本来决定再不愿和这可恨可厌的小鬼说话的,但此刻不知不觉间又说了许多,只是这小鬼''和她说的正是她最愿意说的话题,这小鬼虽然两句话就能将她气得半死,但两句话又可将她的气说平了。第二间屋子全都是粉红的,粉红的墙壁,挂着柄长弓,还挂着口短剑,连剑鞘都是红的。

小鱼儿笑道:``你二姐脾气想必和大姐不同,她想必是个天真直爽的人有时脾气虽然坏些,但心地却是最好的,而且最肯替别人设想。''慕容九妹默然半晌,终于忍不住问道``你怎会知道?''小鱼儿道``慕容家暗器之精妙,天下皆知但你二姐偏偏要使长弓大箭,可见她脾气必是豪爽,喜欢痛快,自然就不喜欢那些精巧的玩意儿。''慕容九妹道``嗯,还有呢''

小鱼儿道``剑长则稳,剑短则险,你二姐用的剑短如匕首,可见她脾气发作时,必是勇往直前,不顾一切。''慕容九妹不由得点了点头,道``我二姐剑法之辛辣险急,可称海内第一。''小鱼儿笑了笑,道``但你二姐夫武功却不高,是么?''他突然间说出这话来,慕容九妹也不禁一怔,诧异地瞧着他,瞧了足足有半盏茶时分,才缓缓点头道:``我二姐夫乃是南宫世家一派单传的独子,南宫世家武功虽然高绝,但我二姐夫却是自小多病,所以\ldots\ldots 唉!''小鱼儿拍手笑道:``这就是了。''

慕容九妹道:``是什么?''

小鱼儿道:``你叁姐出嫁之后,仍将随身的兵刃留在这里.为的自然是不愿以自已因武功来使夫婿觉得惭槐难受,由此可见她夫婿武功必不如他,因此也可见她心地是多么善良,多么肯替别人着想。''慕容九妹默然瞧了他几眼,转身走到第叁间屋子。

这第叁问屋于窗上竟糊着的是极厚的黑纸,屋于里自然光线黝暗,但陈设却是精致,妆台旁有琴案、棋枰,画架上满堆着画,墙上接着极精妙的工笔仕女,题款是``慕容女史'',想来就是她自己的手笔。

小鱼儿目光四转.笑道``你这位叁姐,想必是个才女只是性情也许太孤傲了些,也未免太忧郁,但古往今来的才子才女,岂非惧是如此。''慕容九妹悠悠道``她最不喜欢见到阳光.最喜欢的就是雨声,在雨声中她画出的图画真是不带丝毫人间烟火气,她抚的琴,雨声中听来,更好像是天上传下来的,只可借\ldots\ldots 只可惜我已有许久未听见了。''小鱼儿道:``你叁姐夫呢''

慕容九妹道``他也是武林中的绝顶才子,不但琴模书画,无一不精,而且二十九岁时,便已成为两广武林的盟主。''小鱼儿笑道``如此朗才女貌,好不羡煞人了。''

\hypertarget{ux7b2cux5341ux4e5dux7ae0-ux5f04ux5de7ux6210ux62d9}{%
\chapter{第十九章
弄巧成拙}\label{ux7b2cux5341ux4e5dux7ae0-ux5f04ux5de7ux6210ux62d9}}

小鱼儿随着慕容九妹向一间间房子走过去,走完第八间,慕容九妹神情又大见温和,甚至连眼波都温柔起来,她觉得这``小鬼''实在并不如自己方才想象中那么可僧可厌,谈谈说说,不知不觉已到了第九间。

这间房子什么都是浅碧色的,最精致、最华丽,房子每件东西,都是人间罕睹的珍贵之物。

小鱼儿大眼睛四下转动,突然笑道,``这间房子的主人和前面的完全不同。''慕容九妹目中闪过一丝笑意,神情却是淡淡的,像是漠不关心,只不过随口问问,道``什么不同?''小鱼儿道``这房子里绿色,正表示她自我陶醉、自命不见。这些零零碎碎的东西,也正表示她幼稚、虚荣、俗不可耐\ldots\ldots{}''他话未说完,慕容九妹面上已变了颜色,终于铁青着脸,冲了出去,再也不瞧这可恨的小鬼一眼。

小鱼几忍不住哈哈大笑,道:"我若说错了,你又何必生气慕容九妹头也不回,往前走,小鱼儿跟着她,叁转两转,突然来到一条青石通道中,通道尽头,有扇青铜的门,小鱼儿自然看不见门里的情况,但就只瞧见这扇门,他巳感觉到一种神秘诡谲之意他也说不出这是什么缘故。只见慕容九妹取出柄黄金色的钥匙插入门上一个小洞之中,转了转那扇沉重的门,便无声无息地开了。一股寒气,自门里涌出来。

小鱼几立刻觉出,这间房子和他万春流万大叔的屋子有七分相似之处,屋子四周也堆满了各式各样的药草,自然也有些炼丹制药的铜鼎钢炉只是万春流的屋子乃是以砖瓦建成,这屋予四壁却都是巨大的青石,万春流的屋子四季温暖如春,这屋子却是阴森森的教人发冷。

慕容九妹己将那扇青铜曲门锁起来了,她苍白的面颊,到了这屋子里更变得发青。

小鱼儿笑道``原来咱们的九姑娘还是位女大夫,当真是多才多艺,你带我到这里,莫非又想为我看病。''慕容九妹道``不错。''

小鱼儿道``我的毒已解了,还有什么病?''

慕容九妹道:``你身上多了件东西,若将这件东西割击,你就好多了。''小鱼几笑道``哦!那是什么东西?''

慕容九妹冷冷道:``你的舌头''

小鱼儿伸了伸舌头,赶紧走得远远的,竟道:``我说的话,真能令你如此生气么,那我当真荣幸得很。''慕容九妹冷笑一声,转过了头,道``此间之药草,俱是十分珍贵之物,你万万不可乱动。''小鱼儿笑道:``你想我会不会动?''

慕容九妹笑道:``你若要动,也由你.但这些药草中虽有补气延年的灵药,却也有夺命穿肠的毒草,你若被毒了,可没有人再来救你。小鱼儿又吐了吐舌头,道''你莫吓我,我这人别的也没什么,就是胆子太小,只要被人家一吓,可就吓倒了。``慕容九妹冷冷道''但只要你老老实s实在这里不动便绝没有人能伤你一根毫发,现在是我练功的时候,我得走了。``小鱼儿道''你\ldots\ldots 你要到哪里去,我跟着你。``慕容九妹厉声道''你若再跟着我,不等别人你你,我就要你死``小鱼儿叹了口气,道''其实像你这样漂亮的女孩子,只要笑一笑已是够人神魂颠倒,还要练什么功夫``\ldots 功夫练老了,人也练老了。''慕容九妹也不理他,径自走向另一扇铜门,又取出柄黄金钥匙将门开了一线,回首道``你若要妄入此门一步,就休想再活着出来''小鱼儿笑道:``你门是锁着的,我怎么进得去。''慕容九妹冷笑道,``谅你也进不来的。''

身子一闪,进了钢门,门立刻紧紧关起,``喀啷''一声,又上了锁,竟不让小鱼儿瞧一眼,这门里又是何模样。

小鱼儿也全不着急,懒洋洋伸了个懒腰,喃喃道``女人\ldots 唉,女人,你们最大的毛病,就是将天下的男人都看成笨蛋傻子\ldots。你以为我连这些药草是毒药还是灵药都认不得么?告诉你,我从小就是在药草堆里长大的,我认识的药草可比你多得多。''他一面自言自语,一面东翻翻西瞧瞧,又笑道:``不怪她要吓我,这里的药草,倒真有好些货色,万大叔找了几十年没找到的,这里却有叁四样,嗯,看来我的口福倒不错。''他竟真的选了叁四种药草大嚼起来,慕容九妹若是在旁边瞧着,可真的要急得晕倒过去。

这几种药草中,有些确是稀世之物,小鱼儿其实也未瞧见过,只是万春流曾经绘出图形,教他辨认。这些药草万春流搜寻数十年,却未寻得一味,由此可见价值之珍贵若是炼成丹药,粒便可活人。

此刻像小鱼儿这样的吃法却当真是王八吃大麦,糟蹋粮食,但他一点也不心疼,片刻间便吃了个干净。

他抚着肚子笑道``肚兄呀肚兄,今日可便宜了你。''眼珠子一转,竟还意犹未足,脑筋又动到那些铜鼎中的丹药上去。

他竟把铜鼎全都揭开,瞧了瞧,嗅了嗅,取出一把,像嚼花生米似的吃得津津有味,右手还不停地一把把往怀里塞,塞不下了,他就将剩下的丹药全都混在一起,扮了个鬼脸,笑道``你既然闲着没事,我就找些事给你做做吧。''这一来可真害苦了慕容九妹她若想将这些丹药分门别类,少说也得叁天五天的工夫。

但小鱼儿自已此刻可也不好受,十几种草药、丹药,像是已在肚子里烧起了火来,烧得他身子发热了,嘴唇发焦。他歪着头想了想,自怀中取出极弯弯曲曲的铜丝,伸进那扇铜门的钥匙洞里,笑嘻嘻道``你以为我进不去么?好,我就偏偏进去让你瞧一瞧。''他耳朵凑在钥匙洞上,手拨着钢丝,一面拨.一面听,脸上渐渐露出了笑容,喃喃道``这里\ldots─这里\ldots─对了,就是这里''只听``喀啷''声,钢门立刻开了。

里面的房子,比外面的更冷,寒气又自门缝中袭出。

小鱼儿深深吸了口气,道;``好舒服。''

他此刻全身像是被火在烧,自然越冷越舒服,索性开了门,大步走进去,一面大笑道;``九姑娘,我进来了,你只管练功,我不吵你''话说完了,人也征住,只见这石室中还有个地洞,地洞里全是从冬天就窖藏留存的冰块。

慕容九妹就坐在冰上,双手自腿的外侧弯入腿的内侧抱住了脚,食指点着足心,全身竟是赤裸裸的一丝不挂。小鱼儿活了这么大,见过的事也有不少,但赤裸的少女,却是从未见过的,他无论见到什么都不会吃惊,此刻却也不禁呆呆地怔住了。

慕容九妹眼睛是睁开的,也瞧见了他她眼睛里的惊奇、愤怒、羞急,无论用什么话也不能形容。

但她身子却动也不动,似乎已不能动了。

小鱼儿呆了几乎有半盏茶的工夫!这才转过身子,故意东张西望,道``九姑娘在哪里?我怎地瞧不见呀''达``小鬼''就是这么会体贴女孩子的心意,这句话出来,慕容九妹明知是假的,也可自我安慰一下了。

小鱼儿一面说,一面走,就要退出门,忽然瞧见墙上挂着九幅图画,他又忍不住要停下来瞧瞧。只见第一幅图上,刻画着赤身露体的女子,以手脚倒立在冰上,旁边写着几行小字:化石神功,须处女玄阴之体方能习之,此乃化石神功之入门第一步,叁年有成,口诀如下。"``化石神功,功成九转,肌肤化石,万物不伤,九转功成,无敌天下\ldots\ldots{}''小鱼儿看到这里,巳不禁失声道``这鬼功夫竟活活的要将人练成僵尸,慕容九妹练了这种鬼功夫,难怪对什么人都要冷冰冰的了。''他赶紧去瞧第二张图,只见上面画的人已由倒立而直立,上面写着:``功成二转,由逆为正\ldots\ldots.''小鱼几也懒得往下瞧,他可无心来学这种鬼功夫,人若变成了石头般又硬又冷,纵能无故天下,又有何用?``第叁张图上画着的人形,姿态就和慕容九妹此刻练功时一样,小鱼儿松了口气,喃喃道:''幸好她只练成第叁转就被我瞧见,否则她功夫若是练成了,人也必定要变成个怪物,那就真是害人害己了。"他再也不往下瞧,七手八脚,将挂着的图全扯了下来,慕容九妹仍在瞪着他,目光却由羞愤变成哀求。

小鱼儿也不回头去瞧,口中大声道:``九姑娘,你莫恨我,我这是为你好,你好好一个人,活得快快活活,为什么偏要自己给自已找罪受。''慕容九妹此刻若能说话,若不放声痛驾,便要苦苦哀求,她若能动,只怕早已将小鱼儿吞下肚里。怎奈她既不能言,也不能动,只有眼睁睁瞧着小鱼儿揭起九张图扬长面去,她目中不禁流下眼泪。

小鱼儿将九张图全丢在铜炉里烧了,又弄开外面那扇门的锁,走了出去,居然也不去瞧铁心兰,就越墙走出了这山庄。他做事全凭一时高兴,有时做对,有时做错,但是错是对,他全不管,只觉做了这件事,心里颇是舒服,做完了后果如何,他也全不放在心上。只是他此刻身子一点也不舒服,不但热,而且发起涨来,就像是有人不断往他肚子里填火。

他一口气也不知奔出了多远,一头钻进了树林,凉风穿林而过,自然要比外面凉快得多。

小鱼儿实在走不动了,倒在树下直喘气,心里只希望小仙女此刻莫要来,慕容九妹更莫要来。

他身上又热、又涨、又痒,嘴里干的冒火,喃喃道:``这里要是有个池塘就好了,我现在最需要的就是水''\ldots 水\ldots\ldots{}``突听\ldots 人冷冷道''你此刻最需要的不是水是棺材"小鱼儿但觉脖子一凉,已有一口剑架在他脖子上。

他一惊一怔,苦笑道:``到底还是女人厉害,男人若被女人盯上了,一辈子就休想跑了。''那语声冷笑道``你现在才知道.已嫌太晚了。''小鱼儿道``你是慕容姑娘?还是小仙女?''

那语声道``你还想九丫头救你,你是做梦。''

小鱼儿突然笑了起来,喃喃道``很好\ldots\ldots{}''很好\ldots\ldots 是你,就还算我运气不错。``小仙女自然想不到小鱼儿此刻最怕见的不是她而是慕容九妹,冷笑道''很对,你的运气好极了,偏偏要走这条路偏偏我就在这里等着。"她这话自然是故意来气小鱼儿的,小鱼儿纵然走别的路,还是跑不了的。

小鱼儿脖子动了动,道:你这柄剑很快嘛。"

小仙女道``哼,也不太快,只是我削下你脑袋时,只怕你嘴里还能说话。''小鱼儿笑道:``我那般折磨你,你一剑削下我脑袋,就能出气么,嘿嘿,我若是你,可就没有这么便宜了。''小仙女道``你想受什么罪,只管说吧,我一定包你满意。''小鱼儿道:``至少先得揍一顿再说。''

小仙女冷笑道:``你以为我不敢揍你。''

小鱼儿笑道``你虽能狠一狠心将我杀了,却是舍不得见我挨揍的。''话未说完,脖子上就挨了一拳.背上又挨了一脚。

小仙女咬牙道``很对,我舍不得揍你,很对''她说一声``很对''就揍出一拳,说一声``舍不得'',又踢出一脚小鱼儿被揍得满地打滚,口中却大笑道``舒服\ldots。舒服\ldots\ldots{}''他是真的舒服,可不是假的,他身子正涨得发痒,小仙女拳头打在他身上,倒像是替他捶背,松骨。

小仙女怒道``好,你既舒服,就再打重些。''她话未说完,小鱼儿背上已重重地挨了一拳。

小鱼儿道``不行,还是太轻了''。``再重些。''

小仙女几乎气破肚子,但瞧见小鱼儿面上竟真的全无痛苦之色,她又不觉惊讶、奇怪。她哪里知道小鱼儿体内十几种灵丹妙药的药力已活动开,纵然是铁锤击在他身上也伤不了他的筋骨。小仙女的手倒有些打酸了,小鱼儿还是不住道``舒服,舒服,再重些\ldots\ldots{}''"小仙女想起那日他被痛揍之后,还能奋起击人之事,更是奇怪这小鬼为何如此能挨揍。

突听一人冷冷道:``你打够了么?''

小仙女霍然转身,站在树下的正是慕容九妹。

只见她被头散发,眼睛里满是红丝,指尖不住发抖,小仙女再也想不到她怎会如此模样,大声道:``还没有打够,你要怎样?''慕容九妹道``你若打够了,就让给我。''

小仙女冷笑道``这里可不是你的家了,你若再阻拦我,我也\ldots\ldots{}''慕容九妹道:``你以为我是来救他的么?''

小仙女又怔了怔,道``你不是来救他的,还是来杀他的不成?''慕容九妹道``正是来杀他的!''

突然掠到小鱼儿身旁,抽出一柄匕首,直刺而下!

小鱼儿见到她们两人全来了,心里反倒不怕了既然非死不可,还有什么好害怕的?他瞪着眼睛,瞧着这柄匕首,突见寒光一闪,``叮''的一响,小仙女手里的短剑已架住了匕首。

慕容九妹怒道``你方才本要杀他的,此刻为何要救他?''小仙女冷笑道``你方才本是救他的,此刻为何又要杀他?''慕容九妹道:``你\ldots。你管不着。''

小仙女大声道:``我偏要管。''

慕容九妹手腕一挥,闪电般刺出七刀,道``今日无论是谁来拦阻我,我也是要杀定他了!''小仙女短剑挥出,闪电般接了七刀,道``你方才不许我杀他,我现在也不许你杀他!''

慕容九妹道``你方才苦苦要杀他,此刻却反要救他,莫非\ldots\ldots 莫非是你对他\ldots\ldots{}''小仙女脸绯也似的红了,大声道:``你方才苦苦要救他,此刻反却要杀他,莫非\ldots\ldots 莫非是他对你\ldots\ldots{}''慕容九妹苍白的胜也绯红起来,喝道``你敢胡说!''小仙女喝道``你才是胡说''两人刀剑齐齐击出,``当''的,又硬拆了一招,两人却觉手腕有些发麻,身子也被震得后退数突然间,两人同时惊呼出来。

小鱼儿竟已不见了!

小仙女跺足道:``都是你害得我\ldots\ldots{}''

慕容九妹跺足道``都是你害得我\ldots\ldots{}''

两人同时开口,同时闭口,说出来的竟是同样的一句话,同样的几个宇,两人脸都红了。

小仙女瞧了瞧慕容九妹,慕容九妹瞧了瞧小仙女,小仙女垂下头,慕容九妹也垂下了头。

小仙文终于抬起头来,道``他逃不了的''

慕容九妹也同时始起了头,道``追''

两人红着脸想笑一笑,却又笑不出。

小仙女咬着嘴唇,道``这次追着了,咱们两人同时下手杀他''小鱼儿也知道自己无论凭轻功,凭体力,都是逃不了的,所以他什么地方都不逃,却径自逃回慕容山庄。他从原路跃回,竟笔直走到那石室铜门前,门自然又锁上了,他自然也又轻易地将锁弄开。然后,他将两扇门都从里面锁起,伸展了四肢,舒舒服服地躺在那贮冰的地洞旁,忍不住笑了起来。想起小仙女和慕容九妹方才的模样,他就要笑,这两人在别人眼中是侠女、才女,但在小鱼儿眼中,她们却只不过是个女人,在小鱼儿眼中,世上的男人可能有一百七八十种,但女人却只有一种。

但身子越来越热,嘴唇越来超干,他索性跳下地洞,躺在冰堆里,敲了块冰嚼得``喀吱喀吱''直响,嚼了七八块后,但觉通体生凉,舒服得很,索性就躺在冰上呼呼大睡起来。

此时此地他居然还睡得着,本事当真不小。

睡梦中,突听``克郎''一声,铜门竟似开了,小鱼儿一颗心登时提了起来,动也不敢动,气都不敢喘。

只听小仙女的声音道,``好冷。''

又听得慕容九妹的声音道``昔日家母建造这藏冰窖时,本为了家父怕热,在暑中最嗜冰镇酸梅汤,哪知后来我却做了别的用途。''小仙女又道:``什么用途?''

慕容九妹默然半晌低低叹道``现在,什么用途都没有了。''语声中充满了伤心失望。

小鱼儿听得直发毛.他知道慕容九妹实已恨透了自己,自已若被她们堵在这冰窖里,可是再也休想逃了。

小仙女道``你怕那小鬼还逃到这里来么?''

``嗯。''

小仙女笑道;``你也未免太多虑了.那小鬼又怎会有这么大的胆子。''慕容九妹道:``我真不懂,他会逃到哪里去?''小仙女叹道``那小鬼当真滑溜如鬼,诡计多端,下次见着他时,我话也不跟他说就宰了他,看他还有什么花样使得出来。''语声渐远,又是``克朗''一声,门已锁上了。

谢天谢地她们总算走了,小鱼儿暗笑道"幸好女人都是小处仔细,大处马虎,既要瞧,又不瞧个仔细,否则我真要倒霉他又静静地伏了两盏茶工夫,身上已有些发冷这才一跃而起,他若在冰上调息运气,将药力归纳入元功力必有骇人的增长,只可惜他只是睡了觉就爬起来,这良机竟被他平白的糟塌小鱼儿屏息静气凑眼在那钥匙洞上向外瞧了瞧便发觉小仙女与慕容九妹竟还在外面那屋子里。小仙女斜斜倚在墙上,似乎在出神地想着心思,慕容九妹身子站得笔直,面色苍白得可怕。铁心兰竟也在这屋子里,她坐在药鼎前,正将鼎中的药一粒粒拣出来,分别装到几个铜罐里。她满眶泪水,每捡一粒药,眼泪就落下一滴。

小鱼儿瞧得直皱眉头,暗笑道``我本是要害慕容九妹的,哪知却害了她,想来是慕容九妹恨我入骨,竟把气出在她身上,叫她来做苦工。''顾人玉呢?顾人玉想必是连这屋子都不准进来。

小仙女出了会儿神,突然向铁心兰走过去铁心兰一惊,手里握着一把药丸,洒了满地。

语声自钥匙洞里传进来只听小仙女叹道``你不要怕,我不会难为你了,咱们都是被那小鬼骗苦了的,正是同病相怜。''铁心兰垂下头眼泪滴滴落在衣襟上。

小仙女展颜一笑道``来,快动手我帮你的忙,看来咱们若不将这些药丸整理清楚,九姑娘是不肯给咱们饭吃的了。''慕容九妹冷冷的瞧着她们,面上没有一丝笑容。

过了半晌,小仙女突又道:``那张图\ldots{}''你可真的被那小鬼骗走了。``铁心兰默然半晌,低声道''不是骗,是我送给他的。``小仙女道:''送给他\ldots\ldots 你为什么要送给他?``铁心兰霍然站了起来,大声道''我高兴送给谁就送给谁,这事谁也管不着。``小仙女征了怔,失笑道:''你凶什么?"

小鱼儿暗笑道``小仙女外刚内和,铁心兰却是外和内刚,这两人性子当真是两个极端,而慕容九妹呢她练了那种鬼功夫,外面冷冰冰,心里只怕也是冷冰冰的,这叁人中,最不好惹的就是她了。''又过了半晌,小仙女道:``你还生不生气?''

铁心兰垂下了头.似也有些不好意思,别人若是对她凶恶,她死也不服,别人若是对她好,她反而没法子。

小仙女道``那张图你想必是看过了的。你可记得?''铁心兰道``我。\ldots 我记不清了。''

小仙女道:``我可不是想要那些珍藏,我发誓决不动它们,只是,我想\ldots\ldots 那小鬼必定会到那里去的,你若记得那地方,咱们就可找着他我替你出气。''铁心兰头垂得更低,道``我真的记不得了,我不骗你。''小鱼儿自钥匙洞里往上瞧.正好瞧见她的脸,只见她说话时眼珠子不停地在转,不禁暗笑道她想必是记得那藏宝之地方,只是不肯说出来,这丫头看来老实,嘴里直说不骗人,骗起人来却笃定得很\ldots\ldots{}

心念一转,又忖道``她为何要骗人?\ldots\ldots 莫非是为了我?我对她这么坏,但到现在为止,她非但还是不肯说我一句坏话,听到别人说我坏话,她反而要生气,这是为了什么?''想着想着,他似有些痴了,但瞬间又暗中自语道``我管她是为什么,反正女人都是神经病。''突见慕容九妹快步走了出去,小鱼儿正在奇怪,她又走了回来,手里却拿着个小小的铜勺子。

小仙女道:``这里面是什么?''

慕容九妹道:``铅。''小仙女奇道``铅?你拿铅来要做什么''慕容九妹也不说话,却将那铜勺在火上煨了半晌,目中突然露出一种残忍而得意的光芒,口中缓缓道``里面那屋子,反正也没有用了,我索性将铅将这钥匙洞塞住,这样,谁也休想再进得去,谁也休想再出来!''小鱼儿瞧见她那笑容,巳觉不对,再听到这话,更是心胆皆丧,这慕容九妹好狠毒的手段,竟想将小鱼儿活活关死在里面,她虽然发觉小鱼儿,却绝不说破,只因她生怕小仙女和铁心兰还会救他小鱼儿大骇之下,赶紧想弄开锁冲出去,但慕容九妹已一步掠过来,小鱼儿只瞧见铜勺在钥匙洞外一晃,接着,就什么也瞧不见了,铅汁,已灌了进去,外面的人声也一起被隔断。只听外面突然有人在钢门上踢打起来,这慕容九妹竟生怕小鱼儿在里面敲门,被小仙女与铁心兰听见猜出。

所以她竟自己先敲起门来,小鱼儿再拍门,外面也听不见小鱼儿又惊又伯,跺足大骂道``慕容九妹,你这妖妇,恶婆娘,你的心为何要这么狠,我又没害死你爹妈,又没强奸你,你为什么定要我死?我方才若不是瞧你那瘦骨头全无兴趣,早己乘机修理了你,你现在只怕反不会要我死了。''他破口大骂,什么话都骂了出来,在``恶人谷''长大的孩子,骂人的技术,自然也比别人高明得多。这些话若被慕容九妹听见,不活活的气死才怪,只是四面石墙,钥匙洞又被塞住,小鱼儿骂得虽卖力,外面连一个宇都听不到。

骂了半天,小鱼儿也知自己骂破喉咙也是没用的了,在屋子里乱敲乱转,想弄出条出去的路。怎奈藏冰的屋子,必须建造得分外牢固,不能让一丝热气透入,正是天生牢狱,小鱼儿想尽法子,也挖不出一个小洞。

小鱼儿苦笑道``谁说这屋子没用了,这屋子用来关人,岂非比什么地方都好得多,看来,我只握真要变成条冻鱼了。''他已冷得牙齿打战,只有盘膝坐下,运气相抗,一股真气传达四肢,这才渐渐有了些睡意。小鱼儿本不是个用功的人,方才纵然明知自己将大好机缘白白糟塌了,他也满不在乎。只因他觉得自已是天下第一聪明人,武功好不好都没有关系,反正无论多厉害的人遇着他也无可奈何,他又何必吃苦用功?

但现在情势却逼得他非用功不可,他这才知道那十余种灵药功用当真非同小可,糟蹋了实在有些可惜。药力随着真气流转,功力也跟着增进,他不知不觉间竟巳进入了人我两忘之境,竟将生死之事忘怀了。

\hypertarget{ux7b2cux4e8cux5341ux7ae0-ux4ebaux5fc3ux96beux6d4b}{%
\chapter{第二十章
人心难测}\label{ux7b2cux4e8cux5341ux7ae0-ux4ebaux5fc3ux96beux6d4b}}

这样也不知道过了多久是几个时辰?还是几天?休息的时候他就将怀中的药丸掏出来吃,既不觉饿,也不觉冷。但出去是无法出去的,他迟早也是要活活地被困死在这里,那么纵然练成了绝世的功力,又有何用?小鱼儿想到这里,便要自暴自弃,只是功夫一不练,就冷得厉害,他死活没关系,又何必在活着时多吃苦。

他终究不是神仙,肚子终于饿了,饿得连用功都不能,一饿更冷,他自知死期已不远了。他再也想不到自已这么聪明的人竟也会被人困死,尤其想不到的是,自已竟会死在女人的手上。

这才知道女人并不如自己所想象的那么简单,那么无用,他忽而自责自骂忽而自艾自怨,不住喃喃道"看来好人真是千万做不得的,我若早将小仙文和慕容九妹杀了,又怎会有今日之事于是他又怪万春流,若不是万春流,他彻头彻尾都是个坏人,坏人纵被人恨,被人骂,至少命总比好人活得长些。

他冷得全身发抖,饿得头晕眼花,喃喃道``唉,死就死吧,反正人人都要死的,人死之后,至少也有件好事,那就是他再也不会听到女人的噜嗦了。''但突然间,他竟不再觉得冷了。非但不冷,而且还发起热来,他又惊又奇,张开眼睛,又瞧见桩怪事,那一大块一大块冰,竟也在溶化。

伸手一摸,冰冷的石壁,竟也热得烫手。

小鱼儿跳了起来道:``这是怎么回事?难道慕容九妹那丫头冻死我还不过瘾还要烤熟我\ldots\ldots 不对,她将她姐姐的那几间房间瞧得那般珍贵,又怎会在此引火?''

他围着屋子走了一圈,四面石壁,叁面都烫得像火,只有背山的那面,还只是温热的。

小鱼儿心念一转,恍然道:``是了,想必是慕容家的仇人来了,不但要杀人,还要放火\ldots\ldots 只是你们这些蠢材不知道,你们放火烧了慕容家的破屋子不打紧,却连天下第一个聪明人也要被你们害死了''说着说着,他又跳脚大骂起来。

还不到顿饭工夫,巨大的冰抉全都溶化了,小鱼儿已被泡在水中,想跳脚都无法跳了。水,本来还是凉的,人泡在里面还不觉得难受,小鱼儿既然想不出法子,索性脱了衣服,在里面痛痛快快洗了个澡。他天生不见棺材不流泪的脾气,不到真正走投无路的时候,谁也休想要他着急、害伯。

但现在已到了他真正走投无路的时候了。

水,已渐渐热了起来,像是快要沸滚了,小鱼儿泡在水里,就像是被人抛进热锅里的一条活鱼烫得他在锅中乱蹦乱跳。他只望火能将石壁烧毁,但这见鬼的石壁偏偏坚固得出奇,非但没有毁坏,简直连条裂缝都没有。到后来他什么力气都没有了,竟沉了下去,鼻子一酸,``咕嘟咕嘟'',灌了好几口水。

小鱼儿苦笑道:``好大的一碗鲜鱼汤,叫我一个人独自消受,岂非可措\ldots\ldots{}''突听钢门外有人``叮叮当当''敲打起来。

小鱼儿精神一振,暗道,``好了,这下子总算有人来和我分享这碗鱼汤了''他已想到这大火虽烧不毁铜门,却可将钥匙洞里的铅烧溶,那精巧的机簧,被滚热的铅汁一烫,只怕就不保险,外面只要有人用凿子、钉子之类的东西一敲,铜门九成是要敲开的。

他念头还未转完,铜门果然开了,水势如黄河决提,一下予涌了出去,小鱼儿也不动,任凭水将他冲出。外面两个人再也想不到开了门后会涌出这么大的水,一惊之下,全身己被淋得像是落汤鸡。

他们更是做梦也未想到的是,水里竟还有个人。

小鱼儿被水冲得远远的,就躺在那里,死人般不动,他已被饿得半死,泡得半死,又怎能妄动。眯着眼偷偷瞧了瞧,外面的火,竟已熄了,从这间屋予的门瞧出去只见一片焦木瓦砾仍在冒着青烟。

老房子着火,自然烧得快些。

再瞧这两人,前面一个高大魁伟,满脸横肉,一嘴络腮大胡子,虽被水淋得湿透,看来仍是雄赳赳,气昂昂,就像是条牛似的,小鱼儿瞧见此人,心里很放心,这种四肢发达的人,头脑一定也被肌肉挤得很小,他只要略施小计,保险可教这人服服贴贴。

但另一人他却瞧得有点寒心,这人一身白衣,弯着腰,驼着背,一张脸就像是倒悬的葫芦,再加上一嘴山羊胡子,两只细眉小眼,就算将他放到山羊窝里去,也不会有人瞧出他是人来。

他身子本就轻枯瘦小,再驼背,头还够不着那大汉的胸口,但看来却比那大汉可怕十倍。小鱼儿一瞧这两人,就知道他们十成中有九成必定就是``十二星相''中的``白羊黄牛''了。

他发觉这``十二星相''长得实在都不像人,却像是畜牲,这十二人凑在一起,也不知是怎么找出来的。

两人瞧见小鱼儿,都怔了半晌,那``黄牛''咧着嘴道``谁要听你的话那人准是祖宗没积德,上辈子倒了霉,我早就发誓将你说话当放屁,谁知这次还是要上当。''那``白羊''道``听我的话,才是福气。''

黄牛直着嗓子怪笑道``福气.被淋了一身臭水难道也算是福气,你说这石头屋子里必有宝贝,宝贝却又在哪里?''白羊瞧着小鱼儿,道``这小子就是宝贝。''

黄牛道:``这小子一身嫩肉,若是李大哥在这里,倒可以趁热饱餐一顿,但你这只会嚼草的老山羊,还想拿他怎样?''小鱼儿瞧见这白羊,心里本在发愁,听到这话,精神立刻一振,愁怀大解,突然嘻嘻一笑,道:``老牛老羊,你们近来好么?''黄牛怔了怔,道``这小子认得咱们。''

小鱼儿笑道``闲暇之时,我常听大嘴兄说起,十二星相中,就数黄中最勇,白羊最智,不想今日竟在这里瞧见你们!。''黄牛哈哈大笑道:``过奖过奖\ldots\ldots{}''``突然止住笑声,瞪大眼睛,道''你\ldots。你怎会认得我李。\ldots 李老哥。``他这次不但已将''大哥``改成''老哥``,而且''老哥"这两字说出来时,说得有些结结巴巴。

小鱼儿眼珠于一转,道:``但大嘴兄对我说起时,只说十二星相中有个黄牛乃是他的后辈,听你唤他老哥,莫非是那黄牛的叔伯。''

黄牛红着脸一笑,道``我\ldots。我就是黄牛。''

小鱼儿道:``既是如此,虽在背后,你也该称他大叔才是,你胡乱改了辈份,若是被他知道可不高兴的。''黄牛满脸笑道``是,是,小兄弟,你千万莫要告诉他\ldots。他老人家。''小鱼儿扳着脸道``这小兄弟叁个字,也是你叫得的么?''黄牛道``是是是,我\ldots\ldots 在下─一─''

白羊突然冷笑道``你在下若非跟着我出来,就算被人卖了,还不知是被谁卖的。''黄牛眼睛一瞪,道``这是什么话?''

白羊道``你真相信这小子是李老前辈的小兄弟?\ldots\ldots 哼他年纪简直连李老前辈的儿子都嫌太小了。''黄牛摸了摸头,道``但。\ldots 但他说的倒也不错。''白羊道"呆子,他说的话,有哪句不是你自己卖绘他的\ldots。

请问,他若真是李老前辈的兄弟,哪会在这慕容山庄里。``黄牛道''他\ldots。他只怕被慕容那丫头关起来的。``白羊冷笑道:''这两间屋子是做什么用的,你难道还瞧不出,慕容那丫头又不是疯子,怎会将人关在炼丹藏宝的密室里,这小子既然能在这里慕容家的丹药藏在何处,他必定知道,所以我说他就是个宝贝。``黄牛又摸了摸头,瞧着小鱼儿道''好小子,我还在替你辩驳哪知你却是个小骗子。``小鱼儿冷笑道:''这屋子难道规定是要炼丹藏宝的么?不炼丹时,关人难道不可以?慕容那丫头又不是疯子,这屋子若有藏宝,她又怎会灌一屋子水。``黄牛拍掌道''是呀,不错呀\ldots\ldots 譬如说我这双手,虽可以摸女人的小脸蛋,但也可以打人的耳掴子,炼丹的屋子,为什么就不能关人。``小鱼儿道:''你年纪也和大嘴兄相差无几,但却是他的后辈,我年纪虽和他相差多些,为何就不能是他兄弟。``黄牛再摸了摸头,瞧着白羊道''是呀,他说的不错呀,咱们龙大哥的妹子,岂非也只有十来岁!``白羊冷笑道''世上若真有活了四五十岁,还要上孩子当的人那人就是你,但我\ldots\ldots 哼,他若要我相信,除非\ldots。``小鱼儿招手笑道''你过来,我让你瞧件东西。"他此刻仍水淋淋地躺在地上,白羊方自走到他面前,小鱼儿身子突然一滑,双手双腿连续击出四拳叁脚。

这四拳叁脚几乎是在同一刹那间击出来的,世上唯有一个躺在地上的人,才能将双拳双腿同时击出,世上也唯有李大嘴才练得有这种招式,只因这种招式听来虽厉害,其实却不实用,试问一个好好的人,怎会躺在地上和人动手,除非他是在装病诈死时,要向人猝然偷袭。

而世上除了李大嘴这样外貌老实、内心奸恶的人外,谁也不会挖空心思去创此等招式。

白羊大惊之下,整个人都跳了起来,不像是羊,倒像只兔子──若非小鱼儿已累得半死,他此刻就是只死兔子了。

小鱼儿盘膝坐起,笑嘻嘻道``你此刻相信了么?''白羊喘着气还未说话,黄牛恭敬作了叁个揖,道:``小爷叔..\ldots 无论你年纪多大,就算你刚生出来只有叁天,只要你是李大叔的兄弟,你就是我的小爷叔。''小鱼儿道:``老山羊,你呢?''

白羊目光闪动,仰起了头,缓缓道:``李老前辈在谷中过得还好么?''小鱼儿道:``好人不长命,他却死不了的。''

白羊阴恻恻一笑,道;``谷中的人,一个个俱都长命百岁,李老前辈自然也乐得在谷中享福,是不会再出来受罪的。''小鱼儿眼珠一转笑道``他本来是不会再出来的。''白羊一怔,道``现\ldots─现在呢?''

小鱼儿慢吞吞道:``现在,不但是他,就算是杜大哥、阴大哥、屠大姐\ldots\ldots 嘿嘿,他们若不出来,我又怎敢一个人在外面乱闯。''白羊面色登时变了,道``但。\ldots 但他们\ldots{}''

小鱼儿道:``他们在谷中闷了这许多年,每人又都练了身江沏中谁也没见过的功夫,你若是他们,你出不出来?''白羊垂首道:``是是,阁下\ldots\ldots 前辈可知他们现在\ldots。''他虽然低着头,但目光不住闪动,冷森森的不怀好意,小鱼儿瞧在眼里,微微一笑,道;``他们这些人做事素来神出鬼没,我也不知道他们的行踪。''白羊似乎暗中松了口气,但小鱼儿又已接着道:``说不定,他们现在就在你身后,你也未必知道。''白羊一口气立刻又憋了回去,想回头去瞧,又不敢去瞧。

黄牛却是喜笑颜开,道``若是李大叔真的来了,那就好了,慕容家那几个小丫头纵有叁头六臂,咱们也不怕她来报仇了。''小鱼儿淡淡道``伤们让她逃走了么?''

黄牛叹了口气,道:``咱们这一次虽是那条蛇约来的,其实咱们这些人自己又何尝不是早巳在动慕容山庄的脑筋。''小鱼儿笑道:``慕容家的灵药,确是叫人流口水。''黄牛苦笑道``只可惜慕容那丫头确是鬼灵精,也不知从哪里得知咱们要大举来犯,咱们还没来,她竟已溜了。''小鱼儿吃惊道:``溜了?''

黄牛恨声道``不但人溜走,值钱的东西也被搬得差不多干干净净,连大门也没有锁,只留下条子,说什么妄入者死,哼,简直是放屁''小鱼儿道``不错简直比屁还臭。''

他此刻已猜出慕容九妹是为何要走的了!

小仙女与铁心兰一心以为小鱼儿已溜走.急着去找,慕容九妹知道她们嘴里虽说得凶,心里却是软的,自然再也不肯说出小鱼儿已被关了起来,别人要她去找,她就跟着去找``。''小鱼儿想到这里,不禁又破口大骂道``那丫头不但比屁还臭简直比蛇还毒,你们烧了她的屋子,当真再好也没有,谁动手烧的.我可得请他喝两杯。''黄牛大笑道``放火的虽已走了,但咱们。\ldots{}''

小鱼儿笑道:``咱们却可喝几杯,不对,几百杯''\ldots\ldots 咱们一路走,一路喝,我带你们去找李大嘴,在路上瞧见顺眼的,还可以\ldots 哈哈,还可怎样,你总知道。``黄牛拍掌道:''妙极妙极。"

小鱼儿道``白羊,你呢?''

白羊道``这。\ldots 在下\ldots。咳\ldots\ldots{}''

小鱼儿道:``你若不愿去也没关系,等我遇见大嘴兄时,就说你不愿见他,也就是了。''白羊大叫道:``谁说我不愿去,黄牛,是你说的么?''一把推着黄牛道:``咱们还不走\ldots。咱们还等什么?''这叁人果然是一路走,一路喝,小鱼儿忽然发现.自己喝酒原来也是天才,居然像是永远喝不醉。

有时他简直有些奇怪,那许多杯酒喝下去后,到哪里去了?

他看来看去,也觉得自己没那么大的肚子。

那黄牛白羊两人,对他竟是百依百颁,吃喝歇住,全用不着他费半点心思,早有他两人为他安排得舒舒服服。

他要走就走,要停就停,黄牛白羊两人,也全不问他要到哪里去,``十二星相''中这两个煞星竟会对个孩子如此听话倒真是令人想不到的事。

一路上自然也遇着不少江湖人物,瞧见他们,有的远远行个礼就绕路避开,有的纵不认得他们,但瞧见这两人的奇形怪状,也远远就避之唯恐不及,又有谁敢来噜嗦生事?!

但入了雁门关后小鱼儿突然发现,前面的人瞧见他们,虽远远避开,却有不少人悄悄跟在他们身后。

他们走到哪里,这些人就跟到哪里,个个神情却都是恭恭敬敬,既不说话,也没有半点要找麻烦的样子。

小鱼儿再瞧黄牛白羊,面色竟全无变化,像是什么都没瞧见,小鱼儿也不说破,傍晚时到了剑阁,找了家客栈投宿,小鱼儿道:``大曲酒配麻辣鸡,虽然吃得满头冒汗,但越吃却越有劲。''黄牛大声笑道:``不错,大曲配麻辣鸡.妙极妙极。''平日小鱼儿只要一张口,黄牛白羊两人就动手将东西拿来了,但今日这两人嘴里虽说得好,身子却动也不动。

小鱼儿等了半晌,道``既然妙极,为何不去拿来''黄牛笑道:``从今日起,咱们不必拿了。''

小鱼儿道:``你们不去拿,难道要我去?''

白羊笑道:``怎敢劳动你老人家。''

小鱼儿道:``你们不去拿,又不去吩咐店家,这大曲酒与麻辣鸡难道会从天上掉下来,地下长出来不成?''

黄牛笑嘻嘻道:``你老等着瞧吧。''

小鱼儿在屋里踱了两个圈子只听门外``笃、笃、笃''敲了叁声,霍然拉开门,门外鬼影子却瞧不见一个,但地上却多了个大托盘,盘予里装着一喋麻辣鸡,一碟回锅肉,一碟凉拌四件,碟豆瓣鱼,一大碗老母鸡场,还有一大壶酒劳香甘冽,果然是道道地地的大曲。

小鱼儿眨了眨眼睛笑道``原来你两人还会王鬼搬远法。''黄牛笑道``这不叫王鬼搬运法,这叫孝子贤孙搬运法。''小鱼儿道``哦''

白羊道``这一路上跟在咱们后面的那些人,你老可瞧见小鱼儿笑道;''我只当你们没瞧见哩。``黄牛道''那些小子,就是咱们的孝子贤孙。小鱼儿道:``原来那些人是你们的门下。''黄牛道:``狗屁门下,我连认都不认那些孙子。''小鱼儿道``既不认得,为何要跟着你们。''

黄牛笑道``江湖中人都知道,只要十二星相在哪条道上走,哪条道上就必定有大买卖,这些孙子们自已不敢做大买卖,就总是跟在咱们身后,十二星相从来只取红货,不动金银,这些孙子跟在屁股后面,多少也可分一杯羹。''白羊道:``所以咱们十二星相无论走到哪里,哪里的黑道朋友总是大表欢迎,若有什么风吹草动,不用咱们自已探听,总有人来走报消息。''小鱼儿拊掌笑道``难怪十二星相不发则已,一发必中,原来并不是真的千手千眼,面是有这许多别人不知道的徒子徒孙。''黄牛大笑道:``但这一次,他们却上当了,平白孝敬了许多东西,却是肉包子打狗,有去无回,连血本都捞不回去。''白羊也大笑道:``但这是他们自已心甘情愿的,咱们乐得消受,也不必客气。''他们笑声虽大,语声却小得很。

这一路上自然走得更是舒服,无论他们想要什么,只要把声音说大些,不出片刻,自然就有人送来。

小鱼儿入关之后,竟不再东行,反面又转向西南,通绵阳、龙泉、眉山,竟似要直奔峨嵋。他居然像是认得路的,走到哪里只要问问那地方的名字,就知道方向,根本不向黄牛白羊问路。

蜀中风光,自然与关外草原不同,小鱼儿走得颇是高兴,蜀中的烈酒辣菜,更使小鱼儿一路赞不绝口。到了峨嵋,黄牛白羊一个末留意,小鱼儿竟一个人溜了出去,直到深更半夜时,才施施然回来。

黄牛白羊既不问他去了何处,小鱼儿也一字不提,到了第二日,他也不说走傍晚时又悄悄溜了出去。这样竟一连过了叁天,小鱼儿还不说走,黄牛白羊还是不闻不问,这两人的确已服了小鱼儿,简直比小鱼儿的儿子还听话,看来李大嘴虽然退隐多年,但在这些人心里,对他仍是畏如蛇蝎。

``十大恶人''的声名,果然不是好玩的。

第叁日午后,小鱼儿一个人又到市上兜了个圈子,只见大大小小的酒楼饭铺里,每一家都有几个江湖人坐着。十人中有九人只是在喝着闷酒,非但没有大声吵笑,简直连话都不说一句。

小鱼儿也不知道他们贵姓大名,这些人是黑道?是白道?是成名的英雄?还是无名小中?小鱼儿全不想问。

街道上不时还有些乌簪高髻、立服佩剑的道人走过,他们佩的剑又细又长,神情更是倨傲异常,既像是全不将别人瞧在眼里.但却又不时以锐利的目光去打量别人,他们既像是来市上散步闲逛的,面色偏偏又十分凝重。

小鱼儿知道这些道人必就是``峨嵋''门下,峨媚剑法之辛辣迅急号称天下无双,门下弟子的眼睛自然难免要生在额角头上.何况,这里就在峨嵋山下,正是峨嵋弟子的地盘,他们要在这里招摇过市,作虎视眈眈、巡逻查哨状,也只好由得他们,又有谁敢去管他。

小鱼儿逛了一圈,买了个香袋,又在西街口的卤菜大王那儿切了半斤蹄筋,一斤牛肉,才逛回客栈。

屋予里已摆了一桌配莱,黄牛白羊老老实实地坐在那里等,莱都快凉了,两人却连筷子都不敢动。

小鱼儿道:``这叁天来,你两人简直比大姑娘还老实,简直足不出户,街上热闹得很,你两人也不想瞧瞧。''黄中苦笑道:``瞧是想瞧的,但以我两人的名声,在这蛾媚山下,还是老实点呆在屋子里,太太平平地喝酒好。''小鱼儿道``峨嵋派的杂毛们真有这么厉害?''

黄中叹了口气,举杯道``咱们不说这些,来\ldots{}''小侄敬你老一杯。``小鱼儿却先将两包卤莱打开,笑道:''听说这卤菜大王用的是几十年的陈汤老卤,所以卤出来的莱,滋昧分外不同,你两人不妨先尝尝。``黄牛笑道:''有了孝子贤孙们送来的这许多莱,你老又何必多破费。``小鱼儿道''换换口味,总是好的。"

白羊道``长与赐,不敢辞!''果然夹了块牛肉在嘴里,一面大嚼,一面赞美,等他吃完了,黄牛已吃了五块。

小鱼儿喝了两杯酒虽无酒意,兴致却更高了,笑道``看来蛾嵋派的剑法,果真有两下子,江湖朋友到了这里,连话都不敢说了\ldots。我迟早要见识见识。''黄牛笑道``你老一出手,峨媚杂毛包准吓得满街走。''白羊眼睛盯着那香袋,道``你老莫非真的要上蛾媚山去。''小鱼儿道``我本想和你两人一齐去的,也好叫你两人开开眼界,但你们两人既然不敢露面.我只好一人去了。''黄牛道``你老准备什么时候上山?''

小鱼儿道:``明日清晨。''

黄牛叹了口气,道:``只可惜你老的计划要改变了。''小鱼儿皱眉道``为什么要改变?''

黄牛瞧着他一笑,笑容突然变得十分奇怪。

白羊阴森森笑道:``你这小杂种,你还不知道?''他称呼突然由``你老人家''变成小杂种``,小鱼儿倒当真吃了一惊,''啪"的一拍桌予,霍然站起,怒道:你这老山羊,你敢话犹未了,身子竟软软地倒了下去。

白羊咯咯笑道``小杂种,你现在总知道了吧''

小鱼儿倒在地上,道``酒\ldots\ldots 酒里有毒''

黄牛得意洋洋笑道:``我两人还生怕骗不倒你,所以跟你喝的是同一壶酒,只不过我两人早已服下了解药而已。''小鱼儿道:``你\ldots.。你两人为何要如此?''

白羊道"你只当咱们到慕容山庄去真是为了慕容家的丹药么.哼,那几个小丫头炼出来的药,还不值得十二星相劳师动众\ldots\ldots{}

黄牛道,"老实告诉你,咱们是找你去的\ldots\ldots{}

白羊道``现在普天之下,只怕已唯有你一人知道燕南天的藏宝所在,蛇老七为了要抓住你,早巳在慕容山庄四面都布下了眼线,一面飞鸽传书,将咱们找去,哪知咱们方到那里慕容那丫头竟鬼使神差地走了。''黄牛道``但你却留在庄子里,咱们进去找了一圈竟找不着你,一气之下,就放了把火将屋子烧了。''白羊道``屋子烧光了,咱们才瞧见那两间石室原来你这小杂种也不知为了什么得罪了人家,竟被人家关在水牢里。''黄牛道:``这也难怪,慕容丫头本就喜怒无常\ldots\ldots{}''小鱼儿听得唉声叹气,忍不住问道:``但后来为何只剩下你两人?''黄牛笑道``咱们早巳知道你这小杂种诡计多端,若是逼着你说出藏宝之处,说不定还会想出鬼主意,你若胡说八道,咱们岂非也只有跟着你乱转,一路上若是被你乘机溜了,岂非冤狂。''白羊道``但咱们的黄牛哥算准你只要一能走动,第一个要去的地方,必定就是燕南天的藏宝之处,所以他就做好了这圈套,要你上当。''小鱼儿瞪大了眼睛,瞧着黄牛,道:``是你想出来的主意?''黄牛道``想不到吧?''

\hypertarget{ux7b2cux4e8cux5341ux4e00ux7ae0-ux5c14ux5978ux6211ux8bc8}{%
\chapter{第二十一章
尔奸我诈}\label{ux7b2cux4e8cux5341ux4e00ux7ae0-ux5c14ux5978ux6211ux8bc8}}

小鱼儿中了黄牛、白羊在酒中放的迷药,身子无法动弹,只得叹口气,苦笑道``看来当真是人不可貌相,你这条笨牛居然也有一肚子鬼主意,我可真做梦也未想到。,白羊咯咯笑道''江湖中上过他当的人,真是数也数不清了,你这小杂种又不是头一个.你叹的什么鸟气。``小鱼儿道''但你又怎知我\ldots\ldots"

黄牛道:``你和狂狮铁战的女儿走在一起,自然和十大恶人有关系,我随意说了十大恶人中一个名字,你果然打蛇随棍上,自已往坑里跳。''小鱼儿苦笑道``这才叫歪打正着,算你走运就是。''黄牛道:``我知道你一瞧我两人如此容易上当,必定不会轻易放过的,必定要叫咱们跟着你做牛做马,你这小鬼若是良心好些,咱们反倒要想别的法子了。''小鱼儿叹道``我正也有些奇怪,十二星相是出名的坏蛋,怎会突然变得如此老实听话\ldots。唉!不想我竟也有阴沟里翻船的时候。''黄牛大笑道``你这小鬼自以为已经很聪明了,是么?告诉你,你若想在江湖中混,你还差得远呢''白羊道:``咱们十二星相是何等人物,若不是骗着你玩,又怎会对你这样,哼就算李大嘴自己来了,咱们也不过只是拿他当做个屁。''黄牛道:``咱们本想等你找着那藏宝之地后再拿你开刀,哪知你这小鬼果然滑溜,咱们竟看不住你,所以只好请你喝两杯迷魂汤了。''白羊道``反正咱们此刻已知道那藏宝必定就在峨嵋山,还离已不远了,也不怕你这小鬼再玩花样。''黄牛狞笑道;``你若是好生说出那藏宝之地,说不定大爷一开恩或许饶了你你不是个笨人,想必不会自找麻顿,冤枉多受些活罪。''小鱼儿眼睁睁瞧着他们,突然大笑起来,笑得居然开心得狠,得意得很,白羊大怒道:``小杂种,你只道咱们没有叫你说实话的本事么''小鱼儿笑道``老杂种,你只道我真的上了你们的当么?''黄牛笑道``你还有什么鬼主意,说吧。''

小鱼儿叹了口气,道``我说是愿意说的,只怕你们还未听完,就呜呼哀哉了。''黄牛还是笑嘻嘻道:``真的么?''

小鱼儿也笑嘻嘻道``假的,那包牛肉里没有毒药,一点毒药也没有。''他话未说完,黄牛白羊已再也笑不出来白羊一把拉住他衣襟,变色道:``小杂种,你说什么?小鱼儿笑道,''我说我是个呆子.虽然明天就要去寻宝了,虽然不能让你们跟着,但我还是舍不得毒死你们,所以没有在牛蹄筋里下毒。``他越说没有,白羊面色越是害伯,颤声道:''你。\ldots 你。\ldots 快将解药拿来!``小鱼儿笑道''是是是,我应当将解药拿给伤们,然后等你们来害我\ldots{}``哈哈,莫要忘了,你们要我寻宝,不敢毒我,但我可没有要你们寻宝,难道也不敢毒死你们,哈哈,莫忘了迷药是会醒的,毒药却要人的命。''黄牛居然又笑了,笑嘻嘻拉开白羊的手道``是是是,咽们是呆子,什么都不懂你说咱们中了毒,咱们就真的以为自已中了毒了。''小鱼儿笑道:``当然当然,伤们千万莫要相信,现在你们若是摸一摸第五根肋骨下的乳根穴旁边,那里保险一点毛病都没有,你们也不必摸吧。''他``不必摸''叁个宇还未说完,黄牛白羊两个人的手已不由自主往第五根肋骨下``乳根穴''旁摸了过去。

两人不摸还罢,一摸之下脸色登时变得比墙还白,两人你望着我,我望着你,再也动弹不得。

小鱼儿笑道``没关系,那里虽有些发麻,但两叁盏茶工夫里你们还是死不了的,你们还来得及先杀了我。''他虽然叫他们杀他,但此刻就算再借给他们个胆子,他们也不敢动手,小鱼儿死了,谁给他们解药白羊道``你\ldots 你究竟是怎么样?''小鱼儿笑道:我若是你们,此刻就该乖乖地先将我老人家中的迷药解了,再拍拍我老人家的马屁,让我老人家出出气,然后再发下个金誓,从此永远听我老人家的话,绝不敢丝毫违背黄牛嘎声道:``我若解你的迷药,你不解咱们的毒又如何?''小鱼儿道:``是是是,你不解我中的迷药,我反会替你们解毒了。''白羊黄牛对望一眼,突然向小鱼儿走过去。

小鱼儿悠悠道:``世上有些毒药,是没有现成的药可解的,而且,除了下毒的人之外,谁也不知道那毒性究竟如何,但你们若是不信,不妨试试也可以。''黄牛白羊停住了脚,再也不敢走一步,叫他们拿别的来试都可以,叫他们拿自己性命来试,他们可没这么大的胆子。

两人心中同时忖道``咱们发过誓,服下解药后,难道就不能宰了他么,发誓在咱们说来,岂非比吃白菜还容易。''两人再不说话,一齐跪了下去,发了个又重又毒的誓,恭恭敬敬,将解药喂入了小鱼儿的嘴里。

别的事都可以等,要命的事是等不得的过了半晌,小鱼儿果然已能站起,拍了拍衣服上的土,笑道:``十二星相的解药果然都灵得很。''黄牛干笑道``你老人家的解药想必更灵。小鱼儿道:''什么解药?``白羊黄牛好像被人在肚子上踢了一脚,失声道:''你\ldots{}``你小鱼儿大笑道''莫要着急,我是骗着你们玩的。``他笑嘻嘻自怀中摸出个小瓶子.道:''解药其实在我身上,你们方才为什么不来搜搜\ldots。唉,人有时的确不该太相信别人的话。"白羊黄牛又气又恨,恨不得一手把这小鬼捏死,但还是救命要紧,黄牛先抢过解药,一下子就倒进嘴一大半。

白羊变色道:``你\ldots。你为何服这许多?''

黄牛笑嘻嘻道:"我块头大些,理当多吃些。

白羊狠狠夺过瓶子,将瓶里的药全吃了下去,然后两人瞧着小鱼儿,心里却在想小杂种,瞧你再往哪里跑。

小鱼儿也瞧着他们,道``再摸摸那里还疼不疼''两人一摸,果然不疼了。

白羊笑道``这毒药解得好快!''

黄牛狞笑道``现在你\ldots\ldots{}''往哪里跑``四个字还未说出,小鱼儿突又大笑起来,道''方才我叫你们摸时,那里正是你们气血交流处,纵然轻轻一触,也会又麻又疼现在气血已流过那里,自然不疼了"这下子两人又被气得目瞪口呆,肚子都快被气破了。

白羊嘶声道``小杂种,原来你在骗人。''

小鱼儿笑嘻嘻道``不错,我正在骗你这老杂种,你们也不想想,牛肉又不是我煮的,我怎么下毒?何况,我若真下了毒,为何不将你们毒死''黄牛突也大笑道``算你聪明,但咱们可也不是呆子,告诉你,那迷药虽解,但半个时辰内,你还是无法动用真气,我举手便可取你性命。''小鱼儿道``哦,真的么?''

黄牛狞笑道``假的,我怎舍得宰了你,我只不过要割下你一只耳朵,半个鼻子,砍断你一只手,一条腿。''小鱼儿道:``哎呀,我好怕呀!''

黄牛道:``你不必害怕,我不是李大嘴,不会吃你的,我只不过要把你的肉拿去喂狗。''口中说话,一步步向小鱼儿走了过小鱼儿瞧也不瞧他,口中低低念道:``一、二、叁、四、五、六、七.''他念到``七''字,黄牛巨灵般的手掌已直劈过来,小鱼儿还是动也不动,根本不睬他。黄牛一掌劈出,也不知怎地,身子竟突然摇了起来,面色也变了,突然一个倒裁葱,直挺挺倒了下去。只见他眼睛发直,口吐白沫宛如中了邪─般。

白羊大惊道``这\ldots\ldots 这是怎么回事?''

小鱼儿笑道,``也没什么,只不过牛肉里虽无毒但那解药里却是有毒的,他抢着要多吃些。自然就先例下去。''白羊怒吼一声,飞扑而起,但身子方自扑到空中,就像是根木头似的掉了下去,脑袋立刻肿起了一块。

小鱼儿拍掌笑道:``这下子可变成独角山羊\ldots\ldots{}''笑声未了突然窗外一人叹道``活了这么大年纪,却被个小孩子玩弄于掌股之上,你们这一条羊、一条牛以后还能再见人么?''小鱼儿惊道``什么人?''

只见窗子开了一线,一个人蛇一般自窗缝里滑了进来,全身碧油油的又腻又滑,赫然正是那碧蛇神君小鱼儿眼珠子一转,笑道``好久不见呀,你好吗?坐下来喝杯酒吧。''碧蛇神君阴恻恻笑道:``告诉你,他们在酒中所下的迷药,乃是我独门炼制,这迷药的药性.天下再无一人比我清楚,你纵然想拿话来拖延时间,也是无用的,我就算再让你说一百句话,你还是休想动用真气。''小鱼儿叹了口气道``如此说来,我今天总是劫数难逃,是倒霉定了?''碧蛇神君道``正是''

只听白羊黄牛两人同时哼起来,他两人眼睛还瞧得见,怎奈全身肉都硬了,四肢既不能动想张嘴说话都不行,这迷药可要比碧蛇神君炼制的厉害十倍,碧蛇神君瞧了一眼,也不禁微微变色道``半人半鬼的僵尸散''小鱼儿笑道;``算你还有些眼力,这两位仁兄吃得还生怕不够多半个时辰中,只怕就要变成僵尸,虽然死不了,但以后也只能跳着走路了\ldots\ldots 哈呛,一只羊一只牛满街乱跳,想必好看得很。''黄牛白羊听了这话,头上已往外直冒冷汗,哼的声音更大,碧蛇神君转首瞧了他们一眼,道``两位仁兄可是要小弟先救你们。''黄中白羊拼命点头,头也不过只是微微动了动。

碧蛇神君阴恻恻笑道"一份藏宝,叁个人分不嫌太少了么,何况两位本说好这一路上要给小弟留下标记,但标记又在哪里?

若非小弟早巳知道两位的为人,早巳令人混在那些孝子贤孙中跟来,此刻又怎找得到两位?``黄牛白羊额上的冷汗已比黄豆还大,目中已露出惊恐之色,碧蛇神君目光闪动,纵声长笑道''两位就喜欢装神弄鬼,如今真的变作僵尸,岂非更是有趣``突然顿住笑声,向小鱼儿走了过来小鱼儿笑道:''你若要点我穴道,下手可要轻些,我现在既不能运气相抗,你若一指将我点死,可就没戏唱了。``碧蛇神君狞笑道:那么,我不点伤穴道就是,我只叫碧丝轻轻咬你一口,你非但不会觉得痛,还会觉得痒痒的,酸酸的,那滋味可比抱着女人还舒服。''语声中,只见一条碧光闪闪的小蛇,自他衣袖中滑了出来,蛇身虽只有蚯蚓般大小,但红信闪缩,滑行如风,却足以慑人魂魄!

小鱼儿纵是胆大,此刻面色也不禁变了。

那碧蛇神君衣袖中竟似有个蛇窟,瞬息之间,便有十几条细如蚯蚓、长如筷子的碧丝蛇,接连滑了出来,有的滑上小鱼儿的脸,有的滑上他的脖子,有的滑进他靴子里,还有的竟滑入他衣襟十几条又冷、又滑、又腻的小蛇在自已肉上乱爬,那滋味可真不是人受的。

小鱼儿全身都麻了,纵有力气,也不敢动一动。

碧蛇神君伸出拇、中两指,道``我手指只要轻轻一弹,你便立刻跌入温柔乡里,嘿嘿,十几个女人一齐抱着你,那种销魂蚀骨的滋味,除了你别人也无福消受。''小鱼儿叹道``抱女人若是这样的滋昧,就难怪聪明人都要去当和尚了。''碧蛇神君狞笑道:``你此刻还未尝着,怎知\ldots。''小鱼儿大叫道``拜托拜托,这滋味我也无福消受。''碧蛇神君道``依可是告饶了?''

小鱼儿苦笑道``你要去哪里,我带你去就是。''碧蛇神君目光闪动欢喜得连声音都哑了,道:``那藏宝之地可是真的就在这峨嵋山上?''小鱼儿道``半点也不假。''

碧蛇神君咽了口口水,道:``如此说来,今夜我便可瞧见那批宝藏了。''小鱼几道``你不但可以瞧见,还可以带走。''

碧蛇神君一跃而起,道``既是如此,走吧。''

小鱼儿道``走?\ldots。这\ldots\ldots 这些蛇?''\ldots"

碧蛇神君大笑道``我肯让这些蛇美人抱住你,你真是天大的福气。''小鱼儿苦着脸道:``但有这些小美人抱住我.我哪里还有走路的力气?''碧蛇神君道:``我自知看不住你,只有请她们代劳,只要你乖乖的,她们也必定温柔得很,但你的手若是乱动,她们的樱桃小口只要轻轻咬上你一口,嘿嘿,哈哈''\ldots 突又大笑起来,笑得也不知有多么难听。

小鱼儿只有乖乖地站起来就走,非但不敢乱动,简直连咳嗽也不敢咳嗽一声,他平生也没有如此听话过。

走出门,还可以听见黄牛白羊两人在地上哼哼,那声音像是哀呼、求饶,又像是在咒骂,纵是铁石人听了,也难免要动心。怎奈碧蛇神君的心竟比铁还硬,根本像是没有听见,小鱼儿更是泥菩萨过江,自身难保,哪里还管得了别人。

对面一个店伙走过来,躬身笑道:``少爷你\ldots\ldots{}''话未说完,瞧见小鱼儿的脸,大叫一声,顿时被骇得晕了过去,就像是瞧见活鬼似的。

小鱼儿苦笑道``我现在模样想必好看得很,耳朵上接着两条蛇,脖子里绕着两条蛇,手腕上盘着两条蛇,还有条蛇塞在鼻孔里,耳环、项链、手镯,都全了,他日若有机会,我倒要将这副首饰送给慕容九妹。''他一个人自言自语,碧蛇神君也不理他。

小鱼儿又道``其实那幅藏宝图画得并不十分详细,我花了整整两个晚上,才算将地方摸清,不想却被你捡了便宜。''碧蛇神君道,``那入口是在前山?还是后山?''小鱼儿道``后山''

话未说完,已有一块黑布蒙住了他的头。

碧蛇神君冷冷道``从这里到后山,用不着你领路,你若聪明,就乖乖的跟我走,若想故意招摇过市,引起别人的注意,这心思就白费了。''小鱼儿暗中叹了口气,口中却笑道:``我为何要引起别人的注意?这世上我只有仇人哪有朋友?''碧蛇神君叱道;``闭嘴''

小鱼几叹道``连话都不能说么?\ldots{}''他就像是个瞎子似的,被人牵着走此刻又变成了个哑巴。

碧蛇神君走得快,他只有走快,碧蛇神君走得馒,他也只有走慢,至于已走过什么地方,他全不知道。

走了顿饭工夫,人声渐寂,风渐凉,小鱼儿的手突然被人一拉,像是被拉入一个草堆树丛里。

小鱼儿心念一转,暗道:``这□莫非瞧见了什么他害怕的人碧蛇神君摸在他耳旁沉声道''一出声就要你的命``这句话才说完,约摸六七丈外已有个语声响起''铁心兰这丫头怎地到了这里就突然不见了"娇脆的语声,每说一个字,小鱼儿的心就跳一下──这竟是小仙女的声音,她怎会也到了这里?

接着,就听到另一人道:``只怕她已发觉了我们。''这语声冷漠优美,竟是慕容九妹的。

小鱼儿的心立刻像是打鼓般跳了起来,平时他若知道这两人就在附近,逃得生怕不够快。

但此刻,他却只希望这两人快些走过来,越快越好,他忽然发现这两人虽是他的仇人,却也可算是他的亲人。

只听小仙女道:``咱们一路跟着她,她半点也没发觉,到了此地又怎会突然发觉?瞧她那副痴痴迷迷的模样,心里只有那小鬼,眼里也只知去找小鬼,就算有一队人跟在她后面,她也不会发觉的。''慕容九妹淡谈道``既是如此,你还怕找不着她?''小仙女道``我只怕\ldots{}''只怕\ldots\ldots"

慕容九妹冷笑道``你只怕找不着那小鬼.是么?''小仙女道``对了,我真怕找不着那小鬼─\ldots 真怕不能将他的心挖出来,瞧瞧那究竟是什么颜色?''慕容九妹道``不用瞧你也该细道\ldots。黑的\ldots{}''

语声非但没有走近,反面渐渐远了。

小鱼儿真恨不得大声叫她们回来,但他也知道自己只要一出声,那些蛇美人的``樱桃小嘴''就要一齐咬下来,他可吃不消。

他只有忍着,只要留着命在,什么事总有法子的。

听了她们的话,他已猜出慕容九妹与小仙女必定是先故意将铁心兰放了,然后再一路悄悄跟踪而来。

这是个又简单、又古老的计谋,而这种计谋却偏偏最容易令人上当,但铁心兰,她此刻又到哪里去了?铁心兰到这里自然不是为了那宝藏,她只不过要在这里等小鱼儿,她知道宝藏就在峨嵋山也知道小鱼儿必定会来的,但慕容九妹亲手将小鱼儿关人石牢,自然认为小鱼儿绝对来不了,那么,她为何要来这里?难道这冷漠无情的女人,对这宝藏也有贪念不成?

小鱼儿眼珠直转,怎奈什么也瞧不见,什么也猜不出,只觉碧蛇神君又凑了过来,小鱼儿眼前一亮,黑布已被掀了起来,虽然是深夜,但这一夜的星光夜色有似分外明亮,分外可爱。

小鱼儿不觉也长长松了口气,道:``我现在才知道,做瞎子的滋昧可真不好受。''

\hypertarget{ux7b2cux4e8cux5341ux4e8cux7ae0-ux9634ux9519ux9633ux5dee}{%
\chapter{第二十二章
阴错阳差}\label{ux7b2cux4e8cux5341ux4e8cux7ae0-ux9634ux9519ux9633ux5dee}}

峨媚山山势险峻,正是``高出五岳,秀甲九洲'',尤其是后山,抬头望去,只觉万丈危崖似将临压而下,令人神魄惧为之飞越。

这里正是峨嵋山景最最荒凉的一环,上山不久,但有浓浓的烟霞自脚底生出,到了半山,人已在云雾里。

小鱼儿虽想展开身法,将碧蛇神君摆脱,但有十几条蛇盘在身上,又有谁能走得快,一个时辰后,两人都已在喘气了。

碧蛇神君喘着气道``到了没有?''

小鱼儿道``你还嫌慢么,若是没有我带路,就算你知道这地方,找上七天七夜,也休想找得到。''碧蛇神君突然笑道``你实在是个很能干的孩子,实在比我能干得多。''小鱼儿笑道``这就对了,在没有寻得那宝藏之前,你还是拍拍我马屁的好,等找到宝藏之后,你再将我千刀万剐也不迟。''碧蛇神君柔声道:``你放心,等找到了宝藏,我更不会杀你,我一定会好好的待你,你''\ldots\ldots{}``突然大吼道:''小鬼,出来\ldots。出来\ldots\ldots"原来他说的正得意,小鱼儿竟已不见了。

刹那间碧蛇神君已满头冷汗,大吼道``你若再不出来,我只一声尖哨,你就得死无论你逃到哪里,也是没有用的''夜雾深沉,小鱼儿连影子都瞧不见。

碧蛇神君急得跳脚,又道:``我那碧丝蛇又叫附骨之蛆,着无我的号令,一辈子都要缠着你,直到你死为止,你仔细想想,这样做划得来么''突听身旁``噗嗤''一笑,道``我就在这里,你着急什么?''碧蛇神君瞧了半天,才瞧清那里竟有个洞穴,山藤一条条垂下来,就像是一层层子似的。

小鱼儿不知何时已钻入洞里,又笑道,进来吧,这里就是那宝藏的入口。``碧蛇神君本来满腹怒气,听见这话,火气全没有了,俯身钻了进去,但觉一股阴寒之气扑面而来,他竟不由得机伶伶打了个寒噤,叹道:''也真亏那燕南天找得到这种地方\ldots\ldots{}``小鱼儿道''若不是这种地方,那宝藏还会等着你来拿么?``碧蛇神君展颜笑道''不错,如此幽秘之地,除了有燕南天自己画的地图之外,只怕真的连鬼都找不到\ldots\ldots 燕南天呀燕南天,你花费这许多心血,寻得如此幽秘之地,却不知到头来宝藏还要落在别人手中的"此地既是如此幽秘,那宝藏之珍贵自也可想而知,碧蛇神君想到这里,不禁更是得意,连冷都不觉冷了。

洞穴内伸手不见五指,碧蛇神君燃起了个小小的火折子,火折虽小,光度却甚强,他开怀笑道``你瞧我这火折怎样老实告诉你,为了此行,我已准备许久了,这火折乃是花了叁百两银子向那老火鸦买的,就是燃上个一天一夜,也不会熄灭\ldots\ldots{}''话还未说完,火拆子已突然灭了。

小鱼儿笑道``哦,这火折子原来不会灭的。''

碧蛇神君恨声道:``好个老火鸦,连我的银子也敢骗。''小鱼儿道``这也不能怪他,只怕是你牛吹得太大,连火折子都被你吹灭\ldots\ldots{}''脚下突然踩着样东西,身子踉跄冲向前,碧蛇神君也惊呼了一声,接着,火折又亮起,但火折亮后,两人惊呼之声却更响,眼睛也发了直洞中地下,竟卧着叁具死尸这叁具死尸衣衫华丽手里握着的剑青光闪动,竟似名器,但叁人尸身蜷曲,死得却极惨伸手一探,叁人手足虽已冷,但尸身还是软绵绵的,显见他们死时距离此刻最多也不会超过一个时辰。

碧蛇神君再扳过他们的脸瞧了瞧,他的脸立刻也变得和这叁个死人差不了多少.拿着火折子的手也发起抖来。

小鱼儿忍不住问道``你认得他们?''

碧蛇神君道``金''\ldots 金陵叁纫,其利断金!"

小鱼儿耸了耸肩,展颜道:``反正这叁人已经死了,咱们何必再去多想。''碧蛇神君怒道``他们虽死了,但杀死他们的人却必定还在洞里这人能在刹那间将金陵叁剑一齐杀死,岂非更是怕人''小鱼儿道:``奇怪,他会是谁呢?他怎会知道这秘密?''碧蛇神君咬牙道:"你难道不知道?这难道不是你告诉他的?

燕南天苦心藏宝,地图自然只画了一张,这唯一的一张就在你手里,除了你``\ldots{}''语声未了,手里的火折子突然又灭了。

碧蛇神君这次自然已知道暗中有人做了手脚,倒退叁步,紧贴着冰冷的石壁。

黑暗中一人缓缓道``你猜得不错,杀死金陵叁剑的人确还在洞里.那人就是我''这话声平和缓慢听来完全没有什么奇突之处但也就因为这语声太过平凡,在这阴森诡秘的洞中听来,反而更是可怕。

碧蛇神君这样的角色,竟也不觉打了个寒噤,道;``你\ldots\ldots 你是什么人?''那语声道``你可想瞧瞧我是什么人?''

碧蛇神君咬一咬牙,又将火折亮起。

火光闪动间,只见一个灰衣人缓缓自洞里走了出来,脸上也是灰蒙蒙一片,瞧不见鼻子眼睛,什么都瞧不见,他整张脸就像是个发白的柠檬,那真的要比世上所有丑怪的脸都要可怕十倍。

小鱼儿虽然知道此人面上必定蒙着面具,心里还是忍不住直冒寒气,他蒙着鼻子嘴巴倒也罢了,却为何连眼睛也一齐蒙住?眼睛蒙住了,为何还能在这里行动自如?做瞎子的滋味小鱼儿方才尝过了的。

只见碧蛇神君额角之上又在往外冒汗,道``你\ldots\ldots 你是灰蝙蝠?''灰衣人淡淡笑道``你瞧清楚了么?''

碧蛇神君道"那猫头鹰莫非也\ldots\ldots,一句话未说完,身子突然定住.整个人都像变成个石像,高举着火把的石像,只有一粒粒汗珠自那发青的脸上流下,砰的一声倒了下去。

小鱼儿慌忙接过火把,已瞧见一人自他身后走了出来,这人看来也没有什么奇怪,只是眼睛大得怕人,亮得怕人。

灰衣人微微笑道``灰蝙蝠既然在此,猫头鹰自也不会远的,以后你和前面的人说话时,切记莫忘了留意身后。''那双猫头鹰一般的眼睛,瞪着小鱼儿,咯咯笑道``我真想问问你们,是怎么找到这里来的?''他不说话倒也没什么,这一说话,果然名符其实,正如枭鸟夜啼。

小鱼儿眨了眨眼睛,道;``不是你告诉我的么?''猫头鹰一征道``我告诉你的?''

小鱼儿道``燕南天的藏宝秘图只有一张,不是你告诉我们的,我们怎会找到这里你还要我们帮你的忙,将灰蝙蝠害死,让你一人独吞宝藏,你为何又食言背信?难道你又约了些别的帮手不成?''他瞪着眼睛,叉着腰说的当真是活灵活现。

那猫头鹰脸都气得变了颜色,怒叱道``你小小的年纪,便学会血口喷人,长大了岂非比你师父们还要恶毒!''小鱼儿道``对了,你赶紧杀了我吧,杀了我也好灭口''猫头鹰喝道;``某家正要杀了你为世人除害!''喝声中双掌齐出,十指有如鹰爪,直取小鱼儿胸膛咽喉小鱼儿动也不敢动他实在有点怕那些蛇美人的``樱桃小口'',眼见这一双鹰爪抓来突然人影一闪,那灰蝙蝠已挡在他面前,道``对小孩何苦下毒手?''猫头鹰硬生生收回掌势,变色道``你为何阻止我出手?莫非你真相信了这小鬼的话?''灰蝙蝠淡淡道;``我只是有些奇怪,藏宝图明明只有一张,明明只有你我两人知道,这些人却又怎会来的?''猫头鹰嘶声道:``我与你相交二十中你难道还信不过我?''灰蝙蝠道``瞎子时常受人欺负,疑心病自也难免重些。''猫头鹰跺脚道;"好!想来必是你想独吞宝藏,所以借着这题目,要向我出手,我早己听说瞎子最是难缠,只恨我不听人言,你语声未了,灰蝙蝠已挥掌灭去了火光。

小鱼儿赶紧退后叁步,只听猫头鹰一声惊呼,道:``好好,你真下毒手''接着便是一连串掌风拳击。

小鱼儿暗道:``猫头鹰蚜猫头鹰你还活得了吗?''他算准灰蝙蝠既是瞎子,在黑暗中必定有独特的功夫,猫头鹰纵能在暗中视物,出手时也要先吃个大亏。

只听``喀嚓,喀嚓''几声骨节折断声,猫头鹰惨呼道``你\ldots\ldots 你总有一日要后悔的!\ldots{}''说到最后一字,又是一声闯哼,便再无声息。

然后,灰蝙蝠平和的语声又自响起,一字字道``小娃儿你在哪里?''小鱼儿屏住呼吸,更不敢动了,他知道灰蝙蝠杀了猫头鹰与碧蛇神君后,第二个目标便要轮到自已。

灰蝙蝠的呼吸也渐渐平静,柔声道:``小弟弟你为何不说话呀?你揭破了他的奸阴,我正要谢谢你。''语声中,他脚步竟已向小鱼儿站着的方向移动过来,瞎子总有一种异于常人的触觉,小鱼儿纵然屏佐呼吸,但在这阴森的洞穴中他身上因紧张而散发的热气,已足够将灰蝙蝠引了过来。

只听那脚步声越来越近,小鱼儿满头大汗滚滚而下,靠着石壁的衣衫,也已完全湿透灰蝙蝠柔声道``原来你在这里,你怎么不赶紧跑呀?''小鱼儿紧紧咬着嘴唇,汗珠自沿着他鼻梁流下,他脸上痒得要命,但他连抓也不敢抓,他一生都没有如此害怕过。

只觉灰蝙蝠的手掌已渐渐向他伸了过来,小鱼儿全身的肌肉都绷紧了,却仍然动也不动。

突然一声惊呼,衣袂带风``呼''的一声后退数步,颤声道:``你。\ldots 你颈子上\ldots\ldots{}''原来他手指方自点向小鱼儿的咽喉,缠在小鱼儿头上的毒蛇就给了他一口,别人虽瞧见小鱼儿身上的毒蛇怎奈灰蝙蝠究竟是个瞎子,又怎会料得到有此一着小鱼儿笑道"如今你可尝着我护身蛇神的滋味了么?哈哈!

就凭你这瞎子也想杀我,哪有如此容易"

灰蝙蝠嘶声道``蛇\ldots{}''毒蛇\ldots\ldots"

呼声中发狂般冲了出去,但脚步声还未走出十步,便又听得``砰''的一声,他人己跌倒。小鱼儿又惊又喜,喜的自然是对头已死,惊的却是这``碧蛇神君''所养的毒蛇实在厉害!

他长长吐了口气,喃喃道``唉!本来只要害我的毒蛇,此刻反救了我命,天下的事,有些当真奇怪得令人再也想不到。''他身子软软的,像是已虚脱,要知他方才实是生死一发,他实在是在拿自己的性命来和灰蝙蝠打赌除了小鱼儿这样的人外,又有谁会有如此赌法!

他摸索着去找碧蛇神君的火折子,但手又不敢乱动,这些``蛇美人''的厉害,他已见识过。他不由得轻轻叹息着道``附骨之蛆,若是弄不掉它们,真不如死了算了!''突然问,远处火光闪动,一条锦衣虬髯大汉,高举火把,昂然而入,虽然走在这种阴湿的洞穴,气概仍然不可一世。

小鱼儿自然又吃了一惊,他见了小鱼儿,又见到这满地尸身,面色更是大变,后退叁步,举掌护胸,厉声道:``你是什么人?''小鱼儿眼珠子一转,道;``你是什么人?''

那锦衣大汉厉声道:``你连某家都不认得,还能在江湖中走动么''小鱼儿笑道;``如此说来,你倒像是有些名气!''锦衣大汉喝道:``某家便是西河十七家镖局的联盟总镖头,气拔山河铜拳铁掌震中洲赵全海,这名字你想必定是听过。''小鱼儿微微笑道:``这名字倒长得很,听来倒也威风,但你不知本座是谁?''锦衣大汉赵全海冷笑道:``你算什么东西!''

小鱼儿也冷笑道:``本座便是万蛇之圣,万剑之尊,万王之王,打遍叁山五岳,南七北六十叁省无敌手,惊天动地玉王子你可听过这名字?''他一口气说出这一长串名字,赵全海倒真被唬得怔住了,道,``某家从未听过江湖中有这号人物!''小鱼儿道:``你从未听过,回去问问你师父他想必是知道的,江湖中老一辈的人物,见到我谁敢不低头!''赵全海怒道:凭你这乳臭未干的黄毛小子,也敢如此胡言乱语。某家儿子都比你大得多。``小鱼儿道:''你可知武功修练至登峰造极,便可返老还童。"赵全海又怔了怔,凝目瞧着他,显见已是半信半疑。

小鱼儿道:``今日我杀的人已够多了,再也懒得出手,念在你看来还是条汉子,你快快走吧,本座饶了你。''赵全海怒喝道:``就凭你也想将某家吓走?''

小鱼儿冷笑道:``你且瞧瞧地上死的是些什么人物?''赵全海俯首望去,变色道:"金陵叁剑?\ldots\ldots 灰蝙蝠、猫头鹰?

还有一个。\ldots."

小鱼儿道:``十二星相中的碧蛇神君你不认得?''赵全海倒抽一口凉气道:``他\ldots\ldots 他们难道都死在你手上?''小鱼儿淡谈道:``那也算不得什么?我只问你武功比起这些人如何?''赵全海怔了半晌,挺胸道:``在下费了千辛万苦,方到此间,前辈若要在下这样走了,在下实是心有不甘。''他虽还不走,但不知不觉间已改了称呼。

小鱼儿微微笑道:``你要怎样?''

赵全海道:``只要让在下见识见识前辈的武功,在下拍手就走,绝无留恋。''他生相虽然鲁莽,行事倒也精细,显见成名并非幸致。

小鱼儿神色不动道:``你想见证见证本座武功?那也容易,只要你能将我身上的这些毒蛇全都弄死,而不损及本座毫发,本座就将宝藏让给你也无妨。''赵全海目光闪动,道:``真的?''

小鱼儿道:``前辈对晚辈焉有戏言?''

赵全海大步迈过去,目光眨也不眨地凝注着那些蛇头,小鱼儿心里暗暗欢喜,只望他手下真有两下子。

哪知就在这时,突听一连串刀剑相击声自前面传了过来,别人刀剑相击,每一声之间总有间隔,但此刻这刀剑相击声,却又紧又密,前一声和后一声几乎是同时响起来的,数十声刀剑相击,听来竟如一声。

赵全海霍然回首,变色道:``又是什么人来了!好快的剑!''小鱼儿眨着眼睛道:``莫要怕,只要你站在本座身旁,谁也伤不了你。''赵全海瞧了他几眼,再瞧瞧缠在他耳鼻之间的毒蛇,这种诡异的模样,不由他不信面前的这人实是前辈异士。他瞧了几眼,终于抱拳道:``多谢!''那剑击之声来得好快,方才还在洞口,此刻已到了近前,一个阴沉冷漠的语声冷笑道:``雪花刀,你真要和我拼命么''另一人道;``久闻你剑法之快,关外无双,我早就想见识见识,今日既然又不知怎会被你知道这藏宝之地,看来你我更只有分个生死强弱了!''这语声又尖又细,竟似女子的声音。

小鱼儿忍不住问道;``这雪花刀是女的?''

赵全海叹了口气,道:``她就是昔日江湖中闻名丧胆的叁罗刹之一;刀法实已出神入化,就连历史悠久的叁虎断门刀彭家子弟,都败在她手下。''小鱼儿道:``另一人又是谁?''

赵全海道:``听雪花刀所说的话,这人想来必是长白剑派中巨子,关外神龙剑冯天雨,此人刨法之快,委实可称是关外无双!''小鱼儿叹了口气,道:``本座究竟老了,后辈的成名人物本座多已不知道了。''赵全海双眉深皱,道:``这藏宝之地如此隐秘,却怎会有这许多人来?奇怪\ldots\ldots 奇怪\ldots\ldots{}''只见一片刀光剑影,着地滚来,光芒流动,在火光映影下,看来就仿拂一具十彩变幻的七宝光幢。剑光中有着两条人彤,一个瘦削颀长,满身黑衣,另一人白衣如雪,身材婀娜,掌中一柄柳叶刀,运展如飞赵全海站在那里,已有些不安。

小鱼儿悠悠道:``两人武功虽不错,但破绽还是很多,若是换了本座出手,他两人只怕不能抵挡十招。''只听``呛''的一声龙吟,刀光剑影顿敛,黑衣人、白衣女,已齐地住手,齐地掠到小鱼儿面前。

那白衣女子``雪花刀''徐娘半老,风韵犹存,身材也丝毫不现臃肿,此刻眼波一扫,竟失声道:``全海,你怎地也来了。''赵全海勉强笑了笑,道:``多年不见,你模样看来还未改变。''雪花刀嫣然一笑,道:``谢谢你,在这里见着你,可真是想不到的事\ldots\ldots 十一年\ldots\ldots 嗯,快十二年了,你竟都不来找我,难道你只求成名成业,就不要别的了么''赵全海干咳几声,道:``我\ldots\ldots 我''\ldots."

``关外神龙剑''冯天雨突然笑道:``妙极妙极,原来是老情人见面了,但柳玉如再加上个赵全海,我冯天雨也未见得怕了你们。''``雪花刀''柳玉如眼见有了帮手,根本理也不理他,眼皮扫了赵全海身旁的小鱼儿一眼,道;``你还带了个徒弟来么?怎地如此奇形怪状?''赵全海道:``这位便是\ldots\ldots 玉\ldots\ldots 玉老前辈。''柳如玉眼睛立刻瞪大了,道:``玉老前辈?''

赵全海大声道:"此刻躺在地上的金陵叁剑、灰蝙蝠、猫头鹰、碧蛇神君,就全都是死在这位玉老前辈手下的!这句话说出来,不但柳玉如吃了一慷,冯天雨更是面色大变,退后两步,朝小鱼儿左瞧右瞧,手里的剑握得更紧了。

小鱼儿暗中几乎笑破肚子,面上却正色道,柳姑娘莫非也有份藏宝图么?``柳玉如点头道:''嗯。"

小鱼儿目光移向冯天两,道:``你呢?''

冯天雨冷冷道:``若无藏宝图,我怎会寻到这里。''小鱼儿目光闪动,道:``到目前为止,这藏宝图,已出现六份了,一份宝藏,却有六份藏宝秘图,此次倒真奇怪得很。''冯天雨剑光一展,厉声道:``无论有多少人来,死得只剩最后一个时,便是宝藏的主人!''小鱼儿冷冷道:``你此刻就想死,也没关系,但连那宝藏所在之地都末瞧过一眼就死了,岂非死得太可惜了么?''冯天雨征了征,掌中剑缓缓垂落。

赵全海道:``玉老前辈说的是,无论如何,咱们先进去瞧瞧总是好的,等到瞧见宝藏再拼个你死我活也不迟。''小鱼儿笑道:``究竟还是联盟镖头的见识不同。''他转身走了几步,突又回首道:``烦你瞧瞧那碧蛇神君怀中有些什么好吗''碧蛇神君怀中,果然有叁个紫檀木雕成的小匣子,叁个匣子完全一模一样,上面贴着的黄纸标签却各不相同。

一个匣子上写着``迷魂''一个匣子上写着``解毒'',第叁个匣予上写的赫然正是:``蛇粮''!

小鱼儿接过匣子,简直欢喜得几乎跳起来。

他知道凭这一匣蛇粮,就必定可以将身上这些``蛇美人''引走,但他想了想还是先将匣子拿在手里。

他忽然发觉用这些小蛇来唬人,真是再好也没有了,而此时此刻,他正是要大唬其人的时候。

洞穴竟然很深,而且曲折幽秘、寒气侵人!

小鱼儿当先而行,赵全海高举火把,跟在他身后,柳玉如故意让冯天雨走在前面,冯天面手握长剑,嘴角噙着一丝冷笑。

突然间,洞穴豁然开朗,钟乳四垂,五光十色。

千奇百怪、玲珑剔透的钟乳间,竟插着一大一小两支松枝火把,火光闪影下竟赫然又有五个人在那里。

这五人叁个站着,另外两个却盘膝相对面坐,四只手掌,紧紧贴在一齐,正各以内家真力生死相拼!

只见这两人一个是黄衣和尚,一个是枯瘦老人,两人眼珠却似已将凸出,额上也都已见了汗珠。

站着的叁人,亦是面色凝重,神情紧张,小鱼儿等四人走了进来,这叁人竟连瞧都未瞧上一眼。

小鱼儿再转头一望,赵全海、柳五如、冯天雨的脸色又全都变了,显然他们是认得这五个人的,非但认得,而且还必定对这五人存有畏惧之心.看来这五人无论武功声望,都必定还在他们之上!

赵全海口中正念经般在喃喃自语道:这五个老怪物怎会也到了这里?``小鱼儿微笑道:''一个人能被人称作老怪物,想来就必定有些名堂。``赵全海叹道:''非但有名堂,而且名堂还不小。``小鱼几道:''哦!"

赵全海道:``前辈可听过淮南王家世代相传的大刀鹰爪神功,这一门武功七十年前便已名扬天下。''小鱼几道:``嗯!这我倒听过。''

赵全海道:``那看来瘦小枯干的老人,便是当今鹰爪门的第一名家,人称视人如鸡王一抓。''小鱼儿道:``视人如鸡?这算是什么名字?''

赵全海苦笑道:``名字是他自己取的,意思就是说,无论什么人,在他眼中看来,都好像小鸡一样,老鹰抓小鸡,岂非只要一抓。''小鱼儿失笑道:``好怪的名字,好大的口气\ldots\ldots{}''目光转向那黄衣僧人,只见他身材魁伟,相貌堂堂,坐着也比王一抓高了一个头。

此刻两人四掌相交,那王一抓当真像鹰爪下的小鸡一样,小鱼儿忍住了笑,悄声道;``依你看来这两人谁像小鸡?''赵全海又想笑,又不敢笑,自己面上神色却已变得可笑得狠,干咳一声,清了清喉咙道:``这位黄衣僧人,便是五台山鸡鸣寺的黄鸡大师。''小鱼几终于忍不住笑出声来,道:``像小鸡的偏偏要叫老鹰,像老鹰的偏偏叫做鸡,这两人看来倒真像是天生的活冤家死对头,却不知\ldots\ldots{}''突听一人叱道:``闭嘴!''

这叱声并不甚响,但入耳却极沉重,竟震得小鱼儿耳朵都麻了,再瞧发出叱声的那蓝衣老人,却连头也未回,目光只是凝注着王一抓与黄鸡大师的四只手掌,好像是除了这两人外,世上别的人都未放在他心上。

小鱼儿撇了撇嘴,道:``这小子又是什么角色?''赵全海睑色一阵青一阵白,瞧了瞧那蓝袍老人,又瞧了瞧小鱼儿身上的蛇,终于压低了语声道:"此公便是气功独步海内的一叱开山啸云居士,他与黄鸡大师数十年相交.乃是生死过命的交情。

小鱼儿道:``既是过命交情,为何不助黄鸡和尚出手?''赵全海话压得更低道:``王一抓自然也不是一个人来的,站在他身后的两人,一位掌天南剑派,剑掌出手双绝,另一位便是枪法世家浙东邱门的当今掌门人,邱清波邱七爷,王邱两门,素来是通家之好。''他悄悄喘了口气,接道:``何况以黄鸡大师与王一抓的身份,自也容不得别人助他们出手的。''小鱼儿冷笑道:``狗屁的身份,那王一抓若是一个人来的,啸云老儿不出手才怪''\ldots.``突然大步走了过去,向那邱清波抱拳一礼,笑道:''七弟近来可好?``那邱清被面容清□,神情肃重,但瞧见小鱼儿这副诡异的模样,眼睛不觉也直了,皱眉道:''是谁家的七弟?怎会识得老夫?又怎会来到此处?``小鱼儿笑道:''你不认得我,我却认得你,这次我带了赵全海、冯天雨和雪花刀柳姑娘叁个人来,就是来帮你忙的,你和这位天南剑派的仁兄只管向啸云老儿出手,我负责将这黄鸡和尚送上西天。"邱清波又惊又奇,还在莫名其妙,啸云居士面色却已变了,突然一声长啸,啸声请越,震得火光闪动飘扬。

王一抓、黄鸡大师自也难免被这啸声震得心神分散,四只紧粘在一处的手掌也难免为之震动分离!

刹那间,只见长剑离鞘,银枪出手,黄鸡大师身形已冲天面起,一朵黄云般团出面文。

啸云居士厉叱道:``以王、邱两家的声名,难道真要以多为胜么?''小鱼儿却仰天笑道:``说来你五人倒都是不同凡响的人物,其实也和江湖盗贼差不了许多,谁也信不过谁,大家都有一肚子坏心思''啸云居士脸色铁青,怒道:``你究竟想怎样?''王一抓目光如鹰,沉声道:``究竟你是什么人?''小鱼儿笑道:``你不认得我么?\ldots\ldots 问问他吧。''他随手一指赵全海,两道锐利的目光,便都转到赵全海身上。

赵全海垂下了头,呐呐道;``这位便是玉老前辈,便是\ldots\ldots 便是万蛇之圣、万剑之尊、万王之王,打遍叁山五岳无敌手,惊天动地玉王子\ldots\ldots{}''小鱼儿点头笑道:``虽然少了几个字,也算差不多了!这名字各位若是末听过,那当真是孤陋寡闻得狠。''王一抓怒道:``乳臭末干的小子,也敢用这样的名字!''赵全海道:``这\ldots\ldots 这位玉老前辈内功,已登峰造极,金陵叁剑、灰蝙蝠、猫头鹰和碧蛇神君,就全都是死在这位玉老前辈手上的!''这句话说出来,王一抓等五人自然又都耸然动容。

啸云居士目光逼视赵全海,厉声道:``这些人死在他手上,你怎会知道可是你亲眼瞧见的?''赵全海道;这\ldots\ldots 这自然是我亲眼瞧见的,他们的尸体,此刻就在外面。``他虽未真的亲眼瞧见,但心中实已深信不疑,何况,到了此刻他实已骑虎难下,实在也无法说出''没有亲眼瞧见"这句话来。

王一抓、邱清波、啸云、黄鸡,面面相觑,再去瞧小鱼儿时,目光与神情已与方才大不相同。

要知这些人虽未将赵全海的武功瞧在眼里,但对赵全海说出来的话却也未敢忽视,``两河十七家镖局联盟总镖头''这几字,拿到当铺里去也可当几两金子的。

小鱼儿目光四扫,微微笑道:``一份宝藏却有许多份藏宝秘图,各位难道不觉得此事有些奇怪,难道不想先瞧个究竟。''这番话若是在方才说出来,别人纵然听了,也不会仔细去想,但此刻他身份在别人眼里已不同,说出来的话份量自也不同,王一抓、黄鸡大师心念转动,越想越觉得此事其中实在大有蹊跷?

小鱼儿指起了头,只见山洞顶上,有个缺口,露出一片星光,接着,明月移来,月光自缺口射下。

众人齐地动容道:``时候到了!''

啸云居士撮口一吹,王一抓铁拳反挥,两只松枝火把,登时熄灭,只剩下一点月光照在一株玲珑的石笋上,月光照射处,正是藏宝的入口。

王一抓抢先掠向石笋,但身形方自展动,黄鸡大师长袖已流云般向他卷来,王一抓铁掌如钧,直抓长袖,邱清波银枪已点向啸云胸膛,柳玉如雪花刀,闪电般劈出叁刀,冯天雨也还了两剑,刹那间眼见又是一场混战。

小鱼儿却站得远远的,冷笑道:``你们着急什么?这里面是否有宝藏还说不定啦,等见到藏宝后再拼命,再动手,难道就等不及了么''石笋果然可以移动,火把再燃起,照亮了这神秘的地道入口,也照亮了地道中的十数级石阶。

王一抓、黄鸡大师、邱清波、啸云居士、孙天南、赵全海、冯天雨、柳玉如\ldots。这些人顺序面入,一个盯着一个,一个监视着一个,每个人都是脸色凝重,呼吸急迫,如赴深渊,如临大故。

小鱼儿走在最后,面上虽仍带笑容,但心情也难免有些兴奋,有些紧张,无论如何,此中的秘密,他还是未曾猜透。

突听王一抓``咦──''的一声,接着,黄鸡大师也是``咦──''的一声,这两人俱是一派宗主的身份,若非所见之事委实出奇,又怎会惊得``咦''出声来,孙天南,赵全海等人脚步加快,等他们赶到前面,也不禁"咦──的一声,目瞪口呆,楞在那里,再也说不出话来。

石阶的尽头,哪有什么藏宝.却有几口棺材。

漆黑的棺材,在这黝黯的石室中,闪动的火光下,看来更是诡秘可怖,每具棺材前,都有着灵牌神幔。自地道中吹来阴森森的微风,将鹅黄色的神幔吹得飘飘飞舞,柳玉如但觉身子发冷,不由自主向赵全海靠了过去,暗中一数,那棺材竟有十叁口之多。

小鱼儿委实不敢走快,等他一步步走了进来,赵全海与冯天雨手中所举的两只火把,竟已熄灭。

诺大的石室中,只剩当中一张灵桌上两只烛泪琳漓的白烛,仍是明灭闪动,发出鬼火般的黄光,映着灵脾上的七个宇:``历代祖师之灵位。''这七个宇上还有两个字,却被神幔的阴影所掩,瞧不出来,小鱼儿也不觉倒独了口凉气,道:``这是什么所在?''邱清波沉声道:``衡量地势,中间乃是峨媚后山,闻得峨嵋后山中有处禁地,乃是峨嵋历代掌门人厝灵之所,莫非便是这里?''

\hypertarget{ux7b2cux4e8cux5341ux4e09ux7ae0-ux5947ux5cf0ux8fedux8d77}{%
\chapter{第二十三章
奇峰迭起}\label{ux7b2cux4e8cux5341ux4e09ux7ae0-ux5947ux5cf0ux8fedux8d77}}

黄鸡大师听说这里是峨嵋禁地,不由皱眉道:``当真是这里,你我还是快快退出才是!''啸云居士道:``不错,误入别人禁地,便是犯了武林大忌!''王一抓目光闪动,截口道:``既是如此,各位就请快快退出去吧。''黄鸡大师微一沉吟,终于转身。

冯天雨突然大声道:``大师且慢,莫要中了别人之计。''黄鸡大师道:``计?计从何来?''

冯天雨道:``世上哪里还有比棺材更好的藏宝之地?''黄鸡大师耸然动容,啸云居士与王一抓已双双向居中灵位旁的一口棺材抢去,哪知就在这时,四面石壁突然开了八道门户,八道强烈的灯光,自门中笔直射出,照在小鱼儿、王一抓等人身上。

众人被这灯光一照,一时间竟是动弹不得,眼睛更是无法睁开,隐约只瞧见灯光后人影幢幢,剑光闪动,却瞧不出是什么人来。

一个沉重的话声自灯光后响起,道:``何方狂徒,竟敢擅闯本门圣地!''另一人厉声接道:``擅闯圣地,罪必当诛,还问他们的来历作甚?''这人语音缓慢,但缓缓说来,自有一种凌厉逼人气概!

黄鸡大师失声道:``莫非是神锡道长。''

那语声``哼''了一声,黄鸡大师道:``道长难道已不认得五台黄鸡大师了么''那语声道:``圣地之中,不谈旧谊,咄!''

``咄''字出口,数十道剑光自灯光处急射而出,如雷轰电击,直取黄鸡大师与王一抓等人的咽喉要害!

小鱼儿眼见剑光刺来,竟是不敢闪避──剑光虽狠,蛇吻更毒,他惊惶之下,反而仰天长笑起来。

他这一笑,蜷曲在他身上的毒蛇全部昂首而起,红信闪缩,小小的孩子身上爬满了毒蛇,这模样看来端的比什么都要吓人。

刺向他的两柄长剑,竟不由自主硬生生在半空顿住了剑势,在灯光下出现的人影,是两个紫衣微温的道人,左面一人横剑当胸,厉声道:``你这娃儿鬼笑些什么?''小鱼儿笑道:``我只笑你们峨嵋派自命不凡,却不过只是些不分皂白的糊涂虫而已。''四面兵刃相击声,叱□怒喝声,不绝于耳,他语声说得也不太大。

那道人逼进一步,喝道:``你说啥子?''

峨嵋道人足不离山,说的自然是道地的四川土音。

小鱼儿眨了眨眼睛,道:``什么傻子不傻子,你才是傻子,我且问你,就算是咱们擅闯了禁地,你们又怎会知道的?那道人冷笑道:''峨嵋山岂是容人来去自如之地,有人闯人后山,本派焉有不知之理。``小鱼儿也冷笑道:''只是咱们闯入后被你们发觉,那也算你们的本事,但你们却显然是早有防备在此,难道你们峨嵋弟子真有未卜先知的本事。``那道人厉声道;''这不关你的事。"

小鱼儿道:``这自然关我的事,只因咱们未来之前,早已有人向你们告密,是么?\ldots\ldots 哼,这人又是怎会知道咱们要来的,你们难道想都不想么?''赵全海远远大喝道:``正是,这一切都是告密的那人做成的圈套,好教你我互相火并。\ldots.''话末说完,一声惨呼,显然是身上已挂彩了。

那道人皱了皱眉,沉声道:``啥子圈套?那有啥子圈套?''小鱼儿大声道:``你们只要住手,我自会为你们揭穿这圈套.只听一人喝道:''莫要中了这小鬼的缓兵之计。``那道人亦自喝道:''不错,擒住了他再问话也不迟。小鱼儿知道这两人只要一出手,自己就休想整个回去,他暗中不觉大是后梅,方才为何不先用蛇粮将毒蛇引开,却偏要因着它来唬人。

他情急之下,大喝一声,将紧捏在手里的叁个匣子,劈面向这两个峨嵋道人掷了过去。

但道人剑光一展,叁个匝子立刻分成六半,匣子里的迷魂药,解毒药下雨般落了满地。

道人剑势也不觉缓得一缓,但瞬即扑刺上来。

小鱼儿暗叹一声,苦笑道:``要害人的时候,却莫忘了反面会害到自己''\ldots\ldots{}``心念─闪间,突闻''嗤、嗤、嗤"十数声急风骤响,昏黄的烛光,强烈的灯光,突然─齐熄灭.小鱼儿方在吃惊,有一只手悄悄握住了他的手。

一人在他耳畔轻声道;``随我来。''

小鱼儿只觉这只手虽是冷冰冰的,却有说不出的柔腻,这语声更是说不出的温柔,说不出的熟悉。

他心头不知怎地也会流过一股暖意,低声道:``是铁心兰么?''那语声低低道:``嗯。''

小鱼儿脚下随着她走,口中不觉轻叹了一声,道:``如今我才知道你暗器功夫实在比我强得多,那种在一瞬间便能打灭十几盏灯光的本事,我实在比不上。''铁心兰道:``打灭灯火的不是我。''

小鱼儿怔了怔,道:``不是你是谁?''

灯火熄灭后,虽有一阵静寂,但惊□叱□之声立刻又响起,数十人在黑暗中纷纷呼喝:``谁?''``又是什么人闯了进来''

``掌灯!快!快!''

铁心兰还未仔细回答小鱼儿的话,灯光又自亮起,峨嵋道人贴向石壁,王一抓等人也聚在一起。

灯光下,却多了两个人,只见这两人衣衫雪也似的洁白,头发漆也似的乌黑,那皮肤却更白于衣衫,眸子也更黑于头发。

小鱼儿只当这能在这刹那间熄灯的必是十分了不起的角色,哪知却是两个看来娇柔无力、弱不禁风的绝色少女!

此刻在这峨嵋后山禁地灵堂中的,可说无一不是江湖中顶尖儿的人物,就算那些紫衣道人也都是峨嵋子弟中百里挑一的好手,但这两个白衣少女却似全末将任何人瞧在眼里,两双明亮的秋波,微微上翻,娇美的面容上满带着冷漠傲慢之意。

这种与生俱来、不加做作的傲气,自有一种慑人之力,此刻灯火虽亮起,室中反而变得死一船静寂。

啸云居士突然冷笑道;"居然有女子闯入峨嵋禁地,峨嵋子弟居然还在眼睁睁的瞧着,这倒真是江湖中前所末闻的奇事\ldots\ldots{}

他口中说话,眼角却瞟着神锡道长,神锡道长面沉如水,四下的峨嵋弟子却已不禁起了骚动,有了怒容。

白衣少女却仍神色不动,左面一人身材较细,长长的瓜子脸,尖尖的柳叶眉;冷漠中又带着股说不出的娇俏。

右面的少女身材娇小,圆圆的脸,大大的眼睛,鼻尖上浅浅有几粒白麻子,却使她在冷漠中平添了几分妩媚娇憨。

此刻这圆脸少女眼睛瞪得更大了,冷笑道:``荷露姐,你可听见了。这峨嵋后山,原来是咱们来不得的。''那荷露冷冷道:``天下无论什么地方,响们要来便来,要去便去,有谁能拦着咱们?有谁敢拦着咱们。''神锡道长终于忍不住怒叱一声,厉声道:``是哪里来的小女子,好大的口气!''这一声怒叱出口,峨瞻弟子哪里还忍耐得住,两道剑光如育龙般交剪而来,直刺自在少女们的胸腹。

白衣少女却连瞧也未瞧,直等划光来到近前,纤手突然轻轻一引、一拨,谁也赠不出她们用助是什么手法,两柄闪电般剩来的长剑,竟不知怎地拨了回去,左面的剑竟刺在右面一人的肩上,右面的剑却削落了左面一人的发髻,两人心胆皆丧,楞在那里再也抬不起手。

王一抓、黄鸡大师等人也不禁为之耸然失色。

神锡道长一掠而出,变色道:``这,这莫非是移花接玉?''荷露谈淡道:``亏你还有点眼力。''

圆脸少女冷笑道:``现在你总知道咱们是哪里来的了,你还嫌咱们的口气太大么?''神锡道长面容惨变,道:``峨嵋派与移花宫素无瓜葛,两位姑娘此来,为的是什么?''荷露道:``咱们也不为什么,只想要你将藏南天的藏宝取出来,其实咱们也不想要,只不过想瞧瞧而已。''神锡道长征了征,道:``燕南天的藏宝?''

圆脸少女道:``你还装什么糊涂,好生拿出便罢,否则\ldots。.哼!''神锡道长道:``燕南天与本派更是素无瓜葛,此间怎会有燕南天的藏宝?\ldots\ldots{}''目光四顾,突然惨笑一声,接道:``我明白了,各位想必也是为了这藏宝来的。''王一抓、黄鸡大师俱都闭紧了嘴,谁也不说话,移花宫中居然有人重现江湖,他们还有什么话好说。

神锡道长嘶声道:``这一切想必是个圈套,你我全都是被骗的人,你我若是火并起来,就正是中了别人的毒计!''小鱼儿早已退到圈外,此刻不禁冷笑忖道:``我说这话时你偏偏不信,如今你自己也说出这话来了,这岂非敬酒不吃吃罚酒。''他眨着眼睛,瞧着那两个白衣少女,心里也不知又在转些什么念头,反正他的心思,谁也猜不透。

只所那圆脸少女道:``你的意思是说燕南天的藏宝不在这里?''神锡道长叹道:``贫道简直连听也未听过\ldots。''圆脸少女道:``荷露姐,他说的话,你相信么?''荷露谈淡道:``我天生就不信别人说的话,无论谁说的话,我都不信。''神锡道长道:``姑娘若是不信,那也无可奈何。''圆脸少女冷笑道:``谁说无可奈何,咱们要搜!''神锡道长变色道:``要搜?''

圆脸少女道:``不错,搜!我瞧这几口棺材,就像是最好的藏宝之地,你就先打开来让咱们瞧瞧吧。''她话未说完,峨嵋弟子已俱都勃然大怒,神锡道长更是须发皆张,勉强忍住怒气,沉声道:``棺中乃是本派历代先师之灵厝,天下谁也不能开启。''圆脸少女冷笑道:``这就是了,棺中若真是死人,让咱们瞧瞧有何关系,又不会瞧掉他们一块骨头,你不让咱们瞧,显见有弊!''神锡道长忽喝道:``无论谁要开此灵棺,除非峨嵋弟子死尽死绝!''圆脸少女道:``那要等多久,我可等不及了。''神锡道长喝道:``移花宫欺人太甚,我峨嵋派和你拼了!''反腕拔出长剑,剑光一闪,直取少女咽喉!

他暴怒之下,这一剑正是他毕生功力所聚,当真是快如电击,势若雷露,声威之绍,震人魂魄!

白衣少女毕竟功力还浅,眼见如此声势,竟不敢攫其锋锐,再施展那移花妙手,两人身形一闪,翩翩避了开去!

但这时峨嵋弟子的数十柄长剑,已交剪击来,她两人纵有绝世的心法妙传,也难以敌这数十柄雷霆怒剑!

铁心兰突然松开了小鱼儿的手,道:``你等着莫动,我''小鱼儿瞪眼道:``你要做什么''

铁心兰道:``我迷途荒山,幸得她们收容,你危急被困,又幸得她们出手,此刻她们有难,我怎能坐视不救。''小鱼儿笑道:``移花宫中人纵然有难,还用得着别人解救么?''语犹未了,身后已有人接口道:``你说的不错!''这语声清朗而短促,语声入耳,已有一条人影自小鱼儿身侧掠出,纵在火光之下,小鱼儿也无法瞧清这人是男是女,是何模样,以小鱼儿的眼力,甚至连此人身上穿的衣服是何颜色都未瞧清。

他一生竟从未见到如此迅急的身法,更想不到世上有如此迅急的出手──人影闪过,闪入剑光。

刹那间,只听剑击之声不绝于耳,数十柄长剑一齐落在地上,别人谁也瞧不清这些剑是如何脱手的,只有峨嵋弟子自己心里有数──他们只觉剑上突有─般不可抗拒的力道引来,将自己掌中剑引得与同伴之人掌中剑互相交击,两人都觉得对方剑上之力大得惊人,于是手腕一麻,长剑落地,一个个捧着手腕惊呼后退,心里还是糊里糊涂,仿佛正在做梦似的。

神锡道长掌中剑虽未出手,人已惊得后退一丈,目光四下游顾,除了那两个白衣少女外,哪里还有别的人影\ldots\ldots 但四下火光明灭动,数十柄长剑惧都在地。

神锡道长咬牙顿足,仰天长叹道:``罢了!''反腕一引长剑,竟向自己脖子上抹去,他眼见此等不可抗拒的惊人武功,眼见峨嵋派的声名便要从此断送,也只得一死以求解脱!

谁知就在这时,一只手自他身后伸出,轻轻托住了他的手,另一只手已轻轻将他的长剑接过。

神锡道长掌中小这柄剑,随他出生入死,闯荡天下也不知经历了多少惊心动魄的战役,长剑离手之事,却是从来未有,但此刻也不知怎地,这柄生死不离的长剑,竟会轻轻易易到了别人手中。

神锡道长又惊又怒,一个白衣少年已自他身后缓步走出,双手捧着长剑,从容而揖,含笑道:``道长请恕弟子无礼,但若非贵派道友向妇人家出手,弟子也万万不会胡乱出手的。''灯光下,只见这少年最多也不过只有十叁四岁年纪,但他的武功,他的出手,已非这许多武林一流高手所能梦想,他穿的也不过只是件普普通通的白麻衣衫,但那种华贵的气质,已非世上任何锦衣玉带的公子能及。

他到此刻为止,也不过只说了叁五句话,但他的温文,他的风度,就连阅人无数的``雪花刀''柳玉如见了,也觉心神皆醉,``银枪世家''的邱七爷少年时也曾是风流潇洒的美男子,但见这少年,也只有自愧不如。

一时之间,众人竟都不知不觉瞧得呆了。

神锡道长虽是满心惊怒,此刻竟也似被这种迷人的风度所慑,竟也不觉抱拳还礼,道:``足下莫非亦是来自绣玉谷,移花宫?''白衣少年道:``弟子花无缺,正是来自移花宫,本官中人已有多年未在江湖走动,礼数多已生疏,若有失礼之处,还请各位包涵才是。''他说的话总是那么谦恭,那么有礼,但这情况却像是个天生谦和的主人向奴仆客气,主人虽是出自本意,奴仆受了却甚是不安──有种人天生出就是仿佛应当骄傲的,他纵然将傲气藏在心里,他纵觉骄傲不对,但别人却觉得他骄傲乃是天经地义、理所应当之事。

他面上的笑容虽是那么乎和而亲切,但别人仍觉他高高在上,他对别人如此谦恭亲切,别人反觉难受得很。

神锡道长、黄鸡大师、王一抓、邱清波、孙天南、冯天雨、赵全海,这些人无一不是一派掌门的身份,但不知怎地,在这少年面前,竟有些手足失措,举止难安,几个人口中呐呐,居然说不出应对之词。

荷露眼波流转,忍不住笑了,大声道:``我家公子来了,这棺村可以打开瞧瞧了么?''神锡道长面色又一变,但他还未出言,花无缺已缓缓道:``藏宝之事必属子虚,在下只望各位莫要中了奸人的恶计,而从此化干戈为玉帛,今日之事,从此再也休要提起。''黄鸡大师合什道:``阿弥陀佛,公子慈悲。''

王一抓大声道:``谁若还想争杀,却让别人暗中在一旁看笑话,那才是呆子。''邱清波、孙天南等齐声道:``公子所言极是,在下等就此告退。''神锡道长唏嘘合十,道:"多谢公子!此间本已是个不死不休的杀伐之场,这花无缺公子来了才叁言两语,却已化戾气为祥和,化杀气为和气。

柳玉如眼波转动,始终不离他面目,铁心兰瞧着他,嘴角不知不觉间泛起了一丝钦佩的笑意。

小鱼儿突然``哼''了一声,向地道外大步奔出,铁心兰怔了怔,微微迟疑,终于也快步跟了出去。

只听身后赵全海晚道:``玉大侠,玉老前辈\ldots。.''荷露也在唤道:``喂,那位姑娘,你怎地走了。,神锡道长唤道:''那位小施主,方才多承教言,请稍坐待茶。"几个人呼声混杂,小鱼儿根本听不清楚,何况他纵然听清也不会回头的,他竟一口气走出了那山窟。

洞外虽有薄雾,但明月在天,清辉满地,夜色显得更美。

小鱼儿眼睛却只直勾勾瞧着前面,脚步丝毫不停,直走了几盏茶时分,方自寻了块青石坐下。

铁心兰这才长长叹了口气,道:``藏宝之事,竟会如此结束,倒真是令人想不到的事。''小鱼儿道:``你想得到什么?''铁心兰怔了怔,垂下了头,幽幽道:``我竟为这一文不值的藏宝秘图受了那许多辛苦危难,竟险些一死,如今想来,真是冤枉得很。''小鱼儿道:``你活该。''

铁心兰咬了咬嘴唇,垂首道:``在那慕容山庄,我知道你必有许多苦衷许多困难,才会抛下我不顾,我并不怪你,但你\ldots{}''小鱼儿道:``你怪我又怎样?''

铁心兰霍然抬起头,道:``你\ldots\ldots 你。''你怎么这样说话。

小鱼儿道:``我说话本来就是这样,你不爱听,就莫要听\ldots。哼,别人说话好听,你不会去听别人的么?''铁心兰眼圈已红了,默然半晌,强颜一笑道:``你是什么时候到峨嵋来的?''小鱼儿道:``哼!''

铁心兰柔声道:``你身上怎会有这些蛇?''

小鱼儿道:``哼!''

铁心兰跺了跺脚,也赌气坐了下去,两人背靠着背,谁也不理谁,谁也不动,谁也不说话。

也不知过了多久,小鱼儿终于忍不住了,重重啐了一口,道:``嘿,那小子好神气!''铁心兰像是全没听见,根本不答腔。

小鱼儿憋了半晌,又忍不住了,用背一顶她,道:``喂,聋子,我说的话你听见了么?''铁心兰道:``聋子怎会听得见人说话。''

小鱼儿呆了呆,道:``但\ldots\ldots 你这不是明明听见了么?你听不见人说话,又怎会听见了,你''\ldots."说来说去,他自己也忍不住大笑起来。

铁心兰早已偷偷在笑,此刻也不禁``噗嗤''笑出声来。

笑声中两人不知不觉已并排坐在一起,也不知是铁心兰先移过来的,还是小鱼儿先移过去的。

笑了半晌,小鱼儿突然又道:``那小子实在太神气了!''铁心兰柔声道:``其实那也不是他自己神气,只不过是别人捧着他神气而已。''小鱼儿冷笑道:``你莫以为他自己不神气,他那副样子,不过是装做出来的,好让别人说他谦恭有礼,其实\ldots\ldots 哼,狗屁!''铁心兰笑道:``绣玉谷,移花宫可说是如今天下武林的圣地,他身为移花宫唯一的传人,就算神气,也怪不得他。''小鱼儿道:``哼''``哼哼\ldots\ldots 哼哼哼。''

铁心兰媚然一笑,轻轻摸了摸他的手,瞧见他腕上的毒蛇,又赶紧缩了回来,眨着眼睛笑道:``你有没有发觉,他的眉毛眼睛,可真是像你,简直和你一模一样,不知道的人,还要以为你们是兄弟哩。''小鱼儿道:``我若生得像他那副娘娘腔的模样,我宁可死了算了.''铁心兰含笑瞟了他\ldots 眼,不再说话。

小鱼儿歪起了头,冷笑着又道:``奇怪的是,这种装摸作样、娘娘腔的男人,偏偏有人喜欢他。''铁心兰道:``哦\ldots\ldots 谁喜欢他。''

小鱼儿道:``你。''

铁心兰呆了呆,失笑道:``我喜欢他?你疯了!''小鱼儿道:``你若不喜欢他,怎会瞧他瞧得眼睛都亮了\ldots。.你若不喜欢他,又怎会处处都帮着他说话。铁心兰脸都气红了,咬牙道:''好,就算我喜欢他,我喜欢得要死好么,反正,你也不是我的什么人,你也管不着。"她跺着脚,背又转了过去。

小鱼儿索性坐到地上去了,喃喃道:``哼,装模作样像个小老头子,这种人比什么人都讨厌。''铁心兰也不回头,道:``你不是说他娘娘腔么现在怎么又说他像老头子。''小鱼儿道:``我我说的是他像小老太婆。''

铁心兰突又``噗嗤''一笑。

小鱼儿瞪起眼睛,道:``你笑什么?''

铁心兰慢悠悠地,一字字道:``你在吃醋。''

小鱼儿跳了起来,道:``我在吃醋?笑话,笑话\ldots\ldots{}''突又坐了下去,叹道:"不错,我现在真的有些像是在吃醋。

铁心兰娇笑着扑入他怀里,但瞬即跳起,颤声道:``蛇\ldots\ldots 这些鬼蛇你怎么不弄掉它。''小鱼儿苦着脸道:``我若能弄得掉它们就好了!''铁心兰失声道:``你\ldots\ldots 你自己也弄不掉它?''小鱼儿叹道:``碧蛇神君一死,现在只怕谁也弄不掉它们了,无论谁只要一碰它们,它们立刻就会给我来上一口。''铁心兰着急道:``那\ldots。那怎么办呢?你难道永远带着它们跑?!''小鱼儿愁眉苦脸,呆了半晌,突然做了个鬼脸,笑道:``这样也好,身上缠着蛇,女孩子就不会来缠我了。''铁心兰跺脚道:``人家说正经的,你却还要开玩笑。''她又赌气背转脸,但瞬即又回了过来,笑道:``我有法子了。''小鱼儿喜道:``你有什么法子?''

铁心兰道:``你不给它们东西吃,等它们饿死,它们一死,自己就掉下来了。''小鱼儿像是想了想,点头道:是极是极,这法子简直妙不可言。``铁心兰嫣然笑道:''多谢多谢。"

小鱼儿眨了眨眼睛,道:``只是还有一样你忘了。''铁心兰道:``还有什么?''

小鱼儿道:``这些蛇虽是光头,却不是和尚。''铁心兰呆了半晌,道;``这是什么意思?''

小鱼儿忍住笑道:"不是和尚,就吃荤的\ldots\ldots{}

铁心兰又呆了呆,突然跳了起来,惊呼道:``它。\ldots 它们若是真的饿了,岂非要吃你的肉,喝你的血。''小鱼儿叹了口气道:``你真是天才儿童,到现在才想到。''铁心兰急得眼泪都要流出来了,跺脚道,这怎么办呢?怎么办呢?我看只有``\ldots 只有\ldots。.''到底``只有''怎样,她却说不出,急得在那里直转圈子,转了七八个圈子,突听有人语声传了过来。

只听一人道:``那丫头怎会突然失踪,倒真奇怪。''另一人冷冷道:``她跑得了今天,还跑得了明天么?''这两人语声一入耳,小鱼儿、铁心兰又面色变了。

铁心兰哑声道:``小仙女!''小鱼儿道;``还有慕容九妹!''铁心兰道:``咱\ldots\ldots 咱们快走吧。''

但直到这时,他们才发觉这竟是条死路,叁面俱是直壁削立,唯一的道路,正是小仙女她们要走过来的。

铁心兰手脚都已冰冷,道:``这\ldots 这\ldots。''

小鱼儿道:``咱们先躲一躲再说。''

两人身子刚躲好,小仙女与慕容九妹已走过来了。

小仙女道:``峨嵋山倒真是邪门,诺大的一片山上,除了猴子住的洞外,就只有这里是可以避风的地方。''慕容九妹道:``我看满山乱找也没用,咱们不如先在这里歇歇,等天亮再说。''小仙女早已坐了下来,她坐的正是小鱼儿方才坐的那块石头,两人懒懒的坐下,连眼睛都眯了起来。

小鱼儿和铁心兰不觉暗暗叫苦,这一来要等到什么时候才能进出去,可真是只有天知道了。

也不勿过了多久,小仙女张开了眼睛,道:``你冷不冷?''慕容九妹冷笑道,``你真是娇生掼养的千金小姐,这样就算冷么,就算在冰天雪地之中,我都不会喊冷的。''小仙女耸了耸肩,又闭起了眼睛。

小鱼儿却在暗中撇了撇嘴,暗道:``你自然不怕冷,你也不想想你练的是什么功夫,光着屁股睡在冰下都没关系,别人可没练过你那鬼功夫呀。,又过了半响,小仙女突然站起来,道:''你不怕冷.你有本事,我可受不了啦。``慕容九妹道:''受不了也得受。小仙女笑道:``九姑娘,好姐姐,陪我去找些柴来生堆火好么?''慕容九妹终于慢吞吞站了起来,两人东瞧瞧,西望望,竟向小鱼儿与铁心兰藏身之处走了过来。

小鱼儿暗道:``该死该死,我怎么偏偏选了这地方来躲,这地方怎会偏偏有柴火,当真是倒了穷霉了。''须知他们要躲,自然就躲在枯藤木时后,枯藤木叶自然是最好的引火之物,百般巧合,小鱼儿可像是要倒霉了。

铁心兰掌心早巳流满冷汗,身子也发起抖来。

只见小仙女与慕容九妹越走越近,铁心兰也越抖越厉害,抖得四下枯藤木叶簌簌的直响。

小仙女突然停住脚,道:``你\ldots\ldots 你听,那是什么在响?''慕容九妹冷冷道:"你放心,不会有鬼的。小鱼儿心念一闪,眼珠子一转,突然将头发扯散,自己居然偷偷笑了,也不知在笑什么?

铁心兰见他在这种时候居然还笑得出,简直要气破肚子,急断肠子,只见小仙女又在往前走,口中喃喃道:``就算没有鬼,钻条蛇出来,也够要命的了.''慕容九妹冷冷道:``有我在这里,你什么都不必怕。''她话未说完,突见一个怪物从黑暗中跳了出来。

小仙女吓了一跳,冷汗立刻流出。

慕容九妹却冷叱道:``是什么人装神弄鬼?''

只听这怪物鬼叫道:``慕容九妹''\ldots 慕容九妹,你害我死得好苦,我做了淹死鬼,还要做烫死鬼\ldots\ldots 慕容九妹,慕容九妹,你还我命来!"

\hypertarget{ux7b2cux4e8cux5341ux56dbux7ae0-ux6b7bux4e2dux6c42ux6d3b}{%
\chapter{第二十四章
死中求活}\label{ux7b2cux4e8cux5341ux56dbux7ae0-ux6b7bux4e2dux6c42ux6d3b}}

在月光下,慕容九妹已瞧清了这``怪物''面目,却不是小鱼儿是谁?\ldots\ldots 却不赫然正是那已死在她手上的小鱼儿是谁?

深夜荒山,阴风阵阵,荒山中突然跳出个被头散发,满身是蛇的怪物,面这怪物又正是她亲手害死了的人。

慕容九妹纵有天大的胆子,也是受不了的。

她指着小鱼儿,颤声道,``你\ldots{}''你\ldots\ldots"

第二个``你''字才出口,人已被吓得晕了过去。

小仙女虽然不知道这其中的纠葛秘密,但瞧见小鱼几满身是蛇,瞧见慕容九妹又吓得晕倒\ldots\ldots 她的魂也没有了,惊呼一声,转身就跑,连头都不理回。瞬息间她便跑得踪影不见。

小鱼儿哈哈大笑,道:``蛇兄呀蛇兄,无论你以后是否会害死我,我都得谢谢你,无论如何,你至少已救过我两次命了。''最莫名其妙的自然还是铁心兰,她简直整个人都糊涂了,从黑暗中走出来,瞪大了眼睛瞧着小鱼儿,终于忍不住问道:``你几时被慕容姑娘害死过?什么淹死鬼、烫死鬼我\ldots\ldots 我简直被你弄糊涂了。''小鱼儿笑道:``女孩子还是糊涂些好,女孩子知道越多,麻烦就越多,你只要知道我有两下子就行了。''铁心兰怔了半晌,叹道:``你实在是有两下子,慕容九妹居然会被你吓晕,小仙女居然会被你吓得落荒而逃,这种事告诉别人,别人只怕也不会相信的。''小鱼儿瞧着还是晕迷不醒的慕容九妹,道:``依你看,我会对她怎么样?''铁心兰想了想,道:``你就任凭她晕在这里,一走了之。''她瞧了瞧小鱼儿的脸色,接着又道:"或者,或者你用藤子捆佳她,等她醒来时,打她几下出气\ldots\ldots{}

小鱼儿冷冷道;``妇人之仁,到底是妇人之言。铁心兰道:''这\ldots\ldots 这么凶的法子还不够?``小鱼儿道:''当然不够。"

铁心兰颤声道:难道难道你真要杀了她?``小鱼儿道:''我若不杀她,难道还等她以后来杀我不成?铁心兰跺脚道,``我实在想不到你\ldots 你\ldots\ldots 你竟真的如此狠心。''小鱼儿道:``你现在总该想到了吧。你若不愿瞧,就走得远远的好了。''铁心兰跺了跺脚,一口气冲了出去。

小鱼儿也不理她,眼睛瞪着慕容九妹,喃喃道:``你这个狠心的女人,我若不杀了你,怎对得起自己。''语声微顿,冷笑又道:``我正好要毒蛇咬你一口,看着究竟是蛇毒,还是你毒。''他竟抓起慕容九妹的手,向自己腕上的毒蛇喂去!

这时月光满天洒将下来,正照着慕容九妹的脸。

只见她瘦瘦的瓜子脸,是那么苍白,长长的睫毛,覆盖着眼,虽然是在晕迷着,看来却更是楚楚动人,我见犹怜。

她的手,也是那么柔软,冰冷而柔软,要拿这样人的这种手去喂蛇,又有谁狠得下这个心。

小鱼儿的手有些软了,但想到她将自己关在石牢里,想到她要将自己活活冷死、饿死,小鱼儿的怒火又不禁直冲上来,冷笑着道:``什么事你都怨不得我,你若不想杀我,我绝不会杀你的突听一人缓缓道:''以这样的手段来杀一个女孩子,岂非有失男子汉的身份。``小鱼儿一惊抬头,喝道:''谁?"

``谁''字喝出,他已瞧见了面前的人,正是那温文尔雅的花无缺公子,叁个人远远站在他身后,两个是白衣少女,还有一个竟是铁心兰,叁个女孩子的叁双大眼睛都在瞪着他,像是狠不得将他吞下肚里。

小鱼儿心里也不知已气成什么样予,但面上却只是笑了笑,仍然抓着慕容九妹的手,笑眯眯的道:``你是说我杀不得她?''花无缺和声道:``一个男人,对女孩子总该客气些,就算她有什么对不起你的事,你也该瞧她是女人份上,让她一些。''小鱼儿哈哈笑道:``好个温柔体贴的花公于,世上有你这样的男人,当真是女人的福气,天下的女人真该联合起来送你一面锦旗才是。''花无缺微微笑道:``好说好说。''

小鱼儿道:``但女人若要杀死你时,你又如何,难道你就闭起眼睛来让她们杀?难道你连还手都不还手。''花无缺缓缓道:我若做了对不起她的事,被她杀死,也绝无怨言。``小鱼儿道:''但若有个女人做了对不起你的事,你不杀她?花无缺道:``男人总该让女人些才是。''小鱼儿苦笑道:``你这样的想法,真不知从哪里学来的,照你这样说来,天下的男人简直都该死了,都该一头跳进黄河才是。''花无缺道:``那也不必。''

小鱼儿瞪着他,也不知是该气,还是该笑,也不知他是真的听不懂自己的话,还是假听不懂,也不知他是聪明,还是呆子。

花无缺含笑瞧着他,面上既无怒容,也不着急,他若真像表面看来这般文弱,小鱼儿早已一个耳光掴了过去。

但他那身武功实在有点骇人,小鱼儿只得叹了口气,道:``你的意思是定要我放了她?''花无缺含笑道:``足下放了她诚是英雄所为。''小鱼儿道:``我今日放了她,她日后若来杀我,又当如何?''花无缺沉吟道,``日后之事,谁也无法预测,是么?''小鱼儿道:``好,我要杀她,我就不是英雄,不是男子汉,我就该死,但她若要杀我,却是天经地义的事,我被她杀了也是活该,是么?''花无缺笑道:``在下并无此意,只是\ldots\ldots{}''

小鱼儿大声道:``我不管你是什么意思,今天我打不过你,你放个屁我也只有听着,但以后你打不过我时,我偏要杀几个女人让你瞧瞧。''他重重摔开慕容九妹的手,道:``算你厉害,你抬走吧。''花无缺也不动气,仍然微笑道:``如此就多谢了。''白衣少女已燕子般掠了过来,抱起了慕容九妹。

那团脸少女瞪着小鱼儿,冷笑道:``今天若非公子在这里,我就宰了你,让你知道女人的厉害。''小鱼儿冷笑道:``你随便吧,骂什么都没关系,因为你是女人,女人天生就可以骂男人的,花公子,你说是么?''花无缺笑道:``能被女人骂的男人,才算有福气,有些男人,女人连骂都不屑骂的。''小鱼儿道:``哈\ldots\ldots 哈哈,如此说来,我真是荣幸之至,为了免得让你难受,他日也得找几个女人来让你荣幸荣幸才是。''花无缺笑道:``那时在下必定洗耳恭听。''

小鱼儿眼睛一翻,几乎气炸了肺。

只见荷露拉起了铁心兰的手,道:``姑娘,你也跟咱们一齐走吧。''铁心兰垂首道:``我\ldots\ldots 我\ldots.''

她虽然垂着头,眼角却不住去瞟小鱼儿。

圆脸少女恨声道:``那种男人,你还要理他么,跟咱们走吧。''荷露笑道:``我家公子也正想和你聊聊。小鱼儿大声道:''去去去,你快跟他们去吧,我现在虽然倒霉,但还没什么,你若再跟着我,我才是倒霉透顶了。"铁心兰垂着头,眼角又沁出了泪珠。

圆脸少女拖着她,道:``不理他,我们走。''

花无缺含笑一揖,也转过身子,只见荷露怀中的慕容九妹突然挣着动了起来,口中梦呓般道:``小鱼儿\ldots\ldots 江鱼,放了我\ldots\ldots 放了我吧。''花无缺面色微变,霍然回首凝注着小鱼儿,一字字道:``你就是江鱼,就是小鱼儿?''小鱼儿也不觉怔了怔,道:``我这名字很出名么?''花无缺又瞧了半晌,竟轻轻叹息了一声,道:``抱歉得很小鱼儿瞪大眼睛,道:''抱歉?你为什么抱歉?``花无缺缓缓道:''只因我要杀死你!"

这句话说出来,大家全都吃了一惊。

小鱼儿道:``你头脑有些不正常么?怎地突然又要杀我?''花无缺道:``只因你是江鱼,所以我要杀你,芸芸天下只有一个是我要杀的人,那人就是江鱼,就是你!''小鱼儿怔了半晌,叹道:``我懂了,可是有人叫你杀我的。花无缺道:''正是家师所命。``铁心兰已嘶声大呼道:''你师父为什么要你杀他?为什么?为什么?\ldots\ldots"她想冲过来,却被那圆脸少女紧紧抱住了。

小鱼儿与花无缺面面相对,谁也没有瞧她!

过了半晌,小鱼儿突然笑道:``很好,我本来也想杀死你的,只因我目前实在打不过你,所以才一直忍住,不过,现在\ldots\ldots{}''他双臂突然一振,向花无缺扑了过去,他武功纵非花无缺之敌,但只要让他触及花无缺,他身上的毒蛇,是谁也不认的。

那不但会要花无缺的命,也会要他的命哪知他手臂一震,真气才转,左右双腕,便麻了一麻,他身子还未扑到花无缺面前,眼前已发黑。

他竟凌空跌了下去!

小鱼儿醒来时,首先瞧见一炉香。

这炉香就在他对面,香烟缭绕,氤氲四散,一阵阵送到小鱼儿鼻子里,却非檀香,也非茴香,而是一种说不出的什么香气,乍嗅有些像花,再嗅有些像药,仔细一嗅.又有些像女子的脂粉。

小鱼儿也懒得去分辨,总之他觉得嗅起来舒服得很。

然后,小鱼儿又瞧见一柄刀。

这炳短刀,镶着珠柄,就挂在他睡着的床头,像鲨皮的刀鞘,看起来像是专为装饰用的。

但这间屋子就只有这点装饰,其余都简陋得很,只是四面都打扫得一坐不染,叫人感到舒服得很。

小鱼儿猜不出这是什么地方,他想,这极可能是花无缺为了要在峨嵋山逗留,而临时搭起来的竹屋。

但他又怎会到了花无缺的屋子里?

他方才不是明明中了不可救药的蛇毒,难道花无缺还会救他?花无缺不是一心想杀死他的么?

他转了转头,立刻就瞧见了花无缺。

这时阳光已照满了那以竹架搭成的、简陋的窗子。

花无缺,就坐在阳光下,那眉目,那脸,那安详的神态,那雪白的衣衫,就连小鱼儿也不得不承认他是入间少见的美男子。

他像是已在这里坐了许久许久,但看来却一点也不烦躁着急,他就这样静静的坐着,像是还可以继续坐下去。

这也是小鱼儿佩服的,若是换了小鱼儿,简直连一刻都坐不住,小鱼儿暗中试了试,觉得自己身子好像并没有什么难受,再瞧自己身上那些要人命的毒蛇,居然连一条都瞧不见了,他暗中松了口气,大声道:``喂,可是你救了我?''花无缺淡淡道:``不错。''

小鱼儿道:``那么厉害的蛇毒,你也能救?''

花无缺道:``这仙子香与你已服下的素女丹,万毒俱都可解。''小鱼儿道,``你方才不是要杀我的么?''

花无缺缓缓道:``我现在还是要杀你!只因我必须亲手杀死你!不能让你因为别的事死。''小鱼儿眨了眨眼睛,道:``你为何定要亲手杀死我?''花无缺道:``只因我受命如此。''

小鱼儿默然半晌,道;``她一定要你亲手杀死我?我死在别的人别的事上都不行,这\ldots。你不觉得奇怪么?你不问是为什么?''花无缺道:``我不必问。'',小鱼儿道:``看来你倒听话得很。''花无故道:``本宫令严,无人敢违。''

小鱼儿道:``看来你也老实得很,我问你什么,你就答什么。''花无缺道:任何人无论问我什么,我都会据实以告,我纵要杀死你,但那和问答的话完全是两回事。``小鱼儿道:''你非要亲手杀死我不可?我若杀死了你呢?``花无缺淡淡道:''你杀不死我的。``小鱼儿道:''你敢和我拼一拼么?``花无缺道:''我正是堂堂正正取你性命!"

小鱼儿道:``好,你先退后几步,先让我起来。''花无缺果然站起身子,后退了八九步之多。

小鱼儿缓缓爬起,口中喃喃道:``你这人实在太老实了,但我却不知你是真的老实,还是假的老实,也许你自以为对什么事都太有把握,所以随便怎样都无所谓。''他口中说话,突然抛出了那柄镶珠的匕首,一跃下地!

花无缺淡淡瞧着,神色不变,就这份安详从容的气概,已足以愧煞世上千千万万自命高手的人物。

小鱼儿突然大笑道,``你要我死,那并不困难,但你若定要亲手杀死我,今生今世,再也休想。''突然反转匕首,对准了自己的心窝。

花无缺微微变色,道:``你\ldots\ldots 你这是做什么?''小鱼儿向他做了个鬼脸,笑道:``只要你身子向我这边动一动,我这一刀就刺下去,那么你就一辈子也休想亲手杀死我了,因为我已亲手杀死了自己。''花无缺呆在那里,简直不会动了!他实在想不到小鱼儿竟会有这一着!

若论武功,自是比他强胜许多,但若论临事应变,他又怎能比得上精灵古怪、诡计多端的小鱼儿?

这自然是因为两人生长的环境截然不同──高高在上的``移花仙子'',那精灵诡计,又怎比得上``恶人谷''中的恶徒,小鱼儿使出的这些``绝招'',花无缺当真是做梦也使不出的。

小鱼儿大笑道:``你若还想亲手杀死我,现在就得忍耐,莫要动\ldots\ldots 一动都莫要动\ldots。.''他眼睛瞪着花无缺,一步步往后退。花无缺竟不知该如何应付这种局面,只有站着不动,眼看小鱼儿退出了门,也无可奈何。

但小鱼儿也实在不敢稍有疏忽,虽已退出了门,眼睛还是眨也不眨地盯着花无缺.不敢放松。

门外晨雾弥漫,不知名的山花,在雾中更显得风姿绰约,阳光虽已升起,却仍照不散峨嵋清晨的浓雾。

小鱼儿一步步往后退,退过山花夹列的小径,他除非算准花无缺再也追不着他,否则实也不敢回头。他退得很慢,脚步踏得很稳\ldots。

花无缺突似想起什么,失声道:``江鱼─快快快站住\ldots!''.呼声中,他身子已要往门外冲。

小鱼儿厉声道:``你先站住你只要敢出门一步,我立刻\ldots\ldots\ldots\ldots\ldots\ldots\ldots\ldots{}''花无缺身子硬生生顿住在门口,额上竟已急出冷汗,大声道:``快站住,你已退不得了,后面\ldots\ldots{}''他``后面''两字方白说出,小鱼儿往后退的左脚已一脚踏空,他惊呼之声才出口,人已往下面直坠而落!他身后竟是一道悬崖,云雾凄迷,深不见底,花无缺眼看着小鱼儿直坠下去,也赶不及击拉他了。\ldots.小鱼儿的惊喊声,尖锐而短促,但四山回应却一声声响个不绝,天地间仿佛惧是小鱼儿的惊呼。花无缺身子似已脱力,斜斜倚在门上,眼睛失神地瞧着面前的浓雾,一粒粒汗殊滚滚流下。

这时铁心兰已踉跄冲了出来,四五个白衣少女跟在她身后,铁心兰冲到花无缺面前,道:``是谁在惊呼?是不是他?\ldots\ldots\ldots\ldots\ldots 是不是他?''花无缺点了点头。

铁心兰道:``他──他在哪里?''

花无缺叹息着摇了摇头。

铁心兰瞧见他的神色,后退两步,颤声道:``你─你──你杀了他──你杀了他!''突然冲上去,拳头像雨点般落在他身上。

花无缺仍是动也不动,既不闪避,也不招架,铁心兰拼命击出的拳头,打在他身上,他竟似全无感觉。

白衣少女们惊怒之下,怒喝着齐向铁心兰出手,花无缺反而为铁心兰一一拦住,柔声叹道,``我并没有杀死他,只是他──他自己失足落下了悬崖。''铁心兰身子一震,踉跄地后退,道:``你──你真的没有杀他?''花无缺道:``我一生之中,绝不说半句假话。''铁心兰嘶声道:那你为什么不还手?"

花无缺目光温柔地瞧着她叹道:我知道你此刻心里必定很难受,你纵然伤了我,也是理所应当的事,我绝不会怪你的。"铁心兰怔在那里,心里酸甜苦辣,也不知是何滋味,这花无缺固是如此善良,如此温柔,但小鱼儿──那又凶又坏的小鱼儿,却为什么偏偏比花无缺更令她刻骨铭心,更令她难舍难分,牵肠挂肚。

花无缺目光更是温柔,道:``铁姑娘,你还是歇歇去吧,伤\ldots\ldots\ldots\ldots\ldots\ldots\ldots\ldots\ldots\ldots{}''铁心兰道:``是──我是该歇歇去了,是该去了──''突然疯狂般冲向悬崖,嘶声道:``小鱼儿,你等着,我来陪你一齐歇歇──''但她还未冲到悬崖,花无缺已拉住了她的手,她拼命挣扎,纵然用尽了力气,也是挣扎不脱。

铁心兰泪流满面,大呼道:``放开我──放开我──为什么不让我下去陪他,他一个人死在下面,是多么寂寞──''只听一人悠悠道;``谁死在下面了──?一个人能寂寞寂寞,安安静静的死,是多么幸福。''乳白色的浓雾中,一条婀娜的人影,缓缓走了过来,就像是雾中的幽灵,却正是慕容九妹。

她面容更是苍白,那双灵活而妩媚的大眼睛,也失去了昔日的光彩,竟已像是有些痴呆。

铁心兰咬牙道:``小鱼儿终于死了,你开心么?─他就死在这悬崖下,你可要去瞧瞧他死时的模样。''慕容九妹轻轻摇了摇头,缓缓道:``他不会死在这里的,死在这里的,绝不是他!''她突然咯咯笑了起来,笑道:``他早已死在慕容山庄了,是我亲手杀死了他──一个人是绝不可能死两次的,你们说是么──是么?''她长发在风中飞舞,笑得是那样疯狂。

花无缺怜悯地瞧着她,轻声道:``荷露,这位姑娘方才被骇得太厉害了,到此刻神智还未恢复,你扶她回屋去躺躺吧。''荷露拉起了慕容九妹的手,但慕容九妹仍在咯咯笑道:``我亲手杀死了他,我亲眼瞧见了他的鬼!哈哈,你们瞧见过鬼么──休们能亲手杀死他么?''铁心兰突然狂笑道:``你们谁也杀不死他,世上唯一能杀死他的人,就是他自己──''狂笑突又变成痛哭,她放声悲嘶道:``但他终于杀死了自己──他终于毁灭了自己──为什么聪明的人,总是会自己毁灭自己──''不错,聪明人有时的确会自作聪明,弄巧成拙,到头来虽害了别人,但却也害了自己。小鱼儿远比这种人还要聪明得多──他方才那一脚踏空,竟是假的,竟只不过是做给花无缺看的。

他其实早巳将地势瞧得一清二楚,他整个人看似跌下去的,其实早巳算推了平衡的力量,拿捏得分毫不差。他身子滑下,右手的尖刀便已插入了削壁,左手也立刻拉住了条山藤,整个人都贴在削壁上。

这自然要有很快的眼睛,很细的心,更要有很大的胆子,但若要别人上当,尤其是要花无缺这种人上当,不冒险行么?

到方才铁心兰悲呼痛哭,慕容九妹又笑又叫,花无缺柔言细语,小鱼儿始终贴在壁上,听得清清楚楚。听见这些哭叫呼笑,小鱼儿心里自然也有许多难言的滋味,但他毕竟忍得下这个心,对一切都不闻不问。

到后来人声终于散去了,小鱼儿暗中松了口气,过了半晌,身子悄悄往上爬,眼睛自悬崖边沿悄悄向外望。只见悬崖上果然已没有人了,他正想爬上去──哪知就在这时,身旁竟似有人声响动!

\hypertarget{ux7b2cux4e8cux5341ux4e94ux7ae0-ux6b7bux91ccux9003ux751f}{%
\chapter{第二十五章
死里逃生}\label{ux7b2cux4e8cux5341ux4e94ux7ae0-ux6b7bux91ccux9003ux751f}}

小鱼儿大惊之下,扭头一瞧,才发现那竟不过是猴子,几十只猴子也不知是从哪里来的,竟都学着他的模样,身子爬在削壁上,脑袋悄悄往外伸。峨嵋山的猴子最多,又最喜学人模样,小鱼儿本就听人说过。

但此刻真的让他瞧见了,他不禁又是好气,又是好笑,又不知该如何才能赶走它们,只得撮口道:``嘘──去──''猴子们向他做了个鬼脸,也撮起嘴,吱吱喳喳的叫,有些猴子的脸红得像屁股,做起鬼脸来真可以吓死人。小鱼儿生怕这些见鬼的猴子惊动了花无缺,又不禁有些着急起来,忍不住伸出一只手去赶,去打。他手一伸,就知道坏了。

猛子们突然一窝蜂扑了过来,一齐向小鱼儿伸出手来,若是在平时,小鱼儿自然不怕。

但此刻他身子悬空吊在峭壁上,两只手都用不得力,猴子们往他身上一扑,他就得直滚下去。

他又是害怕,又是着急,又不敢出声呼救,两只手往峭壁上乱爬,手里的尖刀也落了下去,许久才听见``噗''的一声。那峭壁竟是向内陡斜的,所以匕首才会直落到底,那回声许久才传上来,显见这悬崖深得怕人。

小鱼儿满身冷汗,手再也抓不到着力之处,到了削壁向内陡斜之处,他身子也要笔直跌下去,不粉身碎骨才怪。天下第一个聪明人竟会死在一群猴子手上,小鱼儿一念想到这里,真不知是该哭还是该笑。

只见猴子们也往下直跌,但几十只猴子咬咬喳喳一叫,突然一个拉着了一个的手。几十只猴子手拉着手,脚爬着削壁,竟一连串悬空吊了起来,就像是一串葫芦似的,一个也末跌下去。

小鱼儿却已跌下去了,他的手已抓不住任何东西!

他只有闭起眼睛,惨笑道:``完了──小鱼儿竞被猴子杀了───''

但就在这时,突然不知从哪里伸出一只毛茸茸的猴爪来,竟将地胸前的衣襟一把抓住一这只猴爪力道竟大得怕人,只是小鱼儿下落之力更大,猴爪虽抓住了他的衣服,但衣服撕裂,身子还是往下直落!谁知另一只猴爪又闪电般伸出来,抓住了他的头发。

小鱼儿疼得眼泪直流,身子却总算顿住。

只见那一串猴子还在朝他做鬼脸,朝他鬼叫,抓住他的两只猴爪,却是从削壁上的一个洞里伸出来的!小鱼儿暗道:``抓住我的大概是猴王,否则又怎会有这么大力气,猴子对人,可不会有什么好念头,它将我抓上去,却不知要怎样折磨我。''他主意打得真是比天下所有的人都快,这心念一转,立刻暗中运气想先掠上去攀住那个洞,先发制``猴''!

又谁知他身子还未动,那洞里竟突然有个人的语声传出来,语声又尖又细一字字道:``莫要动,一动就将你丢下去!''这又尖又细的语声,听来当真有七分像是猴子,但说的明明是人话,猴子难道也会说人话?这峨嵋山里,莫非真有猴子成了精?

小鱼儿吓得又是一身冷汗,颤声道:``你你究竟是什么?''那语声吱吱笑道:``你是什么,我就是什么。''小鱼儿道:``你\ldots 你是人?''

那语声道,``你猜我是不是人?''

小鱼儿抽了口凉气道:``你要怎样?''

那语声道:``你垂下手,不准动。''

小鱼儿只有乖乖的垂下手,身子已被这``人''凌空直提了上去,就好像是在腾云驾雾一般。那只猴爪竟在他左右双肩各点了一点,点的竟正是他肩头的穴道,他再想抬手也抬不起来!

接着,他真的就像是条鱼似的,被拉入那洞里。

那洞口并不大,但洞里面却并不小。

小鱼儿被拉得全身又酸又疼,脑袋直发晕,张开眼睛,只见一只猴子正咧着大嘴朝他直笑。

这``猴子''可真是不小,竟比小鱼儿矮不了许多。仔细一瞧,这``猴子''身上竟穿着衣服,虽然破破烂烂,但却的确是人穿的衣服,半分不假。再仔细一瞧,这``猴子''全身虽长着毛,股上虽也长着毛,但那眼睛、那鼻予,却又像是人的模样。最奇怪的是,这``猴子''不但长着头发,还长着胡子。

那``猴子''却咬咬笑道:``你现在瞧见了么?我究竟像是什么?''小鱼儿硬着头皮,道:``你有叁分像人。''

那``猴子''道:``但却有七分像猴子,是么?''

小鱼儿道:``若不是亲耳听见你说人话,你简直半分也不像人。''他遇见这怪事,索性豁出去了,心里早巳全忘了``生死''两字,根本全不怕这``怪物''要对他怎样。

但这``猴子''却不生气,反面哈哈大笑道:``告诉你,我本就是人中之猴,猴中之人,你说我是人固然是对的,说我是猴子可也不错。''小鱼儿却不禁怔了征,失声道:``人中之猴\ldots\ldots 猴中之人。\ldots 你难道是\ldots\ldots 是\ldots。.''突听一人冷冷道:``你不要听他鬼话,他根本就是个人,只不过模样本就生得像猴子,再和猴子相处日久,人味儿更小了。''洞中甚是宽阔,阳光自小小的洞口照进来,洞里后面大半地方都是黑黝黝的,什么都瞧不清。这语声正是从黑暗中传出来的,枯涩生冷,听来也完全像是人说的话,小鱼儿又吓了一跳,道,``你呢?你是什么?''只见一个影子缓缓自黑暗中走出,亦是瘦小枯干,满头毛发,看来实也只有叁分像人,但是他的目光却极是清澈,而且像是充满了智慧,除了``人''之外,的确再无一种动物有这样的眼睛。

小鱼儿松了口气道:"不错,你是人\ldots\ldots 但你究竟是什么人?

又怎会在这种地方?又怎会变得如此模样?"

这``人''长长叹息了一声,道:``你问他吧。''

他话未说完,那``猴子''已跳了起来,怒骂道:``问我?我不是被你害的,又怎会活鬼般被困在这里?又怎会变成这副不像人的模样。''那``人''冷冷道:``你本来又像人么?十二星相中,又有哪一个是像人的?''小鱼儿眼睛本在这两``人身上转来转去,心中固是惊骇,也不觉有些可笑、好奇,但听了这话,他却吃了一惊,骇然望向那''猴子``道:''你\ldots\ldots\ldots 你真的是十二星相中人?``那''猴子``挺直背脊,傲然道:''不错,某家正是十二星相中的献果神君!``小鱼儿身子不觉往后退,背贴着石壁,转向那人道:''你。\ldots.你呢?``那人惨笑道:''你小小年纪,绝不会听见过我的名字\ldots\ldots{}``他背脊头也挺直,日中突然射出了光,大声接道:''但十四年前,武林中提起飞花满天,落地无声沈轻虹这名字来,有谁人不知?哪个不晓?"``献果神君''嘿嘿笑道:``放你的臭屁,你从来也不过只是个臭保镖的,一听见咱们十二星相的名字,马上就落荒而逃。''沈轻虹冷笑道:``是么?你十二星相既这般厉害,为何带不走我一分银子,为何也被我困在这里十四年,天天干着急?''这两人互相讥刺,互相嘲骂,小鱼儿又不禁听得呆住了,他这才知道这两人竟非朋友,而是仇敌。

两个仇敌竟同被困在一个山洞里达十四年之久,这日子真不知是怎么过的,小鱼儿委实想不出他们怎能活到现在。

只见两人你瞪着我,我瞪着你,像是已箭在弦上,一触即发,但到后来两人却是谁也未曾出手``献果神君''狞笑道:``你莫忘记,现在已有这小鬼来了,我已不愁寂寞,就算立刻杀了你,也没有什么关系。''沈轻虹冷冷道:``你只因恨我,不想比我先死,所以才活了这么久,我若是真个死了,你也万万活不长的。''小鱼儿忍不住道:``如此说来,你两人只因为互相怀恨,是必一定拼着活下去,所以才能活了这么久的么?''``献果神君''咬牙道:``十二星相怎能比这臭保镖的先死!''小鱼儿道:``这十四年来的日子,你们就始终在打打骂骂中度过?''沈轻虹道;``若不打打骂骂,如何消遣此长日。''献果神君道:``若非如此,我早已宰了他了!''小鱼儿道:``但你两人为何不设法逃出去?''

献果神君道:``我若能走就走了,还用得着你这小鬼来说?''小鱼儿道;你两人若不能出去,却又是如何进来的?``献果神君恨恨道:''只因那批红货就藏在这里,我逼他将我带来!那时我还有些不信,让他先进来。我再进来\ldots。那自然是从绳子上垂下来的。"他也许最因为太久没有和人说过话,也许是因为心里恨得太厉害,所以说话颠叁倒四,不明不白简直教人听不懂。

他眨着眼睛想了想,缓缓道:``他原是镖头,保了批红货,你知道了便要去抢,谁知他竟用了金蝉脱壳之计,先就将红货藏到这里,你去抢只抢了个空是么?''献果神君咬牙道:``说他娘是个老太太,正是一点也不错。''小鱼儿忍住笑道:``只是他机智虽高,武功却非你敌手,所以被你逼得没法子,后来终于将你带到这里。''沈轻虹道:``其中虽有曲折,大致却不差。''

小鱼儿道:``你们两人在悬崖上用绳子一齐垂了下来,他在前,你在后,为的自然是你怕他将绳子割断。''献果神君道:``这臭保镖的什么事都做得出,我自然得时时防备着他。''小鱼儿奇道:``那条绳子却到哪里去了?''

献果神君牙齿咬得``咬咬''作响,恨声道:``我瞧见那批红货,心里一欢喜,就未留意他,谁知这臭保镖的竟以火折子烧了。''小鱼儿叹道:``这端的是绝妙之计,你自然是想不到的,看来他早已有心将你固死在这里,自己早已决定要陪着你死,否则又怎会将你带到这真的藏宝之地。''沈轻虹唏嘘叹道:``不想你小小的年纪,倒真是我的知已,那时我想来想去,也只想出这一个地方能困死他,否则我真是死也不会将他带到这里。''小鱼儿道:``但这些日子来你两人是以何为生,却又令我不解。''献果神君大声道:``这自然又得靠我\ldots\ldots{}''

小鱼儿失笑道:``不错,猴子的别号就叫做献果,你却是献果神君,自然是有法子叫猴儿献果来的。''他话里虽然带刺,``献果神君''听来却反而甚是得意,大笑道:"猴儿们的脾气,天下还有谁比我摸得更清楚,我将石头从洞口抛出去,打它们,它们自然就会将果子从洞口抛进来打我们\ldots\ldots{}

小鱼儿道:``它们抛的若也是石头又如何?''

献果掷君咯咯笑道:``外面悬崖百丈,哪里来的石头''\ldots\ldots{}``小鱼儿点头笑道:''不错不错,猴儿们采果子,的确比捡石头容易得多,但\ldots\ldots 但就只这些,你们也吃得饱么?``献果神君道:''猴儿们吃什么,咱们便也能吃什么,猴儿们的食虽然不多,但咱们可也用不着去吃许多。``小鱼几瞧了瞧他们干枯瘦小的身子,忍住笑道:''这个倒可以瞧得出来的。``献果神君龇牙笑道:''你这小鬼也莫要得意,此后你吃的也就是这些,但你只管放心,这些年来我只瞧见你这么一个人,我绝不会饿死你的。``沈轻虹道:''我瞧这猴子脸也瞧得腻了,就算他要饿死你,我也不答应。"小鱼儿也不理睬,只是瞧着外面出神。

献果神君咯咯笑道:``今后咱们就是一家人了,说不定还要在一起活上个叁五十年,你叫什么名字,也该先说来听听。''小鱼儿道:``江鱼。''

小鱼儿忽然又道;``那批红货现在哪里?''

沈轻虹道:``你想瞧瞧么?''

小鱼儿道:``珍珠宝贝,瞧瞧也是好的。''

沈轻虹道:``好,跟我来\ldots。.''

献果神君喝道:那是我的,你碰一碰就打死你!``他瞪着眼睛发了半天威,终又笑道:''但让这小鱼儿见识见识也好。\ldots 也好让他知道某家有何本领。"一面说话,一面已自黑暗的角落中拎出了两口箱子。

那是两口生了锈的黑铁箱子,但箱子里却是珠光宝气,辉煌耀眼,献果神君眼睛己眯成一条线了,疯狂的笑道:``小鱼儿,你瞧见了么,这些本都是我的\ldots。本都是我的,我只要送你千分之一,已够你吃喝一辈子。''小鱼儿也不理他,只是盯着那些闪闪发光的珠宝出神,过了半晌,突然长长叹息了一声,道:可惜呀可惜!``小鱼儿悠悠道;''我只可惜你们见着我已太晚了些。``献果神君怔了怔道:''我们若是早些见着你又如何?``小鱼儿道:''你们若能早见着我一年,此刻便已在那花花世界中逍遥了一年,你们若能早见着我十年,此刻便已逍遥十年。``献果神君就像是只猴子似的不停的眨着眼睛,道:''你是说\ldots\ldots{}``小鱼儿道,''我是说你们若早见着我,我早已将你们救出去了."献果神君倒退叁步,瞪着小鱼儿,眼睛也不眨了,就好像小鱼儿鼻子上突然长出朵花似的。

献果神君已大笑起来,咯咯笑道:``你这小疯子,小牛皮,你能救咱们出去?''一把抓住沈轻虹,笑得几乎喘不过气,又道:``你听!你听见了么?这小子说能救咱们出去!他自以为是什么人?他只怕自以为自己是个活神仙。''沈轻虹凝目瞧着小鱼儿,瞧着小鱼儿那双透明的大眼睛,瞧着小鱼儿挂在嘴角的笑,一字字道:``说不定他真有法子.''献果神君道:``你\ldots\ldots 你居然相信这小鬼的话。''小鱼儿微笑道:``这只因为阁下脑袋的构造和在下有点不同。''献果神君怒道:``你的脑袋难道比我的管用?''小鱼儿笑道:``岂敢岂敢,在下的脑袋,也未必比阁下管用多少,只不过管用个一二十倍而已。''献果神君跳脚道:``放屁。''

小鱼儿道;``但阁下也莫要生气,像阁下的这种脑袋,也可算是不坏了,至于在下的这种脑袋,普天之下大概还没有第二颗。''献果神君怪叫道:``好,既然如此,你若说不出个法子,老子宰了你。''小鱼儿道:``我叁个月内若不能救你逃出这鬼地方,我脑袋输给你。''献果神君道:``叁个月''\ldots 哈,哈哈,你脑袋只怕有毛病,就算叁年``\ldots\ldots{}''小鱼儿道:``不必叁年,只要叁个月,但叁个月里,我若真的将你弄出这鬼地方了,你又当如何?''献果神君道:``我输你八个脑袋也没关系。''

小鱼儿笑道:``阁下的脑袋,携带既不便,送给李大嘴他也不吃的,一个已嫌太多,若真有八个,倒坑死我了。''他摇着手不许献果神君说话,接着笑道:``阁下若输了,我只要阁下翻几个筋斗让我瞧瞧也就是了。''献果神君暴跳如雷,道:``好,你这小鬼气我\ldots\ldots 好,我若输了,随便你如何就是,但你若输了,我非要你脑袋不可。''小鱼儿道:``一言为定。''

献果神君道:``老子放个屁也算数的。''

小鱼儿道:``但我只要将你救出去,无论用什么法子你可都得由我。''献果神君道:``好,老子全他妈的由你。''

小鱼儿道:``好,叁个月,从现在开始。''

突然抓起最大的一块翡翠,往洞外抛了出去!

\hypertarget{ux7b2cux4e8cux5341ux516dux7ae0-ux5de7ux8ba1ux8131ux56f0}{%
\chapter{第二十六章
巧计脱困}\label{ux7b2cux4e8cux5341ux516dux7ae0-ux5de7ux8ba1ux8131ux56f0}}

绿的翡翠纵在黑暗中也耀眼得很,沈轻虹本来一直含笑瞧着小鱼儿,此刻也不免吃了一惊,献果神君更是要急疯了,一把抓住小鱼儿,道:``你\ldots 你这小疯子,你可知道你在做什么?''小鱼儿笑道:``我自然知道。''

藏果神君跳脚道:``你可知道你抛出这一块翡翠,就等于抛出一栋平墙整瓦的大屋子,就\ldots 就\ldots 就等于抛出叁百条大肥牛。''小鱼儿道:``我自然也知道。''

献果神君道:``你\ldots\ldots 你这也算救我?你这简直是在要我的老命。''小鱼儿叹道:``你若要钱不要命,那也就罢了。''献果神君道:``但你\ldots 你你这又算什么意思?''小鱼儿冷笑道:``我的意思,早知你是不会懂的''。``但你难道也不懂么?''他这最后一句话问的自然是沈轻虹。

沈轻虹面上已有喜色,道:``在下虽有些懂,只是还不能完全明白。''小鱼儿道:``我将这些珍宝抛出去后。那些猴子猴狲们必定抢着去接,它们必定也和这位猴兄一样,见着此等稀奇好玩之物,是万万舍不得抛却的。''沈轻虹笑道:``不错。''

小鱼儿道:``我抛出去一百件珍宝,至少有五十件被它们接去,它们接去后必定带到各地去炫耀。这五十件珍宝,只要有一件被人瞧见,这人必定就要苦苦追寻这珍宝的来处。''沈轻虹道:``若换了我,也会如此的。''小鱼儿道:``这人独力难成,必定要找个同伴,而这种事只要被第二人知道,立刻就会有第叁人知道,有第叁人知道,就定会有第叁百个人知道。只要这消息一传出去,你就不怕没有人能找着这里。''沈轻虹附掌笑道:``不错,就算最无用的人,找寻珍宝时也会突然变得有用的,何况这消息一传出去,各种厉害角色都会赶来的。''小鱼儿叹了口气,道:``现在你懂了么,只要有人能来到这里,咱们就不愁出不去了,如此简单的法子,你们都想不出,可真是奇怪得很。''献果种君脸上的怒容早已瞧不见了,此刻竟一把抱住了小鱼儿,像是发了疯似的狂笑道:``你的的确确当真是天下最聪明的人。''于是,那些价值连城、大多数人一辈子赚来的钱也买不到一件的珍宝.就被小鱼儿像丢烂桃子、香蕉皮似的一件件丢了出去,他每丢一件,献果神君脸上的表情就像是被人砍了一刀似的,也不知是哭是笑。

此后,他每天越丢越多,只丢得献果神君脸皮发青,眼睛发绿,嘴里不停地喃喃嘀咕,道,``聪明人呀聪明人,你可知道你已丢出去多少银子了么?你丢出去的东西若作价成银子,只怕已可将这见鬼的悬崖填平了。''小鱼儿也不理他,到了第七天,献果神君额上已不停地往外直冒汗珠,捏紧了拳头嘶声道:``聪明人呀聪明人,你想出来的这条妙计若是不成功,你可知道你就要如何死法么?''小鱼儿淡淡道:``我丢光了这些珍宝,若是还没有人来,随便你怎样弄死我都没关系。''其实他自己的手也有些发软了,珍宝已不见了一半,还是鬼影子也没有来一个。

献果神君终于一把抢过那箱子,整个人坐在箱子上,大吼道:``不准碰,谁也不准再碰它一碰!''小鱼儿道:``难道你真的要钱不要命?''

献果神君咬紧牙关,道:``我为这些宝贝已吃了十五年的苦,宝贝若被你这小鬼弄光了,我就算能活着出去,又有什么意思?''小鱼儿眼珠子一转道:这话倒也不能说完全没有道理,但你不妨再想想,说不定只要再抛一粒珍珠出去,就有人来了,如此功亏一筏,岂不可惜。``献果神君摸了摸头,道:''这\ldots."

小鱼儿笑嘻嘻瞧着他悠悠道:``说不定只要抛一粒,只要一粒''献果神君终于大吼一声,跳了起来,道:``算你这小鬼的嘴厉害,老子又被你说动了。''有了一粒,就有两粒,有了两粒,就有叁粒\ldots 又好几天过去,还是鬼影子不见一个。

献果神君一把拎住了小鱼儿的衣襟,牙齿咬得吱吱的响,嘶声道:``你这小鬼还有何话说?''小鱼儿道:``说不定只要\ldots{}''

献果神君大吼道:``说不定只要再抛一粒,是么!''小鱼儿嘻嘻笑道:``正是如此。''

献果神君跺脚道:``放你娘的千秋屁,老子已被你害苦了,你还要\ldots\ldots 还要\ldots\ldots{}''两只猴爪般的手,已要去抓小鱼儿的脖子!

就在这时,突听沈轻虹``嘘''的一声,低叱道:``来了!''崖洞边,已探出了半个头来。

果然是人的头。这人的头发,正中央梳成个发髻,但原来戴在头上的帽子此刻却没有了,像是已被风吹落。

这人的眉毛,黑而长,眉尖微微上剔,看来颇有杀气,但眉心却纠结在一起,又像是有许多心事。这人纵有许多心事,却也无法自他眼睛里瞧出来。

他的眼睛大而凸出,眼珠子好像是生在眼眶外的,他的黑眼珠凝结不动,自眼珠上布满了血丝。这双布满血丝的眼睛,就这样瞪着崖洞里的叁个人,空空洞洞的,绝没有丝毫变化,丝毫表情。

这明明是人的眼睛,看来却竟又不像是人的眼睛,如此大的一双眼睛,看来竟全无丝毫生气!小鱼儿与沈轻虹、献果神君自然也在瞪着这双眼睛,瞪着瞪着,也不知怎地,心里竟不由自主生出一般寒意。

这全无丝毫表情、全无丝毫生气的一双眼睛,看来竟是说不出的冷漠、残忍、恐怖诡秘!那疑注的黑眼珠中,竟似带着这种逼人的死亡气息!

献果钟君忍不住大喝一声,道,"你这人是什么东西,你喝声未了,那颗头突然凌空飞了进来!

没有手,没有胸,没有身子\ldots.什么都没有,这赫然只是一颗人头,一颗孤零零的人头。

献果神君喝声已噎在喉咙里,呆呆地怔住,崖洞外却传人了一阵诡秘的猴笑,露出几张带着诡笑的猴脸。

小鱼儿松了口气,带笑骂道:``原来你们这些猢狲在捣鬼!''但这人头却绝计不会是猴子砍下来的。

沈轻虹拾起了人头,凝注着那双煞气凛凛的浓眉,凝注着那双凸出的眼睛,口中喃喃道:``却不知是谁杀死他的?''小鱼儿瞧着洞外将落的夕阳,悠悠道:``杀死他的人,想必就要来了!''但那``杀死他的人''却没有来。

漫漫的长夜已将尽,献果神君又开始坐立不安,蒙蒙的曙色渐渐照入这黝黑的崖洞\ldots\ldots\ldots\ldots 崖洞外突然伸入一只手来!

这只手五指如钩,像是想去抓紧件东西,但却什么也没有抓住,在凄迷的曙色中,这只手看来也是说不出的诡秘。献果神君风一般掠过去,刁住了这只手腕,他并未用什么力气,这只手就被他刁了进来!

但这也只是一只手,一只孤零零的手,已齐肘被人砍断,断处的鲜血已凝结,转变成一种凄艳的死红色,手背上还有条刀疤,长而深,就像是一条蛇蜷曲在那里,想来多年前这只手已险些被人砍断过一次。

诡笑的猴脸在崖洞外摇晃着,像是一张张用鲜血画成的画具,献果神君牙齿咬得直响,嘶声道:``脑袋先到,手也来了,下面只怕就是只臭脚。''小鱼儿道:``这脑袋和手不是同一个人的。''

献果神君冷笑道;``你怎知道?你问过他?''

小鱼儿道:``那脑袋的皮肤又细又嫩,这只手的皮肤却像是砂纸,你就算看不出,摸也该摸出来的。''献果神君道:``哼!''过了半晌,忍不住又道:``这只手莫非就是第二个人的\ldots\ldots{}''小鱼儿道:``不错,这只手就是砍下那脑袋的!''献果神君道:``你又知道了,你瞧见了不成?''小鱼儿道:``你瞧这只手,便该知道必定是孔武有力,若非这么样的手,又怎能一刀就砍下别人的脑袋。''献果神君道:``哼!''

小鱼儿道:``你瞧这只手的模样,也就该知道它被砍断前的那一刻,必定还紧紧握着柄刀\ldots\ldots 不但是刀,还是柄宝刀,所以,手一被砍断,那柄刀立刻就被人抢去了\ldots\ldots 一只有力的手拿着一柄宝刀,砍人的脑袋自然方便得很,想不到的是,这只手不知怎地也被人砍断了。''沈轻虹突然长长叹息了一声,道:``不错,这的确是只有力的手,他手里拿着的也的确是柄宝刀。''献果神君目光闪动,冷笑道:``嘿,你也知道了。''沈轻虹道:``我自然是知道的。那脑袋我虽不认得,这只手我却是认得的。''小鱼儿眉毛一扬道:``莫非是这刀疤?\ldots\ldots{}''

沈轻虹道:``不错,他手上这刀伤正是我留下的,却也是我为他敷的药,看着它收的口,我\ldots\ldots 我又怎会忘记?''他语声中竟似有许多伤感之意。

献果钟君嗤鼻道:``你砍伤了他,又为他敷药,你脑袋莫非有什么毛病不成?''小鱼儿眨着眼睛,道:``这一刀想必是误伤,所以你砍了他之后,心里又后悔得很,所以才会替他敷药,是么?''沈轻虹苦笑道:``正是如此。''

小鱼儿道:``如此说来,这人是你的朋友?''

沈轻虹又长长叹了口气,道:``此人便是昔年江湖上人称铁镖头,金刀手的金刀铁如龙,他与我本是好友,只为了争那总镖头之位,我。\ldots 我竟失手砍了他一刀,到后来我虽想补过,但他。\ldots 他却不告而别了,算将起来,这已是二十年前的事,二十年不见,不想今日竟,竟\ldots\ldots{}''转过头去,咳嗽不已。

献果神君道:``铁镖头,金刀手''\ldots 嗯,这名字我听过,听说他不但比你有种得多,武功也比你强,只可惜没有你诡计多端,所以才会被你砍了一刀。``沈轻虹黯然道:''我确是比不上他。"

献果神君皱起了眉,道:``此人武功本已不错,这二十年来,身受屈辱,想必朝夕苦练,武功自又精进不少,但还是被人一刀砍断了手,砍下他手的那人,岂非又是个厉害的角色,我们要加倍提防才是。''说完了这句话,他再不开口,只是盘膝坐到最黑暗的一个角落里,屏息静气,凝注着那洞口。

洞外面渐明亮起来,微风中也传来了夏日芬芳而温暖的气息,不时有猴子们怪笑着在洞外荡来荡去。

这阳光,这温暖的劳香气息,这无拘无束的自由\ldots\ldots\ldots\ldots\ldots 沈轻虹目中突然流下泪来,他扭转头,嘎声道:``你想。.。真的会有人来么?\ldots。真的会有人找到这里?''小鱼儿道:``会的。''

沈轻虹道:``但来的又会是什么人呢?他又是否会救我们出去?''献果神君狞笑道:``会的,他不救也得救。\ldots 无论他是什么人,我都不管,我只要他垂下来的那条绳子,那条绳子\ldots\ldots{}''沈轻虹道:``但他若要的不是你的人,只是你的珍宝,他若一进来就杀了你,又当如何?''献果神君狞笑道:``他杀不了我的,无论是谁也杀不了我的\ldots\ldots\ldots\ldots 他还未瞧见我在哪里时,我已经先宰了他。''沈轻虹道:``来的若是你的朋友,你莫非也\ldots\ldots{}''献果神君大笑道:``朋友?\ldots\ldots 这世上哪有我的朋友,我七岁之后便再无一个朋友,朋友这两个字我一听就要作呕。''沈轻虹缓缓合起眼,道:``好,很好。''

献果神君一字字道:``你两人若也想活着出去,就千万莫要做出糊涂事''。``你两人什么事都不做也没关系,只要在那人进来时,引开他的注意力,否则\ldots\ldots{}''突然``嗖''的一声一柄剑直飞进来。沈轻虹不等它撞上石壁,便已抄在手中,只见这柄剑青光莹莹,虽非宝器,却也是百炼精钢所铸。

献果神君厉声道:``人呢?''

小鱼儿悠悠道:``人?\ldots\ldots 想必也死了,这炳剑也是你的猢狲兄弟丢进来的,剑的主人若末死,如此利器又怎会落在猴子手里。''沈轻虹轻叹道:``不错,剑在人在,剑亡人亡\ldots\ldots{}''他轻抚着那精致而华丽的剑柄,以金丝镂在剑柄上的,正是``剑在人在,剑亡人亡''这八个字。

小鱼儿道:"配得上使用如此利器的人,想来也是位成名的剑客。

沈轻虹将剑柄送小鱼儿面前,道:``你瞧瞧这剑柄上除了这八个字外,还有什么?''除了八个字外,还有叁个以金丝镂成的圆圈。

小鱼儿眨眨眼睛道:"没有什么,只不过是叁个圈圈而已\ldots\ldots{}

沈轻虹喟然道;``不错,只不过是叁个圈图而已\ldots\ldots 但你可知道这叁个圈圈在武林豪杰眼中又有何等重大的意义?''小鱼儿道:``什么意思?''

沈轻虹沉声道:``就只这叁个圈圈,可使巨万金银易手,可令上千人马改道,可使势不两立的仇人握手言和,可令八拜相交的朋友反脸成仇。''小鱼儿笑道:``这叁个圈圈莫非有什么魔法不成?''沈轻虹道:``没有魔法,这叁个圈圈只是追魂夺命叁环剑客沈洋的标记,就凭这标记,大河两岸便可通行无阻。''小鱼儿道:``哦,这姓沈的居然有这么大的门道?''沈轻虹道:``这叁环剑正是当今天下十七柄名剑之一,那一招叁环套月在沈洋手中使出来,当真可说是\ldots\ldots{}''沈轻虹默然半晌,长叹一声道:``叁环剑客也死在这一役之中,倒真是我意料未及之事,如此看来,被你那些珍宝引来的武林高手,竟有不少。''小鱼儿笑道:此刻在这悬崖上面,必定打得热闹得很,只可惜咱们瞧不见。``沈轻虹黯然道:''不错,此刻这悬崖之上,必定已有许多武林朋友在流血拼命,而这些正都是你造成的后果,你本该为此悔疚才是\ldots。.``小鱼儿大笑道,''这些人为了些破铜烂铁竟不惜拼个你死我活,还说是什么武林高手,在我看来,简直是一群呆子,我不笑他们笑谁?``沈轻虹又自默然半晌,缓缓垂下了头,长叹道:''为了些身外之物而如此拼命,仔细想来,的确是愚不可及,但我\ldots\ldots 我又何尝不是如此。``小鱼儿道:''你若能常常和我说话,以后说不定会变得聪明些的。"这一日又在期待中过去,献果神君眼睛瞪得更大,日色渐暗,他眼晴就像两盏燃烧着碧磷的鬼灯。

子夜后,洞外仍瞧不见人影,但等到这一天的漫漫长夜又将尽时,洞外无边的黑暗中,突然传来了一片喧闹的、刺耳的、诡秘的笑声。这又是猴儿们的笑声。

小鱼儿皱眉道:``狲猢猢狲,半夜叁更,你们还吵什么?''沈轻虹沉声道:``猴性不喜黑夜,这些猴儿半夜如此喧嚷,必有缘故。''话犹未了,只听``叮当、哗啦''一连串响声,猴子们竟又自洞外抛入了十几件东西来,洞窟里一片黑暗,谁也瞧不清它们抛进来的究竟是什么,只听猴笑声渐渐远去,像是已完成了它们的任务。

小鱼儿摸索着,拾起了件东西,道:``这像是柄吴钩剑。''沈轻虹沉吟道:``吴钩剑?\ldots 这种兵刃近年江湖已不多见,吴钩剑的招式也已渐渐失传,但能使用此等兵刃的,却无一不是高手。''小鱼儿道:``看来又有个高手已送命了。''

他摸索着,又抬起件东西,沈轻虹道:``这件是什么?''小鱼儿道;``这东西圆圆的、滑滑的,还带着根练子,像是流星锤,却又不十分像,我也摸不出是什么?''沈轻虹沉吟道:``圆圆的?滑滑的?\ldots\ldots 呀,这莫非是江湖下五门中最歹毒的兵刃五毒霹霹雷霆珠!''小鱼儿道:``五毒霹雳雷霆珠,这名字倒威风得很。''沈轻虹道:``这五毒珠施展起来,招式也和普通流星锤并无不同,只是这铜球内还藏有暗器,若是有敌对方时,暗器使如暴雨般射出,纵是一流的高手,也难免被其所伤,是以这兵刃的主人杨露,在江湖中也可算是个人见人怕的角色。''他虽然告别江湖十五年,但说起武林秘事,仍是如数家珍一般。

小鱼儿笑道:``但看来这姓杨的小子,此番连看家的本领都来不及使出,便己送命了,要他命的人,岂非可算是武林中的超级高手''沈轻虹道:``你再瞧瞧还有什么?但小心些,莫要乱摸,此间既有下五门的高手到来,兵刃上说不定附有剧毒。''小鱼儿笑道:``我这样的人,会中别人的毒么?\ldots\ldots 我手上早已缠着布了,嗯,这里有柄刀像是九环刀。''他的手一抖,便发出一阵震耳的声响。

沈轻虹道:``听这声音,此刀像是十分沉重。''小鱼儿道:``的确重得很,只怕有五十厅。,沈轻虹道:''五十斤的九环刀,先声便足以夺人,看来此人的臂力武功,俱都不在金刀铁如龙之下,莫非是荡魔刀曾伦!``小鱼儿道:''这里还有只判宫笔,份量也重得很,能用如此沉重的兵刃打穴,这人的武功看来也不含糊。``沈轻虹道;''拿来让我瞧瞧。"

小鱼儿笑道:"你瞧得见么?该说让你摸摸才是。沈轻虹手指轻轻滑过冰冷而坚硬的笔杆,笔杆的握手处,像是刻着好几个字,他一个字一个字摸下去。

那上面刻的是``不义者亡''四个宇。

沈轻虹失声道:``果然是生死判赵刚,他\ldots 他难道也会死?''小鱼儿道:``人都会死的,这有什么奇怪?''

沈轻虹道:``但\ldots。但这生死判赵刚,可算是当今江湖中打穴的第一名家,一身小巧功夫,中原武林不作第二人想,又是谁杀了他?又有谁杀得了他!''小鱼儿道:``说不定他没有死,只是丢了兵刃。''沈轻虹叹道:``凡是江湖高手,必定都将自己成名的兵刃视如性命一般,这些兵刃既落入猿猴之手,他们的性命已不保!''这时已有微光照入洞窟,光线虽不强,但以沈轻虹等人的目力,已足以瞧清落在地上的兵刃是何模样。只见地上除了吴钩剑、五毒珠、九环刀之外,还有两柄剑,一根练子银枪,一对虎头钩,叁枚铁胆,两只暗器囊。

沈轻虹掀起一柄剑,这柄剑又轻又巧,刃薄如纸,沈轻虹道:``这是龙凤双飞鸳鸯剑中的雌剑轻凤,那雄剑神龙哪里去了?莫非已被人拆散\ldots\ldots 唉!龙凤剑客一世英雄,江湖人尝言龙风比翼,翱翔九天,谁知到头来还是要龙拆风散遭人毒手!''他叹息着放下了这柄``轻凤''剑,目光缀然,自练予枪、虎头钩等兵刃上一一望了过去,叹息更是沉重,喃喃道:``这些人竟会死在这一役之中,当真令我梦想不到,看来这一役战况之惨烈,只怕已是百年仅有的了。''小鱼儿道:``这些人不但死了,而且显然是同时死的,能同时杀死这许多成名高手的人,可真是了不起。你能猜得出他是谁么?''沈轻虹道:``当今天下能使这许多一流高手同时毙命的人物虽不多,但算来也有七八个,其中武功最高,下手最毒的,自然是推移花宫中的两位官主!''说到``移花宫''叁字,他语声竟也似有些变了,四下瞧了一眼,像是生怕那美如天仙、但却狠如魔鬼的两位宫主突然自黑暗中出现似的。

小鱼儿笑道:``你放心,她们绝不会到这种鬼地方来的。''沈轻虹喘了口气,道:``不错,那两位宫主是天上仙子,又怎会为了区区世俗珍宝出手,下手的绝不会是她们。''小鱼儿道:``除了她们还有谁?''

沈轻虹道:``昔年十大恶人中,武功最高的血手杜杀与狂狮铁战,只怕也有这么样的手段!''小鱼儿道:``这两人也不可能。''

沈轻虹道:``不错,这两人一个已多年不知下落,据闻早已投入恶人谷,至于狂狮铁战么?\ldots\ldots 唉这些人若是被他杀的,连兵刃都早已要被折成一段段的了,又怎会像此刻这般完整。''小鱼儿道:``还有呢?''

沈轻虹道:``还有几人,名字不说也罢。''

小鱼儿道:``为什么?''

沈轻虹道:``只因这几人武功虽强,但轻财仗义,俱都是一代之大侠,那是万万不会做出此等事来的,譬如说当今天下第一剑客燕南天!他老人家要杀这几人,虽然易如反掌,但若非不仁不义之人,他老人家宁可自己受苦,也不会出手的。小鱼儿本就在等他说出''燕南天``这名字,如今听得他如此推祟,胸中不禁热血奔腾,大声道:''好!好男儿!男子汉若活在世上,就要活得像燕南天,教人一提起他的名字,就要竖起大拇指。``沈轻虹瞪着献果神君,大声道:''不但受过他老人家好处的人,人前背后都对他老人家五体投地,就算是他老人家的仇人,背后也不敢对他老人家稍有闲话。``献果神君冷笑道:''嘿嘿,你以为我不敢骂他?``沈轻虹霍然站起,厉声道:''你敢?"

献果神君叹了口气,道:``我虽想骂他两句,却不知该如何骂法。''沈轻虹大笑道:``你听见了么,纵有想骂他老人家的人,也不知该如何骂起,只因他老人家平生实未做过一件见不得人的事,我虽有十五年未见他老人家,但此等上无愧于天、下无愧于人的大英雄,身体必定日更强健,你说是么?''小鱼儿道:``不错,他身子必定十分强健!他活得必定好得很''\ldots\ldots"说着说着,他眼睛像是有些湿了,赶紧垂下头,拾起了一只暗器囊,将里面的暗器全倒了出去。

只见那里面有十叁枚毒针,七枚黝黑无光的铁蒺藜,还有一大堆毒砂,沈轻虹耸然失色,道:川中唐门也有人栽在这里!``小鱼儿道:''下手的这人,既不会是你方才已说过的那几位,又不会是你还没有说过的那几位,那么,他究竟会是谁呢?``沈轻虹叹道:''想来我委实也难以猜测。"

小鱼儿伸了个懒腰,道:"你猜不到也罢,反正他这就要来了,咱们等着瞧吧。

献果神君圆睁的双目中,已露出惊怖之色,虽然他确信自己的武功,在如此黑暗中骤施暗袭,必能得手!但这即将到来的不可猜测的敌人,武功委实太强!委实令人胆寒,他一击若是不中,只怕便难有第二次出手的机会了!

有风吹动,崖洞外突又伸出了一只手来。这只手纤细、柔美,每一根手指都像是白玉雕成,纵是世上最再吹毛求疵的人,也无法在这只手上挑出丝毫瑕疵来。但在这穷崖绝洞外,突然出现这么美的一只手,却显得更是分外诡秘,在沈轻虹等人眼中,这只毫无瑕疵的纤纤玉手,实似带着种凄秘的妖艳之气,实令人不得不怀疑这只手是否属于人的。一时之间,献果神君却似已将窒息.说不出话来。

只见这只手轻轻在洞边的崖石上敲了敲──这只手动了,手指也动了,绝不会再是死人的手。

然后,一个温柔而甜美的语声在洞外银铃般笑道:``有人在家么?''此时此地,这甜笑的语声说的竟是这样的一句话,就好像是邻家的少妇闲来无事走过来串门子似的。献果神君与沈轻虹听在耳里,心中却不禁直发毛,两人面面相觑,简直是哭笑不得,更不知该说什么。

小鱼儿眼珠一转却笑道:``有人在家,有好几个哩!''那语声笑道:``有人在家,就该出来开门呀!''小鱼儿道:``昨天我吃了人家的梨膏糖没付钱,大门己被人扛走了。''那语声银铃般笑道:``我在外面站得腿发软,可以进来坐坐么?''小鱼儿道:``当然可以,但你可得小心些走呀,门槛高得很,莫要弄脏你的新裙子。''那语声道:``谢谢你啦。''

\hypertarget{ux7b2cux4e8cux5341ux4e03ux7ae0-ux8131ux56f0ux5165ux56f0}{%
\chapter{第二十七章
脱困入困}\label{ux7b2cux4e8cux5341ux4e03ux7ae0-ux8131ux56f0ux5165ux56f0}}

一个轻衫绿裙、鬃边斜插着朵山花的少妇,盈盈走了进来,她步履是那么婀娜,腰肢是那么轻盈。她自那百丈危崖外走进来,当真就像是邻家的小媳妇跨过道门槛,就连那朵山茶花还都是稳稳的戴着,仅有歪一点。

黑暗中,献果神君已飞扑而出,挟着一股不可挡的狂风,直扑那看来弱不禁风的少妇。绿裙少妇粹不及防,眼见就要被震出去,但腰肢不知怎地轻轻一折,她身子已盈盈站在献果神君身后。

献果钟君一惊,猛回身,待二次出手。绿裙少妇已向他嫣然一笑,柔声道:``您要我出去,我这就出去,您又何必费这么大的劲,生这么大的气呢。''那妩媚甜笑的笑容,美得像花,甜得像蜜。

献果神君道:``你\ldots\ldots 你\ldots\ldots{}''

他虽然凶横霸道,奸狡毒辣,但面对着如此温柔、如此美丽的女子,心还是不免有些动了,狠话再也说不出口。

绿裙少妇道:``老爷子您着喜欢我留在这里,我就留在这里,替你扫地煮饭补衣服\ldots\ldots{}''``小鱼儿一直在瞪着眼睛瞧她,此刻突然笑嘻嘻道:''我看你不如做我的媳妇吧。``绿裙少妇媚然笑道,''你若真的肯要我做媳妇,我真开心死了,像你这样又聪明、又英俊的丈夫,我找了十年却没找到,只可惜\ldots\ldots{}``小鱼儿道:''只可惜什么?"

绿裙少妇柔声道:``只可惜我的年纪太大了,等你叁十岁的时候,我已经是老太婆了,那时你又想甩了我,又不忍心,岂不是让你为难么?我又怎忍心让你为难呢?''小鱼儿明知她说的全没有一句真话,但不知怎地,听在耳里,心里还是觉得舒服得很,忍不住大笑道:``你不说我年纪太小,只说自己年纪太大,像你这么说话的女子,就算是个杀人不眨眼的母夜叉,我也是喜欢的。''绿裙少妇嫣然道:``不管你说的是真是假,这句话我一定永远记在心里。''献果神君嘎声道:``我若不喜欢留在此处又当如何?''绿裙少妇道:``老爷子若觉得这里太气闷,想出去逛逛,我已在外面备好了梯子,老爷于您随时都可以走。''献果神君嘶声道:``真的?''

绿裙少妇道:``老爷子你若还不放心,只管先上去,然后咱们再上,留下这位少爷最后再带着箱子走,这样老爷子既可放心咱们,咱们也可放心老爷您了。''献果神君心里虽然一万个不愿意听她的话,但她的话实在说得入情入理,实在说入了他的心,实在令他不能不听。就连沈轻虹,心里虽也明知这女子必定是个杀人不眨眼的女魔头,但也像是入了魔似的,听得只有点头。

两人想来想去,找来找去,也找不到她有任何恶意。她说的话委实面面俱到,不但替自己想过,也替别人想过,无论是谁,都再也想不出更好的法子.小鱼儿附掌道:``这法子的确再好也没有,别人若先上去,猴老兄必定不放心,此番猴兄先上去,也要等着最后一批珠宝上来,必定不会割断绳子。''献果神君瞪着那少妇,还是忍不住问道:但你。\ldots 你真的是完全出于善意么?``绿极少妇柔声道:''老爷子您想想我会有什么恶意呢?``献果神君大喝道:''世上真有你这么好的人?``绿裙少妇轻叹道:''我生来就是这样,只知替别人着想,替别人做事,自己也没法子。``献果神君眼珠子转来转去,但左看右看,也实在看不出她究竟坏在哪里,只得跺一跺脚道:''好,无论你是好是坏,先上去再说!"他心中其实早巳迫不及待,那阳光,那暖风,那自由的天地,早已似乎在向他不断地招手。

他探头一瞧,果然有条粗如儿臂的长索从上面直垂下来,这长索若会中断,那么这绿裙少妇自己也要被困在地,只要这长索不会中断,那么,纵有别的诡计,他也要先上去了再说。

献果神君算来算去,只觉已无遗策,当下再不迟疑,纵身一跃,攀住了索头,大笑道:``沈轻虹,你跟着\ldots。.''笑声未了,身子突然一阵扭曲,向那万丈绝壁中直坠了下去,得意的笑声,也变做了凄厉的惨呼。

沈轻虹大惊失色,失声道:``这,这\ldots.''

那绿裙少妇的脸像是也吓白了,颤声道:``这。\ldots 这是怎么回事?''沈轻虹霍然回身,厉声道:``这原该问你才是!''绿裙少妇道:``莫非是他老人家年纪太大,连绳子都抓不住了?沈轻虹忽道:''老实说,你这绳子上究竟有何鬼怪?``绿谣少妇眼睛就像秋水般明亮、婴儿的无辜,柔声道:''这绳子是好好的呀,又没有断,我方才不就是从上面下来的么?你若不信,不妨拉拉看。``沈轻虹果然伸手去拉,小鱼儿突然笑道:''这绳子里若是藏着几根毒针,伸手去拉的人滋味一定不太好受。``他话未说完沈轻虹的手早巳闪电船缩回来,厉声道:''不错,这绳头里必定暗藏毒针,否则献果神君又怎会松手,不想你这女子竟是如此狠毒,我今日才算开了眼了!``绿裙少妇目中泪光莹莹,凄然道:''你们要如此说,我也没法予,既是如此,我\ldots。我只有自己拉给你们瞧吧。"她纤腰一扭,自己果然攀上长索。

沈轻虹眼睁睁瞧着她往上爬,那舞着的绿裙少妇看来已越来越小,他心里又着急,又后悔,要他们跟着这不知究竟是温柔还是毒辣的女子往上爬,他实在有些不敢,但耍他眼睁睁瞧着这机会错过,却又实在令人痛心。

他正在为难,不知是否该冒险一试,哪知就在这时,那不可捉摸的女子竟又轻轻滑了下来。

小鱼儿笑道:``我早已知道你会回来的。''

绿裙少妇柔声叹道:``我本来已想不管你们,但又实在不忍心,唉!我的心为什么总是这么软,简直连我自己都不知道。''她眼被轻轻一扫,对沈轻虹道:``这绳子究竟是好是坏,如今你们总该知道了吧。''到了此刻,沈轻虹委实不知道该相信谁了,他甚至已有些怀疑献果神君真是自己抓不住绳子才跌下去的。

绿裙少妇悠悠道:你若还不相信,不妨用块布包着手。"沈轻虹瞧瞧那绳子,又瞧瞧洞外的青天白日,再瞧瞧这阴森森黝黝的洞窟,想着那十五年苦难的岁月。

这机会委实不容再错过。

他咬了咬牙,最后再瞧了瞧小鱼儿。小鱼儿也皱紧了眉,道:``你莫瞧我,我也没了主意,但是\ldots\ldots 我想这绳子总该不会断的吧,否则她自己也上不去了。''沈轻虹长叹一声,道:``事到如今,无论如何我也要试一试了。''他纵身一跃,攀持而上。

小鱼儿拎起一颗心,眼睁睁瞧着他往上爬,一尺,两尺\ldots\ldots 眼贝他已爬上十余丈,小鱼儿终于松了口气。瞧着那少妇笑道:``你这人究竟是好是坏,到现在我也弄不清了\ldots。.''话未说完,绳子已断了。

沈轻虹惨呼着,挣扎着,自洞口直坠而下,眨眼便瞧不见了,只剩下那凄厉的惨呼响彻四山。

小鱼儿目瞪口呆,怔在当地,呐呐道:``你\ldots\ldots 你\ldots。你真是个骗死人不赔命的女妖怪。''绿裙少妇嫣然笑道:``哦!是么?''

小鱼儿道:``你用绳子里的毒针毒死那老猴子,又将绳子割断一半等着沈轻虹来上当,但以你的武功,你本来不必费这么多心思,就可杀死他们的呀!''缘裙少妇嫣然道:``要自己动手杀人,那多没意思,我一生中从未自己动手杀过一个人,全都是别人心甘情愿去死的。''小鱼儿道:``但我还是不明白,绳子断了,你自己怎么上去。''绿裙少妇道:``这里舒服得很,我已不想上去了。''小鱼儿怔了怔,摸着头苦笑道:``女孩子说的话能教我猜不透的,你是第一个。''缘裙少妇凝注着他,柔声道:``你的朋友被我害死了,你不想报仇?''小鱼儿叹道:``我打也打不过你,骗也骗不过你,怎么样报仇,何况,正如你所说,这不是你迫着他们,面是他们自己心甘情愿送上门来上当的。''绿裙少妇道:``你心里不难受?''

小鱼儿道:``这两人一个是早巳该死了,另一个是十五年前自己不想活了,如今死得正是对门对路,我又难受个什么?''绿裙少妇眼波流转,咯咯笑道:``你这样的孩子,我才真是从来没有见过。''小鱼儿笑道:``好,现在你可以开始骗我了,骗到我死为止。缘裙少妇道:''骗死了你,我一个人在这里岂非寂寞得很\ldots\ldots{}

小鱼儿瞪大眼睛,道:``你。\ldots 你自己难道真的也不上去了?''绿裙少妇道:``我又没生翅膀,又不会飞!''

小鱼儿楞了半晌,苦笑道:``你真是女妖怪!''缘裙少妇道:``我若是女妖怪,你就是小妖怪。''小鱼儿叹道:"这倒不错,一个女妖怪,一个小妖怪,在这鬼洞里过上一辈子了,将来说不定还会生了一大群小小妖怪\ldots。.他话末说完,绿裙少妇已笑得直不起腰来。

突然间,一阵狂笑声远远传了过来。

一个狂笑道:``姓萧的鬼丫头,你跑不了的,老于已知道你从哪里下去的,老子就在这里等着你,除非你一辈子也不上来!''这话声显然是来自云雾凄迷的山头,但听来却如就在你耳畔狂叫一般,震得你耳朵发麻。绿裙少妇面色立刻变了,变得比纸还白。

小鱼儿道:``他是什么人?''

缘裙少妇道:``他\ldots。他不退人,他简直是个老妖怪!''小鱼儿道;``你真那么怕他?''

绿裙少妇摇头叹道:``你不知道,不知道\ldots\ldots 他做出来的事,世上永远没有人能猜得透的。''只听那语声又喝道:``姓萧的,你真不上来么?''绿裙少妇咬住嘴唇,不说话。

过了半晌,那语声又道:``好,老子数到十宇,你若还不上来,等老予捉到你时,担保要你受十天十夜的活罪,若让你少受一刻,老子都不是人!''小鱼儿眨着眼睛,叹道:看来,他果然有叫人连死都死不了的本事。``那语声已大吼道:''现在开始!─!"

绿裙少妇整个人都像是已被吓软了,瘫到地上,动也不能动,鬓旁的山茶花,却簌簌的抖个不住。

那语声已喝道:``二!''

小鱼儿眼珠一转,道:``这□如此凶恶,莫非是十大恶人之─?''绿裙少妇叹道:``十大恶人若和他出起来,简直就像是最乖的小孩子了。''小鱼儿也吃了一惊,道:``他比十大恶人还狠?''只听那语声又喝道:叁"

小鱼儿呆了半晌,道:``他叫什么名字?''

绿裙少妇道:``你不会知道他的。''

小鱼儿道:``他既然比十大恶人还狠,就应该很有名才是。''绿裙少妇长叹道:``咬人的狗是不叫的,你知道么!越是没有名的人才越厉害,他就算做了神鬼难容的事,别人也不知道。''那语声又喝道:``四\ldots\ldots 好,看样子你是真的不上来了,你要不要听听老子捉到你时,要如何对付你。''他像是已在暴跳如雷,狂吼道:``老子捉到你时,先挖掉你一只眼睛,再把盐水灌进去,等到十天后,你全身都要变成咸肉。''小鱼儿苦笑道:``好凶的人,这样的活咸肉,只怕连李大嘴都没有吃过。''缘裙少妇突然道:``伤认得李大嘴?''

小鱼儿眨了眨眼,反问道:``你认得他?''

绿裙少妇默然半晌,悠悠道:``在江湖中混的人,谁不知道他!''只听那语声已狂吼道:``五!..\ldots 你听到了么!五!再数五下,你就要完蛋,你若以为老子捉不到你,你就大错特错了!''绿裙少妇突然站了起来,长叹道:``罢了。与其等着被他捉住,倒不如现在先死了干净。''小鱼儿道;``你\ldots。你怕什么?咱们等在这里不上去,他反正也不敢下来的。''绿裙少妇叹道:``你不知道,他说过的话,从来没有不算数的,他若说能够捉住我,就是真的能捉住我。''小鱼儿道:``你不能死,你死了我一个人在这里多寂寞。''绿裙少妇凄然一笑,道:``你还想活么?''

小鱼儿道:``我活得正有意思,为什么不愿活?''绿裙少妇摇头叹道:``他连你也不会放过的\ldots。.''那话声大叫道:``六!现在已数到六了!''

绿裙少妇道:``他总有法子捉住你,我若死了,他一定要将气都出在你身上,那时你就更惨了!''她一面说话,一面缓步走到洞口。

小鱼儿道;``你要跳下去?''

绿裙少妇道:``依我看来,你也是和我一起跳下去的好。''小鱼儿失声道:``你要我也跳下去?''

绿裙少妇突然回身,凝眸瞧着他,缓缓道:``我一个人死也觉得寂寞得很,你肯陪陪我么?''小鱼儿摸着头,喃喃道:``叫人陪着她一起死,免得她寂寞\ldots 嘿!这种要求倒也少见的。''绿裙少妇悠悠道:``我是喜欢你,才要你赔我一起跳下去,否则,否则\ldots\ldots 你是死是活,我才不管你哩。''那吼声己喊道:``七!''

小鱼儿瞧着她,瞧了很久,才道:``你喜欢我?''缘裙少妇缓缓道:``你是聪明人,这难道瞧不出?''小鱼儿又瞧了她很久,突然大声道:``好!我陪你一起跳下去!''绿裙少妇也像是有些意外,失声道:``真的?小鱼儿道:''我非但陪你跳,还要抱着你跳。``绿极少妇又凝眸瞧着他,也瞧了很久,缓缓道:''好\ldots。你很好。``那吼声道:''八!还有两下子,臭丫头,你的命已不长了!``小鱼儿果然跳上去,紧紧抱住了她,居然还能笑道:''你真香\ldots。我抱着你死,倒真不错。``绿裙少妇突然一笑道:''你真是个可爱的孩子,能被你抱着死,更是件不错的事.``那语声大吼道:''九!臭丫头,你听到了么?老子现在已数到九!``绿裙少妇道:''你抱好了么?抱紧些,我就要跳了!``小鱼儿道:''你跳吧!"

他闭起眼睛,长长叹了口气,道:``死,不知道究竟是何滋味。''绿裙少妇道:``你马上就要知道了''\ldots\ldots"

身子一跃,竟真的向那深不见底的绝壑跳了下去!

他只觉耳朵里都灌满了风,身于往下直坠,这时如说他心里害怕,倒不如说他觉得很有趣、很舒服。无论如何,自百丈高处往下跳,有这种经验的总不多。

也许小鱼儿连``害怕''这两个字都已被吓得忘了,也许他起先根本不相信这绿裙少妇会真的往下跳。

他只觉得越来越快,下半身已似和上半身分了家。这时他心里只有一个念头──他在问自己:``我究竟是聪明?还是糊涂?''就在这时,只听``蓬''的一响,他身子似乎一震,下落的势道突然缓了。

只听绿裙少妇在他耳畔轻笑道:``死的滋味如何?''小鱼儿道,``不错!还不错\ldots\ldots{}''"

他已张开眼,左右一瞧,两旁山壁的树木,都可瞧得很清醒,像是一栋株树都在往上飘。由此可见,他们下落的势道,竟已慢得出奇。

绿裙少妇笑道:``你可知道,你是个幸运的人,虽然尝过了死的滋味,却不必真的死了。''小鱼儿道:``这\ldots\ldots 这究竟是怎么回事?''

绿极少妇道::``抬头瞧瞧。''

小鱼儿一抬头,便瞧见了一样奇怪的东西,这东西像是伞,又不是伞,至少也比伞大了十倍。

这东西竟是从绿裙少妇背后撑出来的,看来像是用无数根细绳系着的一柄五色的大伞。这``伞''兜住了风,他们下落之势自然缓了。

小鱼儿就像是坐在云上往下落似的,那滋味可真妙极了,他忍不住放声大笑,大声道:``这玩意儿真不错,真不知你是如何想出来的。''突然,他只觉身子一震,已落在实地上。那柄``伞''边带着风,带着他们往外滚。

绿裙少妇自裙子里抽出柄小刀,割断了绳子,娇笑道:小鬼,你现在可以放开手了。"

小鱼儿手却抱得更紧,道:``我偏不放开你,你骗得我好苦,我被你骗得差点没发疯,你总该让我多抱抱你,算做补偿。''绿裙少妇笑道,``你这小鬼,你究竟是个聪明人,还是个呆子?''小鱼儿笑嘻嘻道:``这句话我刚刚还问过我自己,我自己也回答不出。''绿裙少妇道:``我瞧你呀,是个不折不如的小呆子。''小鱼儿突然跳起来,大眼睛里闪着光,瞪着她道:``你以为你真骗倒了我?''绿裙少妇也笑眯眯瞧着他,道:``你自己不知道?''小鱼儿大笑道:告诉你,我早就知道你不会死的,所以才陪着你往下跳,你这种人,不像是会自己寻死的人!``绿裙少妇眨了眨眼睛,道:''哦!是么?"

小鱼儿挺起胸,大声道:``告诉你,世上没有一个人能骗得倒我江鱼。''绿裙少妇瞧着他,柔声道:"我现在才发觉你已不是个孩子,而是个大人,是条男子汉,我几乎从未见过像你这样的男子汉\ldots\ldots{}

她眼被里像是充满了赞美之意,小鱼儿的胸脯挺得更高了,他也突然发觉自己不再是孩子,已突然长大了。

绿裙少妇眼波四转,突又长叹道:``我虽然没有死,但到了这里,我又没法子,现在。\ldots 我什么事只有依靠你,你可不能抛下我。''小鱼儿只觉自己从来没有像现在这样强壮,这样有勇气,他觉得自己实在不错,否则她又怎会全心全家地依赖自己。

他大声道:``你只管依靠着我,我绝不会后悔。''绿裙少妇嫣然一笑,道:``你真好,我知道我不会选错人的。小鱼儿笑道:''你当然没有选错,你选得正确极了。``绿裙少妇愉快地叹了口气,道:好,你现在快想个法子,让咱们离开这鬼地方吧。''小鱼儿道:``好。''

他刚说完这``好''宇,嘴虽说得甜,心里却已发苦。

只因他已瞧清了这``鬼地方''。

他实在不知道有什么法于子离开这里。

这里,就像是一个酒瓶的瓶底,就算是有蟑螂那么多脚,那么强的生存力,也休想爬得上去。

奇怪的是,这里并不如他们想象的那么阴湿。这里竟丝毫没有潮气,反而是温暖而干燥的,在上面看到的那凄迷的云雾,距离他们头顶还很高。

他脚下踩着的,也不是沼泽湿泥,而是非常令人愉快的草地,柔软的青草,看来就好像是张碧绿的毯子。明亮的光线中,充满了芬芳的香气。

四面枝叶茂密的树林.树木间还点缀着一些鲜艳的花草,小鱼儿几乎要以为自己突然跌落在仙境里。

这仙境唯一可怕的,就是那无边的静寂,没有风,也没有声音,每一根草,每一片叶子,都是绝对静止的,看来,竟像是没有丝毫生气。

这可怕的静寂.简直要令人发狂!这美丽的``仙境'',竟是块死地``绿裙少妇柔声道:''你已想出了法子么?"

小鱼儿再也笑不出来,不住道:``有法子的,自然有法子的。''缘裙少妇道:"好,我什么都听你的。她温柔地瞧着他,果然不再说话。

小鱼儿背负着手,兜了十七八个圈子,突然大声道:"不对!

不对!"

绿裙少妇道:``什么事不对?''

小鱼儿道:``这里少了样东西?''

绿裙少妇道:``少了东西?什么东西?''

小鱼儿苦着脸道,``那老猴子和沈轻虹两人到哪里去了?飞上天了么?''绿裙少妇道:``他\ldots 他们不是已摔死了么?''

小鱼儿道:``不错,摔死了,但尸身呢?我所有的地方都瞧过,竟瞧不见他们一根骨头,就算是被老虎吃了,也吃得没有这样快呀,何况,这里简直连只猫都没有,哪里会有什么老虎。绿裙少妇脸色也变了,失声道:''你真的没有瞧见他们的尸身?``小鱼儿道:''没有,简直连一根骨都没有。"

他嘴里虽这样说,但还是有些不相信自己,一面说,一面又到四下搜寻起来,绿裙少妇也跟着他找。这地方并不大,他们很快的就找了两叁遍,每个角落,每一株树下,每一块草皮都找遍了。

这里非但没有骨头,甚至连一点血迹都没有──这里简直丝毫没有两个人跌死的痕迹。

小鱼儿突然有些害怕了,道:``这见鬼的地方,莫非真的有鬼!''绿裙少妇身子缩了缩,强笑道:``鬼,哪里会有鬼?''小鱼儿道:``若没有鬼,那两个人哪里去了?就算他们没有摔死,也该在这里呀,何况,他们是绝对不可能不摔死的。''``但这地方必定有古怪,我必定能找出这古怪究竟在哪里!''说着,又到四面去搜索起来,但树还是那几株树,草还是那几片草\ldots\ldots{}``小鱼儿又大叫道:''这里必定还有别的人。``绿裙少妇道:''这鬼地方会有人?"

``因为若是野生的草地,会这么整齐?这么干净?所以,我想这里一定有人住,一定有人时常修剪草地。''绿裙少妇展颜道:``呀,不错,你不但头脑好,眼睛也好\ldots\ldots,这里既然有人住,我就放心了。''她瞬间又皱眉,颤声道;``但\ldots 人呢?''

小鱼儿道:``人''─``人\ldots。.''

他四下去瞧,这里连鬼影都没有,哪里有人?

谜,不可思议、无法解释的谜。

绿裙少妇道:``我\ldots\ldots 我简直想都不敢想了,我一想就要打寒噤。''小鱼儿大声道:``你不必想,由我来想,我想已足够了。''其实他也想不通,他想得头都疼了。

天色,已渐渐暗下来,暗得很早。小鱼儿不停地在四下走,肚子已饿得直冒酸水。

小鱼儿也快急疯了。

他常常说:世上没有办不到的事。

现在,他突然发觉说这话的人不是疯子就是傻瓜。

他更不敢去瞧那绿裙少妇,这女人说一切都依靠着他,她真是选错人了,她眼眼一定有毛病。

到后来小鱼儿简直已发晕了,喃喃道:``睡觉吧,好歹睡一觉再说,最好能一睡不醒\ldots\ldots{}''突然绿裙少妇娇唤道:``过来\ldots\ldots 快过来!''

小鱼儿─回头,已瞧不见她的人,大声道:``你在哪里?你也学会隐身法了么?''绿裙少妇道:``我在这里,在这里!''

这呼声竟是从一株树后传出来的,这株树根粗、很大,叶子特别缘,小鱼儿早就疑心其中有古怪,却瞧不出来。

他飞快地跑过去,只见绿裙少妇跪在那株树后,像是在祈祷似的,动也不动,只是眼睛却瞪得很大。

小鱼儿皱眉道:``你在干什么?拜菩萨?''

绿裙少妇招手道:``你快过来,瞧瞧这里。''

小鱼儿只得也蹲下来,瞧了半晌,道:``这没有什么呀,不过是\ldots。呀,不错,有了!!''他突然发现这株树下半截的树皮,竟和上半截不同,上半截的树皮粗糙,下半截的树皮却光滑得很。

绿裙少妇道:``你瞧,这树皮像是常常被人用手摸的,人为什么要摸这树皮,显然只有一个解释''\ldots 这株树必定就是道门。``小鱼儿展颜道:''你不但头脑好,眼睛也不错。``绿裙少妇嫣然道:''谢谢你。"

小鱼儿眨了眨眼睛,伸手在树上敲了几下,笑嘻嘻道:``有人在家么?''

\hypertarget{ux7b2cux4e8cux5341ux516bux7ae0-ux7a74ux91ccux4e7eux5764}{%
\chapter{第二十八章
穴里乾坤}\label{ux7b2cux4e8cux5341ux516bux7ae0-ux7a74ux91ccux4e7eux5764}}

小鱼儿有个特别的脾气,随时随地都要开玩笑,但他这玩笑开得也并非没有用意,他想试试这株树是空心还是实心。

他做梦也不想里面会有人回应。不错,里面的确没有回应,但那块树皮却突然移动起来,好好的一株树,竟突然现出了个门户!

小鱼儿这一惊倒是不小,整个人都吓得向后飞出去。绿裙少妇也像是吓惨了,竟跪在那里不能动。

树,果然是空的。小鱼儿瞪着那黑黝黝的洞,大声道:``什么人在里面?是人是鬼,都给我滚出来。''树穴里没有声音,一点声音都没有。小鱼儿一步步走过去,拳头捏得很紧,捏得指节都发了白,那双本来就不小的眼睛,瞪得更大。

绿裙少妇颤声道:``不要走进去,里面\ldots\ldots 里面说不定有什么东西。''小鱼儿大声道:``怕什么?这种鬼鬼祟祟的东西,没什么可怕的,他若真的很厉害,为什么不敢出来见人!''绿裙少妇道:``你\ldots 你要进去?''

小鱼儿身子也缩了一下,道:``进\ldots 进去\ldots{}''

他咳嗽一声,大叫道,``自然要进去,这是唯一的线索,我怎么能不查个明白!''皇后。"

小鱼儿呆了半晌,突然大笑起来,笑得几乎喘不过气,他一生中简直从来没有像这样大笑过。

绦裙少妇道:``你开心么?''

小鱼儿大笑道:``我开心,开心极了,我什么疯狂的事都想到过,但却做梦也没有想到我有朝一日竟会做皇后。''缘裙少妇道:``你不愿意?''

小鱼儿瞪大眼睛,道:我为什么不愿意?世上又有几个男人能当皇后?``他突然跳起来往桌于上一坐,大声道:''喂,你们还不过来拜见你们的新皇后么?"那些轻衫少年你瞧着我,我瞧着你,终于一齐走过来。

小鱼儿道:``只要磕叁个头就够了,不必太多。''少年们一齐去望那绿裙少妇,绿裙少妇不停的娇笑,不停的点头,少年们想不磕头也不行了。

小鱼儿道:``磕完头就出去吧,我要和皇上喝酒了,快出去.\ldots。妃子若想和皇后争宠,皇后吃起醋来,是要砍你们脑袋的。''少年瞧着他,那模样倒当真像是瞧见了个妖怪似的,突然一齐转过头,走了干净。

小鱼儿拍手大笑道:``妙极妙极,做皇后的滋味可真不错。''绿裙少妇笑得已直不起腰,咯咯笑道:``你这小鬼真有意思,我在这里十多年,从来也没有这样开心过。''小鱼儿笑道:``从今以后,我天天都要让你开心,开心得要死,你虽然叫迷死人不赔命,我却要迷死你。''绿祖少妇突然不笑了,瞪大眼睛,道:``你''\ldots\ldots 你怎会知道我的名字?``小鱼儿笑嘻嘻道:''我非但知道你这名字,还知道你叫萧眯眯,也是十大恶人中之一,你看来虽然又娇又嫩,其实最少也有四五十了,但你放心,我不会嫌你老的,姜是老的辣,越老我越欢喜。"他连珠炮似的说了一大篇,绿裙少妇已怔在那里。

小鱼儿道:``别站在那里呀,春宵一刻值千金,你该过来和我皇后亲热亲热才是。''绿裙少妇凝眸望着他,缓缓道:``你只说错了一件事。''小鱼儿道:``哦?''

绿裙少妇道:``我今年只有叁十七。''

小鱼儿嘻嘻笑道:``就算你十七也没关系,永远莫要和女人讨论她的年龄,这句话我很小的时候就懂了的。''绿裙少妇道:``别的事你说错都没关系,但你若说错女人的年纪,她可不饶你。''她的手,温柔而美丽,她的笑,也是温柔而美丽。

但这温柔的笑容中却隐含杀机,这双美丽的手顷刻间也能置人死命,这小鱼儿自然是知道的。

小鱼儿却偏偏装做不知道,嘻嘻笑道:``我已知道你是谁,你可知道我是谁么?''萧眯眯眼波流转,道:``你\ldots。.''

小鱼儿道:``十大恶人若也有一个朋友,那就是我,江鱼。''萧眯眯道:``你\ldots\ldots 你竟敢自称十大恶人的朋友?''小鱼儿笑道:你难道以为我是好人不成。"

萧隙眯嫣然道:``你自然不是好人、但你还太小,小得还不能做聪人,我瞧你''\ldots 你只怕是那老妖怪派来的,是么?否则你又怎么知道我。``小鱼儿道:''老妖怪我的确认得好几个。"

萧眯眯道:``好儿个?''

小鱼儿眨了眨眼睛,突然大笑道;``哈哈,小僧从来不近妖孽,阿弥陀佛\ldots\ldots 近妖者杀\ldots\ldots 你杀时小心些,若让血流得太多,肉就不鲜了\ldots\ldots 九幽门下,饿鬼日多,肉纵不鲜,也有鬼食\ldots。你呀,你就是个缺德鬼。''他说了五句话,正活脱脱是哈哈儿,``血手''杜杀,``不吃人头''李大嘴,``半人半鬼''阴九幽,``不男不女''屠娇娇这五人的口气,不但声音相同,语气也相同,正是惟妙惟肖,活灵活现。

萧眯眯眼睛已睁大了,娇笑道:``你这小鬼,你认得他们?''小鱼儿道:``我从小就是在恶人谷长大的。''菌眯眯的手,立刻放下了,拍手笑道:``这就难怪,难怪你是个小妖怪,原来你竟是跟着他们长大的。\ldots 他们常常提起我么?''小鱼儿笑道:``他们叫我遇见你时,要千万小心些,莫要被你迷死。他们说你是六亲不认,见人就要迷的。''萧眯昧咯咯笑道:``你相信他们的鬼话?''

小鱼儿眯着眼笑道:``能见着你这样的人,就算被你迷死,我也心甘情愿的。''萧眯昧娇笑道:``哎哟,小鬼,我没有迷死你,倒真的快要被你迷死了。''小鱼儿大笑道:``现在,你可以请我喝酒了么?''送酒上来的,竟是个孩子。

这孩子生得眉目清秀,但却面黄肌瘦,像是发育不全的模样,看神气像是比小鱼儿大,看身材又似比小鱼儿小。

他缩着脖子,驼着背,捧着盘的两只手,不停地发抖,但一双眼睛,却又不时偷偷在萧眯眯胸前瞟来瞟去。

萧咪咪笑道:``小色鬼,你瞧什么?''

那孩子红着脸,垂下了头,道;``没''。``没有。''萧咪咪媚笑道:``你想亲亲我是么?''

那孩子脸更红人萧咪咪道:``来,想亲就来亲呀,怕什么?''那孩子突然放下盘予,抱住了她。

萧咪咪突然反手一个巴掌,将他打得在地上直滚,小鱼儿抬起头,突然发现这孩予背着脸时,满脸都是杀机,竟令人觉得可怕。

他站起来时,他又变得一副可怜模样,红着脸,垂着头,一步一挨,慢吞吞走了出去,像是路都走不动。

小鱼儿道:``这小孩子也是你的妃子?''

萧咪咪笑道:``你吃醋?''

小鱼儿道:``唉,你简直是摧残幼苗。''

萧咪咪道:``我就是要折磨他,直到他死。,小鱼儿道:''为什么你恨他?他不过是个孩子呀!``萧咪咪道:''他虽是个孩子,但他的爹爹\ldots\ldots 嘿,,普天之下,再没有一个比他那爹爹更毒辣更阴险的人了。``小鱼儿笑道:''哦?他难道比阴九幽还阴险?难道比李大嘴还毒辣?``萧咪咪道:''阴九幽虽险,李大嘴虽狠,别人总还瞧得出,但他爹爹做尽了坏事后,别人还在称他为当世之大侠。``小鱼儿眼珠子一转,笑道:''连你都说这人坏,想来他必定真是个大坏蛋了。``其实他心里想的却是:''你说他是坏蛋,他想必是个好人\ldots"他故意不问这人的名字,萧咪咪居然也不说了,只见那孩于又抱了个盘子走进来。

小鱼儿突然道;``喝酒之前,我先得去清存货。''萧咪咪啐道:``没出息。''

小鱼儿笑道:``皇后方便时,总得有个把子在旁边伺候着他拉起那孩子的手,道:''来,你带我去。``萧咪咪娇笑道:''小心些,莫掉下去先就吃饱了,这里的酒莱还在等着你哩。"那孩子缩着脖子,垂着头在前面走。小鱼儿瞧着他的背影,似乎在想什么。

这地下的宫阙,显然是经过精心的设计,每一寸地方,都没有浪费,长道的弯曲处,就是方便之处。

小鱼儿突然问道:``嗯,你姓什么?''

那孩子道:``江。''

小鱼儿道:``你也姓江?真巧。''你叫什么名字``那孩子道:''玉郎。"

小鱼儿皱了皱眉,眼珠子四面一转,突又笑道:奇怪,这里已是地下,这许多人的大便小便,都流到哪里去了?这地下的地下难道还有通道?``江玉郎道:''下面没有通道,是坟墓。"

小鱼儿道:``坟墓?谁的坟墓?''

江玉郎道:``听说是建造此地工人的坟墓。''

小鱼儿又不禁皱了皱眉头,赶紧站起来,道:``你知道的倒不少,想必已来了许久。''江玉郎道:``─年。''

小鱼儿道:``一年\ldots\ldots 你怎会来的?''

江玉郎道:``阁下怎会来的?''

小鱼儿笑道:``嗯,不错,萧咪咪自然有法子把你弄来的''"。

看来这里必定还有条通向外面的道路,你\ldots\ldots 此知道么``江玉郎道:''不知道。"

小鱼儿道:``你没有查过?''

江玉郎道:``没有。''

小鱼儿道:``你难道不想出去?不想回家?''

江玉郎道:``这里很好,很舒服。''

小鱼儿突然一把抓着他肩头,沉声道:``你这小鬼,我知道你心里恨得要死,时时刻刻都在想法子出去,你瞒不过我的,你若肯与我合作,咱们就能想法子出去!''江玉郎面上毫无表情,淡淡道:``阁下若是方便完了,就请回去用酒。''小鱼儿眼睛盯着他,盯了许久,一宇字道:``我说的话,你记着,每个字都记着!''江玉郎仍然缩着脖子,垂着头,在前面走。小鱼儿瞧着他的背影,还似在想着什么。

两人终于走了回去,萧咪咪笑道:``看来,你存货倒不少,我只当你真的掉下去了。''小鱼几抚着肚子,嘻嘻一笑,道:``这肚子。\ldots.''江玉郎突然截口道:``他方便是假的,他只想要我陪着他捣鬼,只想从我嘴里探听出这里的出路,还叫我跟他一起逃出去。''萧咪咪眼睛一瞪,冷冷笑道:``江鱼你真的想出去?你何必问他,我告诉你好了。''小鱼儿神色不动,却大笑起来,笑道;``我在恶人谷都住了十来年,这地方难道比恶人谷还糟么我不过是试试这小鬼的,你难道信他的?''萧咪咪悠悠道:``其实,不管你是真是假,你问他都没有用的''\ldots\ldots 这地方的出路,除了我,谁也不知道。``她拍了拍江玉郎的头笑道;''想不到你倒很老实。``江玉郎脸又红了,垂头道:''只要能常常在娘娘的身边,我什么地方都不想去了。``萧咪咪笑道:''小色鬼,今天不准再胡思乱想了,乖乖去睡睡吧。``江玉郎瞧了瞧小鱼儿道:''但他\ldots─娘娘难道\ldots。.``萧咪咪道:你想我宰了他?''

江玉郎道:``他\ldots\ldots 他实在\ldots\ldots 萧咪咪轻轻给了他个耳括子,笑啐道:''要吃醋还轮不到你,滚吧。"江玉郎垂着头,转回身,乖乖地走了。萧眯眯根本再也未瞧他,这小鬼她是不放在心上的,无论他想玩什么花样,也玩不过她的手掌心。她只是瞧着另一个小鬼。

小鱼儿嘻嘻一笑,道:``这小子果然是个坏蛋。''萧咪咪道:``他是坏蛋,你也不是好东西。''

小鱼儿道,``我难道不比他好?''

萧咪咪眯着眼笑道:``你可知道我为什么不杀你?''小鱼儿道;``你舍不得杀我的。''

萧咪咪媚笑道:``对了,我真的舍不得杀你,我正要瞧瞧你究竟有多好\ldots\ldots 屠娇娇总教过你几手的,我\ldots。我想试试。''她斜斜地在张软榻上坐下去,春色已上眉梢,柔声道:``你还不过来?难道还要等我再教你?''小鱼儿眼珠子乱转,嘻瞎地笑。

萧咪咪道:``那么。''。你还等什么?"

小鱼儿道:``我只怕\ldots。.''

话还未说完,江玉郎突然又冲了进来,一张脸已变得没有─丝血色,颤声道:``不\ldots 不好,不好了!''萧咪咪怒道:``你想干什么?''

江玉郎道:``死了。\ldots 全都死了。''

萧咪咪变色道:什么人死了?江玉郎道:"你\ldots\ldots 你赶紧去瞧瞧\ldots。他们。\ldots.他们\ldots\ldots。

话未说完,突然晕了过去。

死人,到处都是死人!方才那些轻衣少年,此刻竟没有一人还是活的。

翻开他们的脸,有的七窍流血,有的血肉模糊,就连小鱼儿这么大的胆子,也不禁瞧得心里直冒寒气!

萧咪咪也有些慌了,跺脚道:``这。\ldots 这是怎么回事?''小鱼儿眼珠子一转,道:``莫不是那老妖怪已暗中潜来此地。''萧咪咪道:"不可能,绝不可能!此间入口,绝无人知道。

她嘴里说着``不可能'',人已往门外冲出去,突又回头.厉声道:``你若敢跟着来,我就真宰了你!''小鱼儿苦笑道:你放心,我难说不知道偷看了别人秘密的人,是万万活不长的\ldots\ldots 我还想多活两年哩。"等到萧咪咪从前面的门出去,他人已到了后面的门。他虽然明知萧咪咪必定要到那秘密的出口处查看,他也不想去偷瞧这秘密,只因他想瞧的是另一人的秘密!

他伏在地上,露出半只眼睛。只见那已晕在地上的江玉郎头突然动了,也用一只眼睛往四面瞧,他自然瞧不见门后面的小鱼儿。小鱼儿屏住了呼吸,动也不动。

江玉郎突然唤道:``江公子\ldots\ldots 江鱼,你出来吧。''小鱼儿的心一跳,但咬住牙,终于没有出声。江玉郎又等了等,突然跳起来。他身子突然变得比燕子还轻,比鱼还滑,比狐狸还灵,身子才一闪,已从旁门的一道小门滑出去。

那道小门,正是他方才带小鱼儿去方便时走的门。小鱼儿早已算好方向,他出了那间屋子的小门,小鱼儿也到了这间屋子的小门边,还是用半只眼睛偷偷的瞧。

只见江玉郎身子不停,一头钻进了那方便之处。小鱼儿的身子也像燕子一般掠过去,江玉郎竟掀起了那烘坑的盖子,往里面钻。

突然间,他腰上一麻,裤带已被人拉住。只听小鱼儿笑道:``你想一个人跑,那不成。''江玉朗的脸,这一次是真的吓白了,颤声道:``莫\ldots\ldots 莫要开玩笑。''小鱼儿冷笑道;``谁跟你开玩笑,老实说,你想干什么?''江玉郎道:小\ldots 小人只是想方便方便。"

小鱼儿道:``放屁,方便也不必钻进粪坑里去!''江玉郎道:``我\ldots\ldots、我想''\ldots."

小鱼儿道:``你难道想吃粪?''

江玉郎道:``听说粪是解毒的,我也中了毒,所以\ldots。我小鱼儿冷笑道:''你这小鬼,一张嘴果然厉害,但却休想骗得到我,你再不说老实话,我就拉你去见萧咪咪,而且还告诉她,那些人都是你杀的!``江玉郎身子已抖了起来,道:''我\ldots\ldots 我没有\ldots。.``小鱼儿道:''你杀了他们,将萧咪咪引开,然后再躲在一个秘密的地方,等萧咪咪找不着你时,再偷偷溜出去!``江玉郎道:''你。\ldots 你\ldots。."

小鱼儿道:``老实告诉你,你纵然奸似鬼,也得吃老子的洗脚水,我早就看透你了,你若想活命,就得乖乖跟我合作。''江玉郎终于叹了口气,道:``我服了你,好吧,你说的不错,我那藏身之处,就在这粪坑里,我费了一年的时间,才挖出来的。''小鱼儿道:``真有你的,居然将藏身之处弄在粪坑里,也不怕臭。''江玉郎道:``若要活命,就不觉得臭了。''

小鱼儿叹道:``我见过的坏人也不少,若论忍劲、狠劲,还得叫你这小鬼第一,就连我也不得不佩服你。''江玉郎道:``快,时候不多了,快放手,我带你进去!''小鱼儿放开手笑道:``你将路弄干净些,我\ldots\ldots{}''话犹未了,江玉郎两只脚突然连环踢出,这两脚踢得当真是又准又狠,他看来本不似有这么高的武功。

可惜小鱼儿早已算好他有这一着,他脚再踢出,腰上的穴道已全都被小鱼儿点住了,下半身再也不能动。

小鱼儿冷笑道:``我早就告诉过你,你弄不过我的,还不乖乖往里爬。''江玉郎颤声道:``我\ldots\ldots 我不能动了。''

小鱼儿道;``脚不能动,用手爬!''

江玉郎再不说话,果然乖乖的往里爬。

那粪坑本有一个洞通向地下,竟被他又从旁边挖了条小道,刚好可以容得下他的身子。他就像蛇一般往里爬。小鱼儿也只得捏着鼻子,跟着他爬,幸好爬了一段,就不臭了。小鱼儿摇着头苦笑道:``别人说我是个小妖怪,我看你才真是个小妖怪。真亏你想得出,竟在这种鬼地方下工夫。''这条小小的地道大约有七八尺,然后,里面就是个小小的洞,最多也不过只有七八尺见方。但这洞里,却早巳铺好了四五床棉被,还有两缸水,一坛酒,和一大堆咸肉、香肠、糯米糕,此刻居然还有十几本书。

小鱼儿瞧了瞧,也不禁叹道:"你倒真花了不少工夫,准备得倒真周到。江玉郎缩在角落里,瞧着他,那双眼睛就像蛇一样,闪着光,狡黠的光,狠毒的光,怨恨的光。小鱼儿也瞧着他,他是狐狸也好,是蛇也好,小鱼儿都不怕,小鱼儿并不怕坏人,越坏他越觉有趣。地下静得很幽寂,虽然难耐,但也正代表着安全,这里的确是个安全的地方,小鱼儿想不出有谁还能找得到他。他舒服地在棉被上躺下来,摘下条香肠,嗅了嗅,咬了一曰,香肠的滋味居然不错,很不错。

小鱼儿笑道:``粪坑里的避难所,粪坑里的香肠\ldots\ldots 江玉朗你的确是个天才。''江玉郎垂下眼皮,喃喃道:``天才!天才\ldots\ldots{}''小鱼儿笑道:``在粪坑挖洞,的确是只有天才才想得出的主意,萧咪咪就算查得再紧,但在你方便时可也不能跟着你。''江玉郎木然道;``不错,这的确是天才的主意,但这天才想出这主意后,花了多大的代价,吃了多大的苦,你可知道么?''小鱼儿道:``你说吧,我很喜欢听人诉苦。''

江玉朗道:``你只知道在大便时挖地道非常秘密,但你可知道要大便多少次才能挖出这样的地道!''小鱼儿道:``嗯,确实要不少次。''

江玉朗道:``你可想过一个人一天只能大便多少次?一年又只能大便多少次?大便的次数太多,岂不被人怀疑?''小鱼儿搔了搔头道:``嗯,这\ldots\ldots{}''

江玉朗道:``你可想过一个人在大便时若只是拼命地挖地道,那么他的大便哪里去了?他难道能永远不大便么?''小鱼儿又搔了搔头,苦笑道:``嗯,这的确是个问题,你在大便时若真的大便,就没有时间挖地道,你若挖地道,就没有时间大便了,这怎么办呢?''江玉郎辛涩的一笑,道:``怎么办你永远想不到的,像你这样的大少爷,永远想不到像我这样的小人物能吃怎样的苦。''他瞪着眼,咬着牙,一字字道:``我只有像狗一样,一面工作,一面大便,因为我不能浪费丝毫时间,我学会在最短时间脱光衣服,纵然冷得要死,我也得脱光衣服,因为我不能让大便和泥土弄脏衣服,但是我身上\ldots。.''他突然停住嘴,他似乎想吐。小鱼儿也突然觉得有些恶心,抛下了手里的半截香肠,想说什么,但说了半天,也没有说出话来,江玉朗盯着地上的半截香肠,缓缓道:``你可知道我为什么这样瘦?''小鱼儿道:``你\ldots 嗯\ldots 你\ldots{}''

江玉郎咬牙道:``我瘦,因为我一天到晚在挨饿,为了要尽量减少大便,我只有不吃东西,为了要储存食物,我也只有挨饿。''他露出白森森的牙齿,尖锐地一笑,道:``这就是天才一年来的生活,一年来狗─般的生活才换来这地洞,而你。''``你什么事都没有做,却在这里舒服的睡着。''小鱼儿还在挠头,突然笑道:``你可知道这是为了什么?''江玉郎道,``我但愿能知道。''

小鱼儿笑道:``告诉你,这就因为你虽是天才,我却是天才中的天才,一个人有我这样聪明就可以不必吃苦了。''江玉郎盯着他,良久良久,缓缓垂下头,道:``不钳,我的确不如你,我很佩服你!''这本是句称赞的话,但小鱼儿听了,不知怎地,心头竟突然生出股寒意,竟像是听了句最恶毒的诅咒。不错,这苍白而矮小的少年,也许的确不如他聪明,不如他机警,但若论狠毒,若论狡黠,小鱼儿却差多了。

尤其是那一份忍耐的功夫,小鱼儿更是一辈子也比不上──忍耐是种美德,但有时却又令人觉得可怕。小鱼几也不再说话。

他心里在想:这世上若还有我的对手,就是这小狐狸。但这念头还未转完,他已知道自己错了。这世上他还有个对手,一个更可怕的对手!

他眼前似已泛起了一条人影,那是个文质彬彬,温柔有礼的,又风流体贴,永远不会动怒的人影。

花无缺,无缺公子,他既不狠毒,也不好诈,似乎完全没有什么心机,除了武功外,似乎全无任何可怕之处。但这种"全无可怕之处,正是最最可怕之处一一他整个人就像是大海浩浩瀚瀚、深不可测。

小鱼儿暗中叹了口气,喃喃道,``这小子我的确看不透,能让我看不透的人,大概是不错的了''。

江玉郎瞧着他,想说话,但是忍住了。

小鱼儿笑道:``我不是说你,我是说另一个人。''江玉郎道:``哦。''

小鱼儿道;"这个人看起来并不像是个十分聪明的人,但你无论多聪明,无论玩什么花样,到他面前就没用了,因为你无论对他用什么手段,玩什么花样,他都不会吃亏的,算来算去,吃亏的是你自己。

江玉郎淡淡一笑,道:"这种人我还末见过\ldots\ldots{}

小鱼儿道:``只要你不死,你总会见着的。''

江玉郎木然自语道:``只要我不死\ldots\ldots 只要我不死。''突然面色大变,失声道:``糟糕。''

小鱼儿知道能让他失色的事,必定是件很糟糕的事,脸色不由自主也有些变了,脱口道:``什么事?''江玉郎道:``你\ldots\ldots 你进来时,可反手盖上那粪坑的盖子?''小鱼儿张大眼睛,道:``呀,没有,我忘了。''江玉郎变色道:``萧咪咪瞧不见我们,必定四下搜索,她若瞧见\ldots\ldots\ldots\ldots\ldots{}''小鱼儿展颜笑道:``你也未免太小心,她难道会想到咱们在粪坑里?''江玉郎道:``我自然要小心,只要稍为大意,只要一时大意,就可能招来杀身之祸,你可知道萧咪咪的武功?''小鱼儿苦笑道``我就因为摸不透她的武功,所以不敢和她翻脸\ldots\ldots 假如是笨人,武功高些我也不怕,但她,她简直也是个妖怪。''江玉郎叹道:``她武功之高,只怕远出你想象之外,据说,她一生中有七百多个情郎,其中还包括了七大剑派中的子弟,每人只教她一手武功,就够人受的了。''小鱼儿眼珠子一转,道:``如此说来,倒是真该小心些才好,我还是再偷偷溜出去一趟,把那见鬼的盖子盖上吧。''江玉郎道:``你等一等。''他口中说话,耳朵已贴在土壁上,听了半晌,失色道:``不好,她已经回来了。''突然间,一阵香气从里面飘了出来。

那香气竟像是一只鸡加上酱油五香在锅里烧的味道。

小鱼儿鼻子已耸起来,这味道在他嗅来,当真是世止最可爱的味道了,他咽下几口口水,大声道:``这里面必定是人,鬼是不会吃鸡的,妖怪纵吃鸡,也不会红烧\ldots\ldots 既然是人,就没什么可怕的。''他这话像是说给那绿裙少妇来听,又像是自言自语,壮自己的胆子,绿裙少妇颤声道:``你若真的要进去,就要小心些。''小鱼儿大声道:``我自然会小心的,无论做什么事,我都小心得很,否则只怕已活不到现在了。''嘴里说话,自树下捡了块石予,往洞中抛进去。

只听``笃''的一响,小鱼儿道:``这洞并不深。''绿裙少妇柔声道:``你果然是个很小心仔细的人。''小鱼儿不觉又挺了挺胸,道:``你在这里等着,我进去瞧瞧。''绿裙少妇颤声道:``不\ldots\ldots 不行,叫我一个人留在外面,我怕都怕死了,我要跟着你一齐进去,有你在我身旁,我才放心。''小鱼儿瞧了她两眼,道;``唉,女人,究竟是女人''\ldots\ldots 好,你跟着我来吧,紧紧跟着我,莫要走开。``绿裙少妇道:''你用鞭子都赶不走我的。"

小鱼儿已一脚跨了进去,脚下不觉有些飘飘然。

这株树,里面果然是空的,虽不深,但却十分黑暗。

缘裙少妇紧紧依偎着小鱼儿,颤声道;``奇怪,这里还是没有人。''小鱼儿道:``有人的,一定有人的。''

绿裙少妇道:``这里总共只有这么大地方,人在哪里?''树穴周围不过五尺,果然没有可以藏下一个人的地方。

小鱼儿皱眉道:``奇怪,红烧鸡的香气是从哪里来的?''绿裙少妇道:``这香气像是从下面\ldots\ldots{}''

话末说完.他们站的地方竟突然往下面沉了下去。绿裙少妇整个人都缩进小鱼儿怀里,颤声道:``这是怎么回事?咱们怎么办?''小鱼儿圆瞪着眼睛,大声道:"莫要怕,怕什么,咱们索性就下去瞧个究竟\ldots\ldots{}

两个人的身子不断往下沉,四下仍是一片黑暗,他们就像是站在一个筒子里,一个可以上下活动的筒子。绿裙少妇紧紧抓着小鱼儿的手,她的手又湿又冷,这方才还杀人不眨眼的女子,此刻胆子竟会变得这么小,倒是令人想不通的事。

那``筒子''终于停了,小鱼儿眼前一亮,又出现一道门,一片青蒙蒙的光线,自门外洒了进来。

小鱼儿一伏身,嗖"的窜了出去,外面竟是条地道,两旁是雕刻精致的石壁,壁上嵌着发亮的铜灯。

小鱼儿喃喃道:``好家伙,这地方居然还收拾得华丽得很,看来,此间的主人纵不是妖怪,也和妖怪差不多了。''他刚想回头叫那绿裙少妇出来。突听一声惨呼,原来那铁筒的门突又关了,铁筒竟又往下沉,绿裙少妇的惨呼声不断自筒里传出来。

只听她凄声呼道:``火''\ldots\ldots 救命,救命,火\ldots。.``小鱼儿大擦之下,要伸手去拉,但那就像是间小屋子般大小的铁筒,他又怎么能拉得住。他想随着铁筒往下跳,但那铁筒恰巧嵌在地里,就不动了,只有那绿裙少妇的掺呼声仍不断传上来''火,\ldots\ldots 烧死我了,求求你\ldots。救命呀,火\ldots\ldots"凄厉的呼声,听得小鱼儿全身冷汗直冒。他拳打脚踢,想弄开那铁筒的顶,怎奈那铁筒的顶也是精钢所铸,他用尽气力,也是没有用的。

绿裙少妇的惨呼声已越来越衰弱。``我受不住了''\ldots 求求你,让我快些死吧!..\ldots 求求\ldots。"呼声突然断绝,然后便是死一般的静寂。

小鱼儿也停下了手,痴痴的站在那里。绿裙少妇竟被活活烧死在铁筒里!

这女子虽然狠心,虽然和他没有关系,但却曾全心全意地依靠着他,而结果却落到这种下场。她选错了人,选错人了\ldots\ldots{}

小鱼儿的眼眶已变得湿湿的,突然嘶声大呼道:``你听着,无论你是谁,都仔细的听着,你吓不倒我,也杀不死我的,我却一定要杀死你!''地道里没有回应,根本没有人理他。小鱼儿咬了咬牙,大步向前走去。

地道并不长,尽头处有一扇门,门上面也雕刻着一些人物花草,看来,单只建这条地道,就不知花了多少人力物力,这里的主人肯花这么大的人力物力在地下建造条走道,当真不知是个什么样的怪物。

门,并没有上锁。小鱼儿伸手一推就推开了!

他自己也不知自己怎么会这么大的胆子,竟笔直走了进去,他好像觉得自己绝不会死。只因他若要死,方才就该被火烧死──他只觉得这地道的主人似乎不想杀他,为什么,他却弄不清楚。

他想得并不太多,这就是他思想的秘诀,只要能捕捉着一点主题,其余的就不必想了,想多了反而困扰。

门后面,是一间厅。地道已是如此华丽,厅堂自然更堂皇;在地下竟会有如此堂皇的厅堂,更是件令人想不到的事。除了没有窗子,这里简直和地上富户的花厅没什么两样,陈设得雅致大方,还尤有过之。但厅堂中仍没有人。

小鱼儿喃喃道:``这里的主人虽是个怪物,倒也懂得享受,他若将这里弄得鬼气森获,虽能吓得倒别人.却也苦了自己。''突听一人笑道:``不想阁下倒是此间主人的知己。''这语声虽是男子的口音,但缓慢而温柔,却又有些和女子相似,小鱼儿的溜溜一转身,却瞧不见人,不由大喝道:``什么人?你在哪里?''那语声笑道:``你瞧不见我的,我却瞧得见你。''小鱼儿虽没有瞧见人,却又瞧见一扇门.他一步掠了过去,推开门,又是间花厅。

厅堂的中央,有张桌子,桌子上有只天青色的大碗,那始终引诱着小鱼儿的香气,便是自碗里发出来的。碗里,果然是只烧得红红的鸡。

小鱼儿眼睛又圆了,只听方才那语声又在另一处响起,缓缓道:``江鱼,这只鸡烧得很嫩,是特地为你准备的。''小鱼儿身子一震,大声道:``你\ldots\ldots{}''你怎会知道我的名字?``那语声笑道:''此间的主人,没有不知道的事。``小鱼儿吼道:''你们到底是些什么人?"

那语声道:``你怎知道我们一定是人。''

小鱼儿怔了怔,后退两步,道,``你们究竟想要我怎样?''望补上小鱼儿道;:``嗯''

那语声道:``你可知道你现在是死是活?是人是鬼?现在,你睁大了眼赌,等着瞧吧。''这句话刚说完,四面灯光已亮了起来。小鱼儿发觉自己还是躺在方才倒下去购地方,但四面的椅子上,不知何时,已坐着七八个人。

这七八个人都穿着宽大而柔较的长袍,年纪最多也不过只有二十多岁,每个人都长得清清秀秀、臼白净净。

这七八人虽然都是男人,但看来却又和女子相似,每个人都懒洋洋地坐在那里,瞧着小鱼儿懒洋洋的笑。

小鱼儿道:``你们就是这里的主人?''

七八人一齐摇了摇头。这七八人一个个竟都是有气无力,像是全身没一根骨头,人虽然都是活的,但却和死人差不多。

小鱼儿忍不住大声道:``你们的主人究竟是谁?为什么不出来见我?他若也像你们这种不男不女,要死不活的模样,我还懒得见他哩。''其中一人笑道:``你莫要笑咱们,叁个月后,你也会和咱们一样。''小鱼儿冷笑道:``你活见大头鬼了。''

那人笑道:``你不信?你虽有铁打的身子,也吃不消她。''小鱼儿道:``她,她是谁?''那人道:``她就是我们的女王。''只听一人银铃般娇笑道:``我就是这里的女王!''这笑声听来熟得很,小鱼儿转过头,便瞧见她。她竟是那方才被活活烧死的绿裙少妇。

小鱼儿整个人都呆住了,眼睛瞪得简直比鸡蛋还大。

\hypertarget{ux7b2cux4e8cux5341ux4e5dux7ae0-ux98a0ux5012ux4e7eux5764}{%
\chapter{第二十九章
颠倒乾坤}\label{ux7b2cux4e8cux5341ux4e5dux7ae0-ux98a0ux5012ux4e7eux5764}}

绿裙少妇瞧着小鱼儿咯咯笑道:``天下第一个聪明人,世上真的没有一个人能骗得倒你么?''小鱼儿痴痴地瞧着她,道:``难怪那两人尸身瞧不见了,难怪你能找得到那地道的入口,原来你就是这里的主人,你\ldots\ldots\ldots\ldots 你的确骗倒我了。''绿裙少妇道:``你服了么?''

小鱼儿叹道:``我服了\ldots\ldots 我早就说过,你是个骗死人不赔命的女妖怪,但我却再也想不到你这妖怪竟是从地下钻出来的。''绿裙少妇身子轻盈地一转,笑道,``你瞧我的宫殿如何?''小鱼儿道:``不错,的确不错。''

绿裙少妇眼皮一转,道:``你瞧我这些妃子如何?''小鱼儿瞪大了眼睛。

缘裙少妇咯咯笑道,``男人可以有叁妻四妾,女人为什么不可以?''小鱼儿苦笑了一下突又瞪大眼睛,失声道:``你难道\ldots\ldots 难道要我也做\ldots\ldots 做你的妃\ldots\ldots 妃子?''绿裙少妇瞧着他,嫣然笑道:``不对。''

小鱼儿刚松了口气,绿裙少妇已柔声接道,``我要你做我的皇后。''小鱼儿呆了半晌,突然大笑起来,笑得几乎喘不过气,他一生中简直从来没有像这样大笑过。

绦裙少妇道:``你开心么?''

小鱼儿大笑道:``我开心,开心极了,我什么疯狂的事都想到过,但却做梦也没有想到我有朝一日竟会做皇后。''缘裙少妇道:``你不愿意?''

小鱼儿瞪大眼睛,道:``我为什么不愿意?世上又有几个男人能当皇后?''他突然跳起来往桌子上一坐,大声道:``喂,你们还不过来拜见你们的新皇后么?''那些轻衫少年你瞧着我,我瞧着你,终于一齐走过来。

小鱼儿道:``只要磕叁个头就够了,不必太多。''少年们一齐去望那绿裙少妇,绿裙少妇不停的娇笑,不停的点头,少年们想不磕头也不行了。

小鱼儿道:``磕完头就出去吧,我要和皇上喝酒了,快出去\ldots。妃子若想和皇后争宠,皇后吃起醋来,是要砍你们脑袋的。''少年瞧着他,那模样倒当真像是瞧见了个妖怪似的,突然一齐转过头,走了干净。

小鱼儿拍手大笑道:``妙极妙极,做皇后的滋味可真不错。''绿裙少妇笑得已直不起腰,咯咯笑道:``你这小鬼真有意思,我在这里十多年,从来也没有这样开心过。''小鱼儿笑道:``从今以后,我天天都要让你开心,开心得要死,你虽然叫迷死人不赔命,我却要迷死你。''绿祖少妇突然不笑了,瞪大眼睛,道:``你''\ldots\ldots 你怎会知道我的名字?``小鱼儿笑嘻嘻道:''我非但知道你这名字,还知道你叫萧眯眯,也是十大恶人中之一,你看来虽然又娇又嫩,其实最少也有四五十了,但你放心,我不会嫌你老的,姜是老的辣,越老我越欢喜。"他连珠炮似的说了一大篇,绿裙少妇已怔在那里。

小鱼儿道:``别站在那里呀,春宵一刻值千金,你该过来和我皇后亲热亲热才是。''绿裙少妇凝眸望着他,缓缓道:``你只说错了一件事。''小鱼儿道:``哦?''

绿裙少妇道:``我今年只有叁十七。''

小鱼儿嘻嘻笑道:``就算你十七也没关系,永远莫要和女人讨论她的年龄,这句话我很小的时候就懂了的。''绿裙少妇道:``别的事你说错都没关系,但你若说错女人的年纪,她可不饶你。''她的手,温柔而美丽,她的笑,也是温柔而美丽。

但这温柔的笑容中却隐含杀机,这双美丽的手顷刻间也能置人死命,这小鱼儿自然是知道的。

小鱼儿却偏偏装做不知道,嘻嘻笑道:``我已知道你是谁,你可知道我是谁么?''萧眯眯眼波流转,道:``你\ldots。.''

小鱼儿道:``十大恶人若也有一个朋友,那就是我,江鱼。''萧眯眯道:``你\ldots\ldots 你竟敢自称十大恶人的朋友?''小鱼儿笑道:你难道以为我是好人不成。"

萧隙眯嫣然道:``你自然不是好人、但你还太小,小得还不能做聪明人,我瞧你''\ldots 你只怕是那老妖怪派来的,是么?否则你又怎么知道我。``小鱼儿道:''老妖怪我的确认得好几个。"

萧眯眯道:``好几个?''

小鱼儿眨了眨眼睛,突然大笑道;``哈哈,小僧从来不近妖孽,阿弥陀佛\ldots\ldots 近妖者杀\ldots\ldots 你杀时小心些,若让血流得太多,肉就不鲜了\ldots\ldots 九幽门下,饿鬼日多,肉纵不鲜,也有鬼食\ldots。你呀,你就是个缺德鬼。''他说了五句话,正活脱脱是哈哈儿,``血手''杜杀,``不吃人头''李大嘴,``半人半鬼''阴九幽,``不男不女''屠娇娇这五人的口气,不但声音相同,语气也相同,正是惟妙惟肖,活灵活现。

萧眯眯眼睛已睁大了,娇笑道:``你这小鬼,你认得他们?''小鱼儿道:``我从小就是在恶人谷长大的。''菌眯眯的手,立刻放下了,拍手笑道:``这就难怪,难怪你是个小妖怪,原来你竟是跟着他们长大的。\ldots 他们常常提起我么?''小鱼儿笑道:``他们叫我遇见你时,要千万小心些,莫要被你迷死。他们说你是六亲不认,见人就要迷的。''萧眯昧咯咯笑道:``你相信他们的鬼话?''

小鱼儿眯着眼笑道:``能见着你这样的人,就算被你迷死,我也心甘情愿的。''萧眯昧娇笑道:``哎哟,小鬼,我没有迷死你,倒真的快要被你迷死了。''小鱼儿大笑道:``现在,你可以请我喝酒了么?''送酒上来的,竟是个孩子。

这孩子生得眉目清秀,但却面黄肌瘦,像是发育不全的模样,看神气像是比小鱼儿大,看身材又似比小鱼儿小。

他缩着脖子,驼着背,捧着盘的两只手,不停地发抖,但一双眼睛,却又不时偷偷在萧眯眯胸前瞟来瞟去。

萧咪咪笑道:``小色鬼,你瞧什么?''

那孩子红着脸,垂下了头,道;``没''。``没有。''萧咪咪媚笑道:``你想亲亲我是么?''

那孩子脸更红人萧咪咪道:``来,想亲就来亲呀,怕什么?''那孩子突然放下盘子,抱住了她。

萧咪咪突然反手一个巴掌,将他打得在地上直滚,小鱼儿抬起头,突然发现这孩予背着脸时,满脸都是杀机,竟令人觉得可怕。

他站起来时,他又变得一副可怜模样,红着脸,垂着头,一步一挨,慢吞吞走了出去,像是路都走不动。

小鱼儿道:``这小孩子也是你的妃子?''

萧咪咪笑道:``你吃醋?''

小鱼儿道:``唉,你简直是摧残幼苗。''

萧咪咪道:``我就是要折磨他,直到他死。''小鱼儿道:``为什么你恨他?他不过是个孩子呀!''萧咪咪道:``他虽是个孩子,但他的爹爹\ldots\ldots 嘿,普天之下,再没有一个比他那爹爹更毒辣更阴险的人了。''小鱼儿笑道:``哦?他难道比阴九幽还阴险?难道比李大嘴还毒辣?''萧咪咪道:``阴九幽虽险,李大嘴虽狠,别人总还瞧得出,但他爹爹做尽了坏事后,别人还在称他为当世之大侠。''小鱼儿眼珠子一转,笑道:``连你都说这人坏,想来他必定真是个大坏蛋了。''其实他心里想的却是:``你说他是坏蛋,他想必是个好人\ldots{}''他故意不问这人的名字,萧咪咪居然也不说了,只见那孩于又抱了个盘子走进来。

小鱼儿突然道;``喝酒之前,我先得去清存货。''萧咪咪啐道:``没出息。''

小鱼儿笑道:``皇后方便时,总得有个把子在旁边伺候着他拉起那孩子的手,道:''来,你带我去。``萧咪咪娇笑道:''小心些,莫掉下去先就吃饱了,这里的酒莱还在等着你哩。"那孩子缩着脖子,垂着头在前面走。小鱼儿瞧着他的背影,似乎在想什么。

这地下的宫阙,显然是经过精心的设计,每一寸地方,都没有浪费,长道的弯曲处,就是方便之处。

小鱼儿突然问道:``嗯,你姓什么?''

那孩子道:``江。''

小鱼儿道:``你也姓江?真巧。''你叫什么名字``那孩子道:''玉郎。"

小鱼儿皱了皱眉,眼珠子四面一转,突又笑道:奇怪,这里已是地下,这许多人的大便小便,都流到哪里去了?这地下的地下难道还有通道?``江玉郎道:''下面没有通道,是坟墓。"

小鱼儿道:``坟墓?谁的坟墓?''

江玉郎道:``听说是建造此地工人的坟墓。''

小鱼儿又不禁皱了皱眉头,赶紧站起来,道:``你知道的倒不少,想必已来了许久。''江玉郎道:``─年。''

小鱼儿道:``一年\ldots\ldots 你怎会来的?''

江玉郎道:``阁下怎会来的?''

小鱼儿笑道:``嗯,不错,萧咪咪自然有法子把你弄来的''看来这里必定还有条通向外面的道路,你\ldots\ldots 此知道么``江玉郎道:''不知道。"

小鱼儿道:``你没有查过?''

江玉郎道:``没有。''

小鱼儿道:``你难道不想出去?不想回家?''

江玉郎道:``这里很好,很舒服。''

小鱼儿突然一把抓着他肩头,沉声道:``你这小鬼,我知道你心里恨得要死,时时刻刻都在想法子出去,你瞒不过我的,你若肯与我合作,咱们就能想法子出去!''江玉郎面上毫无表情,淡淡道:``阁下若是方便完了,就请回去用酒。''小鱼儿眼睛盯着他,盯了许久,一宇字道:``我说的话,你记着,每个字都记着!''江玉郎仍然缩着脖子,垂着头,在前面走。小鱼儿瞧着他的背影,还似在想着什么。

两人终于走了回去,萧咪咪笑道:``看来,你存货倒不少,我只当你真的掉下去了。''小鱼几抚着肚子,嘻嘻一笑,道:``这肚子\ldots{}''江玉郎突然截口道:``他方便是假的,他只想要我陪着他捣鬼,只想从我嘴里探听出这里的出路,还叫我跟他一起逃出去。''萧咪咪眼睛一瞪,冷冷笑道:``江鱼你真的想出去?你何必问他,我告诉你好了。''小鱼儿神色不动,却大笑起来,笑道;``我在恶人谷都住了十来年,这地方难道比恶人谷还糟么我不过是试试这小鬼的,你难道信他的?''萧咪咪悠悠道:``其实,不管你是真是假,你问他都没有用的这地方的出路,除了我,谁也不知道。''她拍了拍江玉郎的头笑道;``想不到你倒很老实。''江玉郎脸又红了,垂头道:``只要能常常在娘娘的身边,我什么地方都不想去了。''萧咪咪笑道:``小色鬼,今天不准再胡思乱想了,乖乖去睡睡吧。''江玉郎瞧了瞧小鱼儿道:``但他\ldots 娘娘难道\ldots{}''萧咪咪道:``你想我宰了他?''

江玉郎道:``他\ldots\ldots 他实在\ldots\ldots{}''萧咪咪轻轻给了他个耳括子,笑啐道:``要吃醋还轮不到你,滚吧。''江玉郎垂着头,转回身,乖乖地走了。萧眯眯根本再也未瞧他,这小鬼她是不放在心上的,无论他想玩什么花样,也玩不过她的手掌心。她只是瞧着另一个小鬼。

小鱼儿嘻嘻一笑,道:``这小子果然是个坏蛋。''萧咪咪道:``他是坏蛋,你也不是好东西。''

小鱼儿道,``我难道不比他好?''

萧咪咪眯着眼笑道:``你可知道我为什么不杀你?''小鱼儿道;``你舍不得杀我的。''

萧咪咪媚笑道:``对了,我真的舍不得杀你,我正要瞧瞧你究竟有多好\ldots\ldots 屠娇娇总教过你几手的,我\ldots。我想试试。''她斜斜地在张软榻上坐下去,春色已上眉梢,柔声道:``你还不过来?难道还要等我再教你?''小鱼儿眼珠子乱转,嘻嘻地笑。

萧咪咪道:``那么。''你还等什么?"

小鱼儿道:``我只怕\ldots。.''

话还未说完,江玉郎突然又冲了进来,一张脸已变得没有─丝血色,颤声道:``不\ldots 不好,不好了!''萧咪咪怒道:``你想干什么?''

江玉郎道:``死了\ldots 全都死了。''

萧咪咪变色道:什么人死了?江玉郎道:"你\ldots\ldots 你赶紧去瞧瞧.他们\ldots.他们\ldots\ldots。

话未说完,突然晕了过去。

死人,到处都是死人!方才那些轻衣少年,此刻竟没有一人还是活的。

翻开他们的脸,有的七窍流血,有的血肉模糊,就连小鱼儿这么大的胆子,也不禁瞧得心里直冒寒气!

萧咪咪也有些慌了,跺脚道:``这\ldots 这是怎么回事?''小鱼儿眼珠子一转,道:``莫不是那老妖怪已暗中潜来此地。''萧咪咪道:"不可能,绝不可能!此间入口,绝无人知道。

她嘴里说着``不可能'',人已往门外冲出去,突又回头.厉声道:``你若敢跟着来,我就真宰了你!''小鱼儿苦笑道:``你放心,我难说不知道偷看了别人秘密的人,是万万活不长的\ldots\ldots 我还想多活两年哩。''等到萧咪咪从前面的门出去,他人已到了后面的门。他虽然明知萧咪咪必定要到那秘密的出口处查看,他也不想去偷瞧这秘密,只因他想瞧的是另一人的秘密!

他伏在地上,露出半只眼睛。只见那已晕在地上的江玉郎头突然动了,也用一只眼睛往四面瞧,他自然瞧不见门后面的小鱼儿。小鱼儿屏住了呼吸,动也不动。

江玉郎突然唤道:``江公子\ldots\ldots 江鱼,你出来吧。''小鱼儿的心一跳,但咬住牙,终于没有出声。江玉郎又等了等,突然跳起来。他身子突然变得比燕子还轻,比鱼还滑,比狐狸还灵,身子才一闪,已从旁门的一道小门滑出去。

那道小门,正是他方才带小鱼儿去方便时走的门。小鱼儿早已算好方向,他出了那间屋子的小门,小鱼儿也到了这间屋子的小门边,还是用半只眼睛偷偷的瞧。

只见江玉郎身子不停,一头钻进了那方便之处。小鱼儿的身子也像燕子一般掠过去,江玉郎竟掀起了那烘坑的盖子,往里面钻。

突然间,他腰上一麻,裤带已被人拉住。只听小鱼儿笑道:``你想一个人跑,那不成。''江玉朗的脸,这一次是真的吓白了,颤声道:``莫\ldots\ldots 莫要开玩笑。''小鱼儿冷笑道;``谁跟你开玩笑,老实说,你想干什么?''江玉郎道:``小\ldots 小人只是想方便方便。''

小鱼儿道:``放屁,方便也不必钻进粪坑里去!''江玉郎道:``我\ldots\ldots、我想''\ldots."

小鱼儿道:``你难道想吃粪?''

江玉郎道:``听说粪是解毒的,我也中了毒,所以\ldots。我小鱼儿冷笑道:''你这小鬼,一张嘴果然厉害,但却休想骗得到我,你再不说老实话,我就拉你去见萧咪咪,而且还告诉她,那些人都是你杀的!``江玉郎身子已抖了起来,道:''我\ldots\ldots 我没有\ldots。.``小鱼儿道:''你杀了他们,将萧咪咪引开,然后再躲在一个秘密的地方,等萧咪咪找不着你时,再偷偷溜出去!``江玉郎道:''你\ldots 你\ldots"

小鱼儿道:``老实告诉你,你纵然奸似鬼,也得吃老子的洗脚水,我早就看透你了,你若想活命,就得乖乖跟我合作。''江玉郎终于叹了口气,道:``我服了你,好吧,你说的不错,我那藏身之处,就在这粪坑里,我费了一年的时间,才挖出来的。''小鱼儿道:``真有你的,居然将藏身之处弄在粪坑里,也不怕臭。''江玉郎道:``若要活命,就不觉得臭了。''

小鱼儿叹道:``我见过的坏人也不少,若论忍劲、狠劲,还得叫你这小鬼第一,就连我也不得不佩服你。''江玉郎道:``快,时候不多了,快放手,我带你进去!''小鱼儿放开手笑道:``你将路弄干净些,我\ldots\ldots{}''话犹未了,江玉郎两只脚突然连环踢出,这两脚踢得当真是又准又狠,他看来本不似有这么高的武功。

可惜小鱼儿早已算好他有这一着,他脚再踢出,腰上的穴道已全都被小鱼儿点住了,下半身再也不能动。

小鱼儿冷笑道:``我早就告诉过你,你弄不过我的,还不乖乖往里爬。''江玉郎颤声道:``我\ldots\ldots 我不能动了。''

小鱼儿道;``脚不能动,用手爬!''

江玉郎再不说话,果然乖乖的往里爬。

那粪坑本有一个洞通向地下,竟被他又从旁边挖了条小道,刚好可以容得下他的身子。他就像蛇一般往里爬。小鱼儿也只得捏着鼻子,跟着他爬,幸好爬了一段,就不臭了。小鱼儿摇着头苦笑道:``别人说我是个小妖怪,我看你才真是个小妖怪。真亏你想得出,竟在这种鬼地方下工夫。''这条小小的地道大约有七八尺,然后,里面就是个小小的洞,最多也不过只有七八尺见方。但这洞里,却早巳铺好了四五床棉被,还有两缸水,一坛酒,和一大堆咸肉、香肠、糯米糕,此刻居然还有十几本书。

小鱼儿瞧了瞧,也不禁叹道:"你倒真花了不少工夫,准备得倒真周到。江玉郎缩在角落里,瞧着他,那双眼睛就像蛇一样,闪着光,狡黠的光,狠毒的光,怨恨的光。小鱼儿也瞧着他,他是狐狸也好,是蛇也好,小鱼儿都不怕,小鱼儿并不怕坏人,越坏他越觉有趣。地下静得很幽寂,虽然难耐,但也正代表着安全,这里的确是个安全的地方,小鱼儿想不出有谁还能找得到他。他舒服地在棉被上躺下来,摘下条香肠,嗅了嗅,咬了一曰,香肠的滋味居然不错,很不错。

小鱼儿笑道:``粪坑里的避难所,粪坑里的香肠\ldots\ldots 江玉朗你的确是个天才。''江玉郎垂下眼皮,喃喃道:``天才!天才\ldots\ldots{}''小鱼儿笑道:``在粪坑挖洞,的确是只有天才才想得出的主意,萧咪咪就算查得再紧,但在你方便时可也不能跟着你。''江玉郎木然道;``不错,这的确是天才的主意,但这天才想出这主意后,花了多大的代价,吃了多大的苦,你可知道么?''小鱼儿道:``你说吧,我很喜欢听人诉苦。''

江玉朗道:``你只知道在大便时挖地道非常秘密,但你可知道要大便多少次才能挖出这样的地道!''小鱼儿道:``嗯,确实要不少次。''

江玉朗道:``你可想过一个人一天只能大便多少次?一年又只能大便多少次?大便的次数太多,岂不被人怀疑?''小鱼儿搔了搔头道:``嗯,这\ldots\ldots{}''

江玉朗道:``你可想过一个人在大便时若只是拼命地挖地道,那么他的大便哪里去了?他难道能永远不大便么?''小鱼儿又搔了搔头,苦笑道:``嗯,这的确是个问题,你在大便时若真的大便,就没有时间挖地道,你若挖地道,就没有时间大便了,这怎么办呢?''江玉郎辛涩的一笑,道:``怎么办你永远想不到的,像你这样的大少爷,永远想不到像我这样的小人物能吃怎样的苦。''他瞪着眼,咬着牙,一字字道:``我只有像狗一样,一面工作,一面大便,因为我不能浪费丝毫时间,我学会在最短时间脱光衣服,纵然冷得要死,我也得脱光衣服,因为我不能让大便和泥土弄脏衣服,但是我身上\ldots。.''他突然停住嘴,他似乎想吐。小鱼儿也突然觉得有些恶心,抛下了手里的半截香肠,想说什么,但说了半天,也没有说出话来,江玉朗盯着地上的半截香肠,缓缓道:``你可知道我为什么这样瘦?''小鱼儿道:``你\ldots 嗯\ldots 你\ldots{}''

江玉郎咬牙道:``我瘦,因为我一天到晚在挨饿,为了要尽量减少大便,我只有不吃东西,为了要储存食物,我也只有挨饿。''他露出白森森的牙齿,尖锐地一笑,道:``这就是天才一年来的生活,一年来狗─般的生活才换来这地洞,而你。''``你什么事都没有做,却在这里舒服的睡着。''小鱼儿还在挠头,突然笑道:``你可知道这是为了什么?''江玉郎道,``我但愿能知道。''

小鱼儿笑道:``告诉你,这就因为你虽是天才,我却是天才中的天才,一个人有我这样聪明就可以不必吃苦了。''江玉郎盯着他,良久良久,缓缓垂下头,道:``不错,我的确不如你,我很佩服你!''这本是句称赞的话,但小鱼儿听了,不知怎地,心头竟突然生出股寒意,竟像是听了句最恶毒的诅咒。不错,这苍白而矮小的少年,也许的确不如他聪明,不如他机警,但若论狠毒,若论狡黠,小鱼儿却差多了。

尤其是那一份忍耐的功夫,小鱼儿更是一辈子也比不上──忍耐是种美德,但有时却又令人觉得可怕。小鱼几也不再说话。

他心里在想:这世上若还有我的对手,就是这小狐狸。但这念头还未转完,他已知道自己错了。这世上他还有个对手,一个更可怕的对手!

他眼前似已泛起了一条人影,那是个文质彬彬,温柔有礼的,又风流体贴,永远不会动怒的人影。

花无缺,无缺公子,他既不狠毒,也不好诈,似乎完全没有什么心机,除了武功外,似乎全无任何可怕之处。但这种"全无可怕之处,正是最最可怕之处一一他整个人就像是大海浩浩瀚瀚、深不可测。

小鱼儿暗中叹了口气,喃喃道,``这小子我的确看不透,能让我看不透的人,大概是不错的了''。

江玉郎瞧着他,想说话,但是忍住了。

小鱼儿笑道:``我不是说你,我是说另一个人。''江玉郎道:``哦。''

小鱼儿道;"这个人看起来并不像是个十分聪明的人,但你无论多聪明,无论玩什么花样,到他面前就没用了,因为你无论对他用什么手段,玩什么花样,他都不会吃亏的,算来算去,吃亏的是你自己。

江玉郎淡淡一笑,道:"这种人我还末见过\ldots\ldots{}

小鱼儿道:``只要你不死,你总会见着的。''

江玉郎木然自语道:``只要我不死\ldots\ldots 只要我不死。''突然面色大变,失声道:``糟糕。''

小鱼儿知道能让他失色的事,必定是件很糟糕的事,脸色不由自主也有些变了,脱口道:``什么事?''江玉郎道:``你\ldots\ldots 你进来时,可反手盖上那粪坑的盖子?''小鱼儿张大眼睛,道:``呀,没有,我忘了。''江玉郎变色道:``萧咪咪瞧不见我们,必定四下搜索,她若瞧见\ldots\ldots\ldots\ldots\ldots{}''小鱼儿展颜笑道:``你也未免太小心,她难道会想到咱们在粪坑里?''江玉郎道:``我自然要小心,只要稍为大意,只要一时大意,就可能招来杀身之祸,你可知道萧咪咪的武功?''小鱼儿苦笑道``我就因为摸不透她的武功,所以不敢和她翻脸\ldots\ldots 假如是笨人,武功高些我也不怕,但她,她简直也是个妖怪。''江玉郎叹道:``她武功之高,只怕远出你想象之外,据说,她一生中有七百多个情郎,其中还包括了七大剑派中的子弟,每人只教她一手武功,就够人受的了。''小鱼儿眼珠子一转,道:``如此说来,倒是真该小心些才好,我还是再偷偷溜出去一趟,把那见鬼的盖子盖上吧。''江玉郎道:``你等一等。''他口中说话,耳朵已贴在土壁上,听了半晌,失色道:``不好,她已经回来了。''

\hypertarget{ux7b2cux4e09ux5341ux7ae0-ux4f5cux6cd5ux81eaux6bd9}{%
\chapter{第三十章
作法自毙}\label{ux7b2cux4e09ux5341ux7ae0-ux4f5cux6cd5ux81eaux6bd9}}

小鱼儿耳朵也贴上土壁,静静的听地上面,果然已有声音传下来,各种声音。

萧咪咪自然要发怒,要暴跳如雷,要呼唤、咒骂,小鱼儿虽然听不到她在骂什么话,也可想象得出。

江玉郎道:``我算了许久,算准她本来是绝对想不到我会藏在地下的,她必定以为我已想法子溜了,但那盖子''\ldots.``小鱼儿道:''我想,她在气得快发疯的时候,是不会留意到粪坑的盖子是否盖着的。``江玉郎道:''但愿如此。"

他停了停,又道:``只要她找不着咱们,就必定不会再逗留在上面的,人已死光了,她还留在那里干什么?''小鱼儿道:``不错,她一定会走的。''

江玉郎道:``咱们最多在这里呆半个月,她一定早巳走了,那时,咱们就可以大摇大摆地走出去,也不怕她再来追。''小鱼儿道:``你知道那秘密的出口。''

江玉郎淡淡一笑道:``天下绝没有一件能瞒住所有人的秘密。''小鱼儿笑道:``好,咱们就等半个月吧,在地下住半个月,倒也是件有趣的事,倒也不是每个人都能享受到的。''他又躺下来,眨着眼睛笑道:``只不过\ldots\ldots 抱歉得很,我还是不能解开你的穴道。''江玉郎道:``你\ldots 你真要这样?''

小鱼儿道:``我不能不这样\ldots\ldots 只因为我和你这样的人日夜在一起,我实在有点不放心,实在不能不提防着你。''他一笑道:``我差点忘了告诉你,我点你穴道,所用的手法,你自己是绝对解不开的。''这地洞就像是蛇穴一样,江玉朗也正像是条蛇,和一条蛇一起睡在蛇穴里,能睡着的人大概不多吧。

小鱼儿却睡着了。他吃了条香肠,吃了块糯米糕,还喝了碗酒,他脸红红的,睡得很甜。

壁上自然有个小洞,洞里自然有盏灯,灯光照着他红红的脸,江玉郎的眼睛,也在瞧着这张红红的脸。他暗中在数着小鱼儿的呼吸。已数了四千多下了。小鱼几的呼吸均匀得很。

江玉郎已检查过自己两条腿经脉,这该死的小鬼果然没有说假话,他用的竟不知是哪一派的该死的点穴手法。现在,他睡得很熟,因为他知道江玉朗不敢杀他。

但江玉朗却悄悄伸出了手。小鱼儿仍在睡着,甚至开始轻轻的打呼。

江玉朗眼睛盯着他,手尽量往前伸。小鱼儿呼声越来越响。

江玉朗的手突然拿起了一本书,极快地翻开书,书里面夹着张叠着的纸,江玉郎松了口气,拿出了那张纸。

他轻轻将书放回去,小心地将那张纸叠得更小,想了想,想塞进靴子,最后却终于是藏在发髻里。

这时,他苍白的脸像是发了光。然后,他叹了口气,闭上了眼睛,不久他也睡着了。

小鱼儿的眼睛突然睁开,他睁得很大。灯光照着江玉郎苍白的脸,眼睛里带着些讥嘲,也带着些笑。

这双眼睛像是在说:``你瞒不过我的,你什么事都瞒不过我的。''江玉郎的呼吸也均匀得很。小鱼儿悄悄站起来,伸出一只手,在江玉郎面前晃了十几下,江玉郎呼吸仍然很均匀,完全没有感觉。

这小狐狸的确太累,真的睡着了。小鱼儿轻轻的,慢慢的,伸出了两根手指,去掏江玉郎的头发,但还未触及头发,这两根手指突又改变了方向,向江玉郎的``睡穴''点了过去。

睡着了的江玉郎突然叹了口气,道:``你要拿,就拿去吧,又何苦再点我的穴道。''小鱼儿怔了怔瞬即笑道:``原来你也没有睡着。''江玉郎苦笑道:``和你这样的人在一起,我怎么睡得着。''小鱼儿笑道:``但你假睡的本事却真不错,我竟也被你骗过了。''江玉郎道:``彼此彼此。''

小鱼儿大笑道:``妙极妙极\ldots\ldots 你头发里的东西拿给我瞧瞧好么?''江玉郎苦笑道:``我能说不好么?''

他苦笑着自发留中取出那张纸,指尖已有些颤抖,这张纸他看得比什么都重,但此刻却只有拿出来。对于不能反抗的事,他是从来不会反抗的。

他将纸抛给小鱼儿,仰首长叹道:``我只怕是上辈子缺了很大的德,老天才会让我遇见你。''小鱼儿心里委实充满了好奇。他委实想不出这张纸上究竟有什么秘密,但他相信江玉郎既然如此看重这秘密,这秘密就绝对不是普通的。

他打开这张纸的时候,也不禁有些心跳,但他瞧了一眼\ldots\ldots\ldots\ldots 只瞧了─眼后,竟然笑了起来。

江玉郎瞪着眼睛,道:``你很得意,是么?''

小鱼儿道:``是,是,我得意极了。''

江玉郎咬牙道;"你能瞧见这秘密,的确是该得意的,只因你一生之中,再也不会看到比这张纸更宝贵的东西。

小鱼儿道:"是,是,这张纸的确宝贵得很\ldots\ldots{}

他一面说话,一面竟将那张纸撕得粉碎。江玉郎大概一辈子也没有像此刻这样吃惊过。他的脸色更苍白得好可怕,颤声道:``你\ldots\ldots 你\ldots。你可知道这张纸的价值?''小鱼儿悠悠道:``我非但知道,还瞧见过''\ldots 我自己也有过一张。``江玉郎怔住了,道:''你\ldots\ldots 你自己有过一张?"``我非但自己有过一张,而且还去过那藏宝之处!''原来江玉郎的这张纸,就和铁心兰交给小鱼儿的那张一模一样,就是那骗死各种人不赔命的藏宝秘图。

江玉郎自然不知道这其中曲折,此刻简直被吓呆了,道:``你\ldots\ldots 你去过那藏宝之处!你没有骗我?''小鱼儿道:``我为何要骗你!''

江玉郎呼吸突然急促起来,道:``那宝藏\ldots\ldots\ldots 那宝藏已落入你手中?此刻在何处?''小鱼儿目光闪闪,道;"你先告诉我这张藏宝图是从哪里来的,我再告诉你。

江玉郎两只手紧紧抓着自己的衣服,道:``我说出了,你真的告诉我?''小鱼儿笑道:``你说了我若不说,我就是乌龟。''江玉郎喘了口气,道:``这份藏宝图,我是从我爹爹书房里偷出来的。''小鱼儿道,``你父亲又是从哪里得来的?''

江玉郎道:``不知道,我真的不知道。''

小鱼儿沉吟道:``不错,听说你父亲也是个成名人物,这张图想必是有人送给他的,却不想他竟有个好儿子。''他叹了口气,摇头笑道:``连父亲的东西都要偷,这么好的儿子实在不多。''江玉郎脸居然红也不红,道:``这又算什么,我\ldots\ldots\ldots{}''小鱼儿道:``你一心想得到这宝藏,连父亲也不认了,一个人偷偷溜出来,溜到峨嵋山,哪知却落入了萧咪咪的手中,幸好你遇着她,否则此刻只怕已死了。''江玉郎奇道:``为什么?''

小鱼儿笑道:``你父亲也幸亏有你这样个宝贝儿子,否则就难免要上个大当。''江五朗吃惊道:``上当?''

小鱼儿道:``老实告诉你,这藏宝图根本是假的,根本一文不值,造出这藏宝图的人,只是要寻宝的人自相残杀!''江玉郎完全怔住了,怔了半晌,呐呐道:``这人是谁?''小鱼儿恨恨道:``我也不知这人是谁,但我一定要找出他来,我倒不是要为大众除害,只是他既然令我上了当,我就要他好看。''江玉郎喃喃道:``难怪你要问我这张图是从哪里来的,难怪你\ldots\ldots{}''突然间,一阵呼声从那地道中传了进来。

竟是萧咪咪的声音在呼唤着道:``江玉郎''\ldots.江小鱼两个小坏蛋,你们在下面么?"小鱼儿、江玉郎两个人的手胸都吓凉了,动也不能动。

只听萧咪咪咯咯笑道:``你们不出声也没用,我己知道你们在下面了。''江玉郎颤声道:``她\ldots\ldots 她只怕是在使诈。''

小鱼儿道:``不会,此刻她就对着粪坑在喊,否则咱们是听不见的。''江玉朗叹道:``那盖子\ldots\ldots 我就知道那盖子要出毛病。''小鱼儿叹道:``这女人真厉害''\ldots\ldots"

只听萧咪咪笑道:``江玉郎,你真是个天才,居然想得出躲在粪坑里,也不怕臭。''小鱼儿笑道:``你听,她也说你是天才。''

江玉郎道:``你\ldots\ldots 你还笑得出?''

小鱼儿道:``仔细想想,我为何笑不出?''

江玉郎道:``你\ldots 你不怕她\ldots。''小鱼儿道;``就算她厉害,但咱们在这里等着,她敢爬进来么,以她的脾气,也不会守在外面等着的。''江玉朗想了想,笑道:``呀,不错,她明我暗,她绝不会来冒这个险,就算她等,也等不了许久,咱们总有机会溜出去。''只听萧咪咪道:``两个小坏蛋,出来吧。''

小鱼儿大喊道:``你这老坏蛋,你进来吧。''

萧咪咪道:``你们不出来?''

小鱼儿道:``你为何不进来?''

萧咪咪咯咯笑道;``你们情愿在下面臭死?''

小鱼儿大笑道:``你放心,咱们臭不死的,这里舒服得很,有香肠,还有酒,你要不要下来陪我们喝两杯?''萧咪咪笑道:``你们不怕臭,我却怕臭。''

她语声微顿,又道:``何况,我也不希望你们上来。''小鱼儿大笑道:``是吗?''

萧咪咪道:``你们若上来,我一发脾气,说不定就宰了你们,那样反面让你们死得太痛快了,我要让你们慢慢的死。''小鱼儿大笑道;``你有什么法子让我们\ldots\ldots{}''

话未说完,突然再也笑不出了。

萧咪咪嘻嘻笑道,``笑呀,小坏蛋,为什么不笑了?''江玉郎面色也又变了,两人齐声大呼道:``萧姑娘\ldots 萧姑娘''。

地道中却再也没有声音传进来。江玉郎、小鱼儿对望了一眼,两人都面色如土。

只听``轰''的一声,接着哗啦啦响个不住。

江玉郎颤声道:``完了\ldots{}''

小鱼儿道:``好狠\ldots\ldots 最毒妇人心,我早该想到她有这一着。''江玉朗惨笑道:``现在,再也用不着盖子了\ldots\ldots{}''小鱼儿精神突又一振,大声道:``她虽然将外面堵死了,但咱们还是可以再挖出去。''江玉朗叹道:``她存心将你我困死在这里,必定在上面盖了铁板、石板\ldots\ldots{}''小鱼儿道:``咱们另外换个地方往上挖。''

江玉朗道:``当初建造此地之时,为了防潮,这上面却铺着一尺多厚的石板。''小鱼儿默然半晌,反手拍开了江玉郎的穴道:``想来你也不会再动我的脑筋了\ldots{}''江玉朗木然道;``半个月\ldots\ldots 半个月后,就得饿死在这里。''小鱼儿重重的拍了拍他的肩头,大笑道;``振作些,莫要愁眉苦脸,咱们至少还有半个月好活''\ldots 我本已死过好多次,这半个月已是捡来的。"他虽在大笑,其实笑的声音也难听得很。

江玉郎只怕已有叁个时辰没有动了。

他就这样坐在那里瞪着,两只眼睛发呆,也不知想些什么,小鱼儿打开酒坛,叫了他八次,他也像是没听见。

于是小鱼儿就自己喝了起来。他喝一口,笑一声,喝一口,又叹口气,喃喃道:``一个人知道自己要死了还不喝酒,这人一定是呆子。''江玉郎瞪着他,没有说话。

小鱼儿道:``唯一遗憾的是,咱们都死得太早了些,我现在简直有些后悔,方才本应和萧咪咪风流风流才是,唉,人不风流枉少年\ldots{}''他摇摇晃晃站起来,去摘挂在上面的香肠。

江玉郎冷冷道:``你醉了。''

小鱼儿笑道:``醉死最好,醉死鬼总比饿死鬼好得多\ldots\ldots{}''江玉郎突然一掠而起,一掌向他后颈劈了过去。他身法好轻,出手好快,一掌就想要小鱼儿的命!

\hypertarget{ux7b2cux4e09ux5341ux4e00ux7ae0-ux67f3ux6697ux82b1ux660e}{%
\chapter{第三十一章
柳暗花明}\label{ux7b2cux4e09ux5341ux4e00ux7ae0-ux67f3ux6697ux82b1ux660e}}

但小鱼儿瞧见灯光一花,已霍然转身,刚好接了他这一掌,两个人身子俱都一震,两个人都撞上土壁。

小鱼儿瞪大眼睛,吃惊道:``你''。``你想杀我?''江玉郎道:``一点也不错。''

小鱼儿道:``你我反正是要死的,你为什么\ldots{}''江玉郎道:``这里的食物本够一个月吃的,多了你,就少吃半个月,杀你后,我就可以多活半个月。''小鱼儿道:``为了多活一天你也会杀我?''

江玉郎道:``为了多活一个时辰我也会杀你!''小鱼儿苦笑道:``我虽然知道你是个坏人,但真还没有想到你竟坏成这样子,若论心肠之狠毒,天下只怕得数你第一。江玉郎道:''你呢?``小鱼儿道:和你比起来,我简直就像是个吃长素的老太婆。''这句话他还未说完,他的手已到江玉郎面前。这地洞是如此小,他身子根本不必动,就可以打着江玉朗的脸。

他这一掌也许是真打得快,也许是江玉郎根本没有想到他会出手,所以根本没有闪避。总之,这一掌是着着实实打着了。

只听``啪''的一声,江玉郎半边脸已红了,人已倒下去。

小鱼儿笑道:``你看来虽瘦,脸上的肉倒不少,我若是没看清楚这一巴掌的确是打在你脸上,还真要以为是打着了个胖女人的屁股。''江玉郎捂着脸颤声道:``你\ldots 你要干什么?''

小鱼儿道;``你要杀我,我难道不能杀你?''反手又是一巴掌。

江玉郎的脸,看起来像条死鱼的肚子,颤声道:``你我两个反正都已快死了,你\ldots 你何苦\ldots{}''小鱼儿大笑道:``这话不错,但你提醒了我,我若杀死你,就可多活半个月。''江玉朗垂首道:``我\ldots\ldots 我该死\ldots\ldots 该死\ldots.''他突然将整个人都当做把流星锤似的,一头撞向小鱼儿的肚子,他的脑袋虽不算太硬,但总比肚子硬得多。

小鱼儿早就留心他的一双腿两只手,但说老实话,他实在没有去留意他那颗小脑袋。整个人被撞入角落里,像是个虾米,弯下了腰,捂着肚子,足足有半盏茶时候没有喘气。

江玉郎冷笑道:``现在,你知道该死的是谁了。''他用足力气,一脚向小鱼几下巴踢过去。

小鱼儿呻吟着,仿佛已抬不起头,但等到这只脚到了他面前时,他捂着肚子的手突然闪电般伸出。他这双手就像是抢着去抱了只从宰相千金手里抛出来的绣球似的,抱住了江玉郎的脚,右脚,然后,他把这只右腿拼命的向左一扭。

江玉郎惨叫一声,整个人鱼一般翻了个身,噗地跌在地上,跌了个狗吃屎,鼻血都流了出来。

小鱼儿人已跳在他背上站着,笑道:``现在我的确知道该死的是谁了。''江玉朗趴在地上呻吟着,道,``我服了你,我真的服了你,你什么事都比我强,但我知道你不会真的杀我的,你若要真的杀我,也用不着等到现在。''这小子居然开始乞怜,开始拍马屁,倒不是件容易事,小鱼儿听了却一点也不开心,反而有些毛骨悚然。小鱼儿知道这小子心思其实很想用一把刀子插入他喉咙,或者是什么别的地方,一些比较软的地方。不过他现在没有刀子.纵然有刀子也不行,一个人被别人睬着自己背脊的时候,是割不到别人的喉咙的。

他不过是在等一个机会,好用刀子慢慢的割。

小鱼儿如果算不上是十分穷凶极恶的话,至少可以说是十分聪明,他自然懂得江玉郎的意思,但他明知江玉郎要杀他,却偏偏要给江玉郎这机会,他要看江玉郎到底能用什么法子杀死他"这的确是件有趣的事。对于有趣的事,小鱼儿是从来不愿意错过的。尤其是当他已自知活不长的时候。

小鱼儿有趣地想着,几乎已忘了快要被困死的事。

就在他想得最有趣的时候,江玉郎身子突然用力拱了起来.把站在他身上的小鱼儿弹了出去。若是在平时,这也没什么关系,但这里却是个地洞,一个很小的地洞,高个子在这里几乎不能抬头,于是小鱼儿的头就撞上了上面的顶,``咚''的,就好像打鼓一样,然后他的人也就像鼓槌一样倒下去。

但江玉郎也是过了许久才爬起的。他一爬起来,就扼住了小鱼儿的脖子,阴险地笑道:``我知道你不会真的杀死我的,但我却要真的杀死你。''他手指用力,小鱼儿却一点反应也没有。

江玉郎手指又放松了,他不愿意在小鱼儿晕过去的时候杀他,他要看小鱼儿挣扎着、透不出气来的样子。

小鱼儿竟偏偏不醒。江玉郎腾出一只手,把那个已滚倒在旁边的酒坛子拎起来,把坛子里剩下来的酒全倒在小鱼儿头上。

他酒还没有倒完,小鱼儿的手突然从他两只手中间穿出去,一拳打在他喉咙上。江玉郎疼得脸都变了形,但手里的酒坛还是没有忘记往小鱼儿头上摔下去,小鱼儿自然早已料到他这一着,身子一滚,跟着飞出去一脚,踢在江玉郎某一处重要部位上,酒坛被摔得粉碎,江玉郎身子已蜷曲得像只五月节的棕子,动也不能动,连呼吸都接不上气了。

小鱼儿这一脚的确很有效,但却并不十分漂亮,这简直不能算是招式,从头到尾,他两人根本谁也没有使出一着漂亮的招式。因为在这种老鼠洞一般的地方,谁也使不出漂亮的招式,幸好他们不是打来给别人瞧的,也没有别人能瞧见他们.灯光,像是渐渐暗了。

小鱼儿突然跳起来,道:``不好。''

江玉郎道:``什么不好,我们现在已够坏了,还有什么事更不好?''小鱼儿叹道:``我们还没有被饿死,已经要被闷死了。''地道被堵死,空气中的氧气渐渐稀薄,连灯光都快要灭了,他感觉到呼吸已渐渐不通,眼皮已渐渐发重。

江玉郎颤声道:"我什么都算过了,就没有算到这点\ldots\ldots{}

小鱼儿道:``现在你就算能杀死我,最多也只能活半个时辰了。''江玉朗道:``半个时辰\ldots。半个时辰''他牙齿已打起战来。小鱼儿也是愁眉苦脸,喃喃道:``闷死\ldots\ldots 闷死的滋味不知如何?''江玉郎道:``我听人说过,闷死比什么都痛苦,在闷死之前,人就会发疯,甚至将自己的脸都抓得稀烂!''此刻他还有心情说这些话,只因他觉得只有自己一个人害怕太不公平,他得要小鱼儿也分享这恐怖。

小鱼儿默然半晌,突然笑道:``那也不错,我就怕死得太平常,现在总算能很特别的死了!世上能被闷死的人总是不多。''江玉郎也默然半晌,缓缓道:``但也不少!当初建造此地的人,只怕也是被活活闷死。''小鱼儿眨了眨眼,道:``到现在为止,你还是在尽量想法子刺激我?''江玉郎路冷道:``你实在太开心,我不知你究竟能开心到什么时候。''小鱼儿道:``你真的那么恨我?''江玉郎道:``哼!''小鱼儿道:``你恨我,只因为我什么事都比你强,是么?''江玉郎道:``也好我们生下来就是对头!''

他说这句话的时候,绝不会想到这句话并没有说错。

火光,更弱了。小鱼儿茫然瞧着这点渐渐小下去的火光,喃喃道:``酒!该死的酒,却被你这该死的人糟蹋了,现在,还有什么事能比真正的烂醉如泥更好。''他目光转到地上。地上满是酒坛的碎片,酒,已快干了。但奇怪的是,酒竟非渗入泥土中去的。

这地面自然不平,酒往低处流\ldots\ldots{}

小鱼儿突然跳起来,把一缸水全都倒在地上。水,也在往低处流。

小鱼儿狂呼道:``喂,你瞧\ldots\ldots 瞧!''

江玉郎道:``瞧还有什么好瞧的。''

小鱼儿道:``你瞧这水\ldots\ldots 水一直在流。''

江玉郎道:``水自然要流,自然要往低处流。''小鱼儿指着一个角落,似已紧张的说不出话,吃吃道:``你瞧,水都往这里流,但却没有积在这里。''江玉朗眼睛也瞪大了,道:``不错,水没有积在这里。''小鱼儿道:``水没有积在这里!自然是流了出去,水流了出去,这里自然有个洞,但这里已经是地底下.怎么会有个让水流出去的洞''小鱼儿再不说话,捡起一块碎坛子,在那块地方拼命的挖了起来,江玉郎呆呆地瞧着,一双手在抖。

两个人此刻已更难呼吸了。微弱的火光,突然熄灭,四下立刻一片黑暗,暗得伸手不见五指,江玉郎也不知小鱼儿究竟挖得如何。只听小鱼儿在喘着气,他自己也在喘着气。

突然,砰的一响,像是木板碎裂的声音。接着,小鱼儿大叫道:``洞\ldots。.我又挖出了个洞\ldots\ldots 外面竟是空的!''江玉郎颤声道:``你你没有弄错?''

小鱼儿道:``火折子,火拆子\ldots 看在老天份上,你千万莫要说没有火折子。''有火折子又有什么用?小鱼儿会说出这句话来,只怕是已经晕了头了。

但火折子却亮了起来。小鱼儿人已赫然不见了,那地方已多了个洞,一阵阵阴森森的、带着腐臭味的风,从洞外吹进来。

江天朗呼吸竟渐渐通了,大喜唤道;``江\ldots\ldots 江公子,江兄。''小鱼儿的声音在洞外道:"快过来,快。

这声音中充满惊奇、狂喜。江玉郎几乎像滚一样钻了进去。然后,他就呆立在那里。

这里竟是个八角型的屋子,那八面墙,有的是铁,有的是钢,有的是石板,竟还有一面像是金子。

而谢天谢地,他们这一面恰巧是木板──这一面若不是木板,他们此刻只怕已闷死在那里了。

八角型的屋子里,没有桌子,没有橱子.因为在地底,所以也没有蛛网、积尘,空气也不知是哪里进来的。

屋子里只有绞盘,大大小小、形状不同的机关统盘,有的是铁铸,有的是石造,自然,也有的是金子的。

江玉郎几乎连气都喘不过来,喃喃道:``天呀!天呀\ldots。这里是什么地方?打死我也想不出来!而''\ldots 而这地方竟和我那洞只有一板之隔。"小鱼儿围着这屋子在打转,也惊奇得不知如何是好。这究竟是什么地方?这些绞盘究竟是做什么用的?他看来看去,也看不出这些绞盘的巧妙,这些绞盘一个连着一个.也不知花了多少功夫才做出来的。

小鱼儿一辈子也没有贝过这么巧妙的东西。

江玉郎道:``你瞧出了么?这究竟是什么地方?小鱼儿苦笑道.谁能瞧出才是活见鬼了。''

江玉郎掠过去,用袖子擦一面墙,擦了一会儿,失声道,``天呀,这墙果然是金子。''小鱼儿道:``墙是金子的倒不稀奇,稀奇的是这地方居然能通气,建造这地方的人若是没有发疯,必定另有用意。''江玉郎道:``什什么用意?''

小鱼儿长长叹了口气道:``这只怕是你我这一辈子今所见的最大秘密。''他的手按在一个绞盘上。

江玉郎道:"你\ldots\ldots 你要去搬它?

小鱼儿道:``你能忍得住不搬么?''

他朝江玉郎挤了挤眼睛,笑道:"这里说不定就是地狱的门户,我绞盘一搬,说不定就将鬼都放了出来。

江玉郎咬牙道:``你这笑话不错,真是好笑极了.''两个人突然同时打了寒酸。``吱!''的一声,绞盘已转了。那画石板墙,已突然一转,现出了个门户.小鱼儿大笑道:``你瞧,地狱的门果然现出来了。''其实他自己也知道,他这笑声真不知有多难听。

江玉郎爬回去,取出了那盏灯。

小鱼儿拿着火折子,走到前面,一阵阵腐臭气从门里飘出来,那味道小鱼儿一辈子也没有嗅过。他再也不想嗅第二次。

两个人胆子总算不小,总算走了进去。死尸,这门里竟是一屋子死尸!江玉郎的手在抖,不停的抖,只见这些死尸\ldots\ldots{}

这些死尸的形状,我纵然能说,也还是不说的好,何况,我根本说不出,只怕也没有人能说得出。

这里其实只是一屋子穿着衣服的骷髅,小鱼儿打了个喷嚏,他面前一具骷髅的衣服突然化作了粉灰。

小鱼儿只觉背脊发凉,道,``这些人,只怕已死了几十年。''江王郎道:``他\ldots\ldots 他们都是饿死,你瞧他们的摸样,临死前想必已饿得发疯了,你瞧他\ldots\ldots 他们的手。''小鱼儿想到自己险些也要变成这模样,突然忍不住想吐,竟将方才吃下去的酒肉全都吐了出来。

江玉郎道:``这些人,不知道都是些什么人?''小鱼儿呕出了最后一口苦水,喘息着道:``瞧他们的衣服都很粗俗,想必就是建造此地的工匠。''江玉郎道:``想必是一群呆子。''小鱼儿道;``呆子?''江玉郎道;``若不是呆子,怎会为人建造如此秘密的地方?\ldots。为人建造了如此秘密之地,本就是再也活不成了。''小鱼儿道:``你瞧见这许多人如此惨死,一点都不同情?''江玉郎道:``我若死了,谁来同情我?''

小鱼儿叹了口气,道:``很好,你很好,我在天下恶人集中的地方学了十年,看来还不如你,看来我还得向你学。江玉郎道:''奇怪的是,萧\ldots"话未说完,突听一阵脚步声传了过来。这脚步声缓漫而沉重,似是拖着狠重的东西。

小鱼儿全身的寒毛都悚立起来,他纵然是天下胆子最大的人,此时此刻,也不能不害怕了。

江玉郎的手又在抖,道:``这\ldots\ldots 这''

他心肠虽狠毒,胆子却不大,此刻已说不出话来,``当''的一声,他手里的铜灯也跌落了地上。脚步声似是从上面传来的,已越来越近。

小鱼儿手脚也骇软了,手里的火拆子不知何时也跌落在地,四面立刻又是一片黑暗,该死的黑暗。

沉重购脚步声,像是已踩破他们的苦胆。两个人想往外逃,竟抬不起腿!

突然间,上面露出了个洞,一片昏黄的光线照了下来。小鱼儿、江玉郎都屏住呼吸,动也不敢动。

他们看到了一双脚。

这是纤细的、穿着绣花鞋的脚。脚上面还有一截绿色的裙子,再上面就瞧不见了。

两人偷偷对望一眼,几乎忍不住要同时脱口道;``萧咪咪!''这不是女鬼,竟赫然真的是萧咪咪。

只听萧咪咪的语声喃喃道;``你们就在这里歇歇吧,这地方还不错,虽然稍为挤了些\ldots\ldots{}''语声中,一条人影直落下来。这女妖怪又在害什么人?

小鱼儿、江王郎又是一惊,但瞬即发觉这不过是具死尸──死尸就这样一具具被秘密抛落了下来。

萧咪咪的语声又道:``能住在这么豪华的坟墓里,你们也算死得不冤了,再见吧,两位\ldots。说不定有时我也会想想你们的。''``砰''的,洞又合起,又是一片黑暗。

江玉郎、小鱼儿在黑暗中等了许久许久,才长长透出了一口气,小鱼儿突然哈哈一笑道:``江玉郎,这些死尸就是被你害死的人,你不怕他们找你索命。''江玉郎道:``他们活的时候我都不伯,死了我怕什么!''小鱼儿在脚旁摸着了火折子,火折亮起,照着江玉郎的脸,那几乎也已不像是张活人的脸。小鱼儿笑道:``你不怕,脸怎么骇成这副样子。''江玉郎突然拾起钢灯,大步走了出去。小鱼儿也赶紧跟出去,他可不想被江玉郎关在这里,老实说,从今以后,谁也无法再让他走进这里一步了!

如此``豪华''的地方,他实在吃不消。江玉郎站在一旁,也在呕,他呕的全是苦水。

小鱼儿喃喃道:``我本就怀疑这地方绝不是萧咪咪建造的,女人,怎会有这么大的手笔,现在已可证明我怀疑的果然不错。''江玉郎道;``哼。''

小鱼儿道:``她不知走了什么运,被她发现上面那地方,但找到这里时,她瞧见那许多死尸,就再也不敢往下找了,却不知她找着的只不过是这地下宫阙的一部分而已,说不定只是最差劲的一部分,精采的全在后面哩。''他长长叹了口气,接道:``但这地方又是谁建造的?普天之下,谁有这么大的手笔?''江玉郎冷冷道:``至少,总不会是你吧。''

小鱼儿朝他扮了个鬼脸,道:``你莫要忘记,我武功比你强,还是随时都可以宰了你。''江玉郎情不自禁,后退一步,变色道:``你\ldots\ldots 你\ldots\ldots{}''小鱼儿嘻嘻一笑,道:``但你也莫要着急,我只不过是要你说话客气些。''江玉郎瞪着眼瞧了半晌,垂头道:``我年纪还轻,什么事都不懂,若是说话得罪了你,你总该原谅我一些,我\ldots\ldots 我心里总是把你看我的大哥的。''小鱼儿笑道:幸好你并非真的是我弟弟。"

他举着火折子,围着八角屋子走了一圈,一只手东摸摸,西敲敲,眼珠子不停地转,口中道:``这里八面墙,只有一面是土砖砌成的,其余七团除了石榴和木壁之外,还有金、银、铜,铁,锡。''江玉郎道:``他们用八种不同的东西来造这八面墙,想必也有用意。''小鱼儿道:``不错,你可知道是什么用意?''

江玉郎陪笑道;``我就是不知道.所以才请教大哥你。''小鱼儿瞧了他半晌。缓缓道;``你听着,我告诉你两件事。''江玉郎道:``但请大哥吩咐。''

小鱼儿瞪着眼道;``第一,你以后千万莫叫我大哥,这称呼我听了肉麻。''江玉郎怔了怔,立刻垂下头,道:``是。''

小鱼儿道:``第二,以后也莫要在我面前装傻,我知道你是个聪明人,很聪明,你装傻也没有用的.''江玉郎乖乖地点头道:``是。''

小鱼儿一笑,道:``现在,你且说你猜他们是何用意?江玉朗嗫嚅道:''我不知猜的可对\ldots\ldots 他们造这八面不同的墙,一来表示在八面墙后面,藏着不同的东西。``小鱼儿道:''不错,二来呢?"

江玉郎道:``二来,便和这绞盘有关系,这石绞盘是控制这石壁的,那金绞盘想必就是控制金壁的。''小鱼儿笑道:``很好\ldots\ldots 说下去。''

江玉郎道:``那木壁后是咱们出来的地方,自然不会有什么东西,石壁后是坟墓,咱们也不愿再看了,至于这土墙,看来是实心的,想必也不会有什么巧妙,现在剩下的只有金、银、铜、铁、锡这五面墙了!''小鱼儿道:``不错,这五面墙壁后,必定有些花样。''他眨了眨眼睛,接道:``你说,咱们先试四面墙呢?''江玉郎道,``金的。''

小鱼儿道;``很好,这一次你倒没有说假话,我心里其实也是想先试这面金墙的,其实世上的人又有谁不?''

\hypertarget{ux7b2cux4e09ux5341ux4e8cux7ae0-ux5730ux4e0bux5b9dux85cf}{%
\chapter{第三十二章
地下宝藏}\label{ux7b2cux4e09ux5341ux4e8cux7ae0-ux5730ux4e0bux5b9dux85cf}}

黄金的绞盘转动,黄金的墙壁果然随之移动,现出了道门户,他们人还未定进去,已有一片辉煌的光洒了出来。这金色的墙壁后,竞赫然全都是珠宝,数不清的珠宝,任何人做梦都想不到会有这么多的珠宝!

江玉郎站在那里,整个人都已呆住了,苍白的脸上,竟泛起了异样的红晕,指尖也开始微徽颤抖。

小鱼儿的眼睛却只不过在这些珠宝上打了个转,便转到江玉郎那张激动的脸上,微微笑道:``你喜欢么?''江玉郎道:``我我\ldots\ldots{}''

他初初凸起的一点喉结上下移动,强笑道:``我想,世上没有人不喜欢这些的!''小鱼儿道:``你若喜欢,这些就全算你的吧!''江玉郎惊喜地瞧了他一眼,但瞬即垂下了头,陪笑道:``这宝藏是你先发现的,自然归你所有,我\ldots\ldots 我\ldots\ldots 只要能分我一点,我已感激得很。''小鱼儿道:``我不要。''

江玉郎猝然抬起了头,失声道:``不要?\ldots{}''但立刻又垂下,陪笑道:``我性命都是你所赐,你纵然不肯分给我,我也毫无怨言。''小鱼儿笑道:``你以为我在试探你,存骗你?这些东西饥不能当饭吃,渴不能当水饮,带在身上又嫌累赘,还得担心别人来抢,我为什么要它!''江玉郎呆在那里,再也说不出话来。

小鱼儿也不理他,又在这屋子里兜了圈子,喃喃叹道:``这里的也全都是死的,出路想必也不在这里。''江玉郎突然咯咯笑了起来,笑个不停。

小鱼儿道:``你瞧见了鬼么!''

江玉郎笑道:``这些东西,我也不要了。''

小鱼儿道:``哦,这倒稀奇得很,为什么?''

江玉郎道;``我连人都不知是否能活着走出去,要这些东西作甚?''小鱼儿拍手笑道:``你毕竟还没有笨得不可救药,毕竟还是个聪明人,我就瞧见过有些人不惜为这些东西送命的,你说他们的脑子是否有些毛病。''小鱼儿转动了铜绞盘。

于是,他就瞧见了一生中从未瞧见的那么多的兵器,各式各样的兵器,还有各式各样的暗器。有些兵器,固然是小鱼儿熟悉的,但还有些兵器,小鱼儿非但没有瞧见过,简直还不知道它们的名字。

金铁之气,砭骨生寒,森森的寒光,将他们的脸都照成了铁青色,小鱼儿不禁缩起了脖子。

枪,最长的长达丈八,最短的才不过叁尺,剑,最大的宛如木桨,最小的竟宛如筷子。长枪短剑,整齐地排列着,它们虽然没有生命,却又似蕴含着杀机,令人胆寒的杀机!

普天之下,所有的凶杀之器,只怕尽都在这屋里。

小鱼儿随手拔出了一柄剑,只听``呛啷''一声,剑作龙吟,森森的剑气,直逼他眉睫面来。

他忍不住脱曰赞道:``好剑!''

江玉郎沉声道:``这口剑虽是利器,但在这屋子里,却算不得什么。''江玉郎取出了一件兵刃,道:``你可知道这件兵刃是什么?''这件兵刃骤眼看去,就像是金龙,龙的角,左右伸出,张开的龙嘴里,吐出一条碧绿色的舌头。

小鱼儿道:``看来,这像是条金龙鞭。''

江玉郎道:``不错,这是金龙鞭,但这条金龙鞭,却与众不同,叫做九现神龙鬼见愁,一件兵刃却兼具九种妙用。''小鱼儿道:``有趣有趣,你且说来听听。''

江玉郎道:``这条鞭全身反鳞,不但可粘人兵刃,使对方兵刃脱手,还可粘住暗器,龙角分犄,专制天下各门各派软兵刃,龙舌直伸,打人穴道,那张开的龙嘴,咬人刃剑如探囊取物,除此之外,一双龙眼乃是霹雳火器,龙嘴之内,可射出一十叁口子午问心钉,见血封喉,了不过午,在必要时,那浑身龙鳞,也全都可以激射面出,若不知这件兵刃的底细,只怕神仙也难躲过。''他滔滔说来,竟是如数家珍一般。

小鱼儿叹道:``好个鬼见愁,果然厉害。''

江玉郎道:``只可惜普天之下,这同样的兵刃,一共才只有两件,却不知这一件又怎会出现在这里。''小鱼儿道.``还有一件呢?''

江玉郎道;``这兵刃在江湖中绝迹已久,还有一件,也不知到哪里去了\ldots\ldots 哪一件若是在江湖出现,又不知有多少人的性命要葬送在它手上!''小鱼儿笑眯眯道:``想不到你年纪轻轻,竟对这种绝迹已久的独门兵刃也熟悉得很。''江玉郎眼珠子一转,似乎已觉出自己话太多了,强笑道:``我只不过偶然所人说的''``。你知道家父交游素来广阔,其中自然也有一两个万事通先生的。''小鱼儿笑眯眯瞧着他,淡淡道;``如此说来,这件兵刃你会用了!''江玉郎笑道:``我\ldots\ldots 我若会用就好了。''

他像是满不在乎似的,随手放下了这件兵刃,其实,他的眼睛一直在眨也不眨地盯着小鱼儿的手。小鱼儿也像是满不在乎地笑着,其实他的眼睛也未尝有片刻离开过江玉郎手里的鬼见愁。

达两人虽然还都是孩子,但心计之深,纵然有叁百八十个七十岁的老头子加在一起,也不比上他们一个。

小鱼儿笑道:``如此说来,这屋里的兵刃,无论哪一件拿出去,只怕都可以在江湖中轰动轰动,尤其是这鬼见愁\ldots\ldots 唉,我反正不会使它,不如你拿去吧。''江玉郎不等他话说完,已远远走了开去,笑道:``如此歹毒的兵刃,我可不要它。''小鱼儿笑道;``其实,兵刃究竟是死的,人才是活的,只要人强,无论用什么兵刃都是一样,这种兵刃倒真不要也罢。''他突然拔出一口吹毛断发的利剑,剑光展动,竟将这天下第一歹毒的外门兵刃砍得稀烂。

江玉朗脸上自然还是带着笑的.连连道:"好极了,毁了它最好,免得它落在别人手上害人\ldots\ldots,一面说话,一面转过头去,眼里立刻好像冒出火来。

小鱼儿轻抚着手中的剑,笑道:``好剑呀好剑,我本来也有心将你带在身边,但想了想,还是将你留在这里的好,像我这样的人,纵然空手,也\ldots{}''突听江玉郎惊呼道:``看''。``看这里\ldots\ldots{}''

寒光剑气下,一具骷髅斜斜躺在角落里。这具骷髅不但衣衫腐烂,本应是灰白的骨架,此刻竟也变成乌黑色,在寒光下看去更是可怖。

江玉郎喃喃道:``奇怪.这人怎会死在这里?怎地未被抛入那坟墓?''小鱼儿道:``能进到这屋子里来的,只怕便是此间的主人,此间的主人,自然十成是个武林绝顶高手。''他皱眉道:``但此间的主人,又怎会死在这里!又是被谁杀死的?瞧他躺着的样子,丝毫没有挣扎之态,竟显见是被人一击而死!''江玉郎道:``瞧他骨骼却已变色,又像是中毒而死。''小鱼儿道:``不错。''

两人目光闪动,突然同时失声道:``原来他竟是中了别人的毒药暗器!''两人已发现在那乌黑的骨路上,竟钉着无数根细如牛芒的银针,如此细小的银针,竟能穿透皮肉直针入骨头里。

小鱼儿骇然道:``好厉害的暗器,好歹毒的暗器。''江玉朗道:``这是\ldots。·这不知是谁下的手。''小鱼儿瞧他一眼,道:``你也用不着改口,认得这暗器的人只怕不止你一个,我也认得的。''江玉郎苦笑道:``这天绝地灭透骨穿心针,果然不傀是天下第一暗器\ldots\ldots{}''他眼角突然瞥见兵刃架下,有个金光灿灿的小圆筒,立刻就用身子挡住了小鱼儿的目光,一面弯腰咳嗽,一面移动了过去。

小鱼儿笑道:``你再咳嗽,我也要被你染上了。''他竟真的咳嗽起来,咳得弯下了腰。江工郎等他一弯腰.就飞快地伸出手,伸手拾起那小圆筒,却不知小鱼儿同时也在那骷髅的手掌里轻巧地抽出样东西,塞在衣里。

但那只不过是块竹筒,小鱼儿其实也并末瞧出它有什么用,他只不过觉得,这种人到死时手里还紧握住的东西,若是没有用才怪。

江玉郎勉强忍住心里的欢喜,故意皱眉道:此人若是此间的主人,又怎会被人暗算死在这里?\ldots\ldots 但他若不是此间的主人更没有道理死在这里。``小鱼儿道:''嗯,他若不是此间的主人,根本进不来。``江玉郎道:''那么,这究竟是怎么回事?"

小鱼儿道:``看来,此间还有许多秘密。''

江玉郎叹了口气,道:``许多可怕的秘密。''

小鱼儿笑道:``世上没有可怕的秘密,世上所有的秘密,都是有趣的\ldots\ldots{}''两个人并肩走出了这可怕而又有趣的屋子,两个人都故意用双手举着灯火,表示他们都没有拿走任何东西。

铁壁移动,灯光照入了这寒气森森的铁屋。

江玉郎当先走了进去,目光转处,突然惊呼一声,退了出来。那神情看来就像是只中了箭的兔子。

小鱼儿皱眉道:``这里面又有什么?''

江玉郎脸色苍白,道:``你瞧见会站着的骷髅么?''小鱼儿笑道:``站着的骷髅,这倒有趣。''

他大步走了进去,却也有些笑不出来了。只见这铁屋特别大,特别高,四壁空空,什么也没有,─个人站在里面,就好像站在旷野中似的。

就在这空旷而阴森的屋子中央,孤零零地站着两具骷髅,两具惨白色的骷髅,紧紧拥抱在一起。死人的血肉已化,但骷髅至今犹屹立不倒。

小鱼儿瞧得心里实在也有点儿发毛,口中却笑道:``这只怕是一男一女,瞧他们临死前还抱在一起,舍不得放手,可见他们交情必定不错!说不定是殉情而死。''江玉郎跟了进来,道:``若是交情不错,就不会站着了。''小鱼儿失笑道:``呀,这点我倒没想到,在这方面,你经验的确比我丰富,但这两人若都是男的,却又抱在一起干什么?''他嘴里说话,人己走了过去,站在这两具骷髅面前,像是发了会儿呆,又长叹了口气,道:``这两人果然全是男的。''江玉郎突然笑道:``男人和男人,交情有时也会不错的。''小鱼儿道:``你怎知道?''

江玉郎道:``你过来瞧瞧也知道了。''

这两具骷髅其实并非拥抱在一起的,左面一人的右掌,直插入左面一人的肋骨里,他赤手一抓,便能直透入骨,这是何等的惊人的武功,何等惊人的掌力!但他自己的胸骨却也折断了七八根之多,脖子也被对方捏断,一颗头软软垂下来,倒在对方肩上;这两人竟是在恶斗之下,各施杀手,同归于尽!

江玉朗骇然失声道:``好厉害的鹰爪功;好厉害的掌力!看来这两人想必都是绝顶的武林高手,却不知怎会死在这里!''话犹未了,只听``哗啦啦''一响,两具骷髅却被他语风震例,两个绝项武林高手,此刻便化为一堆枯骨。

小鱼儿沉吟道:``瞧这两人的武功,只怕也是此间的主人之一,两人既然共同隐居在这种秘密之处,情谊必定非浅,为何又要拼个你死我活,结果弄得谁也活不了。''一面说话,一面又自枯骨堆里拾起了两件东西。

江玉郎道:``这地底宫阙里别的人都到哪里去了,难道也都死光了不成?''小鱼儿道:``非但死光,而且还一定要是同时死光的,否则他们枯骨就绝对不会─直留到现在,害得咱们吓一跳。''江玉郎道:``他们若是同时死光,却又是谁下手杀他们的。小鱼儿叹道:''我早就说过,此间必有绝大的秘密。``江玉郎喃喃道:''有趣的秘密。"

小鱼儿道:``很好,你终于学会了。''

这时,他们才发现这阴森森的屋子里,还有五张矮几,几上居然还放着些笔墨、书册。

小鱼儿笑道;``看来这屋子居然是个书房,有趣有趣。''他走过去,将矮几上的书册随意翻了翻,面色突然变了,江玉郎瞧了瞧他,也赶紧去翻另一张矮几上的书册。

瞧了两眼,他面包也变了。这些柔绢订成的书册上,记录的竟是最高深的武功小鱼儿和江玉郎的武功虽惧是名师传授,但此刻仍不禁瞧得冷汗直冒,只因他们忽然发现自己以前所学的功夫和这些武功比起来,简直一文不值,两人手里拿着这绢册,再也舍不得放下来。

良久良久,小鱼儿透了口气,道:``我知道了。这里本来必定有五位绝顶高手,他们五个人一起在这屋子里练武,有了心得,就赶紧在矮几上记录下来。''江玉郎道:``不错,高手练武的所在,屋子必定要特别大了。''小鱼儿道:五位高手,咱们巳瞧见死了叁个,若是我没有猜错,另外两间屋子里,必定还有另外两具尸身。``江玉郎道:''想来必定如此。"

小鱼儿道,``走,咱们瞧瞧去吧。''

江玉郎的眼睛这时才从书上抬起来,失声道:``走?\ldots\ldots 你说走?''小鱼儿道:``你突然听不懂我的话了么?''

江玉朗道,``但这些''\ldots\ldots 这些武功秘笈?\ldots。.``小鱼儿道;''放在这里,它们跑不了的。"

江玉朗垂头道:``好,你说怎样就怎样─\ldots.''突然自怀中取出了那金色的圆筒,狞笑道:``你可认识这是什么?''小鱼儿像是一惊,道:``天绝地灭透骨针。\ldots{}''江玉郎道:``不错,算你还有些眼力''"``我本想出去之后,才用这对付你的,但现在,我却再也容不得你!''小鱼儿道:``你杀了我,一个人留在这里不害怕么?''江玉郎大笑道:``此间这绝世的武功,绝世的宝藏,已全是我的了,我找着出路,立刻便成为天下第一人,我还怕什么?''小鱼儿叹了口气,道:好,既是如此,你杀吧。``江玉郎狞笑道:''你不怕?``小鱼儿突然大笑起来,笑道:''你这针筒是空的,我怕什么?``江王郎变色道:空的!''小鱼儿笑道:``你难道不想想,这针筒若不是空的,怎会被人抛在地上\ldots\ldots 这里面的透骨针早已被他用来将那人杀死了,他杀过人后才会随手将针筒一抛,如此简单的道理,你难道都想不到么?''江玉郎道:``你你''

小鱼儿道:``你方才假扮咳嗽,捡这针筒时,我早就瞧见了,若不是我早就知道这针筒是空的,怎会让你去捡。''他笑了笑,接道:``而且这天绝地灭透骨针,打造最是困难,昔年能制此针的,也不过只有神手匠一个人而已,如今他早已死了,这空的针筒,已是个废物。\ldots 哈哈,简直比废物都不如。''江玉郎满头冷汗,道:``我\ldots,我方才不是真的要\ldots 要杀你,只是''\ldots.``只听''当``的一声,他手里的针筒已落在地上。小鱼儿笑道:''我知道,你只不过是开玩笑的。``江玉郎道:''我始终将你视如兄长,此心可誓天日。"他说的竟像是诚恳已极,居然没有脸红。

小鱼儿笑眯眯瞧着他,道:``现在,你可以出去了么?''江玉郎道:``是。''垂首走了出去。小鱼儿大笑道:``江玉郎呀江玉郎,你真个是乖孩子!''

\hypertarget{ux7b2cux4e09ux5341ux4e09ux7ae0-ux5f53ux4ee3ux4ebaux6770}{%
\chapter{第三十三章
当代人杰}\label{ux7b2cux4e09ux5341ux4e09ux7ae0-ux5f53ux4ee3ux4ebaux6770}}

现在,小鱼儿已在搬动那锡制的绞盘。

小鱼儿道:``石屋子是坟墓,铁屋子练武,金屋于藏宝,铜屋子放兵器,这倒都很合理,这锡屋子里面是什么,你猜不猜得到?''江玉郎眨了眨眼睛,道:``莫非是卧房?''

小鱼儿大笑道:``在锡屋子睡觉,那真是活见鬼了。''那面锡墙已在移动,他话未说完,里面突然扑出来一条猛狮,几乎就扑到站在墙外的江玉郎身上。江玉郎吃了一惊,退出七八尺。

再看那狮子毛发虽存,但皮肉已不见,只剩了一副骨架,一副骇人的骨架,小鱼儿笑道;``这狮子想必是饿极了,一心想扑门而出,临死前还倒在门上,不想却害得咱们江公子又骇了一跳。''说到这里,他人已走了进去,突然失声道:``原来用意在此!''江玉郎跟过来,只见这间灰白色的屋子里,竟是五光十色,琳琅满目,骤然望去,又仿佛是另一宝藏。

仔细一看,才发觉这``宝藏''不过是许许多多颜色不同、大小各异的小瓶子,每一个瓶子的形式都诡异得很。

小鱼儿道:``你总该知道这些瓶子里是什么吧?''江玉郎深深吸了口气道:``毒药!''

小鱼儿道:``不错,他们豢养这头猛狮,正是为了看守这毒药的。''小鱼儿突然弯下了腰,道:``第四人的尸身果然在这里!''江玉郎瞧他只不过捡起了根骨头,想了想,不禁失色道:``他\ldots\ldots 他的尸身,莫非已饱了狮吻?''小鱼儿叹道:``这人也算是时运不济,不但被人害死在这里,尸身还喂了狮子\ldots\ldots{}''江玉郎突然咯咯笑了起来。

小鱼儿道:``什么事如此开心?''

江玉郎笑道:``你回头瞧瞧。''

他手里不知何时已多了黑黝黝的、像竹简般的东西,口中哈哈笑道:``我运气当真不错,居然能找到这宝贝。''小鱼儿眨了眨眼睛,道:这是什么?"

江玉郎道:``你若不认得此物,当真是孤陋寡闻,昔年滇边第一剑客绝尘道长,便是死在这东西手上。''小鱼儿笑道:``我还是不认得。''

江玉朗冷笑道:``告诉你,这就是昔年白水官的五毒天水,无论是谁身上,只要沾着一点,不出半个时辰,便要周身溃烂而死。''小鱼儿笑道:``如此说来,你可得拿远些,莫要溅着我。''江玉郎道:``这一次,你再也休想跑了,我方才已试过,此中满满的盛着一筒五毒天水,只要我手一动,你就完了。''小鱼儿苦笑道:``你难道非杀我不可?''

江玉郎道:``你方才若不多事,由得我把那些武功秘笈取走,我也许会容你多活些时,但现在你已非死不可了!''小鱼儿道:``你莫忘了,我本可杀你的,但却没有下手。''突又大笑道:``但你且先瞧瞧我手里是什么?''他手里拿着的,竟是方才江玉郎抛在地上的``天绝地灭透骨针''的针筒,江玉郎大笑道:"我看你已骇疯了,竟想拿这空筒子来吓人\ldots\ldots{}

小鱼儿笑嘻嘻道:``空筒子?谁说这是空筒子!''江玉郎怔了怔,道:``你\ldots 你自己方才\ldots{}''

小鱼儿笑道:``不错,我自己方才曾说是空筒子,但那不过是我骗你的,试想在那种时候,我不骗你骗谁?你可知道,这天绝地灭透骨针就因为制作费时,是以每个针筒里都有叁套透骨针。''他大笑接道:``这天绝地灭透骨针每筒如只能用一次,用完了又得找那神手匠,还有谁会将它看得那般珍贵,如此简单的道理,你难道都想不到?''江玉郎的手已开始颤抖,道:``你\ldots 你休想骗我,你根本不知道\ldots{}''``小鱼儿冷笑截口道:''我不知道,我自幼生长在恶人谷,对这种歹毒的暗器,知道得会没有你多?``江玉郎的手己软了,颤声笑道:''大哥自然是见多识广,小弟自愧不如。``话未说完,他已将手里的''五毒天水"放了回去。

小鱼儿笑嘻嘻瞧着他,悠悠道:``我若不杀伤,就是我活该倒霉,是么?''江玉郎道,``小小弟年幼无知,胡言乱语,大哥你\ldots\ldots 你想必能原谅的。''他一面说,身子已一面往后直退。

小鱼儿叹了口气,道:``你的确是个聪明人,知道的事的确不少,只可惜比我还差了一点!只差了那么一点点\ldots{}''他手指轻轻一按,手里针筒突然``喀''的一响。

江玉郎全身都软了,几乎吓得晕了过去,但针筒里什么也没有射出来。

小鱼儿已将那五毒天水拿在手里,哈哈笑说:``告诉你,这针筒其实是空的,天绝地灭透骨针一发便是─百叁十根,这小小的针筒理,哪里装得下叁套,如此简单的道理,你却想不到?''江玉郎呻吟─声,真的晕了过去,他自然不是被骇晕,只是被气晕了。

铜灯里油已快干了。

江玉郎乖乖地爬回那地洞,乖乖地加满了油,又带出些清水食物,乖乖地送到小鱼儿面前。等到小鱼儿吃完了,他才敢吃那剩下的,他爹爹此刻若是在旁边瞧见,只怕要气得直翻自眼。只因他对爹爹却从来没有如此孝顺过。

小鱼儿抹着嘴,喃喃道:``只剩下最后一间屋子没有瞧过了,出路,想必就在这屋子,嗯,不错,将出路设在卧房里,正是合理得很。''他终于转动了银绞盘。这银色的墙面后,竟是个奇妙的天地!

这里,才真正是地下的宫阙,萧咪咪那儿间屋子也算奢华的了,但和这里一比,简直像是土窑。

银墙后是条甬道,地上铺着厚厚的柔软的地毯,甬道两旁,有六扇门,门上接着珠,小鱼儿他们走在缤纷的光影里,就像是走人了七宝瑶池,走入了天上的仙境。

小鱼儿却根本瞧也不去瞧它,只是喃喃道:``奇怪,五个人.怎会有六间屋子,难道这里还有第六个人?\ldots。纵有第六个人,只怕也是不会武功的,否则那边又怎会只有五张矮几?''说话间他已走人了第一间屋子。

这屋子布置得竟像是文予闺房,对旁的梳妆台上,居然还放着整套的梳妆用具,床后面还有个马桶。

这一下,小鱼儿倒真是怔住了。他瞪大眼眼,失声道:``是女的?\ldots。这里的主人会是女的,打死我也不相信。''绣花的帐子,略略垂下来。

小鱼儿掀开帐子,床上直直的躺着具骷髅,发髻、环佩,还都完整的留在枕头上,自然是个女子。

第二间屋子,还是间女子的绣房,床上躺着的还是个女的,第叁间、第四间,全都是如此。

小鱼儿直是摇头,苦笑道:``原来这里非但不止五个人,也不止六个人,原来这些武林高手是带着老婆来的。他们被人害死,连老婆也被人害死了。''江玉郎道:``看来这些女子全都是被人点了穴道,然后才慢慢被饿死的。''小鱼儿叹道:``这种死法,大概是世上最不好受的死法了,下手的这人,心肠看来竟比你还毒,手段竟比你还狠。''江玉郎虽然垂下了头,脸却没有红。

他走入第五间屋子,又掀起了床帐,叹道:``人真是奇怪得狠,纵然明知这床上还是副女人骨头,还是忍不住要掀起帐子来瞧一瞧\ldots\ldots{}''他话未说完,就知道自己弄错了,这床上竟有两具尸身,一男一女,男人面朝下,脊椎竟已被打得粉碎,显然是一击之下,便已毙命了。

小鱼儿吐了口气,道:"这才是真正的第五个人\ldots\ldots{}

江玉朗道:``那第六间屋子,只怕就是他的\ldots\ldots{}''小鱼儿掀开了第六间房子的珠,他往屋子里只瞧了─眼,整个人突然被骇得呆在那里。

火光闪动下,一条头戴珠冠、满面虬髯的大汉迎门而坐.双手按在桌子上,竟似要作势扑起。骤眼望去只见他浓眉如戟,环目圆睁,满脸杀气,仔细一瞧,他眼鼻七窍之中,俱都流出了鲜血,只是血迹已干枯,是以瞧不清楚。

小鱼儿叹了口气道:``这人原来也死了。''

江玉郎摘下颗珠子抛过去,击在这虬髯大汉身上,只听``笃''的一声,珠子竟又被弹了回来。

这人的身子竟坚硬如石。

小鱼儿道:``这莫非只是个木偶!''

江玉郎道:``是人,死人。''

小鱼儿叹道:``说他是木偶,他的确像是个人,但说他是人,又怎会硬得像木头一样!''江玉郎一言不发,定过去掀起了帐子。

床上,果然也躺着一个人,女人,绝色的女人。她身子果然也完全如生,一点也没有腐坏,若不是脸色铁青得可怕,她实在可算是世上少见的美女``事实上,江玉郎简直一生中从未见过如此美丽的女子,她脸色纵然铁青,江玉朗纵然明知她是死人,但瞧过一眼后,仍不觉有些痴了小鱼儿叹道:''这女子活着的时候,想必不知要有多少男人被她迷死,萧咪咪和她比起来,简直是个丑八怪。我真不懂,她的尸身为何也\ldots{}``江玉郎沉声道:''这两人的死法和别人不同,他们是中了一种极奇怪的毒而死的,这种毒性竟可以使他们的尸身永不腐烂。``他叹了口气,缓缓接道:看来,她对自己的容貌极为珍惜\ldots\ldots 这原本也是值得珍惜的。''小鱼儿道:``你的意思是说她是自杀的?''

江玉郎道:``别人若要杀她,何苦去寻如此珍贵的毒药?''小鱼儿点头道:``这也有道理,只是\ldots\ldots 这男的又如何!瞧这男子死后数十年还有如此气概,生前想必是个好角色。''江玉郎道:``也许,他就是这里真正的主人。''小鱼儿道:``不错,他看来的确会有这么大的手笔。''江玉郎道:``若说那五个人都是被他杀死的,他自己又是如何死的!他的妻子又为何要自杀?他和那五人又是什么关系?他为何要花费这许多人力物力来造这地下的宫阙?他为何要藏得如此秘密?''小鱼儿苦笑道:``你这么一说,把我的头都说晕了。''两个人虽然都聪明绝顶,但还是打破头也猜不透这秘密,两个人的眼睛虽然都不小,但却谁也没有瞧见枕头边还有本绢册──他们若非瞧见这本绢册,就一辈子也休想猜得出这秘密。

幸好,小鱼儿终于瞧见了。

他翻了两页,突然大呼道:``在这里\ldots\ldots 所有的秘密全部在这里!''浅黄的绢册,秀丽的字迹,显然是女子的手笔。

这正是此刻躺在床上这绝色女子一生凄凉、悲惨、离奇、几乎令人难以相信的遭遇,她临死前揭开了这地底宫阙的全部秘密.自然,她不是写给小鱼儿看的,也不是写给任何人看的,她只不过临死前想将自己,心事倾诉倾诉而已。只是,她死的时候这里己没有活着的人。于是她只有将心事付于纸笔。

她说:她的名字叫方灵姬,她的家本是江南的望族,她们家四代同堂,日子本来过得幸福而平静。但她自己,并没有享受过这享福的日子。

她四岁的时候,她母亲带她到苏州去探亲,等她回去的时候,她们家占地百亩的庄院,已变为一片瓦砾。她们家大大小小叁百多口,已被人杀得干干净净。

仇人,自然要斩草除根,她和她母亲就开始天涯亡命。她虽然没有详详细细叙出这一段经历,但想必是充满了辛酸和艰苦。

在这段艰苦的日子里,她们终于查出了仇人的姓名!

欧阳亭。``当世人杰''欧阳亭!她的仇人竟是当日江湖中享誉最隆的侠士,武功最强的高手之一,家财亿万的富豪。

她母子孤苦伶仃,虽有些武功,但若想寻仇,实无异以卵击石,她母亲忧愤之下,终于一病不起。

叁年后,她竟设法嫁给了她的仇人。她只有用她绝世的美貌,作为她复仇的武器!

但欧阳亭一代人杰,毕竟不是容易被暗算的,她只有忍受着屈辱和愤恨,苦苦等候着复仇的良机。

不幸欧阳亭竟有个最可怕的习惯,他永不和任何人睡在一起,她和他虽是夫妻,竟也不知道他睡在哪里。

小鱼儿瞧了那虬髯珠冠的大汉一眼,道,``这小子想必就是欧阳亭了。''江玉郎叹道:``此人当真不傀为一代人杰,方灵姬虽然恨他入骨,但笔下写来的,字里行间,仍不禁流露出对他的佩服之意。''小鱼儿笑道;``只要假以时日,你就是第二个欧阳亭。''江玉郎不敢答话,转过话题,``奇怪的是,这欧阳亭在人世间既有名誉,又有地位,为何又要建造这地下宫阙?是什么事会让他宁愿过这种暗无天日的日子?''小鱼儿道:``你看下去不就可以知道了么?''

于是,他们接着看了下去!

她说:``欧阳亭为了建造这地下的宫阙,可说是费尽了心血,一年中总有叁个月的时间,他要摒绝一切,来此督工。''``然后,他不知用了什么手段,竟将当时武林中武功最高的五位高手骗到这里,他说服了他们,要他们创造出一套惊天动地,空前绝后的武功,他说,这武功留传后世,他们便可名留千古。''``千古留名''这句话,果然打动了这五大高手的心,他们合五人的智慧与经验,共同探寻武功中最深奥的秘密。

但他们却再也想不到,他们成功的日子,便是死的日子。

她这样写着:到了达``地灵宫''里,他终于不再独睡,只因他对我丝毫没有怀疑之心,他再也想不到我竟是他的仇人。我虽然有了下手的机会,却始终没有下手``我还要等。''

``他还有个野心,在武林的记载和江湖的传说中,古往今来,虽有不少称雄一时的英雄,但却从无一人的武功真的能横扫天下,他便要做这空前绝后、震古铄今的英雄!''``只可怜那被江湖人称为天地五绝的五位高手,显然要成为满足他野心的牺牲品,只因为这五人各有弱点,而抓住别人的弱点,正是他最擅长的,这五人也绝不会想到他的奸谋,只因欧阳亭的慷慨豪爽,天下知名。''``他早已有杀他们的计划,我虽不知道这计划究竟如何,但欧阳亭的毒计,从来都是天衣无缝的。我纵有揭穿他阴谋之心.但却抓不着他的证据,说了别人也不会相信,我怎敢轻举妄动。''``但我早巳准备好杀他的计划,只等他成功之日。''``现在,他成功的日子已抉到了,他眼看便要达到前无古人成功的巅峰。''``现在,在这里等着他的是一杯毒酒,我要和他共饮\ldots\ldots{}''小鱼儿眼睛像是有些湿了,突然将这本绢册远远抛出去,道:她为何要将这些起事写下来,让别人瞧见也难受,这岂非害人么\ldots\ldots 女人,活见鬼的女人!``江玉郎却像是痴了,喃喃道:''人类成功的巅峰\ldots。.生前绝后的英雄\ldots\ldots 唉,可惜呀,可惜!``小鱼儿瞧着欧阳亭的尸身,道:''他杀了天地五绝,正想和他的爱妻共饮一杯庆功之酒,哪知道这杯庆功的酒,却是杯毒酒\ldots\ldots 哈,有趣,有趣。``江玉郎叹道:''这方灵姬倒也是了不起的人物,只是,她既然报了她的血海探仇,为何要陪着她的仇人死呢?``小鱼儿长长伸了个懒腰,道:''我早就说过,女人的心事最难猜测,谁若花工夫去猜女人的心事,他不是呆子,就是疯子,唉\ldots 女人\ldots\ldots{}``江玉郎道:''但她还是不得不杀他,杀了他后,她心里又未尝不痛苦,她只有陪着他死,只因她已没法子一个人活下去。``他长叹一声,悠悠道;''方灵姬之与欧阳亭,岂非正如西施与吴王,唉,国家仇恨与深情厚爱,究竟孰重?只怕很少有人能分得清的。``小鱼儿瞧着他,突然笑道:''有时候我真奇怪,不知你究竟是男是女?``江玉郎怔了怔,失笑道:''你不知道我究竟是男是女?``小鱼儿道:''有时你心狠手辣,六亲不认,但有时你又会突然变得多愁善感。男人,是很少这样的,只有女人的心变化才会这么快,这么多。``他大笑着接道:''若不是我亲耳听见萧咪咪叫你小色鬼,我真要以为你是女扮男装的"。

\hypertarget{ux7b2cux4e09ux5341ux56dbux7ae0-ux76d6ux4e16ux6076ux8d4c}{%
\chapter{第三十四章
盖世恶赌}\label{ux7b2cux4e09ux5341ux56dbux7ae0-ux76d6ux4e16ux6076ux8d4c}}

突听一人娇笑道:``不错,我可以为他证明,他全身上下,每分每寸都是男人,绝没有半分假。''如此娇媚的语声,除了萧咪咪还有谁?

小鱼儿骨头都仿佛酥了,要想回身,只觉一个尖尖的、冰凉的东西低住了他的后脑勺子。

萧咪咪柔声道:``乖乖,不要动,不要回身。''她朝那已吓呆了的江玉朗招了招手,道:``玉郎,你也过来好么\ldots\ldots 嗯,这样才是乖孩子,现在,你也背转身,和他并排站着好么。''小鱼儿只希望江玉朗莫要太乖,只希望他稍为有些反抗,那么,小鱼儿就可以将怀里的``五毒天水''拿出来。

但这见鬼的江王郎却偏偏乖得狠,低着头,垂着手走过来。小鱼儿朝他直打眼色,他也瞧不见。小鱼儿恨得牙痒痒的,但也没法子,一个人若被一柄剑抵住了后脑,他纵有一万个法子也是使不出来的。

但他还没有灰心,他还在等机会,只要让他能取出那``天水'',甚或那针筒,萧咪咪可就完蛋了。萧咪咪没有完蛋,完蛋的是小鱼儿。

她突然伸过手来,将小鱼儿怀里的东西都摸去了,咯咯笑道:``哟,小鬼,看样你们真得了不少好东西,透骨针,五毒水,幸好我没有大意,否则可真惨了。''小鱼儿长长叹了口气,道:``现在我惨了。''

萧咪咪笑道:``还不算太惨,暂时我还不会杀你。''她突然将小鱼儿的右手和江玉郎的左手拉在一起,笑道:``你们是好朋友,先拉拉手\ldots。''小鱼儿只觉江玉郎的手冷冰冰,不停地发抖,满手都是冷汗,其实,他自己的手又何尝不是如此,只听``喀''的一声,两个人的手上,突然多了副手铐,又黑又重的手铐,将两人铐在一起。

萧咪咪银铃般娇笑着,终于走过来,走到他们面前,妩媚的眼波,笑咪咪地瞧着他们,柔声道;``现在,你们真可以算是好朋友了,活要活在一起,死也要死在一起,谁都别想抛下另一个人走。''小鱼儿苦笑道:现在,我倒宁愿他是女的了。``萧咪咪道:''我喜欢你,在这种时候还能说笑的人,世人并没有几个。``江玉郎道;''你\ldots{}``·你\ldots\ldots 你怎会来的?''

萧咪咪眼被一转,笑道:``你们奇怪么?''

小鱼儿叹道:``若不奇怪那才见鬼哩。''

萧咪咪道:``聪明的孩子,你们怎么也突然变得笨了,你想想,你们对我这么好,我怎舍得闷死你们?''小鱼儿道:``我还是不大明白\ldots.''

萧咪咪道:``那时,我虽然明知你躲在下面,但我还是不敢下去的,我根本不知道下面究竟是怎么回事,若是下来,不被你们弄死才怪。''她叹了口气,接道:``你们对我,决不会像我对你们这么客气的。''小鱼儿道:"你的确太客气了,所以你要闷死我们\ldots\ldots{}

萧咪咪娇笑道:``我想,这样也许未必真的能闷死你们,但最少也可以让你们不再防备着我,你们以为我既然要闷死你们,就绝对不会再下来瞧的了,是么?''小鱼儿叹道:``我现在才知道,一个人若没有被闷死,已是非常不幸,假如他再被女人喜欢上,那么他更是倒了穷霉了。''萧咪咪咯咯笑道:``这话真好笑,真要笑死我了!我下次一定要告诉别人,被人讨厌才不倒霉,被人闷死就是走运。''她像是根本不再去听小鱼儿的话,她的心开始完全贯注在这屋子里的东西上。

她将这里每间屋于都仔仔细细搜索了一遍,那种仔细的程度,就好像个妒忌的妻子搜查她丈夫的口袋一样。

然后,她的脸上发了光,眼睛也发了光。她终于找着了她所要找的,那是本淡黄绢册,自然也就是那五大高手心血的结晶。

她将这绢册捧在怀里,贴在脸上,亲了又亲,她吃吃地笑个不停,喃喃道:``心肝呀心肝,我有了你,还怕什么!今后天下武林第一高手是谁?你们可知道?\ldots\ldots 那就是我,萧姑娘。''江玉郎眼睛盯着她手里的绢册,几乎已冒出火。

萧咪咪摸了摸他的脑,咯咯笑道:``说起来,我还得感激你们,若不是你们,我怎会得到它?''烛轻盈地转了个身,看起来真的像是年轻了十几岁。

她接着笑道,``现在,你们领路,每个地方都带我去瞧瞧,那些东西想来都是上天赐给我的,我若客气,肚子会疼的。''其实,萧咪咪自己当真也未想到``上天赐给她''的东西竟会有这么多,她简直连眼睛都花了。

她将每间秘密都瞧了一遍,然后,便瞧着小鱼儿和江玉郎,她的眼睛看来是那么温柔,笑容看来是那么甜蜜。

她柔声笑道:``好孩子,你们可知道我为什么直到现在还没有杀你们?''小鱼儿眼睛却瞧着那面土门士墙,像是没有听见她的话,江玉朗脸色发白,根本已说不出话来。

萧咪咪道;``老实说,叫我一个人在这种鬼地方兜圈子,我实在也有点害怕,所以,我自然要留下你们陪着我。''江玉郎紧咬着嘴唇,脸色更白了。

萧咪咪瞧了小鱼儿一眼,笑道:``现在,你们的任务已完成了,你们两个已连成一个,要再从那地洞爬回去,看样子也困难得很,不如就留在这里吧。''江玉郎嘴唇已咬被了,眼泪已不停地往下流。

江玉郎突然跪了下去,颤声道:``求求你,莫要杀我,只要你放过我,我一辈子都做你的奴隶,无论你要我做什么都可以''\ldots\ldots{}``萧咪咪道:''抱歉得很,只有这件事,我不能答应你,除此之外,你们无论想要怎么样死法,我都可以答应的。``她又瞧了小鱼儿一眼,道:''小鱼儿,你听见了么?``小鱼儿眼眼仍在瞧着那土墙.茫然道:''嗯。``萧咪咪笑道:''有个最特别又最舒服的死法,我可以建议你们,不知你们愿意不愿意?``小鱼儿道:嗯。''萧咪咪道:``我咬死你们,好吗?''

她伸出纤纤玉手,摸着小鱼儿的喉咙,媚笑道:``我只要在这里轻轻咬一口就行了。''小鱼儿眼睛眨也不眨,道:``嗯。''

萧咪咪皱了皱眉,道;``那土墙有什么好看的,你究竟在想什么?''小鱼儿叹了口气,道:``我反正已要死了,想什么都没关系了。''``我倒想听听。''

小鱼儿道:``我看你还是赶紧杀了我算了,免得麻烦。''萧咪咪道:``你越不说,我越要听。''

小鱼儿又叹了口气,道:``你既然要听,我只好说,''他眼珠子一转,接着道:``我在想,既然每扇墙里面都有些古怪的东西,这面士墙后面就绝不可能是空的,但里面究竟是什么呢?''萧咪咪眼睛又亮了,道:``是呀,里面是什么呢?她眼珠子也开始四下转动,喃喃道:''只可惜这里没有土制的绞盘,这土墙不知要怎样才能开开。``小鱼儿眨着眼睛,道:''虽没有土制的绞盘,但上面却有个吊环还未拉过。

萧咪咪喜道:``呀,不错,你快去拉拉看,若不将这土墙开开看,我以后怎么睡得着呢?''小鱼儿满心不情愿地走过去,心里却欢喜得很,他其实也不知道这土墙里是什么东西,但想来必定不会是什么好东西,只是,此时此刻,无论什么东西,都已不可能令他的处境更坏了,他反正是一个死,土墙里面就算藏着妖魔鬼怪又有何妨!

上当的,只不过是萧咪咪。

那铜环吊得很高,拉起来很费力,小鱼儿拉了拉,铜环本来动也不动,但小鱼儿和江玉郎拼命一使力,铜环突然完全落了下来。

接着,只听``轰隆隆''一连串大震,就好像山崩地裂似的,整整一面土墙,突然问完全崩溃!

一股洪水,有如排山倒海一般倒灌了进来!

萧咪咪惊呼一声,面色惨变──她平时面色虽然千变万化,但这一次却变得和平时大不相同。

她就像一个看见老鼠的小丫头似的,拼命跳上了一架绞盘,怎奈那水势来得实在太快,晃眼间已将那绞盘淹没。

此刻她除了想赶紧逃走之外,别的什么都顾不得了,甚至连小鱼儿和江玉郎都可以放在一边,怎奈那唯一的一条逃路──那地道也被水灌了进去。

耍知这块地方和地道那边的出口``厕所''是平行的,所以地道中虽灌满了水,还是无法排泄。

小鱼儿和江玉郎此刻自然也泡在水里,江玉朗的水性竟然高明得很,踩着水就像踩在地上似的。

他瞧着萧咪咪的模样,脸上不禁露出恶毒的微笑,喃喃道:``这女妖怪居然不通水性,妙极妙极。''小鱼儿大笑道:``这就叫歪打正着。''

江玉郎突然回头瞧着他,道:``你会游水么?''小鱼儿的手吊在他手上,声色不动,笑道:``你难道忘了我叫什么名字,天下可有不会游水的鱼么?''他说得实在不像有半分假的,江玉郎瞪了他半响,终于展颜一笑,道:``很好.好极了。''水不停地往里灌,整个屋子都快被灌满了。

萧咪咪非但不会水,而且看来还十分怕水,她此刻简直慌了手脚,手脚乱动,越动越要往下沉。

江玉郎低声道;``她虽不会水,但若沉得住气,不要乱动,也不会往下沉的,何况,她还有一身武功,纵然沉下去,也不会喝着水。''他阴阴地笑了笑道:``但像现在这样,却是非喝水不可,两口水吞下去她就算有天大的本事,也完全没有用了。''那边萧咪咪果然已喝了两口水下去,忍不住嘶声道:``救命呀\ldots\ldots 你们难道真的眼看我死么?''江玉郎柔声道:``我们自然不忍瞧着你死的,只要你先将那秘笈抛过来,我就救你。''他现在自然还不敢过去,只因萧咪咪若是一把拉住他,他也掺了。

但那秘笈若是在水中泡久了,字迹也难免要模溯。

萧咪咪现在倒是真听话,立刻就将秘笈抛了过来,叫道,``快!快来救!''``咕嘟,'',又是一口水灌了进去。

江玉郎赶紧将秘笈接住,小鱼儿也不和他抢,因为他接书的手本和小鱼儿连在一起,他另一只手是把着灯的,只听他咯咯笑道:``傻孩子,你真以为我会救你么?''萧咪咪颤声呼道,``求求求你\ldots\ldots{}''

江玉郎大笑道:``我要在这里瞧着你一口口喝下去\ldots\ldots 等你死的时候,你肚子就会涨得像个球,那模样必好看得很。''萧咪咪大骂道:``你''\ldots 你这狗贼。"

萧咪咪挣扎着想扑过来,但越是挣扎,水喝得越多,不会水的人被泡在水里,那种恐惧和惊慌,若非尝过滋味的人,谁也想象不出。

江玉郎大笑道,``今后天下武林第一高手是谁?萧咪咪你可知道么?\ldots 告诉你,那就是我江大少爷。''小鱼儿冷冷道:``只怕未必。''

江玉郎赶紧接着道:``自然还有咱们的鱼兄。''小鱼儿叹了口气,道:``你我两人,谁也莫要做这梦了,现在唯一的出口已被水淹,你我除非真的有鱼那样好的水性,否则照样也得淹死在这里。''江玉郎怔了怔,立刻又变得面如土色,抓住小鱼儿的手,道:``你·\ldots。你快想想法子。''小鱼儿道:``我早巳想过了,金、银、铜、铁、锡,都是死路,那石头坟墓虽有门道向上面,但那门却是从外面开的。''江玉郎苦笑道:``坟墓的门自然是在外面开的,死人反正不会要出去\ldots。咬,该死,你我难道真的也要死在这里!''小鱼儿道:"也许,咱们还有一条路可走。

江玉郎大喜道:``什么路?''

小鱼儿道,"那木绞盘咱们还未动过\ldots\ldots{}

江玉郎喜色立刻又没有了,恨声道:``你难道忘了,咱们岂非就是从那木墙后面出来的。''小鱼儿悠悠道:``咱们是从下面钻上来的,上面呢?江玉朗大喜呼道:''不错,我为何没有想到!``小鱼儿笑嘻嘻道:''只因为我比你聪明得多。``江玉郎叹道:''此时此刻,还能想到这种事的人,除了你之外,实在不多了"只见萧咪咪头发漂在水上,已完全不动了。

江玉郎潜下了水,扭动了木绞盘,他手上本来一直举着灯的,但此刻一潜下水,四下立刻又是一片黑暗。

只听``吱''的一响大水忽然往外冲,小鱼儿和江玉郎身不由主,也随着水势被冲了出去,心胸突然一畅。

木墙外,赫然正是出口,数百级石阶直通上去,一线天光直照下来,江玉郎欢呼一声,眼泪不觉又往下直流。

石阶尽头,竟然有阳光照下,这的确也出人意外。

江玉郎满心欢喜,却又不禁奇怪,道:``这样的出口倒也奇怪,难道不怕被人发觉么,这里─切既是如此隐秘,出口本也该隐秘些才是。''小鱼儿笑道:``咱们从这里瞧着虽不隐秘,想来必定是隐秘的,若不隐秘,这许多年早该有人寻来了。''突然间,上面竟有语声传了下来。

两人不禁又是一惊,脚步更快、更快,一口气跑上去,只见那出口处盖着那个石板,两旁却留着半寸空隙。

天光,便是自这两条空隙中照下来的,语声也是从这两条空隙中传下来,两人又惊又奇,悄悄往外一瞧。

只见外面竟是个小小庙宇,但这庙宇里供的是什么神像,两人却瞧不见,只因那神像便在他们头顶的石板上,谁能想得到一个小庙的神橡下竟会有世上最神秘、最奇异、也最伟大的地底宫阙,谁能说这出口中不隐秘?

外面,自然有张神案,此刻神案上并没有香烛供札,却赫然有一双腿,这双腿黝黑如铁,上面还长满了黑茸茸的毛,裤管直卷到膝盖,泥脚上穿的是双草鞋,再往上面,他们便瞧不见了。

神案上还有个特别大的酒葫芦,两只半熏鸡.一大块牛肉,一串香肠,一堆豆腐干,一堆落花生。酒香,菜香,混合着那双脚上的臭气,随风一阵阵吹下来,小鱼儿闻了,当真不知是什么滋味他真想冲出去,但瞧见神案对面站着的五个人,却又不敢动了,非但不敢动,还几乎惊出声来,只见最左面站着的是个员外冠,福字履,肚子已渐渐开始膨胀的中年人,身上还接着只香袋。

他旁边一人,衣服也穿得不错,满脸精明强干的样子,但瞧那气概,却必定是那富商的跟班长随。

另外叁个人竟赫然是那``视人如鸡''王一抓,``天南剑客''孙天商,以及那银枪世家的邱清波邱七爷。

他叁人平日是何等飞扬跋扈,不可一世,但此刻一个个却是垂头丧气,满面俱是畏惧惊惶之色。

盘踞在神龛上的这位泥腿客,竟能使这叁人如此畏惧,小鱼儿委实想不出他是何等人物。

小鱼儿既不敢妄动,江玉朗更不敢动了。

只见一双毛茸茸的大手垂了下去,右手虽完完整整,左手却只剩下拇指与食指两根手指。

这双手撕下条鸡腿,用鸡腿向那富商一指,道:``你过来!''那富翁平日保养得法的一张脸,此刻已吓得面无人色,一步一挨,战战兢兢走了几步,颤声道:``小人张得旺叩见大王。''那洪钟般语声大笑道:格老子,老子明明晓得你龟儿子就是城里的土财主王陵川王百万,你龟儿还想骗老子。``他一句话里说了四句''老子``,两句''龟儿子``,正是标准的四川土话,只是说来有些含糊不清,想来因为嘴里正咬着鸡腿。那王百万已噗地跪倒,苦着脸道:''小人身上银子不多,情愿都献给大王,只要大王\ldots\ldots{}``语声大骂道:''放屁,哪个要抢你龟儿子的钱,老子听说你赌得此鬼还精,所以特地把你找来赌一赌的。``王百万喘了口气,陪笑道:''大王若要赌,无论骰子、脾九、马吊、花摊,小人都可奉陪,只是这里没有赌具,小人回城之后,一定准备得舒舒服服的和大王\ldots\ldots{}``那语声拍案道:''哪个和你龟儿子赌这些噜里噜嗦的东西,老子就和你赌猜铜板,是正是反,─翻两瞪眼。``王百万呐呐道:''却不知大王要赌什么,小人赌本带的不多。``那语声道:''老于赌你一只手,一条腿\ldots\ldots"

王百万刚站起来,腿又软了,噗地坐倒,咬牙道:``大王若输了呢?''那语声道:``老子若输了,就割一根手指给你。''王百万道:``这\ldots\ldots 这\ldots\ldots{}''

那语声怒道:``这个什么!老子一根手指,就比你四条腿都贵重得多!''王百万牙齿打战,道:``小人不\ldots·不想赌。''那语声道:``格老子,不赌不行。''

王百万像是也豁出去了,大声道:``世上只有强奸,哪有逼赌的?''那语声咯咯笑道:``老予平生别的坏事不做,就喜欢逼赌,你龟儿子好赌一辈子,今天叫你遇见我恶赌鬼算你走运。''王百万眼睛立刻圆了,失声道:``你\ldots\ldots 你是轩辕''那语声道:``老子就是轩辕叁光,你龟儿子也晓得?''王百万苦着脸道:``城里城外赌钱的人,都拿你来赌咒,谁要赌钱出郎中,就要他遇见轩辕叁光,但·.。但我赌时从未骗过人,老天怎地也让我遇见你。''轩辕叁光大笑道:``你既然知道老子,就该知道老子赌得最硬,从来不赖,你怕个锤子?''只见一个铜板在空中翻了无数个身,``国''的落在神案上,轩辕叁光的大手立刻将之盖住,大声道:``是正是反?猜!快!''小鱼儿也在那里直抽凉气,他实未想到这泥腿大汉,居然竟是``大十恶人''中的``恶赌鬼''轩辕叁光!

他最未想到刚从``十大恶人''手里逃脱,如今竟立刻又遇见\ldots 个,而且,看样子,他遇见的``十大恶人'',竟是一个比一个凶恶!但他方才却看见那制钱的是``通宝''一面朝上,他相信王百万必定也瞧见了,那么这``恶赌鬼''岂非必输无疑!

只见那王百万连嘴唇都白了,嘴张了好几次,还是说不出一个宇,轩辕叁光那只手背上青筋暴露,也像是有点紧张,厉声喝道:``快,再不说就算你输了。''王百万道:``通.通宝。''

轩辕叁光手一翻,大笑道:``龟儿子你输了。''王百万眼睛─闭,小鱼儿也吃了─惊。

他明明看见``通宝''在上,怎地变了,莫非是轩辕叁光故意要王百万看见是``通宝'',等他手盖下去时,就变了过来!

严格说来,这手法并不能算是骗人呀,谁叫王百万要偷看的?小鱼儿暗中叹了口气,苦笑讨道:``这恶赌鬼倒真是厉害!轩辕叁光笑道:''你输了,还不快切下一条腿、一只手来抵账。``王百万嘶声道:''小人\ldots。小人情愿将城里的十七家当铺都过户给你老人家\ldots\ldots 再加上城北那叁家米店,只求你老人家饶了小人这一次。``轩辕叁光咯咯笑道:''你这为富不仁的老畜牲,你以为老子真要你的那条猪腿么?老子虽然是恶人,但却最看不惯你专会在穷人头上打主竟!``他一拍桌子,大声道:当铺和米店老子都收下,快滚去将条子打好。等着老子去拿,反正老子也不怕你龟儿子赖账。王百万道:''是,是\ldots\ldots"屁滚尿流,连滚带爬地逃了。

他那边刚逃,这边他那跟班的已跪了下来,道:小人不过是个低叁下四的人,你老人家想必不屑和小人赌的,求你老人家就放了小人吧。``轩辕叁光大笑道:''你龟儿错了,你知不知道,老予还有个外号叫见人就赌,皇帝老子也跟他赌屁。``那跟班的狠了狠心,道:''你老人家要赌什么?``轩辕叁光道:''老子赌你不知道自己身上有多少个钮扣,你若输了,老子就割下你的鼻子,你若赢了,老于就把那十七家当铺、叁家米店都给你、"那跟班的面色如土,情不自禁用手拖住了鼻子。

轩辕叁光大笑道:``想想看,若凭你自己,一辈子也休想发这么大的财\ldots\ldots 呔,不准往身上看,否则老子就先挖出你的眼珠。''那跟班的果然只敢直勾勾地瞧着前面,道:``但那当铺和米店,现在还在王老爷手里。''轩辕叁光笑道:``你龟儿放心,只要你赢了,老于负责要他给你!那跟班的突然一笑,道:''小人从小有个毛病,专喜欢将扣子吞下肚,所以小人的娘替小人做衣服时,从来不用钮扣,都是用带子系着,长大了也成了习惯!"

\hypertarget{ux7b2cux4e09ux5341ux4e94ux7ae0-ux667aux5f97ux94dcux7b26}{%
\chapter{第三十五章
智得铜符}\label{ux7b2cux4e09ux5341ux4e94ux7ae0-ux667aux5f97ux94dcux7b26}}

那跟班站了起来,拍了拍自己衣裳,道:``所以小人从里到外,从头到脚,身上一粒扣子也没有。''轩辕叁光像是也怔住了,王一抓、邱清波等人看来也想笑,却又笑不出,小鱼儿若不是拼命忍住,早已笑破了肚子。

``这恶赌鬼也有上当的时候。''

轩辕叁光怔了半晌,突也大笑起来,道:``算你龟儿走运,回去等着当大老板吧!''那跟班的躬身一行礼,笑道:``小人叫王大立,日后你老人家进城时,千万莫忘了到小人店里去,小人自当略尽地主之谊。''他四面作了个揖,笑嘻嘻地走了!

轩辕叁光大笑道:``王大立,你这龟儿当真是从头到脚\ldots\ldots{}'',他转眼间赢了百万家财,转眼间又输出去,却像是全不在乎,反而笑得开心得很。

邱清波全身突然变得不自然起来,想必轩辕叁光的目光已转到他身上,他脸上也渐渐发白。

邱清波厉声道:``你若要赌,在下可以奉陪,否则\ldots。''轩辕叁光格格笑道:``不错,堂堂邱公子,自然是吃喝膘赌,样样精通,你要赌什么,花样不妨由你出,老子都奉陪,赌注可要由我!''邱清波笑道:``只望你赌注莫要下得太大,正如你所说,在下正是吃喝嫖赌,样样精通,你也未必赢得了。''轩辕叁光纵声笑道:``你龟儿就是在唬老子!老子从六岁就开始赌,天下无论哪种赌法,老子至少也要比你龟儿强些。''邱清波拎冷道:``无论哪种赌都有假,除了一种。''轩辕叁光道:``你说哪─种?''

邱清波道:``在下腰畔这绣囊中,有几锭紫金锭,你猜是单足双?''轩辕叁光又撕下条鸡腿,一面大嚼,一面道:``听说你的老婆本是苏州第一美人。''他只说了一句,邱清波脸色已变了,失声道:``你,你想怎样?''轩辕叁光道:``老子就赌你的老婆,你输了,就将老婆让给我,老子输了,也将老婆让给你,叁个老婆都让给你,让你占个便宜。''邱清波面如死灰,道:``你,你疯了\ldots\ldots{}''

轩辕叁光大笑道:``老子清醒得很!''邱清波厉声道:``不可以,万万不可以。''轩辕叁光道:``花样是你出的,你现在已非赌不可,反正老子也未必会赢的。''邱清波站在那里,全身颤抖,他若万一真的将老婆输了,以后他还有何面目击见亲戚朋友。

他出身世家,这个人他怎丢得起。

轩辕叁光悠悠道:``现在老子要赌了,你那里面的紫金锭子是\ldots\ldots{}''邱清波狂吼一声,道:``且慢!''轩辕叁光道:``还要等什么?''邱清波厉声道:``你怎可逼使每个人都非和你赌不可?''轩辕叁光笑道:``遇见恶赌鬼,不赌也得睹。''邱清波冷笑道;``但有─种人你却万万不能逼他和你赌的。''轩辕叁光道:``哦,有这种人?''

邱清波大喝道:``当然有。''

轩辕叁光道:``你且说说是哪一种人?''

邱清波道:``死人!''突然反手一掌,向自己``天灵''拍了下去。

世上竟有宁可自杀,不肯丢人的硬汉,这倒是出人意料──世家子弟的行为,有时的确是别人想不通,也想不到的。

轩辕叁光显然也吃了一惊,鸡腿也掉在桌上,他此刻自然只去瞧邱清波的尸身,绝不会去留意王一抓。但小鱼儿却瞧见王一抓与孙天南打了个眼色,也许是邱清被的死激发了他们的豪气。

两人突然飞身而起,向轩辕叁光扑了过来。

小鱼儿瞧得清楚,只见这两人身法既快,出手更狠,王一抓的─双手掌,几乎已完全变成死黑色。

他倒并没有打招呼,他们就是要轩辕叁光措手不及!

以小鱼儿看来,世上能躲得过他们两人全力这─击的人,只怕不多,简直可以说没有几个。

以江玉郎看来,轩辕叁光更是凶多吉少。

只听轩辕叁光怒喝一声,两只拳头飞了出去。

小鱼儿和江玉郎也瞧不清他用的是什么招式,只听得``砰、砰''两声,王一抓和孙天南便飞了出去。

他随手两拳,竟然就将两个武林高手击退,那么狠毒的招式,到了他面前,竟好像完全没有用了。

小鱼儿倒抽一口凉气,只见孙天南如断了线的风筝似的,直飞出窗外,远远跌了下去!

又见王一抓凌空─个翻身,飘落在地,居然拿桩站稳了,只是那张本已干枯的脸,此刻更难看而已。轩辕叁光大笑道:``好,你龟儿子果然有两下子。''王─抓道:``哼。''

轩辕叁光道:``现在你赌不赌?''

王一抓咬一咬牙,道,``赌!''

轩辕叁光道:``老子先赌那孙天南胸口十八根骨头都已断了,若有一根不断的,老子就算输了,输脑袋给你!''王一抓道:``嗯。''

轩辕叁光道:``老子再赌一拳巳打死了你,你若能不死,随便用你那双鬼爪子在老子喉咙抓几个洞都没关系。''王一抓默然了半晌,嘴角泛起一丝惨笑,道:``我输了!''他前面说的几个字,都是闭口音,此刻``了''字一出口,一曰鲜血随之喷出,人也扑地而倒!

江玉朗瞧得手脚冰冷,只见桌子上的两条泥腿,缓缓移了下去,接着,便现出了他的背。

他穿的是件破破烂烂的衣服,身子又高又大,一个肩膀似乎有别人两个那么宽,─个头也有别人两个那么大。

只听他喃喃道:``无趣无趣,老子本想不杀人,这些龟儿子偏要老子杀,老于一心想赌赌,这些龟儿子偏不陪老子赌。''他反手拿起那酒葫芦,拖着脚步走了出去,走到门口,长长伸了个懒腰,叹了口气,喃喃道:``这年头像王大立那样的赌鬼,怎地越来越少了\ldots\ldots{}''小鱼儿这才松口气,吐了吐舌头,道:``这赌鬼好厉害的武功。''江玉郎道:``咱们还不赶紧跑?''

小鱼儿笑道:``格老子,不跑是龟儿子。''

这两句话他竟已学会了──无论是谁,要学另一省的方言,那些骂人的话,总是学得最快的。

两人─搭一档,总算将上面的石扳抬起,一溜溜钻了出去,这才瞧见,供的神像是赵玄坛。

小鱼儿顺手抓起只鸡,边吃边笑道:``只可惜咱们没有瞧见那恶赌鬼的脸,不知道他长得是否和这位赵将军差不多,也许还黑一点吧。''江玉郎道:``求求你,快走吧。''

小鱼儿笑道:``你想追上那赌鬼么?''

江玉郎呆了呆,叹了口气。

小鱼儿道:``吃鸡呀,不吃白不吃。''

突然瞧见江玉郎的眼睛发直,他回过头,便终于瞧见了``见人就赌,恶赌鬼''轩辕叁光的脸。

只见他面如锡底,满胸兜腮大胡子,一双眉毛像是两极构刷,眼睛却像是一只铜铃,他眼睛已只剩下一只,左眼上罩着个黑印罩子,却更增加了他的彪悍、凶猛之气,也增加了几分神秘的魅力。

此刻,这一只铜铃似的眼睛正瞪着小鱼儿。

小鱼儿咧嘴笑了笑道:``这鸡的味道不错,只可惜没有酒。''轩辕叁光目光闪动,像是觉得很有趣,居然将那特别大的酒葫芦送到小鱼儿面前,嘻嘻一笑道:``这酒凶得很。''小鱼儿仰起脖子``咕嘟咕嘟'',一口气喝了十来口之多,伸手抹了抹嘴,居然面不改色,笑嘻嘻道:``这么淡的酒你还说凶?你当我是小孩子!''轩辕叁光笑道:``你这小鬼倒有趣,从哪里来的?''小鱼儿眨了眨眼睛,道:``哪里来的?自然是从窗子里爬进来的。''轩辕叁光道:``从窗子里爬进来偷人家的鸡,还敢理直气壮?''小鱼儿道:``死人可以从窗子里飞出去,活人为什么不能从窗子里爬进来?''轩辕叁光脸色一沉,道:``你早就来了?''

小鱼儿笑嘻嘻道:``不能来么?''

轩辕叁光瞪起眼睛,厉声道:``你小小年纪,到这荒山来作什么?''小鱼儿道:``做什么?找人赌一赌呀!''

轩辕叁光瞪着眼睛瞧了他半晌,哈哈大笑起来:``有趣有趣,实在有趣\ldots\ldots{}''一把将小鱼儿手里的酒葫芦抢了过来,``咕嘟咕嘟''灌了十来口下去。

小鱼儿双手从他手里将酒葫芦抢过来,也灌了十来口,笑道:``你莫小气,烟酒不分家,有酒大家喝。''轩辕叁光目光闪动,狞笑道:``你这小鬼居然不怕我?''小鱼儿也瞪起眼睛,龇牙笑道:``格老子,我既没有当铺输给你,也没有老婆输给你,最多也不过输个脑袋给你,我为什么要怕你?''轩辕叁光大笑道:``你竟敢和老子赌脑袋?''

小鱼儿:``为什么不敢,不过\ldots\ldots 你的脑袋我却不要,你脑袋我嫌太大了,口袋里放不下,提在手里又太重。''只听一人缓缓道:``这脑袋我要。''

轩辕叁光的狂笑声,就像是被人一刀砍断似的突然停顿,小鱼儿也不觉瞪大了眼睛,闭紧了嘴。

这语声虽然缓慢,虽然只说了五个字,但已显示出一种堂堂的气势,一种庄严的慑人之力。

轩辕叁光背对着门,此刻仍没有回头,只因他巳觉出有一般杀气袭人而来,若他一动,先机已失!

他只是缓缓道;``是谁敢要轩辕叁光的头颅?只要真的是英雄好汉,轩辕叁光又何惜将这大好头颅相送!''那人缓缓道:``轩辕叁光果然豪气如云,果然痛快!''一个乌簪高髻、白袜蓝袍的清□道人,随着语声,缓步走了进来,俯右手紧握着悬在左腰的剑柄,剑已出鞘四寸!

虽只出鞘四寸,但却有一般凌厉的剑气逼人眉睫!

轩辕叁光厉喝道:``来的可是峨嵋掌门?''

小鱼儿自然认得这蓝衫人便是神锡道长,但轩辕叁光连头也末回,却又怎会认出了他?

这恶赌鬼莫非连背后都长了眼睛不成!

神锡道长似乎也觉得有些奇怪,沉声道:``阁下怎知是贫道?''轩辕叁光纵身大笑道:``若非一门一派的宗主掌门,谁能有如此堂堂的剑气!''神锡道长缓缓道:``轩辕叁光,果然了得!''

轩辕叁光突然顿住了笑声,道:``只是,道长末入门,剑已出鞘,难道不怕失了你宗主掌门的身份?''神锡道长神色不变,道:``面对名露天下的轩辕叁光,贫道不能不分外小心''轩辕叁光道;``如此说来,道长是一心想要某家的脑袋了!''神锡道长沉声道;``此乃峨嵋圣地,杀人者死!''轩辕叁光狂笑道:``好一个杀人者死!道长莫非要某家为这几块废料偿命不成!''神锡道长道:``贫道并非为人报仇,只是护山之责,责无旁贷!''轩辕叁光厉声道:``很好,只是某家的头颅是在,道长却未必便能随意取去!''神锡道长道:``轩辕叁光先生一生好赌,也不知赢过多少人的大好头颅,此番纵然将头颅输给贫道,想来也不算什么!''轩辕叁光大笑道:``如此说来,道长莫非有意和某家赌一赌!''神锡道长道:``正是如此。''

小鱼儿瞧着神锡道长那已洗得发白的蓝袍,瞧着那瘦削的身子,瞧着他那紧握着剑柄的枯瘦的手指。

就这样一个人,竟使得轩辕叁光连身子都不敢转过来,这又是何等的气概,这又是何等的威风!小鱼儿暗叹忖道;``我就算是天下第一个聪明人,我就算比你聪明百倍,但我能令别人如此怕我么?看来,一个人还是应该好好练成武功,否则他一辈子也休想如此威风,一辈子也休想如此神气''这武林名家的风范,的确是令人羡慕,就算是他说出来的话,那人份量也和普通人绝不相同。

他``正是如此''这四个字说出来,轩辕叁光面上已再无笑容,抗声道:``仍不知要如何赌法?''神锡道长道:``你我但是武林中人,要赌,自然是赌一赌武功之高下!''轩辕叁光怪笑道;``动手拼命,也算是赌么?''神锡道长道;``以身体为赌具,以性命作赌注,世间之豪赌,还有什么能与此相比,这怎能不算是赌?''轩辕叁光历声道:``好,你以什么来换某家的头颅!''神锡道长道:``自然是贫道的头颅。''

轩辕叁光道:``不行,如此赌法,太便宜了你。''神锡道长冷冷道:``贫道自六岁出家,至今位居当代七大剑派之一峨嵋之掌门,门下叁代弟子,两千七百叁十二人,掌门铜符到处,不但本门子弟俯首听命,便是其他的门派,也得给贫道这个面子。''他声色俱厉,叱道:``这样的头颅,还抵不过你的?''轩辕叁光道;``你头倾虽好,只可惜某家要来无用,而你取了某家的头颅,不但维护了你峨嵋圣地的威风,又增长了你自家的声望!''他纵声大笑道:这样算来,某家岂非吃亏太大,这样的赌法,某家不赌!``神锡道长冷笑道:''阁下只怕已是不能不赌了。``轩辕叁光咯咯笑道:''这句话某家不知向别人说过多少次,不想今日竟有人来向我说,只是,你虽想要我的头领,我却想要你的,我难道不能一走了之``神锡道长道:''你走得了么?"

轩辕叁光道:``我走不了?''

神锡道长默然半晌,缓缓道:``你要怎样?''

轩辕叁光道;``除非你拿出一样能抵得过某家头颅之物,否则某家绝不和你赌。''神锡道长道;``普天之下,要有什么样的东西才能抵得过轩辕叁光的头颅?''轩辕叁光缓缓道;``这样的东西委实不多.但你身旁却有一物,勉强也可充数了。''神锡道长微微动容道:``那是什么?''

轩辕叁光厉声道:``那便是你的掌门铜符!''

神锡道长耸然道:``掌门铜符?''

轩辕叁光道:``不错,你胜了我,尽管割下我的头颅,我若胜了你,却留下你的性命,只是你的峨嵋掌位,要让我来过过瘾。''神锡道长面色沉重,缓缓道:``除此之外\ldots\ldots{}''轩辕叁光道;``除此之外,别无他途但某家却还可给你个便宜。''神锡道长道:``如何?''

轩辕叁光道:``某家就这样站在这里,让你砍叁剑,你叁剑若是伤了某家,某家自然就算输了,某家双脚若是离了地,移动了位置,也算输了。''小鱼儿再也想不到他竟会想出如此狂妄的赌法,他算来算去,这样的赌法委实连一分胜的希望都没有。

人站在那里,双脚也不能动,岂非和木头人差不多,神锡道长领导剑法以辛辣见长的峨嵋剑派垂叁十年,剑锋之下,飞鸟难渡。

他难道竟会连个木头人都砍不中?

小鱼儿暗暗笑道:``这恶赌鬼提出这样的赌法来,莫非是吃错药了。''但神锡道长面上还是声色不动,寻思半晌,道:``你还不还手?''轩辕叁光冷笑道:``自然不还手!''

到了这时,神锡道长纵然沉着.面上也不禁露出喜色,大声道,``好,贫道赌了!''轩辕叁光道:``你的铜符在哪里?''

神锡道长想了想,道:``铜符便在贫道腰畔,劳驾小施主取去给他瞧瞧。''他这话自然是对小鱼儿说的,要知道他此刻蓄势已久,正如箭在弦上,满弓待发,若是松开手去取铜符,气势便衰!

何况他捏着剑柄的手若是一松,轩辕叁光便要回过身来,那时情况难免又要有所变化!

他此刻脑中已有必胜之道,自然不愿情况有丝毫变更。

轩核叁光大笑道:``神锡道长,果然精明,但这小鬼却是顽皮得紧,你信得过他么?''神锡道长正色道:``这位小施主年纪虽轻,但来日必将为武林放一异彩,成就必定无人能及,又怎会将区区一面铜牌放在心上。''小鱼儿忍不住大笑道:``我为道长跑跑腿没有关系,道长不必如此捧我。''他嘴里虽然这么说,其实心里也不禁得意非常,当下从神锡道长后面绕过去,取下了他腰间的铜符。

神锡道长沉声道:``但望小施主小心保管。''

小鱼儿笑道:``道长放心,我也不必给他瞧了,反正这铜符绝不会是他的。''轩辕叁光大笑道:``受了别人几句话,立刻就咒我输么?''小鱼儿笑嘻嘻道:``你反正输定了,我咒不咒都一样。''轩辕叁光冷笑道:``看来,只怕你要失望了。''神锡道长叱道:``阁下可曾准备好了。''

轩辕叁光道:"你还未进门时,某家就已准备好了。

神锡道长道:``既是如此,贫道这就出手了!''这句话说出口来,四下突然再无声息,甚至连喘息的声音都没有,每个人唯一能听到的,便是自己心跳的声音。

``呛□''一声,神锡道长长剑出鞘,那森森的剑气,映得他须眉皆碧,映得远处木叶都仿佛有了杀机!

轩辕叁光却仍背着他,山岳般峙立不动。

神锡道长诚心正意,均匀的呼吸叁声,剑锋平平移动,突然间,剑光化为碧绿,一剑刺了出去!

这一剑正是刺向轩辕叁光两腰之间脊椎上的``命门穴'',也正是轩辕叁光全身的中枢所在!

轩辕叁光无论如何闪避,身子都必定要为之倾斜,神锡道长这一剑并非要求伤人,只不过要他身子失去均势。

那么,神锡滋长第二剑便可尽占先机!

小鱼儿暗叹付道:``名家的出手,气派果然不小,若是第一剑便想伤人,岂非显得太小家子气!''只见轩辕叁光熊腰一拧,霍然转过半个身予,腹部猛力收缩,这一剑便堪堪贴着他肚子刺了过来!

但这一剑含蕴不发,后力无穷。

神锡道长不持招式用老,手腕一扭,剑势已变刺为削,平平削向轩辕叁光的胸腹!

他招式变化之间,竟无空隙,小鱼儿瞧得不禁摇头,轩辕叁光只怕连这第二剑都已无法躲过了!

哪知轩辕叁光的腰竟似突然断了,他下半身好像生了根似的钉在地上,上半身却突然倒下。

他整个人就像是根甘蔗似的被拗成两半,神锡道长的第二剑便又贴着他的面目削过!

这一剑当真是避得险极!妙极!

小鱼儿几乎忍不住要拍起手来,谁能想到长得像巨无霸一般的轩辕叁光,竟然也有如此惊人的软功!

神锡道长徽微一笑,剑锋又一转,突然回旋削去,竞闪电般削向轩辕叁光左腿的膝头!

这一剑变化得更快,一眨眼工夫,叁剑都已使出,当真是一气呵成,神锡道长竟似早有成竹在胸,早巳将剑式计算好了,轩辕叁光这一挣、一拆,全都在他计算之中!

轩辕叁光第二剑躲得虽妙,却无异将自己驱人了死路,他此刻身子之变化,已至极限,已变无可变。

何况,他纵然勉强跃起避过这一刻,也还是输了──他已有言在先,只要双脚离地就算输!

小鱼儿暗道:``恶赌鬼呀恶赌鬼,看来你此番脑袋是输定了。''哪知他一念尚未转完,轩辕叁光那就像条毛巾拧续着的身子,突然松了回去,弹了回去。他本来脸朝上,此刻身子一转脸突然朝下,竟张开大嘴,一口咬在神锡道长握剑的手腕上!

神锡道长做梦也想不到他竟有这一着,手腕被咬,痛彻心骨,长剑再也把握不住,``当''的落在地上!

轩辕叁光大笑而起,道:``你输了!''

小鱼儿不禁瞧得怔了,神锡道长更是面如死灰,站在那里,直征了半盏茶工夫,吃吃道:``这\ldots\ldots 这算是什么招式,普天之下,无论哪一门、哪一派的武功中,只怕也都没有这样的招式。''轩辕叁光笑道:``招式是死的,人却是活的,活的人为什么定要用死招式?''神锡道长道:``但你说过绝不还手!''

轩辕叁光大笑道:``不错,我说过不还手,但却未说不还嘴呀!''神锡道长默然半晌,惨然一笑,道:``是,贫道是输了\ldots\ldots。''轩辕叁光摊开大手,笑道:``铜符拿来。''

小鱼儿淡谈道:``这铜符暂时还不算是你的。''轩辖叁光狞笑道:``你这小鬼想怎样''

小鱼儿笑道:``你不是见人就赌么,为何不和我睹一赌,你若赢了我,不但铜符是你的,我的人也是你的,你若输了,这铜符就该给我。''轩辕叁光怪笑道:``你也想赌?''

小鱼儿道:``嗯。''

轩辕叁光道:``你要以你的人来赌这个铜符?''小鱼儿道:``睹得过么''轩辕叁光道:``我赢了你又有何好处?''小鱼儿道:``好处多着哩!一时也数不尽,你无聊时,我可找人来陪你赌,你没有酒喝时,我可替你骗酒来,只要你赢了我,包你一生受用无穷。''轩辕叁光大笑道,``我这老赌鬼有个小赌鬼陪着.倒也的确不错。''小鱼儿道:``你赌了?''

轩辕叁光道:``你要如何赌法?''

小鱼儿笑嘻嘻道:``赌注是我出的,如何赌法,就该由你作主。''轩辕叁光抚掌道:``有意思有意思\ldots\ldots{}''

小鱼儿一只手摸着身上的扣子,笑道:``你可要赌我身上的扣子有多少?''轩辕叁光眼睛一亮,大声道:``好,我就赌你绝不会知道你身上的疤有多少!''江玉郎暗叹一声,忖道:``小鱼儿,这下你可要完了。''他心里虽然开心,又不免有些难受,无论如何,小鱼儿究竟是和他共过生死患难的朋友。

黯然站在一边的神锡道长,此刻神情更是黯然。

小鱼儿的衣襟是敞开的,他脸上是疤,身上更满都是疤,大多数是他小时狮子老虎在他身上留下的杰作,还有小半是刀疤,就算让他脱光衣服,自己去数一数,也未必就能数得清楚。

没有九分胜算的事,轩辕叁光是绝不赌的。

小鱼儿也怔住了,吃吃道:``你真的要赌我身上的疤?''轩辕叁光大笑道:``自然是真的。''

小鱼儿道:``好,我告诉你,我身上的疤一共有一百个。''轩辕叁光道:``整整一百个?''

小鱼儿道,``不错,整整一百个。''

他竟然说的斩钉截铁,像是有十分把握,不但轩辕叁光脸色变了,江玉郎也不禁怔在那里,这小妖怪难道真的知道自己身上的疤有多少?

轩辕叁光怔了半晌,怪笑道:``好,你脱下衣服,让我数数。''小鱼儿居然就真的脱光衣服,让他数,自己也从地上拾起那柄解腕尖刀陪他─起数。

轩辕叁光突然大笑道:``九十\ldots\ldots 你身上的疤只有九十一个,你输了!''小鱼儿道:``哦,九十一个么?只怕未必吧。''他口中说话,手里的刀飞快地在自己身上划了九刀!划得虽然不重,但鲜血仍然流了一身。

轩辕叁光奇道;``这算什么?''

小鱼儿面不改色,道:``这就算你输了。''

轩辕叁光喝道:``放屁,你\ldots\ldots{}''

小鱼儿笑嘻嘻截口道:``九十一道旧疤,再加上九道新疤,正好是一百,你自然输了!''轩辕叁光大怒道:``这也能算么!''

小鱼儿大笑道:``为何不能算?你只赌我身上的疤有多少,却又未曾规定新疤还是旧疤,难道你还想赖么?''

轩辕叁光呆了半晌,突也大笑道;``有意思有意思,你这小鬼的确有意思\ldots\ldots 好,某家就算输给你了。''他转向神锡道长招手笑道:``来来来,还不快来见过你家的新任掌门。''神锡道长神情惨黯,却强笑道:``峨嵋派日渐老衰,正是要阁下这样的少年英雄出来整顿盛顿,贫道已老了.本已早该退位让贤。''小鱼儿笑道:``你真要我做峨嵋掌门?''

神锡道长长髯在风中不住飘动,缓缓道:``铜符能在阁下手中,已是峨嵋之幸,贫道\ldots\ldots{}''话未说完,突然一件东西落在手里,却正是那掌门铜符,小鱼儿的一双眼睛,正笑嘻嘻地瞧着他,道:``做了峨嵋掌门,又要吃素,又要念经,我可受不了,求求你,莫要害我,这玩意儿还是你拿回去吧。''神锡道长又惊又喜,呐呐道:``但,但阁下\ldots\ldots 阁下如此大恩,却教贫道\ldots\ldots 如何\ldots\ldots{}''小鱼儿大笑道:``这又算得了什么?我前程远大,又岂会将这区区铜牌瞧在眼里,这话本是你自己说的,是么?''神锡道长手掌握着那铜符,目注小鱼儿,也不知瞧了多久,突然深深一揖,恭身合十道:``既然如此,贫道就此别过。''

\hypertarget{ux7b2cux4e09ux5341ux516dux7ae0-ux8c8cux5408ux795eux79bb}{%
\chapter{第三十六章
貌合神离}\label{ux7b2cux4e09ux5341ux516dux7ae0-ux8c8cux5408ux795eux79bb}}

他转过身子,竟头也不回的去了。轩辕叁光笑骂道:``这牛鼻子好没良心,居然连谢都不谢你一声。''小鱼儿道:``大恩不言谢,这话你都不知道。''他一面说话,一面撕下块衣襟,去缠肩上的新伤,只是一只手仍和江玉郎铐在一起,行动自然不便。轩辕叁光奇道:``你两人为何如此亲热\ldots。.''小鱼儿笑道:``你若能叫我们不亲热,就算你有本事。''轩辕叁光又拾起那柄刀,突然一刀,向那手铐上砍了下去,只听``铮''的一声,火星四溅,尖刀竟断成两段!

江玉郎叹了口气,小鱼儿笑道:``你瞧,我和他是不是非亲热不可?''轩辕叁光笑道;``那也未必,你若不愿和他亲热,某家不妨砍下他一只手来。''江玉郎面色惨变,小鱼儿已笑道:``纵然砍下他的手,这鬼玩意儿还是在我手上,倒不如留他在我身旁,还可陪我聊聊天\ldots\ldots 轩辕叁光瞧着江玉郎的眼睛.缓缓道:''你若不砍下他的手,只怕总有一日他要砍掉你的!``小鱼儿道:''你放心,他还没有这么大本事.``轩辕叁光大笑道:''你这小鬼很有意思,某家本也想和你多聚聚,只是你身旁这小子一脸奸诈,某家瞧着就讨厌\ldots\ldots{}``他拍了拍小鱼儿的肩头,人忽然已到了门外,挥手笑道:''来日等你一个人时,某家自来寻你痛饮一场。"小鱼儿赶出去,他人竟已不见了,这时夕阳正艳,满山风影如画,小鱼儿想起那地底宫阙,竟如做梦一般。

由这玄坛庙"下山的路并不甚远,两人一口气走了下去,天还没有十分黑,远处山城,灯火数点。

小鱼儿长长松了口气,笑道:想不到我居然还能整个人走下山来,老天待我总算不错.江玉郎一直没有说话,此刻忽然笑道:``不知大哥要往哪里去?''小鱼儿道:``我要去的地方,你也得去。''

江玉郎笑道:``小弟自然追随兄长。''

小鱼儿道:``其实,我也没有什么固定购地方要去,只不过到处逛逛。''江玉郎喜道:``既然到处逛逛,不如先去武汉,那边小弟有个朋友,家传宝剑,削铁如泥\ldots\ldots{}''说到这里,他微微一笑,颤住语声,他知道已用不着再说下去!

小鱼儿果然已大声道:``走,咱们就去找你那朋友。''他走了几步,突又停下,笑道:你身上可带得有银子,咱们总得先到镇上去买几件衣服..\ldots 还得买件衣服搭在手上,否则不被别人看成逃犯才怪.。

江玉郎叹道:``大哥若让小弟自那库中取些珠宝,只要一件珠宝,买来的衣服只怕已够咱们穿一辈子了.。''小鱼儿眨了眨眼晴,笑道:``既然你也没有,看来咱们只好去骗些来了。''话刚说完,突见前面一个人提着灯笼走来,手里提着个大包袱。

小鱼儿和江玉郎使了个眼色,正想走过去.哪知这人瞧见他们,突然放下包袱,远远作了个揖,也不说话,转身就走。

那包袱里竟是四套崭新的衣服,而且好像照着小鱼儿和江玉郎的身材定做的,两人打开包袱都不免吃了一惊。

江玉郎道:``这\ldots。.这是谁送来的?''

小鱼儿皱眉道:``咱们刚下山,有谁会知道?''两人想来想去,也猜不透是谁,只有先换上衣服,这时那山城中已是万家灯火,两人将一件紫缎袍子搭在手上,大摇大摆地走上大街,样子看来倒也神气,肚子却已饿得``咕咕''直叫。

小鱼儿道:``那人既然送了衣服来,为何不好人做到底,再送些银子。''话犹未了,突见一个店家打扮的汉子奔了过来,陪笑道;``两位可是江少爷?方才有位客官寄了五百两银子在柜上,叫小人交给两位,还替两位订好了房间和酒菜。''小鱼儿和江玉郎对望了一服,江玉朗沉声道:``那人性什么?叫什么''店家笑道,``小人也不知道。''

江玉郎道:``他长得是何摸样?''

店家道:``小店里一天人来人往也有不少,那位客官是何模样,小人也记不清了。''他连连作揖,连连陪笑,但无论江玉郎问他什么,他只有叁个宇:``不知道。''洒菜果然早巳备好,而且丰盛得很。

小鱼儿笑道,``这人倒是咱们肚子里的蛔虫,无始咱们要什么,他居然都知道''他嘴里说得虽开心,心里却不免有些担忧,尤其他想到自己和那``黄牛白羊''来的时候,一路上的情况岂非饱和此刻差不多,而自己此刻刚下山还不到一个时辰,怎地就有人知道?此人表面如此殷勤,暗中却不知在打什么鬼主意,他若真的全属好意,又为何不敢露面。

江玉郎眼珠子直转,显然心里也在暗暗狐疑,只是这两人年纪虽轻,城府却深,谁也不肯将心事说出来。

到了晚间,两人自然非睡在一间房里不可。

小鱼儿打了个哈欠,笑道:``你知道我现在最想干什么?''江玉郎笑道:``大哥莫非是想看看书。''

小鱼儿大笑道:``看来你倒真是我的知己!''

他话未说完,江玉郎已将那本从萧主咪手里夺回来的秘笈自怀中取出,小鱼儿想看,他又何尝不想看。

秘笈上所载,自然俱是武功中最最深奥的道理,两人好像都看不懂,一面摇头一面叹气,但眼睛却又都睁得大大的,像是恨不得一口就将这本秘笈吞下肚里,小鱼儿瞧了一个时辰,又打了个哈欠,笑道:``这书难看得很,我要睡了,你呢?''江玉朗也打了个呵欠,笑道:``小弟早就想睡了。''两人睡在床上,睡了一个时辰,眼睛仍是瞪得大大的,也不知在想些什么,若说他们在想那秘笈上所载的武功,他们是死也不会承认的,但到了第二天晚上,刚吃过晚饭,小鱼儿就喃喃笑道:``难看的书,总比没有书看好。''江玉郎立刻也笑道,``眼睛看累了正好睡觉,若是看精采的书,反倒睡不着了。小鱼儿附掌道:''是极是极,早看早睡,早睡早起,真是再好也没有。"其实两人心里都知道对方绝不会相信自己,但却还是装作一本正经。

尤其是小鱼儿,他更觉得这样不但有趣,而且刺激──一个人若是随时随地,甚至连吃饭大便睡觉的时候都要避防着别人害他、骗他,这种日子自然过得既紧张,又有趣,固然过得充满了刺激。

两人就这样勾心斗角,竟不知不觉走了叁天,这叁天居然没有发生什么事,居然太平得很。这叁天里,小鱼儿却时时刻刻觉得有个人在跟踪着他,那种感觉就好像小孩儿半夜走路时,都觉得后面有鬼跟着似的,只要他回头,后面就没有人了,他若倒退着走,那人忽然还是又到了他身后。

小鱼儿猜不透这人是谁,更猜不透这人是何用意,反正只要他觉得缺少什么,立刻就有人送来。

他觉得这人好像是有求于他,在拍他的马屁,但这人究竟有什么事要求他,他还是想不透。

两人沿着岷江南下,这一日到了叙州,川中民丰物阜,景象自然又和贫瘠的西北一带不同。

小鱼儿望着滚滚江流,更是兴高采烈,笑道:``咱们坐船走一段如何?''江玉郎附掌道:``妙极妙极,小弟也正想坐船。''只见一艘崭新的乌篷船驶了过来,两人正待呼唤,船上一个蓑衣笠帽的艄公已招手唤道:``两位可是江少爷?有位客官已为两位将这船包下了。''小鱼儿瞧了江玉郎一眼,苦笑道:``这人不是我肚里的蛔虫才怪。''他索性也不再问这船是谁包下的,只因他知道反正是问不出来的,索性不管叁七二十一,坐上去再说。

船舱里居然窗明几净,除了那白发艄翁外,船上只有个十五六岁的小姑娘,一双大眼睛老是往小鱼儿身上瞟。但小鱼儿却懒得去瞧她,他简直─瞧见漂亮的女人就头疼。到了晚上,江玉郎悄声笑道:``那位史姑娘像是看上大哥了。''小鱼儿打了个哈欠,懒洋洋道:``你长得比我俊,她看上你才是真的,只可惜你非得跟走我不可,否则你这小色鬼倒可去勾搭勾搭。''江玉郎脸红了红,道:``小\ldots。小弟没这个意思。''小鱼儿笑道:``算了,你若没有这意思,怎会提起她,又怎会知道她名姓。''江玉郎脸更红了,吃吃道:``小弟只不过偶然听到的。''小鱼儿大笑道:``你害什么臊,喜欢个女孩子,又不是什么丢人的事。''拿起只枕头盖住眼睛,竟似要睡了。

江玉郎道:``大哥,你不看书了么?''

小鱼儿道:``今天我睡得着,不用看了,你呢?''江玉朗赶紧笑道:``大哥不看,小弟自然也不看。''两人并头睡在一床铺盖上,江玉郎睁大了眼睛瞪着小鱼儿,也不知道了多久,小鱼儿鼻息沉沉,已睡着了。

江玉郎悄悄将那秘笈掏了出来,轻手轻脚,翻了几页,正想看的时候,小鱼儿突然翻了个身,一只手压到书上,一条腿却压到江玉郎肚子上,江玉郎恨得直咬牙,却又不敢吵醒他,只望他再翻个身,将手拿开。

哪知小鱼儿这回却睡得跟死猪似的,再也不动。

江玉郎气得脸发白,眼睛里冒出了火,一只手摸摸索索,突然自被褥下摸出柄菜刀,一刀往小鱼儿头上砍下!

就在这时,只听``嗖嗖''两声,接着,``当''的一响,两粒干莲子自窗外飞了进来,一粒打中菜刀,一粒打中江玉郎的手腕,无论力气、准头,都有两下子,竟像暗器高手发出来的!

江玉郎手却被打歪了,咬紧牙,忍住疼,菜刀虽没有离手,但头上却已不禁疼出了汗殊。小鱼儿像是半睡半醒,咿晤着道:``什么事,谁在敲钟?''江玉郎赶紧又将菜刀藏起来,道:``没''\ldots·没有事。"幸好小鱼儿不再问了,鼻息更沉。

但江玉郎又怎能再睡得着觉?

这两粒莲子是谁打进来的?

达船上怎会有这样的暗器高手?

那咳起嗽来、眼泪鼻涕就要一齐流下的白发艄翁,莫非也会是什么隐迹风尘的武林异人?

那一天到晚只会乱飞媚眼的小姑娘,莫非也有如此高明的身手?竟能以两粒轻飘飘的莲子当做暗器?

这简直使江玉郎无法相信!

但不是他们,又是谁?这船上并没有别的人呀!

何况,就算是他们,他们又为何要在暗中监视?为何要在暗中保护小鱼儿?看他们和小鱼儿根本素不相识。

江玉郎就这样瞪大了眼睛,望着船顶,一夜想到了天亮,还是想不通这其中究竟是何道理。

他刚想睡的时候,小鱼儿已醒了,又推醒了他,笑道:``你睡得好么?''江玉郎强笑道:``好极了,一觉睡到大天亮。''小鱼儿道:``起来吧,睡得太多不好的。''

江玉郎道:``是,是,该起来了。''

他脸上虽在笑.心里却恨不得一拳打过去,到了船头,两眼见小鱼儿精神抖擞的模样,更恨不得─脚将他踢下河里。

那小姑娘已端了盆洗脸水过来,脸上在笑,眼睛在笑,那两只深深的酒窝也在笑──她在笑什么?

江玉郎眼睛盯着这两只端着盆的手,只见这双手又白又嫩,实在不像能发出那般强劲的暗器!

但一个终年劳苦的船家女儿,又怎会有这么一双白嫩的手?这祖孙两人,莫非真的是乔装改扮的!

船是新的,他们的衣裳也是很新,看来,他们扮这船家勾当,还没有多久,也许就是冲小鱼儿才改扮的。

但他们这样做又有何用意?

小鱼儿像是什么都不知道,像是开心得很,洗完了脸,一口气竟喝了四大碗稀饭,外加四只荷包蛋。

江玉郎却什么也吃不下去,只听小鱼儿向那艄翁笑道;``老丈,你贵姓大名呀''那艄翁道:``老汉姓史\ldots\ldots 咳咳,人家都叫我史老头\ldots\ldots 咳咳,我那孙女倒有个名字\ldots\ldots{}''咳咳,她叫史蜀云。"江玉郎暗中苦笑,这每说一句话就要咳嗽两声的糟老头子,也会是个风尘异人、武林高手?

只听那史老头道:``云姑,莫要吃莲子了,吃多了莲子,心会苦的。''江玉郎又是一惊,扭转头,云姑那双又白又嫩的小手里,果然正抓着把莲子,一面吃,一面瞧着他笑。

他的心突然``砰砰''跳了起来,扭回头,又瞧见小鱼儿手里正拿着本书在当扇子,赫然正是那秘笈。

江玉郎这才想起,小鱼儿昨夜是压在上面的,今晨翻了个身,竟乘机将这秘笈拿走了。

他居然将这本天下武林中人``辗转反侧,求之不得''的武功秘笈当作扇子,江玉郎又是气又是着急。

船已驶离渡头,突然一只船迎面过来,史老头用根长长的竹篙,向对面的船头一点,两船交错而过,两只船都斜了一斜!

小鱼儿惊呼一声,道:"哎呀,不好,掉下去了!他手中的那本秘笈竟落在江中,江玉朗的一颗心也几乎掉了下去,只见江水滚滚,眨眼就将秘笈冲得不见了。

小鱼儿苦着脸,顿脚道:``这\ldots\ldots 这怎么办呢?''江玉郎心里恨得流血,面上却笑道:``这些身外之物,掉下去又有何妨。''他心里自然知道这必定是小鱼儿故意掉下去的,小鱼儿想必已背熟了,小鱼儿自然也知道他心里明白。

但两人谁都不说,这就是最有趣之处,除了他两人自己之外,天下只怕再无人能猜得出他两人的心意。

苍穹湛蓝,江水金黄,长江两岸,风物如画。

小鱼儿笑道:``船慢慢走没关系,咱们反正不着急。江玉郎道:''是是,一点也不着急。"

突然间,一艘快船自后面赶了上来,船头插着面镖旗,迎风招展,紫缎金花,绣着的是个狮子。

江玉郎面上立刻露出喜色,眼睛也亮了,突然站起来,大呼道:``金狮镖局是哪一位镖头在船上?''快船立刻慢了下来,船上精赤着上身的大汉们,显然都是行船的高手,船舱中探出了半个身子,大声道:``是哪一位呼唤\ldots\ldots{}''江玉郎招手道:``我,江玉郎,李大叔你还记得么?''船舱中那人紫面短髭.神情甚是沉猛,但瞧见了江玉郎,严肃的面上立刻堆满了笑容,失声道:``呀,这莫非是江大侠的公子,你怎地在这里?''史老头像是什么都没瞧见,仍在驶他的船,但金狮镖局的快船却荡了过来,那紫面大汉竟一跃而过。

小鱼儿轻笑道:``这位仁兄的轻身功夫,看来还得练练。''他说话的声音不大,紫面大汉并末听见,含笑走了过来。

江玉郎笑道:``这位便是江南金狮镖局的大镖头,江湖人称紫面狮李挺,硬功水性,江南可称第一。''他这句话自然是回答小鱼儿``轻功不佳''那句话的,小鱼儿却故意装作没有听见,转头喝茶去了。

只听江玉郎与那李挺大声寒喧了几句,说话的声音突然小了,像是耳语一般,竟像是不愿被小鱼儿听见。

小鱼儿也懒得去听,他就算明知江玉郎要对他不利,他也不想阻拦,他正想瞧瞧江玉郎玩得出什么花样。

自从他叁岁开始,他就没有怕过任何人、任何事,他简直不知道``害怕''是何物,越是危险他越觉得有趣。

到后来,只听那``紫面狮''李挺道:``过了云汉,我便要弃舟登陆,但公子你交托的事,李某决不会耽误的.公于放心就是。''两人又大声说笑了几句,李挺便又一跃面回。

小鱼儿笑道:``小心些呀,莫掉下水里去。''

李挺回头狠狠瞪了他一眼,嘴里像是在说什么:"你该小心些才是\ldots,。但话未说完,两只船又分开了。

江玉郎精神突然像是好起来了,笑道:``江南金狮镖局,除了总镖头金狮子李迪之外,旗下双狮一虎,当真也都可算得上是肝胆相照的义气朋友。''史老头喃喃道:``说什么狮虎成群,也不过是狐群狗党而已。''这句话小鱼儿听见,江玉郎也听见了。但两人却又都像是没有听见。

\hypertarget{ux7b2cux4e09ux5341ux4e03ux7ae0-ux60caux9669ux91cdux91cd}{%
\chapter{第三十七章
惊险重重}\label{ux7b2cux4e09ux5341ux4e03ux7ae0-ux60caux9669ux91cdux91cd}}

船走得果然很慢,小鱼儿一路不住的问:``这是什么地方?这里到了什么地方?''过了云汉,小鱼儿眼睛更大了,像是在等着瞧有什么趣事发生似的,船到奖州,却早早便歇下。

小鱼儿笑道:``现在睡觉,不嫌太早了么?''

史老头``哼''了一声,没有说话。

那云姑却眨着眼睛笑道,``前面便是巫峡,到了晚上,谁也无法渡过,是以咱们今天及早歇下,明天一早好有神精闯过去。''小鱼儿笑道:``呀,前面就是险绝天下的巫山十二蜂了么?我小时听得两岸猿声啼不住,轻舟已过万重山这两句诗,一心就想到那地方去瞧瞧。''云姑娇笑道:``这两句诗虽美,那地方却一点也不美,稍为不小心,就会把命丢在那里,尤其是现在,只怕连两岸的猿猴都叫不出声来了。''小鱼儿奇道:``为什么?''

云姑笑了笑,轻声道,``有些事,你还是莫要问得太清楚的好。''小鱼儿转头去瞧江玉郎,只见江玉郎正垂头在望江水,像是没有听见他们的话,但脸色都已是铁青的了。到了第二天,他脸色更青。小鱼儿知道他心里一紧张.脸色就会发青。

但他却在紧张什么?难道他也算定有事受发生么?

史老头长篙一点,船驶了出去,云姑换了─身青布的短衫裤,扎起了裤脚,更显得她身材苗条。

小龟儿笑嘻嘻地瞧着,也不说话,到了前面,江流渐急,但江面上船只却突然多了起来。

小鱼儿突然发现他们每艘船的船桅上,都接着条黄绸,船上的人瞧见小鱼儿这艘船来了,都缩回了头。

史老头白须飘拂,一心掌舵,像是什么都没有瞧见,云姑两只大眼睛转来转去,却像是高兴得很。

江玉郎却根本不让小鱼儿瞧见他的脸。

突然间,岸上有人吹响了海螺,晌彻四山。

四山回响,急流拍岸,十余艘瓜皮快船,突然自两旁涌了出来,每艘快艇上都有六七个黄巾包头的大汉,有的手持鬼头刀,有的高举红缨枪,有的拿着长长的竹竿,呼啸着直冲了过来!

云始娇呼道:``爷爷,他们果然来了。''

史老头面不改色,淡淡道:``我早知他们会来的。''他神情居然如此镇定,小鱼儿不禁暗暗佩服。

只听快艇上的大汉呼啸着道:``船上的小子们.纳命来吧!''只见两艘小艇已直冲过来,艇上大汉高举刀枪。

云姑突然轻笑道:"不要凶,请你吃莲子\ldots\ldots{}

她的手一扬,当先两条大汉,立刻狂吼一声,撤手抛去刀枪,以手拖面,鲜血泪然自指缝间流出。

大汉们立刻大呼道:``伙伴们小心了,这始娘暗器厉害!''云妨娇笑道:"你还耍吃莲子么?好,就给你一粒。她那双又白又嫩的小手连扬,手里的莲子雨点般澈出去,但却不是干莲子,而是铁莲子。

只见那些大汉们一个个惊呼不绝,有的立刻血流满面,有的兵刃脱手,但还是有大半人冲了上来!

声色不动的史老头到了此刻,突然仰天清啸,啸声清朗高绝,如龙吟风鸣,震得人耳鼓欲裂!

啸声中,他掌中长竿一振,如横扫雷霆,当先冲上来的叁人,竟被他这一竿扫得飞了出去,远远撞上山石,另一人刚要跃上船头,史老头长竿一送,竟从他肚子里直穿过去,惨呼声中,长竿挑起那鲜血淋漓的尸身,数十条大汉哪里还有一人敢冲上来!

这老迈衰病的史老头,竟有如此神威,不但小鱼儿吃了一惊,江玉郎更是惶然失色,满头冷汗。

史老头清啸不绝,江船己冲入快艇群中,那些大汉们鼓起勇气,呼啸着又冲上来,有人跃下水去,似要凿船。

小鱼儿暗道:``糟了!''船一沉,就真的糟了。

但就在这时、一条黄衣黄巾,虬髯如铁的大汉,突然自乱石间纵跃而来,身形兔起鹊落,口中厉声喝道:``住手!快住手!''数十条大汉一所得这喝声,立刻全退了下去。

只见这黄杉客站在一堆乱石上,自水中抓起一条大汉,正正反反掴了七八个耳掂子,顿足怒骂道:``你们这些蠢才都瞎了眼么?也不瞧清是谁在船上,就敢动手。''史老头长篙一点,江船竟在这急流中顿住!

黄衫大汉立刻躬身陪笑道:``在下实在不知道是史老前辈和姑娘在船上,否则有天胆也不敢动手的!这长江一路上,谁不是史老前辈的后生晚辈。''史老头冷冷道:``足下太客气了,老汉担当不起。老汉已不中用了,这长江上已是你们的天下,你们若要老汉的命,老汉也只有送给你。''黄衫大汉头上汗如雨下,连连道:``晚辈该死,晚辈也瞎了眼,晚辈实未想到史老前辈的侠驾又会在长江出现,否则晚辈又怎敢在这里讨饭吃。''史老头冷笑道:``讨饭吃这叁个字未免太谦了,江湖中谁不知道横江一窝黄花蜂做的全是大生意、大买卖。''他眼睛一瞪,厉声道,``但老汉这一艘破船,几个穷人,又怎会被足下看上,这倒奇怪得很,莫非足下是受人所托而来么?''水上的黄花蜂满头大汗,船上的江玉郎也满头大汗。只听黄花蜂连连陷笑道:``前辈千万原谅,晚辈实在不知。''史老头道:``你不肯说,你倒很够义气,好,冲你这一点,老汉也不能难为你。''长竿一扬,江船箭一般顾流冲了下去。

那黄花蜂长长松了口气,望着史老头的背影,喃喃道:``你们知道么,二十年前,不但长江一路全是他的天下,就算是天下叁十六水路的英雄,又有谁不怕他!咱们今天遇着他,算咱们命大,若是换了二十年前,这一带江里的水,只怕都要变红的了。''那大汉机伶伶打了个冷战,道:``他莫非是\ldots。.''黄花蜂大蝎道:``住口,我不要听见他的名字,也但愿莫要再见着他,老天若保佑我不再和他沾上任何关系,那就谢天谢地了。''江上生风,船已出巫峡。

史老头掌着舵,又不住咳嗽起来。

江玉郎瞧着他那在风中飞舞的白胡子.终于忍不住嗫嚅着问道:``老前辈莫非是·\ldots{}''是昔日名震天下的\ldots。.``史老头冷冷道:''你能不能闭上嘴。"

小鱼儿突然笑道:``史老头,我虽然还不知道你是谁,细想来你必定是个了不起的人物,你居然会为我撑船,我不但要谢谢你.实在也有些受宠若惊。''他居然还是叫他``史老头'',江玉郎眼睛都吓直了。

哪知这史老头反面向他笑了笑,道:``你莫要谢我,也不必谢我。''小鱼儿眨了眨眼睛,笑道:``那么我又该谢谁呢?是不是有人求你送我这一程,求你保护我\ldots\ldots 你年高德重,我若猜对了,你可不能骗我。''史老头弯下腰,不住咳嗽。

小鱼儿笑道;``你不说话,就是承认了。''

史老头脑色突然一沉,瞪着他道:``你小小年纪就学得如此伶牙利嘴,将来长大如何得了。''小鱼儿也瞪起眼睛,大声道:``我长大了如何得了,都是我的事,与你无关,你莫要以为是你救了我,我就该怕你,没有你送我,我照样死不了,何况我又没有叫你送我。''史老头瞪了他半晌,突又展颜一笑,道:``像你这样的孩子.老汉倒从未见过。''小鱼儿道:``像我这样的人,天下本来就只有我一个。''他赌气扭转了头,但心头还是在想:``这老头必定大有来历,如今竟降尊绎贵,来做我的船夫,那么,托他来送我的那人,面子必定不小。这人处处为我着想,却又为的是什么?他既然能请得动像这老人般的高手,想来又不致有什么事要求我。''小鱼儿实在想不到这人是谁,索性不想了,转身去看江玉郎,江玉郎竟似不敢面对着他。

小鱼儿突然笑道:``你那位紫狮子听说在云汉就上岸了,是么?''江玉郎道;``大\ldots\ldots 大概是吧。''

小鱼儿笑道:``保镖的勾结强盗,你却勾结了保镖的,叫保镖的通知强盗,来抢这艘船,否则那些强盗又怎会将别的船都挂上黄带子,只等着咱这艘船过去,否则那些强盗又怎会只要我的命,不要银子。''江玉郎汗流浃背,擦也擦不干了,咯咯笑道:``大哥莫非是在说笑么!''小鱼儿大笑道:``不错,我正是在说笑,你也觉得好笑么,哈哈,实在好笑。''他大笑着躺了下去,又喃喃笑道:``奇怪,这么凉快的天气,怎么有人会出汗。''云姑─直在旁边笑眯眯地瞧着他,江风,吹着他零乱的头发,他脸上的刀疤在阳光下显得微微有些发红。

顺风顺水,末到黄昏,船已到了宜昌!

大小船只无论由川人鄂,或是自鄂入川,到了这里,都必定要停泊些时问,加水添柴,采购伙食。

一入鄂境,江玉朗眼睛又亮了起来,像是想说什么,却又在考虑着该怎么样才能说出口。

小鱼儿笑嘻嘻瞧着他,突然跳起来,道:``咱们就在这里上岸吧,坐船坐久了,有些头晕。''他话未完,江玉郎己掩不住满面的喜色。

小鱼儿大声道;``史老头,多谢相送,将船靠岸吧,你虽然有些倚老卖老,但到底还是个好人,我不会忘记你的。''史老头凝目瞧了他许久,突然大笑道:``很好,你去吧,你若死水了,不妨到\ldots{}''小鱼儿摆手笑道:``你不必告诉我住的地方,也不必告诉我名字,因为我既不会去找你,也不想以你的名字去吓唬别人。''船还未靠岸,江玉郎已在东张西望。

史老头喃喃道:``要寻找危险的,就快快上岸吧,你绝不会失望的。''渡头岸边,人来人往,穿着各色的衣裳,有的光鲜,有的褴褛,有的红光满面,有的愁眉苦脸,有的刚上岸,有的正下船。

空气里有鸡羊的臭味,木材的潮气,桐油的气味,榨菜的辣味,茶叶的清香,药材的怪味\ldots。

再加上男人嘴里的酒臭,女人头上刨花油的香气,便混合成一种唯有在码头上才能嗅得到的特异气息。

小鱼儿走夜人从中,东瞧瞧,西闻闻,瞧见这样的热闹,他简直开心极了,就连这气味他都觉得动人得很,江玉郎却仍夜直着脖子,东张西望。

突听人丛外有人呼道:``江兄\ldots\ldots 江玉郎\ldots\ldots{}''江玉郎大喜道:``在这里\ldots\ldots 在这里\ldots\ldots{}''

他分开人丛,大步奔出去,小鱼儿也只得跟着他。

只见渡头外,一座茶棚下,停着叁辆华丽的大车,几匹鞍辔鲜明的健马,几个锦衣华服的少年,正在招手。

江玉郎欢呼着奔了过去,那几个少年也大笑着奔了过来,腰畔的佩剑,盯叮当当地直响。

今鱼儿冷服瞧着这几人又说又笑,却没有人理他,他却像是无所谓,等到他们笑过了,他也笑道,``奇怪,你的朋友怎会知道你要来的江玉郎脸一板,冷冷道:''这好像不关你的事吧``他非但称呼改了,神情也变了,方才还是满嘴''大哥小弟``此刻却像是主子对佣人说话,一个脸色惨白的绿衫少年,皱眉瞧着小鱼儿,就好像瞧着一条癞皮狗似的,满脸厌恶之色.道:''江兄,这人是谁?``江玉郎道:''这人就是世上第一个风流才子,第一个聪明人,女孩子见了他都要发狂的,你看他像么?``少年倒一齐大笑起来,像是世上再没有比这更可笑的事了,小鱼儿却仍然色声不动,笑嘻嘻道:''你的朋友,也该给我介绍介绍呀!``江玉郎眼珠子一转,招着那绿衫少年道:''这位便是荆州总镇将军的公子,白凌霄白小侠,人称绿袍灵剑客.叁十六路回风剑,神鬼莫测。``小鱼儿笑道:''果然是人如其名,美得很。不知道白公予可不可以将脸上的粉刮下来一点让我也美一美。"白凌霄笑声戛地而止,一张白脸变得发青。

江玉郎指着另一位又高又大的黑大汉道:``这位乃是江南第一家镖局,金狮镖局总镖头的长公子李明生,江湖人称红衫金刀,掌中一柄紫金刀,万夫莫故。''小鱼儿附掌道:"果然是相貌堂堂,威风凛凛。但幸好你解释得清楚,否则我难免要误会这位李公予是杀猪的。李明生两只铜铃般的眼睛,像是要凸了出来。

另一个珠冠花衫,眉清目秀,例有七分像是女子的少年,咯咯笑道:``我叫花惜香,家父人称玉面神判,若是没有听过家父的名字,耳朵一定不大好。''小鱼儿瞧了他半晌,突然摇头道:``可惜可惜,花公子没有去扮花旦唱戏实在是梨园的一大损失''花惜香征了征,再也笑不出来。

还有个又高又瘦、竹竿般的少年,叫``轻烟上九霄''何冠军,乃是轻功江南第一的``鬼影子''何无双之子。

最后一个矮矮胖胖,嘻嘻哈哈,但双目神光充足,看来竟是这五人中武功最强的一人,小鱼儿不免特别留意。

江玉郎介绍他时,神情也特别郑重,道:``这位梅秋湖兄,便是当今崆峒掌门人一帆大师关山门的弟子,他武功如何,我不说你也该知道。''梅秋湖哈哈一笑道:``过奖过奖,不敢当不敢当。''小鱼儿想说什么,但瞧他眼膀里似无恶意,竟只是拱了拱手,笑道,``久仰久仰。''他目光一扫,就知道这几个名人之子虽然油头粉脑,一面孔纨□子弟的样子,四人瞧着就讨厌。但瞧他们的眼神步法,却又可发现他们的武功竟都不弱,五人只要叁人联手,自己只怕就不是对手。

这几人瞧着小鱼儿,眼睛里却像是要冒出火来。

忽听一人娇声道:``好个没良心的江玉郎,知道我在这里,也不过来。''车厢中走下个十来岁的女孩子,严格说来,这少女并不难看。只是小鱼儿一瞧就要恶心,但江玉郎瞧了却是眉开服笑,大笑道:``孙小妹,我若知道你也来了.我早就过去了,只怕连李兄也拉不住我。''那孙小妹就像是唱戏似的,张开双臂,扑了过来,一头扑入江玉郎怀里,嘴里哼哼嗯嗯,道:``你这死鬼到哪里击了?我真想死你了.。''少年们拍手大笑,小鱼儿实在忍不住叹起气来,他若不是还没有吃晚饭,只怕此刻早已吐了一身一地。

劲小妹眼睛一瞪,手叉着腰部类声道:``喂!你这人怎么这佯讨厌,还不快走开。''小鱼儿叹道;``我若能走开,真是谢天谢地了。''小鱼儿伏在车窗上,头几乎已伸到车窗外,那位``孙小妹''就坐在江玉郎怀里,小鱼儿实在受不了她那香气。

奸狡深沉的江玉朗,怎会也变得这么浅薄,这么俗!小鱼儿忍不住去瞧他一眼,只见他面上虽笑得像是只呆鸟,但一双眼睛却仍闪动着鸷鹰般的光芒!

他哪里是真的这么浅薄,他原来只不过是装出来的。他若不;装得和这些不知天多高地多厚的纨□子弟一样.他们又怎会将他当做自己的好朋友。

小鱼儿笑了,头又伸出窗外,那``红衫金刀''李明生正在那里得意扬场地打着马,乌油油的鞭子,``□啪''直响。街道上的人瞧见这一群人马走过来,远远就避开了,尤其是小姑娘小媳妇们,更像是瞧见瘟神恶煞一样。

这澡盆看来就像是个特大的木桶,比人还高,桶下面,居然还有生火的地方,桶里的水热腾腾的冒着气。

江玉郎整个人就泡在这大木桶里,眯着眼睛,嘴里还不断发出舒服的呻吟。而小鱼儿呢?小鱼儿却只有站在桶外眼巴巴地瞧着,一只手还得吊在木桶旁边,简直是不舒服已极。

那位总镇之子,``绿袍美剑客''白凌霄就坐在对面,两条腿高高翘在个黄铜衣架上,摸着还未长出胡子的下巴笑道,``这澡盆乃是我家老头子属下一个悍将,自东瀛叁岛带回来的,叫做风吕,据说东瀛岛上的人不讲究吃,也不讲究穿,就是喜欢洗澡,只有洗澡是他们生活中的最大享受,一个澡最少要洗上半个时辰。''江玉郎笑道:``我这澡却洗了有一个时辰了。''他终于爬了起来,娇笑声中,两个胴体健美,赤着双足的短衫少女,已拿了块干布过来,替他擦身子,纤柔的玉手,隔着薄薄的轻布,摩擦着他发红的身子,那滋味简直妙不可言。

少女们娇笑着,替他穿上了雪白的中衣,轻柔的锦抱,江玉郎但觉满身舒畅,长长伸了个懒腰,大笑道:``这样洗澡,我也愿意每天洗上一次。\ldots·洗了这澡,我全身骨头都好像散了,人也好像轻了十斤他的。''小鱼儿叹道:``我却像是重了十斤。''

江玉郎冷冷道:``抱歉得很,此间主人,并没有招待你的意思,你要洗澡,不妨到外面去洗,但在下却不能奉陪。''小鱼儿道:``自然自然,我要洗澡,就得将手砍断,自己出去洗,是么?''江玉郎道:``你总算明白了。''

只听孙小妹在门外娇笑道:``江玉郎,你淹死在澡盆里了么,还不快些出来,我等你吃饭哩!今天花惜香在玉楼东为你洗尘接风。''江玉郎笑道``玉楼东,可是长沙那玉楼东的分店?''孙小妹道:``谁说不是。''江玉郎附掌道:``想起玉楼东的蜜汁火腿,我口水都要流下来了。''``玉楼东''的``蜜汁火腿''果然不愧为名莱,在灯下看来,那就像是盆水晶玛瑙似的,闪动着令人愉快的光芒。

但小鱼儿却不愉快极了。他刚伸筷子,就被白凌霄打了回去,花惜香咯咯笑道:``我根本不认识你,所以也用不着为你洗尘接风,是么?''小鱼儿道:``是极是极,我若要吃,就得割下只手,自己出去吃。''白凌霄大笑道:``你真是越来越聪明了。''

于是小鱼儿就只得看着他们开怀畅饮,看着他们狼吞虎咽,他脸上虽还在笑,肚子却不觉在叫救命了。

突听一阵楼梯响动,几个人大步走上楼来,这几人年纪都在四五十多,穿着俱都十分体面,顾盼之间,也都有些威严,显然不是等闲角色,花惜香、李明生、何冠军\ldots\ldots 这些眼睛长在头顶上的少年们,瞧见这几人,竟全都站了起来,一个个都垂着头低着眉,突然变得老实得很,有的恭声晚道:``师傅。''有的垂首唤道:``爹爹。''

小鱼儿不觉皱起了眉头,哪知这几人却瞧也不瞧他们的徒弟儿子们一眼,反而都走到小鱼儿面前,齐地抱拳笑道:``这位莫非就是江鱼江小侠么?''这一来,小鱼儿更觉奇怪.眨着眼睛道:``我就是。''当先一条白面微须的中年汉子立刻招手道:``店家,快摆上一桌酒菜,我等为江小侠接风。''花惜香、白凌霄,一个个怔在那里,像是呆了。

非但``玉面神判''来了,``鬼影子''何无双、``金狮''李迪,这城里的武林大豪,居然来的一个不漏。

小鱼儿吃完了整整一盆蜜汁火腿,终于忍不住笑道:``儿子们把我当狗屁,老子们却对我客客气气,这究竟是怎么回事,你们可不可以说给我听听。''玉面神判笑道:``犬子无札,江小侠切莫见怪。''又瘦又长、面色铁青的``鬼影子''何无双接口笑道:``我等受了一位武林前辈所托,要我们对江小侠务必要尽到地主之谊,这位武林前辈德高望重\ldots\ldots{}''小鱼儿道:``他究竟是谁?''

玉面神判想了想,笑道:``那位前辈本令我等守秘,为的自然是不愿江小侠回报于他。''小鱼儿笑道:``你放心,我向来不懂得报恩的,报仇么,也许还可能,但报起仇来若太麻烦我也就算了。''玉面神判附掌道:``江湖中人若都有江小侠这样的心胸,为武林开此古来未有的新风气,倒真的是人群之福''\ldots.``小鱼儿道;''现在,你可以说出他是谁了么?``玉面神判缓缓道:''峨嵋掌门,神锡道长!"

小鱼儿拍案道:``原来是他。\ldots 这一路上原来都是他,他倒没有忘记我\ldots\ldots{}''数日疑惑,一旦恍然,于是开怀畅饮,大吃大喝,玉面神判、鬼影子等人只是含笑望着他,谁也没有动筷子。

\hypertarget{ux7b2cux4e09ux5341ux516bux7ae0-ux6c5fux5357ux5927ux4fa0}{%
\chapter{第三十八章
江南大侠}\label{ux7b2cux4e09ux5341ux516bux7ae0-ux6c5fux5357ux5927ux4fa0}}

鱼儿埋头苦吃了半个时辰,总算放下筷子,摸着肚子笑道:``肚兄肚兄,今日我总算对得起你了吧!''玉面神判笑道:``酒菜都已够了么?可要再用些瓜果?''小鱼儿笑道:``我很想,只是肚子却不答应!''玉面神判微微一笑,道:``既是如此,我等总算不负神锡道长之托,已尽过地主之谊了。''小鱼儿眨了眨眼睛,道:``你话里好像有话\ldots\ldots{}''玉面神判霍然长身面起,缓缓道:``阁下不妨先推开窗子看看。''小鱼儿推开窗子一瞧,只见这一段街道上,竟已全无灯火行人,却有数十条劲装大汉,将酒楼团团围住。

再瞧这酒搂之上,也再无别的食客,只有个店小二站在楼梯口,面上满是恐怖之色,两条腿不停地抖。

小鱼儿歪着头想了想,笑道;``这算什么?''

玉面神判脸色一沉,冷冷道:``受人之托,忠人之事,神锡道长托我好生招待你,我等便尽了地主之谊,但还有一人,却托我等来取你的头颅,你看怎样?''小鱼儿哈哈大笑道:``我这颗脑袋居然还有人要,这倒真是荣幸之至,但要我脑袋的这人又是谁?你总该说来听听。''玉面神判冷笑道;``你只需知道他有一个鼻子两只眼睛已足够了。''小鱼儿目光转处,只见江玉郎等人俱是满面喜色,鬼影子等人却是面色凝重,满脸杀气。

这些人早已将他围住,这许多武林高手将他围在中央,他简直连出手的机会都没有,更何况他还有只手是和江玉郎连着的,他根本连逃都不能逃。

小鱼儿长叹一声,苦笑道:``看来,今天我只得将脑袋送给你们了\ldots。一盆蜜汁火腿就换去了我的脑袋,这岂非太便宜了些!''``金狮''李迪呛``的拔出了腰畔紫金刀,厉声进:''你还要我等动手么?``小鱼儿笑道:''用不着了,只是不知道你的刀快不快?若是一刀包险可以切下脑袋,我倒想借来用用。"``金狮''李迪狂笑道:``好,念你死到临头,还有谈笑的本事,某家就把这柄刀借给你!''手扬处,紫金刀夺"的钉在桌上,小鱼儿缓缓伸出手,去拿这柄刀,无数道比刀光更冷更亮的眼睛里,都在瞧着他这只手。

玉面神判冷冷地瞧着他,突然自怀中摸出了对判官笔,那是对十分精巧的兵器,发亮的竹杆上雕着精致的花纹。

小鱼儿的指尖停留在刀柄上,没有拔。

玉面神判缓缓道:``你为何不拔你拔出这柄刀来,就可以一刀砍向我,或是别的人,或是将刀架在江玉朗的脖子上,逼我们放你走。''小鱼儿的手指轻点着刀柄,没有说话。

玉面神判道:``你不敢拔这柄刀的,是吗?只因你自己也知道。只要你拔出这柄刀,只有死得更惨。''小鱼儿觉得自己的手很冷,而且在流汗。

玉面神判叱道;``念你是个聪明人,且给你个速死,咄,去吧!''手腕一抖,判官笔闪电般向咽喉``天突''穴点了出去,这``天突''乃是人身必死大穴之一,纵然被常人拳脚打中,也是难以救治,何况是这等点穴名家掌中的纯钢判宫笔,小鱼儿历经大难不死,岂知竟要死在这里!

眼看这发亮的笔尖已到了咽喉,他竟躲都懒得躲了,躲开这一招,第二招反正还是要来的,既然要死,何不死得痛快些。

哪知就在这时,突听``叮''的一声,一只酒杯自窗外直飞进来,不偏不倚套住了判宫笔的笔尖。

那判官笔击势是何等凌厉,酒杯又是何等容易破碎,奇怪的是,酒杯远远飞来,套住笔尖,居然还是完整的!

玉面神判手腕反似被震得麻了麻,大惊之下,后退叁步,厉喝道:``什么人?''这时新月方自升起,淡淡的月光下,只见对街``老介福绸缎庄''的招牌上赫然坐着一个人。

这人满头蓬头,敞着衣襟,手里提着个特大的酒葫芦,正在嘴对嘴的狂饮,酒葫芦遮去了他的面目,也看不出他是谁.但小鱼儿却已瞧出来了,暗道:``此人来了,又有好戏瞧了。''玉面神判手腕一震,笔尖上的酒杯直飞出去,直打对面那人的胸膛,他自信手上劲力,无论是谁,只要被这酒杯击中,身上必定要多个窟窿,只听又是``叮''的一声,酒杯打在那人身上,片片粉碎。

那人却竟似全无感觉!

玉面神判面色更变了,花措香、白凌霄、李明生等人,拔刀的拔刀,拔剑的拔剑,一时之间刀光剑影大作!

``鬼影子''何无双身子也不见动弹,人突然飞了出去,此人号称轻功江南第一,身手之轻捷果然不同凡俗。

只见他人在空中.手里已有十余点寒光暴射而出。

对街那人突然哈哈一笑,一般闪亮的银光,自口中射了出来,暗器立刻被打飞,银光直射到何无双身上。

这轻功第一的鬼影子竟也被打得飞了回来,回时比去时更快,直飞入窗子,飞过桌面,``砰''的撞在墙上。

那般银光到这时才四溅散开,玉面神判远远便觉得酒气扑鼻,那人嘴里喷出来的,竟只不过是口酒!

他一口酒竟然就将何无双击退,众人不禁都变了颜色,白凌霄等人初生之犊不怕虎,各展刀剑,便要扑过去。

只听``呼''的一声,接着``□□啪啪''一连串声响,白凌霄等人手里的刀剑已全不见了,一个个捂着脸,半边脸色红得像是茄子,就在这刹那之问,这几个人竟已每人重重挨了个耳刮子。

再瞧对面那人,不知何时已端端正正坐在何无双方才坐过的位上,左手仍拿着那酒葫芦,右手却杂七杂八拿了一大把刀剑,白凌霄等人认得,这些刀剑正是自己的,但若问他们怎会到了别人手上?他们只怕谁也回答不出。

江玉郎瞧见这人,面色变得毫无人色,玉面神判心计最深.在未知这人来历之前,生怕李迪等人鲁莽闯祸,当下抢先一步,干笑道:``这位兄台贵姓大名为何无端出手伤人?''那人眼睛一斜,冷冷道:``谁是你的兄台,你是什么玩意儿?''玉面神判勉强忍住怒气,铁青着脸道:``在下萧子春,江湖人称玉面神判。''那人哈哈大笑道;``好个响亮的名头,你配么?''笑声中手一送,将一大把刀剑全送到萧于春面前,雪亮的刀头剑尖,在灯光下像是猛虎的獠牙。

玉面神判一惊之下,不由得伸手去接,再看自己手里那对判宫笔不知何时已到了对方手里。

那``金狮''李迪没有吃过苦头,浓眉一轩,便待发作。江玉郎在桌下扯了扯他袖子,悄悄说了句话。

李迪面色立刻也变得全无人色,失声道:``你\ldots\ldots 你便是恶赌鬼轩辕叁光!''轩辕叁光冷笑一声,也不说话,却自桌上拔起了那柄紫金刀,反手一刀,向旁边一个茶几砍了下去。那茶几上点着只儿臂般粗的蜡烛。

轩辕叁光这一刀砍下去,蜡烛仍是蜡烛,烛台仍是烛台,茶几仍是茶几,他这一刀像是根本砍空了。

但突然间,烛光竟缓缓分了开来,接着蜡烛、烛台、茶几,全都分成了两半,向两边直倒下去。这一刀出手,众人更是面如死灰。

轩辕叁光一扬紫金刀,``夺''的钉入梁上,梁上积尘,簌簌而落,他再也不瞧─眼,一屁股坐下,冷冷道:``儿子们眼见老子来了,怎地还不快摆上酒菜!''他这句话说的虽然无理,但听在众人耳里,再也无人敢顶撞于他。

李迪``砰''的一拍桌子,大喝道:``小二,瞧见老子来,为何还不摆上菜来。''他看来人虽最是粗豪,但做保镖的人,究竟能屈能伸。

那店伙魂魄早巳骇飞了,此刻哪里还禁得起这一声大喝,口中刚说了声``是'',人已直滚下楼去。

少时酒菜摆上,萧子春、李迪抢着要来斟酒。

轩辕叁光眼睛─瞪,道:``谁要你斟酒,除了对面两个姓江的娃儿,全给老子远远站开。''他居然拿起酒壶,替小鱼儿倒了杯酒,又替江玉郎倒了杯酒,小鱼儿满怀欢喜,江玉郎却已骇破苦胆。

轩辕叁光端起酒杯,道;``喝!''

小鱼儿一饮而尽,江玉郎也不敢怠慢,他刚放下杯子,只见轩辕叁光眼睛已在盯他,咯咯笑道:``你可知道这酒叫什么酒?''江玉郎道:``弟\ldots\ldots 弟子愚昧,实在不懂。''

轩辕叁光大声道:``这─杯叫赌酒,无论谁喝了老子倒的酒,都得和老子赌─赌。''江玉郎骇得手一抖,酒杯也摔在地上。

轩辕叁光眼睛一瞪,道:``怎么?你不赌?''

江玉郎道:``吐''。``吐''\ldots\ldots 吐``。''

他骇得舌头都麻了,竟将``赌''宇说成了``吐''。

轩辕叁光大笑道;``好,你龟儿要赌啥?''

江玉郎道:``吐\ldots\ldots 吐什么\ldots\ldots 都可以。''

轩辕叁光道:``好,老子就赌你这条手臂。''

江玉郎两腿一软,从椅子上滑了下去,小鱼儿笑嘻嘻将他拉了起来,道:``你怕什么?反正也未必一定输的。''轩辕叁光厉声道:``坐直了,说,你要怎样赌?''江玉郎目中竟流下泪来,转眼去瞧萧子春等人,但这些人此刻哪里还敢替他出头?

突然间,一人朗声笑道:``轩辕先生若要赌,在下可以奉陪.寻这等黄口孺子来赌,岂非无趣么?''小鱼儿转眼望去,但觉眼睛─亮。

一个青衫秀土巳飘飘走上楼来。

灯光下,只见此人眉清目秀,面如冠玉,他含笑定过来,风神更是潇洒已极,小鱼儿自出道江湖以来,除了那无缺公子外,就再末见过如此令人着迷的人物。

萧子春等人见到他来了,都不禁在暗中长长松了口气,喜动颜色,江玉郎更是欢喜得几乎要跳了起来。

轩辕叁光目光闪电般在他身上一转,也不禁为之动容道:``你是谁?''这人微笑一揖,道:``在下江别鹤。''

轩辕叁光目光闻动,厉声道:``江湖传言,江南一带,出了个了不起的英雄,乃是燕南天之后第一个当得起大侠两宇的人物,莫非就是你?''江别鹤笑道:``那只是江湖朋友抬爱,在下怎担当得起。''轩辕叁光指着江玉郎摇头叹道:``虎父犬子\ldots\ldots 虎父犬子\ldots\ldots{}''突又一拍桌子,大喝道:``他既是你的儿子,你莫非要代他与我赌一赌?''江别鹤道:``轩辕先生若有兴致,在下自当奉陪。不知轩辕先生赌注如何?''轩辕叁光微一思索,浓眉轩起,大声道:``你我两人无论谁输了,便任凭对方处治!''这赌注说出来,众人不禁俱都失色,这``任凭对方处治'',委实令人心惊,胜的一方若令败的一方去做件绝不可能、甚至丢人现眼的事,那岂非比``死''更痛苦百倍,尤其是以江别鹤这样的身分,他若输了,就算想死,也先得做了对方要求之事才能死的。他就算死也不能食言背信。

众人只道江别鹤绝不会答应,哪知他只是淡淡一笑道:``就是这样也好,但如何赌法,还请见告。''轩辕叁光见他如此轻易便答应了这席注,也不禁为之动容,端起面前酒杯,─饮而尽,大笑道;``好,江南太快果然豪气干云,我定了赌注,如何赌法便由得你,这是我的规矩。''江别鹤笑道;``既是如此,在下恭敬不如从命了。''他走过去,搬了张小圆桌来,又将一大碗满满的鱼翅羹放在桌子中央,轩辕叁光瞧得奇怪,道:``这又算了什么?''江别鹤缓缓道:``你我依次往桌上击一掌,谁若要将这碗鱼翅羹震得溅出,或是使得碗落下去,那人便算输了。''他口中说话,一掌向那桌面拍了下去。

他这一掌似乎也未用什么气力,但那坚硬的梨木桌面在他掌下竟像是突然变成了豆腐似的。

他一掌切下,竟穿透了桌面,桌上那碗盛得满满的鱼翅羹,果然还是纹风不动,没有溅出一滴。

江别鹤微微笑道:``你我──掌击下,必定穿透桌面,是以就算你我两人都未将这碗鱼翅羹震倒,到了后来,桌面上惧是掌痕,那中央一块,总要落下去的,谁击下最后一掌,谁就输了,是以桌子越小,胜负便越早。''众人都已被这种掌力惊得呆了,直到此刻才喝出来来,就连小鱼儿也不能例外,他实也未见过这种掌力。

轩辕叁光面色也已变了,站在那里,怔了许久,喃喃道:``这样的赌法,倒真连我也未曾见过。''江别鹤笑道:``在下已击下了第一掌,此刻该轮到轩辕先生了。''轩辕叁光突然仰首狂笑道;``我恶赌鬼平生与人大赌小赌,不下万次,从未有─次还未赌时,便己先认输了\ldots\ldots{}''他突又顿住笑声,目光凝注江别鹤,道,``但这次,我不必赌,已认输了''。·我掌力纵能穿透桌面,却万万不能令这碗见鬼的鱼翅羹一滴也不溅出来。"众人长长嘘了口气,大喜狂欢。

轩辕叁光惨然一笑,背负双手,道:``现在,你要我怎样,只管说吧!''江别鹤微一沉吟,走过去倒了两杯酒,笑道:``在下且敬轩辕先生一杯。''轩辕叁光仰首一饮而尽,``砰''地放下酒杯,厉声道:``现在轩辕叁光是生是死,往东往西,凭阁下吩咐!''

\hypertarget{ux7b2cux4e09ux5341ux4e5dux7ae0-ux5047ux4ec1ux5047ux4e49}{%
\chapter{第三十九章
假仁假义}\label{ux7b2cux4e09ux5341ux4e5dux7ae0-ux5047ux4ec1ux5047ux4e49}}

江别鹤微笑道:``在下要轩辕先生做的事,方才不是已做过了么?轩辕先生的赌注既已付清,为何还要说这样的话。''轩辕叁光又怔住了,呐呐道:``你你说什么?''江别鹤笑道;``输的一方,既是任凭胜方处置,在下就罚轩辕先生一杯酒,此刻轩辕先生酒已放下,正是银货两讫,各无赊欠了。''轩辕叁光木立当地,喃喃道:``你若能杀了我,江湖中谁不钦服,你若要我做件事,无论奇珍异宝,名马灵犬,我也可为你取来,但\ldots\ldots 但\ldots\ldots{}''他长叹一声,苦笑道:``但你却只是要我喝一杯酒。''江别鹤笑道;``若不是在下量小,少不得还得多敬几杯。''轩辕叁光突然举起那酒葫芦,一口气喝了十几口,伸手抹了抹嘴唇,仰天长笑起来,道:``好!果然不愧是江南大侠!我轩辕叁光平生未曾服人,今日却真的服了你江别鹤了!''大步走过去,拍了拍小鱼儿肩头,道:``小兄弟,你的事我已管不了啦,但有江南大侠在此,你再也不必伯那些鼠辈欺负了,我且去了\ldots。再见!''说到``再见''两宇,人已出窗,眨眼便消失在夜色中。窗外凉风习习,一弯新月正在中天。

江别鹤目送他去,喃喃叹道:``此人倒不愧是条好汉!''``玉面神判''萧子春陪笑道,``此人名列十大恶人,江兄不乘机将之除去,岂非太可惜了?''他口中虽以兄弟相称,但神情却比弟子待师长还要恭敬。

江别鹤正色道:``这样的英雄人物,世上有几个?萧兄怎能轻言除去两字,何况,此人除了好赌之外,并无别的恶迹。''萧子春垂首笑道:``是,小弟错了。''

江别鹤笑道:``更何况他只要赌输,使绝不抵赖,纵然输掉头颅,也不会皱一皱眉头,试问当今天下,有他这样赌品的人,能有几个!''小鱼儿突然叹了口气,道:``只可惜轩辕叁光没有听见你这番话;否则他真要感激得眼泪直流了。''江别鹤目光上下瞧了他一眼,展额笑道:``这位小兄莫非也是犬子好友?''小鱼儿道:``好友两宇,我可实在不放当。''

江别鹤目光一闪,已瞧见了他们手上的``情锁'',微微笑道:``这旁门左道的区区之物,我自信还能将之解开,小兄你只管随我回去\ldots\ldots{}''小鱼儿笑道:``我也实在很想随你回去,只是这里还有人等着宰我,怎么办呢?''江别鹤皱眉道:``谁?''

小鱼儿道;``自然都是些威名赫赫的英雄豪杰,七八个成名的大英雄等着宰我一个人,这岂非光荣之至。''江别鹤目光一转,满屋予的人惧都垂下了头,萧子春、李迪等人更是面红耳赤,江别鹤缓缓道:``我可保证,这种事以后绝不会发生了。''突听窗外远处黑暗中有人高歌。歌声随风传来,唱的竟是:``江南大侠手段高,蜜糖来把毒药包,吃在嘴里甜如蜜,吞下肚里似火烧.糟!糟!糟!大下英雄俱都着了道\ldots\ldots{}''江别鹤神色不变,微微笑道;``得名之人,谤必随之,我既不幸得名,挨些骂也是应当的,此等小人,你若去追他,岂非反令他得意。''小鱼儿笑眯眯瞧着他,道:``我小鱼儿也很少服人,今天也倒有些服你了─\ldots\ldots{}''若没有自己去看过,谁也不会相信``江南大侠''住的竟是这样的屋子。那只是叁五间破旧的屋于,收拾得虽然干干净净,一尘不染,但陈设却极为简陋,也没有姬妾奴仆,只有个又聋又哑的老头子,蹒跚地为他做些杂事。

小鱼儿随着他走了两天,才走到这里。

这两天小鱼儿更觉得这``江南大侠''实非常人,一个在武林中有如此大名的人,对人竟会如此客气,这大概除了江别鹤外,再没有人能做到了,和他走在一起,就如同沐浴春风一般,无论是谁,都会觉得很舒服、很开心的。

走进了这间屋子,小鱼儿更不免惊奇。

江别鹤微笑道:"这庄院昔日本是我一个好友诸葛云的,他举家迁往鲁东,就将庄院送给了我,只可惜我却无法保持它昔日的风貌,想起来未免愧对故人。

小鱼儿叹道:``名震天下的江南大侠,过的竟是如此简朴的生活,千百年来,武林中只怕没有第二个了。''江别鹤正色道:``古人说:由俭入奢易,由奢入俭难,这句话我从未忘记。''小鱼儿叹道:``你真是个君子。''

少时菜饭端来,也只是极为清淡的叁四样疏菜,端菜添饭摆桌子,竟都是这领袖江南武林的盟主自己动手的。这样的主活,与他那炫目的名声委实太不相称。

小鱼儿喃喃道:``难怪天下江湖中人都对你如此尊敬,一个人能忍别人之所不能忍,自然是应当成大事的。''江别鹤闪亮的目光转注着他,忽然道:``我看来看去,越看越觉得你像我昔日一位恩兄。''江别鹤叹道:``他是昔日江湖人中温文风雅的典型,也是千百年来江湖上最着名的美男子,我为小儿取玉郎这名字,正也是为了纪念他的。''小鱼儿笑道;``你看我像个美男子?我这人若也可被称为温文风雅,那么天下的男子就没有一个不是温文风雅的了。''江别鹤微笑道:``你也许并不十分温文风雅,但你的确有他那种无法形容的魅力,尤其是你笑的时候,我不相信世上有任何少女能抗拒你微笑时瞧着她的眼睛。''小鱼儿大笑道:``我但愿能有你说的这么好,也但愿能就是你说的那人的儿子,只可惜我爹爹也和我一样,纵然是个聪明人,但绝不是什么美男子,而且他现在也正活得好好的,也许正在他那张逍遥椅上抽着旱烟哩。''他大笑着站了起来,走了出去。江玉郎也只有跟着他。

小鱼儿又笑道:``我实在想陪你多聊聊,却又实在忍不住要去睡了\ldots\ldots 希望你明天能找几个有用的锁匠来,能将这见鬼的情锁打开。''江别鹤叹道:``这一路上我几乎已将鄂中一带有名的巧手锁匠都找过了,我实也未想到这情锁的视簧竟造得如此之妙。''他一笑又道:``但你只管放心,就在这两天我必定能寻得一柄削铁如泥的宝剑\ldots\ldots 到了我这里,你什么事都不必再烦心了。''小鱼儿笑道:``所以我现在只要一沾着枕头,立刻就会睡得像死人似的。''江玉郎现在就像是已突然变成了一个世上最听话、最老实的孩子,老老实实的随他走了出去。

江别鹤温柔地瞧着他们的背影消失,缓缓在袖中摸索着,竟摸着了柄长不过一尺的短剑。

这短剑的剑鞘黑黝黝的,看来毫不起眼,但等到江别鹤抽出这口剑来,屋子里却像是有电光一闪。森冷的剑气,立刻使烛火失去了光彩。

那又聋又哑的老头子,远远站在门口,此刻也不禁打了个冷战,他瞪大了眼睛,像是在说:``你手里的明明已是削铁如泥的宝剑,却又为什么不为他人将那见鬼的情锁削断?''江别鹤抬起头,瞧见他这充满惊疑的目光,像是已瞧破了他的心意,微微一笑,缓缓道:``我此刻自然还不能将那情锁削断,那孩子一肚子鬼主意,谁也猜不到他要干什么,我只有叫玉郎时时刻刻地监视着他\ldots{}''·有了那情锁,他就是想溜想跑,却也是跑不走的了。"可惜他说话的对象只不过是个又聋又哑的老头子,他无论说什么,这老头子都是听不见的。

走廊上,有个小小的灯笼,昏黄的灯光,照着荒凉的庭园,一只黑猫蹲踞在黑暗里,只有眼睛闪着碧绿的光。

小鱼儿和江玉郎走在这曲廊上,脚下的地扳吱吱直响,远远有风吹着树叶,小鱼儿缩起了脖子,苦笑道:``任何人着在这种地方住上十年,不变成疯子才怪。''江玉郎道:``你放心,你用不着住十年的。''

小鱼儿笑道:``你终了说话了\ldots\ldots,方才在你爹爹面前,我还以为你变成哑巴哩!''江玉郎道:``在我爹爹面前敢像你那样说话的人,世上只怕也没有几个.''小鱼儿瞧着那黑黝黝的后园,笑笑道:``这后园你去过么?''江玉郎道:``去过一次。''

小鱼儿道:``你在这里也住了许久,只去过一次?''江玉郎道:``去过一次的人,你用鞭子抽他,他也不会去第二次了。''小鱼儿笑道:``那里面难道有鬼?''

江玉郎道:``那种地方,鬼也不敢去的。''

他打开一扇门,悬起了一盏灯,小小的屋子里,有几柄刀剑,一大堆书,自然,还有张床。

小鱼儿眼珠一转,道:``这就是你的卧房?''

江玉郎长长叹了口气,道:``一年多没有回来,此刻看见这张床,也不觉亲热得很。''小鱼儿笑道:``瞧见你那些宝贝朋友之后,打死我也不相信你以前会老老实实睡在这张床上,你难道真的憋得住?''江玉郎突然一笑,道,``半夜我不会溜出去么?''小鱼儿道:``我自然知道大户人家的子弟,都有半夜溜出去的稚癖,但你爹爹可与别人不同,你怎能逃得过他的耳目?''江玉郎眨了眨眼睛,道:``你可知我为什么要你在这屋子里?''小鱼儿道:``知道。''

江玉郎道:``只因这屋子距离我爹爹的卧房最远,而且窗子最多\ldots\ldots 这本来应该是佣人住的地方,但我却抢着来睡了。''小鱼儿笑道;``据我所知,这只怕是你最聪明的选择了!''回到了自己的卧房,江玉郎终于也放下了心,睡到床上,还没有多久,便已真的睡着,而且睡得很沉。他也用不着再去提防小鱼儿,他也实在累了。小鱼儿也像是睡得很沉。

也不知道了多久,有一阵轻轻的脚步声走了过来,走到门外,停了停,轻轻敲了敲房门。门里没有应声,这人将门推开一线,瞧了瞧,然后这脚步声又走了回去,竟像是走入了那荒凉的后园。

这连鬼都不敢去的地方,他叁更半夜去做什么?

小鱼儿突然张开了眼睛,自头发里摸出了根很细很细的铜丝,竟将这铜丝刺入那``情锁''上的一个小洞里。他耳朵贴在这``情锁''上,将那铜丝轻轻拨动着──他眯着眼睛,聚精会神地,就像是在听着什么动人的音乐。

突然,轻轻``喀''的一响,那鄂中所有的巧匠都打不开的``情锁'',居然被他以一根细细的铜丝拨开了。

他面上不禁露出了得意的笑容,挥动着那只失去自由已久的手随手点了江玉郎的``睡穴''。

江玉郎腿得更不会醒了。

小鱼儿瞧着他得意地笑道:``你自以为聪明,其实却是个呆子,竟一直以为我真的弄不开这见鬼的情锁,你也不想想,我是在什么地方长大的。''``恶人谷''中既然有最出色的强盗,自然也有最出色的小偷,在最出色的小偷手下,世上哪有打不开的锁,但他为什么却一直宁愿和江玉郎锁在一起?宁愿受各种气?他心里究竟又在打着什么主意?莫非他早已猜到江玉郎的父亲必定是个神秘的人物?莫非他早巳猜到这地方必定有一些惊人的秘密?

他要和江玉郎锁在一起,莫非只不过就是要到这里来!而且还可令别人都因此而不再防着他。任何人都以为他是摆脱不了江玉郎的,有江玉郎时时刻刻、寸步不离跟着他,别人自然都放心得很。

但这时,小鱼儿己溜出了窗子。竟向那连鬼都不敢去的后园掠了过去。这时,那脚步声入园已有许久丁。

小鱼儿掠入那圆月形的门时,只瞧见远处有灯火闪了闪,然后,便是一片黑暗,灯火竟似熄灭。

黑暗中,树木在风中摇舞,仿佛是许许多多不知名的妖魔,正待择人而噬,天上虽然有黯淡的星光,但星光却更增加了这园林的神秘和恐怖,风很冷,但小鱼儿掌心却是湿湿的,已沁出了冷汗。

假如是别人,此刻早巳退回去了。但小鱼儿却不是``别人'',小鱼儿就是小鱼儿,天下独一无二的小鱼儿,他若要前进,世上再无任何事能令他后退。

他早己认准了方才那灯火闪动之处,他就直掠过去。但园林中只有枯萎了的树木,颓败了的山石小亭,方才那一点灯火,早巳不知到哪里去了。

走着走着,小鱼儿突然迷失了方向。一阵风吹过,他忍不住机伶伶打了个寒噤,他忽然发觉自己根本不知道该走到哪里去?该找些什么?

就在这时,一条黑影自黑暗中窜了出来!小鱼儿魂都几乎被骇飞了,黑影窜过去,竟是条黑猫!但这黑猫又怎会入了这后园?又怎会突然窜出来?

小鱼儿心念一转,绝不再多想,立刻伏到地上,前面有一堆碎石瓦砾,还有一片枯萎的菊花。

他身子刚伏下来,十余丈外,突然有一扇窗子亮起了灯火,接着,一条人影缓步走了出来。这人手掌着灯,灯光照着他的脸,赫然正是江别鹤!

只听他``咪呜''一声,那黑猫便向他窜了过去,窜入他怀里,他反手扣起了门,抱着黑描走了回去。

小鱼儿伏在地上,连大气都不敢出。灯火,刚刚去远,园林中像是更黑、更冷。小鱼儿又等了许久,才悄悄爬了起来,悄悄走过去,走到前面,才瞧出那里有间小小的花房。

门,已锁上了。

于是小鱼儿又有了机会施展他开锁的本事。

他轻轻推开了门,点着他方才从桌子上偷来的火折子,花房里蛛网密布,角落里堆着些破烂的花盆、枯叶、木炭,此外就什么也没有了──半夜叁更,江别鹤跑到达什么也没有的破屋子里来做什么?

风吹着窗户,吱吱作响,风从破了的窗纸里吹进来,就像是一只冰冷的鬼的爪子,在摸小鱼儿的背脊。小鱼儿真想逃走,逃回床上,用棉被盖住头,这种地方,真是连鬼也不会愿意来的。

但连鬼也不来的地方,岂非最好隐藏秘密!

他目光四下转动,瞧了半晌,也瞧不出这屋子里有什么可疑之处,屋子里到处都积着灰尘,像是已许久没有人来过!但江别鹤方才明明来过,灰尘上怎会没有他的脚印?小鱼儿心一动,俯身摸了摸,那灰尘竟是粘在地上的,除非你用力去搓,否则什么痕迹也不会留下。

小鱼儿几乎跳了起来,他知道这屋子必有地道,但他将每个角落都找遍了,还是找不出有什么机关消息。

他几乎绝望了,仰面长长叹息了一声,蛛网。在风中飘摇,有些蛛网巳被风吹断了,蜘蛛正忙着在重新绘起。但有一张蛛网,任凭风怎么吹,却动也不动。

这种事别人也不会注意,但世上再也没有一件事能逃过小鱼儿的眼睛,他立刻窜了过去!

他发现这面蛛网竟是以极细的钨金丝做成的!他立刻一跃而起,将这面蛛网─拉。

只听``格''的一声,接着,又是一连串格格"声响,蛛网下的一堆枯柴突然缓缓移动,露出一个洞来!小鱼儿也曾见过许多设计巧妙的秘密机关,但却从未见过有任何一处比这更巧妙、更秘密。

除了没有窗子,这实在是一间最标准的书房,就和世上大多数读书人读书的地方完全一样。

书房的左右两壁,是排满了书的书橱书架,中间是一张精雅的大理石书桌,桌上整齐地排列着文房四宝。

除此之外,自然还有盏铜灯,小鱼儿点燃了它,然后,便坐在那张舒服的大椅子上,他开始静静地想:``我若是江别鹤,我会将秘密藏在什么地方?''任何一间书房里,可以收藏秘密的地方都很多,但假如那秘密是一些纸张,最好是藏在什么地方?

最好自然是藏在书里但这里有成千成百本书他又会藏任哪本书里?

自然要藏在别人最不会翻阅的一本书里──虽然,这里绝不会有人走来翻他的书,但他却也会习惯地这样做的。

小鱼儿站了起来,仔细去瞧那书架。他一本本地瞧,书架上有石刻的《史记》、《汉书,还有些手抄的珍本杂记,每本书都已积着灰尘,江别鹤到这里,自然不会是为了看书,这些书上自有积尘,但这里\ldots\ldots 就在这里,却有本书非常干净。

这本书不算薄,小鱼儿抽下来,书皮上写的是:``本草''。

小鱼儿笑了,就发现这本书中间已被挖去了一块,四边却粘在一起,就像是个盒子,书中被挖去的地方,竟放着几张精巧的人皮面具,还有叁两个小瓶子,这显然是易容的工具。

但小鱼儿却对这些完全没有兴趣,他再找,又找出个同样的``书盒子'',这里面也有几只小木瓶。瓶子里装的竟是非常珍贵的毒药!

小鱼儿叹了口气,再找,他又找出一叠数目大得骇死人的银票,还有张很大的名单。他也懒得去瞧那些名字,只瞧见每个名字下却有个括弧,括弧里有的写着``少林'',有的写着``武当'',每一个都写的是名门大派,也许,这些是江别鹤派到这些门派中奸细的名字但小鱼儿却也懒得管它,这些虽然都是惊人的秘密,但却不是小鱼儿所要找的,他失望地坐了下来。

突然,他瞧见书桌旁有些矮几,矮几上堆满了纸,各色各样的纸,他眼睛像是一亮,抓起一叠纸。

纸质很轻,很薄,却带着韧性,这种纸,在当时是非常特殊的,小鱼儿也不过见过一次。但他却知道这种纸的味道!只因他曾经将一张同样的纸吞入肚里。

这叠纸,正和他从铁心兰处得来的那``燕南天藏宝图''的纸质是完全一样的,他死也不会忘记。

他仔细地刮了一小撮尘土,轻轻抹去最上面一张纸上,纸上便现出了花纹,果然正是那藏宝图的图形。

要知那藏宝图为了要求逼真,是用木炭条画的,在上面的一张纸上画过最后一张图后,又恰巧没有再动过这叠纸。

小鱼儿长长叹了口气,哺哺道:``伪造那藏宝图的人,果然就是他!要害得天下英雄自相残杀的人,果然就是他!''他冷笑道:``好一个大仁大义的江南大侠!我早知道你有不可告人的野心,否则你又怎会如此矫情,如此做作?\ldots\ldots 你不但想将天下英雄俱都瞒在鼓里:竟还想将不易收服的人俱都用计除去,好让你独霸天下!''他小心地将一切又重归原位,喃喃又道;``你若不惹我,你的事我本也懒得管的,但谁叫你害得我也上了次大当,我若不教训教训你,岂非对不住自己!''他吹熄了灯,退了出去!将机关也回复原状。

只固他知道此刻就算要揭破江别鹤的阴谋,别人也不会相信的,江别鹤实在装得太好了。所以他只有再等,反正江别鹤是跑不了的。

江玉朗还在沉沉的睡着,甚至连姿势都没有变,他的头埋在枕头里,那副已打开的情锁"也仍挂在手上。

小鱼儿不动声色地上了床,又将手套入``情锁''里,``格''的锁上,此刻他什么都不再想。

他要舒服地睡一觉,养足精神好对付明天的事。但他眼睛还没有闭上,屋子里突然有火光亮起。

小鱼儿一惊,张开眼,便瞧见一个人笑嘻中地站在床头,闪动的火光,照着他苍白的脸,照着他诡秘的笑容\ldots。这人竟赫然是江玉郎!但江玉郎不是明明睡在他旁边么?又怎会站到了床头!小鱼儿跳了起来,再看他身旁的人。

他身旁的人也抬头向他笑,却是那又聋又哑的残废老人\ldots\ldots 小鱼儿怔了半晌,突大笑道:``我明明知道江别鹤是个厉害的人物,怎地还是小估了他?''江玉郎冷笑道:``这也很好笑么?以我看来,你本该痛哭才是。''只见江别鹤缓缓走了进来,含笑瞧着他,柔声道:``你发现了那么重要的秘密本该快快逃走才是,但你居然还能不动声色地回来,你的确有惊人的胆子。''小鱼儿道:``你明明知道我已发现了你的秘密,居然还能不动声色地等我回来,等我再将自己锁起\ldots\ldots 唉,你的确了不起。''江别鹤道:``你小小年纪,居然能骗过了我,居然能找出我的秘密,这实在是我绝未想到的事,的确令人佩服。''小鱼儿道:``你竟能令天下人都相信你是个大仁大义的英雄,竟能令每个人都对你如此尊敬,当真不傀为一代枭雄。''两人你一言我一语,竞互相推崇起来,假如有不相干的人旁边听着,谁也不会猜到他们心里在打什么主意。

江别鹤叹道:``我实在很爱惜你的才智,但你为什么偏偏要来和我作对,你既然知道了那些秘密,我纵然爱惜你,也只有忍痛割爱了。''小鱼儿叹道:``我实在也很爱惜你的才智,很愿意见到你大事成功,但你为什么偏偏要做出那些见鬼的藏宝图来,害得我也上了次当。''江别鹤面上突然微微变了颜色,失声道:``你怎知道那藏宝图与我有关?''小鱼儿道;``若不是那藏宝图,我又怎么来到这里,我又怎会辛辛苦苦地来发掘你的秘密?只要你不惹到我,你的秘密关我屁事!''江别鹤瞧了江玉郎一眼,道;``你什么时候知道的?''小鱼儿笑道:``我瞧见你这犬子身上居然也有张藏宝图,我就问他是从哪里得来的,他说,是从你书房偷来的,那时,我就想,如此重要的藏宝图,你怎能随便放在书房里?那时我心里就已有些疑心。''江别鹤道:``你怀疑得很好。''

小鱼儿道:``我又听人说,这犬子的父亲乃是一代大侠,我又想,常言道:龙生龙,风生风,一代大侠怎会养得出如此卑鄙无耻的儿子。''江别鹤微笑道:``你骂得也很好。''

小鱼儿道;``后来我瞧见你,居然住在这种地方,居然自己搬桌子端莱,身旁只用了又聋又哑的老头子,我又想,这人若不是圣贤,就必定是我从未见过的大奸大恶之徒,因为世上只有这两种人能做出这样的事。''江别鹤笑道:``我自然不太像是圣贤。''

小鱼儿道:``所以我就一心探一探你的秘密。''江别鹤叹道:``你实在太聪明了,这实在是你的不幸\ldots\ldots{}''小鱼儿道:``我若老实些,只怕就能学会装傻了。''江别鹤道:``只可惜你只怕永远学不会了。你可知道今天晚上你并不是唯一想害我的人?''小鱼儿道;``还有谁想害你?''江别鹤道:``昨夜已有人到我卧房里去过了,他先将迷香吹进来,再撬开窗子,显然是要来杀我,只可惜我昨夜并未睡在这里。''小鱼儿道:``不错,你昨夜是和我一起睡在新滩口的客钱里的\ldots\ldots 但你又怎会知道有人曾经进过你的屋子?''江别鹤笑道:``今天我回来时,那屋子里还有残余的迷香气味,窗台上也还留下浅浅的足印,昨夜想来杀我的人,并不是老手。''小鱼儿叹道:``他若是老手,今夜就不会来了。''江别鹤附掌道;``不错,只因他不是老手,所以今夜还会来的。''小鱼儿苦笑道:``所以你就要我睡在你屋子里,代替你被人杀死,你不但可借此杀了我,还可借此捉住那人,那么,你杀他时,还可说是为我报仇,别的人若是知道此事,少不得又要称赞你的仁义。''江别鹤大笑道:``和你这样聪明的孩子说话,当真有趣得很\ldots\ldots 我甚至根本不必说出来,你便已知道我的心意.''

\hypertarget{ux7b2cux56dbux5341ux7ae0-ux51a4ux5bb6ux8defux7a84}{%
\chapter{第四十章
冤家路窄}\label{ux7b2cux56dbux5341ux7ae0-ux51a4ux5bb6ux8defux7a84}}

小鱼儿果然被送到江别鹤卧房的床上。

``情锁''还是他自己打开的,但锁一开,他身上``肺俞''、``心俞''、``督俞''、``脯俞''、``肝俞''、``胆愈''、``脾俞''、``叁熊俞''等八处穴道,立刻就被江别鹤一一点遍。

现在,他睡在床上,腿睁睁瞪着屋顶,心里索性什么也不去想,反而在数着绵羊,一只两只\ldots..但他直数到八千六百五十四只,眼睛还是睁得大大的。

他数着绵羊,心里不由得就想到桃花,想到桃花那红红的、像是苹果般的脸,于是他立刻又想起铁心兰。他从来不知道人类的联想力竟是如此奇怪,你越是不愿意去想一个人,那人总是偏偏会闯入你心里来``铁心兰此刻在哪里?也许正在和那温文风雅的无缺公子开心地谈着话,但我却在这里等死。''小鱼儿闭上眼睛,拼命令自己不要去想她,但铁心兰偏偏还似在他眼前,穿着一身雪白的衣服,站在灿烂的阳光下。这就是他第一眼瞧见她时的模样。

若不是铁心兰,他又怎会得到那见鬼的``藏宝图'',若不是那``藏宝图'',他又怎会来到这里?

他再去数绵羊\ldots\ldots 八千六百五十五\ldots\ldots 八千六百五十六\ldots\ldots 但一只只绵羊的头,竟都变成了铁心兰。

突然间,窗外轻轻一响。接着,便有一阵淡淡的香气飘了进来。

小鱼儿立刻屏住了呼吸,暗道:``来了,终于来了,江别鹤果然算的不错\ldots\ldots 唉,我连手指都不能动,屏住呼吸又有什么用?''他大半个脸都埋在枕头里,只露出半只眼睛。他就用这半只眼睛往外瞧。

只见窗子轻轻开了一线,接着,一条人影闪身而入。这人穿着一身黑色的紧身衣,手上拿着柄闪亮的柳叶刀,行动显得十分轻灵矫捷,而且胆子也真不小。

刀光忽然闪亮了她的脸。小鱼儿恰巧瞧见了她的脸,他立刻骇呆了。这大胆的黑衣刺客,竟是铁心兰!

世上怎会有这样巧的事莫非是小鱼儿看花了眼但他看的实在不错,这人的确是铁心兰。

她一闪进屋子,瞧见床上有人,就也不瞧第二眼,一步窜到床前,一刀向床上的头颅砍了下来。小鱼儿既不能动,也不能喊,心里更不知是什么滋味,他竟要死在铁心兰手里,这岂非是老天的恶作剧!

江别鹤父子就在门外偷偷地瞧着,只待她这一刀砍下,他们立刻就要冲进去──这一刀眼见已砍下去了!小鱼几的头颅见已要离开脖子!

哪知就在这时,突听``格''的一声,铁心兰手里高举着的柳叶刀,竟突然奇迹般一断为二!

江别鹤父子俱都吃了一惊:``是谁有这等身手?''铁心兰更是面无人色,后退两步,似欲觅路面逃。这时窗外已飘入一条人影,就像是被风吹进来的─朵云。淡淡的星光照进窗户。

星光下,只见这人身上穿着件轻柔的白麻长衫,面上带着丝平和的微笑,在淡淡的星光下,看来仿佛是天上的神仙,从头到脚,都带着种无法形容的摄人魅力,但谁也说不出他这种魅力是从哪里来的。

江别鹤竟也不觉被他这种风雅而华贵的气质所摄,竟怔在门外,再也想不起武林中哪有这样的少年。小鱼儿却一眼使认出了他,更几乎晕了过去。

他自然就是世上所有人类最完美的典型──无缺公子。

铁心兰又不禁后退两步,嘶声道:``是你?你\ldots\ldots 你怎会来的?''无缺公子微微笑道;``自从前天你苦心讨来这鸡鸣五鼓返魂香,我就觉有些怀疑,所以这两天来,我一直在暗中跟着你。''铁心兰轻轻跺脚道:``你为什么要跟着我,你为什么要阻拦我杀他?''无缺公子柔声道:``江湖小人人都说江南大侠是位仁义的英雄,你纵然对他有些气恼,也不该如此杀了他。''铁心兰颤声道:``你\ldots\ldots 你知道什么?你可知道他\ldots\ldots 他杀死了我爹爹!''这时,江别鹤终于推门走了进去,满面俱是惊奇之色,像是对什么事都不知道似的.抱拳笑道:``两位是谁?\ldots。·在下平生从未妄杀一人,又怎会杀死姑娘的爹爹,姑娘只怕是对在下有所误会了。''铁心兰眼睛都红了,厉声道:``我爹爹明明留下暗号,告诉我他要来寻你,但到了这里后,使未曾再出去,难道不是被你害死在这里''江别鹤道:``这位姑娘是\ldots。.''

铁心兰大声道;``我姓铁,我爹爹便是狂狮铁战!''江别鹤笑道:``原来是铁姑娘,但在下可以名誉担保,铁老先生确未来过此间,姑娘不妨仔细想想,在下若真的杀了铁老先生,那是何等大事,在下纵要隐瞒,江湖中也必定有人知道的,何况,在下也未必就想隐瞒的。''``狂狮''铁战乃是``十大恶人''之一,江湖中想杀他的人,本就不只一人,若有人杀了他,非但人人称快,而且人人都要称赞几句,江别鹤这番话虽然说的话中带刺,但却大有道理。

铁心兰正和她爹爹一样,是个毛栗火爆的脾气,虽然寻来拚命,但她爹爹究竟是否死在这里她却根本未弄清楚。此刻她听了这番话,心中虽然气恼,却也反驳不得。

江别鹤已向无缺公子抱拳笑道:公子人中龙凤,在下走动江湖数十年,却也从未见过公子这样的人物,不知可否请教尊姓大名?``无缺公子微笑道:''在下无缺,阁下\ldots\ldots"

江别鹤长揖道:``在下便是江别鹤。''

铁心兰突又跳了起来,大声道;``你是江别鹤,那么床上的又是谁?''江别鹤暗笑道:``这女子看来秀气,其实却只怕是个鲁莽张飞,竟直到此刻才问床上的是谁。\ldots.''心念转动,人已走到床边,拍着小鱼儿道:``此乃在下故人之子,今日远道而来,是以在下便将卧榻让给他\ldots\ldots 贤侄快快醒来,见过花公子。''手掌拍动间,他已解开了小鱼儿的穴道,但却又轻轻按在死穴之上,只要小鱼儿说出一个字对他不利,他手掌一用力,小鱼儿第二个字便再也说不出了。

小鱼儿仍埋在枕头里,突然憋着喉咙道:``我早已醒了,只是懒得和他们说话而已。''江别鹤故意皱眉:``你怎可如此无礼?''

小鱼儿道:``江湖中谁不知道你老人家大仁大义的英雄,但他们却要赖你老人家胡乱杀人。这种不明是非的人,我和他有什么好说的。''江别鹤本道小鱼儿纵然被挟,最好也不过开口而己,哪知小鱼儿竟为他辩白起来,这倒是他未曾想到的事。

突听铁心兰失声道:``你\ldots\ldots 你\ldots\ldots{}''瞧了无缺公子一眼,突然一笑,柔声道:``你既没有杀死我爹爹,也就算了,我们走吧。''却不知小鱼儿虽然憋住嗓子,但铁心兰对他朝思夜想,时刻未忘,又怎会听不出他的声音。

她心中正自惊喜交集,突又想到无缺公子若是知道小鱼儿在这里,小鱼儿还有命么?是以立刻拉着花无缺就走。

这几人关系当真是复杂已极,江别鹤纵然是个聪明人,一时之间,却也难以弄得清,反而笑道:``花公子既来寒舍,怎可如此匆匆而去\ldots\ldots{}''花无缺笑道:``在下也久闻江南大侠名,正也要多领教益,只是\ldots\ldots{}''小鱼儿见他要走,本已在暗中谢天谢地,此刻突又所他有留下来的意思,一急之下,忍不住大声道:``只是你若真的要见我江老伯,本该等到明日清晨,再登门拜访,叁更半夜的越窗而来,成何体统?''花无缺面色突然一变,沉声道:``你究竟是什么人?''铁心兰拼命拉他袖子,道:``管他是谁,咱们快走吧。''她直将花无缺放出窗子,才松了口气,哪知眼前人影一花,花无缺已不见了,再瞧他人已到了小鱼儿的床头。

小鱼儿整个头都埋在枕头里,心里不住骂自己该死,江别鹤见花无缺却面复返,更是莫名其妙。

只见花无缺面沉如水,一字字道:``此人可是江鱼?''江别鹤怔了怔,强笑道:``公子可是认得我这位贤侄?''花无缺长长吐了口气,展额笑道:``很好,好极了,你居然没有死。''江别鹤见他如此欢愉,却也想不到他欢喜的只是为了可以亲手杀死小鱼儿,还当他必是小鱼儿的好友,当下笑道:``他自然不会死的,谁若要害他,在下也不会答应。''花无缺悠悠道:``你不答应?''

江别鹤见他神色有异,心里正奇怪,小鱼儿已跳了起来,躲在他背后,向花无缺做了个鬼脸,笑道:``谁若想杀死江南大侠的贤侄,岂非做梦。''花无缺缓缓道:``在下对江南大侠虽然素来崇敬,但却势必要杀此人,别无选择!''江别鹤又是一征,失声道:``你\ldots\ldots 你要杀他?''花无缺叹了口气,道:``在下委实不得不杀。''江别鹤瞧了瞧小鱼儿,不禁暗道一声;``糟,我终于还是上了这小鬼的当了。''要知他话既已说到如此地步,以他的身份地位,那是无论如何也不能眼看别人在他面前杀死他``贤侄''的。

小鱼儿瞧他神色,心里真是开心得要命,口中却叹道:``江老伯,你就让他杀死我吧,这人武功高得狠,反正你老人家也不是他的教手,江湖中人也不会耻笑你老人家的。''江别鹤暗中几乎气破了肚子,面上却微笑道:``花公子当真要令在下为难么?''花无缺沉声道:``阁下但请叁思。''

突然间,江玉郎捂着肚子冲进来,面色苍白得可怕,身子也不住颤抖,指着小鱼儿道:``他\ldots\ldots 他送来的酒中有!''江剑鹤面色也立刻惨变,回身瞪着小鱼儿,厉声道:``我父子待你不薄,你\ldots\ldots 你为何要来害我。\ldots 难怪你自己一滴不尝,原来你竟在酒中下了毒!''这变化不但大出花无缺意料之外,连小鱼儿也怔住了。

但他立刻便又恍然,不禁暗骂:"好个小贼,好阴损的主意这主意的确是个高招,情况一变,变得连江别鹤父子自己都要杀他了,自然再也用不着阻拦花无缺。

只见江别鹤突然自怀中拔出那柄宝剑,怒骂道:``我待你如子如侄,不想你竟为了这区区一柄剑便要置我于死地,你\ldots\ldots 你这种忘恩负义全无天良之人,若是容你活下去,还不知有多少人要死在你手里,我岂能不为世人除害!''手腕一抖,短剑直刺小鱼儿的胸膛。

哪知他剑方刺出,花无缺已轻轻托住了他的手腕。

江别鹤又是一惊,既惊于这少年出手之快,更不知这少年为何又反过头来阻拦于他,失声道:``公子你。\ldots 你为何\ldots\ldots?''花无缺道:``抱歉得很,在下必须亲自动手!''他突听江玉郎惨呼一声,倒在地上。

江别鹤也立刻捂住肚子,惨笑道,``既是如此,在下\ldots\ldots 在下''话未说完,倒退几步``噗''地坐倒椅上。

花无缺叹了口气,自怀中取出个小小的玉瓶,送到江别鹤手里,道:``这仙予香与素女丹─外敷,一内服,可解世间万毒,阁下但请自用,恕在下不能亲自为贤父子效劳了。''他虽有行动,虽在和别人说话,但目光却始终眨也不眨地盯在小鱼儿身上,他已尝过小鱼儿诡计的滋味,这一次哪敢有丝毫大意。

小鱼儿也知道自己这一次只怕是休想再能跑得脱的了,索性盘起双腿,坐在床上,笑嘻嘻地瞧着他道:``我居然没有死,真该恭喜你才是。''花无缺一笑道:``不错,你居然未死,实乃我之大幸。''小鱼儿笑道:``你自信这一次真的必定能杀死我?''花无缺道:``这一次你纵然再想自杀,也是绝无可能的了。''小鱼儿扬了扬眉,道:``哦?''

花无缺缓缓道;``在这样的距离之内,无论任何人的手只要一动,我便可先点下他左右双臂一十八处穴道。''他淡淡说来,就像是在说一件最简单最轻易的事,但小鱼儿却知道他说的绝没有半句假话。

窗外,铁心兰突然将柳叶刀弹得``叮叮''作响,她这柳叶刀本是鸳鸯两柄,断了一柄还剩下一柄。

小鱼儿眼珠子一转,笑道:``你可敢让我自己走出去?''花无缺微微一笑,道:``你想你能逃得了么?''小鱼儿笑道:``你何必多心,我只不过是不愿意被你抱出去而已。''他一跃下床,瞧了江别鹤父子一眼,若是别人,此刻少不得要大声揭破这父子两人的奸谋。但小鱼儿却细道那不过是白费气力,他说的话花无缺根本连一字也不会相信。那是个很老式的窗子,小鱼儿摇摇摆摆地一脚跨了出去,他瞧着铁心兰,铁心兰也在瞧着他,那双美丽的眼睛里究竟含蕴蓄多么复杂的情感?这只怕谁也分不清。

柳叶刀仍被她弹得``叮叮''直响,夜风中已颇有寒意。

小鱼儿笔直向前走,也不回头去瞧花无缺,他知道花无缺必定不会离他很远的,他再瞧也是没有用。他摇摇摆摆走过铁心兰身旁。

突然间,刀光一闪,柳叶刀向小鱼儿身后直劈过去。

刀是劈向花无缺的,花无缺就算有天大的本事.也得先闪避──铁心兰刀法也算一流高手。刀光闪处,小鱼儿己向前一跃面出。

只听铁心兰叱道:``接住''\ldots."

哪钢刀在半空突听``叮''一声,剩下的这柄柳叶刀也突然奇迹般折为两段,自空中直跌下来。

花无缺已又到了小鱼儿身后,道:``你还要往前走么?''他语声仍是那么平和,面上也仍然带着微笑,就像是什么事都没有发生过似的,更绝不去瞧铁心兰─眼。他若去瞧铁心兰,铁心兰怎有颜面见他,他一生中绝不会伤害任何一个女孩子,何况这女孩子是铁心兰。

小鱼儿叹了气,只得再往前走。

他走了几步,忽然叹道:``你对女孩子可真不错。''花无缺笑道:``这是我从小的习惯。''

小鱼儿道:``假如那女孩子很丑呢?''

花无缺道;``只要是女孩子,就全是一样。''

小鱼儿笑道,我真想找个很丑很丑的女孩于来\ldots\ldots 癞痢头、帚把眉、葡萄眼、塌鼻子、缺嘴巴,再加上大麻子\ldots\ldots 我倒要瞧你对她如何?``花无缺道:''抱歉得很,你只怕没有这机会了。``小鱼儿忽又叹了口气,道:''这实在是件令人很难想象的事,你要杀一个人时,居然还能不慌不忙地和他谈笑聊天,这\ldots\ldots 这简直不可思议。``花无缺淡淡笑道:''聊天和杀人,完全是\ldots。.``小鱼儿苦笑道:''完全是两回事,是么?"

花无缺道:``不错,我自己要和你聊天,但我得的命令却要我杀了你,所以这完全是两回事,互相绝没有关系。''小鱼儿四道:``我真不懂,你怎能将这两件事分开的?''花无缺道:``这是我从小所得的教训。''

小鱼儿道:你真是个听话的孩子。"

花无缺笑了笑,道:``你还要往前走么?''

小鱼儿苦笑道:``你要杀我,不是我要杀你,你并不需要征求我的意见。''花无缺缓缓道:``那么\ldots\ldots 就在这里停下吧。''小鱼儿四望一眼,淡淡的星光下,远处龟山巨大的山影朦胧,近处垂杨的枝条已枯萎──。

小鱼儿喃喃道:``奇怪,江南的秋,怎会来得这么早,我江鱼又怎会死得这么早?\ldots\ldots{}''直到花无缺等人俱已去远,江玉郎才跳了起来。

江别鹤也坐直了,瞧着他笑道:``想不到你应变的机智竟还在我之上。''江玉郎垂首道:``孩儿怎及爹爹,孩儿只不过是\ldots\ldots{}''江别鹤叹道:``你在你自己爹爹的面前,并不需要太用心计,就算你智计强胜于我,我难道还会对你怎样不成?''江玉郎道:``是。''

江别鹤抚摸着那玉瓶,皱眉道:``仙子香,素女丹,\ldots\ldots 想不到那花无缺竟是移花宫的弟子,此人出现江湖,我倒要留意些才是。''江玉郎道:``他武功虽高,但却完全不懂事,又有何可怕?''江别鹤叹道:``此人大智若愚,又岂是你所能揣测。''江玉郎笑道:``但那位铁姑娘,却的确有些大愚若智,不过.\ldots{}''她爹爹是否真的没有来过这里?你老人家是否真的没有杀他?``江别鹤冷冷一笑,道:''我虽然真的没有见到过狂狮铁战,但像她那样的女孩子,说出来的话却很少会有假的。``江玉郎皱眉道:''她既没有说假话,而你老人家又真的没有见过狂狮铁战,那么,这究竟是怎么回事?``江别鹤沉声道:''这就是说,狂狮铁战虽然来过,但却改扮成另一种模样,而我竟一时疏忽,没有认出他来。``江玉郎道:''但\ldots\ldots 但那女子又说她爹爹到了这里后,便未曾出去。``江别鹤悠悠道:''不错,他此刻或许在这里。``江玉郎动容道:''在这里?"

江别鹤冷笑一声,长身而起,冷冷道;``你莫要忘记,此间除了我父子之外,还有一个人的。''江玉郎失声道:``你老人家是说那老聋子?''

江别鹤冷笑道;``他难道不能装得又聋又哑么?''江玉朗道:``但你老人家曾经偷偷从他背后走过去,在他耳畔把那面大锣敲得山响,我从前面看,他真的连眼睛都没有眨一眨。''江别鹤道:``有定力的人,纵然山崩于前,也不会眨一眨眼睛的。''江别鹤立刻放低了语声,道;"你老人家可知道此刻他在哪里?说不定已经逃走了也未可知。

江别鹤却放大了声音,厉声道:``他以为我不会怀疑到他,所以必定尚未逃走,此刻我父子只要瞧见了他,就立刻将他杀死,绝不要再给他说话的机会,宁可错杀一百好人,也不要漏掉一个奸细!这句话你切切不可忘记!''江玉郎听他声音说得这么响,心里不禁大是奇怪!

``那老头子若非聋子,听见这话岂非要跑了么?''但转念一想,立刻又恍然!

"爹爹想必已知道他就在附近不远,他若骇得跑了,岂非便可证明他就是狂狮铁战,那时再追也不迟\ldots\ldots{}

只见江别鹤``砰''地一声,推开了门!

\hypertarget{ux7b2cux56dbux5341ux4e00ux7ae0-ux6d41ux6d6aux6c5fux6e56}{%
\chapter{第四十一章
流浪江湖}\label{ux7b2cux56dbux5341ux4e00ux7ae0-ux6d41ux6d6aux6c5fux6e56}}

门外是条走廊,走廊的尽头有间小屋,屋里有炉火,火上烧着壶水,老人正蹲在壶边,等着水沸。他动也不动地蹲在那里,显得那么安详,那么宁静。

他这一生中已``等''了多久?还要``等''多久?对于``等''他自然比少年人有更多的忍耐。

江别鹤厉声道:``很好,你装得很像,但无论如何,我还是要你的命!''他一步窜过去,手掌向老人顶门直击而下。

老人却抬起头来,向他一笑,指着炉子的水壶,像是在说:``水开了,我就替您沏茶。''江别鹤这只手掌终于只轻轻落在他肩上,这老人若是听见他说的一个字,笑容又怎会如此安详。

淡淡的星光,照在花无缺脸上。真是张毫无瑕疵的脸。天下少女们梦里所幻想的白马王子,就该是这模样。

小鱼儿瞧着他,忽然笑道:``你知道么,你无缺这名儿的确取得很好,你的确没有什么缺憾\ldots\ldots 你出身于世上名声最响的武林圣地,你少年英俊,不虑钱财,你的武功可使江湖中每一个人都对你恭恭敬敬,你的美貌、谈吐和风神,又可使天下每一个少女都对你着迷,你的名誉也无懈可击,令人甚至在背后都不能骂你。''他摇着头笑道:``天下若真有一个完美无缺的人,那人就是你。''花无缺微微笑道:``多谢夸奖。''

小鱼儿悠悠道:``但我却忽然发觉,你还是少了样情感,你彻头彻尾是个没有情感的人,你身上流的血,只怕都是冷的。''花无缺淡淡一笑,道:``是么?''

小鱼儿大声道;``你不服么?好,我问你,你可真的懂得什么叫爱,什么叫恨?你可曾尝过爱的滋味?恨的滋味?''他一步步往前走,接道:``你甚至连烦恼都没有,老、病、愁闷、贫苦、失望、悲伤、羞悔、恼怒\ldots\ldots 这些本是全人类都不能避免的痛苦,但伤却一样也没有\ldots\ldots 一个完全没有痛苦的人,又怎能真正领略到欢乐的滋味。,他长叹了一声,缓缓接道:''你既没有真正爱过一个人,也没有真正恨过一个人,你没有痛苦,也没有欢乐\ldots\ldots 别人也许都羡慕你,我却觉得你活着实在没有什么意思。``花无缺默然半晌,神色竟还是那么安详,绝没有任何变化,他只不过是淡淡笑了笑,道;''也许你说得不错,这只怕也是我从小的环境造成的。``小鱼儿苦笑道:''不错,只有移花宫才能造出你这样的人,使你变成个活动的木头人。你虽然对每个人都谦恭有礼,但心里却绝不会认为他们值得尊敬,你虽然对每个女孩子都温柔体贴,但也绝不是真的喜欢她们。``他又长叹一声,道:''就算你要杀人,你心里都未必认为他是该杀的。``花无缺叹道:''这的确是遗憾得很。"

小鱼儿仰天一笑,道:``好,现在我话已说完了,你只管动手吧,我倒要看看,你到底能在几招内将我杀死!''花无缺道:``你可要使用兵器?''

小鱼儿道:``我没有兵器。''

花无缺柔声道:``你若愿使用兵器,我可以陪你到有兵器的地方,让你选择─样。''小鱼儿苦笑道:``你明明知道我纵有武器,也非你敌手,你明明要杀死我,还要对我如此客气,若是别人,必定要认为你是个阴险毒辣的人,但我却知道你不是,因为你连虚伪作假都不会,因为你根本不必作假。''花无缺道:``你实在很了解我。''

小鱼儿道:``你再想找一个这么了解你的人,只怕很难了。''花无缺叹道,``不错。''

小鱼儿抹了发干的嘴唇,道:``我不要用兵器,你动手吧。''花无缺仰头瞧了一眼,秋风吹过,一片枯叶飘落了下来,星光更淡了,大地充满了萧瑟之意。

他叹了一声,悠悠道:``这样的天气\ldots\ldots 小鱼儿接道:''这样的天气,的确很适于杀人。``突听铁心兰冷冷道:''这样的天气,只令我觉得冷得很``\ldots.''她突然走过来,身上竟已是完全赤裸着的!

星光,柔和地洒了她全身。

世上绝对无法再找出一样比这赤裸的少女胴体更美、更眩目的东西来,简直美得令人窒息。一瞬间,小鱼儿和花无缺呼吸都为之停顿。

花无缺颤声道:"你\ldots\ldots 你\ldots\ldots。

铁心兰转身面对着他,悠悠道:``你看我美么?''她起伏着的胸膛,在月光下看来是那么苍白。

花无缺不由自主闭起了眼睛,道:``你\ldots\ldots 你为什么要\ldots\ldots{}''他刚闭起眼睛,铁心兰已扑上去紧紧地抱住了他。

花无缺只觉得一个冰冷的、柔滑的身子,缠住他的身子,他的心房突然猛烈地跳动,手足也颤抖起来。

他一生中从未有这种感觉,他仿佛要晕迷、爆烈\ldots\ldots 他根本不知该如何是好。

铁心兰额声道:``死人,你\ldots\ldots 你还站在这里?''小鱼儿站在那里,像是已发了呆。

铁心兰嘶声道:``你这样\ldots\ldots 你还不走?''

小鱼儿目中突然流下泪来。

这几乎是他平生第一次流泪,他也不知道这是感激的泪?是悲伤的泪?是恼怒的泪?还是羞愧的泪?

花无缺的手根本不敢去碰铁心兰的身子,自然也挣不脱她,额上已有了汗珠,只有连声道:``放手\ldots\ldots 放手!\ldots\ldots{}''铁心兰也是流泪满面,道:``你\ldots\ldots 你再不走,我就死在你面前!''小鱼儿道:``我\ldots\ldots 我\ldots\ldots{}''

他最后瞧了铁心兰一眼──那无辜而纯洁的胴体,那满脸晶莹的泪珠,这必将令他永生不能忘怀。他狂吼一声,发疯似的转身奔了出去。

小鱼儿像一条负伤的野兽,在这秋夜中的原野里狂奔着,也不知究竟奔出了多远,更不知已奔到何处?

他已再没有眼泪可流,他的心乱得就像是他的头发,他一生中从没有这样痛苦这么心乱过。

水田里的稻穗已长出,在晚风中像是大海的波浪,小鱼儿奔入一块稻草中央,在星光下躺了下来。

积水的污泥,浸着他的身子,星光自稻穗间望出去,显得更遥远,更不可捉摸。

他暗问自己:``我能算是个人么?''

``我自以为谁都比不上我,我瞧不起任何人,但别人要杀我时,我却连一点法子也没有。''``我瞧不起女人,尤其是铁心兰,只因我知道她爱我,所以就拼命令她伤心,但到头来都要她牺牲自己来救我!''``我自以为是天下第一个聪明的人,但此刻却像条狗似的被人追逐,像条狗似的夹着尾巴逃。''``我这次虽然逃脱了,但我这一生中难道都要这样逃么?我这一生中难道都要等别人来救我?''``不错,花无缺的计谋也许不如我,但像他这样的人,又何必再用什么计谋?只因他有真实的本事。''``而我\ldots\ldots 我都只想靠聪明、靠运气\ldots。.一个人若只有聪明,而没有本事,那又有什么用?''``我自以为连恶人谷里的人都怕我,所以觉得很了不起,却不知他们怕我,只不过是像父母怕一个顽皮的孩子似的,若是真的动手,我能强得过屠娇娇?李大嘴?血手杜杀?\ldots\ldots{}''小鱼儿就这样躺在水田里,反反复复地想着。

小鱼儿终于爬了起来,他身上满是污泥,脸上也满是污泥,他也不管,只是沿着田埂往前走。

前面有烟火点点,仿佛是个村镇市集。一家小客栈旁的空地上,团聚着一群人,里面锣鼓打得``叮咚''直响,红纸大灯笼也在风中直晃。

这自然是个走江湖的戏班子。

小鱼儿走到前面,蹲下来,一个穿着红衣服,扎着两根小辫子,眼睛大大的女孩子正在那里走绳索。另外还有大大小小、老老少少几个人,有的在旁边舞刀,有的在翻筋斗,有的在打锣,有的在敲鼓。

小鱼儿只是蹲在那里,眼前演着什么,他根本没有看,他只觉得很萧索,只是想看看人们的笑容。也不知过了多久,他模模糊糊感觉到有人欢呼,有人拍手,还有钢钱落在地上的叮叮声响。

然后人群散去了,走江湖的在收拾着家伙,那个穿红衣服的女孩子却像是个公主似的,只是坐在那里喝水。她皱着眉瞧了小鱼儿一眼,那双大眼睛里闪着光,突然从怀里摸出了个铜板,抛在小鱼儿面前,立刻又扭转过去。

戏班子也走了,穿红衣的小姑娘昂着头走过小鱼儿旁边,像是没有在意,伸脚轻轻踢了踢,将那铜板踢到小鱼儿脚下。

这是多么善良的人们,瞧见了别人的穷困,就忘记了自己。

大人们在笑着,讨论着今天的收获可以买多少肉,打多少酒,至于明天──明天是另一个日子,他们用不着去为明天烦恼,明天纵有不幸的事,纵然没有饭吃,且等到明天再去烦恼,今天先喝了酒再说。

这又是多么豁达的人们──小鱼儿此刻想过的,正是这种只有``今天''、没有``明天''的日子。

他捡起了那铜钱,跟在他们后面走,前面不远,就是江岸,江岸停着一艘船,这就是他们的家。

一个蓝布衣裤,敞着衣襟,露着紫铜色胸膛的虬髯老人正在指挥着人将兵刃家伙搬上船去。

他年纪虽已必在六十开外,但身子却仍像少年般健壮,他生活虽然落魄,但钟情间却自有一般威严。

这想来必是戏班子的主人了。

小鱼儿突然赶过去,恭恭敬敬作了个揖,道:``老爷子,我也跟着你走江湖好么?''那老人瞧了他一眼,笑了,摇头道:``走江湖可不是好玩的,要有本事,还得不怕吃苦。''小鱼儿想了想,道:``我不怕吃苦,我会翻筋斗。''老人大笑道:``翻筋斗?干咱们这行的谁不会翻筋斗,翻筋斗原是最简单的玩意几\ldots\ldots 野犊子,你就翻几个让他瞧瞧。''一条浓眉大眼的结实少年笑嘻嘻地走了出来,一挽袖子,也没摆什么姿势,就一连翻了七八个筋斗。

小鱼儿眨了眨眼睛,道:``你最多能翻几个?''那野犊子笑道:``大概二三十个吧。''

小鱼儿道:``但我却可以翻一两百个。''

那老人笑道:``哦!能一口气翻八十筋斗的人,我少年时倒见着一个,那就是李家班头李老大,自从他挨了一刀后,就再没有别人了。''小鱼儿道:``但我却能翻一百六十个。''

老人大笑道:``你若真能翻一百六十个\ldots\ldots 不,只要能翻八十个筋斗,这行饭就能吃上个一辈子了,虽没有什么好的吃,但也有酒有肉。''他话末说完,小鱼儿已翻起筋斗来。

他一身铜筋铁骨,武功虽不能和绝顶高手可比,但翻起筋斗来,那可当真比吃豆子还容易.等他翻到三十个,大家都已围了过来,他翻到六十个时,大家都已在喝彩.在为他打气。

等他翻到八十个时,大家都已瞪大了眼珠,连喝彩都忘了,那穿红衣服的少女大眼睛的光也就更亮了。

小鱼儿直翻了一百多个,才算停住,笑道;``够了么?''老人附掌大笑道:``够了,够了\ldots。太够了,快跟着野犊子上船去,洗个脸,换件衣裳.等着吃宵夜吧,从今天起,你就是咱们海家班的人了。''小鱼儿垂头道:我爹爹妈妈刚死没多久,我在他们坟前发过誓,为他们守三年丧,我\ldots\ldots 我发誓说这三年绝不洗脸。``老人叹了口气道:''可怜的孩于,想不到你还这么孝顺``\ldots 我的孩子们叫我四爹,以后,你也叫我四爹吧。''于是小鱼儿就在这走江湖、玩杂耍的"海家班留了下来,每天翻筋斗,过着新奇即又平凡的日子。

他现在已知道这班子里的人差不多都是海四爹的子侄儿子,野犊子是他的六儿子,也是功夫最好的一个。那穿红衣裳的小姑娘,却是这班子的台柱,她叫海红珠,是海四爹在五十大寿那天生的小女儿,除此之外,他知道的就不多了。

除了翻筋斗,别的事他几乎全都不管,每天除了吃饭、睡觉、翻筋斗外,他就是坐在那里发楞。

谁也不知道他发楞的时候,正是在寻思着武功中最最奥秘的诀窍,普天之下几乎没有几个人懂得武功诀窍。

那本牺牲了无数人命才换得的武功秘笈,他早已背得滚瓜烂熟,他想通了一点,等到晚上别人都睡着了时,就偷偷在江岸无人处去练,别人只觉得他有些奇怪,有些傻,但也没有人去管他。

他翻筋斗的玩意儿既十分叫座,又从不想分银子,他就算有点奇怪,有些傻,甚至有些懒,别人也都可原谅了。

现在,他不再是天下第一个聪明的人,现在,别人都叫他海小呆。

飘泊的人们,终年都在飘泊,从长江这头到那头,从东到西,从南到北,小鱼儿也不知道究竟到过些什么地方。

这一天,船又靠岸了,他正坐在船舷洗脚,背后突然伸过来一只白白的、小小的手,递给他一个桔子。

他接过来剥了就吃,也不回头。海红珠站在他身后,等了很久,他不回头,她只有走过来,在他旁边坐下,也脱了鞋子,在江水中洗脚。

那是双白白的、小小的脚,脚踢起了水花,溅了小鱼儿一身,但小鱼儿却动也不动,也不说话。

海红珠瞟了他一眼,突然``噗哧''一笑,道:``你既然不理我,为何又吃了我的秸子?''小鱼儿道:``我不会说话。''

海红珠笑道:``你不会说话?你难道是哑巴?''小鱼儿冷冷道:``我不配和你说话。''

海红珠柔声道:``你不配,谁说你不配?\ldots。.''她灵活的大眼睛俏巧地转动着,抿着嘴一笑,道:``别人都叫你小呆,但我却知道你是聪明人。不但聪明,而且比别的人都要聪明得多,是么?''小鱼儿现在最怕听的,就是别人说他聪明。

他一皱眉站起来,转头就要走,但这时他突然瞧见一群人,他立刻怔住,就像是被钉子钉在地上,整个人都不能动!

江岸上,正有一群人,踏着青青的草地,谈笑着走了过来,他们穿着鲜艳的、轻柔的春衣,他们面上的笑容是那么开朗而欢愉,春风轻抚着他们的春衣,阳光是那么温暖,而他们正年少!

生命是可爱的,有什么事能令他们忧虑?

这欢乐的一群,正有着小鱼儿最不愿见到的人,那正是花无缺、铁心兰、慕容九和江玉郎。

江玉郎居然也和他们在一起!

此刻,一群衣着鲜明的人正围着花无缺,陪着笑,献着殷勤,他无疑正是这一群人的中心。

但他的笑,却多半是为他身旁的两个娇艳的少女而发的──铁心兰也在笑着,面上似乎充满了幸福的光采。

小鱼儿的心,火一般地燃烧起来。

他平生第─次真正感觉到嫉妒的痛苦,他如今才知道这痛苦竟是如此强烈,竟似要将他的心都揉碎。

海红珠奇怪地瞧着他,再瞧瞧这群人,她似乎已感觉到小鱼儿的悲哀与痛苦,幽幽又道:``我知道你的身世一定有很多秘密,是么?''小鱼儿根本没有听到她的话。

现在,他又瞧见了一身淡绿衣衫的白凌霄。白凌霄正在和花无缺低声谈笑,笑得很愉快。

奇怪,花无缺怎能忍受如此庸俗浅薄的人?"\ldots 唉!花无缺原是什么人都能忍受的,因为他根本末将任何人瞧在眼里,对他说来,世上所有的人全都差不多,他根本不必为他们生气。

海红珠咬着嘴唇,低声道:``你认得他们?\ldots。我知道,你原中是属于他们那一群人的,绝不会属于我们\ldots\ldots 我们,只不过是一群卑贱而可怜的人。''小鱼儿渐渐地往后退,退入了船舱投下的阴影。

他发现铁心兰似乎正在瞧他。

但这只不过是她不经心的一眼而已,她又怎会真的注意─个如此龌龊如此卑贱的少年。

但小鱼儿却不能不注意她,她已长大了些,就像是朵含苞待放的牡丹,既华贵,又娇艳。

而慕容九却更消瘦,瘦得像朵菊花,虽然没有牡丹的娇丽,却另有一种淡淡的幽香,令人沉醉。

她的眼睛也更大了,但眼睛里已失去了往昔那种锐利的光芒,却换了种朦胧的忧郁,她在为什么忧郁?

海红珠轻轻走到小鱼儿面前,目中的忧郁也正和慕容九一样,她幽怨地瞧着小鱼儿轻轻地道:``我现在才知道你为什么不理我,只因我不配和你说话,是么?我又怎比得上那两个女孩子,她们是那么高贵,而我\ldots\ldots{}''小鱼儿突然一把将她搂过来。将灼热的嘴唇重重印在她的嘴唇上,他的血已沸腾,他需要发泄!

在这一刹那间,海红珠只觉天地都已在她面前崩裂。她闭起眼睛,什么都感觉不到了。

她只觉自己似已投身于一团灼热的火焰中,全身也已燃烧起来,烛全身都已融化,灵魂也已融化。这一刹那,已将她的生命全都改变。

但这在别人眼中看来,又是多么不值得重视的小事,岸上的人指点谈笑着,渐渐远去了。小鱼儿突然推开了她,跃下了船舱!

她痴痴地怔在那里,似已永远不能动了,春风仍然吹得很暖,但她的心却开始一寸寸结成冰。

她仍然闭着眼,不敢睁开,她怕那令人迷乱狂醉的美梦在她眼前粉碎,但是她长长的睫毛上已出现了一滴晶莹的眼泪。

夜已深了,谁也不知道夜是何时来的。海红珠更不知道,她几乎什么都不知道了。

灯笼已亮起,人群已聚拢,海四爹已开始用他那独特的豪爽笑声,在大声说着一些吸引人群的话。

无论她有了多大的改变,但生活却必须继续。于是,海红珠又跃了上绳索。

她麻木地在绳索上走着。人群的欢笑声,拍掌声,却似乎已距离她十分遥远,十分遥远"\ldots 只因她的心,已飞驰到远方。

那地方永远春光明媚,在那地方,人们永远能和自己心爱的人守在一起,永远不必再装出卑贱的笑脸。

小鱼儿蹲在兵器架后,他的心也已飞驰到远方,眼前所有的事,他也是什么都瞧不见\ldots\ldots 突然,人群中一声惊叫。海红殊竟自高高绳索上直跌下去!

海四爹、野犊子面色立刻惨变,但却仍要强笑着大声道:``人有失手,马有失蹄,这算不得什么\ldots\ldots 小姑娘,站起来吧,再露两手给爷儿们瞧瞧!''但这时人们的惊呼已变为喧笑!

有人大笑道:``还瞧什么,这妞儿今天心不在焉,只怕已在想汉子了。''``喂,小姑娘想谁呀,是在想我?''

于是人们笑得更开心,也更低贱。

小鱼儿的血又开始沸腾!

但这时,人丛中已有个绿衫少年\ldots 跃而出,却正是白凌霄,他凌厉的目光四下一转冷冷道;``谁若再对这位妨娘说出一个无礼的了,我就割下他的舌头!''另一人厉声接道:``老子就挖他的眼睛!''

这人也随之跃出,竟是那``红衫金刀''李明生。人群立刻静了下来,恶人,永远有人怕的。

海四爹走过来,打着揖笑道:``多谢少爷仗义。''白凌霄冷冷道:``这也没什么!''

自怀中摸出锭大银锞,随手抛在地上:今天眼见你们要白辛苦了,这就给你们买酒喝吧。``李明生大声道:''这可足够买几十坛酒了,爷儿为什么赏你银子,你总该明白。``海四爹面色变了变,但瞬即笑道:''红丫头,还不快过来道谢。``海红珠垂着头走过来,股上像是发了烧,轻轻道:''谢谢少爷``\ldots\ldots{}''白凌霄倔傲的脸上露出了笑容,李明生突然拉住海红珠的手,眯着眼笑道;``咱们的大哥喜欢你,你陪他去喝两杯吧。''海红珠脸色惨白,全身都颤抖起来。

海四爹强笑道:``咱们这孩子年纪还小,等过两年再让她陪少爷喝酒吧。''李明生怪笑道:``过两年?大爷已等不及了。''野犊子冲过来,大声道:``你放开她!''

话末说完,就被李明生反手一个耳光掴在脸上,他半个脸立刻肿了起来,人被打得直跌出去。

白凌霄背负着双手,皮笑肉不笑地说,``我看你还是乖乖地跟我走吧。''背负着的双手突然伸出去摸海红珠的脸。

海红珠已骇得啼哭起来。

突然间,一个人大步定出,一字字道:谁也不能将她带走!"海红珠眼睛立刻发了光──小鱼儿终于出来了!小鱼儿竟会为她出头,她就是死了,也没什么了。

李明生浓眉扬起,狞笑道:``你这脏小子,想找死么!''反手又是一个耳光掴出去。但这耳光却水远也不会掴在小鱼儿脸上。

他的手不知怎地已被小鱼儿捉住,就像上了副铁夹子,骨头都断了,疼得眼泪都流了出来。

小鱼儿厉声道:``去吧!''

喝声出口,手一扬,李明生那好几百斤重的身子,竟被他直摔出去,跌在几丈外,纵然不死,也去了半条命!

人群又惊呼起来,白凌霄面色大变,反手拔剑,``呛''的,长剑出鞘,毒蛇般直刺小鱼儿胸膛!

小鱼儿身子一偏,竟抢入剑光,一掌拍在白凌霄胸膛上,他并未用出全力,但白凌霄却惨呼一声,口中鲜血狂喷而出,整个人就像是一颗草似的软软地倒了下去。淡绿的衣衫上,染满了鲜血画成的桃花!

人群四散而逃,惊呼道:``不好了,杀人了!''小鱼儿呆了呆,他自己实在未想到自己的武功竟如此精进,因惊呼声却使他回过神来。

现在,这里再也不能藏身了!他转身狂奔而出。

海红珠已挣扎着奔出去,嘶声道:``小呆\ldots\ldots 小呆\ldots\ldots 等等我''\ldots\ldots 等等我``。''小鱼儿却头也不回,走得人影不见了。

海红珠踉跄跌在地上,满脸但是眼泪,痛哭着道:``他走了\ldots\ldots 我知道他永远也不会回来了。''海四爹赶过来,扶起了她,他饱经世故的、苍老的脸上,也交织着许多复杂的情感,是惊奇是欣喜,也是不可避免的悲哀。

他轻抚着他爱女的头发,喃喃叹道:``他虽然不会回来了,但这也是没法子的\ldots\ldots 他本就不属于这一群,你又有什么法子拉住他\ldots{}''``海红珠悲嘶道:''但我\ldots。我不能\ldots\ldots 求求你老人家\ldots\ldots{}``海四爹长叹道:''你只有忍耐,像这样的人,非但我拉不住他,世上\ldots\ldots 世上只怕没有任何人能拉住他的``\ldots 你只怕是永远再也见不着他了。''海红珠突然晕倒在他爹爹怀里,永远再不能和自己所爱的人相见,这无论对谁说来,都是不能忍受的痛苦!又何况这情窦初开的女孩子!

\hypertarget{ux7b2cux56dbux5341ux4e8cux7ae0-ux5de7ux8bc6ux9634ux8c0b}{%
\chapter{第四十二章
巧识阴谋}\label{ux7b2cux56dbux5341ux4e8cux7ae0-ux5de7ux8bc6ux9634ux8c0b}}

小鱼儿一口气奔出数里,在荒凉的江岸倒卧下来。今夜,又是满天星光,他做了这件事,总算出了口气,心里似己觉得轻松了些,但却又有另一个沉重的担子加了上去。

他知道自己这一走,海红珠心必定已碎了,他并末存心伤害这纯洁的女孩子,但确已伤害了她。

他仰天笑道:``你莫要怪我,这也是没法子的事''\ldots 我虽然也不愿意走,但我的行迹已露,再也设法子呆在你那里了。"天上的繁屋,就像是海红珠的眼睛,每一只眼睛,都在流着泪,向小鱼儿流着泪,小鱼儿的眼睛却闭起了!。

黎明时,小鱼儿已远远离开了这地方,他茫无目的向前走,更穷、更脏,他都根本不放在心上。

这天,他来到个不算很小的城镇──城镇的大小,其实也和他没什么关系,他根本就远离了人群。

他不走大街,只走陋巷,他不知不觉在一家厨房的后门停了下来,这对他说来,真是种讽刺──所有高贵的香气,都不能令他动心,但这世上最庸俗、最平凡的味道,却诱惹了他。

这厨房最大,香气也最浓,他呆呆地站在那里,也不知过了多久,突然一桶洗碗水倒了出来,倒了他一身。

他既不生气,也不动,现在,他已懂得什么事才值得他生气,像这种事你请他生气,他也不会生气的。

厨房后门里,却探出张圆圆的胖脸来,陪笑道:``对不起,我没有看见你。''小鱼儿笑了笑道:``没关系。''

那张圆脸一笑,缩回了头,过了两盏茶的工大,又探出头来,瞧见小鱼儿还站在那里,竟笑道:``我这里还有些饭,你要是不嫌脏,就进来吃吧。''小鱼儿又笑了笑,道:``好,谢谢你。''

他既没有觉得有什么不好意思,也不客气,走进去就吃,一吃就吃了八碗,吃完了就站起来再笑了笑,道:``多谢。''那圆脸一直在瞧着他,像是觉得这小伙子很有趣,小鱼儿拱了拱手就要走,这圆脸汉子竟笑道:``我这里少个洗碗的人,你要是愿意做,每天少不了有你吃的。''小鱼儿想了想,笑道:``我吃得很多。''

那圆脸笑道:``开饭馆的,还怕大肚汉么。''

小鱼儿想也不想了,一伸手就提起水桶,道:``要洗的碗在哪里?''第二天,小鱼儿就知道这里原来是``四海春饭馆''的厨房,那圆脸汉子自然就是大师傅,名叫张长贵。

于是小鱼儿就开始每天洗碗,他发觉一个人若是躲在饭馆的厨房里,那当真是谁也不会认出他来。

这饭馆生意并不好,客人散得很早,收了炉子,张长贵常会拉小鱼儿陪他喝两杯,聊聊天。

小负儿喝的酒虽不少,但说的话却绝不超过三句。

有一天,锅里的油己热了,张长贵突然肚子痛,抛下钢铲就跑,小鱼儿接着锅铲,替他炒了两样菜。

张长贵回来,不免有些担心,怕炒菜炒得不好。

却不知天下第一名厨也在``恶人谷''里,小鱼儿从小就跟他学了不少手艺,像小鱼儿这样的人,有什么学不好。

过了半晌,外面的堂倌突然唤道,``方才炒的羊肚丝和麻辣鸡,照样再来两盘。''这一次,张长贵自然不会再让小鱼儿动手了,但又过了半晌,四海春的彭老板突然走进厨来,瞪着眼道:``方才有两盘羊肚丝和麻辣鸡是谁做的?''老板居然走进厨房,张长贵心里已在打鼓,硬着头皮笑道:``自然是我做的。''彭老板道:``那味道不对,不是你的手艺。''

张长贵只得如实讲了,彭老板走到小鱼儿面前,左瞧右瞧,瞧了半天,突然挑起大拇指,笑道:``佩服,佩服,瞧不出你小小的年纪,竟能做出那样的莱,连熊老爷吃了都拍手叫好,从今天起,你来掌勺吧。''小鱼儿垂着头,道:``我不会。''

彭老板拍着他肩头,柔声道:``你就帮我个忙吧,从今以后,四海春就得靠你了。''小鱼儿掌勺之后,四海春的生意奇迹般好了起来,远在几百里外的人,都听到了四海春有位名厨。

彭老板已将旁边的铺面都买了下来,加设了房间雅座,厨房里自然也添了人,小鱼儿每天只要动动锅铲。

他甚至连动锅铲时,心里也在想着那本秘笼上的武功奥秘,他简直就像是个得了相思病的少年,昼夜想个不停。

现在,别人都唤他俞大师傅,他说的话就是权威,他不准外人进厨房,就连彭老板都不敢进来。

但有一天,彭老板还是进来了他满脸兴奋之色,搓着手笑道:``俞老弟,今天你可得分外卖力才是──你猜今天有些什么人来了?''小鱼儿淡淡道:``谁?''

彭老板大笑道:三湘地方的一条英雄好汉今天居然赏光来到这里,这不但是我的面子,更是你老弟的光彩。``小鱼儿心一动,道:''他又是谁?"

彭老板挑起拇指,道:``铁无双铁老爷,江湖人称爱才如命,三湘子弟只要提起这名字,谁人不知,哪个不晓。''小鱼儿道:``哦,是么?''

他面色仍是淡淡的,像是丝毫无动于衷,但等到菜炒完,他竟悄悄走了出去,竟第一次走出厨房。

三湘武林盟主,``爱才如命''铁无双,这名字对他的诱惑实在太大,他实在想瞧瞧这竟为了爱才,而敢将李大嘴收为女婿的人,究竟长得是何模样,─个人居然敢将自己的独生女嫁绘李大嘴,这种人连小鱼儿也不得不佩服的。

高高的木屏风,围成一间间雅座。小鱼儿从屏风的缝里瞧出去,只见一个须发皆白、满面红光的锦袍老人,高踞在酒筵的主座上。

他面上笑容虽然可亲,但神情中自有一种尊严气概,那正是惯于发号施令的人所独有的气概,别人再也伪装不得。

小鱼儿只瞧了一眼,便已猜出他必定就是铁无双。

铁无双右面座上,坐着个高颧鹰鼻的中年大汉,目光顾盼之间,也正像是只死鹰一样。

铁无双的左面座上,却赫然坐着那两河十七家镖局的总镖头``气拔山河,铜拳铁掌震中洲''赵全海。

小鱼儿想到此人在那峨嵋山洞中,口口声声将自己唤作``玉老前辈''的神情,险些忍不住笑出声来。

除了这叁人外,酒筵上还坐着八九个衣着鲜明、神情雄壮的汉子,看来也都是江湖中有头有脸的人物。但这其中最令小鱼儿触目的,却是垂手站在铁无双身后的两个紫衣少年。

左面的紫衣少年浓眉大眼,紫黑面膛,就像是条黑豹似的,全身都充满了劲力,不发则已,─发必定惊人。

右面的紫衣少年却是面清目秀,温文有札,看来就像是个循规蹈矩的书香子弟,但他偶而一抬眼,那目光却如刀锋般锐利这两人手持酒壶,代表着铁无双,频频向座上的人劝酒,看来纵非铁无双的子侄,也必是他的弟子。

酒过三巡,赵全海突然长身而起,四下作了个罗圈揖,仰首先喝干了杯酒,然后清了清嗓子大声道:``今日兄弟应铁老前辈之召而来,本该老老实实坐在这里喝得大醉而归,但在未醉之前兄弟心里却有几句话,实在不能不说。''铁无双持须笑道:``说,你只管说,不说话怎么喝得下酒。''赵全海瞪着眼睛,大声道:``段合肥要运往关外的那批镖银,本是咱们两河联镖先派人到台肥去接下来的,江湖中人人都知道此事。''鹰鼻大汉微笑道:``不错,在下也听说过。''

赵全海厉声道:``厉总镖头既然知道此事,便不该再派人到台肥去,将这笔生意抢下来,兄弟久闻衡山鹰厉峰乃是仁义英雄,谁知\ldots{}''哼!"``波''的一声,他手里酒杯竟被捏得粉碎。

``衡山鹰''厉峰神色不动,淡淡笑道:``做买卖讲究货比货,这和江湖道义并没有什么关系,段合肥既然要找三湘镖联,在下也没得法子。''赵全海怒道:``如此说来,你是说咱们两河联镖比不上你们三湘镖联了!''厉蜂冷冷道:``在下并未如此说,这全要看别人的意思。''赵全海胸膛起优,咬牙道:``好\ldots\ldots 很好!..\ldots.''突然转向铁无双,抱拳道:``兄弟今日虽然应召而来,但也知道铁老爷子与''三湘镖联关系深厚,也不想求铁老爷子为兄弟主持公道,只是\ldots─``他''砰``的一拍桌子,大喝道:''只是三湘镖联既然如此瞧不起两河联镖,咱们少不得要和他们斗一斗,尤其是姓厉的。``铁无双突然长身而起,纵声大笑起来,击杯笑道:''赵老弟,我先敬你一杯如何!``赵全海击杯一饮而尽,道:''铁老爷子\ldots\ldots"

铁无双截口笑道:``兄弟你说得不错,老夫世居湘潭,三湘武林中人,可说大多与老夫有些关系.厉峰算起来更可说是老夫的师侄!既然如此,老夫今日若是让老弟你就此负气而去,岂非白混了几十年江湖。''赵全海的手不知不觉已握紧了刀柄,他身旁的四条大汉也变色离座而起,厉蜂面带冷笑,目光却冷锐如刀。

赵全海一字字道:``铁老爷于莫非要将兄弟留在这里?''铁无双纵声笑道:``正是要将你留在这里,听老夫说几句话!''他面色突然一沉,目光转向厉蜂,沉声道:``老夫若要你将这票生意让给两河联镖,你意下如何?''厉峰面色也大变,道:``这\ldots\ldots 这\ldots\ldots{}''

铁无双道:``老夫决不会勉强于你,但这件事老夫已调查清楚,确实是你理亏,你今日若肯接纳老夫之言,老夫便将衡山那片茶林,让作三湘镖联属下的公益\ldots。江湖之中,仁义为先,你还要再思,三思!''厉蜂默然半晌,长叹一声,垂首道:``老爷子的话,弟子怎敢不听,但那茶林乃是老爷子所剩下的少数产业之一,弟子怎敢接受。''。``铁无双附掌大笑道:''只要你肯顾念武林道义,莫教我三湘子弟在江湖中被人背后指骂,我老头子那区区产业,又算得了什么!``赵全海默然半晌,满面愧色,垂首道:''铁老爷子如此大仁大义,而弟子却\ldots\ldots 却\ldots\ldots 弟子实在惭愧,这票生意,还是由三湘镖联承保吧。``厉蜂笑道:''在下不敢,这票生意是两河联镖先接手的,自然还是让两河镖联承保,赵总镖头若是再谦谢,反令在下惭愧。"这两人方才争得面红耳赤,剑拔弩张,恨不得立刻就拼个你死我活,此刻却居然互相谦让起来。

小鱼儿在外面瞧得也不禁大为感叹,暗道:``好个铁无双,果然不愧为领袖武林的人物,非但将一场争杀轻易地消弭于无形,居然还能将别人感化得也变成谦谦君子。''只听铁无双附掌大笑道:``两位既然如此谦让,这趟镖不如就由两阿联镖与三湘镖联联保,岂非更是皆大欢喜。''众人一齐鼓掌称喜,于是干戈化为玉帛。小鱼儿也想走了。

哪知就在这时,赵全海方自举杯笑道:``厉兄,但望此次你我能同心合力,从今以后。''他说到``我''宇,面上肌肉已突然起了阵抽搐,说到``从今以后''手掌也为之抽搐,杯中酒俱已溅出,溅得他一身。

他话未说完,``哗啦啦'',面前碗盏俱都被扫落在地。他人竟也倒了下去!

酒筵前立刻大乱!随他前来的四条大汉,有的失声惊呼,有的赶上去扶起他,突然齐地嘶声道:``不好,中毒\ldots\ldots 总镖头中毒了!''铁无双面色大变,道:``这\ldots\ldots 这是怎么回事?''``两河''属下一条大汉满面悲愤,大喝道:这是怎么回事,该问你才是!``厉峰拍案怒道:''你这是在说谁?他吃过的酒菜咱们也吃过,难道\ldots。"他话未说完,突然也四肢抽搐,跌倒在地上,竟也和赵全海同样的中了毒!

众人更是掠惶大乱,人人自危,每个人都吃了桌上的酒菜,岂非每个人都有中毒的可能!

厉峰既然也中了毒,下毒的自然不会是他,也不会是铁无双了,双方既然都无下毒的理,这毒又是从哪里来的?

小鱼儿虽然旁观者清,一时间却也猜不出这道理。

惊惶大慌之中,小鱼儿忽然瞥见那白面紫衣少年竟悄悄溜了出来,小鱼儿身形一闪,立刻退入了厨房。

此刻厨房中的人也都已惊动面出,再无别人,小鱼儿刚退进去,那紫衣少年竟也悄悄走了进来。

外面正有大事发生,他走进厨房里来作什么?小鱼儿蹲了下去假装往灶里添柴。

那紫衣白面少中根本没有留意到他──像他们这样的人,又怎会去留意一个添火的厨子。

他匆匆穿过厨房,走到后门,轻轻道:``残云。''门外一人应声道:``风卷残云。''

小鱼儿眼角一膘,只见这少年后退两步,门外一条人影一撞而入,满身黑衣,黑巾蒙面,哑声道:``事成了么?''白面少年道:``成了。''

黑衣人道:``好。''

他前后三句话一共加起来才说了九个字,但小鱼儿心头一动,只觉这语声熟悉得很,头埋得更低,几乎要钻进灶里。

黑衣人还是瞧见了他,沉声道:``这人是谁?''白面少年道:``只不过是一个厨子。''

黑衣人道:``留他不得!''

两人身形─闪,黑衣人并指急点小鱼儿背后``神枢''穴,这``神枢''位在``脊中''穴上,乃人身死穴之─。

但小鱼儿却连闪也不闪,只是暗中运气一转,穴道的位置,便向旁滑开了半寸,用的正是武功中最最深奥的``移穴大法'',小鱼儿虽然还未练到炉火纯青,但用来对付这种情况,却已绰绰有余。

那黑衣人一指明明点在他``神枢''穴上,眼看他连声都未出,便跌倒下去,算定此人已必死无疑,冷笑一声,道:``谁叫你耽在这里,你自寻死路,却怨不得我!''黑衣人又道:``快出去,莫要被人猜疑。''

两人再也想不到一个厨子竟身怀绝传已久的武功奥秘,自以为此事做得神不知,鬼不觉,再也不瞧小鱼儿一眼,一个向前,一个向后,急掠而出。

小鱼儿还是伏在地上,就好像真的死了似的动也不动,只是他的心念,却一直在转个不停。这黑衣人的语声,竟和江玉郎有八分相似!

此人若真的是江玉郎,那么,铁无双的弟子又和江玉郎有什么关系?他们进行的究竟是什么阴谋。

小鱼儿心念一转,又想到那日在江别鹤的秘室中,所瞧见的那装着一瓶瓶珍贵毒药的``书匣''。

他那时虽然只匆匆瞧了─遍,但那匣子里的每瓶毒药都未逃过他的眼晴,到如今他还是记得清清楚楚:``销魂散\ldots\ldots 美人泪\ldots\ldots 七步断肠\ldots。夺命丹''\ldots\ldots 一滴封喉\ldots\ldots 散魂水\ldots\ldots 雪魄精\ldots\ldots{}``小鱼儿突然失声道:''雪魄精``\ldots\ldots 不错,必定就是它!瞧那赵全海中毒时的摸样,岂非好像连肌肉都冻僵了。''他立刻跳起来,扯下身上的围裙,用焦炭在围裙上写下副药方──在``恶入谷''长大的人,实在有许多好处。

赵全海、厉峰的脸,已变成一种奇异的死灰色,他们的身子本在颤抖抽搐着,此刻却连动也不会动了。

别的人身子却在不停地颤抖着,也不知自已是否也中了毒?

更不知这毒性要到什么时候才发作。

他们就好像待决之囚般坐在那里,也不敢跑──他们自然知道只要─走动,毒性就发作得更快。

铁无双面上的笑容已不见,不停地踱着方步,搓着手,这纵横数十年的老江湖,此刻也已全失去了主意。

他仰天长叹一声,喃喃道:``这究竟是什么毒?是谁下的毒?''那紫衣白面少年已站在他身后,道;``莫非是这菜馆里的人?\ldots\ldots{}''铁无双道:依我看来,这毒药断非中士所有,否则我行走江湖数十年,怎会连见都未曾见过?若是我猜得不错,这\ldots\ldots{}``突听一人大声道:''你猜的确不错,这毒药确非中土所有,乃是天山雪魄精!``语声中,一人燕子般自屏风上飞掠而过,身子凌空后,抛下了样东西,口中大声接着道;''围裙上所写的药方,可解雪魄精毒,快去配药,还可有救!"他话说得很快,身形却更快,话说到一半时,人已不见,最后那两句话,已是自十余丈外传来的!

铁无双失声道:``好快的身手!''

他一把攫取了那人抛下的东西,只不过是条油腻的围裙,上面果然写着副奇异的药方。

铁无双瞧了两眼,喃喃道,``雪魄精,居然是雪魄精\ldots\ldots\ldots 难怪我猜不到!''众人喜动颜色,齐声道:``如此说来,总镖头岂非有救!''白面少年脸上也已微微变色,口中却冷冷道:``说不定这也是那恶人的诡计!''有人伸手一探赵全海的手,失声道:``不错,那必定又是要来害人的,中了雪魄精毒的人本该全身冻僵而死才是,但他\ldots。.他身上却似火热的。''铁无双沉声道:``你可知道,冻死的人在临死之前,非但不会觉得寒冷,反会觉得如被烈火焚烧一殷,这种感觉若非身历其境,别人永远不会想到的。紫衣白面少年忍不住道:''那么你老人家又怎么知道?``铁无双缓缓道:只因我也险些被冻死过一次。''紫衣白面少年垂下头,再也不敢说话。但他的眼角,还是盯着那条油腻的围裙。

小鱼儿己出了城镇。他自然知道那``四海春饭馆''再也不是他藏身之地了,但是他还不想露面,他还要等!

他要等到自已一露面便已轰动江湖的那一天,他才大摇大摆地走出来,让别人瞧瞧小鱼儿究竟是怎么样的人!

现在,他还是不想管闲事,虽然他明知``四海春''的这件奇案在江湖中必将成为一个谜。

只因他知道以自己此刻的力量,就算去管这件事,也还是没有什么用的,说不定反而要赔上自己一条命。

他又茫无目的地向前走,还是那么脏,那么穷。但此刻,他的心情,他的武功,却已和往昔不可同日而语了。

绝代之英雄,终于已将长成!

这一日他又走到江岸,望着那滚滚江水,他脚步竟不知不觉间放缓了下来,他可是希望再瞧瞧那艘乌篷破船!

他可是希望再瞧瞧船上那些生活虽然卑贱,但人格却毫不卑贱的人?他可是希望再瞧瞧那双明亮的大眼睛?

江上船来船去,却再也找不到那艘破船的影子?他们到哪里去了?还不是在流浪,在飘泊\ldots\ldots 小鱼儿站在江岸旁,痴痴的出了半天神。

突听身后衣挟带风之声响动,─人道:``有劳阁下久候,抱歉得很。''小鱼儿心里虽然奇怪,但也不回头,也不说话。

那人又道:``阁下怎地只有一人前来?还有两位呢?''小鱼儿还是不说话。

那人忽道:``在下等遵嘱而来,阁下为何全不理睬?''小鱼儿终于回头笑道:``你们只怕找错人了吧。''他话未说完,巳瞧清了面前的三个人。

天上星光与江上渔火高映下,只见左面的一人生得又高又大,身上穿件发亮的红衣服,却赫然正是那``红衫金刀''李明生!

中央那人气概轩昂,自然正是他爹爹``金狮''李迪,还有一个紫面短须,却是那``紫面狮''李挺。

小鱼儿瞧见了这三人,还真是吃了一惊,脸上的笑容都险些僵住了,幸好这三人竟末认出他来。

``金狮''李迪皱眉道:``原来是个小叫化子。''

李明生喝道;``你站在这里干什么?''

小鱼儿垂头道:``小人无地可去,所以才站在这里。''李明生道;``你还不快滚,少时只怕\ldots\ldots{}''

话犹未了,``紫面狮''李挺已低叱道:``来了!''江面上,已荡来一叶轻舟。

轻舟上果然有三条人影,黑衣人影!

\hypertarget{ux7b2cux56dbux5341ux4e09ux7ae0-ux5947ux5cf0ux8fedux8d77}{%
\chapter{第四十三章
奇峰迭起}\label{ux7b2cux56dbux5341ux4e09ux7ae0-ux5947ux5cf0ux8fedux8d77}}

小鱼儿远远在江岸旁的草丛中蹲了下来,但却不肯定,他实在穷极无聊,实在想瞧瞧热闹。

轻舟还未靠岸,三条黑衣人已飞擦而来,居然俱都是身手矫健、轻功不弱的武林高手!

当先一人身材魁伟,后面一人矮小精捍,最后的那人腰肢纤细,看来竟仿佛是个女子。

三人都是满身黑衣,黑贴蒙面,几乎连眼睛都掩住,手里都提着长长的黑包袱,包袱里显然是兵器。

他们的兵器为何也要用黑布包着?难道他们连兵器都有秘密。

李家父子已迎了上去,但两方人中间还闻着七八尺,便已停下脚步,面面相对凝神戒备。

``金狮''李迪厉声道:``三位可就是自称仁义三侠的么?''那高大的黑衣人冷冷道:``不错!''

李迪道:``敝镖局的镖车,近年来数次失手,都是三位做的手脚?''李迪冷笑道:``三位既然连连得手,我等又查不出三位的来历,三位便该好生躲藏才是,却又为何要下书将我兄弟约来这里?''黑衣人缓缓道:``江湖中都已知道,赵全海与厉峰已双双中毒,他们的人虽未死,但两河联镖与三湘镖联的威信却大伤。黑衣人道:''三湘与``两河的威信受损,双狮镖局自然要乘机窜起,段合肥那批镖银,自然要着落在你身上了。''听到这里,小鱼儿心才动了,双狮父子也已为之动容。

黑衣人缓缓又道:``这趟镖关系非浅,双狮镖局想也不敢自力承担,必定请得有旁人从中保证,以我三人之力,只怕也动不了它。''紫面狮``冷笑道:''你倒也聪明!``黑衣人厉喝道:''所以我今日就要叫你们也保不了这趟镖,三湘镖联与两河联镖就算倒了霉,你们也休想占便宜!"喝声中,手腕一抖,黑色包袱布抖落在地,露出了三件青光闪闪兵刃,乍看似钩,但钩头部是朵梅花。

``金狮''李迪失声道:``梅花钩!''

黑衣人道:``你们居然还认得这件兵刃,总算不错!''李挺冷笑道:``你们居然敢将这兵刃亮出来,更可算胆子不小,你们难道就不怕你家仇人不声不响地摘走你们的脑袋!''黑衣人道:``没有人会知道''梅花钩又已重现江湖的!"话声中,三人已直扑了上来。

那矮壮的黑衣人当先扑向李明生,此人身法最猛,招式也最猛,看来竟似与李明生有着什么仇恨!

那黑衣女子却掠向``紫面狮''李挺。她身法轻灵巧侠,掌中梅花钩的招式却是迅急狠毒,刺、夺、绞、削,新奇的兵刃,新奇的招式。

``紫面狮''李挺武功虽然老练,但遇着这多门兵刃迅急的招式,一时间竟被逼得手忙脚乱。那边``金狮''李迪也已和那高大的黑衣人交上了手。

这─战已可说是十分激烈,但小鱼儿却瞧得甚是无趣,除了这``梅花钩''有些新奇的招式还勉强值得他一瞧,要知他所练的那武功秘笈,正是天下武功之精华,那李迪等人的武功,实在连比都无法比的。

这其中最惨的就是李明生,四十招下来,他连刀法都未施展开,额头鼻挂都已沁出汗珠。

那矮壮的黑衣人却是越战越勇,突然间拧身错步,青光如落花般洒下,梅花钩已锁住了刀锋。

李明生心胆皆丧,只因他此刻前胸空门已大露,对方只要迎胸一拳击来,他纵然不死,也去了半条命!

哪知那黑衣人却只是反手给他个耳括子,沉声道:``这是先还你的!''李明生被打得踉跄跌倒,再一跃而起,失声道:``还我的?''突然间,只听一声长笑,一条人影闪入了钩光,接着,只听``嗖!嗖!嗖!''三响,三柄梅花钩俱都已冲天飞起,两柄落在地上,一柄落入江里。

三条黑衣人只觉手腕一震,兵刃已脱手,对方用的是什么招式,是如何出手的,这叁人竟全不知道。

三人大惊之下,齐地纵身后退,只见面前不知何时已多了个少年,轻衫飘飘,面白如玉。小鱼儿瞧见这少年,也不免有些吃惊──江玉郎,这面色惨白的、笑容阴森的少年却不是江玉郎是谁?但江玉郎的武功又怎会如此精进?

这问题小鱼儿自然能回答的,江玉郎也背过那武功秘笼,两年来他武功若不精进,那他简直就不是人了。

双狮父子俱都面现喜色。

黑衣人却是又惊又怒。黑衣人顿了顿脚,想是想走,但江玉郎身子一闪,已到了他们面前,挡佳了他们去路,笑道:``这位姑娘也用布蒙住脸,是因为生得太丑?还是太美呢?''那矮壮的黑衣人怒吼一声,挥拳直扑上来。武功的确不弱,李明生绝不是他的敌手,但此刻到了江玉郎面前,却半点用也没有了。

他一拳还未击出,手腕已被江玉朗擒住,轻轻一笑.他身子便飞了出去,险些落入江里。

江玉郎笑道:``你们既不愿说,在下也只有自己来瞧了。''笑声中,他已闪过那高大的黑衣人,到了那少女面前。

黑衣少女的双掌齐出,但两只手不知怎地竟被江玉朗那一只手捉住,她伸腿要踢,膝盖却也麻了。

江玉郎笑道:``但愿姑娘生得美些,否则在下就失望了。''他手掌一扬,黑衣少女的脸拚命向后退,但她面上的黑巾,还是被揭了下来。

于是星光就照上了她的脸,也照着她的眼睛。她眼睛就如同星光般明亮。

小鱼儿目光动处,几乎叫出声来,海红珠.这黑衣少女竟是海红珠!

李明生失声道:``是她!原来是她!''

江玉郎道:``你认得她?''

李明生嘶声道:``她就是那卖艺的女子,白凌霄大哥就是为她死的\ldots\ldots 那矮子想必就是那天被我掴了一拳的人,难怪他要找我报仇!''江玉郎笑道;``更妙了,更妙了,梅花门下,居然做了江湖卖艺的,你们为了避仇居然不借做如此低贱之事,这点我倒也佩服。''那高大的黑衣人也撕下黑巾,果然正是海四爹!他咬紧钢牙,厉声道:``你放开她的手!''江玉郎道:``放开她的手也可以,但我却要先问你,那日一掌就打死白凌霄白公子的人究竟是谁?此刻在哪里?'':海红珠娇呼道:``你想找他,你这是在做梦!''江玉郎微笑道:``哦,做梦?\ldots\ldots{}'';他手掌一紧,海红珠立刻疼出了眼泪,却仍然咬牙呼道:``像你这样的人和他比起来,连提鞋都不配。''说到后来,她声音已颤抖,显然已疼彻心骨,但她死也不肯住口。

海四爹怒吼一声,铁拳直击江玉郎背脊,江玉郎头也不回,身子也是没有动,海四爹的手臂却已被他夹在肋下,再也动弹不得。

海四爹面上青筋暴现,冷汗迸出,手臂似已将折断。他昔日本也是叱咤一时的风云人物,但此刻在这少年面前,武功竟连一成也施展不出,长叹一声,顿足道:``罢了!.\ldots.''突听一人凄声道:``我的神枢穴疼呀,江玉郎,你还我命来!''呼声尖锐凄厉,实在不像是人的声音。接着,一条人影自江岸旁的草丛里飘了出来。

夜色中,只见他披头散发,满身油污,七分像鬼,却连三分也不像人,身子飘飘荡荡,宛如乘风。

他呼声凄厉,模样像鬼,身形更如鬼魅;深夜荒江畔,骤然瞧着这样的``人'',谁能不被骇出冷汗.::小鱼儿格格笑道:``黑心贼,我与你无冤无仇,你却在四海春的厨房里,下毒手害死了我,你陪命来吧。''江玉郎手已松开!身子后退,嘶声道:``你\ldots\ldots 你\ldots\ldots{}''像他这样的人,本不会相信鬼魅之事,但此刻却又实在不能不信,只因他确信自己点着那人死穴时,那人是万万活不成的,而那日在四海春``厨房里的事,天下谁也不知道,此''人"不是鬼是什么?

他牙齿打战,连话竟也说不出来,双狮父子瞧见他怕成如此模样,也不由自主随着他往后退。

小鱼儿道:``你想跑?你跑不了的''\ldots.跑不了的,快拿命来吧!"他龀牙笑着,一步步往前走,身予摇摇荡荡,似将随风而倒!

海红珠也瞪眼瞧着他,突然脱口大呼道:``是你!小呆,是你么?''小鱼儿形状虽然又改变了,但那双眼睛,那双令海红珠刻骨铭心、永生难忘的眼睛,她又怎会认不出。她呼声出口,才想起自己错了,但已来不及。

小鱼儿暗暗顿足道:``该死\ldots.''

江玉郎果然已瞧出其中有诡,身形动处,直扑过来,轻风般地拍出七掌,如落花缤纷,满天飞舞。

海四爹等人瞧见变幻如此奇妙、出手如此轻灵的掌法,都不禁为之失色,海红珠更是为她的``小呆''担心。

小鱼儿却阴森道:``你还想杀我?你已杀死过我一次,再也杀不死我了!''他身子飘飘站在那里,像是根本没有闪避,但江玉郎七掌拍过,他还是好生生的站在那里,这轻灵迅急的七掌竟似没有沾着他一片衣袂。

别的人瞧得目蹬口呆,江玉郎更是心惊胆战,狂吼一声,又是七掌拍出,掌势更急、更狠!但小鱼儿还是动也不动,这七掌还是沾不到他的边。

小鱼儿龀牙笑道:``你再也杀不死我了,此刻你难道还不信?''江玉郎身子颤抖,额上已进出一粒粒冷汗,别的人瞧见这种不可思议的事,也是手足冰冷。

江玉郎的十四掌竟真的像是打在虚无缥渺的鬼魂身上,他们亲眼瞧见怎能不信?怎能不怕?

海红珠瞪大了眼睛,眼里已满是泪水,但这已不再是悲伤的泪,而是惊喜的泪,兴奋的泪。

只见小鱼儿一步步往前逼,江玉郎一步步往后退,他手脚都已似有些软了,竟再无出手的勇气。

双狮父子自然已退得更远了,退着退着,转头就跑,江玉郎也突然全力跃起,凌空一个翻身,逃得比他们还快一些。

小鱼儿也不追赶,瞧着他的背影,喃喃笑道;``我不想杀你\ldots\ldots 实在不想杀你!''

海红珠已扑了过来,颤声呼道,``小呆,我知道还能见着你的,我知道\ldots\ldots\ldots{}''小鱼儿咯咯一笑,道;``谁是小呆\ldots。我是鬼\ldots 鬼\ldots\ldots{}''海红珠刚扑过来,他身子已如火箭般斜斜掠过三丈,凌空再一转折,``扑咚'',落入了江心。

海红珠扑到江边,又痛哭起来,嘶声道,``你若不想见我,为什么要到这江边来\ldots\ldots\ldots 你若想见我,为什么见了我又要走?为什么\ldots\ldots\ldots 为什么?''小鱼儿尽量放松了四肢,飘浮在水面上,冰冷的江水,就像是一张床,天上繁星点点,他觉得舒服得很。

他总算已瞧过了她想见的人,虽然他们的变化不免令他吃惊,虽然他只瞧了一会儿,但这已足够了。

这几天来他怀疑不解的事,此刻总算也恍然大悟。那紫衣白面少年的确是和江玉郎在暗中勾结,而江玉郎却显然是``双狮''镖局的幕后主人。

那么,赵全海与厉峰的被毒,就─点也不奇怪了──他们杯中的酒,正是那白衣少年倒的。他想着想着,突然几根竹篙向他点了过来。

他先不免吃了一掠,但立刻想到:``他们必定以为我是快淹死的人,所以要来救我的。''他暗中好笑,索性闭起了眼睛。只觉得几个人七手八脚地将他拉上了一条船。

一人摸了摸他心口,笑道:``这小子命长,幸好遇见我们,还没淹死。''又有人替他灌下了碗热汤,替他揉着四肢。

突听一个洪亮的语声道:``这人是死的,还是活的?''小鱼儿突然睁开眼睛,笑道:``活的!''

他张开眼睛,就瞧见一条大汉站在眼前,半敞着衣襟;歪带着帽子,一条腿高跨在凳子上,手里拿着又粗又长的旱烟。

此刻他以旱烟指着小鱼儿,大声道:``你既是活的,为何要装死?''小鱼儿还未说话,忽然发现这大汉"胸脯高耸,腰肢很细,虽然浓眉大跟但却并不难看。

小鱼见笑了笑,道:``你既是女人,为何要装成男的?''那大姑娘瞪起了眼睛,怒道;``你知道我是谁?''小鱼儿笑道:``不管你是男的还是女的,你反正是个人,你已经快嫁不出去,再这么凶,还有谁敢娶你!。''他说话本来尖刻,这两年来已极力收敛,但憋了两年多,此刻又不禁故态复萌,这正是江山易改,本性难移。

那大姑娘拍案道:``你敢对我这样说话?''

将小鱼儿擒进来的几个少年,此刻脸都变了颜色,几个人在后面直戳他的脊梁,小鱼儿假装不知道,还是笑道:``为什么不敢?,只要你是人,我就不\ldots\ldots{}''他话未说完,那几个少年已抢着笑道:``这位就是段合肥段老太爷的女公子,江湖人称女孟尝,你总该听过,说话就该小心些。''小鱼儿笑道:``呀,原来你就是段合肥的女儿,你爹爹可是有一批银子要运到关外去?''小鱼儿耸了耸鼻子,道:``这船药材,是你从关外运来的么?''女孟尝眼睛瞪得更大,道:``你怎知道这是船药材''小鱼儿笑道:``我不但知道这是船药材,还知道这些药材是人参、桂皮、鹿角、五加子\ldots。.''他一连说了一大串药名,果然正是这般上所载的药材,说得丝毫不差。

莫说这几种普通的药草,就算将天下各种药草都混在一起,他也照样可以嗅得出的,此刻他一口气说完了,这些人都不禁惊奇得张大了嘴。

女孟尝眼睛里有了笑意,独了口旱烟,``呼''的将一口烟雾喷在小鱼儿的脸上,悠悠道:``想不到你这小子对药材还内行得很。''小鱼儿差点破烟呛出了眼泪,接着眼笑道:``我对药材非但内行,而且敢说很少有人比我再内行的你若真的是女盂尝,就该好生将我礼聘到你家的药铺里去。''女孟尝又抽了口旱烟,这次却未喷到小鱼儿脸上,而是一丝丝吐出来的,等到烟吐完了,突然转身走了进去,口中却道:``替他换件衣服,送他到庆余堂去。''安庆``庆余堂'',可算是皖北一带最大的药铺,小鱼儿在这里,居然做了管药的头儿。他根本用不着到柜上去,所以也不怕人认出他,每天就配配药方,查查药库,日子过得更清闲了。

这时,他才知道,那位``段合肥'',正是长江流域一带最大的财阀,这一带最赚钱的生意,差不多都被他垄断了。那``女孟尝'',就是他独生女儿,她据说还有两个哥哥,但却已死了,所以别人都称她``三姑娘''。

这位三姑娘时常到庆余堂来,但她不理小鱼儿,小鱼儿也不理她,虽然小鱼儿已知道她看来虽凶,心却不错。小鱼儿越不理她,她到的次数越勤了,有时一天会来上两三次,但眼睛还是连瞧也不瞧小鱼儿一眼。

这一天小鱼儿正躺在椅子上晒太阳,初冬的太阳,晒在他身上,他觉得舒服得很,几乎要睡着了。

那位段三姑娘突然走到他面前,用旱烟袋敲了敲椅子背,道:``喂,起来。''小鱼儿笑道:``我的名字可不叫喂。''

三姑娘眼睛又瞪了起来,大笑道:喂,我问你,上次你说的那批要送到关外的镖银,你怎会知道的?``小鱼儿道:''那批镖银怎样?"

三姑娘冷冷道:那批银子已被人劫走了。"

小鱼儿眼睛亮了.翻身坐起来,喃喃道:``奇怪!既是双狮镖局接的镖,怎么还会被人劫走呢?\ldots\ldots{}''三姑娘冷冷道;``双狮镖局的镖,怎么就不能被人劫走?\ldots\ldots 哼,我瞧那个姓李的,根本就是饭桶!''小鱼儿想了想,又道:``劫镖的是些什么人,你可知道?''叁姑娘道;``那批镖银乃是半夜中忽然失踪的,门未开,窗未动,看守镖银的人连屁都末听见,镖银就像生了翅膀飞了。''小鱼儿笑道:``这倒是奇案\ldots\ldots 除非那劫镖银的人会五鬼搬运法,否则就是双狮镖局的人眼睛耳朵有了毛病。''叁姑娘道:``那他们就活该自己倒霉!小鱼儿道:''难道他们要赔?``叁姑娘冷笑道:''当裤子也得赔的。"

小鱼儿又用手模鼻子,喃喃道:``这就怪了\ldots。我本来还以为这是双狮镖局监守自盗,但他们既然要赔,这又是为了什么呢?''三姑娘道:``只因为他们都是饭桶,所以镖银就被人劫走,这道理岂非简单得很。''小鱼儿缓缓道:``看来越是简单的事,说不定其中内幕越是复杂。''三姑娘瞧着他,瞧着他的冷笑,瞧了许久,突然大声道:``你究竟是个聪明的人,还是个呆子?''小鱼儿长长叹了口气,翻过身,把头埋在手弯里,悠悠道,我若是呆子,日子就会过得快活多了。"

\hypertarget{ux7b2cux56dbux5341ux56dbux7ae0-ux6251ux6714ux8fdeux79bb}{%
\chapter{第四十四章
扑朔连离}\label{ux7b2cux56dbux5341ux56dbux7ae0-ux6251ux6714ux8fdeux79bb}}

第二天,还是个晴天,太阳还是照得很暖和。小鱼儿又躺在那张椅子上晒太阳。

他全身骨头都像是已经散了,像是什么事都没有去想,其实,他心里想的事可真是不少。

他心里的事虽然不少,但总归起来,却只有两句话:"那批镖银怎会被劫走?是谁劫走的?他想不通。

这时,三姑娘居然又来了。

小鱼儿眯起了一只眼睛去瞧她,只见她神情像是兴奋得很,匆匆赶到小鱼儿面前,大声道:``喂,你错了。''小鱼儿本来懒得理她,但听见这话,却不禁张开眼睛,道,``我什么地方错了?''三姑娘眼睛闪着光,道:``我刚才听到这个消息,那批镖银已被夺回来了。'';小鱼儿眼晴也睁大了,道:``被谁夺回来的?''三姑娘大声道:``那人年纪和你差不多,但本事却此你大多了,你若不像这么懒,也许还可以赶上他十成中的一成。''小鱼儿已跳了起来,道:``你说的可是江玉郎?''三姑娘怔了怔,道:``你怎会知道?小鱼儿突然大笑道:我知道,我当然知道\ldots。我什么事都知道了\ldots{}''他又笑又叫又跳,三姑娘简直瞧呆了,终于忍不住道:``你难道是个疯子?''小鱼儿突然跳起来亲了亲三姑娘的脸,大笑着道:``只可惜我不是,所以他们倒霉的日子已不远了。''他拍手大笑着,转身跳进了药仓。

三姑娘手摸着脸,瞪大了眼睛,瞧着他,就像是在瞧着什么怪物似的,喃喃道:``小疯子\ldots\ldots 你真是个小疯子。''因为只用一根灯草,所以灯火不亮,小鱼儿出神地瞪着这点灯光,微笑着喃喃道:``江玉郎,你果然很聪明,你假装镖银被盗,再自己去夺回来\ldots..这么神秘的盗案,你居然不费吹灰之力就破了,江湖人有谁能不佩服你,又有谁会知道这只不过是你自己编出来的一出丑角戏。''他轻轻叹了口气,接道:``只有我\ldots\ldots 小鱼儿,但愿你莫要忘了这世上还有我,你那一肚子鬼主意,没有一件能瞒得过我的。''窗外,夜很静,只有风吹着枯枝,飕飕的响。突听一人压着嗓子唤道:``疯子\ldots。小疯子,快出来。''小鱼儿将窗于打开一线,就瞧见了披着一身大红斗篷,站在月光下寒风里的段三姑娘。

三姑娘只是咬了咬嘴唇,道:``我有事。\ldots 有要紧的事要告诉你。那件事果然不太简单。''小鱼儿眼睛一亮,道:``你又得到了消息?''

三姑娘道:``是。\ldots 我刚刚又得到消息,镖银又被人劫走了!''小鱼儿鞋子还没穿就跳出了窗子,这下他可真的吃了一惊,他赤着脚站在冰凉的石扳上,失声道:``你这消息可是真的?''三姑娘道:``半点不假。''

小鱼儿搓着手道:``这镖银居然又会被人劫走,这简直是不可能的事,我实在想不通\ldots\ldots 你可知道劫镖的人是谁么?''三姑娘道:``这一次.和上一次情况大不相同。''小鱼儿道:``有什么不同?难道这一次丢了镖银,他们连赔都不必赔了。''三姑娘缓缓道:``是,他们的确不必赔了。''

小鱼儿跳了起来,大声道;``为什么?''

三姑娘垂下目光,道:``只因为双狮镖局大小镖师,内外趟子手,一共九十八个人,已死得一个不剩,只剩下个喂马的马夫。''小鱼儿以手加额,怔了半晌,忽又大声道:``那江玉郎呢?''叁妓娘道:``江玉郎不是双狮镖局里的人。他夺回镖银,便功成身退,再也不停留片刻,这岂非正是大英雄、大豪杰的行径!''小鱼儿吃吃笑了起来,冷笑道:``好个大英雄、大豪杰!只怕他早巳知道镖银又要被劫,所以就溜了。''三姑娘道:``你是说\ldots\ldots 第二次劫镖的,也是第一次劫镖的那伙人?''小鱼儿眨了眨眼睛,道:``这难道不可能?''

三姑娘道:``第一次劫镖的人,都已被江玉郎杀了,他夺回镖银时,镖银是和劫镖的人头一起送回来的!''小鱼儿击掌道:``好手段!果然是好狠的手段!''叁姑娘凝眸瞧着他,缓缓道:``而且,第二次劫镖的只有一个人\ldots\ldots 双狮镖局的九十八条好汉,全都是死在这一个人的手下!''小鱼儿动容道:``一个人?\ldots\ldots 一个人在一夜间连取九十八条性命,江湖中是谁有如此狠毒如此高明的手段?''叁姑娘道;``据说,那是个须眉皆白的虬髯老人!..\ldots.''小鱼儿道:``有谁瞧见他了?''

三姑娘道:``自然是那死里逃生的马夫。''

小鱼儿道:``那么他\ldots\ldots{}''

三姑娘接口道:``他听得第一声惨呼后,就躲到草料堆里,只听屋子里惨呼一声,接连着断续响了两三盏茶时分\ldots\ldots{}''``小鱼儿失声道:''好快的手!好快的刀!"

叁姑娘叹道:``杀人的时间虽然不长,但在那马夫心中觉得,却仿佛已有好几个时辰,然后他便瞧见一个高大魁伟的虬髯老人,手提钢刀,狂笑着走了出来,这老人穿的本是件淡色衣衫,此刻却已全都被鲜血梁红了!''小鱼儿手模着下巴,悠悠道:``这听来倒像是个说书人说的故事,每个细节都叙述得详详细细,精采动人。\ldots 一个人刚刚死里逃生,还能将细节描述得如此详细,倒端的是个人才。''三姑娘展颜笑道:``当时我听了这话,也觉得他细心得很。''小鱼儿道;``你是什么时候听到这消息的?''

三姑娘道:``就在半个时辰之前。''

小鱼儿道:``这件事又是在什么时候发生的?''三姑娘道:``昨天晚上。''

小鱼儿道:``消息怎会来得这么快?''

三姑娘道:``飞鸽传书\ldots。以此间为中心,周围数千里大小七十九个城镇,都有我家设下的信鸽站!''

小鱼儿突然大声道:``我和这件事又有什么狗屁的关系?你为什么要如此着急地赶来告诉我?你吃饱饭没事做了么?你难道以为我和那劫镖的人有什么关系?''三姑娘跺脚道:``可是!我不是这个意思''

小鱼儿道:``那你是什么意思?''

三姑娘的脸,居然急红了,居然还是没有发脾气。

她垂下了头,轻声道;``只因为你\ldots\ldots 你是我的朋友,─个人心里有什么奇怪的事,总是会去向自己的朋友说的──。''小鱼儿大声道:``朋友?\ldots\ldots 我只不过是你雇的一个伙计,你为什么要将我当做你的朋友?''三姑娘脸更红,头垂得更低,道:``我\ldots\ldots 我也不知道。''小鱼儿瞪着眼瞧了她半晌,突然大笑起来。

三姑娘咬着嘴唇,道:``你\ldots\ldots 你笑什么?''

小鱼儿大笑道:``我认识你到现在,你只有此刻这模样,才像是个女人!''三姑娘垂头站在那里,呆了半晌,突然放声大哭起来,整个人却像是软了,扑倒在橱上,哭得真伤心。

小鱼儿皱了皱眉,道:``你哭什么?''

三姑娘痛哭着道:``我从小到现在,从没有一个人将我看作女人,就连我爹爹,他都将我看成个男孩子,而我。\ldots 明明是个女人。''小鱼儿怔了怔,点头道:``一个女人总是被人看成男孩子,的确是件痛苦的事!\ldots 你实在是个很可怜的女孩子。''三姑娘呻吟道:``我今天能听到这句话就是立刻死,也没有什么了。''小鱼儿道:``但我却一点儿也不同情你。''三姑娘踉跄后退了两步,咬牙瞪着他。

小鱼儿笑道:``你希望别人将你当做真正的女孩予,就该自己先做同女孩子的模样来才是,但你却成天穿着男人的衣服,抽着大烟斗,一条腿跷得比头还高,活像个赶大车骡夫,却教别人人如何将你看成女孩子。''三姑娘冲过来,扬起手就要打,但这只手还没落下去,却又先呆住了,呆了半晌,又垂下了头。

小鱼儿道:``好孩子,回去好生想想我的话吧\ldots\ldots\ldots\ldots 至于那件镖银的事,我现在虽然还没有把握,但不出半个月,我就会将真相告诉你。''他一面说话,一面已跳进了窗户。

他关起窗户,却又从窗隙里瞧出去,只见姑娘痴痴地站在那里,痴痴的想了许久,终于痴痴的走了。小鱼儿摇头苦笑。

下半夜,小鱼儿睡得很熟。正睡得过,突然几个人冲进屋子,把他从床上拉了起来,有的替他穿衣服,有的替他拿鞋子。

这几个人中,居然还有药铺的大掌柜,二掌柜,小鱼儿睡眼惺松,揉着眼睛道:``领钱的日子还没到,就要绑票么?''二掌柜的一面替他扣钮子,一面笑道:``告诉你天大的好消息\ldots\ldots\ldots\ldots\ldots\ldots 太老爷今天居然要见你。''大掌拒也接着笑道:``太老爷成年也难得见一个伙计,今天居然到了安庆,居然第一个就要见你,你这不是走了大运么?''于是小鱼儿糊里糊涂地就被拥上了车,走了顿饭工夫,来到个气派大得可以吓坏人的大宅子,糊里糊涂地被拥了进去。

这大宅院落一层又一层,小鱼儿跟着个脸白白的后生,又走了半顿饭的工夫,才走到后园,花木扶疏中五间明轩,精雅玲珑。

那俊俏后生低声说道:``太老爷就在里面,他老人家要你自己进去。''小鱼儿眨着眼站在门口,想了想,终于掀起子,大步走了进去,第一眼就瞧见了三姑娘。今天的三姑娘,和往昔的三姑娘可大不相同了。

她穿的不再是短脚裤,小短袄,而是百折洒金裙,外加一件蓝底白花的新绸衣。

她脸上淡淡地抹了些胭脂,乌黑的头发,插着只珠凤,两粒龙眼睛大的珍珠,在耳坠上荡来荡去。

她垂着头坐在那里,竟好像有些羞羞答答的模样,她明明瞧见小鱼儿走进来,还是没有抬头,只是眼皮瞟了瞟,轻轻咬了咬嘴唇,头反而垂得更低。

小鱼儿儿乎忍不住要笑出声来──若不是他瞧见她身旁的地上还爬着个人,他早已笑出声来了。

地上铺着厚厚的波斯地毯,一个穿着件宽袍的胖子爬在地上,骤然一看,活脱脱像个大绣球。

他面前有只翡翠匣子,竟是用整块翡翠雕成的,价值至少在万余以卜,但匣子里放着的却是只蟋蟀。

小鱼儿也伏下身子,瞧了半晌,笑道:``这只红头棺材只怕是个刽子手''\ldots\ldots{}``那胖子抬起头,笑得眼睛都眯成一条线了,道:''你也懂蟋蟀?``小鱼儿笑道:''除了生孩子之外,别的事我不懂的只怕还不多。``那胖子附掌大笑道:''好,很好\ldots\ldots 老三,你说的人就是他么?"这人不问可知,自然就是那天下闻名的财阀段合肥了。

三姑娘垂首道:``嗯!''

段合肥笑得眼睛都瞧不见了,道:``很好,太好了,你眼光果然不错''小鱼儿摸了摸头笑道:``这算怎么回事?''

段合肥道:``你莫要问,莫要说话,什么事都有我''。``先把我拉起来,用力\ldots\ldots 嗳,这才是好孩子。''他好容易从地上站了起来,看样子简直比人家走三里路还累,累得直喘气,摸着胸口笑道:``很好。\ldots 很好,你喜欢吃红烧肉吧\ldots\ldots 什么鱼翅燕窝、鲍鱼熊掌都是假的,只有红烧肉吃起来最过瘾。''小鱼儿道:``但是我根本不知道,这是。\ldots.''段合肥摆手道:``你不必知道,什么都不必知道''\ldots 都由我作主就够了,留在这里吃饭,我那大师傅烧的红烧肉,可算是天下第一。"于是小鱼儿糊里糊涂地吃了一大碗红烧肉。到了这里,他的嘴除了吃肉外,好像就没有别的用了,因为段合肥根本就不让他说话。

黄昏后,他回到店里,还是不知道段合肥叫他去干什么,只觉``庆余堂''上上下下的人,对他的态度全变了。

那自然是变得更客气了。

洗过澡,小鱼儿刚躺上藤椅,突听前面传来一阵粗嘎的语声,就像是破锣似的直着嗓子道:``附子、肉桂、犀角、熊胆\ldots\ldots{}''他说了一大串药名,不是大寒,就是大热,接着又听二掌柜那又尖又细的语声,想来是在问他;这些药,你老要多少?``那语声道:''你们这店里有多少,咱们就要多少,全都要,一钱也不能留。``另一人道:''你们这庆余堂想必有药库吧,带爷们去瞧瞧。"这人的语声更响,听起来就像是连珠炮竹。

小鱼儿心念一动,刚站起身子,就瞧见那二掌柜的被两个锦衣大汉接了进来,就好像老鹰抓小鸡似的。

灯火下,只见这两个大汉惧是鸢肩蜂腰,行动矫健,横眉怒目,满脸杀气,遇见这样的人,这二掌柜的能不听话么?

小鱼儿袖手站在旁边瞧着,店里的伙计果然将这两个锦衣大汉所要药材,全都包好扎成四大包。

小鱼儿却悄悄在掌心扣了个小石子,等到他们将药包运出门搬上车子,他手指轻轻一弹,石子``嗖''的飞了出去,打在药包的角上,门外的灯光并不亮,他出手又快,自然没有人发觉。

他又躺回那张藤椅,瞧着天上阀亮的星群,喃喃道:``看来,这只怕又是出好戏''\ldots."夜更静,药铺里的人都已睡了,小鱼儿却仍坐在星光下,在这安详的静夜里;他却似乎在期望着什么惊人的事发生。小鱼儿眯起了眼晴,也似乎将入梦乡。

突然间,静夜中传来─阵急骤的马蹄声,小鱼儿眼睛立刻亮了,侧耳听了听,喃喃道:``三匹马,怎地只有三匹马?''这时健马急嘶,蹄声骤顿。三匹马竟果然俱都在庆余堂前勒缰而停。

接着,便是一阵急促的敲门声,一人大喝道:``店家开门,快开门,咱们有急病的人;要买药。''响亮的呼声中,果然充满了焦急之意。睡在前面的伙计,自然被惊醒,于是回应声、抱怨声、催促声、开门声``。''响成了一片。

那焦急的语声已在大声喝道:``咱们要附子、肉桂、犀角、熊胆;\ldots{}''每样叁斤,快,快,这是急病。``店伙计自然怔了一怔──怎地今天来的人,都是要买这几样药材的?他们的回答自然是;''没有。``那焦急的语声立刻更惊惶、更焦急,甚至大吵大闹起来;''这么大的药铺,怎地连这些药都没有?``这人身材也在六尺开外,一双威光棱棱的眼睛,已满布血丝,那店伙计瞧见这凶相,只有陪笑道;''咱们是百年老店,什么药原都有的,只是这几样药偏偏不巧在两个时辰前偏偏被人买光了,你们不妨到别家试试。``小鱼儿悄悄走过去,从门隙里往外瞧,只见这大汉焦急得满头冷汗涔涔而落,不住顿足道:''怎地如此不巧!这城里几十家药铺,竟会都没有这几样药!"外面的店门半开,门外另一个大汉,牵着两匹健马,马嘴里不住往外喷着白沫,显然是经过长途急驰。

还有一人一马,远立在数尺外。星光下,只见马上人黑巾包头,黑氅长垂,目光顾盼间,星光照上她的脸──这人竟是女子。

店伙计举着烛火,急着要送客。突然,烛火一闪,马上的黑衣女子不知怎地己到了他面前,一双明媚的眼皮,看来竟锐利如刀!店伙计不由得一惊,踉跄后退,烛泪滴在他手背上,烫得钻心,他手一松,烛台直跌下来。

但烛台并未落在地上,不知怎地,竟到了这黑衣女子的手里,蜡烛也未熄灭,嫣红的烛光,正照着她苍白的脸!她的脸苍白得仿佛午夜的鬼魂。

她目光凝注着那店伙计,一字字道:``这些药,是被同一人买去的么?''店伙计也吓呆了,颤声道:``是\ldots\ldots 不是\ldots\ldots 是两个人!''黑衣女子道;``是什么人?''

她缓慢的语声,突然变得尖锐而短促,而且充满了怨毒,就连店伙计都听得忍不住机伶伶打了个寒酸,道:``不\ldots\ldots 不知道\ldots\ldots 咱们做买卖的,哪敢去打听顾主的来历。''黑衣女子锐利的眼睛仍在凝注着他,眨也不眨,似乎要瞧瞧他所说的话,究竟是真?是假?在这么样─双眼睛的注视下,有谁能说假话!

那店伙计的腿己被瞧软了,幸好黑衣女子终于转身,上马,打马\ldots\ldots 蹄声远去,去得比来时更快。

那店伙计就像是做梦一样,猛低头,只见那烛台就放在他胸前地上──这自然不是梦,他俯身拿起烛台``。''烛火突然又一花。这店伙计又一惊,刚拿的烛台又跌落下去。

但这次烛台还是没有跌落在地上,蜡烛也还是没有熄灭──一只手闪电般伸过来,恰巧接住了烛台。那店伙计大吓回头,就瞧见了小鱼儿。

小鱼儿手里拿着烛台,眼睛却瞧着远方,喃喃道,``想不到\ldots。想不到居然是她!''店伙计道:``她\ldots─她是谁?''

小鱼儿道:``她叫荷露,是移花宫的侍女\ldots\ldots\ldots 这些话告诉你,你也不懂得。''突然轻轻一跃,伸手抄住了那张被风卷起的纸,只见纸上写满了药铺的名字。

小鱼儿道:``她将这张纸丢了,显见已经将每一家药铺都找遍,还是买不着那些药\ldots。.''店伙计道:``奇怪,她为什么急着要买这几样奇怪的药?''小鱼儿微笑道:``这自然是因为他们家里有人生了种奇怪的病。''店伙计垂首道:``那会是什么病,居然要这几种大寒太热的药来治\ldots\ldots\ldots\ldots 这种病我简直连听都没有听说过,你听过么?''他抬起头,问小鱼儿。

烛台又被放在地上,小鱼儿已不见了!

\hypertarget{ux7b2cux56dbux5341ux4e94ux7ae0-ux6697ux85cfux5978ux8bc8}{%
\chapter{第四十五章
暗藏奸诈}\label{ux7b2cux56dbux5341ux4e94ux7ae0-ux6697ux85cfux5978ux8bc8}}

小鱼儿掠过几重屋脊,便又瞧见那三匹急驰的健马。

健马奔驰虽急,但又怎及小鱼儿身形之飞掠。马在街上跑,小鱼儿在屋顶上悄悄追随。

他心中也在暗问;``荷露为什么急着要买那几种药?莫非是有人中了极寒或极热的毒?这种毒难道连移花宫的灵药都不能解救?''他心念一转,又忖道:``下毒的人早知道他们要买这几种解药,所以先就将市面上这几种药都买光,显见是一心想将中毒的人置之于死地!..\ldots 下毒的人好狠的手段!但却不知是谁呢?''``中毒的人又是谁呢?难道是花无缺!''

他心思反复,也不知是惊是喜?

健马急驰了两三盏茶工夫,突然在一面高墙前停下,墙下有个小小的门户,像是人家的后门。门,并没有下栓。荷露一跃下马,推门而入。

小鱼儿振起双臂,蝙蝠般掠上高墙,他身形在黑暗中滑过,下面的两条大汉竟然毫没有觉察。

荷露轻喘急行,夜风穿过林梢,石子路沙沙作响,她解下包头的黑巾,发髻上有一明珠。

明珠在星光下闪着光。小鱼儿擦在树梢,缀着珠光。珠光隐人林丛,林中有三五间精舍。

小鱼儿隐身在浓密的枝叶中,倒出不虑别人发觉,他悄悄自林梢望下去,却瞧见了花无缺的脸。

这张俊逸、潇洒、安详、充满自信的脸,此刻却满带焦虑之色,他匆匆赶出门,看到荷露第一句话就问道:``药呢?''荷露手掌里揉着那包头的黑巾,悄声道:``没买到。''她这三个字其实还未说出口来,花无缺瞧见她面上的神色,自己的面色也骤然大变,一把夺过她手里黑巾,失声道:``怎\ldots\ldots 怎地买不到?''这无缺公子平时一举一动,惧是斯斯文文,对女子更是温柔有礼,但此刻却完全失了常态。

小鱼儿瞧见他这神态,已知道受伤的必定是和他关系极为密切的人,否则他绝不会如此失常,如此慌乱。

小鱼儿心里奇怪,暗中猜测,荷露和花无缺又说了两句话,他却没有听见,等他回过神来,两人已走进屋里。

灯光自窗内映出,昏黄的窗纸上,现出了两条人影,一人低垂着头,冠带簌簌而动,似乎急得发抖。这人不问可知,自是花无缺。

另一高冠长髯,坐得笔直,想来神情甚是严肃,小鱼儿瞧了半天也瞧不出这影予究竟是谁?

忽听得一个温和沉稳的语声缓缓道:``吉人自有天相,公子也不必太过忧郁''。``其实,荷露姑娘此番空手而回,在下是早已算定了的。''这语声一入耳,小鱼儿心里就是一跳。

只听花无缺叹道:``这几种药虽然珍贵,但却非罕有之物,诺大的安庆城竟会买不到这几种药,我委实想不透。''那语声接道:``那人算定了他下的毒唯有这几种大寒大热之药才能化解,也算走了公子必定知道这点,他若不将解药全都搜购─空,这毒岂非等于白下了。''这语声无论在说什么,都象是平心静气。从从容容,小鱼儿听到这里,已断定此人必是江别鹤!

想起了此人的阴沉毒辣,小鱼儿背脊上就不禁冒出了一股寒意,花无缺犹还罢了,他若被此人发现,哪里还有生路!小鱼儿躲在木叶中,简直连气都不敢喘了。

只听花无缺恨声道:``不错,此人自是早巳算定了连本宫灵药都无法化解这种冰雪精英凝成的寒毒,只是\ldots\ldots 他和他究竟有什么仇恨?为何定要将他置之于死地!''小鱼儿既猜不透他所说的第一个``他''指的是谁,更猜不透那第二个``他''指的是谁,心里急得要命。

江别鹤已缓缓接道:``此人要害的只怕不是他,而是公子。''花无缺道;``但我自入中原以来,也从未有与人结过什么仇恨,这人为何要害我?\ldots\ldots 这人又会是谁?我实在也想不透。''江别鹤似乎笑了笑,缓缓道:``只要公子放心铁姑娘的病势,随在下出去走一走,在下有八成把握,可以找出那下毒的凶手!''铁姑娘!中毒的人,莫非是铁心兰!小鱼儿这一惊真是非同小可,差点从树上掉下来。木叶``哗啦啦''一阵响动,只见花无缺的影子霍然站起,厉声道:``外面有人,谁?''小鱼儿紧张得一颗心差点跳出腔子来。

只听江别鹤道:``风吹木叶,哪有什么人?在下还是和公子先去瞧瞧铁姑娘的病势吧。''于是两人都离开了窗子。

小鱼儿这才松了口气,暗道,``这真是老天帮忙,江别鹤一向最富机心,今日总算疏忽一次\ldots\ldots{}''想到这里,他心头忽然一寒:``江别鹤一向最富机心,绝不会如此疏忽大意,这其中必定有诈!''小鱼儿当真是千灵百巧,心眼儿转得比闪电还快,一念至此,就想脱走,但饶是如此,他还是迟了!

黑暗中已有两条人影,有如燕子凌空般掠来!

小鱼儿惊慌中眼角一瞥,已瞧见来的果然是江别鹤与花无缺,花无缺衣袂飘风,望之有如飞仙,一双牌子在黑暗中闪闪发光,却是满含恨毒之色,想来必是以为躲在黑暗中的这人与下毒之事有关。

小鱼儿武功虽已精进,但遇着这两人,心里还是不免发毛,只是他出生入死多次,早已将这种生死险难看成家常便饭,此刻虽惊不乱,真气一沉,坐下的树枝立刻``咯嚓''一声断了,他身子也立刻直坠下去。

江别鹤与花无缺蓄势凌空,箭己离弦,自然难以下坠,更难回头,小鱼儿只听头顶风声响动,两人已自他头顶掠过。

他抢得一步先机,哪敢迟疑,全力前扑,方向正和江别鹤两人的来势相反,他算定两人回头来追时,必定要迟了一步,这其间虽仅有刹时之差,但以小鱼儿此时之轻功,江别鹤与花无缺只要这一刹时,也已追不着他了!

哪知江别鹤身子虽不能停,笔直前掠,但手掌却反挥而出,他手里竟早就扣着暗器,数点银星,暴雨般洒向小鱼儿后背!

花无缺身形凌空,突然飞起一足,踢着一根树枝,他竟借着树枝这轻轻一弹之力,整个身子都变了方向,头先脚后,倒射而出!去势之迅,竟和江别鹤反手挥出的暗器不相上下!

小鱼儿但闻暗器破空之声飞来,银星已追至背后!

他力已用光,不能上跃,只得扑倒在地,就地─滚,``噗,噗''一连串轻响过后,七点银星正钉在他身旁地上。

这其间生死当真只差毫发,小鱼儿掠魂末定,还未再次跃进,抬眼处,花无缺飘飘的衣袂,已到了他头顶!

花无缺身子凌空一滚,双掌直击而下!他身形矫捷如龙在天,掌力笼罩下,蝼蚁难逃!

哪知就在这时,钉在地上的七点银星突然弹起,正好打向花无缺,变生突然,花无缺眼看也难以闪避!

江别鹤虽是厉害的角色,却也未料到有此一着,对方竟将他击出的暗器用以脱身,他也不禁为之失声!

只见花无缺击出的双掌``啪''的一合,那七点寒星竟如夜鸟归林,全都自动投入了他的掌心!

这虽是刹那间事,但过程却是千变万化,间不容发!小鱼儿一掌将地上银星震得弹起后,人也借着这一掌之力直弹出去,百忙中犹不忘偷偷一瞥。

而江别鹤瞥见了花无缺这种惊人的内力,也不禁失声道:好!``而江别鹤也正为他这匪夷所思、妙不可言的应变功夫所惊大声道:''朋友好俊的身手,有何来意为何不留下说话!``小鱼儿头也不回,粗着嗓子道:''有话明天再说吧,今天再见了!``他话犹未了,花无缺已冷冷喝道:''朋友你如此身手,在下若让你就此一走,岂非太可惜了!"这话声就在小鱼儿身后,小鱼儿非但不敢回头,连话都不敢说了,用尽全力,向前飞掠。

只见一重重屋脊在他脚下退过,他也不知掠过了多少重屋脊,却竟然还未掠出这一片宅院!

只听江别鹤道:``这位朋友看来年纪并不大,不但身手了得,而且心思敏捷,江湖中出了这样的少年英雄,在下若不好生结交结交,岂非罪过。''他一面说话,一面追赶,竟仍未落后,语气更是从从容容,似是心安理得,算定小鱼儿逃不出他的手去。

花无缺道:``不错,就凭这身轻功夫,纵不算中原第一,却也难能可贵了!''他心里也在暗中奇怪,自己怎会至此刻还追不上。

要知他轻功纵然比小鱼儿高得一筹,但逃的人可以左藏右躲,随意改变方向,自是比追的人占有了便宜。

只听江别鹤又道:``此人不但轻功了得,面且中气充足,此番身形已展动开来,只怕你我难以追及。''小鱼儿听了这话,突然一伏身窜下屋去,哪知小鱼儿更是个鬼灵精,江别鹤不说这话,小鱼儿惊慌中倒未想及,一说这话,反倒提醒了他。

江别鹤暗中跌足,只见小鱼儿在曲廓中三转两转,突然一头撞开一扇窗户飞身跃了进去。

这时宅院中灯火多已熄灭,他虽然不知道屋里有人没人,但这宅院既然如此宏阔,想来自然是空屋子较多。

屋子果然是空的。

小鱼儿刚喘了口气,只听``嗖的一声,花无缺竟也掠了进来,接着又是''嗖"的一声,江别鹤也未落后。

屋子里黑黝黝的,什么都瞧不见的。小鱼儿向前一掠,几乎撞倒了一张桌子。

江别鹤笑道:``朋友还是出来吧,在下江别鹤,以江南大侠的名声作保,只要朋友说得出来历,在下绝不难为你。''这话若是说给别人听,那人说不定真听话了,但小鱼儿却非但知道这``江南大侠是怎么样的人,还知道他们若是知道自己是谁,定非''难为"不可的。

江别鹤道;``朋友若不听在下好言相劝,只怕后悔就来不及了。''小鱼儿悄悄提起那张桌子,往江别鹤直掷过去,风声鼓动中,他已飞身扑向左面一个角落。

他算定左面的角落必定有扇门口,他果然没有算错,那桌子``砰''的落下地,他已踢开门窜了出去。

这间屋子比外面更黑,黑暗对他总是有利的。

小鱼儿藏在黑暗中,动也不敢动,正在盘算着脱身之计,突然眼前一亮,江别鹤竟将外面的灯点着了。

小鱼儿随手始起了椅子,直摔出去,人已后退,``砰''地,又撞出了窗户,凌空一个翻身,撞入了对面一扇窗户。

他这样``砰砰蓬蓬''的一闹,这宅院里的人,自然已被他吵醒了大半,人声四响,喝道:是什么事?什么人?江别鹤郎声道:``院中来了强盗,大家莫要惊慌跑动,免受误伤,只需将四下灯火燃着,这强盗就跑不了的!''小鱼儿心里暗暗叫苦,这姓江的确有两下子,说出的话,正在节骨眼上,要知小鱼儿就希望院中大乱,他才好乘乱逃走,他更希望灯火莫要燃着,灯火一燃,他非但无所逃,连躲都没处躲,正是要了他的命了。

只听四下人声呼喝,纷纷道,``是江大侠在说话,大家都要听他老人家的吩咐。''接着,满院灯火俱都亮了起来.小鱼儿转眼一瞧,只见自己此刻是在间书房里,这书房布置得出奇精致,书桌旁却有个绣花棚子。

他心念一转:``书房里怎会有女子的绣花棚?''江别鹤与花无缺已到了窗外。小鱼儿退向另一扇门,门后突然传出入语声,道;``外面是谁!''这竟是女子的语声。

门后有人,小鱼儿先是一惊,但心念转动,却又一喜,再不迟疑,又一脚踢开了门,闯了进去。

他算定江别鹤假仁假义,要自恃``江南大侠''的身份,决不会闯进女子的闺房,而花无缺更不会在女子面前失礼。

但小鱼儿可不管什么女人不女人,一闯进门,反手就将灯灭了火,眼角却已瞥见床上睡着个女子,他就窜过去,闪电般伸手掩住她的嘴,另一只手接着她的肩头,压低嗓子道:``你若不想受罪,莫要出声!''哪知这女子竟是力大无比,而且出手竟也快得很,小鱼儿的两只手竟被她两只手活生生扣住!

这又是个出人意料的变化,小鱼儿大惊之下,要想用力,这女子竟已将他按在床上,手肘压住了他咽喉!

小鱼儿骤出不意,竟被这女子制住,只觉半边身子发麻,竟是动弹不得,他暗叹一声,苦笑道:``罢了,罢了\ldots。我这辈子大概是注定要死在女人手上的了。''这时江别鹤的语声已在外面响起。

他果然没有径自闯进来,只是在门外问道:``姑娘,那贼子是闯进姑娘的闺房了么?''小鱼儿闭起眼睛,已准备认命。

只听这女子道:``不错,方才是有人闯进来,但已从后面的窗子逃了,只怕是逃向小花园那边,江大侠快去追吧。''小鱼儿作梦也想不到这女子竟是这样回答,只听江别鹤谢了一声,匆匆而去,他又惊又喜,竟呆住了。

小鱼儿终于忍不住道:``姑\ldots 姑娘为什么要救我?''那女子先不答话,却去掩起了门。

屋子里伸手不见五指,小鱼儿也瞧不清这女子的模样,心里反面有些疑起来,一跃而起,沉声道:``在下与姑娘素不相识,蒙姑娘出手相救,却不知是何缘故?那女子''噗哧``一笑,道:''你与我真的素不相识?``小鱼儿道:''与我相识的女人,都一心想杀我,绝不会救我的。``那女子大笑道:你莫非已吓破了胆,连我的声音都听不出了。''她方才说话轻言细语,此刻大笑起来,却有男子的豪气小鱼儿立刻听出来的,失声道:``你,你是三姑娘?你怎会在这里?''三姑娘道:``这是我的家,我不在这里在哪里?''小鱼儿怔了怔,失笑道:``该死该死,我怎未看出这就是段合肥的屋子\ldots\ldots 这见鬼的屋子也委实太大了,走进来简直像走进迷魂阵。''三姑娘笑道:``莫说你不认得,就算我,有时在里面都会迷路。''小鱼儿道:``但那江别鹤与花无缺又怎会在这里?''叁姑娘道:``他们也就是为那趟镖银失劫的事而来的。''小鱼儿叹道:``这倒真是无巧不巧,鬼使神差,天下的巧事,竟都让我遇见了,江别鹤竟会在你家,我竟会一头闯进你的屋子''三姑娘笑嘻嘻道:``他们可再也想不到我认识你。''小鱼儿道:``否则那老狐狸又怎会相信你的话。''要知道江别鹤正是想不到段合肥的女儿会救一个陌生的强盗,所以才会被三姑娘一句话就打发走了。

三姑娘道:但\ldots 但你和江大侠又怎会?怎会?``小鱼儿冷笑道:''江大侠\ldots\ldots 哼哼,见鬼的大侠。``三姑娘奇道:''江湖中谁不知道他江南大侠的名声,他不是大侠,谁是大侠。``小鱼儿道:''他若是大侠,什么乌龟王八屁精贼,,全都是大侠了。``三姑娘笑道:''你只怕受了他的气,所以才会那么恨他,其实他倒真是个好人,听说我家镖银被劫,立刻就赶来为我们出头``\ldots\ldots{}''小鱼儿冷笑道:他这是黄鼠狼给鸡拜年。"

叁姑娘道:``你说他不存好心,但他这又会有什么恶意?''小鱼儿道:``这些人的心机,你一辈子也不会懂的。''三姑娘斜身坐到床上,就坐在小鱼儿身旁,她的心``砰砰''直跳,垂着头坐了半晌,又道:``那位花公子,也是江\ldots。江别鹤请来的''小鱼儿道:``哦?''

三姑娘道:``据说这位花公子,是江湖中第一位英雄,又是天下第一美男子,但我瞧他那副娘娘腔,却总是瞧不顺眼。''小鱼儿听她在骂花无缺,当真是比什么都开心;拉住了她的手,笑道:你有眼光,你说得对。``三姑娘道,''我\ldots\ldots 我\ldots"

她在黑暗中被小鱼儿拉往了手,只觉脸红心跳,喉咙也发干了,连一个字都再也说不出来。

小鱼儿想了想,忽然又道:``你说的那位花公子,他是否有个朋友中了毒?''三姑娘道;``你怎会知道的?''

小鱼儿道:``他的本事这么大,怎会让自己的好朋友被人下毒?''三姑娘道:``昨天下午,那位花公子和江大\ldots\ldots\ldots 江别鹤一起出去了,只留下铁姑娘一个在客房里,却有人送来一份札,要送给花公子,是铁姑娘自己收下的,礼物中有些点心食物,铁姑娘只怕吃了些,谁知竟中毒了。''小鱼儿道:送礼的是谁?"

叁姑娘道:``礼物是直接交给铁姑娘的,别人都不知道。''小鱼儿道:``她难道没有说?''

叁始娘道:``花公子回来了,她已中毒晕迷,根本说不出话了。''小鱼儿皱眉道:``她怎会如此大意,随便就吃别人送来的东西?''想了想,又沉吟道:``那送礼的想来必定是个她极为信任的人,所以她才毫不疑心地吃了\ldots。但一个被她如此信任的人,又怎会害她?''三姑娘叹了口气,道:``那位铁妨娘,可真是又温柔,又美丽,和花公子倒真是一对壁人,她若没救,倒真是件可惜的事。''。

小鱼儿咬住牙道,``你说她和花\ldots。.''

三姑娘道:``他们两人真是恩恩爱爱,叫人瞧得羡慕,尤其是那花公子对她,更是千依百顺,又温柔、又体贴\ldots{}''小鱼儿只听得血冲头顶,人都要气炸了,忍不住大声道:``可恨!''三姑娘道:``你\ldots\ldots 你说谁可恨?''

小鱼儿吐了口气,缓缓道:``我说那个下毒的人可恨。''三姑娘道:``直到现在为止,花公子和江别鹤还都不知道下毒的人是谁\ldots\ldots{}''小鱼儿瞪着眼睛笑,道:``他对她虽然又温柔、又体贴,但却救不了她的性命\ldots。嘿嘿\ldots\ldots 嘿嘿\ldots\ldots{}''三姑娘听他笑得竟奇怪得很,忍不住问道:``你\ldots\ldots{}''你怎么样了?``小鱼儿道,''我很好,很开心,简直从来没有这么开心过。``三姑娘垂下了头,道:''你\ldots。你和我在一起,真的很开心么?"别人说男孩子会自我陶醉,却不知女孩子自我陶醉起来,比男孩子更厉害十倍。

小鱼儿默然半晌,突然又拉起三姑娘的手,道:``我现在求你一件事,你答应么?''三姑娘脸又红了,心又跳了,垂着气,喘着气道:``无论求我什么,我都答应你。''小鱼儿喜道:``我求你将我送出去,莫要被别人发觉。''三妨娘又好像被人抽了一鞭子,整个人又呆住了。

也不知过了多久,她终于颤声道:"你\ldots\ldots\ldots 现在就要走?

好,我送你出去。``三姑娘突然放声大喊道:''来人呀\ldots\ldots\ldots 来人呀\ldots 这里有强盗!``小鱼儿的脸立刻骇白了,一把扣住三姑娘的手,道:''你\ldots。.你这是干什么?``只听衣袂带风之声响动,江别鹤在窗外道:''姑娘休惊,强盗在哪里?"他来得好快!

小鱼儿又惊,又怨,又恨。

``女人\ldots\ldots\ldots\ldots 女人\ldots\ldots\ldots\ldots\ldots 她为了要留住我,竟不惜害我!我早知女人都是祸害,为何还要信任她!''他已准备一冲,只听三姑娘道:``方才我瞧见一人,像是往铁姑娘住的地方\ldots\ldots\ldots\ldots{}''她未说完,花无缺已失声道:``呀\ldots\ldots\ldots\ldots\ldots 不好!我们莫要中了那贼子调虎离山之计,快走!''接着,风声一响,人已去远。

小鱼儿又松了口气,苦笑道:``你真吓了我一跳。''三姑娘悠悠道:``你放心,我不会害你的。我将他们引开,我才好帮你走。''她抓起件大氅,摔在小鱼儿身上道:``披起来,我带你出去。''小鱼儿心里也不知是何滋味,喃喃道:``女人\ldots\ldots\ldots\ldots\ldots 现在简直连我也弄不清女人究竟是种什么样的动物!''

三姑娘道:``你说什么?''

小鱼儿道:``没有什么,我在说\ldots\ldots\ldots\ldots\ldots 你真是我见到的女孩子中最老实的一个。''幸好三姑娘身材高大,小鱼儿披起她的风氅,长短大小,都刚合适,两人就从廊上大模大样走出去。

三姑娘将小鱼儿带到偏门,开了门,回过去,淡淡的星光,正照着小鱼儿那倔强,调皮,却又充满了魅力的脸。

三姑娘轻轻叹了口气,道:``你\ldots\ldots\ldots\ldots 你还会来看我么?''小鱼儿笑道:``我自然会的,我今天就会\ldots\ldots{}''他一面说话,人已匆匆跑了。

三姑娘瞧着他背影去远,犹自呆呆的出神,只觉心中泛起一股滋味,也不知是愁、是喜,竟是她平生从未感觉过的。

小鱼儿匆匆奔回那药铺。

到了那条街上,``庆余堂''的金字招牌在星光下已可隐隐在望,小鱼儿的脚步也立刻缓了下来。

他鼻子东闻西嗅,眼睛东张西望,突然蹲下身子,喃喃道:``是了─一─''只见光亮的青石板上,有一些药末,前面六七尺外,又有一些,小鱼儿眼鼻俱用,一路查了下去。

原来他昨夜以石子将两条大汉买走的两大包药击穿个小洞,正是药包中药漏下,他只要寻得漏下的药末,也自然就可查出那药包是送往何处的,他年纪虽小,做事却极是周到,不但早已伏下这线索,而且早已算定在这深夜之中,街上无人行走,绝不会将漏下的药末踏乱。

到后来根本无需再低头搜索,只凭着清冷的夜风中吹来的一丝药味,他已不会走错路途。

这样走了约莫两盏茶时分,道路竟越来越是荒僻,前面一片池塘,水波粼粼。

只见这池塘不远,果然又有一片庆院,看来纵然不及段合肥的宅院精雅,但依山傍水,气势却更是宏大。那药包竟是径自送到这庄院来的。

小鱼儿微一迟疑,四下瞧了瞧,深夜之中,这庄院里居然还亮着灯火,黑漆的大门也有个牌子!``天香塘,地灵庄,赵。''小鱼儿暗道,``瞧这气派,这姓赵的不但有财有势,而且还必定是个江湖人物,他们深更半夜的不睡觉,想来不会在做什么好事。''他胆子本就大得出奇,再加上近来武功精进,更是满不在乎,竟向有灯光的地方,笔直掠了过去。

那是间花厅。小鱼儿垂在檐下,小指蘸着口水,在窗纸上点了个小小的月牙洞,花厅里正有四个人坐在那里喝酒。

他眼睛只盯住厅左的一个角落,这角落里大包小包,竟堆满了药,自然正是附子、肉桂、犀角、熊胆\ldots。只听一人道:``无论如何,三位光临鄙庄,在下委实受宠之至,在下再敬三位一杯。''这人坐在主座,又高又瘦,一张马脸,扫帚眉,鹰钩鼻,双颧同耸,目光锐利,看来倒有几分威棱。

小鱼儿暗道:``这人想必就是姓赵的。''

又听另一人笑道:``赵庄主这句话已不知说多少遍了,酒也不知敬过多少次,赵庄主再如此客气,我兄弟委实不安。''第三人笑道:其实,我兄弟能做赵庄主的座上客,才真是荣幸之至,我兄弟倒真该好生来敬赵庄主一杯才是。"这两人同样的园脸,肥颈,同样笑眯得起来的眼睛,同样慢条斯理的说话,长得竟是一模一样。

小鱼儿暗笑道:``这两个胖子竟是一个模子里铸出来的。天下的双胞胎虽多,但兄弟两人长得这么像的倒是少有。''这三人他全不认得,他更猜不出他们为何要害铁心兰,他心里正在揣摸,突见第四人回过头来。``这人白发银髯,气派威严,竟是那武林中人人称道、领袖三湘武林的盟主,爱才如命''铁无双。

瞧见此人,小鱼儿倒真吓了一跳。

原来下毒的竟是铁无双!

这就难怪铁心兰那么信任,毫不怀疑地就吃了送来的礼,爱才如命"铁无双这七字,自然是人人信得过的!

想不到这铁无双竟也和江别鹤─样,是个外表仁义,心如蛇蝎之辈,但他为何要害铁心兰呢?

\hypertarget{ux7b2cux56dbux5341ux516dux7ae0-ux5de7ux8bc6ux6bd2ux8ba1}{%
\chapter{第四十六章
巧识毒计}\label{ux7b2cux56dbux5341ux516dux7ae0-ux5de7ux8bc6ux6bd2ux8ba1}}

一时之间,小鱼儿心里已打了十七八个转,正是又惊又疑,只是他纵然不信,事实却又偏偏摆在眼前。

只见那赵庄主又倒了杯酒,举酒笑道:``贤昆仲与铁老前辈惧是今世之英雄,赵香灵何德何能,竟蒙三位不弃,来\ldots 来来,在下再敬三位一杯。''那兄弟两人立刻举起酒杯,铁无双却动也不动。

坐在左首的那胖子眼珠子一转,立刻陪笑道:``我兄弟江湖后辈,无名小辈,怎敢与铁老前辈并驾齐驱,若不是庄主见召,我兄弟哪有资格与铁老前辈饮酒。''另一人也笑道:``正是如此,江湖中人若是听见罗三、罗九竟能赔着铁老前辈在一起喝酒,真不知要羡慕到何种程度。''铁无双哈哈大笑,立刻举杯笑道:``两位太谦了,老夫两耳不聋,也会听得罗氏兄弟行起江湖,侠肝义胆,哈哈\ldots\ldots 哈哈,哈,老夫敬贤昆仲一杯。''小鱼儿暗笑道:``这当真是千穿万穿,马屁不穿,铁无双自命不凡却也受不得两句马屁的!这罗家兄弟马屁拍得如此恰到好处,想来必定不是好东西。''只听那赵香灵笑道;``三位俱都莫要太谦了,铁老前辈固是德高望重,人人钦仰,但贤昆仲又何尝不是当世之杰。''他转向铁无双笑道:``铁老前辈有所不知,罗氏昆仲两位,虽然是近年才出道江湖,但一出手就重创了太湖七煞,接着又做了齐鲁五虎,在太行山上兄弟两人独战三刀十八寇,那一仗更是打得堂堂皇里,轰轰烈烈。''铁无双道:``这倒怪了,这些大事,老夫竟不知道。赵香灵道:''前辈又有所不知,他兄弟两人为着不欲人知,无论做了什么事,都不愿宣扬,就凭这样的心胸,已是人所难得。``铁无双笑道;''好,好,这样的朋友,老夫必定要交一交的,只是\ldots。两位看来显然是孪生兄弟,为何一个行三,一个却行九?``罗叁笑道:''晚辈只是以数字为名,与排行并无关系。``罗九笑道;''其实我是老大,他是老二。"

铁无双附掌笑道:``这倒妙极,别人若是听了你们名姓,只怕谁也不会想到罗九竟是兄长,而罗三却是弟弟。''他语声微顿,又道;``两位如此了得,却不知出自哪一位名师的门下?再也不知两位出道为何如此之晚,直至三年前,老夫才听到两位的名字。''罗九笑道:``我兄弟从小爱武,所以在家里练了几手三脚猫的把式,也没有什么师承,四十岁,老母在堂,我兄弟不敢远游,是以直到家母弃世后,才出来走动的。''铁无双叹道:``不想两位不但是英雄,而且还是孝子。''罗叁笑道:``岂敢岂敢。''

铁无双道:``只是,想那七煞、五虎、三刀、十八寇,但是黑道中有名的硬手,两位既然一一打发了他们,若说不是出自名门,老夫委实难信。''罗九道:``晚辈在前辈面前,怎敢有虚言。''

铁无双笑道;"如此说来,两位更可算得上不世之奇才,自创的武功,竟能也有如此精妙,不知两位可否让老夫开开眼界\ldots\ldots{}

罗三道:``在前辈面前,晚辈怎敢献丑。''

铁无双道:``两位务必要赏老夫个面子。''

罗三道:``晚辈的确不敢。''

铁无双作色道:``两位难道瞧不起老夫,竟不肯给老夫个面子么?''赵香灵赶紧笑道:``铁老前辈人称爱才如命,听得贤昆仲如此奇才,想必早已动心了,两位的确不该扫铁老前辈的兴。''罗三苦笑道:``庄主也\ldots\ldots{}''

赵香灵截口笑道:``说老实话,在下也的确想瞧瞧两位一显身手。''罗九长身而起,笑进:``既是如此,晚辈恭敬不如从命,献丑了。''这兄弟两人人虽肥胖,身材却高得很,两人略挽了挽衣袖,竟在这花厅中施展开拳脚。

这时不但赵香灵与铁无双聚精会神的瞧着,就连窗外的小鱼儿也瞪大了眼睛瞧得目不转睛。

只见这罗九双掌翻飞,使的竟是一路``双盘掌'',罗三拳风虎虎,打的却是一套``大洪拳''。

这兄弟两人拳掌快捷,下盘扎实,身手可说是十分矫健,但招式却毫无精妙之处可言。

要知道``双盘掌''与``大洪拳''正是江湖中中最常见的把式,可说是连赶车的、拍轿的都会使两手。

铁无双竟像是瞧呆了,他不是惊于这兄弟武功之强;而是惊于这兄弟武功之差,这样的武功使出来,实在是在``献丑''。

只见两人使完了一趟拳,脸竟也似有些红了,抱拳笑道:``前辈多多指教。''铁无双道:``嗯\ldots\ldots 嗯\ldots。.''

赵香灵笑道:``罗氏昆仲的武功,当真是扎实已极,这样的武功虽不中看,但却最能实用\ldots\ldots 老前辈以为如何?''铁无双道:``嗯\ldots\ldots 不错\ldots\ldots 不错。''

他嘴里虽然在说``不错'',却已掩不住语气中的失望之意,他对这兄弟两人,委实已再没什么兴趣。

但小鱼儿对这两人的兴趣却更大了。

他心中暗道:``这兄弟两人八面玲珑,深藏不露,竟连铁无双这样的老江湖都瞒过了,竟瞧不出他们的武功绝不只此。这两人如此做法,不但隐藏了自己武功的门路,也消除了别人的警惕,从此不会再对他两人存有戒心,这两人竟宁愿被人瞧不起,这是何等深沉的城府,这种人我倒真要小心提防着才是。''小鱼儿虽已瞧出这两人必定暗藏机心别有图谋,却也猜不透这两人图谋的究竟是什么事,他自然更猜不透这两人的来历。

这时赵香灵又举起酒杯,笑道:``今夜虽然被这件无头公案吵得无法安睡,但能瞧见两位罗兄的身手,又能陷铁老前辈畅饮通宵,倒当真是因祸得福了。''小鱼儿正又暗自讨道:``无头公案?\ldots。什么无头公案?''就在这时,只听庄外突然传人一阵马嘶车声。

铁无双推杯而起,变色道:``莫非又来了!''

语声中他身形已直窜出来!庄外果然驰来一辆马车。开了庄门,车子使直驰而入,但车上却没有人赶车。

赵香灵吩咐家丁,卸下了车上的包裹,刚打开包裹,便有一阵药香扑鼻面来,包里的正是附子、肉桂、犀角、熊脑\ldots"小鱼儿暗自瞧得清楚,当真又吃了一惊,灯光下,只见赵香灵、铁无双面上也都变了颜色。

赵香灵道:``这究竟是怎么回事?一晚上连着七八次,无缘无故的将这药送来,这难道有人在开玩笑,恶作剧?''铁无双皱眉道:``这些药材俱都十分珍贵,谁会将这些珍贵之物来开玩笑。''赵香灵道:``依前辈看来,这是怎么回事?''

铁无双沉吟道:``这其中说不定有什么恶计。''赵香灵道:``但这些药材非但没有毒,而且有的还补得很,送这些药来又害不到咱们的\ldots\ldots 罗兄可猜得出这究竟是何缘故么?''罗九笑道:``铁老前辈见多识广,所言必有道理。''铁无双叹道:``老夫委实也有些莫名其妙。''

他虽然其名其妙,小鱼儿却已猜透了。

他喃喃暗道:``好呀,这原来是你们要栽赃,你们将解药送到这里,好教花无缺以为下毒的人是铁无双,这原来是个连环计\ldots\ldots 好阴毒的连环计,可惜的是,这件事竟遇上了我江小鱼,这真算你们倒大霉了。''他眼珠子一转,竟悄然而去了,他乘着夜色,寻了家专卖脂粉白垩之类的铺子,越墙而入,出来时手里却是满载而归,大包小包提了一手。

于是,天亮时,他已换了副面目,只见他一张白兮兮的脸,两只睡眼泡,一张猪公嘴,活像个妓院里的大茶壶,他从屠娇娇处学来的易容术,果然没有白废。

小鱼儿寻了家最热闹的茶馆,大吃了一顿,他一连吃了两笼蟹黄汤包,四套油炸果子,外带一大碗热汤才住手,他知道今天必定要大出力气,人是吃饱了才有力气的。

茶馆外还有早市,人来人往,热闹得很,一条削长汉子太阳腮上贴着块膏药,手拎着鸟笼,在人丛里转来转去,别人袋里的散碎银子就全都变成了他的。

小鱼儿缀上了他,走到人少处,突然一拍肩头,笑道;``朋友手脚倒蛮快的呀。''那青皮无赖一回头,怒道:``小杂种,你吃饱了撑的得难受么?''反手一个耳光,就往小鱼儿脸上煽了过去,但他一辈子也休想碰着小鱼儿的脸。小鱼儿用两根手指,轻轻刁住他腕子,轻轻一捏,这蛮像样的一条大汉立刻疼得不像样子。

小鱼儿笑嘻嘻道:``谁是小杂种?''

那青皮无赖疼得满身冷汗,道:``我\ldots\ldots 我是小杂种,标标准准的小杂种,小爷,小祖宗,你就饶了我这个小杂种吧,我袋子里的全送给你老人家。''小鱼儿道:``只要你老老实实回答我几句话,我非但不拿你袋里说,说不定还会装满它,你瞧怎么样?''那青皮道:``好''\ldots 自然好\ldots\ldots"

小鱼儿刁着他的手,道:``你可知道天香塘,地灵庄这地方。''那青皮道:``小人若不知道,还能在城里混么?''小鱼儿道:``那赵庆主是怎么样的人?''

那青皮道;``赵庄主家财百万,人缘四海,黑白两道,都很吃得开,只是\ldots\ldots 自从段合肥来了之后,他生意总是被段合肥打垮,他想武的,哪知段合肥居然也养了一群江湖上的朋友,而且字号比他家的更响。''小鱼儿眼珠子一转,喃喃道:``这就对了,\ldots\ldots 赵香灵把铁无双找来,想必是想借铁无双的名头来镇压段合肥的,而这点恰巧又被人利用了。''那青皮也听不清他说的是什么,只是哀求着道:``小爷,你老人家现在可以放手了么?''小鱼儿笑道:``你整天东溜西逛,这城里你必定熟得很,赵家庄里想必也有你的熟人,只要你带我进去见他,让我在庄子里耽一天,我给你三百两银子,你肯么?''这还有不肯的么?为了三百两银,这青皮简直可以把自己的老婆都卖了。

像赵家庄这样的地方,自然是龙蛇混杂,什么人都有,家丁里自然不乏一些混混儿,这些自然就都是那青皮的同伴。

小鱼儿用小手段,就和他们混在一起了,还不到一个时辰,这些人都已将小鱼儿看成好朋友。

使小鱼儿想不到的是,那赵香灵居然一早就来到前厅,精神奕奕,顾盼自得,居然丝毫看不出昨夜曾痛饮通宵的模样。

过了不久,外面川流不息的有人来,看样子都是生意买卖人,见了赵香灵,神情俱都恭恭敬敬。

小鱼儿站得远远的,拉住个家丁问道:"这些人是干什么的?

来得怎地如此早?"

那家丁道:``这些人都是我家庄主派往外面店铺的掌柜,每天早上都要到庄里来报告头一天的生意情况,除了这些人外,我家庄主早上从不见客。''小鱼儿微微一笑,道:``有些客人,你家庄主不见只怕也不行。''那家丁自然听不出小鱼儿话中的深意,笑道:``这,天香塘,地灵庄,难道还有人敢硬闯进来不成。''小鱼儿眨了眨眼睛,道:``段合肥呢?''

那家丁哗道:``那肥猪,我家庄主迟早要将他满身肥肉红烧了来吃。''小鱼儿道:``原来你家庄主与那段合肥冤仇倒大得很。''那家丁道:``他知道我家庆主在哪里有买卖,就在对面也开一家,他知道我家庄主有哪些大主顾,就不惜一切去结纳,咱们天香塘和段合肥委实仇深似海。''小鱼儿笑道:``想不到商场竟也和战场一样,看来在商场上结下的仇人,竟比战场上的仇人恶毒还要深。''那家丁道:``做生意讲究本份,像段合肥用这种卑鄙手段,简直不是人。''说话之间,赵香灵已叁言两语,将那些掌柜的一一打发走,端起碗茶啜了两口,吩咐道:``去瞧瞧客人们,若已起来,调到前厅用茶。''小鱼儿在门房外的树荫下寻了块石头坐下,喃喃道:``若是我猜得不错,现在只怕巳该来了!''就在这时,只听门房那里传来一阵人语声,道,``相烦请将名帖送上贵庄主,就说在下前来拜访。''门房道:``抱歉得很,我家庄主正午从来\ldots。.''语声突然顿住,像是瞧见帖上的名字吓了一跳。

小鱼儿听得那语声、又是紧张,又是欢喜,喃喃道:``来了来了,果然来了。''那家丁已匆匆忙忙上前厅,捧上名帖!赵香灵皱眉接过,但瞧了一瞧,变不禁动容失声道,``江南大侠江别鹤来了。''铁无双肩耸然长身而起,还未说话,厅外已有人朗声笑道,江别鹤前来求见庆主,庄主难道不见么?"两人大步走上厅前石阶,前面一人神采飞逸,正是江别鹤,后面跟着的却是个丰神如玉的美少年。

再后面竟还有四条大汉抬着顶绿呢软轿,轿深垂,也不知里面坐的究竟是何许人也。

赵香灵赶紧抢步迎出,抱拳笑道:``在下不知江大侠光临,有失远迎,恕罪,恕罪。''江别鹤淡淡笑道:这位是花公子,花无缺。"他故意淡淡说来,赵香灵、铁无双、罗九、罗三听见花无缺这三字,都不禁耸然动容。

铁无双目光上下一扫,笑道:``这位兄台竟是近来名震八表的无缺公子,果然是少年英俊,人中之鹤,当真幸会已极。''花无缺路冷道:``幸会幸会。''

赵香灵笑道:``这位铁老前辈,两位想必已不认得了,但这两位罗兄\ldots\ldots{}''当下将罗九、罗三介绍,自然不免又吹嘘了一番。

花无缺却似完全没有听到,鼻子里似乎嗅着了什么气味,突然袍袖一拂,轻飘飘离座面起。

众人只觉眼前人影一闪,他竟已掠入旁边的花厅,目光又一花,他已从花厅掠出,手里抓着一把药,面色更是惨白,嘎声道,``果然在这里。''赵香灵道:``这些药莫非是公子的么?在下正不知是谁送来的,昨夜─\ldots.''江别鹤似笑非笑,截口道:``庄主难道真不知是谁送来的么?''赵香灵瞧了瞧他,又瞧了瞧花无缺的面色,就知道这其中必定牵涉极为严重,强笑道:``这\ldots\ldots 这究竟是怎么回事?''江别鹤道:``这件事说来也简单得很,有人下毒害了花公子未来的夫人,却将市面上的解药全都搜购一空,这是怎么回事?''赵香灵道:``这正是要绝花公子未来夫人的生路。''江别鹤道;``不错,如此说来,搜购解药的人,是否就是那下毒的人呢?''赵香灵道:``自然!''

江别鹤淡淡一笑,道:``这就是了。''

赵香灵想了想,面色突变,失声道:``那''\ldots 那些解药莫非现在花厅之中?``江别鹤一字字道:''正是!"

赵香灵跳了起来,道:``但\ldots\ldots 但在下委实不知此事\ldots。那些解药是昨天有人送来的。''江别鹤道:``是谁送来的?''

赵香灵道:``在下也不知是谁。''

江别鹤冷笑道:``不知是谁?难道还有人会无缘无故的将这些珍贵的药物平白送人么?赵庄主说这话,未免将江某看成小孩子了。''要知这件事说来的确是荒谬已极,的确是绝不可能,赵香灵既无言可辩,满头汗珠滚滚而落。

铁无双长身而起,大声道:``老夫可以身家替赵庄主作保,那药的确是别人送来,赵庄主的确不知那人究竟是谁!''江别鹤瞟了他一眼,淡淡道:``赵庄主若不知道,阁下就想必是知道的。''铁无双怒道:``你\ldots 你说什么?''江别鹤冷冷一笑,再不瞧他,也不答话。

\hypertarget{ux7b2cux56dbux5341ux4e03ux7ae0-ux8ba1ux4e2dux4e4bux8ba1}{%
\chapter{第四十七章
计中之计}\label{ux7b2cux56dbux5341ux4e03ux7ae0-ux8ba1ux4e2dux4e4bux8ba1}}

这时那花无缺才自轿中缩回头来,原来那轿中正是铁心兰,他已将解药喂入铁心兰嘴里。

如此生吞解药,药力虽不能完全发挥,但总可稍解毒性,再加上花无缺以高深的内力相助,果然过了一会儿,轿中便有呻吟声传了出来。

花无缺松了口气,缓缓转过身子,目光缓缓自众人面上扫过,那目光正如厉电一般,直瞧得众人背生寒意。

花无缺一字字道:``是谁下的毒?''

赵香灵抹了抹汗,道,``在下的确不知。''

江别鹤瞧了罗九、罗三一眼,忽然问道:``这药难道真不是铁老英雄与赵庄主买来的?''罗九、罗三对望一眼,罗九缓缓道:``我兄弟什么都不知道。''铁无双怒道:``但你们明明知道,昨夜你们也亲眼瞧见的!''罗叁道:``我兄弟只瞧见药自己来了,却不知是谁送来的,说不定是张三,说不定是李四,也说不定是\ldots。''瞧了铁无双一眼,住口不语。

江别鹤道:``也说不定就是铁老英雄的门下.是么?''罗九、罗三对望一眼,也不答话,竟无异是默认了。

江别鹤目光凝注铁无双,悠悠道:``阁下还有何话说?''铁无双却怒目瞧着罗氏兄弟,厉声道:``你两人怎敢如此?''罗九道:我兄弟只是说老实话。"

江别鹤道;``贤昆仲当真是信义之人,在下好生相敬,但铁老英雄么\ldots\ldots 嘿嘿。''铁无双须发皆张,忽喝道:``老夫怎样?''

江别鹤不再答话,却走到软轿前,唤道:``铁姑娘,铁姑娘醒来了么?''铁心兰的语声在轿中呻吟着道:``嗯。\ldots 我冷得很!''江别鹤道:``铁姑娘可知是被谁下毒的么?''

这句话问出,厅中人惧都紧张了起来。

只听铁心兰道:``我。\ldots 我是中毒了么?我也不知道是谁下毒的\ldots。''赵香灵刚松了口气,铁心兰已接着道,``我只知吃了铁无双送来的两粒枣子,就全身发冷,直打冷战,不到片刻,已晕迷不省人事了。''这句话说出来,人人都变了颜色。

铁无双顿足道:``你\ldots\ldots 你为何要血口喷人?''江别鹤道:``阁下此刻还想狡赖,未免不是大丈夫了。''铁无双怒道:``放屁!老夫与她一不相识,二无仇恨,为何要害她?''江别鹤道;``花公子,你听这话如何?''

花无缺究竟不是常人,到此刻竟还能沉得住气,脸上神色虽更难看,但居然还是动也不动,只是缓缓道:``我等出手之前,总得要人口服心服。''江别鹤笑道:``正该如此。,突然向那抬轿的轿夫招了招手,道:''过来。``那轿夫应命面来,躬身道:''江大侠有何吩咐?``众人正不知江别鹤在这紧张关头,突然令这轿夫前来是为了什么,江别鹤巳微微一笑,道:''铁老前辈方才说的话,你听到了么?``那轿夫道:''小人听得清清楚楚。"

江别鹤道;``你说他是否有加害铁姑娘的道理?''那轿夫道:``没有。''

这时大厅里人人面面相觑,有的认为江别鹤这是故弄玄虚,有人认为江别鹤这是弄巧成拙。

江别鹤不动声色,反而笑道:``那么,这毒不是铁老英雄下的了?''那轿夫道:``是铁老英雄下的。''

江别鹤道:``你为何又说是铁老英雄下的毒呢?''那轿夫道;``只因他虽无相害铁姑娘之意,却有毒杀花公子之心他下毒本是要害花公子的,只不过铁姑娘首当其冲而已。''江别鹤故意皱起眉头,问道:``铁老英雄与花公子也素无冤仇,又为何要害花公子?''他话末说完,铁无双已怒喝道:``正是如此,老夫为何要害人?''那轿夫不慌不忙,缓缓道:``要杀人自然有这几个原因,一是嫉妒,二是仇恨,还有自己若是做了见不得人的事怕被人发觉铁无双怒喝道;''老夫一生顶天立地,你这奴才竟敢道老夫做了见不得人的事!``这一声大喝有如霹雳雷霆,''地灵庄``的家丁都被吓得面目变色,这轿夫居然还是不谎不忙反而笑道;''小人可不敢说这话,这话可是铁老英雄你自己说的。"这轿夫不但口齿伶俐,胆子极大,而且说话恭敬中带着刻薄,竟有与铁无双分庭抗礼之势。

别人都在奇怪,``江南大侠''属下,怎地连个轿夫都是如此厉害的角色,小鱼儿却已瞧出这``轿夫''绝不会是真的轿夫,必是别人打扮成轿夫的模样,他目不转睛地瞧着,越瞧越觉得这轿夫像是一个熟人。

只见铁无双怒极之下,反而狂笑起来。

他仰天狂笑道:``好,好,好,当着许多朋友,老夫倒要听听你这奴才说老夫究竟做了些什么见不得人的事。''那轿夫缓缓道:``见不得人的事也有许多种,譬如说偷鸡摸狗,这种算是小的,劫人镖银,杀人生命,这就算是大的了。''铁无双道:``你\ldots\ldots 你说老夫劫了谁的镖银?''那轿夫道:``譬如说是段合肥老爷的。''

铁无双嘶声道:``段合肥?你\ldots\ldots 你\ldots\ldots{}''

那轿夫道:``城里人人都知道,段老爷子和赵庄主是对头,段老爷子买货的银子若被劫,贷物进不来,这城里岂非就没有人和赵庄主抢生意了。''铁无双怒道:``纵然如此,这和老夫又有何关系?''那轿夫笑嘻嘻道:``铁老英雄若是在暗中动了段合肥的镖银,不但赵庄主要重重酬谢,而且那一笔镖银铁老英雄正也可消受了。''铁无双道:``,好,\ldots\ldots 你再说。''

那轿夫道;``铁老英雄本以为这件事做得神不知,鬼不觉,江湖中纵然有人调查此事,也算计不到铁老英雄。''他一笑接道:``谁知段老爷子竟请出了花公子来,铁老英雄自己也知道花公子不是等闲人物,生怕花公子查出此事,那么铁老英雄日后岂非没脸在江湖混了,所以就先下手为强,要将花公子置之于死地。''他话说得委实越来越露骨,本来还是``假若''、``譬如'',此刻却公然指明就是铁无双了!

铁无双大怒喝道:``好可恶的奴才,老夫先打烂你这张利嘴!''怒喝声中,这暴躁的老人身形已虎扑而起,铁掌扇风,左右齐出,直击这轿夫的左右双颊。

铁无双领袖三湘武林,武功可不等阑,此刻盛怒出手,掌风过处,一丈外衣袂惧已被震得飞起。

奇怪的是,江别鹤就站在那轿夫身旁,他眼看自己属下要挨打,居然像是若无其事,也不出手阻拦!

只听``噗、噗''两声,一声狂吼,一条人影飞出!

这轿夫竟接了铁无双一掌。

而四掌相击,被击出去的竟不是轿夫,而是素来以掌力见长武林的三湘名侠``爱才如命''铁无双!

众人都不禁失声惊呼出来小鱼儿本在苦苦思索这轿夫究竟是谁,此刻见他出手之掌势,掌力竟是极上乘的武林正宗功夫!

小鱼几心念一闪,失声道:``原来是他!''

只是铁无双被震得飞出文余,落下时竟是站立不稳,连退数步,若非赵香灵赶出扶住,他竟要跌倒。

饶是如此,他赤红的脸膛还是已变为惨白,脑膛也起伏不定,显然已受了伤,而且伤还不轻。

江别鹤微微笑道:``铁老前辈毕竟已老了。''

铁无双颤声道:``你\ldots\ldots 你\ldots\ldots{}''

江别鹤道:``前辈还有什么话说,在下等惧都洗耳恭听。''赵香灵大声道:``在下还有话说,试问那毒真是铁老英雄下的,他送礼时怎会将解药放在这里,难道等着阁下来抓人抓赃么?''那轿夫抢先道:``若是凡俗之辈,自然不会这样做的,但铁老英雄纵横江湖数十年,是何等见识,他这样做法,正是叫别人不信此事真是他做的,这岂非说比那种此地无银叁百两的做法高明十倍、百倍。''赵香灵道:``但\ldots 但\ldots\ldots{}''

他平日自命机智善辨,推知此刻竟被这轿夫驳得说不出话来,要知此事若真是铁无双做的,铁无双如此做法,倒的确真是最高明的手段。

江别鹤道:``事已至此,公子意下如何?花无缺缓缓道:''此事着被天下英雄知晓,天下英雄惧都难容。``江别鹤道;''正是如此。"

花无缺目光缓缓扫过众人,然后凝注在铁无双、赵香灵面上,道;``此刻方值正午,我再给两位半天时问,两位可自思该如何了断,今夜子时,我当再来。''微一抱拳竟转身走了出去江别鹤道:``在下素仰铁老前辈侠名,本待好生结纳,谁知唉,''长长叹息了一声,竟也随着走了出去.

众人见他们此刻竟然定了,也不勿是惊是喜,惧都怔在当地。

小鱼儿也不禁暗叹道:``无论如何,两人这一走,倒走得当真不愧大侠身份,只不过那花无缺乃是出自本意,江别鹤却是装出来的。''众人眼睁睁地瞧着花、江等人出了庄门,扬长而去。

铁无双突然狂吼一声,道:``气死老夫''话刚出口,张嘴喷出一口鲜血。

原来他方才对掌时受创极重,只是将一口气强行忍住,他方才一直不说话,正是怕在人前丢脸。

赵香灵见他诺大年纪,仍是如此强傲,心中不觉惨然,强笑道:前辈赶紧到后面歇歇,先将养伤势\ldots\ldots{}``铁无双惨笑道:''今夜子时便是你我大限,养好伤势又有何用?``赵香灵道:''那\ldots\ldots 那只怕也未必,他们人已走了``\ldots.''铁无双长笑道:``他们人虽走了,老夫难道还能逃走不成\ldots\ldots 咳咳,不想老夫一世英名,到老来竟要死于屈辱!''铁无双仰天道:``事到如今,老夫已无处可去,无路可走,与其等到子时,倒当真不如自己先作了了断也罢!''一言未了,竟已热泪盈眶,这老去的英雄又逢末路,怎不令人神伤。

赵香灵骇然道:前辈切切不可如此,事情只怕还有转机铁无双道:``事已至此,我等已是百口莫辩,除非能寻得出那真凶\ldots。但人海茫茫何处去寻那真凶?更何况只有半天的工夫\ldots\ldots{}''赵香灵黯然道:``半天\ldots\ldots 子时\ldots.''

抬眼望去,门外日影已偏西。

铁无双仰天笑道:``江别鹤呀江别鹤,花无缺呀花无缺!老夫并不怪你,事到如此\ldots\ldots 咳咳,你倒也只有如此做了,你们能多给老夫半天时间,已是大仁大义,老夫。\ldots 咳\ldots\ldots 老夫还该感谢于你咳咳。''他一面说话,一面咳嗽,鲜血已溅满衣襟。

赵香灵半推半劝,令人将他扶至后室,转首望向罗九、罗三,惨然道:``贤昆仲难道也无以救我?''罗九微微一笑,道:``铁老英雄忧郁太过,依在下看来,此事倒也简单。''罗九目光一转,附在赵香灵耳旁道:``事到如今,你我只有先下手为强,将段合肥与他女儿擒来,好教江别鹤投鼠忌器,不敢下手!''小鱼儿听了这话,真想过去给他几个耳括,这算是什么主意,这简直是在陷人于死。

赵香灵沉略半晌,道:``此事万万做不得,若是如此做了,天下武林中人,岂非真要以为劫镖、下毒之事惧是我等所为,我等岂非更是百口莫辩。''小鱼儿暗中附掌道:``不错,赵香灵果然不是笨人。''只见罗九却又附耳道:``庄主怎地如此执着,需知如此行事,只不过是暂时权宜之计,一面稳住江别鹤等人.一面去寻访真凶,等真凶寻到,真相大白后,再好生将段家父子送还,那时江湖中谁敢说庄主不是呢?''赵香灵不禁动容,喃喃道:``但在下还是觉得此事\ldots\ldots{}''罗九道:``庄主若不肯行此妙计,以那江别鹤与花无缺的武功,庄主要想逃过今夜子夜之限.只怕是难如登天的了,''赵香灵默然半晌,苦笑道:看来也只有如此了。语声方顿,又道;``只是,那段合肥仆役如云,要想自他庄院中将他父女劫来,也绝非易事,这得有千军万马中取上将首级的本事。''罗九微微一笑,道:``这个倒不用庄主担忧。?罗三道:''此刻花无缺与江别鹤必不会防备有此一着,更不会去防护段氏父女,除了这两人外,别的人都可不虑。``赵香灵喜道:''难道两位肯仗义援手?"

罗九微言道:``食君之禄,怎能不忠君之事。''赵香灵大喜拜道:``贤昆仲如此高义,在下真不知该如何报答才是。''罗九赶紧扶起他,道:``庄主切莫如此多礼。''小鱼儿在一旁瞧得清楚,暗道:"好个罗九,竟使出如此恶计,你这样做法岂非正是要搞得天下大乱,好教你从中取利么\ldots\ldots{}

只听罗九道:``事不宜迟,在下此刻就要去了。''赵香灵道:``资昆仲若有所需,但请吩咐。''

``别的不用,只请庄主派八位家丁,抬两顶小轿跟随着我兄弟。''赵香灵道:``这个容易\ldots\ldots{}''

他吩咐过了,立刻有人应声而出,小鱼儿眼珠子一转,也跟着走了出去,于是小鱼儿也权充了一次``轿夫''。

两顶轿子抬来,罗九却先坐了上去,笑道,``这两个轿子此刻先让我兄弟坐坐,等会儿就要轮到段合肥父女坐了,他父女只怕也不比我兄弟轻。''他坐上轿,放下轿窗,道,``段台肥的庄院,你们可认得么?''一人笑应道。``自然认得,咱们好几次想去放火烧他房子。''罗九道:``咱们这就走。''

七个家丁加上一个小鱼儿,果然抬起轿子就走,那七个家丁还不知此去要干什么,有些不禁在暗中嘀咕。

轿子走了顿饭工夫,远远己可望见段合肥的宅院,见那朱红的大门前也坐着七八个汉子,门里还有七八个。

那家丁道:``前面就是段合肥的猪窝了,罗爷瞧该怎么办?''罗九道:``笔直抬进去。''

这话说出,小鱼儿也不禁骇了一跳:``难道他们不怕江别鹤?''那些家丁们更是惊得呆了,强笑道:``段台肥的守门狗不少,若被他们咬一日,岂非冤狂。''罗九道:``你们只管往里面抬就是,那些守门狗决计咬不着你们。''家丁们互相瞧了一眼,鼓起勇气,忙喝着往前走。

刚走到门口,段宅的庄丁果然迎了过来,吆喝道:``喂,你们是干什么的?站住!''小鱼儿眼珠子一转,喝道:``咱们是来抬猪的,让开!''他这自然是存心捣蛋,好教江别鹤迎出来,罗九就成不了事,出手相救铁无双,他早有成竹在胸。

段宅庄丁果然大骂着冲过来,纷纷喝道:``狗养的,你们是来找死么''\ldots.``赵宅家丁手里抬着桥子,眼看他们冲过来,也不能还手,心里正在着急,突听''嗤、嗤"几响!前面七八个段宅庄丁竟应声倒下,别人什么都没瞧见,还以为是见了鬼了。

小鱼儿眼尖,却瞧见几点乌光自轿中飞出,七八个庄丁每人挨了一下,竟立时倒地,滚了两滚,就不动了!

这罗九当真是好毒的手段!小鱼儿却不免瞧得心惊,赵宅家丁更是目瞪口呆。

罗九笑道:``守门狗不叫了,你们还不走。''家丁诺诺连声,抬起轿子再往前走。

这时门里又有七八人惊呼着奔出,刚奔出大门,又是``嗤、嗤、嗤''几响,又有七八人倒地。

还没出门的一个,转身就跑,大呼道:``来人呀,来人呀,门外有恶鬼闯来了。''小鱼儿暗道:``他如此呼喊,想必可以将江别鹤引出来,这罗氏兄弟难道就毫无顾忌?''罗九、罗三竟真的毫无顾忌,大笑道:伙计们,往前走呀!"这时赵宅家丁一个个惧已勇气大振,放足飞奔。

走进前面一重院子,里面已有二十多人手拿刀斧棒迎出,但暗器飞声响过,前面又倒一片。

一条紫衣大汉变色呼道:``轿子里暗青子扎手,伙计们先退。''这人身手最矫健,武功看来竟不弱。

呼声中,已有五个人箭步窜出,手里竟各各拿着面盾牌,抛了一面给那紫衣大汉,紫衣大汉挥手呼道:``射人先射马,先将抬轿子的做了再说。''刀光闪动间,六个人已飞步而来。

赵宅家丁虽然大声呐喊,但心里已有些发毛,只见武师们各各以盾牌护住前胸,挥刀直劈而下。

突听一声长笑,一人大声道:``且慢!''

一条人影,自轿子里飘了出来,一把抓住那赵宅家丁的后背,将他往后直抛了出去。

那武师一刀砍空,只见一个脸圆圆的胖子笑眯眯的站在面前,一只手指着自己的鼻子,笑道:``各位难道不认得区区在下么?''武师们俱都呆了呆.各各对望了一眼,只道这胖子或许是自己人的朋友,但一眼尚未瞧过,罗九已笑道:``各位既不认得在下,在下也只有不认得各位了!''语声中手掌已毒蛇般伸出,抓住当先那持刀武师的下腕.只听``喀嚓''一声,接着─声惨呼。

那武师的手腕竟被生生拧断!钢刀落地,他人也疼得晕了过去,另五人又惊又怒,─根枪、两把刀交击而下!

罗九目光一扫,笑道:``不想这里竟还有杨家枪的门人,这一招风点头看来至少也有十五年的火候,算得上是好枪法!''那持枪的武师正是北派杨家枪的嫡传弟子,如今一招使出,就被瞧出了来历,不由得暗中─惊,掌中枪也慢了慢。

就在这一惊一慢间,枪尖竟已落入对方掌中。

罗九右手握着枪尖,身形半转以枪杆挡开了右面攻来的一柄剑,却向左面攻来的紫衣大汉笑道:``彭念祖彭老师可好么?''这彭念祖乃是南派``五虎断门刀''的掌门人,而这紫衣大汉却正是他门下弟子,如今听得对方提起自己的师傅,也不由得一怔,道:``你认得他老人家?''罗九笑道:``不认得!''

``不认得''三个字说出,左掌已击上这紫衣大汉的胸膛,将他魁伟的身子打得直飞出去。

也就在这时,那持枪的武师但觉一股大力自枪杆上涌了过来,他想撤手丢枪,却已不及!

只听``噗''的一声,这杆枪的枪柄,竟直插了他的胸膛!他自己掌中的枪竟成了对方的武器!

罗九拍了拍手,笑道:``三位如今可认得区区在下了么?''剩下的三人已吓得面如土色,手里拿着刀枪,却再也不敢动手,这罗九竟在谈笑间便了结了三个身手不弱的武师,出手之阴毒,竟是小鱼儿出道以来所仅见!此刻的罗九,哪里还是昨夜施展大洪拳时的罗九!

小鱼儿昨夜虽已知道此人必定深藏不露,但却也未必想到他的狡诈与毒辣,竟似不在他所认识的``十大恶人''之下!

他心念一转之间,那边站着的三个武师又已躺下了一个,剩下的两人,四条腿已开始发抖。

罗九笑嘻嘻道:``如今二位总该认得在下了吧。''那两人不约而同,颤声道:``认得\ldots\ldots{}''认得``\ldots.''罗九笑道:``两位认得我是谁?''

那两人面面相觑,道:``你\ldots\ldots 你老人家是\ldots。是\ldots。.''罗九道:``我姓罗,叫罗九。''

那两人道:``不错,不错,你老人家是罗九爷。''罗九道:``两位既然认得在下,那真是再好也没有了,就烦两位带我去拜见段合肥段老爷子如何?''那两人你望着我,我望着你,呐呐道:``这\ldots\ldots 这\ldots\ldots{}''罗九面色一沉,道:``这区区小事,两位都不肯答应么?''那两人想了想,终于叹道:``好,就请''\ldots\ldots"

一句话还未说完,只听``嗤、嗤''两响,两道乌光自后面飞来,击中了他们的背脊,两人惨叫倒地。

一人大笑道:``段老爷子已被我请了出来,已用不着你两人带路了!''笑声中罗三大步行走,左手拉着段合肥,右手拉着的正是段三姑娘。

原来罗九在这里动手时,罗三已悄悄溜进了后院,段三姑娘虽也有些武功,但又怎会是这罗三的对手!

四面还剩下三四十个段府的壮丁,此刻眼睁睁瞧着罗三将他们的主人拉出来,竟无人敢出手的!

这神秘的罗氏兄弟两人,果然不费吹灰之力就将段台肥父女绑架了,小鱼儿心里又掠又奇。

``江别鹤呢?江别鹤难道死了?''

只见段合肥已吓得面无人色,罗叁叫他走,他就走,罗叁叫他上轿子,他就乖乖的上了轿子。

那三姑娘眼睛虽然瞪得比铜铃还大,但也毫无抵抗之力,罗叁笑嘻嘻地将她推上轿子,道:``兄弟们,台起轿子走吧。''罗九笑道:``这轿子不小,坐两个人也不嫌挤,各位就辛苦些吧!''这兄弟两人居然也挤进了轿子,直压得轿板咬吱的响。

赵庆的家丁们早巳将这两人视若神明,轿子再重,他们也是心甘情愿的抢着,非但毫无怨言,而且还欢喜得很。

小鱼儿心眼儿又开始打转了!江别鹤始终不露面,莫非是还没有回来?

他们早就该回来的,此刻偏偏还未回来,莫非是早知道罗三罗九有此一着,是以避开了。

他故意要罗三罗九将段合肥父女架走,正是要教这件事闹得更不可收拾,要教铁无双更无法办好!

但罗三罗九又怎知江别鹤不在呢?

``莫非这兄弟两人也早与江别鹤在暗中勾结?''小鱼儿不禁暗叹道:``好一个江别鹤,毒计之中,居然还另有毒计,普天之下,除了我江小鱼外,还有谁能识破他的毒计?''心念转动间,轿子已转过一条街。

突见前面也有一顶轿子走过来,抬轿的正是那能言善辩的``轿夫''后面跟着两匹马,马上人却正是江别鹤与花无缺。

小鱼儿又是一惊,眼珠子转了转,突然大喝道:``前面的轿子快闪开,你可知这轿子里坐的是什么人吗?''赵庄的家丁,瞧见江别鹤与花无缺已是胆战心惊,听见他这一吼,更是吓坏了。

哪知江别鹤居然真的给轿子让开了一条路。

小鱼儿抬着轿子走过去,故意撞了那``轿夫''一下,低声道:``我认得你,你认得我么?''那``轿夫''居然好像没有听见,垂着头走了过去,只有江别鹤策马而过时,狠狠盯了小鱼儿一眼。

轿子交错而过,赵庆的家丁都不禁在暗中松了口气。

小鱼儿冷笑着,暗道:``我猜的果然不错,江别鹤与这两个姓罗的果然早有勾结,所以他就算明知这轿子里的是什么人,也装做不知道。''这一着可当真将铁无双陷入了危境,他若再说自己与劫镖下毒之事无关,天下再也不会有人相信了。

\hypertarget{ux7b2cux56dbux5341ux516bux7ae0-ux63edux53d1ux5978ux8c0b}{%
\chapter{第四十八章
揭发奸谋}\label{ux7b2cux56dbux5341ux516bux7ae0-ux63edux53d1ux5978ux8c0b}}

段合肥父女入了地灵庄,地灵庄上上下下精神俱都一震,一个个喜笑颜开,几年来的闷气这下才算出了。赵香灵虽然也觉得这个事做得有些不妥,但瞧见多年的大对头已成了自己的阶下囚,也不由得心怀大畅。

小鱼儿瞧得不禁暗中摇头;四道:``你们现在尽管笑吧,哭的时候可就快到了\ldots\ldots{}''"只见段合肥父女被几个人拖拖拉拉,拉入了后院,这父女两人落入地灵庄,自然是有罪受的。

赵香灵已摆起慰劳酒,再三举杯道:``贤昆仲如此大义相助,在下实在没齿难忘。''罗三笑道:``区区小事,何足挂齿,只是\ldots\ldots 庄主心中此刻不知是何打算?''赵香灵叹道;``事已至此,在下,只望能将大事化小,小事化无,等到江别鹤来了,将此事好生解释,只要他不再追究,在下便将段合肥放回去也罢了。''罗九忽然冷笑道:``事已至此,庄主还想将大事化小事么?''赵香灵微微变色道;``难道\ldots\ldots 难道不\ldots\ldots{}''

罗九冷冷道:``事已至此.双方已成僵局,庄主再说与此事无关,无论如何解释,江别鹤是再也不会相信的了!''赵香灵失色道:``如此\ldots\ldots 如此贤昆仲岂非害煞在下了。''罗叁冷笑道:``我兄弟出生入死,换来的只是庄主这句话么?''赵香灵赶紧陪笑道;``在下一时失言,贤昆仲千万恕罪,只是\ldots\ldots 在下此刻方寸已乱,委实没了主意,一切还望贤昆仲多多指教才是。''罗九展颜一笑,缓缓道:``不能和,唯有战!''赵香灵失声道;``战!''

罗九道:``正是''

赵香灵道:``但\ldots\ldots 但那江别鹤与花无缺的武功,在下''\ldots 在下\ldots\ldots{}``罗九微笑道:''花无缺与江别鹤纵然武功惊人,但庄主也不必怕他。``罗三道;''庄主岂不闻,不能力敌,便可智取。``赵香灵呐呐道:''却不知该如何智取?"

罗九道:``段合肥父女已在庄主之手,江别鹤投鼠忌器,纵然来了,也必定不敢出手的,庄主你可先将他们稳住。,赵香灵道:然后呢?''罗九目光一扫,悄声道:``地灵庄兄弟,个个身手惧都不凡,庄主不妨令人在这大厅四面埋伏,准备好强弓硬弩\ldots\ldots{}''罗叁微笑接道:``那江别鹤与花无缺只要进了此厅,纵有叁头六臂,只怕也难以活着出去了。''他似乎并无顾忌,说话的声音并不小。

小鱼儿远远听得,不禁暗骂道:``这算什么狗屁主意,那江别鹤怎会中计,赵香灵若是听从了这主意,无异将自己的罪又加深一层,这样江别鹤就算立刻杀了你,江湖中也不会有半个人出来为你说话的了。''赵香灵听了这主意,却不禁动容,道:``贤昆仲以为此计真的行得通么?''罗九道:``自然是行得通的。''

罗叁接着笑道:``此计成功之后,天香塘、地灵庄势必将名震天下,那时只望庄主莫要将我兄弟赶出去就是了!''赵香灵忍不住笑道:``在下怎敢忘记两位\ldots。.''笑声顿住,呐呐道:``只是''\ldots 这样做法,万一不成\ldots\ldots 岂非罗九正色道:``事已至此,庄主难道还有什么别的主意不成?''赵香灵沉吟半晌,苦笑道:``事已至此,看来我已别无选择了,常言道:量小非君子,无毒不丈夫,赵香灵也只好和他们拼到底了!''罗九附掌笑道:``正是正是,庄主这句话说出来,才真是个英雄本色!''罗叁道:``那江别鹤发现段合肥父女被劫后,势必要立刻起来,我等行事得从速才是。''赵香灵霍然长身而起,厉声道:``兄弟们,准备弓箭埋伏,听我掷杯为号,立刻出手!''罗九道:``埋伏好了,你可请铁老英雄出来。''罗叁笑道:``少了铁老英雄,便成不得事了。''江别鹤的计谋,显然进行得十分顺利,赵香灵不但自己一步步走人了陷阱,而且将铁无双也拖了下来。

这样,江别鹤很轻易地就可将铁无双的势力消灭,眼看江湖中反对江别鹤的势力已越来越少了。

这样,铁无双不明不白地就做了那真正劫镖人的替死鬼,江湖中甚至不会有─个人对此事发生怀疑的。

网巳在渐渐收紧了──小鱼儿闭起眼睛,喃喃自语道:``江别鹤的恶计,难道真的无懈可击么?''黄昏。

铁无双已坐上了大厅,他身子虽然仍坐得笔直,但神情看来却很憔悴,目中失去了原有的光彩!

罗九、罗三却是神采奕奕,赵香灵也显得兴奋得很,这地灵庄外表看来似乎很平静,其实却四伏着杀机!

大厅四侧,已埋伏好三十张强弓,二十匣硬弩,院子里却仍有叁五成群的家丁,小鱼儿也混在里面。

突听庄外马蹄声响,众人俱都耸然动容。

蹄声骤住,进来的却是七个劲装佩剑的少年,七人一起抢步直入了大厅,拜倒在铁无双的面前。

这七人正是铁无双的``十八弟子''中的高手,他们闻讯赶来,铁无双固是大感欣慰,赵香灵也不觉喜上眉梢。

小鱼儿瞧见这七人,眼睛也一亮,这七人中为首的一个,正是与那江玉郎暗中勾结的、面色惨白的绿衫少年。

只听他恭声道:``弟子来迟,盼师父恕罪\ldots{}''.``小鱼儿暗喜道:''你来得并不迟,你来得正好,我正在等着你来!``铁无双喜色初露,愁容又起,长叹道:''你等虽来了,却也无济于事\ldots\ldots 此事已非武力可以解决,少时你等切切不可胡乱出手,免得\ldots─"语声未了,突听一声惊呼!

一条人影自大厅后的窗户外飞了进来,``砰''地跃在地上,四肢僵硬,再也动弹不得,只见此人黑衣劲装,手提着一张金背铁胎弓,背后斜插着一壶乌翎箭,却正是赵香灵埋伏在大厅四侧的家丁壮汉。

赵香灵面色惨变,铁无双也惶然失声。

只听又是一声惊呼,又是一个跌入\ldots\ldots 刹时之间,只听惊呼之声不绝于耳,大厅中已有数十人叠了起来,一个个惧是四肢僵硬,动弹不得。

铁无双失声道:``这\ldots\ldots 这是怎么回事''

赵香灵惶然四顾,道:``这\ldots\ldots 这\ldots\ldots{}''

一人冷冷接口道:``这是你弄巧成拙!自作自受!''两条人影飘飘掠了进来,却不是江别鹤与花无缺是谁!

赵香灵``噗''的坐倒椅上,再也站不起来。

江别鹤负手而立,冷笑道:``铁老英雄认为这区区埋伏能害得了江某,也未免将江某瞧得忒低了。''铁无双厉声道:``这究竟是怎么回事?老夫根本全不知情!''江别鹤冷冷道:``若未经铁老英雄同意,赵庄主只怕也不敢如此吧。''铁无双怒喝道:``赵香灵,你说,是谁教你用这卑鄙的手段的?''赵香灵头也不敢抬起,喃喃道:``这\ldots。这\ldots。.''罗九突然长身而起,厉声道:``我兄弟知道铁老前辈与赵庄主乃是英雄,是以不远千里而来,谁知两位竟使出如此卑鄙的手段。''罗三大声接口道:``我兄弟虽然不才,却也不屑与此辈人物为伍,从此以后,地灵庄无论有什么,都与我兄弟毫无关系!''赵香灵大声道:``两位怎可说出这样的话来,达一切岂非都是两位的主意?''罗九冷笑道:``好个赵香灵,你竟敢将此事赖在我兄弟头上么!''罗三冷笑道:``你纵然百般狡赖,只怕也是无人相信的!''赵香灵狂吼一声,道:``你\ldots\ldots 你好,好\ldots。.''花无缺缓缓道:``我虽不为己甚,但事到如今,你两人还有何话说?''铁无双咬牙道:``老夫\ldots。老夫\ldots 气煞老夫也!''吼声中他又自喷出了口鲜血,这老人气极之下,竟晕了过去!

他门下子弟又惊又怒,有的赶过去扶起了他,有的已待拔剑出手,那面色惨白的缘衫少年大声道:``事情未分皂白之前,大家且莫出手''江别鹤正色道:``不错,师父若不义,弟子便不该相随,各位若能分清大义所在,天下武林中人对各位都必将另眼相看。''那绿衫少年道,``但此事究竟如何,还。\ldots.''江别鹤厉声道:``此事实俱在,你们还有什么不信的?''绿衫少年故意惨然长叹一声,道:``师父你休怨弟子无情,只怨你老人家自己做出了此等天理不容之事,弟子为了顾全大义,也只有\ldots\ldots{}''咬牙难受,顿了顿脚,解下腰畔佩剑,掷在地上!

他这一手做得更是厉害已极.江湖人中若知道连铁无双自己的弟子都已认罪,别的人还有何话可说.其余六人一向唯他马首是瞻,见他已如此,有三个人跟着解下佩剑,其余三人虽未解剑,但握剑的手也已垂了下来!

江别鹤郎声道:``除了铁无双与赵香灵外,此事与各位俱都无关,只要各位不助纣为虐,江某也必定不会牵连无辜!''赵香灵牙齿已吓得``喀喀''打战,嘶声道:``我与你究竟有什么冤仇,你要如此害我?''江别鹤缓缓道:``在下与你虽无怨仇,但为了江湖道义,今日却容你不得!''赵香灵突然咬了咬牙,狞笑道:``好,我知道你为了段合肥,要将赵某除去,但你也莫忘了段合肥此刻也在赵某手里,赵某若死,他也是活不成的。''江别鹤冷笑道:``真的么?''

他招了招手,厅后竟也有两顶轿子抬了出来,前面抬轿的,正是那能言善辩的神秘``轿夫''。

江别鹤道:``轿子里坐的是什么人,你可想瞧瞧么?''赵香灵踉跄倒退两步,只见那``轿夫''掀起子,笑嘻嘻地坐在轿子里的,却是那段合肥。

到了这地步,赵香灵已一败涂地,他惨然四顾,突然狂吼一声,疯狂般向厅外奔了出去。

江别鹤也不阻拦,瞧着他冷笑道:``你难道还想逃得了么''赵香灵奔出大厅,黑暗中突然伸出一只手来,将他拉了过去,在他耳边低低说了几句话。

这几句话竟像是仙丹妙药,竟使赵香灵精神一震.这时铁无双悠悠醒来。

花无缺缓缓道:``念在他成名也算不易,就让他自己动手了断吧。''他说话居然还是从从容容,神情也仍旧是那么飘逸而潇洒,他长衫如雪,根本瞧不出丝毫曾经与人动手的痕迹。

他虽可主宰这里所有的事,但一切又仿佛都与他无关似的,他竟连话都没有多说一句。

纵然在乱军之中,他也可保持他那翩翩的风度。

只见江别鹤俯身拾起那绿衫少年的佩剑,缓缓送到铁无双面前,冷冷地瞧着铁无双却没有说话。

他已用不着说话。

铁无双仰天长叹,嘶声道:``苍天呀苍天,我铁无双今日一死,怎能瞑目!''他凄厉的目光,扫过他门下弟子,就连那绿衫少年也不禁垂下了头,铁无双突然奋起,大喝道:``铁某就站在这里,你们谁若认为铁某真的有罪,要取铁某的性命,只管来吧!只怕苍天也不能容你''烛火飘摇中,只见他目光尽赤,须发皆张,一种悲愤之气,不禁令人胆寒,江别鹤竟不觉后退了半步。

那``轿夫''却一步窜了出来,大喝道;``多行不义,人人得而诛之,普天之下,谁都可以取你性命,别人若不忍动手,就由我来动手吧''突听一人道:``江玉郎,你真的敢动手么!''

那``轿夫''身子一震,霍然旋身,只见那赵香灵竟又大步走了回来,他面上虽仍苍白得不见血色,但胸膛却已挺起!说话的声音也响亮了。

他走入大厅中央,众人才瞧见竟还有一人跟在他身后,这人青袍白袜,头上戴着个竹篓,遮住了面目,走起路来,飘飘荡荡,就像是贴在赵香灵身上的幽灵,令人瞧得背脊上不觉直冒寒气。

但那``轿夫''一惊之下,神情瞬即镇定,大笑道:``堂堂的江少侠,怎会来做轿夫,你莫非瞎了眼了!''赵香灵大声道:``江玉郎,你瞒得过别人,却瞒不过我,你劫了段家的镖银后,赶回这里假充轿夫,为的是要取铁老英雄的性命,这样江湖中人都只道铁老英雄是死在个轿夫身上,日后纵有要寻仇之人,也寻不着假仁假义的江南大侠父子了\ldots\ldots 江玉郎呀江玉郎,你父子两人行事当真是千思万虑,滴水不漏!''那``轿夫''纵声狂笑道:``各位听见了么,这竟敢说劫镖的乃是江少侠''\ldots 段老爷子你说这是不是胡说八道的疯子!``段合肥眯着的眼睛里似乎闪过一丝狡黠的光芒,他笑眯眯地瞧着赵香灵,一字字缓缓道:''你这话是从何说起,我镖银第一次被劫,就是江少侠夺回来的,他若是劫镖的人,为何又将镖银夺回?``赵香灵道:''镖银第一次被劫,本是双狮镖局与江玉郎窜通好的,江玉郎若不将镖银送回,他们还是要赔出来。``段合肥道:''他们为何要如此做?"

赵香灵道:``如此做法,不但提高了江玉郎在江湖中声望,而且''``他语声故意顿了顿,段合肥果然忍不住追问道:''而且怎样?``赵香灵缓缓道:''而且第二次镖银被劫时,别人就再也不会怀疑到江玉郎头上。``段合肥道:''如此说来,那双狮镖局中的人,又怎会\ldots。.``赵香灵接口道:''在这恶计之中,双狮镖局里的人,自然不免要做冤死鬼,江玉郎自然要将他们杀死灭口,而且\ldots\ldots{}``段合肥竟又忍不住问道:''而且怎样?"

赵香灵道:``双狮镖局上上下下既然死净死绝,那镖银自然就没有人赔了,于是那诺大一批镖银,就太太平平落入了江南大侠的手中''江别鹤眉心微微一皱,向那``轿夫''瞟了一眼。

那``轿夫''怒喝道:``贼咬一口,入骨三分,你临死居然还要反噬,我却容不得你!''喝声中,已向赵香灵怒扑过去!

他身形之快,当真有如急箭离弦!

赵香灵大惊之下,竟来不及闪避,就在这时,突见人影一花,花无缺竟飘飘挡住了那``轿夫''的去路。

那``轿夫''掌已击出,不及收势,眼见竟要打在花无缺身上,但见他身子突然一扭,左掌向右掌一拍,身子已的溜溜打了个转,指势倒翻而出。

这一手``壮士断腕'',正是内家正宗最上乘的功夫,实比昆仑大九式中的``悬崖勒马''还要高出一筹。

这一手功夫使出,就连铁无双都不禁耸然动容,江别鹤双眉却皱得更紧,只听花无缺微笑道:``好武功!好身手''\ldots.``那''轿夫``吃惊地望着他,呐呐道:''花公子为何要\ldots\ldots。

花无缺悠悠笑道:``无论是谁有话要说,咱们都该听他说完了才是,咱们纵然不信他的话,却也得让他有说话的自由,是么?''那``轿夫''垂下了头,道:``是!''

花无缺转向赵香灵,道,``你无端说出这话,可有什么根据?赵香灵呆了半晌,却又立刻大声道:''双狮镖局中的人,俱是仓猝而死,连一招都不及还手,而这江南双狮武功,要想将这些人全都杀死,也不能令他们全都还不了手的,是么?``他呆了一呆之后,像是突然有人指点了他,口若悬河,侃侃面言,江别鹤两道锐利的目光,已闪电般扫向他背后那''幽灵"的身上。

花无缺缓缓道:``不错,就算武功比我更强的人,纵然能致他们于死,只怕却也不能令他们全都还不了手的。''赵香灵道:``但普天之下,武功更强于公子之上,只怕已没有了,是么?''花无缺微微一笑,道:``纵有也不会多。''

赵香灵道:``是以此事只有一个解释。''

花无缺道:``什么解释?''

赵香灵道:``这必定是一个与李氏双狮极熟的人下的手,他们万万想不到这人会向自己人下毒手,足以猝不及防,连还手俱都不及\ldots.''他咯咯一笑,接着道:``这不问便可知,自然除了江玉郎外再无别个!''花无缺道:``但据那仅存的活口马夫所见,下手的乃是个威猛老人。''赵香灵道:``易容之术,在江湖中,虽仍是奥秘,但会的人却也有不少,他既能假充轿夫,为何就不能改扮成威猛老人\ldots\ldots{}''他语声频了顿又接道;``他故意留下那马夫,正是要借那马夫之口\ldots\ldots 否则他杀人之后,又怎会狂笑而出,否则以他的武功,那马夫就算躲藏,又怎能逃得过他的耳目?''他语声又顿了顿,又接着道:``还有那马夫逃生之后,立刻就将此事绘形绘影地说了出来,而且说的有声有色,巨细不漏,试问一个真的受了如此惊骇的人,说话又怎会如此明白清楚,所以\ldots。那马夫想必也是他的同谋,早已经他指点\ldots\ldots{}''他语声每次顿住时,似乎都在留意倾听着他身后那幽灵``说话,江别鹤目光如炬,冷笑道:''你说的话又是谁指点你的?``赵香灵道:''这\ldots\ldots 这全是我自己想出来的,我\ldots\ldots{}``说到这里,他突然又顿住了声,接着又大声道:''对了,我方才说错了,那马夫说不定就是现在这轿夫,就是江玉朗,而动手的却是江别鹤!``.江别鹤突然仰首大笑起来,道:''我本不愿与你一般见识,但你既如此胡言乱语,我却也容不得你了。``他这话竟不是向赵香灵说的,眼睛也未瞧着赵香灵,他那锐利如刃的目光,正盯在那''幽灵"身上!

突听一声轻叱,,那``轿夫''不知何时已到了那``幽灵''身后,身形凌空,``飞鹰搏兔'',铁掌已闪电般击下!

大厅中人目光俱被江别鹤吸引,谁都没有留意到这``轿夫'',此刻他骤然出手,眼见已是万万不会落空。

谁知他双掌自击下,那``幽灵''竟似早已算定他出掌的方法与部位,头也不回,反手一掌挥出。

这轻描淡写的一掌,竟正是击向那``轿夫''招式中的破绽,也正是他必救之处,他不求伤人,但求自保,双腿一缩一挺,身子凌空倒圈而出,远远落在地上,眼睁睁地瞧着这``幽灵'',竟像是真的见了鬼一般。

众人方才见过他的武功,如今又见他既被人轻轻一掌击退,惧不觉为之大惊,他自己更做梦也想不到自己势在必得的一掌,在别人面前,竟变做儿戏,只见这``幽灵''缓缓转过身子,咯咯笑道:``你认得我么?''那``轿夫''嘶声道:``你\ldots。你是谁?''

那``幽灵''道;``你不认得我,我却认得你\ldots\ldots 我死也不会忘记你!''他语声尖细飘荡,听来当真有几分鬼气。

那``轿夫''竟不觉机伶伶打了个寒战,道:``你''\ldots 你究竟是什么人?``那''幽灵``道:''我早己告诉过你,我不是人,是鬼!``他一步步走过去,那''轿夫"竟不觉一步步往后退。

灯火通明的大厅中,也不知怎地竟像是突然充满了森森鬼气。

那``轿夫''面上肌肉虽动也未动,但一双眼睛却已惊恐欲绝,这样的面容配上这样的眼神,看来更是令人毛骨悚然。

突听那绿衫少年失声道:``呀,不好!我师父''\ldots 我师父\ldots。.他老人家竟自杀了!``这一声惨呼,立刻使众人目光惧都自那''幽灵"身上转了回来──目光转处,人人俱都不禁惊呼失声。

只见铁无双虽仍端坐在椅上,但方才那柄长剑,此刻竟已赫然插入了他咽喉,鲜血已染红了他衣服!

利剑穿喉,他连呼声都不能发出,他双手剑柄,似欲刺人,又似要将长刨拔出,却已无力!

他双服怒凸,目中犹经聚着临死的惊骇与怨毒,他人死去,这一双充满怨毒的眼睛,却似乎是在瞪着那绿衫少年!

众人耸然失色,竟都被惊得呆住了。

江别鹤长长叹息了一声,道:``铁无双不愧是英雄,勇于认错,他这样一死,生前的罪孽与污名总算己可洗清了!''那``幽灵''突然大声道,``放屁!铁无双绝不是自杀的!''

\hypertarget{ux7b2cux56dbux5341ux4e5dux7ae0-ux5e7dux7075ux4e4bux8c1c}{%
\chapter{第四十九章
幽灵之谜}\label{ux7b2cux56dbux5341ux4e5dux7ae0-ux5e7dux7075ux4e4bux8c1c}}

江别鹤怒道:``铁英雄若非自刎,难道还是江某下的手不成?''他顿了一顿,冷笑道:``江某若是下手,早巳下手,又何必等到此刻?''那``幽灵''也冷笑道:``铁无双若是自刎,也早巳自刎了,更不会等到此刻\ldots\ldots 他方才既不肯含冤而死,此刻真相眼见已将大白,他更不会死了''江别鹤厉声道:``铁老英雄若非自刎,还有谁能令他不及还手而死!铁老英雄这样死正是死得清清白白,你难道还要他死后受污名?''那``幽灵''也厉声道:``这里也正和方才赵庄主所说的一样,若是正面动手,自然谁也不能令铁无双不及还手而死,但若下手暗算\ldots。''江别鹤大喝道:``我江别鹤难道还会出手暗算他不成?''那``幽灵''冷笑道:``这次自然不是你,你自己也知道铁无双已在提防着你,纵然出手暗算,也决计无法得手的!''江别鹤道:``若非江某,难道还会是花公子不成?''那``幽灵''道:``我早巳说过,下手的必定是铁无双一个极为亲近的人,铁无双再也想不到他会出手暗算,是以才会遭他的毒手!''那绿衫少年突然大呼道:``是谁害死了我师父,我和他拚了!''那``幽灵''冷冷道:``下手害死你师父的,就是你!''绿衫少年身子一震,大怒道:``放屁,我身负师门至恩,怎会弑师,你\ldots\ldots 你莫非疯了?''那``幽灵''冷笑通:``你既知身受师门重恩,便该好生报答才是,但你却丧尽天良,暗中与江某人勾结!你眼见真相已将大白,便乘着大家全都不会留意你时,一剑刺入了你师父的咽喉,你以为铁无双一死,此事就死无对证,但你却忘了,还有我在这里!''绿衫少年道:``你拿得了证据么?''

那``幽灵''道:``别人拿不出证据,我却拿得出证据,我亲眼瞧见那日在酒中下毒要害赵全海赵总镖头的就是你!''绿衫少年身子已颤抖起来,却更大声喝道:``放屁!那日我师父相请赵总镖头前来与三湘联镖和解,我为何在酒中下毒加害赵总镖头。''那``幽灵''道:``只因你受江玉郎所命,此举不但要使和解不成,还要使你师父遭受污名,这正是个一计害三贤的毒计!''绿衫少年怒喝道:``放屁!你\ldots\ldots 你说的话,谁也不会相信!''那``幽灵''冷笑道:``你还想赖?我亲眼瞧见,亲耳听见你在那厨房与江玉郎商量恶计!''绿衫少年喝道;``你怎会亲眼瞧见\ldots\ldots 你血口喷人,我和你拚了!''他狂吼着扑了上去,便身形方展,``幽灵''突然揭下了头上的竹篓,咯咯怪笑道,``你再瞧瞧我是谁?''灯光下只见他满面泥污,披着散发,望之当真如活鬼。

绿衫少年立顿。后退叁三步,颤声道:``你。\ldots 你\ldots\ldots{}''那``幽灵''一字字:``告诉你,我就是那日被你和江玉郎害死的鬼魂,做鬼也要你的命!''他话末说完,那绿衫少年已发狂般的放声惊呼起来,狂呼道:``鬼\ldots\ldots 鬼\ldots\ldots 真的有鬼!''一面狂呼,一面后退,终于疯狂般奔了出去!

突然间,剑光─闪!

那绿衫少年还末奔到门口,已噗地倒了下去一柄长刨,自他后颈穿入,喉头穿出,竟生生将他钉在地上!

这绿衫少年也是连一声惨呼都末发出,便尸横当地!但这次众人却都瞧见,长剑是江别鹤脱手掷出的!

江别鹤神情不变,缓缓道:``此人神智己丧,若任他冲出去,只怕为害世人,在下只有将他除去了。''那``幽灵''大喝道:``江别鹤,你杀人灭口,还要说好听的话,当真是天理难容!''江别鹤微微一笑,道:``你连真面目都不敢示人,有谁能听信你的话!''这句话正是击中了这``幽灵''的要害──小鱼儿呆了半晌,大声道:``只要我说的话是真的,现不现出面目又有何妨?''江别鹤道:``各位请想,这所说若是真的,为何不敢以真面目见人?''小鱼儿目光四转,只见众人的眼睛,果然都已盯在他脸上,每一双眼睛里,果然都已露出怀疑之色。

江别鹤悠悠接道:``这藏头露尾,危言耸听,居心实不可测他一面说话,一面留意着众人面上的表情,说到这里,突然面对着花无缺,一字字沉声道:''花公子以天下为己任,难道不想知道他们的来历?``花无缺道:''他们?"

江别鹤道:``除了这之外,当然还有那轿夫,在下也正想瞧瞧,他是否真的如这所说乃是犬子玉郎。''众人在混乱之中,多已忘却了那``轿夫''的事,此刻被他一提,方自想起,但放眼四望,不但那``轿夫''踪影不见,就连别的轿夫和段家父子所坐的那两顶轿子,都已不知在何时走了。

小鱼儿不禁暗暗跺足,他虽然聪明绝顶,但经验终还太少,照顾还是不周,竟造成了这致命的疏忽。

江别鹤也似勃然大怒喝道:``那轿夫怎地走了?他什么时候走的?''一直在作壁上观的罗九,此刻突然道:``段老爷子身体不好,紧张过度,委实再也受不了这刺激,是以方才就要他们将轿子抬回去了。''罗三接着笑道:``人太胖了,的确不能紧张,否则难免中风,我兄弟也有这毛病。''江别鹤顿足道:``贤昆仲既然瞧见,就该将那轿夫留下才是,此事若不弄个清楚,在下也难免要担嫌疑!''小鱼儿忍不佼大骂道:``你这老狐狸,若论装模作样的功夫,你当真可算天下第一。''江别鹤冷笑道``有谁知道那轿夫不是和你一路,故意窜通来陷害江某的,否则你又怎会如此轻易地放他一走了之''他居然倒打一耙,居然说的合情合理,众人虽不见得就多信他的,至少已对小鱼儿说的话不再相信。

小鱼儿又气又急,他如今知道这江别鹤果然不是可以轻易对付的人物,轻描淡写几句话,就扭转了逆势。江别鹤还连一根手指都没有动,便已将小鱼儿逼入了死地!

这大厅前后共有十四扇窗户,三道门,每扇窗户高七尺余,宽三尺开外,无论多么魁伟的人都可轻轻易易地钻出去,出路可谓四通八达这大厅虽然宽阔,但每扇窗子距离小鱼站着的地方,最远也不过两三丈,以小鱼儿此刻的武功,轻轻纵身使可掠出。

但小鱼儿却不能走,只因花无缺的眼睛,此刻正盯在他身上。

江别鹤悠悠道;``那轿夫虽已溜走,但阁下却只怕已是溜不走的了,阁下定然不肯以真面目示人,莫非是做了什么见不得人的事。''小鱼儿眼珠直转,却想不出个主意。

花无缺突然道:``朋友若不愿自己动手,在下说不得只好代劳了。''小鱼儿大骂道:``花无缺,我本以为你是个聪明人,谁知你竟然像活土狗似的被人利用,连我都替你觉得丢人。''花无缺也不动怒,只是微笑道:``你若想激怒于我,这心机只怕是白费的了。''江别鹤笑道:``花公子年纪虽轻,涵养功夫却已炉火纯青,要他动怒,除非\ldots\ldots{}''小鱼儿大声道:``要他动怒,除非将铁心兰抢过来是么?''花无缺面色果然微微一变,沉声道:``此事与她无关,阁下最好莫要提起她的名字。''小鱼儿大笑道:``铁心兰可不是你的,你有什么资格不许别人提起她的名字!''也不知怎地,小鱼儿突然觉得身子里有一股热血直冲上来。变得什么也不怕了,一心愿激怒花无缺,一心只想叫花无缺丢人现眼,他明知自己不是花无缺的敌手,却一心想和花无缺拼一拚!那无论胜负生死,至少也可将那满腔热血发散发散!否则整个人只怕都要烧为灰烬!

还因为他确实是个非常非常聪明的人,不但很了解别人,也很了解自己,他知道自己现在不如花无缺,所以他只有忍耐。若没有别人压力,若没有导火线,他也许会一直这样忍耐下去,忍到他能胜过花无缺的那一天。

但此刻情况实在压得他透不过气,而``铁心兰''这三个字正是导火线,他拚命压制住的热血终于突然爆发!

他不但眸子发了光,甚至连瞳孔都异样的张大了!

他狂笑着大声接道:``花无缺,老实告诉你,铁心兰早已有了心上人!她的心早已属于他了,你无论如何也夺不去的,你就算能将她娶为妻子,她的心还是在别人那里!''狂笑声中,他身形突然冲天而起!

就在这刹那间,花无缺手掌已挥出,小鱼儿身形跃起,若是迟了半步,他胸膛只怕便巳被击碎!

大厅的梁木,离地四丈开外!小鱼儿这一跃,竟已攀着了梁木!

他手掌搭在梁上,身子有如秋技上的枯叶般飘荡不定,由下面望上去,似乎随时都会跌落下来!

但江别鹤却已瞧出,这正是轻功中最高妙的身法,他身子看来摇摇欲坠,其实每一动荡中都藏有杀机。

何况他一跃而起,居高临下,虽末抢得先机,却已占有地利,此刻无论是谁,若是跃起迎击,只怕都要遭到当头棒喝!

花无缺却非但没有跃起进击之意,甚至连瞧都没有向上瞧一眼,他只是静静地站在那里,目光竟望着自己的脚尖。

他竟似已处于老僧人定般的绝对静止的状态,对身外的一切事,都似不闻不问,他竟似已站在那里睡着了。

但小鱼儿却知道他此刻心灵正是一片空灵,看似对一切都不闻不问,其实任何人的一举一动都逃不过他的心眼!

小鱼儿在这有利的地位中,他也许还不会出手,但小鱼儿身形只要一展动,先机立失,只怕立刻便要遭他的杀手!

这两人一上一下,一动一静,竟这样僵持着!

别人虽然瞧不出其内的奥妙,但却已感觉出这情况的紧张,嘈乱的大厅竟奇异地静寂下来!

时候过去越久,这紧张的气氛越是沉重。小鱼儿仍在不停的飘荡着,但众人已不再觉得他摇摇欲坠,只觉得这不定的飘荡,竟荡得自己头晕目眩,神情不定。

他们纵然不敢再向上望,但大厅中的烛火却似已随着小鱼儿的飘荡而飘荡,到后来竟连整个大厅都似乎也飘荡起来。

只有江别鹤,他凝望着花无缺,神色仍是那么安详。

花无缺笔直凝立着的身形,就像是惊涛骇浪中的砥柱,不但自己屹立如山,也给别人一份安定的感觉。

别人只觉他屹立不动的身形,竟有一股杀气发散出来,凛凛然逼人眉睫,逼得人连气都透不过来!

这一动一静,正成了强烈的对比。他两人身体相隔虽有四丈,但其间却已不能容一物!

但动的自然终究不能如静的持久。

江别鹤自然知道这点,嘴角不觉已泛起了笑容!

突然,一只燕子自窗外飞了进来。

这是只迷失了方向的孤燕,盲目地冲人了有光和亮的地方,为的只怕是来寻求一份温暖。

它竟飞入了小鱼儿与花无缺相持着的身形之中!

众人也不见小鱼儿与花无缺有任何动作,但这燕子却不知怎地,竟飞不过这无形的杀气。

这燕子竟直坠下来!落下的燕影,掠过了花无缺的脸!就在这时小鱼儿身形突然飞扑而下。

他整个人都似己变成了一个陀螺,在空中不停地旋转。旋转着直落而下,远远望去,他四面八方看来竟似有手脚飞舞。

众人只瞧得眼花缭乱,竟疑有千手千臂的天相天魔,自天飞降!

花无缺却仍未抬头去瞧一眼,小鱼儿凌空一声暴喝,旋转着攻出八腿十六掌!

他招式之快,已非力能所及,看来他一个人身上,竟似有八条腿十六只手掌一齐攻了出来!一齐攻向花无缺!

达一轮急攻虽是虚多实少,但虚实互变,虚招变成实招,只要被他一招击中,那是万无生理。

花无缺突然抬起头来。

飘摇的灯光下,只见他目光闪烁如星,面上似笑非笑,右掌挥出,轻轻一引一拨,看来既非攻招,亦非守势!

只听``劈拚,□通''一连串声响,小鱼儿左掌竟打在自己右掌上,右掌着了自己左掌,左掌之力末竭,又打着自己右拳,右掌之力未竭,又打着自己的左掌,下面也是左腿踢右掌,右腿踢左掌。

他一心制胜的攻势,竟全都打在自己身上,他身子被打得直转,斜斜飘开数尺,``噗''的跌了下去!

江别鹤瞧得眉飞色舞,大声笑道:``好!好一招移花接玉!''只见小鱼儿双掌惧已红肿,胸膛不住喘息,竟已爬不起来。

花无缺瞧着他,微微笑道:``你武功之高,倒也可算是当今武林中的一流高手,内力之强,更出乎我意料之外,只可惜你内力越强,此刻受伤也越重!''他一面说话,一面向小鱼儿缓缓走了过去!

突然,满厅急风骤响,灯火突然灭绝,还有十数道强劲的暗器风声,直打江别鹤与花无缺!

但这样的暗器,还是伤不了江别鹤与花无缺!这两人轻轻一跃,便自闪过。

这时厅堂中已乱成一团,混乱中,只听那罗九大喝道:``请大家站在原地,莫要乱动!''罗三喝道:``莫要被那人乘乱逃走了!''

这些话本是江别鹤要说的,江别鹤听了,不禁暗中点点头,``这罗氏兄弟果然是好角色!''又听得罗九喝道;``我去外面防他逃走,你快点火!''接着,火光一闪,也已亮起了火折子,再瞧方才在地上爬不起的那``幽灵''果然己不见了!

江别鹤面色一变,掠到窗前,窗外夜色沉沉,不见人影。

罗叁跺足道:``这跑得好快,咱们快追吧!''

花无缺缓缓道:``此间出路如此之多,要追只怕也无从追起!''江别鹤皱眉道:``难道就让他这样逃了?''

花无缺道:``以他方才出手之力,被我移力击伤了他自己的手足,他本是无法逃的!''江别鹤恨恨道:``这自然是那将灯光击灭的人,出手救了他。''罗三道:``家兄只怕已去追赶,却不知追不追得着!''花无缺缓缀道:``令兄只怕是追不着的。''

罗三道:``哦?''

花无缺道:``那暗中出手的人,既能在我等面前将人救走,自然有出类拔萃的身手,我等既被他以暗器阻延了片刻,只怕是再也追不着他的了!''罗三苦笑了笑,道:``不错,那人既能在花公子面前将人救走,家兄自然是追不着他的!''灯光一灭,小鱼儿就知道是救星到了,他正想挣扎着爬起,已有一人抱起了他,穿窗而出!这人的轻功竟是江湖中的顶尖身手,轻轻几掠,已在十余丈外。

凉风扑面,小鱼儿的手脚仍在隐隐发疼,他想起了花无缺那惊人的神秘武功,心里更不禁暗暗吃惊。

方才那一瞬间,委实是生死一发,惊险绝伦,若不是这人出手相救,小鱼儿是万万逃不了的,但这人却是谁呢?

小鱼儿忍不住道:``承蒙阁下出手相救,多谢多谢。''那人脚下不停,口中道:``嗯!''他将小鱼儿挟在肋下,小鱼儿也瞧不见他的面目。

过了半响,小鱼儿又道:``你可知道,我并不是什么好人,你为何要救我?''那人笑道:``你也不坏。''

小鱼儿道:``但我却不认得你,你是谁呢?''

那人道:``你猜。小鱼儿道:''听你语声,你年纪并不太大。``那人笑道:''却也不小了。"

小鱼儿道:``你自然不会是神锡道长。''

那人道:``哦。''

小鱼儿道:``你若是神锡道长,就不会叫我猜了,出家人绝不会像你这样鬼鬼祟祟。''人家救了他,他居然还要骂人,只因他一心想逼这人多说几句话,好听出他的语声是谁。

哪知这人只是笑了笑,道:``你说的不错。''

小鱼儿还是听不出他的声音,眼珠子一转,道:``你莫非是轩辕三光?''那人笑道:``我不认识那赌鬼。''

小鱼儿忍不住大声道:``你究竟是人是鬼?''

那人笑道:``你永远猜不出我是谁的。''

小鱼儿道:``你莫以为我的手脚真不能动,你若再不说,我就点了你的穴道,绑住你,看你究竟是谁。''一面说话,他的手果然已按住了那人的腰眼。

那人道:``你莫忘了,我可是你的救命恩人。''小鱼儿道:``我可不领你的情!有些人出手救人,也是没有存好心的,你从别人手中救了我,说不定是为了要利用我,也说不定是为了要把我害得更惨。''那人大笑道:``你这人果然难以对付,我阅人无数,倒真未见过像你这么难对付的人\ldots\ldots{}''说话间已掠入了一扇窗子,将小鱼儿放了下来。

这窗子竟似是通夜开着的,屋子里居然还点着灯,灯光下,小鱼儿终于瞧见了这人的脸。

这人竟是那神秘的罗九!

小鱼儿吃惊得瞪大眼睛,喃喃道:``是你\ldots\ldots 怎会是你?''罗九笑道:``我就知道你是永远猜不着的。''

小鱼儿道:但\ldots\ldots 但我方才明明还听见你在那大厅中喝话。``罗九笑道:''那是我兄弟罗三,他一个人装着两个人说话的声音,别人以为我留在那里还未走,自然想不到出手救你的人是我了。``小鱼儿大笑道:''果然是妙计,这连我都上了当,那些人想不上当更不可能了!``罗九笑道:''要江别鹤那老狐狸上当,可真不是件容易的事。``小鱼儿目光灼灼地瞧着他,道:''不错,要江别鹤上当真不容易,但你却能令江别鹤也上当。``小鱼儿眼珠子一转,道:''那么,我再问你,我和你一不沾亲,二不带故,你为何要救我?``罗九道:''在下只是仰慕兄台的为人,不忍见兄台被逼,是以忍不住要冒险出手相救了。``小鱼儿冷笑道:''你只怕是看见我有两下予,想利用利用我罗九大笑道:``兄台如此说,未免错怪好人了。''小鱼儿道:``人与人之间,本来大多就是互相利用,你想利用我,又岂知我不想利用你,你若有所求,只管说就是,我绝不怪你。''罗九附掌大笑道:"兄台倒当真是快人快语,在下好生佩服\ldots\ldots{}

他突然顿住笑声,逼视着小鱼儿,沉声道:``在下瞧兄台所作所为,无一不是想揭破江别鹤的假面目,而在下也的确早有此心,是以才\ldots\ldots\ldots{}''小鱼儿道:``是以才找上了我,是么?''

罗九大笑道:``兄台若能与在下联手,江别鹤纵然奸猾如狐,此番只怕也要无所遁形了。''他眼睛盯着小鱼儿,小鱼儿眼睛也盯着他,缓缓道:``你明明帮着铁无双和赵香灵,却又在暗中和江别鹤勾结,你明明和江别鹤勾勾搭搭,却又要在暗中结识我,这究竟是为了什么?好,我也不管你究竟存何居心,只要你是真心想揭破江别鹤的假面目,我就和你联盟握手,在这件事上我总支持你到底!''

\hypertarget{ux7b2cux4e94ux5341ux7ae0-ux610fux6599ux4e4bux5916}{%
\chapter{第五十章
意料之外}\label{ux7b2cux4e94ux5341ux7ae0-ux610fux6599ux4e4bux5916}}

这间屋子乃是间小小的阁楼,但布置得却极为精雅,厚厚的地毯上织着琉璃的花纹,人走在上面,绝不会发出丝毫声音。

小鱼儿这时才有空四下打量,只见桌上摆着些奇异而贵重的珍玩,壁上也接着精巧的饰品。有的是黄金铸成的小刀小剑,有的是白玉塑成的小人小马,还有些丑恶的怪兽妖魔,美丽的仙子神女。

罗九笑道:``兄台看这屋子如何?''

小鱼儿道:``这究竟是谁的屋子,你就随意闯了进来。''罗九笑道:``这就是蜗居。''

小鱼儿骇了一跳,道:``这就是你的家?你不怕江别鹤找来?罗九笑道;''兄台大可放心,小弟这居处,是谁也不知道的。``小鱼儿笑道:''你倒真是深谋远虑,居然在这里也布置了一个这样的地方\ldots\ldots{}``罗九道:''此处虽乃我兄弟所有,但却非我兄弟布置的。``小鱼儿道:哦?''

罗九神秘地一笑,道:``布置此地的人,兄台见了,必定极感兴趣。''小鱼儿道:``为什么?''

罗九笑道;``只因她乃是绝世的美人。''

小鱼儿大笑道;``美人\ldots\ldots 我见了美人就头疼得要命。''罗九笑道:``兄台虽然无视于美色,但是她\ldots\ldots 她却和别人不同,她不但美,而且还带着一种说不出的神秘之感,想来必定会合兄台的脾胃。''小鱼儿笑道:所你说得这么妙,我倒也想瞧瞧了。``罗九拉了拉系铃的绳索,笑道:''兄台立刻就可以瞧见了。小鱼儿道:``能布置出这种地方的人,想来必定有些和别人不同之处\ldots。.''心念一转,突然改变话题,道,``江别鹤他可是还住在那破屋子里么?''罗九笑道:``虽然还是那地方,但屋子却已不破了。''小鱼儿道:``他不是不愿别人为他修建的么?如今为何又改变了主意?''罗九道:``但这次是花无缺为他修建的,而且花无缺自己也住在那里。小鱼儿叹道:''不想花无缺居然被这种人缠上了,我倒真有些为他可惜``罗九赔笑道:''江别鹤外表做得那么仁义,不知他真面目的人,谁不愿和他结交为友?花无缺武功虽然不错,但究竟少年无知``\ldots\ldots{}''小鱼儿冷笑道:``花无缺聪明内蕴,深藏不露,你若以为他少年无知,那你就是无知了。''罗九目光闪动,道:``兄台莫非与花无缺相知颇深?''小鱼儿微微笑道:``你知不知道这句话!对一个人了解最深的,常常是他最大的仇人!''他突然感觉到身后一种异样的感觉,霍然回头──一个人幽灵般站在他身后,灯光,正照着她的脸。

这果然是张绝美的脸,她柳眉轻颦,大大的眼睛里,像是弥漫着烟雾。

她眼睛瞧着小鱼儿,却像是没有瞧着小鱼儿,她虽然好生生站在那里,但看来却像是在做梦。她赫然竟是慕容九。

小鱼儿一眼瞧过,也不禁瞧得呆了。

罗九却像是没有留意到他神情的改变,却笑道,``这位梦姑娘,就是布置此间的人。''小鱼儿道:``梦姑娘?''

罗九道:``我瞧见她的时候,她就是这样子,迷迷糊糊的一个人东逛西走,我问她愿不愿意跟我回来,她笑嘻嘻地点了点头,我问她叫什么名字,她还是笑嘻嘻点了点头\ldots\ldots 唉,她整天像是在做梦似的,所以就叫她梦姑娘。''小鱼儿自然知道她受的是什么刺激,为何会变得如此模样,但他却只是轻轻叹了口气,道:``梦姑娘\ldots\ldots 这名字倒不错。''罗九瞧了他两眼,忽然道:``兄台莫非认得她?''小鱼儿道:``你瞧她可认得我么?''

慕容九眼中一片迷雾,像是什么人都不认得。

罗九笑道:``兄台自然不会认得她的,只是\ldots\ldots 兄台你瞧她怎样?''小鱼儿眼珠子一转,道:``我说好又有什么用,你难道舍得将她送给我?''罗九笑道:``兄台既然已与在下结盟,在下所有之物,便是兄台所有之物,何况我兄弟又老又懒又胖,兄台总该知道,这老、胖、懒三个宇,正是好色的最大克星吧。''小色儿大笑道:``你既如此慷慨,我倒也不便客气了。''突听笑声起自窗外,一人穿窗而入,正是罗三。

罗九道;``你怎地也回来了?那江别鹤可曾怀疑到我?''罗叁笑道:``他自然做梦也不会怀疑到你我身上,此刻铁无双已死,赵香灵更骇得千依百顺,唯命是从,他嘴里不说,心里早高兴得不知该如何是好了。''小鱼儿突然道:``死了的那人并不是唯一的人证。''罗九、罗三对望了一眼,同时道:``还有谁?''小鱼儿道:``你莫忘了,还有他儿子江玉郎。''罗九道:``但江玉郎又怎会揭穿他老子的阴谋?''小鱼儿懒懒地一笑,道:``我也许会有法子的。''他长长打了个哈欠,整个人从椅子上溜了下来,倒在那又软又厚的地毯上,喃喃地道:``温暖的太阳,辽阔的大草原\ldots。.这地毯真像是那草原上的长草,又轻,又软,又暖和,人若能在上面舒舒服服的睡上个三天三夜,只怕就应该是非常满足的了。''罗九笑道:``兄台只管睡吧,在这里,绝不会有什么人来打扰的。''一个人若无论在什么情况下都睡得着,这人真是非常有福气──小鱼儿无疑是有福气的。

他也不知睡了多久,醒来的时候,烛火已死了,像是白天,但厚厚的窗掩住日色,屋里的光线朦胧。朦胧中,有一双亮晶晶的眼睛正在凝注着他。

小鱼儿躺在那里,动也没有动。

他瞧见慕容九就坐在他身旁的地毯上,像是刚刚坐下来,又像是自昨夜起就一直坐在那里。

小鱼儿也睁开了眼睛瞧着她,竟不觉瞧得痴了,他没有说话,自然更没有期望她说话。

哪知幕容九竟突然道:``我好像在什么地方瞧过你,我好像认得你。''小鱼儿的心─跳,道:``你认得我?''

慕容九道:``嗯。''

小鱼儿道:``你可记得在什么地方瞧见过我?''慕容九叹道;``我已记不清了\ldots\ldots 我只是有这种感觉。''小鱼儿笑了,转着眼珠子,道:``你可记得你自己么?''慕容九突然双手捧着头,道:``我也不记得,我不能想,我一想就头痛。''小鱼儿道:``那你就不要想吧,你最好不要想,想起来反而不好。''慕容九道:``你。\ldots 你莫非知道我以前是谁?''小鱼儿笑道:``我也记不清了,我只知道,你现在这样子,比以前可爱得多。''还是夏天,小室中热得令人懒洋洋的提不起精神,虽然没有风,空气中却有一阵淡香传来。

小鱼儿一觉睡醒,全身都充满了过剩的精力,他瞧着那圆润的、莹白的足踝,竟不觉连想起那日在冰室中她赤裸的胴体\ldots.在这焕热的夏日黄昏里,他突然兴起了一种邪恶的感觉。

他突然笑道:``但你无论如何,还是想知道自己以前是什么样子,是么?''慕容九道:``我假如能想起以前的事,就算立刻死了都愿意。''小鱼儿道:``好,你先脱光,我替你想法子。''幕容九眼睛睁得更大,颤声道:``脱\ldots\ldots 脱光衣服。''小鱼儿道:``你一定是遇着了什么可怕的事,才变得这样子,只因那件事的恐怖,现在还像恶魔似的盘踞在你身体里。''慕容九轻轻点着头,道:``嗯。''

小鱼儿道:``所以,你要想起以前的事,就得先将身体里的恶魔赶走,你要赶走这恶魔,就得先解除一切束缚。''慕容九像是听得痴了,不断地点着头。

小鱼儿笑嘻嘻地道:``衣服就是人最大的束缚,你先脱光衣服,我才可以帮你把恶魔赶走,这道理简单得很,你总该听得懂,是么?''慕容九道:``但\ldots。但\ldots\ldots{}''

小鱼儿的手已摸到她的足踝,笑道:``你听我的话,绝不会错的\ldots\ldots{}''他话未说完,慕容九突然跳了起来,手里已多了柄精光闪闪的匕首,直逼着小鱼儿的咽喉。

小鱼儿失声道:``你这是干什么?我不是在帮你的忙么?''慕容九缓缓道:``有人告诉我,无论谁想碰我的身子,我就该拿这把刀对付他。''小鱼儿眼珠子一转,喃喃苦笑道:``难怪罗家两兄弟不敢碰你──难怪他们要将你送给我。''慕容九道:``你说什么?''

小鱼儿道:``你可认识他们么?''

慕容九道:``我好像不认识。''

小鱼儿道:``但你却认识我,你为什么不相信我而相信他们呢?''慕容九低着头想了想,匕首已跌落在地毯上。

小鱼儿一把将她拉了下来,压在她身上,慕容九完全没有反抗,小鱼儿的手已拉开了她的衣襟,嘴里自言自语,喃喃道;``假如一个人差点杀死你,你无论对她怎样,也不能算说不过去吧。''他的嘴在说话,手也在动。

突听一人冷冷道:``不可以!''

小鱼儿一惊,那厚厚的窗后,已飞出一条银丝,毒蛇般缠住了他的手,以小鱼儿此刻的武功,竟没有闪开,竟没有挣脱。

接着,一条瘦小的人影,鬼魅般自窗里飞了出来,直扑小鱼儿,小鱼儿一个筋斗翻了出去,反手去扯那银丝。

那又细又长的银丝,虽被他扯得笔直,他竟扯不断。

他自然也瞧清了那瘦小的人影,全身都被一件黑得发光的衣服紧紧裹住,一张脸也蒙着漆黑的面具,只留下一双黑多白少的眸子,这双阵子不停地眨动,看来就好像鬼脸窥人,也说不出有多么诡秘恐怖。

小鱼儿失声道:``你是黑蜘蛛!''

黑蜘蛛身形已展,硬生生又自顿住,冷冷道:``你是谁?竟认得我!''小鱼儿笑道:``黑老弟,你难道不认得我了?''黑蜘蛛眼睛一亮,道:``呀,是你!你竟会变成这模样?''小鱼儿笑嘻嘻道:``你不愿意以真面目示人,我难道就不能变变面貌么?''黑蛛蜘目光灼灼,道:``一个人在做如此卑鄙的事的时候,被我撞见,居然还能笑嘻嘻地对我说话。\ldots 像这样的人,除了你之外天下只怕没有第二个。''小鱼儿笑道:``这又怎能算卑鄙的事''\ldots\ldots 只要是年轻力壮的男人,谁都可能做出这样的事来。"黑蜘蛛瞪着眼瞧着他,似乎在奇怪!一个人做出这样的事后,怎么还能如此理直气壮,竟像是真的丝毫没有恶意。

小鱼儿接着笑道:``何况,这种事本来就没什么的,只有一个存心龌龊的人,才会将它瞧得变了样,像我这样的人,做了它固然不会觉得难受,不做它也不会觉得难受的。''黑蜘蛛突然笑了,道:``像这种胡说八道的话,自你嘴里说出来,竟一点不令人觉得可恶,这是什么道理呢?''小鱼儿道:``这因为我根本不是个可恶的人呀。''突听门外一阵脚步声传来,黑蜘蛛身形一闪,又到了窗后,银丝也跟着飞了回去。

小鱼儿就站在那里,嘴里却发出沉沉的鼻息,那人似乎在门外听了半晌,然后,脚步声又退了回去。

但拉开窗,黑蜘蛛却不见了。

窗外日色将落未落,犹未黄昏,小鱼儿喃喃道:``白天,还是白天,这黑蜘蛛在大白天里就能飞檐走壁,来去自如,难怪江湖中人都将他当做怪物。''慕容九痴痴地站在那里,轻轻道:你也觉得他奇怪?``小鱼儿转过头,盯着她,道:''给你那把刀的,就是他?他难道不怕被人发觉?``慕容九咬着嘴唇,像是想了许久,才慢慢道:''他们虽然也怀疑有人常在附近,但想尽方法还是瞧不见他的人影,他来的时候,总是只有我单独一个人。"

小鱼儿皱了皱眉头,道:``他常来看你,他常在附近。\ldots 莫非他也对这罗家兄弟起了怀疑?这兄弟俩能令这种人花如此多工夫在他们身上,究竟是什么样的身份?''他低着头兜了两个圈子,猛抬头,便瞧见慕容九竟已脱光了衣服,赤裸裸地站在那里。

朦胧中,她青春的胴体,就像缎子似的发着光,她修长而坚实的双腿,紧紧并拢着,她柔软的胸膛,悄然挺立\ldots\ldots 穿着衣服的慕容九,看来虽是那么纤弱,但除却衣服,她全身每一寸都似乎含蕴着慑人的成熟魅力。

这是小鱼儿第二次瞧见她赤裸的胴体,第一次是在那充满了诡秘意味的冰室中,而此刻\ldots\ldots 小室中香气迷蒙,光影朦胧,空气中似乎有一种逼人发狂的热力,小鱼儿额上不觉迸出了汗珠,喉咙也干燥起来,嗄声道:``你这是干什么?''幕容九痴痴地瞧着他一步步走了过来,道:``我要你帮我赶去身子里恶魔''。"``小鱼儿大声道:''你身子里并没有什么恶鬼,我那是骗你的。``慕容九道:''我知道有的,它现在已经在我身子里动了,我已可感觉得出。"她痴痴地笑着,雪白的牙齿就像是野兽般在发着光,她苍白的面颊已嫣红,她眼睛里也发出了异样的光。

小鱼儿竟不觉后迟了半步,大叫道:``胡说,快穿起衣服来,否则\ldots\ldots{}''慕容九道:``我不穿衣服,我要你帮我\ldots\ldots{}''"

她突然扑到小鱼儿身上,两手两脚,就像是八爪鱼似的紧紧缠住了小鱼儿,于是两人一起倒在地上。

她冰冷的身子,突然变得火山般灼热,嘴唇狠命压着小鱼儿的脸,胸腔喘息着,小鱼儿手掌轻轻抚着她光滑的背脊。

他突然掀起幕容九的头发,将她压在下面,然后抽过条毯子,将她裹粽子似的裹了起来,紧紧绑住。

慕容九眼睛里满是惊骇之色,嘶声道:``你。\ldots 你为什么这样?''小鱼儿笑嘻嘻地瞧了她一眼,又提起她脱下来的衣服瞧了瞧,将桌上一壶冷茶,慢慢地从她头上淋下去,笑嘻嘻道:``记着,女孩子不可随便脱衣服的,她至少也该等男孩子替她脱,下次你若再这样,看我不打你的屁股!''慕容九被冷茶淋得几乎喘不过气来,大声道:``你这恶棍,放开我\ldots。''小鱼儿不再理她,将倒干了的茶壶用她的衣服包住,轻轻放在她胸膛上,推开门,``咚、咚、咚''走下了阁楼。

小鱼儿在楼下走了一遍,只瞧见两个呆头呆脑的傻丫头,却找不着那罗九和罗三兄弟两个人。

小鱼儿走进了厨房,洗了个脸,又用昨天剩下来的材料,将自己的脸改成另一副样子,才大摇大摆地走出去。

这房子竟在闹市之中,小鱼儿在街头的成衣铺买了套新衣服换起来,又在旁边的酒楼痛痛快快吃了一顿,抬头仰望天色,笑道:``天快黑了,我活动的时候又快到了\ldots.''他对自己方才做的那件事觉得很得意,此刻全身都痛快得很,充满活力,只觉不好好干一场,未免太对不起自己。

这时天色已将入暮,小鱼儿走到那药铺去逛了一圈,还买了个紫金锭,药铺里果然没有一个人认得他。于是,小鱼儿直奔郊外他本想先到段合肥家去的,但临时又改变了主意,只因他瞧见有许多武林人物匆匆出城,想来是赶到天香塘去的。

要知``爱才如命''铁无双成名数十年,数十年来,蒙他提拔、受他好处的人也不知有多少。

小鱼儿远远便瞧见,``地灵庄''里灯火辉煌,人影幢幢,偌大的庭院里,几乎已挤满了各色各样的人物。

庄门外,也停满了各色各样的车马,小鱼儿匆匆走过去,突又停下脚步,马群中有匹马嘶声分外响亮,竟像``小仙女''的胭脂马。

``小仙女''张菁莫非也来了?!

小鱼儿嘴角不禁泛起了微笑:``这两年来,她怎样了?是不是还像以前一样,穿着火红的衣服,骑着马到处跑来跑去?到处用鞭子打人?''他实在想瞧瞧那又刁蛮、又泼辣、又凶恶、又美丽的小女人,这两年来,她至少总该长大了些,却不知是否比以前懂事了些。

但院子里的人实在太多,小鱼儿东张西望,非但没瞧见她的影子,简直连一个穿红衣服的姑娘都没瞧见。

``她若来了,必定抢眼的很,我怎会瞧不见她?像她这种人在十万个人里也该被人一眼就瞧出来的。''小鱼儿暗中嘀咕,心里竟不禁有些失望。

\hypertarget{ux7b2cux4e94ux5341ux4e00ux7ae0-ux5047ux4ec1ux5047ux4e49}{%
\chapter{第五十一章
假仁假义}\label{ux7b2cux4e94ux5341ux4e00ux7ae0-ux5047ux4ec1ux5047ux4e49}}

铁无双的棺木,就放在大厅中央,赵香灵哭丧着脸站在一旁,居然还为他披麻带孝,活脱脱一副孝子的模样。

吊丧的客人,却都挤在院子里,叁五成群,交头接耳,指指点的也不知在谈论些什麽。

突听庄院外一阵骚动,人声纷纷道:江大侠竟也来了!

江大侠行事素来仁义,我早就知道他会来的。

院子里的人立刻分立两旁,让出了条路,一个个打躬作揖,有几个直恨不得跪下去磕头。

七、八条蓝衣大汉,已拥着江别鹤大步而入。

只见他双眉深锁,面色沉重,笔直走到铁无双灵前,恭恭敬敬叩了叁个头,沉声道:``铁老英雄,你生前江某虽然与你为敌,但那也是为了江湖道义,情非得已,你英灵非遥,也该知道江某的一番苦心,而今以后,但望你在天英灵能助江某一臂之力,为武林维护正义,春秋四祀,江某也必定代表天下武林同道,到你灵然,祝你英魂安息。''这番话当真说得大仁大义,掷地成声,群豪听了,更不禁众人一声,称赞江别鹤的侠心。

小鱼儿听了却不禁直犯恶心,冷笑暗道:``这才真的叫猫哭老鼠假慈悲''\ldots\ldots{}``一念尚未转过,突听一人大声冷笑道:''这才真叫猫哭老鼠假慈悲,杀了别人还来为人流泪。"语声又高又亮,竟似是女人的声音。

众豪杰都不禁为之动容,向语声发出的方向瞧过去。只见说话的乃是个黑衣女子,头戴着马连坡大草帽,紧压着眉目,虽在夏夜中,却穿着长可及地的黑缎披风,这许多人瞪眼去瞧她,她也毫不在乎,也用那发亮的大眼睛去瞪别人。

她身旁还有个长身玉立的华衣少年,神情却像是个大姑娘似的,别人瞧他一眼,他就臊得不敢抬头。

小鱼儿一眼使瞧出这两人是谁了,心里不觉又惊又喜!``她果然来了,她居然还是那六亲不认的老脾气,一点儿也没变。''这时人丛中已有好几个涌了过去,指着那黑衣女子骂道:``你是何方来的女子,怎敢对江大侠如此无礼。''那黑衣女子冷冷道:``我高兴说什么就说什么,谁管得着我?''虬髯大汉喝道:``江大侠宽宏大量,老子今天却要替江大侠管教管教你!''喝声中他已伸出一双蒲扇般大小的巴掌抓了过去,黑夜女子冷笑着动也不动,她身旁那腼腆的少年却突然伸臂一格!

这看来霸王般的大汉,竟被这少年轻轻一格,震得飞了出去,群豪耸然失声,又有几人怒喝着要扑上去!

那少年双拳一引,摆了个架式,竟如山停岳峙,神充气足,他不出手时看来像是个羞羞答答的大姑娘,此刻乍出手,竟隐然有一代宗匠的气派,群豪中有识货的,已不禁为之骇然动容。

那黑衣少女冷笑道:``你尽管替我打,出事了有我!''那少年看来倒真听话,左脚前踏半步,右拳已闪电般直击面出,当先一条大汉,又被震得飞了出去。

突听-声轻叱,一人道;``且慢!住手!''

叱声未了,江别鹤已笑吟吟挡在这少年面前,江别鹤捻须笑道:``若是在下双眼不盲,兄台想必就是玉面神拳顾人玉顾二公子。''小鱼儿暗道:``这江别鹤当真生了一双好毒的眼睛。''只见顾人玉还未说话,那黑衣女子已拉着他的手,冷笑道,``咱们犯不着跟他攀交情,咱们走!''``走''字出口,两条人影已飞掠而起,自人丛上直飞出去,黑缎的斗篷迎风飞舞,露出了里面的一身火红的衣服。

群豪中已有人失声道;``这莫非是小仙女!''

但这时两人已掠出庄门,一声呼哨,蹄声骤响,一匹火红的胭脂马急驰而来,载着这两人飞也似的走了。

江别鹤目送他两人身影远去,捻须叹道:``名家之子弟,身手果然是不同凡俗。''突见一条泥腿子,手里高挑着根竹竿,快步奔了进来。

竹竿上高挂着副白布挽联,挽联上龙飞风舞地写着:``你活着,我难受。''``你死了,我伤心。''

这十二个字写得墨迹淋漓,雄伟开阔,似是名家的手笔,但语句却是奇怪之极,不通之极。

群豪又是惊奇,又是好笑,但瞧见挽联上写的上下款,脸色却都变了,再无一人笑得出来。只见那上款写的是------``老丈人千古。''下款赫然竟是:愚婿李大嘴敬挽。"

小鱼儿吃一惊,仔细瞧瞧,这挽联写的竟真有些像李大嘴的笔迹,李大嘴莫非也已真的出了``恶人谷''?他几时出来的?此刻在哪里?

江别鹤迎面挡住了那泥腿汉子,沉声道:``这挽联是谁叫你送来的?''那泥腿汉子眨着眼睛道:``黑夜中我也没有瞧清他是何模样,只觉他生得似乎甚是高大,相貌凶恶得很,有几分像是庙里的判官像。''江别鹤道,``他除了叫你送这挽联来,还说了什么话?''那泥腿汉子支支晤晤,终于道;``他还说,他老丈人虽要宰他,但别人宰了他老丈人,他还是气愤.他叫那宰了他老丈人的人快洗干净身子,我忍不住问他为什么要人家将身子洗干净,他刚开大嘴一笑,回头就走了。''江别鹤面色一变,再不说话,大踏步走了出去。

那泥腿汉子却还在大声道:``你老爷予难道也不懂他说的什么意思么?你老爷子\ldots{}''这时群豪已又骚动,淹没了他的语声,纷纷道:``十大恶人已销声匿迹多年,此番这李大嘴一露脸,别的人说不定也要跟着出来了。''又有人道:``除了李大嘴外,还有个恶赌鬼,就算别的人不出来,只这两人已够受的了,这该怎么办呢?''惊叹议论间,谁也没有去留意那泥腿子,只有小鱼儿却跟定了他,只见他将那挽联送了上灵堂,一路东张西望走了出去,小鱼儿暗暗在后面跟着,两人一先一后,走了段路,那汉子突然回身笑道:"我身上刚得了三百两银子,你跟着我莫非想打闷棍么??

小鱼儿也笑嘻嘻道:你究竟是什麽人?假冒李大嘴的名送这挽联来,究竟安的是什麽心思?

那汉子脸色一变,眼睛里突然射出逼人的光,这眼光竟比江别鹤还深沉,比恶赌鬼还凌厉。

但一瞬间他又阖起了眼帘,笑道:``人家我叁佰两银子,我就送挽联,别的事我可不知道.''小鱼儿笑道:我跟在你後面,你怎会知道,你明明有一身武功,还想瞒我?

那汉子大笑道:你说我有武功,找有武功早就做强盗去了,还会来干穷要饭的.

小鱼儿大声道:你不承认,我也要叫你承认!

他一个箭步蹿过去,伸手就打,那知这汉子竟真的不会武功,小鱼儿一拳击出,他竟应声而倒。

小鱼儿还怕他在使诈,等了半晌,这汉子躺在地上动也不动,伸手一摸,这汉子四肢冰冷,心没气,竟已活活被打死了。

小鱼儿倒的确没想到这人竟如此禁不起打,他无缘无故伸手打死了人,心里也不免难受的很,呆了半晌,长叹道:你莫怪我,我出手误伤你,少不得要好生殓葬於你,虽然好死不如歹活,我总也要你死得风光些。

他叹息着将这汉子的身扛了起来,走回城去,走了还不到盏茶时分,突觉脖子湿的还有臊味。

小鱼儿一惊:死人怎会撒尿?

他又又怒,手去擦,``死尸''就掉了下去,他飞一脚去,那死尸"突然平白飞了起来,大笑道:我今天请你喝尿,下次可要请你吃屎了。

笑声中一个斤斗竟翻出数丈,再一晃就不见了。

这人轻功之高,竟不在江别鹤等人之下,等到小鱼儿去追时,风次草动,那里还有他的影子。

小鱼儿从小到大,几时吃过这麽大的哑吧亏?当真差点儿活活被气死,他连这人究竟是谁都不知道,这气自然更没法出。

小鱼儿气得呆了半晌,又突然大笑道:幸好他只是恶作剧,方才他若想杀我,我那里还能活到现在,我本该高兴才是,还生什麽鸟气。

他大笑着往前走,竟像是一点也不生气了,对无可奈何的事,他倒真是想得开。

街上灯火辉煌,正是晚最热闹的时侯。

小鱼儿又买了套衣服换上,正在东游西逛的磨时间,突然一辆大车急驰而过,几乎撞在他身上。小鱼儿也不觉多瞧了两眼。

只见这大车骤然停在一家门面很大的客栈前,过了半晌,几个衣帽光鲜的家丁,从客栈里走出来,拉开车门,垂手侍立在一旁,似乎连大气都不敢喘。

又过了半晌,两个人自客栈中款步而出,四面前呼後拥的跟着一群人,弯腰的弯腰,提灯的提灯。灯光下,只见左面的一人,面色苍白\ldots 身材瘦弱,看来像似弱不禁风,但气度从容,叫人看了说不出的舒服,身上穿的虽然颜色素,线条简单,但一巾一带莫不配合得恰到好处,从头到脚找不出丝毫瑕疵。

右面的一人,身材较高大、神采较飞扬,目光顾盼之间,咄咄逼人,竟有一种令人不可仰视之感。

这人的衣服穿得也较随便,但一套随随便便的普通衣服穿在他身上,竟也变得不普通不随便了。

两人一前一後走上了大车,既没有摆姿势,也没有拿架子,但看来就彷佛和别人有些不同,彷佛生来就该被人前呼後拥,生来就该坐这样的车子。

直到车子走了,小鱼儿还站在那里,喃喃道:这两人又不知是谁?竟有这样的气派\ldots\ldots{}

要知这样的气派,正是装也装不出、学也学不会的。

这安庆城中,此刻竟是侠踪频现,小鱼儿在这一夜之中,所见的竟无一不是出类拔萃、不同凡俗的人物。

小鱼儿叹道:``只可惜我到现在为止,还不知道这些人究竟是谁,也不知道他们是为了什麽来的,但无论如何,这皖北一带,从此必定要热闹起来了''。

小鱼儿逛了半天,不知不觉间又走回罗九那屋子。

此刻夜虽已歇,但距离夜行人活动的时候还是太早,小鱼儿想了想,终於还是走了进去。

在楼下坐了半天,小鱼儿站起来刚想往外走,突然见罗九,罗三从外面奔进来。

罗九,罗三瞧见他又是一惊,後退两步,盯着他瞧了几眼,罗九终於释怀而笑,抱拳道:``兄台好精妙的易容术,看来只怕可算得上是海内第一了''.

小鱼儿笑嘻嘻道:``两位到那里去了?回来得倒真早。''罗九笑道:``今日有贵客降临,江别鹤设宴为他们接风,我兄弟也忝陪未座,所以竟不觉回来迟了''。

罗叁道:``有劳兄台久候,恕罪恕罪''。

小鱼儿,笑问道:``贵客!是谁''?

罗九道:"这两人说来倒当真颇有名气,两人俱是「九秀庄」慕容家的姑爷,一位是「南宫世家」的传人南宫柳,一位是江湖中的才子,也是两广武林的盟主秦剑。

小鱼儿眼睛亮了,道:慕容家的姑爷!妙极妙极。

罗叁道:能娶到慕容家姑娘的人,当真是人人艳羡,这些人本身条件,也委实不差,就说那南宫柳,虽然体弱多病,但看来也令人不敢轻视。

罗九道:听兄台说话,莫非认得他们?

小鱼儿道:我虽不认得他们,方才却瞧见了他们\ldots\ldots 这两人可是一个睑色苍白\ldots 衣服考究,另一个得意扬扬,像是刚捡着叁百两银子似的。

罗九笑道:不错,正是这两人。

罗叁道:不但这两人,听说慕容家的另六位姑爷,这两天也要一起赶来,另外还有位准姑爷「玉面神拳」顾人玉\ldots\ldots{}

小鱼儿眼睛又一亮,道:顾人玉难道也是和他们一起来的?

小鱼儿眼珠子转了转,又道:这些人全赶到这里来,你可知道是为了什麽?

罗三道:据说,慕容家里有一位姑娘失琮了,而这位姑娘据说曾经和花无缺在一起,所以他们都赶到这里来打听消息。

小鱼儿拍手笑道:这就对了,我早就猜到他们八成是为这件事来的。

罗叁道:兄台难道也认得那位姑娘?

罗九眼睛盯着他,道:兄台莫非知道那姑娘的下落?

小鱼儿连瞧都没有向阁楼那方向瞧一眼,板着脸道:我怎会知道,我难道还会将人家的大姑娘藏起来不成。

罗九笑道:小弟焉有此意,只是\ldots\ldots{}

小鱼儿笑嘻嘻道:说不定这只是她自己跟情人私奔了,也说不定被人用药迷住\ldots\ldots 他又歪着头想了想,突然大笑道:这倒有趣的很,的确有趣的很。

罗九打了个哈哈,往阁楼上瞧了一眼,笑嘻嘻道:兄台这半日又到那里去了?

小鱼儿道:这半天我倒真瞧见了许多有趣的事,也瞧见了许多有趣的人,其中最有趣的一个是\ldots\ldots{}

他虽然吃了个哑吧亏,但丝毫不觉丢人,反而将自己如何上当的事,源源本本说了出来,一面说,一面笑,竟像是在说笑话似的".

罗九、罗三听了,虽也跟着在笑,但却是皮笑肉不笑,两人的睑色竟似都有些变了!

两人悄悄使了个眼色,罗九道:却不知那人长得是何模样?

小鱼儿道:那人正是一副标标准准的地痞无赖相,你无论在任何一个城的茶楼赌馆,花街柳巷里,都可以见到,但无论任何人都不会对这种人多瞧一眼的,这也就正是他厉害的地方,不引人注意的人,做起坏事来岂非特别容易。

罗九,罗三两人又交换了个眼色,罗九突然站起来,走进房里,小鱼儿只听得房里有开抽屉的声音,接着,是一阵纸张的簌簌声,然後,罗九又走了出来,手里拿着卷已旧得发黄的纸。

这张纸非但已旧得变色发黄,而且残破不全,但罗九却似将之瞧得甚是珍贵,谨谨慎慎地捧了出来,小小心心地摊在小鱼儿面前桌上,却又用半个身子挡住在小鱼儿视线,像是怕被小鱼儿瞧见。

小鱼儿笑道:这张破纸摔又摔不碎,跌又跌不破,更没有别人会来抢,你怎地却将它瞧得像个宝见似的。

罗九正色道:这张纸虽然残破,但在某些武林人士眼中,却正是无价之宝,兄台若以为没有人会来抢,那就大大错了。

小鱼儿嘻嘻笑道:哦,如此说来,这张纸莫非又是什麽「藏宝图」不成?若真的也是张「藏宝图」,我可瞧郡不愿瞧上一眼。

罗三笑道:江湖中故意害人上当的「藏宝图」的确有不少,一万张「藏宝图」里,真有宝藏的,只怕连一张也没有,听兄台如此说,莫非也是上过当来的。

罗九道:但此图却绝非如此\ldots\ldots{}

小鱼儿道:你将这张纸拿出来,本是让我瞧的,为何又挡住我的眼睛。

罗九陪笑道:兄弟平日虽将此图珍如拱璧,但兄台此刻已非外人,是以在下才肯将它拿出来,只是\ldots\ldots 但望兄台答应,瞧过之後,千万要保守秘密。

小鱼儿也忍不住动了好奇之心,却故意站起来走到一旁,笑道:你若信不过我,我不瞧也罢。

罗三大笑道:我兄弟若信不过兄台,还能信得过谁\ldots\ldots{}

小鱼儿道:你先告诉我这张图上画的是什麽,我再考虑要不要瞧它。

罗九沉声道:这张图上,画的乃是「十大恶人」的真容?

小鱼儿眼睛一亮,却又故意笑道:十大恶人我虽未见过,但听这名字,想来只怕个个都是丑八怪,这又有什麽好瞧的,别人又为何要抢它?

罗九叹道:兄台有所不知,这「十大恶人」,个个都有一身神鬼莫测的本事,个个俱都作恶多端,江湖中曾经受他们所害的人,也不知有多少\ldots\ldots{}

罗叁接道:但这十人非但个个行踪飘忽,而且个个都有乔装改扮的本事,有些人虽然被他害得家破人亡无路可走,却连他们的真面目都未瞧过,这又叫他们如何去寻仇报复,如何来出这怨气。

小鱼儿笑道:我明白了,别人想抢这张图去,只是为了要瞧瞧他们长得究竟是何模样,好去报仇出气。

罗三附掌道:正是如此。

小鱼儿道:但他们跟我却是无冤无仇,你又为何要我来瞧\ldots\ldots{}

罗九神的一笑,道:兄台真的和他们无冤无仇麽?

小鱼儿眼珠子一转,道:你莫非是说那装死的无赖,也是「十大恶人」之一?

罗九且不答话,闪开身子,指着那张图上昼的一个人,缓缓道:兄台不妨来瞧瞧,那无赖是不是他?

发黄的纸上,工笔画出了十个人像,笔法细腻,栩栩如生,一人白衣如雪,面色苍白,正是``血手''杜杀。

杜杀身旁,作仰天大笑状的,自然就是``笑里藏刀''小弥陀哈哈儿,再过去就是那满面媚笑的``迷死人不赔命''萧咪咪及手里捧着个人头,愁眉苦脸在叹气的``不吃人头''李大嘴\ldots\ldots{}

还有一人虚虚的站在一团雾里,不问可知,便是那``半人半鬼''阴九幽,阴九陛身旁一个人却有两个头,左面一个头是小姑娘,右面一个头是美男子,这自然就是``不男不女''屠娇娇。

这些人小鱼儿瞧着不知有多少遍了,只见此图画得不但面貌酷似,而且连他们的神情也昼得唯妙唯肖。

小鱼儿不禁暗中赞赏,又忖道:这张图却不知是谁画的?若非和他们十分熟悉的人,怎能昼得如此传神?

接着,他就瞧到那衣衫落拓、神情极轩昂的``恶赌鬼''轩辕三光,再旁边一人满脸虬髯,满脸杀气,一双眼睛更像是饿狼恶虎,正待择人而嗤,手里提着柄大刀,刀头上鲜血淋漓。

小鱼儿故意问道:此人长得好怕人的模样,却不知谁?

罗九道:他便是``狂狮''铁战。

罗叁道:此人模样虽然凶恶,其实却可说是「十大恶人」中最善良的一人,人家只要不去惹他,他也绝不去惹别人。

小鱼儿道:但别人若是惹了他昵。

罗叁道:谁若惹了他,谁就当真是倒了三辈子的霉了,他若不将那人全家杀得鸡犬不留,再也不肯放手的。

小鱼儿失笑道:这样的人还算善良,那麽我简直是圣人了。

他中虽在答应着别人的话,心里却不觉想起了铁心兰,想起了那似嗔似笑的嘴角,似幽似怨的眼睛\ldots\ldots{}

他心里只觉一阵刺痛,赶紧大声道:这两人又是谁?

这两人显然是一双孪生兄弟,两人俱是瘦骨嶙峋\ldots 双颧凸出,一人手里拿着个算盘,一人手里拿着本帐簿,穿着打扮,虽像是买卖做得极为发达的富商大贾,模样神情,却像是一双刚从地狱逃出来的恶鬼。

罗九笑道:这兄弟一胞双生,焦不离孟、孟不离焦,「十大恶人」虽号称「十大」,其实却有十一个人,只因江湖中人都把这两人算成一人。

罗叁道:这兄弟两人复姓欧阳,外号一个叫做「拚命占便宜」,一个叫「宁死不吃亏」,兄台听这外号,就可知道他们是怎麽样的人了。

罗九道:十大恶人声名虽响,但大都俱是身无馀财,只有这兄弟两人,却是富可敌国的大财主,大富翁。

罗叁指着画上另一人道:``但这人性格却和他兄弟全然相反,这人平生最喜欢害人,一心只想别人上当,至於他自己是否占着便宜,他却全然不管。''小鱼儿笑道:这样的人倒也少见的很,他\ldots\ldots{}

突然失声道:呀!不错,他果然就是那装死的无赖!

画上的人,有的坐着,有的站着,只有这人却是蹲在画纸最下面的角落里,一只手在挖脚丫,一只手放在鼻子上嗅。

画上别的多多少少总有些合大人物的气概,只有这人猥猥琐琐,嘻皮笑脸,活脱脱是个小无赖。

罗九眼睛一亮道:兄台可瞧清楚了?

小鱼儿大声道:一点也不错,就是他!他的脸虽也改扮过,但这神情、这笑容\ldots\ldots 那是万万不会错的。

罗叁叹道:在下一听兄台说超那无赖的行为,便已猜着是他了。

罗九道:此人姓白,自己取名为白开心。

罗叁道:江湖中又他加了个外号,叫「损人不利己」白开心。

小鱼儿失笑道:这倒的确是名符其实,冒名送挽联、装死骗人,这的确都是「损人不利己」的事,别人虽被他害了,他自己也得不着便宜。

小鱼儿突然又道:你兄弟听我一说,就想起他来,莫非和他熟得很?

罗九摸了摸下巴,笑道:我兄弟虽不才,却也不至於和这种人为伍。

小鱼儿笑嘻嘻瞧着地,道:我看你兄弟非但和他熟得很,也和「十大恶人」熟得很,否则怎会对他的行事如此清楚,这张图又怎会在你手里?

罗九面色变了变,罗三已长笑道:不瞒兄台说,「十大恶人」与我兄弟实有不共戴天之仇,我兄弟的父母,便是死在他们的手里。

小鱼儿倒颇觉意外,道:哦\ldots\ldots 真有此事?

罗九道:我兄弟为了报仇,是以不惜千方百计寻来此图,又不惜千方百计,将他们的性格行为,打听得清清楚楚。

小鱼儿道:既是如此,你为何不将此图让大家都瞧瞧,好教别人也去寻他们的晦气,你为何反而替他们保守秘密!

罗九恨声道:我兄弟为了报仇,已不知花了多少心血,我兄弟每日俱在幻想着手刃仇人时的快活,又怎肯让他们死在别人的手里!

小鱼儿想了想,点头道:不错,这也有道理\ldots\ldots 很有道理。

罗九仔仔细细,将那张纸又卷了起来,道:是以兄台下次若再遇着那白开心时,千万要替我兄弟留着。

罗叁接道:兄台若能打听出他的下落,我兄弟更是感激不尽。

小鱼儿目光闪动,笑道:好,白开心是你的,但江玉郎却是我的,你兄弟也得为我留着才是,最好莫要叫别人碰着他一根手指。

罗九笑道:那是自然。

小鱼儿道:老子请客,儿子自然作陪,你今日想必是见过他的了。

罗九道:奇怪就在这里,江别鹤请客,江玉郎并未在席上。

小鱼儿哈哈笑道:这小贼难道连露面都不敢露面了麽?否则遇着南宫柳这样的人物,他爹爹还会不赶紧叫他去结纳结纳。

罗九立刻陪着笑道:那小贼只怕已被兄台吓破了胆。

小鱼儿往阁楼上瞟了一眼,笑道:瞧见一个被自己打死的人,又在自己面前复活了,无论是谁,只怕都要被吓得神智不清,见不得人了。

他这句话中自然另有得意,只是罗九兄弟却再也不会想到这会和阁楼上的女孩子有关,更不会想到这``神智不清''的女孩子就是慕容九。

两人只见小鱼儿眼睛往阁楼上瞟,於是两人齐地站了起来,打了个哈哈,笑道:时侯不早,兄台只怕要安歇了。

小鱼儿大笑道:不错,正是要安歇了。

他站起身了,大笑着往外走了出去。

\hypertarget{ux7b2cux4e94ux5341ux4e8cux7ae0-ux88c5ux50bbux88c5ux75af}{%
\chapter{第五十二章
装傻装疯}\label{ux7b2cux4e94ux5341ux4e8cux7ae0-ux88c5ux50bbux88c5ux75af}}

罗九兄弟了愣了愣,指了指那阁楼,道:兄台今夜难道不睡在上面?

小鱼儿走出了门,回头笑道:那上面有蜘蛛,我睡不着,还是明天再来吧\ldots\ldots 若有江玉郎的消息,两位千万莫忘了为我打听打听。

罗九瞧着他扬长而去,喃喃道:蜘蛛?蜘蛛\ldots\ldots 你瞧这小子是否有些毛病?

罗三道:他有个见鬼的毛病,他这不过是在装疯扮傻,你我可莫要阴沟里翻船,利用他不成,反被他利用了。

罗九格格笑道:这小子虽是一肚子坏水,但比起咱们来又如何?

罗三大笑道:天下的坏人虽多,又有谁比得上咱们?

这时夜已很深,罗九兄弟的居处本就极为偏僻,此刻已无人迹。小鱼儿在街道转了两个圈子。

只见这附近一带,大都是平房,除了那小阁楼外,只有东面五六丈外有座楼房,高出屋脊。

小鱼儿踱了过去,绕着墙角,又兜了个圈子,等到这楼房灯火全都熄灭,他轻轻一跃而上,在屋脊背後的黑暗处伏了下来。

天上月明星稀,地上人声静寂,远远望去,那小阁楼窗户半开,灯火朦胧。慕容九正托着香腮坐在灯畔,幽幽的出神。

突然间,只听衣袂带风之声轻响,一条黑衣人影,鬼魅般掠上屋脊,也伏到屋脊上,向阁楼那方遥望。

小鱼儿暗笑道:不出我所料,果然来了!

慕容九在那边想得出了神,这人影在这里也瞧得出了神,竟全未发觉还有人在旁边瞧着他。

只见他一双黑多白少的眸子在夜色中闪闪发光,但全身上下除了这双眼睛外,别的地方都在黑暗中。

这人竟是黑蜘蛛。

他平时那般灵动的目光,此刻竟似蒙着一层迷惘\ldots 一片惆怅,他就这样痴痴的瞧着,静静的伏在星光下,也不管露水湿透他衣裳。

小鱼儿突然噗哧一笑,道:如此星辰如此夜,为谁风露立中宵?

话声未了,黑蜘蛛已到了他面前,轻叱道:谁?

小鱼儿笑道:除了我还有谁?

黑蜘蛛目光闪电般一转,终於松懈下来,道:又是你!

小鱼儿笑道:两地阻隔,不过五丈,阁下为何不一掠而去?

黑蜘蛛笑道:我\ldots\ldots 我岂是为了她来的?

他面目虽不能见,但语声已颇不自然。

小鱼儿却不说破,反而笑道:你不是为了她,是为谁?

黑蜘蛛道:自然是那姓罗的兄弟两人。

小鱼儿笑道:哦,是麽?

黑蜘蛛道:【这兄弟两人身世诡秘,行动异常,我暗中缀着他两人,已有两\ldots 三个月了,为的就是要揭破他们的秘密。

小鱼儿道:这罗九兄弟的事,值得你来管麽?

黑蜘蛛冷笑道:江湖之中,无论是黑白两道,无论善人恶人,都是这兄弟两人要害的对象,这两人竟似要挑拨得天下武林中人全都自相残杀,好让他们坐收渔利,到目前为止,已不知有多少人死在他们手上。

小鱼儿道:哦!

黑蜘蛛道:你可知道两个月前渤海帮与黄海帮的火拚?一个月前崂山帮与快刀门的恶斗?这两场流血残杀,就全都是他兄弟两人挑拨出来的小鱼儿道:既是如此,你为何还不出手?

黑蜘蛛道:一来是我拿不着他们的证据,二来他所害的那些人,也全不是好东西,三来我心想揭破他们的底细再出手?

小鱼儿道:你猜他们会是谁呢?

黑蜘蛛道:我本来疑心他们乃是「十大恶人」中之一,後来\ldots\ldots 我调查之後,才知道``十大恶人''中,并没有这两个人。

小鱼儿笑了笑,道:也许没有\ldots\ldots 但\ldots\ldots 如此说来你并非为着那位姑娘了。

黑蜘蛛默然半晌,道:也并非完全没有关系。

小鱼儿道:你可知道她是谁?

黑蜘蛛叹道:我只知道她是个可怜的女孩子,不幸落人了这恶徒的手里。

小鱼儿道:所以你要保护她?

黑蜘蛛道:天下的可怜人,我都要保护的。

小鱼儿道:既是如此,你为何不将她救出来带走?

黑蜘蛛发亮的眼睛突然黯了下来,中却大笑道:你可知道我过的是怎麽样的生活?\ldots\ldots 我终年流浪,居无定所,吃了上一顿,远不知下一顿在那里,今天晚上活过了,也不知道明天是否能活下去,我活着没有家,死也不知要死在那里?

小鱼儿道:以你的本事,你本可活得舒舒服服的,是麽?

黑蜘蛛道:但我既已选择了这种生活,就只有过下去,到现在是想改也无法改了\ldots\ldots 就算我自己不想再过这种日子,别人也不许\ldots\ldots{}

他握紧拳头,嘶声道:像这样的生活,她是万万不能过的!

小鱼儿淡淡一笑,道:只要你喜欢她,她也喜欢你,就算过再苦的日子,也是开心的。

黑蜘蛛目中射出了凄厉的光,惨笑道:谁说我喜欢她!像我这样的人,不配喜欢任何人!也不能\ldots\ldots{}

小鱼儿叹道:我本来以为你连血都是冷的,但现在\ldots\ldots 现在我才知道你其实是个多情的人!

黑蜘蛛霍然站了起来,叱道:你小小年纪,懂得什麽,不准再说了。

小鱼儿笑道:别人说出了心事,也不必这麽凶呀!

黑蜘蛛了他半晌,突然大笑起来,拉起他的手,道:我近来又结交了个朋友,今天买了两壶的酒,烧了一锅好肉,我请你也去吃他一顿如何?

小鱼儿笑道:好,能做你朋友的人,想必也有趣的很。

两人急掠了一阵,小鱼儿始终跟在黑蜘蛛身後。

黑蜘蛛回首笑道:近来你功夫倒精进的很。

小鱼儿笑道:好说好说。

黑蜘蛛道:我交的另一个朋友,也文武全才,样样精通,你瞧见他必定也是欢喜的。

小鱼儿道:哦!他叫什麽名字。

黑蜘蛛笑道:有才能的人,也并非一定全都有名!他姓古名叫月言,虽是无名之辈,但却比那些成名人物强胜何止万倍。

说话之间,已掠出城,只见前面一片树林,隐隐火光闪动,走到近前,便可瞧见一个荒废的祠堂。

火光,便是自荒祠中出来的。

到了这里,已可嗅着一阵阵扑鼻的肉香。

小鱼儿道:看来你那朋友非但文武全才,而且还是个好厨子。

黑蜘蛛道:江湖中的浪子,除了偶而大吃一顿之外,还有什麽别的享受?

两人一掠入林,只见荒祠中旺旺的生着一堆火,火上吊着个大铁锅,锅里肉香正浓,锅旁碗筷已备,碗里也倒满了酒,但却瞧不见人。

黑蜘蛛四下瞧了瞧,高声唤道:古老弟\ldots\ldots 古老弟,我又为你带来个朋友,快来见见。

小鱼儿笑道:【来你这好做人大哥的脾气,还是改不了。

只听黑蜘蛛唤了一阵,四下却无回应,他又出去找了一圈,也找不着人,索性坐了下来,笑道:我这古老弟屁股是尖的,永远坐不住,此刻也不知野到何处去了,咱们也不必客气,先吃了再说吧。

小鱼儿早已举起筷子,笑道:正合我意。

但他只吃了一块肉,就放了筷子,嘴也不动了,竟似还未将那块肉下去,那边黑蜘蛛却早已七\ldots 八块下了肚。

吃到第十来块时,就用一大嘴酒将嘴里的肉冲下肚子,这才抬头瞧着小鱼儿,裂嘴笑道:这肉又鲜又嫩,滋味可真不错,你为何不加紧动筷子?

小鱼儿却将嘴里的肉吐在地上,道:【这肉吃不得。】

黑蜘蛛脸色一沉,道:为何吃不得?这肉可不是偷来的。

小鱼儿突然一笑,道:你可知道这是什麽肉吗?

黑蜘蛛惊呼一声,刚吃进去的一块肉立刻吐了出来,失声道:你说什麽?

小鱼儿道:老实告诉你,我是在「恶人谷」长大的,这肉若不是从刚死的人身上割下来的,我就吃下我的鼻子。

他等着来瞧黑蜘蛛将吃进去的肉呕出来,那知黑蜘蛛反而大笑道:如此说来,煮这肉的莫非是李大嘴麽?

小鱼儿道:也许就是他。

黑蜘蛛道:嗯,不错,古月言这\ldots\ldots「古月言」岂非就是「胡说」,他早已告诉我他是「胡说」我居然到现在才想起来。

小鱼儿道:你不想吐!

黑蜘蛛笑道:既已吃下去,吐也无用了。

小鱼儿道:你还笑得出?

黑蜘蛛大笑道:能和李大嘴这种人交交朋友,岂非是件有趣的事,无论他是好是坏,总算是个角色,江湖中像这样的角色可不多。

小鱼儿心里不禁暗暗赞美!这人倒洒脱得很,绝不会装腔作势,教人恶心。中道:但这位「胡说先生」却也并非一定是李大嘴。

黑蜘蛛道:不是李大嘴是谁?

小鱼儿道:我还知道一个人,他装作李大嘴,也许正是要你吃人肉,然後再吐得瞒地都是,只要你上了当,他就开心\ldots\ldots{}

说到这里,语声突然顿住,低声道:也许他还不止要你吐,也许他还另有阴谋。

黑蜘蛛刷地将面具拉了下来,冷冷道:外面的朋友!既然来了,为何还不进来?

小鱼儿的耳朵虽灵,黑蜘蛛的耳朵也不错!话声未了,荒祠外已有一条人影飞掠进来。

闪动的火光中,只见这人窈窕的身子,穿着件比火还红的衣裳,发光的眼睛里,也充满了怒火。这人竟是小仙女。

三更半夜,小仙女竟会跑到这荒祠来,小鱼儿虽未免吃了一惊,但却仍然不动声色,坐在那里。

黑蜘蛛显然也未想到闯进来的会是个年轻的美女,也惊得愣住了,小仙女更未将这两人瞧在眼里。

她掌中剑一挥。竟以那纤细的剑尖挑起了沉重的铁锅,将锅里的肉全都泼在地上只见金光一闪肉锅里竟有只女子用的金钗。

小仙女立刻尖声叫了起来,门外又有一人跃入,却是顾人玉,小仙女扑在他身上,嘶声道:宛儿的金钗\ldots\ldots 宛儿的金钗果然在锅里。

顾人玉一只大眼睛狠狠的瞪着小鱼儿,厉声道:你说!这锅里是什麽?

小鱼儿倒真未见过这大姑娘般的少年如此凶狠,知道他必定动了真怒,也知道锅里煮的这人必定和他们有些关系。

但小鱼儿却想不通他们怎会寻到这里来的,又怎会知道肉锅里有只金钗,他心中生疑,中却笑道:你说锅里的是什麽?

顾人玉脸涨得通红,却说不出话来。

只听一人缓缓道:世上肉食众多,两人为何偏嗜人肉,同类相食,两位难道连畜牲都不如麽?

这人虽在骂人,但嘴里却绝不吐半个脏字,而且语气也是平平和和,竟像是与人闲话家常似的。

随着话声,两个人缓缓走了过来,目中虽有怒气,神情也从容,正是那南宫柳与秦剑。

小鱼儿还是笑嘻嘻道:你说我们在吃人,但你们又怎会知道的,莫非是有人告密?

秦剑还未答话,小仙女已扑了过来,跺脚骂道:自然有人要告密,你们做出这种天理不容的事,谁能看得过去。

南宫柳缓缓道:像宛儿那般聪明可爱的女子,男子正当万般珍惜才是,两位却将之煮而食之,岂非焚琴煮鹤大煞风景。

小仙女忍不住大喝道:这种人你还和他们多说什麽\ldots\ldots{}

南宫柳还是缓缓道:事已至此,两泣还有什麽话说。

黑蜘蛛霍然长身而起,厉声道:在下还有话说\ldots\ldots{}

秦剑目光一闪道:阁下莫非就是江湖传言中的黑蜘蛛。

黑蜘蛛道:正是!

秦剑皱眉道:看来江湖传言,终不可信,不想黑蜘蛛竟是你这样的人物。

黑蜘蛛大声道:江湖传言虽不可信,密告之言更不可听,我且问你,若非亲手煮肉的人,又怎会知道这金钗在锅里秦剑\ldots 南宫柳对望了一眼,南宫柳缓缓道:阁下的意思,莫非是说此事乃是别人故意做来嫁祸於你的?

黑蜘蛛道:自是如此。

南宫柳缓缓点了点头,道:这话也道理。

小仙女跺脚道:二哥,你要放过他们,我可不能放过他们,这难道不可能是别人在暗中瞧见他们杀人煮肉,而来告密的。

南宫柳道:那自然也有可能。

小仙女大声道:宛儿既然可能是被他们杀来吃的,九妹自然也\ldots\ldots 也\ldots\ldots 她语声突然哽咽,竟再也说不下去。

秦剑目光灼灼的瞪着小鱼儿与黑蜘蛛,沉声道:此事虽有可疑,但两位若不能拿出证据证明无辜,今日只怕得请两位随我等回去了。

黑蜘姝冷笑道:阁下说话倒客气的很,叫我随阁下回去也无妨,只是阁下也要拿出证据来,凭什麽要带我回去。

小仙女厉喝道:这金钗难道还不是证据!你还想赖?

黑蜘蛛眼睛一瞪,还未说话,那知小鱼儿竟突然嘻嘻笑道:我几时赖过。

小仙女一剑已待刺出,闻言倒不禁愣了愣,道:你承认了?

小鱼儿向小仙女笑嘻嘻道:你说的那九妹,可是位眼睛大大、睑色苍白,约莫十八、九岁,平日喜欢穿淡绿衣衫的姑娘?

小仙女颤声道:你\ldots\ldots 你\ldots\ldots 你将她怎麽样了?

小鱼儿大笑道:我己将她怎样,这还用说麽?

黑蜘蛛大骇道:这小子疯了,满嘴胡说八道。

小鱼儿笑道:这又有什麽大不了的事,你怕什麽?

南宫柳与秦剑就算再沉得住气,此刻面上也不禁变了颜色。

小仙女跳起脚道:你听,你听\ldots\ldots 他自己都承认了!

她又哭又叫,还未忘了出手,刷的一剑,毒蛇般刺出,那边顾人玉更是眼睛都红了,狂吼一声,击出了三拳。

这三拳一剑,自然都是向小鱼儿致命处下的手,剑如闪电,拳似雷霆,左右夹击间不容发?

\hypertarget{ux7b2cux4e94ux5341ux4e09ux7ae0-ux683dux8d43ux5ac1ux7978}{%
\chapter{第五十三章
栽赃嫁祸}\label{ux7b2cux4e94ux5341ux4e09ux7ae0-ux683dux8d43ux5ac1ux7978}}

若换了两年前,小鱼儿不死在拳下,也要死在剑下,但现在的小鱼儿,却已非昔日吴下阿蒙。

只见仙左手一分,右手竟沿着小仙女的剑脊轻轻一抹,小仙女只觉跟前一花,掌中剑被一股大力吸引,本是刺向小鱼儿的一剑,此刻竟向顾人玉刺了过去,顾人玉大骇变招,嗤的,衣袖已被划破。

这一招普普通通的``移花接木'',到了小鱼儿手中,竟已化腐朽为神奇,看来竟已和移花宫威震天下的``移花接玉''有异曲同工之妙,这只因武功进入某一阶段後,便有些地方大同小异。

但南宫与秦剑一时却瞧不出其中奥妙,耸然失声道:你可是移花宫门下?

小鱼儿却不回答,反而大笑闪到黑蜘姝身後,道:我虽也吃了些肉,但主谋的却不是我,你们怎地专来找我?

顾人玉与小仙女见他明明已占先机,却不乘胜追击,反而避开了,两人急怒攻心,也不问情由,举剑又攻了上去。

这一次两人招式更毒,出手也更加小心,但首当其冲的却已非小鱼儿,而是黑蜘蛛了。

黑蜘蛛又惊又恼,此刻情况,又怎容他解释。

刹那间只见剑光闪动,拳影翻飞,小仙女与顾人玉已攻出十馀招,黑蜘蛛也还了三掌。

在小仙女快速的剑法、顾人玉雄浑的拳势下,黑蜘蛛怎能分心,简直连开都无法开。

小鱼儿却在他身後笑道:对了,这样就对了,和他们打,怕什麽!

黑蜘蛛气得连连怪叫,一心想将小鱼儿摆脱,但小鱼儿却像影子似的黏在他身後,还不时拍手笑道:好!这一剑果然了得\ldots\ldots 嗯,顾家神拳果然也不错,黑蜘蛛呀黑蜘蛛,我瞧你打不过他们的了!

小仙女与顾人玉方才急怒之下,心神大乱,所以才会被小鱼儿一出手就占得了先机。

而数十招後,两人心也定了,手也稳了,顾人玉拳势虽沉猛,出手还未免嫩些,小仙女终日找人打架与人交手的经验,却是比谁都老到,一柄剑东挑西刺,又快又毒,非但自己抢攻,而且也将顾人玉拳法中的疵漏全部补了过来,而顾人玉扎扎实实的招式,正好弥补了她剑法中沉猛之不足。两人俱是武林正宗,不用事先预习,配合得恰到好处。

黑蜘蛛声名虽着,却非以功力见长,此刻遇着他两人一快一慢\ldots 一刚一柔,这种天生的搭档,渐渐已有些应付不了。

何况还有小鱼儿在他身後,明是帮忙,暗中捣蛋。

南宫柳袖手一旁,微微颔首道:人玉果然是个天生练武的胚子。

秦剑道:但菁妹终是比他高出一筹。

南宫柳道:这你就看错了,人玉此刻出手虽嫩些,但那只是因为他家教太严,不敢惹事,根本没有交手的机会,若让他在江湖中多闯闯,不出三五年,他的名声必定要远远超过菁妹之上。

秦剑道:二哥果然法眼无双,难怪江湖中人一经南宫公子题名之後,立刻身价百倍。

南宫柳道:今日你我要留意的倒非黑蜘蛛,而是这面色蜡黄的少年,此人行态诡异,做事也不循常规,若我瞧得不差,他必定是一个成名人物易容改扮的。

这南宫公子武功是高是低,虽还不知,但就凭这份眼力,当真已不愧是雄踞江南百馀年之武林世家的传人。

说话间,那边强弱便已分明。

以黑蜘姝身法之诡异灵动,顾人玉与小仙女本难占得上风,但小鱼儿始终黏在黑蜘蛛身後,黑蜘蛛就总觉得後面像是堕着个秤锤似的,身形变化之间,自然要大受影响,这时已屡遇险招。

小鱼儿故意叹气道:不好不好,堂堂的黑蜘蛛,今日看来竟要败在两个小娃儿手上了。

其实小仙女和顾人玉也是江湖中的成名人物,并非小娃儿,小鱼儿这样说,只不过要故意激怒黑蜘姝。

黑蜘蛛脾气刚烈,明知如此,还是被他激的怒吼道:你这疯子,你倒底要怎样?

小鱼儿悄声道:打不过,难道不会逃吗?

黑蜘蛛更是暴跳如雷,道:放屁!我老黑岂是这种人!

小鱼儿道:黑蜘蛛享名天下,本就是以身法之诡\ldots 飘忽见长,今日你偏偏舍己之长,与人交手,岂非是个呆子。

黑蜘蛛嘴里虽仍骂不绝口,心里已觉得他说的有道理,只因他此刻一分心说话,肋上已险些中了一剑。

小鱼儿悠悠道:今日你自己若能全身而退,也能带我一齐走,江湖中人知道了,非但不会耻笑於你,还会佩服得很。

黑蜘蛛跺了跺脚,道:好!

他好字方出,小鱼儿已自他身後冲了出来,``断玉分金'',双掌左右斜斜分击而出。

顾人玉与小仙女骤出不意,竟被这一招逼得後退两步。

就在这时,黑蜘蛛袖中已有一线银丝飞出,直穿出门,搭上祠外的一株古柏之上,他人也跟着飞了出去。

小鱼儿早已拉住他衣角,跟着飞出,他身形轻若飞絮,虽藉了黑蜘蛛携带之力,黑蜘蛛却不觉负担。

只见他身形有如被线拉着的纸鸢似的,飘上了古柏,双足一点,人又从枯树上飞出,跃上第二株柏树,那根银丝也跟着飞出,搭上了更前面第三株柏,黑蜘蛛身子在第叁树上一点,跃上第四株,银丝又搭在第五树上\ldots\ldots{}

等到秦剑等人追出时,两人身形已在数十丈外,一闪後便在黑暗中消失无影,唯有语声远远传来,道:你们若不服,明夜叁更,不妨再来这里!

黑蜘蛛身形不停,只掠到城垛下,才在黑暗中歇住。

小鱼儿附掌道:好个黑蜘蛛,果然是来去如电,倏忽千里,这一手银丝飞蛛的轻功,果然是独步江湖\ldots 天下无双!

黑蜘蛛道:哼,你拍我的马屁,也没有用的。

小鱼儿大笑道:我知道你必定一肚子闷气,不过想让你消消气而已。

黑蜘蛛道:我且问你,明明不是你做的事,你为何要揽在自己头上,还拉上了我,而你躲後面,让我来背黑锅。

他越说越火,大声道:这也不用说它,最可恨的,你明明可以光明正大的动手,却又偏偏要逃,害得我也陪着你丢人,这究竟是为了什麽?

小鱼儿笑嘻嘻道:你还不明白麽?我这自然是要害你。

黑蜘蛛愣了愣,道:害我?

小鱼儿道:咱们这一逃,我可以一走了之,但你黑蜘蛛有名有姓,日後传说出去,说你黑蜘蛛也和李大嘴一样吃人,你还能混麽?

黑蜘蛛大怒道:那你为什麽要害我?

小鱼儿嘻嘻笑道:这只因我要把你拖下水,你才会为我出力,但你也莫要气恼,我瞧你不错才这样害你的,有些人想求我害他,我还没功夫哩。

黑蜘蛛厉声道:你害了我,我该捏死你才是,怎肯替你出力。

小鱼儿笑道:若是换了别人,我害了他,他自然要找我算帐,但你黑蜘蛛可和别人大不相同?这一点我清楚得很。

黑蜘蛛瞪了他半晌,突然放手大笑道:好,你这小子,倒真是知道老黑的脾气!我老黑遇着这种怪事,的确是明知上当,也不肯放手的。

小鱼儿笑道":若非如此,黑蜘蛛就不是黑蜘蛛了。

黑蜘蛛道:你如此做法,除了拖我下水外,难道没有别的用意?

小鱼儿道:自然有的,想那南宫柳与秦剑,眼高於顶,自命不凡,我平日时若想约他们出来,他肯麽但现在我要他明夜三更来,他绝不会迟到半刻。

黑蜘蛛道:好,现在我既已被你拖下了水,他们也被你抓了尾巴,这出戏究竟该怎样唱下去,你说吧。

小鱼儿道:那泣「胡说」先生偷偷将人宰了,要你来吃,却又偷偷去密告别人来抓你,这样的手段叫做什麽?

黑蜘蛛恨恨道:这自然就叫做嫁祸栽赃。

小鱼儿道:这种专门嫁祸栽赃的害人精,你说该如何对付他?

黑蜘蛛咬牙道:我若再见着他时,不一把捏死他才怪。

小鱼儿道:你可知道这样的害人精,除了胡说先生之外,还有不少,而且他们所作所为,委实此胡说先生还要可恨,却又该如何对付他们!.黑蜘蛛道:捉来一个个捏死就是了。

小鱼儿笑道:捏死他们还算太便宜了,何况,你若想捏死他们还不容易。

黑蜘蛛道:你说的究竟是什麽人?

小鱼儿一字字道:江别鹤

黑蜘蛛几乎跳了起来,失声道:江南大侠怎会做这样的事?

小鱼儿凝目瞧着他,道:你信不过我?

黑蜘蛛也瞧着小鱼儿,道:你这人藏头露尾\ldots 鬼鬼崇崇,做起事来更是古灵精怪,花样百出,天下又有谁能信得过你!

他叹了气,缓缓接道:我相信你,只因你虽是坏小子,却非伪君子!

小鱼儿叹道:不错,最可恨的人就是伪君子,那江别鹤就是其中最可恨的一个。

黑蜘蛛道:你想如何对付他?

小鱼儿眼睛发亮,道:以其人之道,还治其人之身。他们会栽赃嫁祸别人,我就要栽赃嫁祸他们,这就叫以牙还牙。

黑蜘蛛道:如何还法,你且说来听听。

小鱼儿眼睛盯着他,道:你可知道阁楼上的那位姑娘是谁?

黑蜘蛛突然扭转头,道:我早就说过,不知道。

小鱼儿缓缓道:我现在告诉你,她就是慕容家的九姑娘!

黑蜘蛛眼睛立刻直了,失声道:她就是慕容九小鱼儿道:不错,如今南宫柳、秦剑,小仙女都在急着找她,他们若发现有人将她藏了起来,少不得要找那人大干一场。

黑蜘蛛的眼睛也发了亮,道:所以,你就想将这件事栽在江别鹤身上。

小鱼儿附掌大笑道:我正是也想叫他尝尝被人栽赃的滋味。

黑蜘蛛道:但那江别鹤老谋深算,又怎会上你的当?

小鱼儿笑道:那江别鹤纵然狡如狐狸,只要你帮忙,我也有法子要他上当?

他一跃而起,拉起黑蜘蛛,道:时候已不多,咱门快去办事吧。

两人飞掠入城。

一路上,黑蜘蛛不住喃喃自语道:我到现在为止还是不懂,那胡说宰食了慕容家的人又害了我,却对他自己有何好处。】

他猜想那宛儿必定与慕容家有关,八成就是慕容姑娘陪嫁的贴身侍女。

小鱼儿笑道:你说的那位胡说先生,并非李大嘴,而是白开心,他有个外号叫「损人不利已」,只要别人上当受罪,就是他平生快事。

黑蜘蛛失声道:世上那有这样的人?

小鱼儿道:你说没有,他偏偏就有,他们知慕容家的姑爷来找慕容九,所以就将那宛儿偷来宰了,好让慕容家的那些姑爷认为慕容九也已被人家吃下肚,所以他们才找不着,他们伤心难过,白开心就开心了。

黑蜘蛛叹道:世上既有白开心这样的人,又偏偏有你这样的人,你们两人害来害去,倒楣的只是我老黑而已。

小鱼儿道:今夜若不是有我,你更惨了,当时人赃俱获,就算有一百张嘴,也休想能辩说得清。

黑蜘蛛道:但无论如何你总不该承认\ldots\ldots{}

小鱼儿笑道:我又几时承认了,我几时说过慕容九已被我吃下肚里?我只不过\ldots\ldots「我已将她怎样,还用说麽?」「也没什麽大不了,你怕什麽!」\ldots\ldots{}

黑蜘蛛想了想,不禁失笑道:不错,当时你虽好像说了,其实却等於没有说\ldots\ldots{}

小鱼儿道:其中的巧妙就在这里。

说话间,他竟将黑蜘蛛又带回了那阁楼外。

此刻四下灯火俱寂,只有那阁楼里灯光还亮着,慕容九伏在桌上,想是因为想得出神,不觉睡着了。

小鱼儿道:这位姑娘最听你的话,你叫她带着刀,她就带着刀,你叫她杀人,她就杀人,现在,我只要你叫她写张条子。

黑蜘蛛奇道:此时此刻,突然写起什麽条子来了?

小鱼儿道:你叫她写:「若要赎我的性命,请带白银八十万两,至他们所约之处,千万勿误,否则便是他人俎上之肉了」。

黑蜘蛛骇然道:八十万两!

小鱼儿道:八十万两数目虽不少,但以南宫柳与秦剑的身家,却也算不得多,别人一日之间筹不出来,他们想必有法子的。

他一笑接道:何况,这字条又的确是慕容九自己的笔迹\ldots\ldots 其中问题是,你必须对他们说八十万两,全要白银,金子珠宝郡不行。

黑蜘蛛道:我对他们去说?.

小鱼儿笑道:自然要你去对他们说,这字条自然也要你送去\ldots\ldots 黑蜘蛛来去无踪,倏忽千里,送这样的信,世上还有比你更好的人麽?

黑蜘蛛默然半晌,叹了气,道:好吧\ldots\ldots 我只是不懂,为何定要白银?

小鱼儿道:这其中自然又有奥妙,你到时就会懂的。送信之後,你等着瞧热闹就是。

黑蜘蛛道:到时你难道真的自己去接银子?

小鱼儿道:到时去接银子的,已是我送去的替死鬼了。

黑蜘蛛道:那麽\ldots\ldots 秦剑与南宫柳若瞧见不是你而是别人,岂非又难免怀疑。

小鱼儿笑道:秦剑与南宫柳又不知道我是谁\ldots\ldots 他们见到我这张蜡黄的睑,又瞧见那手「移花接木」,还以为我是「移花宫」门下改扮的哩,而此刻那真的「移花宫」弟子却正是和江别鹤在一起。

黑蜘姝想了又想,叹道:原来你每一举动都有用意,像你这样的人,世上若是再多几个,别人的日子如何能过得下去。

小鱼儿大笑道:你放心,像我这样的人,天下是再也不会有第二个了。

凌晨时,那庆馀堂的掌柜糊里糊涂的被小鱼儿从床上拉了起来,送了封信到段叁姑娘处。

天亮时,小鱼儿已回复成药伙计的打,倒在庆馀堂里他原来那张小床上,睡了一大觉。

然後,段三姑娘就来了。

这一次,她已没有在窗子外面叫,直接就闯了进来,从床上拖起小鱼儿,又是欢喜,又是埋怨,跺脚道:这两天,你到那里去了,知不知道人家多着急。

小鱼儿揉着眼睛,笑道:你若真的为我着急,就该帮我个忙。

三姑娘幽幽道:你要我做什麽,我几时不肯答应你。

小鱼儿道:但这件事,你绝不能向第叁人露半个字。

三姑娘垂下头,道:你难道还信不过我?

小鱼儿展颜笑道:好,我先问你,这两天你可瞧见了那江玉郎麽?

三姑娘道:没瞧见。

小鱼儿眼睛瞪着她,道:你再想想,江别鹤周围的人有没有一个可能是江玉郎改扮的。

叁姑娘果然想了想,断然道:没有,绝无可能,这两天江玉郎绝不在这里。

小鱼儿松了气,道:这就是了,女子的感觉虽然有些莫名其妙,但有时却是对的,你既然如此肯定,江玉想必不会在这里了。

三姑娘幽幽道:你叫我来,就是要问他麽?

小鱼儿笑道:"这只因他和你有很大的关系。

三姑娘嗔道:你莫要胡说,我和他有什麽关系?

小鱼儿沉声道:你可知道,你家的镖银,就是他动手劫的。

三姑娘失声道:真的?

小鱼儿道:他这两天突然走了,一来是想避开我,二来就是要去将那批镖银换个地方藏起来,只因他以为我知道的密比我实在知道的多。

三姑娘眨着眼睛道:你究竟是谁?他为什麽这麽怕你?

小鱼儿笑道:严格说来,他到现在为止也还不知道我是谁?

三姑娘默然半晌,轻轻道:我不管你是谁,我都\ldots\ldots{}

小鱼儿赶紧打断她的话,道:只要我猜的不错,只要他不在这里,我的计划就能成功\ldots\ldots 你必须替我留意着,他若万一回来了,你就得赶紧告诉我。

三姑娘道:你究竟有什麽计划?为何定要他不在这里,你的计划才能成功。

小鱼儿拉起她的手,柔声道:这些事你以後总会知道的,但现在却请你莫要问我。

世上若有什麽事能令女子闭起嘴,那就是她心爱的男人温柔的话了。三姑娘果然闭起了嘴,不再问下去。

她只是垂头,悠悠道:你\ldots\ldots 没有别的话对我说?

小鱼儿道:今天晚上,起更时,你在你家後园的小门外等我\ldots\ldots{}

三姑娘的眼睛立刻闪起了喜悦的光,颤声道:今夜\ldots\ldots 後园小门?

小鱼儿道:不错,你千万莫要忘了,千万要准时到那里。

三姑娘娇笑道:我绝不会忘,就算天塌下来,我也会准时到的.

她娇笑着转身而去,满怀着绮丽而浪漫的憧憬。

小鱼儿在街上东游西逛,走过许多饭酒楼,他也不进去,却在东城外找着了家又脏又破的小面馆。

这小面馆居然也有个很漂亮的名字,叫:思乡馆。

小鱼儿走进去吃了一大碗热汤面、四个荷包蛋,却叫店里那看来已有三年没洗澡的山东老乡去买了些笔墨,七,八十张纸。

他用饭碗那麽大的字,在纸上写下了:开心的朋友,今夜戌时,有个姓李的在东城外的「思乡馆」等着你,你想不来也不行的。同样的句子,他竟一连写了七\ldots 八十张,又雇了两个泥腿汉子,叫他们去贴在城里大街小巷的显眼处。

那山东老乡实在瞧得奇怪,忍不住道:这是在干啥?俺实在不懂。小鱼儿笑道:该懂的自然会懂,不该懂的自然不懂。.那山东老乡摸着头皮道:谁是该懂的?

小鱼儿却已笑嘻嘻走了,竟又到估衣铺去买了身半新旧的黑布衣服,到杂货去买了些油墨\ldots 石膏、牛皮胶。

然後,他就寻了家半大不小的客栈,痛痛快快睡了一觉!这一觉睡醒,天已快黑了。

小鱼儿对着镜子,像是少女梳妆般在脸上抹了半天,又穿起那套衣服,在镜子前一站\ldots\ldots{}

这那里还是小鱼儿,这不活脱脱正是李大嘴麽。

小鱼儿自己也瞧得很是满意,哈哈笑道:虽然还不十分一样,但想那白开心已有二三十年未见过李大嘴,黑夜之中,想必已可混得过去。

他生得本来不矮,经过这两年来的磨炼,身子更是结实,挺起胸来,不但面貌已与李大嘴有九分相似,就算身材也和那魁伟雄壮的李大嘴差不了多少,纵是和李大嘴天天见面的人,若不十分留意,也未见就能瞧得破绽。

他将换下来的衣服卷成一条,塞在被窝里,从外面瞧进来,床上仍然像是有个人在睡觉。

然後他又用桌上的秃笔写了封信,这封信竟是写江别鹤的,他用左手歪歪斜斜的写着:江别鹤,你儿子和镖银都已落在大爷我的手里了,你若想谈条件,今夜三更,到城外的祠堂里等着吧。

他将这封信紧紧封起,又在信封上写着:江别鹤亲拆,别人看不得。小鱼儿将信收在怀,喃喃笑道:江玉郎不在城里,八成是去收藏那镖银去了,只要他今天晚上不回来,江别鹤就算是狐狸,瞧见这封信也得中计,他心里就算不十分相信,到了三更时也必定忍不住要去瞧瞧的。他得意的笑着,从窗溜了出去。

小鱼儿走到「思乡馆」时,暮色已很深了。

这时虽正是吃饭的时候,「思乡馆」里却没什麽人,就连那山东老乡都已瞧不见,只有一个客人正坐着喝酒。

这人穿着件新缎子衣服,戴的帽子上还有粒珍珠,穿着锥像个富商士绅,神态却还是个地痞无赖,竟不肯好好坐在那里,却蹲在凳子上喝酒,一双贼眼不住转来转去,又像是随时提防着别人来抓的小偷。

小鱼儿大步走了进去,哈哈笑道:好小子,你果然来了,许多年不见,你这王八旦倒还未忘记有个姓李的朋友,来得倒准时。】他从小和李大嘴长大,要学李大嘴说话的神情腔调,自然学得唯妙唯尚,活脱脱是一个模子里铸出来的。

那人却板着脸,瞪着眼道:你是谁,咱不认得你。小鱼儿笑道:你想瞒我,你虽然穿得像是个人,但那副猴头猴脑的贼相还是改不了的。那人果然大笑起来,道:你这吃人不吐骨头的混球蛋,多少年不见,你对老子说话,难道就不能稍为客气些麽?小鱼儿在他对面坐了下来,桌子上有两副杯筷,却只有一碗红烧肉,小鱼儿皱了皱眉头道:你这穷贼实在愈来愈穷了,快叫那山东老乡来,待老哥哥我叫你痛痛快快的吃上一顿。白开心道:他不会来的.

小鱼儿道:为什麽?他在那里?

白开心笑嘻嘻指着那只碗,道:就在这只碗里。小鱼儿神色不动,哈哈笑道:你倒会拍老子的马屁,还未忘记老子喜欢吃什麽,只是瞧那山东老乡好几年没洗澡的样子,只怕连肉都已臭了。白开心嘻嘻笑道:我早已把他从头到脚洗得乾乾净净再下锅的。他举杯敬了小鱼儿一杯酒,又倒满了一杯。

小鱼儿笑道:你这儿子倒真孝顺。

他只得挟起一块肉,但刚吃了两块,又吐了出来,瞪眼道:这是什麽鸟肉敢混充人肉?

白开心附掌大笑道:姓李的,你果然还是有两下子,这张鸟嘴竟一吃就能尝得出是不是人肉来,你也不想想,老子会杀人给你吃麽?他自然本是想用这方法试试来的人是否真的李大嘴,小鱼儿肚子里暗暗好笑,却不说破,瞪眼道:你不孝顺老子孝顺谁?那山东老乡人虽脏些,肉倒还结实,老子早已有心将他红烧来吃了,你却将他弄到那里去了?白开心道:他早已回家去了,老子已将他这家店买了下来\ldots\ldots 哈哈,他受了老子里面灌铅的假银子,居然还开心的很,以为上当的是老子。小鱼儿叹道:这家破面馆你要来鸟用也没有,你却骗苦了他,又害得老子吃不着好肉,你那``损人不利己''的贼脾气,当真是一辈子也改不了。白开心嘻嘻笑道:老子的脾气改不了,你那贼脾气又改得了麽?狗是改不了要吃屎的\ldots\ldots 你躲在狗窝里这许多年,突然又钻出来干什麽?小鱼儿眼睛一瞪,大声道:我先问你,你假借老子的名头,送了付挽联给铁无双,又假借老子的名头,将人家的小丫头炖来吃了,究竟想干什麽?白开心愣了愣,道:你全知道?

小鱼儿大笑道:你想,还有什麽事能瞒得过老子的。白开心笑道:那些人太没事干了,老子瞧得不顺眼,所以找些事他们做,炖了肉请人来吃,却又去告他一状,要他们两家都闹得人仰马翻,老子才开心\ldots\ldots 你凭良心说,老子这一手做得妙不妙。小鱼儿冷笑道:只可叹姓秦的和那南宫小儿,活到这麽大了,随随便便来个人告诉他们一件事,他们居然也相信,若换了是我,你来告状,老子就先将你扣下来,问问你别人吃人肉,你又怎会知道。白开心道:老子不会写信麽?为何定要自己去?小鱼儿道:一封无头信他们就相信了?

白开心道:他们纵不相信,好歹也得去瞧瞧。小鱼儿一拍桌子,笑道:正是这道理!我正是要你说出这句话来。白开心眼珠子转动,道:你又在打什麽鬼主意,要叫老子上当?小鱼儿笑道:你冒了老子的名,老子暂且也不罚你,只要你再写封信给那姓秦的与南宫小儿,他们既已证明了你第一封信说的不假,你第二封法,他们自然更相信了。白开心道:什麽信?

小鱼儿道:自然也是害人的信,若不是害人的信,你想来也不肯写的。白开心展颜笑道:要害人嘛,老子还马马虎虎可以答应你,却不知要害的是谁?小鱼儿道:你只要告诉他们,今夜三更,到段合肥家的後院客房里去瞧瞧,自然会瞧见令他们感到有趣的东西\ldots\ldots 但必定要在正三更,早也不行迟也不行,至於要害的是什麽人,你迟早会知道的。白开心道:老子若不肯写呢?

小鱼儿冷道:我知道你肯写的,你看可以害人的事不做,你还睡得着觉麽?何况,你若不写这封信,老子总有法子叫你\ldots\ldots 突然取出写给江别鹤的那封信,拿在手里,一掌击灭了桌上的油灯,白开心面色变了变,道:你干什麽?

\hypertarget{ux7b2cux4e94ux5341ux56dbux7ae0-ux7565ux65bdux5de7ux8ba1}{%
\chapter{第五十四章
略施巧计}\label{ux7b2cux4e94ux5341ux56dbux7ae0-ux7565ux65bdux5de7ux8ba1}}

小鱼儿悄声道:有人来抓咱们了,准备逃吧!话犹未了,窗外有刀光闪动。

只听有人喝道:姓李的、姓白的!你们作恶多端,今天再也休想跑了,出来受死吧。黑暗中人影幢幢,这思乡馆竟也被人团团围住。

白开心喃喃道:奇怪,这些人竟会知道咱们在这里\ldots\ldots{}

小鱼儿悄声道:这人满口仁义道德,必定是江别鹤。岸白开心道:嗯。

小鱼儿道:咱就从他这边冲出去。

白开心道:从武功最强的人那边冲出去?你莫非疯了?小鱼儿微微一笑,道:我自有道理。

这时外面已又喝道:你们再不答覆,咱们就冲进去了。其实这些人对``十大恶人''也颇为忌惮,一时之间,是谁也不敢冲入这黑黝黝的屋子里的。

小鱼儿霍然站起,大喝道:李大嘴来也,你们等着吧!提起张凳子往东面门外掷了出去人却已从西面窗蹿出。

这李大嘴三个字,果然有些吓人,凳子飞出来,东面一阵大乱,几刀不问青红皂白就砍了出去,全都砍到凳子上口小鱼儿蹿出窗外,也有两柄刀直劈而来,小鱼儿一声虎吼,飞起一脚将左面的一柄刀飞!

他身子却已自右面一人头上掠过,顺势一脚,在那人头上,那人登时矮了半截。

这一着鸳鸯双飞脚,本非什麽玄妙的武功,但在他手里稍加变化,却立时制住了两个高手。

要知他在那窟中所得,正是普天之下,各门各派的武功精妙所在,他融会贯通之後,无论那一派的招式到了他手里,他都可化腐朽为神奇,却教别人再也猜不出他的武功来历。

只听有人惊呼道:这姓李的果然厉害,大家要小心\ldots\ldots 话未说完,只听啪的一响,接着又是一阵大笑,说话的人想是已被白开心打歪了嘴巴。

小鱼儿一招北派鸳鸯双飞脚踢倒了两人,跟着又用招南派冲天炮,一拳将一条大汉打得飞上半空。

突见跟前剑光闪动,迅急辛辣,神定气足。

一人冷笑道:李大嘴,你武功虽不错,今日还是休想逃走。三句话功夫,已刺出八剑,剑剑俱是杀着。

小鱼儿连瞧都不必瞧,已知道是江别鹤来了,连连闪过八剑,却不还手,只是压低声音道:你想知道你儿子和镖银的下落麽?江别鹤掌中剑果然缓了一缓,失声道:你说什麽?

小鱼儿将那封信穿在江别鹤的剑尖上,道:你先瞧瞧再说。江别鹤也不知是收回剑来瞧信,还是刺出剑去伤人,稍一犹豫间,小鱼儿已自他身旁蹿了出去。

白开心也怪叫着跟着掠出。

江别鹤竟眼睁睁瞧着他两人逃走,等到别的入围过来时,小鱼儿和白开心早没了影子。

小鱼儿和白开心蹿入个暗林中,方自停下。

白开心瞧着小鱼儿冷笑道:这些人怎会知道咱们在那里?小鱼儿眨了眨眼睛,笑道:自然是有人密告的。白开心冷笑道:密告的人,只怕是你自己吧。小鱼儿道:若是我,我为何还要助你逃出来。别人又不是瞎子,难道不是见那告示上饭碗那麽大的字。白开心冷笑道:那些话,这些人又怎瞧得懂。

小鱼儿笑嘻嘻道:自然有人瞧得懂的。

白开心变色道:谁?难道咱们的老朋友也有人到了城里?小鱼儿想了想,道:我不妨告诉你,有两个人,一个叫罗九、一个叫罗三,一心想找咱们的麻烦,对咱们的事知道得清楚得很。白开心皱眉道:这两人长得是何模样?

小鱼儿道:胖胖的,高高的,两个人长得一模一样,是个双胞胎。白开心道:我只认识个瘦瘦的双胞胎,却不认得胖的。小鱼儿道:你不认得他们,他们都认得你。白开心怒道:你既早已知道他们瞧得懂那张告示,既然早已知道他们要告密,为什麽偏偏还要这样做?小鱼儿笑嘻嘻道:我正是要他们告密,正是要叫他们找人来抓咱们,这样我才能将那封信交到江别鹤手上\ldots\ldots 我若用别的法子将信交他,他未必重视,但这封信既是李大嘴亲手交他的,份量可就不同了。

白开心道:但你又怎知道江别鹤必定会来。

小鱼儿道:他自命大侠,听说有十大恶人在城里,他能不管麽?只要他来了,听到我的话後,就必定不放咱们走的。

白开心默然半晌,叹了气,道:你样样事都算得这麽准,只怕连真的李大嘴都不如你。

这次小鱼儿却不禁愣了愣,咯咯笑道:什麽真的李大嘴,老子难道是假的不成?

白开心突然大笑起来,道:你能将李大嘴的模样腔调学得这麽像,简直连我都有点佩服你,简直有些舍不得瞧着你死在我面前,只可惜你已是非死不可的了!

小鱼儿皱了皱眉,道:非死不可?

白开心怪笑道:你喝的那杯酒里,老子早已下了独门水晶断肠散,本来还可多活半个时辰,但方才那麽一折腾,只怕现在就要你的命了!

小鱼儿怒道:你这恶贼,我和你拚了?

他跳起来想扑过去,但身子才跳起,便咚的跌在地上,脸色发白,双手捂着肚子,颤声道:不好,我\ldots\ldots 我\ldots\ldots 已不行了\ldots\ldots{}

白开心手舞足蹈,格格大笑道:你如今总该知道十大恶人可不是好对付的吧。

小鱼儿嘶声道:但\ldots\ldots 但你又怎知道我\ldots\ldots 我不是李大嘴?我不信你能瞧得出。

白开心道:你将李大嘴一举一动,都学得唯妙唯肖,想必是认得他的,是麽?

小鱼儿疼得全身都抖了起来,道:是\ldots\ldots 是。

白开心道:你可听见他说起过我麽?

小鱼儿呆了呆,道:没\ldots\ldots 没有。

白开心道:这只因他与我仇深似海,他将我恨之入骨,连我的名字都不愿提起,又怎会将我当做朋友,和我在一张桌子上喝酒。

他大笑接道:.你以为「十大恶人」既然都是恶人,大家臭味相投,想必全是朋友,却不知「十大恶人」中也有互相恨得入骨的冤家对头\ldots\ldots 你千算万算,终於还是算错了一着,这一着就够要你的命了!

小鱼儿呻吟着道:原来你早已知道我不是李大嘴了,但你为什麽\ldots\ldots 为什麽\ldots\ldots{}

白开心嘻嘻笑道:老子一直在装糊涂,只是为了想瞧瞧你倒底是何居心?也想逗着你玩玩,如今老子已玩够了,你就等死吧。

小鱼儿突然惨笑道:我今日虽然死在你手上,但是你有件事\ldots\ldots{}

他身子突然一抽搐,整个人仰天躺到地上,虽然拚命想说话但嘴唇启动,却发不出声来。

白开心道:老子有什麽事,你说呀?

小鱼儿挣得满头大汗,道:你\ldots\ldots 你

他虽然用尽力气,但声音却仍小得像蚊子叫。

白开心忍不住走过去,低下头来,道:你说大声些,老子听不见。

小鱼儿突然大吼道:我说你是个大笨蛋?

吼声中,他出手如风,已点了白开心身上十来处穴道,白开心刚被吼声骇得一震,人已躺了下来。

小鱼儿一跃而起,大笑道:十大恶人纵然一个个精似鬼,但遇见了我,还是要上当的,你如今总该知道,老子不是好对付的了吧。

白开心躺在地上,眼睁睁的瞧着,他实在想不到这世上竟有比十大恶人还要诡计多端的。

小鱼儿又笑道:老子虽然拿不准那杯酒里是否有毒,但对你十大恶人,总是要提防一着的,你以为老子喝下了那杯酒,其实老子却不过将酒藏在舌头下,早已随着那块假人肉一齐吐出来了!

白开心道:我\ldots\ldots 我怎麽未瞧出?

小鱼儿笑道:这种骗人的本事,老子五岁时就学会了,老子莫说将小小一杯酒藏在嘴里,就算嘴里藏着个大鸭蛋,你也是瞧不出的。

白开心像是瞧见了鬼似的,颤声道:你\ldots\ldots 你究竟是什麽人?

小鱼儿笑道:你也知道害怕了麽!老子这样的人,原是谁都要害怕的,你若要问老子是谁乖乖替老子办完事後,老子也许会告许你。

白开心听这比鬼还厉害的人居然并无杀死自己之意,只不过要替他办事而已,不禁大喜道:是,是\ldots\ldots 小子这就立刻去写信。

小鱼儿大笑道:你如今已从老子变成小子了麽\ldots\ldots 但老子若这样就放了你这样的小子,还未免有些不放心。

他双手背在身後,早已悄悄搓了个泥团在手里,此刻突然捏着白开心的鼻子用力塞进他嘴。

白开心只觉一粒又黏又湿\ldots 还微微带着种说不出臭气的东西,从喉咙里滑下了肚,不禁大骇道:这\ldots\ldots 这是什麽小鱼儿笑道:你有你独门的水晶断肠散,我也有我独门的黑煞催命丸\ldots\ldots{}

白开心变色道:黑煞催命丸?我\ldots\ldots 我怎地从未听过这名字?

小鱼儿悠然道:你自然没有听见过这名字,这是我苦心研究多年,最近才配成的,天下无药可解,服後七个时辰之内,全身发黑发肿,再过半个时辰,便全身溃烂而死,变成一滩又黑又臭的脓水。

他信口说来,说得当真是活灵活现。

白开心满头冷汗涔涔而落,颤声道:你\ldots\ldots 你不是还要我做事麽?

小鱼儿笑道:当然,我自己是有独门解药的。

白开心道:我和你无冤无仇,求求你\ldots\ldots{}

小鱼儿眼睛一瞪,大声道:你七个时辰之内,若能将我吩咐的那件事办得妥安当当,若能令我满意,再来这里等着,我自然会救你的。

他顺手拍开了白开心的穴道。

白开心却仍软瘫在地上,似乎连站起来的力气都没有了,道:你\ldots\ldots 你不会将我忘记的吧?

小鱼儿冷冷道:时候已不多,你还不快去,只怕就来不及了。

白开心不等他话说完,已从地上跳了起来,就像是匹被人在屁股上砍了一刀的野马,风也似的远去。

小鱼儿瞧着他去远,哈哈笑道:人人害怕的十大恶人,原来也是容易上当的。

起更前,小鱼儿又回到那阁楼上。

罗九、罗叁兄弟果然都不在,慕容九正坐在地毡上,手里提着个无锡泥娃娃慢声低唱道:小宝贝,快快睡,窗外天已黑,小鸟回家去,乌鸦也休息\ldots\ldots{}

小鱼儿笑了笑,接着唱道:到天亮,出太阳,又是鸟语花香\ldots\ldots{}

慕容九顿住歌声,茫然瞧了他半晌,呐呐道:你是谁,我不认得你。

小鱼儿柔声笑道:你忘了麽?我就是昨天教你如何去打跑心里那恶魔的人啊!

慕容九道:呀原来是你,你模样看来怎地有些变了?小鱼儿故意悄声道:我为了怕恶魔来找我,所以故意扮成这样子,好教他找不着,你可千万不要对别人说。慕容九连连点头道:我知道,我懂得,那恶魔厉害得很,千万不能被他找着小鱼儿笑道:我知道你会懂的,你是个很聪明的女孩子。

慕容九嫣然一笑,她忧郁的脸上出现笑容,就像阴沉的天气里突然出现了阳光,鲜艳的花朵也在这一瞬间开放。

小鱼儿瞧了两眼,心里竟似有些异样的感觉,他立刻知道不能再瞧下去了,赶紧道:现在,我要带你去一个地方,不久你就可以瞧见比我本事还大,能帮你赶走那恶魔的人了。

也不知怎的,慕容九竟对他顺从的很,立刻就站了起来,走了两步,眨了眨眼睛忽然又道:那麽\ldots\ldots 你昵?

小鱼儿苦笑了笑,道:以後,你只怕就瞧不见我了。

慕容九立刻停下脚步,道:若是以後瞧不见你,我就不走了。

小鱼儿愣了愣,心里也不知道什麽滋味,赶紧大声道:你心里那恶魔被赶走之後,你自己也不会愿意再见着我的,那时,会有许多别的人天天陪着你。

慕容九想了想,道:那麽,就让这恶魔待在我心里吧。

小鱼儿鼻亍竟像是有些酸了起来,大声笑道:傻孩子,你难道想一辈子这样吗?

慕容九凝目瞧着他,咬着嘴唇道:这样其实也没什麽不好,何况,只要你天天来陪着我,你也可以将那恶魔赶走的,是麽?

小鱼儿揉了揉鼻子,板着脸道:你这样不听话,我怎会来陪你。.慕容九垂下了头,幽幽道:你一定要我去,我就去,但是你\ldots\ldots{}

小鱼儿终於忍不住叹了口气,道:只要你记得今天的话,我以後还是会去瞧你的\ldots\ldots{}

小鱼儿替慕容九披起了件长大的披风,走到段宅後园的小门外时,段叁姑娘早已在那里等着了。

她的眼睛闪着光,一颗心跳个不停,身子虽然正冷得发抖,但一张脸却在发烧,烧得连耳根都红了。

她远远就瞧见小鱼儿了,狂喜着迎了上去,到了小鱼儿面前才发现小鱼儿身後竟还有个人。

她一颗心立刻沉了下去,咬着嘴唇道:你\ldots\ldots 你不是一个人来的。

小鱼儿也不知究竟是真的不懂她心里的感觉,还是装着不懂,扬起了眉毛,瞧着她嘻嘻一笑道:我本来就没有说要一个人来呀!

三姑娘这时才瞧见他的脸,失声道:你\ldots\ldots 你是什麽人?

小鱼儿笑道:你方才能认出了我,现在怎地又不认得了。三姑娘已听出了他的声音,但还怀疑着,呐呐道:方才我只是感觉\ldots\ldots 感觉到是你来了,但你的脸\ldots\ldots{}

小鱼儿压住声音,道:我有件密的事要做,所以不能不扮成这样子,你可千万莫要告诉别人,这件事只有你一个人知道。

他虽然根本没有说出这件事是什麽事,但他知道少女们一听到只有自己一个人知道她心爱男人的秘密时,别的事就再也不会追究了。

三姑娘果然又愉快了起来十小鱼儿毕竟对她不错,否则又怎会将这没有人知道的秘密告诉她。

她立刻也压低声音,道:你放心,绝不会告诉别人的小鱼儿皱起了眉头,道:但这件事,我还需要人帮忙.

叁姑娘急忙问道:我能帮忙麽?

小鱼儿道:本来可以找别人的,但是你\ldots\ldots 你若肯帮忙,那当然再好也没有。

叁姑娘更开心了,道:我早就说过,无论你要我做什麽事,我都答应你。

她心爱的男人不找别人帮忙,只找她,可见对她确实和别人不同,她简直开心得要死。

小鱼儿瞧她的神色,知道事情已绝不会有问题了,这才沉声道:其实,这件事也并没有什麽困难,只要你将这人带到你屋里,等到三更时,才悄悄将她放到江别鹤屋里,找个地方藏起来。

叁姑娘道:这容易得很,我一定能做到。

小鱼儿道:但你却要记住两件事,第一,你千万不能让任何人瞧见她,第二,你必须要在准叁更时才将她藏好,千万不能太早,更不能迟。

叁姑娘笑道:你放心,我绝不会误事的。

她这时才留意到慕容九?

\hypertarget{ux7b2cux4e94ux5341ux4e94ux7ae0-ux5de7ux5999ux5b89ux6392}{%
\chapter{第五十五章
巧妙安排}\label{ux7b2cux4e94ux5341ux4e94ux7ae0-ux5de7ux5999ux5b89ux6392}}

慕容九全身都笼罩在黑色的披风里,连头也被盖着,三姑娘也瞧不出她长得是何模样,迟疑了半晌,终於忍不住问道:这人是谁?

小鱼儿含着道:她和我做的那件事关系很大,你以後就知道的。

他将慕容九推到三姑娘面前,道:你们两人赶去吧慕容九回头瞧着他,似乎还想说什麽,但小鱼儿已赶紧走了,叁姑娘瞧着他们的神情,面上不禁露出了怀疑之色,但终於只是叹了口气,道:喂,你随我来吧。

小鱼儿早早便赶到那祠堂,在四面巡视了一遍,他所约的人都还没来,他在四面略为布置了一下,便寻了个最佳地势,藏了起来。

然後,他将这事从头到尾再想了一遍。

秦剑和南宫柳接到慕容九的字条後,必定会来的。

江别鹤瞧了那封信,也是非来不可。

秦剑那批人身带着八十万两现银,江别鹤那一批人却要来寻镖银,这两批人在这里碰面後,还会没有热闹瞧麽?

黑夜之中,两边人心里郡焦急得很,一言不合,不打起来有鬼。

就算他们还未打起来,但等到三姑娘将慕容九送到江别鹤的屋子,慕容九的人听了白开心的密告,去找出她来之後,慕容九的人还会放过江别鹤麽?江别鹤纵然厉害,慕容家可也不是好惹的。

小鱼儿这个计划,又岂是一举两得而已。

第一\ldots 他以其人之道,还治其人之身,让江别鹤也尝尝被人栽赃的苦头,他心里总算能出了恶气。

第二、南宫柳\ldots 小仙女这些人昨夜冤枉了他,他也要他们吃些苦头,他算准他们接到白开心的密告後,必定要分两批人到段宅的後园去瞧瞧,但这祠堂也是不能不来的,来的人最多不过是秦剑\ldots 小仙女与顾人玉,这三人纵能制住江别鹤,少不得也是要吃些苦的。

第三,他终於将慕容九送回她自己亲人身旁,她日後神智纵不恢复,但在亲人身旁,总不会再被人欺负。这样,小鱼儿也了却一桩心事。

第四、江别鹤上过这次当後,纵然不死,也必定要老实得多,白开心等人,也必不敢再多事。这样,江湖中又有些太平日子了。

第五、段家的镖银也可能因此而物归原主,段家父女对他总算不错,他这样也等於报了他们的恩了。

第六、铁无双所受的冤枉,也因此可以洗清,也免得这爱才如命的老人,死後还落个污名。

他灵机一动间想出个计划,竟一举而六得,这计划实行起来纵然困难些、复杂些,却也是值得的了。

小鱼儿思前想后,越想越觉得这计划是天衣无缝,妙到极点,江别鹤纵然心计深沉,只怕也想不出这样的妙计来。

江别鹤、秦剑、南宫柳、白开心、罗九、罗三\ldots\ldots 有关这计划的每一个人,虽然都是厉害透顶的角色,但却都被他利用了而不自知,他绝不相信世上有任何一个人能将他的妙计瞧穿。

小鱼儿愈想愈是得意,忍不住喃喃笑道:谁敢说我不是天下第一个聪明人,谁敢讲我不是天才。

喂,跟我走吧。

三姑娘将这话又说了一次,说得声音更大,慕容九却还是在瞧小鱼儿身影消失之处,痴痴的出神。

三姑娘冷冷道:他人已走了,你还瞧什麽?

慕容九歪着头想了想,幽幽笑道:不错,他人已走了\ldots\ldots 但你知不知道,他以後还会来看我的。

叁姑娘大声道:他骗你的,他将你送来这里,就不再理你了。

慕容九嫣然一笑道:他绝不会骗我的,我知道。

她充满自信的抬起头,月光便照上了她那微笑着的脸,那充满对未来幸福憧憬的月亮眼波。

三姑娘虽是女人,也不禁瞧得痴了,颤声道:你\ldots\ldots 你怎知道他不会骗你?

慕容九微笑着道:他将我送到这里来,只是为了要将我心里的恶魔赶走,然後,他就会来找我的。

三姑娘瞧着她那张痴迷而美丽的脸,缓缓道:你什麽都不记得了麽?

慕容九道:嗯。

三姑娘道:若不是因为你神智不清,他就不会将你送来了?

慕容九道:我知道他也舍不得离开我的。

三姑娘道:等\ldots\ldots 等你好了後,他\ldots\ldots 他就来找你?

她语声竟已因嫉妒而微微发抖,这麽强烈的嫉妒,已足以使一个女人不惜做出任何事来。

慕容九却全不知道,嫣然笑道:他一定会来找我的。

三姑娘道:他\ldots\ldots 他还说了些什麽?

慕容九迷惘的眼睛也发了光,笑道:他说,我是个聪明的女孩子,只要我听话,他就会天天陪着我,我自然会听话的,你说我应不应该听他的话昵?三姑娘突然吼声道:不应该,不应该?慕容九楞住了。

三姑娘狂吼道:你非但一点也不聪明,也一点都不漂亮,你只是个疯子,又丑又怪的疯子他绝不会喜欢你的?

慕容九终於忍不住放声大哭起来,掩面道:我不是疯子,我不是疯子\ldots\ldots{}

叁姑娘道:你不是疯子,我问你,你可知道自己是谁麽?

慕容九她拚命想,也想不起自己是谁,只觉得忽然头疼欲裂,竟拚命打着自己的头,痛哭着道:求求你,莫要问我了,我不知道,我不知道\ldots\ldots{}

三姑娘冷笑道:一个人连自己是谁都不知道,不是疯子是什麽?

慕容九嘶声狂呼道:我是疯子,我是疯子\ldots\ldots 他不会喜欢我的,不会喜欢我的\ldots\ldots{}

呼声中,她竟痛哭着狂奔了出去。

三姑娘直瞧着她身影走得不见了,才松了气,她嘴角不禁泛起了一丝残酷的胜利微笑。

小鱼儿千算万算,终於还是忘记了一件事。他竟忘了天下绝没有任何一个女人不是嫉妒的。

小鱼儿在黑暗中静静的等着,竟始终瞧不见一个人影荒郊中自然听不见更鼓,他也不知到了什麽时候。

但他却还能沉得住气,这时远处终於有了人声。

小鱼儿精神一振,喃喃道:先来的不知是谁?两批人虽然都很急,但江别鹤大约总比较沉得住气,按理说先来的应该是秦剑。

只听人声中竟还杂着有滚滚的车轮声,隐隐的驴叫声小鱼儿暗道:来的果然是秦剑一夥人,竟以驴车将银子运来了\ldots\ldots{}

心念一转,突又发觉不对。

秦剑\ldots 南宫柳那样的世家公子,要用车来运送银子,也必定是用马拉,绝不会用驴子的。

这时车马已来到他视线之内。

来的竟非秦剑和南宫柳一夥人,也不是江别鹤,竟是五六个披头散发,穿着麻衣孝服的乡下妇人。

驴车上载的也不是银子,而是棺材。

小鱼儿不禁呆住了,半路上怎地突然杀出了个程咬金,深更半夜的,这些乡下妇人跑到这里来干什麽?

只见这几个妇人走入了祠堂,竟一起跪在地上,放声大哭了起来,左面的一个妇人磕着头哭道:我死去的公公呀,你在天上有灵,替我评评这个理吧,我为你门家守寡守了几十年,好不容易守到儿子长大,指望他好生孝敬我,让我下半辈子享享清福,那知他竟被人害死了,你叫我下半辈子怎麽过呀?

这妇人年看来已有四\ldots 五十岁,虽然穿着孝服,但看来却还是端端正正,她一面哭,身旁的一个年轻妇人就不住替她背,也痛哭着道:姨奶奶,你可千万不能哭坏身子,你伤心死了,家产可就全落到别人手里了,你又何必让别人得意。

这一边一哭,右面那妇人也不甘示弱,立刻痛哭着道:死去的公公、婆婆呀,你们在天上有灵,就替我撕烂那贱人的嘴吧,儿子虽然不是我生的,但总是我们家的骨血,要算只能算我的儿子,那贱人名不正、言不顺,又算什麽东西,她冤任我,只不过是想谋夺家产罢了。

这妇人年纪较大长的也较丑,看来虽然瘦骨伶仃,但哭起来的声音却此什麽人都大。

她一哭,身旁立刻也有个较年轻的妇人陪着哭道:大奶奶,你千万莫哭坏了身子,大家都是有眼睛的人,绝不会让那恶毒的妇人将家产霸占去的。

小鱼儿听了几句,心里已明白了。

到祠堂里来评理倒也没什麽不该,千不该、万不该,只是不该在这节骨眼上撞到祠堂来。

小鱼儿实在也未想到天下竟有这麽巧的事,不禁又是好气,又是好笑真想将这些妇人赶走。

他心里正在暗骂,突见几条黑衣人影,悄然掠了过来,几个人俱是黑衣劲装,黑衣蒙面。

小鱼儿心里一跳:江别鹤来了。

那几个妇人还在边哭边骂,全未发觉祠堂里已多了几个人,几个黑衣人冷冷的站在後面,也不说话。

只见那大奶奶和姨奶奶本是各骂各的,此刻已变得对骂了起来,那大奶奶指着姨奶奶骂道:你这贱人,仗着几分狐媚,迷死了我的丈夫,现在你儿子也死了,这是老天报应你,你还敢骂我?

那姨奶奶怎肯示弱,立刻也反唇骂道:你这醋子,丑八怪,自己也不撒泡尿照照自己,还想和人争风吃醋,我丈夫就是被你气死的!

大奶奶怒道:谁是你丈夫,不要脸,丈夫明明是我的。

姨奶奶冷笑道:你才不要脸,嫁给他那麽多年,连个屁都没有放出来,若不是我,他死了连个上坟的人都没有。

这姨奶奶竟是能说会道,骂起人来又尖酸\ldots 又刻毒,那大奶奶被她气得全身发抖,突然一个耳光蝈了过去。

姨奶奶脸上挨了一巴掌,大骂道:好,你敢打人,我和你拚了。

她扑上去,就揪住了大奶奶的头发。

她们身旁那两三个年纪较轻的妇人,赶着来劝架,但到了後来,你一耳光,我一巴掌,劝架的反而打得更凶。

几个妇人揪头发\ldots 扯衣服,竟打做了一团,竟滚在地上,越滚离那几个黑衣人越近。

那几个黑衣人倒也奇怪,眼瞧着她们在面前打,竟也像是没有瞧见似的,还是冷冷的站在那里。

就在这时,只听嗤,嗤,嗤一连串声响,竟有几十道乌光自那些打架的妇人堆里暴射而出。

这些暗器来得竟是又毒又快,那几个黑衣人全在暗器笼罩之下,眼见是没有一个人能逃得了的!

小鱼儿早已觉得有些不对了!

这几个妇人虽是蓬头散发,脸上也是又粗又老,但每个人的手,却都是十指尖尖又白又嫩。

小鱼儿发现这点,眼睛立刻一亮,暗道:慕容家的姑娘,果然厉害,江别鹤看来这个当是上定的了。

他这念头刚转完,暗器已暴射而出。谁知那些黑衣人居然也似早已料到有此一着。

暗器飞出,这几人便已冲天而起,呛的,凌空拔出了刀剑,寒光如流星,向那些妇人笔直刺下!

这些妇人竟也无一是弱着,身子一滚分开,闪过了凌空刺下的一剑,跃起时掌中都已多了件兵刃。

为首那黑衣人冷笑道:好个无知的妇人,竟敢在我面前玩弄奸计,你们还差得远些,我早已调查过,这祠堂一家的後代,都已死净死绝\ldots\ldots 你们究竟是什麽人,若不说出来历,今日休想有一个能活着走出去。

小鱼儿暗叹道:这江别鹤果然是只老狐狸,无论做什麽事之前,竟都先将每一着都提防着,将每件事都调查得仔仔细细,绝不肯放松一步。

只见那大奶奶冷冷一笑,道:咱们是为着什麽来的,你难道还不知道?

这句话本来很容易答覆,甚至可以说不答覆都没关系,但这黑衣人心机深沉,别人听来简简单单的一句话,经过他一想,却变得复杂的很。

他若说知道,就无异承认这镖银确是他动手劫下的,对方若只不过是做个圈套诱他吐实,他岂非便要上当了。

那些妇人见他迟疑不敢作答,心里也不免动了疑心,那大奶奶和姨奶奶交换了个眼色,姨奶奶道:你究竟是什麽人?难道不是为那封信来的。

黑衣人这次再不迟疑,冷笑道:若不是为那封信,我怎会来到这里?

姨奶奶道:如此说来,那些银子你是非要不可了?

黑衣人心里再无怀疑,厉声道:不但要银子,还要人?

大奶奶面色微微一变,怒道:你要了银子,还要人?

黑衣人道:两样缺一不可?

那姨奶奶大怒道:你凭着什麽,敢如此强横霸道?

黑衣人冷笑道:就凭我掌中这柄利剑?

双方愈说火气愈大,小鱼儿却愈听愈是开心,只希望他们快些动手打起,打得愈凶愈好。

只见那大奶奶和娆奶奶又交换了个眼色。

那姨奶奶大声道:老实告诉你,银子和人,你一样也休想要得到,银子咱们根本未带来,人昵\ldots\ldots 你若想要人,咱们就要你的命?

黑衣人目光一转,冷笑道:我早已说过,银子和人,缺一不可,如今就先取过银子再说吧话声未了,已悄悄在身後打了个手式。妇人们虽未瞧见他的手式,小鱼儿却瞧得清清楚楚。

另四条黑衣人自然也瞧见了,前面两人突然出手,刀光闪动处,竟活生生将那匹拉车的驴子砍倒在地!

後面两人却提起了车上的棺材,往下一倒,只听哗啦啦一声巨响,棺材里倒下了无数银子。

虽在黑夜之中,这许多银子仍是灿栏生光,耀人眼目,那几条黑衣大汉骤见这许多银子,竟不觉呆了。

为首那黑衣人纵声笑道:我早已说过,你们若想弄鬼瞒我,还差得远哩!

这银子自然正是他的镖银无疑。

说话间他已悄悄打了第二个手式,那几条黑衣大汉挥刀便待扑上,这时,就在这时,突听又是嗤,嗤,嗤!连串声响,那装银子的棺材里,竟也暴射出数十道乌光,向黑衣人们飞出!

那几条黑衣大汉惨呼一声,俱都扑倒在地。

只有为首那黑衣人站得较远,应变也较迅,剑光飞舞,震飞了暗器,但瞧见他属下竟无一幸免,目光也不禁露出惊怒之色,大喝道:好狠毒的妇人,竟敢\ldots\ldots{}

那大奶奶冷笑截道:对付你这样狠毒的人,自然也只有用这种狠毒的法子!

几个人渐成合围之势,砰的一声,棺材底被震得飞起,又有个人跃出来,站在黑衣人身後,厉声道:你还有什麽话说?

那黑衣人孤零零被围在中央,竟是丝毫不惧,反而冷笑道:想不到你们行事倒也周密,我们未免低估了你们,只是你们此刻便得意,还嫌太早了些!

自棺材里跃出的那人一身紧衣,身材婀娜,面上虽仍蒙着层轻纱,但小鱼儿还是一眼就认出她是小仙女。

想是因为她性子急躁,又不会装假啼哭,所以别人才先要她藏在棺材里免得露出马脚误事。

此刻她在棺材里憋了一肚子闷气,早已忍不住了,一剑刺向那黑衣人的後背,叱道:废话少说,你纳命来吧?

那黑衣人背後竟似生着眼睛,头也不回,反手一剑上撩,将她掌中的剑几乎脱手震飞?

小仙女手腕被霞得又酸又麻,才知道面前这黑衣人竟是自己平生未遇的强敌,又惊又怒,大喝道:你死到临头,还敢逞强!

黑衣人藉长剑一挥之势退到墙角,冷冷笑道:死到临头的究竟是谁,你们不妨瞧瞧吧!

大家不由自主随着他目光转头一瞧,只见这荒祠外竟多了无数条黑衣人影,一个个俱已张弓搭箭。

窗户里,墙隙间,已布满了黑黝黝的闪亮箭镞。妇人们不禁俱都为之失色。

黑衣人冷冷道:这祠堂外已伏下一百四十张铁胎弓,每张弓俱有三百石力气,我数到三,你们若还不放下掌中的兵刃,束手就缚,後果如何,你们自己也该想像得到!

一百四十张铁胎强弓,若是分成两批,轮流不断发射,纵是顶尖的武林高手,最多也不过只能抵挡一时而已。

这些妇人们心里自然也知道,自己这群人中,纵或有一两人能冲得出去,但别的人却只怕都要丧生在箭下!

几个人又聚在一起,窃窃私议,小仙女和那姨奶奶语声忽停,似要硬闯,大奶奶却紧紧抓住她们的手。

黑衣人冷眼旁观,悠然道:一!

大奶奶突然道:银子和人就都你如何?

黑衣人冷冷道:你先将人\ldots\ldots{}

话声未了,突然一阵惊呼,祠堂外的黑衣人,已有几个倒了下去,严密布下的箭阵,刹那间便已大乱。

那姨奶奶眼睛一亮,娇呼道:三妹、菁妹、还不动手,等待何时!呼声中,一柄闪亮的短剑,已向黑衣人直刺过去!

小鱼儿一听那大奶奶说出那句话来,就知道再也不能让他们谈判下去否则这事就要揭穿了!

他一念至此,掌中早已准备好的尖石,便直击出去!

他手法又快,藏身之处又隐,十馀人被打得头破血流,满地翻滚,竟无一人瞧出那些暗器是从那里发出的。

这时那姨奶奶短剑已化做一片寒光,转瞬间便刺出了十剑,她虽是妇道人家,但剑法之敏捷辛辣,纵是浪迹江湖时刻找人拚命的黑道强豪、白道游侠,也都难及她万一。

黑衣人骤然间剑势竟被她逼住,暗中不禁吃了一惊。

这姨奶奶剑法不但辛辣,而且招招都有不惜和对力两败俱伤的姿态,放眼江湖,这样的女子委实没有几个。

再瞧那大奶奶,平剑当胸,在旁掠阵,竟无出手夹攻之意,女子和男人动手,总是吃亏些。

是以女子纵然以多为胜,江湖中也没有人会说闲话的,这姨奶奶到了这种地步,居然还是自恃身分,不屑以二敌一,这麽大气派的女子,在江湖中更如凤毛麟角,绝无仅有。

黑衣人愈瞧愈奇怪,愈想愈吃惊。

更令他吃惊的是,那两个丫头暗器手法竟也准得吓人,只要手一扬,外面立刻就有一、二人惊呼着倒下去。

小仙女更早已冲了出去,百来个黑衣大汉,此刻倒下至少已有四五十个,剩下的自顾尚且不瑕,那里还有功夫放箭。

小鱼儿瞧得张大了嘴,几乎要笑出声来,他吃了江别鹤几次亏,这气到今天才总算是出了。

又是数十招拆过,那姨奶奶剑出更快、更毒,剑剑不离黑衣人的要害,剑尖已堪堪到了黑衣人的咽喉。别人看着,都只道她已占了上风。

却不知那黑衣人心机最多,此刻又在想着心事,掌中剑虽在展动,只不过是虚应故事,但求护身而已。此刻他心意贯通,突然朗声大笑,平平一剑削出。

那姨奶奶顿觉对方一柄轻瓢飘的长剑,竟骤然变得千钧般重,剑还未到,已有一股大力涌来。她应变不及,只有挥剑迎了上去。

她剑法虽辛辣,内力却与这黑衣人相去甚远,黑衣人这一剑力已用足,她舍己之长,用己之短,挥剑迎上,这无异以卵击石。

这只因她委实太小瞧这黑衣人的武功,等到发觉时却已迟了,纵然明知吃亏,也只有硬着头皮一拚。

那大奶奶瞧得清楚,失惊道:千万别和他斗力!

她纵然不屑以多为胜,此刻事态紧急,也说不得了,喝声中长剑挥出,也迎击了上去只听呛的一声龙吟,火花四下飞溅。

大奶奶和姨奶奶以二敌一,竟还是力不能及,两人但觉半边身子发麻,掌中剑几乎脱手飞去?

小鱼儿瞧得暗暗顿足道:这些丫头们不用自己拿手的功夫,反和人家斗力气,岂不是自找倒楣麽?

只见这大奶奶和姨奶奶身子凌空飘开了两丈,几乎已退到墙上,两人临危不乱,掌中早已扣好了暗器。

慕容家姑娘的轻功暗器,天下扬名,黑衣人若是求胜心切,贪功追来,只怕就很难全身而退了。

谁知黑衣人一击未成,竟立刻住手,朗声笑道:今日我什麽都不要了,就此别过。一面说话,身子已向後退。

这一着倒是连小鱼儿都大感意外,那大奶奶和姨奶奶见他明明占了上风,却反而要走了,不禁更是奇怪。

姨奶奶忍不住道:你方才死命逼人,此刻却想一走了之,这是为了什麽?

黑衣人大笑道:方才我不知你们是谁,若是走了,日後再也难以寻找,那时我自然是万万不肯走的!

姨奶奶道:现在呢?

黑衣人冷笑道:慕容家的姑娘有名有姓\ldots 有家有业,我今日要不回东西来,以後日日到府上拜访,还怕要不回来麽!

姨奶奶变色道:你已瞧出了咱们的来历?

黑衣人道:慕容二姑娘剑法辛辣,天下皆知,我若再瞧不出,就真是瞎子了!

那姨奶奶突然自头上扯下一把头发\ldots 一张面具,震出了一张白生生的脸,只见她杏眼圆睁,柳眉带煞,冷笑道:你认出了我,我却不认得你,日後正是再也找不着你了,你想想,今天咱们还能让你走麽?

一人大声接道:他走不了的!

小仙女已挡在黑衣人身後,堵住了门。

黑衣人厉声狂笑道:我今日若走不脱,方才也不会说那番话了!

慕容双喝道:我们要看看你如何走得脱!

这位慕容二姑娘,脾气果然急躁,方才虽吃了个亏,此刻竟丝毫不惧,挥剑又扑了上去。

只听当的一响,那大奶奶竟拦住了她的剑。

慕容双怒道:三妹,你难道要放他走,你难道不想寻回九妹了麽?

慕容珊珊道:我看此事,其中似乎有些蹊跷。

慕容双道:什麽蹊跷?

慕容珊珊道:此人既将我等约来,便应早已知道我们是谁,但他却直到此刻才知道我们的来历,这岂非有些奇怪麽?

慕容双愣了愣,还是跺脚道:这有什麽奇怪,谁知道他这不是在装佯。

小仙女应声道:不错,先制住他再说。

那黑衣人一直留神倾听,此刻突然大声道:三位且莫动手,你我只怕都中了别人挑拨之计了。

话声未了,突听哗啦啦一阵响,一只香炉,从屋梁上滚了下来,还带着拉下了一大条白布。

那白布士竟写着:江别鹤,你作恶多端,到现在想赖也赖不掉了!

白布上碗大的黑字,虽在黑夜中也瞧得分明。几人见了,俱是大吃一惊。

慕容双失声道:你\ldots\ldots 你竟是江别鹤?

黑衣人目中露出惊讶之色,他听了慕容姑娘的对话,已知道自己虽然精打细算,今日还是落入了别人的圈套,却连那真正在暗中主谋的人是谁都不知道。

他心机素多,别人只想起了一件事,他已想起了十件,这有时反而害了他,只因他心里有事就忘了答话。

慕容双冷笑道:堂堂的江南大侠,竟也做出这样的事来,倒买是令人想不到的。

黑衣人还未答话,只听又是哗啦啦一阵响,一个香炉盖从梁上滚了下来,又带下条白布。

白布上还是写着海碗那麽大的字``江别鹤,你藏的人已被寻着了,你还有什麽话说''。

这些布条,自然是小鱼儿方才早已准备好的,他将布条一端钉在梁上,用香炉包着布条的另一端,又在香炉下系着条又长又细的线,从屋梁上绕到他藏身之地,只要线一拉,香炉滚下来,布条自然也就随着落了下来。

方才他听得慕容珊珊愈说愈不对了,再说下去,他这妙计便要被揭穿,所以赶紧将线一拉。

他算定秦剑等人此刻必定已在江别鹤屋里寻着了慕容九,等到他们将慕容九带来,江别鹤纵有一百张嘴,也休想辩说得清了,这计划原是万无一失,他做梦也想不到其中竟会出了差错。

\hypertarget{ux7b2cux4e94ux5341ux516dux7ae0-ux4f5cux6cd5ux81eaux6bd9}{%
\chapter{第五十六章
作法自毙}\label{ux7b2cux4e94ux5341ux516dux7ae0-ux4f5cux6cd5ux81eaux6bd9}}

两张布条落下後,就连慕容珊珊心里也再无怀疑,小仙女和慕容双更是满面杀气,恨不得将江别鹤先宰了再说。

那黑衣人既未承认自己就是江别鹤,却也未否认,竟是一言不发,眼睛只是瞪着对方的几柄剑。

慕容双瞪着眼睛,道:三妹,现在你说怎样?

慕容珊珊叹了口气,道:先拿下他再说吧。小仙女等不及她这话说完,掌中剑已刺了出去。

她剑法迅急泼辣,慕容双剑法辛狠毒辣。

慕容珊珊的剑法虽然急不如小仙女,狠不如慕容双,但眼睛敏锐,头脑清楚,每刺一剑,必是对力的必救之处?

这三个人三柄剑,可说都不是好惹的,而且自幼同堂练剑,招式配合得更是滴水不漏。

那黑衣人武功虽高,却也难以应付,挡了几招,剑法突转凌厉,已是以进为退,想夺路而逃了。

怎奈对手虽是三个女子,与人交手经验之丰富,并不在任何人之下,他剑法一变,三个人已全都瞧破了他的心意。

他不走还好,这一想走\ldots 对力更是认定他无私也有弊,小仙女与慕容双更是不要命的缠了过来。

她们带来的二个丫头,应付外面剩下的黑衣大汉们,竟也是绰绰有馀。

黑衣人头上汗珠,已湿透了蒙面的黑巾,这才知道名动天下的慕容姐妹,果然不是好斗的。

他却不知道剑法还非慕容姐妹所长,暗器轻功,才是她们的绝技!只是此刻她们生怕他见隙而逃,是以才没有抽身使出暗器。

只听嗖的一声,慕容珊珊一招分花拂柳,迎面刺来,剑光闪动不歇,也不知是虚是实。

她这一招其实不在伤敌,只在眩乱对力的眼目,好教别人出手,但黑衣人若不闪避,虚招立刻变成实招。

黑衣人不假思索,斜身扬剑,小仙女与慕容双果然已在等着他了,剑光如惊虹交剪,左右刺来。

她三人所使出的这叁招,并非什麽高妙的招数,但配合却实在绝妙无此,三招普普通通的剑式一齐刺来,威力何止大了三倍,闪动的剑光,竟将对方的所有去路全都封死,眼看是避得开这一剑,也避不开那一剑的。

谁知黑衣人一招挡开了慕容珊珊的剑後,竟突然松手,抛却了掌中剑,出手如风,已捏着了慕容珊珊的手腕?

这一招变得委实险极,也委实妙极,若非他这样的人,也想不出这样的招式,就连小鱼儿瞧得都几乎失声喝采!

黑衣人另一只手已到了她咽喉,叱道:你们还要不要她的命?

这时黑衣人虽然背後全是空的,小仙女与慕容双的两剑,随时都可以将他身子刺上几个窟窿。

但慕容珊珊性命已被别人捏在掌下,她两人又怎敢出手,两柄剑抵住黑衣人的身上,竟不敢刺下去!

慕容双跺脚道:快放手,否则我就宰了你!

黑衣人冷笑道:你们若不放手,我就宰了她!.小仙女道:你先放,我们就放。

黑衣人大笑道:男儿不该与女子争先,还是你们先放吧!

慕容双怒道:我们怎能信得过你?

黑衣人冷冷道:我也未见能信得过你们!

双方谁也不敢出手,却也不敢放手,这样僵持了一会儿,小仙女与慕容双性子急躁,早已急出了满头大汗。

慕容珊珊反倒似不着急,缓缓道:二姐你们切切不可放手,他是决计不敢伤我的。

黑衣人冷笑道:我素来沉得住气,就这样耗下去也没关系。

慕容双怒极之下,剑尖忍不住向前一移,那边慕容珊珊立刻就透不过气来。

小仙女怒吼道:你究竟要这样耗到几时?

黑衣人道:直到你们放手为止。

小仙女满头大汗,似已急得不知该如何是好!

小鱼儿苦笑暗道:傻丫头,你着急什麽,你难道还怕没有帮手来麽?\ldots\ldots{}

就在这时,远处三条人影一闪,刹那间使到了跟前,果然是南宫柳\ldots 秦剑与顾人玉来了!

小鱼儿\ldots 慕容姐妹俱都大喜,但那黑衣人有恃无恐,竟也不甚惊惶!秦剑来了,更不会让慕容珊珊死的。

他只要挟持着慕容珊珊,就不愁走不出去。

秦剑见到爱妻被人挟制,面色果然大变,顾人玉江湖经验最嫩,瞧见这情况,更是呆住了。

小仙女跺脚道:呆子,你还不过来帮忙?

黑衣人大喝道:谁敢过来?

秦剑道:这\ldots\ldots 这究竟是怎麽回事,朋友有话好说。

黑衣人厉声道:此事纯属误会,但事已至此,我纵然解释,你们也是不会相信的,什麽话只有等我先走出去再说了!

这时南宫柳已瞧见梁上挂着的布条,失声道:阁下莫非真的是江大侠?

小仙女喝道:什麽狗屁的大侠,此人正是江别鹤!

慕容珊珊喘了气,道:你们先别管我,先问问九妹可曾找着了麽?

南宫柳叹了口气,道:我等方才已到江大侠的居所去了一次\ldots\ldots{}

小鱼儿听到这里,一颗心已拎了起来,他们若在江别鹤住所寻着了慕容九,又怎还会对他如此客气,称他为大侠!

慕容珊珊也已着急道:九妹难道不在那里?

秦剑急道:你先别管九妹,你自己\ldots\ldots 你自己\ldots\ldots{}

南宫柳苦笑道:九妹并未在江大侠那里,我等只怕是全都被人捉弄了!

小鱼儿这一惊才是非同小可,几乎要从藏身之处跳了出来,慕容九怎会不在那里,莫非是他们找错了地方?

秦剑道:我等方才也已见过了花无缺公子和铁心兰姑娘,都说九妹早已失踪,绝不会和江大侠有关!

慕容双愣在那里,剑已不觉垂下。

小仙女喃喃道:铁心兰想来是不致於帮江别鹤说话的。

慕容珊珊叹了口气,道:我也早已觉得此事有些不对,试想江大侠若存心要我们赎金,为何要自己出头?纵然他自己来了,又怎会不知道我们是谁?何况,他要将九妹藏起,地方也多得是,又何必藏在自己的居处?

秦剑顿足道:这件事你既然早已想到,为何还要与江大侠动手。

他见到那黑衣人还未松手,自然只得先责备妻子的不是。

慕容双却不服道:他\ldots\ldots 江大侠自己一句话不说,咱们怎会知道。

慕容珊珊眼珠子一转,突然问道:但\ldots\ldots 阁下是否真的是江别鹤江大侠?

这句话问出来,众人又不觉动了疑心。

只见黑衣人终於缓缓放下了手,微笑道:误会既已解开,在下是否江别鹤都是一样的了。

他竟还不揭开蒙面的黑巾。

秦剑早已蹿到慕容珊珊身旁,悄声道:你没事麽?】

慕容珊珊一笑握住了他的手,眼睛却还是盯着那黑衣人,道:贱妾等伤了江大侠那麽多属下,实是罪该万死,但望江大侠恕罪。

她故意将江大侠三个语声说得特别重些,而且一连说两次。

黑衣人还是既不承认,也不否认,笑道:双方既已出手伤亡在所难免,又怎能怪得了夫人,只是,那暗中陷害我等的人,却实在可恨!

说到这里,他一双冷森森的眼睛,突然盯到小鱼儿藏身之处,众人的目光也不禁随之望了过去。

慕容双大声道:不错,那人的确是不能放过!

小仙女道:我若找着了那人,先割下他的舌头、挖出他的眼睛,再问问他为什麽要使出这害死人的毒计。

几个人一面说话,一面将小鱼儿藏身之处隐然围住,这许多顶尖高手将一个人围住,无论是谁,也是休想逃得了的!

小鱼儿掌心也不觉沁出了冷汗,他知道这些人若是抓住了自已,那後果真也是不堪设想。他弄巧成拙,害人不着,竟害着自己。

就在这一瞬间,他脑筋已动了几百次,却也想不出一个法子能逃得了。

这时那黑衣人已冷笑道:到了这时,阁下还不出来麽?

慕容双恨声道:你既然早已知道他藏在这里,为何不早说?

黑衣人道:那时我见到暗器自这里飞出,击伤了在下的同伴,还以为是夫人们预先将人埋伏在这里的。

小鱼儿暗骂道:这双狗眼,倒当真是毒得很。

他骂尽管骂,却已知道自己此番是劫数难逃的了,要想从这些包围中冲出去,那岂非是做梦。

只听黑衣人冷冷道:朋友再不自己出来,在下便要令人发箭了!

慕容双突然抢过柄弓箭,大声道:且叫你见识见识慕容姑娘弓箭上的本领!

小鱼儿那天参观着慕容双闺房後,便已知道她在弓箭上必有非凡的身手,他可不愿蹲在这里做她的箭靶子,就在这时,突听一人咯咯笑道:【这里好热闹呀,莫非是在看戏麽?

众人不由得齐转头望去,只见一人长袍披发,咯咯的痴笑着,幽灵般走了过来不是慕容九是谁!

慕容九方才到那里去了?此刻又怎会来到这里?这的确连小鱼儿也瞧得愣住了。

慕容姐妹喜交集,失声呼道:九,你可想死我了?呼声中两人已扑过去抓住了慕容九的手。

慕容九瞧了她们一眼,目中却满是茫然之色,咯咯笑道:你们是谁?我不认得你们呀?

慕容双颤声道:九妹,你\ldots\ldots 你难道连二姐都不认得了麽?话未说完泪珠已夺眶而出。

慕容珊珊也是热泪盈眶,流泪道:九妹,你怎地会变得如此模样?

慕容九痴痴的瞧着他们,也不说话。

顾人玉终於忍不住走过去,颤声道:九妹!你认得我麽?

小仙女顿足道:【他连二姐三姐都不认得了,又怎会认得你?

顾人玉垂下头来,眼泪已滴在地上。秦剑与南宫柳亦是满面惨痛之色。

慕容双顿脚道:是谁将她害成这样子?是谁?

小仙女突然大哭道:她见了小鱼儿死而复活,所以才吓成这样子的,其实小鱼儿根本没有死,是故意吓吓她的。

慕容双大喝道:谁是小鱼儿?他现在那里?

小仙女道:现在只怕是死了。

慕容双愣了愣,道:你方才说他未死,此刻又说他死?他到底死了没有?

小仙女道:他本来没有死,後来却跌到悬崖死了。

语声微顿,又道:但这人一肚子鬼主意,一身鬼本事,别人明明算定他死了,他却常常没有死,没有亲眼瞧见他的身,谁也不敢说他是否真的死了!

黑衣人突然道:他还没有死。我最近又瞧过他的。

慕容双大声道:你知道他在那里?

黑衣人冷冷道:依我看来,他此刻只怕就在\ldots\ldots{}

他像是已猜出藏着的便是小鱼儿,小鱼儿心又拎了起来,那知他一句话还未说完,慕容九突然大声道:小鱼儿\ldots\ldots 小鱼儿!我想起来了!

大家又是惊又是喜,慕容双颤声道:你\ldots\ldots 你什麽都想起来了麽?

慕容九痴痴的瞧过她,缓缓道:你是二姐。

慕容双狂呼一声,抱住了她,竟欢喜得放声痛哭了起来。

慕容珊珊也不觉喜极而泣,道:九,九\ldots\ldots 天可见怜,你终於好了。

慕容九笑道:三姐\ldots\ldots 三姐,我还能见着你们?我这是在做梦麽?

姐妹们又笑又哭,哭成一团,小鱼儿在一旁偷偷瞧着,眼睛竟也不觉湿了,心里也不知是何滋味。

只听那黑衣人突然叹道:那江小鱼将令妹害成如此模样,江湖中谁也放不过他的。

他留在这里不走,原来就是为对付小鱼儿的,生怕慕容姐姝欢喜中忘记这事,赶紧又提醒了一句。

慕容双果然顿住哭声,恨恨道:我若知道那小贼现在那里,不宰了他才怪。

慕容九突又截道:这事其实是怪不得小鱼儿的。

这句话说出来,大家又吃了一惊,最吃惊的当然还是小鱼儿自已,其次就是小仙女了。

她忍不住问道:不怪他怪谁?你岂非恨他入骨的麽?

慕容九凄然一笑,道:我见他死而复活,当时骇了一跳,虽然有些迷迷,但过了没有多久,便已渐渐清醒了过来。

慕容双奇道:你既然早已清醒,为何方才不认得我们?

慕容九道:那是被江别鹤害的!

这句话说出来,连小鱼儿也糊涂了。江别鹤又怎会害她?

只听慕容九接道:他见我清醒,就又以迷药迷住了我,他想乘我晕迷时,逼我和他成亲,为的也是想做慕容家的女婿,他日日夜夜看着我,直到方才,我见他不在,才偷偷溜出来的。

众人方才虽已认为江别鹤受了冤枉,但此刻这话亲从慕容九嘴里说出来,那还会假麽?

慕容双怒喝道:好个可恶的江别鹤,咱们竟险些被他骗过了!

南宫柳亦自怨道:难怪我等方才寻不着她,原来她已自己逃出,幸亏老天有眼,叫他逃来这里,这当真是天网恢恢,疏而不漏。

喝声中几个人又将那黑衣人团团围住。

小鱼儿瞧得可真是又惊又喜,但却又是满头雾水、一肚子糊涂,事情竟会演变到这地步,小鱼儿就算真的是天下第一个聪明人,却再也想不通是怎麽回事。

只听慕容双喝道:江别鹤,你到现在还有何话说?

谁知那黑衣人竟突然放声大笑起来,道:谁说我是江别鹤?

他顺手抹下了蒙面的黑巾,露出一张满是虬髯的脸,众人俱都瞧过江别鹤,这张脸果然不是江别鹤的,大家不禁都愣住了。

慕容双失声道:你究竟是谁?

慕容珊珊道:你若不是江别鹤,江别鹤在那里?

黑衣人大喝道:江别鹤就在这里!

他竟突然冲入小鱼儿藏身之地,呼道:江别鹤,你出来吧。呼声中一掌闪电般拍下!

\hypertarget{ux7b2cux4e94ux5341ux4e03ux7ae0-ux610fux5916ux4e4bux5916}{%
\chapter{第五十七章
意外之外}\label{ux7b2cux4e94ux5341ux4e03ux7ae0-ux610fux5916ux4e4bux5916}}

小鱼儿见黑衣人闪电般一掌拍下,又是一惊,百忙中迎了一掌,喝道:``你才是江别鹤易容改扮的,骗得了谁?''那黑衣人竟也喝道:``你才是江别鹤易容改扮的,骗得了谁?''小鱼儿眼珠子一转,破口大骂道:``江别鹤,你这恶贼,你这混帐王八蛋,屁精活乌龟!''他算定江别鹤也是个人物,怎肯自己骂自己。

哪知黑衣人也大骂道:``江别鹤,你这恶贼,你这混帐,王八蛋,屁精活乌龟!''小鱼儿大笑道:``我就算不能逼出你的原形,听你自己骂自己,倒也出了我胸中一日恶气,哈哈,自己骂自己乌龟,可笑呀可笑。那黑衣人竟也大笑道:''我就算\ldots。."他竟然将小鱼儿说的话,一字不改、原封不动的说出来,小鱼儿骂得越来越开心,他也骂得毫不逊色。

两人一面骂,一面打,众人都不觉瞧得呆。

慕容珊珊道:``江别鹤武功人称江南第一,想必不差。''只见两人拳来脚往,不但功力俱都极深,招式也是千变万化,奇诡绝伦,竟都是顶尖儿的高手!

一时之间,谁也分不出他们武功谁强谁弱。

只听``砰砰蓬蓬''之声不绝于耳,无论什么东西只要挨着他们的拳风,立刻就被打得粉碎。

只见两人从里打到外,从近打到远。

要知这黑衣人虽不愿被人瞧破来历,小鱼儿也是如此,两人抱着同样的念头,自然越打越远。

两人招式看来虽仍凌厉,其实都不愿再缠战下去,突然齐地一纵,一个往东,一个往西。

两人身法俱快,慕容双等人虽然追来,却已追不着了,何况他两人分头而逃,大家也不知该去追谁!

就在这时,突见一个人自树林中暗影掠了出来,竟拦住小鱼儿的去路,指着小鱼儿怪笑道:``这才是江别鹤,这才是真的。''月光下瞧得清楚,这人竟是那``损人不利己''的白开心!

小鱼儿又惊又怒,喝道:``你疯了么?你不想要解药救命了?''白开心嘻嘻一笑,道:``谁救谁的命,你害了我,我不害你?''突然一个筋斗,倒纵了出来走得瞧不见了。

这时慕容姐妹等人早已赶来,几柄剑已将小鱼儿围住。

慕容双怒道:``江别鹤,这次若是再让你逃了,我就不姓慕容。''小鱼儿跳脚道:``谁是江别鹤?王八蛋才是江别鹤!''慕容珊珊冷笑道;``你不是江别鹤,为何要逃?''小鱼儿怔了怔,这句话他实在回答不出。

慕容双应声喝道:``是呀,你若不是江别鹤,为何不让我们检查检查你的脸!''她们上过一次当,再也不肯上当了,嘴里说话,手也不停,掌中剑刺出去一剑比一剑狠毒。

小鱼儿道:``我堂堂男子汉,怎能让你们女子碰我的脸,常言道:男儿脸上有黄金,女人手上有粪,我脸上怎能沾着粪土。''他一急之下,索性胡说八道起来,也正是想借此激怒她们,自己才好有机会冲出去。

慕容双果然大怒道:``放屁,你脸上才有粪土。''小仙女道:``你少时落在姑奶奶手中,不将你泡在粪缸去才怪。''小鱼儿道:``就算泡在粪缸里,也不能让女人摸来摸去。''众人已猜出他心意,知道他故意胡言乱语来打岔,谁也不再理他,只有那顾人玉最老实,忍不住道:``我不是女人,你让我检查检查如何?''小鱼儿道:你原来不是女人么?我还以为你也是她们的妹妹哩。``他自己说着,自己也不觉好笑,刚笑出来,''嗤"的,前胸衣裳已被划破,若不是他武功精进,肠子只怕已被划出来。

既到这种时候,他反正已豁出去了,瞧见秦剑与南宫柳并未动手,只是在旁掠阵,便又笑道:``慕容家的女婿,江湖中是人人羡慕的,都说你们艳福不浅,依我看来,却不如娶个麻子跛脚还好得多。''他嘴里说得开心,肩头又着了一剑,虽末伤着骨头,但剑锋过处,鲜血已泪泪然流了出来。

只听秦剑冷笑道:``秦某本不想以多欺你,但你如此,我也说不得了。''话声中已刺出三剑,这三剑功沉力猛面面惧到,正好补上慕容姐妹剑法之沉稳不足。

他心里虽暗叫苦,嘴里还是不饶人,大笑道,``南宫柳,你为何不也一起上来呀,难道你武功原也见不得人,只是靠老婆在江湖中混的么?''南宫柳面色果然微一变,突然沉声道:``腹结、府舍\ldots 市风、渎中\ldots\ldots 环跳\ldots\ldots\ldots{}''话末说完,已有三柄剑照着他所说的部位刺了出去,``嗤''的一声,小鱼儿``环跳''穴旁已被划破了条血口!

此刻他冷眼旁观,嘴里淡淡道来,正是小鱼儿难以闪避、难以招架的破绽之处。这一来小鱼儿更是手忙脚乱。

只听南宫柳接着道:``灵门、中府,阴市、梁邱\ldots 承扶!''刷、刷、刷三剑过后,小鱼儿承按``穴旁果然又挨了一划,他心里本在暗自思忖着道:''我听你先说出部位,难道不会躲么?"谁知等着别人说出来时,他竟是偏偏躲不开。

南宫柳纵横全局,对小鱼儿的出手已了如指掌,所指点出来的部位,自然正是小鱼儿之必救之地。

南宫柳又道:``幽门、通谷\ldots\ldots 府会、归来\ldots\ldots 涌泉!''这``涌泉''穴乃在脚底之下,小鱼儿听得不禁一怔,心想:``你们的剑难道还能刺在我足底么?''只见慕容珊珊剑势击来,直刺``府会''、``归来''两穴,他本可躲避,怎奈别的剑已封住了他去路。

他危急之中,不及细想,只有飞起一脚,去踢慕容珊珊握剑的手腕,慕容珊珊剑虽退去,但慕容双``刷''的一剑刺来,正恰巧刺在他``涌泉''穴上,小鱼儿穿着皮靴,这一剑伤的虽不重,但他却已不觉冷汗涔涔而落。

南宫柳悠然道:神堂、心俞\ldots 委中、阴谷\ldots\ldots\ldots 缺宣!``这一次小鱼儿更加注意,全神贯注,防护着''缺宣``穴,谁知后背一凉,''会阳"穴旁中了一剑。

而南宫柳正恰巧在此时道:``会阳!''

小鱼儿不禁暗叹一声:``罢了\ldots\ldots{}''

哪知就在这时,远处突然传来慕容九的惨呼声:``救命呀\ldots\ldots 江别鹤\ldots\ldots 你这恶贼\ldots\ldots 三姐\ldots 二姐\ldots 救命\ldots\ldots{}''呼声一声比─声远。

慕容珊珊大骇道:"不好,我们将九妹忘记在那祠堂里了\ldots\ldots{}

小仙女道:``江别鹤在那边。''

顾人玉道:``这人果然不是江别鹤!''

纷纷呼喝间,已都向慕容九呼声传来处飞过去,只有南宫柳走得最慢,竟向小鱼儿微一抱拳,道:``阁下身手非凡,似是集各门之长,卓然自成一家,只是出手间还不能浑然圆通,似是易露破绽,想是因为阁下旁骛太多,不能专心于武,日后若能改去此点,我纵在旁指点,也是无用的人。''小鱼儿怔了怔,道:``你为何要对我说这些话?''南宫柳道:``阁下实非江别鹤,江别鹤出手必不致如此生疏。''小鱼儿怒道:``你早看出来了,为何不早说?''南宫柳道:``在下虽早已瞧出,但那时还想瞧瞧阁下究竟是谁,是以也未说破,此刻既是九妹有难,自又当别论了。''小鱼儿叹了口气,道:``只怕是我骂了你两句,你就故意叫我受些苦吧。''南宫柳微笑道:``在下若非心中也有些不安,又怎会对阁下说那番知\ldots\ldots{}''微一抱拳,也展动身形追去了。

南宫柳已走得没有影子,小鱼儿还是在反复咀嚼着他方才说的那番话,越想越觉滋味无穷!

``\ldots 想是因为阁下旁骛太多,不能专心学武\ldots\ldots{}''小鱼儿叹了口气,喃喃道:``他这话倒还真是说到我节骨眼上了,看来这些武林世家的子弟的确有些门道的,倒也轻视不得。''他呆了半晌,放开大步,向前走去,只想先寻着那``损人不利己''的白开心好好算一帐。

他一面走,一面又忍不住喃喃自语道:``白开心怎会突然不怕死了,连解药也不想要?\ldots\ldots 慕容九又是怎么回事?此刻又是否真的被江别鹤劫去了?''小鱼儿越想越糊涂,索性不再去想了,但觉满身伤口,都发起疼来,就在树林里找了株大树坐下歇歇。

这时星群渐稀,东方渐渐露出了曙光,树林里面渐响起了啾啁鸟语,大地显得说不出的和平宁静。

小鱼儿闭起眼睛,喃喃道:``我只怕真的是闲事管得太多了,但一个人光吃饭不做事也不行呀,何况,事情找上门来时,想躲也躲不了的。''谁知就在这时,突听一人呼唤着道:``小鱼儿\ldots\ldots 江小鱼\ldots.你在哪里?''小鱼儿跳了起来,苦笑道:``事情果然真的找上门来了\ldots\ldots 却不知来的这人是谁?又怎会知道我在这树林子里?''只听那人又道:``小鱼儿,我知道你就在这树林子里,你快出来吧,我有很要紧的话要对你说\ldots\ldots 你还不出来么?''这声音竟似慕容九。

小鱼儿眼睛一亮,笑道:``若是慕容九,来得倒正好,我正想我她,她就来了。''只见一人披发长袍,踏着乳白色的晨雾飘飘而来,看来就像是乘云飞降的山林女神,可不正是慕容九。

小鱼儿突然跳到她面前,大声道:``喂!''

慕容九像是骇了一跳,抚着胸口,娇嗔道:``你又想吓死我?小鱼儿上下瞧了她两眼,笑道:''半天不见,你看来越发漂亮了。``慕容九抿嘴笑道:''半天不见,你看来也越发越英俊了。``小鱼儿笑嘻嘻道:''你不恨我了。"

幕容九道:``女人的心,常常会变的,你难道不懂么?''小鱼儿道:``我正是上过女人的当了。''

慕容九笑道:``谁让你上当的!谁骗过你?莫非是\ldots\ldots 那位铁姑娘?''小鱼儿心里一痛,大声道:``不是!是慕容九。''慕容九咯咯笑道:``我几时骗过你了?''

小鱼儿眼睛里发着光,一字字道:``你不是慕容九!''慕容九大笑道:``我不是慕容九是谁?难道你也发了昏,竟不认得我了。''小鱼儿瞪着眼睛瞧了她半晌,突然跳起来,翻了个筋斗,落在地上,又揉了揉眼睛,终于大笑道:``我想来虽绝不会是你,但却又一定是你。''慕容九笑道:``你到底说我是谁呀?''

小鱼儿一把抓住她,大笑道:``你是屠姑姑\ldots\ldots\ldots 屠娇娇!''那``慕容九''也瞪着眼睛瞧了他半晌,突也大笑道:``小鬼头,到底是你聪明,果然被你瞧出来了,普天之下,除了你之外,只怕谁也瞧不破我的。''小鱼儿道:``不错,只是。\ldots 我又不相信屠姑姑真的会到这里来,我简直做梦也想不到你会离开恶人谷。''屠娇娇竟叹了口气,缓缓道:``天下有许多事,都是想不到的。''小鱼儿瞪大眼眼,道:``我实在想不到屠姑姑竟也会叹气了,也想不出你怎会离开了恶人谷,更想不到你怎会知道我的事,而扮成了慕容九?''他心里想不通的事实在太多,忍不住一口气问出来。

屠娇娇笑道:``你连珠炮似的问我这么多,叫我怎么回答你呀?''小鱼儿道:``这一两年来,根本就没有人知道我在哪里,你又怎会知道我的事,又怎会扮成慕容九呢?''屠娇娇笑道:``我离谷之后,虽然听见过一些你的得意杰作,但确实不知道你躲到哪里去了!打听也打听不出。''小鱼儿得意的眨了眨眼睛,笑道:``你当然打听不出,我若想躲起来,谁能知道我在哪里。''屠娇娇道:``我找来找去找不着,前几天却在无竟中见到了你!我非但见过你,还跟你说过话。''小鱼儿摸着头,苦笑道:``这倒怪了\ldots\ldots 我居然还跟你说过话?\ldots\ldots\ldots{}''屠娇娇咯咯笑道:``你那时好凶呀,直瞪着眼睛叫我滚,我可真是不敢惹你,只好被吓得乖乖的远远滚开了。''小鱼儿跳了起来,瞪着眼睛大笑道:``我知道了,你就是\ldots\ldots 就是\ldots。''屠娇娇悠然笑道:``我就是罗九兄弟楼下的那傻丫头。''小鱼儿大笑道:``我实在佩服你,你实在装得真像,我真是做梦也想不到。''他大笑了一阵突又顿住笑声,问道:``但在那天之前,你并没有见过我是么?''屠娇娇道:``没有。''

小鱼儿道:``你当然也不会算到我会到罗九家里去的。''屠娇娇笑道:``我又不是神仙,自然算不出的。''小鱼儿道:``那么你又怎会扮成个傻丫头,躲在那里等我?''屠娇娇目中突然现出了凶恶的光芒,一字字道:``我为的是那罗九兄弟!''小鱼儿恍然道:``我知道了,他兄弟本和你有些仇恨。''屠娇娇道:``我此番出谷,除了找你之外,还一心要找两个人。''小鱼儿道:``你要找的,就是他们?''

屠娇娇也不回答,只是缓缓接着道;``二十年前,十大恶人中,有五个被逼恶人谷,那时情形十分危急,他们走得十分仓促,所以有许多重要的东西,都来不及带走。''小鱼儿点头道:``不错,你和李叔叔、杜叔叔等人,纵横江湖多年,自然不会是身无长物,而能被你们瞧得上眼的东西,自然也必定珍贵得很。''屠娇娇道:``你知道,我们在江湖中根本没有朋友,只有十大恶人中另外那五个人,勉强可算是和我们臭味相投。''小鱼儿微笑道:``这点我当然清楚得很。''

屠娇娇道:``所以,我们只有将东西交给他们,但那狂狮铁战总是疯疯癫癫,发起疯来时,连自己的命都可以不要,何况是别人交给他的东西,那损人不利己白开心非但靠不住,而且又和李大嘴是对头。''小鱼儿笑道,若是交给恶赌鬼轩辕三光,又怕他输光。``屠娇娇忍不住也笑道:''是呀,这恶赌鬼虽然赌了一辈子,虽然自命赌得比谁都精,但还是常常输得几乎连裤子都没有,总是等到天光、人光、钱也光时才肯罢手,他那轩辕三光的名字,正也是出此而来的。``小鱼儿笑道:''常言道:久赌神仙输,何况他还只不过是个赌鬼而已,还够不上神仙的资格,又怎么能不输。``屠娇娇道:''那时,大家本决定要将东西交给迷死人不赔命的萧咪咪,但她却又偏偏不知躲到哪里去丁,我们竟找她不着。``屠娇娇又接着道,''所以我们想来想去,只有将东西交给那欧阳兄弟。``小鱼儿道:''依我看,这兄弟两人更靠不住,这兄弟既然连拚命都要占人便宜,你们将东西交给他们,岂不是送羊入虎口。``屠娇娇苦笑道:''那时我们虽也想到这点,但这欧阳兄弟平生最怕的就是从不爱占人便宜只爱杀人的血手杜杀,所以咱们便认为他们绝不敢将东西吞没的,谁知这两兄弟一打算盘,想到血手杜杀既已逃到恶人谷不敢出头,为何还要怕他,竟真的将东西吞没下去了。``小鱼儿道:''所以你一出谷,就找他们。"

屠娇娇道:``正是!''

小鱼儿眨着眼睛道;``那欧阳兄弟莫非和罗九兄弟有什么关系不成?''屠娇娇一字字道:``罗九兄弟,就是欧阳兄弟!''小鱼儿失声道:``难怪他们手段那么毒辣,我早巳疑心他们的来历绝不寻常\ldots\ldots 不过,据我所知,他们和那欧阳兄弟长得一点也不像呀''屠娇娇道:``这些年来,他们故意将自己养得又肥又胖,整个人都像是肿了起来,他人本来比鬼还瘦,这一发起胖来,连脸上的样子都变了,简直没有人再认得出他们,这兄弟当真比谁都精,竟想出了个最好的易容之法。''小鱼儿拍手道:``不错,用这天生出来的一身肥肉来易容,当真是再好不过,他们想出来的这法子,当真妙绝天下!''屠娇娇道:``所以,我就将他们选来的一个傻丫头,拖出去宰了,再扮成傻丫头的模样,他们果然没有瞧出来,但我却瞧出了他们的破绽,早已瞧出他们就是欧阳兄弟,只是我若立刻揭穿,既怕他们跑了,又怕他们不肯说出那批东西的下落。''小鱼儿道:``所以,你还要等到查出那批东西的下落后再动手。''屠娇娇道:``本来我虽不知道那痴痴呆呆的少女就是慕容九,但已觉得她有些奇怪了,所以我在闲着无聊时,就早巳照着她的脸做了副面具,否则在方才那么短的时间里,我手边什么都没有,又怎能扮成她的模样。''小鱼儿眼珠子转动,突然冷笑道:``你做成这面具,只怕并不是为了闲着无聊吧。''屠娇娇笑道:``那么,你说我是为了什么呢?''小鱼儿道:``你本想在必要时,将她也宰了,扮着她的模样,那罗九兄弟更不会提防于她,你要查什么事,也就更容易了。屠娇娇笑道:''究竟是你这小鬼聪明,我的心意也只有你猜得中。"

\hypertarget{ux7b2cux4e94ux5341ux516bux7ae0-ux5929ux964dux602aux5ba2}{%
\chapter{第五十八章
天降怪客}\label{ux7b2cux4e94ux5341ux516bux7ae0-ux5929ux964dux602aux5ba2}}

小鱼儿道:``你这主意打得虽妙,谁知慕容九竟被我带走了,你要这面具也无用,所以乐得做个顺水人情,用它来救了我。''屠娇娇笑道:``我一瞧是你,就知道你必定又在弄鬼,所以时时刻刻都要留意着你,今天早上,你和那黑蜘蛛来叫慕容九写信,我就听到了。''她娇笑着接道:``若不是我在外面为你们把风,只怕今天早上你们就被那欧阳兄弟撞破了。''小鱼儿心里吃了一惊,面上却笑道:``就算被他们撞破,也没什么关系。''屠娇娇笑道:``你倒真是死不领情。''

小鱼儿道:``你就是听到了那封信,所以才知道我们晚上会到那祠堂里去\ldots\ldots{}''屠娇娇道:``除此之外,我还遇见了一个人。''小鱼儿失声道:``白开心?''

屠娇娇笑道:``你在手上搓泥丸子时,我已瞧见了。''小鱼儿喃喃道:``奇怪,你就在附近,我怎会听不见?''屠娇娇笑道,``以你现在的能耐,本是应该听得见的,只不过那时白开心正面对着我,我早已和他悄悄打了个手式,叫他故意大叫大喊,分散你的注意力,何况你那时心里正在得意,又怎会留意别的。''

小鱼儿苦笑道:``看来一个人无论在什么时候,都不该太得意的。''话声微顿,突又失笑道:``难怪白开心方才竟不向我要解药,原来你早巳告诉他那不过是泥丸子,他吃了我手上的泥,自然要害我一害来出气了。''屠娇娇笑道:``这件事若不是样样凑巧,又怎会便宜了你。''小鱼儿正色道:``这件事看来虽然凑巧,其实也不完全是凑巧的,每件事都有前因后果,这样的结果正是再合理也没有。''屠娇娇笑道:``算来算去,只苦了江别鹤。''

小鱼儿大笑道:``要害人,自然就要害他这样的人才有意思,若是去害个老老实实的规矩人那倒不如坐在家里数手指头算了。''屠娇娇沉思着点了点头,微微道:``这话倒也有道理,害坏人确实比害好人有趣得多,绝不敢宣扬出去,何况,就算别人知道你害了他,也只有佩服你,没有人会找你算帐的。''小鱼儿笑道:``所以,你若学我,只害坏人,不害好人,这样既可过足害人防瘾,又不必躲躲藏藏怕人找上门来算帐,岂非又风光、又体面、又上算。''屠娇娇吃吃笑道:``上算的事,当真都被你这小鬼一个人做尽了。''小鱼儿道:``但我还是想不到你怎会离开恶人谷的。''屠娇娇又叹了口气,道:``天下有许多事,都是想不到的。''这同样的一句话,她竟说了两次,而且每说这句话时,竟都忍不住要长叹口气出来。

小鱼儿心念一动,道:``莫非恶人谷里,竟发出了什么令人意想不到的变故不成。''屠娇娇长叹道:``的确严重得很。''

小鱼儿着急道:``究竟有什么事,你快说呀\ldots\ldots{}''屠娇娇缓缓道;``你可知道\ldots\ldots{}''

突听``嘶''的一声轻响,一条人影,自树梢飞来,大声道:你们原来在这里,却找得我好苦。"来的这人,正是黑蜘蛛。

黑蜘蛛长叹道:``我险些连你们的人都瞧不见了。''小鱼儿这才发现他那一身比缎子还亮的黑衣,此刻竟满是泥污,头发也零乱不堪,不禁失声道:``你怎会变得如此模样?''黑蜘蛛道:``我去送那信时,只见南宫柳屋里一个人也没有,于是我就悄悄进去,将信放在桌上\ldots\ldots{}''他话末说完,小鱼儿已顿足道:``你为何要走进屋,将那封信抛下去不就成了么?他们的贴身丫头都被人家宰来吃了,对自己的居处又怎会不分外警戒?''黑蜘蛛苦笑道:``我正是太大意了些,刚将信放在桌上,就突然有条长鞭卷来,将信卷了过去,我知道不妙,想夺路而走时,门窗已全被人堵住了。''小鱼儿叹道:``他们故意将那屋子空着,正是要诱你进去上当的,否则那南宫柳和幕容双住的屋子,会容人大摇大摆的来去自如么?''黑蜘蛛又接着道:``我当时一惊之下,便要冲出去,谁知那些人竟无一弱者,暗器尤其佳妙,我非但冲不出去,反而眼看就要受伤被制。''``慕容家的暗器,果然是名下无虚。\ldots 但你既自他们包围中冲出来,岂非比他们还要强得多。''黑蜘蛛长叹道:``若凭我一人之力,哪里能冲得出来。''小鱼儿讶然道:难道还有人帮你的忙不成?``黑蜘蛛道;''我正眼见不敌,突然有个人飘了进来,顾人玉家传神拳,武功可算不弱,但被这人袍袖轻轻一拂,就直跌了出去!``小鱼儿失声道:``这人武功竟如此厉害。''

黑蜘蛛叹道:``此人武功之高,当真是我平生未见,我简直连做梦都未想到世上竟有武功如此厉害的人。''小鱼儿动容道:``连你都服了他,这倒难得得很。''黑蜘蛛道:``这人袍袖拂了拂,就将暗器全都反射出去。力道竟比他们用手发出来时还强,他们大惊闪避时,这人已带着我掠了出去。''他苦笑接着道:我竟被他夹在肋下,动都动不得,只见他身子轻轻一纵,便凌空飞出去七八丈,就好像腾云驾雾似的。``小鱼儿笑道:''你简直越说越神了,世上哪有轻功如此高明的人。``黑蜘蛛沉声道:''非但你此刻不信,就连我虽亲眼瞧见,都几乎有些不相信自己的眼睛,但你不妨想想.这人武功若非大得吓人,能将我夹在肋下吗?``小鱼儿叹道:''不错,能将你夹在胁下的,世上简直不可能有这样的人。``屠娇娇听到这里,竟也忍不住道:''他长得是何模样?``黑蜘蛛道:''这人身材并不高大,但却有无穷的力量,我被他夹了盏茶时刻,竟是全身麻木连动都动不得了。``屠娇娇听得这人''身材并不高大``,已松了口气.小鱼儿却追问道:''他的脸呢?``黑蜘蛛道;''他脸上戴着个狰狞丑陋的青铜面具,一双眼睛更是说不出的鬼气森森,我素来自命胆大包天,但瞧了他一眼,手心竟不觉直冒冷汗。"小鱼儿也不禁被他说得寒毛悚然,全身都凉风飕飕,像是要打冷战。

黑蜘蛛道:``他夹着我奔上座小山,又掠上株大树,才放在一根树桠上,我全身麻木,动也动不得,也根本不敢动,生怕一动就要掉下来。''小鱼儿道:``他呢?''

黑蜘蛛道:``他自己也坐在一枝树枝上,冷冷的瞧着我,也不说话,那树枝柔弱不堪,连婴儿都能折断,他坐在上面,却似舒服得很。''小鱼儿叹道:``这倒的确是个怪人\ldots\ldots 莫非武功特别好的人,都有些怪毛病。''屠娇娇笑道:``那么你想必就要倒霉了。''

黑蜘蛛道;``的确如此,他等了半天,又点了我两处穴道,竟将我留在那棵大树上,袍袖一展,已走得瞧不见影子。''说到这里,突然像是想起了什么,瞪着屠娇娇道:``慕容姑娘神智已恢复了么?''屠娇娇格格笑道:``我神智恢复了么\ldots\ldots 我也不知道呀?''突然转身,飞也似的走了。

黑蜘蛛还想追,小鱼儿已拉住了他笑道:``你让她走吧,你且莫管她,先说说你在那树上的事吧。''黑蜘蛛目中满是迷悯,呆了半晌,终于接着道:``那时风越来越大,将我的身子吹得直摇,树枝也像是快断,我连根手指都动不了,当真是提心吊胆。''小鱼儿道;``后来你是怎么从树上下来的呢?''黑蜘蛛苦笑道:``我心里正在想着报仇,那人竟已来了,而且像是看透了我的心思,突然问我:你可是想报仇么?''小鱼儿笑道;``你心里在想什么,我也能瞧得出来,你嘴里就算不说话,仅那双眼睛却已将什么都说出来了。''黑蜘蛛道:``我被他说破了心思,就更是狠狠的瞪着他,心想就算被他踢下来,也比在树上活受罪的好,谁知他竟反而笑了,又道.我救了你的性命,你不先想该如何报恩,就想如何报仇么?''小鱼儿笑道;``这句话倒也问得妙极。''

黑蜘蛛道:``当时我也被他问住了,仇固然要报,恩也是要报的,我老黑怎能做忘恩负义之徒,只是他武功既然那么高,我非但无法报仇,简直连报恩也不知该从何报起,这报恩有时实比报仇还困难得多。''小鱼儿道:``你这番心意只怕又被他瞧破了。''黑蜘蛛又叹道:``果然是被他瞧破了,我还未说话,他已说道:你不知该如何报恩,是么?我哼了一声,他又道:你能替别人送信,难道就不能替我送信?我忍不住问道:我替你送了信,就算报了恩么?他居然点了点头,取出封信,叫我送给\ldots 你猜送给谁?''小鱼儿道:``这我倒猜不透了。''

黑蜘蛛道;``他竟要我将信去送给花无缺。''

小鱼儿眼睛发亮,笑道:``这倒真的越来越有趣了,他和花无缺又有何关系?为何要你为他送信,他自己明明可以直接和花无缺说话的呀。''黑蜘蛛道:``也许他不愿和花无缺见面。''

小鱼儿道:``他就算不愿和花无缺见面,以他那样的轻功,就算将信送到花无缺的床头,花无缺也是不会发觉的。''黑蜘蛛突然又道:``也许他只是知道我无法报恩,所以想出这件事来叫我做。''小鱼儿沉吟道;``这倒有可能,像他那样的怪人,的确可能会有这种怪念头,你固然不愿欠他的情,他可能也不愿让别人欠他的情\ldots。''黑蜘蛛道:``正是如此,我不欠人,自也不愿别人欠我,彼此各不相欠,日子过得才舒服,我若知道有人一心想报我的恩,我也会难受得很。''小鱼儿笑道:``如此说来,你两人脾气倒是同样的古怪了,这就难怪他会救你\ldots\ldots 但那封信上写的是什么,你可瞧见了么?''黑蜘蛛怒道:``我老黑难道还会偷看别人的信么?他解开我的穴道后,我立刻就将信送给花无缺,连信封上写着什么,我都未去瞧一眼。''小鱼儿笑道:``你果然是个君子,但花无缺瞧过那封信后,总该说了些话吧。''黑蜘蛛道:``就是因为他瞧过信后,说的话十分奇怪,所以我才急着找你。''小鱼儿立刻追问道:``他说了什么?''

黑蜘蛛道:``他说:我与江别鹤相识虽不久,但却已相知极深,又怎会被别人谣言中伤,就认为他是恶人,这位前辈也未免过虑了。''小鱼儿皱眉道:``那怪人却又是江别鹤的什么人?为何要这样帮江别鹤的忙?''黑蜘蛛道:``花无缺说了这番话后,我正想问他:这位前辈是谁?''谁知他已先问我:你已瞧见了这位前辈,真是福气,却不知他老人家长得是何模样,脸上是不是真的戴着青铜面具?``小鱼儿道:''花无缺既然没有见过他,又怎会听他的话?``黑蜘蛛道:''我本来也觉得奇怪,移花宫主巳嘱咐他,要他日后若遇见─位铜先生,就万万不能违抗这人的话,无论铜先生说什么,他都必须听从。``小鱼儿道:''原来那怪人叫``铜先生,这名字倒真和他一样古怪!''黑蜘蛛道:``移花官主还说,这铜先生乃是古往今来江湖中第─位奇人,武功更是高绝天下,移花宫主竟说她自己比起这铜先生来,都要差得多。''小鱼儿动容道:``移花宫主那么高傲的人,也会说这样的话么?若连移花宫主都对他如此服气,这铜先生的武功倒的确是可怕得很了。''黑蜘蛛道;``但花无缺既对那铜先生言听计从,日后对江别鹤必定更要帮忙到底,有他那样的人帮江别鹤的忙,也够你头疼的了。''小鱼儿淡淡一笑,道:``那倒没什么关系。''

黑蜘蛛瞪着眼瞧了他半晌,突然道:``再见,我的恩虽已报过,仇却还未报哩!''小鱼儿失声道:``你要去找那铜先生报仇?''黑蜘蛛冷冷道:``不行么?''

小鱼儿道:``但\ldots\ldots 但他的武功\ldots\ldots{}''

黑蜘蛛怒道:``他武功强过我,我就不去报仇了么?我老黑难道是欺善怕恶的人?''他一面大喊大叫,人已飞掠而去.现在,小鱼儿心里又多了三样解不开的心事。

第一,那真的慕容九到哪里去了。''

第二,恶人谷"究竟发生了什么惊人的事?

第叁.那``铜先生''究竟是何许人也?和江别鹤又有什么关系?为什么定要说江别鹤是个好人?

这时天已大亮,小鱼儿巳将脸上面具弄了下来,大白天,他可不愿以李大嘴的面目见人。

大路上行人已渐渐多了起来,但十个中倒有九个多是自西往东去的,而且看来大多是江湖朋友,有的袖子还系着黑布,一个个面上都带着兴奋之色,嘴里嘀嘀咕咕也不知在说些什么。

小鱼儿心中正觉奇怪,就在这时,突然有一辆形式奇特、装饰华丽的马车,自道旁驶来,骤然停在小鱼儿面前。

车门打开,一个人探出头来,道:``快上来。''日光照着她的脸,她容貌清秀,但皮肤看来却甚是粗糙,正是那改扮成慕容九的屠娇娇,小鱼儿跳上马车,只见车厢里装饰得更是华丽,坐垫又厚、又柔软、又宽大,坐上去舒服得很。

小鱼儿忍不住笑道:``你倒真是神通广大,又从哪里变出这么辆马车来了?''屠娇娇也不回答,却反问道:``我等了你好半天,你怎地到此刻才出来,你和那黑蜘蛛,究竟有些什么事好说的。''小鱼儿笑道:``我们在谈论着一位铜先生,你可听见过这名字?''屠娇娇失声道:``救他的那怪人就是铜先生?''小鱼儿道:``你知道这人?''

屠娇娇像是怔了怔,但立刻就大声道:"我不知道这人,我从未听说过这名字。

\hypertarget{ux7b2cux4e94ux5341ux4e5dux7ae0-ux60caux4ebaux4e4bux53d8}{%
\chapter{第五十九章
惊人之变}\label{ux7b2cux4e94ux5341ux4e5dux7ae0-ux60caux4ebaux4e4bux53d8}}

小鱼儿见屠娇娇提到铜先生时,说话吞吞吐吐,闷在心里,也不再追问,只见这辆大车也是由西往东而行,正和那些江湖朋友所走的方向一样。

他忍不住道:``这些人匆匆忙忙,是要去干什么的?''屠娇娇道:``瞧热闹,天下武功最高的门派弟子,和江湖中地位最高、势力最大的一个集团斗法,你说这热闹有没有趣?''小鱼儿眼珠子一转,道:``莫非是花无缺和慕容家的姑爷们?''屠娇娇道:``南宫柳和秦剑去找江别鹤算帐,花无缺却一力保证江别鹤是清白的,双方相持不下,只有在武功上争个高低了。''小鱼儿眼睛发亮笑道:``这场架打起来,倒当真是有趣得很,不过,这件事今天凌晨才发生的,怎地已有这么多人知道了?''屠娇娇笑道:``这只怕就是江别鹤叫人去通知他们的,他算定自己这面有了花无缺撑腰,必胜无疑,自然要多找些人去看热闹。''小鱼儿叹道:``不错,慕容家虽强,但比起花无缺,还要差一些\ldots\ldots 这世上难道就真的没有人能对付花无缺么?''屠娇娇含笑瞧着他,道:``只有你。''

这问题实在不愿意再谈下去,幸好此刻正有个他不愿意谈的问题,他眼珠子一转,立刻改口道:``你方才的话被黑蜘蛛打断了,恶人谷里究竟发生了什么大事?''屠娇娇叹了口气道:``你可记得谷里有个万春流?''小鱼儿笑道:``我怎会不记得,小时候,他天天将我往药汁里泡,泡得我头晕脑胀,我现在揍人的本事是未见得如何,挨揍的本事却不错,正是他将我泡出来的。''屠娇娇道:``你可记得万春流屋里,有个人叫药罐子?''小鱼儿心里吃了一惊,面上却不动声色,笑道:``我自然也是记得的,他吃的药比我还多,万春流只要采着一种新的药草,总是先让他尝尝的。''屠娇娇眼睛盯着他的脸,一字字道:``十个月前,万春流和这药罐子,都失踪了!''小鱼儿一颗心几乎要跳出腔子外来,但你就算鼻子已贴住他的脸,也休想瞧出他脸上肌肉有一些颤动。

他只是淡淡一笑,通;``这又算得什么大事,你们穷紧张些什么?''屠娇娇也笑了笑,道:``你可知道那药罐子是谁?''小鱼儿茫然睁大了眼睛,道:``谁?''

屠娇娇道:``你可听说过,昔日江湖中有个人,他一剑挥出,可以令你在十丈外能感觉出他的剑风,也可以将你的胡子头发都削光,而你却一点也感觉不到。''小鱼儿笑道:``这人我听说过,他好像是叫燕南天,是么?''屠娇娇叹道;``除了燕南天,哪里还有第二个。''小鱼儿道:``但他岂非早巳死了?''

屠娇娇道:``他没有死!他就是那药罐子!''

小鱼儿故意失声道;``药罐子竟然就是天下剑法最强的燕南天,这倒真是令人想不到的事,但燕南天剑法若是真的那么高,又怎会变成那种半死不活的模样?''屠娇娇叹道:``这还不是为了你的缘故,咱们为了要从他手上将你救下来,所以才不得已而伤了他。''她说的居然活灵活现,小鱼儿若非早巳听万春流说起过这件事的秘密,此刻只怕真要相信她的话了。

他暗中叹了口气,忖道:``燕南天虽是我的恩人,虽是大侠,但却和我毫无情感,你们虽是恶人,但这么多年来,已和我多少有了些感情,我怎忍心为了他而找你们复仇,你们又何苦还要骗我!''严格说来,小鱼儿虽不能算是个十分好的人,但却是热血澎湃、感情丰富、表面虽硬、心肠却软得很的人。

小鱼儿心里叹着气,面上却笑道:``为了我?他又和我有什么关系?''屠娇娇道:``这件事说来话长,以后慢慢再说吧,只要你记住,咱们是为你得罪了燕南天,燕南天此番一走,咱们就连恶人谷也不敢耽下去了。''小鱼儿道;``为什么?''

屠娇娇道:恶人谷虽被江湖人视为禁地,但燕南天若要闯进来时,天下又有谁拦得住他。他上次已上过了一次当,这次必定更加小心。``她狡黠而善变的眼睛里,竟也露出了恐惧之色,长叹着接道:''这次他再来时,咱们这些恶人,只怕就要都变成恶鬼了小鱼儿目光闪动,道:``你想\ldots\ldots 他武功难道又恢复了么?''屠娇娇恨恨道:``他武功现在纵末恢复,但那万春流想必已试出某种药草可以治愈他的伤,否则又怎会带他逃出恶人谷去!''小鱼儿悠悠道:``但也许此刻已治好了,是么?''屠娇娇身子竟不由得一震,盯着小鱼儿道:``你希望他现在已治好了!''小鱼儿神色不动,缓缓道:``虽不希望如此,但无论什么事,总得先作最坏的打算才是。''屠娇娇默然半晌,终于叹道:``不错,说不定他此刻武功早已恢复了,说不定他现在已经在找咱们\ldots.''眼睛转向车窗外,再也打不起精神说话。

车马越走越快,赶车的皮鞭打得``□啪''直响,似乎也急着想去瞧瞧那一场必定精采万分的龙争虎斗。

三面低坡下,有个小小的山谷,这时山坡上已高高低低站着几百个人,甚至连树桠上都坐着人。

车马停在山谷外,小鱼儿也瞧不见山谷里的动静。

只听人声纷纷议论着道:``那看来斯斯文文的弱书生,难道就是移花宫的传人么?我真瞧不出他能有多么高的武功。''``据说当今江湖上,武功没有人能比得上他,甚至连江大侠都对他佩服得狠,这话不知是真是假。''有人叹道:``他年纪轻轻,武功既是天下第一高手,人又生得那么漂亮,普天之下,只怕谁也比不上他了。''议论纷纷间,尽是一片赞美羡慕之声,小鱼儿听得一肚子闷气,屠娇娇瞧着他微微笑道:``你听了这活,心里可是有些不舒服?''小鱼儿瞪着眼道:``谁说我不舒服,我舒服极了。''屠娇娇大笑道:``他虽是天之骄子,但咱们的小鱼儿却也不比他差,末来的江湖中,只怕就是你两人的天下了。''小鱼儿突然推开了门,道:``我可要去瞧热闹了,你呢?''屠娇娇道:``你去吧,我就在这里等着,不过\ldots。你却要为我做件事。''小鱼儿道:``什么事?''

屠娇娇道:``设法子去把那欧阳\ldots 罗九兄弟,弄到这车上来,你可能办得到。''小鱼儿笑道:``只要你这车子够大,我就算要把山谷里的人全都弄上拿来,也简单得很。''他跳下车子大步而去,突然转头盯了那赶车的一眼,那赶车的正摸着颔下的一摄络腮胡子,瞧着他嘻嘻的笑。

小鱼儿毫不费事地就挤进了人丛,爬上山坡。

山坡上,百棵大树,坐在上面,正可纵观全局,只可惜,此刻上面已坐满了人,小鱼儿眼珠子一转,突然摇头,叹道:``真奇怪世上竟有这么多不怕死的人,竟敢坐在毒蛇穴上,若被毒蛇在屁股上咬一口\ldots\ldots{}''他话未说完,林上的人已吓得跳了下来,乱了一阵,却发现方才叹气说话的人,已舒舒服服的坐在树上了。

这些人忍不住道:``喂,朋友,你说这株树是个蛇穴,自己怎敢坐上去。''小鱼儿笑嘻嘻道:``哦?我方才说过这话么?''那些人又惊又怒,却听小鱼儿喃喃又道:``有江南大侠与慕容家的姑娘们在这里办正事,若想在这里乱吵,那才是活得不耐烦了哩。''那些人面面相觑,只得忍下了一肚子火,有些人又爬上了树,挤不上去的也只好自认晦气。

只见山谷内的空地上,停着辆马车,那花无缺正悠闲地靠着车门,似乎正在和车厢里的人说话。

江别鹤却坐在他身旁一块石头上,也不住的和四面瞧热闹的人微笑着打招呼,看不出丝毫``大侠''的架子。

小鱼儿也瞧见了那``罗九''兄弟,这两人又高又胖,站在人丛里,比别人都高出一个头。

但慕容家的人却连一个也没有来,四面的江湖朋友已开始有些不满,都是觉得他们的架子实在太大。

花无缺看来丝毫不着急,面上的笑容也非常愉快,每当他眼睛望进车厢中去时,那一双锐利的目光,也变得分外温柔。

小鱼儿不禁捏紧了拳头,心里说不出的别扭:``车厢里的人是谁?难道花无缺真的和铁心兰寸步不离,将她也带来了?''突见人群一阵骚动,十二个身穿黑衣、腰束彩带的彪形大汉,抬着三顶绿呢大轿奔了进来。

每顶大轿后还跟着顶小轿,轿上坐着的是三个明眸妩媚的俏丫头,轿子停下,三个俏丫头下了小轿,掀起大轿的门,大轿里便盈盈走出三个艳光照人的绝代佳人来。

这三人正是慕容双、慕容珊珊和``小仙女''张菁,三个人今天都是宫鬓华服,刻意修饰过,就像是高贵人家出来作客的大小姐少奶奶似的,哪里像是要来与人争杀搏斗的女中豪杰、江湖高手。

在山坡上等着瞧热闹的江湖朋友,大多人只闻慕容九姐妹的声名,仅见过她们真面目的,却少之又少,此刻但觉眼睛一亮,十个人中,倒有九个惊得呆住了,就连小鱼儿都几乎瞧不出那文文静静地走在最后的大姑娘,便是昔日跃马草原,瞪眼杀人的小仙女。

花无缺的眼睛,果然已从车厢里移到她们脸上,他那眼神与其说是赞赏,倒不如说是惊奇还恰当些。

慕容珊珊,莲步轻移,走在最前面,裣衽笑道:``贱妾等一步来迟,有劳公子久候,还请恕罪。''她说的是这么温柔客气,花无缺又怎会在女子面前失礼,立刻也长长一揖,躬身微笑道:``不是夫人们来迟,而是在下来得太早了。''慕容珊珊笑道:``今日天气晴朗,风和日丽,风雅如公子,自当早些出来逛逛的,只恨贱妾等俗务羁身,不能早来奉陪。''两人嫣然笑语,竟真的像是早巳约好出来游春的名门闺秀和世家公子似的,哪里瞧得出有丝毫火气。

只听花无缺道:``南宫公子与秦公子只怕也快要来了吧。''慕容珊珊笑道:``他们家里有事,已先赶回去了。''慕容双接口道:``慕容家的事,向来是不容外人插足的。''花无缺又呆住了,道:``但\ldots\ldots 但夫人们岂非\ldots,''慕容双笑道:``我姐妹虽是他们的妻子,但妻子的事,也是和丈夫无关的,我慕容姐妹,又怎会嫁绘个爱管妻子闲事的丈夫。''慕容珊珊笑道:``公子只怕也不愿娶个爱管丈夫闲事的妻子吧。''这姐妹两人你一句,我一句,竟将花无缺说得呆在那里,作声不得,小鱼儿却暗笑忖道:``谁娶了慕容家的姑娘做妻子,果然是好福气,明明是南宫柳与秦剑自己不敢和花无缺动手,但被她们这一说,就非但丝毫不会损了他们的名声,人家反要称赞他们真是个善体人意的好丈夫哩。''只是,他们既放心肯让自己的爱妻前来,想必是深信她们有致胜的把握,小鱼儿不赞又在暗中猜测!

江别鹤也真沉得住气,直到此刻,才微笑着道:``南宫公子与秦公子若不来,此事岂非无法解决了么?''慕容双眼随转到他身上,脸上的笑容立刻不见了,瞪眼道:``谁说无法解决?''花无缺乱咳一声,苦笑道:在下又怎能与夫人们交手?``慕容珊珊笑道:''公于若不愿和贱妾等交手,就请公子莫要再管贱妾等与江别鹤之间的事,江别鹤又不是孩子了,难道还不能料理自己的事么?"她笑容虽温柔,但话却说得比刀还锋利,群豪听了都不禁耸然失笑,只道江别鹤无论如何,都是忍不下这句话的。

谁知江别鹤还是声色不动,微笑道:``江湖朋友都知道,在下平生不愿出手伤人,何况是对夫人们?更何况只是为了些小误会。''慕容双大声道:``江别鹤,你听着,第一,这绝不是误会!第二,你也未必能伤得了我们,你只管出手吧!''江别鹤淡淡笑道:``这件误会暂时纵不能解开,但日久自明,在下此刻又怎能向夫人抡拳动脚,夫人就算宰了在下,在下也是不能还手的。''这句话说的更是漂亮已极,群豪闻言有的也忍不住喝起彩来,就连小鱼儿也不禁在暗中赞叹:``普天之下,对付人的本事,只怕是谁也比不上江别鹤的,尤其是这种场合里,才显得出他的本事。''慕容双大喝道:``你明知花公子不会让咱们宰了你,所以才故意说这种漂亮话。''突听一人大喊道:``至少江大侠绝不会自己溜回家去,却让老婆出头来和人家吵架。''小鱼儿瞧得清楚,这呼喊的正是那化名罗九的欧阳丁,慕容姐妹却瞧不见他,也不知说话的是谁。

她们索性装作没有听见,心里却知道不能再和江别鹤说下去了,双方手段既然差不多,索性彼此包涵几分还好些。

小仙女突然大声道:``这样说来说去,是非黑白,还是分不清,不如还是动手吧,就让我来领教花公子高招如何?''花无缺上下瞧了她一眼,笑道;``你想我能和你动手么?''慕容珊珊笑道:``花公子想来定然是不肯和妇女之辈动手的了。''花无缺笑道:``在下若是不慎,乱了夫人们的容妆,已是罪过,何况真的与夫人们动手。''慕容双大声道:``此事必须解决的,公子若没有法,我倒有一个。''花无缺道:``请教。''

幕容双道:``贱妾等说出三件事,公子若能做到,贱妾等便从此不再寻这江别鹤,但公子若无法做到,便请公子莫再管江别鹤的事!''所到这里,小鱼儿恍然大悟,秦剑与南宫柳故意不来,慕容姐妹故意如此打扮,正是要拘住花无缺不能真的出手,她们才好拿三件事来难住花无缺,只要花无缺一上当,这一仗便算输了!

但花无缺却也不是呆子,微一沉吟,笑道:``夫人说出的三件事,若是根本无法做到的又如何?''小仙女大声道:``这三件事说出后,你若无法做到,咱们就做出来让你瞧瞧,这样总该算是公平得很吧。''慕容珊珊道:``这三件事自然是不分男女,人人都能做到的,贱妾等只不过是想领教领教公子的武功与智慧而已。''花无缺笑道:``若是如此,在下便从此退出江湖。''小鱼儿早已算定慕容姐妹说出的那叁件事必定是百灵精怪、极尽刁钻之能事,此刻不禁暗笑道:``花无缺呀花无缺,你一答应,只怕就要上当了!她们挖空心思想出来的事,连我都只怕未必能做到,何况你!''需知花无缺那句话说得虽轻松,但``退出江湖''四字,份量却实在太重,他此刻声名正如日之方升,此后数十年的江湖生涯,必定多彩多姿,绚丽无比,但他今日若输了,这一生便将默默以终,是以他自己虽然充满自信,旁边瞧热闹的人却不禁为他紧张起来,只见慕容姐妹悄悄商议了一阵。

慕容双终于笑道:``贱妾等要公子做的第一件事,便是请公子以金鸡独立姿式站着,然后再令人来推,若是推不倒公子,公子便算赢了。''花无缺笑道:``但不知夫人要多少人来推呢?''慕容双眼波一转道:``随便多少人!譬如说,两百个吧!''花无缺略一沉吟,竟含笑道:``好,就是如此。''这句话说出来,群豪又不禁耸然动容,两百个人加在一起,那力量何等巨大,纵然两百条普通壮汉,加起来的力量也绝非花无缺一个人所能抵挡的,何况也还要以金鸡独立的姿式站着。

``这件事有什么稀奇,只要花些脑筋,任何人都能做的,你只要贴着山壁而立,莫说两百人,就算两万人也是推不倒你的。''小鱼儿只当花无缺也想通了这点,谁知他并不走向山壁,竟在空地上:就曲起一腿微微笑道:``在下数到三时,夫人便可令人来推了。''慕容姐妹交换了个眼色,目中都不禁露出欣喜之色,齐声道:``遵命。''这时山谷外几百个,包括小鱼儿在内,都以为花无缺输定了,有的人甚至已在叹息。

以花无缺之武功而论,百十壮汉,的确不是他的敌手,但这种硬拼力气的事,却毫无技巧可言,既不能惜力使力,也不能躲让闪避,别人有一百斤力气推来,你也必须要一百斤力气能抵挡。

只听花无缺道:``一、二、三\ldots\ldots{}''数到``三''字,他踏在地上的一只脚,竟突然下陷了半寸,那坚硬的石地在他脚上,竟变得像是烂泥似的。慕容珊珊瞧得心里暗吃一惊,挥手道:``花公子已准备好了,你们还等什么?''抬轿的十八条彪形大汉,立刻快步奔来,他们显然是早经训练,奔行之中,第二人的手已搭上第一人的肩头,第三人搭上第二人的\ldots\ldots 十八个人脚步越来越快,冲向花无缺,推了出去。

这─推之力,非但聚集了这十八个人本身的力量,还加上他们的冲力,力量之大,可以想象。

\hypertarget{ux7b2cux516dux5341ux7ae0-ux5929ux4e4bux9a84ux5b50}{%
\chapter{第六十章
天之骄子}\label{ux7b2cux516dux5341ux7ae0-ux5929ux4e4bux9a84ux5b50}}

谁知那十八条大汉一推之后,花无缺非但未曾跌倒,连后退都没有后退,他身子竟又往下陷落了几寸。

十八条大汉用的力量越大,他身子也就住下陷得越快,十八条大汉满头汗珠滚滚而落,用尽了全身力气。

花无缺身子竟已下陷了两尺,半条腿都已没入石地里,但他面上却仍带着微笑,竟似没有花丝毫力气,就好像站在流沙上似的。

群豪如瞧魔法,瞧得目瞪口呆,几乎以为自己眼睛花了──他脚下站着的难道不是真的石地面是流沙。

小鱼儿也瞧得呆了。

花无缺用的这法子虽然比他所想的要笨得多,也困难得多,但这样的法子却只有更令人吃惊更令人佩服。

小鱼儿想了想,连自己也不知道究竟是花无缺所用的这法子聪明,还是自己所想的那法子聪明了。

只见花无缺身子下陷已越来越慢,显然是那十八条大汉推的力量也已越来越微弱。

到后来花无缺不再下陷时,那十八条大汉突然跌倒在地,竟已全身脱力,再也站不起来了。

花无缺竟已用``移花接玉''的功夫,巧妙地转变了他们的方向,他们的力量本是往后退的,但经过花无缺的转变后,已变成向下压了,是以他们看来虽是在推花无缺其实却无异在推那地面。

群豪自然不懂其中的巧妙,但越是不懂,对花无缺的武功就越是惊讶佩服,终于忍不住暴雷般的喝起彩来。

慕容姐妹面上也不禁变了颜色,只听花无缺微笑道:``夫人们还要另找他人来推么?''慕容珊珊强笑道:``公子神通果然不可思议,贱妾佩服得很。''小仙女撇了撇嘴,大声道:``这第一件事就算你能做到,还有第二件呢。''花无缺微微一笑,身子自地拔起,有风吹过,他那条腿上所穿的半截裤子,立刻化为蝴蝶般随风而去。

群豪喝彩声历久不绝,等到喝彩声过后,那车厢里还在响着清脆的掌声,小鱼儿听得一颗心立刻绞了起来。

他虽然不得不承认花无缺的武功,确实值得``她''拍掌的,只是他想到这一点,却不免更是难受。

花无缺已微笑道:``那第二件事是什么,还请夫人吩咐。''慕容珊珊眼珠一转,笑道;``安庆城里,有家专售点心的馆子,叫小苏州,不知公子可知道么?''花无缺微笑道;``江兄曾带在下去尝过几次。''慕容珊珊道;``这小苏州所制的八宝饭、千层糕,甜而不腻,入口即化,当真可说是妙绝天下。''花无缺笑道:``在下虽然对此类甜食毫无兴趣,但在下却有位朋友,对这两样东西,也是赞不绝口的。''小鱼儿自然知道他所说的这``朋友''是谁,想到铁心兰和他在一起吃八宝饭的样子,小鱼儿乎气得跌下树来。

慕容珊珊已娇笑道:``贱妾等对这两样东西非但赞不绝口,简直已是魂牵梦索,时刻难忘了,不知公子可否劳驾去一趟,解解贱妾的馋。''这件事也未免太不合情理,也太容易。

花无缺心里也奇怪,但对于女子们的要求,他从来不愿拒绝,他怔了怔,终于笑道:``在下若能为夫人们做点事,正是极幸之至。''慕容珊珊道:``但这两样东西,却要乘热时才好吃。''花无缺沉吟道:``在下买回来时,只怕还是热的。,慕容珊珊笑得更甜道:''但公子此去,两只脚却不能沾着地面,不知公子能做得到么?"这句话说出来,群豪才知道她们出的难题,原来在这里,但两只脚不沾地,却又怎能到安庆城来回一次?

小鱼儿却又忍不住要笑了,暗道:``这位慕容姑娘出的题目,简直越是荒唐了,两只足不沾地,难道不能坐车去、骑马去么?''这件事又是个诡谲狡计,但花无缺若做不到,等到慕容珊珊做出来时,以花无缺的为人,也只好认输的。

只见花无缺突然脱下鞋子,露出一双洁白的罗袜,笑道:``在下双足是否沾地,此袜可为证。''话声未了,他身形已像轻烟般掠起。

他既没有坐上车子,也没有骑上马,却掠到一株大树前,折下了两段树枝,左手的树枝在地上一点,已掠出三丈,右手的树枝接着一点,人已到了六丈开外,只听他语声远远传来,道:``夫人稍候片刻,在下立即回来。''他竟将这一手``寒凫戏水''的轻功,运用化境,别人纵然使用这手轻功,但要在片刻间来回数里,也是绝不可能的。

议论之间,时间像是过去得很快,只见远处人影一闪,花无缺已到了近前,嘴里果然衔着东西。

他两根树枝点地,身子倒立而起,胸底向天,一双洁白的罗沫,果然还是干干净净,点尘不染。

欢呼声中,花无缺身子一翻,两只脚已套入方才脱下的那双鞋子里,抛去树技,将那包东西送到慕容珊珊面前,笑道:``在下幸不辱命,请夫人乘热吃吧。''慕容珊珊勉强挤出一丝笑容,道:``多谢公子。''她接过纸包,拆了开来,里面果然是热气腾腾的八宝饭和千层糕,她只得拿起一块,慢慢吃下去。

这又甜又香的千层糕,吃在她嘴里,却像是有些发苦。

不错,花无缺用的又是个笨法子,但小鱼儿非但不能说他笨,甚至也不禁在暗中有些佩服。

他用第一个``笨法子''显示出他惊人的内力,再用这第二个``笨法子''显示出他超群拔俗的轻功。

他用的若不是这两个``笨法子'',群豪此刻非但不会拍掌,简直已要将臭鸡蛋、桔子皮抛在他身上了。

慕容珊珊好容易才将一块千层糕吞下去,她简直从未想到千层糕也会变得这么样难吃的。

花无缺不动声色,等她吃完,才笑道:``那第三件事呢?''小仙女早已忍不住了,大声道:``有间屋,门是关着的,你全身上下都不许碰着扇门,也不许用东西去撞,能走进这屋子么?''小鱼儿暗笑道:``这第三件事简直比第二件还要荒唐,他手胸不能去碰那扇门,难道就不能打开窗子进去么?''但他此刻也知道花无缺必定是不会用这法子的,只见花无缺沉吟了半晌,道:``此地并无房屋,不知这马车慕容双道:''马车也行,你手不许碰马车的门,能走进马车里,就算你胜了。``花无缺目光转向慕容珊珊,道:''是这样么?``慕容珊珊想了想,笑道:''马车和屋子是一样的。``花无缺微笑道;''在下做到此事后,夫人还有无意见?``慕容双瞧了慕容珊珊一眼,慕容珊珊道:''公子若能做到此事,贱妾等立刻就走。"她实在想不出还有什么事能难得倒花无缺,若是动武,更非花无缺的对手,不走又能如何?

花无缺笑道:"既是如此,夫人但请瞧着\ldots\ldots 他一面说话,一面已走向那马车。

小鱼儿暗道;``这小子难道能用隔山打牛一类的劈空掌力,将这马车的门震裂不成?''只见花无缺走到马车前,突然道:``铁姑娘,开门吧。车厢里人银铃般娇笑着道:''这就开了。"群豪先是惊讶,后是奇怪,终于忍不住大笑起来,连小鱼儿都儿乎忍不住要笑起来,但听见那银铃般的娇笑声,他实在笑不出。慕容姐妹眼睁睁瞧着花无缺走进车门,也呆住了。

只听花无缺在车厢里笑道:``在下并未违背夫人们的规矩,已走进马车来了,夫人是否同意在下已胜了?''慕容姐妹张口给舌,竟说不出话来。

花无缺用的这法子,竟比慕容姐妹和小鱼儿所想的还要聪明,还要荒唐,在他等到最后才用出来,群豪已非但不会对他轻视,觉得失望,反而只有更佩服他的机智,一个个纷纷欢呼道:``花公子自然该算是胜了,谁也没有话说。''慕容珊珊再想勉强挤出一丝笑容,也没法子了。

她跺了跺脚,转身走上轿子,慕容双也跟着她,小仙女狠狠瞪了江别鹤一眼,狠狠道:``你莫要得意,我不会有好日子给你过的。''江别鹤微笑着瞧着她,也不说话。

十八条大汉又抬起了三顶大轿、三顶小轿,逃也似的走出了这山谷。

江别鹤笑道:``花兄的机智与武功,当世已不作第二人想,小弟当真叹为观止了。''群豪欢声雷动,花无缺自车厢中抱拳答礼,于是这辆马车也在这欢呼喝彩声中,驶了出去。

小鱼儿瞧着这辆马车,想到车厢里的铁心兰,竟呆住了,一颗心像是手巾似的被绞住,过了半晌,突又呼道:``我几时对她这么好的?我为何要为她痛苦?这不是活见鬼么?''铁心兰在他身边时,他丝毫也不觉得什么,但等到铁心兰到了旁人身旁,他竟突然觉得铁心兰比什么都重要。

小鱼儿呆了半晌,突见人丛里走过两个又高又大的胖子,他这才想起已答应过屠娇娇的事。

他跃下树,挤了过去,轻轻拍了拍那罗九"欧阳丁的肩头,欧阳丁霍然回过头,脸色已变了。

小鱼儿笑道:``你总是如此紧张,为何还不瘦,倒也是件怪事.''欧阳丁认出了他,面上才露出笑容,道;``最难消受美人恩,在下总无美人恩可以消受,只有以吃来打发日子,自然要越来越胖了。''小鱼儿眼珠子一转,笑道:``两位原来早已知道是我将那位姑娘带走的?''欧阳丁笑道:``除了兄台之外,她还会跟着谁走?''欧阳当笑道:``只是小弟却想不到兄台竟对那傻丫头也有兴趣,居然将她也带走了。''但两人这一次算盘都没有打对,更未想到那``傻丫头''竟是屠娇娇,以为那``傻丫头''也是被小鱼儿带走的。

小鱼儿自然也不说破,笑道:``有总比没有好,两个总比一个好,是么?''谈笑间三人已走出山谷,快走到屠娇娇的马车前。

小鱼儿突然停下脚步,道:两位请走吧,晚上再见。``欧阳丁笑道:''兄台莫非又要去会佳人了么?``小鱼儿神秘的一笑,道:''也许是\ldots\ldots"他有意无意间往那马车瞟了一眼。

欧阳丁眼珠子一转,大笑道:``在下等反正无事,正想陪兄台聊聊。''小鱼儿故意着急道:``找还要到别处去,两位。\ldots.''欧阳当大声道:``兄台只怕是要到别处去吧。''欧阳丁已冲到那马车前,一把拉开了车门,拍手笑道:``我猜的果然不错,佳人果然就在这里。''这兄弟两人一个拼命要占便宜,一个宁死也不吃亏,见到自己寻到的``美人儿''被别人弄走了,越想越觉这亏实在吃得太大了,不占些便宜回来,以后简直连觉都睡不着,兄弟两人竟不约而同,坐上了马车。

欧阳丁笑道:``兄台也请上来吧,我兄弟两人反正是打不走的了。''小鱼儿肚子里暗暗好笑:``你这宁死不吃亏,看样子今天已经是非吃亏不可的了。''他愁眉苦脸地坐上马车,叹道:``早知如此,方才就该避着你们才是,怎地还跑去招呼\ldots\ldots 唉,这只怕是瞧热闹瞧得晕了头了。''于是车马启行,向前直驰。

欧阳兄弟笑得更是得意,在那又厚又软的车座上舒服地坐了下来,却不知对面坐的就是要命的瘟神。

屠娇娇低垂着头,仿佛羞羞答答的模样,其实却是不愿这张脸被对面的人瞧得太清楚。

欧阳丁大笑道:``一日不见,姑娘怎地变得如此漂亮。''欧阳当笑道:``新承雨露,花朵自更娇艳,你难道连这道理都不懂。''这两兄弟虽然时时刻刻都在提防着别人,但此刻在这马车里,背后就是车壁,他们还有什么好提防的。

小鱼儿虽然知道屠娇娇要骗这两人上车,必定是要向他们算帐了,但也想不出她要如何下手。

只见屠娇娇始终羞答答的坐着,并不急着出手,也没有找小鱼儿帮忙的意思,竟像是早已胸有成竹。

小鱼儿只觉这热闹比方才还有意思,简直等不及地想瞧瞧屠娇娇如何出手,欧阳兄弟又是如何对付。

这时车马越走越快,已远离人群,转入荒郊。

欧阳丁忍不住问道:``兄台的香巢,怎地这么远呀?''小鱼儿大笑道,``你若想吃李子,就该沉得住气。''欧阳当大笑道:``是极是极,只不过\ldots\ldots{}''

屠娇娇突然抬起头来,娇笑道:``只不过那李子酸得很,你们只怕吃不下去。''欧阳兄弟齐地怔了怔,似已觉得有些不对劲了。

欧阳丁哈哈笑道:``姑娘什么时候变得如此会说话了!''屠娇娇笑道;``很久了,大概已经有二十年了。''欧阳兄弟脸色又变了变,两人已准备冲下车去。

小鱼儿瞧得暗暗皱眉:``屠娇娇做事怎地也变得如此沉不住气了,她这两句话说出,也不怕打草惊蛇么?\ldots{}''就在这时,只听``噗''的一声,那宽大的车座下,又厚又软的垫子里竟突然伸出四只手来!

两人只觉肘间一麻,双臂也被这四只手捏住,有如加上了道铁箍,痛彻心骨,再也动弹不得了!

欧阳丁惊极骇极,颤声道,``兄\ldots\ldots 兄台,你\ldots\ldots 你为何如此?''小鱼儿又是惊奇,又是好笑,道:``这不关我的事,你们莫要问我。''欧阳丁转向屠娇娇,道:``难道这\ldots 这是姑娘的主意?''屠娇娇笑道:``不是我是谁呢?''

欧阳兄弟听得这语气,脸上吓得更无一丝血色。

欧阳当道:``你\ldots 你究竟是什么人?屠娇娇笑道:''你方才认不出我,是真的,现在还认不出我,就是装佯了。``欧阳当道:''我。\ldots 我兄弟怎会认得姑娘?屠娇娇道;``你不认得我,为何会如此害怕?''欧阳丁强笑道:``害怕?谁害怕了\ldots\ldots{}''

欧阳当咯咯干笑道,``我兄弟自然知道娇姑娘这是开玩笑的。''屠娇娇叹了口气,道:``欧阳丁,欧阳当,你们再装佯也没有用了\ldots─''欧阳丁道:``屠大姐,你也觉得有趣么!瘦子竟会变得如此胖了。''屠娇娇笑道:``你们只怕是吃了发猪菜。''

欧阳丁道:"不错不错,我兄弟真像是吃了发猪莱了,哈哈\ldots\ldots{}

屠娇娇眼睛一瞪,冷冷道:``现在已经到了,你们该将发猪菜的菜吐出来的时候,是么?''两人嘴里不停地打着``哈哈'',却连什么话都不说,小鱼儿知道这两人不知又在打什么坏主意了。

突听车垫下一人笑道:``欧阳兄弟这二十年来除了养得又白又胖外,不想还学会了你这打哈哈的本事,我看你不如收他们做徒弟算了。''阴阳怪气的语声,竟是白开心。

一人大笑道:``哈哈,我若是收了这两个徒弟,只怕连裤子都要被他们算计去,只能光着屁股上街了,哈哈。''这两个``哈哈''声音又洪又亮,正是货真价实、``童叟无数的''笑里藏刀小弥陀"哈哈儿来了。

欧阳兄弟本来还在打着脱逃的主意,一听藏在车垫下的竟是这两个人,他们还有什么希望逃得掉。

欧阳丁干笑道:``小弟不想竟将两位兄长坐在屁股下,真是罪过。''白开心的车垫下笑道:``那倒无妨,屠大姐将这下面弄得比我家的床都舒服,还有酒有肉\ldots\ldots{}''哈哈儿接着笑道:``只是我想到你们两张肥屁股就在头上,却有些吃不下了。''欧阳当道:``两位不放开手,小弟使无法站起来,小弟不站起来,两位便只能在下面蹲着\ldots\ldots 屠大姐,你说这怎么办呢?''屠娇娇笑道:``这还不容易办么?只要你们把发猪菜吐出来,他们立刻就放手。''白开心道:``再不然就将你两人宰了也行。''

哈哈儿道:``哈哈,这主意倒也不错。''

欧阳丁叹了口气,道:``屠大姐交给我兄弟的东西,我兄弟早就想送到恶人谷去的,只是\ldots{}''屠娇娇冷笑道:``只是东西却不见了,是么?''欧阳丁哭丧着脸道:``屠大姐猜的一点也不错,你们入谷的第二年,那批东西就全都被人抢走了,我兄弟生怕屠大姐怪罪,所以只好\ldots 只好\ldots{}''屠娇娇完全不动声色,甚至连眼睛都没有眨一眨,悠然道:``这理由的确不错,但抢东西的是谁呢?''欧阳丁叹了口气,道;``路仲达。''

屠娇娇突然格格笑起来.道:``哈兄,你说他们这谎话说的好么?''哈哈儿道;``哈哈,果然不错,他明知咱们没法子去问路仲达的.''白开心嘻嘻笑道:``这种事就叫做死无对证。欧阳当道:''若有半句虚言,就叫我天诛地灭,不得好死,下辈子投胎变个母猪,红烧了来让哈兄下酒.。

小鱼儿暗笑道:"这人赌咒当真好像吃白菜似的,一天也不知说过多少次,否则又怎能说得如此流利\ldots\ldots{}

只见屠娇娇仰起了头,全不理睬,哈哈儿和白开心在车垫下也不说话,却有阵咀嚼声传出,显见白开心已在吃起肉来。

欧阳兄弟你一句我一句,说得满头大汗,几乎连嘴都说破了,屠娇娇却像是一句也没听见。

小鱼儿越瞧越有趣,本来想走,也舍不得走了,这时车马突然停下,接着,车窗外就露出一张脸。

这张脸冷漠苍白,白得已几乎变得像冰一样透明。

欧阳兄弟瞧见了这张脸,就好像被别人抽了鞭子似的,整个身子都缩成一团,欧阳丁道:``原\ldots\ldots 原来杜\ldots\ldots 杜老大也来了!''

\hypertarget{ux7b2cux516dux5341ux4e00ux7ae0-ux9634ux72e0ux6bd2ux8ba1}{%
\chapter{第六十一章
阴狠毒计}\label{ux7b2cux516dux5341ux4e00ux7ae0-ux9634ux72e0ux6bd2ux8ba1}}

欧阳兄弟方才还是滔滔不绝,能说会道,此刻见了杜杀,竟连几个字都说不清楚。

小鱼儿瞧见``血手''杜杀这张冰一般的脸,心里不知怎地,却生出一种亲切之感,忍不住笑道:``杜大叔,你好么?''杜杀道:``好!''

他只瞧了小鱼儿一眼,在这一瞬间,他目中的冰雪似乎有些溶化,但等到这双眼睛盯在欧阳兄弟身上时,寒意却更重了。

他拉开了车门,话也不说,另一只手已掴在欧阳当脸上,正正反反,捆了二十几个耳光,这才冷冷道:``你还认得我么?''欧阳当却连哼都不敢哼,还陪着笑道:``小\ldots\ldots 小弟怎敢不\ldots 不认得杜老大?''杜杀冷笑着反手一掌,切在他右膝``犊鼻''穴上,照样给欧阳丁也来了一掌,转过身子,厉声道:``下来吧!''欧阳丁道:``小\ldots。小弟腿已不能动了,怎么下去?''杜杀道:``腿不能动,用手爬下来!''

欧阳兄弟互望了一眼,果然乖乖地爬了下去。

马车停在一栋荒宅外,赶车的却已不见了。

几人进了荒宅,只见残败破落的大厅里,竟生着堆火,火上煮着锅东西,也不知是什么,还有好几个瓦罐子,零乱地放在地上,像是做菜用的佐料。

一个人箕踞在火堆旁,正是那赶车的,这么大热的天气,他坐在火旁,头上竟没有一粒汗珠。

屠娇娇笑道:``小鱼儿,你还不快过去见见你的李大叔,这些年来,他天天在想着你哩,只不过不知道他是不是想吃你的肉?''小鱼儿笑嘻嘻道:``看样子,李大叔莫非在生气么?''李大嘴忍不住哈哈一笑,拉起小鱼儿的手,笑道:``不想你这小鬼倒还记得这句话。''这时欧阳兄弟才呻吟着爬了进来,``血手''杜杀冷冷地跟在他们身后,只要他们爬得慢了些就重重给他们一脚,简直把这两人看得比猪还不如。

哈哈儿大笑道:``二十年来,咱们兄弟还是第一次聚了这么多,当真是盛会难逢,不可不好生庆祝庆祝。''屠娇娇格格笑道:``江湖中若有人知道咱们这班老伙伴又聚在一起,不如该如何想法?''哈哈儿笑道:``他们只怕连苦胆都要吓破了。''李大嘴正色道:``苦胆千万不可吓破,否则肉就苦得不能吃了。''小鱼儿眼珠子四下转动,瞧着这些人,想到自己童年时的光景,心里也不知是什么滋味。

这些人虽然是恶人,但在他眼中,每个人多少都有些可爱之处,真要比江别鹤那种伪君子可爱得多。

小鱼儿觉得实在开心得很,但想到这些人每个都和瘟神一样,此番重出江湖,又不知有多少人要倒霉了,他心里不觉又有些发愁。

他实在不能眼睁睁的瞧着,他得想个法子。

只听屠娇娇道:``现在,只差阴老九了,不知他遇见了什么事,怎地还未赶来?''欧阳丁爬在地上,赔笑道:``小弟瞧见诸兄又复重聚,实是不胜之喜。''屠娇娇道:``是呀,但咱们的钱已被你骗光了,哪里还有钱买酒。''欧阳丁道:``屠大姐只要放了小弟,小弟必定立刻去找那姓路的,拚了命也要将那批东西抢回来。''话未说完,杜杀的钢钩已钩入了他肩头,将他整个人都钩了起来,欧阳丁再也忍不住杀猪似的惨呼道:``杜老大,小弟并末说谎,你饶了小弟吧。''杜杀冷冷道:``东西在哪里?说!''

欧阳丁道;``真\ldots\ldots 真的被路仲达\ldots.''

杜杀一拳捣在他脸上他``达''字出口,一嘴鲜血也随着喷了出来,里面还夹着三颗牙齿。

小鱼儿明知这欧阳兄弟比谁都坏,但瞧见他们这副模样,也觉大是不忍,正想设法帮他们个忙,欧阳丁已大呼道:``我说了,我说了,那批东西还在,路仲达根本连手指出没有碰到,我方才全是说谎的,你们饶了我吧。''小鱼儿叹了口气,喃喃道:``你明知要说的,为何不早说,难道非要人家用这种法子对付你不可?这也怪不得别人心狠手辣了。''杜杀道:``东西既在,在哪里?欧阳丁颤声道:''我说出来后,你们还要杀我么?``哈哈儿道:''哈哈,咱们本是如弟兄一样,怎会杀你们?``欧阳当道:''这话要杜老大说,我兄弟才放心。"``血手''杜杀虽然心狠手辣,但平生言出必行,从未说过半句谎话,这点江湖中人都是知道的。

只听杜杀冷冷道:``你说出之后,我等绝不伤你性命!''欧阳丁长长松了口气,道:``那批东西就藏在龟山之巅的一个洞穴里\ldots\ldots{}''欧阳当抢着道:``小弟还可为诸兄画一幅详细的地图。''地图画好,众人俱是喜动颜色,四双手一起伸了出去。只听一连串``□啪''声响,你打我的手,我打你的手,四双手又一起缩了回去──只有四双手,只因``血手''杜杀的手除了杀人外,是从不轻易伸出来的。

李大嘴终于大声道:``此图还是交给杜老大保管,否则我绝不放心。''突听一人悠悠道:``不错,除了杜老大外,我也是谁都不放心的。''缥缥渺缈的话声中,窗外已多了条人影。

哈哈儿道:``哈哈,阴老九果然是聪明人,等咱们费了好半天力后,他才来抢便宜。''阴九幽冷冷道:``你们费了力,难道我没有?''屠娇娇笑道;``你费了什么力?难道被鬼缠住脱不了身?''阴九幽一字字道:``我正是遇见鬼了。''

阴九幽目光在小鱼儿身上打了个转,突然阴恻恻的一笑,道:``小鱼儿,你猜是什么鬼?''小鱼儿眼珠子一转,笑道:``能缠住你的鬼,倒也少有,但能令你害怕的人,倒有一个\ldots\ldots{}''屠娇娇跳了起来,失声道:``你莫非遇见了燕南天?''阴九幽诡笑道:``我若遇上他,还能来么?\ldots\ldots 我只不过远远瞧见他了,瞧见他骑在马上,生龙活虎,比以前好像还要精神得多。''小鱼儿听得又惊又喜,李大嘴、哈哈儿、白开心、屠娇娇,脸上全都变了颜色,尤其是屠娇娇,一步冲过去,道:``他\ldots\ldots 他是往哪里去的?''阴九幽道:``我怎知他要到哪里去?说不定是往这里来的。''这句话说出来,名震天下的``十大恶人''们竟连地都坐不住了,李大嘴首先站了起来,道:``这里的确不是久留之地,咱们走吧。''哈哈儿道:``走自然要走,谁不走我佩服他。''欧阳丁颤声道:``求求你们,将我也带走吧,我\ldots\ldots 我也不愿见着燕南天。''这``燕南天''三个宇,竟像是有着什么魔力,竟能使这些杀人不眨眼的人物坐立不定,失魂落魄。

小鱼儿瞧得又是惊喜,又是羡慕,暗叹道:``一个人若能做到像燕南天这样,这辈子也就不算白活了\ldots\ldots 我自以为己蛮不错,但比起他来,又能算什么?''但燕南天也是个人呀,燕南天能做到的事,江小鱼为什么不能做到,江小鱼又有什么不如人的地方?

一时之间,小鱼儿但觉心中万念奔涌,忽而觉得心灰意懒,忽又觉得热血澎湃,豪气顿生\ldots"忽听欧阳丁狂呼一声,鲜血飞激,他一条手臂,一条大腿,竟已被屠娇娇生生剁了下来。

欧阳当嘶声道:``杜老大,你\ldots\ldots 你答应过的\ldots\ldots 你\ldots\ldots{}''屠娇娇笑道:``杜老大只答应不要你性命,并未答应别的呀。''她一面说话,一面又将欧阳当的一手一腿剁了下来,又将罐子里一满罐白糖,全都倒在他们身上。

欧阳当大呼道:``你\ldots\ldots 你干脆给我个痛快,杀了我吧!''屠娇娇笑道:``杜老大说道不杀你,我怎能杀你!''欧阳了咬牙道:``你\ldots\ldots 你好狠的心,好毒的手段!''屠娇娇咯咯笑道:``你现在虽然这么说,但我若落在你手上,你只怕比我还要狠上两倍。''她娇笑着走了出去,竟再也不瞧他们一眼。

欧阳兄弟的惨呼,竟像是没有一个人听见.现在,夕阳满天,已是黄昏。

小鱼儿独立在夕阳下,屠娇娇、白开心、李大嘴、杜杀、阴九幽都已走了,临走之前,都和小鱼儿说过一些话,但说的是些什么,小鱼儿并没有认真去听,他只知道他们都已到龟山去了,并没有要小鱼儿随行,小鱼儿更没有跟他们去的意思,他只听他们说:``小心提防着燕南天,好生将江别鹤斗垮,你跟着我们走,也有些不便,我们日后定会来找你。''小鱼儿并没有认真去听他们的话,只因他不知从什么时候开始,他的心突然被``燕南天''三个字充满。

``燕南天,我为什么不能学燕南天?而要学屠娇娇、李大嘴。\ldots 我恨一个人时\ldots\ldots 为什么不能学燕南天那样,堂堂正正地找他,与他决斗,反击学屠娇娇和李大嘴,只知在暗中和他捣鬼!''欧阳兄弟的惨呼声,犹不住我中传来,小鱼儿突然转身向那荒宅直掠而去。

欧阳兄弟倒卧在血泊中,成千成万虫蚁,已从荒宅中四面八方涌了过来,他们身受之惨,实非任何言语所能形容。

他们瞧见小鱼儿来了,惧都颤声呼道:``求求你,赏我一刀吧,我死也感激你。''小鱼儿叹了口气,竟将两人提了出去,寻了个水井,将他们两人身上的虫蚁冲了个干净。

欧阳兄弟再也想不到他竟会来相救,四只眼睛呆望着小鱼儿,目光中既是惊讶,又是感激。

小鱼儿喃喃道:``我突然变得慈悲起来了,你们奇怪么?我虽然知道你们都不是好东西,但要你们这样慢慢的死,却也未免太过份了些。''欧阳丁凝注着他,道:``你''\ldots 你若肯救我,我\ldots\ldots\ldots 必定重重报答你。小鱼儿笑道:``只要你能活下去,我一定救你,但我可不要你什么报答。''欧阳丁瞧着他,就像是从未见过他这个人似的,突然道:``那批宝物并非藏在龟山。''他忽然说出这句话来,小鱼儿怔了怔。

欧阳丁那张令任何人见了都要生恻隐之心的脸,竟又露出一丝狡恶的狞笑;咬牙道:``我在那种情况下说出来的话,任何人都不会以为是假的了,是么?我正是要他们认为如此,否则那些恶鬼又怎会上我的当!''小鱼儿道:``他们最多也不过空跑一趟而已,也算不得是上当。''欧阳当疼得嘴唇上的肉都在打颤,此刻却仍在大笑道;``我兄弟要他们上当,岂只空跑─趟而已。''欧阳丁狞笑道:``这一趟他们纵能活着回来,至少也是将半条命留在龟山上。''小鱼儿皱眉道:``为什么?''

欧阳当阴阴笑道:``我兄弟告诉他们的那个地方,没有藏宝,却有个恶魔,这恶魔已有许多年未露面了,他们做梦也不会想到他会藏在龟山。''欧阳丁道:``咱们就算死了,但他们也没有好受的,遇见了这恶魔,他们身受之惨,只怕比咱们还惨十倍。''小鱼儿摇头笑道,``你们既已要死了,何苦要害人?''欧阳丁大笑道:``我明知他们反正是放不过我的,索性多吃些苦,多受些罪,把他们也拖下水,我欧阳丁正是拚命也要占便宜的。''欧阳当大笑道:``我兄弟两条命,要换他们五条命,这买卖做得连本带利都有了,我欧阳当正是宁死也不吃亏。''小鱼儿瞧见他们这副一面疼得打滚、一面还要大笑的模样,全身都起了鸡皮疙瘩,摇头苦笑道:``你们这简直不是明知必死才害人的,简直是为了害人,而宁可去死,像你们这样的人,倒也少见得很。''只见这拼命害人的两兄弟,虽在大笑,但笑声却渐渐微弱,欧阳当滚到欧阳丁身旁,道:``老大,响们真要将那藏宝之地告诉这小子么?''欧阳丁道:``这小子天生不是好东西,得了咱们的那宝藏后,害的人必定更多了,咱们死后,能瞧着这小子用咱们的宝藏害人,也是乐事一件。''小鱼儿叹道:``别人说,人之将死,其言也善,你们死到临头,也不肯说两句好话么?''欧阳当道:``咱\ldots\ldots 咱们活着是恶人,死了也要\ldots\ldots 做恶鬼。''欧阳丁道:``告诉你,那真的藏宝之处,是在\ldots\ldots\ldots 汉口城,八宝里,巷子到头右面的叁栋小屋子里,那门是黄色的。''欧阳当咯咯笑道:``他们都以为咱们必定也将财宝藏在什么荒无人迹的秘密山洞里,却不愿咱们偏偏要将财宝藏在人烟稠密之处,叫他们做梦也想不到。''两人的语声,也越来越微弱,简直不大容易听得清楚了,那伤口也渐渐不再有血流出来。

小鱼儿忽然一笑,道:``很好,现在你们若要去做恶鬼,只管去做吧,但你们却莫要忘了,做恶鬼是要上刀山、下油锅的,那滋味并不好受。''欧阳当身子突然缩成一团,嘶声道:``我不是恶人。''。也不愿做恶鬼,我。\ldots 我不愿下地狱。``小鱼儿道:''你现在才想起说这话,不嫌太迟了么?``欧阳当大呼道,''求求你,用我们的财宝,去为我们做些好事吧。``欧阳丁道:''不错不错,我们坏事做得太多了,求求你为我们赎罪吧。``小鱼儿摇头四道:''奇怪,很多人都以为用两个臭钱就可以赎罪,这想法岂非太可笑了么?若是真的如此,天堂上岂非都是有钱人,穷人难道都要下地狱。``欧阳兄弟齐声惨呼道:''求求你,帮个忙吧!"欧阳兄弟全身颤抖,已说不出话来,只是拚命点头。

小鱼儿摇头道:``若让天下的恶人,全都来瞧瞧你们现在的样子,以后做坏事的人,只怕就要少得多了。他叹了口气,接道:''但无论如何,我总会为你们试试的,你们现在才知道忏悔,虽已迟了,但总比死也不肯忏侮好一点,你们只管放心死吧。"每个人一生之中,都会有一个特别值得怀念的日子。

小鱼儿自然也有这样的一天.小鱼儿在这一天里,突然发现了许多事\ldots\ldots\ldots 这些事他以前并非完全不知道,只是从未仔细想而已。

这一天纵然对─生多姿多形的小鱼儿说来,也是特别值得怀念的,就在这一天里,他经历到从来未有的伤心与失望,也经历到从来未有的兴奋与刺激,假如他以前始终还只是个孩子,这一天却使他完全成长起来。

现在,小鱼儿将脸洗得干干净净,到成衣铺里,换上套天青色的衣服,临镜一照,自己对自己也觉得十分满意。

于是他又找了家地方最大、生意最好的饭馆,饱餐了一顿,来自四面八方的江湖朋友,仍因在安庆城没有走,这状元楼里几十张桌子,倒有一大半坐的是武林豪杰。

小鱼儿带着欣赏的心情,瞧着他们大块吃肉、大碗喝酒,他觉得这些粗豪的汉子们,委实都有他们的可爱之处。

只听他旁边桌子一人笑道:``欧阳兄今天晚上想必还是要到这状元楼来的了。''那``欧阳兄''哈哈笑道:``承蒙江大侠瞧得起,倒也发给俺一张帖子,今天晚上正是少不得还要到这里来喝上一顿。''他语声故意说得很大,四下果然立刻有不少人向他瞧了过来,那眼光既是羡慕又有些妒忌。

小鱼儿瞧得又好笑,又好气.江别鹤居然还有脸来请其客,被请的人居然还引以为荣,这实在要令小鱼儿气破肚子。

靠窗的一桌上,突然又有人讶然道:``江大侠今天晚上请客,正是要为花公子庆功,花公子此刻却怎地要走了?难道他竟不给江大侠面子。''另一人道:今天风和日丽,天色晴朗,花公子想必正是带着他未来的妻子出城踏青,绝不会是真要走的。"只见一辆大车,自东面来,车窗上竹半卷,隐约可以瞧见一个乌发堆云的丽人倩影。

花无缺风神俊朗,白衣如雪,骑着匹鞍辔鲜明的千里马,随行在车旁,不时与车中人低低谈笑。

小鱼儿一眼瞧过,几乎又变得痴了。

这时酒楼上的人大多数涌到窗前凭窗下望,不觉又发出一片艳羡之声,有的人竟含笑招呼道:``花公子你好?''花无缺抬起头来,淡淡一笑。

酒楼上的人唯恐他瞧不见自己,一个个的头都拼命向外伸,小鱼儿却生怕被他瞧见,赶紧缩回了头。

直到花无缺的车马过去,酒楼上的人都回到座上,小鱼儿仍痴痴地坐在那里,忽然喃喃自语道:``我这样躲着他,究竟要躲到几时,我难道真得一辈子都躲着他么\ldots\ldots{}''想到这里,忽然站起身子,冲下楼去。

\hypertarget{ux7b2cux516dux5341ux4e8cux7ae0-ux60c5ux6709ux72ecux949f}{%
\chapter{第六十二章
情有独钟}\label{ux7b2cux516dux5341ux4e8cux7ae0-ux60c5ux6709ux72ecux949f}}

小鱼儿根本全不管别人用什么眼光瞧他,提着衣襟越跑越快,片刻间便已追上了花无缺的车马。

车马这时正是出城,突听一人大呼道:``花无缺慢走!''花无缺微徽皱了皱眉头,自动勒住马,铁心兰刚从车窗里探出半个头,小鱼儿一个箭步窜了过来。

小鱼儿会突然出现,就连花无缺都不免大吃一惊,几乎不相信自己的眼睛,铁心兰更骇呆了。

小鱼儿拼命忍住,绝不去瞧铁心兰一眼,只是眨也不眨地瞪着花无缺,突然哈哈一笑,道:``你以为我是送死来的,是么?''花无缺叹了口气,道;``不错。''

面对着这样的人,小鱼儿也有些笑不出来了,大声道;``你既然这么想杀我,为何不来找我却等我来找你。''花无缺缓缓道:``我自已本不愿杀你,所以也并不急着找你,但此刻我既然见着你,却还是非杀你不可!''铁心兰这时才回过神,突然拉开车门,自车厢里冲了下来,挡在小鱼儿面前,大声道:``这次是他自己来找你的,至少这次你不能杀他。''小鱼儿突然用力一推,将她推得撞在车上。花无缺脸色变了变,终于忍住没有开口。

铁心兰瞧着小鱼儿,颤声道:``你\ldots\ldots 你为什么要这样对我?''小鱼儿连瞧也不瞧她一眼,瞪着花无缺冷笑道:``这铁姑娘听说是你未过门的妻子,为何要来管我的闲事,我根本连认都不认得她。''铁心兰用力咬住了嘴唇,虽然嘴唇己被咬得出血,虽然眼睛里已有泪珠在打转,却还是不离开。

花无缺心里只觉阵阵刺痛,故意不再去瞧铁心兰,淡淡道:``这次你不要别人帮你忙了么?''小鱼儿仰天大笑道:``我若要人帮忙,为何来找你?''他突又顿住笑声,大声道:你心里自然也知道,我这种人,是绝不会为了送死而来找你的,那么,我是为何而来的,你心里必定又在奇怪.``花无缺道:''正是有些奇怪。"

小鱼儿道:``你以为我杀不死你,我也以为你杀不死我,若是这样拖下去,拖到两百年后也不知究竟是你对,还是我对,我心里着急,你只怕比我更急,所以,我今天来,正是为了要和你做个了断!''花无缺目光闪动微笑道:``你想如何来做了断?''小鱼儿道:``你只要说个地方,三个月后,我必定去找你一决生死!没有分出生死强弱前,谁也不许逃走!''小鱼儿长长吐了口气,又道:但在这三个月的约期末到之前,你纵然瞧见了我,也得装作没有瞧见,更不能来寻我动手!"花无缺沉吟不语。

小鱼儿大声道:``我若不来找你,这三个月,你反正是找不着我的,这条件你并没有吃亏,你为何不肯答应?''花无缺缓缓道:``你说出这条件,其中想必又有诡计。''小鱼儿瞪眼道:``你\ldots\ldots 你不答应?''

花无缺忽然勒过马头,道:``三个月后,我在武汉一带,你必定可以找到我的。''小鱼儿大声道:``很好,你如此信任我,我必定不会使你失望!''话未说完,也掉转头,大步而出。

铁心兰只望他会回头来瞧一眼,但他始终也没有回过头来,只到他身影完全消失,铁心兰还痴痴地站在那里.花无缺静静地坐在马上,也没有催她。

也不如过了多久,铁心兰才缓缓上了马车,拉开车门瞧见花无缺仍坐在马上等她,她心里也不知是什么滋味。

花无缺本是为了要让铁心兰散散心,才劝她出城走走的,但此刻出得城来,两人心里反面都打了个结,眼见再难化解得开。

铁心兰不停地将车窗上的竹卷起来,又放下去,城郊外虽然风物如画,但她再也没有心情去瞧上一眼。

前面一丛花树,千千万万朵不知名的山花,开得正盛,一道小溪流过花林,溪水在初秋的太阳下闪闪发光。

远处,有个穷汉,正仰面卧在小溪旁晒太阳,近处虫鸣阵阵,鸟语花香,地上的泥土,软得像毯子。

花无缺下丁马,站在一栋花树下,又出起神来,微风吹动着他雪白的长衫。

铁心兰轻轻拉开了车门,走在柔软的泥土上,瞧着花无缺的背影,也痴痴地出了会神,突然道:``你明知那其中必有诡计,为何还要答应他?''花无缺似叹了口气,但没有回头,也没有说话。

铁心兰自他身旁走过,自低枝上摘下了一朵小花,揉碎了这朵不知名的山花,突然回过头,面对着他,道:``你为何不说话?''花无缺淡淡一笑,终于缓缓道:``沉默,有时岂非比什么话都好?''铁心兰霍然扭转身子,道;``这两年来,你处处照顾着我,若不是你,我早已死了,我这一辈子,从来也没有人像你对我这么好。''花无缺瞧着她脖子后随风飘动的发丝,没有开口。

铁心兰轻叹着接道:``我这一生中,也从没有人像他对我那么坏,但是我\ldots\ldots 我也不知为了什么,一瞧见他,就没了主意。''花无缺闭起了眼睛,道:``这些话,你本来不必对我说的。''铁心兰肩头不住颤抖,道:``我也知道这话不该说的,但若不对你说个明白,我心里更难受,更觉得对不起你。''花无缺柔声道:``这怎能怪你?你又有什么对我不起?''远处那穷汉,长长伸了个懒腰,喃喃道;``年纪轻轻,为了这种小事就痛苦不堪,等你们长大了,就会知道世上比这种更痛苦千万倍的事,还多着哩!''花无缺本未留意他,更未想到自己在这边的轻言细语,竟会被远在数丈外的人听在耳里,就连铁心兰也不觉止住了低泣声,抬起头来。

那穷汉打了个呵欠,突然翻身掠起。

只见他面上瘦骨嶙嶙,浓眉如墨,满脸青惨惨的发渣子,在阳光下亮得刺眼,骤眼瞧去,也瞧不出他有多大年纪。

花无缺出道以来,天下的英雄,谁也没被他瞧在眼里,但也不知道怎的,这懒洋洋的穷汉,竟似有一种说不出的慑人之力,他身影虽非十分魁伟,但无论谁在他面前,都不禁要自觉渺小。

那穷汉瞧见花无缺,也似吃了一惊,喃喃道,"莫非就是他?

否则怎会如此相像,别人的事我可不管,但是他\ldots\ldots 我岂能不成全他的心意?"花无缺与铁心兰也末听清他说的是什么,这穷汉已走了过来,他懒洋洋地走着,像是走得很慢。

但只走两步他竟已到了花无缺面前,这时花无缺才将他瞧得更清楚了些。

只见他身上穿的是件已洗得发白的黑布衣服,脚下穿着双破烂草鞋,一双筋骨凸出的大手长长垂了下来,几乎垂过膝盖.腰畔扎着条草绳,草绳上却斜斜插着柄早已生了锈的铁剑。

这穷汉已上上下下仔细地打量了花无缺几眼,突然咧嘴一笑,道:``你心里可是很喜欢这位姑娘?''花无缺实未想到他竟会问出这句话来,怔了怔,呐呐道:``这\ldots\ldots{}''

那穷汉喝道:``什么沉默比说话好,全是狗屁,你不说出来,人家怎知你喜欢她。''花无缺的脸竟红了红,更说不出话来,他从来以含蓄为美,但也不知怎地,这种粗俗不堪的话,自这穷汉嘴里说出来,竟另有一种豪迈之气,令人不觉心动神驰。

铁心兰的脸虽也红了,却忽然道:``有些话,他不必说,我也知道。''那穷汉闪电殷的眼睛,立刻瞪在她的脸上,哈哈大笑道:``很好,不想你竟比他痛快得多,这样的女孩子,莫说是他,就是连我见了都有些欢喜。''那穷汉道:``你喜不喜欢他?''铁心兰道:``我不\ldots\ldots{}''

她抬头瞧了花无缺一眼,又垂下丁头,接着道:``我也不是不喜欢,只是\ldots\ldots{}''那穷汉不等她再说,已大笑道;``既不是不喜欢,自然就是喜欢了,你两人既彼此欢喜,就由我来作媒,今日就在这里成了亲吧!''他这句话说出来,花无缺与铁心兰又不觉大吃一惊。

花无缺失声道;``阁下莫非在开玩笑么?''

那穷汉眼睛一瞪,大声道:``这怎会是开玩笑,你瞧此地,鸟语花香,风和日丽,你两人在这里成亲,岂非比什么地方都好得多。''他越说越得意,又不禁大笑道:``红烛之光,又怎及阳光之美,世上所有红毯,更都不比这泥土的芬芳柔软,你两人就这阳光下、泥土上,快快拜了天地,岂非人生一大乐事,就连我都觉得痛快已极!''花无缺听他自说自话,也不知是该恼怒,还是该欢喜,铁心兰呆呆地怔在那里,更是哭笑不得。

她此刻虽有心一口拒绝,却又不忍心去伤花无缺的心。

花无缺瞧了瞧她的神色,却忽然道:``阁下虽是一番好意,怎奈我等却难从命。''那穷汉笑声顿住,瞪眼道:``你不答应?''

花无缺长长叹了口气,道:``是。''

那穷汉又大笑道:``我知道了,这不是你不愿意只是你怕她不愿意,但她既未说话,你又何苦多心。''花无缺想了想,缓缓道:``有许多话,是不必说出来的。''那穷汉叹道:``你明明喜欢她喜欢得要命,但为了她,却宁可硬着心肠不答应。这样的多情种子,倒真不傀是你爹爹的儿子。''花无缺也听不懂他这话是什么意思,那穷汉已瞪着铁心兰道;``像这样的男人,你不嫁给他嫁给谁?''花无缺虽然明知地是为了自己,此刻也不觉怒气发作,冷笑道:``在下什么人都见过,倒真还没有见过如此逼人成亲的。''那穷汉道:``你如此说话,想必是以为我宰不了你,是么?''``是么''两字出口,突然拔出腰畔的剑,向身旁的一株花树上砍了过去,这柄剑已锈得不成模样,看来简直连树枝都砍不动,谁知他一剑挥去,那合抱不拢的巨木,竟``喀咳''一声折为两段!

铁心兰生怕花无缺开口得罪了他,只因此人武功实在深不可测,就连花无缺都未必是他敌手。

要知铁心兰心肠最是善良,虽不愿花无缺伤了小鱼儿,也不愿别人伤了花无缺,不等花无缺开口,抢先道:``我答应了。''花无缺突然道:``我绝不答应。''

花无缺明知铁心兰不是真心情愿的,他越是对铁心兰爱之入骨,便越是不肯令铁心兰有半分勉强。

花无缺冷冷道:``我不答应,就是不答应,你若要杀我,只管动手就是!''铁心兰失声道:``你\ldots\ldots 你难道不喜欢我?''

花无缺再也不瞧她一眼──他看来虽和小鱼儿全无丝毫相同之处,但使起性子来,却和小鱼儿完全一模一样。

那穷汉瞪眼瞧着他,道:``你宁可终生痛苦,也不答应?''花无缺道:``绝不答应。''

那穷汉喝道:``好!我与其让你终生受苦,倒不如现在就宰了你!''剑光一展,向花无缺直刺过去!他这一剑自然末尽全力,但出手之快,剑势之强,环顾天下武林,已无一人能望其项背。

只听``啪''的一声,花无缺虽然避开了这一剑,束发的玉冠,却已被剑气震断,满头头发,都被激得根根立起!这一剑之威,竞至如此!实是不可思议!

铁心兰失色惊呼道:``前辈快请住手,他不肯答应只是为了我,我心里才真是不肯答应的,前辈你要杀,就杀了我吧!''她惊骇之下,不禁吐了真言,花无缺只觉心里一阵刺痛,出手三掌,竟不顾一切,抢入剑光反扑过去。

谁知那穷汉反而收住剑势,哈哈大笑道:``姓江的果然都是牛一般的脾气,只是你却比你爹爹还呆,试想她若真的不肯答应你,真的不喜欢你,又怎肯为你死。''花无缺怔了一怔,铁心兰也跟着怔住了,道:``他自然不姓江,他叫花无缺。''那穷汉摸了模头,满面惊讶之色,哺哺道:``你不姓江?这倒真是件怪事,你简直彻头彻尾像个姓江的,你简直和他长得一模一样。''花无缺也忘了出手,只觉这人简直有些毛病。

那穷汉叹了口气,苦笑道:``你既不姓江,成不成亲,就全都不关我的事了,你要走就走吧。''他竟然真的什么都不管了,喃喃苦笑着转身而去。

花无缺、铁心兰两人面面相觑,谁也弄不懂这究竞是怎么回事,只见那穷汉一面走,一面还在自言自语,道:``这少年居然不是江小鱼,奇怪奇怪\ldots\ldots{}''铁心兰又惊又喜,失声道:``前辈莫非以为他是江小鱼,才逼着我们成亲的么?''那穷汉说道:``我虽然是不忍见着你们为情受苦!但若非认定他是江小鱼,我实在也不会多管闹事。''那穷汉忽然回过头来,瞧了瞧铁心兰,又瞧了瞧花无缺,突然大笑道:``我明白了,我明白了,原来你说的那对你最坏的人,就是江小鱼,你两人本来是会成亲的!就为了江小鱼,才弄成这般模样。''铁心兰幽幽叹息一声,垂下了头。

那穷汉用手敲头失笑道:``我本来想成人好事,谁知却将这件事越弄越糟了\ldots\ldots{}''他一生精研剑法,再加上终年闯荡江湖,奔波劳苦,从来也未能领略到儿女柔情的滋味。

花无缺听得这笑声,心里又是愤怒,又是酸苦,突然道:``你就想走了么?''那穷汉笑道:``我知道你心里不舒服,就让你打两拳出出气吧。''花无缺冷笑道:``你武功纵然强绝天下,却也万万受不了我一掌,你若不招架,可是自寻死路!''语声中一掌拍了出去。

这一掌看来虽轻柔,但所取的部位,却是毒辣无比,而且掌心深陷,蓄力不吐,显然一发便不可收拾。

那穷汉是何等眼力,耸然道:``果然好掌力!''他天性好武,此刻骤然遇见此等少年高手,也不禁想试试对方功力究竟如何,巧掌竟迎了上去!

谁知花无缺掌势突变,来势如矢的一掌,竟突然向右掌引,转变之巧妙亦是令人不可思议。

这一着正是``移花宫''独步天下的``移花接玉'',花无缺一招使出,只道对方这一掌必定要反打在自己身上。

谁知那穷汉身形的溜溜一转,竟将这普天之下、无人能破解的``移花接玉''轻轻化解。

花无缺这才真的大惊失色,动容道:``你究竟是谁?''那穷汉突然仰天大笑道:``我一生总以未能一试移花接玉武功为恨,不想今日竟在此地遇见了移花官的门下\ldots\ldots{}''洪亮的笑声,震得四面枝头山花雨一般落下。

铁心兰悚然道:``前辈莫非与移花宫有什么过不去么?''那穷汉嘎然顿住笑声,喝道:``我正是与移花宫仇深如海,我十年磨剑,为的正是要将移花宫门下,杀尽杀绝!''花无缺突然失声道:``燕南天!你是燕南天!''``移花宫''最大的对头,就是燕南天,普天之下,除了燕南天之外,也没有别人敢和``移花宫''为仇作对!

\hypertarget{ux7b2cux516dux5341ux4e09ux7ae0-ux5251ux6c14ux51b2ux9704}{%
\chapter{第六十三章
剑气冲霄}\label{ux7b2cux516dux5341ux4e09ux7ae0-ux5251ux6c14ux51b2ux9704}}

花无缺和铁心兰正发楞间,只见那穷汉目中光芒一闪,道,``我正是燕南天!''花无缺默然半晌,忽然缓缓脱下自己的长衫,仔仔细细叠好,缓缓走到铁心兰面前,双手交给铁心兰。

铁心兰自然也知道他交给自己的虽然只不过是件衣服,但其中却不知有多么沉重、多么复杂的含意。

花无缺道:``能与燕南天一战,正是学武的人毕生之愿,就是移花宫门下,也以能与燕南天一战为荣。''铁心兰压低声音,道:``你\ldots\ldots 你难道不能走么?我替你挡住他,他绝不会杀我的!''花无缺微微一笑,道:``我这一战并非为了自己,而是为了移花宫\ldots\ldots{}''语声嘎然而止,但言下未竟之意,却又不知有多么沉重。

他缓缓转过身子,忽又回首道:``我还要你知道,我要杀江小鱼,也非为了自己,也是为了移花宫,三个月后,你见着他时,不妨告诉他,我虽然一心要杀他,对他却始终没有怀恨之意,希望他\ldots\ldots 他也莫要恨我。''铁心兰泪流满面,嘶声道:``你为什么做事都要为着别人?你这─生难道是为别人活着的,你\ldots\ldots 你难道不该为自己做些事么?''花无缺已转过身子,仰首望天,突然一笑,道:"为着我自己?──我又是谁呢──?这是他第一次在别人面前表露了自己的悲痛,这虽然是很简单的两句话,但其中的悲痛却比山更重。

铁心兰瞧着他,流泪低语道:``别人都说你是世上最完美、最幸福、最令人羡慕的人,又有谁知道你的痛苦,别人都说你是最镇定、最冷静,又有谁知道你连自己都已迷失,别人都想过你的日子,又有谁知道你竟是为别人活着。''燕南天始终在一旁瞧着,此刻突然大笑道,``花无缺,你果然不愧为移花宫门下!无论这一战你是胜是负,移花宫之声名,都因你面不坠!''花无缺道:``多谢。''

燕南天大声道:``但我也要你知道,除了你外,世上还有许多人,他们所做的事,也并非为了自己的,永远只知为自己活着的人,他们心里也未必便能快乐,甚至说不定比你还要悲哀得多!''花无缺凝目瞧着他,缓缓道:``你要杀死,莫非也是为了别人么?''燕南天默然半晌,突然仰天长啸,似也含蕴着满腔抑郁的悲愤,难以向人叙说。

花无缺叹了口气,突然自怀中抽出一柄银剑。

铁心兰也曾见他交手多次,却从未见他用过兵刃,她几乎以为``移花官''门下都是不用兵刃的。

只见他掌中这柄银剑,剑身狭窄,看来竟似比筷子还细,却长达五尺开外,由头至尾,银光流动,似乎时刻都将脱手飞去!

燕南天目光闪动,对这怪异的兵刃,只淡淡瞧了一服,厉声道:你兵刃既已取出,为何还不出手?``花无缺左手中指轻弹,银剑''铮"的一声龙吟。龙吟未绝,剑已出手!

这柄剑不动时,已是银光流动,眩人眼目,此刻剑光一展,宛如乎天里泼下一盆水银来。

燕南天持剑而立,如山停岳峙,花无缺一剑刺来,他竟是动也不动,但见银光一旋,剑势突然变了方向。原来花无缺那一划本是虚招。

花无缺以虚招诱故,不料对方竟如此沉得住气。

花无缺竟一连使出七剑虚招。

这一连七剑正是``移花宫''剑法中的妙着,虽然皆是虚招,但在如此眩目的剑光下,谁也不敢拿稳这是虚招的,谁都会忍不住去招架闪进,无论他如何招架闪避,却早己全都在这七剑的计算之中。

怎奈燕南天竟丝毫不为这眩目的剑光所动,这七剑虚招中的妙用,在燕南天面前,竟完全发挥不出。

花无缺第七剑方自击出,燕南天掌中铁剑便已直刺而出,穿透满天光影,直刺花无缺胸膛。

这一剑平平实实,毫无花样,但出剑奇快,剑势奇猛,正是自平淡中见神奇,自扎实中见威力!

花无缺剑法纵有无数变化,却也不得不先避开这一着,但闻剑风呼啸,燕南天已刺出叁剑!花无缺避开叁招,才还了一剑。

只见满天银光流动,燕南天似已陷于流光之中,其实这满天闪动的剑光根本无法攻入一着。

花无缺围着燕南天飞驰不歇,燕南天脚下都未移动方寸,花无缺剑如流水,燕南天却如中流之砥柱。

这两人剑法一个极柔,一个极刚,一个飞云变幻,一个刚猛平实,一个如水银泻地,无孔不入,一个却如铁桶江山,滴水不漏。

花无缺看来虽然处处主动,其实处处都落在下风,铁心兰瞧得目眩神迷,已不知身在何处。花林中繁花如雨,落了满地。

小鱼儿寻了个客栈,想好生睡一觉,但翻来覆去,再也睡不着,索性穿起衣服,逛了出去。

偌大的院子,除了小鱼儿外,只有一间屋子住着有人,像是刚搬进来的,屋子里不住有语声传出,门窗却是关得紧紧的。

突见一个青衣大汉闯进了院子,手里还拿着根马鞭,像是赶车的,一走进院子,就大声呼唤着道;``江别鹤江大爷可是在这里么?''小鱼儿吓了一跳,江别鹤怎地也到了这里?他是为什么来的?小鱼儿来不及多想,闪身藏到根柱子后。

只见那屋子的门开了一半,里面有人道:``谁?''那赴车的道:``小人段贵,就是方才送花公子出城的\ldots\ldots\ldots{}''话未说完,江别鹤走了出来,那门却又立刻掩起。

江别鹤皱眉道:``你怎地回来了?又怎会寻到这里?''段贵道:``花公子在城外像是遇着麻烦了,小人赶着回来禀报,恰巧碰到送江大爷到这里来的段富,才知道江大爷到这里来访客了。''江别鹤微微一笑,道:``花公子纵然遇着麻烦,他自己也能对付的,还用得着你着急?''段贵道:``但\ldots 但那人看来却很扎眼,铁姑娘看来像是很着急,小人想,铁姑娘是知道花公子本事的,连铁姑娘都着急了,这麻烦想必不小。''江别鹤沉吟道:``既是如此,我就去瞧瞧吧。江别鹤回首向着屋内道:''至迟今夜,弟子必定再来\ldots\ldots"一面说话,一面已随着段贵匆匆走了出去。

小鱼儿本想瞧瞧那屋子里究竟是谁?行迹为何如此神秘?但想了想,这人反正要在此等江别鹤的,也不急在一时。

他实在想先瞧瞧是谁能给花无缺这么大的麻烦?

小鱼儿和花无缺非但没有交情,而且简直可以说是对头,但也不知怎地,花无缺的事,总是能令小鱼儿心动。

门外有辆马车刚走,江别鹤想必就坐在车子里。小鱼儿尾随了去,但大街上不能施展轻功,两条腿的究竟没有四兵腿的走得快,出城时,马车已瞧不见了。

马车出城,江别鹤在车厢中大声问道:``花公子可曾与那人动过手么?''段贵道:``好像接了一掌。''

江别鹤皱眉道:``这人能接得住花公子一掌,倒也有些功夫,却不知他长得是何模样?''段贵道:``这人又高又大,穿的比小人还破烂,但样子却神气得很。''江别鹤眉头皱得更紧,道:``这人有多大年纪?''段贵道:``看来好像四十上下,又好像有五十多了,但\ldots\ldots 但又好像只有叁十出头,你瞧他有多大年纪,他就像有多大,小人实在没见过这么奇怪的人。''江别鹤皱眉沉吟,面色已渐渐沉重。

段贵忽然又道:``对了,那人腰上,还有柄铁剑,但却已生锈了\ldots\ldots\ldots{}''他话未说完,江别鹤已耸然变色,呆了半晌,沉声道.``你将车远远停下切莫走得太近,知道么?''段贵心里虽然奇怪,不知道他为什么远远就要将车停下,但江大爷的话,他可不敢不听。距离花林还有十余丈,车马便已停住。

只见漫天剑气中,一条人影兔起鹤落,飞旋盘舞,另一条人影却稳如泰山磐石,动也不动。

此刻花无缺身法仍极轻灵,剑气仍盛,似乎并无败象,但江别鹤又是何等眼力,一眼便瞧出花无缺剑式虽极尽曼妙,其实根本攻不进一招!那击剑破风声,更是一强一弱,相隔悬殊。

江别鹤面色更是惨变,喃喃道:``燕南天!这必定是燕南天!''江别鹤知道燕南天此刻只不过是想多瞧瞧移花宫独创一格之剑法的变化而已,否则花无缺早已毙命剑下!

那段贵自然瞧不出此等高深剑法的奥妙,也正是因为他根本什么都瞧不出,所以才更着急。

段贵见到那纵横的剑气,早已为花无缺急出一身大汗,道:``江大爷难道不去助花公子一臂之力么?''江别鹤道:``自然要去的。这车门怎地打不开了,莫非有什么毛病?''段贵跳下车座,去开车门。车门一下子就打开了.一点毛病也没有。

段贵笑道:``江大爷只怕是太过着急,所以连车门都打不开''\ldots"话未说完,突然瞧见江别鹤的一张脸,似已变成青色,眼睛蹬着段贵,目光也似已变为惨青色。

江别鹤阴森森一笑,缓缓道:``一个人最好莫要多管闲事,否则活不长的。''段贵骇得腿都软了,转身就想逃,突觉领子已被一把抓住,整个人都被拖入了车厢。

段贵牙齿格格打战,道:``江\ldots\ldots\ldots 江大爷,小人可\ldots 没\ldots。.没有得罪你老人家,你\ldots\ldots{}''话未说完,一柄短剑已插入他肋下,直没至柄。

江别鹤一分分缓缓拔出了短剑,生怕鲜血会溅上他衣服,短剑拔出,仍如一泓秋水,杀人也不见血。这正是削断``情锁''的那柄宝剑!

江别鹤长长吐出了口气喃喃道,``现在,没有人会知道我曾到过这里,也没有人会知道我眼见花无缺必死而不救了!我侠义的名声,可不能为了这蠢小子而受损\ldots\ldots 你用一条命来保全我江南大侠的名声,死也不算冤枉的。''他一面说话,一面已悄悄溜下马车,转身回去。花林里恶战方急,自然没有人会发现他。

郊外无人,小鱼儿兜了个圈子,终于瞧见了那花林里纵横的剑气,接着才瞧见那辆马车。

他没有瞧见江别鹤。江别鹤莫非还留在马车里?马车为何停得这么远?

小鱼儿本无心去追究这些,只想站得远远地瞧瞧花林里的恶斗,瞧瞧花无缺剑法与众不同的变化,留做以后对付他的准备。

自然,他也想瞧瞧能和花无缺一战的人是谁。

但他突又瞧见那紧闭着的马车门,门缝里在向外流着鲜血──江别鹤莫非已死了?否则这又会是谁的血?

小鱼儿又是兴奋,又是好奇,忍不住想去瞧瞧。

他一拉开中门,就发现段贵那张狰狞扭曲的脸,接着,就瞧见那双满含恐惧、满含惊惶的眼睛。而江别鹤却已不见了。

小鱼儿本也不禁一惊,怔住,但随即恍然而悟──江别鹤用心之狠毒,没有人比小鱼儿更清楚。

他也立刻就发现花无缺此刻情况之危急,铁心兰为花无缺焦急担心的神态,又不禁令他心里一阵刺痛。

突听一声长啸,直冲云霄!一道剑光,冲天飞起,花无缺踉跄后退,终于跌倒!

燕南天竟以至钝至刚之剑,将花无缺掌中至利至柔之剑震得脱手飞去!花无缺但觉气血反逆,终于不支跌倒!

但在这刹那之间,也不知为了什么,小鱼儿但觉热血冲上头顶,竟忘了他与花无缺之间的恩恩怨怨,情仇纠缠\ldots\ldots 他竟突然忘了一切,不顾一切,竟突然飞扑过去!

燕南天长啸不己,铁剑再展。铁心兰失声惊呼──就在这时,突见一条人影如飞掠来,挡在花无缺面前,大声道:``谁也不能伤他!''铁心兰瞧见这人竟是小鱼儿,张大了嘴,惊得呆住。

燕南天目光如电,在小鱼儿身上一转,厉声道:``你是谁?竟敢来撄燕某之剑锋!''铁心兰终于回过神来,大声道:``他就是江小鱼呀!''燕南天失声道:``江小鱼?江小鱼就是你?''他一双眼睛盯在小鱼儿脸上更是不肯放松。

小鱼儿也盯着他,迟疑着道:你``\ldots 你难道就是燕南天燕伯伯?''铁心兰道:``他正是燕老前辈。''

小鱼儿像是又惊又喜,突然扑过去,抱住燕南天,道:``燕伯伯,我可真是想死你了\ldots.''燕南天目中似有热泪盈眶,喃喃道:``江小鱼\ldots\ldots 江小鱼,燕伯伯又何尝不想你?''铁心兰瞧见孤苦飘零的小鱼儿突然有了亲人,而且竟是名震天下的燕南天,心里当真是又惊又喜,热泪又不觉要夺眶而出。

只见燕南天突又推开小鱼儿,沉声道:``你可知道这花无缺乃是移花宫门下!''小鱼儿道:``知道。''

燕南天厉声道:``你可知道杀死你父母的人,就是移花宫主?''小鱼儿身子一震,失声道,``这难道竟是真的?''他很小的时候,虽然曾经有个神秘的人,将他带出``恶人谷'',告诉他这件事,他却总觉得这个人行踪太诡秘,说的话未必可信,所以他一直都没有认为``移花宫''真的是自己不共戴天的仇人。

但如刻这话从燕南天嘴里说出来,他却不能不信了燕南天瞪着小鱼儿,道:``你为何要救他?''小鱼儿道:``我\ldots\ldots\ldots 我\ldots{}''

他自己也实在不知道自己为何要救花无缺,就算``移花宫''和他并无仇恨,他本来也是万万不该救花无缺的!

燕南天突然将铁剑抛在地上,喝道:``你亲手杀了他吧!''小鱼儿身子又是一震,回头去瞧花无缺。

只见花无缺竟已被燕南天剑气震得晕了过去,一条残花,落在他脸上,鲜红的花,衬得他面色更是苍白。

小鱼儿瞧着这张苍白的脸,心里竟泛起一种难言的滋味,他也不知为了什么,竟突然大声道:``我不能杀他!''燕南天怒道:``你为何不能杀他?你已知道他是你仇人门下!何况他又一心要杀你!''小鱼儿道:``我\ldots\ldots 我\ldots。.''

他叹了口气,突又大声道:``我已和他约定,在三个月后决一生死!所以不能让燕伯伯杀死他,更不能在他受了伤时,将他杀死!''燕南天怔了怔,突然仰天大笑道:``好!你果然不愧为江小鱼,果然不愧为我那江二弟的儿子\ldots\ldots 二弟呀二弟,你有子如此,九泉之下,也该瞑目了!''他欢乐的笑声,突又变得无限悲抢。

小鱼儿但觉胸中热血奔腾,突地跪下,嘶声道:``燕伯伯,我发誓今后再也不会丢我爹爹的人了!''燕南天抚着他的肩头,黯然道:``你可是自觉以前所作所为,有些对不起他?''小鱼儿低垂着头,哽咽道:``我\ldots.''

燕南天道:``你用不着难受,更用不着自责,无论谁生长在你那种环境中,都要地你坏得多,何况,据我所知,你用的手段或有不对,却根本未做什么坏事。''燕南天又大笑道:``燕南天能见到江枫有你这样的儿子,正是毕生之快事''他笑声中带着泪痕,显见得心里又是快乐,又是酸楚,铁心兰瞧着他们真情流露,不觉低下了头,眼泪一连串落在地上。

她心里又何尝不是悲欢交集,难以自处。小鱼儿的痛苦还有燕南天了解安慰,她的痛苦又有谁知道?

她死也不能让花无缺杀死小鱼儿,但小鱼儿若是杀死花无缺,她也会难受得很,她只望两人能好好相处。

谁知道他们竟偏偏又是不共戴天的仇人,这仇恨显然谁也化解不开,眼见着他们必有一人,要死在另一人手下!否则这仇恨永远也不能终止!

更令她伤心的是,为了小鱼儿,她不惜牺牲一切,而小鱼儿却似连瞧都不屑再瞧她一眼。

这时燕南天已将小鱼儿拉到花树下坐下,忽然道:``你可知道屠娇娇和李大嘴等人,已离开了恶人谷?''小鱼儿道:``知道。''

燕南天目光闪动,道:``你莫非已见过他们?''小鱼儿点了点头,忽又笑道:``燕伯伯,你饶了他们好么?''燕南天怒道:``我怎能饶了他们!''

小鱼儿道:``他们虽然想害你老人家,但终究没有害着,何况,他们到底将我养大了,更何况他们早巳改过。''燕南天想了想,叹道:``为了你,只要他们此后真的不再为恶,我就饶了他们!''小鱼儿大喜道:``他们听见这消息,简直要高兴死了,以后哪里还会害人!''燕南天瞧了铁心兰一眼,微微笑道:``你现在也该过去和那位妨娘说话了吧,我也不能老是霸占住你。''小鱼儿脸沉下来,道:我不认得那位姑娘,简直连见都未见过。"铁心兰再也忍不住,失声痛哭起来,她痛哭着奔问小鱼儿,但还未到小鱼儿面前,突又转过身子,抚面狂奔而去。

小鱼儿唆紧牙关,也不去拉她。

燕南天瞧着铁心兰奔远,又回头瞧着小鱼几,道:``这究竟是怎么回事?你们这些年轻人的事,我可真弄不清。''小鱼儿也似呆住了,久久都不说话。

燕南天仔细瞧了他两眼,突然长身而起,笑道:``你是要自己闯闯,还是要跟着我?''小鱼儿这才回过神来,展颜笑道:``跟着燕伯伯虽然再好也没有,但别人瞧见燕伯伯就逃,我老是没事做,也没什么意思。''燕南天大笑道:``你果然有志气!''

小鱼儿道:``但我却又想和燕伯伯多聊聊\ldots。.''燕南天道:明日此刻,我还在这里等你,现在我忽然想起有件事要做,已该走了!"他微笑着拍了拍小鱼儿的肩头,拾起铁剑,一掠而去,转眼已无踪影。

小鱼儿倒未想到他说走就走,他竟末留意燕南天所去的方向,是和铁心兰一路的。

他轻轻拾起了花无缺面上的落花,握起花无缺的手掌,暗暗将一股真气自他掌心传了过去,过了半晌,花无缺一跃而起,目光茫然四转,瞧见小鱼儿,吃惊道:``你怎会在这里?''小鱼儿微笑瞧着他,也不说话,听他说话的语声,小鱼儿已知道他方才真气骤然被激反逆,因而晕迷,但究竟功力深厚,并未受着内伤。

花无缺想了想,道:"你救了我?小鱼儿还是不说话。

\hypertarget{ux7b2cux516dux5341ux56dbux7ae0-ux795eux638cux632bux654c}{%
\chapter{第六十四章
神掌挫敌}\label{ux7b2cux516dux5341ux56dbux7ae0-ux795eux638cux632bux654c}}

花无缺默然瞧了他许久,缓缓转过身子,似乎不愿被小鱼儿瞧见自己面上的变化。

他霍然转回身,大声道:``你为何要救我?''

小鱼儿缓缓道:``别人要杀我时,你也曾救过我的。''花无缺道:``但那只因我要亲手杀你!''

小鱼儿眼睛里闪着光,道,``你又怎知我不是要亲手杀死你呢?你莫忘了,我和你在三个月后,还有场不见不散的死约会!''花无缺默然半晌,又长长叹了口气,喃喃道,"不见不散,不死不休\ldots。.。

小鱼儿忽然大笑起来,道:``所以在这三个月里,你我非但不是仇人,而且简直可以算做朋友了。''他笑的声音虽大,但笑声中却似有许多感慨。

花无缺目光凝注着他,久久都未移动,嘴角忽然也泛起一丝笑容,所有的言语,俱在不言之中。

两人同时走出花林,只见繁花大多已被剑气震落,满地惧是落花,有的被风吹动,犹在婀娜起舞。

花无缺忍不住长叹了一声,谁知小鱼儿的叹息声也恰在此时发出,两人忍不住对望一眼,相视一笑。

花无缺心中暗道:``能和此人做三个月的朋友,想必也是人生一快。''他素来深沉寡言,心里这么想,嘴里并末说出。

谁知小鱼儿已笑道:``能和你做叁个月朋友,倒也是人生一大乐事\ldots\ldots{}''花无缺怔了怔,终于忍不住大笑起来。他这一生,几乎从未这样笑过。

只见一辆马车远远停在林外,那匹马显然也是久经训练,是以虽然无人驾驭,此刻仍未走远。

小鱼儿披开车门,指着门里的尸身,道:``你可知道这车夫是被谁杀死的?''花无缺瞪大眼睛,道,``谁?''

小鱼儿想了想,笑道:``我现在说了,你也不会相信,但以后你自然会知道的。''江别鹤一袭青衫,周旋在宾客间,面上虽然满带笑容,但眉目间却隐有忧色,似乎有些心事。

来自合肥的名武师``金刀无敌''彭天寿,年纪最长,被让在首席,此刻手捋着颔下白髯,笑道:``江大侠此刻莫非在惦念着花公子么?''江别鹤苦笑道:``我也知道他绝不会出什么事,但也不知怎地,心中却总似有些警兆\ldots\ldots{}''他长叹一声,接道:``但愿他莫要出事才好,若是他真的遇了危险,我却在此开怀畅饮,却叫我日后还有何面目去见朋友。''群豪间立刻响起一阵赞叹之声。

突听一人大笑接道:``不错,谁若能交着江别鹤这朋友,那真是上辈子积了德了。''爽朗的笑声中,一个身材挺拔,神情洒脱,面上虽有一道又长又深的疤,但看来却带着种说不出的魅力的少年,大步走了上来。

他年纪虽不大,气派却似不小,笑容看来虽然十分亲切可爱,目光顾盼间,竟似全未将任何人瞧在眼里。

群豪竟无一人认得这少年是谁,心里却在暗暗猜测,这想必又是什么名门大派的传人,武林世家的子弟。

江别鹤瞧见这少年,面色却突然大变,失声道:``你\ldots\ldots 你怎会也来!''小鱼儿笑嘻嘻道:``我来不得么?''

江别鹤还未说话,已瞧见了跟小鱼儿同来的人──花无缺也已走上楼,竟微笑着站在小鱼儿身旁。

小鱼儿居然会到这里来,江别鹤已是一惊,花无缺居然还活着,江别鹤又是一惊。

小鱼儿和花无缺同行而来,而且还似乎已化敌为友,江别鹤这一惊更当真是非同小可。

群豪瞧见花无缺,俱都长身面起,含笑招呼,谁也没有发现江别鹤已惊得怔在那里,久久都动弹不得。

他憋了一肚子话想问,却苦于有的话不便问,有的话不能问,怔了许久,才想起该向花无缺表示自己的关心和焦急.只可惜这时他无论想表示什么,都已迟了。

首席的上位,还有几个位子是空着的,大家让来让去,谁也没有坐下去,小鱼儿却大喇喇走过去,坐了下来。

他好像天生就该坐这位子的,别人瞪着了,他脸也不红,眼也不眨,举起酒杯瞧了瞧,忽然笑道:``江大侠请客,难道连酒都没有么?''江别鹤干咳了两声,道;``拿酒来!''

小鱼儿道:``瞧江大侠的模样,好像对我这客人不大欢迎?但我可不是自己要来的,而是花无缺请我来的。''江别鹤面色又变了变,却大笑道:``花兄的客人,便是我的客人。''小鱼儿笑嘻嘻道:``如此说来,花无缺的朋友,也就是你的朋友了?''江别鹤道:``正是如此。''

小鱼儿脸色突然一沉冷冷道:``但花无缺的朋友,却不是我的朋友!''此刻群豪听了小鱼儿和江别鹤的一番话,已全都知道小鱼儿简直和江别鹤连一点关系也没有。

``金刀无敌''彭天寿第一个忍不住了,哼了一声,冷冷道:``这位小朋友,说话倒难懂得很。''``我的意思是说,我若也拿花无缺的朋友当我的朋友,那我可就倒了霉了!花无缺自己人虽不错,他交的朋友\ldots\ldots 嘿嘿,嘿嘿。''小鱼儿冷笑道:``他交的朋友非但见死不救,而且\ldots。.''彭天寿怒道:``你这是在说谁?''

小鱼儿道:``谁是花无缺的朋友,我说的就是谁!''彭天寿怒道:``江大侠也是花公子的至交好友,难道你\ldots{}''小鱼儿冷冷道:``我说的至少不是你!只因你想和花无缺交朋友还不配哩,你最多也不过只能拍拍江别鹤的马屁罢了。''彭天寿``叭''的一拍桌子,厉喝道;``你可知道老夫是谁?''小鱼儿道:``这倒的确不知道。''

彭天寿还未说话,旁边已有人帮腔道:``你连金刀无敌彭老英雄都不知道,还想在江湖混么?''小鱼儿道:``彭老英雄的名字,若是换成马屁无敌,岂非更是名副其实。''在江别鹤的酒筵上,彭天寿本来还有些顾忌,但直到此刻,江别鹤非但全未劝阻,简直好像没有听见这等吵闹似的。

彭天寿自然不知道这是江别鹤希望小鱼儿结的仇家越多越好,还道江别鹤有心替他撑腰。

听了``马屁无敌''这四字,他哪里按捺得住,虎吼一声,隔着桌子便向小鱼儿扑了过去。

小鱼儿根本就是存心闹事来的,笑嘻嘻地瞧着彭天寿扑过来,突然举起筷子,轻轻一点。

彭天寿只觉身子突然发麻,再也使不出力,``砰''的一声,整个人竟都跌在桌子上,碗筷杯盏溅了一地。

小鱼儿笑嘻喀道:``江别鹤,你难道舍不得上菜,要拿这马屁精来当冷盘么?''群豪中和彭天寿有交情的也不少,坐得远远的,已在纷纷呼喝,坐得近的,已想动手了。

花无缺静静地瞧着江别鹤,江别鹤还是全无丝毫劝解之意,这些客人竟像是全非他请来的。

只因他此刻正也希望情况越乱越好,只听哗啦啦─声,彭天寿从桌上滚了下来,桌子也翻了,几个人冲上来,全都被小鱼儿拎住脖子,甩了出去,店小二一旁惊呼,忙着收碟子收碗,酒楼上顿时乱做一团,但群豪瞧见小鱼儿的武功后,反而没有一个人真的敢过来动手了。

江别鹤这才皱眉道:``花兄,你瞧这事,该当如何处理?''花无缺淡淡一笑,道:``我不知道。''

江别鹤再也想不到他会说出这句话来,不禁又是一怔,只听拳风震耳,小鱼儿已一拳直击过来,大喝道:江别鹤,他瞧见花无缺有难,赶紧溜走,还怕那赶车的泄露你的不义,竟将他也杀死灭口,今天我别的不想,只想痛痛快快揍你一顿,你就接着吧。"一面说,一面打,说完了这番话,已击出数十拳之多。

江别鹤居然只是闪避,也不还手,等他说完了,才冷冷道:``阁下血口喷人,只怕谁也难以相信。''小鱼儿喝道:``告诉你,那赶车的虽然挨了你一剑,但却没有死\ldots\ldots\ldots{}''江别鹤面色不禁一变。

小鱼儿忽然后退几步,大喝道:``你瞧,他已从那边走过来了!''群豪不由自主,全都沿着他手指之处瞧了过去。

江别鹤却冷笑道;``你骗不过我的,他\ldots\ldots{}''说到这里骤然住口,脸色突然变得苍白。

小鱼儿大笑道:``我的确是骗不过你的,别人都回头.只有你不回头,因为只有你知道他是活不了的,是么?''他方才乱七八糟的闹了一场,一来是要镇住别人,再来也是要让情况大乱,要江别鹤定不下心来,否则他又怎会上这个当。

江别鹤目光一扫,只见群豪面上果然都已露出惊讶怀疑之色,他一步窜到花无缺面前,道:``花兄,你是相信他,还是相信我?''花无缺叹了一口气,道:``此事不提也罢\ldots\ldots{}''小鱼儿大声道:``无论提不提此事,我要和他打架,你是帮他,还是帮我?''花无缺苦笑道:``你两人若是定要比划比划,谁也不能多事插手。''小鱼儿就在等他说这句话,立刻大声道:``好,假如有别人插手,我就找你!''话未说完,又是一拳击出。

江别鹤瞧他方才打了数十拳,也未沽着自己一片衣服,看来武功也不过如此,冷笑道:``既然阁下定要出手,也怪不得江某了!''两句话说完,小鱼儿又已攻出四拳之多。

只见江别鹤一掌击出,掌风凌厉,掌式都是飘忽无方,小鱼儿橡是用尽了身法才堪堪避开。群豪又忍不住为江别鹤喝起彩来。

江别鹤知道江湖中人,胜者为强,只要自己伤了小鱼儿,也就不会有人再来追究方才杀人的事了.他精神一震,冷笑着又道:``江湖朋友全都在此见着,这是你自取其辱,并非江某以大压小。''小鱼儿像是只顾得打架闪避,连斗嘴的余力都没有了,拆了还不到二十招,他已屡遇险招。

江别鹤本来一直怀疑他就是在暗中和自己捣鬼的那人,是以怀有戒心,此刻见他武功竟是如此稀松平常,疑心顿减,攻势也顿时松了下来,微笑道:``你虽然不知好歹,无理取闹,但我念在你年幼无知,也不愿太难为你,只要你肯赔罪认错,瞧在花兄面上,我就放你走如何。''他这话说得非但又是大仁大义,而且也卖给花无缺个交情.不折不扣正是``江南大侠''的身份。

小鱼儿不住喘气,像是连话都说不出了。

其实他早巳算定,在这许多人面前,江别鹤只要能摆摆``大侠''的身份,就绝不会放弃这种机会的。

他算准了在这许多人面前,自己装得越弱,江别鹤越不会使出煞手,否则岂非是失了``大侠''的风度。

江别鹤出手果然更平和了,群豪却有人呼喝道:``对这种人,江大侠你又何必太客气。''方才挨过小鱼儿揍的,更是随声附和。

江别鹤像是被逼无奈,叹口气道:``你年纪轻轻,我实在不愿伤你,但若不给你个教训,连别的朋友也瞧不过眼的''\ldots."说话间,小鱼儿又被逼退几步。

江别鹤微笑道:``我这一着分花拂柳后,便要取你胸膛,你可得小心了!最好莫要闪避招架,否则我出手一重,难免要你了你。''小鱼儿道:``多承指教!''

只见江别鹤一招``分花拂柳''后,右掌突然斜击而出,掌式如斧开山,直取小鱼儿胸膛。这一掌说来虽然没什么奥妙,但掌式变化之快,却是无与伦比,纵然他已先将自己招式喝破,但群豪还是想不到他掌式竟能变到这部位来,眼见小鱼儿是再也避不开这一掌的了。

群豪又不禁喝起彩来。小鱼儿突然出手硬接了这一掌!

江别鹤突觉一般大力涌来,再想使出全力,已来不及了,``砰''的一声,他身子竟被震得飞了起来!

小鱼儿忍了多年的怒气,终于在这一掌里发泄!

只见江别鹤身子撞入人丛,站在前面的几个人,也被他撞得一起跌倒,踉跄后退几步,才坐到地上!

群豪喝彩声夏然顿住,一个个张口结舌,怔在那里,只见小鱼儿拍掌大笑,竟穿过窗户扬长而去了!

小鱼儿虽未能真个痛揍江别鹤一顿,但江别鹤大大出了个洋相,也算出了口气,心里觉得再愉快也不过。

``见好就收''这句话,小鱼儿当然清楚得很。

群豪就算还不十分相信江别鹤真的``见死不救,杀人灭口'',至少心里已有些怀疑。

他在街上逛了一围,又溜进了那客栈,在白天订好的那间屋子里歇了一会儿,等到院子里没有人声,才溜出来。

只见住着那神秘人物的屋子,门窗仍是紧紧关着的,屋子里已燃起了灯火,却瞧不见人影。

小鱼儿四下瞧了一眼,纵身掠上了屋脊,悄悄溜到这间屋子的屋檐上,伏在屋檐的暗影里,动也不动。屋子里也没有丝毫声音。这神秘的人物已睡着了,还是已走了?江别鹤和他已订有后约,他怎么会走呢?何况屋子里的灯,还是亮着的。

小鱼儿沉住了气,等在那里,他算定江别鹤绝不会不来,满天星光,夜凉如水,等着等着,他几乎睡着了。

突听``嗖''的一声,一条人影,轻烟般掠来,那轻功之高,小鱼儿简直连见都没有见过。

他简直瞧不见这人的身形,心里刚吃了一惊,只听房门轻轻一响,这人竟已走进了屋子。

屋子里还是没有声音。

这人的轻功竟如此高明,莫说自己比不上,就连花无缺比他也似差了一筹,武林中又怎会有这样的人物!

这样的人物在和江别鹤勾结,岂非可怕得很!小鱼儿想着想着,突然又瞧见一个人溜进了院子。

只见他一路东张西望,悄悄走了过来,也走到这间屋子前面,轻轻咳嗽了一声,敲了敲门。

屋子里立刻有人祝声道:``谁?''

这黑衣人低声道:``是晚辈。''

听这声音,小鱼儿才知道是江别鹤来了,精神不由一振,这时门开了一线,江别鹤已闪身走了进去。两人说了几句话,小鱼儿也末听清。

忽听江别鹤道:``晚辈今日倒瞧见了惊人之事。''那人道:``什么事?''

江别鹤道:``燕南天并未死,而且又出世了!''江湖中无论是谁,听到这消息都难免要大吃一掠,那人却似无所谓,语声似是淡淡的,道:``哼,燕南天不死最好,他若死了,反倒无趣了。''小鱼儿越听越谅讶,这人非但对燕南天毫不畏惧,反倒有和燕南天较量较量的意思。

江湖中敢和燕南天一较高低的人,有谁呢?小鱼儿简直连一个也想不出来。

只听江别鹤又道:``除了燕南天之外,那江小鱼居然也现身了!''那人对江小鱼的兴趣,竟似比对燕南天浓厚得多,道:``他武功怎样?比起花无缺如何?''江别鹤笑道:``他武功纵然比不上花无缺,但动起手来,诡计多端,只要稍为疏忽,便要上他的当。''那人居然好像微微笑了笑,道:``我正担心他武功太差,如今才放心了!''小鱼儿听得更是奇怪,他再也想不通这人为何对他如此有兴趣,难道这么样的人会认得他?

只听那人又道:``江小鱼武功无论多强,都有花无缺去对付,用不着你担心。''江别鹤叹了口气,道:``但现在花无缺却似和江小鱼交起朋友来了\ldots\ldots\ldots{}''那人冷笑道:``这两人是天生的冤家对头,不死不休,就算交朋友,也绝对交不长的,这点你只管放心。''小鱼儿吃了一惊!这人怎会对花无缺和自己的事如此清楚?

知道这件事的人实在并不多呀。

江别鹤似乎笑了笑,道,``既是如此,前辈对弟子不知究竟有何吩咐?''那人道:``我只要你\ldots。.''

语声突然低了下去,小鱼儿连一句话都听不清了,只听得这人说一句,江别鹤就答一声;``是。''等到这人说完了,江别鹤笑道:``这几件事,晚辈无不从命。,那人冷冷道;''这几件事对你也有好处,你自然要从命的!``江别鹤沉吟着,又笑道;''前辈只吩咐了一声,晚辈立刻就遵命而来,但直到此刻为止,却连前辈的高姓大名都不知道。``那人叱道;''我的名字,你用不着知道,你只要知道普天之下,除了我之外,已没有别人能帮你的忙,若没有我,你非但做不成大侠,简直连活都活不成了!``江别鹤默然半晌道:''是。"

那人道:``你现在可以走了,到时候我自然会去找你。''江别鹤道:``是!''

那人又道:``我交给你办的几件事,你若出了差错,那时不用燕南天和江小鱼动手,我自己就宰了你!你知道么?''江别鹤道,``是!''

\hypertarget{ux7b2cux516dux5341ux4e94ux7ae0-ux795eux51faux9b3cux6ca1}{%
\chapter{第六十五章
神出鬼没}\label{ux7b2cux516dux5341ux4e94ux7ae0-ux795eux51faux9b3cux6ca1}}

只见江别鹤垂首走出了门,身法立即变快,四顾无人,一闪就出了院子,小鱼儿眼珠子一转也悄悄自屋檐上溜开。

小鱼儿直跃出几重屋脊,才敢一掠而下,从角门穿出院子,找着厨房,炉火还有余烬,上面还烧着一壶水。

他拎着这壶水,才大摇大摆地走回去,那间屋子里的灯火,果然还是亮着的,小鱼儿过去,拍门道:``客官可要加些茶水么?''他一心想瞧瞧这神秘人物的真面目,竟不惜涉险,扮成茶厨,也不管这人会不会认得出他,屋予里竟没有应声。

他壮起胆子,轻轻推门。门竟没有拴上,他一推就开了。

只见桌上燃着灯,灯旁有个盘子,盘子里有个茶壶,四只茶杯,茶壶和茶杯全没动过。

再瞧那张床,床上的被褥,也是叠得整整齐齐的。

这神秘的人虽然住在这屋子里,但却连动都没有动这屋子里的东西,他显然只不过是借这间屋子来和江别鹤说话而已。

小鱼儿却喃喃道:``壶里不知还有茶没有,我不如先给斟上吧,也免得客人回来没水喝。''他一面说,一面已走进房子。

一走进门.他才发觉屋子里竟弥漫着一种如兰如馨的奇异香气,他竟像是一步踏上了百花怒放的花丛中。

但除了这奇异的香气外,屋子里却再也没有丝毫可疑的痕迹,这屋子简直好像从来就没有人住过。

但这屋子却打扫得一尘不染,连床底下的灰尘,都被打扫得干干净净,桌子、椅子、衣橱,都像是被水洗过。

就连那石板铺成的地,都被水洗得闪闪发光。

那神秘的人物,既然只不过用这屋子作谈话之地,并不想在这里住,也没有沾这里的东西,却又为何要将这屋子洗刷得如此干净,而且还在屋子里散布出如此神秘、又如此珍贵的香气。

这神秘的人物,莫非有种特别的洁癖。小鱼儿不禁又皱起了眉头,喃喃道:``这么爱干净的人,倒也少见得很''突听一人冷冷道;``你是谁?来干什么?''

这声音竟赫然就是从小鱼儿身后发出来的!小鱼儿心里这一惊当真不小,嘴里却含笑道:``小的是来瞧瞧,客官是不是要添些茶水。''那人道:``你是这店里的伙计?''

小鱼儿赶紧道:``是。''

那人道:``白天来的,好像不是你。''

小鱼儿道:``钱老大当日班,小的王三是值夜的。''那人突然冷冷一笑,道:``江小鱼果然是随机应变,对答如流,只可惜你出娘胎,我就认得你,你在我面前装什么都没有用的!''小鱼儿大骇道:``你是谁?''那人又不说话。

小兔儿霍然转身,身后空空的,那扇门还在随风而动!门外夜色深沉,哪里有人的影子!那人莫非又走了!

小鱼儿又惊又奇,刚松了口气,谁知身后又有人冷冷道:``你瞧不见我的!''那人竟又已到了他身后!小鱼儿连转五六个身,他身法已不能说不快了,但那人竟始终在他身后,就好像贴在他身上的影子似的。小鱼儿就算胆子再大,此刻也不禁被骇出了身冷汗。此人轻功如此,武功可想而知,小鱼儿知道自己非但万万不能抵敌,连逃都逃不了的。

他眼珠子一转,索性站住不动了,笑嘻嘻道:``你若不愿被我瞧见,为何要来呢?''那人道:``你想不出?''小鱼儿眨着眼睛,道:``我想,你总不会是要杀死我吧。''那人道:``你怎知我不杀你?''小鱼儿道:``一个马上要死的人,就算瞧见你的真面目,也没什么关系,所以你若要杀我,就不妨让我瞧瞧了,是么?''他已隐约觉出这人的确没有杀他之意,胆子不觉大了起来,瞧里说着话,突然一步窜到衣橱前。

那衣橱油漆本就很新,又被仔细擦洗了一遍,更是光亮如镜,小鱼儿身子往下一蹲,一个白衣人影,便清清楚楚地映在衣橱上。

只见这人长发披肩,白衣如雪,神情飘飘有出尘之概,但面上却戴着个狰狞恐怖的青铜面罩。

小鱼儿又不禁骇了一跳,失声道:``你原来就是铜先生!''小鱼儿只觉他一双眼睛正狠狠瞪着自己------这双眼睛的光射到衣橱上,再反射出来,仍是冷森森的令人悚栗。

小鱼儿强笑道:``那日黑蜘蛛说你武功如何如何之高,我还有些不信,今日一见,才知道他不是吹牛的。''铜先生冷笑道,``你用不着奉承我,我既不想杀你,就永远不会杀你。''小鱼儿道:``永远不会?''铜先生道:``嗯!''小鱼儿松了口气,笑道:``我见你这样爱干净,又弄出这香气,本来以为你是个女人\ldots\ldots 幸好你不是女人,否则你就算说不杀我,我也不相信。''铜先生道:``你不相信女人?''小鱼儿笑道:``妇人之言,绝不可听,谁若相信女人,谁就倒霉了!''铜先生突然怒道:``你母亲难道不是女人?''小鱼儿道:``天下的女人,有谁能和我母亲相比,她又温柔,又美丽\ldots。''他虽从未见母亲之面,但在每个孩子的心目中,自己的母亲,自然永远是天下最温柔、最美丽的女人。

他说着说着,不觉闭起了眼睛,依着他的幻想,描叙起来,他口才本好,此番一描叙,更是将自己的母亲说得天下少有,世间无双。

铜先生冷漠的目光中,却似突然燃起了火焰。

小鱼儿也未瞧见,犹在梦呓般道:``世上别的女人,若和我母亲相比,简直连粪土也不如,我\ldots。.''话未说完,突觉脖子上一阵剧痛,身子一麻,整个人竟都已被这``铜先生''提了起来!

以小鱼儿此时的武功,竟无还手抗拒之力!

只见铜先生目中满是怒火,冰凉的手掌,越来越紧,竟似乎要将小鱼儿的脖子生生拗断。

小鱼儿大骇道:``你\ldots 你说过永远不杀我的.说出来的活怎能不算。''铜先生道:``只因你满嘴胡说八道,令人可恨。''小鱼儿道:``我几时胡说八道了?''

铜先生道:``你母亲是好是坏,是美是丑,你根本未见过,如此为她吹嘘,不是胡说八道是什么!''小鱼儿道;``你\ldots\ldots 你怎知我未见过我母亲的面?''铜先生冷笑道:``我不知道谁知道?''

小鱼儿忍不住道:``我母亲长得是何模样?''

铜先生道:``你母亲跛脚驼背,又麻又秃,乃是世上最丑最恶的女人,世上无论哪一个女人都比她好看得多。''小鱼儿大怒道:``放屁放屁,你才是胡说八道!''话末说完,脸上竟挨了两个耳掴子。

铜先生这两掌虽未使出真力,但已将小鱼儿两边都打得肿了起来,鲜血不住自嘴角沁出。但小鱼儿仍是骂不绝口。

他虽未见过母亲,但只要一想起母亲,心里就会有种说不出的滋味,是痛苦,也是温馨。

他平日虽然最喜见风转舵,所以这``铜先生''若是辱骂了他,他自知不敌,也绝不会反抗还嘴。但辱骂了他的母亲,他却不能忍受。

铜先生耳括子打个不停,小鱼儿还是骂个不停,他牛脾气一发,什么死活都全然不管不顾。

铜先生咬牙道;"你再敢骂,我就杀了你。,

小鱼儿满嘴流血,嘶声道:``只要你承认我母亲是最温柔、最美丽的,我就不骂你。''铜先生道:``你\ldots\ldots 你死也不肯承认你母亲是最丑最恶的女人?''小鱼儿立刻点头。

铜先生道:``你\ldots\ldots 你情愿为她死?''他眼睛里充满怨毒,语声却渐渐颤抖。

只见这``铜光生''站在那里,全身抖个不停。

小鱼儿偷偷瞧着他,却也不敢妄动,过了半晌,才终于忍不住道:``我母亲究竟与你有什么仇恨,你要如此骂她?''铜先生竟似完全没有听见他的话。

小鱼儿再不迟疑,纵身一跃,跳出窗户,转首瞧了瞧,那铜先生似乎并没有追出来,小鱼儿心里虽然有许多怀疑不解,此刻却也顾不得了,展开身法,没命飞掠,眨眼间使已掠出了客栈。

突听身后一人冷冷道:``你还不承认?''

小鱼儿身子刚掠起,又跌下,他知道只要被这人追着,便如附骨之蛆,再也休想甩得脱了,突然大喝道:``你有本事,就宰了我吧!''喝声中,他猝然转身,双拳雨点般击出,但他连对方的人影都未瞧见,背后一麻,身子又跌到地上。

花无缺本不喜欢喝酒,今夜也不知怎地,竟然自酌自饮起来,而且酒到杯干,喝得迷迷糊糊的,往床上一倒,便睡着了。

这时窗外正有人在呼唤!``花无缺!醒来!''

声音虽轻细,但每个字却似能送人花无缺耳朵里。

花无缺定了定神,便推开了窗子,窗外夜色朦胧,一个白衣人影,鬼脸般站在五六丈外。

淡淡的星光映照下,这人的脸上似乎发着青光,仔细一瞧,才发觉他脸上竟戴着个狰狞的青铜面具.花无缺一惊,失声道:``莫非是铜\ldots\ldots 铜先生?''那人点了点头,道:``出来!''

铜先生已飘上了屋脊。花无缺跟了过去,掠过屋脊,越过静寂的街道。

铜先生头也不回,忽然冷冷道:``移花宫门下,怎地也贪酒贪睡起来!''花无缺怔了怔,垂下头不敢说话。

只见这铜先生从头到胸,从未动弹,飞掠却迅急无比,整个人都仿佛在驭风而行一般。花无缺瞧见这样的轻功,也不禁暗暗吃惊。

只听铜先生又道:``你自然已知道我是谁了。''花无缺道:``晚辈出宫时,家师已吩咐过,只要见到先生,便如见家师,先生所有指示,晚辈无不遵命。''铜先生道,``你出宫时,宫主还曾吩咐了你什么?''花无缺终于沉声道:``家师要我亲手杀死一个叫江小鱼的人!''铜先生像是笑了笑,道:``很好!''

他不再说话,也始终未曾回过头来,只见去路渐僻,渐渐到了个山坡,山坡上有株枝叶浓密的大树,铜先生身形突然飞掠而起,口中却道:``你在树下站着!''短短五个字说完,他身子已站在树梢,满天星光,衬着他一身雪白的衣裳,看来更觉潇洒出尘,高不可攀。

突见铜先生自浓密的枝叶中,提起一个人,叱道:``接稳了!''叱声方自入耳,已有一个人自树梢急坠而下。

这大树高达十余丈,一个重量虽不满百厅,自树梢被抛下来,那力量何止五百厅。

花无缺更猜不出他抛下的这人是谁,也没有把握能否接得住这人的身子,刹那间不及细想,也飞身迎了上去。

花无缺突然出手,捞住了这人的衣带,但闻``嘶''的一声,这人衣裳已被撕破,花无缺也被这下坠之力带了下来。

但等到落地时,下坠之力已减,花无缺口中吆喝一声,临空一个翻身,又复将这人身子直抛上去。

等到这人第二次落下时,花无缺伸出双臂,便轻轻托住,满天星光,映着这人苍白的脸,紧闭着的眼睛。

这人猛然竟是小鱼儿!花无缺虽然深沉镇定,此刻也不禁惊呼出声。

铜先生犹自站在树梢,冷冷道:``他是否是江小鱼?''花无缺道:``不错。''

铜先生道:``好,你杀了他吧!''

花无缺心头一震,垂首瞧着晕迷不醒的小鱼儿,嘴里只觉有些发苦,一时之间,竟呆住了。

铜先生缓缓道:``你若不愿杀一个没有反抗之力的人,不妨先解开他的穴道!''花无缺茫然伸手,拍开了小鱼儿的穴道,小鱼儿张开眼睛,瞧见了花无缺,展颜笑道:``是你救了我?''花无缺呆在那里,一个字也说不出。

小鱼儿笑道:``我早就知道你会来救我的,我们是朋友。''花无数也不知为了什么,心里只觉一酸,竟扭转了头去。

突听一人冷冷道:``花无缺,你为什么还不动手?''"小鱼儿这才瞧见站在树梢的铜先生,倒抽了口凉气,转首面对着花无缺,眼睛瞪得大大的。\ldots{}

花无缺长长叹了口气。小鱼儿默然半晌,苦笑道:``我知道你不敢违抗他的话\ldots\ldots 好,你动手吧!''花无缺也默然半晌,一字字缓缓道:``我现在不能杀你!''小鱼儿一喜,铜先生怒道:``你忘了你师父的话么?''花无缺长长吐了口气,道:``我已和他订了三个月之约,未到约期,绝不能杀他!''铜先生喝道:``你的师父若是知道这事,又当如何?''花无缺霍然抬头,大声道:``师命虽不可违,但诺言也不可毁,纵然家师此刻便在这里,也不可能令晚辈做食言背信的人!''铜先生怒道:``花无缺你莫忘记,见我如见师,你敢不听我的话?''花无缺叹道:``先生无论吩咐什么,弟子无不照办,只有此事,却万万不能从命。''铜先生忽然大喝道:``你不杀他,只怕并非为了要守诺言,只怕还另有原因?是么?''花无缺心里又是一震,他自己也不知道自己坚持不杀小鱼儿,到底是完全为了要守诺言,还是另有原因。

方才小鱼儿无助地躺在他怀里,他心里竟忽然泛起一阵难言的滋味,他瞧着小鱼儿的脸,忽然觉得这不是他的仇人,而是已相交多年的亲密的朋友。

他手臂上感觉到小鱼儿微弱的呼吸,又觉得这不是他要杀的人,而是他本应全力保护的。

直到小鱼儿跌到地上,这份奇异的感觉,还留在他心里,再瞧见小鱼儿那充满信任的笑容,他现在又怎能动手!

花无缺长长叹了口气,他自己心里,却丝毫不觉和小鱼儿有何仇恨,他自己也说不出这种奇异的感觉,是在什么时候发生的。

这份感觉,像是久久以前便已隐藏在他心底,只不过等到小鱼儿的肌肤触及他的肌肤时,才被引发。

他瞧着小鱼儿,心里喃喃自语:``江小鱼,江小鱼,你心里在想什么?你想的可是和我一样?''小鱼儿也在凝注着他,心里的确也在沉思。

铜先生自树梢瞧下来,瞧见这并肩站在一起的两个人,冷漠的目光,又变得比火还炽热,厉声道:``花无缺,莫要再等三个月了!现在就动手吧!''小鱼儿突然抑首狂笑道,``为什么不能再等三个月?你怕三个月后,他更不会动手了吗?''铜先生嘶声道:``我怕什么!你两人是天生的冤家对头,你们的命中已注定,必有一个人要死在另一人的手上!''小鱼儿大吼道:``既然如此,你现在为何还要逼他,你若想我现在就死,就自己动手吧\ldots。·你自己为何不敢动手?''铜先生像是被人一刀刺在心上,长啸着一掠而下。

\hypertarget{ux7b2cux516dux5341ux516dux7ae0-ux9ad8ux6df1ux83abux6d4b}{%
\chapter{第六十六章
高深莫测}\label{ux7b2cux516dux5341ux516dux7ae0-ux9ad8ux6df1ux83abux6d4b}}

花无缺面上变了颜色,只道他将向小鱼儿下手,谁知他竟长啸着扑入树林,举手一掌,将一棵树生生震断!

只见他身形盘旋飞舞,双掌连环拍出,片刻之间,山坡上一片树木,已被他击断了七八株之多,连着枝叶倒下,发出一阵震耳的声响。

小鱼儿瞧见这等惊人的掌力,也不禁为之舌矫不下。

他知道这铜先生的武功,若要杀他,实是易如反掌。他也知道这铜先生对他实已恨到极点,恨不得将他碎尸万段,千刀万剐,但铜先生竟偏偏不肯自己动手,宁可拿这些木头来出气。

这究竟是为的什么?岂非令人难解!

心念闪动间,铜先生已掠到花无缺面前,厉声道:``你定要等到三个月后才肯杀他,是么?''花无缺深深吸了口气,道:``是!''

铜先生忽然狂笑起来,道:``你既重信义,我身为前辈,怎能令你为难,你要等三个月,我就让你等三个月又有何妨?''这变化倒又出人意料之外,花无缺又惊又喜。

铜先生顿住笑声,道:``现在,你走吧。''

花无缺又瞧了小鱼儿一眼,道:``那么他\ldots.。''铜先生道:``他留在这里!''

花无缺又一惊,道:``先生难道要\ldots\ldots{}''

铜先生冷冷道:``无论他会不会失信,这三个月里,我都要好好的保护他,不使他受到丝毫伤损,三个月后,再将他完完整整地交给你\ldots\ldots{}''小鱼儿笑嘻嘻道:``要你如此费心保护我,怎么好意思呢?''铜先生道:``保护你这么样一个人,还用得着我费心么?''小鱼儿笑道:``你以为我很容易保护,你可错了,我这人别的毛病没有,就喜欢找人麻烦,江湖中要杀我的人,可不止一个。''铜先生道:``除了花无缺外,谁也杀不了你!''小鱼儿叹了口气,道:``你话已说得这么满,在这三个月里,我若受了损伤,可真不知道你有什么面目来见人了。''铜先生喝道:``在这三个月里,你若有丝毫损伤,唯我是问。''小鱼儿大笑道:``那我就放心了,在这三个月里,我无论做什么,都没关系了,反正任何人都伤不了我。''铜先生冷冷道:``你只管放心,在这三个月里,你无论什么事,都做不出的。''小鱼儿眨了眨眼睛,笑嘻嘻道:``那倒未必\ldots\ldots{}''花无缺想到小鱼儿的刁钻古怪,精灵跳脱,铜先生武功纵高,若不想上他的当,怕真不容易。想到这里,花无缺竟不知不觉笑了起来。

铜先生怒道:``你还不走?等在这里做什么?''

小鱼儿截口道:``你放心走吧,三个月后,我会在那地方等你的!''他转向铜先生,笑着又道:``但现在我想和他悄悄说句话,你放不放心?''铜先生冷冷道:``天下根本没有一件可令我不放心的事。小鱼儿皱了皱鼻子,笑道:''你本事虽不算小,但牛也未必吹得太大了。``铜先生忽道:''你敢无礼?"

小鱼儿大笑道:``我为何不敢,在这三个月里,反正没有人能伤到我的,是么?''铜先生气得呆在那里,竟动弹不得。

小鱼儿走到花无缺面前,悄声笑道,``只可惜他戴个鬼脸,否则他现在的脸色一定好看得很。''他虽然故意压低声音说话,但却又让这语声刚好能令铜先生听到,花无缺几乎忍不住又要笑出来,赶紧咳嗽一声,道:``你要说什么?''小鱼儿道:``明天下午,燕南天燕大侠在今天那花林等我,你能不能代我去告诉他,我不能赴约了。''他这次才真的压低了语声。

花无缺皱了皱眉,道:``燕南天?\ldots.''

小鱼儿叹道:``我知道你跟他有些过不去,所以你纵不答应我,我也不会怪你。''花无缺忽然一笑,道:``这三个月,你我是朋友,是么?''小鱼儿目视了他半晌,笑道:``你很好,结交你这朋友,总算不冤枉。''花无缺默然许久,淡淡道:``可惜只有三个月。''他故意装出淡漠之色,但却装得不太高明。

小鱼儿笑道:``天下有很多出人意料的事,这些事每天都有几件发生,说不定我过两天就能看见你也未可知。''花无缺叹道:``我总不相信奇迹。''

小鱼儿笑道:``我若不相信奇迹,你想我现在还能笑得出么?''忽听铜先生冷冷道:``奇迹是不会出现的!花无缺,你还不走么?''小鱼儿瞧着花无缺走得远了,才叹息着道:``一个人若是非死不可,能死在他手上,总比死在别人手上好得多了。''铜先生喝道:``你不恨他?''小鱼儿道:``我为何要恨他?''铜先生道:``他的尊长,杀死了你的父母!''

小鱼儿道:``我父母死的时候,他只怕还未出生哩!他师父做的事,与他又有何关系,他师父吃了饭,难道还能要他代替拉屎么?''小鱼儿说出这番话,铜先生竟不禁怔住了。小鱼儿凝目瞧着他,忽然笑道:``我问你,你为何要我恨他?铜先生忽道,''你恨不恨他,与我又有何关系?``小鱼儿道:''是呀,我恨不恨他,和你没关系,你又何苦如此关心?``铜先生竟没有说话。小鱼儿微笑道:''他竟要亲手杀死我,而又说不出原因来,我本已觉得有些奇怪,现在更是越来越奇怪了。``铜先生道:''你虽不恨他,他却恨你,所以要杀你,这有什么好奇怪的?``小鱼儿笑道:''你以为他真的恨我么?"

铜先生身子竟似震了震,厉声道:``他非恨你不可!''小鱼儿叹道:``这就是我所奇怪的,你和他师父,要杀我都很容易,但你们却都不动手,所以我觉得你们其实也并不是真的要我死,只不过是要他动手杀我而已,你们好像一定要看他亲手杀我,才觉得开心。''铜先生道:``要他杀你,就是要你死,这又有何分别?''小鱼儿道:``这是有分别的,而且这分别还微妙得很,我知道这其中必定有个很奇怪的原因,只可惜我现在还猜不出而已。''铜先生道:``这秘密普天之下,只有两个人知道,而他们绝不会告诉你!''小鱼儿眼睛里像是有光芒一闪,却故意沉吟着道:``移花宫主自然是知道的''\ldots.``铜先生道:''自然。``小鱼儿大喝道:''移花宫主便是姐妹两人,你既然说这秘密天下只有两个人知道,那么你又怎会知道的?``铜先生身子又似一震,大怒道:''你说的话太多了,现在闭起嘴吧!"他忽然出手,点住了小鱼儿的穴道,小鱼儿只觉白影一闪,连他的手是何模样,都未瞧出。

这神秘的``铜先生'',非但不愿任何人瞧见他的真面目,甚至连他的手都不愿被人见到!

花无缺心里又何尝没有许多怀疑难解之处,只不过他心里的事,既没有人可以倾诉,他自己也不愿对别人说。

天亮时,闷酒又使他朦胧睡着,也不知睡了多久,院子里忽然响起了一阵骚动声,才将他惊醒了。

他披衣而起,刚走出门,便瞧见江别鹤负手站在树下,瞧见他就含笑过来,含笑道:``愚兄昨夜与人有约,不得已只好出去走了走,回来时才知贤弟你独自喝了不少闷酒,竟喝醉了。他非但再也不提昨夜在酒楼上发生的事,而且称呼也改了,口口声声''愚兄"``贤弟''起来,好像是因为那些事根本是别人在挑拨离问,根本不值一提------这实在比任何解释都好得多。

花无缺目光移动,道:``现在不知是什么时候了?''江别鹤笑道:``已过了午时。''

花无缺失声道:``呀,我一觉睡得竟这么迟\ldots\ldots{}''他一面说话,一面匆匆回屋梳洗。

江别鹤也跟了过去,试探着道:``愚兄陪贤弟出去逛逛如何?''花无缺笑道:``小弟已在城里住了如此久,江兄还担心小弟会迷路么?''江别鹤在门口又站了半天,才强笑道:``既是如此,愚兄就到前面去瞧瞧段姑娘了。''他似乎已发觉花无缺对他有所隐瞒,嘴里不说,心里已打了个结,走到院子里,就向两个人低低嘱咐了几句。

那两条大汉齐声道:``遵命。''

江别鹤瞧着他们奔出院外,嘴角露出一丝狞笑,喃喃道:``花无缺呀花无缺,我虽然一心想结纳于你,但你若想对不起我,就莫怪我也要对不起你了!''花无缺像是在闲逛。只见他在一家卖鸟的铺子前,听了半天鸟语,又走到一家茶食店,喝了两杯茶,吃了半碟椒盐片。路上立刻就有个人,回去禀报江别鹤。

江别鹤沉吟道:``喝茶\ldots。.他一个人会到茶馆里去喝么?难道他约了什么人在那茶馆里见面不成?''那大汉道:``花公子在那茶馆里坐了很久,并没有人过去和他说话。''又过了半晌,一人回禀道:``花公子此刻在街头瞧王铁臂练把式。''江别鹤皱眉道:``那种骗人的把式,他也能看得下去?\ldots 你们可瞧见那边人从里,有什么人和他说话么?''那大汉道:``没有。''

江别鹤道:``现在谁在盯着他?''

那大汉道:``那条街是宋三和李阿牛在管的\ldots\ldots{}''话未说完,宋三已慌慌张张地奔了回来,伏地道:``花公子忽然不见了!''江别鹤赫然震怒,拍案道:``你难道是瞎子么?光天化日之下,行人往来不断的街道上,他绝不能施展轻功,又怎会突然不见?''宋三颤声道:``那王铁臂和徒弟练完单刀破花抢,就轮到他女儿耍流星锤,谁知她正使到一招云里捉月,流星锤的链子忽然断了,小西瓜般大小的流星锤,冲天飞了出去,瞧把式的人都怕它掉下来打着脑袋,惊呼着四下飞逃,那把式场立刻就乱了。''江别鹤道:``流星锤的链子,是怎么断的?''

宋三道:``小的不知道。''

江别鹤冷冷道:``你只怕是瞧王铁臂的女儿瞧晕了头吧?''宋三以首顿地道:``小\ldots\ldots 小的不敢。''

江别鹤厉声道:``你这双眼睛既然如此不中用,还留着它干什么?''话未说完,已有两条大汉将宋三拖了出去,宋三脸如死灰,却连求饶的话都不敢说出来。

过了半晌,后面便传人一声凄厉的惨呼!

江别鹤却似根本没听见,只是喃喃自语道:``花无缺哪里去了?他为何要躲着我?莫非他真的和江小鱼有约,要来对付我?这两人若是联成一路,我该如何是好?''他话声说得很轻,目光已露出杀机,冷笑道:``宁可我负天下人,莫令天下人负我\ldots。江别鹤呀江别鹤,这句话你千万忘记不得!''花无缺出了城,嘴角带着微笑,现在若有人问他:``那流星锤是怎会断的?''他一定会笑得很大声------能用一粒小石头打断那精铁铸成的链子,他对自己的手力也不禁觉得很满意。

花无缺到达花林时,锦绣般的紫花,已被昨日的剑气摧残得甚是萧索,阴霾掩去了日色,风中已有凉意。

花无缺想到自己又要和燕南天相对,嘴角的笑容竟瞧不见了,但他纵然明知此行必有凶险,也是非来不可。

花无缺踏着落花,走入花林,燕南天并末在林中,却有个白衣如雪的女子,垂头斜倚在花树旁,似乎在细数着地上的残花。

她背对着花无缺,花无缺只能瞧见她苗条的身子,和那乌黑的、长长披落在肩头的柔发。

花无缺虽然瞧不见她的脸,但一眼瞧过去,便已瞧出她是谁了------铁心兰,铁心兰怎么还在这里?

他想不到在这里见到铁心兰,他也不知道自已是不是应该招呼她,他的心里似乎有些发苦。

她心头似有许多心事,根本不知道有人来了,凉风轻抚着她的发丝,她的头发像缎子般光滑。

良久良久,才听得幽幽长叹了一声,喃喃道:``花开花落,顷刻化泥,人生又何尝不是如此?''花无缺本不想惊动她,也不忍惊动她,又想悄悄转身走出去,但此刻却也不禁发出一声轻轻的叹息。

铁心兰似惊似喜,猝然回首,道:``你\ldots\ldots{}''她只说了一个字.她瞧见来的竟是花无缺,便立刻楞住了。

花无缺心中纵有许多心事,面上却只是淡淡笑道:``你好么?''在这一瞬间,他实在想不出别的话来说。又有谁知道他在这一句淡淡的问候里,含蕴着多少情意。

铁心兰也似不知该说什么,只有轻轻点了点头。

过了半晌,花无缺又微笑答道:``你想不到来的是我,是么?''铁心兰垂下头,悠悠道,``瞧见你没有受伤,我实在很高兴。''她说话的声音几乎连自己都听不见,但花无缺每个字都听得清清理楚,他心里一阵刺痛。

他努力想使自已的笑容变自然些,但无疑是失败了,幸好铁心兰并没有瞧见他的笑容。

她仿佛根本不敢看他。又过了半晌,铁心兰才又叹息着道,``我本来有许多话想对你说,却不知该怎么说才好。''花无缺的微笑更苦涩,柔声道:``有些人是很难被忘记的,有时你纵然以为自已忘却了他,但只要一见着他,他的一言-笑,就都又重回到你心头\ldots.''铁心兰道:``你你能原谅我?''她霍然抬起头,目中已满是泪珠。

花无缺也不敢瞧她,垂首笑道:``你根本没有什么事要求人原谅的,我若是你,说不定也会如此。''铁心兰道:``但我实在对不起你,你\ldots。你为什么不骂我?不怪我?那样我心里反而会好受些,你的同情和了解,只有令我更痛苦。''她语声渐惭激动,终于哭出声来。

\hypertarget{ux7b2cux516dux5341ux4e03ux7ae0-ux4e49ux8584ux4e91ux5929}{%
\chapter{第六十七章
义薄云天}\label{ux7b2cux516dux5341ux4e03ux7ae0-ux4e49ux8584ux4e91ux5929}}

花无缺默然半晌,仰天叹道:``我永远也不会恨你,我虽然不能和你\ldots\ldots 和你在一起,但我终生都会将你当妹妹一样看待的。''他笑了笑,接着又道:``还有,我要告诉你,我也从来没有恨过江小鱼,他虽然和我命中注定要做仇敌,但他是我平生唯一真正的朋友,你\ldots\ldots 你能和他在一起,我也觉得很高兴\ldots\ldots{}''铁心兰忽然大呼道:``大\ldots\ldots 大哥,我这一辈子,永远感激你,真正的感激你。''她泪中带笑,实不知是悲是喜。花无缺也不如是悲是喜。他知道铁心兰这一声``大哥''唤出,便是终生无法更改的了,纵然已多多少少建立起一些情感,但这份情感,也被这一声``大哥''完全改变,这一声``大哥''唤得虽亲近,却又是多么疏远。

花无缺仰面向天,终于忍不住长长叹息,道:但愿他莫要对不起你\ldots\ldots 莫要对不起你!"这是一种愿望、一种祈求,也是一种铭誓,一种自我的舒放和宽解------这两句话中情感的复杂只怕也是别人难以了解的。

但无论如何,现在他们的心里总已比较坦然,大哥"这两个字就是一堵堤防,令他们觉得自己的情感已不致泛滥。

铁心兰终于嫣然而笑,道:``大哥,你怎会又到这里来的?''花无缺沉吟着道:``我受人之托,来找一个人。''铁心兰已追问道:``你莫非是要来找燕大侠的?''花无缺只好点头。铁心兰眼睛一亮,道:``莫非是他托你来的?''花无缺道:``是。''铁心兰道:他。\ldots·他自己为何不来"

花无缺不答反问,道:``燕大侠为何不在,你反在这里?''铁心兰垂下了头,道:``昨天晚上,燕大侠找到了我,对我说了许多话,又叫我今天在这里等他,你知道,燕大侠说的话,是没有人能拒绝的。''花无缺道:``他对你说了些什么?''

铁心兰的脸红了红,咬着嘴唇道:``燕大侠说,要我\ldots\ldots 我和他先聊聊,然后\ldots.''突听林外一人大笑道:``你们小两口已淡了么,我此刻来得是否太早?''花无缺霍然转身,只见燕南天长笑大步入林,瞧见了他,笑声骤顿,脸色一沉,厉声道:``你怎会在这里?你怎会来的?''他目光闪电般在铁心兰面上一扫,又道:``小鱼儿呢?''铁心兰不觉又垂下头,道:``我不知道,他说''花无缺接口道:``江小鱼托我来禀报燕大侠,他今日只怕不能来赴约了。''燕南天怒道:``他为何不能来''

花无缺长长吸了口气,道:``他已被人拘禁,只怕已是寸步难行''他知道自己这番话如果说出来,后果必然不堪设想,他话未说完,铁心兰果然已惨然变色。

燕南天暴怒道:``是谁拘禁了他?''

花无缺迟疑着,终于道:``一位武林前辈,人称铜先生的!''燕南天忽喝道:``铜先生?燕某闯荡江湖数十年,还未听说江湖中有铜先生此人,这名字莫非是你造出来的!''他一步窜到花无缺面前,又喝道:``莫非是你暗算了他,你居然还敢到这里来冒充好人!''花无缺昂然道:``在下受人之托,忠人之事,是以燕太侠你只要问我,我知无不言,但燕大侠您老对在下人格有所怀疑,在下\ldots.''花无缺一字字道:``在下纵不是燕大侠敌手,好歹也要和燕大侠再较一较高低''燕南天仰天狂笑道:``你还敢如此说话?你好大的胆子!''花无缺缓缓道:``在下胆子纵不大,却也不是贪生畏死的懦夫!''燕南天喝道;``你既不怕死,燕某今日就成全了你吧!''喝声未了,铁心兰也已冲过来,嘶声道:``燕大侠,我知道他.无论如何他绝不会是说谎的人!''燕南天厉声道,小鱼儿已落入别人手里你还在为他说话!难怪小鱼儿不愿理睬你,原来你也是个善变的女人!``铁心兰眼泪又已夺眶而出,颤声道:''江小鱼若有危险,晚辈就算拼了性命,也要救他的,但燕大挟说花\ldots\ldots 花公子说谎\ldots\ldots 晚辈死也不能相信。``燕南天冷笑道:''你要为小鱼儿拼命,又要为花无缺死,你究竟有几条命?``铁心兰流泪道:''燕大侠无论如何责骂,就算认为晚辈是个是个水性扬花的女人,晚辈也没法子``她扑倒在地,嘶声道:''晚辈只求燕大侠放过了花公子,日后燕大侠若是发现他是在说谎,就算将晚辈碎尸万段,晚辈也是甘心的。``燕南天厉声笑道:''好!你居然要以性命为他作保,只不过像你这样朝三暮四的女人,你的性命又能值得几文?"这一代名侠,本就性如烈火,此刻为小鱼儿担心情急之下,更是怒气勃生,不可遏止。

花无缺变色道:``燕南天,我敬你是一代英雄,总是对你容忍,想不到你竟对一个女孩子说出这样的话来,这样的英雄,嘿嘿,又值得几文?''燕南天已怒喝着一拳击出。花无缺也展动身形,迎了上去。

铁心兰知道这两人一动起手,天下只怕再难有人能化解得开,想到自己为小鱼儿和花无缺所受的屈侮与委屈,竟没有一个人能了解,想到自己的一番苦心,未了落得个``朝三暮四''的骂名外,竟毫无作用\ldots\ldots 她终于忍不住欲声大哭起来。悲恸的哭声,更惨于杜鹃啼血。

拳风,掌风,震得残花似雨一般飘落。

这几乎是江湖中新旧两代最强的高手决斗!这几乎已是百年来江湖中最惊心动魄的决斗!

上一次,他们用的是剑,这一次虽是空手,但战况的紧张与激烈,却绝不在上次之下!燕南天的拳势,就和他的剑法一样,纵横开阔,刚强威猛,招式之强霸,可说是天下无双!移花宫的武功,本是``以柔克刚''、``后发制人'',花无缺这温柔深沉的性格,本也和他从小练的就是这种武功有关。

但现在,他招式竟已完全变了!

他竟使出刚猛的招式,着着抢攻!只因若非这样的招式,已不足以将他心里的悲愤渲泄!这一战,已非完全为了他的性命而战!而是为了保护他这一生中最关心的人而战!

他虽然中是个温柔沉静的人,但铁心兰悲恸的哭声,却已激发了他血液中的勇悍之气!

他这勇悍的血液,是得自母亲的------他那可敬的母亲,为了爱,曾毫不畏惧地含笑面对死亡。

``移花宫''冷峻的教养,虽己使花无缺的血渐渐变冷了,但爱的火焰,却又沸腾了它!他忽然觉得生死之事,并不十分重要。

重要的是,他要和燕南天决一死战,他要以自己的血,洗清他最关心的人的冤枉,也洗清自己的冤枉。

激烈的掌风,似已震撼了天地。

花无缺双掌抢攻、直插、横截、斜击,招式刚强中不失灵活,但燕南天拳风却像是一道铁墙。花无缺竟连一招都攻不进去!

他头发已凌乱,凌乱的发丝,飘落在苍白的额角上,但他的面颊却因激动而充血发红。

任何人若也想以刚猛的招式来和燕南天对敌,那实在是活得不耐烦了。

他的掌式虽锐利得像钉子,但燕南天的拳势就像是铁锤,无情的铁锤,无情地敲打着他。

他只觉已渐渐窒息,渐惭透不过气来,燕南天飞舞的铁拳,在他眼中已像是越来越大,越来越大\ldots。他知道这次燕南天不会放过他!但他并不放弃,并未绝望,只要他还有最后一口气,至死,也绝不迟缩!

谁知燕南天竟忽然一个翻身,退出七尺,厉叱道:``住手!''他眼见已可将花无缺逼死掌下,却忽然住手。

花无缺不觉怔了怔,忍不住喘息着道:``你为何要我住手?''燕南天目光灼灼,逼视着他,一字字道;``我虽然从未听见过铜先生这名字,也并不相信世上真有铜先生这人存在,但我却已相信你并未说谎。''花无缺道:``哦?\ldots\ldots{}''燕南天道:``你若说谎,必定心虚,一个心虚的人,绝对使不出如此刚烈的招式!''花无缺默然半晌,仰天一笑,道:``你现在相信,不觉太迟了么?''燕南天沉声道:``你若觉得燕某方才对你有所侮辱,燕某在此谨致歉意。''花无缺长叹道:``是错就错绝不推诿,果然是天下之英雄,在下纵想与你一决生死,此刻也无法出手了!''燕南天厉声道:``但我却还是要出手的!''花无缺又一怔,道:``为什么?''

燕南天道:``你纵未说谎,我还是不能放你走,无论那铜先生是谁,他定与你有些关系,是么?''花无缺想了想,道:``是。''

燕南天道:``他拘禁了江小鱼,可是为了你?''花无缺苦笑道:``我并未要他如此,但他却实有此意。''燕南天喝道:``这就是了,他既然留下了江小鱼,我就要留下你!他什么时候放了江小鱼,我就什么时候放你!''他踏前一步,须发皆张,厉声说道:``他若杀了江小鱼,我就杀了你!''花无缺面色一变,却又长长叹了口气,道:``这说来倒也公平得很。''燕南天道:``燕某行事,素来公正。''

花无缺冷笑道:"但你对铁姑娘说的话,却太不公平,她说到这里,他才忽然发现,花树下已瞧不见铁心兰的人影,这已心碎了的少女,不知何时走了!

燕南天喝道:``你是自愿留下,还是要燕某再与你一战?''花无缺脸色铁青,一字字道:``你此刻要我走,我也不会走了。铁心兰若因此有三长两短,你纵放得过我,我也放不过你!''燕南天大笑道:``好,很好!在我找着铁心兰和江小鱼之前,看来你我两人,是谁也分不开谁了,是么?''花无缺道:``正是如此!''

铜先生抱着小鱼儿,又掠上树梢。

这株树枝时繁密,树的尖梢,方圆竟也有一丈多,树枝坚韧而有弹力,足可承受起百十斤的重量。

铜先生将小鱼儿放在上面,只不过将枝时压得下陷了一些而已------浓密的枝叶就好像棉褥般将小鱼儿包了起来,除非是翱翔在天空的飞鸟,否则绝不会发觉有人藏在这里。

小鱼儿身子虽不能动,脸上却仍是笑噶嘻的,道:``这倒真是再好也没有的藏身之处,如此看来,倒可以舒舒服服地睡上一觉了。''铜先生冷冷道:``你最好老老实实睡一觉。''

小鱼儿道:``你要走了么?你这人又孤僻,又特别喜欢干净,我就知道你不会永远守着我的。''铜先生冷笑道:``你也休想跑得了,等到我此间的事做完,就将你带到一个更安全之处。''小鱼儿道:``我连手指都不能动,你就是将我放在路上,我也跑不了的。''铜先生道:``你明白这点最好。''

小鱼儿眼珠子转了转,道:``若是下起雨来,我这人身体不太好,一淋就要生病,我生病倒没有什么,但若病坏了身子,岂非于你的名声有损?你答应过,绝不让我受到丝毫损伤的,是么?''铜先生冷冷道:``你无论生多大的病,我都能治得了你。''小鱼儿想了想,又道:``我身子比牛还重,这树枝若是承受不起,突然断了两根,我若摔了胳膊跌断了腿,你难道也能接起来么?''铜先生道:``这树枝纵然断了两根,你还是跌不下去的。''小鱼儿张大了眼睛,笑道:``若有什么老鹰之类的大鸟,从我头上飞过,把我的眼珠子当做鸽蛋,一口啄了去,你难道能补上么?''铜先生忽道:``你这人怎地这么烦!''

小鱼儿笑道:``我生来没别的本事,就会惹人烦,你若嫌烦,为何不宰了我,死人就不会惹麻烦了。''铜先生一生中,当真从来没有遇见这么讨厌的人,若是别人如此,他早已将之剁成八块了。

他身子已气得发抖,却只好取出块丝帕,盖在小鱼儿脸上,厉声道,``这样好了么7''小鱼儿深深吸了口气,笑道:``你这手帕好香呀,莫非是什么大姑娘送给你的定情物?''铜先生大怒道,``你为何不能闭起嘴来?''

小鱼儿道:``你若点了我的哑穴,我岂非就不能说话了么?但你自然也知道,哑穴不能点过三个时辰的,否则就会气绝而死。''他笑着接道::``所以你若点了我的哑穴,每隔三个时辰,就得回来为我换一次气,那样岂非更麻烦了。''铜先生咬牙道:``你知道的倒不少。''

小鱼儿道:``除此之外,倒有个比较不麻烦的法子。''他语声故意顿了顿,才接着道:``那就是三十六着,走为上策,你一走了,无论我说什么,你都听不见了,岂非落个耳棍清净。''钢先生不等他话说完,已掠下树梢。

小鱼儿故意叹了口气,喃喃道:``他总算走了,但愿那位仁兄莫要来得太早,先让我好好睡一觉。''他话未说完,铜先生又掠了上去,一把掀开了蒙着他脸的丝帕,厉声道:``你说的那位仁兄是谁?''小鱼儿故意失惊道:``呀,我说的话,被你听见了么?''铜先生冷冷道:``百丈之内,飞花落叶瞒不过我的。''小鱼儿又叹了口气,道:``我被你藏在这树上,任何人都瞧不见我,又怎会有人来救我呢?我方才不过自己说着玩玩而已。''铜先生道:``你以为谁会来救你?''

铜先生沉思了半晌,失声道:``不错,花无缺说不定会回来瞧瞧的。''他不再说话,又抱起小鱼儿,掠下树梢,他自以为心思灵敏,却未瞧见小鱼儿正在偷偷的笑。

小鱼儿根本就未指望有人会来救他,他知道若是耽在树上,就什么逃走的机会都没有了,只有拼命缠着铜先生,缠得他发昏,只要他稍为一大意,自己就有逃定的机会。

若论武功,小鱼儿自然不及铜先生,但若斗起心眼儿来,两个铜先生也不是小鱼儿的敌手。

他抱着小鱼儿掠到树下,却又迟疑起来。

小鱼儿道:``你要把我送到哪里去呀?你总不能一直抱着我站在这里吧。''``哼!''

小鱼儿笑道:``我已经有好几天没洗澡了,你抱着我不嫌脏么?''他话末说完,铜先生的手已一松。

小鱼儿``砰''的跌在地上,大叫道:``哎哟,不好了,骨头跌断了!''铜先生一脚踢在他胯骨上,踢开了他下半身的穴道,喝道:``站起来跟我走!''小鱼儿只觉两条腿已能动了,却呻吟着道:``我骨头都断了,哪里还能站得起来,这下子你非抱我不可了!''铜先生怒道:``你骨头是什么做的,怎地一跌就断?''小鱼儿道:``就算没有跌断,被你一脚也踢断了\ldots 哎哟,好痛!''他索性大呼大喊,叫起疼来。

铜先生目光闪动,忍不住道:``真的断了么?''

小鱼儿呻吟着道:``你不信就自己摸摸看。''

铜先生迟疑着,终于俯下身子,视探着小鱼儿的腿骨。

小鱼儿道:``不对,不是这里.''

铜先生道:``是哪里?''

小鱼儿道:不是大腿,还要再上面一些。

铜先生的手,突然缩了回去,就好像被毒蛇咬了一口似的,只见他笔直站在那里,胸膛却不住喘息。

小鱼儿笑嘻嘻道:``你为什么连摸都不敢摸,难道你是女人么?''铜先生大喝道:``住嘴!''

小鱼儿吐了吐舌头笑道:``你要我住嘴,就算不愿点我的哑穴,也可用布塞住我的嘴呀!''他的确可以塞住小鱼儿嘴的,但小鱼儿自己既然先说出来了,他再这样做,岂非丢人么?

铜先生冷冷道:"我为何要塞住你的嘴,我正要听你说话.。

小鱼儿``噗哧''一笑,道:``想不到我的话竟这么好听,你既然这么喜欢听,何不也坐下来,咱们可以聊个舒服。''铜先生怒目瞪着小鱼儿,简直无计可施,他本觉世上绝没有自己不能对付的人,谁知就偏偏有个江小鱼,他这一生中,第-次觉得头疼起来。

\hypertarget{ux7b2cux516dux5341ux516bux7ae0-ux6697ux85cfux5978ux8bc8}{%
\chapter{第六十八章
暗藏奸诈}\label{ux7b2cux516dux5341ux516bux7ae0-ux6697ux85cfux5978ux8bc8}}

燕南天与花无缺并肩走出了花林。

花无缺忽然道:``铁心兰是往哪里走的?你也未曾瞧见么?''燕南天道:``没有!''

花无缺仰首望天,轻叹道:``江小鱼此刻也不知是在哪里?''燕南天道:``他是何时落入那铜先生掌中的?''花无缺道:``昨天晚上.''燕南天默然半晌,忽然又道:``江湖中又怎会有个铜先生?他纵有那么高的武功,我怎会未曾闻及?\ldots·你可知道他的来历?''花无缺道:``在下只知他武功之高,不可思议,却也不知他的来历。''燕南天冷笑道:``若是我猜的不错,他必定是别人化名改扮的。''花无缺道:``但普天之下谁会有那么高的武功?燕南天道:''移花宫主``花无缺淡淡笑了笑,道:''家师为何要改扮成别人?家师又为何要瞒住我?这对她老人家又有何好处?燕大侠你可想得出什么原因来么?"``我想不出\ldots\ldots{}''他语声微顿,又道:``你想,那铜先生会将江小鱼带到何处去?''花无缺也长长叹了口气,道:``在下也想不出.''这时小鱼儿已睡着了,铜先生乘着夜色,将小鱼儿又带到那客栈的屋子里,他实在想不出能将这作怪的少年带到何处。小鱼儿躺在床上,呼呼大睡,铜先生却只有坐在椅子上瞧着,他就像个木头人做的坐在椅子上,动也不动,只见小鱼儿鼻息沉沉,似睡得安稳已极,就像是个睡在母亲旁边的孩子似的,嘴角还带着一丝微笑。

他醒着时,这张脸上,不但充满了一种逼人的魅力,也充满了飞扬洒脱、精灵古怪的神气。但此刻他睡着了,这张脸却变得有如婴儿般纯真.铜先生瞧着他这张纯真而英俊的脸,瞧着他脸上那条永远不能消除的刀疤,整个人突然都颤抖了起来。

他手掌紧握着椅背,握得那么紧,冷漠的目光,也变得比火还热,像是充满了痛苦,又像充满了仇恨。

只听``啪''的一声,柚木的椅靠,竟被他生生捏碎!

小鱼儿缓缓张开眼来,揉着眼睛向他一笑,道:``我睡了很久了么?''``很\ldots\ldots 很久了。''他拼命要使自己语声平静,却还是不免有些颤抖。

小鱼儿笑道:你一直坐在这里守着我?``小鱼儿身子虽不能动,腿一挺,就跳下床来,笑道:''我占了你的床,让你不能睡觉,真抱歉得很。``铜先生盯着他的腿,厉声道,''你\ldots\ldots 你的腿没有伤?"小鱼儿朝他扮了个鬼脸,就要往外走。

铜先生喝道;``你要到哪里去?''

小鱼儿笑嘻嘻道:``我有个毛病,一睡醒就要\ldots·就要上茅房。''铜先生怒道:``不许去!''

小鱼儿苦着脸道;``不许去,我就要拉在裤子上了,那可臭得很。''铜先生几乎要跳了起来,大喝道:``你\ldots\ldots 你敢?''小鱼儿悠悠道:``一个人无论有多凶、多厉害,他就算能杀人、放火,但可也没法子叫别人不拉屎的。''铜先生瞪着他,目中简直要冒出火来。

小鱼儿却还满不在乎,笑道:``你要我不拉屎,只有一个法子,那就是立刻杀了我,否则\ldots·否则我现在就已忍不住了。''他一面说话,一面就要蹲下去。

铜先生赶紧大呼道:``不行\ldots\ldots 这里不行\ldots。.''小鱼儿道:``你让我出去了么?''

铜先生狠狠一跺脚,道:``你滚出去吧!''

小鱼儿不等他说完,已弯着腰走出去,笑道:``你若不放心,就在茅房外看着我吧。''铜先生的确不放心,的确只得在茅房外等着。

他简直连做梦都未想到过,自己这一辈子,居然也会站在茅房外,等着别人在里面拉屎。

过了几乎快有半个时辰,小鱼儿才摸着肚子,施施然走了出来,铜先生简直快气疯了,怒道:``你死在里面了么?小鱼儿笑道:''好几天的存货,一次出清,自然要费些工夫。"铜先生气得也不知该说什么,只好扭过头去。

小鱼儿却笑道;``现在咱们该去吃饭了。''

铜先生大怒道:``你\ldots\ldots 你说什么?''

小鱼儿笑道:``吃饭拉屎,本是最普通的事,这又有什么好奇怪的?\ldots\ldots 你难道从未听见过一个人要吃饭么?''铜先生怔了半晌,突然冷笑道:``我虽不能禁止你\ldots。·你上茅房,但却能禁止你吃饭的。''小鱼儿道:``你不许我吃饭?''

铜先生厉声道:``我给你吃的时候,你才能吃,否则你就闭起嘴!''小鱼儿眨了眨眼睛,笑道:但嘴却是长在我脸上的,是么?所以,我要吃饭的时候,你就得给我吃,否则我就永远不吃了,我若活活饿死,你的计划也完了\ldots\ldots 你明白了么?``铜先生一步窜过去,揪住小鱼儿的衣襟,嘶声道:''你你敢对我如此说话?``小鱼儿嘻嘻笑道:''我虽打不过你,但要饿死自己,你可也没法子,是么?"铜先生气得全身发抖,却只好装作没有听见。

燕南天和花无缺自然没有找到铁心兰,更找不着小鱼儿,他们茫无目的地兜了两个圈子,燕南天突然道:``你喝酒么?''花无缺微笑道;``还可喝两杯。''

燕南天道:好,咱们就去喝两杯!"

两人便又入城,燕南天道:``江浙菜甜,北方菜淡,还是四川菜,又咸又辣又麻,那才合男子汉大丈夫的口味,你意下如何?''花无缺道:``这城里有家扬子江酒楼,据说倒是名厨。''这时夜市仍未收,街上人群熙来攘往,倒也热闹得很,扬子江酒楼上,更是高朋满座,座无虚席。

江别鹤正一个人喝着闷酒。

这两天令他烦心的事实在太多,小鱼儿、花无缺\ldots\ldots 还有他儿子江玉郎,竟直到此刻还未回来。

突见一个大汉匆匆奔上楼,撞倒两张椅子,才走到他面前,悄声道:``花公子来了,就在下面,好像也要上楼来喝酒。''江别鹤道:``他一个人么?''

那大汉道:``他还带着个穿得又破又烂的瘦长汉子,好像是他话未说完,江别鹤面色已惨变,霍然长身而起,颤声道:''快\ldots。·你想法子去挡他们一挡。"但这时花无缺与燕南天已走上楼,花无缺已面带微笑,向他走了过来。

江别鹤手扶着桌子,似已吓得站不住了。

只听花无缺笑道:``不想江兄也在这里。''江别鹤道:``是\ldots\ldots 是\ldots\ldots{}''他眼睛直勾勾地瞪着燕南天,只觉喉咙发干,双腿发软,一个字也说不出,竟似已吓破了胆。

燕南天上下瞧了他两眼,笑道:``这位就是近来江湖盛传的江南大侠江别鹤么?''江别鹤道:``不不敢。''燕南天道:``好,咱们就坐在一起,喝两杯吧。''他拉过张椅子,就坐了下来,只觉桌上杯子、盘子一直不停地动,原来江别鹤全身都在发抖。燕南天皱眉道:``江兄为何不坐下?''江别鹤立刻直挺挺地坐到椅上。

燕南天笑道:``燕某足迹虽未踏人江湖,却也久闻江兄侠名,今日少不得要痛痛快快和你喝上两杯。''江别鹤赶紧倒了三杯,强笑道:``晚辈先敬燕大侠一杯。''他用酒杯挡住脸,心里却不禁更是惊奇!原来江小鱼还未将我的事告诉他,但他\ldots\ldots 他又怎会不认得我了?这二十年来,我容貌未改变许多呀.他眼角偷偷自酒杯边缘瞧出去,又自暗付道:``但他的容貌却改变了许多,莫非\ldots 莫非是''突听燕南天道:``江兄这杯酒,为何还不喝下去?''江别鹤赶紧一饮而尽,哈哈笑道:``晚辈也早已久仰燕大侠侠名,不想今日得见,当真荣幸之至。''燕南天大笑道:``不错,你我初次相见,倒真该痛饮一场才是。''听到``初次相见''四个字,江别鹤心里虽然更奇怪,却不禁长长松了口气,大笑道:``正是该痛饮一场,不醉不归。''燕南天拍案笑道:``好个不醉不归\ldots 来,快拿三十斤酒来!''铜先生和小鱼儿走出客栈,夜已很深,长街上已无人迹,两旁店铺也都上起了门板。

小鱼儿背负双手,逛来逛去,好像开心得很,笑道:``你别着急,饭铺就算打烊,只要你肯花银子,连鬼都会推磨,何愁饭铺不为你开门。''铜先生忍住怒火,道:``这里就有家饭铺,你叫门吧。''小鱼儿道:``这家饭铺叫三和楼,是江浙菜,不行嗯,这里还有家真北平,一定是北方菜,也不行.''铜先生怒道:``为何不行?你难道不能将就些?''小鱼儿正色道:``不行,一个人可以对不起朋友,但却万万不能对不起自己的肠胃,因为朋友在你倒霉时,都会跑的,但肠胃却跟你一辈子。''铜先生狠狠盯着他,过了半响,才缓缓道:``世上人人都怕我,你\ldots 你为何不怕?''小鱼儿笑道:``我明知你绝不会自己动手杀我的,我为何要怕你。''铜先生霍然扭转身,大步而行。

小鱼儿大笑道:``其实你也不必生气,你明知你越生气,我就越开心,又何必定要和自己过不去呢?''只见前面一处楼上,还有灯光,招牌上几个斗大的金字,也在闪闪发着光。

``扬子江酒楼,正宗川菜。''

但这时扬子江酒搂上却已没有人了,几个伙计,正在打扫收拾。

几个人一抬头,全都吓得呆住------一个戴着铜鬼脸的人,不知何时已走上楼来,正冷冷地瞧着他们。

小鱼儿却笑嘻嘻道:``你们发什么呆,这位大爷脸上戴的虽然是青铜,腰里却多的是金子,财神爷上门,你们还不赶紧招呼?''那店伙吃吃道:``抱\ldots\ldots 抱歉得狠,小店已经打烊了。''铜先生冷冷瞧着他,忽然一把揪住他的头发。

那店伙身子就好像腾云驾雾似的,直飞了出去,等他定过神来,才发觉自己竟已坐到横梁上。身子虽未受伤,胆子却几乎吓破,头一晕,直栽了下来。若不是小鱼儿接着,脑袋不变成烂西瓜才怪。

铜先生冷冷道:``不管你们打烊没有,他要吃什么,你们就送什么上来只要少了一样,你们这四个人休想有一个活着!''四个店伙哪里还敢说个``不''字。

小鱼儿大笑道:``愉快愉快,和你这样的人出来吃饭,当真再愉快不过。''他舒舒服服地坐了下来,道:``先来四个凉菜,棒棒鸡,凉拌四件,麻辣蹄筋,蒜泥白肉,再来个肥肥的樟茶鸭子,红烧牛尾,豆瓣鱼------''他说一样菜,店伙们就点了一下头,四个店伙的头都点酸了,小鱼儿才总算叹了口气,笑道:``深更半夜的,也不必弄太多菜了,马马虎虎就这几样吧,但酒却要上好的,竹叶青还是花雕都行,先来个二三十斤。''几个店伙听得张口结舌,这些菜二十个人都够吃了,这小子居然才``马马虎虎'',几个人怔了半晌,才吃吃道:``抱歉\ldots。小\ldots\ldots 小店的酒,已经被方才三位客官喝光了。''铜先生冷冷道:``喝光了就到别处去买,三十斤,少了一斤,要你的脑袋!''四个店伙只有自叹倒霉,刚送走了三个瘟神,又来了两个恶煞。

不到半个时辰,酒菜都送了上来,果然一样也不少,小鱼儿立刻开始大吃大喝,铜先生却连坐都不肯坐下来。

小鱼儿笑嘻嘻道:``你为何不坐下来,你这样站着,我怎么吃得下?''他举起酒杯,又笑道:``这酒菜倒都不错,你为何不来吃一些,你若气得吃不下,饿坏了身子,我心里也不舒服的.''铜先生根本不理他。

小鱼儿夹起块樟茶鸭,一面大嚼,一面叹着气,道:``嘴是长在你身上的,你不吃,我也没法子,但你这样,既不吃,又不睡,怎么受得了呢?''铜先生忽然出手一掌,将旁边一张桌子拍得片片碎裂,他心中怒气实是无可宣泄,只有拿桌子出气。

小鱼儿笑道:桌子又没有得罪你,你何苦跟它过不去\ldots\ldots 依我看,你不如放了我吧,也免得自己受这活罪。``铜先生怒喝道:''放了你,休想!"

小鱼儿仰起脖子,喝了杯酒,哈哈笑道:``老实告诉你,其实你现在就算放了我,我也不走的,睡觉有人保镖,喝酒有人付帐,这么开心的日子,到哪里找去?''铜先生瞪眼瞧了他半晌,一字字道:``我正是要你现在活得开心些,这样你死时才会更痛苦。''小鱼儿放下筷子,瞪眼瞧着他,忽又叹道:``我问你,我和你素不相识,你为何如此恨我?你既如此恨我,又为什么不肯自己动手杀了我?''铜先生仰首望天,冷笑道:``这其中秘密,你永远也不会知道的!''小鱼儿叹道:``一个人若是永远无法知道自己最切身的秘密,这岂非是世上最残忍、最悲惨的事。''铜先生厉声笑道:``不错,这正是世上最残忍、最悲惨的事,我敢负责担保,这悲惨的命运,你逃也逃不了的,只因世上绝对没有人能揭穿这秘密,所以你现在只管开心吧,只要你真能开心,你不妨尽量多开些心。''燕南天、花无缺、汀别鹤,三个人都像是有些醉了,三个人摇摇晃晃,在灿烂的星光下兜着圈子。

江别鹤一生中从未喝过这么多的洒,但燕南天要喝,他却只有陪着,虽然到后来燕南天每干一杯时,他杯子里的酒最多也不过只有半杯。

只听燕南天引吭高歌道:``五花马,千金袭,呼儿将出换美酒,与尔共消万古愁\ldots\ldots 万古愁\ldots\ldots{}''歌声豪迈而悲怆,似是心中满怀积郁。

燕南天仰天长叹道:``怎地这世上最好的人和最坏的人,都姓江呢?''江别鹤吃吃道:``此\ldots。此话怎讲?''

燕南天叹道:``我那江二弟,温厚善良,可算世上第一个好人,但还有江琴\ldots\ldots{}''说到``江琴''两字,江别鹤忽然机伶伶打了个寒战,燕南天更是须发皆张,目眦尽裂,厉声接道:``我那江二弟虽将江琴视如兄弟手足一般,但这狠心狗肺的杀才,竟在暗中串通别人,将他出卖了!''江别鹤满头冷汗涔涔而落,口中却强笑道:``那江\ldots\ldots 江琴竟如此可恶?''燕南天双拳紧握,嘶声道:``只可惜这杀才竟不知躲到哪里去了,我竟找不着他\ldots\ldots 我若找着他时,不将他骨头一根根捏碎才怪。''江别鹤又打了个寒噤,酒也似被吓醒了一半,只觉燕南天捏着他双手越来越紧,竟似要将他骨头捏碎。

江别鹤忍不住强笑道:``晚\ldots\ldots 晚辈并非江\ldots·江琴,燕大侠莫要将晚辈的手也捏碎。''燕南天一笑松了手,只见前面夜色沉沉,几个夜行人狸猫般的掠入一栋屋子里,也不知要干什么勾当。

花无缺酒意上涌,似也变得意气风发,笑道:``三更半夜,这几人必定不干好事,我瞧瞧去。''燕南天忽道:``有我在此,还用得着你去瞧么?''他纵身一掠,跃上墙头,厉声道:``冀人燕南天在此,上线开扒的朋友,全出来吧!''喝声方了,黑暗中已狼窜鼠奔,掠出几个人来。

藏南天喝道:``站住,一个也不许跑!''

几个夜行人竟似全被``燕南天''这名字骇得呆了,一个个站在那里,果然连动都不敢动。

燕南天厉声道:``有燕某在这城里,你们居然还想为非作歹,难道不要命了!''他独立墙头,衣抉飞舞,望之当真如天神下降一般。

那几个人瞧见他如此神威,才确信果然是天下无敌的燕南天来了,几个人骇得一起拜倒在地,颤声道:``小人们不知燕大侠又重出江湖,望燕大侠恕罪。''燕南天喝道:``但江大侠在这城里,你们难道也不知道。''几个人瞧了江别鹤一眼,嘴里虽不说话,但那意思却明显得很,无论江别鹤多么努力,但江别鹤这``大侠''比起燕南天来,还是差得多。

燕南天喝道:``念在你们坏事还未做出,每个人打自己二十个耳括子,快滚吧!''那几人竟真的扬起手来,``噼噼啪啪''打了自己二十个耳光,又磕了个头,才飞也似的狼狈而逃。

江别鹤瞧得又是吃惊,又是羡慕,又是妒忌,忍不住长叹道:``一个人能有这样的声名,才算不虚此生了。''花无缺却微笑道:``普天之下,有这样声名的人,只怕也不过燕大侠一个。''燕南天轩眉道:``花无缺,你还不服我?''

花无缺微笑道:``他们若知道移花宫有人在此,只怕跑得更快的。''燕南天瞪了他半晌,忽然大笑道:``要你这样的人佩服,当真不是容易事。''他跃下墙头,又复高歌而行。江别鹤悄悄拉了拉花无缺衣袖,悄声道:``贤弟,燕大侠似已有些醉了,你我不如和燕南天别过,赶紧走吧。''花无缺微笑道:``我只怕要和江兄别过了。''

江别鹤怔了怔,道:``贤弟你\ldots\ldots 你难道要和燕大侠同行么?''花无缺道:``正是。''江别鹤掌心沁出冷汗,道:``令师若是知道,只怕有些不便吧.''花无缺微笑道:``家师纵然知道,我也是要和他一起走的。''江别鹤怔了半晌,道:``你\ldots\ldots 你们要去哪里?''花无缺道:``去找江小鱼。''

江别鹤身子又是一震,暗暗忖道:``燕南天现在就算还未认出我,就算还将我看成朋友,但再见到江小鱼后,我还是要完了。''三个人兜了两个圈子,也到了``铜先生''歇脚的客栈,江别鹤眼珠子一转,忽然笑道:``这客栈燕大侠可要再进去喝两杯么?燕南天大笑道:''你果然善体人意走,咱们进去``到了屋里,燕南天吩咐''拿酒来",江别鹤却找了个借口出去,偷偷溜到铜先生那屋子。

他自然是想找铜先生对付燕南天,只可惜铜先生偏偏不在屋子里。虽还留着那淡淡的香气,但他却说不定早巳离开此地。

江别鹤满心失望,回房时,燕南天又已几斤酒下肚了。他酒量虽好,此刻却也不免有些醉意。花无缺也是醉态可掬,江别鹤心念一转,溜出去将肚子里的酒全都用手挖得吐出来,再回去频频劝饮。

到后来燕南天终于倒在床上,呼呼大睡。花无缺喃喃道:``酒逢知己,不醉不归,来,再喝一杯\ldots\ldots{}''话未说完,也伏在桌上睡着了。

\hypertarget{ux7b2cux516dux5341ux4e5dux7ae0-ux5343ux94a7ux4e00ux53d1}{%
\chapter{第六十九章
千钧一发}\label{ux7b2cux516dux5341ux4e5dux7ae0-ux5343ux94a7ux4e00ux53d1}}

江别鹤静静坐了半晌,瞪大了眼睛,瞧着燕南天。花无缺伏在桌上,也是动也不动。

江别鹤只听得自己的心跳声,越来越响------他若想从此称霸江湖,现在的确是机会到了。

但这机会,却又未免来得太容易!他紧握着双拳,掌心也满是冷汗。``江别鹤呀江别鹤,你若错过了这机会,就再也不会有这样的机会了,你今天若不杀他们,迟早总要死在他们手中,你怕什么?犹豫什么?他两人都已醉了,你为何还不动手7''想到这里,江别鹤霍然站起,却又``噗''地坐了下去!``不行!不能心存侥幸,世上绝不会有如此容易的事!''他手掌抖得太厉害,不得不紧紧抓住椅子!

但这种事连我自己都不相信,他们自然更不会相信了,他们就因为不相信,所以才没有丝毫提防之心。"江别鹤眼睛里发出了光!

``不错,花无缺和燕南天万万想不到我会杀死他们的,这实在是千载难逢的机会\ldots\ldots 江别鹤呀江别鹤,此刻怎会拿不定主意\ldots\ldots?你现在只要一出手,天下就是你的!..\ldots.''江别鹤不再迟疑,一步窜到桌前,铁掌直击下去!

就在这时,花无缺突然跳了起来,大喝道:``江别鹤,我总算瞧清了你的真面目,江小鱼果然没有冤枉你!''喝声中,他纵身扑了过去。

谁知燕南天竟比他还快了一步。

江别鹤手掌击下,燕南天铁掌已迎了上去!

只听``啪''的一声,江别鹤身子已被震飞,重重撞到墙上,只觉满身骨节欲裂,一时间竟站不起来。

花无缺怔了一怔,失笑道:``原来你是假醉!''燕南天大笑道;这区区几杯酒,怎能醉得倒我?我也正是要瞧瞧这厮,喝了又吐,吐了再喝,究竟是何用意?``他倏然顿住笑声,大喝道:''江别鹤,你现在还有何话说?``江别鹤惨笑道:''罢了\ldots\ldots 我苦练二十年的武功,竟接不了燕南天的一掌,我还有何话说?``燕南天厉声道:''我与你无冤无仇,你为何暗算我?``江别鹤故意长长叹了口气,道:''双雄难以并立,你我不能并存,你这大侠若活在世上,哪里还有我这大侠立足之地!``他咬了咬牙,大声接道:''方才我见到那些人瞧见你后,便不将我放在眼里,我已下定了决心,要除去你!如今我武功既然不敌,夫复何言?``燕南天怒道:''你武功就算能无敌于天下,就凭你这心胸,也难当大侠二字。``江别鹤道:''你\ldots·你要怎样?"

燕南天厉声道:``你虚有大侠之名,心肠竟如此恶毒,手段竟如此卑鄙,燕某今日若不为江湖除害,日后还不知有多少人要死在你手上!''江别鹤道:``你要杀了我?''燕南天道:``正是''

喝声中,他一掌闪电般击出。

江别鹤就地一接,避开了他这一掌,突然大笑道:``你若杀了我,普天之下再无一人知道江琴的下落\ldots。这一辈子就休想再能找得到他了!''燕南天一震,失声道:``你你知道江琴的下落?''江别鹤缓缓站了起来,悠然道:``正是。''

燕南天冲了过去,一把揪着他衣襟,嘶声道;``他在哪里?''江别鹤站在那里,也不闪避,悠悠道:``你可以杀死我,却不能令我说出他的下落。''燕南天手掌一架,怒喝道;``你可要试试?''江别鹤微笑道:``你身为一代大侠,若也想以酷刑逼供,岂非有失你大侠的身份?''燕南天怔了怔,手掌不由自主缓了下来。

江别鹤微笑又道:``你若真的想要我说出来,除非答应我两件事。''燕南天怒道:``你还要怎样?''

江别鹤缓缓道:``我要你答应,非但今日好生送我出去,日后也永不伤我毫发!''燕南天默然半晌,狂吼道:``好,我答应你\ldots。我不信除了燕某之外,世上就再无别人能伤你!''江别鹤微微一笑,道:``还有,我说出江琴的下落后,你必定要严守秘密,绝不能让第四人知道江琴在哪里。''燕南天大声道:``这本是我自己的事,我正要亲手杀死他,为何要让别人知道。''江别鹤嘴角泛起一丝诡秘的笑容,道:``很好,但你若不能杀死他呢?''燕南天忽道:``我若不能亲手杀死他,别人更不能杀他!''江别鹤转过头道:``花公子你呢?''

花无缺长长吐了口气,道:``这本是燕大侠的事,他既已答应,我自无异议。''江别鹤仰天大笑道:``很好,好极了。''

燕南天道:``江琴究竟在哪里?''

江别鹤缓缓顿住笑容,瞧着燕南天,一字字道:``就在这里!''燕南天身子一震,道:``你你\ldots\ldots{}''

江别鹤大笑道:``我就是江琴,但你却已答应,永不伤我毫发!''燕南天就像是被人抽了一鞭子,踉跄后退,双拳紧捏,全身都颤抖了起来,花无缺也不禁为之怔住。

江别鹤狂笑道:``你一心想知道江琴的下落,所以才答应放了我,如今虽已知道江琴的下落,却永远不能杀他了。''他笑声声嘶力竭,仿佛觉得世上再也没有比这更好笑的事,燕南天目光尽赤,突然狂吼扑上去,道:``你\ldots\ldots 你这恶贼,我岂能容你!''江别鹤瞪起眼睛,厉声道:``堂堂的大侠燕南天,难道是食言背信的人!''燕南天身子一震,整个人都呆在那里。

只见他须发怒张,眼角似已崩裂,全身骨节都不住响动,终于踉跄后退几步,跌坐在床上,惨然道:将\ldots\ldots 好我答应了你,你走吧。``燕南天突又跳-
了起来,嘶声道:''你若再不走,小心我改变了主意!``江别鹤抱拳一揖,笑道:''既是如此,在下就告辞了,多谢多谢,再见再见。"他大笑着扬长而去,屋子里立刻变得一片死寂,只有燕南天沉重的呼吸声,屋顶也沉重得像是要压了下来。

也不知过了多久,花无缺忽然长叹一声,道:``燕大侠,我此刻终于服了你了。''燕南天惨然一笑,道:``我以拳剑胜你两次,你不服我,我一声叱咤,但令群贼丧胆,你也不服我,如今我眼睁睁瞧着仇人扬长而去,竟无可奈何,你反而服了我么?''花无缺正色道:``我正是见你让江别鹤走了,才知道燕南天果然不傀为一代之大侠,你要杀他,本是易事,世上能杀江别鹤的人并不少,但能这样放了他的,却只怕唯有燕南天一人而已!''他长叹接道:``所以,世上纵有人名声比你更令人畏惧,纵有人武功比你更高,仍却也唯有你,才能当得起这大侠二字!''燕南天惨笑道:``但你可知道,一个人若要保全这大侠两字,他使要忍受多少痛苦,多少寂寞''花无缺长笑道:``我如今终于也知道,一个人要做到大侠两字,的确是不容易的。他不但要做到别人所不能做的事,还要忍别人所不能忍------''他游目瞧着燕南天,展颜一笑,道:``但无论如何,那也是值得的,是么?''江别鹤走出了院子,立刻就笑不出了,他知道今天虽然骗过了燕南天,但以后的麻烦,正还多着哩。

风吹着竹叶,沙沙的响,江别鹤闪身躲入了竹林,他是想瞧瞧燕南天和花无缺的动静。

他想,这两人现在必定不知有多么懊恼愤怒,他恨不得能瞧见燕南天活活气死,他才开心。

但过了半晌,屋子里却传出燕南天豪迈的笑声,这一次挫败虽大,但燕南天却似并未放在心上。

笑声中,只见燕南天和花无缺把臂而出,腾身而起,身形一闪,便消失在浓重的夜色里。

他们要到哪里去?是去找江小鱼么?这三个人本该是冤家对头,现在怎地已像站到同一条战线上来了。

江别鹤虽然猜不透其中的真相,但``怀疑''却使得他的心更不定、更痛苦,他咬着嘴唇,沉思了半晌,还拿不定主意。

突见人影飘动,一个狰狞的青铜面具,在闪着光。

铜先生居然又回来了。

江别鹤大喜,正想赶过去,但就在这时,也看清了铜先生身旁的人,竟然是小鱼儿!

江小鱼脸喝得红红的,满脸笑容,像是开心得很------铜先生竟然和江小鱼走到一起了,而且两人还像是刚喝完了酒回来!

他现在一心想倚靠这神秘的铜先生来对付燕南天和花无缺,这几乎已是他唯一可以致胜的希望。

他再也想不到,铜先生会和江小鱼在一起,这一老一少两个怪物,是什么时候交上了朋友?

铜先生本来明明要杀江小鱼的,现在为何改变了主意?

莫非他已被江小鱼的花言巧语打动了?

江别鹤又惊、又怒、又是担心恐惧,直到铜先生和小鱼儿走进屋子,他还是呆呆地怔在那里。

他忽然发觉自己竟己变得完全孤立,到处都是他的敌人,竟没有一个可以信赖的朋友。

他疑心病本来就大,现在既已亲眼目睹,更认定燕南天、江小鱼、花无缺、铜先生,四人已结成一党,要来对付他。这时夜已更深,竹时上的露水,一滴滴落下来,滴在他身上、脸上,甚至滴入了他的脖子里。

他却浑然不觉,只是不住暗中自语:``我要击败这四人,该怎么办呢?我一个人的力量,自然不够,还得去找帮手,但我却又能找得到谁?''竹叶上忽然有条小虫,掉了下来,却恰巧掉在他头上,江别鹤反手捉了下去,只见那小虫在他掌心蠕蠕而动,就像是条小蛇。

他面上忽然露出喜色,失声道:``对了!我怎地未想起他来!他一个人力量纵还不够,但再加上那老虎夫妻和我,四个对四个,岂非正是旗鼓相当!''他大喜着掠出树林,突然想起铜先生和江小鱼还在对面的屋子里,他大惊止步,掌心已沁出冷汗。

但对面屋子里却丝毫没有反应,屋里虽燃着灯,窗上却瞧不见人影,铜先生和小鱼儿,竟已走了。

小鱼儿走出屋子时,也末想到江别鹤就在外面瞧着他。

屋子里灯已熄了,小鱼儿虽然什么都瞧不见,却发觉屋子里的香气,比他们出去时更浓了。

这屋子里难道已有人走进来过?

小鱼儿正觉奇怪,突听铜先生冷冷道:``你怎地现在才来?''黑暗中竟响起了个女子的声音,道:``要找个能令你满意的地方,并不容易,所以我才来迟了。''这声音自然比铜先生粗戛生硬的语声娇柔多了,但语气也是冰冰冷冷,竟似和铜先生一副腔调。

小鱼儿又惊又奇,暗道:``想不到铜先生这怪物也会有女朋友,而且说话竟也是和他一样阴阳怪样,两人倒真是天生一对。''他摸着了火折子,赶紧燃起灯。

灯光亮起,小鱼儿才瞧见一个长发披肩的黑袍女子,她面上也戴着个死眉死脸的面具,却是以沉香木雕成的,此刻灯光虽已甚是明亮,小鱼儿骤然见着这么样一个人,仍不禁骇了一跳。

这黑袍女子也在瞧着小鱼儿,忽然道:``你就是江小鱼?''小鱼儿瞪大眼睛,道;``你\ldots\ldots 但我怎么不认得你?''黑袍女子道:``你既知世上有铜先生,为何不知本夫人?''小鱼儿道:``木夫人?不错,我好像听到过这名字。''他记得黑蜘蛛向他说起铜先生时,也曾提起过木夫人这名字,还说这两人是齐名的怪物。

木夫人瞧瞧小鱼儿,又瞧瞧铜先生,道,我早已来到此地,但你两人\ldots\ldots"``我和铜先生喝酒去了,有劳夫人久候,抱歉得很。''小鱼儿笑嘻嘻道:``铜先生对我真好,怕我饿坏了肚子,就带我去喝酒,知道我喜欢吃咸吃辣,就带我去吃川菜------这么好的人,我当真还未见过。''木夫人眼睛里既是惊奇,又似乎觉得有些好笑。

小鱼儿这才发现,她语声虽和铜先生同样冷漠,但这双眼睛,却比铜先生灵活得多,也温暖得多。

他眼珠子一转,立刻叹了口气,又接着道:``只不过铜先生实在对我太关心了,一心只想看我,自己连饭也不吃,觉也睡不着,我真怕累坏了他,所以夫人若是铜先生的好朋友,不如代铜先生照顾我吧,也好让他休息休息。''木夫人道:``大\ldots\ldots 大哥若是烦了,就将他交给我也好。''她目中笑意虽更明显,但语声仍是冰冰冷冷。

只见铜先生身子突然飘起,``啪''的-掌,掴在小鱼儿脸上,这一掌打得并不重,但打的地方却妙极。

小鱼儿一点也不觉得疼,只觉得头脑一阵晕眩,身子再也站不住,踉跄后退几步终于倒了下去。

晕迷中,只听铜先生冷冷道:``这一次,谁也休想从我身边带走他了,他活着时,我固然要看着他,就算他死了,我也要看着他,直到他尸身腐烂为止。''木夫人道:``但我\ldots\ldots{}''铜先生冷笑道:``你也是一样,你对我也不见得比别人忠心多少。''木夫人道:``你。\ldots·你连我都不相信?''

铜先生一字字道:``自从月奴将江枫带走的那天开始,我就已不再信任何人了!''木夫人默然半晌,缓缓垂下了头,道:我知道你还在记着那一次,你总以为我要和你争夺江枫\ldots{}``.''铜先生厉声道:``你也爱他,这话是你自己说的,是么?''木夫人始起了头,大声道:``不错,我也爱他,但我并没有要得到他,更没有要和你抢他,我这一生从来没有和你争夺过任何东西,是么?''她冷漠的语声竟突然颤抖起来,嘶声道;``从小的时候开始,只要有好的东西,我永远都是让给你的,从你为了和我争着去采那树上唯一熟了的桃子,而把我从树上推下来,让我跌断了腿的那天开始,我就不敢再和你抢任何东西,你还记得吗?''铜先生目光刀一般瞪着她,良久良久,终于长长叹息了一声,也缓缓垂下了头,黯然道:``忘了这些事吧,无论如何,我们都没有得到他是么?''木大人默然良久,也长叹了一声,黯然道:``大姐,对不起,我本不该说这些话的,其实我早已忘记那些事了。''只可惜小鱼儿早巳晕过去了,根本没有听见她们在说什么。

小鱼儿还未醒来,就已感觉出那醉人的香气。

他以为自己还是在那客栈的屋子里,但他张开眼后,立刻就发觉自己错了,世上绝没有任何一家客栈,有如此华丽的屋子,也绝没有任何一家客栈,有如此芬芳的被褥,如此柔软的床。

接着,他又瞧见站在床头的两个少女。

她们都穿着柔软的纱衣,戴着鲜艳的花冠。

她们的脸,却比鲜花更美,只是这美丽的脸上,也没有丝毫表情,也没有丝毫血色,看来就像是以冰雪雕成的。

小鱼儿揉了揉眼睛,喃喃道:``我莫非已死了,这莫非是在天上?''轻纱少女动也不动地站在那里,目光茫然瞧着前方,非但好像没有听见他的话,简直就好像根本没有瞧见他。

小鱼儿眼珠子一转,嘻嘻笑道:``我自然没有死,只因我若死了,就绝不会在天上,而地狱里也绝不会有你们这么美丽的仙子。''他以为她们会笑的,谁知她们竟还是没有望他一眼。

小鱼儿揉了揉鼻子,道:``你们难道瞧不见我么?\ldots.我难道忽然学会了隐身法?''轻纱少女简直连眼珠子都没有动一动。

小鱼儿叹了口气,道:``我本想瞧瞧你们笑的,我想你们笑的时候一定更美,但现在,我却只有承认失败了,你们去把那见鬼的钢先生找来吧。''轻纱少女居然还是不理他。

小鱼儿跳了起来,大声道:``说话呀!为什么不说话?你们难道是聋子、瞎子、哑巴?''他跳下地来,赤着脚站在她们面前瞧了半晌,又围着她们打了两个转,皱起了眉头,喃喃道:``这两个难道不是人?难道真是用冰雪雕成的?''他竞伸出手,要去拧那轻纱少女的鼻子。

这少女忽然轻轻一挥手,她纤长的手指,柔若春葱,但五根涂着风仙花汁的红指中,却像是五柄小刀,直刺小鱼儿的咽喉。

小鱼儿一个筋斗倒在床上,大笑道:``原来你们虽不舍说话,至少还是会动的。''那少女却又像石像般动也不动了。

小鱼儿道:``你们就算不愿跟我说话,也总该笑一笑吧.老是这么样紧绷着脸,人特别容易变老的。''他又跳下床,找着双柔软的丝履,套在脚上,忽然缓缓道:``从前有个人,做事素来马虎,有一天出去时,穿了两只鞋子,都是左脚的,他只觉走路不方便,一点也不知道是鞋子穿错了,等他到了朋友家里,那朋友告诉他,他才发觉,就赶紧叫仆人回家去换,那仆人去了好半天,回来时却还是空着一双手,你猜为什么?''说到这里,小鱼儿已忍不住要笑,忍笑接着说,那人也奇怪,就问他仆人为什么不将鞋子换来,那仆人却道,不用换了,家里那双鞋子,两只都是右脚的。"他还未说完,已笑得弯下腰去。

但那两个少女却连眼皮都未抬一抬。

小鱼儿自己也觉笑得没意思了,才叹了口气,道:``好,我承认没法子逗你们笑,但我有个朋友叫张三的,却最会逗人笑了,有一天,他和另外两个人去逛大街,瞧见-
位姑娘站在树下,就和你们一样,冷冰冰的,张三说他能逗这姑娘笑,那两个朋友自然不信,张三就说:我用一个字就能把她逗笑,再说一个字义能令她生气,你们要不要和我打赌,赌-桌酒。那两个朋友自然立刻就和他赌了。''小鱼儿口才本好,此刻更是说得眉飞色舞,有声有色,那两个少女眼睛虽还是不去瞧他,但已忍不住想听听这``张三''怎能用一个字就能将人逗得发笑,再用一个字逗得别人生气。

只听小鱼儿接着道:``于是张三就走到那姑娘面前,忽然向那姑娘旁边一条狗跪了下去,道;爹。那少女见他竟将一条狗认作爹爹,再也忍不住笑了起来,谁知张三又向她跪了下去,叫了声妈。那少女立刻气得满脸飞红,咬着牙,张三果然就赢了这东西。''他还未说完,左面一个脸圆圆的少女,已忍不住``噗哧''一声,笑出声来,小鱼儿拍掌大笑道:"笑了!笑了!你还是笑了只见这少女笑容初露,面色又已惨变。

铜先生不如何时又走了进来,冷冷地瞧着她,冷冷道:``你觉得他很好笑?''那少女全身发抖,``噗''地跪了下去,颤声道:``婢\ldots\ldots 婢子并没有找他说话''

\hypertarget{ux7b2cux4e03ux5341ux7ae0-ux6b7bux91ccux6c42ux751f}{%
\chapter{第七十章
死里求生}\label{ux7b2cux4e03ux5341ux7ae0-ux6b7bux91ccux6c42ux751f}}

铜先生厉声道:``但你却为他笑了,是么?''

那少女竟吓得话也说不出,忽然掩面痛哭起来。

铜先生缓缓道:``你出去吧。''

那少女嘶声道:``求求你\ldots\ldots 求求你饶婢子一命,婢子下次再也不敢了。''小鱼儿吃惊道:``饶她一命?\ldots 你\ldots\ldots 你难道要杀了她?''铜先生冷冷道:``杀,倒也不必,只不过割下她的舌头,要她以后永远也笑不出。''小鱼儿大骇道:``她只不过笑了笑,你就要割下她的舌头!''铜先生冷冷道:``这只能怪你,你本不该逗她笑的。''小鱼儿大叫道:``我只不过说了个笑话给她听,你\ldots\ldots 你何必吃醋!''铜先生忽然又是一掌掴了出去,小鱼儿竟躲闪不开,被他-
掌打得仰面跌倒,口中却还是怒喝道:``你打我没关系.但千万不能因为这件事罚她。''铜先生目中又射出了怒火,道:``你\ldots 你竟然为她说话?''他竟似已怒极,连身子都气得发抖。小鱼儿大声道:``这件事本不能怪她,要怪也只能怪我。''铜先生颤声道:``好好!你宁可要我打你,也不愿我罚她,你\ldots\ldots 你倒也和你那爹爹一样,是个多情种子!''说到``种子''二字,他忽然狂吼一声,反手一掌击出,那圆脸少女被打得直飞出门外,一滩泥似的跌在地上,再也动弹不得!

小鱼儿跳了起来,大喝道:``你\ldots 你竟杀了她!''铜先生全身发抖,忽然仰首狂笑道:``不错,我杀了她,她再也不能偷偷和你逃走。''小鱼儿又惊又怒,道:``你疯了么?她几时要和我偷偷逃走?''铜先生道:``等你们逃走时,我再杀她,便已迟了!''小鱼儿瞪大眼睛,嘶声道:``你疯了,你简直疯了\ldots\ldots 我本以为你脾气虽然冷酷,却并不是个狠毒残忍的人,谁知你竟能对一个女子下此毒手。''他越说越怒,忽然扑过去,双拿飞击而出。

这时小鱼儿武功之高,已足可与当代任何一个武林名家并列而无愧,盛怒之下击出的两掌更融合了武当、昆仑两大门派掌法之精萃,小鱼儿此刻不但已可运用自如,而且已可将其中所有威力发挥。

谁知这足以威震武林的两掌,到了铜先生面前,竟如儿戏一般,铜先生身子轻轻一折,整个人像是突然断成两截。

他手掌便也在此时反击而出,若非亲眼瞧见,谁也不会相信一个人竟能在这种部位下出手的。

小鱼儿只觉身子一震,整个人又被打得跌在地上,他虽未受伤,但却被这种奇妙的武功吓呆了。

铜先生俯首望着他,冷笑道:``像你这样的武功,最多也不过能接得住花无缺五十招而已,我本以为你还可与他一拼,谁知你竟如此令我失望。''小鱼儿咬牙道,``我能接得他多少招,关你屁事。''铜先生竟不再动怒,反而自怀中取出一卷黄绢,缓缓道:``这里有三招可以破解移花宫武功的招式,你若能在这三个月里将它练成,纵不能胜了花无缺,至少也可多挡他几招。''他居然要传授小鱼儿武功,这真比天上掉元宝下来还要令人难以置信,小鱼儿张口结舌,道:"你\ldots\ldots 你是什么意思?铜先生将绢卷抛在他面前,冷笑着走了出去。

小鱼儿大喝道;``你究竟是要花无缺杀我,还是要我杀花无缺?你究竟有什么毛病?''铜先生霍然转身,冷冷道;``你这一生,已注定了要有悲惨的结局,无论你杀了花无缺,还是花无缺杀了你,都是一样的。''铜先生已头也不回地走了出去,``砰''的关上了门,小鱼儿怔了半晌,抬起头,却发现犹自呆立在房中的少女,眼里已流下泪来,但这一次小鱼儿却再也不敢找她说话了,他实在再也不忍瞧见一个活生生的美丽少女,为他而死。

那少女呆呆地站着,任凭眼泪流下面颊,也不伸手去擦,小鱼儿叹了口气,将那绢卷展开。

那上面果然是三招妙绝天下的招式,每一招都锋利、简单而有效,正是花无缺那种繁复招式的克星。

绢卷上不但画着清晰的图解,还有详细的文字说明,若不是对``移花宫''武功了如指掌的人绝对无法创出这样的招式。

``移花宫''的武功,本是江湖中最大的秘密,铜先生又怎会对它如此了解,这岂非是件奇怪的事。

但小鱼儿却没有想到这点,他此刻简直什么都不愿想,只是瞧着那卷书,呆呆地出神。

少时有人送来饭莱,居然是樟茶鸭、豆瓣鱼、棒棒鸡每一样都是通道地地的川味,还有一大壶上好的陈年花雕。

小鱼儿一笑,尽管饱餐了一顿,却留下一碟红烧牛尾,半只樟茶鸭子不动,像是自言自语,喃喃道:``这两样菜不辣的,你吃不吃都随便你。''那少女始终站在那里,连指尖都未动过,此刻竟忽然转过身,用手撕着那半只鸭子就薄饼,吃了个干净。

她若不吃,本在小鱼儿意中,她此刻居然大吃起来,小鱼儿倒不免大感奇怪,竟瞧得呆了。

只见那少女吃完一只鸭腿时,便已似吃不下了,但还是拼命勉强自己将半只鸭子吃光。

她嘴里咀嚼,眼睛却眨也不眨地盯着那桌子上的一具计时秒漏,一粒粒金黄色的细沙落下来,时间便也随着流了过去。

小鱼儿不禁苦笑,时间,现在对他实在太宝贵了,但他却只有眼见时间在他面前流过,全没有一点法子。

突见那少女走了过来,走到他面前,悄声道:``你还吃得下么?''她竟忽然开口说话了,小鱼儿不觉吓了一跳。

那少女又道:``现在说话没关系,没有人会来的。''小鱼儿这才笑了笑道:``我肚子都快撑破了,连一只蚂蚁都吞不下了。''那少女道:``你最好还是多吃些,这两天,我们只怕都没有东西吃了。''小鱼儿又吃了一惊,道:``为什么?''

那少女眼睛里射出了逼人的光芒,一字字道:``只因我们现在就要开始逃,在逃亡的途中,绝不会有东西吃的,甚至连水都喝不到。''小鱼儿简直吓呆了,吃吃道:``逃?\ldots\ldots 你是说逃走?''那少女道:``不错,我方才拼命的吃,就为的是要有力气逃走!''小鱼儿道:``但铜先生\ldots\ldots{}''

那少女道:``现在正是他入定的时候,至少在两个时辰之内,不会到这里来。''小鱼儿道:``你能确定?''

那少女道:``他这习惯数十年来从未改过,据说十多年前,也有个身份和我一样的女子,就是在这时候,带了一个人逃走的。''小鱼儿恍然道:``难怪他方才那般愤怒,原来他就是怕历史重演\ldots。''那少女目中又泛起了泪光,道:``你可知道方才被他杀死的女孩子是谁?''小鱼儿动容道:``那莫非是你的\ldots 你的''

那少女目中终于又流下泪来,颤声道:``她就是我嫡亲的妹妹。''小鱼儿怔了半晌,惨然道:``对不起,我方才中不该逗她笑的。''那少女恨恨道:``我妹子跟了他七年,他为了那么小的事,也能下得了毒手,而你与我妹子素不相识,反而为她争辩,甚至不惜为她拼命\ldots。.''小鱼儿道:``你就是为了这原因,所以才冒险救我的?''他忽然拉起她冰冷的手,沉声道:``但经过十多年前的那次事后,他防守得必定十分严密,我们能逃得出去么?''那少女道:``若是在他的禁宫中,我们实在连一分逃走的机会都没有,但这里,却只不过是他临时歇脚的地方.''这时她脸上初次露出一丝苦涩的微笑,拉着他道:``何况,这地方不但是我找到的,而且是我布置的,我们虽不是一定能逃得出去,但好歹也得试一试,那总比在这里等死的好。''小鱼儿四下瞧了一眼.忍不住道:``这里究竟是什么地方?''那少女道:``这是个庙。''

``这里竟是个庙?''他眼睛瞧着四下华贵而绮丽的陈设,鼻子里嗅着醉人的香气,实在难以相信,这里竟会是个庙宇。

那少女道:``这里本是个冷冷清清的古刹,经过我们一整天的布置后,才变成这样子的。''小鱼儿叹道:``你们的本事可真不小。''

他忽然一笑,又道:``但时间宝贵得很,我们为何还不走,你若是想聊天,等我们逃出去之后,时间还多着哩。''那少女道:``我们要等人来收去这些碗筷后才能走,否则立刻就会被人发现,我们已不在这个屋子里。''小鱼儿笑道:``不错,我小地方总是疏忽,好像每个女孩子都比我细心得多。''那少女凝注着他,缓缓道:``你认得的女孩子很多么?''小鱼儿苦笑道:``我真希望能少认得几个你呢?你认得的男孩子''那少女冷玲道:``我一个都不认得。''

小鱼儿笑道:``你现在总算已认得我了,我姓江,叫江小鱼,你呢?''那少女默然半晌,缓缓道:``你不妨叫我铁萍姑。''小鱼儿像是怔了怔,苦笑道:``你也姓铁?为什么姓铁的女孩子这么多\ldots.''话未说完,铁萍姑挥手打断了他的话。

只听门外轻轻一响,小鱼儿赶紧倒在床上,已有个面色冷峻的紫衣少女,带着个青衣妇人走了进来。

铁萍姑站在那里,根本不去瞧她。

这紫衣少女却走到她面前,冷冷道:``你妹妹已死了。''铁萍姑也冷冷道:我知道。"

紫衣少女道:``你伤心么?''

铁萍姑道:``我若伤心,你开心么?''

紫衣少女霍然扭转身,一双冷酷而充满怒火的眼睛,恰好对着小鱼儿,小鱼儿却向她扮了个鬼脸。

这时那青衣妇人已将碗筷全都收了出去。

紫农少女忽然道:``你也可以出去了。''

小鱼儿怔了怔,强笑道:``你说我可以出去了?''紫衣少女又转身盯着铁萍姑.冷笑道:``你自然知道我说的是你,你为何还不走?''小鱼儿一惊,心跳都几乎停止。

铁萍姑却冷冷道:``谁叫我走的?''

紫衣少女冷笑道:``你现在已可以换班了,我叫你去休息休息还不好。''铁萍姑不再说话,转身走了出去。

小鱼儿眼睁睁瞧着她往外走,心里虽着急,却一点法子也没有,只见紫衣少女眼睛已又盯在他身上,一字字道:``你不愿意她走?''小鱼儿打了个哈欠,笑道:``她走了最好,她那副晚娘面孔我已瞧腻了,你虽然也未必比她好看多少,但换了个新的总比旧的好,我天生是喜新厌旧的脾气。''紫衣少女冷笑道:``你眼睛若敢盯着我,我就挖出你眼珠子。''小鱼儿见到铁萍姑已悄悄退了回来,故意大笑道:你嘴里虽说不愿我瞧你,心里却是愿意的,说不定你还希望我能抱一抱你,亲一亲你,否则你为何定要将她调走,自己留在这里?``紫衣少女气得脸上颜色都变了,颤声道:''你\ldots\ldots 你敢对我如此说话?``小鱼儿吐了吐舌头,笑道:''你可不是雌老虎,我为何不敢,我还想咬你一口哩。"他瞧见铁萍姑已到了这紫衣少女身后,更故意要将她气得疯。

紫衣少女大喝道:"你莫以为我不能杀你,我至少可打断你话未说完,她头忽然垂了下来,接着,整个人就噗地倒了下去,连哼都没有哼出一声。

铁萍姑一掌已切在她脖子上。

小鱼儿跳了起来,道:``你不怕别人发现\ldots\ldots{}''铁萍姑冷冷截口道:``时机难得,我只好冒一冒险了,何况,在这里的人,都不会关心别人的事,她就算三天不露面,也不会有人找她的。''她一面说话,一面已将那张床移开了半尺,伸手在墙上摸索了半晌,墙壁立刻出现了一道窄门。

铁萍姑一推而入,沉声道:``快跟着我来。''

入壁后,居然还有一条地道,曲折深邃,也不知通向哪里,一阵阵阴森潮湿之气令人作呕。

小鱼儿又惊又喜,捏着鼻子走了段路,才忍不住叹道:``想不到庙里居然也会有复壁地道,你是什么时候发现的?''铁萍姑道,``我收拾这间屋子时,已发现了。''她接着又道;``据我猜想,这古刹乃是五胡作乱时所建,那时流寇盗贼横行,人命更贱于猪狗,很多人都削发出家,借以避祸,但庙宇中也非安全之地,所以寺僧才建了这些复壁地道,以躲避散兵流寇的杀掠。''小鱼儿叹道:``你的确和我所认识的其他女孩子有些不同。你有头脑\ldots\ldots 这世上有头脑的女孩子,已越来越少了,而且有些人就算有头脑,却偏偏懒得去用它,她们总认为只要有张漂亮的脸就够了。''铁萍姑像是又笑了笑,道:``但这却只能怪男人。''小鱼儿道:``哦?''

铁萍姑道;``只因男人都不喜欢有头脑的女孩子,他们都生怕女孩子比自己强,所以越是聪明的女孩子,就越是要装得愚笨软弱,男人既然天生就觉得自己比女人强,喜欢保护女人,女人为何不让他们多伤些脑筋,多吃些苦。''小鱼儿大笑道:``如此说来,愚笨的倒是男人了,''\ldots 但你连一个男人也不认得,又怎会对男人了解得这么清楚?铁萍姑道:``女人天生就能了解男人的,但男人却永远不会了解女人的。''小鱼儿叹了口气,道:``这话倒的确不错,一个男人若自以为能了解女人,他受苦的日子就不远了。''这时两人心中其实都充满了恐惧和不安,所以就拼命找话说,只因说话通常都能令人紧张的神经松弛、镇定下来。

在这黑暗阴森的地道中,自己都不知道自己生命能否保全的时候,两人若再保持沉默,那岂非更令人难以忍受?

地道已越来越潮湿,越来越黑暗.

小鱼儿伸手去摸了摸,两旁已不再是光滑的墙,而是坚硬、粗糙、长满了厚绒青苔的石壁。

他也感觉到,地上亦是坎坷不平,忍不住问道:这庙宇的复壁难道是连着山腹的么?"铁萍姑并未回答,却亮起了精巧的火拆子。

这里果然已在山腹中,纵横交错的洞隙,密如蛛网,风,也不知从哪里吹进来的,吹得人寒毛直竖。小鱼儿笑道:``在这种地方,铜先生就算有通天的本事,想找到咱们也不容易。''铁萍姑道:``但我们要想走出去,只怕也不容易。''小鱼儿吓了一跳,失声道:``你\ldots\ldots 你难道也不知道出去的路?''铁萍姑道:``我当然不知道。''

小鱼儿骇然道,``那么你\ldots。·你为什么说咱们可以逃得出去?''铁萍姑道:``只要有路,我们自然就有逃出去的希望。''小鱼儿苦着脸道:``姑娘你未免将事情瞧得太简单了,你可知道,山腹中的这些洞隙,有的根本是没有路通出去的。''铁萍姑道:``也还有的是可以通得出去的,是么?小鱼儿道:''纵然有路,但这些洞穴简直比诸葛亮的八阵图还要复杂诡秘,有时你在里面兜上三个月的圈子,到最后才发现自己又回到原来的地方``他长叹道:据我所知,古往今来,被困死在这种山腹里的冤死鬼,若是聚在一起,阎王老子的森罗殿只怕也要被挤破了。''铁萍姑在前面走着,却连头也不回,冷冷道,``既是如此,再加两个也不多.''小鱼儿道;``你------你难道不着急?''

铁萍姑冷冷道:"你若着急,现在回去,还来得及\ldots\ldots{}

小鱼儿怔了征,苦笑道,``你别生气,我并没有怪你,只不过\ldots{}''.``铁萍姑霍然回过头,大声道:''你以为我不知道这里的危险?但无论如何,我们总有一半的机会能逃出去,这总比坐在那里等死好得多,是么?``小鱼儿吐了吐舌头,笑道:''早知道你这么生气.那些话我就不说了。``铁萍姑狠狠盯了他半晌,忽然叹道:''我真想不到你竟是个如此奇怪的人。``小鱼儿笑道:''我也真未想到,你的脾气竟这么大。"他嘴里在不停地说着话,眼睛也没有闲着。

这时,他忽然发觉石壁上浓厚的青苔里,隐约仍可瞧见刻着个箭头,铁萍姑目光闪动,显然也瞧见了。

她立刻沿着这箭头所指的方向,走了过去,走了十余丈转角处的石壁上果然又有个箭头。

但小鱼儿却还是站在那里,动也不动。

铁萍姑皱眉道:``现在我们既然已可走出去了,你为何站着不动?''小鱼儿笑嘻嘻道:``你若沿着这箭头走,再走片刻,就可见到铜先生了,但我可不愿再见到他那副尊容。''铁萍姑一惊,道:``这些箭头难道不是指路的?''小鱼儿道:``箭头虽然是指路的,但指的却绝不是出去的路。''铁萍姑道:``你怎知道?''

小鱼儿道:``这些箭头,必定是以前庙里的和尚刻上去的,是么?''铁萍姑道:``不错\ldots\ldots{}''小鱼儿道:``他们也为的是怕迷失路途,被困死在这里,所以才刻这些箭头的,是么?''铁萍姑道:``不错。''小鱼儿道:``他们为了躲避流寇,所以才躲到这里,等他们知道流寇走了之后,你想他们要到什么地方去呢?''铁萍姑道:``自然是回到庙里去.''

她脱口说出了这句话,才恍然大梧,失声道:``不错,这些箭头指的一定是回庙去的路,他们只不过是想在这山腹里躲避一时,又怎会去标明出路。''小鱼儿拍手笑道:``我早已说过,你是个很有头脑的女孩子,你终于明白了,我看你方才想不通,只怕也是故意装出来的。''铁萍姑忍不住垂下头,一张脸已红到耳根了。她忽然将火折子交到小鱼儿手上,道:``你\ldots\ldots 你带路吧。''小鱼儿叹了口气,喃喃道:``所以越是聪明的女孩子,就越是要装得愚笨软弱,所以你现在就要我多伤些脑筋,多出些力''。``他话未说完,铁萍姑已红着脸,跺着脚道:''这件事就算是你对了,也没什么了不起。``小鱼儿笑嘻嘻瞧着她,瞧了许久,慢吞吞笑道:''我就是要你脸红、生气,你生起气来,才真正像是个女孩子,我实在受不了你那冷冰冰的样子。``铁萍姑想要板起脸,小鱼儿却已大笑着转身走了,于是她刚板起来的脸,又忍不住嫣然一笑喃喃道,我的脸真红了么?我实在连自己都不知道自己脸红时是什么样子,这只怕还是我生平第一次''小鱼儿沿着箭头而行,每隔十多丈,到了转角处,他就发现另外一个箭头在那里。

只不过箭头指的是前,他就往后,箭头指的是左,他就往右,每走过一个箭头,他就将那箭头设法毁了去,铁萍姑随他走了半晌,忍不住道:``你这样走,能走得出去么?''小鱼儿笑道,``我虽不知能否走得出去,但这样走,至少距离那庙宇越来越远了。''但这时洞隙已越来越窄,小鱼儿有时竟已走不过去,到了这时,指路曲箭头也没有了。

小鱼儿叹了口气,道:``现在,咱们看来只有碰运气了,索性闭着眼睛往前走吧。''他一面说话,一面已熄去了火折子。

铣萍姑不再说话,只觉自己的手已被小鱼儿拉住。

她的心突然跳了起来,在黑暗中,这心跳得似乎特别响,铁萍姑的脸不禁又红了,简直恨不得找个地缝钻下去。

只听小鱼儿悠悠笑道:``一个人的心若是要跳,谁也没法子叫它停住。''铁萍姑``嘤咛''一声,要去拧他的臂,但手却又忽然顿住,痴痴地发起怔来,她忽然发觉多年以来,这竟是自己第一次意会到自己也是有血有肉的。

狭隘地洞里,举步艰难,有时甚至要爬过去,在黑暗中走这样的路,可真不是件舒服的事。

铁萍姑衣服已被刮破了,也许身上已有些地方在流血,但她却丝毫不觉得痛苦,一个人竟像是走在云堆里。

每走一段路,小鱼儿就打亮火折子,瞧瞧四周的情况,但到了后来,火折子的光焰,已越来越弱。

小鱼儿知道火已将尽,更不敢随意动用了,他知道在这种地方,若是完全没有火光,那更是死路一条,于是路就走得更苦了。

铁萍姑的脚步,终于也沉重起来。接着,她就感觉到全身疼痛,头晕眼花,又饿又渴。

她自然不像小鱼儿那铁打的身子,怎能受得了这种苦,若不是小鱼儿始终在和她说说笑笑,她简直连一步都走不动了。其实小鱼儿自己又何尝走得动?若是换了别人,到了他这种绝境之中,纵不急得发疯,也难免要呼天怨地了。

但小鱼儿却是天生的怪脾气,要他死,也许还容易些,要他着急愁苦,要他笑不出,那却要困难得多。

铁萍姑终于忍不住道:``我们歇歇再走吧。''

小鱼儿沉声道;``绝不能歇下来,一歇,就再也休想走得动了。''铁萍姑道:``但我\ldots·我现在已\ldots.''

小鱼儿笑道:``你想,我们在这千古以来、极少有人来过的神秘洞灾里拉着手散步,这是多么美、多么风流浪漫的事,别人一辈子都不会有这种机会,我们为何不多享受享受。''铁萍姑幽幽道:``只可惜我\ldots\ldots 我不是你心上的人。''小鱼儿笑道:``谁说不是的,此时此刻,除了你之外,世上还有和我更亲近的人么?''铁萍姑又``嘤咛''一声,整个人忽然倒入小鱼儿怀里,她的脸烫得就像是一团火,这火,是从她心底发出来的!

\hypertarget{ux7b2cux4e03ux5341ux4e00ux7ae0-ux67f3ux6697ux82b1ux660e}{%
\chapter{第七十一章
柳暗花明}\label{ux7b2cux4e03ux5341ux4e00ux7ae0-ux67f3ux6697ux82b1ux660e}}

铁萍姑根本就没有接触过男人,她青春的火焰,本已抑制得太久了,更何况一个人到了生死边缘时,理智本就最容易崩溃。

铁萍姑实在也想不到自己会倒入小鱼儿怀里,但此刻已倒下去了,她也丝毫不觉后悔。

她只觉得小鱼儿的手,已轻轻搂住她肩头。

铁萍妨颤声道:``人生,人生真是多么奇妙,我现在才知道\ldots 我两三天前还不认得你,但现在\ldots 现在\ldots{}''小鱼儿忽然道:``你可知道,我现在想什么?我现在最想瞧瞧你的脸。''铁萍姑道:``不要\ldots\ldots 求求你不要''

但火折子却已亮着了,铁萍姑以手掩住脸,她的脸又羞红了。

她颤声道:``火折子快没有了''

小鱼儿笑道:``火折子虽然珍贵,但能瞧见你现在这模样,无论牺牲多么珍贵的东西,都是值得的。''铁萍姑的手缓缓垂下,道:``真的?''

小鱼儿笑道,``只可惜现在没有镜子,否则我也要让你知道,你现在的模样,要比以前那种冷冰冰的样子美丽多少。''铣萍姑眼波也凝注着小鱼儿,悠悠说道:``我们若真的走不出去你会怪我么?''小鱼儿道:``怪你,我怎会怪你?''铁萍姑道:``你在那里,本还不会死的,但现在\ldots\ldots{}''小鱼儿笑道:``若这么说,你本该怪我才是,若不是我,你又怎会受这样的苦。''铁萍姑嫣然笑道:``连我自己都已不将我当做女人,何况别人呢?别人也许会将我看成仙子甚至魔女,却绝不会将找看成女人的。''小鱼儿笑道:``但你却不折不扣是个女人,我可以用一千种法子来证明。''铁萍姑笑道;``我现在自己也知道了,所以我现在就算死,也是快乐的。''火折子,渐渐只剩下一点豆大的火焰。

铁萍姑凝注着这火焰,眼皮已越来越重,低语着道:``我也知道,你这样对我,并不是真的喜欢我,只不过是为了安慰我,让我得到最后的快乐。''小鱼儿笑道:``你\ldots\ldots 你想得太多了。''铁萍姑嘴角泛起一丝微笑,轻轻道:``但我还是感激你,我只是只是真的累了,求求你让我睡吧,这一睡纵然永不醒来,我也满足了\ldots.''小鱼儿瞧着铁萍姑眼帘渐渐阖起,也不禁叹了口气。

就在这时,突然``梭噜''一声,竟有一连串又肥又大的老鼠,首尾相接,从他们面前跑过去。

铁萍姑一惊,张开眼来,身子已吓得缩成一团。

小鱼儿却是满面喜色,大声道:``你不必睡,我们已得救了。,铁萍姑道:''但这只不过是些老鼠。"

小鱼儿道:``你瞧,这些老鼠又肥又大,绝对不是在山腹里的,这里连一颗米都没有,绝养不了这么肥的老鼠。''铁萍姑眼睛也亮了,道:``你说这些老鼠是从山外跑进来的?''小鱼儿道:``不错,这里必定已接近山腹的边缘,山路必定就在附近。''他一面说话,一面已向鼠群窜来的方向走过去。

幸好这时火折子还未完全熄灭,他不久就发现一个不大不小的洞,洞外还隐隐有淡淡的光线透入。

他立刻将铁萍姑拉了过去,从这小洞里钻了过去。

外面竟然是个宝窟,一箱箱金银珠宝堆在那里,虽然并不算太多,可也绝不算少了。

小鱼儿怔了征,笑道:``我又不是财迷,老天却偏偏总是要我发现一些神秘的宝藏,我真不懂世上的宝藏怎会有这么多。''铁萍姑手扶着一只箱子,忽然道:``这里并不是什么神秘的宝藏,这些箱子搬进来,还没有几天,上面连积灰都没有。''他抬起手来一瞧,手上果然没有沾着什么尘垢。

他忽然发现每只箱子的箱盖里,都贴着张红纸,纸上竟写着``段合肥藏''四个字。

这个发现几乎叫他眺了起来。

这些财宝,想必就是江别鹤父子设计抢去的东西,被江玉朗藏到这里来的,他想必认为这地方秘密已极,却不想竟偏偏被小鱼儿发现了。

小鱼儿又惊又喜,简直要放声欢呼起来。

铁萍姑的身子却突又靠了过来,悄声道:``外面有人!''只见一道影如门户的石隙处,竟隐隐有灯光传入,小鱼儿悄悄走了过去,果然发现外面一块巨石旁,有两个人相对而坐。

面对着这边的一人,面色惨白,赫然竟是江玉郎,坐在江玉郎对面的一人,身材甚是魁伟,却瞧不清面目。

那块大石头旁,摆着许多酒肉,但两个人却都没有吃喝,只是聚精会神地看着前面的这块大石头,两只眼睛睁得大大的,眨也不眨。

铁萍姑忍不住悄声道:``这石头有什么好看的,这两人为何看得如此出神?莫非是疯子不成?''小鱼儿咽了好几口水,叹道:``据我所知,这人非但不疯,而且头脑还比别人都清楚。''铁萍姑道:``你认得他?''

小鱼儿眼睛还是盯着那些酒肉,道:``嗯。''

铁萍姑道:``那么他们为什么死盯着这块石头呢?''小鱼儿笑道:``也许他们希望这石头上能长出花来。''他眼睛终于自酒肉上移开,移到这石头上。

只见这石头上方方正正,一点出奇的地方也没有,但石头中间,却划着条线,线的左右两边各放着一小块肥肉。

这两人的眼睛,就盯着块肥肉,动也不动。

小鱼儿也被他们弄棚涂了,忍不住笑道:``我以前是知道这小子没毛病的,但现在却说不定了,难道他竟忘了肉是用嘴吃的,不是用眼睛看的。''铁萍姑也忍不住咽了两口口水,悄声道:``你若认得他,不如去教教他吧。''小鱼儿苦笑道:``我又何尝不想去教他吃肉,只可惜我现在只要一走出去,他就要吃我的肉了,他早已恨不得吃我的肉了。''铁萍姑叹了口气,又忍不住道:``另外一个人呢?''小鱼儿道:``这人我还瞧不出是谁,好像是\ldots\ldots{}''话末说完,突见一只老鼠从黑暗中窜出来,窜上那块大石头,将那大汉面前的一块肥肉衔了去,又飞也似的逃走了。

江玉郎面色立刻变了变,苦笑道:``好,这一次又是你赢了。''那大汉大笑道:``现在,你已欠我一百三十万两,你那里面的东西,已快输光了吧!''江玉郎冷冷道:``你放心,还多着哩。''

那大汉狂笑道:``老子正赌得过瘾,你若这么快就输光,老子不捏出你蛋黄来才怪。''他大笑着,又割下一小块肥肉,放在石头上。

铁萍姑这才恍然大悟,忍不住笑道:``原来这两人是在赌钱,谁面前的肉被老鼠衔走,谁就赢了,这样的赌法,倒也是天下少有。''小鱼儿笑道:``但这样的赌法却公平得很,谁也休想作弊。''铁萍姑道:``若是老鼠不来,怎么办呢?''

小鱼儿道:``老鼠不来,反正就等着,这人的赌瘾最大,只要是赌,你叫他等几天八夜也没什么关系。''铁萍姑失笑道:``不错,此刻看来他们就已不止睹了几天几夜了。''小鱼儿道:``你可要知道背对着我们的这人是谁么?他就是恶赌鬼轩辕三光,不赌到人光、钱光,他是绝不肯站起来走的。''铁萍姑动容道:``恶赌鬼?莫非是十大恶人中的\ldots.''铁萍姑沉默了半晌,忽又问道,``你可知道这十大恶人究竟是些什么人?''小鱼儿笑道:``你这话可算真问对人了,世上比我更知道十大恶人的,还真不多。''他扳着手指,道:``十大恶人,就是血手杜杀,笑里藏刀哈哈儿,不男不女屠娇娇,半人半鬼阴九幽,不吃人头李大嘴。''说到这里,铁萍姑身子似乎微微一震,面色也变了,但小鱼儿却并没有瞧她,只是接着道:``还有狂狮铁战,迷死人不赔命萧咪咪,恶赌鬼轩辕三光,损人不利已白开心,再加上欧阳丁、欧阳当兄弟。''铁萍姑道:``照你这样说来,岂非有十一个人了。''小鱼儿笑道:``只因这欧阳兄弟向来秤不离砣,砣不离秤,两个人无论干什么,都是一起的,所以只能算做一个人。''铁萍姑缓缓垂下了头,道:``这些人是否真的都十分恶毒?''小鱼儿笑道:``其实世上比他们更恶毒的人,还不知有多少,只不过,这些人做事特别不正常,毛病特别大而已。''铁萍姑道:``这话是什么意思?''

小鱼儿道:``譬如说,这不吃人头李大嘴,平日看来,他不但很和气,而且还可说是个文武双全的才子,但他毛病一发作起来,却连自己的老婆都能吃下肚去,见过他面的人,谁也想不到他能做得出这种事来。''说到``李大嘴''这名字,铁萍姑竟又微微一震,怔了半晌,才轻轻问道:``你难道认得他们的?''小鱼儿笑道:``我非但认得他们,老实告诉你,我还是跟着他们长大的。''铁萍姑又怔了怔,道:``你\ldots\ldots 你可知道他们现在哪里?''小鱼儿道:``只怕是在龟山一带。''

他忽然顿住语声,笑道:``你为何问得这么清楚?''铁萍姑勉强笑了笑,道:``我只不过是好奇而已,谁想得到世上有这么奇怪的人?''他们说话的声音自然很小,江玉郎和轩辕三光此刻已赌得连自己生辰八字都忘了,自然更不会听到他们的话。

只见江玉郎忽然一笑,道:``你我赌了七八天,还是谁也没有输光,你不烦么?''轩辕三光道:``不烦,不烦,再赌上三年六个月,老子也不会烦的。''江玉郎道:``但这样赌下去,我却有些烦了。''轩辕三光立刻瞪起眼睛,大声道:``你烦,也要陪老子赌下去。''江玉郎笑道:``我并不是说不赌,只不过是想将赌注增大而已。''轩辕三光大笑道:``老子赌钱,向来是嫌小不嫌大,越大越过瘾,你要赌多大,说吧。''江玉郎缓缓道:``阁下身上带的东西,既然值七八十万两,此刻又赢了我一百三十万两,你我这一注,就赌两百万两吧。''轩辕三光抚掌笑道:``一注见输赢,这倒也痛快,只是\ldots\ldots{}''他忽然顿住笑声,大喝道:``老子早巳看过,你那洞里最多也不过只有两三百万,此刻已输了一半,你哪里还有这么多银子来跟老子赌?''江玉郎道:``洞中存银,至少还有一百万。''

轩辕三光道:``还差一百万呢?''

江玉郎道:``还差一百万,以人来作数。''

轩辕三光狂笑道:``格老子,就凭你这龟儿子,也值得了一百万?''江玉郎面色不变,微微笑道:``在下纵不值一百万,却有值一百万的人。''轩辕三光道:``在哪里?''

江玉郎笑道:``阁下难道还要先估估价么?''

轩辕三光瞪眼道:``当然要先估估价,上了赌桌六亲不认,就算是儿子跟老子赌钱,帐也要算清楚的,一文钱也差错不得。''江玉郎微笑道:"既是如此,在下这就去将她带来就是\ldots\ldots{}

轩辕三光身后,一块凸出来的岩石上,有盏铜灯,此刻江玉郎端起了这盏铜灯,大步走了出去,一面微笑道:``阁下但请放心,在下立刻就回来的。''轩辕三光笑道:``老子自然放心得很,你龟儿家当都在这里,又急着翻杠,不回来才怪''他这才撕下条鸡腿,就着酒大嚼起来。

已瞧得出神的铁萍姑,忽然叹了口气,道:``这些人赌起钱来,一赌就上百万两银子,他们的银子简直好像是偷来的。''小鱼儿笑道:``谁说这些银子不是偷来的?''

铁萍姑道:``纵然是偷来的,也要费些力气,一下子就输出去,岂不可惜。''小鱼儿道:``这就叫来得容易去得快,何况,一个好赌的人,连老婆儿子输出去,都不会心疼的。''铁萍姑也不禁笑道:``难道他也要把老婆拿来和别人赌么?''小鱼儿道:``他就算有老婆,也不值一百万,这小子到底在玩什么花样,就连我也猜不出了,能值一百万的人,到底不多呀。''这时江玉郎已拉着一人走了进来,被他拉着的人,身材苗条,竟是个女子,只是脸上覆着层面纱,瞧不出面目。

轩辕三光皱眉道:``你怎要带来个女人?''

江玉郎微笑道;``当然是女人,若是男人,就不值钱了。''轩辕三光大笑道:``但从你这龟儿子手上送了来的剩货,只怕连一文都不值。''江玉郎正色道:``这位姑娘虽然跟着我走了几天,但我却绝未动过她的毫发。''轩辕三光道:``你这馋猫会不偷嘴吃,老子不信。''江玉郎笑道:``阁下若不信,一试便知。''

他将铜灯又放到山石上,但这次并末放在轩辕三光身后,却放到他自己身后,灯光从他肩上照下来,正好照在轩辕三光面前。

一盏灯无论放在哪里,都是件小事,自然谁也不会在意,但小鱼儿却不禁皱起了眉头,喃喃道:``这小子又想搞什么鬼,他将这盏灯带进带出,绝不会没有用意的。''江玉郎满肚子坏水,自然谁也没有小鱼儿清楚。

只见那蒙黑纱的女子,始终木然地站在那里,江玉郎伸手掀开她的面纱,她还是痴痴地站着不动。

灯光下,她的脸果然美得不带丝毫烟火气,轩辕三光、铁萍姑瞧见这张脸,但觉眼前一亮。

小鱼儿瞧见这张脸,却险些惊呼出声来。

慕容九,这女子竟是慕容九,她被三姑娘赶走后,一路痴痴迷迷的到处乱闯,她梦游般笔直走出了城,别人虽然瞧着奇怪,但见她衣服华贵,人又美得邪气,也不致有人敢动她的歪主意。

谁知竟偏偏误打误闯,被江玉郎听见这消息。

他立刻想到这女子必是慕容九,所以就立刻放下别的事,赶回头,恰巧在路上迎着了已饿得发晕的慕容九。

江玉郎自然不怕她泄漏秘室,就带着她去起出赃银,藏到这里,又谁知螳螂捕蝉,黄雀在后,轩辕三光竟早巳在身后盯上他了!

这时轩辕三光瞧见慕容九的脸,也不禁怔了半晌,方自叹道:``美女,果然是美女,只可惜近二十年来,老子已对任何美女都不感兴趣了,你还是带着她走吧!''江玉郎微笑道:``这位姑娘虽美,但值钱的地方却不在她这张脸上,在她的身份。''轩辕三光大笑道:``她难道还是位公主不成?''江玉郎道:``虽不是公主,却也和公主差不多。''轩辕三光怒道:``她究竟是谁?你这龟儿子说话怎地总要兜圈子?''江玉郎缓缓道:``她便是九秀山庄的慕容九姑娘。''轩辕三光也不禁一怔,动容道:``慕容家的九姑娘,怎会落在你手里?''江玉郎道:``她被恶人所害,神智迷失,不知下落,慕容家的八位姐妹,八位姑爷,都寻她不着,在下运气好,却在无意中找到了她。''他一笑接道:``阁下请想想,若有人将她送回她姐姐、姐夫那里,秦剑、南宫柳等人又将如何感激,那谢礼还会少得了么?''轩辕三光想了想,一拍手道:``好,老子就跟你赌了!''突听一人大喝道:``赌不得!''

小鱼儿忽然这么一叫,不但轩辕三光和江玉郎大吃一惊,就连铁萍姑都不免吓了一跳。

小鱼儿也不着急,先附在铁萍姑耳畔,悄声道:``你跟我出去,喜欢吃什么,就拿起来吃,千万莫要讲客气,我现在已有对付这小子的法子。''他说完了话,才施施然走了出去,笑道:``躲在粪坑下吃大便的朋友,难道已忘了我么?''江玉郎瞧见小鱼儿,真比瞧见鬼还要吃惊,倒退两步,失声道:``你\ldots\ldots 你怎会在这里?''小鱼儿笑道:``老子阴魂不散,跟定了你这龟儿子。''他聪明绝顶,学什么像什么,学起轩辕三光的口音,更是惟妙惟肖,轩辕三光用力一拍他肩头,大笑道:``若是别人从里面钻出来,老子也要吃一惊,但你这鬼精灵,你就算从地上钻出来,老子也不会奇怪的。''轩辕三光笑弯了腰,小鱼儿却早已大吃大喝起来,慕容九痴痴地瞧着他,又似相识,又似不识。

江玉郎瞧见小鱼儿身后居然也跟着个绝世美女,那吃相居然也和小鱼儿一样,像饿死鬼投胎似的。

他瞧得眼睛都直了,简直不知该如何是好。

只听轩辕三光好不容易忍住了笑,喘着气道:``小兄弟,老子赌了一辈子,这次你为何说老子赌不得。''小鱼儿嘴里塞满了肉,道:``只因你一赌,就要上当。''轩辕三光道:``老子是老赌鬼,这龟儿子顶多也不过算是个小赌鬼,他怎能令老子上当,何况这赌法最公平不过,谁也作不得弊,除非他也是个老鼠精。''小鱼儿悠悠说道:``你说这赌法最公平,你也赢了许多次了,是么?''轩辕三光道:``不错。''

小鱼儿道:``你可知道你是怎么会赢的?''

轩辕三光道:``老子这两天运气好。''

小鱼儿道;``不是。''

轩辕三光皱眉道:难道还有什么别的原因不成?``小鱼儿道:''只因为\ldots."

他故意瞧了江玉郎一眼,立刻摇头道:``不行,我不能说。''轩核三光跳了起来,道:``你为何不能说?''

小鱼儿道:``这两天我体力不好,我怕这小子来跟我拚命。''轩辕三光怒道:``这龟儿子若是敢动你一根手指,老子不把他骨头一根根拆散才怪。''小鱼儿道:``我若和他打架,你帮我忙么?''

轩辕三光道:``当然。''

小鱼儿展颜一笑,道:``好,这样我才能放心说了。''他笑嘻嘻接着道:``你总该知道,老鼠是最怕光的,到了晚上,才敢露面,但只要一点起灯,它们就没有戏唱了。''轩辕三光笑道:``想不到你对老鼠们也了解得很。''小鱼儿笑道:``鱼和老鼠,正是同病相怜,一见到猫就头疼,我不了解它们谁了解?''轩辕三光又笑得喘不过气来,道:``但这\ldots\ldots 这又有什么关系?''小鱼儿道:``这里的老鼠,想必都是刚从外面搬进来的,外面只怕是来了只恶猫,把它们赶进了洞,谁知这山洞里并没有老鼠饭店,它们若非快饿疯了,也不敢到你们面前来抢肉吃的。\ldots.''轩辕三光笑道:``这还要老子不动,谁若忍不住要动,老鼠就不敢来吃他面前的肉了。''小鱼儿道:``但你还忘了一点,方才这盏灯,是在你身后,你的身子挡住了灯光,所以你才会连赢几次.''轩辕三光拍掌道:``果然不错,你果然是个鬼精灵,连这种事都想得到。''过半晌轩辕三光恍然道:``老子懂了,这龟儿子现在已把灯换了个地方,这灯光正好照在老子面前的肉上,他算定老子这一次要输,所以才要赌大的。''小鱼儿笑道:``正是如此,他现在不但可以把输了的银子捞回来,还可捞你一票。''轩辕三光又气又笑,道:``若不是你来提醒,老子今天竟要在阴沟里翻船了。''小鱼儿转脸瞧着江玉郎,笑道:``如何?我说的不错吧?''江玉郎面上早已变了颜色,口中却冷笑道,``你定要以小人之心度君子之腹,我也没法子。''小鱼儿大笑道:``江玉郎,你那一肚子坏水,别人不知道,我还会不知道么?你在我面前,还装什么蒜?''江玉郎冷冷道:``我只怕是时运不济,才会遇见了鬼。''小鱼儿大笑道:``不错,你遇着了我,当真是倒了八辈子霉了,如今我人赃并获,你就跟我到段合肥那里说话吧。''江玉郎瞧了瞧他,又瞧了瞧轩辕三光,垂首道:``事已至今,我也没有什么话说了,只不过\ldots\ldots{}''他突然一把扭过慕容九的手腕,闪身到慕容九身后,狞笑道:``只不过你们不想要这位慕容姑娘的命么?''

\hypertarget{ux7b2cux4e03ux5341ux4e8cux7ae0-ux5cf0ux56deux8defux8f6c}{%
\chapter{第七十二章
峰回路转}\label{ux7b2cux4e03ux5341ux4e8cux7ae0-ux5cf0ux56deux8defux8f6c}}

小鱼儿暗中吃了一惊,却大笑道:``你若想以慕容九来要挟我.你就错了,你莫非不知道她老是想要我的命,我又怎会要救她。''轩辕三光也跟着大笑道:``老子早就对女人没兴趣,她的死活,更和老子没关系。''江玉郎不动声色,微笑道;既是如此,两位为何不向我出手呀?``轩辕三光道:''老子并不想宰你。"

小鱼儿也笑道:``吃大便的朋友,我杀你还怕脏了手哩。''江玉郎笑道:``既是如此,在下就要告退了,这位慕容姑娘,自然也要跟着在下走的。''小鱼儿大笑道:``你走吧!你带走了慕容九,还怕没有人找你算帐。''江玉郎冷笑道:``这倒不劳阁下费心,若有人问起我来,我便说带走慕容姑娘,只为的是害怕她遭了你的毒手,若不是江小鱼,慕容九此刻又怎会变成如此模样?''小鱼儿叹道:``有其父必有其子,你们父子两人,别的本事没有,栽赃耍赖,混充好人的本事,倒真还没有别人比得上。但你抢了段台肥的银子,事实俱在,你总赖不掉的吧。''江玉郎道:``什么银子,我两手空空,哪里有银子,现在银子是谁的,就是谁动手抢去的,这道理岂非更简单了。''轩辕三光忽道;``你龟儿子想赖起老子来了!''江玉郎冷笑道:``你说我赖你,我就说你赖我,咱们倒不妨看看,江湖中人是相信你恶赌鬼的话,还是相信我江玉郎的话。''轩辕三光也被气得怔住了,苦笑道:``你龟儿子若早生几年,十大恶人哪里还有老子的份。''江玉郎大笑道;``过奖过奖,在下只不过\ldots。.''话声未了,突听几声惨呼,自外面传了进来。

这惨呼声非但分外凄厉,而且历久不绝,发出惨呼的人,不但像是瞧见了一些残忍之极、恐怖之极的事,而且还像是在遭受着某种非人所能忍受的痛苦,这样的惨呼声听在耳里,足以令任何人的血液都为之凝结。

江玉郎的面色变得最快,也变得最惨。拉着慕容九,就想转身奔出小鱼儿大喝道:``来的人既能令他手下发出这样的惨呼,必定可怕得很,你要出去送死没关系,但慕容九\ldots。.''他语声突然顿住,黑暗中,已现出了五条人影!

这时虽然还没有人能瞧见他们的面目,但他们带进来的那种鬼气森森的邪气,已令每个人掌心都沁出了冷汗。

黑暗中,只听得一阵阵令人寒毛悚栗的``吱吱''声,响个不绝,五条人影已缓步走了过来。

小鱼儿首先看到的,是他们那一双惨碧诡异、闪闪发光的眼睛,接着,便瞧见了他们惨变的脸色。

这五人身子里流的血,都好像是惨碧色。

五个人俱都穿着长可及地的黑袍,右手里拿着根鞭子,左手里却提着个铁笼,那听来令人作呕的吱吱声,便是从铁笼里发出来的。

轩辕三光大喝道:``朋友们是什么人?干什么来的?''他喝声有如霹雷,震得山谷回应不绝,正是借着这喝声露了手气功,想先给对方个下马威。

谁知五个黑衣人却连眼睛都没有眨一眨,碧森森的目光,在小鱼儿等人面上不停的打转,也不说话。

江小鱼早已退了回来,大喝道:``九秀山庄的九姑娘和恶赌鬼全都在这里,朋友们若是识相,还是快快退出去吧,再迟想走也走不了啦!''他更是机伶,一看苗头不对,就赶紧先将轩辕三光和慕容九的名头抬出来吓人,这两人名头实在也不小,何况,就算吓不退对方,也是别人的名字,全不关他的事,对方要找也不会找他了。

五个黑衣人仍然声色不动,脚下也未停。

铁萍姑忽然惊呼一声,拉住小鱼儿的手,颤声道,``老鼠\ldots\ldots 笼子里好多老鼠。''几十只老鼠在铁笼里吱吱乱叫,小鱼儿虽不怕老鼠,但瞧见那几十双发光的眼睛,毛茸茸的一大堆老鼠,也不觉全身都起了鸡皮疙瘩。

为首的黑衣人嘿嘿一笑,道:``不错,老鼠\ldots\ldots 在下五人此来找的只是老鼠,与人无关,各位只要站着不动,在下必定秋毫无犯。''他话虽说得客气,但语声却比老鼠叫更令从作呕。

轩辕三光忍不住问道:``捉老鼠干什么?''

那黑衣人嘿嘿笑道:``敝上非鼠肉不欢,是以令在下等四处搜捕,但此间方圆百里内的老鼠都已流窜入山,是以在下等才一路追捕过来。''小鱼儿恍然失笑道:``难怪这山洞里老鼠特别多,原来就是被他们赶来的,我本来还以为外面来了只恶猫哩。''轩辕三光面色却微微一变,似乎想起个人来,厉声道:``朋友们的主子是谁?''那黑衣人不再答话,却挥了挥手。

五个人嘴里便同时发出了吹竹之声,这声音宛如吹竹,却又不似,听得人又觉恐怖,又是恶心。

铁萍姑早已掩起了耳朵,小鱼儿也听得牙痒痒的,全身不舒服,但他好奇之心最重,见了这种怪事,一心只想瞧个究竟。

轩辕三光双目圆睁,目中却有惊恐之色。

小鱼儿忍不住悄声问道:``这喜欢吃老鼠的朋友是谁?你知道么?''轩辕三光道:``嗯。''

他像是想起了件十分可怕的事,竟想得出了神,小鱼儿在他耳朵边说的话,他竟连一个字也没有听见。

就在这时,土石下异声骤起,像是有几千几百只老鼠,在吱吱乱叫,拼命要往外面逃窜出来!

黑衣人立刻将手提的铁笼,分成五个方位摆开。

就在这时,一大群老鼠,已从山有的裂隙中,黑暗的角落里,潮水般奔了出来,多得简直数也数不清。

小鱼儿一辈子瞧见过的老鼠,加起来也没有此刻十分之一多,他简直做梦也想不到世上竟有这么多老鼠。

此刻奔来的若是一大群饿狼、一大群虎豹,小鱼儿也末见得会如何害怕,但这一大群老鼠,却令他脸色发白,身子发冷,刚吃下的酒肉,直在胸口里往外冒,几乎就要吐出来。

他虽然还能忍住,但铁萍姑却已忍不住了,``哇''的一声,吐了满地,老鼠从他们胸旁奔过,几个一等的武功高手,竟都忍不住跳起来,跳到那块巨石上,挤成了一堆,铁萍姑双手掩着了脸死也不肯再张开眼睛。

但小鱼儿眼睛却仍睁得大大的。

几千几百只老鼠就在自己脚底下奔过去,这景象究竟不是人人都能看得到,他怎舍得不看。

只见黑衣人口中吹竹之声不停,手里长鞭飞舞,将老鼠一群群的赶进铁笼,铁笼虽不小,却也并不太大,但老鼠一群群的跑进去,就像是填鸭子似的,塞不进去也要塞,一只叠着一只,一群叠着一群。

直到五只铁笼子都塞得水泄不通,看来已像五个大肉团的时候,黑衣人才放下鞭子,停住了哨声。

剩下的老鼠竟也立刻就如蒙大赦一般,又四面八方地逃了回去,眨眼间又逃得个不剩。

山洞里立刻又恢复了平静,铁萍姑偷偷瞧了一眼,才敢放下手,脸上已满是冷汗,就像是刚做完一场噩梦似的。

小鱼儿长长叹了口气,苦笑道:``我如今才知道,老鼠竟如此可怕。''轩辕三光干咳几声,道:``格老子,成千成百只耗子,看起来真和十只八只差得多了,四川耗子虽多,但老子也没有看过有这么多的。''江玉郎咯咯笑道:``在下倒不是害怕,只不过觉得有些恶心。''为首的那黑衣人大笑道:``这位朋友说的不错,老鼠非但不可怕,而且还美味得很。''小鱼儿苦着脸道:``美味?''

黑衣人怪笑道:``你若不信,一试便知。''

他竟从笼子里捞出只毛茸茸的老鼠来,往小鱼儿手里送。

小鱼儿赶紧摇手笑道:``君子不夺人所好,老鼠既是如此美味,还是留给阁下自用吧。''那黑衣人嘿嘿笑道:``可惜可惜,想不到阁下看来胆子虽大,却连只老鼠都不敢吃,否则阁下尝过老鼠肉之后,再吃别的肉就味同嚼蜡了。''小鱼儿身上鸡皮疙瘩又冒了出来,大声道:``朋友既然已找到了老鼠,此刻总该走了吧。''江玉郎忽然阴恻测笑道:``你素来最爱多管闲事,这次怎地不管了?''小鱼儿笑道:``若有人喜欢吃老鼠,那是他自己的事,我为何要管,正如你喜欢吃大便,我也是管不了的。''江玉郎面色微微一变,转眼去瞧那黑衣人道:``朋友真要走了?''那黑衣人道:``在下早已说过,此来只是为了老鼠,与人无干。''江玉郎叹了口气,道:``难道朋友就不知道,这里有比老鼠更好的东西么?''那黑衣人眼睛在慕容九和铁萍姑身上一转,怪笑道:``本门弟子,都觉得女人不如老鼠可爱\ldots-江玉郎将慕容九拉到一边,远远躲开小鱼儿和轩辕三光,才笑嘻嘻道:''金银珠宝难道也不比老鼠可爱么?``那黑衣人眼睛一亮,道:''金银珠宝?在哪里?``江玉郎眼角往后洞瞟了一眼,口中却笑道:''有这两位在此,我不敢说。``小鱼儿叹了口气,苦笑道:''我真奇怪,以前为何不早把你宰了。``江玉郎大笑道:''就凭你要杀我,只怕还不容易。``只见那黑衣人互相打了个眼色,提起了铁笼,就往后洞走,小鱼儿闪身挡住了他们的去路,笑嘻嘻道:''后面没有老鼠,各位还是请回吧。``那黑衣人嘿嘿笑道:''朋友最好知道,你虽不敢吃老鼠,老鼠却敢吃你的。``小鱼儿笑道:''我已有好几天没洗澡了,肉脏得很,老鼠只怕也吃不下去。``那黑衣人大笑道:''好,你这人有趣得很,而且胆子也不小"``小''字说出口,他掌中皮鞭已挥了出去。

这鞭子又黑又亮,也不知是什么做的,份量却不轻,黑衣人手劲更不小,鞭子飞出来,又急又重,鞭风嘶嘶直响。

但小鱼儿一伸手就抓住了鞭梢,笑道:``朋友还不知道,我虽然对老鼠有些头疼,但人,我却是不怕的。''那黑衣人脸色早已变了,用力想夺回鞭子,但鞭子却好像已长在小鱼儿手上了,他用尽吃奶的力气,也动不了分毫。

小鱼儿笑嘻嘻道:``老鼠既不认得我,我也不认得老鼠,你们就算把天下的老鼠都捉去吃光,我也不管你们,但你们若想打别的主意,我却要不客气了。''那黑衣人冷笑道:``你不来惹咱们,咱们也不惹你,但你若想挡咱们的去路,咱们却要不客气了!''他话一说完,口中突又发出了吹竹声。

他身旁两个黑衣人就拉开手中铁笼的门,铁笼里塞得满满的老鼠,立刻像箭一般窜了过来。

小鱼儿一惊,几十几百只老鼠,已窜上他身子,在他身上又叫又咬,小鱼儿又是吃惊,又是恶心,挥也挥不去,赶也赶不走,抓鞭子的手只得放开了。

五根鞭子立刻没头没脑的向他抽了过来。

小鱼儿满身都是老鼠,哪里还能施展得开手脚,只得一面躲,一面退,口中不住大呼道:``轩辕三光,你还不来帮忙么?''但轩辕三光的脸色也发了青,迟疑着,慢慢走过来。

那黑衣人厉声道:``轩辕三光,你既已猜出我等是何人门下,你还敢出手?''轩辗三光怔了怔,竟然退了回去。

小鱼儿大喝道:``轩辕三光,你难道也像女人,怕老鼠?''轩辕三光竟索性转过头去,不瞧他了。

小鱼儿身上老鼠非但没有少,而且越来越多,身上又疼又痒又麻,已不知被老鼠咬了多少口。

那五根鞭子,更毒蛇般抽了过来。

小鱼儿这才真的有些慌了。

他无论遇着什么事,都能沉着对付,但这满身毛茸茸的大老鼠,却令他手慌脚忙,简直不知该如何是好。

江玉郎忍不住大笑道:``自命为天下第一聪明的人,竟连老鼠也对付不了\ldots\ldots 江小鱼,你几时想到过你会死在老鼠手里。''小鱼儿身上巳挨了几鞭子,不禁长叹道:``我实在没有想到过\ldots\ldots{}''突然间,只见人影一闪,一个黑衣人已被人挟颈一把抓住,从后面抛了出去,手里的鞭子也被人夺走。

另四个黑衣人惊呼忽吼,四条鞭子向来的这人抽过去,却不知怎地,鞭子竟不听话了,你的鞭子抽我,我的鞭子抽你。

四个人竟自己打起自己人来。

小鱼儿大笑道:``花无缺,想不到你居然来了。''来的人自然正是花无缺,除了他``移花接玉''的功夫外,还有谁能令这四个人自己打自己。

小鱼儿见他来,自然松了口气,江玉郎见他来了,却也开心得很,只道花无缺救下小鱼儿,只不过为的是要自己动手杀他而已。

花无缺鞭子飞舞,已将小鱼儿身上的老鼠全部赶走。

那五个黑衣人已全都吓呆了,张口结舌,呆呆地瞧着花无缺,手里的鞭子再也不敢抽出去。

为首的那黑衣人吃吃的道:``朋友是谁?为何来多营闲事?''花无缺淡淡道:``你纵不认得我,也该认得这手功夫吧?''那黑衣人想了想,变色道:``移\ldots\ldots 移花接玉。''那黑衣人跺了跺脚,又道:``既是移花宫的人到此,在下等只有告退。''小鱼儿笑道:``你们弄了我一身老鼠屎,此刻就想走么?''那黑衣人冷笑道:``这话只怕还轮不到阁下来说,就凭阁下\ldots\ldots 哼!''花无缺道:``你们瞧他不起?''

花无缺微微一笑,又道:``既是如此,莫要老鼠帮忙,你们不妨再和他打一场,五人齐上也无妨,我绝不出手。''那黑衣人狞笑道:``只要阁下不出手,这小子\ldots\ldots{}''话未说完,小鱼儿一拳已击出,他明明瞧见小鱼儿这拳打出来,竞偏偏躲不开,鞭子还未飞出,人已被打得飞了出去。

另四个黑衣人齐地扑过来,但小鱼儿指东打西,片刻间五个人都被他打得东倒西歪,鼻青脸肿。

花无缺微笑道:``各位此刻已知道他的厉害了么?''五个黑衣人哪里还有一个说得出话来,竟都倒在地上,连爬都爬不起来了,小鱼儿大笑道:``想不到竟不如老鼠,竟如此经不得打。''黑衣人既不敢答腔,也不敢动。

那边轩辕三光却直向小鱼儿使眼色,打手势,意思竟是要小鱼儿放他们走,小鱼儿皱了皱眉头,道:``我现在手已不疼了,还不快站起来。''黑衣人非但没有站起来,身子反而缩成了一团。

小鱼儿大笑道:``五个这么大的人,居然还好意思赖在地上,难道还要等你们师娘来,抱你们起来么?''黑衣人本来还在颤抖,此刻却连动都不动了。

轩辕三光忽然窜过来,一把拎起个黑衣人,只瞧了一眼,脸色便已改变,缓缓将黑衣人又放了下去,叹道:``他们只怕永远也站不起来了。''轩辕三光将他们的尸体一动,只见口、鼻、五宫中,便有鲜血渗出来,就连这血,也都是惨碧色的。

小鱼儿也不禁怔住了,道:``这五人挨了两拳,难道就气得自杀了么?''花无缺皱眉道:``他们也许是以为你放不过他们,所以自己先就\ldots\ldots{}''小鱼儿跺足道:``他们就算弄了我一身老鼠屎,我也不会杀他们的呀,这些人难道是老鼠吃多了,人也变得像老鼠一样想不开。''轩辕三光苦笑道:``这些龟儿子说死就死,死得倒真快。''小鱼儿道:``是呀,难道他们嘴里早就含着毒药,随时都准备死不成。''轩辕三光皱着眉蹲下,将这黑衣人的嘴扳开,立刻就有一般掺碧色的、浓得像墨汁似的苦水,从他嘴里流出来,还带着种令人作恶的臭气。

轩辕三光叹道:``你说的不错,这些杂种竟是将毒药藏在牙齿里的。''小鱼儿皱眉道:但他们为什么要自杀呢?我既没有杀他们的意思,也不想逼问他们的口供,他们难道真是活得不耐烦了么?"轩辕三光对这黑衣人全身都搜了一遍,只搜出了些银子,此外连一条汗巾都没有。

这些人身上除了银子外,竟是什么都不带。

轩辕三光想了想,忽又一把撕开他的衣襟,失声道:``你想不通的事,回答就在这里。''只见这黑衣人胸膛上,赫然有十个大字。

这十个惨碧色的字,竟像是用碧磷烧出来的,几乎已烧及骨头,伤痕深深印在肉里,无论用什么法子,都休想除去。

这十个字写的是:``无牙门下士,可杀不可辱。''小鱼儿道:``无牙门下士,可杀不可辱这算什么见鬼的意思?''轩辕三光叹道:``这意思就是叫他们打不过别人时,赶快自杀,免得丢他们主子的人,他们现在若不自杀,回去死得只怕更要惨十倍。小鱼儿道:''你是说他们怕回去受主子的酷刑,所以宁可现在自杀,是么?``轩辕三光道:''正是。``小鱼儿道:''但他们在这里挨揍,他们的主子根本不知道呀,只要他们自己不说,难道我还会说出去不成。``轩辕三光道:''这些龟儿子也许正是以小人之心,度君子之腹,以为你------``花无缺忽然道:''不是这原因。"

小鱼儿道:``你说是什么原因?''

花无缺缓缓道:``我瞧见他们时,他们本有七个人的。''轩辕三光拍手道:``这就对了,他们五个人进来,还留着两个人躲在暗处,那两人见势不抄,恐怕已暗中溜了,这五人算定他们回去一定要报告的,与其到那时凌迟受罪,倒不如现在落个痛快的好。''小鱼儿瞪着花无缺道:``你进来时,没有瞧见那两个人么?''花无缺苦笑道:``我听见你的呼喊声,立刻就闯了进来,并没有去留意别的。''小鱼儿忽然一拍脑袋,大叫道:``不好,我们竟被这些鬼老鼠弄晕了头,五六个大活人从我们身边溜走,我们竟全都不知道。,轩辕三光四下瞧了一眼,也失声道:''不错,那姓江的小杂种,果然溜了。``小鱼儿跺足道:''你进来时,我还瞧见他的,那时他脸上像是还有欢喜之色,以为你要来宰我,后来想必是-
发现情况有点不对,就立刻开溜唉,这小子一向是个鬼精灵,我本该特别盯着他才是的。``花无缺默然半晌,淡淡一笑,道:''他自己走了倒也好。``小鱼儿瞪眼道:''你是早已瞧见了他的,是么?``花无缺道:好像瞟过一眼。''小鱼儿道:``但你还是放他走了。''花无缺叹道:``我和他总算交友一场\ldots\ldots{}''小鱼儿大叫道:``但你为何要让他将幕容九一起带走呢?''

\hypertarget{ux7b2cux4e03ux5341ux4e09ux7ae0-ux53e3ux871cux8179ux5251}{%
\chapter{第七十三章
口蜜腹剑}\label{ux7b2cux4e03ux5341ux4e09ux7ae0-ux53e3ux871cux8179ux5251}}

花无缺听小鱼儿说慕容九已被江玉郎带走,不由怔了怔,道:慕容姑娘?\ldots\ldots{}

慕容姑娘也和他在一起麽?

小鱼儿道:你\ldots\ldots 你没有瞧见?

花无缺也不禁顿足道:我只见到有个女子在他身边,再也未想到会是慕容姑娘,那时我一心只顾着你,再加上灯光太暗,竟未瞧清她的脸。

轩辕叁光忽然一拍小鱼儿肩头,道;但和你一起出来的那姑娘竟会也溜了呢小鱼儿皱眉道:是呀!她为什麽也溜了呢?难道她怕见到花无缺?

花无缺道;这位姑娘又是什麽人?

小鱼儿道:她叫铁萍姑\ldots\ldots 你认不认得她?

花无缺道:我连这名字都未听到过。

小鱼儿用手指敲着脑袋,道:你既不认得她,她为何要溜呢?我实在想不通\ldots\ldots{}

铁萍姑的确是有理由的,而且理由充足得很。

花无缺本来也是认得她的,他没有听见铁萍姑这名字。只不过是因为她那时并不叫铁萍姑。铁萍姑自然更认得花无缺。

她一眼瞧见花无缺,脸色突然改变,赶紧扭过了头,等到她确定花无缺并没有留意她,她就以最快的速度溜了出去。

这时已近黄昏,满天夕阳,映着青葱的山岳,微风中带着香,铁萍姑深深吸了口气,心里也不知是什麽滋味。

十多年来,这是她第一次得到自由,第一次可以单独自立,她想做什麽,就可以做什麽,想到那里去,就可以到那里去。

但她反而不知该如何是好了。江玉郎跟着她溜了出来。

他瞧见花无缺,本来很欢喜,但他又瞧见花无缺对小鱼儿的神情竟似已变了,他立刻就发觉情况不对。

铁萍姑会溜走,江玉郎本也觉得很奇怪。铁萍姑一展身形,江玉郎更是一惊。

这少女轻功之高妙,固然惊人,最奇怪的是她身形飞掠间,竟带着一种独特的高贵的姿势,和花无缺超群拔俗的身法有几分相似。

江玉郎的眼睛立刻眯起来了,他又是惊讶,又是奇怪,眼珠子一转,竟也立刻拉着慕容九追了下去。

江玉郎是从来不肯放过任何机会的,但他也末发觉,螳螂捕蝉,黄雀在後,还有两个人在身後跟着他。

等到小鱼儿花无缺和轩辕叁光出来时,除了那些身外,洞外已没有一个活人的影子了。

小鱼儿瞧着这些身,叹道:这些人虽是江玉郎带来的,江玉郎虽可抛下他们不管,但咱们\ldots\ldots{}

轩辕叁光道:这些事你莫管,埋死人,是我的拿手本事。

小鱼儿笑道:那麽,你叫我做什麽呢?

轩辕叁光叹道:你就得要准备去对付一个你生平从来没有遇见过的,最毒最狠最令人恶心,也最令人头疼的对头了。

小鱼儿道:你莫非是说那没有牙的小子!

轩辕叁光道:我说的正是魏无牙。

小鱼儿道:那五个人又不是我杀死的。

轩辕叁光道:你以为他很讲理麽!只要你沾着他门下一点,他就跟你没有完。

小鱼儿深深吸了口气,道:你将这位无齿之徒说得这麽厉害,他到底是谁呀!

轩辕叁光道:你可听见过十二星象这名字!他就是十二星象中的子鼠\ldots\ldots{}

小鱼儿失笑道:我当你说谁,原来是十二星象\ldots\ldots 十二星象中的人,我也领教过了,倒也未见得能拿我怎样。

轩辕叁光道:十二星象之所以成名,就是因为魏无牙,他们声名最盛时,江湖中人听到十二星象这名字,晚上连觉都睡不着,那时你只怕还末生出来哩。

小鱼儿笑道:你这麽样一说,我倒幸好远末生出来了。

轩辕叁光道:不说别人,就说我们十大恶人,总算是天不怕地不怕的,但听到魏无牙这叁个字,还是要头疼好几天。

小鱼儿这才为之动容,道:连十大恶人鄱头疼的角色,想必是有些门道了。

花无缺忽然道;我倒也听到过这名字。

小鱼儿笑道:难道连移花宫都对他头疼不成。

花无缺缓缓道;我出宫时,家师曾要我特别留意两个人,其中一人就是魏无牙。

小鱼儿道:还有一个呢?

花无缺苦笑了笑,道;还有一位是燕南天燕大侠。

小鱼儿默然半晌,道;他现在那里?

轩辕叁光道:十二星象最近几年所以抬不起头来,就是因为魏无牙十多年前忽然不见了,有人说他是因为被移花宫主所伤,所以躲起来的,也有人说他是为了要练一种神秘的武功,所以才不愿见人\ldots\ldots{}

小鱼儿道;你想\ldots\ldots 他会躲到那里去呢!

轩辕叁光叹道:他要躲起来,只怕连鬼都找不着。

小鱼儿皱着眉头,喃喃道:他莫非就躲在龟山\ldots\ldots 那损人不利己兄弟两人,临死前说的人,莫非就是他\ldots\ldots{}

他忽然一拍轩辕叁光肩头,笑道:你埋过死人之後,还想去干什麽呢?

轩辕叁光道;我本想去找人赌一场,但想起魏无牙又出现了,老子竟连赌兴都没有了。

小鱼儿道:那麽就麻烦你把洞里的银子,去送给段合肥吧,同时告诉段合肥,这些银子本是谁藏起来的。

他一笑接道;只要你还给他,然後再把银子赢回来都没关系,段合肥很喜欢斗蟋蟀,也很喜欢吃肉,你若和他赌吃肉,他一定会奉陪。

轩辕叁光就算想拒绝,也来不及了,小鱼儿话还没有说完,已拉着花无缺飞也似的走开。

轩辕叁光只得摇头苦笑道:格老子,要想拒绝江小鱼求你的事,真他妈的不容易。

小鱼儿一面走,一面将自己这段经过说了出来。

花无缺自然听得满心惊奇,连他也弄不懂这位铜先生究竟在搞什麽鬼了,他也不禁渐渐开始怀疑铜先生的来历。等他说出自己经过的事,小鱼儿也觉得奇怪得很,忍不住道:燕大侠既然要等到找着我时才肯放你,那麽现在又怎会只有你一个人呢?他到那里去了?

花无缺道:这两天也不知怎地,我忽然变得心神不定起来,好像有什麽灾难要降临似的,我一生中从来也没有这种情形发生。

小鱼儿笑道;这两天有灾难的是我,你怎会心神不定起来,这倒也奇怪得很。

花无缺道:燕大侠想必也发现我神情有异,就问我想干什麽,我就说想出来走走\ldots\ldots 我本以为燕大侠不会答应我的,谁知他竟答应了。

小鱼儿失声道:你要走,他就让你走了麽!

花无缺道:不错。

小鱼儿叹道:燕南天到底是燕南天,到底和那铜先生不同,老实说,你遇见他这样的人,实是你的运气。

花无缺默然无语,他心里佩服一个人时,嘴里本就不会说出,何况他佩服的竟是移花宫的对头呢。

小鱼儿忽又笑道:但你也不愧是个君子,他才会放心你,他遇着的若是我,只怕也不会这麽容易放我走了。

花无缺一笑,道:你为何要认为你自已不是君子呢?

小鱼儿默然半晌,缓缓道:这也许是因为我从小就没见过一个君子,我根本就不知道君子是什麽样子的,等我见着一两个君子时,他们又总是要令我失望\ldots\ldots{}

花无缺笑了笑,道:燕大侠还在等着我,你\ldots\ldots{}

小鱼儿忽然截口道:你见着他时,就说并末见到我,好吗?

花无缺奇道:为什麽?你难道不跟我去见他?

小鱼儿道:我\ldots\ldots 我想到龟山去,但他却一定不会让我去的。

花无缺更奇怪,道:你要去龟山?为什麽?

小鱼儿道:我要去救人。

花无缺讶然道;莫非是十大恶人中的?但他们\ldots\ldots{}

花无缺道:但他们\ldots\ldots{}

小鱼儿苦笑道;他们虽不是好人,但我却是被他们养大的,我若不知道这事也就罢了,现在既已知道,就不能不管,何况\ldots\ldots 我还想顺路去找找那铁萍姑,她武功虽不错,但简直没出过门,根本不知道世情之险恶,随时随地,都会上人家当的,她既然救了我一次,我好歹也要救她一次\ldots\ldots{}

他做了个鬼脸,笑道:你要知道,欠女人的账,那滋味可不是好受的。

铁萍姑也不知是否被那一阵阵油香菜香引过来的,总之,她已走入了这小镇,而且她也已发觉自己肚子饿得发慌。她在那山洞里,虽然也吃了些东西,但一个人在饿了两叁天之後,食欲又岂非那麽容易就能满足的。小酒的桌子,在灯光下发着油光,十几只绿头苍蝇,围着那装满卤菜的大盘子飞来飞去。

这种地方,在平时用八人大轿来抬,铁萍姑都不会走进去的,但现在,她就算爬,也要爬进去。

致萍姑现在的样子,的确不像是个好客人。

她脸上又是灰,又是汗,头发乱得像是麻雀窝,衣服更是又脏又破,看来就算不像个刚从监狱里逃出来的女犯,也像是个大户人家的逃妾,只可惜她也和世上大多数的人一样,只看得见别人身上的脏,却看不见自己的。

小店里只有叁个客人,都瞪大了眼睛瞧着她,铁萍姑却再也想不到这些人是为什麽在瞧自己。

店伙终於走过去,勉强笑着道;姑娘来碗面好吗?小店的阳春面,一碗足足有半斤。

铁萍姑深深吸了气,道:面,我吃不惯,你给我来一只粟子烧鸡,一碟溜鱼片,一碟炸响铃,半只火腿去皮蒸一蒸,加点冰糖,一碗笋尖炖冬菇汤\ldots\ldots 哦,对了,把那边盘子里的卤菜,给我切上几样来。

这些菜,在她眼中看来,实在平常得很,她已觉得很委屈自己了,以她现在旺盛的食欲,她简直可以吃得下一匹马。

但旁边叁个客人听她说了一大串,却忍不住笑出声来,那店伙更是瞪大眼睛,直摸脑袋。

铁萍姑瞪眼道:怎麽,你们这店,难道连这几样菜都没有麽?那店伙慢吞吞道:菜是有的,但小店却还有个规矩!

铁萍姑道:什麽规矩?

小店本轻利微,禁不得赊欠,所以来照顾的客人,都得先付账。

铁萍姑怔住了。她身上怎麽会带着银子,她只知道银子又脏又重,她简直没有想到银子会这麽有用。

那店伙皮笑肉不笑,道;吃饭是要付账的,这规矩姑娘难道都不懂麽?

旁边那叁个客人哈哈大笑,其中一人笑道:姑娘不如到这边桌子上来,一起吃吧,这里虽没有栗子烧鸡,但鸭头却还有半个,将就些也可下酒了。

铁萍姑只希望自己根本没有生出来,没有走进这鬼铺子,她只觉坐在这里固然难受,这样走出去却更丢人,简直不知道该如何是好了。

江玉郎就在这时走了进来,这时候当真选得再妙没有。

他走到铁萍姑面前,恭恭敬敬行了个礼,双手捧上了十几个黄澄澄的金锭子,陪笑道;姑丈知道表姊出来得匆忙,也许末及带银子,所以先令小弟送些零用来。

那店伙立刻怔住了,旁边叁个客人也怔住了。

最发怔的,自然还是铁萍姑,她自然认得江玉郎就是小鱼儿嘴里的小坏蛋,却想不通这究竟是怎麽回事。

她只好眼瞧着江玉郎在她身旁坐下来慕容九就好像是个傀儡,痴痴地笑着,痴痴地随着他坐下。

那店伙却变得可爱极了,弯着腰,陪着笑,送菜送酒,不到片刻,卤菜就摆满了一桌子。

江玉郎用热茶将铁萍姑的筷子洗得乾乾净净,陪笑道;这卤菜倒还新鲜,表姊你就将就吃些吧。

铁萍姑突然来了个这麽样的表弟,当真也不知是好气还是好笑,但江玉郎却实在太懂得女孩子的心理了,他在铁萍姑最窘的时侯,替她作了面子,铁萍姑怎能不感激。

饭吃完了,铁萍姑风风光光的付了账,心里也不免开心起来,但剩下来的金子,她却又不好意思拿了。

她始终没有和江玉郎说过一句话,现在也没有理他,就迳自走出去,江小鱼既然讨厌这个人,这人必定不是好东西。

铁萍姑在前面走,江玉郎就在後面跟着。

铁萍姑终於忍不住道:你还想干什麽?

江玉郎陪笑道:我只是怕姑娘一个人行走不便,所以想为姑娘效效劳而已。

铁萍姑道:我的事,用不着你来费心。她嘴里虽这麽说,心却已有些动了。

只见道路上人来人去,没有一个人是她认得的,远处灯火越来越少,更是黑暗得可怕。

她实在不知道该往那里去她忽然发觉,一个人若想在这世上自由自在地活着,实在不如她想像中那麽容易。

江玉郎许久没有发出声音,他莫非已走了麽铁萍姑忽然发觉自己竟怕他走了。

她赶回头,江玉郎还是笑嘻嘻地跟在她身後。

她心里虽松了气,嘴里却大声道:你还跟着我作什麽江玉郎笑道:天色已不早,姑娘难道不想休息休息麽?

铁萍姑咬着嘴唇,她实在累了,但该到什麽地方休息呢?

江玉郎眼睛里发着光,笑道:姑娘就算不愿在下跟着,至少也得让在下为姑娘寻家客栈。

这次,铁萍姑又说不出拒绝的话了。

但找好客栈後,铁萍姑立刻慎重地关起门,大声道:你现在可以走了,走得越远越好。

这次江玉郎居然听话得很,铁萍姑等了半晌,没有听见他动静,长长松了气,倒在床上。

她想着江小鱼,想着花无缺,又想着江玉郎\ldots\ldots 江小鱼为什麽会和他是对头?他的人好像并不太坏嘛。但铁萍姑实在太累了,她忽然就睡着了。

第二天早上一醒来,她立刻又觉得肚子饿得很。

铁萍姑好几次想要人送东西来,每次又都忍住,她越想忍肚子越是饿得忍不住。

突听店小二在门外陪笑道;江公子令小人为姑娘送来了早点,姑娘可现在吃麽?

吃完了,铁萍姑终於才发自己的模样有多可怕,她恨不得将桌子上的镜远远丢出去,她全身都觉得发。

就在这时,店小二又来了。这次他捧来了许多件柔软而美丽的崭新衣裳一套精致的梳装用具,高贵的香粉,柔软的鞋袜,这些东西,铁萍姑能拒绝麽?

等到铁萍姑穿上这些衣袜,梳洗乾净的时候,江玉郎的声音就出现了。不知在下可否进来?

现在,铁萍姑肚子里装着是人家送来的食物,身上穿着的,是人家送的衣服鞋袜。她还能不让他进来麽?

到了这天中饭时,江玉郎自然还没有走,铁萍姑也没有要他走的意思了,她现在只觉自己实在少不了他。

这自然也是个小客栈,小客栈的小饭厅里,只有他们两个人,据江玉郎说:那位慕容姑娘不舒服,所以没有起来。

其实呢,是江玉郎点了她的睡穴,把她卷在棉被里,她虽然只不过是个傀儡,江玉郎也不愿意她来打扰。

小客栈里自然不会有什麽好菜,但江玉郎还是叫满了一桌子,还要了两壶酒,他笑着道:姑娘若不反对,在下想饮两杯。

铁萍姑也不说话,但等到酒来了,她却一把夺过酒壶,满满倒了一大杯酒,一仰脖子乾了下去。

她只觉一股又热又辣的味道,顺着她脖子直冲下来,烫得她眼泪都似乎要流出来。她几时喝过酒的。

江玉郎瞧得肚子里暗暗好笑,嘴里却道:姑娘若是没有喝过酒,最好还是莫要喝吧,若是喝醉了\ldots\ldots 唉。他装得满脸诚恳之色,真的像是生怕铁萍姑喝醉。

其实他恨不得她马上就醉得人事不知。

铁萍姑仰起脖子乾了一杯,江玉郎在旁边只是唉声叹气,其实却开心得要死。

一杯酒下肚,铁萍姑只觉全身又舒服又暖和,简直想飞起来,等到喝第四杯酒时,她只觉这酒实在是世上最好喝的东西,既不觉得辣,也不觉得苦,喝到第五杯时,她已将所有的烦恼忘得乾乾净净。

这时江玉郎就开始为她倒酒了。江玉郎笑道:想不到姑娘竟是海量,来,在下再敬姑娘一杯。

铁萍姑又乾了一杯,忽然瞪着江玉郎,道:你究竟是个好人,还是恶人?

江玉郎微笑道:姑娘看在下像是个恶人麽?

铁萍姑皱眉道:你实在不像,但\ldots\ldots 江小鱼为什麽说你不是好东西。

江玉郎苦笑道:姑娘跟他很熟麽?

铁萍姑道:远好\ldots\ldots 不太熟。

江玉郎道:姑娘以後若是知道他的为人,就会明白了\ldots\ldots 唉,那位慕容姑娘,若不是他,又怎会变成如此模样。

铁萍姑怔了半晌,又倒了杯酒喝下去。

江玉郎笑道;此情此景,在下本不该提起此等令人懊恼之事。

铁萍姑忽也吃吃笑了起来道:不错,我们该说些开心的事,你有什麽令人开心的事,就快说吧,你说一件,我就喝一杯酒。

江玉郎是什麽样的才,若要他说令人开心的事,叁天叁夜也说不完,他说了一件又一件。

铁萍姑就喝了一杯又一杯,她一面笑,一面喝。

到後来江玉郎不说她也笑了,再到後来,她笑也笑不出,一个人从椅子上滑下去,爬都爬不起来了。

江玉郎眼睛里发了光,试探着道:姑娘还听得到我说话麽?铁萍姑连哼都哼不出了。

江玉郎把她从桌子下拉了起来,只觉她全身已软得像是没有一根骨头,江玉郎要她往东,她就往东,要她往西,她就往西。

突听一人大笑道:兄台好高明的手段,在下当真佩服得很。

江玉郎一惊,放下铁萍姑,霍然转身。只见一高一矮两个人,已大笑着走了进来。

\hypertarget{ux7b2cux4e03ux5341ux56dbux7ae0-ux4ebaux9762ux517dux5fc3}{%
\chapter{第七十四章
人面兽心}\label{ux7b2cux4e03ux5341ux56dbux7ae0-ux4ebaux9762ux517dux5fc3}}

小厅里的光线暗得很,这一高一矮两个人,站在灰蒙蒙的光影里,竟带着种说不出的邪气。

他们长得本没有什麽特别的地方,但那神情,那姿态,那双碧森森的眼睛,就好像本非活在这世上的人?

江玉郎心里已打了个结,脸上却不动声色,微笑道:两位说的可是在下麽?

矮的那人吃吃笑道:在下也曾见到过不少花丛圣手、风流种子,但若论对付女人的手段,却简直没有人能比得上兄台一半的。

江玉郎哈哈笑道:两位说笑话的本事,倒当真妙极。

矮的那人阴森森笑道;现在这位姑娘,已是兄台的手中之物了,眼见兄台立刻便要软玉温香抱个满怀,兄台难道就不愿让我兄弟也开开心麽?

高的那人冷冷道:在下只是说,兄台若想真个销魂,多少也要给我兄弟一些好处,否则\ldots\ldots{}

江玉郎眼珠子一转,脸上又露笑容,道:两位难道也想分一杯羹麽矮的那人笑道:这倒不敢,只是兄台既有了新人,棉被里那位姑娘,总该让给我兄弟了吧。

江玉郎大笑道:原来两位知道的还不少。

高的那人冷冷道:老实说,自从兄台开始盯上这位姑娘时,一举一动,我兄弟都瞧得清清楚楚。

江玉郎大笑道:妙极妙极,想不到兄台倒是对在下如此有兴趣,快请先坐下来,容在下敬两位一杯。

高的那人道:酒,可以打扰,下酒物我兄弟自己随身带着。他竟自袖子里拎出只老鼠,放在嘴里大嚼起来。

江玉郎怔了怔,笑道:原来阁下乃是和那五位朋友一路的,这就难怪对在下如此清楚了。

高的那人冷冷道:在下等除了要请兄台将慕容家的姑娘割爱之外,还要向兄台打听一件事!

江玉郎道:什麽事?

高的那人目中射出凶光,道:洞里的那叁个人,究竟是些什麽人?和你又有什麽关系?

江玉郎展颜笑道:那叁人一个叫轩辕叁光,一个叫江小鱼,一个叫花无缺,两位方才既然瞧见了,总该知道他们都是在下的仇人吧?

那人阴恻恻一笑,道:很好,好极了

江玉郎试探着道:方才那五位朋友,难道已被他们\ldots\ldots{}

那人道:不错,已被他们杀了!

江玉郎松了口气,道:如此说来,在下与两位正是同仇敌忾,在下理当敬两位一杯。

那人道:很好,兄台喝了这杯酒,就跟我兄弟走吧!

矮的那人接道:至於这位姑娘,兄台净可在路上\ldots\ldots 哈哈,我兄弟必定为兄台准备辆又舒服又宽敞的车子。

江玉郎讶然道:两位要在下到那里去?

那人笑道;我兄弟就想请兄台劳驾一赵,随我兄弟一同回去,好将那叁人诱来。

江玉郎忽然笑道:两位意思,在下已全部了解,两位既是想将叁人诱去复仇的,岂非也与在下有利,在下又怎会不答应?

矮的那人大笑道:兄台果然是个通达事理的人,在下也理当敬兄台一杯。

高矮两人举起酒杯,一饮而尽。

但他们的脖子刚仰起来,酒还没有喝下喉咙,江玉郎掌中酒杯已嗤的飞出,打在高的那人咽喉上?

那人狂吼一声,酒全都从鼻子里喷出,人却已倒下。

矮的那人刚大吃一惊,还未来得及应变,江玉郎双掌已闪电般拍出。

他出手虽不如小鱼儿,但也是够狠的了,只听波波两声,矮的那人也随着倒了下去。

江玉郎拍了拍手,冷笑道:就凭你们两人也想将我带走,你们还差得远哩?

只见两人直挺挺躺在地上,动也不动了,但两人却都还没有死,江玉郎只不过点了他们穴道而已。铁萍姑又从椅子上滑了下来,在这越来越暗的黄昏里,她飞红了的面靥,看来实在比什麽都可爱。於是他高声唤入了店伙将两个喝醉的朋友送到隔壁房间,和那位生病的姑娘躺在一起。虽然这两人全没有丝毫喝醉的样子,但做店小二的大多是聪明人,总知道眼晴什麽时候该睁开,什麽时候该闭起。

店小二离开有灯的帐房,站在黑暗的小院子里,他当然并不是有意要来偷听别人的秘密,但这房间里假如有什麽微妙的声音传出来的话,他当然也不会掩起自己的耳朵的,他并不想做一个君子。

那就像乌龟遇见变故时,将头缩回壳里一样只要他自己瞧不见,他就觉得安心了。

这时,铁萍姑酒已醒了。

她只觉全身都在疼痛,痛得像是要裂开,她的头也在疼,酒精像是已变成个小鬼,在里面锯着她的脑袋。

然後,她忽然发觉在她身旁躺着喘息着的江玉郎。她用尽一切力气,呼出来。她用尽一切力气,将江玉郎推了下去。

江玉郎伏在地上,却放声痛哭起来!应该痛哭的本是别人,但他居然先下手为强了。

江玉郎痛哭着道;我知道我做错了,我知道我对不起你,只求你原谅我\ldots\ldots{}

铁萍姑紧咬着牙齿,全身发抖,道;我\ldots\ldots 我恨不得\ldots\ldots{}

江玉郎道:你若恨我,就杀了我吧,我\ldots\ldots 我实在控制不住自己,我也醉了,我们本不该喝酒的。

他忽然又扑上床去,大哭道:求你杀了我吧,你杀了我,也许我还好受些。

铁萍姑本来的确恨不得杀了他的,但现在\ldots\ldots 现在她的手竟软得一丝力气也没有,她本来伤心怨恨,满怀愤怒,但江玉郎竟先哭了起来,哭得又是这麽伤心,她竟不知不觉地没了主意。

江玉郎从手指缝里,偷偷瞧着她表情的变化,却哭得更伤心了,他知道男人的眼泪,有时比女人的还有用。

铁萍姑终於也伏在床上,放声痛哭起来。除了哭,她已没有别的法子。

江玉郎目中露出得意的微笑,但还是痛哭着道:我做的虽不对,但我的心却是真诚的,只要你相信我,我会证明给你看,我这一辈子都不会令你失望的。

他又已触及了铁萍姑的身子,铁萍姑并没有闪避,这意思江玉郎当然清楚得很。

他忽然紧紧抱着了她,大声道:你要麽就原谅我,要麽就杀了我吧\ldots\ldots 你可以杀死我,但却不能要我不喜欢你,我死也要喜欢你\ldots\ldots{}

铁萍姑还是没有动,江玉郎知道自己已成功了,他伏在铁萍姑耳旁,说尽了世上最温柔最甜蜜的话,他知道她现在最需要的就是这些。

铁萍姑哭声果然微弱下来,她本是孤苦伶仃的人,她本觉得茫然无主,无依无靠,现在却忽然发觉自己不再孤单了。

江玉郎忍不住得意地笑了,柔声道:你不恨我了?

铁萍姑鼓起勇气,露出头来,咬着嘴唇道:只要你说的是真的,只要你莫忘记今天的话,我\ldots\ldots{}

忽然间,一声凄厉的惨呼,从隔壁屋子里传来,这惨呼声虽然十分短促,但足以令人听得寒毛悚栗。

江玉郎以一个人所能达到的最快速度装束好一切,箭一般窜出屋子,他好像立刻就忘记铁萍姑了。

江玉郎窜了出去,却没有窜入惨呼声发出的那屋子,却先将这屋子的叁面窗户都开。然後,他燃起盏油灯,从窗户里抛进去!

油灯被摔碎在地上,火焰也在地上燃烧起来。

闪动的火光,令这间暗而潮湿的小屋子,显得更阴森诡秘,他瞧见慕容九还是好好的在棉被里,不觉松了口气。

但他这气没有真正松出来时,他又已发现,那一高一矮两个人已不见了,他们已变成了两堆血!

这景象竟使江玉郎也打了个寒噤,却又安下心。

那危险而残暴的人,此来若只是为了要杀这两人的,他又为何反对又为何要担心害怕呢这时,已有一个人在闪动的火光中出现了。

这人的一张脸,在火光下看来好像是透明的,透明得甚至令人可以看到他惨碧色的骨骼。

他那双眼睛,更不像人的眼睛,而像某一种残暴的食人野兽,在饿了几天几夜後的模样。

江玉郎并不是个少见多怪的人,更不容易被人骇住,但他见到这个人时,却似乎连心跳都已停止!

这人也冷冷地瞪着江玉郎,一字字道:是你点了这两人的穴道?

江玉郎勉强挤出一丝笑容,道:正是在下,在下本不知要拿他们怎麽办,阁下此番解决了他们,在下简直不知该如何感激才好。

他已发觉这人远比想像中还要危险得多,所以赶紧拉起交情来,但这人还是冷冷瞪着他,忽然一笑,露出野兽般的雪白牙齿,缓缓道:我就是他们的主人!他们本是我的奴隶!

江玉郎倒抽了口凉气,道:但你\ldots\ldots 杀死他们的,并不是我。

这人忽然自血堆里拎起了一具体,撕开了它的衣服,闪动的火光中,只见那体上有十个发着碧光的字:无牙门下士,可杀不可辱!

江玉郎几乎呕吐出来,失声道:这\ldots\ldots 这是什麽意思,我不懂。

这人缓缓道:这两人既已被你所辱,我只有杀了他们,免得他们再为我丢人现眼。

江玉郎叹道;有时我也杀人的,但我总是要有一个十分好的理由,譬如说\ldots\ldots{}

在地上燃烧的火焰,突然熄灭了,四下立刻又黑暗得如同坟墓,但这人的眼睛,却仍在黑暗中闪着碧光。

只听他冷冷道:譬如说什麽?

江玉郎道:譬如说,当我知道一个人要杀我的时候,我通常会先杀了他!

他的眼睛也在闪着光,随时都在准备着出手。

他虽然深信这人不是个好惹的人物,却也深信自己也并不见得比这人好惹多少。

谁知道这人却忽然笑了。

他笑的声音,就像是一只老鼠在啃木头似的,令人听得全身都要起鸡皮疙瘩,他大笑着道:我要杀人时,就不跟他多话的。

江玉郎讶然道:你为何不想杀我?

这人冷冷道;你若能在七天之内,带我找到轩辕叁光江小鱼和花无缺,你不但现在不会死,而且还会长命得很?

江玉郎沉吟道;他们也是我的仇人,你若能杀得了他们,我自然很愿意带你去找他们,只可惜要杀这叁个人,并不是件容易事,被他们杀,倒容易得很,你若杀不成他们,反被他们杀死我岂非也要被你连累。

这人厉声道:你要怎样才相信我能杀得了他们?

江玉郎道:这就要看你有什麽法子能令我相信了。

这人冷笑道:我何止有一千种法子可以令你相信,你若想见识见识无牙门下的神功,我不妨先让你瞧一种\ldots\ldots{}

他似乎挥了挥手,便有一种碧森森的火焰,飞射而出,射在墙上,这火焰光芒并不强烈,射在墙上,立刻便熄灭,也根本没有燃烧。

但火焰一闪後,这人已到了院子里。

他根本没有从窗户掠出,却又是怎麽样出来的呢?江玉郎一惊之下,忽然发现墙上已多了个大洞。

江玉郎这才吓呆了,这人的轻功虽惊人,倒没有吓着他,但这种虽不燃烧,却能毁灭一切的火焰,他实在连见都没有见过。

这人已到了他身旁,闪动的目光,已固定在他身上,一字字道:你还想见识别的麽?

突听一人也狂笑着道:无牙门下的神功,我看来却算不得什麽!狂笑声中,已有条人影如流星急坠?

\hypertarget{ux7b2cux4e03ux5341ux4e94ux7ae0-ux5357ux5929ux5927ux4fa0}{%
\chapter{第七十五章
南天大侠}\label{ux7b2cux4e03ux5341ux4e94ux7ae0-ux5357ux5929ux5927ux4fa0}}

这人的身形也不算十分高大,但看来却魁伟如同山岳!

那无牙门下似也被他气势所慑,倒退叁步,厉声道:是谁敢对无牙门下如此无礼?

冀人燕南天!这五个字就像流星,能照亮整个大地!

只听燕南天喝道:你是魏无牙的什麽人?他现在那里?

那人胆虽已怯,却仍狂笑道:你用不着去找家师,无牙门下的四大弟子,每一个都早已想找燕南天较量较量了,不想我魏白衣运气竟比别人好\ldots\ldots{}

江玉郎忽然怒喝道:你是什麽东西,竟敢对燕大侠如此无礼!

喝声中,他竟已扑了过去,闪电般向魏白衣击出叁掌,这叁掌清妙灵动,竟是武当正宗!

武当掌法也正是当时武林中最流行的掌法,江玉郎偷偷练好了这种掌法当然没安什麽好心。

他叁掌全力击出,竟已深得武当掌法之精萃。

魏白衣狂笑道:你也敢来和我动手?

他只道叁招两式,已可将江玉郎打发回去,却不知江玉郎虽是个懦夫,却绝不是笨蛋。

他实在低估了江玉郎的武功。骤然间,他被江玉郎抢得先机,竟无法扭转劣势。

江玉郎知道燕南天绝不会看他吃亏的,有燕南天在旁边掠阵,他还怕什麽,他胆气越壮,出手更急。魏白衣武功虽然诡秘狠毒,竟也奈何不得他。

突见魏白衣身形溜溜旋转起来,四五道碧森森的火焰,忽然暴射而出!却看不出是往那里射出来的!

燕南天暴喝一声,一股掌风卷了出去,卷开了江玉郎的身形,震散了碧森森的火焰,也将魏白衣震得踉跄後退。

这时喝声已变为长啸,长啸声中,燕南天身形已如大鹏般凌空盘旋飞舞,魏白衣抬头望去,心胆皆丧,他再想躲时,那里还能躲得了。他狂吼着喷出一口鲜血,仰天倒了下去!

燕南天一把拎起他衣襟,厉声道:魏无牙在那魏白衣睁开眼来,瞧了瞧燕南天,狞笑道;无牙门下士,可杀不可辱\ldots\ldots{}

这次他开口说话时,嘴襄已有一股腥臭的惨碧色浓液流出,等他说完工这要命的十个字,他便再也说不出一字来了。

燕南天放下了他,长叹道:想不到魏无牙门下,又多了这些狠毒疯狂的弟子\ldots\ldots{}

他忽然转向江玉郎,展颜笑道:但你\ldots\ldots 你可是武当门下。

江玉郎这时才定过神来,立刻躬身陪笑道:武当门下弟子江玉郎,参见燕老前辈。

燕南天扶起了他,大笑道:好,好,正派门下有你这样的後起之秀,他们就算再多收几个疯子,我也用不着发愁了。

江玉郎神情更恭谨,躬身道:但今日若非前辈怡巧赶来,弟子那里还有命在。

他说恰巧两字时,心不知有多愉快,燕南天若是早来一步,再多听到他两句话,他此刻只怕也要和魏白衣并排躺在地上了。

燕南天笑道:这实在巧得很,我若非约好个小朋友在此相见,也不会到这来的。

他拍着江玉郎肩头,大声笑道;他叫花无缺,你近年若常在江湖走动,就该听见过这个名字。

江玉郎神色不变,微笑道:晚辈下山并没有多久,对江湖侠踪,还生疏得很。

他一直留意着,直到此刻为止,铁萍姑竟仍无动静,这使他暗中松了一口气,接着又道:弟子方才来到时,那魏白衣要对一位慕容姑娘下手,这位姑娘此刻还躺在屋,前辈是否要去瞧瞧。

燕南天动容道:慕容姑娘?\ldots\ldots 莫非是慕容家的人他嘴说着话,人已掠进屋去。

慕容九自然还在棉被躺着。

屋子黑暗,但燕南天只瞧了两眼,便道:这孩子是被他点着哑穴了,这穴道虽非要穴,但因下手太重,而且已点了她至少有六七个时辰。

江玉郎失声道;已有六七个时辰了麽?如此说来,这位姑娘元气必然要亏损很大了。

燕南天沉声道:不错,她气血俱已受损甚巨,我此刻若骤然解开她穴道,她只怕就要等叁个月才能恢复过来。

江玉郎道:那\ldots\ldots 那怎麽办呢?

燕南天道:我行功为她活血时,最忌有人打扰,若是中断下来,她非但受损更大,我也难免要吃些亏的,但有你在旁守护着,我就用不着担心了。

江玉郎陪笑道:前辈只管放心,弟子虽无能,如此小事自信还不致有了差错。

燕南天大笑道:我若不放心你,远会冒这个险麽\ldots\ldots 紫髯老道的徒弟,我再不放心还能放心谁?

於是他盘膝坐在床上,双掌按上慕容九的後背,屋子虽然还是很暗,却也能看出他神情之凝重。

江玉郎站在他身後,嘴角不禁泛起一丝狞笑。

铁萍姑为什麽直到此刻还没有动静?只因她早已走了。江玉郎的甜言蜜语,虽然平息了她的愤怒,却令她自己感觉得更羞辱,她清醒过来时,只觉得自己好像被自己出卖了。

她恨自己,为什麽不杀了江玉郎,她恨自己为什麽下不了手,她知道方才既末下手,便永远再也不能下手。

她恨自己,为什麽如此轻易地就被人夺去了一生中最珍贵的东西,而自己却偏偏又好像爱上了这可恶的强盗。

铁萍姑一口气冲了出去。这客栈本就在小镇的边缘,掠出了这小镇,大地显得更黑暗,她瞧不见路途,也辨不出方向。

忽然间,黑暗中有两条人影走了过来,这两条人影几乎是同样大小同样高矮,就像是一个模子里铸出来的。

他们远远就停了下来,铁萍姑自然看不清他们的身形面貌,但在如此寂静的深夜,纵然是轻轻的语声,听来也十分清晰。

只听其中一人道:江小鱼,你真的不愿见他麽?

江小鱼这叁个字传到铁萍姑耳朵,她几乎忍不住要飞奔过去,投入他的怀抱。

但她知道自己现在没有资格再投入别人的怀抱了。她只有咬紧牙关,拚命忍住。

微风中果然传来了江小鱼的语声!他笑着道:你又说错了,我不是不愿见他,只不过是现在不愿见他。

花无缺道:你怎麽知道他一定会阻拦你!也许\ldots\ldots{}

小鱼儿道;当然他也许会让我去的,但我却不愿冒这个险,这件事我既已决定要做,就非做不可!

花无缺道:但你既已陪我来到这\ldots\ldots{}

小鱼儿道:燕大侠会在什麽地方等你

花无缺点了点手,道:就在前面小镇上的一家客栈裹,这小镇只有一家客栈,我绝不会找错地方的。

听到这,铁萍姑的心又跳了起来\ldots\ldots 江玉郎此刻还在那客栈,而他们也要到那客栈去。

她虽然恨江玉郎恨得要死,但一听到江玉郎有了危险,她就忘了一切,莫名其妙地对他关心起来。

只听小鱼儿缓缓道:我本来想要你陪我到龟山去的,但我知道你,既然约了别人,就决不会失信,是麽花无缺默然半晌,道;你我今日一别,就不知\ldots\ldots 他骤然顿住语声,也不愿再说下去。

小鱼儿重重一捏他肩膀,低声道:无论如何,你我总有再见的时侯\ldots\ldots 他话末说完,已大步走了出去。

花无缺想了想,也追了过去,道:现在时候还早,我也送你一程。

铁萍姑眼瞧着两条人影渐渐去远,她身子头抖着,咬着牙,突又跳起来,向那客栈飞奔回去。

只见窗子是开着的,窗里窗外,地上倒着叁个人的身,一条陌生的大汉正在为床上的一位姑娘推拿运气。

江玉郎眼睛里闪动着奇异的光,嘴角带着残酷的笑,正盯着那大汉的後背缓缓抬起了手!

铁萍姑冲到窗子前,也末弄清这里究竟是怎麽回事,便脱口道:江玉郎你\ldots\ldots{}

江玉郎这叁个字一出口,燕南天已霍然转过来,面上已变了颜色,他已迟了!

江玉郎的手掌,已重重击在他後心上?

燕南天狂吼一声,一口鲜血喷出!俪满了慕容九纤的身子,江玉郎也被这一声狂吼惊得踉跄後退,退到了墙角。

只见燕南天须发皆张,目尽裂,嘶声喝道:鼠辈,我救了你性命,你竟敢暗算於我?

江玉郎骇得腿都软了,身子贴着墙角往下滑,噗地跌在地上竟连爬都没有力气爬起来。

燕南天紧握着双拳,一步步走过去,喝道:你究竟是什麽人?为何要暗算我?说!

江玉郎那里还敢抬头望他,却偷偷去瞧窗外的铁萍姑,眼睛里再也没有夺人的神采,有的只是乞怜之意。

铁萍姑瞧见江玉郎竟以如此毒辣的手段暗算别人,又惊又怒,但她瞧见这双乞怜的目光,心却又软了。

她也不知怎地,迷迷糊糊就掠了进去,迷迷糊糊的击出了一掌又是一声狂吼,燕南天终於倒了下去!

江玉郎大喜跃起,笑喝道:你要知道我是谁麽好!我告诉你,我就是江南大侠的少爷江玉郎!什麽武当弟子,在我眼中简直不值一个屁?

燕南天一惊,一怔,终於缓缓阖起眼帘,纵声狂笑道:好!好!某家纵横天下,想不到今日竟死在你这贱奴的鼠子手上!

江玉郎狞笑道:你既出言不逊,少爷我就要令你在死前还要多受些罪了!

铁萍姑一直呆呆地望着自己的手,此刻突然用这只手拉住了江玉郎,道;他现在已经快死了,你何必再下毒手。

江玉郎笑着去摸她的脸,道:好,你叫我饶了他,我就饶了他\ldots\ldots{}

铁萍姑推开了他的手,道:花无缺就要来了!

江玉郎脸上笑容立刻全都不见,失声道:你已瞧见了他?

铁萍姑咬了咬嘴唇,道:还有江小鱼!

江玉郎再不说话,拉起铁萍姑就走,走出门,又回来,从床上扛起慕容九只要是对他有利的东西,他永远都不会放弃的。

他们居然很容易地就走出了这小镇,然後,江玉郎忽然问道:你说你见到了花无缺,你怎会认得他?

铁萍姑目光凝注着远方,默然许久,终於一字字缓缓道;只因我也是移花宫门下\ldots\ldots{}

小鱼儿和花无缺在路上慢慢走着,夜色很浓很静,他们甚至可以听到大地沉默呼吸。突然,远处传来了一声狂吼!

小鱼儿和花无缺骤然停下脚步。两人都没有说一个字,就向吼声传来处扑了回去。

只见那家客栈门口,有个人伏在门楣上呕吐这正是客栈的主人,他眼睛瞧着,耳朵听着一连串残酷的冷血的谋杀在他店里发生,但却完全没有法子,只有呕吐,似乎想吐出心里的难受与羞侮。

小鱼儿和花无缺还是没有说话,只交换了个眼色,便齐地扑入那客栈中。在那间有灯的屋子里看到了倒卧在血泊中的燕南天!

这就像一座山突然倒塌在他们面前,这就像大地突然在他们跟前裂开,他们立刻像石头般怔住!

燕南天挣扎着,睁开了眼睛。他逐渐僵硬的脸上,绽开一丝苦涩的笑,道:你\ldots\ldots 你们来了\ldots\ldots 很好\ldots\ldots 很好\ldots\ldots{}

花无缺终於过去,跪下,嘶声道;晚辈来迟了一步?

燕南天凄然笑道:我死前能见到你们,死也无憾了!

小鱼儿早已自血泊中抱起了他,大声道:你不会死的,没有人能杀得死你!

花无缺竟大叫起来,道:是谁下的毒手?是谁?

燕南天道:江玉郎!

花无缺长长吸了口气,一字字道:我一定要杀了他,为你复仇!

燕南天又笑了笑,转向小鱼儿。

小鱼儿也始终在凝注着他,此刻忽然大声道:用不着他去杀江玉郎,江玉郎是我的,无论前辈你是什麽人,我都会不顾一切,为前辈复仇的!

花无缺又怔住了,失声道;无论前辈是什麽人?\ldots\ldots 前辈不是燕大侠是谁?

燕南天却已大笑起来。他笑得虽然很痛苦,额上已笑出了黄豆般大的汗珠,但他仍笑个不停,他瞧着小鱼儿笑道:我自以为能瞒过了所有的人,谁知终於还是没有瞒过你。

花无缺又叫了起来,道:前辈难道竟不是燕南天燕大侠?

燕南天道;燕南天只是我平生第一好友\ldots\ldots{}

花无缺失声道:那麽前辈你\ldots\ldots?

燕南天道;我姓路。

小鱼儿道:路仲远?前辈莫非是南天大侠路仲远!

路仲远微笑道:你听过我的名字?

小鱼儿叹道;弟子五岁时便听过前辈的侠名了,那血手杜杀,虽然几乎死在前辈手中,但对前辈却始终佩服得很。

花无缺道:但\ldots\ldots 但路大侠为何要冒燕大侠之名呢?

路仲远道:只\ldots\ldots 只因燕\ldots\ldots{}

他呼吸已更急促,气力已更微弱,此刻连说话都显得痛苦得很。

小鱼儿道:此事我已猜出一二,不如由我替路大侠来说吧,若是我说的不错,前辈就点点头,若是我说错了,前辈不妨再自己说。

路仲远目中露出赞许之色,微笑点头道:好!

小鱼儿想了想道;燕大侠自恶人谷逃出後,神智虽已渐渐清醒,但武功一时还不龙完全恢复,是麽?

路仲远点点头。

小鱼儿道:他出谷之後,便找到了路大侠,是麽?

路仲远道:不错。

小鱼儿道:在一路上,他已发现江湖中有大乱将生,只恨自己无力阻止,於是他便想求路大侠助他一臂之力,是麽?

路仲远道:是。

小鱼儿道;他又生怕自己武功失传,是以一见路大侠,便将武功秘诀相赠。

路仲远不等他说完,已摇头挣扎着道:我十多年之前,曾受挫於魏无牙之手,那时我才发觉自己武功之不足,是以洗手归隐\ldots\ldots 他面上又露出痛苦之色。

小鱼儿立刻接下去道;是以这次燕大侠求前辈重出,前辈便生怕自己武功仍有不足,便要燕大侠将自己的武功秘诀相授,是麽?

路仲远含笑点了点头。

小鱼儿道:路大侠就为了这缘故,又不愿掠人之美,所以此番重出江湖,便借了燕大侠的名号。

他笑着接道:以路大侠的身分地位,自然不愿用燕南天的武功,来增加南天大侠的声名,不知弟子猜得可对麽?

路仲远含笑道:除此之外,还有一点。

小鱼儿又想了想,道:莫非是燕大侠算定自己一离开恶人谷後,恶人谷的恶人便要倾巢而出,他更怕这些人在江湖中为非作歹,知道这些人唯有燕南天叁个字才能震慑得住,所以便求前辈暂时冒充一番。

路仲远用尽一切力量,忍着痛苦问道:你果然是个聪明人,但\ldots\ldots 但我\ldots\ldots 我自信不但已学会了燕南天的武功,而且还请万春流将我的面容改变了许多,对於燕南天的音容笑貌,我自信也学得不差,我实在不懂怎麽会被你瞧破了?

前辈一见着我时,本该立刻提起万春流的,但前辈却完全忘记了这个人,是以那时我已开始怀疑了。而且前辈的神情,却仍和十馀年前传说中的燕大侠完全一样,这不但已超出人情之常,而且简直是不可能的事。他凄然接道:因为我深知燕大侠在那十几年所忍受的痛苦,在经过那种痛苦後,没有人还能保持不变的!

路仲远也不禁凄然道:不错,燕南天的\ldots\ldots 的确已改变了许多。他语声微弱得几乎连小鱼儿都听不清了。

他心里还有句话未曾说出他若是真的燕南天又怎认不出今日的江别鹤就是昔年的江琴!

但他既然答应了江别鹤,就只有保守这秘密。

小鱼儿长长叹了气,道:现在我只求前辈告诉我,燕大侠、燕伯父,现在究竟是在那里?路仲远没有回答,他已再次闭起眼睛。

\hypertarget{ux7b2cux4e03ux5341ux516dux7ae0-ux65e0ux7259ux95e8ux4e0b}{%
\chapter{第七十六章
无牙门下}\label{ux7b2cux4e03ux5341ux516dux7ae0-ux65e0ux7259ux95e8ux4e0b}}

现在,南天大侠路仲远已安葬了,在这清凉的小镇上,安葬的仪式虽然是不可避免地十分简单,但却也是十分隆重的。

小鱼儿和花无缺,沉重地肃立在路仲远的墓前,以一杯浊酒,吊祭这一代大侠的英魂。

暮色苍茫,大地萧索,秋,像是已极深了,直到夜幕垂下,星光升起,他们才黯然离去。

花无缺仰天唏嘘,叹道;盗寇末除,江湖末宁,路大侠实在死得太早了些\ldots\ldots 他甚至连燕大侠的下落,都末及说出,便含恨而殁。

小鱼儿苦笑道:也许是因为他不愿任何人去打扰燕大侠的安宁,也许是\ldots\ldots 燕大侠早已仙去,他不愿说出来,令我伤心。

花无缺黯然道;但愿我今生远能见到燕大侠一面,否则\ldots\ldots{}

小鱼儿忽然挺起胸来,大声道:你当然还能见着他,他当然不会死的,他还没有见到我扬名天下,他又怎能放心一死?

花无缺凝目瞧着他,展颜一笑,道:不错,燕大侠若是不愿死时,谁也无法要他死,甚至阎王老子也不能例外,我终有一日,能再见着他的。

小鱼儿仰天笑道;说得好,你说话的口气,简直和我差不多了,再过七十五天,就算我死了,你也可以替我活下去。

花无缺神情骤然又沉重了下来,他沉默许久,忽然道:现在你就要赶去龟山?

小鱼儿道;咱们一起去,我保证让你瞧一出又紧张又热闹的好戏。

花无缺垂下了头,道:可惜我不能陪你去了。

小鱼儿怔了半晌,大声道:咱们已只剩下七十五天了,你竟不愿陪着我?

花无缺望着远方的星光,缓缓道:我这件事若是做成,你我就不止可以做七十五天的朋友。

小鱼儿凝注了他半晌,大声道:你莫非想回移花宫?

花无缺叹道;我只是想去问清楚,她们为何定要我杀死你。

小鱼儿大笑道:你以为她们会告诉你?

花无缺默然良久,淡淡一笑,道:江小鱼,难道你已被命运屈服了麽?

小鱼儿一惊,大笑道:好,你去吧,无论如何,你我总还有一次见面的时侯,这已足够令人想起就开心了!

在这里,花开得正盛、菊花、牡丹、蔷薇、梅、桃、兰、曼陀罗、夜来香、郁金香\ldots\ldots{}

这些本不该在同一个地方开放更不该在同一个时候开放的花,此刻却全都在这里开放了。

这里本是深山,绝岭,本该弥漫着阴黯的云雾寒冷的风,但在这里,阳光如黄金般在花朵上,气候更温柔得永远像是春天。

无论任何人到了这里,都会被这一片花海迷醉,忘记了红尘中的困扰,更忘记了危险,忘记了一切。但这里都正是天下最神秘最危险的地方,这里就是移花宫!

但这时,却有个少女,正不顾一切要爬上来。

她穿的本是件雪白的衣裳,但现在却已染满了泥污和血迹,她容貌本是美丽的,但现在却已憔悴得可怕。

无论任何人都可看出,她是花了多大的代价,忍受了多大的痛苦才能到这神秘的地方来的。

到了这里,她整个人都已崩溃,她嘴唇已乾裂,肚子已发酸,已站不起来,她只有爬。

她爬,也要爬上来。自山下爬上来的少女,正是铁心兰?

她当然也知道移花宫的神秘与危险,但她不顾一切也要来,为的也只是要向移花宫主问一句话为什麽定要花无缺杀死江小鱼?

现在,她瞧见了这一片灿烂的花海,心里不觉长长松了口气,无论如何,所有的痛苦都已过去了!她晕了过去,她以为自己永远再也不会醒了?

醒来时,她发觉自己是安静地躺在一张柔软而带着香气的床上,阳光已不见,灯光却似比阳光更辉煌。她闭起眼睛,等她再张开时,她就瞧见了花无缺。

花无缺也正在温柔地望着她,在这辉煌的光线里,他看来更如神话中的王子,那麽英俊那麽脱,那麽高不可攀。

铁心兰呻吟一声,道:花无缺,你真的是花无缺麽花无缺温柔地笑了笑,柔声道;是我,我就站在你身畔,你用不着害怕了!

铁心兰突又挣扎着要爬起来,嘶声道:求求你,带我去见移花宫的宫主好麽了我不顾一切来到这里,为的只是想求她见我一面。

花无缺苦笑道:我回来,也是想求见她老人家的,只可惜,她们都早已不在宫里了。

铁心兰倒在床上,失声道:她们都出去了?

花无缺道:两位宫主全都离宫而出,这本是很少有的事。

铁心所凄然道:我的运气为什麽总是这麽坏,我\ldots\ldots 我\ldots\ldots 她语声哽咽,用丝被蒙住了头,再也说不下去。

花无缺呆了半晌缓缓道:我想\ldots\ldots 我是知道你来意的,我也正是为了同一件事,想回来问她老人家,想不到她们离宫都已有许久了。

铁心兰在被里轻轻啜泣,忽又问道:这些日子里,你是否已见过他?

用不着说出名字,别人也知道她说的他是谁。

花无缺柔声笑道:他现在很好,你用不着为他担心。

他虽然尽力想装得平淡,但笑容中仍不免有些苦涩之意。

铁心兰终於自被里伸出了头,呐呐道:你可知道,他现在在那里?

花无缺努力想笑得偷快些,柔声道:我知道,只要你身子康复,我就可以带你去找他。

铁心兰凝注着他,眼泪又不觉流下面颊,头声道:你\ldots\ldots 你为什麽永远对我这麽好,你\ldots\ldots 你\ldots\ldots{}

忽然间,屋外传来了一阵奇异的声音,这声音既不尖锐,也不凄厉,却令人听得忍不住要为之毛骨悚然。

这声音骤听如同铁锯锯木,再听又如蚕食桑叶,仔细一听,又如刀剑相磨,简直令任何人听得都要牙脚软。接着,就听得少女们的鹫呼声。

花无缺也微微变了颜色,道:我出去瞧瞧。

他深知移花宫门下,纵然大多是少女,却绝没有一个会大鹫小怪的,能令她们鹫呼出声来,事情绝不简单。

铁心兰摸了摸身上已穿得甚是整齐,也跳下了床,道我跟你一起去。

两人赶出去,只见少女们都躲在宫檐下,一个个竟都吓得花容失色,有的甚至连身子都发起抖来。再见那一片花海中,正有无数个东西在窜动。

铁心兰夫声道老鼠!那里来的这麽多老鼠!

果然是老鼠!

成千成百只简直有猫那麽大的老鼠,正在花丛中往来流窜,啃着花枝,吞食着珍贵的花朵。

移花宫门下虽然都有绝技在身,怎奈全都是女子,老虎她们是不怕的,但见了这许多老鼠,腿都不禁软了。

花无缺一步窜了出去,变色喝道来的可是魏无牙门下?

四下寂静无声,也瞧不见人影,这一片也不知费了多少心血才培养成的花海,转眼间已是狼藉不堪,花无缺既惊且怒,但面对着这麽多老鼠,他也没法子了。

在移花宫中,他既不能用火烧,也不能用水淹,若是要去赶,这些老鼠根本就不怕人。他再也想不到名震天下的移花宫,竟拿这一群动物中最无用、最卑鄙的老鼠无法可施。

这时黑暗中才传来一阵狂笑声。

一个尖锐的语声狂笑着道:只可惜移花宫主不在家,否则让她们亲眼瞧见这些宝贝鲜花进了咱们老鼠的肚子,她们只怕连血都要吐出来了。

花无缺此刻神情反而镇定了下来,既不再惊慌,也不动怒,就好像连一只老鼠都没有瞧见似的。

他脸上带着微笑,缓缓道;无牙门下的高足既已来了,何不出来相见?

只听黑暗中那人大笑道;这小子倒沉得住气,你可知道他是谁麽?

花无缺还是身形不动,淡淡道:在下花无缺,正也是移花宫门下!

那人道:花无缺,我好像听见过这名字。

话声末了,那黑暗的角落里,突然闭起了一片阴森森的碧光,碧光闪动,渐渐现出了两条人影。

这两人俱是枯瘦颀长,宛如竹竿,两人一个穿着青衣,一个穿着黄袍,脸上却都是碧油油的像是戴了层面具。但不知怎地,却令人一见就要起鸡皮疙瘩,一见就要怍呕。

那青衣人碧森森的目光上上下下瞧了花无缺几眼,阴阴笑道:阁下居然知道我兄弟是无牙门下,见识已不能算不广,所以你这麽年轻就要死,我实在不免要替你可惜。

黄衣人笑道:他叫魏青衣,我叫魏黄衣,我们本不想杀你,怎奈家师此番复出,第一个要毁的就是移花宫,我们也没法子。

少女们听到这说不出有多丑恶的笑声,瞧见被老鼠围在中间的两个人,竟无一人敢出手的。

只见魏青衣肩头微微一动,花无缺身形立刻冲天飞起,接着,立刻便有一丝碧光自魏青衣掌中飞出!

但这时花无缺身形早已了过去,碧光过处,一个少女已惨呼着倒地,花无缺却不回头,双掌已击向魏青衣头顶!

魏青衣再也想不到他来得竟如此快,脚步倒错,平平一掌撩了上去,魏黄衣亦自斜斜一掌击出。

谁知花无缺这凌空一掌,竟也是虚势,掌到中途,他手肘突然缩了回来,不去接魏青衣的一掌,反而空空划了个圈子。

魏青衣只觉掌势突然脱力,就在这旧力落空新力末生的刹那间,另一股奇异的力量已将他掌势引得往外一偏,也不知怎的,击出这一掌,竟迎上了魏黄衣料斜击过来的一掌?

拍的一声,双掌相接,接着又是喀嚓一声,魏青衣这已脱了力的一条手臂,竟生生被魏黄衣震断了!

花无缺竟以出其不意的速度,冒险的攻势,妙绝天下的移花接玉神功一着便占了上风!

一掌接过,魏青衣、魏黄衣两人俱是大失色。

魏黄衣虽末受伤,但见到自己竟伤了同伴,惊慌更甚,一脚踩在老鼠堆上,鼠群一慌,四下奔出。

只见花无缺一招得手,竟又含笑站在那里,并末跟着抢攻,只因他方才一招便已试出这两人的功力,实是非同小可,他自知侥幸得手,绝不贪功急进,他还要等着这两人再次上钩。

这时鼠辈已散布开来,再次往四方流窜。

铁心兰突然咬了咬牙,自窗框上拆下段木头,咬着牙奔出去,举手一棍,将一只老鼠打得血肉横飞。

本来往四下流窜的老鼠,此刻竟都向铁心兰围了过来,铁心兰心已发寒手已发软,但仍咬着牙不退缩。

躲在宫檐下的少女们,终於有一个奔出来只要有一个出来,别的人也就会跟着出来了。

她们只要打死一只老鼠,胆子也就壮了。

十几个又娇柔又美丽的少女,流着汗,喘着气,忘记了一切,全心全意地在和一群老鼠拚命!鼠辈终於败了,大多被打死少数逃得不见踪影。

少女们瞧着地上狼藉的鼠又瞧自己手,她们几乎不相信这些老鼠真是她们打死的。

这简直就好像做了一场噩梦!

然後,她们有的抛下棍子开呕吐有的疯狂般大叫大笑起来,也有的拥抱起别人,放声痛哭。

这些情况,都是移花宫不会发生的但现在却发生了,只因她们经过这一番恶战後,已不知不觉地放松了自己。

只有铁心兰,她停下了手,立刻就去找花无缺!

花无缺竟已不见了?

魏青衣魏黄衣也已不见了!

铁心兰踉跄地四下搜寻着,心里又是惊慌,又是害怕,她方才专心对付老鼠,竟忘了瞧一瞧这边的战况!

花无缺的武功虽高,但这两人既敢闯到移花宫来,又岂是弱者,花无缺以一敌二,未必真是他们的对手。

铁心兰几乎要急疯了。忽然间,她发觉残花丛中,似躺着一个人的身。

只见他右臂已肘而断,胸前有个血淋淋的大洞,一张阴森碧绿的脸上,也已被人打肿了。

这模样也不知有多麽狰狞可怕,铁心兰那里还敢再看!她赶紧移开目光,不觉瞧见了魏青衣的一只左手。

只见他这只鬼爪般的手掌食中两指上,竟带着两粒血淋淋的眼珠子!显然是被他自眼眶中生生挖出来的!

她眼泪不觉已夺眶而出?

忽然间,她听得有一阵沉重而急促的像是负伤野兽般的呼吸声,自一片山崖下传了上来。

她立刻扑了过去!只见一个人满面流血,双臂箕张,喘息着蹲在一株树下,一双眼睛已变成了两个血洞!

但这人也不是花无缺,而是魏黄衣土他显然是在移花接玉的奇妙功夫下,被他自己的同伴挖去了眼珠!

\hypertarget{ux7b2cux4e03ux5341ux4e03ux7ae0-ux840dux6c34ux76f8ux9022}{%
\chapter{第七十七章
萍水相逢}\label{ux7b2cux4e03ux5341ux4e03ux7ae0-ux840dux6c34ux76f8ux9022}}

铁心兰见那满面流血的人不是花无缺,虽然松了气,但瞧见这比豺狼更凶悍的人,瞧见这残酷而诡秘的情况,身子仍不禁发起抖来。

幸好她立刻又瞧见了花无缺!花无缺此刻正远远站在魏黄衣对面的另一株树下。

他全身每一根神经每一根肌肉,都在紧张着,一双眼睛,更瞬也不瞬地瞪着魏黄衣的一双手。

两个人虽然全都站着不动,但这情况却比什麽都要紧张,就连远在山崖上的铁心兰,也已紧张得透不过气来。突听魏黄衣一声狂吼,向花无缺了过去!他虽然已经没有眼睛可看,但还有耳朵可听!

这一扑不但势道之威猛无可此拟,而且方向准确已极?

但就在这刹那间,花无缺左右双手,各自弹出一粒石子,他自己却闪电般从魏黄衣胁下窜了出去!

只听喀嚓一声,花无缺身後的一株此面盆还粗的大树,已被魏黄衣的身子生生撞断!他竟还末倒下,一个虎跳,又转过身来。

他的头向左右旋转,嘶声狞笑道:花无缺,我知道你在那里,你逃不了的,今日就是你我两人谁也休想活着走,我要和你一起死在这里!

他其实根本不知道花无缺在那里,花无缺又到了他对面,他的头却不自觉地左右转动。

铁心兰瞧着他这样子,觉得既可怕又可怜,若不是花无缺此刻犹在险境,她实在不忍心再瞧下去。花无缺也显然大是不忍,竟忍不住叹了气,黯然道:我实在不忍和你动手,我劝你还是\ldots\ldots 魏黄衣突然跳起来,狂吼道:我用不着你可怜我,我\ldots\ldots 我就算找不到你,也用不着你\ldots\ldots 他声音已说不下去,却开始拚命去捶打自己的胸螳,嘴里轻哼着,虽不是哭,却比哭更凄惨十倍。

铁心兰瞧得目中竟忍不住流下泪来,魏黄衣就算是世上最恶毒残暴的人,她也不忍再看见他受这样的罪。她忍不住叹道:你快走吧,我知道花\ldots\ldots 花公子绝不会阻拦你。

魏黄衣嘶声笑道:走\ldots\ldots 你难道不知道无牙门下,可杀不可辱\ldots\ldots{}

狂笑声中,他忽然用尽所有的潜力,飞扑而起,向低崖上的铁心兰扑了过去,嘶声狞笑道;你不该多话的,我虽杀不了花无缺,却能杀死你?

铁心兰已被他这疯狂的模样骇呆了,竟不知闪避。

魏黄衣话声末了,人已揍上低崖,两条铁一般的手臂,已挟住了铁心兰,疯狂般大笑道:我要死,至少也得有一个人陪着我!

铁心兰只觉全身都快要断了,那张流满了鲜血的脸,那两个血淋淋的黑洞,就在她面前,她骇得连惊呼声都发不出来!

只听蹼的一声,魏黄衣狂笑声突然断绝,两条手臂也突然松了,倒退半步,仰天跌下了低崖。

花无缺已在她面前,铁心兰再也忍不住扑入花无缺怀里,放声痛哭起来。

花无缺抚着她的头发,黯然道:我本不忍杀他的,我\ldots\ldots{}

铁心兰痛哭道:我错了,我本不该多嘴的,否则你也不必勉强自己来杀一个没有眼睛的人,我\ldots\ldots 我为什麽总是会把事情弄得一团糟。

花无缺柔声道:你认为你错了麽?你只不过是心太软了,错,并不在你,你本想将每件事都做好的,你已尽了你的力量了。

铁心兰啜泣着道:你总是对我这麽好,而我\ldots\ldots 我\ldots\ldots{}

花无缺不敢再看她,转过眼,俯首凝视低崖下魏黄衣的身,长长叹了口气喃喃道:无牙门下,好厉害的无牙门下,江小鱼,你对付得了麽?

他轻轻一句话,就将话题转到小鱼儿身上。

铁心兰果然身子一震,她心里对花无缺的感激与情意,果然立刻变怍了对小鱼儿的关心。

花无缺叹道:无牙门下的弟子,已如此厉害,何况魏无牙自己?江小鱼呀江小鱼,我实在难免要替你担心。

铁心兰再也忍不住失声问道:江小鱼,她难道已经\ldots\ldots{}

花无缺这才回过头,沉声道:他此刻只怕已到了龟山,只怕已快见着魏无牙了!

第二天,花无缺就带着铁心兰直奔龟山。

他有意无意间,始终和铁心兰保持着一段距离,行路时跟在铁心兰身後,吃饭时故意找件事做,等铁心兰快吃完时再上桌,晚间投宿时,他也不睡在铁心兰的邻室,却远远再去找个房间。

他们的心情都像是很沉重,终日也难得见到笑容。

他们走了两天,这一日晚间投宿,花无缺很早就回房睡了,但他却又怎会真的睡得着。

花无缺凝注着飘摇的烛光,心里想到小鱼儿,想到铁心兰,想到移花宫主,又想到那神秘的铜先生。

每个人都在他心里结成个解不开的死结,他实在不知自己该如何处理。

只听门外忽然响起了轻轻的敲门声。

花无缺只当是店伙来添加水了,随道;门没有关,进来吧。

他再也想不到推门进来的竟是铁心兰。

灯光下,只见她穿着件雪白的衣服,乌黑的头发,长长披落,她的眼睛似乎微微有些肿,眼波看来也就更朦胧。

但她低垂着头,朦胧的眼波,始终也末抬起。花无缺的心像是忽然被抽紧了。

铁心兰垂着头道:我\ldots\ldots 我睡不着,心里有几句话,想来对你说。

请坐。他实在不知道该说什麽话,只有说请坐这两个字,却不知道这两个字说得又是多麽冷淡多麽生疏。

她迟疑了许久,像是鼓起了最大的勇气,才幽幽道:我知道这些日子来,你故意很冷淡我、很疏远我。

花无缺默然半晌,沉重地坐下来,长叹道:你要我说真话?

迟早总要说的话,为什麽不现在说?

花无缺自烛台上剥下了一段烛泪,放在手指里重捏着,就好像在捏他自己的心一样。

你知道,人与人之间在一起接近得久了,就难免要生出感情,尤其是在困苦与患难中。

他一个字一个字地说着,说得是那麽艰苦。

铁心兰出神地瞧着他手心里的烛泪,却好像他在捏着的是她的心。

我不是怕你对不起他,而是怕我自己,我\ldots\ldots 他咬了咬牙,接着道:我不忍把你的情感拖入矛盾里,假如我和你接近得太多,不但我痛苦,你也会痛苦。

铁心兰的头又垂了下去。目中已流下泪来。

她忽然抬起头,含泪凝注着花无缺,大声道:但我\ldots\ldots 我是个孤苦的女孩子,我只想把你当做我真的兄长,我希望你能相信我\ldots\ldots{}

花无缺没有说话。

铁心兰道:我此刻来只是要告诉你,你不必疏远我,也不必防我,只要我们心里光明坦荡,就不怕对不起别人,也不必怕别人的想法。

花无缺终於展颜一笑,道:我现在才知道你很有勇气,这勇气,平常虽看不出,但到了必要时,你却此任何人都勇敢得多?

铁心兰长长吐了口气,也展颜笑道:我把这些话说出来,心里真的愉快多了,我真想喝杯酒庆祝庆祝。

花无缺霍然站起,笑道:我心里也痛快多了,我也正想喝杯酒庆祝庆祝。

两人将心里憋着的话都说了出来,就好像突然解开了一重枷锁。只可惜客栈中已没有酒菜,於是两人走上街头。

长街上的灯光已疏,店也都上起了门板,只有转角处一个面摊子的炉火尚未熄,一阵阵牛肉汤的香气,在晚风中显得分外浓冽。

铁心兰笑道:坐在这种小面摊上喝酒,倒也别有风味,却不知道你嫌不嫌脏?

花无缺微笑道:你真的把我看成只肯坐在高楼上喝酒的那种人麽?

铁心兰嫣然一笑;还末走到面摊子前,已大声道:给我们切半斤牛肉,来一斤酒。

面摊旁摆着两张东倒西歪的木桌子,此刻都是空着的,只有一个穿着黑衣服的瘦子,正蹲在面摊前那张长板凳上喝酒。

朦朦胧胧的热气与灯光下,这黑衣人瘦削的脸,看来简直比那小木橱里的卤菜还要乾瘪。但是他的一双眼睛,却比天上的星光更亮。

他箕踞在板凳上,一面啃着鸭头,一面喝酒,神思却已似飞到远方。

一个落拓的人,坐在简陋的面摊上喝着酒,追悼着逝去的青春与欢乐,这本是极普通的情况,铁心兰和花无缺也没有留意他。

也们天南地北的聊着,但後来他们忽然发现,无论他们聊什麽都好像总和小鱼儿有些关系。

花无缺笑道:如此良宵,有酒有肉,这本已足够了,但我却总还觉得缺少了什麽,现在我才知道缺少的是什麽了。

铁心兰垂下了头,道:你是说\ldots\ldots 缺少一个人?

花无缺叹道:没有他在一起,你我岂能尽欢

铁心兰默然半晌,抬头道:你想,我们叁个人会不会有在一起喝酒的时候花无缺道:为什麽不会有?

他一笑举杯,道:来,你我且为江小鱼乾一杯。

江小鱼,这叁个字说出来,那黑衣人突然抛下了鸭头,放下了酒杯,目光闪电般向他们扫了过去。

铁心兰一饮而尽,脸更红了。她脸上虽有笑容,目中却似含有泪光,悠悠道:我若也是个男人,那有多好\ldots\ldots{}

他抬起头,忽然发觉一个乾枯瘦削的黑衣人,已走到面前,一双发亮的眼睛,不停地在他们脸上打转。

花无缺和铁心兰都怔住了。

这黑衣人上上下下,打量着他们几眼,忽然向花无缺道:你就是花无缺?

花无缺更惊奇道:正是,阁下\ldots\ldots{}

黑衣人根本不听他说话,已转向铁心兰,道:你就是铁心兰!

铁心兰点了点头,已吃惊得说不出话来。

黑衣人眼睛瞪得更大,道:你们方才可是为江小鱼乾了一杯?

她知道小鱼儿仇人不少,她以为这黑衣人也是来找麻烦的,谁知这黑衣人竟拉过张凳子,坐了下来,道:好!你们为江小鱼乾一杯,我最少要敬你们叁杯!

他竟举起那酒,为他们各倒了杯酒。铁心兰和花无缺望着面前的酒,也不知是喝好,还是不喝好。

黑衣人自己先仰脖子乾了一杯瞪眼道:喝呀!你们难道怕酒中有毒不成?

花无缺还在怀疑着,铁心兰已大声道:对不起,我们没有和陌生人喝酒的习惯,你若要敬我们的酒,至少总得先说出你是谁?

黑衣人道:你也莫管我是谁,只要知道我是江小鱼的朋友就好了。

铁心兰瞪眼瞧了他半晌,道:好,你既是江小鱼的朋友,我就喝了这一杯。

黑衣人转向花无缺,道:你呢?

花无缺微微一笑,道:在下喝叁杯。

黑衣人大笑道:好,你很好,很够朋友。

他和花无缺对饮了叁杯,又道:你在这样的星光下,和这样的美女坐在一起喝酒,心里居然远没有忘记江小鱼,好\ldots\ldots 好,我再敬你叁杯!

那酒已差不多快空了,这黑衣人眼睛虽然清亮,但神情间却似已有些醉意,再不管别人喝不喝,也不和别人说话,只是自己一杯又一杯地往肚子里灌,不时仰望着天色,似乎在等人。

他等的是谁?

铁心兰凝目瞧着他,忍不住又道:你真的和江小鱼是朋友?

黑衣人瞪眼道:江小鱼又不是什麽了不起的大人物,我为何要冒认是他朋友?

他语声顿了顿,忽然又道:你们若是瞧见他时,不妨代我向他问好。

铁心茁试探着又道:我们见着小鱼儿时,说你是谁呢?

黑衣人沉吟道:你就说是他大哥好了。

铁心兰忽然长身而起,厉声道:你究竟是什麽人黑衣人道:我不是刚告诉你\ldots\ldots{}

铁心兰冷笑道;放屁,小鱼儿绝不会认别人是他大哥的,你休想骗我。

黑衣人忽然大笑起来,道:好,好,你们当真不愧是小鱼儿的知己不错,我一心想要他叫我一声大哥,但他却总是要叫我兄弟。

铁心兰忍不住又道:喂,我看你像是有什麽心事?是麽?

黑衣人又瞪起眼睛,道:心事?我会有什麽心事?

铁心兰道:你若真将我们当成江小鱼的朋友,为何不将心事说出来,也许\ldots\ldots 也许我们能帮你的忙。

黑衣人忽然仰天狂笑,道;帮忙!我难道会要别人帮忙!他高亢的笑声中,竟也充满了悲痛与愤怒。

铁心兰还想再问,却被花无缺以眼色止住了。远处传来更鼓声,已是二更叁点。

黑衣人突又顿住笑声,凝注着花无缺与铁心兰,道:好,你就每人敬我叁杯酒吧,这就算帮了我的忙了。

六杯酒下肚,黑衣人仰天笑道;我本当今夜只有一个人触自度过,谁知竟遇着了你们,陪我痛饮了一夜,这也算是我人生一大快事了\ldots\ldots{}

黑衣人霍然站起,像是想说什麽,却连一个字也没有说,扭过头就走。

他走到面摊子前,把怀里的东西全都掏了出来,竟有好几锭金子,和十几粒珍珠,他随手抛在面摊上,道:这是给你的酒钱,全给你。

面摊老板骇得怔住了,等他想说谢时,那黑衣人却已走得很远,昏黄的灯光,将他的影子长长拖在地上。

一他看来是如此寂寞如此萧索。

花无缺缓缓道:在他临死前的晚上,他本都以为要独自度过的,他竟找不到一个朋友来陪他度过最後的一天。

铁心兰夫声道:临死的晚上最後一天

花无缺叹道:你还瞧不出麽\ldots\ldots{}

他忽然顿住语声,拉着铁心兰掠了出去。

那黑衣人脚步踉跄,本像是走得极慢,但,银光一闪後,他就忽然不见了,竟像是忽然就被夜色吞没。

掠过几重屋脊,花无缺就将铁心兰放下,道:我去追他,你在这里等着!

铁心兰只有等着。但她的一颗心却总是静不下来。

这黑衣人是谁?他为何要死他和小鱼儿\ldots\ldots 人影一闪,花无缺已到了她面前。

花无缺道:你踉我来!

两人又飞掠过几重屋脊,铁心兰又忍不住问道:你怎知他已快死了?

花无缺叹道:他随时在留意着时刻,显见他今天晚上一定有件要紧的事要去做。

铁心兰道:这我也发觉了。

花无缺缓缓道:但他既是江小鱼的朋友,我们又怎能坐视他去送死?

铁心兰咬了咬嘴唇,道:他轻功已是顶尖好手,就算打不过别人,也该能跑得了的,但却完全不抱能逃走的希望,他那对头,岂非可怕得很。

花无缺沉声道:所以你要分外小心,有我在,你千万不要随意出手。

铁心兰忽然发现前面不远的山脚下,有座规模不小的庙宇,气派看来竟似富豪人家的庄院。

此时此刻,这庙宇的後进,居然还亮着灯火。

铁心兰道:他难道就是到这道观里去了。

花无缺截口道:他进去时,行动甚为小心,以他的轻功,别人暂时必定难以觉察,所以我就先赶回去找你。

铁心兰放眼望去,只见这道观里灯光虽末熄,但却绝没有丝毫人声,更看不出有丝毫凶险之兆。

花无缺皱眉道你在这里等着,我进去看看。

铁心兰却拉住了他,沉声道:我看这其中必定还有些蹊跷,说不定这也是他和别人串通好的陷阱,故意要将我们诱到这里来的!

花无缺淡淡一笑,道:此人若是真的要诱我入伏,我更要瞧个究竟了。

他轻轻甩脱铁心兰的手,人影一闪,已没入黑暗中。

铁心兰望着他身影消失,苦笑道:想不到这人的脾气有时竟也和小鱼儿一模一样。

花无缺从黑暗的檐下绕到後院,又发觉这灯火明亮的後院,已不再是庙宇,无论房屋的格式和屋里的陈设,都已和普通的大户人家没什麽两样。

最奇怪的是,整个後院里都听不见人声,也瞧不见人影,但在那间精致的花厅里,豪华的地毡上,却横卧着一只吊睛白额猛虎。

这花席看来本还不只这麽大,中间却以一道长可及地的黄幔,将後面一半隔开,猛虎便横卧在黄幔前。

这花临为何要用黄幔隔成两半黄幔後又隐藏着什麽秘密?

他自黑暗中悄悄掩过去,这个并非完全因为他胆子特别大,而是因为他深信自己的轻功。

他行动间当然绝不会发出丝毫声息。谁知就在这时,那彷佛睡着的猛虎,竟突然跃起,一声虎吼,响彻天地,满院木叶萧萧而落。

\hypertarget{ux7b2cux4e03ux5341ux516bux7ae0-ux51a4ux5bb6ux8defux7a84}{%
\chapter{第七十八章
冤家路窄}\label{ux7b2cux4e03ux5341ux516bux7ae0-ux51a4ux5bb6ux8defux7a84}}

花无缺的轻功纵然妙绝天下,怎奈这老虎既不必用眼睛看,也不必用耳朵听,它只要用鼻子一嗅,无论什麽人走进这後院,都休想瞒得过它那黑衣人既然已入了後院,此刻只怕已凶多吉少了。

花无缺一惊之後,又不禁叹息。

只见满厅灯火摇动,那猛虎已待扑起,虎威之猛,当真是百兽难及,就连花无缺心里也不禁暗暗吃惊。

但这时黄幔後却传出了一阵柔媚的语声,轻轻道:小猫,坐下来,莫要学看家狗的恶模样吓坏了客人。

这猛虎竟真的乖乖走了过去,坐了下来,就像是忽然娈成了一只小猫。

花无缺不觉已瞧得呆住了,却见黄幔後又伸出一只晶莹如玉柔若无骨的纤纤玉手来,轻抚着虎背。

只听那柔媚入骨的语声带着笑意:足下既然来了,为何不进来坐坐呢花无缺暗忖道:那黑衣人方才所经历的,是否正也和我此刻一样他是否走进去了他进去之後,又遭遇到什麽事?

他断定那黑衣人既抱着必死之心前来,就绝对不会退缩的,这花厅纵然真是虎穴,他也会闯进去!

想到这里,花无缺也不再迟疑,大步走了过去!

他正面带着微笑,一步步走进去,就好像一个彬彬有礼的客人,来拜访他的世交似的,黄幔後传出了银铃般的笑声,道:好一位翩翩浊世佳公子,不敢请教高姓大名。

花无缺抱拳一揖,道:在下花无缺,不知姑娘芳名?

黄幔後嘻嘻笑道;徐娘已嫁,怎敢再自居姑娘\ldots\ldots 贱妾姓白。

花无缺道:原来是白夫人。

白夫人道:不敢,花公子请坐。

花无缺竟真的坐了下来,道:多谢夫人。

这也是花无缺改不了的脾气,只要别人客客气气地对他,他就算明知道这人要宰了他,也还是会对这人客客气气的。

只听白夫人又笑道:公子远来,贱妾竟不能出来一尽地主之谊,盼公子恕罪。

花无缺道:能与夫人隔帘而谈,在下已觉不胜荣宠。

白夫人忽然大笑道;我已经算很客气的了,不想你竟比我更客气,咱们这样客气下去,我既不好意思间你是为何而来的,你也不好意思说,这些客气话,不如还是免了吧。

花无缺微微一笑道:先礼而後兵,正是君子相争之道,以在下之见,还是客气些的好。

白夫人道:你我无冤无仇,你甚至连我的面都末见到,你怎知我要和你先礼後兵呢?我并没有和你兵的意思呀。

花无缺道:陌生之人,寅夜登堂,夫人纵以干戈相待,固亦理所当然也。

白夫人娇笑道:我虽然不知道你的来意,但看你文质彬彬,一表人才,又是满腹诗书,出口成章,怎麽看也不像个坏人的样子,你若像刚才进来的人那副样子,我纵然不会难为你,但别人却放不过你了。花无缺长长吐了气,沉声道;多蒙夫人青睐,怎奈在下却偏偏是为了方才那人而来的。

白夫人道;哎约,你难道和那个鬼鬼祟祟的小黑鬼是朋友?

花无缺道:夫人若能将他的下落赐知,在下感激不尽。

白夫人道:我就算将他的下落告诉了你,你有这本事救他出去麽花无缺道:在下在夫人面前,倒也不敢妄自菲薄。

白夫人大笑道;好,好个不敢妄自菲薄,既是如此,你就先露一手给我瞧瞧吧,我看你是不是真有能救他出来的本事。

花无缺微微一笑,道:如此在下就献丑了。

他坐着动也没有动,但整个人却突然飞了起来,那张沉重的紫檀大椅,也好像黏在身上了。

白夫人大笑道;好,有你这样的本事,难怪你说不敢妄自菲薄了,只恐怕\ldots\ldots{}

花无缺娥眉道:只恐怕什麽?

白夫人又接着道:我们这里有两个客人,却瞧着那小黑鬼不顺眼了,他们也不知道为了什麽,说着说着就打了起来!唉,你那朋友样子虽然凶,却又偏偏不是我那两个朋友的对手。

花无缺失声道:他莫非已遭了别人毒手?

白夫人道;你那朋友好像是被我的朋友带走了,但带到那里去了,我可也不知道。

花无缺不觉呆住了,一时间竟不知该怎麽做才好。

他也摸不清这位白夫人是何等身分,更摸不清她说的话是真是假,何况,他就算明知她说的是假话,也是无可奈何。

他走也不是,不走也不是,正在发怔。

谁知白夫人却又忽然噗哧一笑,道:但你也莫要发愁,你若真的要找他,我是可以带你去的。

花无缺喜道:多谢夫人。

白夫人竟又叹了口气,道:只不过我被人关在这里,动也不能动,又怎麽能带你去呢?

花无缺瞧着那在致手抚摸上,驯如家猫的猛虎,呐呐道:夫人既是此间的主人,此虎又是夫人所养,夫人却是被谁关在这里的,在下实在百思不得其解。

白夫人叹了口气道:这事说来话长,你先掀起这帘子,我再告诉你。

花无缺迟疑着道:莫非是个陷阱?

白夫人道:你还说自己本事大,竟连这帘子都不敢掀麽?花无缺霍然长身而起,一把将那帘子掀了开来。帘子一掀,他更吃得说不出话来。

这花厅前面一半,陈设精雅,堂皇富丽,但被黄幔隔开的後面一半,却什麽陈设也没有,满地都是稻草,只有在角落里放着只水槽这那里像是人住的地方,简直像是猪窝、马厩。

这情况已经够令人吃惊的了,更令人吃惊的是,这华衣美妇的脖子上,还系着根铁,铁的另一端,深深钉人墙里。

花无缺也像是被钉子钉在地上了,再也动弹不得。

白夫人瞧着他凄然一笑道;你现在总该明白我为什麽不能带你去了吧。

花无缺暗中叹了口气,道:这?\ldots\ldots 这究竟是谁做的事,是谁\ldots\ldots{}

白夫人垂下了头,一字字道:我的丈夫!

花无缺几乎跳了起来失声道:你的丈夫

白夫人凄然道:不错,我的丈夫是天下最会吃醋最不讲理的男人,他总是认为只要他一走,我就会和别的男人勾叁搭四。

花无缺呆望着她,那里还说得出话来。

白夫人道;你看我的衣服打扮还不错,又觉得奇怪,是麽?

她长叹着接道:若有别人瞧了我一眼,他就要将那人杀死,你现在已瞧过我了,你就算不救我出去,他也要找你算帐的。

花无缺苦笑道:在下平生最恨的,就是欺负妇人女子的人,莫说在下还有求於夫人,就算没有此事,在下无论如何也要将夫人救出去的。

铁心兰伏在黑暗中,等了许久。

忽然间,她听到一声惊天动地的虎吼,但虎吼过後,四下又转於静寂,什麽动静都没有了。

这没有动静却此什麽动静都令铁心兰担心。

她又等了半晌,越等越着急,到後来实在忍不住了,终於自藏身处跃出,她无论如何也想去瞧个究竟。

铁心兰枞身跃上了墙头。她刚跃上墙头,突然有灯光一闪,那是特制的孔明灯,一道光柱闪电般从她脸上掠过。

接着,黑黝黝的大殿里,就有一人缓缓笑道:我当是谁呢,原来是铁心兰姑娘。

铁心兰这一惊,几乎在墙头上冻结住了,嘶声道;你是谁?

姑娘走进来瞧瞧,就会认得我是谁的。

铁心兰又惊又疑,那里敢贸然走进这阴森黝黯的大殿。

那人阴恻恻一笑,接着又道:姑娘既已来到这里,还是进来瞧瞧的好,否则,连姑娘的那两个朋友都走不了,凭姑娘的本事,难道能走得了麽?

铁心兰全身鄱头抖了起来难道连花无缺都已落人别人的陷阱,遭了毒手?

黑暗中那人缓缓道:石阶旁的柱子下,有盏灯,还有个火摺子,姑娘最好点着灯才进来,别人都说我在灯光下看来,是个非常英俊的男人。

铁心兰又在犹疑:这又是什麽诡计?

但无论如何,灯光通常都能带给人一些勇气,黑暗中的危险总比较大於是她寻着灯,燃起。铁心茁紧紧握着灯,一步步走进了大殿。

大殿中那里有什麽人?巨大的香炉,褪色的黄幔,魁伟而狞狰的神像\ldots\ldots 灯光又像是忽然黯淡了。

铁心厕忍不住打了个寒噤,大声道;你究竟是什麽人?为何要躲起来?

没有人回答,也瞧不见人影。莫非那木雕的神像,在向一个平凡的女子恶作剧?

铁心兰不敢抬头,却又忍不住抬起头,巨大的山神,箕踞在一只猛虎身上,似乎正在瞧着她狞笑。

铁心兰几乎忍不住要抛下灯,转身逃出去。铜灯又变得冰冷,她的手已开始发抖。

忽然,神幔後爆发出一阵狂笑声。

一人大笑道:铁心兰呀铁心兰,你的胆子倒当真不小。这语声赫然竟似那木塑神像发出来的。

但铁心兰反自沉住气了,她也冷笑道:你既敢请我进来,为何又躲在神像後不敢见我。

那人大笑道;女人的胆子,有时倒的确此男人大得多,我本想骇你一跳的,谁知道竟被你瞧破机关了。

随着笑声,一个人缓缓自神像後转了出来,飘摇的灯光,照着他苍白的脸,锐利的眸子。他果然是个十分英俊的男人。

但铁心兰瞧见了这个男人,却此瞧见什麽恶魔都要吃惊。

他失声而呼,道:江玉郎,是你!

江玉郎微笑道:不错,是我,我方才跟你开了个玩笑,你受惊了麽?

铁心兰一步步往後退,道:你\ldots\ldots 你要怎样?

江玉郎却微笑道:我们是老朋友了,你看见我远怕什麽?

铁心兰连脚趾都冰冷了,脸上却勉强挤出一丝微笑,道;谁说我还在害怕,我也高兴得很。

她嘴里说着话,脚下还是在往後退,她突然将手里的灯,往江玉郎脸上摔了过去飞一般逃出了大殿。

她突然撞人一个人怀里!

铁心兰用不着用眼瞧,已知道这人是谁了,这人穿的衣裳又软又滑,滑得像一条满身都是腥涎的毒蛇。

这人的一双手也是又软又滑。他竟然轻轻搂住了铁心兰,柔声道:你为何要逃你难道怕我?

铁心兰整个人都软了,整个身子鄱发起抖来。她竟已没有力气伸手去推。

江玉郎轻抚着她肩头,缓缓道:告诉我,你怕的究竟是什麽?

铁心兰努力使自己心跳平静下来。於是她跺着脚道:我不理你了,你刚刚吓得我半死,我为什麽要理你?

她知道自己绝不是江玉郎的敌手,她知道此时此刻,唯有少女的娇嗔,才是她唯一可用的武器。

江玉郎果然笑了,大笑道:你真是个可爱的女人,难怪小鱼儿和花无缺都要为你着迷了。

铁心兰抢着道:你以为你自己比不上他们两人江玉郎眯着眼道:你以为我比他们两个人如何?

铁心兰道:他们还都是孩子,而你\ldots\ldots 你却已经是男人了。

江玉郎大笑道:你果然有眼光,只可惜你为何不早让我知道!

他将铁心兰抱得更紧,铁心兰简直快要吐出来了。

但她却只是娇笑道:你难道是呆子,你难道还要等我告诉你。

在这微带凉意的晚风中,在这寂寂静静的黑暗里,怀抱中有个如此温柔如此美丽的女人\ldots\ldots 江玉郎纵然厉害,只怕心也软了吧。

铁心兰的声音更温柔,缓缓道:现在,我不妨告诉你,其实我早已\ldots\ldots{}

她已准备了许久,此刻她只臂已蓄满真力,她用尽全身力气,向江玉郎腰眼上打了过去。

但她的手刚一动,左右肩头上的肩井穴,已被江玉郎捏住了,她的力气连半分都使不出来。江玉郎,这恶魔,竟早已看透了她的心意。

她只觉江玉郎的手沿着她背脊滑了下去,沿着背脊又点了她七八处穴道,她立刻连手指都无法动弹。

但江玉郎的手却还在她身上不停地动着,嘴里咯咯笑道:我知道你已喜欢我了,今天晚上我可不能辜负你的好意。

他冰冷柔滑的手,已从她衣服里滑了进去。铁心兰全身的肌肤都在他手指下战栗起来。

这是她处女的禁地,如今竟被这恶毒的男人侵入。她只觉灵魂已飞出了躯壳,心已飞出腔子。

她只想死!从江玉郎嘴里发出来的热气,熏着她耳朵。

只听江玉郎吃吃笑道:你不用怕,我会很温柔地对你,非常非常地温柔\ldots\ldots 你立刻就会发觉,小鱼儿和花无缺和我比起来,的确还都是孩子。

铁心兰咬着嘴唇,没有喊出来,她知道此时此刻,呼喊和挣扎非但无用,反而会激起江玉郎的兽性。

她已准备接受这悲惨的命运。她闭起眼睛,眼泪涌泉般流了出来。

谁知就在这时,江玉郎的手竟然停住不动了,铁心兰还末觉察这是怎麽回事时,江玉郎竟已将她推开。

她无助地倒了下去,倒在地上。她立刻便瞧见了一个女人。

这女人雪白的衣服,苍白的脸,眼睛瞬也不瞬地着江玉郎,冷冰的眼睛里,既没有愤怒,也没有悲哀。

江玉郎拍了拍手,强笑道:这丫头当我是呆子,居然想骗我我怎能不给他个教训。

那女子还是冷冷地瞪着他,不说话。

你吃醋了麽?他笑嘻嘻地去摸她的脸,又道:你用不着生气,更用不着吃醋,你知道我心里真正喜欢的只有你?

那女子动也不动地被他摸着,就像是块木头。

那女子终於开了口。她瞪着江玉郎,一字道:不管你是不是骗我,从今以後,我只要看见你再动别的女人一根手指,我就立刻杀了你,然後再死在你面前。

\hypertarget{ux7b2cux4e03ux5341ux4e5dux7ae0-ux5c71ux541bux592bux4eba}{%
\chapter{第七十九章
山君夫人}\label{ux7b2cux4e03ux5341ux4e5dux7ae0-ux5c71ux541bux592bux4eba}}

江玉郎吐了吐舌头,笑道:你真是会多心,有了你这麽漂亮的老婆,我还会打别人的主意麽?他搂起铁萍姑的脖子,在她面颊上亲了亲。

她垂下头,眼睛似已有些湿湿的,轻轻接着道:你知道,你不但是我平生第一个男人,也是我平生第一个对我如此亲切的人,无论你这麽做是真是假,只要你永远这样对我,我就已心满意足了,你就算做别的坏事,我\ldots\ldots 我\ldots\ldots 她咬着嘴唇,再也说不出话来。

铁心兰瞧着她,听到她的话,心里不禁暗暗叹道;这是个多麽寂寞的女人,又是个多麽可怜的女人,她甚至已明知江玉郎对她是假的,假的她竟也接受,她难道已再也不能忍受孤独?

铁心兰心里又是难受,又是同情。

大殿的神座下竟有条秘道。

这条秘道可以通向几间地室,铁心兰就被铁萍姑送入了一间很舒服的地室里来了。

她立刻发现,那黑衣人早已在这屋子里了他整个人软瘫在一张椅子上,显然也已被人点了穴道。

令铁心兰吃惊的是坐在这黑衣人对面的少女。

这少女有一双十分美丽的大眠睛,只可惜这双本该十分清澈的大眼睛里,此刻竟充满迷惘之色。

她呆呆地望着那黑衣人,似乎在思索着什麽?那黑衣人也在望着她却似瞧得痴了。

慕容九怎会也在这里?铁心兰忍不住惊呼出声来。

江玉郎瞧着他们哈哈大笑道:这里也有个你的老朋友,是麽?

铁心兰咬紧牙,总算忍住没有再骂出来。

江玉郎走到黑蜘蛛身旁大笑道:蜘蛛兄,又有位朋友来看你了,你为什麽不理人家?

黑蜘蛛这才像是自梦中醒来,瞧见了铁心兰,吃惊道;你?\ldots\ldots 你怎地也来了?

铁心兰苦笑道:我们本来\ldots\ldots 本来是想来助你一臂之力的。

江玉郎仰头狂笑道:只可惜普天之下,只怕谁也救不了你们!

铁心兰咬牙道:你莫忘了,还有花公子\ldots\ldots{}

江玉郎似乎笑得喘不过气来,大笑着道:花无缺此刻还等着别人去救他哩。

花无缺终於解开了白夫人颈上的锁。

他长长松了口气,道:夫人现在可以起来了麽?

白夫人身子却已软软的倒在稻草上,喘着气道:我现在怎麽站得起来?

花无缺怔了怔,道:怎会站不起来?

白夫人叹了气,道;呆子,你难道看不出来,我现在简直连一丝力气都没有。

她称呼竟已从公子变为呆子了。花无缺只有伸手去扶她的膀子。

但白夫人却像已瘫在地上,他那里扶得起,若不是他两条腿站得稳,只怕早已被白夫人拉倒在稻草堆上了。

他只好去扶白夫人的腰肢。

白夫人却又浑身扭曲起来,吃吃笑道:\ldots\ldots 痒死我了,原来你也不是好人,故意来逗我。

花无缺脸又红了,道:在下绝非有意。

白夫人咬着嘴唇,道:谁知道你是不是有意的!

花无缺简直不敢看她的眼睛,扭过头去道:夫人再不起来,在下就要\ldots\ldots{}

白夫人腻声道;呆子,你这麽大一个男人,遇见这麽点小事就没主意了麽?

花无缺叹道:夫人的意思要在下怎样?

你扶不起我来难道还抱不起我来麽?她面泛红霞,丰满的胸膛不住起伏\ldots\ldots{}

若是换了江玉郎,此刻不扑上去抱住她才怪,若是换了小鱼儿,此刻却只怕要一个耳光掴过去,再问她是什麽意思了。

但花无缺,天下的女人简直都是他的克星。他既不会对任何女人无礼,更不会对她们发脾气。

他甚至直到此刻,还末觉出这娇慵无力的女人,实在此旁边那吊睛白额猛虎还要危险十倍。

花无缺默然半晌,叹了口气,柔声道;夫人此刻若真的站不起来,在下就在这里等等好了。

白夫人眼波流转,笑道:我若是一个时辰都站不起来呢?

花无缺道:在下素来很沉得住气。

白夫人噗哧一笑,道:我若是叁天叁夜都站不起来,你难道等叁天叁夜?

花无缺居然还是不动气,微笑道;在下知道夫人绝不会让在下等叁天叁夜的。

她忽然轻呼一声,跳起来扑入花无缺怀里。

花无缺这才吃了一惊,道:夫人,你\ldots\ldots{}

不好,我\ldots\ldots 我丈夫回来了。

花无缺也不禁变了颜色失声,道:在那里?

白夫人全身发抖,道:在\ldots\ldots 就在\ldots\ldots{}

只听外面一人大吼道;就在这里!

砰的一声,左边一扇窗户,被震得四分五裂,一条大汉从粉碎的窗框间直飞了起来。

他身上穿着件五色斑斓的锦衣,面色黝黑,满脸虹须如铁,一双眼睛更是神光炯炯,令人不敢逼视。

花无缺早就想推开白夫人了,但白夫人却紧紧搂住他的脖子,死也不肯放松,像是已经怕得要命。

那大汉自然已瞧得目尽裂,怒喝道:臭裱子,看你做的什麽事?

他一跃入大厅,那猛虎就摇着尾巴走过去,就好像只驯服的家犬。但这大汉却一拳将这重逾数百斤的猛虎打得几乎飞了起来,出去一丈多远,跳起脚怒骂道:好个不中用的东西,我要你看着这臭女人,你却只知道睡懒觉。

这猛虎竟连半分虎威也没有了,翻了个身站起来,乖乖的蹲在那里,瞧那垂头丧气的模样,简直连只病猫都不如。

花无缺简直瞧呆了,忍不住道:阁下暂且息怒,听我一言\ldots\ldots{}

他不说话还好,一说话那大汉更是暴跳如雷狂,吼道:我听你什麽我听你个屁,老子前脚一走,你们这双狗男女就不干好事,老子早就知道这臭裱子是天生的贱货,竟会看上你这种小兔崽子!

白夫人却大声道;老实告诉你,我们在一起已经有两叁年了,只要你一出去,我们就亲亲热热的在一起,你又能怎麽样?

那大汉仰面狂吼,拚命腿着自己的胸膛,吼道:气死我了!

但花无缺却比他还要愤怒十倍,嗄声道:白\ldots\ldots 白夫人,我和你无冤无仇,你\ldots\ldots 你为何要如此\ldots\ldots{}

白夫人柔声道:好人,你怕什麽,事情反正已到这种地步了,咱们不如索性跟他讲个清楚反而好,是麽?

花无缺气得手都发起抖来,道:你\ldots\ldots 你\ldots\ldots{}

那大汉厉喝道;讲清楚也没用,你们这对狗男女若想要老子做睁眼王八,那是在做梦!

他狂吼着扑过来,一拳击出!

拳风虎虎,竟将满厅灯火都震得飘摇不定,花无缺的衣袂,也被他拳风激得飘然飞舞!

他实在不想打这场冤枉架,身形一斜,轻轻避了开去!

那大汉更是狂怒,喝道:好小子,难怪敢偷人家的老婆,原来有两下子!

喝声中又是叁拳击出。花无缺展开身形,连连闪避,能不还手,他实在不想还手。

但这大汉非但拳重力猛,而且招式也十分险峻毒辣,武功之高,竟远出花无缺意料之外。

花无缺也实在被逼得不能不回手了。他左拳拍出,右手巧妙地划了半个圆弧。

这正是妙绝天下的移花接玉神功。无论是谁,被这种奇异的力量一引,发出的招式,都会反击到自己身上。

谁知那大汉一声虎吼,身子硬生生向後一挫,竟将发出去的拳势,硬生生在半途顿住!

他出拳力道那般猛烈,後防必已大空,此时发出的力道骤然回击,本是任何人也禁受不住的?

花无缺更末想到这人竟能破得了移花接玉神功,除了燕南天之外,这只不过是他所遇见的第二个人!

他委实不能不吃!这大汉功力之深厚,竟不可思议!

那大汉瞧着他狞笑道:原来是移花宫出来的,难怪这麽怪了\ldots\ldots 但你这麽点功力,又怎能奈何我白山君,叫你师娘来还差不多?

他拳式再度展出,力道更强更猛,竟像是真的末将威震天下的移花接玉放在眼里。现在他更不能不还手了。

这白山君的武功,实已激起了他的敌忾之心,他骤然遇见了这麽强的对手,也不免想分个强弱高低!

白夫人在一旁拍手娇呼道:对,不要怕他,为了我,你也该和他拚了!

这呼声听在花无缺耳里,虽然越想越不是滋味,但现在他已好像骑上了虎背下都下不来了。

他简直猜不透这白夫人打的究竟是什麽主意!白山君拳势越来越凶猛。

他每一招每一拳击出,彷佛都已拚尽了全力,再也没有馀力可使了,但他第二拳发出,力道却又和头一拳同样凶猛。

但花无缺身形如惊鸿、如游龙,满厅瓢舞,白山君拳势虽猛,空自激得他衣袂飞舞,却还是将他无可奈何。

白夫人娇笑道:好人,我真还末看出你有这麽好的功夫,有你这样的情郎,我还怕什麽你赶紧宰了这老家伙,我们就可以安安稳稳地做一对永远夫妻了。

她越说越不像话,花无缺既不能封住她的嘴,又没法子不听,纵然定力不错,却也难免为之分心。那白山君的拳式,却又根本容不得他稍有分心。

白夫人忽然失声惊呼道:哎约,小心他下一着虎爪抓心!呼声中,白山君果然虎吼一声一爪抓来。

这一招也末见得特别厉害,花无缺向後微一错步,就避开了,心里倒不觉有些奇怪,不知道白夫人为何要突然惊呼起来。

他知道这其中必定有花样的。

但这时却已没有时间来让他想了,他脚步刚往後一退,左右双膝的腿弯里,已各中了一点暗器。

他直到身子倒下,还不知道这暗器竟是白夫人发出来的,白夫人却已过来,抱住了白山君的脖子,娇喘着道:我本来以为已爱上了别人,但你们一打起来,我才知道真正爱的永远是你,我宁可将天下的男人都杀光,也不能看别人动你一根手指。

花无缺叹了气,闭上眼睛,心里直发苦:唉,女人\ldots\ldots{}

他现在才懂得小鱼儿为什麽会对女人那麽头疼了。

只听白山君狂笑起来,笑声越来越近,终於到了他身旁,他眼睛闭得更紧,既不想说,也不想听,更不想看。

白山君却狂笑道:你现在总该知道咱老婆的厉害了吧,谁若沾上她,不倒楣才怪,你年纪轻轻,不像个呆子,怎地偏偏做出这种事来?

花无缺咬紧牙关,也不想辩驳。白山君却一把拎起他衣领,拖起就走。

只觉白山君竟将他放到一张短榻上,又对他翻了个身,面朝下,接着,竟将他的裤子脱了下来。

花无缺骇极大呼,道:你\ldots\ldots 你想干什麽他拚命仰起头,张开眼睛!

只见白山君笑嘻嘻地站在短榻旁,面上绝没有丝毫恶意,手里拿着一块黑黝黝的马蹄铁,缓缓道:我那老婆暗器之歹毒,昔年连燕南天听了都有些头疼,你两条腿各中一枚,我若不用这吸铁星将它吸出来,你这辈子就休想走路了。

花无缺又惊又疑,道;你\ldots\ldots 你为何要救我?

白山君忽又大笑起来,道:你以为我真相信我老婆的话麽?

这时他已自花无缺腿穹里吸出了两根细如牛毛的小针,针虽小,但钉在花无缺腿里时,他全身竟连一丝力气都没有,连手指都动弹不得。

此刻针被吸去,花无缺立刻就奇迹般恢复了力气,翻身一掠而起,眼睛睁睁望着白山君,道:你既不信她的话,方才为何\ldots\ldots 为何要那般恼怒?

他简直好像坠入五里雾中,再也摸不着头绪。

白山君拍了拍他肩头,笑道;小伙子,我知道你也被弄糊涂了,好生坐下来听我说吧。

花无缺苦笑道:在下倒的确想请教请教。

白山君竟也叹了口气,竟也苦笑道;你可知道,世上有一种奇怪的人,别人若是爱她敬她她就觉得痛苦,若是百般凌辱虐待於她,她反而会觉得舒服快乐。

花无缺既觉惊奇,又忍不住觉得有些好笑,道:世上真有这样的人?

白山君苦笑道;自然是有的,我老婆就是其中的一个。

她\ldots\ldots 她怎会这样子的?

白山君叹道:据说她从小就是如此,非但从小就喜欢别人虐待她,而且她自己还要虐待自己,到了老年时,这脾气更是变本加厉,竟连普通居室都待不下去,非要将住处布置成马厩一般,而且还要我用铁锁住她。

花无缺叹道:原来这竟是她自愿如此的,在下本还以为是\ldots\ldots{}

白山君道:我虽然知道她这毛病,但有时还是不忍下手,也不愿意动手,所以她就时常会故意激怒我,为的就是想我揍她。

花无缺叹道:今日之事,想来也必定就是为了这原故了。

白山君道:她年华逐渐老去,总以为我会对她日久生厌,移情别恋,所以时常又会故意令我嫉妒?\ldots\ldots,其实白夫人那些做作全都是多馀的,阁下爱妻之心,自始至终,从来也未曾改变过,是麽?

白山君仰首大笑道;不错,我只顾了她的欢喜,却令朋友你吃了个大亏,只事实是在我夫妻之错,是打是罚,但凭朋友你吩咐如何!

花无缺整了整衣裳,微笑道:实不相瞒,在下本来对此事也委实有些恼怒,但听了阁下这番话,却非但对阁下的处境甚是同情,对阁下如此深挚的伉俪之情,更是十分相敬,何况,在下本已作了贤伉俪的阶下囚,本只有任凭阁下处置的。

他语声忽然顿住,只因他刚走了两步,忽又发现自己虽然已可行动无疑,但一口气到了腰上便再也无法提起。

花无缺缓缓道;阁下又何苦要在我腰畔暗施手脚?

白山君像是吃了一惊,失声道;真的麽?那想必是我方才为你拔针时,一不小心,又将那游丝针插入你腰畔什麽穴道里去了。

花无缺悠悠道:就在笑腰穴下。

白山君像是着急得很,搓着手道:若在笑腰穴附近,那就麻烦了,我实在不敢胡乱替你拔针,否则若是又一不小心,令那游丝针窜入你笑腰穴里,便是神仙也救不了的,只有眼看着你狂笑叁日,笑死为止。

花无缺默然半晌,道:既是如此,在下只有告辞,去另外设法了。

白山君叹道:你现在若是随意走动,那游丝针也会跟着你气血而动,窜入你笑腰穴里,你纵然十分小心,也走不出七十步的。

花无缺停下脚步,缓缓转过身,静静地凝注着他,良久良久,才长长叹了气,苦笑着摇头道:贤夫妇的行径,的确令人难解得很,尊夫人不愿为人,却愿做马,这且不去说她,而阁下\ldots\ldots{}

白山君凝注着他,过了很久,才缓缓道:你真的直到此刻还不知道我是谁?

花无缺道:在下见识一向不广。

白山君笑道:不错,移花宫门下,自然不会留意江湖侠\ldots\ldots 但十二星象这名字,你难道也从末听人说过?

花无缺恍然失声道:不错,虎为山君,难怪阁下不但以虎自命,还蓄虎为奴,马为虎妻,难怪尊夫人不愿为人愿做马了。

白山君大笑道:你此刻既然已知道我是谁,便该知道十二星象中人,与移花宫乃是死敌,你既已落人我手中,难道不害怕麽?

花无缺神色不动,淡淡道:阁下若要动手,方才便不必救我,阁下方才既然救了我,想必是有求於我,阁下既然有求於我,我难道还会害怕麽白山君又自大笑起来,他笑着笑着忽又沉下脸,泛声道:不错,我的确有求於你,只要你说出移花接玉这功夫的秘密,我不但立刻放了你,而且你若有所求,我必也件件应允。

花无缺忽也笑了起来,道:阁下若以为移花接玉的秘密,如此容易便可得到,阁下就未免会大大失望了。

白山君变色道:你难道敢不说?

花无缺悠然道:世上令人开口的法子有很多,有的以生死相胁,有的以酷刑逼供,有的以财色相诱,阁下不妨都试试看,看是否能令在下开。

白山君默然半晌,忽又一笑,道:我既然无法可想,也不愿白费气力,看来只有一走了之。你愿意留下,就留下,愿意走就走,我也管不了你了。不过你万一要找我时,只要大叫一声,我就会来的。他竟然真的说走就走,话末说完,已扬长而去。

这一着又出了花无缺意料之外,一时间竟有些示知所措,只见白山君刚走出门,又回过头来,笑道:但你也莫要忘记,千万莫要走出七十步,否则大笑而死的滋味,可实在比什麽死法都要难受得多。

\hypertarget{ux7b2cux516bux5341ux7ae0-ux4e49ux65e0ux53cdux987e}{%
\chapter{第八十章
义无反顾}\label{ux7b2cux516bux5341ux7ae0-ux4e49ux65e0ux53cdux987e}}

花无缺眼见着白山君从这扇门里走出去,他本来也可以跟着走出去的,但他却只怔在那里,动弹不得。

他知道白山君的话绝不是一意吓唬他,他虽然还可以走出去,却也不愿以性命来作赌注,赌自己是否能走出七十步。

就在此时,忽听一声虎吼

厅房中窗户本是紧闭着的,但一声虎吼过後,腥风突起,灯火摇摇欲灭,满堂桌椅,也似将随风而倒!

花无缺不由得耸然色变,虎已入了厅堂。

这平阳之虎,竟又已恢复了森林之王的威势,虎步虽慢,但每一步都似乎带着千钧之力!

只可惜他此刻连真气都不能提起,简直可说是手无缚鸡之力,何况搏虎?猛虎,既已长驱而入,他只有一步步往後退。

那猛虎已逼到他面前,虎尾已如旗杆般耸起,接着而来的是一扑一掀一剪,又岂是此刻的花无缺所能抵挡?

花无缺额上冷汗已滚滚落下!眼见他此刻若不向白山君呼救,便难免要被虎爪撕裂,一饱虎吻。

他虽不愿死,将性命看得十分珍贵,但像他这麽样的人,却又怎甘心向别人呼救呢?又是一声虎吼,几上花瓶震落,当的摔成粉碎!

江玉郎已狂笑着走了出去。铁心兰听着他得意的笑声,手脚俱已冰冷。

她知道江玉郎心肠虽毒,胆子却小,若非有十分的把握能制住花无缺,他此刻绝不会这麽得意,这麽放心!

眼泪,已一连串从她眼睛里流了出来。

突听黑蜘蛛冷笑道:到底是女人,死,又有什麽大不了何必哭得如此伤心?

铁心兰咬着嘴唇,道;你\ldots\ldots 你以为我是在为自己伤心?

黑蜘蛛忽然瞪起眼睛,道:你难道是为了那姓花的?

铁心兰垂下了头,黑蜘蛛大声道:若是小鱼儿死了,你也会如此伤心?

铁心兰霍然抬起头,瞧了他半晌,凄然一笑,道:他若死了,你以为我还能活得下去麽?

既然如此,你为何又要为别人伤心\ldots\ldots 一个女人只能为一个男人伤心,别的男人是死是活,她都不该放在心上。

铁心兰长长叹息了一声,黯然道:我的心事,你不会懂的,永远都不会懂的,任何人都不会懂的。

铁心兰转目去瞧慕容九慕容九仍然痴痴地站在那里,连手指都没有动过,就像是永远也不会动了。

铁心兰凄然一笑,道:你自己岂非也是为了救别人而来的?

黑蜘蛛大喊道:不错,我是为了救她而来的!但我是心甘情愿地为她而死,除了她之外,别的女人就算死在我面前,我也未必会伸一伸手的?

铁心兰凝住着他幽幽道:但你无论对她多麽好,多麽真情,她也不会知道的。

黑蜘蛛怒目瞪着她,一字字道:我告诉你,我对她好,用不着她知道,也用不着她同样来对我好,我爱她就是爱她,绝没有任何条件!

铁心兰颤声道:就算她以後不爱你,甚至根本不理你,还是要爱她?

黑蜘蛛大声道:不错,我爱她,并不是为了要她嫁给我,只要她能好好的活着,我死了也没有什麽关系。

铁心兰默然半晌,目中又流下泪来,黯然道:一个女人一生中,若能得到这样的情感,她死了也没有什麽关系了,她已可心满意足\ldots\ldots{}

她抬起头,忽然发现慕容九此刻竟也已泪流满面。

铁心兰又惊又喜,大声道:你已能听得懂我们的话?你已能懂得他的意思了麽?

慕容九目中虽有泪珠不停地流下来,但目光仍是一片痴迷,黑蜘蛛面上本已泛起了兴奋喜悦的光芒,此刻光芒又已黯淡。

铁心兰柔声道:你用不着难受,她现在神智虽仍痴迷不醒,但你的真情,显然已感动了她,只要你的心不变,总有一天,她会完全领受的。

突听一人咯咯笑道:总有一天\ldots\ldots 嘿嘿,只怕这一天永远也不会来了。

江玉郎竟又摇摇摆摆走了进来。

铁心兰吃惊道:你还想来干什麽?

江玉郎笑嘻嘻道:我自然是来看你的。他摇摇摆摆走到铁心兰面前又伸手去摸她的脸。

铁心兰骇极大呼道:你\ldots\ldots 你莫忘了,那位穿白衣服的姑娘\ldots\ldots{}

江玉郎大笑道:我自然不会忘记她,所以我已给她吃了一服安神的药,现在她已安安稳稳地睡了,你就算喊破喉咙,她也不会听到。

铁心兰全身又不觉头抖起来,大呼道:只要你碰我一根手指,我就\ldots\ldots 我就告诉她。

江玉郎格格笑道:不会,你不会告诉她的,我保证她醒来的时候,你已经不能说话了。

他的手已从她肩头缓缓滑到胸膛。

铁心兰连血都凉了,头声道:求\ldots\ldots 求求你,不要这样,求求你杀了我吧。

江玉郎笑道:杀你?我现在为何要杀你?江小鱼和花无缺的情人,我若不享受享受,我怎对得起他们。

他大笑着将铁心兰抱了起来狞笑着又道:老实告拆你,我不惜一切,也要得到你,倒也不是真的看上了你,我只不过是因为花无缺和江小鱼\ldots\ldots{}

铁心兰已听不到他的话,她已晕了过去。

黑蜘蛛虽然将牙齿咬得岐吱作响,却也只有眼见江玉郎抱着她走出门,眼看着她就要被人蹂躏猛虎作势欲扑,花无缺已眼见要丧生虎爪。

就在这时,他忽然发现身旁挂着的一幅昼,竟然紧紧贴在墙上的,下面的昼轴,也紧嵌在墙里。

花无缺已无瑕思索,伸手将昼轴一拖一扳,整幅昼便突然陷入,现出了一重门户,他立刻闪身而入。

又是一声震天动地的虎吼。但花无缺已将这秘密的门户阖起。

花无缺虽也想瞧瞧门里的情况,却又实在不敢妄自多走一步他每走一步,下一步就可能是致命的一步!

但这时门里竟有颤抖的呼声传了出来求求你,不要这样,求求你杀了我吧!

这赫然竟是铁心兰的呼声。

花无缺热血冲上头顶,再也不顾一切,大步走了过去!

江玉郎洋洋得意,刚想将铁心兰抱出门,忽然发现一个人站在门,档住了他的去路。

灯光照着这人苍白愤怒而英俊的脸,竟是花无缺白山君和白夫人却踪影不见?

江玉郎就像是挨了一鞭子,立刻踉跄後退了几步。

花无缺怒目瞧着他,此刻只要还有一丝真气能提得上来,花无缺也不能再容这阴毒卑鄙的小人再活在世上。

幸好江玉郎也不知道他已无力伤人,纵然再借给江玉郎一个胆子,也万万不敢向他动手的。

花无缺只有在暗中叹了口气,缓缓道:你还不放下她?

江玉郎满脸陪笑已恭恭敬敬将铁心兰放在椅子上。

花无缺道;我也不愿伤你,你\ldots\ldots 快走吧?

江玉郎如蒙大赦,一溜烟逃了出去,嘴里犹自陪着笑道;小弟遵命\ldots\ldots 小弟遵命!

黑蜘蛛忍不住狂吼一声,道;姓花的,你这是什麽意思?这样的人,你为何不宰了他?

花无缺苦笑道:杀之既污手,放了也罢。

他生怕江玉郎还在偷听丁自然不肯说出真正的原因。

黑蜘蛛怒道:你怕沾污了你那双宝贝的手,我却不怕,你快解开我的穴道,我去找他算帐。

花无缺怔了怔,他现在又怎有力量为别人解开穴道?他只有装作没听见。

黑蜘蛛大怒道:你难道也不愿沾着我?我难道也会弄脏你的手?

花无缺只有垂着头,向铁心兰走过去,又走了十几步,才走到身旁,他只觉这段路简直长得可怕。

黑蜘蛛冷笑道:好,很好,原来你竟是这样的人,我们真看错了你上像你这样的人手指若沾着我,我反倒会作呕。

花无缺暗中叹了口气,无话可说。

他平生从末被人如此辱骂,此刻却只有忍受,只因他此刻若是说出真相,万一被江玉郎听见大家便谁都休想活得成了,江玉郎此刻唯一畏惧的就是他,而他对江玉郎,又何尝不是步步提防。

这时铁心兰悠悠醒转。

她一眼瞧见了花无缺,泪眼中立刻发出了光,喜极而呼道:你来了!你果然来了,我就知道没有人能伤得了你,我早已知道你一定会来救我们的。

黑蜘蛛冷笑道:我若要这种人来救我,倒不如死了还好。

铁心兰大奇道:你\ldots\ldots 你为何要对他这样说话?

突听一人道:花公子现在自顾尚不瑕,那有力气救你们,你们难道还瞧不出来麽?你们又何苦逼他?

狂笑声中,江玉郎又大摇大摆走了进来。花无缺竟眼睁睁瞧着他走进来,一句话也说不出。

铁心兰简直骇呆了,嘶声道:这\ldots\ldots,这是真的麽?

花无缺长长叹了口气,缓缓道;江玉郎,我不愿杀你,你难道真要来自寻死路?

江玉郎大笑道:不错,我就是要来自寻死路,我现在就要将铁姑娘抱走,死在她身上。

他嘴里虽说得狂,但心里多少还是对花无缺有些畏惧,绕过了他,才敢走进铁心铁心兰,一把抱了起来。

铁心兰大惊呼道;你\ldots\ldots 你敢\ldots\ldots{}

江玉郎瞧见花无缺还末出手,胆子更大了,大声笑道:我为何不敢?难道我们的花公子还敢对我怎样!

他抱着铁心兰,一步步退着往外走,眼睛还是瞪着花无缺。

花无缺汗如雨下?

他现在已走了五六步,下一步便可能迈入鬼域!

汪玉郎放声狂笑,道:花无缺呀,花无缺,你为什麽不过来你那一身自命天下无敌的武功,到那里去了?你难道真要眼看着我将你的情人抱上床麽?

他已退到门,却故意停了下来。

花无缺全身都颤抖起来,死,固然可怕,更可怕的是,他知道自己若是死了,铁心兰悲惨的命运还是无法改变?

江玉郎的手,又袭上铁心兰的胸膛,奸笑道:你瞧,这是多麽软的胸膛,多麽嫩的皮肤,这处女的身子,本来是完完全全属於你的,现在,却完全归我了,我要怎麽样享受,就可以怎麽样享受!

花无缺突然一步步走了过去!

他就算明知必死,他就算明知救不了铁心兰,但他也不能眼见着铁心兰被人如此侮辱!

江玉郎笑声忽然顿住了。

他瞧着花无缺已铁青得可怕的脸,吃惊道:你\ldots\ldots 你敢过来?

花无缺深深吸了口气,道:放下她?

江玉郎目光闪动,忽然发现花无缺的脸色虽沉重,但脚步却是轻瓢瓢的,像是一个完全不会武功的人走路的样子。

江玉郎立即又放声狂笑起来,大笑道;花无缺,你吓不了我的!我早已看出,你已被白山君夫妻所伤,武功连一分都使不出来了,是麽?

花无缺咬着牙不说话,还是一步步往前走!

他自然知道江玉郎说的不假,也知道自己正在步入死路,但他现在已只有死路一条,别无选择的馀地!

江玉郎厉声喝道;好小子,你真有种!但你若敢再往前走一步,我就宰了你!

花无缺暗中叹了口气,又往前走了一步。他忽然发觉死亡并不如想像中那麽可怕?

铁心兰忍不住嘶声大呼道:花无缺,求求你,莫要过来吧,我\ldots\ldots 我没有关系,我对你更没有什麽好处,你何必将我放在心上。

汪玉郎狞笑道:你莫忘记,一个人是只有一条命的?

花无缺缓缓道;不错,生命的确可贵,它绝没有任何东西可以交换\ldots\ldots{}

他微微一笑,接着道:所以,我若要为一个人而死,也绝不需要你有交换条件,她是否对我好,她是否爱我,都没有什麽关系。

铁心兰已痛哭失声,再也说不出话来。

黑蜘蛛终於忍不住大喝道;一条好汉子!我黑蜘蛛平生从未向人低头,但对你\ldots\ldots 我方才错怪了你,现在郑重向你致歉,你\ldots\ldots 你好生去吧?

花无缺傲笑道:多谢。

他又往前走出一步!江玉郎似乎也已被他这种不顾一切的勇气吓呆了,他再也没有想到花无缺竟也会和小鱼儿一样,必要时竟真的会拚命的!生命,在别人看来固然是珍贵无比,但在他们眼中,竟似看得轻淡得很。

\hypertarget{ux7b2cux516bux5341ux4e00ux7ae0-ux751fux6b7bux4e24ux96be}{%
\chapter{第八十一章
生死两难}\label{ux7b2cux516bux5341ux4e00ux7ae0-ux751fux6b7bux4e24ux96be}}

江玉郎见花无缺缓缓向自己走来,终于狞笑道:``好,你既然要死,我就索性成全了你吧!杀个把人,想来也不会妨碍我享受的兴致的''他掌心已扣着一把暗器,正待发出去!

谁知就在这时,突见花无缺身子剧烈的颤抖,如被针刺,接着放声狂笑了起来』笑声有如疯狂,江玉郎更想不到温文尔雅的花无缺,也会发出这疯狂般的笑声,忍不住失声道:``你疯了么?''花无缺逼出最后一步时,突觉一根针刺入了他全身最脆弱最柔软的地方,一阵奇异的滋味,又痛又痒,直钻人心里。

他竟突然忍不住疯狂的大笑起来,竟再也遏制不住,但那股被隔断了的真气,却骤然为之畅通!

江玉郎又惊又奇,满把银针,暴雨般撒出!

花无缺狂笑叱道:``你\ldots\ldots 你敢!''

叱声中举手划了个圆圈,漫天暗器,突然如泥牛入海,无声无息的一起消失,也不知到哪里去了!

黑蜘蛛动容道:``好一着移花接玉!''

江玉郎吓得面如土色,大声惊呼道:``你方才难道是在装模作样?''花无缺道:``不错哈哈还不放下她来!''江玉郎颤声道:``我我放下她,你就放了我?''花无缺大笑道:``放放''

江玉郎他一言既出,重逾千斤,再也不敢哆嗦了,放下铁心兰转身就跑,一眨眼便无踪影!

花无缺不断地狂笑着,心里却已凉透!白山君的话,竟果然不是假的!

花无缺紧咬着牙,却也止不住笑声,他只有暂时不去想这件事,俯身拍开了铁心兰的穴道.

铁心兰瞪大了眼睛讶然道:``你将我们都骗过了,害得我们为你着急,你就觉得很好笑么?''花无缺知道铁心兰又误会了,却又不能解释,到了这种时候,他还怕铁心兰知道真相后,会为他伤心.

他只有转过身子,先拍开黑蜘蛛的穴道.

黑蜘蛛也大怒道:``你觉得这玩笑开得很好笑么?''花无缺暗中叹了口气,又有谁能瞧见他心里的痛苦!别人只能瞧见他好像在得意地大笑着,他拉起铁心兰狂奔而出.

黑蜘蛛到底江湖历练较丰,终于也发现有些不对,皱着眉想了想,忽又发现慕容九在呆望着他.

他立刻抛开一切心事,也拉起慕容九奔了出去!

铁心兰是从这条地道进来,自然知道秘室的出口.

他们乘着黑暗的夜色,奔入旷野,满天星群渐隐,山麓下林木沉寂,花无缺的笑声听来也就更刺耳.

铁心兰又忍不住道:``你可以不笑了么?''

花无缺的心已快碎了,几乎忍不住要将真相说出来.

但他忽又想到,与其要让铁心兰等着看他的惨死之况,倒不如还是被她永远误会下去的好.他反正已快死了,又何必还要叫别人伤心.

铁心兰跺了跺脚,道:``你你再要这样笑下去我就走了!''花无缺暗中叹了口气,嘴里却大笑道:``你走吧!哈哈我反正已知道你爱的不是我哈哈哈,你快走吧!''铁心兰身子一震,颤声道:``你真要我走?''

花无缺狂笑着道:``是!''

铁心兰呆视着他,一步步往后退.花无缺还是在不停地狂笑着.

他已明知必死,他眼见着他最珍惜的人离他而去,他拼命救出来的人,也丝毫不谅解他,但他还是只有不停的笑,不停的笑

寂静黑暗的山林中充满了他这凄凉而疯狂的笑声,最后一粒孤星,也沉重地落入死灰色的苍穹里

花无缺眼泪终于也忍不住流下面颊.

他从小生长的,便是一个冷酷无情的世界,他从来也不知道流泪是什么滋味,但现在他却在狂笑中落下泪珠!

忽然间,铁心兰又来到他面前,静静地瞧着他.

花无缺赶紧悄悄擦干了泪痕,大笑道:``你又回来做什么?''铁心兰面上已带着有恐惧之色,颤声道:``告诉我,这究竟是怎么回事?''花无缺道:``什么事?\ldots 哈哈,我只是觉得你好笑!哈哈哈,你难道连赶都赶不走?''铁心兰道:``我知道你绝不是这样的人,我不能走!''花无缺道:``你不走?哈哈,好,我走!''

他还没有转过身,铁心兰已一把抱住了他,嘶声道:``告诉我,你\ldots 你是不是受了种很奇怪的伤?''花无缺大笑道;``我怎会受伤?''

铁心兰只觉他的手已冷得像冰一样,大骇道,``你为何不肯说实话?''花无缺心如刀割,却还是只有笑,不停地笑。

铁心兰又流下泪来,道:``我知道你是为了我,才变成这样子的,你\ldots{}''花无缺狂笑道:``我为了你\ldots\ldots 哈哈,你还是快去找江小鱼,快去快去!''铁心兰嘶声道:``我不去,我谁也不找,我一定要陪着你,无论谁也不能要我走。''花无缺道:``江小鱼呢?''

铁心兰泪如泉涌,颤声道;``小鱼儿?\ldots\ldots 我早已忘记他了。''花无缺大笑道:``但你还是忘不了他的,哈哈\ldots\ldots 爱,并不是交换,哈哈哈,你若爱一个人,无论他怎样对你,你都是爱他的。''铁心兰:``我\ldots\ldots\ldots 我\ldots\ldots\ldots{}''她终于扑倒在地上,放声痛哭起来。

花无缺仰天笑道:``你还是去找他吧\ldots。好生照顾他,知道么\ldots\ldots 哈哈\ldots\ldots 但望你们一辈子过得快快活活\ldots\ldots{}''他笑声忽然渐渐远去!铁心兰始起头时,花无缺已踪影不见了。

她知道自已是永远追不上他的,只有痛哭着嘶声呼:``花无缺,你这混帐\ldots\ldots 你若这样死了,我能嫁给小鱼儿么?你若这样死了.我们这一生,又怎么会再有一天快活?''她用尽力气放声大呼道:``花无缺,花无缺\ldots\ldots 你回来吧!''但这时哪里会再有花无缺的回应?只有冷风穿过树林,发出一声令人断肠的呜咽\ldots\ldots 天亮的时候,花无缺生命就将结束!他知道自己的生命简直比一只寒风中的秋蛾还要短促!

但他难道就这样等死么?

花无缺本已绝望地坐下来,此刻却又一跃而起。

他仰天狂笑道;``花无缺呀花无缺,你至少现在还是活着的!你至少还可用这短促的生命做一番事!你就算要死,也不该死得无声无息!''天地间晌彻了他高亢的笑声。

他返身又向那山君庙飞掠了过去。大殿仍然黑暗而阴森。

花无缺一掠而入,飞起一脚,特那山君神像踢了下来,狂笑着道:``白山君,你出来吧!''花无缺狂笑着提起神案,重重摔在院子里,大笑道:``白山君,你听着,我虽然要死了,但我也要将你们这些阴毒的人全都杀死,为世人除害!''突听一声虎吼,那吊睛白额猛虎箭一般窜了进来。

花无缺狂笑着迎上去,身形一避,先让过这猛虎不可抵挡的一扑之势,反身一掌,砍在虎颈上!

花无缺身形展动,如游龙天骄。那猛虎哪里能沾着他半片衣袂,叁扑之后,其势已竭!

花无缺再拍出一掌,猛虎竟已伏在地上,动弹不得!

后院里竟也是寂无人影!

花无缺满腔悲愤,竟是无处发泄,一脚踢开门户,抓起桌子,远远掷出,桌子被摔得粉碎:但纵然这整个庄院都被他毁去,却又有何用?

花无缺狂笑大呼道:``白山君白山君!你在哪里!你为何不肯出来与我一战!''他此刻但求一战,纵然不敌战死.也是心甘情愿的!

花无缺但觉一股热血直冲上来,随着狂笑溅出了点点鲜血,有如花瓣般洒满了他的衣衫。

他只觉自己气力似已将竭,身子也摇摇欲倒!他那一般怒气,也似已由厉而衰,由衰而竭。

花无缺忽然发现,此刻只希望有个人在他身旁,无论是谁都没有关系,他实在不愿意寂寞而死!

他只希望战死!却偏偏没有人理睬,他希望死在人群中,却似乎竟已没有力气走出去!

花无缺跟地后退,噗地倒在椅上,目光茫然凝注着逐渐降临的曙色,只希望死亡也跟着曙色而来。他实已心灰意冷,他竟在等死!

但他却还是忍不住要笑,不停的笑,疯狂的笑,笑出了他自己的生命,却笑不出他心头的悲愤!

他可以逃避一切,却又怎能逃避自己的笑声,这笑声就像是附骨的毒蛆,一直要缠到他死而为止!

他现在甚至已不措牺牲一切,只求能停住这该死的笑声,他拼命掩起耳朵,却又怎会听不见自己的笑声。

这笑声简直令他发疯,为了使笑声停止,他已准备结束自己的生命!

就在这时,苍茫的曙色中,忽然现出了一条人影!

晨雾迷漫,如烟氤氲,花无缺终于看清了她的脸,那美丽的脸上,似乎也带着绝望的死色!

白夫人!这人竟是白夫人!她终于还是出现了!

花无缺本来以为自己一见了她就会冲过去的,谁知此刻竟只是呆呆地坐着,呆呆地望着她。

花无缺又以为她一定是要来杀他的,谁知她也只是静静地站在他面前,静静地瞧着他。

花无缺忽然狂笑道:``你来的正好,既来了为何还不出手?''白夫人只是瞧着他竟不说话。

``原来你只是来看着我死的么?''白夫人还是不说话。

``很好,无论你为何而来,我都很感激你,我正在觉得寂寞。''白夫人竟忽然长长叹息了一声,黯然道:``可怜的人,你竟连求生的勇气都没有了么?''花无缺心里一阵绞痛,嘶声笑道:``你一心只求我速死,却反来要我求生,你难道还觉得我的痛苦不够?''白夫人道:``但我也知道我是对不起你的,只求你能原谅我。''花无缺狂笑道:``你为什么要说这些话?难道又想来骗我么?白夫人黯然垂首,道:''我也知道你是绝不会相信我的,但\ldots\ldots 但你能跟我去瞧一样东西么?"花无缺动也不动地坐着,笑声已嘶哑。

白夫人抬头凝注着他,颤声道:``我只求你这一砍,无论如何,这对你也不会再有什么伤害是么?''她目中竟似真的充满了哀求之色。

花无缺嘶声笑道:``不错,我既已将死,还有什么人能伤害我?''他终于还是跟着她走了出去。

穿过几间屋子,花无缺赫然发现竟有个人倒悬在横梁上,全身鲜血淋漓,一柄长刀穿胸而过。

花无缺失声道:``白山君死了!''

狂笑声掩去他语声中的惊讶之意,他语声中甚至还有些失望,却绝没有高兴的意思,他虽想与白山君一战,虽想特此人除去,但骤然见到此人死状如此之惨,想到一个人生命之短促,竟不觉兴起兔死狐悲之感。

白夫人缓缓道:我要你亲眼瞧见他的尸身,也正是因为我觉得对不起你\ldots。.``花无缺道:''你杀了他?"

白夫人瞪然长叹了一声,道,"不错,是我杀了他!花无缺踉跄而退,一个字也说不出来。

白夫人偷偷瞟了花无缺一眼道:``我那么样对你,只因我一心还在想挽回他的心,我为了他,不惜伤害任何人,不惜做出任何事\ldots。''她目中泪珠又一连串落了下来,几乎泣不成声。

花无缺道:``但你既然如此对他,为何又杀了他?''她忽然返身扑到花无缺怀里,放声痛哭道:``他竟丝毫不念夫妻之情,他\ldots\ldots 他。\ldots 他竟要杀我!''花无数竟没有推开她。

在这种情况下,他还是不忍推开一个在他怀中痛哭的女人──一个痛哭的女人,伏在一个狂笑者的男人怀里痛哭,旁边还例悬着一具鲜血淋漓的尸身,这情形之怪异诡秘,当真谁也描叙不出。

花无缺道:``所以\ldots\ldots 你就杀了他。''

白夫人道:``我本来虽然不惜为他而死的,但他真要来杀我时,我却再也忍受不住,二十年来历受的折磨和委屈,二十年来的冤苦和悲痛,全都在这一瞬间发作出来,我忍不住抽出了刀,一刀向他刺了过去!''她惨然接道:``我本也以为这一刀大概伤不了他,谁知他从未想到我会反抗,竟毫无防备之心,我这一刀,竟真的\ldots\ldots\ldots 真的将他刺死!''花无缺又能说什么?他笑声已渐渐嘶哑,腿已渐渐发软。他一身气力,竟已都被笑了出去!

花无缺忽然道:``过去的事,不必再提,我\ldots\ldots\ldots 我绝不会再恨你\ldots\ldots\ldots{}''白夫人道:``你原谅了我?''

花无缺点了点头,又道:``你话已说完了么?''白夫人道:``我该说的都已说了,你\ldots\ldots 你难道没有话要对我说?''花无缺道:``我\ldots\ldots 我只望你\ldots\ldots\ldots\ldots{}''

他自然希望白夫人能止住他这要命的笑声,但到了这地步,他竟然还是无法在女人面前说一句恳求的话。

白夫人静静瞧了他半晌,黯然道:``其实用不着你说,我也早该为你起出笑穴中那根销魂针的,但你方才用力过度,针已入穴极深,我也无力为你起出来了。''花无缺心里一阵绞痛,突然推开了白夫人转身而行,到了此刻,他知道自己的命运已注定,只有笑死为止!谁知白夫人却又拦住了他的去路,道:``你现在还不能走。''花无缺再也忍不住怒气上涌,却又勉强压了下去,道:``事已至此,你为何还要留下我?''白夫人道:``世上还有个能救你的人,我虽然无力救你,但都能将你的性命延长叁天,叁天内我就可以带你去找到那个人,如若想活下去,你就该有勇气去求他!你年纪轻轻,求人并不可耻,不敢活下去才真正可耻。''花无缺嘎声笑道:``我纵去求他,他也未必会救我,我又何苦\ldots\ldots\ldots{}''白夫人截口道:``我很了解那个人,只要你去,他一定会救你的。''她缓缓接道;``何况,你并不是去求他,你只不过去治病而已,一个人生了病而不去就医,这人并不可敬,反而可笑!''她翻来覆去的解说,花无缺心终于动了,一个人无论多么不怕死,有了生机时还是不愿意死的。

花无缺终于点了点头。对如此真挚的恳求,他永远都无法拒绝的。

\hypertarget{ux7b2cux516bux5341ux4e8cux7ae0-ux6e29ux67d4ux9677ux9631}{%
\chapter{第八十二章
温柔陷阱}\label{ux7b2cux516bux5341ux4e8cux7ae0-ux6e29ux67d4ux9677ux9631}}

花无缺和白夫人已走了,大厅里更沉寂、更阴森,曙色斜照着尸身上的鲜血,鲜血竟被映成了惨碧颜色。

这时江玉郎却悠然踱了进来,附掌笑道:``前辈端的是智计过人,弟子当真佩服得五体投地。''倒悬在梁上的``死人''突然哈哈一笑,道:``此计虽妙,也只有姓花的这种人才会上当,若换了你我,只怕再也不会如此轻易就相信女人的话。''这``死人''此刻竟已自粱上翻身跃下,右手拔起了自前胸刺入的刀柄,左手拔出了自后背刺出的刀尖。

原来这柄刀竟是两截断刀,贴在白山君身上的。

花无缺晕晕迷迷地坐在车子里,白夫人给他吃了种很强烈的宁神药,药力发作,他就昏昏欲睡。

幸好这车厢还舒服得很,他既不知道白夫人从哪里叫来的这辆车子,也不知道赶车的是谁,更不知道车马奔向何方。

一个垂死的人,对别人还有什么不可信任的!

叁天后的黄昏,车马上了个山坡,就缓缓停下,推开车窗,夕阳满天,山坡上繁花如锦,仿佛图画。

极目望去,大江如带,山坡后一轮红日如火,夕阳映照下舱江水,更显得无比的灿烂辉煌。

花无缺暗叹忖道:``我此番纵然无故而死,但能死在这样的地方,也总算不虚此行了。''只听白夫人长长叹息了一声,谣然道:``那人脾气甚是古怪,我\ldots\ldots 我不愿见他。''她开了车门,扶着花无缺下车,遥指前方,道:``你可瞧见了,那边的山亭?''只见红花青树间,有亭翼然,一缕流泉,自亭畔的山岩门倒泻而下,飞珠溅玉,被夕阳一映更是七采生光,艳丽不可方物。

花无缺九死一生,骤然到了这种地方,几疑置身天上,淡淡的花香随晚风吹来,他痴了半晌,才点头道:``瞧见了。''白夫人道:``你转过这小亭,便可瞧见一面石门藏在山岩边的青藤里,石门终年不闭,你只管走进去无妨。''花无缺暗叹忖道:``能住在这种地方的,自然不会是俗人,我有幸能与高人相见,本是人生乐事,只可惜我现在竟是如此模样。''花无缺道:``他叫什么名字?''

白夫人道;``她叫苏樱。''

花无缺暗叹道:``苏樱\ldots\ldots 苏樱\ldots\ldots 我与你素不相识,却要求你来救我的性命,你只怕会觉得可笑。''白夫人道:``你见着她后,她也许会问你是谁带来的,你只要说出我的名字''\ldots 对了,我的本名是马亦云。``花无缺道:''我记得。"

白夫人凄然一笑,道:``我此后虽生如死,你也不必再关心我,从今以后,世上再没有我这苦命的女人。\ldots.''她语声忽然停顿,转身奔上了马车,车马立刻急驰而去,花无缺怔了半晌,心里也不知是何滋味。

这女人害得他如此模样,但此刻他却只有感激,只有信任,绝没有丝毫怀疑和忿恨。

车马转过几处山坳,突又停住,山岩边、浓荫下,已来了叁个人,却正是铁萍姑、江玉郎和白山君。

花无缺已走入了那已被苍苔染成碧绿色的石门。

石门之后,洞府幽绝,人行其中,几不知今世何世。

花无缺只恨自己的笑声,偏偏要破坏这令人忘俗的幽静,他用力掩住自己的嘴,笑声还是要发出来。

走了片刻,人洞已深,两旁山壁,渐渐狭窄,但前行数步,忽又豁然开朗,竟似已非人间,而在天上。

前面竟是一处幽谷,白云在天,繁花遍地,清泉怪石,罗列其间,亭台楼阁,错综有致。

远远一声鹤唳,叁五白鹤,伴有一二褐鹿徜徉而来,竟不畏人,反而似乎在迎接这远来的侠客。

花无缺正已心动神移,那白鹤却已衔起了他衣袂,领着他走在青石路上,繁花深处。

只见─条清溪蜿蜓流过,溪旁俏生生坐着条人影。

她垂头坐在那里,似乎在沉思,又似乎在向水中的游鱼诉说着青春的易逝,山居的寂寞。

她漆黑的长发披散肩头,一袭轻衣却皎白如雪。

花无缺竟不由自主被迎客的白鹤带到了这里,岸上的人影与水中人影相互辉映,他不觉又瞧得痴了。

白衣少女也回过头来,瞧了他一眼。她不回头也罢,此番回过头来,满谷香花,却似乎顿然失去了颜色,只见她眉目如画,娇靥如玉,玲珑的嘴唇,虽嫌太大了,广阔的额角,虽嫌太高了些,但那双如秋月,如明星的眼珠,却足以补救这一切。

她也许不如铁心兰的明艳,也许不如慕容九的清丽,也许不如小仙女的妩媚\ldots\ldots 她也许并不能算很美。

但她那绝代的风华,却令人自惭形秽,不敢平视。

此刻,她眼中带着淡淡一丝惊讶,一丝埋怨,似乎正在问这鲁莽的来客,为何要笑得如此古怪。

花无缺的脸竟不觉红了起来,道:``在\ldots\ldots 在下花无缺,特来求见苏樱苏老先生。''白衣少女缓缓接着道:``我就是苏樱。''

花无缺这才真的怔住了。他本以为这``苏樱''既能治他的不治之伤,必然是江湖耆宿、武林名医、退隐林下的高手。他再也想不到这苏樱竟是个年华未满双十的少女。

苏樱眼波流转,淡淡道:``山居幽僻,不知哪一位是阁下的引路人?''花无缺道:``这\ldots\ldots 在下''

他实末想到白夫人竟要他来求这少女来救他的性命,面对着这淡淡的笑容,冷淡的眼花,他怎么好意思说出恳求的话来?

苏樱道:``阁下既然远道而来,难道连一句话都说不出么?''她话虽说得客气,但却似对这已笑得狼狈不堪的来客生出了轻蔑之意,嘴里说着话,眼珠却又在数着水中的游鱼。

花无缺忽然道:``在下误入此间,打扰了姑娘的安静,抱歉得很\ldots\ldots{}''他微微一揖,竟转身走了出去。

苏樱也末回头,直到花无缺人影巳将没人花丛,却突又唤道:``这位公子请留步。''花无缺只得停下脚步,道:``姑娘还有何见教?''苏樱道:``回来。''

这叁个字虽然说得有些不客气了,但语声却变得说不出的温柔,说不出的婉转,世上绝没有一个男子听了这种语声还能不动心。花无缺竟不由自主走了回去。

苏樱还是没有回头,淡淡道;``你并未误入此间,而是专程而来的,只不过见了苏樱竟是个少女后,你心里就有些失望了,是么?''花无缺实在没有什么话好说。

苏樱缓缓接道;``就因为你是这种人,觉得若在个少女面前说出要求的事,不免有些丢人,听以你虽专程而来,却又借词要走,是么?''花无缺又怔住了。

这少女只不过淡淡瞧了他一眼,但这一眼却似瞧入他的心里,他心里无论在想什么竟都似瞒不过这一双美丽的眼睛。

苏樱轻轻叹了口气,道:``你若是还要走,我自然也不能拦你,但我却要告诉你,你是万万走不出外面那石门的!''花无缺身子一震,还未说话,苏樱已接着道:``此刻你心肠已将被切断,面上已现死色,普天之下,巳只有叁个人能救得了你,而我\ldots。.''她淡淡接着道:``我就是其中之一,只怕也是唯一肯出手救你的,你若对自己的性命丝毫不知珍惜,岂非令人失望!''这是间宽大而舒服的屋子,四面都有宽大的窗户,此刻暮色渐深,明烛初燃,满谷醉人的花香,都随着温暖的晚风飘了进来,满天星光也都照了进来,苏樱支起了最后一扇窗户,那双纤纤玉手,似已白得透明了。

没有窗户的地方,排满了古松书架,松木也在晚风中散发出一阵阵清香,书架的间隔,有大有小,上面摆满了各色各样的书册,大大小小的瓶子,有的是玉,有的是石,也有的是以各种不同的木头雕成的。

这些东西摆满四壁,骤看似乎有些零乱,再看来却又非常典雅,又别致,就算是个最俗的人,走进这间屋子来,俗气都会被洗去几分。

但这屋子里却有个很古怪的地方,那就是这么大一间屋子里,竟只有一张椅子,其余就什么都没有了。

这张椅子也奇怪得很,它看来既不像普通的太师椅,也不像女子闺阁中常见的那一种。

这张椅子看来竟像是个很大很大的箱子,只不过中间凹进去一块,人坐上去后,就好像被嵌在里面了。

花无缺已走了进来。

他只觉这少女的话说来虽平和,但却令人无法争辩,又觉得她的话说来虽冷漠,但却令人无法拒绝。

苏樱已在那唯一的椅子上坐了下来。

花无缺只有站在那里,心里真觉得有些哭笑不得。

椅子的扶手很宽,竟也像个箱子,可以找开来的。

苏樱一面已将上面的盖子掀起,伸手在里面轻轻一拨,只听格"的一声轻响。

花无缺面前的地板,竟忽然裂了开来,露出了个地洞,接着,竞有张床自地洞里缓缓升起。

苏樱淡谈道:``现在已有床可以让你躺下了,你还要什么?''花无缺道:``我\ldots\ldots 我想喝茶。''

这句话本非他真正想说的,仅却不知不觉地从他嘴里说了出来,他实在也想试试这少女究竟有多大的本事.苏樱道:``呀,我竟忘了,有客自远方来,纵然无酒,但一杯茶的确是早该奉上的了。''她说着话,手又在箱子里一拨。

只听壁上书架后忽然响起了一阵水声,接着,木架竟自动移开,一个小小的木头人,缓缓从书架后滑了出来。

这木僮手上竟真的长着只茶盘,盘上果然有两只玉杯,杯中水色如乳,苏樱微微一笑,道:``抱歉得很,此间无茶,但这百载空灵石乳.勉强也可待客了,请。''花无缺忍不住道:``诸葛武侯的木牛流马,其巧妙只怕也不过如此了。''苏樱淡谈笑道:``孔明先生的木牛流马,用于战阵之上倒是好的,若用于奉茶待客,就未免显得太霸气了。''言下之意,竟是连诸葛武侯也末放在她眼里。

这时夜色已浓,星光已不足照人面目,书架里虽有铜灯,但还未燃起,花无缺忍不住又道:``难道姑娘不用动手,也能将灯燃起么?''苏樱道:``我是个很懒的人,懒人常会想出很多懒法子\ldots\ldots{}''她的手又轻轻拨了拨,铜灯旁的书架间,立刻伸出了火刀火石,``呛''的一声,火星四溅。

那铜灯竟真的被燃起了。,苏樱微笑道,``你瞧,我就算坐在这里不动,也可以做很多事的。''花无缺大笑起来──真的大笑起来,笑道:``以我看来,纵然是自己燃灯倒茶,也要比造这些消息机关容易得多,你这懒人怎地却想出这最麻烦的法子?''也不知怎地,他竟一心想折折苏樱的骄气,他本不是这样的人,此刻也许是笑得心里失去了常态。

苏樱却冷冷道;``像我这样的人,难道也会替你倒茶么?''花无缺道:``你为何不用个丫环女仆,这法子岂非也容易得多?''苏樱冷冷道:``我怕沾上那些人的俗气。''

花无缺又没有话说了,苏樱静静地凝注着他,缓缓接着道:``你说这些话,只因你觉得我太强了,所以想压倒我,是么?我不妨告诉你,世上没有人能压倒我的,我永远都是高高在上,你不必白费心机。''花无缺大笑道:``其实你只不过是个弱不禁风的女孩子,任何人一掌就可以推倒你。''苏樱道:``你居然看我不会武功,你的眼光倒不错。''花无缺道:``多谢。''

苏樱道:``你的武功很不错,是么?''

花无缺道:``还过得去。''

苏樱道:``但现在却是你求我救你,我并没有求你救我,由此可见,世上有很多事,并不是武功可解决的,人所以为万物之灵,只因为他的智慧,并不是因为他的力气,若论力气,连匹驴子都要比人强得多。''花无缺只觉怒气上涌,又要拂袖而去了,苏樱却就在这个时候嫣然一笑,盈盈走过来,柔声道:``现在,你老老实实地躺下去,我给你服下一瓶药后,你这可恶的笑声,立刻就可以停止了。''面对着如此可爱的笑容,如此温柔的声音,世上还有四个男人能发出火来。何况她说的这句话,又正是花无缺最想听的.花无缺并不是怕死,但这笑\ldots\ldots\ldots 他现在真想不出世上还有什么比``笑''更可怕的事。

笑声终于停止了。花无缺服了药后,已沉沉睡去。

突听一人娇笑道:``好妹子,真有你的,无论多么凶的男人,到了你面前都会乖得像只小狗\ldots\ldots{}''随着娇笑声走进的,正是白夫人。

苏樱瞧也没有瞧她一眼,淡淡道:``你为何现在就来了,你不放心我?''白夫人笑道:"只不过大家都知道妹妹你心高气傲,所以要我来求妹妹,这次委屈些,只要这小子说出了移花接玉的秘密,咱们立刻就将这小子杀了给妹妹出气\ldots\ldots{}

苏樱到这时才冷冷瞟了她一眼.道:``你觉得我对他这法子不好。''白夫人又赔笑道:``不是不好,只不过。\ldots 咱们现在是要骗他说出秘密,所以\ldots\ldots\ldots{}''苏樱冷冷道:``你觉得我应该对他温柔些,应该拍拍马屁,灌灌他迷汤,必要时甚至不妨脱光衣服,倒人他怀里,是么?''自夫人娇笑道;``反正这小子已快死了,就让他占些便宜又有什么关系。''苏樱已冷冷接道:``老实告诉你,我对他若真用这样的法子,他也是万万不肯说的,用这种法子来对付你的丈夫还差不多。''白夫人道:``但\ldots\ldots\ldots 但是\ldots\ldots{}''

苏樱道:``对付他这样的人,就要用我这样的法子,他才服贴,只因我这样对付他,他就万万想不到我有事求他,也就万万不会提防我,否则我怎会故意让他看出我不会武功?你总该知道我虽不屑去学这些笨玩意几,但要我装成一流高手的样子,我还是照样可以装得出的。''白夫人展颜笑道:``我现在才懂了,妹妹你的手段,果然非人能及。''苏樱懒懒的一笑,道:``你懂了就好,现在你们快躲远些吧,明天这时候,我负责令他老老实实的说出移花接玉的秘密。''

\hypertarget{ux7b2cux516bux5341ux4e09ux7ae0-ux81eaux4f5cux81eaux53d7}{%
\chapter{第八十三章
自作自受}\label{ux7b2cux516bux5341ux4e09ux7ae0-ux81eaux4f5cux81eaux53d7}}

第二天花无缺醒来时,笑声果然已停顿了,只觉得全身软软的没有丝毫力气,躺在床上竟连坐都坐不起来。

屋于里一个人也没有,四面花香鸟语,浓荫满窗。

突听屋子后一人在怪叫道:``出去出去,我说过我不要吃这捞什子的草根树皮,你为何总是要给我吃。''又听得苏樱柔声道:``这不是草根树皮,这是人参.那人又吼道,''管他是人参鬼参,我说不吃,就是不吃。,苏樱竟笑道:``也没见过你这样的人,好好好,你不吃,我就拿出去。''她这样的人也会受人家的气,花无缺听得实在有些奇怪,忍不住暗暗猜测,不知道给她气受的这位仁兄,究竟是怎么样一位人物。

过了半晌,只见苏樱垂着头走了进来。

她一走进屋子,立刻又恢复了她那种清丽脱俗、高高在上的神情,只不过手里还是捧着碗参汤。

花无缺暗叹道:"那人不吃,她难道就要拿来给我吃么?他现在虽的确很需要此物,但心里却暗暗决定,她若将这碗参汤拿来给他吃,他也是不吃的。

谁知苏樱却走到窗口,将那碗参汤都泼出窗外,她为``那位仁兄''做的东西,竟宁可拨掉,也不给别人吃。

苏樱已走到床边,淡淡道:``现在你是否觉得舒服多了?''花无缺这才又想起大笑不止时那种难以忍受的痛苦,才觉得现在实无异登天一般,不由得叹道:``多谢姑娘。''苏樱道:``现在你还不必谢我。''

花无缺动容道:``为\ldots\ldots 为什么''

苏樱道:``你现在笑声虽已停止,但那根针还是留在你气穴里,只不过被我用药力逼得偏了些,没有触入你的笑穴,但你只要一用力,旧疾还是难免复发。''花无缺吃惊道:``这\ldots\ldots 这便又该如何是好?''他现在宁可牺牲一切,也不愿再那么样笑了。

苏樱道:``这根针入穴已深,纵以黑石一类宝物,也难将它吸出来了,只有你自己用内力或许还可将它退出。''花无缺道:``但\ldots\ldots 但我现在连一丝气力都使不出来。''苏樱冷冷道:``你现在自然使不出来,你若能使得出来,也就不必来找我了。''花无缺道:``姑娘难道有什么法子,能令我真气贯通无碍。''苏樱淡淡道:``自然有的,此刻你只要将你所练内功的要决告诉我,我便要在旁助你一臂之力,使你真气贯通,逼出毒针。''她说的是那么轻松平淡,就好像这本是件最普通的事,好像只要她一盼咐,花无缺就会说出自己内功的秘密。

只因她知道自己只有这样说法,花无缺才不会想到这一切都是他们费了无数心力所做成的圈套。花无缺果然没有想到。

但``移花接玉''的行功秘诀,却是天下武林中最大的秘密,要他骤然说出来,他还是不免犹疑。

苏樱静静瞧了他半晌,悠然道,``你难道是怕我偷学你的内功么?''花无缺道:``在下并无此意,只不过\ldots\ldots{}''

苏樱淡淡一笑,道:``像我这样的人,若是有一份爱武的心,此刻纵非天下第一高手,只怕也差不多了。''她叹了口气,冷冷接道:``你们这些练武的人.总将自己的武功视若珍宝,又怎知这件事在我眼中看来,简直不值一文。''话未说完,她竟己拂袖而去。

花无缺失声道:``姑娘慢走。''

苏樱头也不回,冷冷道:``说不说虽由得你,但我听不听,还不─定哩。''花无缺叹了口气,道:``在下所练内功,名曰移花接玉,乃是\ldots\ldots{}''黄昏来临时,白山君夫妇已带着江玉郎和铁萍姑,在谷外的小亭里等了许久了,四个人面上已不禁都露出了焦急之色。

江玉朗忍不住笑道:``我实在想不出这位苏妨娘究竟是位怎么样的人?两位前辈竟对她如此倾倒。''白夫人笑道:``小伙子,我告诉你,你见了她时,只怕连话都说不出来了。''江玉郎笑道:"前辈未免也说得太玄了。难道在下竟如此他突然顿住语声;张大了嘴,说不出话来。

只见一个身披霓裳羽衣的仙子,在满天夕阳中,飘飘而来,一只红顶雪羽的白鹤昂然走在她前面,一只驯鹿,依依跟在她身后,温柔的暮风,吹乱了她的发丝,她伸出手来轻轻一挽\ldots\ldots 就是这么样轻轻一挽,已是令天下的男人都为之窒息,只是这么样─幅图画,已非任何人描叙得出。

她生得也许并不十分美,但那绝代的风华,却无可比拟,江玉郎只觉神魂惧醉,哪里还能说话。

白夫人含笑瞟了他一眼,迎了上去,笑道:``好妹子,你果然来了。''白山君也迎了过来,笑着道:``移花接玉的秘密,妹子你想必也问出来了。''苏樱道:``不错,我问出来了。''

白山君夫妇大喜道:多谢多谢\ldots。.``─苏樱冷冷道:''你现在还不必急着来谢我。``白夫人道:''那么\ldots\ldots 那么\ldots 妹子你难道已将移花接玉的诀窍写下来了么?``白山君道:''是是,妹子自然会写下来给我们的,老太婆你急什么?``苏樱淡谈道:''我现在也不准备写下来给你们。``白山君怔了怔,道:''那么\ldots。那么妹子你的意思是。\ldots{}``白夫人陪笑道:''妹子你要到什么时候才肯告诉我们呢?``苏樱道:''也许叁天五天,也许一年半载,也许十年八年,等我玩够了,我自然会告诉你们的。``白山君夫妇面面相觑,伍了半晌,白夫人陪笑道:''好妹子,你别开玩笑,若是等十年八年岂非急也把人急死了。``苏樱道:''你们急不急死,是你们的事,与我又有何关系。``自夫人着急道:''但\ldots\ldots 但妹子你不是已答应了我\ldots\ldots{}``苏田冷冷截口道:''我只答应你,要叫花无缺说出移花接玉的秘密,并未答应将这秘密告诉你。"白山君夫妇怔在那里,再也说不出话来。

苏樱缓缓转过身子道:``深山无以待客,我也不留你们了,你们还是回去吧。''白夫人道:``妹子请留步。''

苏樱淡淡道:``你们总该知道,我说出的话永无更改,何苦再多事。''白夫人叹了口气,道:``我只想问问那姓花的现在怎么样了?''苏樱皱眉道:``但你们只管放心,我也绝不会放了他,他这辈子只怕是再也休想见人了。''说完了这句话,她再也不回头,扬长而去。

白山君夫妇竟只是眼睁睁瞧着,谁也不敢拦阻。

过了半晌,铁萍姑叹了口气,道:``这位姑娘好大的架子。''江玉郎却道;``这丫头既然手无缚鸡之力,前辈为何不拿下她来。''白山君叹了曰气道:``老头子拿她当宝贝一样,谁若碰着她一根手指,老头子不拼命才怪,我夫妇现在还不想惹那老头子,也只好放她一马了。''自夫人也叹道:``何况,你莫看她手无缚鸡之力,但鬼心眼却还是真多,我们这几个人,倒真还未必能制得住她。''江玉郎微微一笑却不说话。

白山君瞧了他半晌,眼睛里忽然发出了光,道:``你莫非不服气?''江玉郎瞟了瞟铁萍姑一眼,微笑不语。

白山君重重一拍他肩头,大笑道:``好小子,我早就听说你对女人另有一套,你去试试,那丫头正在有些春心荡漾,说不定真的会告诉你。''江玉郎眼角瞟着铁萍姑,笑道:``在下对女人有何本事,前辈说笑了。''白夫人已搂住了铁萍姑,娇笑道:``好妹子,你就让他去吧,嫂子我保证他不敢对你变心,他若敢变心,嫂子我就叫小白将他的脑装咬下来。''江玉郎大摇大摆走进了山谷,晚风入怀,花香扑面,他身子只觉有些轻飘飘的,骨头仿佛没有四两重。

对于女人,他自觉已是老手,尤其这种年纪轻轻的小姑娘,只要他一出马,那还不是手到擒来。

更令他放心的是,这位姑娘连一点武功也不会,他就算不成功,至少也能全身而退,少不了半根汗毛。

何况,到了必要时,他还可以来个霸王硬上弓,那时生米煮成熟饭,还怕这姑娘不对他服服贴贴地俯首称臣。

更何况,就算这位苏姑娘脾气拗些,死也不肯说,反正便宜已让他占过了,吃亏的永远是别人,绝不会是他。他算来算去,越想越开心,简直开心得要飞上天了。

突听一人冷冷道;``你是谁?凭什么冒冒失失地闯人这里来?''原来他开心得过了头,竟未发觉苏樱早已在冷冷瞪着他。

一瞧见苏樱,江玉郎立刻做出一副可拎兮兮的模样垂下了头,嗫嚅着道:``在下冒昧闯入,实在无礼\ldots。.''苏樱道:"你既知无礼,此刻就该快些退出去。

江玉郎本已准备好满肚子花言巧语,本以为足可打动任何一个少女的心,谁知苏樱面前竟好像坚着道冰墙,令他根本无孔可入。

他满肚子话竟连一句也没有说出来,苏樱已冷冷转身走了回去,江玉郎眼珠子打转,突然大声道:``姑娘慢走,姑娘你好歹要救在下一命。''苏樱果然回过了头,皱眉道:``你若有病,就该去看医生,此间既未悬壶,也未开业,你来干什么?''江玉郎黯然道:``别人若是救得了在下的命,在下又怎敢来麻烦姑娘,只叹世间的名医虽多却都是欺世盗名之辈,他们若有姑娘的一成本事,在下\ldots\ldots 唉,在下也不必千里迢迢地进来打扰姑娘了。''常言道:``千穿万穿,马屁不穿'',这点江玉郎知道得比谁都清楚,苏樱面色果然大为和缓,嘴里却还是冷冷道;``你又怎知我能治得了你的病?是谁告诉你的?''江玉朗道:``这\ldots\ldots 这是在下的一位父执前辈,不忍见在下无救而死,才指点在下─条明路,而且将在下带来这里。''他头垂得更低,苦笑接道:``这位前辈不许在下说出他的名讳,但在下在姑娘面前,又怎敢说谎,指点在下前来的,就是白山君白老前辈和他的夫人。''苏樱面色果然更是和缓,摇头道:``这两口子倒真是会替我找麻烦。''江玉郎窥见她的面色,已知事情大为有望,于是打蛇随棍上,竟``噗通''跪了下来,道:``在下这病,别人反正也救不了的,姑娘今日若不肯\ldots\ldots\ldots 不肯可怜可怜我,我就索性死在姑娘面前吧。''苏樱一双明如秋水的眼睛,在他脸上凝注了半晌,轻轻叹了口气,道:``你倒真是会缠人\ldots。.''她嘴里说着话,竟又转身走了。

江玉郎大声道:``姑娘走不得,姑娘好歹也得救在下一命。''苏樱回眸一笑,道:``呆子,我走了,你难道不会跟我来么?''这一笑,已笑得江玉郎骨头都酥了,这一声``呆子'',更叫得江玉郎心头痒痒的,也不知该如何是好。

苏樱分手拂柳,又将他带到那间明亮的敞轩中,烛火已燃,那张床也还在那里,但床上的花无缺,却已不知何处去了。

只听苏樱道:``现在,你不妨告诉我,你得的是什么病?是哪里觉得不舒服?''江玉郎哪里有什么病,情急之下,脱口道:``在下\ldots 在下肚子疼得很厉害。''苏樱忽然沉下了脸,冷冷道:``但我瞧你却不像疼得很厉害的样子。''江玉郎怔了怔,若是换了别人,此刻只怕已要脸红了,但江玉郎究竟不傀为说慌的名家,眼珠子一转,立刻陪笑道,``在下在姑娘面前,怎敢放肆,何况,无论是谁,见到姑娘这样天仙般的人物,也会将疼痛浑然忘却了的。''这句马屁看来又拍得恰到好处。

苏樱展颜一笑,道:``你看到我既然就能止疼,那还要医什么?''江玉郎涎脸笑道:``在下若能常伴姑娘左右,疼死也无妨,只不过\ldots\ldots 只不过\ldots\ldots{}''他内功中已有很深的火候,此刻在暗中运气一逼,额角上立刻有一连串黄豆般大小的汗珠流了下来。

苏樱竟似也有些着急,道:``你瞧你,疼成这样子,还不快躺下来。''她轻轻扶起江玉郎的手,江玉郎``装羊吃老虎'',竟整人都向她身上依了过去,在她耳朵边吹着气道:``多谢姑娘。''苏樱居然也不生气,江玉郎胆子更大,一双手也按了上去,谁知苏樱却一扭腰逃了,哮着嘴道,``你若不乖乖的躺上床,我就不理你了。''江玉郎赶紧道:``是是,我听话就是。''

苏樱``噗哧''一笑,道:``听话的才是乖孩子,姐姐买糖给你吃。''她轻嗔薄怒,似嗔似喜,当真是风情万种,令人其意也消。

江玉郎心里更痒得也不知该如何去搔才好,却指着肚子道:``我疼''\ldots 疼得更厉害了,你快来\ldots\ldots 快来瞧瞧。``苏樱果然走过来道:''你哪里疼?"

江玉郎拉起她的手来揉肚子,道:``这里\ldots。就在这里。''苏樱一双柔若无骨的纤手竟真的在他肚子上轻轻揉着,柔声道:``你现在觉得好些了么?''江玉郎闭起眼睛,道:``好些了\ldots\ldots\ldots 但你不能停手,一停手我就疼。''苏樱的手竟真的不停地揉着,不敢停下。

江玉郎心里又是得意,又是好笑,暗道:``别人都说这位苏姑娘是如何如何的厉害,但在我看来,也不过是个初解风情的黄毛丫头而已,只要我略施妙计,还不是一样立刻手到擒来。''忽觉一阵如兰如馨的香气扑鼻而来,苏樱一只纤纤玉手,已到了他嘴边,手里还拿着粒清香扑鼻的丸药,柔声道:``这是我精心配成的清灵镇痛丸,不但可止疼,而且还大补,你现在吃下去,肚子立刻就不疼了。''江玉郎摇头道:``我不吃。''

苏樱皱眉道:``为什么不吃?''

江玉郎道:``我一吃,肚子就不疼了,我肚子若是不疼,姑娘岂非就不肯\ldots\ldots 不肯替我揉了。''苏樱嫣然一笑,道:``小坏蛋\ldots\ldots 好,你吃下去,我还是替你揉的''这一声``小坏蛋''更将江玉郎的魂都叫飞了,索性撒娇道:``这药苦不苦?''苏樱抿嘴笑道:``这药非但不苦,而且还甜得很。简直就像糖一样,来,乖乖的张开嘴,我喂你吃下去。''江王郎闭着眼张开嘴,心里真是舒服极了。

突听一人在远处大喊大叫,道:``酒呢?没有酒了,苏樱小丫头,快拿酒来。''苏樱皱了皱眉头,竟停下了手,道:``你乖乖的躺在这里,我去去就来。''她竟似有些着急,话未说完,就匆匆走了出去,又回头道,``你若站起来乱跑,我可就不理你了。''远处那人又在大叫道:``姓苏的丫头,你耳朵聋了么?怎地还不来。''苏樱竟笑道:``来了来了,我这就替你拿酒去。''江玉郎心里暗暗奇怪:``这位苏姑娘倒也有意思,别人都对她那么样恭敬,她却冷冰冰的爱理不理,这人一日一声丫头,简直没拿她当人,她反而像是服气得很,却不知这位仁兄究竟有何本事,竟能令她如此听话。''他真想爬起来,偷偷去瞧瞧,但转念一想,现在事眼看已有望,莫要轻举妄动坏了大事。

于是他索性又闭起眼睛,想到这如花似玉的美人,眼看已在抱,那天下武林中人人垂涎的秘密,眼看已快到手了。

他几乎忍不住要笑了出来,喃喃道:``白山君呀白山君,你以为我听到这秘密后,会告诉你么?你若真的以为我会告诉你,你可就是天下第一个大笨蛋了。''只听一人笑道:``你说谁是天下第一个大笨蛋?''江玉郎暗中一惊,但瞬即笑道:``谁若敢说姑娘是丫头,谁就是天下第一个大笨蛋。''苏樱笑道:"那只过是个老糊涂、老酒鬼,咱们犯不不理他。

江玉郎听得一个``老''字,已大是放心,听得``咱们''两个字,更开心得忍不住笑出来,大笑道:``是是是,咱们不理他。''苏樱道;``你笑得这么开心,肚子不疼了么?''江玉郎立刻皱起了眉头,道:``疼''\ldots 疼得更厉害了,求姑娘再替我揉揉。``苏樱抿嘴一笑,又替他揉起肚子,江玉郎只觉得全身发软,简直是要登天,揉了半响,苏樱缓缓又道:''其实,你心里本认为我才是天下第一个大傻蛋,是么?``江玉郎一怔,笑道:''我怎敢这么想,我难道晕了头了?``苏樱缓缓道:''你认为我很年轻,又没见过什么男人,一定很容易上男人的当,你觉得你对女人很有一手,略施妙计,就可以令我投怀于抱,而且将那移花接玉的秘密,老老实实的告诉你\ldots\ldots\ldots 是么?``江玉郎这才大吃一惊,强笑道:''哪\ldots\ldots 哪有这样的事,姑娘你\ldots\ldots 你太\ldots。.``苏樱淡淡截口道:''何况,你知道我丝毫不会武功,就算看透了你的心意,也没法子拿你怎样,所以你胆子就更大了,是么?``江玉郎大惊之下,想翻身跃起,但不知怎地,全身竟软软的连一丝力气都没有了,不禁大骇道:''姑娘千万莫要错怪了好人,在下绝无此意。``苏樱道:''你不但有这意思,而且到了必要时,还想来个霸王硬上弓,反正我也无力抗拒,那时生米煮成熟饭,我还能不乖乖的听话么?``江玉郎肚子里有几条蛔虫,她竟都能数得清清楚楚,江玉郎一面听,一面流汗,颤声道:''姑娘不能冤枉我,我若有此意,就叫我不得好死。``苏樱嫣然一笑,道:''到了这时,你还想你能好死么?``江玉郎大骇道:''我\ldots 我\ldots 姑娘\ldots\ldots 哎哟!"

苏楔的手还在替他揉着肚子,此刻突然用力─按,江玉郎大吼一声,疼得全身都出了冷汗。

他竟也不知道自己怎会变得如此怕疼的。

苏樱笑道:``你要我替你揉肚子,我就替你揉肚子,你可知我为何如此听话?''江玉郎颤声道:``在\ldots\ldots 在下不知道,求姑娘莫要揉了吧\ldots\ldots{}''苏樱笑道,``现在你觉得疼了,就要我莫要揉了么,但我知道你的肚子很疼,病很重,怎能忍心不替你揉。''江玉郎大叫道:``我\ldots 没有病\ldots 一点病也没有。''苏樱脸色一沉,道:``你没有病?为何要骗我?''她的手又一按,江玉郎大呼道:``我有病,有病\ldots\ldots。''苏樱展额笑道:``对了,你不但有病,而且病得很重,而且越来越重,到后来纵然是一片纸落在你手上,你也会觉得有如刀割。''江玉郎大骇道:``求\ldots。求姑妨娘救我,救救我\ldots{}''."苏樱的手还是在轻轻地揉着,但江玉郎却丝毫也不觉得舒服了,他只觉全身骨头,都像是要被揉散。

只听苏樱叹道:``现在我也没法子救你了,只因我方才拿错了药,拿给你吃的,不是清灵镇痛丸,面是百病百疼催生丸\ldots\ldots{}''江玉郎大骇道:``百病百疼催生丸?这是什么药?''他实在一辈子也没听过这样的药名。

\hypertarget{ux7b2cux516bux5341ux56dbux7ae0-ux610fux5916ux4e4bux53d8}{%
\chapter{第八十四章
意外之变}\label{ux7b2cux516bux5341ux56dbux7ae0-ux610fux5916ux4e4bux53d8}}

苏樱哭道:``只因有病的吃了这药,病势立刻加重十倍,没有病的吃了这药,也立刻百病俱生,而且全身都疼得要命\ldots。''江玉郎嘶声道:``妨娘\ldots 在下与姑娘无冤无仇,姑娘为何要如此害我。''苏樱笑道:``你不是说已病入膏肓了么!我不愿将你当成个专门说谎的无耻之徒,所以好心给你吃下这药,你真的生了病,就不算说谎了\ldots 而且,我还怕你病得太慢,所以又好心替你揉肚子,帮药力发散。''她叹了口气,悠然接道:``你看,我对你这么好,你还不谢谢我。''江玉郎又惊又怕又疼,头上汗如雨落,颤声道:``苏姑娘\ldots.苏前辈,我\ldots\ldots 小人现在才知道你的厉害了,求求你瞧在白山君夫妻的面上,饶饶我吧。''苏樱道:``哎哟,我倒忘了你是白山君夫妇的朋友。''江玉郎道:``姑\ldots 姑娘千万忘不得的。''

苏樱叹道,``不错,你既是他们的朋友,我就不能眼见你病死在这里了,我好歹也得救救你\ldots\ldots 只可惜这药并非毒药,所以也没有解药,你又吃了下去\ldots\ldots 这怎么办呢?''江玉郎道:``求求姑娘,姑娘一定有法子的.''苏樱拍掌道:``有了,我想起个法子来了。''

江玉郎大喜道:``什么法子?''

苏樱道:``我只要剔开你肚子,将那药丸拿出来。''江玉郎骇道:``剖开我肚子?''

苏樱柔声道:``但你放心,我一定会轻轻的割,轻轻地将那药丸拿出来,你一定连丝毫痛苦都没有。''江玉朗忍不住苦着脸道:``肚子剖开,人已死了,还会觉得痛么?''苏樱抚掌笑道:``你真是个聪明人。''

她格格笑道:``这就是我们家祖传的止疼秘方,手疼割手,脚疼割脚,头疼切脑袋,肚疼剖肚子,担保你妙手成春,药到命除。''她一面说,一面又走了开去,喃喃道:``刀呢\ldots 刀呢\ldots\ldots 江玉郎大骇喊道:''姑娘\ldots 姑娘千万莫要\ldots\ldots{}``苏樱道:''你不要我替你治病了么?"

江玉郎嘎声道:``不要了,不要了。''

苏樱叹了口气,道:``你既不要,我也没法子,但这可是你自己的主意,不能怪我不救你,对不对?''江玉郎道:``对对对,对极了。''

苏樱道:``现在你可知道,谁是天下第一个大傻蛋么?''江玉郎苦着脸道;``是我,我就是天下第一个大傻蛋,大混帐,大\ldots\ldots{}''他竟忍不住放声痛哭了起来。

苏樱笑道:"没出息,这么大个男人还哭,真叫我见了难受她的手又在那椅子的扶手里轻轻一按.那张床竟忽然弹了起来,将江玉郎整个人都弹起,床后却露出个地洞,江玉郎惊呼一声,人已落在洞里,像坐滑梯般滑了下去。

苏樱微微笑道:``一个哭,一个笑,这两人倒是天生一对,就让你们去作作伴吧\ldots\ldots{}''语声中床又落下,地洞也合起。

只听远处那人又大叫道:``一个人喝酒没意思,姓苏的丫头,你还不过来陪陪我。''苏樱叹了口气,苦笑道:``他才真是我命中的魔星,我为什么看见了他就没了主意\ldots。.''这敞轩后繁花如锦,小山上佳木葱笼,山坡下有个山洞,里面灯光亮如白昼,布置得比大户人家的少女闺房还要舒服。

但洞口却有道铁栅,铁栅比小孩的手臀还粗。

此刻山洞里正有个人坐在桌子旁一杯杯地喝着酒,只见他蓬着头,赤着脚,身上穿着件又宽又大的白袍子,看来滑稽得很。

他脸冲着里面,也瞧不清他的面目,只听他不住大喊道:``姓苏的丫头,你还不来我就\ldots。.''苏樱柔声道:``我这不是来了么?也没见过你这么性急的人。''那人一拍桌子,大吼道:``你嫌我性子火急了么?我天生就是这样的脾气,你看不惯最好就不要看!''苏樱垂下了头,眼泪都似要掉了下来。

那人却忽又一笑,道:``但我若不想你,又怎会急着要你来,别人常说,一日不见,如隔叁秋,但我简直片刻也不能不见你。''苏樱忍不住破涕为笑,咬着嘴唇笑道:``我知道我这条命,迟早总是要被你气死的。''那人大笑道:``千万死不得,你死了,还有谁来赔我喝酒?''他大笑着回过头来,灯光照上了他的脸。

只见他脸上斑斑驳驳,也不知有多少刀疤,骤看像是丑得很怕人,但仔细一看,他脸上却像是连一条刀疤也没有了,只觉他眼睛又大又亮,鼻子又直又挺,薄薄的嘴唇,懒洋洋的笑意\ldots.这人不就是那令人割不断、抛不下、朝思夜想、又爱又恨的小鱼儿吗?

苏樱瞧见小鱼儿转过身,她眼睛里也发着光,柔声笑道:``你既然要我来陪你喝酒,为什么不把酒杯拿来''小鱼儿眨着眼睛,笑嘻嘻道:``你既然要来陪我喝酒,为什么不进来''苏樱摇了摇头,笑道:我在外面陪你喝,还不是一样么?``小鱼儿正色道:''那怎么会一样,你一定得坐在我旁边,陪我说话,我的酒才喝得下去,我方才不是说过,我有多么想你。苏樱眼波流动,面上微微现出一抹红晕,垂头笑道:``反正我在外面,你一样还是能看得到我的。''小鱼儿忽然跳了起来,大骂道:``你这臭丫头,死丫头,谁要你来陪我喝酒,你快滚吧!''苏樱居然丝毫也不生气,却笑道:``反正你拍我马屁,我也不进去,你骂我,我还是不进去的。''小鱼儿吼道:``你为何不进来难道怕我吃了你?我又不是李大嘴。''苏樱笑道:``我知道你不吃人的,但我一开门进去,你就要乘机冲出来了,是么?''小鱼儿撇了撇嘴,冷笑道:``你又不是我肚子里的蛔虫,你怎知道我的心意?''苏樱只是轻轻的笑,也不说话。

小鱼儿在里面绕了几个圈子,忽又在她面前停了下来,笑道:``我知道你是个好人,而且对我很好,我骂你,你也不生气,但你为什么偏偏要将我关在这里呢?''苏樱幽幽道:``你是个爱动的人,性子又急,我若不将你关起来,你一定早就走了,但你的伤却到现在还没有好,若是一走动,就更糟了。''小鱼儿笑道:``原来你还是一番好意。苏樱嫣然一笑,谁知小鱼儿又跳了起来,大吼道:''但你这番好意,我却不领情,我是死是活,都不关你的事,你莫以为你救了我,我就该听你的话,感激你\ldots。.``苏樱垂下了头,道:''我\ldots\ldots 我并没有要你感激我,是么?``小鱼儿又在里面兜了七八个圈子,忽又一笑,道:''说老实话,你为什么要救我,我可真有些弄不清。``苏樱默然半晌,悠悠道:''那天,我恰巧到天外天去\ldots\ldots{}``她刚说了一句,小鱼儿又跳起脚来,怒吼道:''什么天外天,那里只不过是个老鼠洞而已。``苏印哧─笑道:''好,就算是老鼠洞,你也不必生气呀。``小鱼儿大声道:''我为何不生气,现在我一听老鼠两个字就头疼。``苏樱道:''但这两个字是你自己说的,我并没有说。``小鱼儿扳着脸道:''我听人说都头疼,自己说自然头更疼了。``苏樱忍住笑道:''你不会不说么,又没有人强迫你说。``小鱼儿道:''我不说又嘴痒,我\ldots."

说到这里,他自己也忍不住要笑了起来,自己也觉得自己实在是蛮不讲理,转过头,忍住笑道:``你为何不说下去?''苏樱道:``那天我恰巧到天\ldots\ldots 到老\ldots\ldots{}''

她忽然发觉自己既不能说``天外天'',也不能说``老鼠''两个字,自己也不觉好笑起来,只有咬着嘴唇道:``那天我到那地方去,本是去拿要他们替我采购药草,谁知却见到了你,你刚巧也到了那里。''小鱼儿道:``我会到那鬼地方去,算我倒霉,你遇见我,也算你倒霉。''苏樱一笑,道:``但那天我看见你的时候,你却连一点倒霉的样子都没有,你身上穿的衣服虽然破破烂烂,但那神气却像是穿着世上最华贵、最好看的衣服。''小鱼儿坐了下来,跷起了脚,道:``还有呢?我不但很神气,长得也不难看呀.''苏樱抿嘴笑道:``不错,你长得的确不难看,尤其是你的眼睛''小鱼儿大声道:``我的眉毛,我的鼻子,我的嘴难道就不好看么?''苏樱吃吃笑道:``你从头到脚,没有一个地方不好看。\ldots 这够了么?''小鱼儿喝了口酒,笑道;``嗯\ldots\ldots 这还差不多\ldots\ldots{}''苏樱已笑得喘不过气来:``我本不是个很容易吃惊的人,但我见到你时,我\ldots\ldots{}''小鱼儿大笑道:``你见到我时,眼睛都直了,嘴也张大了,活像瞧见了大头鬼似的,那时我真想往你嘴里塞个大鸡蛋。''苏樱``噗哧''一笑,道:``那只因我心里实在奇怪。你怎会找到\ldots\ldots\ldots\ldots 找到那地方的。''小鱼儿默然半晌,皱起了眉头,道:``那其中自然有个缘故,但你\ldots\ldots 你却不必知道,因为无论我是怎会找到那鬼地方的,都不关你的事。''苏樱叹了口气,道:``还有令我奇怪的是,你到了那里,竟一点也不害怕。''小鱼儿冷笑道:``那有什么好害怕的,比那地方更恐怖、更骇人的地方,我都见得多了。''苏樱道:``但你见过比\ldots\ldots 比魏无牙更可怕的人么?''小鱼儿像是忽然说不出话了,那只拿着酒杯的子,也像是有些发抖,连杯子里的酒都快溅了出来。

苏樱又叹了口气,道:我从七八岁的时候开始,差不多每隔两叁天就要见他一面,但直到现在为止,我一见他的面,还是好像要发抖。``小鱼儿将酒杯摔在桌上,大声道;''我不是怕他,我只是觉得恶心,他那张脸:那副模样看来简直不是人\ldots\ldots 他看来简直就像是老天用一只老鼠、一只狐狸、一匹狼斩碎了,再用─瓶毒药、一碗臭水揉在一起造成的活鬼。``苏樱忍不住又笑了,道:''你这张嘴可真缺德,但你实在也将他形容得再妙也没有了。``小鱼儿''哼``了一声,忽也笑了,道;老实说,我见到你们时,心里真觉得有些好笑,你们两人坐在一起,看来就像香酥鸽子旁摆着堆臭狗屎,世上再也找不出比这更不相配的事了。''苏樱垂下了头,默然半晌,幽幽道:``他虽然不是个好人,但对我\ldots\ldots 对我却一直很好。这十年来,他简直没有拂过我的心意,我无论要做什么,他全都答应。''小鱼儿道:``哼,丑八怪拍小美人的马屁,那本是天经地义的事。''苏樱又默然半晌,展颜一笑,道:``他看见你忽然闯来,而且还有胆子瞪着眼睛向他穷吼,他实在也骇了一跳,这么多年来,我还没有见过有人能令他脸上变了颜色的,但他瞧见你时,却迦眼睛都好像发绿了。''小鱼儿仰首狂笑道:``他只怕本以为洞口的那些破铜烂铁能够拦得住我的,谁知那些东西在我眼里,简直就像是小孩子玩的把戏。''苏樱道:``他就是因为你能闯下他布下的十八道机关消息,所以才对你有些顾忌,所以你虽然对他穷吼,他还是坐着不动小鱼儿截口道:''他既然已知道我的厉害,为何还要令那些蠢才来送死。,苏樱道:``他自己不动手,却要他门下弟子去动手,为的只是想先试出你的武功来,他也明知那些人不会是你对手的;''小鱼儿又大笑道:``你以为我不知道他心意?所以我才偏偏不让他瞧出我的武功路数来。''苏樱一笑,道:``魏无牙实也未想到连他都瞧不出你的武功路数来。''小鱼儿道:``所以他就一直坐着不出手,是么?''苏樱道:``嗯。''

小鱼儿道:``他就能眼瞧着那些人被我活活打死?''苏樱叹道:``那些人虽也是他的门徒弟子,但却都还未能登堂入室,并非他心爱的那几个,何况,别人的死活,他根本就不放在心上,只要对他自已有利,就算要他将他儿子的脑袋切下来送人,他也不会皱一皱眉头的。''小鱼儿怒道:``我早就知道这家伙不是人!谁知他竟连畜生都不如。''苏樱叹道:``谁知后来你还是上了他的当了。''小鱼儿瞪眼道:``你懂得什么,若论斗智,就凭他还差得远哩。''苏樱道:``但是你\ldots\ldots 你还是\ldots\ldots{}''

小鱼儿也叹了口气,道:``斗智他虽斗不过我,斗力我可就斗不过他了,不瞒你说,我实未想到这畜生的武功,竟有那么厉害。''苏樱道:"据说在二十年前,他武功已可算是天下有数的几个高手之一,十二星象能横行江湖,可说全靠他一人之力\ldots\ldots{}

小鱼儿道:``他这倒不是吹牛,十二星象中的人,我也见过两个,武功比起他来,简直连他一成都赶不上。''苏樱道:``二十年前,他本已以为可以无敌于天下,后来遇着了移花宫主,大约吃了个大亏,所以才闭门洗手,躲到这里来,这二十年他日日夜夜的苦练武功,据他说,现在就算移花宫主姐妹两个一起来,他也未必怕她们了。''小鱼儿大笑道:``他这就是吹牛了,莫说移花官主自己来,就算移花宫主的徒弟来了,也管叫他吃不了,兜着走。''苏樱眼波流动,道:``移花宫主有几个徒弟?''小鱼儿道:``女的我不知道,男的却只有一个。''苏樱目光凝注着他,道:``你\ldots\ldots 你和他是朋友?''小鱼儿长叹道:``本来是可以和他交朋友的,但现在\ldots\ldots 现在却好像非和他做仇人不可。''苏樱嫣然一笑,道:``很好,好极了!''

小鱼儿瞪眼道:好什么?"

苏樱含笑垂下了头,不再说话。

小鱼儿自然不懂她的心意,更不知道花无缺眼见就快死了,瞪着眼瞧了她半晌才接着道;``我也知道他要我坐下,本来是想以诡计害我的,我只怕和他斗力,不怕和他斗智,所以也就立刻坐了下来。''苏樱又笑了笑道:``他那张椅子上,本有机关,只要他的手一按,坐在椅子上的人就要掉下刀坑去,纵然武功再强,只怕也活不成了。''小鱼儿道:``真的这般厉害?''

苏樱道:``他不但武功颇高,旁门杂学更是样样精通,他以为只要发动机关,你必死无疑,所以才不愿费力和你动手。''小鱼儿道:``他自己只怕也想不到他发动机关之后,我还是好好的坐着末动。''苏樱道:``那时不但他奇怪,我也奇怪极了。''小鱼儿大笑起来,道:``老实告诉你,我早己看出那张椅子上有古怪了,所以我看来好像已坐下,其实我的屁股根本就没挨着椅子。''苏樱嫣然笑道:``你真是鬼灵精。''

小鱼儿道:``我借此骂了他两句,谁知这老畜牲竟比我还沉不住气,竟跳起来就和我动手,我一见他出手,就知道要糟了。''苏樱道:``但你还是和他拼了好一阵,那一场大战,我简直从来也没有见过。''小鱼儿叹道:``这老畜牲倒的确有两下子,不但武功高,招式狠,而且出手又贼又滑,我就算武功比他高,也占不了他的便宜。''苏樱道:``他自己也这么样说,就算武功比他高的人,也未必能胜得了他,只因他无论使出什么招式,自己先立于不败之地。''小鱼儿道:``就因为他出力还是先留叁分余力,所以我才能和他支持那么久,但我心里也知道,只要我稍一不慎,就得死在他手里。''苏樱叹道:``他手下的确从来没有活口。''

小鱼儿道;``我既然知道迟早总要遭他的毒手,连逃也逃不了,心里就在打主意了,我就算要死,也不愿死在这种人手里。''苏樱道:``所以你就\ldots\ldots 你\ldots\ldots{}''

小鱼儿道;``所以我就一步步向后退,退到墙角。''苏樱道:``那墙角也有个机关,只要你踩到那里,立刻有飞刀射出。''小鱼儿笑道:``你以为我不知道么?''

苏樱讶然道:``你知道?你知道为何还要去?''小鱼儿大笑道:``我就因为已瞧出墙角有机关,就因为已瞧出他要将我诱到那里去,所以才故意好像被他逼得无路可退,一脚踩上那机关,等飞刀射出来时,我也故意装成无法闪避的模样去接那一刀。''苏樱竟也愕住了,失声道:``为什么?你为什么故意要上这个当。''小鱼儿笑道:``只因我不愿死在他手上。''

苏樱道:``但你可知道,那飞刀上也有剧毒?''小鱼儿道:``飞刀上就算有毒,也比他那双鬼爪子好多了,我若被他那鬼爪子抓中,必死无疑,所以我才宁可去挨一刀。''他大笑接道:``我算准他见我挨了一刀后,就不会再动手了,否则我只有和他打到死为止。\ldots 现在你总该知道,我并不是真的上了他的当吧。''苏樱瞧了他半晌,长长叹了口气:``若论应变时智计之灵巧,手段之奇秘,心眼儿动得之快,世上只怕真没有几个人比得上你。''小鱼儿板起脸道:``你难道还不晓得我是天下第一个聪明人么?''

\hypertarget{ux7b2cux516bux5341ux4e94ux7ae0-ux8272ux80c6ux5305ux5929}{%
\chapter{第八十五章
色胆包天}\label{ux7b2cux516bux5341ux4e94ux7ae0-ux8272ux80c6ux5305ux5929}}

苏樱``噗哧''一笑,过了半晌,悠悠道:``但你若非遇见我,你这天卜第─的聪明人,还是一样活不了,你\ldots。,你该怎么样感激我才是。''谁知小鱼儿却冷笑道:``你纵然不救我,也还是会有人来救我的。''苏樱又怔了怔,道:``谁?''

小鱼儿道:``张叁李四,王二麻子,我现在也不知道是谁,但到时候总会有人救我的就是,你看我像个短命的人么?''苏樱轻咬着嘴唇,道:``如此说来,我倒是不该救你的了。''小鱼儿道:``哼。苏樱道:''我本该等着瞧瞧,看有哪个笨蛋会来救你。``小鱼儿大笑道:''不错,来救我的都是笨蛋,你说的简直对极了。``苏樱跺脚道,''你\ldots\ldots\ldots 你\ldots\ldots\ldots"

小鱼儿跷起了脚,悠然笑道:``何况,就算没有笨蛋来救我,我也照样死不了的。好人不长命,坏蛋活千年,这句话你难道没有听过?''苏樱终了还是忍不住笑了,吃吃笑道:``你呀\ldots\ldots 你这小坏蛋,可真叫人见了没法子.''小鱼儿笑嘻嘻道:``说来说去,你实在不该救我的,现在你自己只怕都有些后悔了。''苏樱道:``后悔?─\ldots 我无论做什么事,从来都没有后悔过。''她缓缓接道:那日你身中毒刀之后,没多久就晕迷不醒,魏无牙算定你必死无疑,就要叫人将你抬出去喂老鼠。``小鱼儿吐了吐舌头,失声道''喂老鼠?``苏樱道:''嗯。``小鱼儿全身都痒了起来,却还是笑道:''好运气呀好运气``苏樱嫣然道:''你如今也知道你自己运气不错了么?``小鱼儿笑道:''不是我运气不错,而是那些老鼠运气实在不错。``苏樱楞然道:''你说老鼠的运气不错?"

小鱼儿正色道:``我全身上下,里里外外,连筋带皮带骨头,早就已坏透了,老鼠若是真的吃了我,不上吐下泻才怪。''他话未说完,苏樱已笑得弯下了腰。

小鱼儿道:``你觉得很开心么?''

苏樱笑着笑着,忽然不笑了,痴痴地怔了半晌,竟然幽叹道:``你可知道,我从生下来到现在,从没有这么样开心的笑过。''她眼圈忽然红了,垂下头,不再说话。

小鱼儿瞧了她很久,耸了耸鼻子,笑道:``你莫难受,我嘴里虽这么样说,心里还是很感激你的。,苏樱垂首道:''我知道你嘴里虽说得坏,其实心里。\ldots 心里却是善良的,但有些人嘴里虽说得漂亮.一颗心却比什么都丑恶。``小鱼儿仰首大笑道:''你以为你很聪明?你以为你能看透别人的心事?``苏樱摇了摇头,不说话了,过了半晌,才缓缓接道,''那日我本来也没有机会救你,但魏无牙恰巧来了个很重要的客人,就将那人迎入里面说话去了,因为他─向不愿意别人见着我。``小鱼儿笑道:''只因为人人都比他生得漂亮,他当然怕别人将你抢走。``这句话像又触动了苏樱的心事。她又垂下头,又过了半响才接着道:''他离开之后,我才能叫他那两个小徒弟将你抬到这里来,我对他们说,有种花一定要用死人做肥料才会开得鲜艳\ldots\ldots{}

小鱼儿笑道:``这种话那两个笨徒弟虽相信,魏无牙难道也会相信么!''苏樱道:``他的徒弟都对他畏之如虎,见了他,简直连一个字都不敢说。''小鱼儿伸了个懒腰,道:``你难道是觉得我这么聪明的人死了实在可惜,所以才救我的。''苏樱一笑,道:``我也不知道究竟是为了什么才会救你,也许\ldots\ldots 也许是因为你见了魏无牙时那种神气,也许是因为你中了毒刀后,还瞧我一笑\ldots 临死前还要对我笑的人,我怎么能眼看他真的去死。''小鱼儿抚掌大笑道:``我那一笑,笑得果然有用极了。''苏樱道:``难道\ldots\ldots 难道你对我那─笑,就是为了要我救你的?''小鱼儿竟嘻嘻道:``否则我人都快死了,还有什么好笑的。''苏樱咬着嘴唇道:``你\ldots 你为什么不骗骗我,就说是因为见了我之后,神魂颠倒,所以才不觉笑了出来\ldots\ldots{}''小鱼儿道;``现在你既已救了我,我为什么还要骗你,何况\ldots\ldots 你生气时的模样,比笑的时候还要好看得多。''苏樱忍不住又``噗哧''一笑,道:``你究竟是为了什么去找魏无牙的?''小鱼儿道:``我那天不早就说过了么?\ldots\ldots 我去找魏无牙,只因为要去救我的朋友。''苏樱道:``你怎知道你的朋友在那里?''

小鱼儿道:``我的朋友在一路上都留下了暗记,标志说是到那\ldots\ldots 那见鬼的天外天去了。''苏樱默然半晌,缓缓道:``但我却可以告诉你,这叁个月来,根本就没有一个人到过那地方去,只有你\ldots\ldots 你是第一个闯进那地方去的人!''小鱼儿跃了起来,大声道:``绝不会的!''

苏樱道:``你怎知那不是假的?''

小鱼儿道:``那些标志除了他们自己之外,绝没有别人做得出来。''苏樱叹了口气道:``他们也许是因为自己不敢闯入那地方去,所以叫你去为他们探路,为他们打前锋,他们也许是瞧着你不顺眼,所以叫你去送死!''小鱼儿倒在椅子上,两眼茫然瞪着前面,喃喃道:``绝不会的,绝不会的\ldots\ldots 他们从小将我养大,现在为什么要害我?\ldots。.为什么要害我?''他突又跳起来,冲到铁栅前,大声道:``让我出去,快让我出去,我要去找他们问个明白。''苏樱柔声道:``你现在伤势还没有好,毒也还没有完全去尽,怎么能出去\ldots\ldots 你是天下第一个聪明人,怎么如此沉不住气?''突听一人阴恻恻笑道:``好温柔呀!好体贴!''小鱼儿吃了一惊,嗄声道:``什么人?''

苏樱竟是丝毫不动声色,甚至连嘴角的肌肉都没有牵动一根,只是缓缓转过身子,悠然道:``此间少有佳客,无论什么人来了,我都是欢迎的。''花丛中一人格格笑道:``只可惜在下来得很不是时候,是么?''苏樱微笑道:``阁下不想出来也无妨,只是好花多刺,刺上有毒,阁下若有什么叁长两短,莫怪我不懂得待客之道。''这次她话末说完,花丛中已有个人就好像屁般后被人踢了一脚似的,连蹦连跳的窜了出来。只见这人一张叁角脸,鹰鼻鼠目,那模样叫人一看就恶心,身子却偏偏穿着一身亮闪闪的锦绣衣衫。见了苏樱,竟当头一揖,道:在下小小的开了个玩笑,不想竟让苏姑娘小小的吃了一惊,恕罪恕罪。"小鱼儿见到这人原来是苏樱认得的,原来只不过是在找她开玩笑,心里也就定了下来。

但这人样子讨厌,说话更讨厌,小鱼儿又恨不得``小小的''给他个耳括子,再``小小的''加上一脚。

苏樱也沉下了脸,冷冷道:``你来干什么?你师父难道没有告诉你,这地方不是你们随便来得的!''那人丝丝笑道:``在下小小的胆子,怎敢冒昧闯人苏姑娘的洞府,但这次却是师父他老人家自己叫我来的。''苏樱眼珠一转,道:``他叫你来的?他叫你来干什么?''那人眼睛眯成了一线,笑道;``他老人家叫我来瞧瞧,那一定要用死人做肥料的花,究竟开得有多漂亮,只因他老人家有位客人,也想瞧瞧这种奇怪的花。''这句话说出来,苏樱和小鱼儿都不免吃了一惊。

苏樱冷冰冰的脸色,立刻和缓了,微笑道:``既是如此,我就带你去瞧瞧那种花吧。''那人道:``现在我却不用去瞧了,肥料既然还在喝酒,那花自然还没有开出来,是么?''苏樱眼波流动,媚然道;``那么你\ldots\ldots 你想怎么办呢?''``在下小小的胆子,怎敢对师父说谎,除非\ldots。.那人笑眯眯道:''除非姑娘能令我的胆子大起来。``苏樱笑道:''你的胆子要怎么样才能变大呢?``那人眯着眼瞧着苏樱道;''常言道:色胆包天!这句话姑娘难道没听过?``苏樱脸色微微一变,但还是笑着道,''你不怕你师父吃醋?``那人格格笑道:''不错,师父的确很会吃醋的,他老人家若是知道在和肥料喝酒\ldots。嘿嘿,那时他对姑娘你只怕就不会很客气了。``苏樱咬嘴唇,道:''其实你又何必要挟我,我本来就想和你"她嘴里说着话,一只手有意无意向铁栅上扶了过去。

那人突然大笑道:``姑娘难道想将肥料放出来,杀了我灭口么\ldots\ldots 嘿嘿,只要姑娘的手一碰上去,我立刻就走,不用片刻,师父就会来的!''苏樱的手果然放了下来,笑道:``你这人倒真是多心。但这里总不是\ldots。总不是说话的地方呀,我们到屋里去吧!''那人赶紧摇手道:``不用不用。\ldots 在下早已听说过,姑娘那屋子里机关巧妙,若是随姑娘进去了,在下这小小的性命只怕就保不住了!''苏樱柔声道:``那么你\ldots。你难道想在这里\ldots\ldots{}''她媚笑着,一步步过去。

谁知那人却突然倒退了好几尺,道:``莫要过来\ldots\ldots;苏樱吃吃笑道:''你既然要我。\ldots 为何又不让我过去呢?``那人诡笑道:''在下自然是要姑娘过来的,只不过却要请姑娘先脱了衣服,而且要脱得干干净净,一件不剩。``苏樱道:''我不会武功,你难道还不知道?"

那人道:``姑娘虽不会武功,但那心眼儿之多,在下怎吃得消,只不过\ldots\ldots{}''他笑嘻嘻接道:``姑娘若是脱光衣服,在下就放心了,一个女人若是光赤赤的一丝不挂,她就玩不出什么花样来了。''小鱼儿在一旁瞧得几乎已气破肚子,这人简直比狐狸还奸,比蛇还滑,无论谁遇着这样的人那真是倒霉透顶。

只见苏樱嫣然一笑,一双纤纤玉手,竟真的去解衣服。

小鱼儿忍不住大声道:``气死我了。''

苏樱柔声道:``你绝不会气死的,我也绝不会\ldots。.''突听``嗖''的一声,一道尖锐之极、猛烈之极的风声响过,那人吃了一惊,霍然转身,后面却什么也没有。

他楞了半晌,缓缓回过身来,喃喃道:``我难道遇见了鬼接着,一根青竹''嗖"的飞来,竟活生生将他钉在地上,鲜血雨点般飞溅出来,这人在地上一阵抽搐,永远也不能动了!

就连小鱼儿这样的眼光,竟都未瞧出这人是怎么倒下的,杀他的人出手之快,当真是骇人听闻!

苏樱面色苍白,道:``是\ldots\ldots 是哪位前辈出手相救,请出来容我当面拜谢。''风吹木叶,飕飕作响,四下竟寂无回应。

小鱼儿大声道:``到了这时候,你还不放我出来,让我出去瞧瞧?''苏樱叹了口气,道:``我现在若是让你出来,就等于在害你,我这一生中从来没有关心过别人的死活,只有你。''小鱼儿怒道:``我偏要死,你又怎样?''

苏樱嫣然一笑,道:``我这人下了决心,永远再也不会更改\ldots\ldots 你现在就算真的自杀,我想尽法子,也要将你救活的。''小鱼儿道:``你''\ldots 你简直不是人,是个女妖精。``苏樱抿嘴笑道:''女妖精配小坏蛋,岂非正是天生一对么?"说着说着,她自己脸也红了,红着脸逃了开去。

小鱼儿瞧着她,竟似变得痴了,喃喃苦笑道:``天下竟会有这样的女人,倒也少见得很,看样子她竟像是要跟定我了,这倒是件麻烦事。''只听苏樱远远道:``你在这里等着,我去瞧瞧那位前辈究竟在哪里,立刻就回来的。''小鱼儿忍不住道:``那人武功深不可测,你\ldots\ldots 你要小心了。''苏樱笑道:``你放心,你还没有死,我也舍不得死的,何况,这位前辈既然救了我,又怎么会对我有恶意。''语声渐渐去远,没入树影花丛中。

小鱼儿摇头叹道:``这人看来比谁都柔弱,又有谁能想到她竟有这么大的胆子,这么硬的脾气?''苏樱分花拂柳,一面走,一面笑道:``这地方看来虽美,其实到处都有杀人的陷阱,前辈你救了我,万一在这里受了伤,却叫我怎么好意思?''她面对着一个行踪诡秘、武功深不可测的高手,竟还是一点也小顾及自身的安危,反而口口声声怕别人受了伤,只可惜那人就算听见,也丝毫不领她的情,还是给她个不理不睬苏樱叹了口气,喃喃道:``这人倒真奇怪得很,既然救了我.却又不敢见我,这是为了什么呢?''那敞轩中灯火仍是亮着的,也瞧不见人影,那``椅子''也还好生生的在那里,不像有人动过的样子。

苏樱转了一圈,又回到那山洞去──这一下她脸色终于大变,那山洞前的铁栅竟已被人开启,里面的小鱼儿竟巳不见了!

他难道真的不顾一切,逃了出去?

不会的,他绝不会是自己逃走的,这铁栅他绝对无法开启,能开这铁栅的,算来只有魏无牙和他的首徒魏麻衣。

难道他们也到了这里,将小鱼儿劫走了?

若是换了别人,想到此点,必已惊惶失措,不如该如何是好了,但苏樱反而镇定了下来。

小鱼儿若真的被魏无牙劫走,那么方才救她的那武林高手又到哪里去了?难道他救人后,立刻就走了不成?

何况,若真是魏无牙来了,小鱼儿又怎会全未发出丝毫声音,就老老实实的被他们劫走呢?

苏樱暗暗叹了口气,突听远处传来了惊呼怒骂声。这声音竟正是小鱼儿发出来的。

小鱼儿目送苏樱远去,刚端起酒杯,突听``当''的一声,一粒石子击在铁栅上,火星四溅。接着,铁栅竟缓缓向上升了起来。

小鱼儿又惊又喜,一时间竟怔住了,黑暗中却已幽灵般现出一条人影,长袍高冠,目光森森冷冷瞧着小鱼儿,却不说话。

小鱼儿长长吸了口气,道:``你是来救我的?''那人道:``嗯''

小鱼儿道:``杀了魏无牙的徒弟,也是你么?那人道:''嗯。``小鱼儿道:''但你究竟是什么人?为什么要来救我?``那人冷笑道:''你若不愿出来,我再将这铁栅放下也无妨\ldots\ldots{}

小鱼儿眼珠子一转,笑道:``你可得知道,无论你是为了什么救我,我都不领情的,更不会感恩图报。''那人道:``你若会感恩图报,我就不会来救你了。小鱼儿笑道:''话既然说清楚了,我好歹就让你救我一次吧。"别人救了他,他非但不领情,反面像是要别人感激他似的,那人竟也丝毫不以为忤。

小鱼儿一跃而出,喃喃笑道:"苏樱姑娘,抱歉了,以后有空,我说不定也会来看看你的,你对我的一番好意,我也心领了\ldots\ldots{}

只见那人身形飘飘荡荡,宛如驭风而行。

小鱼儿跟在后面,笑道:``阁下的轻功很不错嘛;。但你究竟要将我带到哪里去?''

\hypertarget{ux7b2cux516bux5341ux516dux7ae0-ux5229ux4ee4ux667aux660f}{%
\chapter{第八十六章
利令智昏}\label{ux7b2cux516bux5341ux516dux7ae0-ux5229ux4ee4ux667aux660f}}

那人道:``到了你自然就知道的。''

小鱼儿忽然停下脚步,道:``你莫以为你救了我,我就会跟你走,你此刻若不说明白,那么抱歉得很,你走你的路,我就要走我的路了。''那人回头一笑,道:``难怪别人说你难缠难惹,如今看来,倒真的\ldots。''他话声忽然停顿,压低声音道:``小心,有人来了,说不定就是魏无牙。''小鱼儿真吃了一惊,道:``人在哪里?''

那人拉住他的手,忽又冷冷一笑,道:``就在这里!''小鱼儿又一惊,已觉得半身发麻,原来那人已扣住了他的脉门,五指如铁,小鱼儿哪里还能挣得脱,失声道:``你\ldots\ldots 你这是干什么?''那人也不说话,左手又闪电般点了他好几处穴道。

小鱼儿怒道:``你疯了么,既然救了我,为何又来暗算于我?''那人冷笑道,``就因为你想不到,否则我又怎能得手?''他嘴里说着话.竟用条带子将小鱼儿吊在树上。

小鱼儿又惊又怒,怒骂道:``你这疯子、畜牲,你究竟想怎样?''那人却再也不瞧他一眼,拍了拍手,扬长而去了。

小鱼儿忍不住怒骂道:``疯子,疯子。\ldots 我怎地总是撞见些疯子。''苏攫听见小鱼儿的怒骂声,亦是又惊又喜,无论如何,小鱼儿总算还在这山谷里,她正想追过去。

突听黑暗中一人冷冷道:``你不必找了,我就在这里!''一人随着语声缓缓走出来,瘦骨嶙峋,麻衣高冠,双颧高耸,鼻如兀鹰,目光睨睥之问,充满冷漠倨傲之意。

苏樱竟不觉怔了怔,才长长吐出口气,道:``原来是你!''麻衣人道:``哼!''苏樱嫣然一笑,道:``方才我就觉得杀人的手法很像你,但我却想不到\ldots。.''麻衣人冷冷道:``你想不到我会来,是么?''

苏樱叹了口气,道:``我的确没有想到,自从你和老头子斗翻之后,已经有四年\ldots\ldots 四年叁个月没听过你的消息了。''麻衣人仰面望天,道:``你倒还记得我。''

苏樱垂下了头,道:``我怎么会忘记你,你一向对我那么好。''麻友人忽然怒道:``谁说我对你好,普天之下,我从来也没有对谁好过。''苏樱道:``你难道没有?''

麻衣人长长吸了口气,大声道:``不错,我也是为了你,我瞧不惯他已半截入了土的人,还要\ldots\ldots 还要把你当做他的禁脔,别人只要瞧你一眼,他就要发疯。''苏樱默然半晌,道:``但你现在还是回来了。''麻衣人冷笑道:``我要来就来,要去就去,谁管得了我。''苏樱道:``不错,连老头子都有些含糊你,你走了之后,他常说这一生收的弟子虽多,但所得到他真传的,却只有你一个。''麻衣人冷笑道:``你以为我的功夫是他教给我的么?哼\ldots\ldots 魏无牙自私自利,苛刻成性,还有谁不知道,他收那么多徒弟,只不过是想用些不要钱的佣人而已,几曾将真功夫教给别人\ldots\ldots 他只不过传授了我几手皮毛功夫,就要人家去为他拼命,为他死!''苏樱道:``那么你的功夫''

麻衣人冷冷道:``我的功夫只不过是一点一滴偷来的\ldots。在他练功的时候,我在暗中偷偷的瞧,偷偷的学来的。''苏樱叹道:``他对徒弟的确不好,但对你\ldots。你现在为什么又要回来呢?麻衣人道;''我\ldots\ldots 我只不过是想回来瞧瞧。``苏樱眼波流动,微笑道:''你回来还是为了想看看我,是么?``麻衣人大声道:''现在我已知道,你这人根本无情无义,无论别人对你多么好,你既不会放在心上,也不会感激。``苏樱似是十分委屈,垂头道:''我\ldots\ldots 我真是这样的人么?``麻衣人道;''哼。"

苏樱道:``但你杀了魏十八,还是为了我,你看不惯他那么样欺负我,由此可见,你还是对我很好的,是么?''麻衣人突然大笑起来。

苏樱眨了眨眼睛,道:``你笑什么?''

麻衣人戛然顿住笑声,一字字道:``老实告诉你,我早巳对你死了心了!我虽不屑去做那些揭人隐秘、无耻密告的事,但无论你喜欢谁,我都再也不会放在心上!''苏樱静静地瞧了他半晌,也缓缓道:``那么,你为什么要将我喜欢的人劫走呢?''麻衣人冷冷一笑,道:``这原因你不久就会知道,现在你想不想先去瞧瞧他?''苏樱道:``你说我想不想?''

麻衣人道:``好,你跟我来吧!''

小鱼儿瞧见苏樱竟和这麻衣人一起来了,而且两个人看来还好像很熟,他又是惊讶,又是诧异,忍不住怒喝道:``这疯子究竟是什么人?你认得他?''苏樱瞧见小鱼儿竟已被人吊在树上,不觉叹了口气,苦笑道:``天下第一个聪明人,怎会变成这样子的?''小鱼儿怒道:``只因我没想到这人竟是个疯子,做的事实在令人莫名其妙。苏樱道:''他就是魏无牙门下,武功最高的弟子,江湖中人提起无常索命魏麻衣来,谁不心惊胆战,否则怎会连你都上他的当。``小鱼怔了半晌,长长叹了口气,道:''这人竟会是魏无牙的徒弟,看来我真的遇见鬼了。``魏麻衣冷冷道:''既然遇见了,你还有什么话说?``小鱼儿向他扮了个鬼脸道:''话是没有了,屁倒还有一个,你想不想闻闻?"他头下脚上,高高吊起,人的脸若是反过来看,本已十分滑稽,此刻他又做了个鬼脸,那样子可实在令人不敢恭维。

苏樱忍不住``噗哧''笑出声来。

魏麻衣纵是满心气恼,但瞧见他这副样子,竟也忍不住要笑,当下扭转了头,瞪着苏樱道:``你喜欢的就是这人么?''若是换了别的女人,纵然满心喜欢,也万万不好意思当面说出来,但苏樱却连头都未垂下,道:``不错。''魏麻衣冷笑道:``我本当你眼界很高,谁知你喜欢的却是这种疯疯癫癫的笨蛋。''苏樱笑道:``他本来就不错,否则我\ldots\ldots 我又怎会被他迷上呢!''魏麻衣怔了怔,道:``连这样的话,你也说得出口。''苏樱道:``我为何不敢说出心里的话?这又不是什么丢人的事,若是鬼鬼祟祟、偷偷摸摸,心里喜欢了别人,嘴里却不敢说,那才叫丢人哩\ldots。你说是么?''魏麻衣蜡黄的一张脸,竟也像是红了红,冷笑道:``你虽喜欢他,怎奈他却未必喜欢你。''苏樱道:``只要我喜欢他,无论他喜不喜欢我都没关系,更用不着你来费心。''魏麻衣道:``哼,你\ldots\ldots{}''他也想反唇相讥,怎奈``哼''了一声,就说不出话来。

苏樱一笑又道:``何况,就算他现在不喜欢我,我也有法;叫他喜欢我的。''听到这里,小鱼儿已忍不住大笑道:``好,说得好,我简直现在就有些喜欢你了。''魏麻衣面上一阵青一阵白,厉声道:``既是如此,他若死了,你必定十分伤心,是么?''苏樱微微一笑,道:``我早就知道你要以他来要挟我的,你究竟想要什么?难道还不好意思说?''

魏麻衣瞧着她那如春水般的眼波,瞧着她那在轻衣下微微起伏的胸膛,只觉心跳加速,嘴唇发干,道:``\ldots\ldots 我要你\ldots\ldots{}''突然大喝一声,身形急转,在自己胸膛大打了七八拳,眼睛再也不敢去瞧她,大声道:``我只要你说出你昨日听到的秘密!''苏樱忽然笑道:``其实你就算要的是我,我也会将自己给你的,只恨你竟没有这个胆子,将大好机会平白错过。''魏麻衣怒吼一声,转身抓住她的肩头,嘶声道;``你\ldots\ldots 你这臭丫头,小贱人,你\ldots.你\ldots.你\ldots\ldots\ldots 他说了一句,又说不出来,忽然出手一掌,向苏樱脸上掴了过去,谁知苏樱竟不闪避,反而转脸迎了上去,道:''你要打,就打吧,但你忍心打得下手么?"只见淡淡的星光,自树梢漏下,照射在她脸上,她星眸如丝,鲜花般的面颊更似吹弹就破。

魏麻衣这一掌竟硬生生地在半空中顿住,再也打不下去。

苏樱却将整个身子都偎了过去,闭着眼道:``你打呀,你怎么不打了?''魏麻衣身子似乎发起抖来,心里恨不得立刻就将这软玉温香抱个满怀,偏偏又没脸真的伸出手去。

小鱼儿瞧得又好气,又好笑,突见苏樱一只春葱般的纤纤玉:手上,不如何时已戴起了个发亮的戒指。

他头上脚上,眼睛正对着这戒指,星光下瞧得清楚,这戒指上竟有根又尖又细的银针。

苏樱扭动着腰肢,嘴里含含糊糊的,也不知说些什么,这只戴着戒指的手,却向魏麻衣脖子上搂了过去。

魏麻衣脖子上的细皮,只要被这根银针划破一丝,他就再也休想活了,而他此刻心跳气喘,眼睛发红,一颗心已飘飘荡荡地不知飞到哪里去了,怎么想得到这要命的无常巳离他不到半寸。

谁知小鱼儿竟然大喝道:``小心她的手!她手指有毒针!''魏麻农狂吼一声,举手一掌,将苏樱推出数尺。

苏樱身体撞到树上,瞪眼瞧着小鱼儿,失声道:``你\ldots\ldots 你疯了么?''苏樱咬着嘴唇,不说话,魏麻衣又惊又怒,但实也不懂小鱼儿为何反来救他,是以瞪着眼站在那里,也没有说话。

只听小鱼儿笑道:``我救他,只因我也想听听你那秘密。''苏樱道:``\ldots\ldots\ldots 你说什么?''

小鱼儿接道,``你宁可将自己肉身布施,也不肯说出这秘密,可见连你自己都将这秘密瞧得比自己的身子还要紧得多。''苏樱道:``他不敢杀我的,只因他杀了我后,就再也休想知道那秘密了。''小鱼儿截口笑道:``我倒想听这秘密,只有让他要挟你,你才不得不说出来,他若被你杀了,这秘密只怕你再也不会说出来,我岂非也听不到了。''苏樱跺脚道:``但我既然救了你,这秘密,难道以后不肯告诉你么?''小鱼儿笑道:``那是两回事,你见我要死,心里着急,才会将这秘密说出来,等我被救下来后,你却又怕我走了,那时你就会用这秘密来钓住我,说不定要等到什么时候才肯说出来,我怎么能等得及。''他大笑接道:``老实告诉你,你救了我后,我说不定立刻就要走的,那时我岂非永远也听不到这秘密了,我心里岂非要难受一辈子。''这番话说出来,就连魏麻衣听了,都有些哭笑不得,苏樱更听得几乎气破肚子,大声道,``这秘密既如此重要,你若也要一旁听见了,他怎会放过你,你。\ldots 你自愈天下第一个聪明人,怎地连这点都未想到。''小鱼儿大笑道:``朝闻道,夕死可矣,我只要能听到如此精彩的秘密,死了也没什么关系。''苏樱瞧了瞧小鱼儿,又瞧了瞧魏麻衣,忽然娇笑着道:``有趣呀有趣,天下竟有这样的人,这样的事,我本来绝不会为了任何人说出这秘密,但为了你\ldots\ldots{}''小鱼儿道:``为了我,你愿说么?''

苏樱转向魏麻衣,脸立刻沉了下来,缓缓道:``其实我就算将移花接玉的秘密告诉你,也没有用的,你反正学也学不会,破也破不了''\ldots\ldots{}``魏麻衣还未说话,小鱼儿已变了颜色,失声道:''你说什么?

移花接玉的秘密?"

苏樱道:``不错,移花接玉的秘密,也就是武学中最大的秘密,他们师徒就为了这秘密,二十年来食不知味,睡不安枕。''小鱼儿瞪大了眼睛,道:``你\ldots。你知道移花接玉的秘密?''魏麻衣早己沉不住气了,嘎声道:``只要你说出来,学不学得会就是我的事了。''苏樱道:``好,你听着\ldots。.''

一句话还未说完,突听小鱼儿放声大喊道:``天灵灵,地灵灵,玉皇大帝圣旨令,观音菩萨柳枝瓶,外加阎王老子,牛头马面,你们快来救我呀。''他穷吼鬼叫,又叫又嚷,苏樱说些什么,魏麻衣一个字也听不见了,一步窜过去,大怒吼道:``你小子疯了么?''小鱼儿朝他扮了个鬼脸,笑嘻嘻道:``我没有疯,只是这秘密我已不愿听了。''这句话说出来,苏樱又怔住了。

魏麻衣更是暴跳如雷,吼道:``你本来拼命想听这秘密,如能听到移花接玉的秘密,就是死了也不冤,如今为何反而不想听了?''小鱼儿笑道;``别的秘密我倒也想听听,但这移花接玉的秘密么\ldots\ldots 嘿嘿,我叁岁就知道了,再听岂非无趣。''魏麻衣怔了怔,道:``你\ldots\ldots\ldots 你也知道?''

小鱼儿道:``这秘密若是由苏樱说出来,你练到一百岁也休想练得成,何况你连五十岁都未必活得到。''苏樱吃吃笑道;``这话倒也不错。''

小鱼儿道:``但这秘密若由我说出来,不出叁天,你就可练成,只因我所知道的,乃是移花接玉功的速成捷径。''魏麻衣听得脸都热了起来,忍不住动容道:``只要你真能说出来,我\ldots\ldots{}''小鱼儿正色道:``我也不要你感激我,只要你放了我就是。''魏麻衣道:``是是是,在下一定\ldots。.''

小鱼儿截口道;``好,你听着,我一面说,你一边练。''小鱼儿道:``移花接玉的行功要诀,第一步就是要你手为脚,倒立而起,昂起头,分开双足屏息静气。''魏麻衣皱眉道:``这算什么功夫?''

小鱼儿正色道:``你要知道,移花接玉的最大奥妙,就是一切都反其道而行,练功的姿势,自然也得要如此。''魏麻衣虽然有些怀疑,但只要能学到移花接玉,他委实不惜牺牲一切,只要有一点机会,他也不肯错过,苏樱抿嘴在一旁瞧着,也不说话。

只见魏麻衣身子一挺,已倒立而起,双足微分,头抬得高高的,那模样活脱脱脱是一只蛤蟆。

小鱼儿扳着脸瞧着,脑上连一丝笑容也没有,道:``膝盖再弯些,头再拾高些。''魏麻衣倒真听话得很,立刻照话做了,道:``这样行了么?''小鱼儿道:``马马虎虎,将就使得了。''

说完了这句话,就再也没有下文。

要知魏麻友纵然内力深湛,但这姿势实在要命,武功再高的人摆出这种姿势,也不免吃力得很。

盏茶工夫过后,魏麻衣头上已快流汗,忍不住道:``还要等多久?''小鱼儿道:``好,现在你真气巳沉至胸膛,第一步已可算准备好,第二步的功夫未做前,先得放个屁。''魏麻衣怒道:``我看你简直在放屁。''

她虽然又惊又怒,但生怕前功尽弃,还是不敢站起。

小鱼儿道:你要知道,屁乃人身内之浊气,我要你放屁,正是要你先将体内浊气驱出,然后才能开始练功夫。"

\hypertarget{ux7b2cux516bux5341ux4e03ux7ae0-ux6c5dux5978ux6211ux8bc8}{%
\chapter{第八十七章
汝奸我诈}\label{ux7b2cux516bux5341ux4e03ux7ae0-ux6c5dux5978ux6211ux8bc8}}

魏麻衣听小鱼儿要他放屁,心中一想,这倒也有理,只好放了个屁,要知内功高明的人,本可随意控制自己身体里的气脉,放个屁并非难事,苏樱早巳掩住鼻子,转过身去,肩头不停的在动,像是忍不住要笑,小鱼儿却仍是一本正经,:``这个屁要脱下裤子来放才算的。''魏麻衣道:``脱。脱\ldots\ldots{}''

他脸已胀得通红,连话都说不出了。

小鱼儿道:``这一步就叫做脱了裤子放屁,放个痛快。''要知他非但不是呆子,而且阴沉狡猾,只不过想学``移花接玉''的心太热了一些,头未免有些晕了,正是所谓``利令智昏'',小鱼儿才会有机可乘.此刻魏麻衣越听越不对,翻身跃起,怒道:``这\ldots。这究竟算什么功夫?''小鱼儿还是板住脸,道,"这就叫呆子放屁功,那比移花接玉可要厉害多了.,魏麻衣双拳紧握,全身发抖,简直活活要被气死。苏樱也忍不住笑得花枝乱颤。

小鱼儿这才放声大笑道:``呆子,你想我真会移花接玉还会被你用在树上么?你让我上了个当,我若不也让你上个当,怎么对得起你.''苏樱娇笑道:``但你\ldots\ldots 你这样做也未免太缺德了。''小鱼儿大笑道:``要想占我便宜的人,总得吃些亏的。''魏麻衣怒吼道:``你要我上当,我就要你的命!''怒吼声中,扑了过去。

小鱼儿却大呼道;``天灵灵,地灵灵,天兵神将,大鬼小鬼,再不出来救驾我就要骂了。''``像你这样的人,鬼也不会来救你的。''魏麻衣手指已向小鱼儿哑穴点了过去。

就在这时,突听黑暗中一人阴恻恻道:``你又不是鬼,怎知鬼不会来救他?''这语声缥缥渺渺,若断若续,连一点生气都没有,哪里像是活人发出来的声音,而且语声发出时,本在西面,一句话说完,已到了东面。

深夜荒林,骤然听见这样的声音,真叫人不寒而栗。

只见黑暗的苍弯下,树梢头,果然有条灰白色的影子,一身麻衣在风中猎猎飞舞,看来当真是鬼气森森,不像活人。

魏麻衣究竟不是等闲人物,瞧见对方的影子后,反而沉住了气,一步步走过去,冷冷道:``阁下既然想做鬼,我就成全了你吧!''语声中,已有一蓬银雨,向树梢暴射而出。

由下往上,本难使力,但魏麻衣的腕力当真不同凡响,这一蓬银雨去势之急,竟比强弩硬箭还急几分。

树梢上的影子惊呼一声,落叶般飘了下来。

魏麻衣冷笑道:``看你还装神弄鬼。''"

话犹未了,只听一人哈哈笑道:``死一次是鬼,死两次还是鬼你再往这里瞧瞧。''魏麻衣大惊回首,那灰白色的影子赫然竟已到了左面十丈外的树梢上,一双灰白色的眼睛,正俯首瞪着魏麻衣冷笑。

魏麻衣纵是艺高人胆大,此刻手脚也不禁有些发冷,就在这时,突听身后一人哈哈大笑道:``这么大一个人,难道也会被鬼吓着么?''魏麻衣霍然翻身,只见一个满脸笑容的圆脸和尚,摇摇摆摆走了过来,魏麻衣蓄气作势,厉声道:``你难道也是鬼么?''那和尚哈哈笑道:``和尚不是鬼,和尚是捉鬼的和尚。''魏麻衣冷笑道:``既然如此,和尚你就将那鬼捉来吧。''那和尚道:``那不是鬼\ldots 哈哈,鬼不在那里。''那和尚的手突然往旁边黑暗的林中一指。

魏麻衣情不自禁,随着他手指之处瞧了过去。只见黑暗中不知何时,已坐着条人影,手里拿着白生生一件东西,正吃得津律有昧。

魏麻衣眼观四面,心里在筹思着对敌之策,要如何才能将对方几人一连击倒,嘴里却笑道:``但鬼哪有如此好吃的?''那和尚道:``哈哈,他不信。\ldots 你为何不让他瞧瞧。''树林里那人嘻嘻一笑,将手里的东西向魏麻衣抛了过来,魏麻衣不由自主的伸手一抄。

他只党这东西软软的,嫩嫩的,仔细一瞧,竟是半截手臂,上面牙印宛然,而且是已煮熟了的。

这下子魏麻衣真的吃丁一惊,只觉半边身子都麻了,赶紧将这半条人臂远远抛了出去。

树林里那人又伸手接住,嘻嘻笑道:``这地方人都有老鼠臭,不能吃的,我好容易才找到一个能吃的人,节省着吃了叁天,只剩下达半截手了,你若抛了岂非可惜?''一面说着,一面又放怀大嚼起来,嚼得吱吱喳喳的响。

魏麻衣几乎忍不住吐了出来,情不自禁地往后退,嘎声道,``各\ldots\ldots 各位究竟是什么人?究竟要想怎样?''突听又是一人冷冷道:``这里只有我一个人,你有什么话,找我来说吧!''语声中一人大步走了过来,身子又高又瘦,白衣如雪,袖长及地,一张惨白的脸,冷得像冰,简直比鬼难看得多。

魏麻衣厉声道:好,你既是人,我也要让你变鬼!"他出手当真是快如闪电,话声中招已递出。

这一抓他五指已贯满真气,若是被他抓着,铁石也将洞穿,那白衣人竟似变招不及,闪避无力。

魏麻衣一抓就抓住了他的手,突然手里冷冷冰冰,抓住的哪里是只人手,大惊之下,白衣人已狞笑道:``撒手!''只听``嘶''的─声,他长袖一分为二,魏麻衣但见对方的``手''已自他掌心划过,鲜血立涌而出。这白衣人的手,竟是只钢钩!

魏麻农手掌虽不重,但生怕对方钩上有毒,更是不敢激战,身形倒纵,便待冲出。

忽然间,又听得一人怒喝道:``无牙门下,岂是临阵脱逃的人,不管他们是人是鬼,你怕什么?''只见这人身形瘦小如童子,一张也说不出有多难看的脸上,却生着一副很好看的胡子,长须飘飘,几乎已飘到地上。

他头戴金冠,长袍上碧光闪闪,看来又是可笑又是可怕,树林里那吃人的鬼惊呼一声,道:``魏无牙来了!鬼也害怕,还是溜吧!''这时树林里连人带鬼都逃了个干净,只有小鱼儿吊在树上,苏樱也早巳不知走到哪里去了。

魏麻衣叹了口气,苦笑道:弟子如今才知道,无论如何,还是比不上师父的。``魏无牙冷笑道,''你知道就好。"

他袍袖一挥,又道:``那人伤了你哪里?可有毒么?伸出手来让我瞧瞧。''魏麻衣缓缓伸出手,突然一掌向魏无牙击出。

这一掌出手很急,魏无牙却似早巳算准他有这一着,身子一闪,后退一丈开外,怒叱道:``好个孽徒,敢对师父如此无礼。''魏麻衣狂笑道:``你易容的本事虽不错,但想扮魏无牙,还差得远哩!''那魏无牙也哈哈笑了起来,道:``好,居然被你瞧破了,但我且问你,我学得哪点不像?''魏麻衣大笑道:``你难道不知道他天生残废,两条腿有如婴儿,走起路来就像爬一样,他生怕别人瞧见.是以从不自己走路只听哈哈一笑,那和尚又从黑暗中跳了出来,招手笑道:''小娇儿这次可栽了跟头了。"

那吃人的鬼也忽然出观,大笑道:像魏无牙那么丑怪的人,天下也找不出第二个,的确是谁也扮不像的,我早就知道你下的苦功都白费了。``那人身子一长,忽然长高了两尺,道:''现在我只想该用什么法子,让魏无牙走两步瞧瞧。``魏麻衣忽然翻身,箭一般掠回小鱼儿身旁,抽出一柄碧绿的匕首指着小鱼儿的咽喉,喝道:''你们可是来救他的么?``那吃人的鬼大笑道:''你要杀他,你杀得了他么?``笑声中,倒吊在树上动也不能动的小鱼儿,突然能动了!非但能动,而且动作简直比闪电还快。他两只手─动,就点了魏席衣的几处穴道.魏麻衣大骇之下,连还手都来不及,全身已被制往,小鱼儿顺手夺过他的匕首,指着他的咽喉,哈哈笑道:''你又上了我的当了。"

魏麻衣只有瞪着眼,咬着牙,到了这地步,他还有什么话好说,小鱼儿笑嘻嘻瞧着他,道:``你现在总该知道,我的便宜是不好占的了吧,你若占了我的便宜,我迟早连本带利都要收回来的。''那吃人的鬼摇摆摆摆走了过来,在魏麻衣脖子上嗅了嗅,面上忽然露出大喜之色抚掌笑道:``妙极妙极,这人身上已没有什么老鼠臭了,若多加些葱姜佐料,用上好的酱油来红烧,已勉强可以吃得。''

魏麻衣目中满是惊惧之色,瞪着他嘎声道:``你\ldots\ldots 你莫非是不吃人头李大嘴?''那吃人鬼仰天笑道:``我已有二十年未在江湖走动.不想还有人记得我的名字。''魏麻衣全身都软了,别人若要吃他,他还未必相信,但李大嘴若说要吃他,那可就不是说笑的了。

小鱼儿笑嘻嘻道:``你何苦再骇他,若是骇破了苦胆,肉岂非吃不得了。''突见一个人自树梢凌空翻下来,一身白麻衣衫飘飘飞舞,落到魏麻衣面前,瞧着他咧嘴一笑道:``你只认得不吃人头李大嘴?可认得我么?''这人就是方才被魏麻衣用暗器从树梢打下去的,一顶白麻冠上,还留着根银针,显见方才虽未真的被打中,少不得也要骇一大跳。

魏麻衣瞧了他一眼,闭上眼睛,叹道:``装神弄鬼的人,我早该想到你是半人半鬼阴九幽的。''那人却折了段树枝,拨开他的眼皮,道:"你再睁大眼睛瞧瞧;阴九幽是在哪里。魏麻衣只有张开眼睛,望了过去,只见树梢上还飘飘荡荡地站着条麻衣人影,打扮得和面前这一个人一模一样。

方才装鬼的,原来是两个人,难怪``瞻之在前,忽焉在后,瞻之在左,忽焉在右'',说穿了竟是一文不值。

魏麻衣长叹了一声,苦笑道::十大恶人,今日究竟来了几个?``那人道:''也不太多,只不过六个,老子就是损人不利己白开心,你小子可曾听过老子的大名?``魏麻衣冷冷道:''我早已听说,白开心在十大恶人中,可算是最没用的一个,只不过是江湖中人勉强拿来凑数的。``白开心脸色变了变,仍瞬即大笑道:''你莫要挑拨离间,老子今年已四十八,再也不会上这种当了。``那和尚拍手道:''白开心果然长成大人了,只不过你明明已五十二,为何说四十八,你又不是女人,何必瞒岁呢。``白开心瞪眼道:''我老婆还未娶着,若不瞒几岁,还有谁嫁给我。``他又拍了拍魏麻衣肩头,又道:''你可得记着,这和尚笑里藏刀,最不是东西。``魏麻衣叹道:''好一个笑里藏刀哈哈儿!"

他眼睛向那面色惨白的白衣人瞧了过去,道:``你是\ldots\ldots 你是\ldots\ldots{}''白友人长袖一翻,露出了双手──右手竟是一只雪亮的钢钩,左手上光芒闪闪,其红如血!

魏麻衣失声道:``血\ldots\ldots 血手杜杀!''

杜杀道:``哼!''

魏麻衣惨笑道:``好,好,好,原来十人恶人真的到了六个,我魏麻衣落在你们手里,还有什么话说?''杜杀冷冷道:``不错,你只有死!''

他一步步走过来,光芒闪动处,钢钩向魏麻衣咽喉划了过去。

李大嘴赶紧拉着他的手,道:``这使不得。''

杜杀厉声道:``你想怎样?''

李大嘴笑道:``杜老大的事,小弟怎敢拦阻,只不过,他身上的肉本已不多,若先杀了他再煮,失血过多,肉更没有滋味了。''杜杀道:``哼。''

他缓缓放下了手,魏麻衣却已颤声呼道:``李大嘴,你我究竟同是武林一脉,你杀了我,我死而无怨,但你又怎能\ldots\ldots 怎能\ldots\ldots{}''他只觉一阵呕心,胃里的东西都吐了出来。

李大嘴捏着魏麻衣身上的肉,喃喃道:``像这么大一个人,用两斤酱油,一斤料酒,十文钱的葱姜只怕就够了,自然还要加五文钱的五香八角。''魏麻衣全身都麻了,终于颤声道:``求求你,我\ldots\ldots 我\ldots\ldots 求求你好么\ldots\ldots{}''李大嘴两只手一提,将魏麻衣整个人都提了起来,笑道:``各位,小弟肚子饿了,要先走一步\ldots\ldots{}''他话未说完,魏麻衣已狂吼一声,晕了过去,哈哈儿拍手笑道:``吓昏了,吓昏了,李大嘴果然有两下子。''白开心摸着魏麻衣的头,道;``这小子醒了后,想必会乖乖的听话了,咱们要挑魏无牙的老鼠洞,也就全要靠这小子帮忙。''哈哈儿道:``正是如此,否则咱们何必花这么多功夫来吓他。''小鱼儿伸了个懒腰,笑道:``只苦了我,害得我在树上多吊了半个时辰。''屠娇娇瞧了他半晌,忽然道:``那姓苏的丫头明明已要说出移花接玉的秘密了,你为何反而要拦住她?''白开心道:``是呀,你为何要拦住她,你不是要和花无缺拼命了么?若能知道移花接玉的秘密,岂非就能稳操胜券?''小鱼儿懒洋洋一笑,道:``我知道他武功的秘密后,再和他打架还有什么意思?''白开心瞪了他半晌,长长叹了口气,道:``你原来是个好人。''他忽又大笑起来,拍手笑道:``由哈哈儿、李大嘴、杜老大、屠娇娇、阴九幽,这五个人养大的孩子,居然会是个好人\ldots\ldots 狐狸窝里出了条牧羊狗,你们五个不觉得丢人么?''阴九幽、杜杀面色都微微变了。

\hypertarget{ux7b2cux516bux5341ux516bux7ae0-ux98d8ux5ffdux65e0ux8e2a}{%
\chapter{第八十八章
飘忽无踪}\label{ux7b2cux516bux5341ux516bux7ae0-ux98d8ux5ffdux65e0ux8e2a}}

李大嘴却立刻大笑道:``你也学会了屠娇娇的一手?也来挑拨离间了?''屠娇娇嘻嘻笑道:``他挨了小鱼儿一顿,他心里一直不服气哩。''哈哈儿道:``不服气又有什么用?哈哈,十个白开心也斗不过一个小鱼儿的,你若是想出气,还是死了这条心吧。''白开心也不生气,笑嘻嘻道:``我又有什么不服气的?有一天狐狸若是被狗吃了,那我才是服气哩。''这句话说出来,连李大嘴脸色都变得有些难看了。

小鱼儿却似没有瞧见,拍手大笑道:``损人不利己,果然是损人不利己。''话犹未了,只听一人银铃跋笑道:``十大恶人,也果然名下不虚,我真佩服极了。''一栋四人合抱的大树干上,忽然开了个门,原来这株树竟是空心的,里面正好藏人,谁也休想找得着。

苏樱从树里面盈盈走出来,盈盈一礼,笑道:``名震天下的十大恶人来了,贱妾竟有失远迎,恕罪恕罪。''哈哈儿大笑道:``姑娘千万别客气,咱们这些人是天生的贱骨头,有人对咱们一客气,咱们就以为他要来动坏主意了。''李大嘴忽然跳了起来,大嚷道:``走吧.走吧,快走吧,再不走我就受不了啦!''屠娇娇道:``你受不了什么?''

李大嘴道:``瞧见这丫头的一身细皮白肉,我简直连口水都快流了出来,但又明知道小鱼儿绝不肯让我吃了她的,再不走我岂非要发疯。''嘴里说着话,已背着魏麻衣,如飞似的走了出去。

白开心也跳了起来,道:``我也要走,瞧着这娇滴滴的美人儿,我这光棍也实在有些心动,不如还是快走,眼不见为净,也免得和小鱼儿争风吃醋。''话声中,凌空一个翻身掠出叁丈外,眨眼就不见了。

哈哈儿也随了出去,一面笑道:``不错,再不走连和尚都要动凡心了。''屠娇娇格格笑道:``幸好我还有一半是女人,否则\ldots\ldots\ldots{}''瞟了小鱼儿一眼,娇笑着掠上树梢一闪不见。

阴九幽阴恻恻笑道:``姑娘若做人做腻了,不妨来找我,做鬼有些时比做人有趣得多,这年头漂亮的女鬼,更吃香得很。''苏樱抿嘴笑道:``多谢指教,但我现在却活得还蛮有趣哩。''阴九幽指着小鱼儿,大笑道:``你若是爱上了这个人,用不着多久,就会觉得活着无趣的\ldots。.''等这句话说完了,笑声已远在十余丈外。

杜杀瞪着小鱼儿,笑道:``你还要在这里耽多久?''小鱼儿笑道:``只怕用不着多久的。''

杜杀道:``你知道在哪里可找得着我们?''

小鱼儿道:``知道。''

杜杀道:``好''

他人己掠出林外,突又回首道:``小心些,漂亮的女子若要吃人时,连人头都要吃下去。''苏樱娇笑道:``前辈只管放心,我的胃口一向不好,一向是吃素的。''树林里忽然静了下来,苏樱含笑瞧着小鱼儿,道:``魏麻衣将你吊在树上后,这些人已来了?''小鱼儿笑道:``他们来得正巧。''

苏樱道:``但你还是装成不能动的样子,来骗我。小鱼儿笑道:''我本来可不是要骗你的,魏麻衣让我上了一次当,我怎么能就那样放过他,我好歹也得要他知道厉害。``苏樱道:''你本来虽不是为了骗我,但后来还是骗了我了。``小鱼儿耸了耸肩,道:''你若要这么想,我也没法子。``苏樱道:''你知道我对你很好,所以就利用这点来骗我,让我为你担心,为你着急,我不顾一切来救你,你反而以此来要挟我说出心里的秘密。``她眨也不眨地凝注着小鱼儿,眼被沉得像黑夜中的海水,小鱼儿扭转头,忽又回头一笑道:''我早就说过,我并不是好人,谁若对我好,谁就要倒霉了。``苏樱叹了口气,缓缓道:''世上大多数人,都生怕自己变得太坏,但你却偏偏相反,你竟好像生怕自己变得太好了,总要做些事来证明你自己不是好东西\ldots\ldots 这究竟是为了什么呢?这只怕连你自己也想不到的,是么?``小鱼儿笑道:''这只怕是因为我天生是个坏胚子。``苏樱瞧了他半晌,忽也一笑,道:''但你可知道,你并没有自己想象中那么坏么?``小鱼儿笑道;''你且说来听听吧。"

苏樱缓缓道;``这只因你从小是跟着那些坏人长大的,所以在你心里面,总觉得自己绝不可能变得太好。''苏樱顿了顿又接着说:``而且,你还认为自己若是变得太好.就有些对不起那些将你养大的人,所以有时你不得不做些坏事来证明自己\ldots\ldots{}''小鱼儿突然大笑起来,打断了她的话,截口道:``你和我见面还没有几天,就以为很了解我了?''苏樱道:``我本来也并不太了解,但见了那些人后,就明白了。''小鱼儿道:``哦?''

苏樱微笑道:``那些人真可算是坏人中的天才,已坏得炉火纯青了,他们竟能将一件卑劣低下、或是很恶毒残酷的事,做得令人反而觉得很有趣。''小鱼儿道:``你用不着这样骂他们,他们可没有得罪你。''苏樱一字字道:``你难道现在还未发觉,是他们将你诱入那\ldots\ldots\ldots\ldots 那老鼠洞去的。''小鱼儿又大笑起来,道:``笑话,这才是笑话,他们为何要骗我?''苏樱道:``这也许是因为他们已发觉,你并不是和他们一样的坏,他们认为你说不定会反叛他们,所以就故意做下那些标志暗号,将你诱入那老鼠洞,要想假魏无牙之手,将你除去\ldots\ldots{}''小鱼儿顿住笑声,大声道:``那么我问你,他们既要害死我,方才为何又来救我?''苏樱眼波流动,道:``这也许是因为他们忽然又觉得你有用了,杀了可借,也许是因为他的并不愿亲手杀死你''\ldots\ldots{}

小鱼儿忽然跳了起来,大声道:``放屁放屁,你说的话,我一个字也不相信。''苏樱叹了口气,道:``我也不一定要你相信,只要你多加提防,也就是了。''小鱼儿哈哈一笑,道:你叫我多加提防?我看你倒真该多加些提防才是。``苏樱叹了口气,道:''你说的不错,这地方以后只怕真要变成是非之地了,看来我只怕也没法子再在这里耽下去,但是你\ldots\ldots\ldots 你难道发现了什么?``小鱼儿悠然道:''一个被吊在树上的人,瞧见的总要比别人多些的。``苏樱道:''你究竟瞧见了什么?"

小鱼儿道:``我瞧见两个人。''

苏印哧一笑.道:``就算瞧见二十个人,也并不是一件什么稀奇的事。''小鱼儿道:``但这两个人却稀奇得很。''

苏樱道:``哦?''

小鱼儿道:``这两个人早已藏在那边的小山石后面了,我的朋友来救我时,他们已经在那里,但他们却好像根本不愿管这边的闲事,等到你和魏麻衣一走进这树林子,他们就立刻飞出似的溜到那边的屋子里去,轻功居然是一等一的高手\ldots\ldots\ldots{}''苏樱非但没有吃惊,却反而笑了。柔声道:``原来你还是关心我的。''小鱼儿冷笑道:``你若喜欢自我陶醉,我也没法子,但现在可不是你自我陶醉的时候,那两个人\ldots\ldots{}''苏樱又打断了他的话,媚然道:``你不必为我担心,那是一对很有趣的夫妇,常常喜欢做一些自作聪明的事,男的一个还好些,女的一个总认为自己比别人都聪明得多,其实却是个神经病。''小鱼儿板着脸道:``自以为比别人聪明的人,大多是有些毛病的,但我却是例外,只因为我的确比别人聪明得多。''苏樱道:``他们已走了么?''

小鱼儿道:"不但走了,而且还带走了两大包东西\ldots\ldots{}

苏樱怔了怔.道:``什么时候走的?''

小鱼儿道:``就在刚刚你笑得最开心的时候。''他故意叹了口气,接着道:``现在,只怕你也笑不出了吧。''谁知苏樱眼珠子一转却又笑了。

她笑着道:``他们偷走的不是两包东西,是两个人。''这下子小鱼儿倒真的怔往了,失声道:``偷走了两个人?是活人?''苏樱道:``不能算活人,但也不能算死人,只能算是两个半死不活的人。''小鱼儿长长吐出口气,通:``看来这夫妻两人的确是有点毛病''苏樱忽又笑道:``但他们却等于帮了你一个忙。''小鱼儿又怔住了。

苏樱接着道:``他们偷去的两个人中,有一个就是要和你拼命的仇人。''小鱼儿的一颗心开始往下沉,嘎声道:``你\ldots\ldots 你,你是说\ldots 花无缺?苏樱笑道:''不错!"小鱼儿就像是─只被人踩着了尾巴的猫,跳起来大叫道。

``你说花无缺被人偷走了?你为什么不早说?''苏樱苦笑道:``我怎知他被人偷走?你为何不早些告诉我?''小鱼儿突然左右开弓,打了自己两个耳光,道:``不错,我为何不早些告诉你!我为何不拦住他们?\ldots\ldots{}''他一面叫苦,一面就像疯了似的穿出树林去。

苏樱想拦住他时,他早已走得连影子都瞧不见了,树林里就只剩下苏樱─个人,痴痴的怔了许久,喃喃道:``苏樱\ldots\ldots 苏樱\ldots\ldots 你难道就这样让他走了么?''

她忽然像是下了很大的决心,匆匆转身奔回去屋去,嘴里还在不住的喃喃自语,道:``小鱼儿\ldots\ldots 小鱼儿\ldots\ldots 我不会让你就这样走了的,只因我知道再也找不到你这样的人了,所以无论你走到哪里,我都要找到你。''她身形刚消失在迷朦的小屋中,树林边的一棵大树下,突然有一块石头向旁边移动了起来。

石头下面竟露出了个地洞!洞里边竟钻出个人来!

他目送着苏樱身形消失,嘴角泛起一丝恶毒的微笑,喃喃道:``你用不着担心,无论那小子走到哪里,我都会帮你找着他的''山坳后的隐蔽处,忽然传出一声长嘶,原来竟有辆马车藏在那里,赶车的竟是铁萍姑。

她双眉深深地皱着,看样子倒并非完全因为等得心焦,而是因为心里实在有着太多、太复杂的心事。

突听``嗖嗖''两声,马车上的木叶,也微微摇了摇。

铁萍姑沉声道:``是前辈们回来了么?''

只听白山君的声音道:``是我们。''

白夫人的声音笑道;``你放心,你的玉郎现在正好好躺在这里哩。''铁萍姑骤然一带绳,马车便直冲了出去。

又转过几处山坳后,入山反而越来越深了,原来马车并非向山外走,反而是向山深处行。

这时马车里却传出了江玉郎的呻吟声。

他身子已缩成一团,忽而颤声道:``冷\ldots\ldots 冷,冷死我了。''但还未过多久,他却又是满头大汗,不住嘶声呼道:``热,热直热得要命。''这段路上,他竟是忽而冷得要死,忽而热得要命,也不知折腾了多少次,白夫人不禁摇头叹息,道:``那丫头也不知下了什么毒,竟将这孩子折磨成如此模样。''白山君忽然冷笑道:``这小子和咱们既非亲,又非故,只不过是慕名投奔而来的,你又何苦为他如此难受!''白夫人摸了摸他的脸,嫣然道;``傻老头子,你以为我真是为了他难受么?我只不过是觉得那丫头的手段太厉害了而已,你瞧咱们这位花公子\ldots\ldots{}''白山君竟也叹了口气,道:"这姓花的如此模样,才实在是令人担心。花无缺竟似已变得痴了。他痴痴地坐在那里,不言不动,目光中也是一片茫然之色,就像是全身都已麻木,什么知觉都没有。

此刻花无缺简直和死人一般无二,只不过比死人多了口气面已,别人无论问他什么,他似乎完全没有听见。

森森林木中,竟有间小小的石屋,像是昔日苦行僧人面壁修行之地,却被白山君寻来作藏匿之处。

花无缺竟是被人抱进来的。他非但听不见别人的话,竟连路都不会走了。

白夫人瞧着他,皱眉道:``你看他是真的已变得如此模样,还是装出来的?''白山君道:``这倒难说得很''

铁萍始一直抱着江玉郎,坐在石屋外的树下,她竟还是不敢面对花无缺,竟不敢进来。

此刻白山君目光闪动,忽然冲出去,道;``他现在是发冷还是发热?''铁萍姑叹了口气,道:``他现在只觉全身都在疼,也不知是话未说完,突觉双肩一麻,左右肩头上的''肩井"大穴,竟已被白山君闪电般出手点住。

白山君道:``听说你是从移花宫中逃出来的,是么?''铁萍姑咬了咬牙,道:``你\ldots\ldots 你既然已知道,为何还要来问我。''白山君狞笑道:``既是如此,我就借借你的身子一用。''他竟抓起铁萍她的头发,一把提了起来。

铁萍姑怀里的江玉郎,立刻呻吟着跃在地上,却颤声笑道:``无\ldots\ldots 无妨,前\ldots\ldots 前辈只管借去吧!''这人果然是又狠又毒,到了什么样的时候,就说什么样的话,知道呼痛也没有人理他时,他也就不喊疼了白山君拉着铁萍姑冲进石屋,冲到花无缺面前,厉声道:``你认得这女子是谁么?''花无缺眼睛直直地瞧着铁萍姑,既不摇头,也不点头。

白山君狞笑着,他的手突然一撕,将铁萍姑前胸的一片衣襟撕下,露出了那初为妇人后,丰满而柔软的胸膛。

铁萍姑紧紧咬着牙,既末哀求,也未惊呼,只因她早已学会逆来顺受,知道呼救哀求都没有用的。

花无缺坐在那里,面上也是全无表情,一双眼睛也还是瞪得大大的,茫然瞧着铁萍姑。

白山君厉声道:``你还不认得她?好,我再叫你瞧清楚些!''只听``嘶、嘶''几声,铁萍姑处子般苗条坚挺,却又有妇人般成熟诱人的胴体,已赤裸棵站在花无缺的面前。

她两条修长而紧夹在一起的腿,已和胸膛同样在深山空林的寒风中,微微颤抖了起来,她目中虽已流出了羞侮委屈的眼泪,却又流露出火一般的悲愤和怨毒,恨根地瞪着白山君。

白山君却只是瞪着花无缺的眼睛。

但花无缺的目光却丝毫没有回避,还是茫然瞪着铁萍姑,那诱人的胸膛,那光滑的小腹,那修长的腿\ldots 在花无缺眼里,竟好像完全是木头似的。

白山君怒道:``你眼见你的同门这般模样,还是不闻不问,也不怕将你们移花宫上上下下的人全都丢光了么?''他吼声虽大,花无缺却似连一个字都末听见。

白山君狞笑道:``好,你既不怕丢人,我索性让你人再丢大些。''他抱起铁萍姑赤裸的身子,竟要\ldots\ldots。

\hypertarget{ux7b2cux516bux5341ux4e5dux7ae0-ux5b88ux682aux5f85ux5154}{%
\chapter{第八十九章
守株待兔}\label{ux7b2cux516bux5341ux4e5dux7ae0-ux5b88ux682aux5f85ux5154}}

白夫人一直在含笑旁观,这时才走过来,拍拍白山君的肩头,笑道:``够了够了,你难道真想假戏真做,来个假公济私、混水摸鱼不成,这出戏再唱下去,我可要吃醋了。''她又拍了拍铁萍姑的身子,笑道:``这只是在唱戏,你莫生气。''铁萍姑闭上眼睛,眼泪终于一连串流了出来。

白夫人皱眉道:``你看你这死老头子,把人家小姑娘气成如此模样。''白山君哈哈笑道:``她若生气,不妨把我的衣服也脱光就是。''白夫人解下外面长衫,将铁萍姑包了起来,柔声道:``男人看见漂亮女人,总不免想占占便宜的,你也用不着难受。\ldots.''她将铁萍姑抱出去,轻轻放在江玉郎身旁,笑道,``还是你们小两口亲热亲热吧。''她也不知是有意,还是无意,竟未解开铁萍姑的穴道,像是知道铁萍姑经过这番事后,就会偷偷逃走的;江玉郎虽已疼得面无人色,却还是佯笑道``到底是小孩子。人家开开玩笑,就要哭了。''铁萍姑忍不住痛骂道;``你\ldots 你\ldots\ldots 你究竟是不是人?''江玉郎目光转处,见到白山君夫妻都在屋子里没有出来,他这才长长叹了口气,压低声音道;``人在屋檐下,不得不低头,我们现在落到如此地步,若是还要逞强,还想活得下去么?''铁萍姑咬牙道:``我不怕死,我宁可死也不愿被人像狗一样的欺负。''江玉郎道:``不怕死的,都是呆子。但你可想报仇出气么?''铁萍姑道:``当然。''

江玉郎微笑道:``那么你就该知道,死人是没法子报仇出气的!''白山君夫妇坐在屋子里,你看着我,我看着你,神情都不免有些沮丧,他们辛辛苦苦,绞尽了脑汁,才将花无缺从苏樱那里又偷了回来,为的自然只是想再设法从花无缺口中探出秘密。

而此刻他们的苦心竟全都白废了。

白夫人长长叹了口气,站起来定出了屋子,白山君也没有心情来问她要到什么地方去了,只是瞪着花无缺苦笑。

过了半晌,突听白夫人在外面惊呼道:``你快出来瞧瞧,这是什么?''白山君箭一般冲出屋子,只见江玉郎和铁萍始并头躺在那里。像是睡着了,白夫人却站在树下发呆。

树下面什么都没有,只有一堆落叶而已。

白夫人面上却显得又是惊奇,又是兴奋,道:``瞧这是什么?''只见落叶堆里,有个小小的洞窟,像是兔窟,又像是狐穴。

白山君道:``但这只不过是个洞而已,你难道从来没有瞧见过一个洞么?''白夫人忽然扭过头,瞪大了眼睛瞧着他,就好像白山君脸上忽然生出了一棵银杏树来似的。

白山君笑道:``你难道连我都从来没有瞧见过。''她竟弯下腰,将洞旁的落叶都扫了过去,只见这地洞四面,都十分光滑平整,而且下面没有别的出路。

白大人道:你再仔细瞧瞧这个洞。白山君动容道:``我懂了!这个洞是人挖出来的!''白夫人拍手道:``这就是了,但这么小的洞,又有谁能藏在里面?''白山君皱眉道:``但他已有二十年没露过面,听人说早已死了。''白夫人淡淡道:``你想,像他这种人会死得了么?谁能杀得了他?''白山君叹了口气,道:"不错,好人不长命,祸害活千年。

白夫人吃吃笑道:``你还在吃他的醋?''

白山君板着脸道;``就算你的老情人快来了,你也用不着在我面前笑得如此开心。''白夫人勾住丁他的脖子,悄笑道:``老糊涂,我若是喜欢他,又怎么会嫁给你?\ldots\ldots 来\ldots\ldots{}''白山君却一把推开了他,大声道:``不来。''

白山君狠狠在那堆落叶上踢了一脚,又道:``想起这小子说不定就在左右,我什么兴趣也没有了。我要留在这里。''白夫人道:``为什么?''

白山君一字字道:``守株待兔。''

江玉朗简直难受得快死了,哪里能真的睡着──他只不过是闭起了眼睛,在装睡而已。

他听到这夫妻两人竟为了地上有个洞而大谅小怪,心里也不免很觉惊奇,听到这夫妻两人在打情骂俏,又觉得好笑,再听到他们说这小洞里竟能藏人,他几乎忍不住要失声问了出来:``这么小的洞,连五岁小孩子都难以在里面藏身,一个大人又怎么能藏得进去呢?难道这人是侏儒不成?''最后他又听到白山君说:``守株待免!''

江玉郎心念一闪,暗道:他们等的这人,莫非就是十二星象中的兔子不成?``要知道''十二星象"虽是江湖巨盗,武林煞星,但偏偏又觉得做牛做马,大是不雅,所以又引经据典,为自己找了个风雅的名字。

鼠号``无牙''、牛号``运粮''、虎乃``山君''、兔号``捣药''、龙为``四灵之首''、蛇乃``食鹿神君、猪为''黑面``、马虽名''踏胃``,又号虎妻''、羊号``叱石''、鸡乃``司晨''、猴名``献果''、狗号``迎客'',这十二个风雅的名字,正是出自诗韵``十二星象''中的``兔子''姓胡,自号``蟾宫落药''取的自然就是``月中捣药'',却始终不知道这人是男是女。

只因江湖中简直就没有几个人能瞧见过这胡药师真面目的,所以根本没有人知道他长的是何模样!

白山君果然坐在树下,``守株待兔''起来。

白夫人静静地瞧了他半晌,忽然一笑,道:``你在这里苦苦等着,免子若是不来呢?''白山君道:``他既已来过,必然知道你会回到这里,有你在这里,他还会不来么?\ldots\ldots 嘿嘿,说不定他早已在暗中你偷跟着咱们,想等机会见你一面。''白夫人吃吃笑道:``我已经是老太婆了,还有什么好看的?''白山君冷笑道:``情人眼里出西施,别人看来,你虽然已是老太婆,但在他跟里,你说不定还是个小美人哩。''听到这里,江玉郎实在觉得好笑,他想不到这一对老夫老妻,居然还在这里拿肉麻当有趣。

突听白山君一声轻呼,道:``来了!''

江天朗再也忍不住张开跟,偷偷一望,只见一段比人头略为粗些,叁尺多长的枯木,远远滚了过来。

这段木头不但能自己在地上滚,而且还像长着眼睛似的,遇到前面有木石阻路,它居然自己就会转弯.深山荒林之中,骤然见到这种怪事,若是换了平时,江玉郎就算胆子不小,也一定要被吓出冷汗来的。

但现在他已知道这段枯木必定与那胡药师有关,已猜出胡药师说不定就藏在这段枯木里,所以也不觉得有什么可怕了,只不过有些奇怪而已:``这段木头比枕头也大不了多少,人怎能藏在里面?''白山君却眨也不眨地瞪着这段枯木,眼睛似乎要冒出火来,两只手也紧紧捏成了拳头。

白夫人轻轻按住了他的手,娇笑道:``老朋友许久不见,可不能像以前一样,见面就要打架。''那段枯木竟哈哈一笑,道:``多年不见,想不到贤伉俪居然还恩爱如昔,当真可喜可贺。''白山君大声道:``你怎知道咱们还恩爱如昔,你莫非一直在暗中偷看''那枯木笑道,``若非恩爱如昔,怎会有这么大的酸劲,这道理自是显而易见.根本用不着看的,是么?''笑声中,这段枯本已滚到树下。

枯木中竟忽然伸出个头来。

江玉郎虽然明知木头里有人,但猝然间还是不免吓了─跳──枯木上忽然生出个人的头来,这无论如何,都是件非常骇人的事。

只见这颗头已是白发苍苍,但颔下胡子却没有几根,一双眼睛又圆又亮,就像是两粒巨大的珍珠。

最奇怪的是,这颗头非但不小,而且还比普通人大些,枯木虽然中空,但这人头塞进去,还是紧得很。

不但头大,耳朵更大,而且又大又尖,和兔子的耳朵几乎完全一摸一样,只不过大了两倍。

一个侏儒,又怎会有这么大的头,这么大的耳朵?

江玉郎不由得更吃惊了,虽然还想装睡,却再也舍不得闭起眼睛,再看铁萍姑,眼睛又何尝不是瞪得大大的。

白夫人吃吃笑道:``十多年不见,想不到你还是如此顽皮?''这人哈哈─笑,道:``这就叫,江山易改,本性难移。''白山君冷笑道:``你若以为女人还喜欢顽皮的男人,你就错了。''这人笑嘻嘻道:``哦,现在的风气难道改了么?我记得顽皮的男人一向是很吃香的。''白山君道:``顽皮的男人,自然还是吃香的,但顽皮的老头子\ldots\ldots 嘿嘿,让人见了只有觉得肉麻,觉得恶心。''白夫人见到现在还有男人为她争风吃醋,心里实在说不出的开心:``看来我还没有老哩。''但面上却故意做出生气的模样,板着脸道:``你们两人谁若再斗嘴,我就不理谁了.''白山君大吼道:``你莫忘了,我是你的老公,你想不理我也不行。''白夫人娇笑道:``你瞧你,我又没有真的不理你,你何必紧张得这样子。''只见她眼睛发亮,脸也红润起来,像是忽然年轻了十几岁。

那人叹了口气,笑道:``白老哥,看来你真是老福气,看来只怕等你进了棺树,我这小嫂子还是年轻得跟大姑娘似的。''白山君怒吼道;``你想咒我死么?就算我死了,也轮不到你。''吼声中,一拳击了出去。

只听``蓬''的一声,那段葳木竟被他拳风震得粉碎,一个人自枯木中弹了出来,``嗖''的,穿上树梢。

江玉郎竟连这人的身形都没有瞧清楚。

只见这人一颗大脑袋从树叶里探了出来,笑嘻嘻道:``人无害虎心,虎有伤人意''\ldots 但白老哥,我这次来,可不是为了来和你打架的。``白山君吼道:''你是干什么来的?我这老虎虽不吃人,吃个把兔子却没关系。``那人悠然笑道;''你若伤了我,只怕这辈子再也没耳福听到移花接玉的秘密了。``白山君怔了怔,脸上立刻堆满了笑容,大笑道:''胡老弟,你和我老婆是老朋友了,难道忘了她的脾气?``那人道:''她的脾气怎样?"

白山君道:``她最喜欢别人为她吃醋,我既然是她的老公,自然时常都要想法子让她开心,其实\ldots\ldots{}''话未说完,``吧''的,脸上己挨了个掴子.白夫人瞪着眼道:``其实怎样?''白山君也不生气,笑嘻嘻道:``其实我也是真喜欢你的,只不过也很喜欢那移花接玉。''白夫人眼珠一转,也笑了.她又向树上一瞪眼睛,笑骂道:死兔子,你还不跟老娘下来么?``那人大笑道:''是,老娘,我这就下来了。"

他随着笑声一跃而下,哪里是侏儒,竟是个昂着七尺的伟丈夫,看来比白山君还高一个头。

江玉郎瞧得眼珠子都快掉出来了,他实在想不出这么大一个人,怎能藏入那么一小段枯技中去。

突见白山君走过来,望着他笑道,``原来你早已醒了。''江玉郎连脸都没有红,笑道;``弟子迷迷糊溯的,并没有睡得很沉。''白山君道:``告诉你,这位就是名满天下的胡药师,江湖中人,谁不知道胡药师锁子缩骨功,乃是武功绝传.天下无双。,江玉朗失声道:''锁子缩骨功?难道就是昔年无骨道人的不传之秘么?白山君笑道,``算你小子还有些见识,现在你总该明白了吧。江玉郎道:''弟子明白了。白山君忽然一瞪眼睛,道:"既然明白了,还不快走远些,难道也想听那秘密?他心里虽一万个舍不得走,但又非走不可,铁萍姑也咬着牙站起来,扶着他走入那石屋里。

有风吹过,吹起铁萍姑身上的袍子,露出了一双修长笔直坚挺,白得令人眼花的玉腿。

胡药师的眼睛似乎发直了,笑道:``这小姐儿的腿可真不错。''白山君走过去,悄声笑道;``她不但腿长得好,别的地方\ldots\ldots 嘿嘿。''话未说完,耳朵忽然被人拧住。

白夫人咬着牙笑骂道;``老色鬼,看你如此不正经,在外面一定瞒着我也不知搞了多少女人了,是不是?快说!''胡药师笑道:``据我所知,白老哥对你倒一向是忠心耿耽的。''白夫人瞪了他一眼,道:``你用不着为他求情,你也不是好东西。''胡药师道;``哎哟,那你可真是冤枉好人了。''白夫人``噗哧一笑,放了手,笑道:''男人呀\ldots{}``十个男人,倒有九个是色鬼。''白山君抚着耳朵,笑道:``闲话少说,言归正传,胡老弟,你可真的知道那秘密么?''胡药师大笑了几声,才接着道:``我瞧见你们将魏老人的大徒弟魏麻衣拉到这里来,喃喃了半天,又叫他去找一个姓苏的女子。''白夫人道:``苏樱,就是魏老头的命根子,你不知道么?''胡药师笑道:``现在我自然知道了,当时我却很奇怪,你们自已有路,为何叫别人去走,后来我又瞧见你们也在暗中悄悄跟了去。''白夫人道:``那丫头不愿学武,但魏老头的消息机关之学,却全都传给了她,而且据说青出于蓝,比魏老头还要高明得多!''

\hypertarget{ux7b2cux4e5dux5341ux7ae0-ux5de7ux8ba1ux5b89ux6392}{%
\chapter{第九十章
巧计安排}\label{ux7b2cux4e5dux5341ux7ae0-ux5de7ux8ba1ux5b89ux6392}}

胡药师接着道:``我对消息机关之学总是学不会,所以也不敢胡乱走动,就找了地方躲起来。过了半晌,就瞧见魏麻衣将一个小伙子骗到我躲着的树林里去,而且还将那小伙子点了穴道,吊了起来。''白山君奖道:那时我们远远听得有人在骂街,想必就是那小伙子在骂魏麻衣了。``白夫人皱眉道:''这小伙子长得是何模样?"

胡药师道:``年纪大约二十不到,身材和我差不多,满脸都是伤疤,应该其丑不堪,但也不知怎地,却看来一点也不讨厌,反而很讨人喜欢。''白夫人道:``据说近年来江湖中出了个小魔星,叫什么鱼的,好像是小鱼,此人武功虽不十分高,但却精灵鬼怪,又奸又滑,只要惹着他的人,没有不上他的当的,连江别鹤那样的人,见了他都头疼。''胡药师默然半晌,微笑道:``不错,那小伙子就是此人,他实在是个鬼精灵,魏麻衣也算是个厉害角色了,但后来却被他捉弄得团团乱转''\ldots\ldots{}``白山君忍不任插口道:''这人又和移花接玉的秘密有何关系?``胡药师道:''我问你,现在天下有几个人知道移花接玉武功的秘密?``白夫人道:''知道的人虽也有几个,但会说出来的人却一个也没有。``胡药师笑道:''这就对了,不过,现在我却有个法子令其中一人说出来。``白夫人道:''你能让谁说出来?"

胡药师道;``苏樱!''

白夫人叹了口气道:``你若能令那丫头说出来,我就能令瓶子也开口了。胡药师微笑道:''你不相信?"

白夫人又叹了口气,道,``好吧,你有什么法子,且说来听听。''胡药师沉声道:``我这法子,就着落在那条小鱼的身上。''白夫人皱眉道:``这是什么法子?我不懂。''

胡药师道:``那姓苏的丫头,已对小鱼着了迷,只要我们能抓着那条小鱼,无论要苏樱说什么,她都不敢不说的。''白夫人道;``这法子只怕靠不住吧,据我们所知,那丫头的心比石头还硬,天下简直没有一个男人能让她瞧在眼睛里。''胡药师道:``一定行得通的,我亲眼瞧见过它行通了。''白夫人悠悠道;``只不过,咱们若想让那条小鱼入网,只怕还不容易。''胡药师哈哈笑道:"这张网可就要嫂子你来做了.。

白夫人嫣然一笑,向他送了眼波,道,你放心,越是调皮的男人我越有法子对付的。"花无缺还是痴痴地坐夜石屋里,就像是个本头人。

江玉郎和铁萍姑走进来时,外面正在讨论她那一双玉腿,听得这亵猥的笑声,铁萍姑眼泪不禁又快落了下来。

铁萍姑忽然紧紧抓住江玉郎的手,嘎声道:``我们为何不乘这时候逃走?''江玉郎道:``你若─个人逃走.也许还可以逃出两叁里去,但还是要被抓住,你若背着我,只怕连半里路都逃不出。''铁萍姑道:``那么你\ldots。你想怎样?''

江玉郎道:``等着,等机会,忍耐,拼命忍耐\ldots.''他忽然一笑,接道:你可知道。若论这忍耐的功夫,普天下只怕没有一个人能比得上我。``这话倒当真不假,此人当真是又能狠,又能忍,否则多年前他只怕已死在''迷死人不赔命"萧咪咪的地府中了。

铁萍姑垂下头不再说话。这时白山君夫妇和胡药师已大步走入。

白夫人一直走到江玉郎面前,轻轻去揉他的双肩,柔声道:``这样还疼不疼?''江玉郎道:疼\ldots\ldots 疼还是疼的,只不过已\ldots\ldots 已像是好些\ldots\ldots\ldots."话末说完,忽然杀猪般的惨叫起来。白夫人揉着他肩头的一双手,竟忽然贯注真力。

江玉郎的疼虽有一半是在装假,也有一半是真的,此刻白夫人掌上真力,由他左右双肩的穴道里逼了进去,他全身立刻宛如无数根尖针所刺,上上下下,所有骨节像是都散了。

白夫人还是满面笑容,柔声道:``你是不是觉得舒服了些?''江玉郎惨呼道;``求求你\ldots\ldots\ldots 放\ldots 救手\ldots\ldots{}''铁萍姑也冲了过来,向白夫人扑了上去。但白山君出手如电,已把她手臂拗了过来。

白夫人笑道:``我只不过揉了操他骨头,你已如此心疼,我若杀了他,你岂非要发疯?''其实铁萍姑现在已要发疯了,疯狂般大呼道;``你们不能这样\ldots。你们不能\ldots\ldots{}''白夫人悠悠道:``只要你答应帮我们一件事,我就立刻放了他。''铁萍姑想也不想,立刻道:``我答应,我答应\ldots\ldots{}''白夫人叹了口气,喃喃道:``想不到男女之间,爱的力量竞有这么大。''她终于放了手,轻轻拍了拍江玉郎的脸,又笑道:``小伙子,看来你只怕真有两手,能令一个女人如此死心踏地的跟着你,这本事可真不小。''胡药师忽然笑道:``苏樱对那条小鱼着迷的程度,比她还厉害得多。''白山君大笑道:``如此说来,咱们这件事是必然行得通了。''白夫人道;"现在你留在这里,这两人就都交给你了\ldots\ldots{}

折山君道:``你只管放心就是。''

铁萍姑还伏在江玉郎身上,轻轻啜泣着。

白夫人拉起了她,道:``你跟我走吧\ldots。但你千万要记住,你若是不听话,坏了我们的大事,你这情郎就要死在你手上了!''小鱼儿心里虽然急得像火烧,但走得并不快。

他知道走快也没有用的,走快了反而会错过一些应该留意的事,但他现在却连丝毫线索也不能错过。

夜晚虽已过去,但半山云雾凄迷,目力仍起难以及远,远处的木叶都似飘浮在云雾里,瞧不见枝干。

连哈哈儿、李大嘴等人留下的暗号,现在都很难找得到,要想追查武林高手留下的足迹,自然更是难如登天了!

但遇着越是困难的事,小鱼儿反而越是沉得住气,他先找了个小溪,在溪水里洗了洗脸,又定下心来,运气调息了片刻,看看自己的伤势是否巳痊愈。

他真气活动了一遍,觉得自己已和未受伤前没有什么两样,只不过躺在床上太久,脚下有些轻飘飘的。

他不禁微笑起来,喃喃道:``那丫头将我受的伤说得那般严重我就知道她是在吓我,不让我走\ldots\ldots 唉,女人,谁若相信女人的话,谁就要一辈子做女人的奴隶。''但想到苏樱的温柔与情意,他心里还是不免觉得甜甜的,无论如何,一个人若被别人爱上,总是件十分愉快的事。

魏无牙的洞府在西面一个隐密的山洞里。

小鱼儿虽然天不怕,地不怕,但刚吃了魏无牙的一个大亏,余悸犹在,还是不敢往西面去。

他坐在溪旁的石头上,出了半晌神,正不知自己该往哪里去找花无缺,突见溪水上游,有样红红的东西随波流了下来。

小鱼儿既然不肯放过任何线索,此刻自然也不肯错过这样东西,他立刻折了段树枝,跃到前面一块石头上,将这件东西挑起来。

原来这竟是条女人的裙子,上面还绣着花,做工甚是精致,看来像是大家妇女所穿着的。

但裙腰处却已被撕裂了,竟似被人以暴力脱下来的。

小鱼儿皱眉道:``如此深山中,怎么有穿这种裙子的女人?这女人难道遇上了个急色鬼?''他本来以为这又是魏无牙门下的杰作,但魏无牙的洞府在西面,溪水的上游却在东南方。

就在这时,溪水中又有样东西飘了过来,也是红的。这却是一双女人的绣花鞋。

但现在小鱼儿不但已动了好奇心,而且也动了义愤之心,只觉这急色鬼未免太不像话了,好歹也得给他个教训才是。

溪水旁有一块块石头,上面长满了青苔,滑得很,但以小鱼儿的轻功,自然不怕滑倒。

他从这些石头上跳过去,走出叁五丈后,又从水里挑起个鲜红的绣花肚兜,更是已被扯得稀烂。

小鱼儿皱眉道:``好小子,你不觉这样做得太过份了么?要知女人虽然大多不是好东西,但欺负女人的男人,却更不是好东西。''又往前走了一段,水里竟又飘来一只肚兜。这只肚兜是天青色的,也已被撕裂。

小鱼儿失声道;``原来还不止一个女人,竟有两个。''他脚步反而停了下来,他忽然觉得,深山之中,绝不会跑出这么样两个女人的,穿着这种裙子的女人,在大街上都很难遇得到。

就在这时,上游处传来了一声惊呼!呼声尖锐,果然是女人的声音。

小鱼儿站在石头上,又出了半晌神,嘴角竟露出一丝神秘的笑容,喃喃道:``女人,女人\ldots\ldots 为什么我无论走到哪里,都会遇见些奇怪的女人呢?''溪水尽头,有峰冀然,一条瀑布,自上面倒挂而下,下面却又有一块巨石,承受了水源。

瀑布灌注巨石上,方自四面溅开,落入溪流中。那巨石上却有两个女人。

她们的身子竟已几乎是全裸着的,飞瀑自强巅直灌而下,全都冲激在她们身上,这般水力,显然是十分强大的。

她们修长而结实的玉腿,已被流水冲激得不住伸缩痉挛,满头秀发,乌云般散布在青灰色的石头上。

小鱼儿到了这里,也不禁瞧得呆住了。

这景象虽然惨不忍睹,却又充满了一种罪恶的诱惑力,足以使全世界上任何一个男人面红,心跳,不能自已。

水雾、流云、清泉、飞瀑、赤裸的美女,惨无人道的酷刑\ldots 这简直荒唐离奇得不可思议。

小鱼儿喃喃道:``这是谁干的事?这人简直是个天才的疯子!''只听那两个女子不住的呻吟着,似已觉出有人来了,颤声呼道:``救命''\ldots 救命\ldots。.``小鱼儿大声道;''你们自己不能动了么?"

那女子只是不住哀呼道:``求求你\ldots 救教我们!''小鱼儿道:``是谁把你们弄成这样子的?他的人呢?''那女子呼声渐渐微弱,嘴里像是在说话,但小鱼儿连一个字也听不清,他现在站的一块石头距离她们还有两丈远近。

两丈多距离以小鱼儿的轻功,自然一掠而过,天下所有的男人,若有他这样的功夫,若瞧见这样的情况,却一定会掠过去的。

谁知小鱼儿既不救人,也不走。

他竟在石头上坐了下来,瞪着眼睛瞧着──这做法实在大出常情常理,除了他之外,世上再也没有第二个人会做得出来。

石头上的女人,自然就是白夫人和铁萍姑。现在,白夫人也怔住了。她所安排的每一个计谋,每一个陷阱,本都是奇诡、突兀、周密,有时几乎是令人难以相信的。

她所布置的每一个计划中,都带着种残酷的、罪恶的诱惑力,简直令人无法抗拒,不得不上当。

这一次,她知道对方也是个聪明人,自然更加倍用了心机,她算准无论是谁,被人在树上吊了许久,一定要喝些水──尤其是聪明人,更会找个地方喝水的,因为聪明人在办事之前,总会令自己心神冷静下来。

只要是男人,瞧见溪水中有女人被强暴的证物流过来,都会忍不住要溯流而上,瞧个究竟。

于是她就在这里等着,展露着她依然美丽诱人的胴体,她认为天下绝没有一个男人,瞧见这情况而不过来的。

但她还是不能完全放心,还是怕岁月已削弱了她胴体的诱感力,所以她又将铁萍姑也拉了下来。

她知道``小鱼儿''这名字,就是从江玉郎嘴里听来的,自然也知道铁萍始曾经救过小鱼儿一次。

因为江玉郎去投靠他夫妻时,她不但仔细盘究过江玉郎的来历,对江玉郎带来的这女孩子更没有放松。

江玉郎为了取信于她,只有将有关铁萍姑的每一件事都说了出来──江王郎自然绝不会为别人保守秘密。

所以她更认为小鱼儿绝没有不过来的道理。滴水尚且能穿石,何况奔泉之力;这块石头自然已被飞瀑冲得又圆又滑,只有在石头的中央,有一块凹进去的地方,其余四边滑不溜足。

任何人也没法子在这上面站得住脚。

白夫人就躺在这块凹进去的地方,只要小鱼儿到这块石头上来救她,她只要轻轻一推,小鱼儿就要落入水里去。

而胡药师此刻就潜伏在水下,将一枝芦苇插在嘴里,另一端露出水面,以通呼吸,小鱼儿一掉下水,就等于鱼入了网──一个人落水时,自然免不了手脚舞功,空门大开,胡药师却是全神贯注,自然是手到擒来。

奔泉之下,滑石之上,这地势又是何等凶险,小鱼儿就算有天大的本事,只要一过来,也设法子不掉下去。

白夫人先将自己安排在这种险恶之地,正是置之死地面后生的绝计,但她简直连做梦也未想到,小鱼儿竟既不过来也不走,竟只是远远坐在那里瞧着,简直就好像在看戏似的。

再看小鱼儿悠悠闲闲地坐在那里,竟脱下鞋子,在溪水中洗起脚来,面上神情,更是说不出的开心得意。

又过了半晌,他居然拍手高歌起来!

"有清泉兮濯足,不亦乐乎!

有美人兮娱目,不亦乐乎!

人生至此,夫复何求?"

白夫人听得简直气破了肚子,忍不住切齿骂道:``这小于简直不是人\ldots 他难道已瞧破了我的计划吗?''后面一句话,自然是在问铁萍姑,只因此间水声隆隆如万蹄奔动,她说话的声音就算再响些也只有铁萍姑能听得到。

铁萍姑本是满心羞怒,这时却不禁暗暗好笑,故意道:``他一定已看破了。''白夫人恨声道:``这计划可说是天衣无缝,他怎么瞧破的呢?''铁萍姑道:``有许多人都说他是天下第一个聪明人,这话看来竟没有说错。''她功力本不如白夫人,本已被奔泉冲压得无法喘息,但此刻心情愉快,不但能将话一口气说了出来,而且说得声音还不小。

白夫人冷冷道:``你可是想向他报讯么?但你最好还是莫要忘记,休的情郎是在我手里,这件事不成,你就要做未过门的寡妇了。''

\hypertarget{ux7b2cux4e5dux5341ux4e00ux7ae0-ux5c06ux8ba1ux5c31ux8ba1}{%
\chapter{第九十一章
将计就计}\label{ux7b2cux4e5dux5341ux4e00ux7ae0-ux5c06ux8ba1ux5c31ux8ba1}}

一提起江玉郎,铁萍姑的心立刻就沉了下去,她虽不愿小鱼儿上当,但却更不忍让江玉郎死,铁萍姑再也不敢开口。

过了半晌,白夫人却又问道:``我知道你救他一次,是么?''铁萍姑道;``嗯。''

白夫人道:``现在他为何不来救你?''

铁萍姑道:"也许\ldots\ldots 也许他没有认出我\ldots\ldots{}

白夫人沉吟着道:``不错\ldots\ldots 男人瞧见一个赤裸的美女时,眼睛就只会瞪着她的身子,往往就不会去瞧她的脸了。''铁萍姑的脸火烧般飞红了起来,她忽然感觉到小鱼儿的眼晴像是一直瞪着她,她恨不得立刻掩起自己的胸膛,自己的腿\ldots\ldots 但为了江玉郎,她却连动也不敢动。

白夫人冷冷道:``现在,你赶紧将头偏过去一些,叫两声救命\ldots\ldots 叫得声音不能太响,但也不能太小,要做出声嘶力竭的模样知道么?''铁萍姑立刻嘶声呼道:``救命\ldots\ldots 救命\ldots\ldots{}''

她将头偏过去一半,竟发现小鱼儿已洗完了脚,手支着头,半躺在那块石头上,竟像是已睡着了。

白夫人自也瞧见了,切齿道:``好个小贼,他心里究竟在打什么主意?''只听得石头下一个人道:``我说的不错吧,这条鱼是很难入网的。''原来胡药师也忍不住了,自水里露出大半个头来。

白夫人赶紧道:``快下去,莫被他瞧见。''

胡药师笑道:"他就算有天大的本事,难道目光还能拐弯么?

怎能瞧到石头后面来?"

白夫人叹了口气,道:``依你看,他是不是已瞧破这计划了呢?''胡药师道:``那么他为何不过来?''

白夫人道:``这小子也许是天生的多心病,对任何事都有些疑心,所以先不过来,在那边耗着,看咱们是什么反应?''胡药师苦笑道:但咱们在这里受罪,他却在那边享福,这样耗下去,咱们怎么能耗得过他?``白夫人道:''不耗下去又能怎样?这小子简直比鱼还滑溜,这次咱们若被他瞧破,下次再想要他入网更是难如登天了。``胡药师长长叹了口气,道:''既是如此,看来咱们只好和他耗下去了,但你又还能耗多久呢?``白夫人默然半晌苦笑:''事到如此,只有耗一刻是一刻了\ldots\ldots{}

谁知就在这时,小鱼儿突然站了起来。

白夫人又惊又喜,嘎声道:``快下去,鱼只怕已快上钩了。''胡药师不等她说完,于是就已潜入水中,将那芦苇又探出水面。

只听小鱼儿喃喃道:``这只怕不是做假的,否则她们一定忍不了这么久。''一面说着话,一面已套上鞋子,又将脚伸入水里泡了泡,显然也是怕那边石头上太滑,所以先将鞋底弄湿.白夫人知道他立刻就要来了,心里的欢喜真是没法子形容,铁萍姑却几乎忍不住要哭起来。

这时她几乎已忘了江玉郎,几乎忍不住立刻就要放声大呼,叫小鱼儿莫要过来上当,只不过是在这种生死存亡的一刹那间,潜伏在人们心底深处的道德心,往往会忽然战胜私心利欲。

只可惜白夫人也深深了解这一点的,竟一字字沉声:``记住,莫忘了你的情郎。''铁萍姑心里一寒,猛然咬住了自己的舌头,只觉一阵痛彻心腑;呼声虽未唤出,眼泪却流了出来。

突听小鱼儿大呼道:``姑娘们莫要害怕,我来救你们!''呼声中他身形已跃起,向这边石头上窜了过来。

小鱼儿蓄气作势,准备了许久,白夫人只道他这一跃必定是身法轻灵,姿态美妙,谁知他身法既不轻灵,姿态也难看得很。

一个人费了许多苦心气力张网,总希望能捕着条大鱼,这条鱼"看来竟真的小得很。

白夫人暗中叹了口气:``聪明人果然大多是不会用苦功的,早知他功夫这样糟,我又何苦白费这么多力气。''心念闻动间忽听``噗通''一声,水花四溅──小鱼儿这一跃竟没有跃上石头,竟跌到水里去了。

又听得``咕嘟咕嘟''几声,他竟像是被灌了几口水下去,从鼻子里向外面直冒水泡,到后来竟放声大呼起来。

救命\ldots。救命\ldots\ldots 淹死我了``\ldots.''

来救人的人,此刻反而喊起救命来。

白夫人真是又好气,又好笑,她实在想不到这小子非但武功糟透,而且水性比武功更糟.这时小鱼儿这呼救声都已发不出,却有一连凉气泡泡从水里冒出来,眼看这条小鱼儿竟要被淹死。

白夫人暗骂道:``若不是我还用得着你,今天不让你活活淹死才怪。''她这时已不再顾忌,正想坐起来,但上面的水力实在太大,她力气却已快被耗尽丁,刚坐起半个身子,又被水力冲倒。

那根芦苇却已从石头后头转了过来,白夫人瞧见胡药师既然已来捉鱼了,她就索性省些力气。

水很清,胡药师在水里张开眼睛,只见这条小鱼儿此刻竟像是已变成了条落水小狗,眼见他一伸手就能捉住。

谁知小鱼儿也不知怎地一使劲,竟从水里冒了上去。

他手指像是轻轻一弹,弹出了一粒黑暗的小弹丸,竟不偏不倚,恰巧落在那根空心芦苇中;胡药师正在吸气,突觉一粒东西从芦苇中落了下来,在水里闷了这么久,他吸气的时候自然很用力,等到他再想往外面吐气时,已来不及了。

小鱼儿竟已飞快的伸出手,将这根芦苇从他嘴里拔了出来,``咕嘟''一声,这粒东西已被他吞了下肚。

只觉这东西又咸又湿又臭,还带着臭咸鱼味。刚张开嘴想吐,水已灌了进来,被灌了两口水下去后,就算吞下团狗屎,也休想吐得出了。

白夫人只听得水声哗啦哗啦``的响,正不知是怎么回事,小鱼儿已拔出了那根芦苇,顺手就点了她足底的''涌泉"穴。

等到胡药师像只中了箭的癞蛤蟆,从水里跳出来时,白夫人却己变成匹死马,躺在石头上不能动了。

只见胡药师掠到石头上,立刻张开了嘴,不停的干呕,连眼泪鼻涕都一齐被呕了出来。

再瞧小鱼儿,不知何时已回到那边的那块石上,笑嘻嘻地瞧着他们,就像什么事全都没有发生过似的。

白夫人这才知道钓鱼的人反而被鱼钓去了。

她又惊又怒,嘎声道:``快\ldots\ldots 快解开我的穴道。''胡药师一面揉眼睛,一面喘着气道:``什\ldots\ldots 什么穴道?''白夫人道:``涌泉穴。''

胡药师刚想出来,小鱼儿已在那边悠然笑道:``我若是你,我是万万不会救她的。''胡药师一只手果然在半空中停顿,嘎声道;``为什么?''小鱼儿笑道:``你现在还有救人的工夫么?不如还是先想法子救救自己吧。''胡药师面色惨变,道:``方才那\ldots\ldots{}''究竟是什么东西?``小鱼儿笑嘻嘻道:''不是毒药,难道还是大补丸么?"胡药师整个人都软了。

小鱼儿又道:``你若想我救你,最好先乖乖的坐在那里不要动''。

白夫人道:``无论如何,你先解开我的穴道再说,我们再一起逼他拿出解药来。''小鱼儿道:``就凭你们两个,连我的屁都逼不出来的。''两人你一句,我一句,胡药师已被说得怔在中间,也不知究竞该听白夫人的,还是该听小鱼儿的。

铁萍姑却瞧得又是惊奇,又是欢喜,也怔了半晌,才忽然想起:``此时不逃,更待何时?''当下一个翻身从石头上滚了下去,落在水里。

那边白夫人已经快急疯了,道:``你\ldots 为什么还不动手?''胡药师叹了口气,苫笑道:``我虽想救你,但究竟还是自己性命要紧。''白夫人瞪着眼睛,气得再也说不出话来。

这时铁萍姑已挣扎着游了过来,刚想跳到石头上,忽又想起自己身上简直是一丝不挂,怎么见得了人?

小鱼儿的眼睛却偏偏向她瞟了过来,还笑了笑。铁萍姑恨不得将头都藏在水里。

小鱼儿道:``你想叫我转过头去,是么?''铁萍姑赶紧点了点头。

小鱼儿道:``好,我就转过头去,但我却要先问你一句,你方才躺在那里也不害羞,此刻为什么忽然害羞了?''铁萍姑吃吃道:``我\ldots\ldots 我只是\ldots\ldots{}''

小鱼儿悠悠道:``你方才只是想让我上当,是么?只可惜上当的不是我,而是别人。''这句话就像是条鞭子,抽得铁萍姑脸又发了白,颤声道:``你\ldots 你怎么这样冤枉我?''小鱼儿冷笑道:``我冤枉你\ldots\ldots 哈哈,我倒要请教你,你方才身子既然能动,嘴既然能说话,为什么不警告我一声,叫我莫要上当?''铁萍姑道:``这只因我。\ldots 我\ldots\ldots.''她终于发现自己实在无话可说,眼泪不觉流了下来。

小鱼儿道:``你用不着哭,我可不是花无缺,从来没有他那样怜香惜玉的心肠,你眼泪尽管哭成河,我也不会同情你的。''铁萍妨全身都发起抖来,嘶声道:``我并没有要你多原谅,我\ldots 我也绝不会求你\ldots。.''小鱼儿忽然瞪起眼睛,大声道:``但我还是要问你,你为什么要出卖我?为什么?为什么?\ldots\ldots{}''铁萍姑忽也放声大吼起来,嘶声道:``只因为我觉得你是个自高自傲、自私自利、自命不凡的大混蛋,你自以为比谁都强,我就希望能眼见你死在别人手上!''小鱼儿呆了半晌,竟又笑了,笑嘻嘻道:``女人声音喊得越大,说的往往越不是真话,你这样说,我反而认为你不是故意害我了,你一定别有苦衷,也许我真该原谅你才是。''铁萍姑张口结舌,倒反而怔住了,只觉得这个人所做所为,所说的话,简直没有一件不是要大出人意外的。

小鱼儿缓缓接道:``这也许是因为你有什么亲近的人.落在他们手上,你为了要救那个人的性命,只好出卖我了。''他叹了口气,接着道:``若真是如此,我倒不能怪你,因为我知道女人为了她的心上人,往往会连她自己也不惜出卖的。''这句话已说入铁萍姑心里,铁萍姑眼泪忍不住又夺眶而出,她再也想不到这可恶的小鱼儿竟如此能体谅别人的苦衷,了解别人的心意。

小鱼儿柔声道:``但这人是谁呢?他值得你为他如此牺牲么?''铁萍姑流泪道:``你\ldots\ldots 你是认得他的,我不能说出他的名字。''小鱼儿面色已变了,却还是柔声道:``你说的可是江玉郎?''这次铁萍姑真的闭住嘴了。但现在闭住嘴,岂非已等于默认.小鱼儿忽然跳了起来,大吼道:``好,好,好,你竟为了江玉郎那小杂种而出卖我,你可知道这小子有多混帐,他就算被人砍头一百次,也绝不嫌多的,''铁萍姑又骇呆了。

小鱼儿瞪眼瞧着她,过了半晌,忽又叹道:``其实我还是不该怪你的,那小子满嘴甜言蜜语,莫说是你,就算比你更聪明十倍的女人.也会上他当的。''铁萍姑茫然站在水里,简直有些哭笑不得了。

只见小鱼儿已变得神平气和,笑嘻嘻站了起来,向胡药师道:``很好,你很聪明,一直没有乱动手,只是像你这般聪明的男人,却娶了一个老是爱脱衣服的老婆,实在未免有些泄气!''胡药师叹了口气.道:``我没有老婆。''

小鱼儿怔了怔,大笑道:``妙极妙极,如此说来,你简直比我想象中还要聪明了\ldots\ldots 但她这种女人若没有老公,却一定会发疯的,她的老公呢?''他眼珠子一转,立刻又笑道:``他的老公自然在看着江玉郎了,是么?''胡药师只有叹道:``正是如此。''

小鱼儿身形忽然跃起,又向那边大石头上窜了过去,这次他轻轻一掠,轻轻飘飘站在石头上绝不会再掉下水了。

白夫人咬着嘴唇,嘴唇都咬出血来。

小鱼儿笑嘻嘻瞧着她,道:``像你这样的老太婆,身上的肥肉还不算太多,这倒不容易,但你既有了老公,又有情人,为什么还要找上我呢?''白夫人咬牙道:``你既如此聪明,为何猜不出?''小鱼儿想也不想,立刻道:``因为你们叁个人中,必定有一个偷偷瞧见了苏樱为我着急的摸样,你们就想用我来要挟苏樱,叫她说出花无缺不肯说的事。''他话末说完,白夫人已怔位了,她虽然叫他猜,却再也未想到这该死的小鱼儿竟真的一猜就猜中,就好像在旁边瞧见了似的。白夫人满嘴都是苦水,却吐不出来。

小鱼儿道:``但你就算要让我上当,本来也不必自己脱光衣服,如此折磨自己的,这只怕是因为你本来就有这毛病,喜欢让别人瞧你脱得赤条条的模样──有些疯子喜欢对着女人小便,他们的毛病只怕就和你一样。''自夫人气得嘴唇发抖,忍不住破口大骂起来。

她简直已将世上所有悲毒的话都骂出了口,小鱼儿却像是连一句都没有听见,再也不瞧她一眼。

那边铁萍姑泡在水里,既不敢钻出来,也不勿该如何是好.溪水冷冽,她冻得嘴唇都发了白,心里又是悲哀,又是痛苦,又是羞惭,只觉活下去再也没什么意思,正想一头撞死算了。

小鱼儿忽然大声道:``你知道铁姑娘是我的救命恩人,也是我的好朋友,但她现在却在水里泡着,不敢出头,你说我心里难受不难受?''他忽又说了这种话来,铁萍姑也不知是惊是喜。

胡药师道:``阁下想必是。\ldots 是有些难受的。''小鱼儿怒道:``你既知我心里难受,为何还不脱下你的衣服为她送过去。''胡药师再也不敢多话,只好脱下外衣,远远抛绘铁萍姑,铁萍姑接在手里,也不知是穿上的好,还是不穿的好。

只听小鱼儿道:``铁萍姑在穿衣服时,你若敢做看一眼,我就挖出你的眼珠子来知道么?''胡药师又是好气,又是好笑,暗道:我方才难道还没有看够,现在你就算要我看。我又怎会有这么好的心情,这么好肠胃口。"铁萍姑终于还是将衣服穿了起来。

小鱼儿忍着笑喃喃道:不知她衣服穿好了没有?胡药师忍不住道:穿好了。``小鱼儿忽然又怒道:''想不到你还是偷看了!``胡药师道;没''。没有。"

小鱼儿哈哈一笑道:``其实你既早巳什么都瞧见了,现在就是又偷瞧了一眼,也没有什么关系,你用不着害怕的。''胡药师眼睁睁瞧着小鱼儿,也是满肚子苦水吐不出来。

他武功不弱,头脑也不坏,本来也很是自命不见,谁知此刻竟被个还未成年的半大孩子耍得团团乱转,他简直很不得不顾一切,先和这可恶的小鬼拼个死活再说。

小鱼儿目光闻动,忽然拍了拍肩头,笑道:``你用不着难受,只有呆子才会不爱惜自己性命的,你为了要我救你而委屈求全,正是你的聪明处。''胡药师叹了口气,渐渐又觉得自己伟大起来,``我能如此委屈求全,岂非正是人所难及之处,这又有什么丢人呢?''一念至此,方才那要和小鱼儿拼命的心,早已不知飞到哪里去了。

小鱼儿笑得更开心,道:``现在,你只要再为我做一件事.我就将解药给你。''胡药师叹道:``既是如此,愿闻所命。''

小鱼儿道:带我去找她的老公。"

胡药师想到花无缺还在白山君掌握之中,以花无缺相挟,也不怕小鱼儿不拿出解药来。

一念至此,他眼睛又亮了,立刻躬身道:``遵命!''胡药师瞧了白夫人一眼,忍不住又道:但她呢?"小鱼儿笑道:她既然喜欢脱光了洗澡,就索性让她在这里洗干净吧。

不到顿饭工夫,那石屋已然在望,风吹林木,沙沙作响,屋子里却是静悄悄的,听不到丝毫声音。

小鱼儿忽然出手,拧转了胡药师的手腕,沉声道:``他们就在那屋子里?''胡药师道:``不错。''

小鱼儿皱眉道:``叁个大活人在屋子里,怎地一点声音都没有?''铁萍姑忍不住道:``我\ldots\ldots 我先去瞧瞧。''

小鱼儿另一只手却飞快地拉往了她,沉着脸道:``既已到了这里,你还急什么!''铁萍姑嗫嚅道:``你苦念我也\ldots 也对你有些好处,只求你莫要杀了他。''小鱼儿瞪眼道:``不杀他!还留着他害人么?''铁萍姑头垂得更低,目中却流下泪来。

小鱼儿默然半晌,恨恨道:``看来这小畜牲将你骗得真不浅,但我早已跟你说过,我不是君子,你若指望我有恩必报,你就打错算盘了。''铁萍姑幽幽道:``你嘴里说得虽凶恶,但我却知道你的心并非如此,你\ldots\ldots 你\ldots\ldots 你不会杀他的,是么?''小鱼儿跺了跺脚,忽然重重一摔胡药师的手,厉声道;``叫他们出来,听见了么?''胡药师咳一声,高声唤道:``白大哥,出来吧,小弟回来了。''空山传声,回音不绝。但石屋里似是静悄悄的,没有回音。

小鱼儿皱眉道:``这姓白的难道是聋子。胡药师目光闪动,道:''不如让在下进去瞧瞧吧。``小鱼儿想了想,沉声道:''好,你先走,莫要走得太快,只要你稍有妄动,我就先扭断你的手!"胡药师叹了口气,一步步走过去,走到门口,就瞧见江玉郎一个人蜷曲在角落里,全身直发抖!

白山君和花无缺竟已不见了!

\hypertarget{ux7b2cux4e5dux5341ux4e8cux7ae0-ux5404ux901eux673aux950b}{%
\chapter{第九十二章
各逞机锋}\label{ux7b2cux4e5dux5341ux4e8cux7ae0-ux5404ux901eux673aux950b}}

胡药师和铁萍姑俱是又惊又奇,但小鱼儿见了江玉郎,却只觉气往上撞,别的什么都不再顾及。

江玉郎也瞧见了他们,干笑道``原来是鱼兄驾到,当真久违了''小鱼儿破口大骂道``谁跟你这小畜生称兄道第。只可惜那次大便没有淹死你,否则燕大侠又怎会死在你这小畜生手上。''他越说越怒,忽然扑过去,拳头雨点般落下。

江玉郎竟是全无还手之力,痛极大呼:``鱼兄千万手下留情,小弟已病入膏肓,禁不得打的。''小鱼儿怒喝道:``你若怕挨揍,为何不少做些伤天害理的事,''铁萍姑在一旁流着泪瞧着也不敢劝阻,他拳上虽末出真,但江玉郎已被打青眼肿,铁萍姑虽扭转头去,不忍再看,但也已知道小鱼儿并没有杀他之意了,否则用不着两拳就可将他活活打死,又何必多花这许多力气。

江玉郎大呼道``萍儿,你为什么不拉着他,你对他有救命之恩,他不会不听你话的,你难道真忍心瞧我活活被打死么?''铁萍姑暗叹道:``不是我不去救你,只望你经过这次教训后,能稍为过才好,只要你有稍为改过之心,就算要我为你而死,也是心甘情原的。''却听江玉郎忽然狂笑起来,大声道``好,你有种就打死我吧,这辈子就休想再见着花无缺了。''小鱼儿的拳头立刻在半空中顿住,他这才想起白山君和花无缺本该也在这屋子里的。小鱼儿一把将他从地上拎了起来,历声道``花无缺在那里?你说不说?''

江玉郎悠然道"你若想见他,就该敬敬,好生求教于我。

小鱼儿拳头又捣了出去,大喝道``小杂种,我求你个屁。''江玉郎冷笑道:``好,你打吧,但拳头却是问不出话来的,人若是我,难道挨了两拳就会说么?我说出后你难道不打得更凶。''``我打你?我几时打过你了?''他竟拍了拍江玉郎身上尘土,扶他坐起来笑道``江兄久违了,近来身子还好么?''江玉郎哈哈大笑道:"还好还好,只不过方才被条疯狗咬了几口。

小鱼儿大笑道"疯狗素来只咬疯狗的,江兄既没有疯,也末必是狗,怎会有疯狗咬你。

江玉郎也大笑道:``如此说来,倒是小弟看错了。''小鱼儿哈哈笑道:``江兄想必是思念小弟,连眼睛都哭红了,所以目力有些不清。''江玉郎道:``不错,小弟时时在想,鱼兄近来怎样了呀,会不会忽得了羊癫疯,坐板疮?一念至此,小弟真是忧心如焚、哈哈,忧心如焚。''小鱼儿笑道:``小弟本当江兄这样的人,必定无病无痛,谁知今日一见,江兄却好象得了羊癫疯了,否则为何在地上发抖。''两人针锋相对,一吹一唱,竟好象在唱起戏来。

胡药师在一旁瞧着,又是好笑,又不禁叹息``看来长江后浪推前浪,这句话倒当真一点也不错,昔日江湖中,虽也有几个随机善变,心计深沉的历害角色,但和这两少年一比,实在差得多了。''他更想不出白山君和花无缺会到那去?白山君若将花无缺带走为何又将江玉郎留在这里?只听小鱼儿又道:``荒山寂寂,江兄一个人坐在这里,难道不怕有什么不开眼的恶鬼找上门来向江兄索命么?''这倒不劳鱼兄费心,小弟近日是手头有些拮据,若有什么冤魂恶鬼真的敢来,小弟正好将他卖了,换几两银子打洒喝、何况,小弟方才本也不是一个人坐在这里的。"他这最后一句话,才总算转入正题。

小鱼儿却故作不解,道:``哦,却不知方才还有谁在这里?''江玉郎笑嘻嘻道:``其中有个姓花的,鱼兄好象认得?''小鱼儿道``是花无缺么?小弟正好想找他有些事,却不知他此刻到那去了?''江玉郎正色道:``小弟知道他和鱼兄有些事,生怕他再来找鱼兄你的麻烦,本想为鱼兄略效微劳,一刀将他宰了。''小鱼儿哈哈笑道``江兄若真的宰了他,小弟也省事多了、杀人总比问话容易得多的,是么?''江玉郎也笑道``小弟后来一想,鱼兄若要亲手杀他,小弟这马屁岂非就拍在马腿上了么?是以小弟只不过喂他吃了些迷药。''胡药师忍小住道``白\ldots\ldots 白山君也中了你的迷药么?''江玉郎笑嘻嘻道"中得也不太多,大约再过叁五天,就会醒来的。一个人若被迷倒叁五日之久,纵然醒来,只怕也变得成痴呆废人。

"

小鱼儿眼珠子一转,忽然大笑起来,江玉郎立刻也陪着他大笑,两个人笑得几乎连眼泪都流了出来。

铁萍姑和胡药师瞧得发呆,也不知他两人笑什么。``只见小鱼儿捧腹大笑道''有趣有趣,我简直要笑破肚子了。``江玉郎道''鱼兄笑的是什么?"

小鱼儿忽然不笑了,眼晴瞪着江玉郎,道``江兄看来纵非大病将死,也差不多了,却能将两个七八十斤的大男人背出去藏起来,这岂非是简直是最荒唐的笑话么。''江玉郎大笑起来,道``鱼兄的幻想力当真是丰富的得很,只可惜那位花公子\ldots\ldots{}''小鱼儿终于还是有点着了急,忍不住道``花公子怎样了?''胡药师叹了口气,道``花公子不但被点了穴道,而且还象是受了很大的刺激,神智已有些痴迷,只怕\ldots\ldots 只怕是无法走动的了。''小鱼儿歪着头,用手敲着自已的额角,一连敲了十七八下,嘴角又露出了一丝微笑喃喃道``他们倒下后,你就将他们背了出去?''江玉郎道``小弟这病,时发时愈,发作时固然痛苦不堪,莫说背人,简直连让人背都受不了,但没有发作时,背个人还是没有问题的。''小鱼儿眼睛向胡药师瞟了过去,胡药师点了点头。

江玉郎笑道:``小弟说的不假吧?''

小鱼儿笑嘻嘻道:``不假不假\ldots\ldots 但你将人背出去后,为什么又回来呢?难道你身上有些发痒,等着要在这里挨揍么?''江玉郎神色不动,也不生气,却笑道:``萍儿还在他们手里,小弟就算知道鱼兄要来。要将小弟碎尸万段,也还是要在这儿等着见萍儿一面。''小鱼儿撇了撇嘴。笑道:``江玉郎几时变成如此多情的人了。有趣有趣,实在有趣\ldots\ldots{}''铁萍姑已再也忍不住,扑倒在江玉郎脚下,放声痛哭起来。

小鱼儿叹了口气,喃喃道:"傻丫头,这小子若说他放的屁是香的,你难道也相信他么?

只听铁萍姑流着泪道:``你伤得重吗?痛不痛?''江玉郎轻轻摸着她的头发,柔声道:``我就算痛,只要瞧见你也就不觉得痛了。''小鱼儿忽然大叫起来,道:``好了好了,我全身的肉都麻了,你这大情人的戏还有没有演完么?''江玉郎道:``鱼兄有何吩咐?''

小鱼儿叹了口气,苦笑道:"现在货在你手里,你就是老板,要什么价钱,就开出来吧。

江玉郎慢吞吞笑道:``小弟这病,多蒙苏姑娘之赐\ldots\ldots 鱼兄和这位苏姑娘的交情却不错上么?''小鱼儿叹道:``我若不认得她,怎会有这许多麻烦。''江玉郎笑道:``这也算不了什么麻烦,只要鱼兄将苏姑娘接来,为小弟治好这病,小弟也立刻会将花公子请出来,治好他的病。''小鱼儿叹道:``好,走吧。''

江玉郎道:``小弟也要陪着去。''

小鱼儿嘻嘻一笑,道:"我也舍不得将你一个人孤令令抛在这里的。

"

胡药师忽然道:``这一趟不去也罢。只因那位苏姑娘马上就要到这里来了。''江玉郎怔了怔,皱眉道:``你怎知道她就会到这里来?''胡药师笑了笑,道:``正如这位铁姑娘跟阁下一样,苏姑娘对小鱼\ldots\ldots 公子也是一往情深小鱼公子一走,她也就跟着出来了。''江玉郎抚掌大笑道``担苏姑娘就算已出来寻找鱼兄,却也末必能找到这里。''胡药师微笑道``这倒不劳阁下担心,她一定能找得到的。''江玉郎想了想,笑道``不错,你们本要以鱼兄来要胁于她,自然已故意在一路上都留下线索,叫她找到这里。''小鱼儿叹了口气,道``既是如此,咱们就在这里等她吧。''白夫人在石关头上一分一寸地移动着,终于按准了地方,籍着飞泉的冲激之力,解开足底的道。

她勉强支起半个身子,正不知该如何是好,忽然发现岸上的杂草中,竟有双眼睛在瞬也瞬的瞪着她。

这人脸上满是泥垢,看来已不知有多久没洗过脸了,但一双眼睛却仍是又大又亮,像是正瞧得有趣得很。

白夫众眼波一转,反而将胸膛挺得更高了些,娇笑道``小子,你难道从末看过女人冼澡么?''那人象是已瞧得痴了,茫然摇了摇头。那人忽然一笑,道:``你用不着怕我,我\ldots\ldots 我也是女的。''她嘴里说着话,人已自草纵中站了起来,只见她衣服虽也又脏又破,但却更亲出了她身上曲线之诱人。

白夫人怔拄了,而且神情间似有些失望,这少女非但不丑,而且仿佛是人间绝色。

白夫人一直瞪着她,嫣然一笑,试探着问道:``瞧姑娘的模样,莫非赶了很远的路么?''少女垂首道:``嗯。''

白夫人道:``这里山既不青,水也不秀,姑娘巴巴的赶到这里来,是为了什么呢?''少女眉宇间忽然泛起一股幽之色,痴痴的呆了许久,黯然道:"我\ldots\ldots 我是来找人的?

白夫人心里一动,道:``你一定不会认得他,他也不一定在这里。''无论如何,一个孤伶伶的少女,竟敢深入荒山来找人,总是件不寻常的事,这其中虽难免有些蹊跷。那少女却似已要走了。``白夫人赶紧又笑道:''姑娘你叫什么名字?可不可以告诉我?``少女红着脸一笑,道:''我叫铁心兰。"

口口口铁心兰终于在溪水旁坐了下来。

她觉得这妇人竟敢在清溪中裸浴,虽然末免太大胆了些,但却是如此美丽,如此亲切。

这许多天来,她一直在伤心,矛盾,痛苦中,她到这里来,自然是为了找小鱼儿,找花无缺。

但真的找到了他们又怎样?她自已实在也不知道。

铁心兰第一次觉得心情轻忪了些。情不自禁脱她那双鞋底早已磨穿了的鞋子,将一双纤美的脚伸入溪水。

已走得发酸,发胀的脚,骤然泡入清凉的水里,那种美妙的滋味,使得她整个人都象是飘入云端。她忍不住轻轻呻吟一声,瞌起了眼。

白夫人一直在留意着她的神情,柔声笑道:``你为什么也学我一样来痛痛快快洗个澡。''铁心兰脸又红了,道:``在这里洗澡?''

白夫人道:``我每天都要在这里洗一次澡的,除了你之外,却从来没有碰见过什么人。''铁心兰咬着嘴唇,道:``这里真的\ldots\ldots 真的很少有人来?''她显然也有些心动。

白夫人笑道:``若常有人来,我怎么敢在这里洗澡?''铁心兰的心更动了,瞟了白夫人一眼,又红着脸垂下头道``我\ldots\ldots 我还冼洗脚算了。''铁心兰还在犹疑着。

白夫人已闭起眼睛,笑道:"快呀,还怕什么\ldots\ldots 她实已脏得全身发痒了,这实在是任何人都抵抗不了的诱惑。

她躲在草纵中,飞快的脱下衣服,虽然没有人偷看,但阳光却已偷偷爬上了她丰满的胸膛。

她全身都羞红了,一颗心也几乎跳了出来,飞快地跃下小溪,钻入水里,那清凉,而又微带温暖和水,立刻将全身都包围了起来。

她这才松了口气,笑道:``好了。''

白夫人张开眼睛瞧着她,笑道:``舒服么?''

铁心兰点着头道嗯。"

白夫人道:``好,现在我要下来了,你扶着我。''她也直到此刻才真的松了口气,轻轻滑入了水中。

水势果然很急,她双腿发软若没有人扶着她,她实在无力游上岸,纵然不被淹死,也难免要被水冲走。

铁心兰赶紧扶着她,着急道:``你\ldots\ldots 你难道要走了?''白夫人笑道:``我只是到岸上去替你望风,你放心地洗吧。''铁心兰这才放了心,笑道:``可是你千万不能走远呀。''白夫人吃吃笑道"有你这样和小美人儿在洗澡,我舍得走远么?

铁心兰连耳根子都红了,简直连手都不敢伸出水来,她发现女人的眼晴,有时竟也和男人差不多可怕。

白夫人却已藉着她的扶携之力,终于上了岸,笑道:``好。我要穿衣服了你也不准偷看。''其实铁心兰早已闭起了眼睛,根本就不敢看,一看到她那白得诱人的胴体,铁心兰的心就好象跳得再也无法停止\ldots\ldots 她又发现女人的裸体不但对男人是种诱惑,有时对女人也一样。

衣服虽然又脏又破,也总比不穿的好,白夫人的脸皮就算比城墙还厚,也不敢光着身子到处乱跑的。

铁心兰闭着眼睛等了半晌,只听白夫人道:"这件衣服料子倒不错,只可惜实在太脏了些。

铁心兰忍不住张开眼一瞧,哧得脸都白了,失声惊呼道:``你怎么能穿我的衣服?''白夫人笑嘻嘻道:``我不穿你的衣服,穿谁的衣服?''铁心兰颤声道:``你穿走了我的衣服,我怎么办呢?''白夫人笑道:``你就在这里多洗一会吧,这里来来往往的人,反正不少,虽然都是男人,但男人也不见得全是色鬼,说不定也会有个把个好心的,会将裤子脱下来借给你穿\ldots\ldots{}''她不说还好,这么样一说,铁心兰简直急得要哭了出来。白夫人却笑得弯下了腰,娇笑着又道:``你穿过男人的裤子么?虽然大些,却又宽敞,又通风,比你小时候穿的开裆裤还要舒服得多。''铁心兰飞红了脸,嘶声喝道:``你这女疯子,恶婆娘,把衣服还给我。''她象是忍不住要从水里冲出来。白夫人却已再也不理她,笑嘻嘻扬长去了。

铁心兰怒极大骂道:``你简直不是人,是畜生,是母狗\ldots\ldots{}''白夫人头也不回,笑嘻嘻道:``你骂吧,用不着再骂几声,附近的男人就会被你引来了。''铁心兰果然哧得连一个字都不敢骂出口。

她身子蜷曲在水里,眼泪已流了下来,她本不想信一个大人也会象孩子似的被急哭,现在才知道这世不原是什么事都可能发生的,想到这里,她简直恨不得立刻死了算了。

\hypertarget{ux7b2cux4e5dux5341ux4e09ux7ae0-ux5978ux72e1ux65e0ux5339}{%
\chapter{第九十三章
奸狡无匹}\label{ux7b2cux4e5dux5341ux4e09ux7ae0-ux5978ux72e1ux65e0ux5339}}

溪水左边,有片树林,白夫人穿过树林,匆匆而行。

忽然间,她发现竟有件衣服,在前面树枝上飘荡,水红色的底,绣着经霜愈艳的秋海棠,在阳光下看来就像是真的。

一整套漂亮的,考究的女人衣服,这诱惑对白夫人未免太大了,她实在不愿穿着身上这套破衣服,去见她的丈夫。白夫人的心动了。

她眼睛盯着那衣服,脚步已渐渐慢了下来,只不过心里还是有些犹疑,不敢伸手去拿衣服。

白夫人告诉自己:「这其中说不定有诈,我麻烦已够多了,何必再惹这些麻烦。」一念至此就简直看都不愿再看一眼。

但那海棠绣得实在太好,衣服的缝工又是那麽精致,那料子,那水色,更是说不出的令人中意。

白夫人终於还是下了决心,暗道:「这大不了也只是件衣服而已,难道还会长出牙齿来,咬我一口不成。」

这果然只不过是件衣服,既没有毛病,也没有古怪,任何人将它从树上拿下来,都不会有麻烦。

白夫人再也不客气了,立刻脱下破衣服,穿上新的,柔软的绸缎,摩擦着刚洗乾净的身子,就好像情人的手一样。

但这双手却太不老实了,白夫人忽然觉得身上发起来,开始时,就好像有只小从领子里爬进来,沿着她背脊往下爬。

到後来,这小就像是变成了十只,百只,千只\ldots\ldots 在她身上每一个角落爬来爬去。

得要发疯,连路都走不动了,两只手拚命的去抓,但越抓越,不但身上,连心里也了起来。

她又像舒服,又像难受,又想哭,又想笑\ldots\ldots 到後来竟真的整个人都倒在地上,吃吃地笑了起来。

突听一人银铃般笑道:「这件衣服,你穿着还舒服麽?」原来毛病还是在这件衣服上。

只见一个人从远处盈盈走过来,身上只穿着件月白中衣,在淡淡的阳光下看来,无论谁的魂魄都要被勾去。她竟是苏樱。

白夫人眼珠子都快掉了出来,失声道;「是你?这衣服是你的?」

苏樱微笑道:「我做好了刚预备第一次穿,你说好看麽?」

白夫人却已得说不出话来,只是拚命靠着树干摩擦着身子,颤声道:「衣服上有什麽?」

苏樱悠悠笑道:「也没有什麽,只不过是一点儿药而已,过几天就会慢慢褪了的。」

白夫人就好像被人踩着脖子,嘶声惨呼起来。

现在她已得发狂,直恨不得找人用鞭子狠狠的抽她一顿,连一时半刻都等不了,若是再过几天,她真情愿一头撞死算了。

白夫人疯狂般把衣服都扯了下来,嘶声道:「我和你无冤无仇,你为什麽要如此害我?」

苏樱冷冷道;「你再仔细想想,有没有得罪过我?」

白夫人虽然已又脱光了衣服,但还是得要命,爬在地上,扭动着身子,流着泪哀求道:「好姑娘,好妹子,我知道错了,求求你饶饶我吧?」

苏樱笑;「那麽我问你,花无缺是不是被你偷去了?」

此时此刻,白夫人那里还敢不承认,立刻点头道:「是我,我该死。」

苏樱沉下了脸,道:「你将他藏到什麽地方去了?」

白夫人道:「就在後山,那小山谷里,有间小屋子\ldots\ldots」

苏樱默然半晌,一字字问道:「你可是真的将他藏在那地方了?」

白夫人苦笑道;「在姑娘你的面前,我几时敢说过假话?」

苏樱面色竟彷佛微微变了变,摇头叹道;「荒山之中,竟会有间盖得那般坚固的石屋,你们难道不觉得奇怪麽?」

白夫人也没有心情再追究这件事情,只是苦苦哀求道:「我现在什麽都说了,你总该饶了我吧!」

苏樱淡淡一笑,道:「你方才是从那里来的」

白夫人怔了怔,道:「那边的小溪。」

苏樱道;「那麽你就再回去吧」

铁心兰手脚都快冻僵了,一双眼睛却不停的四下乱转,只怕有什麽野男人忽然间闯了过来。

幸好四下静悄悄的,瞧不见人影。

铁心兰也想偷偷爬起来溜走,但一个赤条条的大姑娘,又能到那里去呢?万一迎面来了个男人\ldots\ldots 她简直想也不敢再想下去。

忽然间,前面竟又有一个赤条条的女人,狂奔过来,「噗通」一声,跳入溪水里不住喘息。

铁心兰又鹫又喜,本还不好意思去瞧,但眼角瞟去,却发现这女本苋然就是方才将自己衣服骗走的那个。铁心兰吃鹫得瞪大眼睛,说不出话。

铁心兰忽然扑过去抓住她的头发,大喝道:「我的衣服呢?还给我。」

只听一人微笑道:「这就是你的衣服麽?」铁心兰扭转头瞧见了苏樱。

苏樱站在溪水旁,就像是一朵初开放的莲花似的。

铁心兰只觉得自己这一生中,从来没有见过如此美丽的女人,她虽也是女人,竟也瞧痴了。

苏樱笑道:「你若不想再洗了,就起来穿上它吧!」

铁心兰虽然还是害羞,但也不能不起来了,飞快的接过衣服,一溜烟似的躲入杂草丛去。

白夫人陪着笑道;「我也想起来了。」

苏樱淡淡道;「你想起来就起来吧!也没有人拦着你。」

白夫人爬到石头上,谁知她的上半身刚一离开水被风一吹,就又了起来,得简直要她的命。

苏樱笑道「只要你觉得不的时候,随时都可以起来的。」

白夫人道:「那\ldots\ldots 那要等到什麽时候?」

苏樱微笑道:「也许一半天,也许叁两天\ldots\ldots 反正你喜欢洗澡,就索性洗个痛快些吧」

白夫人怔在水里,几乎晕了过去。

这时铁心兰已穿好衣服走出来,盈盈一礼,道:「多谢姑娘。」

她身上穿的衣服虽然又破又烂,佳人出浴,白足如霜,皓腕胜雪,嫣红的面靥,可爱得如同苹果。

苏樱情不自禁拉起了她的手,娇笑道:「这样美的女孩子,真是我见犹怜,男人本该一排排跪在你面前求你才是,你何苦反而来找他们。」

铁心兰脸又红了,嗫嚅着道:「我\ldots\ldots 我\ldots\ldots」

苏樱笑道:「是什麽人有如此好的福气」

铁心兰道;「他\ldots\ldots 他\ldots\ldots」

苏樱笑道:「你用不着对我说出来,反正我也不会认得他的。」

铁心兰随着她走了半晌,轻轻叹息道:「你也最好还是莫要认得他的好。」

苏樱失笑道;「为什麽难道认得他的人,都要倒楣麽?」

铁心兰竟点了点头,道:「嗯!」

苏樱骤然回过头,张大了眼睛看她道:「他叫什麽名字?」

铁心兰也没有留意她神情的变化,轻叹道「他姓江,别人都叫他小鱼儿。」

小鱼儿叁个字,使得苏樱的心立刻像打鼓般跳了起来,她发现走在她旁边这少女,竟然就是她的情敌。

望着铁心兰花一般的面靥,她心里只觉酸酸的:「小鱼儿呀,小鱼儿,你的眼光倒真不错。」

只见铁心兰忽然笑了笑,道;「他这人有时可以把你气死。」

苏樱眨了眨眼睛,笑:「你很恨他?」

铁心兰垂首道:「我有时的确很恨他,但有时\ldots\ldots」

苏樱一笑,接着道:「但有时却又喜欢他,喜欢得要命是麽?」

铁心兰咬着嘴唇,只是吃吃的笑。

苏樱瞪着眼出了一会儿神,忽然大声道:「但他却未必喜欢你,是麽?」

铁心兰呆呆的出了会儿神,眼波渐渐变得更温柔了,嘴角也露出一丝甜蜜的微笑,垂下头轻轻道:「他有时对我虽然不好,但有时\ldots\ldots 有时对我也不错的。」

苏樱的心就像是被针在刺着,恨不得把铁心兰的心挖出来,在上面也刺十七、八个洞,叫她以後永远再也不敢想小鱼儿。

铁心兰全末瞧见她的表情,目光痴痴的瞧着天边的一朵云,这朵云像是已变成了小鱼儿笑嘻嘻的脸。

苏樱扭转头不去看她,故意大声道:「他就算有时对你很好,但也并不一定就能证明他喜欢你,也许,他对每个女孩子都一样,也许,他对别人比对你更好。」

铁心兰轻轻道:「只要他对我好,他对别人怎样,我都不会在意。」

苏樱道:「你不吃醋麽?」

铁心兰笑了笑,道;「有许多男人,天生就不是一个女人所能独占的,小鱼儿就是这样的人,我既然很了解他,就不该吃醋。」

苏樱一心想刺伤铁心兰,谁知铁心兰竟一点儿也不生气,她自己倒反而快被气死了,过了半晌,忍不住又道;「这也许是因为你认得的男人只有他一个,所以才会对他如此死心塌地,你若多认识几个男人,就会发现比他更好的,还多的是。」

铁心兰神色忽然变了,头垂得更低。

苏樱这才发现她神情的变化,眼睛一亮,又道:「除他之外,你心里难道还有一个人麽?」

铁心兰红着脸不说话。

苏樱笑了,道;「我猜的一定不错,这就怪不得你不吃他的醋。」铁心兰的脸更红了。

苏樱银铃般笑着,却道:「一个女人,心上若有了两个男人,虽然很伤脑筋,倒也有趣得很\ldots\ldots」

铁心兰垂首弄着衣袂,过了半晌,忽然道;「我这一生,本来已决定交给小鱼儿了,无论他对我是好是坏,我都绝不会有所改变,谁知道\ldots\ldots」

苏樱眼珠子一转,笑道:「另外一个男人却实在对你太好,让你没法子抗拒是麽?」

铁心兰目中流下泪来,颤声道;「但他对我好,并不是为了占有\ldots\ldots」

苏樱道:「他越是这样做,你反而越是觉得对他歉疚,是麽?」

铁心兰道:「嗯!」

苏樱道:「我知道,他也一定和小鱼儿一样,又聪明,又风趣,又可爱,有时却又有点儿讨厌\ldots\ldots 只有一点点讨厌。」

铁心兰道;「你错了。」

苏樱道:「哦?」

铁心兰道:「他和小鱼儿是极端相反的男人,简直连一点相同的地方都没有,他对女孩子,永远都是彬彬有礼,连一句玩笑都不会开。」

苏樱道:「这种看家狗似的男人,我就一点儿也不喜欢。」

铁心兰道:「但\ldots\ldots 但\ldots\ldots」苏樱笑道:「但有人却很喜欢的,是麽?」

铁心兰的脸又红了,道:「我\ldots\ldots 我并不是喜\ldots\ldots 喜欢他,只不过他非但救过我的命,而且对我更是\ldots\ldots 更是\ldots\ldots」

她说话的声音简直比蚊子叫还轻,而且吞吞吐吐,断断绩绩,就像是嘴里含着个鸡蛋似的。

苏樱娇笑着替她接了下去,道:「她不但救了你的命,而且对你更是照顾得无微不至,你就算不喜欢他,也不能不感激他,是麽?」

铁心兰咬着嘴唇,呆了半晌,忽然道:「就算我喜欢他,他也不会喜欢我。」

苏樱笑道:「他若不喜欢你,为什麽要对你这麽好难道他脑袋有毛病麽」

铁心兰垂头道;「他照顾我,也许只是为了小鱼儿。」

苏樱这次才真的像是吃了一鹫,失声道:「他为了小鱼儿才对你好,这我倒不懂了。」

铁心厕幽幽道;「他说希望我和小鱼儿能\ldots\ldots 能在一起。」

苏樱道:「他难道是小鱼儿的朋友。」

铁心兰想了想,道;「有时,他们的确可以算是很好的朋友,若知道对力有了危险,会连自己性命也不要,赶去相救,但有时他们却又要拚得你死我活。」

苏樱忽然明白她说的这人是谁了,怔了半晌,喃喃道:「这件事的确妙得很,简直妙极了。」

苏樱眼波流动,忽又拉起她的手,柔声道:「我一瞧见你,就觉得很投缘,你若也不讨厌我不知你肯收我这个妹妹麽?」

如此温柔的请求,自如此美丽的女孩子嘴里说出来,又有谁能拒绝。

铁心兰就这样做了苏樱的姊姊。

阳光娇艳,山林碧荫浓得化不开,啁啾的鸟语伴着流水,微风中隐约有醉人的花香菸。

铁心兰从来也想不到自己也会这麽开心的,这些日子来,她几乎已认为自己再也不会有开心的时候。

苏樱拉着她的手,笑道:「现在你既然是我的姊姊,就再也不能让你这样去找小鱼儿了。」

铁心兰道;「为什麽?」苏樱道:「男人都是贱骨头,你越是急着去找他,他就越得意,你若不睬他,他反而也许会爬着来找你。」

铁心兰嫣然一笑,道:「那麽\ldots\ldots 你想要我怎样做呢?」

苏樱道:「你什麽都不必做,只要静静的等着就好,我自然有法子让他来找你。」

铁心兰垂首道:「但你连认识都不认得他\ldots\ldots」

苏樱道:「现在被你一说,我已经想起来了,他是不是一个眼睛很大的小伙子,脸上虽然有很多疤,但看起来却不讨厌,整天嘻皮笑脸的,走起路来,扬扬得意,好像总觉得自己很神气,很了不起。」

铁心兰嫣然道:「你那里知道,他还说自己是天下第一聪明人哩。」

想起小鱼儿,苏樱的心里也觉得甜甜的,娇笑道:「他若说自己是天下第一厚脸皮,那倒是一点也不假。」

铁心兰道:「你什麽时候看到他的」

苏樱道:「没多久,才不过一两天。」

铁心兰叹了口气,道:「但这人连一时半刻也静不下来,你一两天以前看见他,现在他早已不知到那里去了?」

苏樱笑道:「你放心,只要他在这山里,我就有法子找得到他。」

她不等铁心兰说话,又接着道:「为了安全起见,我现在就要带你去个地方。那里的主人可算是我的义父,他的人长得虽然凶恶,但心却是很好的,尤其是对我,更好得不得了。」

铁心兰笑道:「连我这做乾姊姊的,都恨不得把心掏出来给你才好,何况他做乾爹的呢。」

苏樱撇了撇嘴,道;「你要把心给我,你的心不是给了小鱼儿麽?」

她看见铁心兰红了脸,就又笑了,道:「我那乾爹姓魏,他若知道你是我的姊姊,一定会好好照顾你,只不过你莫忘记,他模样看来是很怕人的。」

铁心兰道:「我若觉得他可怕,少看他两眼也就是了。」

苏樱拍手笑道:「不错,这法子的确再好也没有了。」

她拉着铁心兰走出树林,空山寂寂,天地间彷佛充满了一种安宁祥和之意,令人觉得只要能活着,就是件幸福的事。

走了半晌,苏樱忽然停下脚,道;「嗳呀!我差点儿忘了,我还有个约会哩。」

苏樱眼珠子一转,又道:「从这里一直往山上走,用不了多久,你就会瞧见,一片槐树林,那里面就是我乾爹住的地方了。」

铁心兰道:「你\ldots\ldots 你难道叫我一个人去麽?」

苏樱道:「一个人去也没关系,你只要走进愧树林,自然就有人出来接待你。」

铁心兰道;「但他们又不认识我。」

苏樱想了想,自头上拔下了恨珠钗,道;「你只要将这珠钗给他们看,说是我叫你去的,他们就一定会对你恭恭敬敬,为你安排好一切。」

铁心兰虽然不愿意,但还是去了。

她现在就像是一片没有恨的浮萍,瓢到那里算那里,她自己也不知道自己该怎麽做,自己也拿不定主意。

苏樱瞧着她走远了,刚轻轻吐出气,突听一人叹道:「可怜的傻丫头,自己被人责了郡不知道。」

另一人道:「哈哈,这位苏姑娘没有将她卖给你,所以你就来假慈悲了麽?」

第叁人咯咯笑道:「我本来还觉得那姓铁的丫头满不错的,但和这位苏姑娘一比,那简直就好像变成个大笨瓜了。」

第四人大笑道:「咱们的小鱼儿可不能娶个大笨瓜做老婆。」

笑语声中,山石後木叶间,忽然钻出四个人来,这四人模样,一个比一个奇怪,也不知怎麽会凑到一齐的。

只见第一人蓬头垢面,穿着身又油又腻,破破烂烂的衣服,就像是个穷要饭的,但手里却偏偏拿个价值不菲的翡翠鼻烟壶。

第二儿圆圆的脸,圆圆的肚子,年纪虽然不小,看来却还像个孩子,一直不停的在哈哈大笑像是个弥勒佛。

第叁人满头珠翠,脸上的粉足有半寸厚,像是带着个假面具似的,叫人恨本瞧不出她本来长的是美是丑,是老是少。她打扮得明明是个女的,但身上却穿着件男人的衣服,脚下面偏又套着双红缎珠花的绣花鞋。

第四人却是个身材魁伟的伟丈夫,目光闪动,顾盼自雄,只不过一张嘴大得可怕,看来像是可以塞得进他自己的拳头。

\hypertarget{ux7b2cux4e5dux5341ux56dbux7ae0-ux673aux667aux7eddux4f26}{%
\chapter{第九十四章
机智绝伦}\label{ux7b2cux4e5dux5341ux56dbux7ae0-ux673aux667aux7eddux4f26}}

苏樱虽然不知道这四人就是顶顶大名的白开心哈哈儿屠娇娇和李大嘴,但却是见过这四人的。

她也曾亲眼瞧见,这四人如何对付魏麻衣,现在这四人忽然一齐出现,将她围住,她就算一向喜怒不形於色,脸色也不禁有些变了。

李大嘴大笑道:``苏姑娘,你用不着害怕,这两天我的胃口都不太好,要吃你,至少也得再等几天。''

屠娇娇咯咯笑道:``像这样聪明致的女孩儿,就算你舍得吃,我也不答应的。''

白开心道:``以我看来,还是吃了算了。''

哈哈儿道::处。''

白开心道:``我至少可以放心些,不至於被她卖了。''

苏樱眼波流动,忽然笑道:``四位难道是来为铁心兰打抱不平的麽?''

屠娇娇叹了气,道:``说起来,那傻丫头倒的确满可怜的。''

苏樱笑道;``四位若是觉得我让她去上当,方才为何不拦住她。''

白开心板着脸道:``她既不是我女儿,也不是我老婆,她上不上当,与我又有何关?我为何要来多事。''

哈哈儿道:``何况,让她到魏无牙那里去也不错,哈哈,魏无牙要是看中了她,那就简直更妙不可言了。''

苏樱嫣然道:``既是如此,四位是为了什麽来的呢?''

李大嘴道:``我们来找你,只不过是为了谈一项交易。''

苏樱道:``交易?什麽交易?''

哈哈儿道:``哈哈,自然是彼此有利的交易,却不知你肯不肯答应''

苏樱笑道:``若是彼此有利的交易,我怎麽会不答应呢''

屠娇娇道;``好,我问你,你想嫁给小鱼儿,是不是''

``哈哈,你这人真是名副其实的人不利己,李大嘴将她吃了,与你又有什麽好苏樱笑了笑,道:``我并不是想想就算了,我是非嫁他不可。''

屠娇娇道:但你有把握让他娶你麽?''

苏樱笑道:``越没有把握的事,做起来就越有趣,是麽卜''

屠娇娇道:``好,现在我们可以帮你的忙,叫小鱼儿娶你,但你却也要答应我们一件事。''

苏樱眼珠一转,笑道:``你们真有把握让他娶我。''

屠娇娇道;``当然有把握,你莫忘了,小鱼儿是我们养大的,我们怎会不知道他的脾气。''

苏樱道:``那麽,你们又要我做什麽事呢?''

屠娇娇道:``将他活着带入魏无牙的洞去,再活着带出来。''

苏樱道:``你们为什麽要这样做呢?''

屠娇娇道:``只因我们要叫他去拿件东西。''

苏樱想了想,道;``他若不肯去?''

屠娇娇笑道:``他本来就算不一定会去,但现在却是非去不可的,只因为你帮了我们的忙,你将铁心兰送到那里去。''

苏樱悠悠道:``若是我不答应呢?''

李大嘴咯咯笑道:``你若不答应,我的胃口立刻就会变好的。''

苏樱嫣然一笑道:``我相信我身上的肉,无论怎麽样做,都很好吃的,只不过我要劝你,切切不要红烧,这麽嫩的肉,红烧实在太可惜了,最好是用来涮锅子,肉才能保持鲜嫩。''

李大嘴等人,听得面面相觑,反倒不禁呆住了。

李大嘴乾笑两声,道:``你倒提醒了我,涮人肉的滋味,的确可算是天下第一,我倒买的已有许久未曾过。''

苏樱道;``你最好在我还活着的时候,就将我身上的肉片切下来,而且作料中,切切不可放醋,因为人肉本来就有些酸的。''

李大嘴乾笑道:``多承指教,我吃人吃了无数,想不到竟还没有你内行。''

他走了两步,只见苏樱悠然坐在那里,怎麽看也不像要被人吃下肚子里的,倒像是等着别人送上门给她吃。

屠娇娇忽然道:``李大嘴,你先过来一下,我有话跟你说。''

她将李大嘴拉向一边,悄悄道:``你吃过这样的人麽?''

李大嘴笑嘻嘻瞧了坐在那边的苏樱一眼,忍不住低声骂道:``这丫头看起来,就像是喜欢被老子吃下去似的,真不知她肚子里在打什麽鬼主意''

屠娇娇道:``你想,她若非胸有成竹,怎会如此笃定,而且还像是生怕死得太舒服了,竟劝你活着将她凌迟,你想,世上有这样的人麽''

李大嘴默然半晌,道:``你的意思是\ldots\ldots{}''

屠娇娇道:``依我之见,还是算了吧``咱们能活到现在,并不是件容易的事,莫要阴沟里翻船,栽在这小丫头手里,那才冤哩。''

李大嘴沉吟着道;``这话倒也不错,:''

只听苏樱娇笑道;``你还不过来,再等下去,我的肉都要变老了。''

李大嘴大笑道:``你的肉太酸,我懒得吃了。''``想不到我的肉竟是酸的,莫非是平时吃醋吃得太多了。''她盈盈站了起来,俭衽道;``你先生既然不肯赏脸,我只有告辞了。''

突听白开心喝道:``我和他不一样,他好吃,我好色,好吃的人,胆子总比较小些,但好色的人就不同了\ldots{}''

他一步步向苏樱走过去,大笑道;``常言道,色胆包天,这句话你总该听过的吧?''

苏樱情不自禁,向後退了半步,但面上还是带着微笑,道:``阁下若觉得光棍做得无趣了,我倒可替你做个媒。那边小溪里,有位美人在出浴,她不但长得千娇百媚,比我好看多了,而且风情万种,知情识趣。''

白开心吃吃笑道:``我就看上了你,别的人我都不要。''

他嘴里说着话,一双大手已向苏樱抓了过去。

苏樱肚子里就算有一千条绝顶妙计,此刻却也连一条都便不出来了,女人若碰见急色鬼,那真是什麽法子也没有。

只听``哧''的一声,苏樱的衣服已被白开心撕了一块下来。

就在这时,突又听得一人缓缓道:``男子汉,大丈夫,怎麽能欺负女人,''

这语声平和而缓慢,但他的人却来得快如风,疾如电。

白开心只见一条人影自天而降,他大鹫之下,还掌击出。

李大嘴等人,但见人影一花,但闻一声清脆的掌声,白开心的身子,已像是一个球似的挂在树枝上。

再看苏樱身旁,已多了个手采翩翩的美少年,衣衫虽然有些狼狈,但却仍掩不住有一种清贵高华之气流露出来。

这人虽然救了苏樱,但见苏樱瞧见他,脸色反而变了,失声道:``花无缺!''

花无缺淡淡一笑,目光向李大嘴等人扫了过去,缓缓道:``还有那一位想动手的麽?''

李大嘴等人也骇呆了。花无缺虽不认得他们,但他们却是认得花无缺的。

他们曾经眼看着花无缺,以一身超凡绝俗的武功,将慕容姊妹吓走,又在一招间将白开心抛在树上。

李大嘴大笑道:``咱们也早就看这色鬼不顺眼,公子此刻教训了他,这是再好也没有。''

屠娇娇也笑道;``只可惜公子出手还太轻了些\ldots{}''

哈哈儿道:``哈哈,公子若将他抛得更远些,让咱们再也瞧不见才好。''

白开心挣扎着想从树上跳下来,嘴里大叫道:``我只不过想摸一摸她而已,但那大嘴巴却要吃她的肉哩。''

他们不去对付外人,反倒先窝里翻起来,花无缺倒买还没有见过像这样的人,忍不住叹了口气,道;``各位倒买是够义气得很\ldots\ldots{}''

一句话末说完,李大嘴已怒吼着向白开心扑了过去,白开心似是闪避不及,竟被他一拳打出叁丈外,怪叫道;``大嘴狠,你敢打人?''

李大嘴吼道:``二十年前,我就想打死你这王八蛋了!''

他一面骂,一面追过去,谁知白开心的脚忽然一勾,他也倒了下去,两个人竟都猿在地上,扭成一团。

只听``砰砰蓬蓬''的拳头声,``混帐王八''的怒骂声,骂的话固然不堪入耳,打架的姿态更是不堪入。

花无缺本还以为他们是什麽武林高手,此刻看来,却简直连可以为了叁文钱而打破头的泼皮无赖还不如。

哈哈儿却在一旁拍掌大笑道:``好,打得好,哈哈,快抓他的头发,对了,抓紧些。''

屠娇娇道:``但也不能让他们这样打下去,若是打死了一个,咱们岂非还得花钱为他收,还是过去拉开他们吧。''

一这时李大嘴和白开心已猿到那边的树後面去了,两个人都已打得像狗一般在喘息,但还是不肯住手。

屠娇娇和哈哈儿也赶了过去,一面呼道:``莫要打了\ldots\ldots 再打就要打出人命来了呀!''

於是这两个人也到了树後,似乎在拉架。

花无缺瞧着他们,只有摇头苦笑他遇见这样的泼皮无赖,除了摇头之外,还能干什麽?

苏樱忽然微微一笑,道:``花公子,你上了他们的当了。''

花无缺道:``上什麽当?''

苏樱微笑道;``你以为他们这真是在打架麽?''

花无缺怔了怔,道:``难道这是\ldots{}''

苏樱抿嘴笑道:``这不过是他们在想法子逃走而已,那两人的武功虽然不怎麽样,但若真的要拚命,叁百招内,谁也休想碰着对力一根手指。''

花无缺纵身掠了过去,树後果然连人影都瞧不见了。

树皮上,却留下了四行字;``手下留情,多谢多谢,不辞而别,惶恐惶恐,不够胆量,也许也许,不够义气,未必未必。''

花无缺呆了半晌,忍不住苦笑道:``果然上当,惭愧惭愧。''

苏樱笑道;``这四人的诡计多端,实在少见得很,像花公子这样的忠厚君子,若不上他们的当,那才是怪事。''

花无缺忽也一笑,道;``忠厚君子,倒也未必未必,,,\ldots\ldots 方才也有几个人就上了我的当。''

苏樱道:``哦?谁?''

她话问出来後,自己也明白了,笑道;``不错,上当的必定就是白山君夫妇,是麽?''

花无缺微笑点头,道:``正是他们。''

苏樱眼珠一转,道:``我虽然以药力将你困住,但那药对人却没有什麽害处的,只要一吹风药力就解了,只不过那时他们必已点了你的穴道,你还是不能逃走。''

她微微一笑,接着道:``你是不是故意装成中毒很深的模样,让他们对你不如提防,你却在暗中以``移花接玉的内力,打开了穴道,扬长而去。''

花无缺笑道:``姑娘的聪明智慧,实在也少见得很。''

花无缺面上的笑容忽然不见了,叹了气道:``姑娘你虽然是智计无双,但在下却知道还有一个人,,,;就算姑娘你遇见他,只怕也要吃亏的。''

苏樱垂下了头,也叹了口气,幽幽道:``你说的不错,我非但知道你说的这人是谁,而且也吃过他的亏了。''

花无缺面上不禁露出鹫异之色,刚想问个清楚,苏樱忽又笑道;``温良如玉的花公子,如今也会以诡计骗人,只怕也就是跟这个人学的\ldots\ldots 我说的是麽?''

花无缺忍不住笑道:``这就叫做:近朱者赤,近墨者黑。''

苏樱道:``但君子毕竟总是君子,所以我虽然那麽样对待你,你非但没有向我报复,反而救了我。''

花无缺脸色忽然沈了下来,道;``你可知道我为什麽要救你''

苏樱望着他忽然改变的脸色,也像是有些吃鹫,但还是笑着道:``我已说过,这就因为你是君子。''

花无缺沉着脸说道:``我必需告诉你叁件事,第一,移花接玉的秘密,绝不容许外人知道,谁知道了,只有死!圭。是移花宫的禁例,谁也不能例外。''

苏樱虽然还在笑着,笑声听来却没有那麽悦耳了。

花无缺道:``第叁,移花宫的门下无论要做什麽事,都必需自己动手,绝不容别人干涉,也绝不能假手於外人。''

苏樱道:``第\ldots\ldots 第叁呢?''

花无缺道:``第叁,我也是移花宫的门下,无论如何,我也不能破坏移花宫的规矩。''

苏樱叹了口气,道:``如此说来,你救了我,只不过是为了要亲手杀我而已,是麽?''

花无缺扭过头不看她,一字字道;``纵然情非得已,却也势在必行。''

苏樱道:``那麽\ldots\ldots 那麽我也要告诉你叁件事。''

她不等花无缺问她,就接着道;``第一,你莫要忘记,我本来有许多机会可以杀你的,但我却没有动手,你现在若杀了我岂非不义?''

花无缺虽然没说什麽,却忍不住叹了口气。

苏樱道:``我虽然知道了移花接玉的秘密,但我绝不会练这种功夫,也绝没有告诉过别人,你若杀了我,岂非不仁。''

花无缺已微微动容。

苏樱道:``第叁,你莫忘了,我是个女人,而且手无缚鸡之力,一个大男人以强欺弱,来欺负一个弱女子,这非但无礼,简直是无耻了。''

花无缺已不觉垂下了头。

苏樱见他神情的变化,眼睛已发了光,嘴里却冷冷道:``你若一定要做这种不仁不义无礼无耻的事,我自然也没法子,但铁心兰若是知道了,她一定会对你失望得很。''

花无缺霍然抬起头。

苏樱悠悠道:``不错,铁心兰\ldots\ldots 她总是对我说,你是最温柔、最有礼的男人,我本来也很相信的,但现在\ldots\ldots{}''

她故意叹了气,住口不语。

花无缺指尖已有些发抖,道:``你\ldots\ldots 你认识铁心兰?''

苏樱抬起头,淡淡道:``我和她也不算太亲密,只不过刚刚结拜为姊妹而已。''

花无缺像是忽然挨了一鞭子,呆了半晌,摇头道:``不可能\ldots\ldots 这绝不可能她在那?''

苏樱道:``我就算告诉你她此刻在那里,你也不敢去找她的。''

花无缺目光一闪,变色道:``魏无牙,你将她送到魏无牙那里去了午''

苏樱笑道:``魏无牙对别人虽凶恶,但对我们姊却很好的。''

花无缺跺了跺脚,霍然扭转身,嗄声道:``移花宫的秘密,你绝不告诉别人?''

苏樱道:``若有第二个人知道,那时你再杀我也不迟。''

花无缺长叹道:``那时虽已迟了,但\ldots\ldots 但我还是相信你。''他又跺了跺脚,身子已向前窜出。

\hypertarget{ux7b2cux4e5dux5341ux4e94ux7ae0-ux9634ux9669ux6bd2ux8fa3}{%
\chapter{第九十五章
阴险毒辣}\label{ux7b2cux4e5dux5341ux4e94ux7ae0-ux9634ux9669ux6bd2ux8fa3}}

苏樱见花无缺的身形已向前窜出,忽然又道:``和你关在一起的那个人,叫江玉郎,你认不认得他?''

花无缺顿住脚步,不觉又叹了气,道:``我但愿不认他才好。''

苏樱叹道:``你为什麽不杀了他呢``留这个人活在世上实在是後患无穷。''

花无缺道:``他此刻既伤且病,我怎能向他出手?''

苏樱苦笑道:``这就是君子的毛病,但你若没有这毛病我只怕也\ldots\ldots{}''

她瞧见花无缺又旋动身形,立刻大声道:``等一等我还句话要告诉你。''

花无缺只得再次停下来,道:``什麽话?''

苏樱嫣然一笑,道``铁心兰并没有看错,你实在是个温柔又可爱的男人,也实在对她好得很。''

大家都知道,小鱼儿的性子有多麽急,要一个性子急的人坐在那里等人,实在是要他的命。

小鱼儿已急得像是只火里的蚱蜢,不停地走来走去,不停地向胡药师问;``你算准苏樱一定能找到这里来麽?''

胡药师本来很有把握,断然道:``是''

但等到後来,胡药师也有些着急了,忍不住道:``在下中的毒,只怕快发作了吧?''

小鱼儿忽然跳起脚大喝道:``告诉你,苏樱若不来,我再也不会为你解毒的。''

胡药师苦着脸道:``苏姑娘是否前来,和在下又有何关系你下的毒若是发作了;''

小鱼儿大声道:``毒性发作了,算你倒楣,你死了也活该,谁叫你说苏樱一定会来的''

他现在的确是蛮不讲理,只因他已快急疯了。

胡药师此他更急,刚乾了的衣服,又被汗湿透了。

只有江玉郎,却像是一点也不着急,他笑嘻嘻坐在那里,苏樱来不来,好像都和他没关系似的。原来他忽然发现,那见鬼的药力已开始在消散,他身子已渐渐舒服起来,渐渐开始有了力气。

小鱼儿眼睛都快望穿了,还是瞧不见苏樱的影子,终於忍不住道;``走,不管她来不来,咱们先去找她去。''

江玉郎悠悠道;``现在若先去找苏姑娘,再转回来救花公子,花公子只怕已\ldots\ldots{}''

他故意顿住语声,小鱼儿果然忍不住跳了起来,大喝道:``只怕已怎样?说!''

江王郎慢吞吞道;``卖不相瞒,我藏起花无缺的那地方,并不太舒服,而且有点不大透气,时间若是隔得太长,说不定会闷死人的。''

小鱼儿跳起来就想扑过去,但扑到一半,就硬生生停了下来,脸上的怒容立刻变成了笑容,哈哈笑道:``江兄是聪明人,总该知道花无缺若死了,对江兄你也没什麽好处。''

江玉郎叹了口气,道:``这个小弟自然明白的,只不过\ldots\ldots{}''

小鱼儿立刻道:``你救了他,我负责要苏樱将解药给你。''

江王郎苦笑道;``小弟现在已想通了,只觉世情皆是虚幻,生生死死,也只不过是一场梦而已,是否能拿到解药,小弟卖已不放在心上。''

他忽然说出这一番大道理,小鱼儿瞪大了眼睛瞧着他,道:``你\ldots\ldots 你真的是江玉郎麽妙极妙极,江兄原来是个老和尚投胎转世的。''

江玉郎又叹了气,道:``小弟虽已不再将这副臭皮囊放在心上,只不过\ldots\ldots{}''

他转头瞧了铁萍姑一眼,黯然道:``只不过她\ldots\ldots 她对我的恩情,却令我再也抛不开,放不下。''

铁萍姑痴痴地望着他,目中已是泪光莹莹,却不知是鹫讶,是欢喜,是相信,还是不信?

江玉郎叹道:``小弟经此一劫,再也无意与诸兄逐鹿江湖,只盼将恩仇俱一刀斩断,和她寻个山林隐处,安安份份的度此馀年,可是\ldots{}''他惨笑着接道:``可是小弟虽有此意,怎奈以前做的错事页在太多,小弟也自知鱼兄绝不会就此放过我的,是麽川小鱼儿正色道;``常言道,放下屠刀,立地成佛,江兄如此做法,小弟佩服还来不及,又怎麽会再找江兄的麻烦呢?''

江玉郎沉吟了半晌,缓缓道:``鱼兄博闻广见,想必知道野生蕈菌中有一种叫女儿红的。''

铁萍姑到这时才忍不住问道:``这女儿红又是什麽?''

小鱼儿道:``这女儿红乃是生在极阴湿之地的一种毒菌,据说无论谁吃了,不出叁五天,就会得一种怪病。''

铁萍姑道:``什麽怪病''

小鱼儿道:``这种病开始时也没什麽,只觉不过有些晕晕欲睡,精神恍惚,就好像得了相思病似的,除非每隔几个月,能找到一株婆草连根吃下去,否则这相思病就要越来越重,不出一年,就完蛋大吉。''

铁萍姑虽也觉得这名字取得妙不可言,有趣已极,但想到一个人若不幸吃下了这麽样一粒毒菌,那可实在是无趣极了。

只听小鱼儿笑着又道:``此时此刻,江兄忽然提起此物来,难道是想要小弟也害一害这相思病麽?''

江玉郎这次竟连狡赖都没有狡赖,很简单地回答道:``正是。''

小鱼儿却笑了,道:``这麽珍贵的东西,一时之间,你能到那里去找来给我吃?''

江玉郎道:``小弟若是去别处寻找,就算找个叁年五载,也末必能找得到,但凑巧的是,这附近就偏偏有一株,只要鱼兄答应,小弟立刻就可去为鱼兄掘来。''

铁萍姑终於也忍不住失声道:``你疯了麽?怎麽能说得出这种话?他\ldots\ldots 他怎麽可能答应你?''

江玉郎也不理她,缓缓接着道;``鱼兄想必知道,那恶婆草虽也和女儿红一样,十分稀罕珍贵,但却可以用人工来培养的,而小弟又恰巧知道培养它的法子。''

小鱼儿眼珠子直转,竟没有说话。

江玉郎又道:``这里的事办完之後,小弟就立刻找个地方隐居起来,专心为鱼兄培植恶婆草,鱼兄若想身体康健,自然也就会好生保护小弟的性命了。''

胡药师这才知道,他打的如意算盘,竟是要以这件事来要胁小鱼儿,要小鱼儿以後永远不敢找他的航烦。

但这想法却实在未免太天真了些,胡药师几乎忍不住要笑了出来,眼睛瞧着江玉郎暗笑道;``你难道以为小鱼儿是呆子麽?这种事你就算杀了我,我也不会答应的,何况这条比泥鳅还滑溜的小鱼儿?''

只见小鱼儿眼珠子转了半天,笑嘻嘻道:``你信不过我,我又怎信得过你?我怎知道你会为我培植恶婆草,又怎知这恶婆草一定能吃到嘴呢''

江王郎叹道:``小弟的病毒也一直不解,鱼兄要杀我,还是容易得很。''

小鱼儿道:``但我若找不到你呢?''

江玉郎笑道:``鱼兄若真的要找,小弟就算上天入地,也躲不了的。''

像小鱼儿这样的聪明人,竟会问出这麽笨的两句话来,江玉郎回答得更是妙不可言,说的话等於没说一样,而小鱼儿却偏偏像是相信了,只不过又问了一句:``我吃下了这女儿红你就去救花无缺?''

江玉郎道:``小弟若是矢言背信,鱼兄随时都可要小弟的命。''

小鱼儿叹了口气,道:``好,我答应你。''

小鱼儿竟真的答应了他。任何人都不会答应的事,他竟偏偏答应了。

胡药师呆呆地瞧着小鱼儿,暗道;``疯子,疯子,这人原来是疯子,别人说太聪明的人,有时往往会变成疯子,这话听来倒是一点也不错。''

铁萍姑也是目瞪呆,吃鹫得说不出话来。

江玉郎果然掘来了一株看来十分鲜艳的女儿红。小鱼儿果然笑嘻嘻吞了下去。

他抹了抹嘴,竟大笑道:``妙极妙极,想不到这女儿红竟是人间第一美味,我这一辈子,简直没有吃过这麽鲜嫩的东西。''

到了这时,江王郎目中也不禁露出狂喜之色,却故意叹了气,道:``绝代之佳人,大多是倾国倾城的祸水,致命之毒物,也常常是人间美味,唯有页药,才是苦口的。''

小鱼儿一把拉住他的手,笑道:``好听的话,大多是骗人的,江兄还是少说两句,紧去救人吧。''

石屋所在地,本来已十分荒僻,江王郎带着小鱼儿再往前走,地势就越来越是崎岖险峻。

他的毛病偏偏又发了,走两步,就喘口气,再走两步,又跌一跤,两条腿就像弹琵琶似的抖个不停。

小鱼儿实在快急疯了,到後来终於忍不住将他抱了起来,道:``那地方究竟在那里,你说出来,我抱你去。''

江玉郎道:``如此劳动鱼兄,小弟怎麽敢当。''

小鱼儿``嗤''的一笑,道:``没关系,你骨头轻得很,我抱你并不费力。''

铁萍姑跺脚道:``求求你们两个人,莫要再斗嘴了好不好''

江王郎叹道:``我怎敢跟鱼兄斗嘴,只不过,\ldots;''

他语声忽然顿住,手向上面一指,道:``鱼兄可瞧见上面那洞穴麽?''

小鱼儿随着他手指向上瞧去,只见生满了苍苔的山壁上,果然有个黑黝黝的洞穴,洞口还有一片石头凸了出来。

江王郎道:``这地方还不错吧''

小鱼儿道:``你为什麽不用块石块将洞口堵上呢?''

江玉郎道:``花公子现在已是寸步难行,小弟反正也不怕他逃走?''

小鱼儿忽然瞪起眼睛,高声道:``洞口既没有堵上,他怎麽会闷死?''

江玉郎神色不变,淡淡道:``也许不会被闷死,但荒山上的洞穴里,总难免有些毒蛇恶兽,:''

他话末说完,小鱼儿己纵身掠了上去。

江王郎道:``鱼兄不妨先将小弟放下来,看看这地方对不对。''

一这片石台上也长满了苍苔,滑不留足,小鱼儿放下了他,他连站都不敢站起来,忙到洞口前瞧了瞧,忽然大呼道;``花公子,小弟等来救你了,你听得见麽?''

只听洞穴回声不绝,却听不见花无缺的回应。

江玉郎皱起眉头,道:``花公子,你\ldots\ldots 你\ldots\ldots 你怎麽样了,怎地\ldots\ldots{}''

小鱼儿跺了跺脚,一把将江王郎拉到後面去,自己伏在洞口,极目而望,洞穴里黑得伸手不见五指,他什麽也瞧不见。

江玉郎道;``鱼兄,可瞧见花公子了麽?''

小鱼儿道:``你这小子究竟在玩什麽花样,为什麽\ldots\ldots{}''

话犹未了,忽觉一股大力自脚跟撞了过来,他一声呼尚未出口,身子已落叶般向洞穴中直坠了下去。

方才连路都走不动的江玉郎,此刻却忽然变得生龙活虎起来,一跃而起,向洞穴中呼道;``鱼兄,:小鱼儿,,:''

小鱼儿没有回应,过了半晌,才听得``咚''的一声。这洞穴竟深得可怕。

江玉郎仰天大笑道:``小鱼儿\ldots\ldots 小鱼儿,你毕竟还是不如我江玉郎,毕竟还是上了我的当了?''

铁萍姑从下面往上望,石台上发生了什麽事,她也瞧不真切,此刻听到江玉郎得意的笑声,才吃鹫道:``你将小鱼儿怎麽样了?''

江玉郎大笑道:``我不害死他,难道还等他害死我麽''

铁萍姑又鹫又恐,嘶声道:``你不是已改过了麽了不是只想和我安度馀生,怎地又\ldots\ldots{}''

她一面说着话,一面就想往上掠去,但身子刚跃起,忽又想到自己身上只穿着胡药师的一件长衫,里面却是空空的,若是跳起来,下面的胡药师的眼福就真不浅了,她只有赶紧落下来,掩住衣衫,不停地跺脚。

胡药师也吃鹫得呆住了,过了半晌,忍不住道:``小鱼儿既已中了女儿红的毒,你以後岂非正可以此要胁他,要他乖乖的听命於你,你现在就害死了他,岂非可惜。''

江玉郎笑道:``你想不通,小鱼儿也想不通的,所以他才会上当,方才那女儿红只不过是个钩子而已,你现在可想通了麽?''

胡药师不觉得又怔住了,只觉这江玉郎心计之深,手段之毒,做出来的事之凶狠狡诈,简直叫人梦想不到。

江王郎哈哈大笑道:``小鱼儿呀小鱼儿,你常常自命自己是天下第一个聪明人,如此你总该知道,天下第一个聪明人,倒底是谁了吧。''

胡药师忍不住又道:``但花无缺呢他难道也被你害死了''

江玉郎笑道:``你以坞花无缺很呆板麽?告诉你,他也会骗人的,他故意装出那副痴痴呆呆的模样,让你们不再提防他,他却乘机溜之大吉。''

胡药师怔了半晌,苦笑道:``那麽,白山君呢?''

江玉郎道:``那时我病发作得厉害,迷迷糊糊的,也没有瞧清楚,好像是瞧见他去追花无缺了。''

胡药师忽然跳起来,鹫呼道:``不好,我中的毒药力还末消散,我还得找他要解药。''

江玉郎忽然冷冷一笑,道:``很好,你就下去找他吧?,''

冷笑声中,忽然出手一掌,向胡药师拍了过去。

胡药师刚掠上石台,身子还末站稳,一口忾也没有换过来,若是立刻再跳下去,虽可避开这一掌,但真气既末换转,跳到地上後,纵不跌伤,身子也必定站不稳,那时江玉郎若再乘势进击凌空扑下,他再也难闪避。

石台上滑不留足,胡药师算准江玉郎在台上发招,下盘必不稳固,下盘若不稳,出手的力道就必定不会太强。

江玉郎一掌拍出,胡药师竟不避不闪,拚着挨他一掌,下面却飞起一脚,向江玉郎下盘横扫过去。

一这一招以攻为守,攻敌之所必救,正是绝顶厉害的妙着,但若非久经大敌的武林老手,就绝不敢使出这样的险招。

江玉郎笑道:``好个兔二爷,果然有两下子!''

他身形忽然一跃而起,双腿却已凌空出。

胡药师再也想不到他在这种地方,还敢用这种招式,大鹫之下,要想闪避已来不及了。

要知道胡药师方才出的一脚,此刻还末及收回,下盘更是不稳,江玉郎的脚尖,已踢向他咽喉。

他只有用手去接,手的力量,怎及脚大,他就算接得住这一脚还是难免要被江玉郎下去但江玉郎的脚若被他抓住,自也难免要被他一齐拖下去,这一着用的虽近无赖,但情急之下他也顾不得许多了。

谁知江玉郎身子凌空,竟还有馀力变招。

只见他只腿,刹那间竟一连出七八脚之多,胡药师莫说抓不到他,简直连他出腿的方位都已分辨不出。

他这才知道江玉郎不但凶狠狡猾,非人能及,武功之高,竟也大出他意料之外,他知道自己再也无法抵抗,不禁长长叹了口气,身子突然在石头上一稂,竟纵身向那深不可测的黑洞跳了下去。

铁萍姑痴痴地站在那里,动也不动,江玉郎着意卖弄,凌空翻身,就像是一只大蝴蝶似的落在她身旁,她也像是没有见到。

江玉郎笑嘻嘻道:``方才我出的那几脚,你可瞧见了麽?''

铁萍姑看也不看他,淡淡道:``瞧见了。''

江玉郎道:``那是北派谭腿中的精华``卧鱼八式,和胡家堡的盅影脚,武当派的``流星步,昆仑派的``飞龙式,四种武林绝技混合在一,变化而成的,我替它取了个名子,叫``踢死人不赔命,天下无双魔脚,你说妙不妙?''

铁萍姑冷冷道:``妙极了。''

江玉郎笑道:``你有个武功如此高明的夫婿,难道不高舆麽?''

铁萍姑忽然扭转头,直奔了出去。

江玉郎赶紧掠过去挡在她的前面,笑道:``你这是干什麽?咱们已有很久没在一齐,现在我的病已好了,咱们正可以好好的温存温存,你为什麽不理我?''

铁萍姑冷笑道:``你还是找别人温存去吧,像你这样人既聪明,武功又高的大英雄,大豪杰,我怎麽高攀得上?''

江玉郎笑道:``我去找别人去找谁我喜欢的只有你呀''

他一把抱起了铁萍姑,就去亲她的脸。

铁萍姑挣也挣不脱,跺脚道;``你\ldots\ldots 你\ldots\ldots 你放不放手''

江玉郎谜着眼笑道:``我不放手,我偏不放手,你打死我,我也舍不得放手的。''

他的手已伸进了袍子,铁萍姑的挣扎终於越来越没有力气,头声道:``你先放手,我问你一句话。''

江玉郎笑嘻嘻道:``你问呀,我又没有堵住你的嘴!''

铁萍姑道:``我问你,你害死了小鱼儿,难道还不过瘾,为何又要害死胡药师?''

江玉郎道:``我看见那小子对你色迷迷的模样,简直快气疯了,恨不得当时就宰了他。''

铁萍姑道:``你\ldots\ldots 你杀他,难道是为了我''

江王郎笑道;``也不知为了什麽,只要别人瞧你一眼,我就气得要死,何况他居然想打你的主意\ldots\ldots 除了我之外,谁敢动你一根手指,我拚命也要宰了他的。''

他嘴里说着,手动得更厉害。

铁萍姑脸上的怒容早已不见了,面颊上已泛起了红晕,不但语声头抖,身子也头抖起来。

江玉郎将嘴唇凑到她耳朵上,低低说了两句话。

铁萍姑立刻红着脸挣扎道:``不行,不可以在这里\ldots\ldots{}''

江玉郎道:``这里连鬼都没有一个,有谁会瞧见,来吧\ldots{}''

话还没有说完,铁萍姑也不知怎地,竟忽然从他怀抱里直飞了起来,同时又发出了一声鹫呼。

江玉郎也骇了一跳,情不自禁,随着她的去势向上面瞧去,只见铁萍姑白生生的两条腿在空中不停的挣扎飞舞,但身子却如旗花火箭般向上直冲,竟飞起有七八丈高,不偏不倚,落在一棵树上。

一这棵树自山壁间斜斜伸出来,铁萍姑的袍子竟恰巧被树枝勾住,赤裸裸的身子肚像是条白羊似的被吊了起来。

江王郎再也想不通她是怎麽会被吊上去的,忍不住大呼道:``快跳下来,我接住你。''

铁萍姑却像是己被吓呆了,竟连动都不会动。脸上已没有一丝血色,眼睛里的神色更是怖欲绝。但她的眼睛却没有瞧着江玉郎。

江玉郎忍不住又随着她的目光瞧了一眼,这才发现自己面前不知何时竟已站着个长发披肩的白衣人。只见她雪白的衣衫飘飘飞舞,身子却如木头人般动也不动,面上也戴着个木头雕成的面具,看来就像是忽然自地底升起的幽灵。

她随手一抛,就能将铁萍姑抛起八、九丈高而且不偏不倚地挂在树上,这份手力武功,简直骇人听闻。

一个男人正在兴致勃勃时,若被人撞破好事,那火气当真比什麽都来得大,江玉郎只觉一肚子鄱是火,把别的事全都忘了,大怒道:``你这人有什麽毛病,好生生的为何来找我的麻烦''

白衣人远是站在那里,既不动,也不说话。江玉郎火气更大,忍不住窜过去一拳击出心白衣人还是不动,只不过袍袖轻轻一拂,江玉郎击出去的一拳,也不知怎地,竟忽然转了回来。

只听``砰''的一声,这一拳竟打在他自己头上。

江玉郎脸立刻被打肿了,但头恼却被打得清醒过来,只觉两条腿畿乎再也站不住,颤声道``你\ldots\ldots 你莫非就是移花宫主?''

白衣人冷冷道:``凭你这样的人,也配说移花宫主四个字?''

江玉郎``噗''地跪在地上,嗄声道:``小人的确不配说这四个字,小人该打。''

他的确是聪明人,不等白衣人出手,就自己打起自己来,而且下手还真重,打的实在不轻。

白衣人冷冷的瞧着,也不开口。

\hypertarget{ux7b2cux4e5dux5341ux516dux7ae0-ux5978ux72e1ux8be1ux8bc8}{%
\chapter{第九十六章
奸狡诡诈}\label{ux7b2cux4e5dux5341ux516dux7ae0-ux5978ux72e1ux8be1ux8bc8}}

她不开口,江玉郎的手就不敢停,只见他一张又白又俊的脸,恍眼间就变得像猪肝一样,顺着嘴角往下直淌鲜血。

铁萍姑瞧得心都碎了,忍不住道:``宫主,求求宫主饶了他吧。''

白衣人这才抬起头来,道:``你为他求情,又有谁为你求情?''

铁萍姑头声道:``婢子自知罪孽深重,本就不敢求宫主饶恕的。''

白衣人道:``很好,那麽我问你,你将小鱼儿带到那里去了?''

铁萍姑道:``小鱼儿他\ldots\ldots{}''

她忽然想到自已若说出真相,宫主若知道小鱼儿已死在江玉郎手上,江玉郎只怕立刻就要被碎万段了。

白衣人道:``小鱼儿他怎麽样了?你为何不说?''

铁萍姑道:``他\ldots\ldots 他也到了这里,只怕是在东面那一带。''

白衣人道:``好,我这就去找他,但愿你说的不假。''

江玉郎这时已被自己打得躺在地上,但还是不敢停手。

白衣人叱道:``够了,停手吧。''

江玉郎挣扎着爬起来,叩头道:``多\ldots\ldots 多谢宫主。''

白衣人道:``现在,我要你在这里看着她,若有人伤了她,我就要你的命,若有人将她救走,我也要你的命,知道麽?''

江玉郎道:``小人知道。''

等到江王郎抬起头时,白衣人已又如幽灵般消失了。

他忍不住叹了口气,苦笑道:``这就是移花宫主,原来移花宫主就是这样子的,想不到我今日竟见着了她,只怕是走了运了。''

铁萍姑叹道;``幸好今日来的只是小宫主,若是大宫主来了,你我此刻只怕都活不成了。''

江玉郎出神地凝注着远方,也不知在想些什麽。

铁萍姑道:``但等她回来,你我还是活不成的,你害了小鱼儿,她绝不会饶你。''

江玉郎道:``为什麽?她本来不是要花无缺杀小鱼儿的麽?''

铁萍姑道:``不错,但她只许花无缺自已亲手杀小鱼儿,却不许别人动小鱼儿一根手指,就连她自己,也绝不伤小鱼儿的。''

江玉郎讶然道:``这又是为了什麽?倒是件怪事!''

铁萍姑道:``我也猜不透这是什麽道理,她们姐妹本来就是个怪人,无论如何,你现在快将我放下去吧,我半身发麻,已被她点了穴道。''

江玉郎叹道:``我就算救了你,咱们两人还是逃不脱她掌握的。''

铁萍姑道:``但咱们好歹也得试一试,等她回来了,反正也只有一死,现在若是逃走找个地方藏起来,说不定还可过几天快活的日子。''

江玉郎垂下头没有说话,过了半晌,忽又抬头道:``但你若不告诉她小鱼儿是被我害死的,她也就不会杀我了,是麽?''

铁萍姑怔了怔,道:``也许\ldots{}''

江玉郎道;``你方才既已骗过了她,为什麽不再骗下去呢?''

铁萍姑道:``但\ldots\ldots 但我\ldots\ldots{}''

江玉郎柔声道:``你既然反正是要死的,为何要我陪你一死呢?你若真的对我好,就该牺牲自己来救我,我一定永远也忘不了你。''

铁萍姑整个人都呆住了,她实在再也想不到江玉郎会说出这样的话来这实在不是人说的话。

忽听一人咯咯笑道:``妙极妙极,我已有很久没听过这麽妙的话了。''

另一人笑道:``这位仁兄若是女的,萧咪咪见着他也一定要自愧不如。''

第叁人道:``哈哈,两个萧咪咪,只怕也抵不上他一个。''

第四人大笑道;``自从欧阳兄弟死後,你们一直担心找不到人来凑数,现在不现成的就有一个在这里麽。''

笑声不绝,山坳後已走出四个人来。

只见这四人一个嘴巴特大,一个不男不女,一个满脸笑容,还有一个像叫化子的,背上却背着只麻袋。

一逅麻袋竟不停的在蠕蠕而动,而且里面还不停地有叫吟之声发出,这叫吟声也奇怪得很。

发出叫吟的人,虽像是很痛苦,很难受,却又像是很舒服,听得人忍不住从心里了起来。

那叫化子模样的人,左手还提着根树枝,竟将树枝当鞭子,不时往那麻袋上抽上一鞭。

他一鞭抽下去,麻袋里的叫吟声就更销魂,嘴里还含含棚糊的说着话,隐约可以听出,她居然是在哀求道:``求求你\ldots\ldots 抽重些好麽?求求你\ldots\ldots{}''

那叫化子模样的人却偏偏放下鞭子,不肯再抽了,反而向江玉郎笑道:``世上居然有人喜欢挨打,你可瞧见过麽?''

江玉郎倒买还没见过这样的人,简直连听都没听见过,他虽然最善应变此刻也不禁呆住了。

树上的铁萍姑又羞又急,竟不觉晕了过去。

来的这四人,无疑就是李大嘴屠娇娇白开心和哈哈儿了,但麻袋里这喜欢被人打的却又是谁泥?

李大嘴已走到江玉郎面前,咧嘴一笑,道:``这位朋友,你贵姓呀?''

江玉郎虽不知道这些人是什麽来头,但见到他们的模样一个此一个诡秘,倒也不敢再得罪他什。

他乾咳一声陪笑道:``在下蒋平,却不知各位尊姓大名。''

李大嘴笑道:``兄台年纪虽轻,想必也听说过``十大恶人的名字?''

哈哈儿道:``哈哈,你瞧见他这张嘴,也该知道他是谁的。''

江玉郎目光从他们脸上瞧了过去,掌心已不觉出了汗。

屠娇娇咯咯笑道:``小兄弟你只管放心,咱们来找你,并没有什麽恶意。''

江王郎忽地一笑,道:``各位俱是武林前辈,自然不会找在下这无名後辈麻烦的,在下非但十分放心,而且今日得见武林前辈的芊采,更实在高兴得很。''

屠娇娇吃吃笑道:``你们瞧,这孩子多会说话,嘴上就好像抹了蜜似的。''

哈哈儿道:``哈哈,这样的人,连我和尚见了都欢喜也就难怪树上的这位小姑娘,不惜为他玩命了。''

江玉郎正色道:``树上那位姑娘,与在下虽然相识,却不过只是道义之交而已,那里有什麽男女之情,前辈说笑了!''

屠娇娇道:``既然是道义之交,人家赤条条地被吊在树上,你为什麽不去救她呢?''

江玉郎叹了气道:``在下虽有相救之心,怎奈,\ldots 怎奈男女授受不亲,如今她不幸遭人羞侮赤身露体,在下若是去救她,岂非多有不便。''

屠娇娇道:``如此说来,你倒是个正人君子了。''

江玉郎道:``在下虽然浪迹江湖,但这礼义两字,倒也末敢忘记。''

屠娇娇忽然咯咯大笑了起来,指着江玉郎道:``你们瞧,他是不是有两下子,莫说萧咪咪,就连欧阳兄弟见了他,也非得拜他做师傅不行。''

哈哈儿道:``哈哈,欧阳兄弟说话,叁句中至少远有一句是真的,但他一共只说了四旬半话却有四句是假的。''

江玉郎道:``前辈又说笑了,在前辈面前,在下怎敢说谎。''

哈哈儿道:``你不敢说谎麽?哈哈,这就又是一句谎话。''

屠娇娇打断了他的话,娇笑道:``你说的句句都是实话?好,那麽我问你,你若是蒋平,有个叫江玉郎的小坏蛋,却又是谁妮?''

谎话被人当面揭穿,还能面不改色的人,每一万人中,大约只有一两个,江玉郎自然就是其中之。他非但脸不红,色不变,反而笑了起来。

屠娇娇瞧着他,似乎越来越觉得他有趣了,也笑着问道:``你笑什麽?''

江玉郎道:``要在前辈们面前说谎,岂非简直好像鲁班门前弄大斧,孔子庙前卖百家姓,但在下却偏偏自不量力,这还不可笑麽?''

哈哈儿拍手大笑道:``说得好,说得好,哈哈,这马屁实在刚好拍在咱们屁股上,拍得恰到好处,舒服极了。''

江玉郎道:``前辈们末和在下说话之前,想必早已将在下的底细都摸清了。''

屠娇娇笑道:``不错,咱们非但早已知道你叫江玉郎,是江南大侠的宝贝儿子,也知道这位小情人本是移花宫的门下。''

屠娇娇道:``你可知道咱们为什麽会对你如此关心?''

江玉郎微微一笑,道;``莫非前辈们想替在下做媒麽?''

屠娇娇笑道:``我若有女儿,宁可嫁给李大嘴,也不会嫁给你,李大嘴至少远不会吃她的脑袋,但是你,吃了人只怕连骨头都不会吐出来。''

江王郎微笑道:``前辈过奖了,在下怎比得上李老前辈''

李大嘴道:``你也用不着客气,我吃人最多只不过是一个个的吃,但你吃人却是一队队的往下吞,在狮镖局的那些人,不是被你一夜之间全都吞下去了麽''

江玉郎还是面不改色,笑道:``前辈们将在下调查得如此清楚,是为了什麽呢''

屠娇娇道:``你也许不知道,自从欧阳兄弟两人死了後,``十大恶人』其实剩下九个了。''

屠娇娇又道:``除了欧阳兄弟已经一命呜呼外,这些年来,恶赌鬼好像渐渐要改邪归正,做好核子了,狂师铁战的毛病也越来越大,没有别人和他打架时,他就打自己,那位``迷死人不赔命的萧咪咪,更不如在那个洞里藏了起来,所以咱们此番出山之後,忽然发觉``十大恶人的名头,在江湖中已渐渐不大能吓唬人了。''

江王郎自然是知道萧咪咪在什麽地方的萧咪咪已被他和小鱼儿关在地牢里,这辈子只怕再也休想出头。

但他只是淡淡笑道:``前辈莫非是想找个人来代替欧阳兄弟的位置''

屠娇娇道:``不错,咱们若想重振『十大恶人』的名声,非找个生力军不行。''

江玉郎目光闪动,笑道:``但这人倒的确难找得很,据在下所知,江湖中够资格能和前辈并驾齐驱的人,只怕还没有几个。''

屠娇娇瞧着他微微笑道:``远在天边,近在跟前,你就是一个。''

江王郎赶紧道:``在下怎当得起。''

哈哈儿道:``哈哈,你用不着客气,你年纪轻轻,已有这麽样的成就,再过两年,只怕连咱们都没法子和你相此。''

江玉郎像是觉得有些受宠若鹫,连声道:``不敢当,不敢当,前辈们如此抬举在下,却叫在下如何报答呢?''

李大嘴抚掌大笑道:``有意思,有意思,你能说出这句话来,就表示你这人真在够意思得很,也不枉咱们对你另眼相看了。''

白开心忽然道:``但小伙子你可千万莫上他们的当,他们拉你入伙,只不遇是要你为他们做件事而已。''

一逼位仁兄``损人不利己''的外号,果然是名下无虚,他半天不说话,一开口就必定是拆人台的。

江玉郎微笑道:``前辈虽是一番好意,但在下若能有机会为前辈们效劳,正也是不胜荣宠之至,前辈们有何吩咐,只管说出来就是。''

屠娇娇道:``武林中有个极厉害的人物,叫魏无牙,他就住在这山上,你自然也知道的,但你可知道,他那老鼠洞里现在来了位贵客麽?''

她话锋一转,忽然转向魏无牙身上,江玉郎脸上的微笑立刻瞧不见了,咳嗽两声,乾笑道:``这世上若只有一个在下不愿打交道的人,那就是魏无牙了,就算天下的人都死尽死绝,在下也不愿和他有任何来往,他洞里是否来了位贵客,在下既不会知道,也绝不想知道。''

屠娇娇道:``只可惜这位贵客却偏偏是你认得的。''

江王郎不禁怔了怔,道:``我认得?我怎会认得?''

屠娇娇道;``魏无牙平生没有一个朋友,就连他们『十二星象』中的人,瞧见他都像是见了鬼一样,避之唯恐不及。''

江玉郎笑道:``这正是:老鼠过街,人人喊打,愿意和毒蛇猛兽为伍的人,在下倒也见过畿个,但愿意和老鼠交朋友的人,只怕连一个都不会有。''

屠娇娇笑道;``你错了,愿意和老鼠交朋友的人,也有一个的。''

李大嘴接着道:``事实上他简直已将魏无牙哄得服服贴贴,他无论说什麽,魏无牙都听他的,魏无牙这辈子从来也没有对别人这麽好过。''

江王郎笑道;``如此说来,这位仁兄的本事倒的确不小。''

屠娇娇;``你可知道这人是谁麽?''

江玉郎脸上终於露出了惊奇之色,道:``在下实在想不出有神通如此广大的朋友。''

屠娇娇吃吃笑道:``谁说他是你的朋友\ldots\ldots 你虽没有神通如此广大的朋友,却有个神通广大的老子,你难道忘了麽?''

江于郎这才真的怔住了,失声道:``是我爹爹?''

屠娇娇道:``不错,魏无牙的贵客,就是江南大侠江别鹤。''

江玉郎怔了半晌,长叹道;``想不到家父居然和魏无牙交上了朋友。''

他嘴里虽在长叹,目中却忍不住露出了欢喜之色。

屠娇娇笑道:``和魏无牙交上朋友又有什麽不好,有了这麽硬的靠山,就算移花宫主想找他的麻烦,他也用不着害怕了。''

江玉郎几乎忍不住要笑了出来,试探着问道:``那麽,前辈的意思是要在下做什麽呢?''

屠娇娇和李大嘴对望一眼,李大嘴道:``你若成了魏无牙的贵客,在那洞中自然就可随意走动\ldots\ldots{}''

江玉郎道:``前辈莫非是要在下打听件什麽事?''

李大嘴抚掌笑道:``不错,和你这麽样有头恼的人说话,的确是件令人愉快的事。''

李大嘴和屠娇娇又交换了个眼色,屠娇娇笑道:``那也不是什麽大不了的事,只不过,咱们有几只箱子,据说已落在魏无牙手里,你不妨顺便去瞧瞧箱子是不是真的在那里若在那里,是在什麽地方?然後咱们再一齐想法子把它弄出来。''

江玉郎目光闪动,显然对这件事也越来越有兴趣了,但脸上却怍出不大关心的模样,淡淡笑道:``却不知那是几只什麽样的箱子?箱子里装的是什麽''

哈哈儿道:``哈哈,那只不过是几只破铁箱子而已,是黑色的,看起来又笨又重,那麽笨重的箱子,别人绝不会有,所以你一看就会知道的。''

屠娇娇笑道:``箱子里本来装着有些珠宝,但魏无牙说不定早已将珠宝拿出来了。''

江王郎道:``箱子既已是空的,前辈们为何还要苦苦寻找?''

屠娇娇叹了口气,道:``在别人眼中,那虽然只是几破铁箱子,但在咱们眼中,它却是无价之贲。''

江玉郎的眼睛更亮,道:``无价之宝''哈哈儿道:``哈哈,这无价之宝,却是一两银子也页不出去的,只不过因为箱子上的油漆有些不同,所以在咱们眼中才变得十分珍贵。''

屠娇娇道;``你可知道那油漆是用什麽调成的麽''

她不等江玉郎回答,就又接着道:``那是用血调成的,是用咱们仇人的血调成的,咱们这肚二人都已老了,老得连雄心都已消磨,只有那几箱子,还可以令咱们重想起以前那些光辉灿燎的日子,所以咱们无论如何,也不能让它落在别人手里。''

江王郎像是已听得呆住,半晌没有说话。

屠娇娇道:``若是世俗的珍宝,无论有多少,既已落在魏无牙手里,咱们也就算了,犯不上冒险去老虎头上拔毛,咱们就算等着要花钱,到别的地方去抢,岂非容易得多麽?''

李大嘴握紧拳头,小声道:``但这畿口箱子若丢了,咱们这辈子就完蛋大吉,所以,小兄弟你无论如何,也得帮咱们这个忙,咱们一定忘不了你的好处。''

江王郎垂头瞧着自己的手,就好像他从来也没有瞧见过这双手似的,简直瞧得出神极了。

李大嘴道:``小兄弟,你难道不信咱们的话?''

江王郎道:``那畿箱子在别人眼中既是不值一文,魏无牙也必然不会看重的,他若已取出箱子里的珍贸,说不定早已将箱子抛却。''

屠娇娇道:``咱们也曾考虑过这问题,所以魏无牙若已将箱子抛却,就烦小兄弟你打听打听,他将箱子抛到什麽地方去了?''

她一笑接着道:``咱们现在虽已是自己人,但也不会要小兄弟你白辛苦的,只要事成,咱们一定想法子去弄万两黄金,和几个夭娇百的美人儿来让你享受享受,而且还保证替你保守所有的密。''

江玉郎满面俱是欢喜之色,道;``前辈可是要在下立刻就去麽?''

屠娇娇道:``自然是越快越好。''

江玉郎忍不住往树上瞧了一眼,道:``那麽她\ldots{}''

屠娇娇道:``但现在你总该已知道,你和她缠在一齐,是只有麻烦,没有好处的。''

江玉郎叹了口气,道:``就算有好处,也不会有麻烦多。''

屠娇娇笑道:``正是如此,何况,她长得虽不差,身材也不错,但只要你事成之後,我负责替你找十个此她更迷人的小姑娘来。''

她附在江玉郎耳边娇笑道:``而且我还可以先教给她们畿手,可以让你欲仙欲死的功夫。''

江王郎似乎已笑得阖不拢嘴来,道;``既是如此,在下立刻就走,只不过,在下事成之後,该如何和前辈们联络呢?''

屠娇娇道:``无论事成不成,叁天之後,你到洞口兜个圈子,咱们自然会想法子和你说话的。''

江玉郎道;``好,就是这样,一言为定。''

他什麽都不再说,也不再瞧铁萍姑一眼,立刻就飞也似的走了。

\hypertarget{ux7b2cux4e5dux5341ux4e03ux7ae0-ux80f8ux6709ux6210ux7af9}{%
\chapter{第九十七章
胸有成竹}\label{ux7b2cux4e5dux5341ux4e03ux7ae0-ux80f8ux6709ux6210ux7af9}}

李大嘴望着江玉郎走远,才皱眉道:``这小子走得那麽快,我看有些不保险。''

哈哈儿道;``哈哈,他这是怕移花宫主来找他算账的,所以赶紧想躲到那老鼠洞里去。''白开心冷冷道:``我看他对咱们说的话,未必就真的相信了,你们若认为他真的会为你们找箱子,那才是做梦。''屠娇娇笑道;``我说的话既合情,又合理,他为什麽不信何况,这小子又贪财,又好色,万两黄金十个大美人儿难道还打不动他?''白开心道:``他就算找着箱子,未必会交给你们的。''屠娇娇笑道:``他不交给咱们,要那几日空箱子又有什麽用?

哈哈儿大笑道:``不错,这小子是个聪明人,只要用几日空箱子来换黄金美人,这麽划算的事他难道还会不做。''白开心也忍不住笑了,道;``但换过来之後,我一定要告诉他这几日又旧又破的空箱子,究一竟有什麽好处,我们要瞧瞧他那时的脸色。''

哈哈儿道;``哈哈,那时他脸色一定比你的屁股还要难看得多。''

说起屁股两字,白开心的眼睛已向树上瞧了过去,腿着眼笑道:``喂丁小姑娘,上面的风很大,你不怕着凉麽?''铁萍姑仍然晕迷不醒,李大嘴却皱眉道:``你这小子背上还背着一个,又想打别人的主意了麽?''白开心笑嘻嘻道:``这位小姑娘孤苦伶仃,又偏偏遇着个没有心肝的薄情郎,实在怪可怜的,我不去安慰她谁去安慰她。''屠娇娇笑道:``很好,你快去安慰她吧但等到移花宫主找上门来时,你可莫怪咱们不帮你的忙了。''白开心咳嗽一声,嘻嘻笑道:``老实说,像她这麽样痛苦的人,我也安慰不了的,何况,我袋子里已有了一个,年纪虽然大些,但姜是老的辣,老的才去火。''屠娇娇笑道:``你现在总算懂得些男女之间的门道了,只可惜男人却是年轻力壮的才好,否则我\ldots\ldots{}''白开心大笑道:``幸好我年纪大些,否则若被你看上,那才真是天大的麻烦。''

屠娇娇瞪限道:``有什麽麻烦?''

白开心笑道:``别的麻烦也没什麽,只不过,谁也弄不清你那几天是男的,那几天是女的,若是弄错了时辰,岂非危险得很。''

李大嘴抚掌大笑道:``妙极妙极,想不到你这样的俗人,也能说出如此妙不可言的话来,莫非是这些日子来,已渐渐受了我的感化。''

白开心道:``不错,古人说得好上同气相应,近朱者赤,这些日子来,小弟能和李兄这样的风雅之士朝夕相处,说话自然也渐渐变得有味起来。''

一这两人本是天生的冤家对头,虽然两人都名列十大恶人,但见面的时候并不多,而一见面不是斗,就是斗手。

白开心在江湖中的仇家也并不少,但他就为了李大嘴,是以宁可在江湖中像野狗般东藏西躲也不肯躲到恶人谷去。

他此刻竟忽然说出这种话来,李大嘴倒不禁怔住了。

屠娇娇笑道:``你们两个混蛋闹够了麽?若是闹够了,就快回去吧!''

哈哈儿道:``不错,杜老大只怕已在那边等得急了,哈哈,你两人总该知道,杜老大若是生起气来,那就不是闹着玩的了。''

白开心叹了气,道:``想不到冷冰冰的杜老大,居然会对那小鱼儿这样好,还生怕小鱼儿找不着,一定要留在那里等,他若知道小鱼儿永远再也不会去了,一定伤心得很,咱们还是赶紧回去,好生安慰安慰他吧。''

李大嘴大笑道:``你以为小鱼儿真的已被那江玉郎害死了麽?''

白开心瞪眼道;``你方才难道没有听见?''

李大嘴笑道:``你放心,江玉郎若能真的害死小鱼儿,他就不是小坏蛋,是活神仙了。''

哈哈儿道:``只怕连活神仙都害不死小鱼儿的,哈哈,我第一个放心得很。''

屠娇娇笑道:``小鱼儿若是死了,我少不得也要掉两滴眼泪的,又怎会如此开心?''

白开心道;``既是如此,你们为什麽也要害他,故意留下那些漂志,骗他到那老鼠洞去,这岂非存心要他死在那大老鼠手上麽?''

屠娇娇笑道:``这只因咱们知道就算那大老鼠也弄不死他的白开心冷笑道;``你只怕没有这麽好的心吧?你只不过是怕们,所以就想借刀杀人,要他的命?''

李大嘴怒道;``你这张狗嘴,为什麽永远说不出人话来?''

白开心怒道;``老子说的难道你敢不承认?''

屠娇娇嘻嘻笑道:``咱们就算承认也没关系,但我告诉你,会为他掉眼泪的\ldots\ldots{}''

这时竟真的有一滴眠泪从树上掉了下来,幸好他们已离开了一一和燕南天勾结在一齐,来害你算他是被咱们害死的,我还是垣树林子,谁也没有注意。

铁萍姑并没有真的晕过去,只不过,在她这麽样悲惨的处境下,她除了假装晕过去之外,还有什麽更好的法子?他们说的每一句话,她都听到了。

她再也末想到江王郎对她竟完全都是虚情假意,更末想到江玉郎竟会如此轻易地抛弃了她。

她的心早已碎了,只等他们走光之後,才忍不住放声大哭起来,她恨不得现在立刻就能死去。

她自已也想不到自己怎会对这小畜牲如此多情。

一这也许是因为她在移花宫里忍受的寂寞太久,压制的情感太多,所以一旦发作,就不可收拾,她本来从不如流泪的滋味,但现在眼泪却流个不停。

也不知过了多久,她忽然发觉又有双眼睛在瞬也不瞬地瞧着她,但这双眼并不如别人那麽贪婪,那麽可恨。

一这双眼非但美丽,而且明亮得就像是春天晚上升起的第一颗星,叫人见了,几乎忍不住要迫她朝拜下去。铁萍姑从来也没有见到如此动人的眼睛。这双眼睛的主人笑了。

她柔声笑道:``这位姑娘,你贵姓呀?''

铁萍姑竟不由自主答道;``我姓铁。''

铁萍姑瞧着她那绝世的风姿,瞧着她身上那华美的衣衫,想到自己狼狈的模样,忍不住闭起眼睛,眼泪又落了下来。

那少女柔声道;``你一定很不愿意在这样子时见到我,但你也用不着难受,这世上的坏人实在太多,像我们这样的女孩子,都免不了要受人欺负的,你若是知道,世上比你遭遇更悲惨的人还多得很,你也许就不会这麽样难受了。''

铁萍姑忍不住道:``世上难道真还有\ldots\ldots 还有比我更不幸的人''

那少女道;``怎麽会没有呢你可知道,世上每一个城市里,都有一些可怜的女孩子,被一些她素不相识,甚至是她们厌恶的人在蹂躏,但她们还不能像你这样尽情一哭,她们还得装出笑脸,去讨好那些蹂躏她们的人。''她的确很会安慰别人,只因她很了解人们的心。

铁萍姑果然不再哭了,过了半晌,忍不住道:``你能不能将我救下去?我一定\ldots\ldots 一定重重谢你。''

那少女叹了口气,道:``你用不着谢我,我也很想救你的,只可惜我连梯子都爬不上去,这麽高的树,我简直连瞧着都头晕。''

铁萍姑道:``你\ldots\ldots 你难道一点武功都不会?''

那少女笑道:``你好像很奇怪,是麽?其实这世上不会武功的人比会武功的人可多得多了,大多数正常的人都不会武功的。''

铁萍姑长长叹息了一声,黯然道;``那麽你\ldots\ldots 你还是快走吧?''

那少女道;``我至少可以为你做些事,你冷不冷?我在下面生堆火好麽?''

铁萍姑方才又是羞恼,又是悲惨,又是害怕,竟忘了寒冷,现在才觉得全身都已冷得发抖,山风吹在她身上,就像是刀割一样。

只见那少女果然拾了些枯枝,又自怀中取出个很精巧的火子,在树下生起一堆火来。

那少女笑了笑,道:``我叫苏樱。''

``苏樱,你就是苏樱?''铁萍姑又吃了一鹫,忍不住失声呼了出来。

铁萍姑默然半晌,嗄声道:``你到这里来,是不是想找一个人''

苏樱也有些鹫讶了,道:``你怎麽会知道?难道你\ldots\ldots 你也认得我要找的那个人?''

铁萍姑黯然道:``不错,我认得他。''

苏樱叹了口气,苦笑道:``世上所有美丽的女孩子,好像都认得他,你说奇怪不奇怪看来我竞争的对手倒不少哩。''

铁萍姑道:``我不会和你竞争的,以後只怕也永远没有人和你竞争了。''

她一句话末说完,眼泪又落了下来。

苏樱脸上忽然变了颜色,失声道:``你这句话是什麽意思?''

铁萍姑流泪道:``他\ldots\ldots 他已被人害死了''

苏樱全身的血液,像是一下子就结成了冰。

她木然怔了半晌,苏樱忽又笑了,大笑道:``你一定是弄错了,小鱼儿怎麽会被人害死世上又有什麽人能害得死他?他不害死别人,已经很客气了。''

铁萍姑凄然道;``我本来也不信世上有人能害得了他的,但这次却不能不信,因为这次是我自己亲眼瞧见的。''

苏樱全身都发抖了,顶声道:``你亲眼瞧见的?是\ldots\ldots 是谁害死了他?''

铁萍姑道:``那人叫江玉郎,他将小鱼儿推到那边山壁上的洞里去了,那山洞深不可测,何况小鱼儿还中了毒\ldots\ldots{}''

她话末说完,苏樱已向那边山壁奔了过去。

一这山壁笔立千尺,宛如刀削,那洞穴离她又至少有十丈,其间虽然也有可以落脚的地方,但轻功稍差的人也难跃上,何况丝毫不会武功的苏樱。平日此谁都镇定的苏樱,此刻不禁也失常了。

她早已泪流满面,跺着脚道:``我为什麽不学武功?谁说武功是没有用的\ldots\ldots{}''

铁萍姑道:``你能上得去麽?''

苏樱道:``无论如何,我也要想法子上去的,而且我一定有法子上去!''

她说这句话时,语声忽然变得无比坚定,说完了这句话,她立刻就擦乾了眼泪绝不再哭泣!

她就算要哭泣,也要等到以後,因为她知道现在不是哭泣的时候,她知道眼泪并不能帮助她解决任何事。

铁萍姑瞧见她的转变,也看出她的决心,心里不禁暗暗叹息:``想不到这弱不禁风的女孩子一苋有这麽强的自信,这麽大的决心,而我呢?\ldots\ldots{}''

胡药师的运气不错。

他掉下去的这山洞,页在比他想像中还要深得多,这山洞外面最多只有十丈,里面却深了不止六倍。

从五十丈高的地方跌下去,就算这人的轻功已天下无双,还是一样难免要摔得四分五裂。

胡药师自己也以为自己是必死无疑的了土他还未来得及再转第二个念头,只听``噗通''一声身子已跌入水中,这山洞底下,原来是一池水。

胡药师先吃了一鹫,但鹫吓立刻就变成了欢喜,他既没有摔死,小鱼儿自然更不会跌死了。

他想从水里跳起来,但水却不浅,育一头栽进水里,喝了两口又咸又臭的水,几乎呛得他透不过气来。

只听小鱼儿笑嘻嘻道:``我正觉得寂寞,有朋自天上掉下来,不亦悦乎,只可惜这里没有酒,也只好请你喝两口臭水了。''

山洞里虽然很暗,但总算有天光从那里透进来。胡药师揉了揉眼睛,已瞧见小鱼儿了。

只见小鱼儿坐在旁边一块大石头上,他肚子里装满了无可救药的女儿红,又被人推到这插翅也难飞出的洞里来,但他脸上居然还是笑嘻嘻,非但一点也不发愁,而且还像是开心的很。

胡菜师也游过去爬上石头,忍不住问道:``你\ldots\ldots 你难道不发愁''

小鱼儿笑道:``发愁若能使我逃出去,我早就发愁了。''

胡药师默然半晌,吃吃道;``那解药浸了水之後,还能用麽?''

小鱼儿道:``你放心,那解药我藏得很妥当,水浸不透的。''

胡药师咳嗽两声,乾笑道;``现在鱼兄和在下同在危难之中,已可算得是同病相怜的患难之交,鱼兄现在总该将解药赠给在下吃了。''

小鱼儿道:``不可以。''

胡药师道;``为\ldots\ldots 为什麽?''

小鱼儿笑嘻嘻道;``我解药不给你,你就会一直听我的话,我将来就算养个儿子,也不会像你这样乖的,有这样乖的人在旁边,岂非是件很令人愉快的事,我为什麽要将解药给你呢?''

胡药师苦着脸道:``但\ldots,:但在下\ldots\ldots{}''

小鱼儿道:``你只管放心,你中的毒暂时绝不会发作的。''

他们说话的声音自然很小,因为空谷传音,山洞里又有水,说话的声音一大,外面立刻就会听见的。

但他们却末想到,外面说话的声音,这里竟也能听得见,在外面的人,瞧见四野无人,更绝不会想到隔墙有耳,是以说话时自然也不会有什麽顾忌。

江玉郎在那里向铁萍姑花言巧语时,小鱼儿骁得只是摇头叹气,胡药师几次要说话,都被他拦住了。

忽听铁萍姑一声鹫呼,小鱼儿正以为她不知被江玉郎怎麽欺负了,但这时却已响起江王郎的呼声。

接着,他又听到江玉郎、铁萍姑和移花宫主说的那些话听到了这些话,小鱼儿就像个石头人似的怔住了。

他这时才知道铁萍姑是移花宫的门下。

过了半晌,只听小鱼儿喃喃道:``原来铁萍姑竟是移花宫门下,难怪她那天一见到花无缺,就悄悄溜走了二那麽``铜先生和``木夫人就一定是移花宫主改扮的了,这也难怪移花宫主要花无缺听铜先生和木夫人的话,但移花宫主好生生的为什麽要改扮成别人呢?''

他将前因後果,每件事都仔仔细细想了一遍,想得头疼了起来,但却越想越糊涂,越想越不明白。

想到名震天下,人人畏之如鬼的移花宫主,竟被他支得团团乱转,甚至在厕所的外面等他大使,他又忍不住笑了出来。

突听胡药师笑道:``妙极妙极,移花宫主刚走,``十大恶人又来了好几个,我看江玉郎这小子以後也没有什麽好日子过了。''

小鱼儿这才回过神来,听了半晌,展颜笑道:``来的是``不男不女屠娇娇,``不吃人头李大嘴,``笑里藏刀哈哈儿,和``人不利己』的白开心。''

胡药师道:``你和他们很熟麽''

小鱼儿道:``天下只怕再也没有此我跟他们再熟的人了。''

胡药师精神一振,道;``那麽你现在为何还不赶快要他们来救你?''

小鱼儿笑道;``等一等,我还要听听他们究竟在搞什麽鬼。''

等到他们说出魏无牙的贵客就是江别鹤,小鱼儿又是一鹫,这才知道那天他重伤垂死时,无牙洞里来的人就是江别鹤,若非江别鹤到了,苏樱还末必能将他救走,想到这里,小鱼儿不禁又笑了。只听胡药师又道:``奇怪,他们为何要将几日箱子看得如此重要呢?''

小鱼儿笑道:``少年戒之在斗,老年戒之在贪,一个年纪越大,对钱财也就看得越重,竟似乎已忘记人若死了,是连一文钱也带不走的。''

胡药师道:``但他们要的只是畿口箱子呀。''

小鱼儿微笑着,不再说话了,但眼睛里却发出了光,过了半晌就龉得屠娇娇他们说起他了。

听到那些漂志果然是他们设下来骗他的陷阱,小鱼儿脸色不禁又变了,默然半晌,摇头苦笑道:``想不到竟不出苏樱所料,连你们都想要我的命,但你们可知道,我早已知道燕大叔的秘密了麽,我并没有想要你们的命呀?''

他叹了几气,忽又开心起来,笑道:``只不过一个人死了後,若能赚得屠娇娇几滴眼泪,也真算不容易了。''

小鱼儿最大的本事,就是无论在多麽恶劣的情况下,他都有法子让自己变得开心起来。

胡药师却再也没有这样的本事,他现在自然也已知道小鱼儿是不会要屠娇娇他们出手相救了。

胡药师愁眉苦脸地怔在那里,再也打不起精神来。

小鱼儿却拍了拍他肩头,笑道:``你放心,就算他们不来救我,也有人会来救我的。''

胡药师还想再问,这时外面却已传来苏樱说话的声音。

听到後来,胡药师忍不住叹了气,道:``苏姑娘对鱼兄你当真是情深一往,有这麽样的佳人垂青,鱼兄你的福气页在不错。''

小鱼儿竟也叹了气道:``你若觉得这是福气,我就转让给你吧。''

胡药师只有笑了笑,过了半晌,忍不住又道:``但在下实在想不出她有什麽法子?''

小鱼儿笑道:``你若能想得出她的法子,也就不会像现在这麽样倒楣了。''

突听铁萍姑大声呼道;``苏姑娘,这石壁滑不留足,你爬不上去的。''

听她的语声,似乎很为苏樱着急,显见得苏樱一定爬得很狼狈,很艰苦,小鱼儿也不禁叹息道;``她那双脚一定又白又嫩,若被割破了,倒可惜得很。''

胡药师也叹道:``看她的模样那麽娇弱,倒真想不到她有这麽大的决心。''

小鱼儿道;``但像她那样的聪明人,竟会用这麽笨的法子,却叫我失望得很。''

一这时外面根本听不见苏樱的声音,铁萍姑却不时发出一声鹫呼,显见得苏樱的处境必定真是危险得随时都可能跌下去的。

胡药师微笑道:``一个女子若对男人有了情意,根本就不必有什麽理由,而且,女人们的理由,男人根本永远也不会明白的。''

小鱼儿叹道:``不错,只要碰见女人,我也只有自认倒楣的!''

突听铁萍姑一声欢呼。又听得苏樱大声道:``小鱼儿,我来找你了,你听得见我说话麽午,''

一这语声竟已是从上面洞口发出来的,空谷回应,小鱼儿非但能听得到,而且耳朵都快要被震破了。胡药师刚想说什麽,小鱼儿已将他的嘴掩住,悄声道:``你千万不能回答她,否则她说不定会跳下来的。''

只见苏樱的脸,已在洞口露了出来,只不过洞太深,洞里的光线又太暗,所以小鱼儿虽能看到她,她却看不到小鱼儿。

小鱼儿甚至可以看到她的脸已被划破了,满脸湿淋淋的,也不知是汗水,还是眼泪。

苏樱嘶声道:``小鱼儿,你为什麽不回答我的话?你\ldots\ldots 你怎会这麽样没用,连江玉郎那样的小畜牲都能害得死你,岂非丢人丢到家了。''

小鱼儿附在胡药师耳畔悄声笑道:``她这是在用激将法,想要我说话,我就偏偏不上她这个当。''

苏樱又呼道:、:、、、``我辛辛苦苦救了你,你又这样糊里糊涂地死了,你怎麽对得起我,你,你简直太令我失望了。''

小鱼儿还是不说话。这次苏樱也说不出什麽了,忍不住放声大哭起来。胡药师平日看她一举一动,风姿都那般优美,无论遇着什麽事,神情都那样镇定,再也想不到她也会像这麽样号淘大哭,哭得就像孩子一样。

只听铁萍姑道:``你自己方才还说过,世上遭遇比我们更悲惨的人,还多得很,连我都不再哭了,你又何必哭呢?''

苏樱痛哭着道:``你放心,我哭过这一次,以後就不再哭了,所以这次我一定要痛痛快快的哭一场,你也用不着再劝我。''

也不知过了多少,苏樱的哭声非但没有停止,反而越哭越伤心,竟真的像是要将所有的眼泪都在这一次哭出来。铁萍姑嗄声道:``求求你,莫要再哭了好麽,你若再哭,我\ldots\ldots 我也\ldots\ldots{}''

话末说完,她自己也已失声哭了出来。

苏樱却忽然不哭了,道:``你我萍水相逢,总算还很投缘,我希望你以後能想法子用石块将一这山洞填满,免得有别人再来打扰我们。''

铁萍姑道:``你\ldots\ldots 你怎麽能死呢据我所知,你和小鱼儿又没有什麽山盟海誓,你为什麽要为他死。''

苏樱淡浃道:``我并不觉是要为他死,我只觉得活着没什麽意思了。''

胡药师动容道:``鱼兄,到了这地步,你还不说话麽?''

小鱼儿叹道:``你以为她真会死麽?她这只不过是吓吓人的,你难道不知道,女人最大的本事,就是一哭二闹叁上吊。''

胡药师道:``但是她\ldots\ldots{}''

话末说完,突听铁萍姑一声鹫呼。苏樱已从上面坠了下来。

\hypertarget{ux7b2cux4e5dux5341ux516bux7ae0-ux751fux6b7bux4e24ux96be}{%
\chapter{第九十八章
生死两难}\label{ux7b2cux4e5dux5341ux516bux7ae0-ux751fux6b7bux4e24ux96be}}

小鱼儿这才真的吃了一鹫,用尽全力,一跃而起,想凌空抱起苏樱的身子,但苏樱下坠之势却实在太猛,小鱼儿武功纵已非昔比,还是接不住的,只听``噗通''一声两个人同时掉在水里。

水花溅起,过了半晌,才瞧见小鱼儿湿淋淋地从水里钻了出来,抱着苏樱,跳到石头上。

胡药师忍不住微笑道:``她并不是故意说来吓吓人的,是麽?''

小鱼儿叹了口气,苦笑道;``这丫头倒买和别的女人有些不同,我简直忍不住要开始怀疑她究竟是不是真的女人了。''

他本以为苏樱这下子必定早已吓得晕了过去。谁知``这丫头''的身子虽此春天的桃花还单薄,神经却坚轫得像是雪地里的老竹子,此刻非但没有晕过去,而且还像是觉得很舒服、很有趣的样子,正瞪着一双大眼睛,在瞬也不瞬地瞧着小鱼儿。

小鱼儿怔了怔,忽然一松手,将苏樱抛在石头上,大声道:``我问你,你这究竟是什麽意思,我和你根本连狗屁关系都没有,你为什麽要为我死?难道你要我感激你?一辈子做你的奴隶?''

苏樱悠悠道;``我也不想要你做我的奴隶,我只不过想要你做我的丈夫而已。''

小鱼儿又怔了怔,指着苏樱向胡药师道:``你听见没有?这丫头的话你听见没有?脸皮这麽厚的女人,你只怕还没有瞧见过吧?''

苏樱笑道:``无论如何,他现在总算瞧见了,总算眠福不错。''

小鱼儿瞪着眼瞧了她很久,忽然叹了气,摇头道;``我问你,你为了一个男人要死要活,一这男人却一见了你就头疼,你难道竟一点也不觉得难受麽?''

苏樱嫣然道:``我为什麽要难受?我知道你嘴里虽然在叫头疼,心里却一定欢喜得很,你若一点也不关心我,方才为什麽要跳起来去抱我呢''

小鱼儿冷冷道:``就算是一条狗掉下来,我也会去接它一把的。''

苏樱笑道;``我知道你故意说出这些恶毒刻薄的话,故意作出这种冷酷凶毒的模样来,只不过是心里害怕而已,所以我绝不会生气的。''

小鱼儿瞪眠道:``我害怕我怕什麽''

苏樱悠然道:``你生怕我以後会压倒你,更怕自己以後会爱我爱得发疯,所以就故意作出这种样子来保护自己,只因为你拚命想叫别人认为你是个无情无义的人,但你若真的无情无义,也就不会这麽样做了。''

小鱼儿跳起来道:``放屁放屁,简直是放屁。''

苏樱笑道:``一个人若被人说破心事,总难免会生气的,你虽骂我,我也不怪你。''

小鱼儿瞪眼瞧着她,又瞧了半晌,喃喃道:``老天呀,老天呀``你怎麽让我遇见这样的女人。''他嘴里说着话,忽然一个斗跳入水里,打着自己的头道:``完蛋了,完蛋了,我简直完蛋了,一个男人若遇见如此自作多情的女人,他只有剃光了头做和尚去。''

苏栖笑道:``那麽这世上就又要多了个酒肉和尚,和一个酒肉尼姑了。''

小鱼儿也不禁怔了怔,道:``酒肉尼姑?''

苏樱道:``你做了和尚,我自然只有去做尼姑,我做了尼姑,自然一定是酒肉尼姑,难道只许有酒肉和尚,就不许有酒肉尼姑麽?''小鱼儿叫吟一声,连头都钻到水里去。

胡药师瞧得几乎笑破肚子,暗道:``这小鱼儿平时说话简直可以将人气死,不想今日也遇着克星了,这位苏姑娘可真是聪明绝顶,早已算准一个女人若想要小鱼儿这样的男人对她服贴,只有用这种以毒攻毒的法子。''

只见小鱼儿头埋在水里,到现在还不肯露出来,他似乎宁可被闷死,也不愿被苏樱气死。

苏樱也不理他,却问胡药师道:``你现在总该已看出来,他是喜欢我的吧。''

胡药师只有含含糊糊``嗯''了一声。

苏樱笑道:``你想,他若不喜欢我,又怎麽将头藏在我的洗脚水里,也不嫌臭呢''

话末说完,小鱼儿已一根箭似的从水里窜了出来。

此刻水已越涨越高,只有这边一块石头还露在水面上,苏樱就坐在这石头中间,小鱼儿若不坐到她身旁,只有再跳下水去。

小鱼儿只有坐到她身旁,苏樱笑着问道:``你不是天下第一聪明人麽?又怎会上了江玉郎的当呢?''

小鱼儿道:``我高兴,我就喜欢上他的当,你管得着麽?''

苏樱柔声道:``我知道你绝不会上他的当,你只不过是故意逗着他玩的,是麽''

她的确聪明得很,知道自己现在已将小鱼儿气够了,若再不适可而止,只怕小鱼儿就要真的恼羞成怒,那就反而弄巧成拙了,是以语锋一变,忽然变得说不出的温柔。

小鱼儿冷冷道:``你用不着拍我马屁,这次我的确是上了他的当,一个人偶而上一次当,也算不了什麽。''

苏樱知道他火气已渐渐平了,但现在最好还是不要惹他,她不等小鱼儿说话,就转向胡药师道:``这件事你一定知道的,你告诉我吧。''

胡药师咳嗽一声,道:``这件事要从花无缺说起,他\ldots\ldots{}''

他说到``女儿红''时,苏樱忍不住失声道:``他难道真将那棵``女儿红吃了下去?卜胡药师叹道:``真吃了下去,就因为他吃了这毒草,所以才认为江玉郎不会再害他,所以才会被推下这里。''

苏樱道:``原来他这只不过是为了救花无缺,才愿这麽样做的,一个人能为了救朋友而牺牲自己,宜在是了不起,了不起''

她说着说着,身子忽然发起抖来,终於嘶声道:``但你难道就没有想到,花无缺也许早已自己走了,江玉郎只不过是在以谎话来要胁你。''

小鱼儿道:``我自然想到了。''

苏栖顶声道:``但你可知道这``女儿红的毒性若是发作起来简直此死还难受。''

小鱼儿瞧见她着急,就再也不生气了,笑嘻嘻道:``我日子过得买在太开心了,有人能让我难受难受,倒也不错。''

苏樱瞪大了眼睛瞧着他,道;``你\ldots 你难道一点也不着急?''

小鱼儿笑道:``已经有你在替我着急了,我自己何必再着急呢?''

苏樱怔了半晌,叹道:``人人都算准你要上当时,你偏偏不上当,人人都想不到你会上当时你反而上当了,我有时实在猜不透你这人究竟在打什麽主意?''

小鱼儿跷起了腿,大笑道:``我打的主意,就是要别人都猜不透我,一个人做的事若都已在别人意料之中,他活着岂非也和死了差不多。''

苏樱苦笑道;``不错,你死的时候,一定有很多人会大吃一鹫的,只可惜那时你自己已瞧不见了。''

小鱼儿笑嘻嘻道;``那倒不见得,说不定那时我正在棺材里偷看哩。''

苏樱跳下去时,铁萍姑也晕了过去。

这几天来,她吃的苦买在太多,身子实在衰弱不堪,再也受不了任何刺激。

晕晕迷迷中,她彷佛听到那山洞里有人语声传出来,但她也不能确定,她对自己已无信心。

她想起了在移花宫中,那一连串平淡的岁月,那时她虽然认为日子过得太空虚,太寂寞,但现在\ldots\ldots 现在她就算想再过一天那样的子,也求之不得了。

她又想起了和小鱼儿在那山洞里所度过的两天,在那黑暗的山洞里,没有食物,没有水,甚至连希望都没有。她的肉体虽在忍受着非人所能忍受的折磨,精神却是愉快的,只要小鱼儿握住她的手,任何痛苦都像是变成了甜蜜。

当然,她也想起了江王郎。江王郎虽然可恶,虽然可恨,但却也有可爱的时候,尤其令人忘不了的,就是他那温柔的抚摸,轻柔的蜜语。

有了这麽多爱和恨纠纽在心头,想死又怎会容易?铁萍姑满面泪痕,连这麽大的风都吹不乾了。她遥望着苏樱方才跳下去的洞窟,凄然道:``为什麽她能死得那麽容易,而我就不能呢?我为什麽不能有她那样的决心?她不是此我有更多理由活下去?''

铁萍姑伸出舌头,用力咬了下去。

铁萍姑没有死,却忽然晕了过去,等她醒过来时她第一眼就瞧见了那狰狞可的青面具。

邀月宫主也正在冷冷地瞧着她,那冷漠的目光,实在比那狰狞的面具更可怕,但最怕的,还是她说的话。只听邀月宫主道:``你那男人已走了麽?''

奴萍姑垂首道:``是。''

邀月宫主道:``但他却没有救你。''

一这两句话又在像两枝箭,刺穿了铁萍姑的心,她虽然永远也不想再提起这件事,却不敢不回答。她只有强忍住眼泪道:``他\ldots\ldots 他不敢救我。''

邀月宫主冷笑道:``他既然敢逃走,为什麽不敢救你?''

铁萍姑终於忍不住又流下泪来。

邀月宫主道:``你用不着流泪,这是你自作自受,你早该知道男人没有一个好东西为什麽还要上他们的当?''

铁萍姑忽然大声道:``男人也并非没有好的,有的人做事虽然古怪,但心地却善良得很。''

邀月宫主道:``你说的是谁?''

铁萍姑道:``我说的就是江小鱼。''

邀月宫主冷漠的目光忽然像火一般燃烧起来,反手一掌掴在她脸上,嘶声道:``你可知道姓江的没有一个是好东西,江小鱼更和他不要脸的爹娘一样。''

铁萍姑道:``我只知道他又善页,又可爱\ldots\ldots{}''

邀月宫主怒喝道:``你再说他一个字,我就立刻杀了你。''

铁萍姑道:``你可以封住我的嘴,不让我说话,但却没法子让我不想他,他现在已死了,你若杀了我,我反而立刻就可以去会见他,这也是你阻拦不住的。''

邀月宫主身子忽然剧烈地头抖起来,只因她又想了江枫和花月奴临死的情况,花月奴临死前说的话,正也好像铁萍姑现在说的一样。她却不知道铁萍姑说这些话,只不过是为了要激怒於她,铁萍姑自然知道移花宫对叛徒的处置多麽残酷,自从花月奴的事件发生後,邀月宫主的心肠已变得比任何人都残酷毒辣。铁萍姑现在所求的,只不过是速死而已。更令邀月宫主愤怒的是,小鱼儿竟已死在别人手里,她十多年来所费的心血竟完全白费了。只因这二十年来,花月奴临死前所说的话,江枫临死的表情,仍都像烈火般鲜明,时时刻刻都在燃烧着她的魂。

一这痛苦简直已将令她发疯了,她还是拚命忍受着,只因她知道总有一天,江枫的两个儿子会落人她一手造成的悲惨命运。

她幻想堵花无缺亲手杀死小鱼儿後的情况,她也不知想过多少次,只有在想着这件事时,她的痛苦才会减轻。但现在,小鱼儿竟已死在别人手里?

铁萍姑虽然瞧不见她的脸色,但从来也没有见过一个人的目光竟会变得如此可怕,只见她竟似再也站不住了,斜斜地倚在树干上,过了半晌,目中竟似泛起了泪光,铁萍姑连做梦也没有想到过。她为的是什麽?

又过了半晌,只听邀月宫主缓缓道:``小鱼儿真的死了麽?''铁萍姑点了点头。

她遥望着远处的目光忽然向铁萍姑瞧了过来,铁萍姑竟忍不住机伶伶打了个寒噤,道:``但\ldots\ldots 但杀死他的人,并不是我。''

邀月宫主道:``不错,你并没有杀他,但若不是你将他带走,他又怎会死在别人手里。''

铁萍姑声道:``我知道我错了,你杀了我吧。''

邀月宫主一字字道;``我要你也忍受二十年的痛苦,从今以後,每天我都会很小心地将你身上的肉割下一片来,现在我就要先挖出你的眼睛,让你什麽也瞧不见,先割下你半截舌头,叫你什麽也说不出。''

铁萍姑自然知道这不是吓人的,移花宫主若要人受二十年的罪,那就绝不会少一天。

就在这时,突听山谷间窖起了一片大笑声!

\hypertarget{ux7b2cux4e5dux5341ux4e5dux7ae0-ux6c34ux843dux77f3ux51fa}{%
\chapter{第九十九章
水落石出}\label{ux7b2cux4e5dux5341ux4e5dux7ae0-ux6c34ux843dux77f3ux51fa}}

``想不到小鱼儿竟有这麽大的本事,他死了後,竟连移花宫主都会为他伤心。''

笑声自四面八力一齐吕起,就连邀月宫主都辨不出他的人在那里。

但她的神情反而立刻镇定下来,沉声道;``是什麽人敢在此胡言乱语?''

那人却仍大笑道:``你连我的声音都听不出了麽?你莫非已忘记了,我在大使时,你还在门闻过我的臭气哩!''

邀月宫主身子一震,道:``你就是小鱼儿?你没有死?你在那里?''

小鱼儿笑道:``我就在你面前,你都瞧不见我麽?''

邀月宫主目光一转,道:``你可是在这山腹中?''

小鱼儿道:``我就是出不来,所以才只好在这里等你来救我,我算准了你一定会救我的,是麽?''

邀月宫主又深深呼吸了两次,道:``不错,我一定会将你救出来的。''

小鱼儿道:``但你若不立刻放了铁萍姑,我就情愿死在这里。''

邀月宫主怔了怔,怒道:``你敢?''

小鱼儿道:``我为什麽不敢?我现在想活就活,想死就死,移花宫主就算有通天的本事,可也拿我没法子,是麽?''

邀月宫主又被气得发起抖来。

小鱼儿道:``现在,我和花无缺的约会已经到时候了,你总不愿意我就这样死了吧?''

邀月宫主跺了跺脚,道:``好,我放了她,绝不伤她毫发就是''

小鱼儿道;``我死了之後,你再杀她我也没法子,但我活着的时候,总要瞧着她也舒舒服服地活着才能放心。''

邀月宫主怒道:``你究竟要怎样''

小鱼儿道:``这山洞虽深,但下面都是水,无论谁跳下来,都绝不会摔死。''

他话还末说完,邀月宫主已提起铁萍姑抛了出去。

她随手一抛,竟已将铁萍姑的身子抛出十馀丈,不偏不倚,抛入那洞窟,看来竟比童子抛球还容易。

过了半晌,只听``噗通''一声。

又听得小鱼儿大笑道:``妙极妙极,想不到不可一世的移花宫主,竟是个呆子,你现在己将她交给了我,我更用不着听你的话了,是麽?''

邀月宫主又鹫又怒,竟气得说不出话来。

小鱼儿道:``现在花无缺又不在这里,我就算出来了,又有什麽用?你见到我就生气,我瞧见你也不舒服,倒不如在这里还落得个眼不见为净。''

邀月宫主道:``但叁月之期已经到了。''

小鱼儿道:``不错,约会的时候到了,所以你快去将花无缺找来吧,我在这里等你。''

邀月宫主道:``你在这里等?''

小鱼儿道:``这山洞就像是个大酒子,就是你掉下来,也休想逃得出去的,你还有什麽不放心麽?''

他大笑着接道:``何况,就算你不放心也没法子,现在只有我才是当家的,我若不想出去,就算十个移花宫主,也没怯子请我出去的。''

移花宫士竟真的无法可施,过了半晌,道:``花无缺是不是也已到了这里?''

小鱼儿笑道:``不错,他已到了这里,只不过这山上的老鼠洞很多,你一时片刻也未必找得着他,若是找的时候太久,我只怕就要被饿死了,所以,你最好还是先弄些东西给我吃,我的味,你是知道的,是麽?''

邀月宫主道:``不错,我是知道的。''

她声音都气得变了,忽然一掌拍出,只听``喀嚓''一声,那株合围巨树,已被她一掌拍断。

山腹里的水,涨得更高了,露出水面的石头,已比一张圆桌大不了多少,小鱼儿胡药师苏樱和铁萍姑,四个人只好都挤在这块石头上。

外面的树被邀月宫主拍断,小鱼儿笑得更开心,但除了他之外,每个人都是心事重重,谁也笑不出来。

铁萍姑瞟了小鱼儿一眼呐呐对苏樱道:``我\ldots\ldots 我说我对他\ldots,:对他很好,那只不过是故意气移花宫主的,其实我\ldots\ldots{}''

苏樱大笑道:``你用不着再解释了,我又不是醋婷?何况对小鱼儿好的人又不止你一个,你就算对他好也没关系。''

她嘴里虽然说``没关系'',但话里酸味,谁都可以嗅得出来,小鱼儿眨了眨眼睛也大笑道;``你对我好,我对你也不错呀,若不是为了你,我现在多多少少也可以听出一些有关移花宫主的秘密了。''

铁萍姑脸红得连头也不敢抬起。

苏樱又觉得有些不忍了,打着岔道:``移花宫主又有什麽秘密?''

小鱼儿道;``我想知道她和我们家究竟有什麽仇恨,她既然将姓江的恨之入骨,为什麽又偏偏不肯自己动手,而且还要扮成什麽见鬼的``铜先生』,逼着要花无缺来杀我,她不但骗了我,而且对她自己的徒弟也鬼鬼祟祟的,到现在为止,花无缺只怕还不知道铜先生就是他的师傅。''

苏樱想了想,苦笑道:``这些事的确奇怪,而且简直毫无道理。''

小鱼儿叹了口气,道:``这其中的道理,也许只有她们姐妹两人自己知道,但看来我只要活着,她们是绝不会说出来。''

苏樱微笑道;``也许你就是要移花宫主认为你已经死了,所以才竦蔓让江玉郎将你推下来,也许你自己知道这洞里都是水,是跌不死的。''

小鱼儿道:``我怎会知道洞里都是水?''

苏樱笑道:``那时太阳还末下山,也许正好有一线日光照进来,反映出下面的水光。''

小鱼儿笑道:``就算是这样,但我总也该知道,这麽深的洞,一掉下来就出不去了的。''

``你自然有法子的,而且法子远不止一个。''苏樱抿嘴一笑,又道:``外面说话的声音,洞里既然听得很清楚,外面有什麽人走过,你一定也知道的,那麽,你又不是哑巴,为什麽不能叫人救你。''

胡药师怔了怔,道;``但,,:但那时候他并不知道这山洞是可以传声的。''

苏樱道:``你也许不知道,但他从小在山谷中长大的,对这件事自然知道得很清楚。''

胡药师叹道:``如此说来,在下实在是孤陋寡闻得很了。''

苏樱道:``但这法子却有个漏洞。这里山势荒僻,万一没有人走过,他岂非就要被困死在这里,万一走过的不是他的朋友,而且是他的仇人,他又怎敢呼救。''

胡药师摸着头道:``是呀,万一没有人走过,万一走过的都是他仇人,那又怎麽办呢?''

苏樱道;``所以他还有第二个法子。''

苏樱又道;``你莫忘了,这座山就在长江口,这山腹里的水,就是江水,江水有潮汐涨落,潮涨的时候,这里的水也跟着涨,潮落的时候,这里的水也跟着退了。''

胡药师瞪着眼呆了半晌,苦笑道:``不错,这道理在下本来也该,也能想得出的。''

苏樱道:``江水既然能流到这里来,那麽这地方必定就有个出口直通长江,只要等到潮水退下去的时候,就可以找到这出口\ldots\ldots{}''

她仿微一笑,这才转过头向小鱼儿一笑,道:``我说的法子对不对呀?''

小鱼儿冷冷道;``你以为你很聪明麽?真正聪明的女人都知道,她无论和那个男人说话时,憧得的事都该比那男人少一些,你的毛病就是懂的买在太多了,这麽样的女人,大多数男人都不敢领教。''

苏樱嫣然道:``但你却并不是大多数男人,像你这样的人,天下只有一个\ldots\ldots 何况,这些道理你也知道的,我懂的还是比你少一些。''

小鱼儿忍不住大笑起来,笑了半晌,又叹了口气,喃喃道:``如此看来,我迟早总有一天要被这丫头迷上的。''

就在这时,忽然间又有样东西从上面直落了下来,胡药师和铁萍姑都吃了一鹫,小鱼儿却微笑道:``移花宫主,果然听话,已将咱们的晚饭送来了。''

邀月宫主送来的东西可真不少,满满地塞了一大包,小鱼儿一面吃着,一面已发觉山腹中的水在开始往下退了。

水还没有退完,胡药师已跳了下去,四面寻找箸出,小鱼儿却往石头上一躺,竟真的呼呼大睡起来。

苏樱轻轻摸着他漆黑的头发,幽幽道:``他宾在累了,这几天来,他吃的苦实在不少。''

他回头向铁萍姑一笑,道;``若是换了别人,吃了他这麽多苦,受了他这麽多打击,纵然不意志消沉,也一定会怨天尤人的,但是你看他,他竟像是一点也不放在心上。这样的男人,你又怎麽能怪我喜欢他。''

铁萍姑笑了笑,眼泪却已快流了出来,苏樱可以为自爱的男人而骄傲,但是她呢?她的男人带给她的,却只有羞侮和不幸。

过了半晌,苏樱忽又问道:``你认不认得铁心兰''

铁萍姑道:``我知道她也对小鱼儿很好,可是\ldots\ldots{}''

苏樱抢着道:``可是他除了小鱼儿外,还能喜欢别人但除了小鱼儿外,却再也不会爱上任何人了,所以我绝不能让她将小鱼儿抢走,无论用什麽子我也要\ldots\ldots{}''

就在这时,突听胡药师大呼道:``在这里,就在这里我到了!''

一这山中果然有条直通长江的出口,看来虽是条很曲折崎岖的地道,但一个不太胖的人还是可以爬过去的。

苏樱摇醒了小鱼儿,笑道:``你要睡,出去後再好生睡,现在咱们已经可以走了。''小鱼儿道:``我为什麽要走你难道没有听见我要在这里等花无缺麽?''

苏樱失声道:``你\ldots;你真的要等他''

小鱼儿瞪眼道:``当然是真的,这约会叁个月以前就约好了。''

苏樱道:``但\ldots\ldots 但他来了之後,移花宫主一定会逼着他跟你打架的。''

小鱼儿笑道:``打架这两个字用得不妥,像咱们这样高手相争,应该说是比武才对。''

苏樱着急道:``但你们并不是比武,你们是要拚命呀。''

苏樱又将他身子扳了过来,跺脚道:``但你\ldots\ldots 你现在还不是他的对手,因为我知道那``移花接玉功之神奇,宜在是天下第一,,,:''

小鱼儿忽然一笑,悠悠道:``但你可知道,普天之下,只有我一个人知道破解移花宫武功的招式。''

苏樱怔一怔,失声道;``你真的知道\ldots\ldots 你怎麽会知道。''

小鱼儿笑嘻嘻道:``自然是有人教给我的,``移花宫''武功的秘密,天下再也没有别人知道得此他更清楚了。''

``移花宫主又怎麽将破解她自己武功的招式教给你?她难道疯了麽?''苏樱怔了半晌又道;``但就算你能破解``移花宫的武功,你也绝不会杀了花无缺的,是麽?''

小鱼儿道:``我杀不杀他,和你又有什麽关系?''

苏樱道;``当然有关系,你不杀他,他就要杀你,你留在这里,就是\ldots\ldots{}''

小鱼儿忽然跳起来,大孔道:``你们谁高兴走,谁就走,反正我是在这里等定了!''

胡药师本来兴高采烈地站在那边出口旁,只等着出了这山洞,解药就可到手,听了小鱼儿这句话,只觉两腿发软,连站郡站不住了,手扶着山壁,呆望着小鱼儿不停喘着气忽然嘶声道;``在\ldots,:在下有些不\ldots\ldots 不对了。毒\ldots\ldots 毒性只怕已发作。''

苏樱道:``是他下的毒麽?''胡药师拚命点头。

苏樱眼珠子一转,道:``那毒药是什麽味道?''

胡药师苦着脸道:``咸咸的,湿湿的,还有些\ldots\ldots 有些臭气。''

苏樱忽然笑了道:``他只不过是故意吓吓你的,那一定不是毒药,你方才觉得毒已发怍,只怕你自己心里在作怪。''

胡药师怔了怔,道:``不是毒药是什麽?''

苏樱笑道:``我也不知道是什麽,说不定就是他脚上搓下来的泥丸子。''

胡药师脸上阵红阵白,突然转过身,像只被人踢了一脚的野狗似的,一头钻了出去,飞也似的逃了。

他只望这辈子再也莫要见着小鱼儿,他宁可遇着一百个大头鬼,也不想再遇到小鱼儿了。

苏樱的眼睛移到铁萍姑身上,道:``你也不想走麽?''

铁萍姑垂下头,不知该说什麽。

但她若走,又买在不知道该走到那里去,天地虽大却好像没有她这麽样一个人的容身之地。

苏樱道;``你难道不想再见江玉郎''

铁萍姑道:``我\ldots\ldots{}''

她本来以为自己一定可以断然说出:``我绝不再见他!''但也不知怎地,话到嘴边,她竟说不出了。

苏樱像是已看透她的心,微笑道:``我知道你一定想再见到他,因为你就算不再会喜欢他,难道你还会不想报复麽?''

铁萍姑叹了口气,道:``可是我却不知道该如何报复。''这句话她本来不想说的,但不知怎地,竟说了出来。

苏樱道;``你可知道你现在为什麽会难受,那只因为你觉得他对不起你,他抛弃了你,你觉得他根本末将你放在心上,所以你的心才会碎,是麽?''

铁萍姑黯然无语,因为苏樱的话,买已说到她心里去了。

苏褂道:``你若想报复,就要让他难受,让他觉得是你抛弃了他,让他觉得你根本就末将他放在心上,到了那时,他就会像条狗似的来求你了。''

铁萍姑垂着头想了许久,眼睛渐渐发了光。

苏樱道;``现在你懂得我的意思了麽?''

\hypertarget{ux7b2cux4e00ux767eux7ae0-ux53ccux9a84ux518dux805a}{%
\chapter{第一百章
双骄再聚}\label{ux7b2cux4e00ux767eux7ae0-ux53ccux9a84ux518dux805a}}

铁萍姑道;``我懂了。''

苏樱一笑道:``很好,只要你照着我的话来做,不怕他不来找你,等他来找你的时候,就是你出气的时候到了。''

铁萍姑也不禁笑了笑,忽又叹道:``但是我\ldots\ldots 我现在,\ldots{}''

苏樱道:``你觉得自己现在孤零零的一个人,身无长物,又没有倚靠,是以心里有些害怕,是麽?''

铁萍姑黯然点了点头。

苏樱笑道:``你莫忘了,你是个很美丽,很动人的女孩子,年纪又轻,这已经是女人最大的财产了,就凭这样,你就可以将世上大多数男人摆在你的手心里,就凭这些,你无论走到那里鄱可以抬起头来的。''

铁萍姑果然抬起头来,微笑道;``谢谢你。''

她瞧了小鱼儿一眼,似乎想说什麽,但却什麽也没有说出来,就走了,头也不回地走了。

小鱼儿怔了怔,大吼道:``你把别人都弄走了,自己为什麽不走?''

苏樱嫣然道;``走我为什麽要走这地方不是很舒服麽''

小鱼儿道:``求求你,你快走吧,我现在一个头已经有别人叁个那麽大了,你若再不走,我说不定马上就要发疯。''

苏樱淡淡道:``你若是看到我就生气,不会自己走麽?''

小鱼儿呆了半晌,反而笑了,大笑道:``好,小丫头,我服了你了,我从生下来到现在,还没有一个人让我这样生气过,我总算遇见了对手。''

苏樱也不理他,却将方才吃剩下来的东西,又仔仔细细地包了起来,嘴里自言自语道:``这地方潮湿得很,东西再放几天,只怕就要发霉了。''

小鱼儿道;``就算发莒了又有什麽关系,你难道还想带出去麽?''

苏樱这才回头一笑,道:``你以为移花宫主立刻就能将花无缺找来麽?''

小鱼儿瞪直眼瞧了半晌,忽然跳到她面前,道:``你知道江王郎是在骗我,那麽你一定见过了花无缺,对不对?''

苏樱在石头上坐了下来,盘起了腿,也瞧了小鱼儿半晌,才悠悠道;,;他,也知道他到什麽地方去了,但是,现在我却不能告诉你。''

小鱼儿叫了起来,道:``你为什麽不能告诉我?''

苏樱道;``因为我怕你生气。''

小鱼儿大声道:``我若生气我就是王八蛋。''

苏樱摇头笑道:``因为你绝不会变成王八蛋的,任何人都不会忽然变成王八蛋,是麽?''

小鱼儿道:``好,我若生气,你叫我干什麽,我就干什麽?''

苏樱嫣然一笑,道:``好,我告诉你,花无缺现在去找铁心兰去了。''

小鱼儿失声,道:``他去找铁心兰去了?他怎会知道铁心兰在那里?''

苏樱道:``我告诉他的。''

小鱼儿这才真的吃了,道:``你告诉他的?你怎会知道铁心兰在那里?怎会认得她的?''

苏樱笑道:``我已经和她结拜为异姓姐,你难道不知道麽?''小鱼儿张大了嘴,再也说不出话来。

苏樱道;``你是不是已有很久没见过铁心兰了?''

小鱼儿道:``嗯。''

``不错,我的确见过了苏樱道;``你可知道,这两个月来,铁心兰一直和花无缺在一齐''

小鱼儿微笑道:``他们能在一倒不错,我本来一直在担心着她,现在可放心了,我知道花无缺一直对她很好的。''

苏樱的眼睛里发了光,却垂下头去,道:``你为何不问我铁心兰现在在那里?''

小鱼儿笑道:``你总不会将她送到那老鼠洞里去吧?''

苏樱道:``她正是在那里。''

小鱼儿脸上的笑容像石头般僵住了,然後,他整个人跳起来有叁丈高,跳到苏樱面前的石头上,大吼道``你这死丫头,你怎麽能将她送到那里去''

苏樱道:``她是我的姐妹,在那地方正安全得很,谁也不会欺负她。''

小鱼儿大怒道:``但花无缺此番去找她,那大老鼠怎会放过花无缺,你,:你这不是在害人麽,我\ldots\ldots 我\ldots\ldots 我\ldots\ldots{}''

他气得连话都说不出来了,一把拧起苏樱的手,吼道:``今天我若不狠狠揍你一顿,实在对不起他们,''

苏樱微笑道:``你说过不生气的,男子汉大丈夫,怎麽能在我这种小丫头面前食言背信。''

小鱼儿怔了怔,又跳起叁丈高。

苏樱柔声道:``其穴你也不用着急,花无缺死不了的,何况,他一心要杀死你,本来就不能算是你的朋友,他若不能来,你岂非也用不着为难了麽?''

小鱼儿用力打着自己的头,高声道;``你以为你这是在帮我的忙?以为他死了我一定很开心?老宜告诉你,他若真被魏无牙害死了,我就\ldots\ldots{}''

突听外面一人大呼道:``小鱼儿,你在那里,你听得到我说话麽?''

一这赫然竟是花无缺的声音。

小鱼儿和苏樱全都怔住了。花无缺竟好生生来了,而且来得这麽快。

小鱼儿大声道:``花无缺,我就在这里。你放条绳子下来,我就可以上去了。''

过了半晌,只见花无缺的头已在上面的洞口伸了出来,面上的神情既是欢喜,又是关切。

小鱼儿更已笑得合不拢嘴来,大笑道:``好小子,两个月没见,我们都没有变。''

花无缺已垂下条长索,笑道:``你在下面我看不见你,你快上来吧。''

苏樱看着这两个人,心里真是奇怪极了,这两人随便怎麽看,也不像是立刻就要拚命的冤家对头。

只见小鱼儿刚窜上绳子,又跳了下来,板着脸道:``姓苏的小丫头,你现在还不想走麽?''

苏樱垂头道:``你一个人走吧,我丁想看见你被人杀死的样子。''

小鱼儿大吼道:``你不想看,我就偏要你看,不想走,我就偏要你走,看你有什麽法子反抗我。''

苏樱身子往後退,道:``你;你敢?''

她脸上虽然装出很生气的样子,其实心里也不知有多麽高兴,因为她知道她的手已渐渐开始能摸到小鱼儿的心了。

花无缺垂手站在邀月宫主身旁,脸上已变得木无表情。

对花无缺说来,邀月宫主不但是他的严师,也是他的养母,他从小就末见到她面上露出过一丝笑容。

他也从不敢在她面前有丝毫放肆之处,因为他心里不但对她很尊敬,很感激,而且也有些畏惧。

现在,小鱼儿终於见到邀月宫主的脸了。

她已除下了那可怕的青铜面具,可是她的脸却比那面具更冷漠,任何人都无法在她脸上看出任何喜怒哀乐的表情。

小鱼儿再也想不到这威镇天下垂叁十年的人,看来竟是如此年轻,更想不到一个如此美丽的人,竟会让人看过一眼便不敢再看。

就连小鱼儿瞧她一眼後,也觉得有一股寒意自脚底直升了上来,彷佛在寒夜中忽然瞧见了一个美丽的幽灵。

他甚至没有注意到铁心兰也在她身旁。

铁心茴却已兴奋得在发抖了,她瞧见小鱼儿自山石上一跃而下,立刻就忍不住向小鱼儿奔了过去。

但只奔出两步,她身子忽然僵硬了,她忽然想起了花无缺,她怎能一见到小鱼儿,就抛下花无缺?

她站在小鱼儿和花无缺中间,也不知是该进,还是该退,她只希望自己根本就没生到这世上来。

这时小鱼儿也瞧见她了,正笑着招呼道:``好久不见,你好麽?''

铁心兰竟完全没有听见他的话,忽然扭转头,垂首奔到那边一株大树下,这棵树也恰巧正在小鱼儿和花无缺中间。

苏樱的眼睛却始终在留意着小鱼儿,她发现小鱼儿虽然还在笑着,但笑容也僵硬得很。再看花无缺,竟也低着头始终末曾抬起。

苏樱不禁在暗中长长叹了口气瞧这叁人间复杂而微妙的关系,她除了叹气外,还能怎样?

邀月宫主比刀更利,比冰更冷的眼睛,也始终瞪着小鱼儿,小鱼儿长长吸了口气,也抬起头瞪着她,微笑道:``你送来的东西都不错,只可惜没有辣椒,下次你若再请我吃饭,可千万不能忘记我喜欢吃辣的。''

邀月宫主脸上并没有什麽表情,花无缺却吃鹫地抬起头来,他实在想不到世上居然有人敢对邀月宫主这样说话。

邀月宫主道;``现在我再给你叁个时辰,你在叁个时辰内,不妨调息运气,养精蓄锐,但却不准离开这里!''

小鱼儿拍手笑道;``移花宫主果然不愧为移花宫主,丝毫不肯占人便宜,知道我累了,就让我先休息休息。''

邀月宫主却已转过身,道:``无缺,你随我来。''

小鱼儿道:``我想和花无缺说两句话,行不行?''

邀月宫主头也不回,冷冷道;``不行!''

小鱼儿大声道;``为什麽不行,你难道怕我告诉他你就是铜先生?''

一这时花无缺也转过身去,也没有回头,但小鱼儿却可以见到他听到了这句话全身都震了一震。小鱼儿笑了,因为他的目的已达到。

只见邀月宫主走到最远的一棵树下,才转回身来,像在和花无缺说话,但花无缺却始终是背对这边的。

苏樱柔声道:``叁个时辰并不长,你还是好生歇歇吧。''

一这时正是清晨,太阳已刚刚升起。

苏樱将四下的落叶都收集起来,铺在树下,拉着小鱼儿坐上去,就好像一个妻子在为丈夫铺床似的。

铁心兰还站在那边树下,泪珠已在眼眶里打转。她忽然觉得自己活在这世界上,竟好像已变成多馀的。

她方才既没有走到小鱼儿这边来,现在更不能走过来了,她方才既没有回到花无缺那边去,现在也更不能回去。

她也知道在这种情况下,小鱼儿和花无缺两个人,都绝不会走到她这边来,移花宫主已用冰凉的手,将这两个人的友情撕成两半,这两人之间若不再有友情,那麽她的处境岂非更悲惨,更难堪。

她知道自己现在最好就是远远的走开,走得越远越好,那麽无论任何事鄱不能伤害到她了。

但现在她生命中最亲近的两个人,立刻就要在这里作生死之决斗,她又怎麽能走?怎麽忍心走呢?

小鱼儿在落叶上躺了下来,闭起了眼睛。

别人有的紧张,有的痛苦,但他却悠悠闲闲地跷起了腿,嘴里还含含糊糊哼着山歌,这些事一竟好像和他没有关系。

苏樱站在他身旁,俯首瞧着他,瞧了半晌,轻轻叹了气,道:``你瞧见铁心兰了麽?''

小鱼儿道;``你没有看见我方才已经和她打过招呼。''

苏樱咬着嘴唇,道:``但是她\ldots\ldots 她实在可怜得很,你实在应该去安慰安慰她。''

小鱼儿霍然张开眼睛,瞪着道;``我为什麽要过去安慰她?她为什麽不能过来?''

苏栖叹道;``她现在的确很为难\ldots\ldots{}''

小鱼儿道:``她为难,我就不为难麽?何况,她为难也是她自己找的,谁叫她站在那边不肯过来?又没有钉子钉住了她的脚。''

苏樱又叹了口气,道;``你既然不肯过去,我就过去吧。''

小鱼儿道:``你会不会唇语?''

苏樱道:``不会。''

小鱼儿叹道:``我现在若能听出移花宫主在对花无缺说什麽,那就好了。''

苏樱道:``你就算听不见,也应该想像得到的,她现在还不是在告诉花无缺,要用什麽法子才能杀你。''

小鱼儿沉默了半晌,缓缓道:``方才我在洞里时,花无缺还和我有说有笑的,但等我出来他一竟不理我了,简直连看都没有看我一眼。''

苏樱道:``你若在移花官长大,你见了移花宫主,也会变得没主意的。''

小鱼儿苦笑道:``这样看来,``恶人谷反而此``移花宫好得多了,恶人谷里的至少还是人,移花宫却只是一群活鬼,一群行走肉。''

苏樱笑了笑,柔声道:``你歇歇吧,我过去说两句话就回来。''

小鱼儿瞪眼道;``你为什麽一定要过去?我现在也不好受,你为什麽不在这里陪着我?''

苏樱眼波流动,嫣然道;``你难道不想知道,她和花无缺两人是如何从那老鼠洞里出来的麽?''

落叶上的泪珠已乾了,但铁心兰的眼泪却还没有乾,她听见苏樱的一双脚在向她走过来,就咬紧牙关,绝不让眼泪再流下来。

苏樱悄悄走到她身旁,她却连头也没有抬起,风,次着她的头发,一片落叶正在她紊乱的发丝里挣扎着,要想飞起。

苏樱轻轻拈起了这片枯叶,悄然道:``你在生我的气?是麽?''

过了很久,铁心兰才缓缓站起来道:``你用不着难过,我若知道你就是我的情敌,我也不会对你说真话的卜''

苏樱长长叹了口气,拉起了她的手,嫣然笑道:``我真没想到你是这样的女孩子,我现在只希望你是个又凶又狠又狡猾的女人,那样我心里就会好受得多了。''

铁心兰瞪着她瞧了半晌,忽然道:``可是无论怎样,你也不会为我放弃小鱼儿的,是麽?''

这句话问得更不聪明,她连自己也不知道怎会问出这句话来。

苏樱也直视着她的眼睛,道:``不错,我不会伪了你放弃他的,只因我若放弃了他,也许反而会令你更为难,是麽?''

铁心兰的头又垂了下来,这句话就像是一根针,直刺入她心里,使得她再也不知道该说什麽。

直到她手里的落叶已被她揉得粉碎,她才黯然道:``我穴在不该对你说那句话的,小鱼儿也许根本就没有将我放在心上,也许只有你才配得上他。''

苏樱道;``小鱼儿并没有忘记你,他若真的末将你放在心上,现在早已走过来了。''

铁心兰怔了怔道;``你\ldots\ldots 你为什麽要告诉我?你为什麽不让我死了这条心?''

苏樱凄然一笑,道:``这也许是因为我太想得到小鱼儿了,所以才不愿让他以後恨我,我要让他自己选择,他喜欢的若是你,我就算杀了你,也没有用的。''

铁心闹头垂得更低,她仔细咀嚼着这几句话的滋味,但觉心里充满了酸苦,只因她的心情已越来越矛盾,越来越复杂,她在暗中间着自己;``小鱼儿选择的若是我,我是否真的会很快乐呢?''

苏樱忽又一笑,道;``你可瞧见了我义父麽?他是不是长得很可怕?''

铁心兰道:``我没有瞧见他。''

\hypertarget{ux7b2cux4e00ux767eux96f6ux4e00ux7ae0-ux610fux5916ux4e4bux53d8}{%
\chapter{第一百零一章
意外之变}\label{ux7b2cux4e00ux767eux96f6ux4e00ux7ae0-ux610fux5916ux4e4bux53d8}}

苏樱讶然道:``你到了那边树林,难道没有人来接你麽?你是不是找错了地方?''

铁心兰叹了口气道:``我没有找错地方,我到了那里,只见到处都有老鼠在窜来窜去,我就吓得立刻躲到树上去,谁知树上竟吊着个死尸,远远瞧过去,还可以瞧见有几具死尸吊在树上,我正不知该如何是好时,花\ldots\ldots 花公子就来了。''

苏樱整个人都怔在那里,手心已出了汗。

铁心兰叹道:``以我看来,那边一定发生了很大的变化,你\ldots\ldots 你最好还是瞧瞧去。''

苏樱不等她话说完,已转身奔出,但奔出几步,又停了下来,无论如何,魏无牙总是她的恩人,魏无牙若是有什麽不幸,她是万万无法置之不理的,但现在\ldots\ldots 现在小鱼儿正在瞧着她,她又怎麽能走呢?

她怔在那里,也不知该如何是好了。

苏樱终於已回到小鱼儿身旁,无论什麽事发生,都不能让她此刻抛下小鱼儿一个人在这里。

小鱼儿笑了笑,道:``看你这样子,移花宫主莫非已杀死了魏无牙麽?''

苏樱还没有回答这句话,风中忽然飘来了一条人影。

她也和邀月宫主同样冷漠,同样美丽,只不过她那双明如秋水的眼睛,还多少有些柔和之一意。

她的身子似乎比落叶更轻,瓢落在花无缺身旁。花无缺立刻拜倒在地。

小鱼儿瞪大了眼睛,道:``这只怕就是那怜星宫主了,简直和她姐姐是一个模子铸出来的,只不过比死人多了口气而已。''

苏樱苦笑道:``但这姐妹两人能令江湖中人连她们的名字都不敢提起,她们若只比死人多口气,江湖中就一定都是死人了。''

小鱼儿大笑道;``你错了,一个人活着,就要会哭会笑会高兴会悲伤,也会害怕,像她们这样的人,活着才没意思。''

他故意直着喉咙大笑,就是想要移花宫主听见。但移花宫主姐妹两人,连瞧也没有往这边瞧一眼。

小鱼儿哈哈笑道:``我将她们当死人,说不定她们也已将我当成死人,所以我无论说什麽,她们都不会生气。''

一这句话他虽笑嘻嘻的说了出来,但听在苏樱耳里,却也不知有多麽辛酸,她几乎流下泪来。

她实在看不出小鱼儿有希望能活下去,他就算能战胜花无缺,就算能杀了花无缺,也得死在移花宫主手里?,小鱼儿道:``你笑一笑麽?只要你笑一笑,我死了也开心。''

苏樱果然笑了,可是她若不笑也许还能忍得住不流泪,现在一笑起来,眼泪也随着流下。

一阵风卷起落叶,怜星宫主忽然到了小鱼儿面前,冷冷道:``时候已快到了,你知道吗?''

小鱼儿道:``我倒希望时候快些到,否则我只怕要被眼泪淹死了。''

小鱼儿眼珠子一转,又笑道;``我倒有一句话想问问你?''

怜星宫主道:``什麽话?''

小鱼儿道:``像你这样漂亮的女人,为什麽直到现在还没有嫁人呢?难道这麽多年来,竟没有一个男人爱上你麽?''

怜星宫主霍然转过身,小鱼儿可以瞧见她脖子後面的两根筋都已颤抖起来,满头青丝,也忽然在西风中飞舞而起。

过了半晌,只听她一字字道:``站起来!''

小鱼儿这次倒听话得很,立刻跳了起来道:``现在就要动手了麽?''

只见那边树下的花无缺,也缓缓转过身来。

苏樱忽然抓住小鱼儿的手,道:``你\ldots\ldots 你难道没有什麽话要对我说''

小鱼儿道;``没有。''苏樱手指一根根松开,倒退两步,泪珠已夺眶而出。

怜星宫主道;``花无缺,江小鱼,你们两人都听着,从现在开始,你们两人都向前走十五步,走到第十五步时,便可出手,这一战无论你两人谁胜谁负,都绝不许有第三人从旁相助,无论谁敢来多事,立取其命,绝不宽恕。''

苏樱忍不住大声道;``你也不出手相助麽?''

怜星宫主还末说话,邀月宫主已冷冷道:``她若敢多事,我也要她的命''

苏樱道:``那麽你自己若出手了呢?''

邀月宫主道:``我就自己要自己的命''

苏樱擦了擦眼泪,大声道:``小鱼儿,你听见了麽?移花宫主话出如风,想必不会食言,求求你无论如何也莫要败给他好麽?''

她却不知道今日一战,战败者固然只有死,战胜者的命运却比死还要悲惨,小鱼儿若能死在花无缺手下,那就比花无缺幸运得多了。

天色阴暝,乌云已越来越重,枝头虽还有几片枯叶在与西风相抗,但那也只不过是垂死的挣扎而已。

小鱼儿已开始往前走。花无缺也开始缓缓移动了脚步。

邀月怜星苏樱铁心兰,四双眼睛,都在眨也不眨地瞪着小鱼儿和花无缺的脚步。

这四人的心事虽然不同,但却都同样的紧张。

铁心兰知道片刻之间,这两人就有一个要倒下去,她也不知道自己究竟希望倒下去的是谁。

在她心底深处,她也知道这两人若有一个倒下去,那麽她就不会再有矛盾,不必再作抉择,事情也就会变得简单得多。

她甚至拒绝承认自己有这种想法,只因这想法实在太自私太卑鄙太无情太狠毒\ldots\ldots 苏樱的心里倒只有痛苦,并没有矛盾,因为她已决定小鱼儿若死了,她绝不单独活下去。

她虽然知道小鱼儿获胜的机会并不大,但她还是希望有奇迹出现,希望小鱼儿能将花无缺打倒。而怜星和邀月两人呢?现在她们的计划已将实现,她们的忍耐也总算有了收获,她们心里的仇恨,也眼见就能得到报复。

她们只有幻想着这两人倒下一个时,才能将这痛苦减轻,只因唯有等到那时候,她们才能将一这惊人的秘密说出来,这秘密已像条沉重的铁链般将她们的心灵禁锢了二十年,她们唯有等到将一这秘密说出来之後,才能自由自在,否则她们就永远要做这秘密的奴隶。

而现在,她们还是只有等待。

谁知小鱼儿刚走了三步,忽然回头向苏樱一笑,道:``对了,我刚想起有句话要告诉你。''

苏樱心头一阵激动,热泪又将夺眶而出无论如何,小鱼儿对她总算和对别人有些不可。

她忍住泪道;``你\ldots\ldots 你说吧,我在听着。''

小鱼兄道:``我劝你还是乘着年轻时快嫁人吧,否则越老越嫁不出去,到了五十岁时,就也会变成和她们一样的老妖怪了。''

这竟是小鱼儿临死前所要说的最後一句话。到了此时此刻,他竟然还能说得出这种话来。

苏樱只觉一颗心已像是手帕般绞住了,过了半晌,咬紧牙颤声道:``你放心,我绝不会等那麽久。''

他轻描淡写一句话,就将苏樱的心绞碎了,更令怜星和邀月两人气得全身发抖,面无血色。

但他自已却像是根本没有说这句话似的。

最奇妙的是,到了这时,每个人心里竟还是希望他能打倒花无缺,苏樱固然一心想他得胜,铁心兰也不忍见到他被击倒时的样子。

也不知为了什麽,她总是认为花无缺比较坚强些,所以也就不妨多忍受些痛苦,所以她宁可伤害花无缺,也不忍伤害小鱼儿。

更奇妙的是,就连邀月和怜星两人竟也希望小鱼儿得!她们自己也许不会承认,但却是事实。只因花无缺若打倒了小鱼儿,那麽她们就要在花无缺面前说出这秘密,她们养育花无缺虽是为了复仇,但这许多年以来,她们还难免对这自己见着长大的孩子多多少少生出些感情。

她们还是在暗中数着小鱼儿的脚步!``十一,十二,十三\ldots\ldots{}''

邀月宫主嘴角不禁泛起一丝残酷的微笑。

现在,小鱼儿和花无缺已迈出第十四步了。

小鱼儿的眼睛一直在瞪着花无缺,花无缺面上虽全无任何表情,但目光却一直在回避着他。

无论他们走得多麽慢,这第十五步终於还是要迈出去的,怜星和邀月宫主情不自禁,都紧握起手掌。

但铁心兰和苏樱却连手都握不紧了,她们的手抖得是这麽厉害,抖得就像是西风的枯叶。

就在这时,小鱼儿忽然倒了下去!

在如此紧张,紧张得令人窒息的一刹那中,小鱼儿竟莫名其妙,无缘无故的忽然倒了下去。

花无缺整个人都怔住了,铁心兰也怔住了,苏樱更怔住了,他们全身上下本已紧张得充满了血,现在,全身的血又像是一下子忽然被抽干,脑子也忽然变得茫茫然一片真空,竟没有人知道该如何处理这突然发生的变化。

就连邀月和怜星宫主都怔住了,脸上神色也为之大变。

只见小鱼儿身子倒在地上後,就忽然发起抖来,越抖越厉害,到後来身子竟渐渐缩成一团。

怜星宫主跺了跺脚,道:``你这究竟是怎麽回事?''

邀月宫主怒道;``他这是在装死,杀了他,快杀了他。''

花无缺垂首道:``他已无还手之力,弟子怎能出手?''

邀月宫主道:``他既不敢跟你动手,就是认输了,你为何不能杀他?''

花无缺垂着头,既不出手,也不说话。

只听邀月宫主厉声又道:``你为何还不出手,难道他每次一装死,你就要放过他``你难道忘了本门的规矩,你难道连我的话都敢不听?''

花无缺满头汗珠滚滚而落,垂首瞧着小鱼儿,颤声道:``你为何不肯站起来和我一拚?你难道定要逼我在如此情况下杀你?''

小鱼儿忽然咧嘴一笑,道;``你赶紧杀了我吧,我绝不怪你的,因为这并不能算是你杀死了我,杀死我的人是江玉郎。''

邀月宫主变色道:``你这话是什麽意思?''

小鱼儿叹了口气,道:``因为我若没有中毒,现在就不会无力出手,也就未必会死,所以现在我就算死了,你也不必觉得抱歉,因为我根本就不是死在你手上的。''

他眠睛忽然瞪着邀月宫主,一字字道:``江玉郎才是真正杀死我的人。''

邀月宫主和怜星宫主两人对望了一眼,又不禁怔住了。

过了半晌,怜星宫主才厉声问道:``你中了他什麽毒?''

小鱼儿道:``女儿红。''

怜星宫主长长吸了气,瞧着邀月宫主沉声道:``看他这样子,倒的确是女儿红毒发时的迹象。''

邀月宫主脸上已不见一丝血色,过了半晌,忽然冷笑道:``此人诡计多端,你怎可听信他的话。''

小鱼儿道:``信不信由你,好在我中毒时,有很多人都在旁边瞧见的。''

邀月宫主立刻问道:``是些什麽人?''

小鱼儿道;``有铁萍姑,和一个叫胡药师的人,自然还有下毒的江玉郎。''

怜星和邀月又对望了一眼,两人忽然同时掠出,一阵风吹过,两人都已在十馀丈外的树下。

邀月宫主和怜星官主同时掠到树下。

怜星宫主道:``你的意思怎样?''邀月宫主嘴唇都发了白,闭着嘴不说话。

怜星宫主道:``这江小鱼若真的已中了江玉郎的毒,那麽的确不该算是死在花无缺手上,这麽一来,我们的计划岂非就变得毫无意义?''

邀月宫主颤声道:``我\ldots\ldots 我已忍受了二十年的痛苦;''

怜星宫主的目光也随着她的手缓缓垂落,道:``你忍受了二十年的痛苦,这二十年来我难道很快活''

过了半晌她又接着道;``但我们这二十年的罪绝不是白受,因为普天之下,只有我们两人知道这秘密,只有我们两人才知道他们本是兄弟,我们自己若不将这秘密说出去,他们两人到死也不会知道。''

邀月宫主脸色也渐渐和缓,道:``不错,他们永远也不会知道。''

怜星宫主道:``所以他们迟早必有一天,会互相残杀而死的,他们的命运已注定了如此,除了我们两人之外,谁也不能将之改变。''

她一字字接着道:``而我们两人却是绝不会令它改变的,是麽?''

邀月宫主道:``不错。''

怜星宫主道:``所以我们现在根本不必着急,我们等着虽然难受,但他们这样又何尝不痛苦?我们正好瞧着他们为自己的命运挣扎,就好像一只猫瞧着在它爪下挣扎的老鼠一样,何况,我们既已等了二十年,再多等三两个月又有何妨?''

邀月宫主冷冷道:``我知道你的意思,你要先解了江小鱼所中的毒,再令花无缺杀他,你要他完完全全死在无缺手上,是麽?''

怜星宫主目中闪动着欣慰的笑意,柔声道:``不错,因为只有这样,才能令无缺痛苦悔恨,觉得生不如死,你若令他现在就杀了江小鱼,他就会自己宽恕自己,甚至会去杀了江玉郎为小鱼儿报仇,那麽我们的计划也就变得毫无意义。''

邀月宫主默然半晌,道;``但你可知道江小鱼是否真的中了毒呢?''

怜星宫主道``这一点我们立刻就能查出来的。''

小鱼儿仍倒在地上,抖着,铁心兰、苏樱和花无缺却并没有在看望他,他们的眼睛,都眨也不眨的瞪着移花宫主。

只可惜他们非但什麽都看不出,而且连一个字也听不到,他们只能瞧见邀月宫主冷冰冰的一张脸上,充满了怨毒,充满了杀气,他们越瞧越是心惊,三个人掌心不觉都为小鱼儿捏着一把冷汗。

也不知过了多久,才看见移花宫主姐妹两人缓缓走了回来,花无缺想迎上去,但脚步力动,又停了下来。

只见邀月宫主走到小鱼儿面前,沈声道:``你中毒时,铁萍姑也看到的,是麽?''

小鱼儿道:``嗯!''

邀月宫主道:``好,你叫她出来,我问问她。''

小鱼儿咧嘴一笑,道:``你以为那山腹中只有这一条山路麽?''

邀月宫主冷笑道:``若有别的出路,你为何不走?''

小鱼儿也冷笑着道;``我不走,只因我不愿对花无缺失约,但铁萍姑却早已走了,你若是不信,为何不自已下去瞧瞧。''

他话还没有说完,邀月宫主的身形已飞云般掠上山崖,方才花无缺垂下去的那条绳子还未解下。

邀月宫主游鱼般滑入那洞穴,过了片刻,又轻风般掠了出来,面上的神色,似乎觉得有些意外。

小鱼儿笑道:``你现在可相信了麽?''

邀月宫主道:``哼。''

小鱼儿道:``那麽你就也该知道,我若不愿和花无缺动手,方才就也早已和铁萍姑一齐走了,用不着等到现在才来装死。''

邀月宫主沉默了半晌,道:``那麽你可知道江玉郎现在在那里?''

小鱼儿道;``我当然知道,只怕我说出那地方,你也不敢去找他。''

小鱼儿却偏偏还要再激她一句,冷冷又道:``也许只有这地方是你不敢去的,因为我还没见过不怕老鼠的女人。''

邀月宫主目光一闪,道:``你说的莫非是魏无牙?他也在这山上?''

小鱼儿冷笑道;``他当然在这山上,你是真不知道?还是假不知道。''

只见邀月宫主神情仍然毫无变化,小鱼儿虽然故意想激恼於她,但她却根本无动於衷。

由此可见,魏无牙这个人在她心目中根本无足轻重,反而是小鱼儿在她心里的份量重得多。

到了这时,苏樱也觉得越来越奇怪了,暗道:``无论如何,魏无牙总是江湖中有数的厉害人物,而且他也不惜隐姓埋名,二十年来练就一种对付移花宫的武功,可见他和移花宫之问必有极深的仇恨,但移花宫主却根本末将这人放在心上,而小鱼儿连移花宫主的面都末见过,移花宫主却连他的一点小事也不肯放过,甚至不惜忍气吞声,只为要花无缺亲手杀他,这究竟是为了什麽?''

她渐渐也觉得这件事实在很神秘很复杂。

只听小鱼儿道;``好,我带你去,但我现在实在走不动,谁来扶我一把?''

花无缺和铁心兰似乎都想伸过手来,但花无缺发现移花宫主正在冷冷瞧着他,立刻就回头去瞧瞧铁心兰,像是想要铁心兰来扶小鱼儿,但铁心兰发现花无缺在瞧她,却立刻垂下了手。

苏樱嫣然一笑,柔声道:``你若不嫌我走得慢,就让我来扶你吧。''

苏樱扶着小鱼儿已走出很远了,花无缺还站在那里发怔,铁心兰头垂得更低,眼泪已又流了下来。

怜星宫主瞧了瞧花无缺,又瞧了瞧铁心兰,忽然拉起铁心兰的手柔和道:``你跟我走吧。铁心兰做梦也末想到移花宫主竟会来照她,不知是惊是喜,只觉一股柔和的力量自掌心传来,已身不由己地随着怜星宫主掠了出去花无缺见到怜星宫主竟拉起铁心兰的手也是又惊又喜,但忽又不知想起了什麽,眉宇间又泛起一种凄凉之意。

只听邀月宫主缓缓道:``你现在总可以走了吧。''

这虽然只不过是很普通的一句话,但听在花无缺耳里,却又别有一番滋味,只因他发觉移花宫主已看破了他的心事。

他的心事却又偏偏是不足为外人道的。

\hypertarget{ux7b2cux4e00ux767eux96f6ux4e8cux7ae0-ux5947ux5cf0ux518dux8d77}{%
\chapter{第一百零二章
奇峰再起}\label{ux7b2cux4e00ux767eux96f6ux4e8cux7ae0-ux5947ux5cf0ux518dux8d77}}

小鱼儿道:``无论如何,魏无牙总算对你不错,你也承认他是你的乾爹,现在移花宫主要去找他,你非但不着急,反而来带路,这是什麽道理?''

苏樱不说话了,过了半晌,才轻轻叹了气。

小鱼儿道:``我知道你心里一定藏着件事没有说出来,莫非铁心兰方才\ldots,:''

他忽然顿住了语声,只因这时怜星宫主已拉着铁心兰从後面赶上来了,小鱼儿眼珠子一转,忽然向铁心兰笑道:``咱们已有多久没见面了?只怕已经有两个多月了吧?''

铁心兰似乎末想到小鱼儿会忽然对她说话,骤然之间,竟像是有些手足失措,红着脸说不出话来。

小鱼儿又转过头向苏樱笑道:``你看,才两个多月不见,她和我就好像变得很生疏了,我问她一句话,她居然连脸都红了起来。''

苏樱叹了口气,悄声道:``她已经够难受的了,你何必再来折磨她。''

小鱼儿又转过去向铁心兰笑道:``你听见没有,她说我这是在折磨你,我只不过是在向你问问好而已,这也能算我折磨你麽?''

铁心兰只有摇了摇头,眼圈不觉又红了起来。

小鱼儿叹了口气,道;``我想,这两个多月来,一定发生了许多事,因为我发现才只不过两个月不见,你竟已变了许多。''

铁心兰只觉心头一阵刺痛,眼泪不觉又流下面颊,只因她也发觉自己实在是变了。

以前,她只要见到小鱼儿,无论在什麽情况,无论有什麽人在旁边,她都会不顾一切,奔向小鱼儿的。以前,她只要见到小鱼儿,就会忘记一切。

但现在花无缺在她心里的份量的确是一天此一天加重了,只因这两个月来,的确是发生了许多事。

她就算能忘记花无缺曾经再三救了她生命,但她又怎能忘记她受伤时,花无缺对她的照顾与体贴?

何况,她就算能忘记这些,又怎能忘记在那一段漫长的旅途中,所发生的许许多多令人忘不了的事。

她只要一闭起眼睛,似乎就能看到花无缺在痛苦地狂笑着,狂笑着叫她莫要再理他,为的却只是不愿见到她为他痛苦。

一个人在自知必死时,还在挂念着别人的欢乐与悲伤,反而将自己的生死置之於度外。这样的情感,又是何等深挚?这样的情感,又有谁能忘记呢?

怜星宫主始终在一旁凝注着她,忽然冷冷道:``你是不是也觉得自己有些变了?''

铁心兰道;``我\ldots\ldots 我\ldots\ldots{}''

她还末说出第二个字,已是泣不成声。

怜星宫主转向小鱼儿,冷冷道:``你用不着再问她了,应该已知道她的回答。''

她不等小鱼儿说话,忽又一笑,道;``但你也许还是宁愿不知道的,是麽?''

小鱼儿却向她咧嘴一笑,道:``你若是以为我很难受,那才是活见鬼哩。''小鱼儿真的不难受麽?这恐怕也只有他自己才知道。

苏樱实在走不快,走了半个多时辰,远远望去,才能见到那一片浓密的树林,小鱼儿道``前面那一片树林後,就是魏无牙的老鼠洞了,,:''

他话末说完,就瞧见一只又肥又大的老鼠,自树林中窜了出来,一溜烟钻入旁边的乱草中。

过了半晌,又听得草丛一阵窖动,如波浪般起伏不定,竟像是有许多只老鼠在跑来跑去。

小鱼儿娥眉道;``魏无牙一向将这些老鼠当宝贝,现在为什麽竟让他们到处乱跑?''

苏樱嘴里虽末说话,心里却更担心,此刻她已断定魏无牙洞中必已有了极大的变故,否则这些老鼠的确不会跑出来的。

山风吹得更急,她脚步也不觉加快了,阴暝的天色中,只见一个人凌空吊在树上,随着风不住晃来晃去。

小鱼儿娥眉道:``奇怪,魏无牙大门口怎麽有人上吊?''

这人果然是吊死的!

他身上并没有什麽伤痕,但左边脸上,却又红又肿看来竟是在临死前被人重重掴了个耳光。

怜星宫主娥眉道:``这人是魏无牙的门下?''

小鱼儿也不答话,却解开了这人的衣襟。

只见他胸膛上果然有两行碧磷磷的字。

``无牙门下士,可杀不可辱。''

小鱼儿道:``现在你总该知道了吧,这想必是因为有人想闯入魏无牙的老鼠洞,他拦不住,反被人重重打了个耳光,他生怕魏无牙收拾他,所以就吓得先上了吊,看来上吊还不止他一个哩。''

上吊的果然不止一个,这一片树林中,竟悬着十多条死,每个人左边脸都己被打肿,有的连颚骨都已被打碎了。

小鱼儿喃喃道;``这人好大的手劲,随手一耳光,就将人的脸都打碎了,却不知是什麽人呢?居然敢上门来找魏无牙的麻烦,胆子倒也不小。''

他低下头,才发觉地上到处都是一颗颗带着血的牙齿,显见这人随手一掌,非但打肿了别人的脸,打碎了别人的骨头,竟将别人满嘴牙齿都打了下来,这十馀人看来竟连还手之力都没有。

小鱼儿不禁暗暗吃惊,他知道魏无牙门下弟子武功俱都不弱。默然半晌,喃喃道:``看来出手打他们的人,武功至少要比我高出好几倍。''

苏樱心里越来越忧虑,只因她知道魏无牙的武功并不比小鱼儿高出很多,这人的武功若此小鱼儿高出数倍,魏无牙就难免要遭他的毒手了。

小鱼儿道:``但这人却显然末用出真功夫,只是随手拍出,他们非但抵挡不住,甚至连躲都躲不开,由此可见这人出手之快,实在要比我快得多,他随手一个耳光打出来,已可将人的骨头都打碎,可见他内力此我强得多。''

苏樱回首望去,只见移花宫主面色凝重,显然也认为小鱼儿的评论正确,过了半晌,邀月宫主忽然道;``你看他们死了有多久了?''

这句话竟是向小鱼儿问出来的,可见这目空一世的移花宫主,现在也开始对小鱼儿的见解重视起来。

小鱼儿道:``一个人死了一个半时辰後,体温会完全冷却。''

怜星宫主道:``那麽你认为是在什麽时侯发生的''

小鱼儿道;``昨天黄昏以前。''

怜星宫主道:``你怎知道?''

小鱼儿道;``因为我知道两个半时辰以前,那位铁姑娘曾经到过这里,这些人若没死,就一定会将她接入那老鼠洞里,那麽花无缺来找她时,就少不了要和魏无牙打起来,你们来找花无缺时,也少不了要和魏无牙冲突。''

怜星宫主瞧了花无缺一眼,道:``不错。''

小鱼儿道:``但你们显然并不是在这里找到花无缺的,由此可见,那时花无缺和铁姑娘是自己离开这里,是麽?''

怜星宫主道:``那麽,他们为什麽不可能是在两个半时辰之前死的?为什麽一定是在昨天黄昏之前?''

小鱼儿道:``现在正是午时,两个半时辰之前,天还未亮''

他忽然向怜星宫主一笑,接着道:``你若要来找魏无牙的麻烦,会不会在天黑时来呢?''

怜星宫主默然半晌,缓缓道:``不会。''

小鱼儿道:``不错,你一定不会的,因为你若在天黑时来找人,岂非失了自己的身分,何况天越黑,就对魏无牙这种人越有利,你在魏无牙住的地方找他动手,已失了地利,若在晚上来,又失了天时。''

怜星宫主望了邀月宫主一眼,虽然没有说什麽,但瞧她目中的神色竟似已露出些赞赏之意。

小鱼儿道:``瞧这人的出手的气派,就知道他行事一定很光明正大,何况,能练到他这种武功的人,也绝不会是呆子,所以我可以断定,他绝不会是晚上来的,既然不是晚上来的,就必定是在昨天黄昏之前。''

他拍了拍手,笑嘻嘻道;``各位觉得我的意见还不错吧?''

邀月宫主冷冷道;``这道理本来就很明显简单,谁都可以看出来的。''

小鱼儿大笑道:``你既然也瞧得出来,为什麽还要来问我呢?''

邀月宫主沉下了脸,再也不理他,身子飘动,已向林木深处掠了过去,小鱼儿在她後面扮了个鬼脸,笑道;``你也用不着生气,其卖我知道你嘴里虽不说,心里却是很佩服我的。''

穿过树林,前面一片山壁,如屏风般隔绝了天地。山壁上生满了盘旋纠缠的藤萝,尽掩去了山石的颜色。

邀月宫主看不见有什麽山穴石洞,只有回头道:``魏无牙的住处在那里?''

她说话时的眼睛虽望着怜星宫主,其实她也知道怜星宫主同样是不知道的,这句话自然是在问小鱼儿。

小鱼儿却故意装作不懂,却仰首望了天,喃喃道:``我本来以为要下雨,谁知天气又好起来了。''

邀月宫主瞪了他一眼,厉声道;``魏无牙的洞穴在那里?''

小鱼儿好像怔了怔,道;``如此简单明了的事,你怎麽又要问我呢?''

邀月宫主脸又气得苍白,却无话可说。

只见小鱼儿扶着苏樱走过去,将前面一片山藤拨开。

这片山藤长得最密,但却有大半已枯死,拨开山藤,就露出一个黑黝黝的洞穴,里面连光都瞧不见。

小鱼儿道:``这就是了,各位请进。''

魏无牙声势赫赫,仆从弟子如云,谁也想不到他竟会住在这麽样一个连狗洞都不如的小山洞。

大家都不禁觉得很惊奇,尤其是花无缺,他见到苏樱的洞府已是那麽幽雅精致,以为魏无牙的住处必定更可观,忍不住道:``这就是魏无牙住的地方?''

小鱼儿笑道:``不错,你奇怪麽?''

花无缺还想说什麽,但望了邀月宫主一眼,就垂下头去。

小鱼儿嘴里说着话,已当先钻了进去,只见他身子摇摇晃晃,脚步也跟跄不稳,显见得还是没有丝毫气力。

邀月宫主皱眉叱道;``站住!''

小鱼儿道:``为什麽我要站住?这老鼠洞中也不知发生了些什麽稀奇古怪的事,说不定一进去就得送死,我先为你们探探路不好麽?''

怜星宫主道;``正因为先行者必有凶险,所以才要你站住。''

小鱼儿大笑道:``想不到你们竟如此关心我,多谢多谢,可是我既然中了那见不得人的毒,活着反正已无趣得很,死了倒正中下怀。''

邀月宫主冷冷道:``你死不得的。''

小鱼儿只觉风声飕然,邀月宫主已自他身旁不及一尺宽的空隙掠过他前面,连他的衣袂都没有碰到。

见到这样的轻功,小鱼儿也不禁叹了口气,喃喃道:``魏无牙现在若已死了,倒是他的运气,否则若是落在这两位大宫主手上,就难免也要像我一样,连死都死不了啦。''

大家随着邀月宫主走了数十步後,向左一转,这黑暗狭窄的洞穴,竟豁然开朗,变为一条宽阔的甬道。

甬道两旁,都砌着白玉般晶莹光滑的石块,顶上隐隐有灯光透出却瞧不见灯是嵌在那里的。

铁心兰花无缺和移花宫主等人,实未想到这洞中竟别有天地,面上多多少少都不禁露出些惊奇之色。

小鱼儿笑嘻嘻道:``你们现在就奇怪了麽?等你们到里面去一瞧,那更不知道要有多麽奇怪了,我虽未去过皇宫,但想来皇宫也未必会此魏无牙这老鼠洞漂亮。''

他又说又笑,还像是生怕别人听不见,甬道里面回声不绝,到处都是他嘻嘻哈哈的笑声。

怜星宫主冷冷道:``你不说话,也没有人会将你当哑巴的。''

小鱼儿道;``你怕魏无牙听到麽?''

他不等怜星宫主说话,接着又笑道:``我若要来找人麻烦,就一定要光明正大的走进来,若是偷偷摸摸的怕人听见,就算不得英雄好汉。''

怜星宫主也不答话,却缓缓道:``魏无牙,你听着,移花宫有人来访,你出来吧。''

她说话的声音并不高昂,但却盖过了小鱼儿的笑声,一字字传送到远处,可是除了她自己的回声外,就再也听不到一丝声音。

苏樱面上的神情不禁更是忧虑。

魏无牙此刻怕已凶多吉少,他若还没有死,用不着等小鱼儿大声说笑,更用不着怜星宫主喊话叫阵,这甬道中的机关必定早已发动了。

突见邀月宫主停了脚步,道:``你看这是什麽?''

大家随着她望去,才发觉这甬道的地上,竟留着一行脚印,每隔三尺,就有一个,就算是用尺量着画上去的,也没有如此规律整齐。

一这甬道中地上铺的石头,也和两壁一样,平滑坚实,就算是用刀来刻,也十分不容易。

但这人的脚印竟比刀刻的还清楚。

怜星宫主道:``此人为的是来找魏无牙,又何苦将功力浪费在这里拿地上的石头来出气。''

小鱼儿摇了摇头,笑道:``以我看来,说这话的才真有点笨哩。''

怜星宫主怒道:``你说什麽?''

小鱼儿道;``据我所知,单只这一条甬道里,就至少有十畿种机关埋伏,每一种都很可能要你送命。''

怜星宫主道:``你怎知道?''

小鱼儿笑了笑,道:``因为我至少已经尝过了十三种。''

他接着又道;``此人既然要来找魏无牙的麻烦,必然对魏无牙知道得很清楚,走在这甬道里必定步步为营,全身功力,也都蓄满待发,你瞧他脚步间隔,如此整齐,就可想见他那时的情况。''

怜星宫主道:``不错,一个人武功若练到极峰,那麽等他功力集中时,一举一动,都必定自有规律。''

小鱼儿道:``但他并不知道机关要在何时发动,是以他集中的功力随时都在跃跃欲动,便不知不觉在地上留下了脚印。''

他瞧了怜星宫主一眼,笑着接道:``由此可见,此人并不是呆子,只不过功力太强了些而已。''

怜星宫主沉着脸竟不说话了。

邀月宫主道:``但这甬道中的机关却一直并未发动,是麽?''

小鱼儿道;``不错,机关发动後,无论是否伤了人,都会有痕迹留下来的,要等人收拾过後才能复原,而这人走进来後,这洞里的人就好像已死光了,否则我们走到这里,至少要遇见十来种埋伏。''

邀月宫主道:``但此人来时,洞中必定还有人在,机关又为何始终末曾发动呢?''

小鱼儿眼珠转了转,道:``我虽末见到这人走进来时的情况,但可以想见他必定也和我们一样,一面走,一面亮着字号,``魏无牙你听着,我某某人来找你了!这里的机关未曾发动,想必是因为魏无牙一听他的名头,就大吃一惊,知道就算将机关发动也没有用的,又生怕激恼了此人,所以就索性做大方些。''

她们姐妹两人对望了一眼,心里似乎突然想起一个人来上只有小鱼儿才知道她们是想错了。

苏樱忽然道:``看这人的脚印,比平常人至少要大出一半,可见他的身材必定很魁伟,他随随便便一跨出,就有三尺远,可见他的两条腿必定很长。''

她发现每个人的眼睛都已望在她脸上,似乎都在等她说下去。

她就接着道:``据我所知,普天之下,只有一个人的功力如此强猛,而传说中他的身材也和此人一样。''

移花宫主姐妹又对望了一眼,怜星宫主沉着声道;``谁。''

苏樱道:``大侠燕南天!''

移花宫主自然也早已想到此人就是燕南天了,但骤然听到``燕南天''三个字,这冷静得有如冰湖雪水般的两姐妹,面上也不禁为之动容,姐妹两人都不禁向小鱼儿对望一眼,目光却立刻收了回来。

小鱼儿的眼睛也在留意着她们神情的变化。

这其中只有小鱼儿知道此人绝不会是燕南天,因为燕南天纵然还活着,功力也不会恢复得这麽快。

但眼珠子一转,却拍手道:``不错,这人必定就是燕南天大侠,除了燕大侠外,还有谁有这麽高的武功,这麽大的力气。''

邀月宫主忽然道:``此人绝不会是燕南天!''

邀月宫主冷冷道;``他纵然末死,必定也已和死差不多了。''

怜星宫主道:``不错,此人最是好名,以前他每隔一两个月,总要做一件让人人都知道的事,他若还没有死,这二十年来,为什麽全没有他的消息?''

苏樱眠波流转,缓缓道:``你们为什麽不进去瞧瞧,说不定他还在这里没有走哩。''

这句话还末说完,移花宫主姐妹两人飞也似的掠过甬道。

连花无缺和铁心兰也被他们抛下了。

\hypertarget{ux7b2cux4e00ux767eux96f6ux4e09ux7ae0-ux83abux6d4bux9ad8ux6df1}{%
\chapter{第一百零三章
莫测高深}\label{ux7b2cux4e00ux767eux96f6ux4e09ux7ae0-ux83abux6d4bux9ad8ux6df1}}

铁心兰恰巧又站在花无缺和小鱼儿中间,她连头也不敢抬起,神情看来是那麽悲惨,那麽可怜。

花无缺目中也充满了矛盾和痛苦之色,他抬趄头,似乎想说什麽,但一个字也没有说出来,垂下头急步前行。

谁知小鱼儿忽然扑在他面前,笑道:``谢谢你。''

花无缺默然半晌,勉强一笑,道:``你并没有什麽该谢我的。''

小鱼儿叹了口气,道:``现在三个月已经过去,我知道你已不再将我当做你的朋友,但你却还是为我保守了一些秘密,我自然应该谢谢你。''

花无缺又沈默了许久,他每说一句话,都变得好像非常困难,过了半晌,才听他缓缓道:``你用不着谢我,这只不过因为我生来就不是个喜欢多嘴的人。''

小鱼儿道:``但这件事你本该告诉你师傅的,而你却连一个字都没有说,这自然是为了我,只有朋友才会互相保守密,仇人\ldots\ldots{}''

花无缺面上的肌肉一阵抽搐,厉声道:``但我却不是这样的小人?''他说完了这句话,身子已闪过小鱼儿,冲了进去。

小鱼儿又叹了口气,喃喃道;``就因为你太君子了,所以才没有反抗的勇气,你为什麽不能学学我,也做个叛徒呢\ldots\ldots{}''

铁心兰忽然掩面狂奔而出。

苏栖立刻大声呼唤她,她不理也不睬,她心里只有一个念头,那就是远远离开这里,远远离开这些人。

小鱼儿笑了笑道:``一个人若是决心要走,谁也拉不住他的。''

他虽然在笑,但谁也想不到小鱼儿的笑容竟也会如比凄惨。

苏樱道:``但你一定可以拉住她的。''

小鱼儿忽然跳了起来,大声道:``你想要我怎样?你难道要我用铁子锁住她?难道要我跪在地上,痛哭流涕地抱住她的腿!''

苏樱呆呆地瞧着他,目光渐渐朦胧,眼角缓缓流出了两滴晶莹的泪珠,沿着她苍白的脸,滴在她衣服上。

小鱼儿扭过头不去瞧她,冷冷道;``她走了你本该开心才是,哭什麽呢?''

苏樱流着泪道;``现在我只希望也能像她一样,远远的走开,再也看不到你为她生气,为她难受伤心。''

小鱼儿大笑道;``我伤心?我难受?我为什麽要难受?''

苏樱道:``只因这次是她要离开你,而不是你要离开她。''

一这简简单单的两句话,其中却含蕴最深刻最复杂的道理,正如一根针,直刺入小鱼儿的心底。

小鱼儿又跳了起来,道:``既然如此,你为什麽不走呢?''苏樱只有用眼泪来代替回答。

小鱼儿忽然一把搂住了她,嘴唇重重压在她的嘴唇上,他抱得那麽紧,似乎要将苏樱整个人都揉碎。

苏樱似已完全崩溃了,但忽然间,她又用力去小鱼儿的身子,用力推着他的胸膛,嘶声道``放开我,放开我。''

小鱼儿道:``你,你难道不喜欢\ldots\ldots{}''

他忽然放开手,用手掩着嘴,嘴唇上似已泌出鲜血,脸色也变了,也不知是愤怒还是惊奇,苏樱已跟跄退到墙角,不住喘息。

小鱼儿终於长长叹了口气,苦笑道:``我现在才知道我弄错了。''

苏樱目中又流下了泪来,头声道:``你没有错,我也并不是不愿你\ldots\ldots 你抱我,但现在我却不愿你抱着我,心里还在想着别人。''

小鱼儿呆了半晌,刚抬起头,话还没有说出口来,却发现怜星宫主不知何时已站在甬道尽头冷冷的瞧着他们。

在这地方的中央,有一张很大很大的石椅,是用一整块石头虽塑成的,虽然是石头,但比玉质更晶莹,连一丝杂色都看不到,这洞中阴寒之气砭人肌肓,但只要坐在这石椅上,立刻觉得温暖如春。

像这样的石椅,普天之下,只怕再也找不出第二只了,但现在这石椅却已被一剑劈成两半!

邀月宫主和花无缺就在这石椅前,凝注着这石椅被劈开的切口,面色看来都十分凝重。

邀月宫主沈着脸没有说话,过了半晌,忽然自宽大的白袍中,抽出一柄墨绿色的短剑。

剑长一尺七寸,骤看似乎没有什麽光泽,但若多看两眼,便会觉得剑气森森,逼人眉睫连眼睛都难睁开。

邀月宫主对这短剑也似十分珍惜,以指尖轻抚着剑脊,又沉吟了许久,才将剑交给花无缺,道;``你且用九成力在这石椅上砍一剑。''

花无缺道:``是。''

他用双手接过剑,才发觉这短短一柄剑份量沉重,竟远出他意料之外,而且指尖一触剑身,便觉一股寒气直透心腑。

花无缺不敢再问,右手持剑,左足前踏,``有凤来仪'',剑光如匹练般向那石椅劈了下去。

他几乎已将全身买力都凝注在手腕上,莫说这柄剑还是切金断玉的利器,就算他手里拿着的只是柄竹剑,这一剑击下,也足以碎石成粉!

只听``当''的一声,火星四激,这一剑竟只不过将石椅劈开了一尺多而已,剑身就嵌在石缝里。

花无缺手握剑柄,呆了半晌,额上已泌出冷汗。

劈开这石椅的人,就算用的是一柄和他同样锋利的宝剑,功力也至少要此他高出数倍!

世上竟有这样的高手,这实在令人难以想像。

邀月宫主似乎叹了气,缓缓道:``久闻青玉石石质之坚,天下无双,如今看来果然不错,此人能将青玉石一劈为二,剑法倒也不差。''

花无缺忍不住道;``此人剑法虽高,但他的功力只怕更\ldots\ldots{}''

邀月宫主截断了他的话,冷冷道:``这椅背高达五尺,他一剑竟能劈开,而你一剑却只能劈开尺馀,你就认为他的功力至少要比你强三倍,是麽?''

花无缺道:``弟子惭愧。''

他接着又道;``弟子一剑将石椅劈开时,自觉馀力仍甚强,至少可再劈下三尺,谁知剑下一尺後馀力即尽,由此可知,越往下劈越是艰难。''

邀月宫主道;``不错。''

花无缺道:``弟子将这石椅劈开一尺时,只用了三分气力,但再往下劈了三寸,却用了七分气力,此人一剑将石椅劈开五尺,功力又何止比弟子高出三倍。''

邀月宫主淡淡一笑,道;``你错了,你用不着妄自菲薄,普天之下,绝无一人功力能此你高出三倍的,只是你不明白这其中道理何在而已。''

花无缺垂首道:``是,弟子愚昧。''

邀月宫主道;``此人能一剑劈开石椅,而你不能,并不是因为他功力此你高出数倍,只不过是因为他使剑的手比你巧而已。''

此话道理看来虽浅显,其实却正是武功中至深至奥之理,花无缺仔细咀嚼着其中滋味,只觉受用无穷,又惊又喜。

邀月宫主道:``此人不但手法比你巧,出手也此你快,只因``快,就是``力,所以他才能你之所不能,你若和他动手,五十招内,他就可封住你的剑势,一百招内,他只怕就已可取下你的首级来!''

花无缺额上又泌出冷汗。

邀月宫主道:``除此之外,他这一剑劈下时,必是满怀愤怒,只想取人性命,并末考虑到这一剑是否能将石椅劈成两半,出手的气势就自不同,而你出手时,却只是斤斤计较着能将石椅劈开多少,气势已比人弱了七分,你和人动手时若也如此,那就危险得很了。''

一这一席话只说得花无缺不敢抬头,汗透重衣。

突听一人拍手笑道;``移花宫主妙论武功,果然精辟入微,令人闻之茅塞顿开,就连我都忍不住有点佩服你了。''

小鱼儿已笑嘻嘻走了进来,若是换了别人,嘴上被咬破一块,必定少不得要遮遮掩掩。

但小鱼儿却一点也不在乎,眼珠子一转,悠然盯在那柄墨绿色的短剑上,耸然动容道:``这难道就是传说中那柄上古神兵''碧血照丹青``麽?''

邀月宫主冷冷道:``你眼力倒不错。''

小鱼儿道:``据说自古以来,所有神兵利器在冶造时,都要以活人的血来祭剑之後,才能铸成,还有些人竟不惜以身殉剑,是以干将莫邪始,每一柄宝剑的历史,必定都是凄恻动人的故事!''

邀月宫主道:``现在并不是说故事的时候。''

小鱼儿也不理她,接着道;``只有这柄碧血照丹青』,用一个人的热血来祭剑,剑还是不成,铸剑师的妻子儿女都相继以身殉剑,也没有用,铸剑师悲愤之下,自己也跃入法炉,谁知他自己跳下去後,炉火竟立刻纯青,又燃烧了两口後,才有个过路的道人将剑铸成,据说此剑出炉後,天地俱为之变色,一声霹雳大震,那道人吃了一鹫,被霹雳震倒,竟恰巧跌倒在这柄剑上,就做了这柄剑出世後的第一个牺牲品。''

说到这里,小鱼儿才笑了笑,道:``这些话当然只不过是後人故神其说,并不足信,试想那些人既已死尽,这故事又是谁说出来的呢?''

邀月宫主道;``不错,这些事并不足信,但有一件事你却不能不信。''

小鱼儿道:``什麽事?''

邀月宫主道:``那铸剑人自己跃入法炉时,悲愤之下,曾赌了个恶咒,说此剑若能出炉,以後只要见到此剑的人,必将死於此剑之下''她目光冷冷的凝注着小鱼儿,一字字接着道:``唯有这件事,你不能不信''

苏樱听得忍不住机伶伶打了个寒噤,情不自禁,转过了头去不敢再向那不祥的凶器看一眼。

花无缺忽然``呛''的自石上抽出了剑,双手送到邀月宫主面前,邀月宫主目光闪动,淡淡道:``你留着它吧。''

花无缺脸色变了变,垂下头去,道:``弟子\ldots\ldots{}''

他话还没有说出来,小鱼儿又大笑道:``你将剑送给他,可是想要他用这柄剑来杀我麽?但你莫忘记,那铸剑师的恶咒若是真的很灵,你也免不了要死在这柄剑下的?''

邀月宫主的面色也忽然为之惨变,目光忽然刀一般转到花无缺身上,但这时怜星宫主已抢着道:``无缺,你去将铁心兰找回来。''

花无缺似乎又吃了一惊,失声道:``她\ldots\ldots{}''他瞧了小鱼儿一眼,立刻又闭上了嘴。

邀月宫主道:``她已走了,但以她的脚力,必定不会走得太远,你一定能追得上的。''

花无缺垂首道:``但弟子\ldots\ldots 弟子\ldots\ldots{}''

怜星宫主厉声道:``你怎样?你难道连我的话郡不听?''

花无缺又瞧了小鱼儿一眼,虽然满面俱是痛苦为难之色,却还是不敢再说什麽,笔直冲了出去。

小鱼儿却似完全没有留意到他,道:``你们进来时,这老鼠洞里已没有人了麽?''

邀月宫主方才听了那句话後,到现在彷佛还是心事重重。

怜星宫主沉声道:``一个人鄱没有。''

小鱼儿皱眉道:``那麽魏无呀呢他难道已经逃走了麽''苏樱虽末说话,却忍不住露出惊喜之色。

小鱼儿眼珠子一转,道``你能不能扶着我到四下去瞧瞧?''

魏无牙就算是世上最残酷恶毒的小人,但做起事来却当真不愧为大手笔,竟几乎将这座山的山腹都挖空了。

除了这一片宫殿般的主洞外,四面还建造了无数间较小的洞室,一间间排列得就像蜂房似的。

苏樱扶着小鱼儿一间间走过去,只见每间洞室都很整洁,甚至可以说都很华丽,而且还都有张很柔软、很舒服的床。

小鱼儿叹了口气,道:``我大概已经有两三年没有在这麽舒服的床上睡过觉,想不到这些小老鼠的日子倒比我过得舒服。''

苏樱道:``魏\ldots\ldots 魏无牙对门下的弟子虽然刻薄寡恩,但只要他们不犯错,日常生活上的享受倒的确还不错。''

小鱼儿道;``但老鼠为什麽要搬家呢?他们难道早已算准了有猫要来麽?魏无牙的本事就算不小,总也不能未卜先知吧。''

苏樱默然半晌,道:``不错,这人既是突然而来的,魏无牙就绝不可能知道,他若在仓促间逃走,就绝不会走得如此乾净。''

小鱼儿道:``何况,他在这里苦练了二十年的武功,又建造了这许多机关消息,为的就是要准备对付燕大侠和移花宫主。''

苏樱点了点头,道;``不错,他的确有这意思。''

小鱼儿道:``但他自己现在却偏偏走了,这是为了什麽呢?这道理你能想得通麽?''

苏樱苦笑道:``我想不通。''

小鱼儿道:``除此之外,我还有件想不通的事。''

苏樱道:``哦。''

小鱼肝道:``那天我受了重伤时,魏无牙忽然匆匆而出,去迎接一位贵客,现在我才知道,这位贵客就是江别鹤。''

苏樱道:``不错。''

小鱼儿道:``江别鹤虽然是江南大侠,但``江南大侠』这四个字,在魏无牙眼中,只怕连一文都不值。''

苏樱道:``看来只怕是早就认得的,否则江别鹤既不找上门来,也根本就找不着他。''

小鱼儿道:``所以我就又想不通了,江别鹤崛起江湖,只不过是近年来的事,魏无牙却已在这里隐居了十七八年,他们是怎麽会认得的呢?''

他叹了口气,接着又道:``这两人既已勾结在一起,魏无牙如虎添翼,本该更不会走的,但却偏偏走了,所以我想,这件事其中必定有些阴谋,说不定根本就是他们故意布置出来的圈套,我一走进来,就觉得这地方有些不对了。''

突听一人道:``有什麽不对?''

这语声忽然自他们身後发出来,但苏樱和小鱼儿非但都没有吃惊,甚至根本没有回头去瞧一眼因为他们知道移花宫主必定会跟在他们身後的,也知道以移花宫主的轻功,他们必定觉察不到。

小鱼儿道:``这地方虽然连个人影都没有,但我却觉得到处都充满了杀机,好像已走进了座坟墓,再也出不去。''

怜星宫主冷冷道:``这只不过是你疑心生暗鬼而已。''

小鱼儿道;``这也许只不过是我的疑心病,但无论如何,我却不想再留在这地方了,你们若不想走,我可要先走一步\ldots\ldots{}''

他的话还末说完,突听一人咯咯笑道:``你现在要走,只怕已来不及了。''

小鱼儿这一辈子虽然活得还不算长,但各式各样的笑声倒也听过不少,但无论多麽难听的笑声,若和这笑声一比,简直就变得如同仙乐了,他也知道普天之下,只有一个人的声音会如此难听。

移花宫主和苏樱都已悚然失色。

小鱼儿也忍不住叫了起来,道:``魏无牙还在这里!''这洞中的人既已走光了,魏无牙怎还在这里?

只听那人咯咯笑道:``不错,我还在这里!我在这里等候各位的大驾已有多时了。''

这笑声就是从隔壁的一间石室中传出来的。

但在这刺耳的笑声中,这洞室的石壁忽然奇迹般打开,一辆很小巧的两轮车已自石壁中滑了出来。

这辆车子是用一种发亮的金属造成的,看来非常灵便,非常轻巧,上面坐着个童子般的侏儒。

他盘膝坐在这辆轮车上,恨本就瞧不见他的两条腿。

他的眼睛又狡猾,又恶毒,带着山雨欲来时那种绝望的死灰色,但有时却又偏偏会露出一丝天真顽皮的光芒,就像是个恶作剧的孩子。

他的脸歪曲而狞恶,看来就像是一只等着择人而噬的饿狼,但嘴角有时却又偏偏会露出一丝甜蜜的微笑。

小鱼儿说的不错,这人实在是用毒药和蜜糖混合成的,你明明知道他要杀你时,还会忍不住要可怜他。

移花宫主一眼瞧见他,竟也不禁骤然顿住身形,不愿再向他接近半寸,正如一个人骤然见到一条毒蛇似的。

魏无牙悠然道;``你方才说的并不错,这里实在已是一座坟墓,你们再也休想走出去了!''

邀月宫主变色道:``你说什麽?''

魏无牙道:``这里就是整个洞府的机关枢纽所在地,现在我已将所有的出路全都封死,莫说是人,就算一只苍蝇也休想飞得出去了。''

小鱼儿大骇之下,就想赶出去瞧瞧,但忽又停住,因为他知道魏无牙既然说出这话来,就绝不会骗人的。

他眼珠子一转,却笑道;``你将所有的出路全都封死了?''

魏无牙道:``不错。''

小鱼儿笑道:``那麽,难道你自己也不想出去了麽?''

魏无牙道:``我正是已不想再出去。''

\hypertarget{ux7b2cux4e00ux767eux96f6ux56dbux7ae0-ux89c1ux5229ux5fd8ux4e49}{%
\chapter{第一百零四章
见利忘义}\label{ux7b2cux4e00ux767eux96f6ux56dbux7ae0-ux89c1ux5229ux5fd8ux4e49}}

小鱼儿大笑道:``你说的话,有谁会相信?就算你要将她们活活葬在这里,你也可以找别人来发动这机关,为什麽自己要来陪葬呢?''

魏无牙淡淡道:``这只因我要亲眼瞧见她们死,亲眼瞧见她们临死前的痛苦之态,我还要亲眼瞧瞧她们被饿和恐惧折磨时,是不是还能保持这样圣女的模样!''

小鱼儿望了移花宫主一眼,只见这姐妹两人就像是忽然变成了两个石像,连动都不动。小鱼儿眼珠子一转,忽又大笑道:``但你这样做,一定是因为自知还不是她们的对手,否则你就可以真刀真枪的杀了她们,用不着自己也来陪葬了,是麽?''

魏无牙叹道:``不错,我本以为这二十年来,武功已精进许多,已足可将她们置之於死地,但见到江别鹤时,才知道自己错了。''

小鱼儿又不觉怔了怔,道:``你为何要等见到他时,才知道自己错了?''

魏无牙道:``二十年前,江别鹤的武功根本还不入流,但现在却已可算得上是江湖中的一流高手,这二十年来,连他的武功都进步了这麽多,何况移花宫主,我和移花宫主的武功若是同样在进步,那麽我再练二十年,还是一样胜不过她们,何况,她们有姐妹两人,我却只有孤零零一个。''

他笑了笑,接着道:``所以我想来想去,只有用这一手了。''

小鱼儿道:``既然如此,她们现在要杀你,还是简单得很,你\ldots\ldots{}''

魏无牙冷冷道:``这些门户俱是万斤巨石,现在已被封死,连我自己也是开不开的。''

小鱼儿也石头般怔住,再也说不出话来。

魏无牙道:``何况,你们就算明知这里的门户都已被封死,还是难免要抱万一的希望,而我就是你们唯一的希望,所以我算准你们绝不敢杀了我的?''

他忽又笑了笑,道:``樱儿,你为什麽躲在外面不敢进来?''

苏樱垂首走了进来,脸色也苍白得可怕。

魏无牙瞪着她瞧了半晌,又瞧了瞧移花宫主,道:``我一向对你不错,你可知道是为了什麽''

苏樱垂首道;``我\ldots\ldots 我不知道。''

魏无牙笑道:``你若瞧瞧这两位宫主,再自己照照镜子,就会知道了。''

小鱼儿心里一动,这才发现苏樱和移花宫主的容貌竟有七分相似之处,她们都是绝世的美人,面色又都是那麽苍白,神情又都是那麽冷漠,看来简直就像亲生母女同胞姐妹差不多,苏樱也不知是惊是喜,动容道:``你老人家对我好,难道就是为了我长得很像她们?''

魏无牙道;``不错,否则天下的孤女那麽多,我为何要将你一个人救回来?我一向对你百依百顺,就因为我要将你养成冷漠高傲之态,我要你一个人住在那里,就因为我要养成你孤僻的性格,,,:''

苏樱道:``你老人家想尽法子,难道只为了要便我变得和她们一模一样麽?''

只听小鱼儿拍手大笑道:``我现在才明白了,原来你的心上人竟是移花宫主,就因为你得不到她们,所以因爱生恨,才会对她们恨之入骨。''

他是世上最聪明的丑侏懦,竟会爱上世上最最高贵,最最美丽的女人,这种事实在不可思议,妙不可言。

小鱼儿越想越好笑,笑得连气都喘不过来。

魏无牙却一本正经,缓缓道:``二十多年前,我专程赶到移花宫去,向她们两位求亲\ldots\ldots{}''

小鱼儿喘着气笑道:``你\ldots\ldots 你向她们求亲?''

魏无牙正色道;:笑?''

小鱼儿道;``是是是,这件事实在再相配也没有,只可惜她们非但不答应,还要杀了你,你们的仇恨,就是这样结下来的,是麽?''

魏无牙叹了口气,虽然没有说话,却已无异默认。

再看移花宫主姐妹两人,已气得发抖。

小鱼儿眼珠子一转,笑嘻嘻道:``有这样的大英雄大豪杰来向你们求亲,正是你们的光荣,你们为何竟不肯答应呢?我实在觉得很可惜。''

魏无牙大笑道:``你用不着激怒她们,要她们向我出手,她们就算杀了我,你也没什麽好处,你若真是个聪明人,就该劝她们莫要杀我才是,等我自己饿得受不了时,说不定会想出个法子,将封死的门户再打开的。''

小鱼儿瞪着他瞧了半晌,道:``不错,你现在的确不能死,我还有很多事要问你。''

魏无牙道:``你第一样要问我的,会是方才究竟有谁来了能一剑将青玉石椅劈开的人,究竟是谁?对不对?''

小鱼儿道:``不对,这件事我已用不着问你,只因我现在已经明白了。谁也没有来。''

魏无牙大笑道:``谁也没有来?在甬道上留下脚印的难道是我麽?''

小鱼儿道:``甬道上那些脚印只是你自己刻出来的,所以才会那麽整齐。''魏无牙目光闪动,道:``外面树林中那些人又是谁杀死的呢?''

小鱼儿道:``自然就是你自己杀死的,你打他们的耳光,他们自然不敢还手,也不敢躲避,你要他们上吊,他们就不敢跳河。''

魏无牙道;``如此说来,那青玉石椅难道也是被我自己劈开的麽?''

小鱼儿道:``你既然能将青玉石削成椅子,你手里就一定有柄削铁如泥的宝剑。这宝剑既能将青玉石削成椅子,就一定能将椅子劈成两半\ldots\ldots 这道理岂非明显得很麽?''

魏无牙叹了气,道:``不错,这道理实在很明显了。''

小鱼儿道;``你将树林中的那些徒弟杀死,又在甬道上刻下那些脚印,就是为了要引诱我们走进来。''

魏无牙道:``这也很有理。''

小鱼儿道:``但你又生怕我们一走进来,发现这里已没有人,就立刻又走出去了,所以你就将那石椅劈成两半,叫我们心中猜疑,而且\ldots,:''

他歇了气,才接着道:``这里的门房既然全都是千斤巨石做成的,要将它们完全封死,也绝对不是一时半刻间能做得到的。''

魏无牙接着道:``所以我就要将你们的注意力全都吸引到那张石椅上,我才有时间从从容容将门户封死,是麽?''

小鱼儿抚掌道:``正是如此。''

魏无牙忽然大笑起来,笑得几乎从轮椅上跑到地上。

小鱼儿瞪眼道:``你笑什麽?我猜的难道不对麽?''

魏无牙大笑道:``对对对,实在太对了,你实在是天下第一聪明人。''

小鱼儿笑道;``对於这一点,我倒是从来不敢自谦。''

魏无牙道:``只不过我也有几句话要问你。''

小鱼儿道:``哦?''

魏无牙道:``你到我这地方来过,总该知道,这里到处都是奇珍异宝,现在为什麽连一件都没有了呢?''

小鱼儿怔了怔,道;``这\ldots\ldots 这自然是你要你的徒弟带出去了。''

魏无牙道:``我为什麽要他们带走?我既已决心死在这里,为什麽不将这些珍宝拿来陪葬,却将它们送给别人,我既然从来也未将我的徒弟当做人,为什麽要让他们落个大便宜\ldots\ldots 这其中道理你想得通麽?''

小鱼儿眼睛忽然一亮,道:``这只因你想看我们死了後,再走出去。''

魏无牙道:``我若有这样的打算,更不该将珍宝送走了,只因我此刻若想走出去,一定要等你们全都死光,我难道还怕你们这些已快死的人来抢我的珠宝麽?''

小鱼儿这次才真的怔住了。``如此说来,这地方难道真有位武林高手来过麽?来的这人是谁?''

魏无牙道:``这人是你认得的。''

小鱼儿道:,你怎知我认得他?''

魏无牙悠然道:``只因他曾经问起过你。

小鱼儿面上变了变颜色,忽然大笑道:你难道要告诉我,来的这人是燕南天麽?''

魏无牙眼睛盯着他,一字字道;``不错!来的这人正是燕南天!''

小鱼儿怔了许久,忽又大笑起来,道:``燕南天若来过,你怎麽还能活在世上害人?''

魏无牙冷笑道:``你以为他武功比我高?''

小鱼儿面色又变了变,但瞬即展颜笑道:``他若真的来过,甬道上的脚印就是他留下来的,石椅自然也就是被他神剑所劈开,这一剑之威,足以惊动天地,就凭你这身本事,只怕还难伤得了他一根毫发\ldots\ldots 你的本事我是知道的。''

魏无牙默然半晌,长长叹了口气,道:``不错,单只他那一剑之威,已足可睥睨天下,我实在还不是他的敌手。''

小鱼儿道:``他若真的来过,为何没有杀了你呢?''

魏无牙缓缓道:``这自然有交换条件。''

小鱼儿道:``什麽条件?''

魏无牙道;``我答应交给他一个人,他就答应不伤我性命。''

小鱼儿追问道:``你答应将谁交给他''

魏无牙道:``江别鹤!''

小鱼儿又吃了一鹫,失声道;``江别鹤?燕大侠竟肯为了江别鹤,饶了你的性命?''

魏无牙道:``不错。''

小鱼儿道;``他为什麽要救江别鹤?''

魏无牙笑道:``他不是为了要救江别鹤,而是要杀他。''

小鱼儿不禁又是一怔,道;``他和江别鹤又有什麽仇恨?''

魏无牙默然半晌,缓缓道:``你可知道江别鹤的本来面目是谁麽''

小鱼儿道:``是谁?''

魏无牙道:``他本来就是你父亲的书童江琴,从小就在你们家长大,你父亲和他名虽主仆,其实却无异兄弟。''

小鱼儿吃惊得张大了嘴,合不拢来。忍不住问道;``江琴既然和先父也情同手足,燕大侠又为何要杀他?''

魏无牙道:``江枫非但是天下少见的美男子,也是数一数二的大富翁,江湖好汉们早已想打他的主意了,只是碍着燕南天,所以迟迟不敢下手。谁知道江枫忽然鬼迷心窍,竟和移花宫门下一个女徒弟私奔了,这女人也就是你的母亲。''

小鱼儿怒道:``你说话用字最好放文雅些。''

魏无牙毗牙一笑,悠然接着道:``这两人虽然已爱得发晕,不顾一切,但也知道移花宫主是绝不会放过他们的,所以两人一逃回来,江枫就将家财送的送,卖的卖!自己只带着些随身细软准备亡命天涯,隐居避祸。''

小鱼儿怒道;``所以你们这些臭强盗就红了眼睛。''

魏无牙道:``不错,江枫的计划,是要江琴先轻骑去找燕南天,他自己再带着你母亲穿过一条久已废置的古道,赶快和燕南天会合,这计划本来不错,他走的路本来也很秘密,只可惜江琴还没有去找燕南天时,就先找到咱们``十二星象了。''

小鱼儿狠狠道;``难怪你认得江别鹤,原来你们早已狼狈为奸,干过买卖。''

魏无牙一笑道:``这件事我虽然知道,但却没有出手,因为我就算不出手,也不怕他们得手後不分给我,而且我那时也正有别的事不能分身。''

小鱼儿道:``出手的是被燕大侠宰了,他们早该明白燕大侠的手段,为什麽还要出手?''

魏无牙道;``他们本来打算将这笔账算在移花宫主身上的,让燕南天认为这是移花宫主动的手,再加上江琴又将你父亲带出来的东西开了张清单,这麽大的买卖,``十二星象又怎肯放过?''

小鱼儿咬牙道:``但江琴也该知道『十二星象』是什麽角色,这买卖既然已归了十二星象,他还有什麽便宜好占的?''

魏无牙笑道:``他的贪心并不大,只要占其中两成,他也知道我们``十二星象』做买卖最公道,只要答应分给他的,就绝不会赖账。而且,你父亲虽然将他当自己兄弟,但在别人眼中,他还只不过是个江枫家里的一个奴才,你父亲若不死,他就一辈子也休想出头。''

他微微一笑,接着道:``这人的贪心虽不大,野心却不小,一心只想在江湖中成名立万,所以他就非先害死你父亲不可。''

小鱼儿只觉手脚冰凉,默然半晌,道:``但我父亲後来并不是死在``十二星象''手上的,是麽?''

魏无牙道;``後来的事,我知道得并不太详细,我只知道等燕南天赶去的时候,你父母都死了,只有你还活着。''

小鱼儿强忍住心里的悲痛,道:``无论我父母是被谁动手杀死的,这原因总是江琴而起。他若不出卖我父亲,这些人就一定找不到他老人家的,是麽?''

魏无牙道:``正是如此。''

小鱼儿道:``既是如此,燕大侠那时为何不杀了他呢?''

魏无牙道:``燕南天那时只怕还不知道江琴是罪魁祸首,等他知道的时侯,江琴早已溜了,从此之後,江湖中就再也没有听见过江琴的消息,也没有再听到燕南天的消息,後来我才听说燕南天已死在恶人谷。''

他又叹了口气,苦笑道:``谁知这消息竟是放屁,燕南天非但没死,而且武功又精进了不少,那江琴摇身一变,竟变成江南大侠了。''

小鱼儿默然半晌。他实在也想不通燕南天怎会忽又现身的?他的病势怎会忽然痊愈?难道是忽然出现了什麽奇迹?还是另外又有个像``南天大侠''路仲远那样的人,又借用了``燕南天''这名字?这人会是谁呢?

\hypertarget{ux7b2cux4e00ux767eux96f6ux4e94ux7ae0-ux52feux5fc3ux6597ux89d2}{%
\chapter{第一百零五章
勾心斗角}\label{ux7b2cux4e00ux767eux96f6ux4e94ux7ae0-ux52feux5fc3ux6597ux89d2}}

苏樱忽然问道:``这位燕大侠是不是已经将江别鹤杀死了呢?''

魏无牙道:``还没有。''

苏樱道:``燕大侠为什麽还不杀他?''

魏无牙道,``因为他要将江别鹤留给小鱼儿,要小鱼儿亲手复仇。他一天找不着小鱼儿,江别鹤就一天不会送命,他十年找不着小鱼儿,江别鹤就十年不会送命。''

苏樱失聋道:``如此说来,江别鹤岂非\ldots,:岂非,,;''她的话虽没有说完,意思却已很明显。

魏无牙大笑道``不错,江别鹤永远也送不了命的,因为燕南天永远也找不着小鱼儿了,他武功虽比江别鹤高明十倍,但却远不及江别鹤诡计多端,他将江别鹤这种人带在身侧,就好像拉着只老虎满街跑似的,迟早总有一天,他的命也要送在江别鹤手上。''

小鱼儿大怒道;``他饶了你性命,你却这麽样对付他,你还算是个人麽?''

魏无牙抑住了笑声,恨恨道:``他虽然没有杀我,却将我的徒弟全都赶走,而且要他们将我的珠宝全都带走,这岂非和杀了我一样?''

小鱼儿这才完全明白了,忍不住笑道:``只怕他非但赶走了你的徒弟,连你那些宝贝老鼠也被赶走了,是麽?''

魏无牙咬着牙,道;``哼。''

小鱼儿道:``原来你是自觉活着没意思了,才想出这最後一着来的,但你平时若对你那些徒弟稍微好些,他们又怎会在你有困难时离你而去?''

魏无牙忽又阴恻恻一笑,道:``但现在既已有你们陪着我死,我已经很心满意足了。''

突听移花宫主唤道;``江小鱼,你过来。''

小鱼儿本来似乎不愿过去了,但想了想,还是过去了,走了两步,又回过头来望了望苏樱。

苏樱本来似乎要先看看魏无牙的反应,但忽又改变了主意,只是向小鱼儿嫣然一笑就跟了过去。

移花宫主姊妹两人站在``大厅''的中央,神情虽然还是那麽骄傲而冷漠,但看来已似忽然变得很渺小,很孤独,很可怜。

但她们还是笔直的站着,没有坐下来。她们几乎从来也没有坐下来过。

邀月宫主霍然转过身子,像是生怕自己再瞧见小鱼儿一眼之後,会忍不住出手将他杀了。

怜星宫主缓缓道:``我们方才已将这小洞四面都探查了一遍。这四面的门户的确已全都被闭死了。''

小鱼儿道;``我根本用不着去看,也知道这绝不会是假的。''

怜星宫主默然半晌,道:``这门户俱是万斤巨石,绝非人力所能开启,但我想,魏无牙绝不会甘心将自己困死在这。''

小鱼儿道:``你难道想要我将这条逃路找出来麽?''

怜星宫主又沉默了半晌,缓缓道:``我想,你也许有法子能自魏无牙口中探听出来。''

小鱼儿道:``你以为我真有那麽大的本事?''

怜星宫主道:``他若不肯说,你就杀了他!''她瞟了苏樱一眼,又道:``我看得出他对你已恨之入骨,若有机会亲手杀你,他绝不会错过。''

小鱼儿道:``这话倒是不错,只可惜我若和他动手,送命的不是他,而是我。''

怜星宫主道:``我也知道你此刻武功还不及他,但只要我教你三个时辰的武功,他就万万不会是你的对手了。''

小鱼儿道;``哦,你真有这麽大的把握?我有点不信。''

怜星宫主淡淡道:``本门武功的神奇奥妙,又岂是你们所能想像。''

小鱼儿忽然不说话了。他歪着头想了半天,竟又大笑起来。

怜星宫主怒道:``你以为这是在说笑麽''

小鱼儿道;``我为什麽要平白费这麽大力气,去和魏无牙动手呢?''

怜星宫``又不禁怔了怔,道:``但你若能将他击倒,再以死相胁,他只怕就会将最後一条逃路说出来的。''

小鱼儿道:``我为什麽要逃出去?这不是很舒服麽''

怜星宫主气得脸色发白,话也说不出来。

小鱼儿悠然道:``我反正也中了毒,迟早总是要死的,就算你们能解了我的毒,我还是难免要死在花无缺手上,既然我算来算去,都是非死不可,倒不如索性死在这,我看这坟墓倒也堂皇富丽。''

怜星宫主一直瞪着他,等他说完了,又瞪着他许久,忽然道:``我若保证你绝不会死在花无缺手上呢?''

邀月宫主忽然厉声道;``你和无缺这一战势在必行,绝无更改\ldots\ldots{}''

小鱼儿叹道:``既然如此,那就没法子了,我们大家只好一在这等死吧。''

怜星宫主道:``但你莫忘了,我若能令你的武功胜过魏无牙,就也能胜过花无缺,你若能杀了魏无牙,就也能胜过花无缺!''

小鱼儿眨了眨眼睛,道:``花无缺是你们从小养大的,非但是你们的徒弟,简直已和你们的儿子差不多了,我却是你们的仇人之子,若非我明知武功比你们差得太远,说不定我早就要了你们的命了,现在你们竟要传授我武功,要我去杀死你们的徒弟,这种话天下只怕再也没有一个人会相信。''

怜星宫主望了她姊姊一眼,邀月宫主道:``这其中自然有\ldots\ldots{}''

小鱼儿目光闪动,等着她说下去,谁知她刚说了几个字,忽又顿住语声,小鱼儿追问道;``你们若要我相信,也容易得很,只要你们将这其中的原因说出来,你们无论要我做什麽,我都可以答应。''小鱼儿眼睛盯着她,悠悠道:``你们难道情愿让魏无牙看见你们临死前的丑态,也不肯说出这秘密麽?我可以告诉你们一个人临死的时候,那样子非但很难看,而且还很可笑。''

邀月宫主咬了咬牙,忽又转过身。怜星宫主也随着她缓缓转过身去,两人既不愿再瞧小鱼儿一眼,也不愿再听他说一个字了。

小鱼儿木头人般愣了半晌,忽然转向苏樱道:``这件事前前後後你已知道了不少,是麽?''

苏樱叹道:``我现在已知道江伯母以前本是移花宫的门下,後来\ldots\ldots 後来\ldots\ldots{}''

小鱼儿咬着牙道:``我父母无疑是死在她们手上的,她们当时没有斩草除根,现在却想杀了我,以免留下後患。可是她们为什麽一定要花无缺动手杀死我呢?她们若肯自已动手我现在早已不知死过多少次了。''

苏樱道:``她们本来以为你们会很恨花无缺的,你不找她们复仇,就一定会找花无缺,谁知你的思想却开明得很,竟认为上一代的仇恨,和下一代无关,所以她们只好逼着花无缺来杀你了。据我看来,你和花无缺之间,必定还有一种极复杂的关系。''

小鱼儿眼睛一亮,又皱眉道:``但我和花无缺之间却又绝不可能有什麽关系的,我一生下来就被带到恶人谷去了,在这世上,我根本没有什麽亲人。''

洞窟中静寂得穴在和坟墓没什麽两样,从石壁间透出来的灯光很柔和,月光般照着小鱼儿的脸。这本是张明朗骄傲,倔强,充满了魅力的脸,但现在看来,却显得说不出的黯淡,说不出的疲倦。苏樱痴痴的瞧着,目中似乎隐隐泛起了泪光。

也不知过了多久,只听小鱼儿喃喃道:``苏樱,你要知道,我并不是怕死,但要我就这样糊糊涂地死了,我实在不甘心\ldots\ldots 实在不甘心!''

苏樱道:``这地方门户若真的全都封死了,整个洞窟就该和坟墓般变得密不通风,可是\ldots\ldots 直到现在我们还没有气闷之感,而且不通气的地方,连火都燃烧不起来。''

小鱼儿用拳头打了打手掌,道:``好,只要他真的还留下一条路我就有法子要他说出来。''

苏樱忽然一笑,道:``你不是已经不想出去了麽?''

小鱼儿向她扮了个鬼脸,道:``那只是我故意要胁她们的,这秘密还没有水落石出之前,我非但自己舍不得死,还舍不得让她们死哩。''绝望之中,忽然又有了一线生机,两人的精神都不禁变得振奋起来,两人正想往前走,忽然身後传来一声叹息。``你们不用找了,我就在这里!''

那本来放着青玉椅的石台,现在忽然移开了魏无牙推着轮车,从下面缓缓滑了上来。

``我知道你现在心里一定又在打主意,要想法子令我说出那些通风之处在那,那麽我劝你,这心思你也不必白费了。因为那时我造那些气孔时,就怕老鼠会从气孔中逃出去。''

小鱼儿沉思了半晌,忽又问道:``你是怕我们死得太快了麽?''

魏无牙嵘嵘笑道:``这就对了,我费了许多力气,才将你们弄到这地方来,怎麽舍得一下子就将你们闷死?我当然希望你们死得越慢越好,这样我才能慢慢欣赏你们临死时忍不住要做出来的种种丑态,我敢担保世上绝没有一件事比这更有趣的了。''他似乎越想越有趣,笑得整个人都扭曲起来。

小鱼儿居然也笑了,道:``我们想问问你,你认为我们会做出什麽丑态来。''

魏无牙眼睛闪着光,笑道:``你总该知道,移花宫主姊妹是从不肯随便坐下来的,无论什麽地方她都嫌脏,但我敢担保,不出三天,她们就会躺在那些臭男人睡过的床上了,她们平时什麽东西也不肯吃,但再过几天,就算有只死老鼠她们说不定也会吞下去,也说不定会将你们两人煮来吃了,你信不信?''

小鱼儿大笑道:``她们若真会将我吃下肚,倒也妙极,我情愿葬在她们两人的肚子。''

他虽在哈哈大笑,暗中却已不禁毛骨悚然,因为他知道魏无牙所说的话,并不是完全不可能。

只听魏无牙笑着又道:``还有,我知道你们这四个人还都是童男童女,还没有一个真正过人生的乐趣,到了快死的时候,说不定会忽然觉得这麽一死未免太划不来了,说不定就会想那件事是何效味。''他眠睛充满了猥亵之意,脑子似乎已在幻想着那时的情况,蜷曲着身子狂笑着接道:``到了那时,你这小伙子只怕就要变成宝贝了。''

``你为什麽不想这磁味呢?难道你已经不行了麽?''小鱼儿盯着他的两条蛇曲的腿,冷笑道:``原来你早就不行了,所以才会变成这麽样一个疯子,我本来觉得你很可恨,现在才发觉你原来很可怜。''

魏无牙忽然狂吼一声,向小鱼儿扑了上来。小鱼儿身形急转,双掌反切。谁知魏无牙的身上忽又多出十根短剑,划向他的手腕。原来他每根手指上都留着三四寸长的指甲,平时是蛇曲着的,与人动手时,真气贯汪指尖,指甲便剑一般弹出。灯光下,只见这十根指甲隐隐闪着乌光,显然淬着剧毒,小鱼儿只要被他划破一点油皮,就无救了。

他这一扑之势,竟藏着三种变化後着,每一种变化都出人意外,招式之怪异狠毒,竟是天下无双。苏樱已忍不住惊呼出声来。只见小鱼儿身子就地一猿,已滚出两丈外,这一着破法更非正统武功,只是小鱼儿随机应变临时创出的。

谁知魏无牙身子一转,竟又落回那轮车上。小鱼儿正想扑过去时轮车忽然围着他兜起圈子。

刹那间,小鱼儿只觉自己前後左右,都是魏无牙的人影,竟比那威震天下的``八卦游身掌''还要厉害三分。

但一个人步法无论多麽巧妙,也没有轮子转得快的。小鱼儿只觉头晕眼花,几乎不用魏无牙出手,他就要倒下去了。小鱼儿忽然长啸一声,冲天而起。这一招竟是昆仑派的镇山绝技``飞龙大八式''。普天之下,唯有``飞龙大八式''能破解魏无牙这种功夫,除此之外,纵是武当少林的掌门大师,也难免要被魏无牙困死。

谁知他身形方自凌空飞起,魏无牙竟又迎面扑了过来,十根闪闪发着乌光的指甲,又划到他咽喉。这人竟生像是已变了小鱼儿的影子,小鱼儿竟连变招都已不及,猝然间竟使出了少林的``千斤坠''。

要在身形上冲时突然落下,也并不是件容易事。但小鱼儿偏偏就在这间不容发时落了下来。

谁知他身子刚落下,只听``嗖,嗖,嗖''急风破空,三道乌光,分由三个不同的方向射了过来。

原来魏无牙身子虽已飞起,但那轮车却还在不停的转动,这三道乌光,竟是转椅中射出来的。这一着才真的出小鱼儿意料之外,若是换了中原武林任何一门一派的高手,此番都难免要丧在这三根乌骨箭下!

只见他身子忽然一折一扭,全身的骨头竟像是都忽然分开了,三道乌光就在这一刹那间擦着他的衣裳飞过。

魏无牙固然是怪招百出,令人难斗,这轮车中也不时射出一两件暗器来,更令人防不胜防。

但见魏无牙忽而和这轮椅溶为一体,忽而又分开来各自进攻,不到叁十招,小鱼儿觉得吃不消了。

小鱼儿脚步一错,忽然轻瓢瓢拍出两掌。这两掌看来也没有什麽奇妙之处,但也不知怎地,魏无牙竟险些闪避不开,他再也想不到小鱼儿这一招是从那学来的。

更令他想不通的是,小鱼儿的招式竟忽然变了,每一招都变得轻飘飘的,像是一点气力也没有。但每一招发出来,却都是攻向魏无牙自己也想不到的破绽,而且招式看来全无变化,其实却变化无穷。

\hypertarget{ux7b2cux4e00ux767eux96f6ux516dux7ae0-ux96beux4ee5ux6349ux6478}{%
\chapter{第一百零六章
难以捉摸}\label{ux7b2cux4e00ux767eux96f6ux516dux7ae0-ux96beux4ee5ux6349ux6478}}

苏樱本来已经快急疯了,此刻面上却露出了微笑。原来就在小鱼儿最危险的时候,他忽然发现了移花宫主,这姊妹两人竟也在远处过起招来。她们所用的招式一正一反,一攻一守,每一招击出时都很慢,像是生怕别人瞧不清楚。

小鱼儿就算再笨,也知道她们是在传授自己武功了,此时此刻,他就算想拒绝也无法拒绝。

他随意将邀月宫主力才使出的一招拍了出来,果然令魏无牙大吃一惊,等到魏无牙再攻来时,他就以怜星宫主所使的招式来解救。但也不知怎地,十来招过後小鱼儿竟轻轻松松的就占了上风。

等到魏无牙也发觉她们时,已被小鱼儿逼得连气都透不过来,他再也想不通自己如此奇诡的招式,怎会被如比平淡的招式克制住。他却不知移花宫主这种招式,并非平淡,而是简练,她们实已将最繁复的变化加以精淬,将无数个变化化为一个。三十招过後,魏无牙声势已弱,变化已穷。

谁知就在这时,突听``叫''的一声。这声音似乎是山洞外传来的,但回音却震动了整个山窟。小鱼儿一惊,又一喜,魏无牙的轮车已滑开三丈。

一这时山外``咕咚''之声不停的传了进来,怜星宫主目中早已忍不住露出喜色。

魏无牙道:``这既无食物,也无饮水,你们就算有天大的本事,最多也只能维持十天不死,等到外面的人进来时,你们恐怕已剩下一把骨头。''

小鱼儿忽然大声道:``既是如此,我们就非杀你不可了?''

魏无牙道:``不错,杀了我,你们也可免得在我跟前出丑,只不过\ldots\ldots 你们现在杀了我,却未免太可惜了。你们不妨先随我去看几样东西。''

小鱼儿望了移花宫主一眼,道:``好,我就跟你去瞧瞧,反正也不怕你在我面前玩花样。''

魏无牙道:``在移花宫主和天下第一聪明人面前,我还有什麽花样好玩的。''他推动轮车向地道中滑了下去。移花宫主姊妹就像影子般跟着他。

只见魏无牙这时已滑入了一扇很窄的石门一这道石门莫非就是他留下来的秘密出麽?小鱼儿赶紧奔了过去,一走进去,就不禁大失所望,石门後竟是一间六角形的石室,再也没有别的门户。这间石室中光线特别黯,小鱼儿隐隐约约只能看出面有一口很大的石棺,远有许多石像。小鱼儿忍不住问道:``这些石像是什麽玩意儿?''

魏无牙吃吃笑道;``这些全都是我的精心杰作,我去点起灯,让你们看清楚些。''他笑声中一竟带着种说不出的奇怪味道,小鱼儿一听这笑声,就知道这些石像必然有些古怪。

一这时魏无牙已滑到墙角,取出了个火摺子,将嵌在石墙中的十来盏铜灯,一盏盏燃了起来。

他燃起第四盏灯时,小鱼儿已看呆了。

一这些石像竟全都雕成移花宫主姊妹和魏无牙自己的模样,而且都和真人差不多大小,自成一组,每一个的姿态都不同。

第一组石像是移花宫主姊妹两人跪在地上,拉着魏无牙的衣角,在向他苦苦哀求。

第二组石像是魏无牙在用鞭子抽着她们,不但移花宫主姊妹面上的痛苦之色栩栩如生就好像活的一样。

第三组石像是移花宫主姊趴在地上,魏无牙就踏着她们的背脊,手还举个杯子在越到後来,石像的模样就越不堪入目,而每一个石像却又都雕得活灵活现,纤毫毕露小鱼儿忍不住叹了口气,喃喃道:``想不到这疯子竟是个如此伟大的天才。''

移花宫主姊妹早已气得全身发抖了,此刻忽然扑上去,提起个石像,摔得片片粉碎,只见这些坚硬的石像,到了移花宫主的手里,竟有如纸扎的一般,无数件心血的结晶,瞬间便化为一片碎石。

魏无牙却只是在那静静的瞧着,动也不动。怜星宫主终於扑到他面前,怒喝道:``你这畜牲,这次你还想要我放过你麽?''

喝声中,她已拎起了魏无牙的衣襟,将他从轮车上提了起来,向石壁用力掷了出去。

只听``砰''的一声,魏无牙居然摔得粉碎可是一个人的血肉之躯,又怎会被摔成``粉碎''呢?

怜星宫主怔了怔,才发现这个``魏无牙''原来竟也是用石头雕成的,只不过穿着衣服而已。

真的魏无牙竟不知在什麽时侯溜走了。

这石室仅有的一道门已被封闭,四面石壁,也就是山壁,移花宫主用那麽重的石像去摔,石壁也纹风不动,其坚固可想而知。

苏樱默然半晌,道:``他既然已将我们困死,为何还要将我们骗到这来呢?''

小鱼儿苦笑道:``这理由太多了,第一,他将我们困在这裹,他自己就可以自由活动,甚至可以大吃大喝,等我们饿死後,就可以走了。他用的这法子,就叫``置之死地而後生'',一计中还有一计,主要的目的,只怕还是想将我们骗到这里来,在外面说的那些话,做的那些事,全都是在做戏。''

苏樱垂下头,黯然叹息。小鱼儿苦笑着又道:``现在我们就好像是一群关在笼的猴子,只好做把戏给他看了。''

苏樱再也说不出什麽了,过了半晌,小鱼儿又笑了起来,喃喃道:``我临死前会变成什麽样子,宜在连我自己都想像不出,这倒有趣得很。我说不定会将你吃下去,你怕不怕?''

苏樱柔声道:``那麽我们两个就永远变成一个,我怕什麽?''

小鱼儿注视着她的脸,页久良久,才叹息着道;``只可惜你太聪明了些,否则说不定我真的会喜欢你了。''

苏樱红着脸,咬着嘴唇道;``我听说女人生了孩子後,就会变得笨些的。''

若是换了平时,小鱼儿听到这话一定会放声大笑起来,但此刻他只是觉得心襄泛起一阵甜蜜的温柔之意,又带着种说不出的酸楚,他也不知道这究竟是什麽滋味,只知道这种滋味他平生也没有领略过。

也不知过了多久,小鱼儿忽然站了起来,走到那青石棺材前,将棺材盖抬了起来,挡在棺材前面,又将四面的碎石在棺材两旁一块块堆起。

移花宫主也不知他这是在干什麽,两人越瞧越奇怪,虽然忍住不想问,却希望苏樱问他。但苏樱眼睛充满了柔情蜜意,含笑瞧着小鱼儿,也不开口,竟似乎很了解小鱼儿的用意。

只听小鱼儿嘻嘻一笑,道:``吃喝拉撒、睡,乃是一个人五样非做不可的事,现在我们虽没有吃喝,但以前吃喝的东西还是要出来,我们既没法子让它留在肚子,也不能让它拉到裤子上,所以只有用这法子了。''

移花宫主脸都气红了,偏偏又说不出话来。只见小鱼儿已将碎石在棺材两边堆成两道墙,再加上那棺材盖子,就活脱脱是个现成的茅房了他拍了拍手,笑道;``在下一向敬老尊贤,两位若要用,就先请吧。''移花宫主红着脸跺了跺脚,拧转身去。

小鱼儿又瞧着苏樱,笑道:``你呢?''

苏樱脸也红了,道;``我\ldots\ldots 我现在不\ldots\ldots 不想。''

小鱼儿笑道:``既然如此,我就不客气了。''他嘴说着话,人已进去,过了半晌,才慢吞吞走了出来,一面叹着气,一面喃喃道:``舒服舒服,这麽舒服的事世上只怕还没有几样。''

他走回去坐下,闭起眼睛,似乎要睡着了,苏樱终於也忍不住悄悄爬起来,向那边走。谁知她身子刚动,小鱼儿左边一只眼睛忽然张开了,笑嘻嘻道;``你想了麽?''

苏樱红着脸啤道:``你真是个小坏蛋。''

又不知过了多久,怜星宫主的脸潮渐胀红了,再过片刻,她两条腿似乎已在轻轻发抖。只听小鱼儿鼻息沉沈,似已睡着。怜星宫主忽然一阵风似的飘了进去,她就算在和最厉害的对头交手时,也没有用过这麽快的身法。、谁知小鱼儿却忽然``噗哧''一笑,道;``你现在只怕不会再说我无礼,反要感激我了吧。''

小鱼儿笑不出的时候,移花宫主姊妹终於也在地上坐了下来,这只不过是三两天之间的事,但在他们感觉中,却如同十年。就在这时,屋顶上忽然露出饭碗般大小的洞,还有样东西自洞落了下来,掉在地上,竟是个柚子。

苏樱瞧着这柚子,眼睛已发直了,她从末想到一个柚子竟能令她如此动心,只见移花宫主姊妹的眼色,竟也为这一个柚子而改变。怜星宫主眼睛盯着这柚子,已缓缓站了起来。

突听小鱼儿大笑道:``想不到不可一世的移花宫主,如今竟连别人丢在地下的东西也要捡起来吃了,有趣呀有趣。''怜星宫主身子忽然僵住,指尖却已在发抖。但她的眼睛还是盯着那柚子动也不动。

小鱼儿笑道:``但我若捡别人丢在地上的东西吃,却没有人会笑我的,因为我脸皮本来就和城墙差不多厚。''他嘴裹说着话,已跳起来将那柚子搂在手。

只见小鱼儿将柚子劈开两半,带着清香的水汁,溅得他满脸都是,他伸出舌头来舐了舐,喃喃道:``好甜,好香,看来一个人的脸皮厚些,倒不是件坏事。''他忽然转头向苏樱一笑,又道:``但你的脸皮一向也不薄,这柚子也该分一半给你的,是麽?''

苏樱忍不住嫣然一笑,柔声道:``我有时真奇怪,一个人有了张强盗的嘴,却偏偏还有颗善一艮的心。''

小鱼儿将剩下的半边柚子又闻了闻,忽然站起来,走到移花宫主姊妹面前,笑嘻嘻地将半边柚子递出去,道:``这一半已是你们的。我知道你们绝不肯吃别人丢掉的东西,但这半个柚子却是我恭恭敬敬送来的,你们已可放心吃了。''移花宫主面面相觑,竟都怔住。

过了半晌,怜星宫主忍不住道;``你\ldots\ldots 你为什麽要这样做?''

小鱼儿默然半晌,缓缓道;``一个人在快要死的时候,还能保持自己的身份,不肯丢人,这种人连我也很佩服的。''只见小鱼儿笑嘻嘻走了过来,脸上既没有得意之色,也没什麽难受,就好像他刚吃过一百个柚子,才将吃不下的半个送给别人似的。

苏樱将这半个柚子也分成两半,柔声道:``你既然已将这半个柚子送给我,这就是我的,我自然也要送一半给你''

小鱼儿道:``我不要。因为你那一半比我大,我要你那一半。''

苏樱噗哧一笑,道:``我若生个孩子像你,我不被他气死才怪。''

\hypertarget{ux7b2cux4e00ux767eux96f6ux4e03ux7ae0-ux4ebaux6027ux5f31ux70b9}{%
\chapter{第一百零七章
人性弱点}\label{ux7b2cux4e00ux767eux96f6ux4e03ux7ae0-ux4ebaux6027ux5f31ux70b9}}

永远高高在上,令人不可仰视的移花宫主,终於也渐渐变得和别人同样平凡,小鱼儿到这时侯,才觉得她们原来也是个人,也有人的各种需要,也有人的各种情感,甚至也有眼泪。现在,她们会不会将那秘密说出来?

苏樱揉了揉眼睛,悄悄道:``我们现在难道连一点希望都没有了麽?''

小鱼儿默然半晌,也压低语声,道:``我们若能沉得住气,静静的等死,也许还有一丝希望。''

苏樱道:``既然静静的等死,还有什麽希望?''

小鱼儿道:``魏无牙要我们慢慢的死,就是要我们痛苦,疯狂,甚至自相残杀,因为只有这样他才能得到发泄,但我们现在却都很镇静,我们若是就这样静静的死了,他一定不甘心,一定还会有别的举动,那就是我们的机会到了。''

苏樱眨了眨眼睛,道:``所以我们现在一定要想个法子来逼他。''

移花宫主也听不到他们在说什麽,过了半晌,只见小鱼儿忽然站了起来向她们姊妹两人恭恭敬敬行了个礼,然後又长叹一声,道:``我江小鱼能和移花宫主死在一起,葬在一起,总算有缘。现在大家反正都快死了,我们昔日的恩怨,也从此一笔勾消,你们为何定要花无缺杀我,究竟有什麽秘密,我都不想问了。''移花宫主也不知道他为何忽然说出这种话来,只有张大了眼睛瞧着他,等他再接着说下去。

小鱼儿道:``现在花无缺既然不在这里,我们看来也不会有逃出去的希望,我只求你们让我痛痛快快的死了吧。死,我并不怕:但等死却实在令我受不了。''移花宫主姊妹神情骤然沉重下来。

他一面说话,一面偷偷向移花宫主挤了挤眼睛。邀月宫主怔了怔,怜星宫主已悄悄拉了拉她衣襟,道:``好,你死吧。''

苏樱忽然道:``我这裹有两粒毒药,是魏无牙为他徒弟们准备的。''

小鱼儿道:``这种毒药的厉害我知道,只要一粒已足够了。''

苏樱凄然一笑,道:``你死了,我是连一时一刻也活不下去的,你难道还不知道?''

小鱼儿默然半晌,道:``好,要死就一起死吧,也免得黄泉路上寂寞。''

突听一人大声道:``死不得,死不得,你们少年恩爱,多活一天,就有一天的乐趣,若是现在死了,岂非太冤枉了麽?''小鱼儿和苏樱对望一眼,心暗道;``他果然沉不住气了。''

只听魏无牙又道:``你们若是觉得心烦闷,喝几杯酒就会好的,哈哈\ldots\ldots,这就算我送给你们的台沓酒吧。''话声中,上面那小洞中已抛下了一只酒瓶,小鱼儿刚伸手接着,就又有一只酒瓶落了下来。片刻间,小鱼儿怀已抱着十二瓶酒,瓶子还都不小。

小鱼儿将瓶酒放在移花宫主面前,道:``还是老规矩,一人一半。你们若真是素来酒不沾唇,现在更该喝两杯了,一个人若到了临死时还不知道酒的滋味那实在是白活了一辈子。''片刻之间,他自己已经半瓶酒下了肚。

这酒若是十分辛辣,移花宫主姊妹也许还能忍得住不去喝它,但这酒却偏偏是上好的竹叶青,清香芳洌,教人嗅着都舒服,碧沉沉的酒色,更教人看着顺眼,若有人真能忍得住不喝,那才真是怪事。

怜星宫主瞧了邀月宫主一眼,终於忍不住开了酒瓶,浅浅啜了一口。这一口不喝也还罢了,一口喝了下去,但觉一股暖意直下丹田,却又忍不住打了个寒噤。接着,她全身的血液又热了起来,眼睛也亮了一这一口不喝也还罢了,一口喝下去,那还能忍得住不喝第二口?

只见小鱼儿用力敲着酒瓶,引吭高歌道:``君不见,黄河之水天上来,奔流到海不复回,君不见\ldots\ldots{}''这正是李白的千年绝唱``将进酒'',移花宫主虽然也曾念过,却总觉得这不过只是个酒鬼疯言疯语。

但此刻怜星宫主几日酒下了肚,只听了两句,已觉得这首长歌的确是气势磅礴,古来少有。

再等到一曲终了时,怜星宫主已不觉热血奔腾,热泪盈眶,不知不觉间,已将一瓶酒都喝了下去,嘴犹自喃喃道;``五花马,千金裘,呼儿将出换美酒,与尔同消万古愁\ldots\ldots 来,江小鱼我敬你一杯,与你共消这万古愁吧。''

苏樱已不觉看呆了,她想不到怜星宫主竟将一瓶酒喝下去,再想不到她会变成这样子。这实在已不像怜星宫主,就像是另外换了个人似的。

邀月宫主虽也喝了两口,但见她第二瓶酒又喝下去一半,不禁皱眉去夺她酒瓶,道;``你已经醉了,放下酒瓶来。''

怜星宫主忽然叫了起来,道:``我不要你管,我偏要喝!你已经管了我一辈子,现在我已经快死了,你还要管我?''

邀月宫主又惊又怒,但听到她最後一句话,又不禁长长叹息了一声,也喝了口酒,黯然道:``不错,我自己反正也已离死不远,何必再来管你''

怜星宫主这才转过头向小鱼儿一笑,道;``来,我再敬你一杯,你实在是个很可爱的孩子。''

小鱼儿好像并不在意,随问道:``既是如此,你为什麽还要杀我呢?''

邀月宫主面色忽然变了,怜星宫主却只是嘻嘻笑道:``这秘密等你死了之後,我一定会告诉你的。''到了这种时候,她还能忍住不说出这秘密来。

小鱼儿道:``一言为定,可是,;你若比我先死呢?''

怜星宫主道;``那麽你就陪我死吧,我在黄泉路上,一定会告诉你。''

小鱼儿叹道:``能和你一死,倒也算不虚此生了。你以为只有魏无牙一个人为你疯狂麽?

像你这麽可爱的人,我\ldots\ldots 我实在\ldots\ldots''他没有再说下去,却用眼睛盯着她的脸。

怜星宫主眼波流动,忽然指着苏樱道;``我难道比她还可爱麽?''

小鱼儿道:``她怎麽能和你此,你若肯嫁给我,我现在就娶你。''

两人越说越不像话,简直拿别人都当做死的,像是全未看到苏樱的脸已发白,邀月宫主更已气得全身发抖。

只见怜星宫主笑着笑着,人已到了小鱼儿怀里,娇笑道:``我一生都没有这麽样的开心过,我\ldots\ldots{}''邀月宫主不等她说完,已飞身掠了过来。

突听小鱼儿压低声音,悄悄道:``你想不想活着出去,想不想杀了魏无牙出气!''邀月宫主怔了怔,小鱼儿声音更低,道;``你若想,就照我的话做,先扑灭这所有的灯火。''

魏无牙果然一直在外面偷看,他看到怜星宫主扑入小鱼儿怀时,眼珠子都快凸了出来,全身都紧张得在发抖,掌心也在淌着汗。谁知就在这时,灯火竟忽然灭了。

石室中骤然黑暗得伸手不见五指,什麽也看不见。魏无牙几乎急得跳了起来。

只听黑暗中发出各种声音,先是怜星宫主的娇笑,邀月宫主的怒喝,接着又是一阵掌风激荡。黑暗中此刻偏偏连一点声音也没有了,这没有声音实在比什麽声音都要诱惑,都要急人。魏无牙简直要急疯了。他苦心安排了一切,就为的是等着瞧这一幕,为了这件事,他也不知花了多少心血,甚至已牺牲了一切。

但现在他却偏偏什麽也看不到。他疯子似的推动着轮车,去取了盏灯,想将灯光从那小洞中照进去,谁知灯光一移到洞口,就又被打灭了。

只听小鱼儿喘息着笑道:``不准你偷看。''

魏无牙心就像是有一把火在烧,又像是有无数条小在爬来爬去,终於咬了咬牙狞笑道:``你不让我看,我也要看我死也非看不可。''

他算定邀月宫主此刻必已被打倒,怜星宫主和小鱼儿此刻也绝不会有功夫来对付别人了。只剩下个苏樱他自然不放在心上。

他等了几十年,好容易才等到今天,这机会他怎肯错过山於是他又拿了盏灯,扳开了门上的枢纽。沉重的石门,无声无息地滑了开来。

魏无牙简直紧张得连气都透不出了,手在发抖,灯也在抖,他用力推动轮车,无声无息地滑了进去。谁知就在这时,黑暗中忽然爆发起一阵狂笑声。

只听小鱼儿狂笑着道;``魏无牙,你终於也上了我一次当了!''

魏无牙大惊之下,心胆皆丧。灯光映照处,他赫然发现小鱼儿什麽也没有做,正笔直站在他面前,他想後退,邀月宫主却已挡住了那道门户。

小鱼儿笑嘻嘻道:``你栽在天下第一聪明人手,难道还觉得冤枉麽?这若有人为我作传立碑,少不得也会将你带上一笔,你岂非也可名垂千古了。''

魏无牙下一口苦水,嗄声道:``你\ldots\ldots 你现在想要怎麽样!''

小鱼儿沉下了脸,冷笑道:``你现在难道还想要我们相信这的出路已全都被封死?''他嘴说着话,已一步步向魏无牙走了过来,再看邀月宫主,目中已射出刀一般的杀气。

``只不过你是想要我带你们出去麽?那容易得很。''魏无牙笑道:``我现在已经在往外面走了,你难道看不见?''

小鱼儿讶然道:``你现在\ldots\ldots{}''他语声忽然顿住,就像是忽然见到鬼似的,满脸俱是惊惧之色,喉咙格格的,却说不出话来。小鱼儿指着魏无牙,手指不停的发抖。

邀月宫主站在魏无牙身後,也看不到魏无牙的脸。

只听小鱼儿嗄声道:``你\ldots\ldots 你过来\ldots\ldots 过来看看他。''邀月宫主赶紧掠到魏无牙面前,也骇得呆住了。

灯,还在魏无牙手,火焰不停的闪动。闪动的火光下,只见魏无牙一张脸色变成死黑色,眼睛和嘴都紧紧闭着,嘴角和眼角一丝丝的往外面冒着鲜血。

邀月宫主也情不自禁,後退了半步,骇然道:``他难道竟自杀死了。''只见魏无牙扭曲的嘴角,彷佛带着一丝恶毒的微笑。邀月宫主站在那襄,也呆住了。

只见苏樱苍白着脸,走到魏无牙的身前,恭恭敬敬拜了几拜,目中已流下了几滴眼泪。她一逅是在为魏无牙悲哀还是在为自己悲哀突听小鱼儿惊呼一声,道:``不好。''喝声中,他已自那石门中奔了上去。

邀月宫主和苏樱对望了一眼,也不知他又发现了什麽事,但此刻大家已唯小鱼儿马首是瞻,小鱼儿惊呼出声,她们面上也不禁变了颜色。

一这时怜星宫主鼻息沈沉,似已熟睡,原来方才在那一片令人迷乱的恙暗中,邀月宫主已点了她的睡穴。此刻邀月宫主抱起了怜星,随着小鱼儿掠出。

掠出地道,那巨大的洞窟中仍是静悄悄的,并没有发生什麽变化,甚至连四面的灯光都没有熄灭。但小鱼儿站在那,脸上却已看不到一丝血色。

小鱼儿沉着脸道;``你可听到了什麽声音?''

苏樱道:``没有听到呀?''四下静寂得如同坟墓!

小鱼儿长长叹了口气,道:``就因为你什麽声音都听不到,这才可怕。''他话末说完,苏樱也已耸然变色。

花无缺若在外面挖掘地道,就一定会有``叫叫咚咚''的敲石声传进来,但此刻四下静无声音,他显然已住手。他们连最後一线希望都断绝了。

只见苏樱已在一旁坐了下来,用手抱着头,似在苦苦思索。小鱼儿就站在她对面,静静的瞧着她。

小鱼儿痴痴的瞧了半晌,走过去拍了拍她肩头,道;``你在想什麽?''苏樱仰起头嫣然一笑,眼波如雾夜的星光,看来是那麽遥远,那麽蒙胧,美丽得令人不可捉摸。

她轻轻抱着小鱼儿的腿,道:``我在想,魏无牙必定为他自己留下了一条最後的出路,这已是绝无疑问的事,但我们为何找不着呢?''她咬着嘴唇,缓缓接道;``我已在四面都很留意的探查过,这每一条出路的确都被封死了,山壁上假如还有暗门,我也一定能看得出来的。''

小鱼儿忽然笑了笑,道:``这最後一条出路在那,我已经知道了。''

一逼句话说出来,苏樱和邀月宫主几乎都忍不住跳了起来,邀月宫主已风一阵掠到小鱼儿面前动容道:``在那?''

小鱼儿同手指点着道:``那边角落有块凸起的山石,石头下有个比较大的气孔。你们总该看到了吧。''

邀月宫主道:``那气孔虽比别的大些,力圆仍不及一尺,人怎麽能钻得出去?''

小鱼儿长长叹息了一声,道:``我们只知道魏无牙必定会为自己留下最後一条出路,却都忘记了一件事。''

苏樱脸色立刻变了,道:``不错,我们的确都忘了最重要的一件事。''

小鱼儿一字字道:``我们都忘了魏无牙是个畸形的侏懦人士那气孔我们虽无法出入,他却可以钻得出去,他虽然留下了一条出路,我们也只有瞧着乾瞪眠。''

邀月宫主身子一震,几乎再也站立不稳,现在他们所有的希望都已断绝,除了死之外,已无路可走。

\hypertarget{ux7b2cux4e00ux767eux96f6ux516bux7ae0-ux8ba1ux8131ux5371ux56f0}{%
\chapter{第一百零八章
计脱危困}\label{ux7b2cux4e00ux767eux96f6ux516bux7ae0-ux8ba1ux8131ux5371ux56f0}}

她现在也终於知道魏无牙的计划,果然周密,果然绝无漏洞,这计划中最妙的地方,就是他虽留下了出路,别人却无法走得出去,他虽然留下了食物,别人却再也休想吃得到嘴。那是一笼看到都恶心的活老鼠。

邀月宫主只觉两条腿轻飘飘的,已无法支持下去,终於也倒了瓶酒坐下去一口喝了起来。

小鱼儿也抱起个酒子,拉着苏樱走了出去,苏樱心中虽也充满了悲忿与绝望,却又充满了柔情蜜意。

谁知小鱼儿刚走了两步,忽然失声道:``槽了!方才,我们还有希望,所以大家也只有一条心都想逃出去,正如风雨共舟,自然齐心协力,但现在所有的希望都已断绝,她就不会放过我了。''话刚说完,跟前人影闪动,邀月宫主已到了他们面前,小鱼儿苦笑着瞧了瞧苏樱,喃喃道:``我猜的不错吧\ldots\ldots 有时我真希望自己也能猜错几件事才好。''

只听邀月宫主冷冷道:``你们的话已说完了麽我再给你们片刻时间,你们快说吧。''

只听小鱼儿忽然大笑道:``好,我们迟早总要拚个死活的,但你既说了要让我们再说几句话,你就不能像魏无牙一样在旁边偷听。''

他拉着苏樱走到角落,嘀嘀咭咭说了几句话,一面说,苏樱一面点头,到最後才听得小鱼儿道;``你明白了麽?''

苏樱黯然道:``我明白了,但你\ldots 你也得千万小心呀''

邀月宫主冷笑道:``再小心也没有用的,过来吧。''

小鱼儿笑嘻嘻道:``你要杀我,你为什麽自己不过来?''邀月宫主脸上又气得变了颜色,谁知小鱼儿这句话刚说完,身子已凌空扑起,闪电般攻出三掌。

这三掌当真是凌厉无匹,强劲绝伦,武林中只怕已极少有人能逃得过他这``杀手三招''。但在邀月宫主眼,却看得有如儿戏一般,她身子似乎全末动弹,小鱼儿这三掌竟连她的衣角都沾不到。

苏樱只瞧了一眼,已知道小鱼儿绝非邀月宫主的敌手了,她似乎不忍再看,竟垂着头走了出去他果然越打越起劲,果然丝毫没有畏怯之意,每一招使出,都带着虎虎的风声,可见是已用出了十成劲力。但无论他用出多麽厉害的招式,邀月宫主只要轻轻的一挥手,就将他的攻势化解於无形。

奇招连变,直到此刻为止,她既没有使出``移花接玉''的功夫来,也没有使出一着杀手。

小鱼儿眨了眨眼睛,忽又笑道:``你究竟是想杀我?还是在跟我闹着玩的?''他不等邀月宫主说,又笑着道:``你是不是想等到摸清我使力的方法之後,才要我死?''

邀月宫主微微动容,皱眉道:``我为什麽要摸清你使力的方法?''

小鱼儿道:``因为你若摸不清我力量发出的方向,就使不出!移花接玉』的功夫来,是不是?''他的嘴在不停的说着话,手也在不停的挥动攻击,但一双眼睛,却始终瞬也不瞬的瞪着邀月宫主。

邀月宫主面上的神情果然又有了变化,却冷冷道:``我要用!移花接玉的功夫时,自然会用的,用不着你着急。''

小鱼儿大笑道:``你也用不着再骗我了,我早已看破了你那!移花接玉』的秘密,你要不要我说结你听听?''

邀月宫主冷笑道:``就凭你,只怕还不配说起``移花接玉』这四个字。''

小鱼儿道:``我为什麽不配?!移花接玉』又有什麽了不起,那只不过也是种借方使力的功夫罢了,和武当的四两拨千斤'',少林的!沾衣十八跌』也差不了多少,只不过因为你的出手特别快,而且能在对力力量还末充分使出来之前,就抢了先机先将他的力量拨回去,所以在别人眼中看来,就变得分外神奇,再加上你们自己故作神奇,故弄玄虚,将本来很简单的一件事,故意渲染得十分复杂,十分神秘,所以别人就更认为这种功夫了不起了。''

他滔滔不绝,说到这,才歇了口气。邀月宫生面上已露出惊讶之色,厉声道:``你还知道什麽!''

小鱼兄道:``我虽然还不知道你是用什麽手法将别人经脉中的真气拨回去的,但这也无关紧要,因为我已知道了你这种功夫最大的关键,就是要先摸清对力的真气是从什麽地方,什麽方向发出来的!''

邀月宫主道:``哼。''

小鱼兄道:``因为普通一般人的力量,大多是发自丹田附近几处穴道,所以你不费什麽事,就可以将他的力道摸清,但是我\ldots\ldots{}''

他大笑着接道:``我学的武功却和任何人都不同,我的师傅至少也有七、八十个,甚至连你自己也是其中之一,就因为我学的武功太杂,所以内功也不佳,说来是我最大的缺点,但和你动手时,这反而帮了我的大忙了。''

邀月宫主道:``你以为\ldots\ldots{}''她只说了三个字,就又顿住了语声。

小鱼儿道:``就因为我的内功不佳,出手又没有规矩,所以你一时间竟摸不清我内力发出的方向,就根本使不出``移花接玉』的功夫来。''

邀月宫主一声冷笑中,她纤纤十指,已向小鱼儿``曲泽''``天泉''两穴之间点了过去,手势如采花拂柳。

这两处穴道属``手厥阴经'',小鱼儿此刻攻出两招,力道正是由此而发,显然她已摸清了小鱼儿真气流动的方位。

谁知小鱼儿身形一转,转开叁尺,连一点事也没有。这百发百中万无一失的``移花接玉''功使到小鱼儿身上,竟变得一点用也没有了。

邀月宫主这才真的吃了一惊,她既已看准了小鱼儿出手的力道发自``手厥阴经'',那就万万不会错的。

只听小鱼儿大笑道:``你想不到吧,告诉你,你以为我那两招用了很大力气,其实我却是一点力气也没有用,你想借我的力气打我自己,但根本连一点力气也没有,这就是我对付``移花接玉''功的法子,你说这法子好不好?''

邀月宫主变了变颜色,冷笑道:``很好,也亏你想得出这麽笨的法子来。你出手若不用力气,就根本无法伤人,自己实已立於不胜之地,两人交手,若根本无法求胜,难道远不算笨麽?''

小鱼,了黜头,笑嘻嘻道:``不错,我自己也觉得这法子的确很笨,但对付你这样的人,有时越笨法子,往往会越有用,何况,是你想杀我,我根本就不想杀你,我只要能令你伤不了我,就已经很满意了。''

邀月宫主厉声道;``我不用!移花接玉』的功夫,难道就杀不了你麽?''

小鱼儿道:``我正是想瞧瞧你倒底还有什麽本事能杀得了我!''

他话还末说完,已觉得有一股劲气面而来,接着,邀月宫主的一双手就彷佛已化为七、八双手了。小鱼儿只觉得跟前到处都是邀月宫主的掌影,也分不清那只是卖,那只是虚,更不知道如何招架闪避。

他宜在想不到一个人的手动作怎会这麽快。他虽然勉强躲过了几招,但连他自己也不知道邀月宫主下一招攻出时,他是否还能躲得开了。

她只差还末使出最後致命的一击!突听小鱼儿大喝:``等一等,我还有最後一句话要说。''

邀月宫井根本不理他,闪电的出手,但一招使出後,却又忽然顿住,只不过手掌仍不离小鱼儿方寸之间,目光始终不离小鱼儿面目,冷冷道:``此时此刻,你还想玩什麽花样?''

小鱼儿叹道:``现在你总也该知道,无论如何,我都再也逃不了的,也绝不会再有人来救我,我已没怯子不死在你手。那麽,到了这种时侯,你总该将那秘密告诉我了吧。''

他满脸都是渴望企求之色,看来真是说不出的可怜,谁也想不到小鱼儿竟也会露出这样的可羊怜像。邀月宫主瞧着他,许久没有说话。

邀月宫主忽然道:``你死了之後,我一定将这秘密告诉苏樱。''

小鱼儿嗄声道:``你\ldots\ldots 难道就不可告诉我吗?''

邀月宫主道:``不能!''这回答又变得和以前同样坚决,全无商量的馀地。

小鱼儿长叹一口气,道:``你这人真比强盗还凶,连我临死前最後一个要求都不肯答应。我若要求别的事,你肯不肯答应呢?''

邀月宫主犹疑了半晌,终於缓缓道:``那也要看你要求的什麽事。''

小鱼儿道:``我要小便,行不行?''

在这种时候,他居然提出这种要求来,宜在令人哭笑不得,邀月宫主苍白的脸都似乎被气得发红。

小鱼儿道;``我方才酒喝得太多,现在已憋不住了,你若还不肯答应我!我只好在这就地解决了。''

邀月宫主怒道;``我现在就杀了你?''邀月宫主咬着牙瞪了他半晌,忽也冷笑道;``好,你去吧,我就不信你现在还可玩得出什麽花样。''

小鱼儿道:``这地方就是死路一条,我难道还会七十二变,能变个苍蝇飞出去麽!''

他又回到方才那地室,只见魏无牙的身已渐渐开始乾瘪缩小,那模样看来更是令人作呕。

小鱼儿眨了眨眼睛,道:``你不进来?难道不怕我跑了麽?''

邀月宫主也不理他,这地室只有这一个出口,她自然知道小鱼儿就算有多大的本事,也无路可逃的。

过了半晌,只听面``哗啦哗啦''的响了起来,邀月宫主这一辈子几曾听过这种``可怕''的声音。她的脸不禁又红了,只恨不得紧累堵住耳朵,幸好任何人小便都不会太长的,她忍耐最多也只不过是片刻间的事。

谁知过了半天,那声音还在``哗啦哗啦''的响着。又过了两叁盏茶功夫,那声音还在个不停。

邀月宫主越等越不耐烦,越等越奇怪。邀月宫主忍不住道;``江小鱼,你为何还不出来?''

里面却只有``流水''的声音,竟没有人答话。

邀月宫主虽然明知小鱼儿无路可逃,还是不免有些惊疑,又呼唤了两声,听不到回答,就不禁暗暗忖道:``这鬼灵精难道真的找到了另一条出路他已知道在此,所以才使出这诡计自己逃出去,却将我们困死在这里!''想到这,她手足都已冰冷,再也顾不得别的事,冲了进去。

不,这时并没有什麽变化,那声音还是在``哗啦哗啦''的,只不过有``墙''挡住视线,也看不出小鱼儿是否还在面。邀月宫主一冲进去,就挥手发出一股真气。

只听``哥''的一声,那以碎石和棺材盖隔成的三面墙,就都已被震倒,面果然没有小鱼儿的影子。

只有几只酒瓶,被人用布带困在一齐,从上面那气穴襄吊下来,吊在半空中,瓶底都被开了个小洞。瓶的酒,就都流入那棺材,响个不停。

邀月宫主一鹫之下,眼角忽然瞥见有条人影窜了出去。原来小鱼儿一直躲在那道门的後面,邀月宫主的注意力全被那边吸引住时,他就一溜烟窜了出去。邀月宫主发现他时,他已溜到门外。

等到邀月宫主想追出去时,那石门已无声无息的阖了起来,连小鱼儿的大笑声都被隔断。邀月宫主这才真的吓呆了。

她平生无论遇着什麽事,从来也没有鹫呼出声,更没有哀求过别人,但此刻她却忍不住大呼道:``江小鱼,开门,让我出去。''

过了半晌,小鱼儿的声音就自上面那气穴中传了下来。只听他笑嘻嘻道:``让你出来我难道会让你出来杀我麽?''

邀月宫主咬着嘴唇,道:``我\ldots\ldots 答应绝不杀你就是''

小鱼儿已大声道:``你就算不杀我,我也不会放你出来的,只因你不杀我,我却要杀你,你莫忘了我和你之间的仇恨并不小。''邀月宫主心里一震,再也无话可说。

\hypertarget{ux7b2cux4e00ux767eux96f6ux4e5dux7ae0-ux660eux7389ux795eux529f}{%
\chapter{第一百零九章
明玉神功}\label{ux7b2cux4e00ux767eux96f6ux4e5dux7ae0-ux660eux7389ux795eux529f}}

邀月宫主几乎连头都已垂了下去。

忽听小鱼儿道:``我并不是真的想让你死得这麽惨的,只要你肯答应我一件事,我立刻就让你出来。''

邀月宫主脱口道:``什麽事?''这句话她说出,已知道小鱼儿要她答应的是什麽事了。

小鱼儿果然道:``只要你说出那秘密,我就立刻放了你。''

邀月宫主叹息道;``你\ldots\ldots 你休想\ldots\ldots{}''

小鱼儿道:``你难道情愿同魏无牙死在一麽?以後若是有人到这里来,发现你们同穴而死又会有什麽想法''他笑着接道;``那时别人一定要说,邀月宫主看来虽然冷若冰霜、高不可攀,其穴却也有个秘密的情郎,两人竟到这种地方来幽会,而且,:''

他一笑顿住语声,故意不再说下去。邀月宫主身子早已在发抖。

小鱼儿道;``你不妨再考虑考虑吧,你什麽时候说出来,我就什麽时候放你,反正我听了这秘密後,也活不长的。''

邀月宫主没有说话她至少已不再拒绝了一直伴在小鱼儿身旁的苏樱却叹息了一声,道``到了这种时候,你为什麽一定要逼她说出那秘密来呢?她说出来之後,於你又有什麽好处那只不过使你更添些烦恼而已''

小鱼儿且不回答,却反问道:``你总该也知道,我和花无缺之间,总有一个人要死在对力手上不是他杀死我,就是我杀死他。但我却不相信世上真有命中注定的事,我一定要想法子将它改;,所以我只有逼她说出这秘密来,我若知道她为何一定要我们拚命,我就有法子解决。''

苏樱黯然道;``可是\ldots\ldots 可是现在你们的命运岂非已经改变了麽!现在,你既无法杀他,他更法杀你,只因你\ldots\ldots 你已将死在这里。''

小鱼儿道:``谁说我一定要死在这里?我这人天生福气不错,无论遇着什麽危险,到时候逢凶化吉,我可以跟你打赌,一定会有人来救我的。''

苏樱默然半晌,道:``本来花无缺是一定会想法子来救你的,但现在,他自己也不知道遇到什麽意外了,否则他绝不会停手的。''

小鱼儿拘掌笑道:``不错,他最可能遇见的人,就是李大嘴他们了,因为他们在这里有个约会,这两天一定会来的。''

苏樱说道:``那麽,你以为他们会想法子进来救你麽?''

小鱼儿苦笑道:``当然不会,我现在也知道他们总以为我会和别人勾结,来对付他们,所以就巴不得我早些死了才好。但他们总以为有一批珠宝被魏无牙藏了起来,若不进来绝不死心,我算准他们不出一天就会进来。''

苏樱道:``他们有法子能进得来麽?''

小鱼儿道:``凭他们那几个人的本事,这里就算是铜墙铁壁,他们也有法子能进来的。''

苏樱终於展颜一笑,道:``我只望你这次莫要猜错才好。''话末说完,外面响起了``叫叫咚咚''的开山声。

小鱼儿拘掌大笑道;``你现在总该相信我的本事了吧。''

邀月宫主激动的情绪似已惭渐静了下来,正在静静的闭目调息,且已渐渐进入了物我两忘的状态。

小鱼儿道;``看来现在我只有告诉她,花无缺已经快进来了。''

苏樱眼睛一亮,道:``不错,我们先告诉她花无缺已经快进来,再告诉她,她若不肯说出那秘密,我们就将这地方封死,我想,她就算将这秘密看得十分重要,也绝不会将它看得比自己性命更重要的。''她的话声还末消失,身後忽然响起了另一个人的声音。

只听怜星宫主一字字道:``你错了,她实在将这秘密看得比性命还重要得多。''这声音虽然十分缓慢,十分平和,但听在小鱼儿和苏樱耳里,却简直好像半空中忽然打下个霹雳灯光下怜星宫主的脸色苍白如纸。怜星宫主继续道:``也许我永远莫要醒过来反倒好些。''

她神色仍是一片迷惘,似乎连自己在说什麽都不知道。

小鱼儿眼珠子一转,忽然笑道:``看样子你好像很难受,其实,喝醉酒也不是什麽丢人的事,这世上每天至少有几十万人喝醉酒的,你何必难受呢你以为自己做仕了什麽事你喝醉後立刻就睡着了,只不过说了几句梦话,像是做了个梦而已。''

怜星宫主顿时吐出气,眼睛里渐渐有了光辉,苍白的脸上也渐渐有了神采,喃喃道:``不错,我的确做了个梦,而且是个很奇怪的梦。''

苏樱瞧着他,目光充满了赞赏之意,像是深深以他为骄傲每个少女都希望自己的情人慷慨、热情而仁慈。小鱼儿为了求生,虽然也做出过一些不择手段的事,但却有一颗对人类充满了热爱的仁慈的心。

过了半晌,怜星宫主才缓缓道:``现在她已不能杀你了,你放了她吧。''她说这句话时的口气很奇怪,非但丝毫没有勉强之意,而且竟像是个局外人在劝解似的。

小鱼儿瞧了她两眼,什麽话也没有说,就拉着苏樱,走到那机关枢纽的所在之地,怜星宫主一竟没有跟来。

他们忍不住要下去瞧瞧,但他却再也末想到邀月宫主竟真的留在那石室中没有出来,而且反而已靠着石壁坐下。怜星宫主正远远站在一旁,出神的瞧着她,面上的神情看来既有些鹫奇,又有些欣羡,甚至还有些妒忌。

小鱼儿越看越觉得奇怪,怜星宫主的表情虽奇怪,邀月宫主的脸色却更奇怪,她一张脸非红非白,竟已变成透明的。灯光映照下,她肌肉里的每一恨筋络,每一恨骨头都彷佛能看得清清楚楚,这一张绝顶美丽的脸,竟变得说不出的诡秘可怕。

苏樱骇然道:``这是怎麽回事,难道她已经,已经走火入魔了?''小鱼儿摇摇头,还没说话,怜星宫主已悄悄退了出来,站在那里痴痴的出神,也不知在想些什麽。苏樱和小鱼儿就在她对面,她也像是没有瞧见。

小鱼儿不住搭讪着道;``一个人的脸会变成透明的,这倒也少见得很,这难道也是你们练的功夫麽?''

他见到怜星宫主如此模样,以为她绝不会回答这句话的,谁知怜星宫主虽然还是没有望他一眠,却缓缓道:``不错,!明玉功』练到最後一层,就会有这种现象。''

小鱼儿试探着问道;``那麽,这种功夫一定很厉害了?''

怜星宫主道:``这种功夫共分九层,只要能使到第六层,已可与当代第一流高手一争长短,若能使到第八层,就可无敌於天下。二十年前,我们已练到第八层了,本来要将这功夫练到第八层,至少也得要花三十二年苦功,但我们却只练了二十四年,这进境实已超越古人,我们以为最多再过四、五年,就可练至颠峰。''

小鱼儿知道她谈锋已被引起,就不再开口,只是静静等着她说下去,过了半晌,怜星宫主果然又叹息着接道:``谁知这二十年来,我们的功夫竟一直没有进境,竟似已只能到此为止,再也无法更上一层楼。''

苏樱又忍不住的道;``但你们\ldots\ldots 你们为什麽练不成呢''

怜星宫主凝注着小鱼儿,许久没有说话,像是在考虑是否应该回答他这句话,小鱼儿也只有沉住气等着。又过了很久,怜星宫主终於长叹了一声,缓缓道:``这乃因前二十四年,我们练功的时侯心无旁骛,但到了後二十年,我们却也像凡俗中人一样,也有了烦恼和病苦,再也无法像以前那麽专心一意了。''

小鱼儿默然半晌,喃喃道:``二十年前?\ldots:二十年前\ldots\ldots{}''他仍然停住了话声,怜星宫主的脸色渐渐又变得苍白,只因她发现小鱼儿已猜出二十年前令她们烦恼和痛苦的是什麽事了二十年前,岂非就是她们第一眼瞧见江枫的时候。

苏樱忽然道:``现在\ldots\ldots 现在邀月宫主莫非已练到第九层了麽?''

怜星宫主道:``不错。''她目中又露出一丝羡慕和妒忌之色,幽幽道:``我买在想不到她苦练二十年不成,居然能在这种时候,这种地方练成了,我\ldots\ldots 我实在为她高兴。''

小鱼儿咬了咬嘴唇,笑道:``这只怕是因为我帮了她的忙。''

怜星宫主叹道;``只怕正是如此,因为她被你困在那地方之後,才真的断绝了生机,到了这种时候,人的思想往往会有意想不到的变化,也许在一刹那间,她便已豁然贯通了,她自己只怕也想不通会有这种意外的收获。''

外面的开山声还在不停的响着。小鱼儿耳里听得这``叫叫咚咚''的声音,心里也不知是什麽滋味,邀月宫主若已真的天下无敌,此番出去後,他的日子只怕更难过了。

谁知就在这时,开山声竟突又停顿下来。苏樱和怜星宫主不禁为之耸然失色,忍耐着等了很久,只望这声音会再度响起。但她们却失望了。

过了一天,外面还是连丝毫动静也没有,这一天简直比一万年还长。这次连小鱼儿也无法猜得出,能令十大恶人住手的实在不多。现在他们根本已毫无希望。

\hypertarget{ux7b2cux4e00ux767eux4e00ux5341ux7ae0-ux6076ux4ebaux6076ux8ba1}{%
\chapter{第一百一十章
恶人恶计}\label{ux7b2cux4e00ux767eux4e00ux5341ux7ae0-ux6076ux4ebaux6076ux8ba1}}

花无缺并没有找到铁心兰。铁心兰竟忽然神秘地消失了。

以花无缺的轻功,无论铁心兰往那里走,他都必然能追得到,但他寻遍了整个龟山,都找不到铁心兰的影子。等他失望地回去时,魏无牙的洞穴已被封闭。

一这变化实在令花无缺吃惊得不知所措,他狂呼大喊,也没有人回答,移花宫主和小鱼儿显然已被封锁在这洞穴中,否则绝不会不告而去,花无缺只觉手足发麻,竟不知该如何是好。

等他自半山的樵子手中借来一柄铁锹和一柄斧头的时候,日色已渐渐西沉,夕阳晚照,晚霞如血\ldots\ldots 他用尽全身力气,动手开山,开始时,山石在他铁锹下似乎十分脆弱,但後来却越变越坚硬,坚硬如铁。

他知道气力也已渐渐不支了,但他却不能停下来,他也不知道洞穴中究竟发生了什麽事,他简直要发疯。这时暮诂苍茫,夜色已临,苍茫的暮色,忽然冉冉出现了一条人影,她也不说话,只是静静的站在那里,痴痴的望着花无缺。花无缺虽然没有听到她的声音,但本能上却似已觉察出什麽,缓缓停住了手,很快的转过身。

然後,他也就像这人影一样怔在那里,不会动了,他再也想不到此刻站在他面前的人,竟是他苦寻不着的铁心兰。在他满山遍路的去追寻铁心兰时,他的思潮正也就像他的脚步一样,始终都没有停下来过。

他想起许多许多话,要对铁心兰说。但此刻,他已面对铁心兰,他反而连一句话都说不出了。铁心兰也没有说什麽,甚至连目光都不敢接触他,却悄悄垂下了头,垂头弄着被风吹起的衣角。

``你\ldots\ldots 你方才到那里去了''

铁心茴头垂得更低,道:``我什麽地方都没有去,我一直都在这里。''花无缺嘴角动了动,像是想笑,却没有笑出来。

於是他也垂下头,道:``原来你根本就没有走远,难怪我找不到你了\ldots\ldots{}''

绒心靥眨了眨眼睛,道:``你方才见到了魏无牙麽?''

花无缺道:``我没有见到,里面一个人也没有,但我以为魏无牙一定躲起来了,乘他们没有防备时,将出路全郡封死。''

铁心兰垂头笑了笑,道:``看来现在你的疑心病也不小。''花无缺也不禁垂下头一笑,这才发现自己还是握着铁心兰的手,他的心一跳,立刻就想将手松开。

谁知铁心兰有意无意间,竟也握起了他的手,道:``这山洞被你师傅封死的,她似乎不愿意别人再进去,我只恨\ldots\ldots 只恨方才为何不进去看看。''花无缺只觉自己的心跳得很厉害,长长呼了口气,勉强笑道;``其实那里面也没有什麽好看的。''

铁心兰道:``听说魏无牙一生最喜欢搜集奇珍异宝,有许多东西都是世上很少能见到的,你难道也没有瞧见麽?''

花无缺道:``我什麽都没有瞧见,也许他把东西全带走了。''

铁心兰道:``也许你根本没有注意。''

花无缺还想说什麽,忽然发现她的目光变得很奇怪。她的眼睛本来清澈而纯净的,只不过这些子来,又添了些忧郁的神色,令人见了心碎。但现在,她的眼睛竟变得彷佛鹰车般锐利,狐狸般诡谲,而且远带着种令人毛骨悚然的邪气。

在夜色中看来,她的身材体态,她的神情面貌,都和铁心兰一般无二,只有这双眼睛,,\ldots,这双眼睛无论如何也不会是铁心兰的。花无缺只觉心里一寒,就想後退。但这时已经太迟了!

花无缺只觉掌心一麻,接着,麻木就传遍了四肢。他拚尽最後一丝力量,反手切了过去,可是这``铁心兰''的身子已像风一般退了两三丈。他再想追过去,手脚已无法动弹。

只听``铁心兰''笑道:``花无缺呀花无缺,看来你比小鱼儿还差得多哩,要是小鱼儿,我说不到三句话他只怕就看出我来了。''

花无缺心念闪动,突想起了``不男不女''屠娇娇这名字,但此刻他连站都站不住了,一句话尚未说出,人已倒了下去。

只听一人冷笑道:``你也用不着太得意,依我看来,你那点易容术也稀松得很,到最後还不是被人家看破了麽?''

屠娇娇笑道:``不错,他到最後是看出来了,但那也只不过是因为我没有时间多学学铁心兰的样子,我总共也不过只将她研究了半个时辰而已,只要能给我半天功夫,就算白天,这小子也末必能瞧得出我来。''

花无缺已隐隐约约猜出这几人是谁了,也知道自己此番落在这几人手里,简直有如肥羊到了屠场。但他并没有为自己的处境担心,因为他知道移花宫主和铁心兰他们的处境,一定比他还要险恶得多。

李大嘴大笑着走过来,将花无缺上上下下,从头到脚,都仔仔绌细瞧了一遍,嘴里``啧啧''

连声,喃喃道;``好,好,简直太好了,这麽好的肉,十万人中也末见得有一个,只不过稍微嫌瘦了一点点而已,若是红烧,油就太少了。''

他嘴里说着话,口水似乎要流了下来,一面已伸出手,像是要去捏花无缺的肚子,就像是老太婆上菜市场买鸡似的。花无缺又急又怒,却又偏偏无法阻止。杜杀忽然出声道:``住手!''

李大嘴的手缩回去一半,笑道:``我现在又不宰他,只不过捏一把有什麽关系?』杜杀冷冷道:``此人不失为当世之英雄,我虽不能以武功胜他,至少也该以礼相待,你杀了他倒无妨,却不能羞侮於他!''

花无缺直到此刻才听到句人话,忍不住长长叹了气,道:``多谢。''

花无缺默然半晌沉声道:``在下既已落在各位手中,便已将生死置之度外,``尊敬』两字更不敢奢望,只不过铁心兰\ldots\ldots{}''他眼睛盯着杜杀一字字道:``铁心兰是否也落在各位手里了?''他不问别人,只问杜杀,因为他已看出这五个人中,唯有这满面杀气的人是不会说假话的。

杜杀果然道:``是?''

花无缺还是不理别人,只盯着杜杀,道:``阁下若肯放了她,在下死而无怨。''

杜杀道;``我不妨告诉你,她父亲本是我的八拜之交,我怎会难为她。铁战虽也名列``十大恶人』,但除了性情狂傲外,若论他的所做所为,和他那把硬骨头,绝不会在那些自命侠义的角色之下,\ldots;''

花无缺长叹了一声,道:``阁下既如此说,我就放心了,只想再请教阁下,家师\ldots\ldots{}''他刚说了两句,屠娇娇已笑道:``这件事你也该放心了,她们都被魏无牙困死在这山洞里,除非有什麽人能从日莲和谷那里借来柄开山巨斧,否则他们这辈子也休想出得来。''

花无缺全身发冷,道:``这话可是真的?''杜杀沈声道:``我并未见到他们出来。''花无缺闭起眼睛,不再说话。

\hypertarget{ux7b2cux4e00ux767eux5341ux4e00ux7ae0-ux5947ux5f02ux8d4cux573a}{%
\chapter{第一百十一章
奇异赌场}\label{ux7b2cux4e00ux767eux5341ux4e00ux7ae0-ux5947ux5f02ux8d4cux573a}}

屠娇娇道:``魏无牙既能将她们困在里面,必定早已计划周详,那山洞里就绝不会有任何吃喝的东西留下来。''

李大嘴道;``不错,魏无牙一定早已算准了要将她们饿死在里面。''

屠娇娇道:``但你又能饿多久呢?''

李大嘴眼睛一亮,道;``光只是没有东西吃,我至少还可以挨十半个月,但没有水喝,两天都受不了的。''

屠娇娇笑道:``正是如此,无论多麽强的人,光是两天没水喝,得要躺下去,移花宫主就算比别人都强些,也必定挨不过三天、''

哈哈儿拘掌道:``哈哈,是叨,我们为何不能等上个三五天後再进去呢?''

话末说完,白开心已一个斤斗自树林翻了出来,大笑道:``是呀,我们为何不能等叁天後再进去取,哈哈,屠娇娇呀屠娇娇,你实在比我想像中还要聪明得多。''

花无缺虽闭着眼睛,耳朵却没有闭着,这些话听入他耳里,他的心已不觉沉了下去,彷佛已沉入万劫不复的无底深渊里。

只听屠娇娇道:``现在大家既已决定留在这里不走,就有几件事要做了。''

白开心道:``不错,咱们既已决定留在这里,就该将那两个妞儿也带到这里来,那个半人半鬼的怪物虽然答应在那边看着她们,我还是有些不放心。''

屠娇娇道:``正是如此,那两位姑娘我说不定还用得着她们,所以,哈哈儿,就烦你去将她们带到这里来吧。''

白开心``哼''了一声,道:``那麽我呢?你要我去干什麽?''

屠娇娇道:``你去找一些吃喝的东西来,最少也要够咱们叁天吃的。''

李大嘴跳了起来,道:``你为何要他去?这小子根本就不懂得吃,啃个冷馒头就可以过一天了,他弄固来的东西,只怕连狗都不闻。''

屠娇娇笑道:``不错,色鬼大多不讲究吃的,但总也比要你去好,你先去弄条肥肥胖胖的烤人回来,咱们就只好饿肚子了。山下的小镇里,好像有家铁器,你到那里去弄几件开山的家伙来,依我看,要想将这山洞打通,只怕还不是件容易事。''

哈哈儿道:``哈哈,若是容易,移花宫主她们岂非早就打出来了。''

三个人分头而去,最先回来的是哈哈儿。他拉着一匹骡子,骡子拉着一块大石头。

花无缺正满心焦急地等着铁心兰,哈哈儿却只不过带回一匹骡子来,花无缺既是惊奇,又是失望。

就在这时,更奇怪的事发生了一这块石头中,竟忽然发出一种很奇异的叫吟声,还夹着吃吃的笑声。

花无缺畿乎不相信自己的眼睛,更不相信自己的耳朵。屠娇娇瞟了他一眼,忽然道;``你可瞧见了这块石头麽这是一块魔石,它会吃人,所以又叫做吃人石,你那位铁姑娘就被它吃进肚子里去了。''

花无缺咬着牙,忍耐着不说话。花无缺心里就算一万个不信,但眼睛还是忍不住要往那边看。他眼睛虽在看着,心里还是一万个不相信。

谁知屠娇娇一扬手,那块石头竟真的开了。石头中竟真的有两个人。竟赫然是那白夫人和铁心兰。

此时此刻,此情此景,花无缺倒买的吃了一惊,但哈哈儿和屠娇娇都已一齐拍手大笑起来。

花无缺也终於发现,这块石头原来是用帆布架起的,然後又将真苔一块块的粘在帆布上。制怍得本来已可乱真,再加上夜色如此黝黯,所以花无缺的目光纵敏锐,一时间也末看清。

揭开帆布,里面竟是个精钢铸成的架子,就像是个铁笼,白夫人和铁心兰就被关在这铁笼里。铁心兰曲在角落里,只手掩盖着脸,彷佛既不愿让人看到她,她也不愿意看到任何人。白夫人的身子却几乎是完全赤裸着的,而且不停的在扭动着,不停的在笑,又不停的在叫吟。

花无缺只看了一眼,就闭起眼睛不忍再看。他既不忍看到铁心兰的模样,也不忍看到白夫人的模样,铁心兰令他伤心,白夫人却实在令他觉得有些呕心。

屠娇娇悠然笑道:``铁心兰,铁姑娘,你可知道我们是在对谁说话麽''铁心兰还是以手蒙着脸,不肯抬头。

哈哈儿道:``你为什麽不张开眼睛来瞧瞧呢,我保证你只要张开眼睛,准会吓一跳。''

花无缺只望铁心兰莫要张开眠睛来,莫要看到他此剧的模样,他永远不愿铁心兰为了他伤心。但铁心兰的手已滑落,头已抬起。

她身子立刻颤抖起来。她冲过来,手抓着铁栅,目光充满了悲痛与绝望,她并没有呼号呐喊,但她的眼色却更令人心碎。花无缺闭起眼睛,只望大地忽然裂开,将他永远吞没。

就在这时,白开心已回来了。

他带回了两大包东西,不停地在喘着气,嘴里喃喃道:``我居然会辛辛苦苦去为你们找东西来,这简直连我自己都不相信。''

杜杀道:``李大嘴呢?为何还不回来?你没有和他一到那小镇去?''白开心叫了起来,道:``我怎麽会和那大嘴狼走一条路,他若能上西天,我宁可下地狱。''

屠娇娇道;``那麽,这些吃的东西你是从那里找来的?''

白开心道;``就在山脚的那庙里。你难道以为庙里的和尚都是吃素麽?告诉你,你的运气不错我找的这间庙,是个酒肉和尚开的。连老板带伙计都不吃一两肉,,,,;他们要吃就一斤一斤的吃''

他自麻袋中摸出块肉大嚼起来,喃喃又道:``嘴是用来吃东西的,不是用来骂人的,谁若用错了地方,倒楣的是他自己。''

笼子里的白夫人忽然跳了起来,瞪着那两只麻袋。她身已布满了一条条伤痕,有的是鞭子抽出来的,有的是她自己抓的,她实在已被折磨得不像个人,已完全没有人的尊严。就连她的目光看来都已像是只野兽。

屠娇娇拿出个馒头,道:``你也想吃麽抱歉得很,我却非要你们挨饿不可。''

白夫人没有说话,只因她身上的奇痒又发怍了。

杜杀皱眉道:``你为同要他们挨饿!''

屠娇娇微笑道:``只因我要拿她们做个试验,看她们饿到什麽时侯才没有力气,到了那时,我们就可以开始挖洞了。''

最後回来的是李大嘴。他回来的时候,天已经完全亮了。他奔驰了一夜,非但丝毫没有疲倦之意,反而显得很兴奋。

白开心撇着嘴,冷笑道:``你们瞧瞧他得意的模样,就活像牛魔王吃到了唐僧肉。''

屠娇娇抢着道:``你莫听他放屁,快说说你遇见了什麽奇怪的事吧。''

杜杀冷冷道:``究竟是什麽事''

李大嘴道:``我下山的时候已经快到子时,我以为那小镇上的人一定都睡着了,谁知那小镇上却是灯火通明,满街上都是人来人往,竟比京城的庙会还热闹。所以我也觉得奇怪,拉了个人一问,才知道原来是有两个人在镇上摆了个赌场,不但镇上的人通宵去赌,连附近几百里地的人都闻风而来,所以这本来很荒凉的小镇,竟变得比通商大埠远热闹。''

哈哈儿道:``哈哈,开赌场是一本万利的生意,咱们不如也去凑凑热闹,我和两个小子打打对台吧。''

李大嘴笑了笑,道:``像他们那样的赌场,咱们只怕还开不起。只因他们开赌场为的恨本不是赚钱,而是为了要过赌瘾,到那里去赌钱的人,若是赢了,庄家照赔不误,若是输了,只要叩个头就可走路,据说远不到三天,做庄的那两位仁兄已赔了十几万两。''

白开心张大眼睛,道:``杀头的生意有人做,赔本的生意没人做,这两人莫非有毛病?''

李大嘴悠然道:``这两人也没有什麽别的毛病,只不过赌瘾大得骇人而已,只要有人陪他们赌,他们就就乐不可支,输嬴他们根本就不放在心上。''

哈哈儿忽也一拍巴掌,道:``哈哈,我知道了,这样的赌鬼世上的确再也找不出第二个。''

杜杀皱眉道:``真的是轩辕叁光?''

李大嘴道:``我看见了他,他却没有看到我,只因那时他眼睛里除了骰子和牌九,就算是他亲爹,他都不会认得了。他那里赌注倒也妙得很,磕一个头算一两,打一记屁股算五钱,他若嬴了,赌场里就立刻响起了一片``扑通扑通的磕头声,劈哩拍啦的打屁股声,再加上他得意的笑声,真是热闹得很。''

屠娇娇道:``他若输了呢?''

李大嘴道:``他若输了,倒赔的是一锭一锭的银子拿出来赔人家,一文都不少。''

杜杀忽然道:``和他一齐做庄的那人,你认不认得?''

李大嘴笑道:``人瘦小枯乾,其貌不扬,我连见都没见过。

屠娇娇悠然道:``这倒说不定,也许我对这人倒蛮有兴趣哩''

白开心笑道:``我对这人的兴趣也不小,倒真想看看他是怎和那恶赌鬼交上朋友的,恶赌鬼输的银子,说不定就是他在掏腰包。''

屠娇娇眼珠子一转,笑道:``既然我们两个都对他很有兴趣那麽今天晚上我们就去看看他巴。''

虽已夜深,小镇上果然仍是灯火通明,街上走着的人,大多都是喜气洋洋,但十个中倒有九个看来不像规矩人。

屠娇娇现在的模样,却规矩得很,她打扮得就像是个银子不多,气派却不小的穷酸秀才。白开心自然只好做她的跟班了。

屠娇娇选了个卖云吞面的摊子坐下来,要了一碗面,一个卤蛋,外加一碟卤牛肉。白开心只有在旁边看着的份。

那面摊的老板是个老头子,一面捞面,一面搭讪着道;``你家也是赌钱的麽?''

屠娇娇也笑了笑,道:``开赌场的那两人,你可曾见过?''

那老头子叹了气,道:``那是两个疯子,你家,尤其是瘦的那个,不赌钱的时候,就像是刚死了亲爹似的,成天哭丧着脸,一赌起来,立刻就精神百倍了,我看他这次已赌了三天三夜,连手却没有转过,你家。''

屠娇娇道:``他们输得起麽?''

那老头子道:``据说他们整整带了两大车的银子来的,你家说,这不是祖宗缺了德,才生出一这种败家子麽。''那湖北佬说话倒是客气,一口一个``你家'',叫人听得受用得很。

说话间,他们已随着畿个人走进了小镇里唯一的一家客栈,客栈并不大,现在几乎已经快被挤破了。轩辕三光的赌场就在这家客栈里。

屠娇娇走进去,只见到处都是人挤人,人推入,她的个子本不高,根本就看不到轩辕三光的人在那里。但她终於听见轩辕三光的声音。

只听一人大笑着吼道:``格老子,你们这些龟儿子一个个的上来好不好,再挤就连你们的蛋黄都要挤出来了。''屠娇娇虽已有二十年没听过他的声音,但一听到这``格老子''三个字,已知道准是恶赌鬼无疑。

屠娇娇眠珠子一转,拉着白开心挤到墙角,忽然出手点了前面两个人的穴道,那人连``哼''

都没有哼一声就倒了下去,别的人竟连看都没有往这边看一眼,屠娇娇居然就站到这两人的身上去。於是她就终於见到那``恶赌鬼''轩辕三光了。

现在他们赌的是``单双'',一张八仙桌上,着块白布,白布中间划着条黑线,左面的是单,右面的是双。

骰子开出来,若是``单'',那麽押在``双''上的人就得磕头打屁股,这种赌钱的法子,当真是简单明了,痛快得很。

他半边衣裳已褪了下来,头发也乱了,却用条又脏又臭的毛巾扎着头,满面俱是油光,眼睛里满是血丝,看来活脱脱就像是个杀猪的。

他面前还摆着几个夹着肉的馒头,显见得非但没睡觉,连饭都来不及吃,而那馒头也不过只咬了一口而已。他模样看来买在狼狈得很,但脸上却是兴高采烈,声音虽已嘶哑了,但还是在直着嗓子穷吼。

屠娇娇眼睛盯在轩辕叁光旁边一个人的身上,白开心终於也随着她目光望了过去。只见这人果然是又黑又瘦,其貌不扬,可是一双满布血丝的眼睛,看来却仍然是炯炯有光。

只听轩辕叁光大吼道:``龟儿子们,快下注吧,老子要开了。''桌上单双两边,都押着东西,有的押几个铜板,有的押两块石头,还有的就在破纸上写几个字。桌子旁边,还有两个人在磕头,显然是输得太多了。

轩辕叁光手里摇着个破碗,骰子在碗里不停的响,那又黑又瘦的汉子在一旁瞪着眼瞧着,头上直冒汗。突听轩辕叁光大喝一声,道:``开!''``砰''的,破碗已在桌子上揭了开来。

\hypertarget{ux7b2cux4e00ux767eux5341ux4e8cux7ae0-ux60caux4ebaux8c6aux8d4c}{%
\chapter{第一百十二章
惊人豪赌}\label{ux7b2cux4e00ux767eux5341ux4e8cux7ae0-ux60caux4ebaux8c6aux8d4c}}

人丛中立刻爆发出一片欢呼,有人大笑道:``七点,是单,我赢了。''轩辕三光大笑道;``有赢家就有输家,入你先人板板,输钱的龟儿子先来磕头吧!''也自桌上拈起一串铜钱,一面数,一面笑道:``格老子,五十个,你龟儿子居然想嬴老子们五十两银子\ldots\ldots 是那一个,快出来磕头。''他一连问了三次,人丛里却没有人答应。话犹未了,那又黑又瘦的汉子忽然凌空飞了起来,就像是只大鸟似的,盘旋一转,提起了一个人的头发。

那人惊呼道:``不是找押的\ldots\ldots 不是我押的\ldots\ldots{}''但是那瘦汉子脚尖在另一人肩上只轻轻一点,竟然就将这么大一个人凭空提了起来,``嗖''的掠了回去。

屠娇娇沉声道:``此人不但轻功高明,而且身法古怪得很,我简直连见都没见过。''白开心沉吟著道:``我们好像见过,只不过\ldots\ldots{}''屠娇娇冷笑道;``只不过现在已经忘记了,是么?''这时那黑瘦汉子已将一个太阳穴上贴著狗皮百药的青衣汉子摔在桌子上,那人还在大叫道:

``不是我,你看错了。''

轩辕三光一把拎起他来,怒喝道:``格老子,你龟儿你以为老子们的眼睛不管用么,你龟儿不妨问问这里的人,老子们几时看错过。''他越说越气,反手一个耳光掴了过去,一面打,一面骂道:``赌奸赌滑不赌诈,你龟儿连这规矩都不懂,还敢来赌钱\ldots\ldots 快滚你妈的臭蛋吧。''他的手一扬,竟将这人自人丛上直抛了出去,果然没有一个人敢赖帐了,赌场里立刻就``劈里拍啦'',``噗通噗通''的响了起来,再加上轩辕三光的哈哈大笑声,听起来果然热闹得很。

屠娇娇摇著头笑道:``我看这''恶赌鬼``现在已经该改个外号了。奇怪的是,这黑小子怎会也跟著他一齐发疯呢了难道他们这些银子是从天上掉下来的么?''她笑了笑,又接道:``这也许是因为这小子太年轻,还不懂得银钱的可爱,等他到了我这样的年纪,他就会知道世上再也没有比银钱更可爱的东西了。''这时轩辕三光又在大吼道;``龟儿子们,都押好了么?老子又要开了。''他``吧''的一声刚将那只破碗盖在桌上,突听一人道:``且慢,等我一等。''这声音娇柔清脆,竟是女子的囗音,听来说话的人还在门外,但一个字一个字的传进来,竟将四下乱嘈嘈的人声都压了下去。

轩辕三光咧嘴一笑,道:``赌场里的规矩,你既然来迟了,就得押下一把,但看在你说话的声音很好听的份上,就等你一等。''那声音银铃般笑道:``多谢。''她的笑声比说话的声音更好听,大家都不禁想瞧瞧来的是何许人也,前面的人都扭过头,伸长脖子去望。

他们什么也没有瞧见,只见靠著门的一群人忽然惊呼著向两旁倒了下去,又听得一个男人的声音喝道:``闪开,让条路出来。''接著,大家就郡瞧见五六个铁塔般的锦衣大汉,手里提著皮鞭子,横冲直闯的走了进来。

说话声中,外面又有四条锦衣大汉走了进来,两人抬著很大的二口箱子,箱子的份量似乎很重,他们将箱子抬到赌桌前,也叉起手往两旁一站。

轩辕三光一双眼珠子滚来滚去,大笑道:``想不到我们这小庙里竟来了大菩萨。''他重重一拍那黑瘦汉子的肩头,又笑道:``兄弟,你不是总说赌得不过瘾么?看样子过瘾的已经来了!''那黑瘦汉子面上什么表情也没有,嘴里也不说一个字若不是他的眼睛还没有闭上,别人一定要以为他已经睡著了。就在这时,己有三个艳光照人的少妇姗姗而来。

赌场里本来还是乱烘烘的,但她们三个人一进来后,四下忽然变得一点声音都没有了。每个人都张大了嘴,眼睛发直,连呼吸都几乎停顿,只因这三位少妇实在太美,美得简直令人连气都透不过来。除了衣服的颜色不同外,这三位少妇看来几乎就是一个模子里铸出来的,连走路的步子都完全一样。这时她们已姗姗走到轩辕三光面前,嫣然一笑。

当中的紫衣少妇道:``有劳久候,抱歉得很。''轩辕三光笑道:``没得关系,我已有很久没有跟美人赌钱了,再等等都没得关系。''锦衣大汉们已自外面搬进来三张椅子,用衣襟擦得乾乾净净,再恭恭敬敬的请那三位少妇坐下。

轩辕三光拍了拍手,道``好,现在姑娘们已经可以下注了,请!''那紫衣少妇向身旁的锦衣大汉微微点头,那大汉立刻打开一只箱子,大家只觉银光耀目,照得眼睛都花了。

轩辕三光的眼睛也立刻亮了起来,笑道:``原来姑娘们竟真的是准备来好好赌一场的,姑娘们找到了我,实在真是找对了人了!''那紫衣少妇道:``这里限不限注的!''

轩辕三光大笑道;``你只管放心,随便你押多少,庄家都照赔不误。''紫衣少妇道:``这样最好。''

她挥了挥,道:``五万,双!''

这``五万''两个字说出来,别人只当自己的耳朵有了毛病,但那大汉却真的将五万两白花花白银子堆了上去。

白开心忍不住问道:``你看这三个美人儿真是来赌钱的么?''屠娇娇摇了摇头,道:``像她们这样的人,就算要赌钱,也不会巴巴的赶到这里来。''白开心道:``那么,她们难道是想来找这赌鬼麻烦的么?''屠娇娇沉吟著道;``我现在也还看不透她们的用意,反正你等著瞧吧,这''恶赌鬼``今天绝不会有好日子过的。''这时那黄瘦汉子也似乎忽然自梦中惊醒了,黑脸上已冒出了红光,轩辕三光更是不停的摩拳擦掌,不住道:``好,要得,硬是要得,硬是过瘾。''他一双蒲扇般的大手忽然将那破碗攫了起来,口中大喝道:``开!''两粒骰子都是红的,一粒是么点,一粒是四点。

人丛中立刻传出了一阵叹息声:``五点,单,庄家赢了。''那紫衣少妇却连眼睛都没有眨,好像输出去的只不过是五个小钱,她竟又轻轻挥了挥手,淡淡道:``五万,还是双。''轩辕三光大笑道;``对,有赌不为输,再来。''骰子在碗里``格郎格郎''的响,突听``吧''的一声,轩辕三光将那只破碗用力掀了起来。

两粒骰子都是黑的,一粒是三点,一粒是六点。又是单。

那紫衣少妇竟一连押了六把``双''。骰子开出来一连六次竟都是``单''!两口大箱子已空了一口,赌场里的人头上都冒出了汗。但那紫衣少妇竟还是面不改色。

她身旁的两人,嘴角竟始终带著微笑,既没有说一句话,也没有皱一皱眉,甚至连坐的姿势都没有变一变。

锦衣大汉道:``还有二十万。''

紫衣少妇淡淡道;``这次全押上吧!''紫衣少妇的樱唇中只轻轻吐出了一个字:``双!''她押的还是双!人丛中已忍不住发出了骚动声,但骰子声一响,别的声音立刻全都安静了,甚至连喘息的声音都没有。

轩辕三光``吧''的又将破碗盖在桌子上,用两只大手紧紧包住,眼睛瞪著那紫衣少妇,道;``这次你真的还是押双么?好,要得,连老子都服你了。''他``老子''两个字终于还是说了出来,可见此刻连这``恶赌鬼''的心里都开始紧张起来。那黑瘦汉子的眼睛彷佛已比方才大了一倍,瞬也不瞬的盯著轩辕三光的一双手,额上也已在冒汗。

只听一声大喝:``开!''

骰子开出来又是单这次连轩辕三光都怔住了,他实在连自己都不相信自己有这么好的运气,骰子竟一连开出了七次单人丛中又是惊呼,又是叹息。

但那三位少妇却还是面不改色,甚至连头上的珠花都没有头动,三个人只瞟了那两粒骰子一眼,就站了起来,一言不发,静静的转过身子,静静的走了出去。

轩辕三光忽然道:``姑娘们且慢走。像姑娘们这样的赌客,虽非千载难逢,也是天上少有的。一个赌鬼遇见姑娘这样的对手,若是轻轻放过了,这赌鬼就该打下十八层地狱。姑娘们难道不想翻本?''紫衣少妇笑了笑,道:``只可惜我们今天已输光了,过两天吧。''轩辕三光道:``赌场里本来讲究的是现赌现赔,绝不赊欠,但对姑娘们这样的赌客,却可以例外。''他``啪''的一拍桌子,笑道;``姑娘们尽管押吧,无论要押多少,只要一句话就算数。''紫衣少妇眼角瞟了她身旁的姊妹两人一眼,悠然笑道;``你信得过我们?''轩辕王光大笑道;``只要姑娘肯赌,我还怕姑娘会少了我一两银子么!''紫衣少妇沉吟著,三个人又交换了个眼色,终于一齐转回身,又缓缓走回那张赌桌前。屠娇娇微笑著悄声道:``我早就知道这恶赌鬼不肯放她们走的。''

\hypertarget{ux7b2cux4e00ux767eux5341ux4e09ux7ae0-ux60c5ux6709ux72ecux949f}{%
\chapter{第一百十三章
情有独钟}\label{ux7b2cux4e00ux767eux5341ux4e09ux7ae0-ux60c5ux6709ux72ecux949f}}

只见轩辕三光满面红光,开心得直搓手笑道:``姑娘们这次押多少.''紫衣少妇笑道:``你虽信得过我们,我们却不愿破坏赌场的规矩,何况,空口说白话,赌起来也没什么意思。我们的银子虽已输光,人却远未输出去。''轩辕三光怔了怔道:``人!''

紫衣少妇微笑道:``人,有时也可怍赌注的,赌鬼若是拿到把好牌,就恨不得将人都睡上去作赌注,阁下赌了五十年,难道连这都不懂?''.``妙极妙极,我这赌鬼赌遍天下,到今天才总算遇见了对手。姑娘要怎么赌,只管说吧,我总奉陪就是。''紫衣少妇道:``我们的赌法也简单得很,也是押一个,赔一个。''轩辕三光目光在她们三人身上一转,大笑道:``但像姑娘们这样的人,在下却赔不出来。''紫衣少妇道:``我们若赢了,你们两位中只要有一个跟著我们走就行了。''轩辕三光眼睛瞪得更大,道;.``姑娘们若是输了又如何?''紫衣少妇微微一笑,道:``我们若输了,我们姊妹中自然也有一人要跟著你们走的。''这句话说出来,赌场里又起了骚动,大家都觉得这样赌法,轩辕三光也未免太上算了些。他们若能嬴得这么一个千娇百媚的美人儿,固然是艳福齐天,他们就算输了,能跟著这么样三个人一齐走,也等于一步走入温柔乡了。

白开心瞪著眼道;.``这三人难道看上了这恶赌鬼么?否则为何要如此赌法?''屠娇娇皱眉道:``到现在连我都越来越不明白了,实在想不通她们是为什么来的。''只听轩辕三光不停的大笑道;.``要得,要得,硬是要得\ldots\ldots{}''紫衣少妇等他笑完了,才缓缓道:``如此说来,我们的赌注你已同意了?''轩辕三光笑道:``我还有什么不同意的?''紫衣少妇道:``那么你这位伙伴呢?他也同意么?''她这句话虽是问轩辕三光的,但目光却已瞟向那沉默寡言,令人难测的神秘黑瘦汉子。除了在开宝的时候,他脸上会有些激动的神色,目中会射出些狂热的光芒外,其他的时候,他始终只是呆呆的坐在那里,什么表情也没有,非但好像已脱离了这赌场里烦嚣的人群,简直已像是脱离了这个世界。

轩辕三光笑道;.``我这老弟跟我一样的毛病,什么都不喜欢,就喜欢赌,只要是赌,无论赌什么他都同意。''紫衣少妇眼珠子一转,道:``但我还是要听他自己说一句话。''轩辕三光用手拍了拍他肩头,道:``好,你就自己说一句吧。我们若输了,你肯不肯跟她们走?''黑瘦汉子想也不想,道:``好。''

紫衣少妇立刻追问道:``无论到那里,你都肯去么?''黑瘦汉子长长叹了口气,道:``无论到那里都没关系,在我说来无论任何地方都是一样。''轩辕三光笑道;.``你们莫看我这位老弟有些呆头呆脑的,其实他却是个响当当的男子汉,只要说出来的话,就绝不会反悔!''紫衣少妇嫣然一笑,道;.``我绝对相信。''

轩辕三光大笑道:``既是如此,姑娘们就来押吧。''他一把攫起了那破碗,瞪著紫衣少妇道.``这次你押单还是押双?''紫衣少妇道:``双!''她居然还是押双,就好像输不怕似的。

人群中不禁又.``嘘''的发出一声叹息,大家好像都算定她这次还是有输无嬴,非输不可。

只听.``吧''的一声,轩辕三光已将碗放了下来,但一双大手还是盖在碗上,没有掀起来。

在摇骰子的时侯,他一点也不紧张,因为赌徒只要一听到那清脆的骰子声,就立刻忘记了一切但现在,骰子停了下来,他却不禁有些紧张了巳无论怎么算,这赌注都实在不小。

那三位美丽的少妇却还是神色不动,面带微笑,竟好像还是没有将这场赌的胜负放在眼里,就连轩辕三光都不禁有些佩服她们,别的人更全都屏住了呼吸,整个赌场里静得连一恨针掉在地上可以听见。

猛听得一声大喝:``开!''开出来的骰子,又全都是红的。是一对四。少妇们这次终于押中了!赌场中竟有人情不自禁欢呼了起来,赌徒们毕竟也是人,人都是同情弱者的,赌徒们也大多都同情输家,只要赢家不是他们自己。轩辕三光反倒又不紧张了,反倒笑了起来。他若输不起还有资格算得上赌鬼么?他大笑著道:``好好好,赌神爷在收徒弟了,所以一定要让你们赢一次,若是总叫你们输,你们以后也不会赌得起劲的。''紫衣少妇嫣然一笑,道:``如此说来,这一把是我们赢了。那么,做庄的就该赔呀!''她的手已指向那黑瘦汉子,微笑著接道;.``就请阁下跟著我们走吧。''黑瘦汉子沉默了半晌霍然站起来,大步走出。

轩辕三光一把拉住他,道:``你\ldots\ldots 你真的要走?这里的赌本,还有一半是你的。''黑汉瘦子道:``全给你。''他连自己的身子性命都全不顾惜,又何况这些身外之物呢!轩辕三光叹了口气,黑瘦汉子已转出赌桌,木立在少妇们的面前,紫衣少妇嫣然一笑,道:.

``你放心,你跟著我们走,绝不会吃亏的。''黑瘦汉子好像又已神游物外,什么话都听不见了。

轩辕三光一直瞪著她们,忽又大喝一声,道:``且慢!''喝声中,他魁伟的身子竟已凌空飞起,就好像一只大鸟似的,掠到门口,挡住了那三个少妇的去路。

轩辕三光冷笑道:``我现在才知道三位竟是为了我这黑老弟来的,你们究竟想拿他怎样?想将他带到什么地方?''紫衣少妇也冷笑著道:``这些事,你都管不著,你自己说过.''赌奸赌滑不赌赖``,现在你既已输了,难道还想赖么?''恶赌鬼的脸竟像是有些发红,忽又问道:``你们若输了,难道真肯跟著我走不成?''紫衣少妇淡淡道:``我们姊妹若输了,自然会有人跟著你走,反正我们家姊妹多得很\ldots\ldots{}''轩辕三光的眼睛忽然眯成一条线,上下瞧了这少妇几眼,道:``你们的姊妹真的多得很?有没有九个?''紫衣少妇沉默了半晌,缓缓道:``不多不少,正是九个。''这句话说出来,轩辕三光谜著的眼睛忽又睁开,而且瞪得比铜铃还大,那死气沉沉的黑瘦汉子身子一震,一张脸陡然变得通红,全身的血像是全都冲上了头顶,也瞪著那少妇道:``你\ldots\ldots 你是慕容\ldots\ldots{}''紫衣少妇微微一笑,道:``我是七娘,这是我六姊\ldots\ldots 这是八妹。''她身旁的两位少妇也嫣然一笑,年纪较大的那人道:``你虽未见过我们,我们却久已知道你了。''那黑瘦汉子的脸色忽又变成苍白,脚下一步步向后退。

慕容七娘微笑道:``我们也知道你说出来的话如白衣染皂!永无更改,你既然输了,就一定会跟著我们走的。''轩辕三光忽然仰首大笑起来,大笑著道:``江湖传言,都说慕容九姊妹非但都找到个万中选一的好丈夫,而且姊妹九人个个都有两下子。江湖中人也都知道,慕容姊妹中武功最高的是二姊慕容双,最能干的是七娘,但最聪明.最美丽的却还是么妹慕容九。''听到.``慕容九''这名字,那黑瘦汉子的脸忽又胀得通红。

轩辕三光道:``我还知道这位九姑娘运气没有她八位姊姊好,有一年竟莫名其妙的忽然矢踪了,她八位姊夫虽然都是赫赫有名的世家子弟,而且可说是交游满天下,但找了好几年都没有将她找到。但我这黑老弟却将她找著了,而且就像个呆子似的将她护送回去,谁知别人却丝毫不领他的情,反而好像以为慕容九就是他拐走的,竟将他当成个小偷般盘问了两三天,只差没有打屁股,上夹棍了。''慕容七娘道:``二姊和三姊不是要盘问他,对他更没有丝毫恶意,只不过想问清楚九妹这些年来的遭遇而已。''慕容八娘道:``所以他临走的时候,她们坚持要重重酬谢他。''轩辕三光道:``不错,他走的时候,她们一定要送他五千两金子,这实在不算少数了,若打发叫化子,至少可以打发一两万个。''他脸色早已发青,此刻忽然跳了起来,大吼道:``但我这黑老弟却不是叫化子,他为了你们那九妹,有好几次差点连命都送掉了,吃的苦更不知有多少,他难道就是为了你们那几两破铜烂铁么?你们姊妹都是聪明人,难道真不懂他的意思?''慕容七娘叹了口气,苦笑道;.``我们并不是不懂,只不过\ldots\ldots{}''轩辕三光冷笑道:``只不过慕容姊妹嫁的都是金龟婿,我这黑老弟却既没有钱,又没有势,更不是什么世家子弟,你们自然不能将慕容九嫁给他。''说著说著,他又跳了起来,怒吼道:.

``但我这黑老弟又有那点配不上她?他虽然不是什么大亨,但却是个顶天立地的男子汉,你们的姊妹能嫁到这样的老公,正是你们祖宗积了德!''他指手划脚,大叫大嚷,手指几乎已快指到慕容七娘的鼻子上,慕容七娘居然没有发脾气。

她反而叹息道:``我们也知道他是个很好的人,并不辱没九妹\ldots\ldots{}''轩辕三光冷笑道:``据我所知,黑老弟将她送回去的时候,她病势已有了起色,你们就因为认定她的病会好的,是以才舍不得将她嫁给他。''慕容七娘叹道;.``那时我们的确认为她的病会好的,因为那时她好像已认得大姊了,谁知这位黑\ldots\ldots 黑老弟走了之后,她的病情又忽然恶化,非但连大姊都不认得了,而且整天不说一个字一句话。''慕容六娘也叹了口气,道;.``她只要一开囗,就必定是问:『他走了么?』到后来她连这句话都不说了,每天只是坐在那里流泪。''那黑瘦汉子自然就是骄傲而孤僻的黑蜘蛛。他就像是个木头人似的站著,听到这里,他僵木的面容忽然扭曲起来,就彷佛有人用针在他心上刺了一下。

轩辕三光却大笑道:``原来那位九姑娘也是个多情人,这也不枉黑老弟对她那么好了。''慕容七娘叹道:``到了这时,我们才知道她的心意,我们自然也知道世上无论什么事都能勉强,只有这.''情``之一字是谁也勉强不得。''轩辕三光附和道:``你们总算还不太糊涂。''

慕容六娘叹道:``九妹已病得那么厉害,却还能领受到他的情意,可见他对九妹必是情深意重,人心都是肉做的,到了这种时候,无论他是什么人,我们都不会反对他了。''慕容八娘道:``所以我们就出来找他。但我们也知道他的行踪一向很瓢忽,正发愁不知是否能找得到他,幸好那时五姊夫恰巧经过武汉,恰巧瞧见你和他的一场豪赌。''慕容七娘笑了笑,道;.``我五姊夫就是.''神眼书生``骆明道,他多年前曾经见过你一次,只要被他看过一眼的人,他就永远不会忘记。五姊夫本来也认不出他的,但为了要找他,三姊早已为他昼了很多幅像,五姊夫一瞧见画像,立刻就想起他在什么地方见过这人了。''慕容八娘道;.``我们听了五姊夫的话,就立刻赶到武汉这边来,幸好你们两位的豪赌已在这一带出了名,所以我们很快就找到了你们。''轩辕三光瞪眼道;.``但你们莫要弄错了,我这黑老弟跟我不一样,他并不是赌鬼,他只不过是心情不好,所以才赌的。''慕容七娘笑了笑,道:``他的心情,我们都很了解,我们也知道他是个心高气傲的人,我们若就这样来找他,他一定不会跟我们走的。所以我们才想出赌的法子。''轩辕三光忍不住问道;.``但你们若又输了,那怎么办呢?''慕容七娘道:``我们若输了,我们姊妹中就要有一人跟著你们走,对不对?所以我们若输了,就会要九妹跟著你们走,我们知道你们决不会亏待她的,只要她快乐,谁跟谁走岂非都是一样么?''轩辕三光大笑道:``我只要能亲眼看到我这位黑老弟和那位九姑娘成亲,能喝到他们一杯喜酒,就算叫我三个月不赌都没关系。''他忽又顿住笑声,摇著头道:``不行不行,这杯喜酒只怕是喝不得的。慕容家的姑娘成亲,喜筵上一定全都是有名有姓,有头有脸的人物,我这.''恶赌鬼``若是忽然闯去了,岂非大煞风景。''慕容七娘笑道:``你放心,这杯喜酒少不了你的,我们就算什么人都不请,也一定要请你。''轩辕三光拍掌大笑道:``要得,我若不去,我就是龟儿子。''他忽又挥手道:``抬走抬走,将那些银子全都抬走,连一两都不要留下来。''慕容七娘道:``这\ldots\ldots 这是为了什么?''

轩辕三光笑道:``要喝喜酒,自然就得送礼,你们若不收,就是看不起我,就是不准备请我喝喜酒了。''慕容七娘嫣然笑道:``纵然如此,你也该留下一些做赌本才是呀。''轩辕三光道:``千万留不得,我这人天生是不输光不肯停手的脾气,所以我自从发了笔横财后,简直就没有一天好好睡过觉,我越是拚命想输光,越是输不光,现在好容易有机会将它送出去,你们若不完全收下来,就又害苦了我了。''黑蜘蛛终于笑了笑,忽又悄声道:``小鱼儿必定远在山上,你若看到,莫要忘记告诉他\ldots\ldots{}''轩辕三光笑道:``你放心,我若看到他,一定会要他去喝你喜酒的。''原来他们交成好朋友并非完全是为了赌,而是为了小鱼儿,因为他们始终都认为小鱼儿是个好朋友。

轩辕三光将他们送到门口,忽又笑道:``七姑娘,你以后若是手痒,千万莫要忘记来找我,像你这样的赌客,我平生实在没有遇见几个。''

\hypertarget{ux7b2cux4e00ux767eux5341ux56dbux7ae0-ux90aaux4e0dux654cux6b63}{%
\chapter{第一百十四章
邪不敌正}\label{ux7b2cux4e00ux767eux5341ux56dbux7ae0-ux90aaux4e0dux654cux6b63}}

银子一搬走,赌场里的人立刻也跟著散了。轩辕三光望著已然发白的天空,长长伸了个懒腰,喃喃道:``格老子,真他妈的是天光、人光、钱光,反正不弄到鸟蛋精光,老子也睡不著觉。''他忽然发现赌场里的人竟还没有走光,还剩下四个人,有两个人躺在地上,像是已睡著了。

另外两个人却在笑嘻嘻的望著他。

轩辕三光眼睛一瞪,道:``你们两个龟儿子为什么还不走,难道还想跟老子赌?''那两人中有个比较高的抢著笑道:``这里只有一个半龟儿子,还有半个是龟女儿。''轩辕三光眼睛瞪得更大,瞪著那矮的一人。屠娇娇笑嘻嘻道;.``这里只有一个龟儿子,我却是你祖奶奶。''她也不知道轩辕三光现在已认出她是什么人了,但却末想到轩辕三光不等她话说完忽然好像条被人踩著尾巴的猫似的,飞一般夺门而出。

屠娇娇他们追出去的时侯,轩辕三光已连人影都瞧不见,街上的人,却都扭著头往左面瞧。

轩辕三光显然就是从左面逃走的。

屠娇娇笑了笑,道:``你放心,那赌鬼的轻功一向并不高明,咱们一定能追得上。''话刚说完,轩辕三光忽然又从左面街角后倒退了回来,退得竟比逃的时侯还要快得多。

一退到这条街上,他就转过身子,向这边逃了回来,只见他满脸俱是惊慌之色,一头又冲回了赌场。屠娇娇他们自然又立刻跟了进去。

白开心笑道;.``你这是干什么?难道撞见了鬼么?''李大嘴正将眼睛凑在门缝上,向外面偷看,嘴里道:``正是撞见了大头鬼。''他的神情看来更累张,连脸色都有些发白了。屠娇娇和白开心对望了一眼,也忍不住将眼睛凑到门缝上,向外面望了出去,果然看到左面那边的街角后已转出两个人来。

走在前面的一人,身材很高,肩膀很宽,但却骨瘦如柴,身上穿著件短蓝布袍子,空空荡荡的看来就活像是个纸扎的金刚,只要被风一吹,他整个人都像是要被吹到屋顶上去。他不但人长得很奇怪,脸也长得很奇怪,因为他脸上皱纹虽不少,但却连一根胡子也没有。也没有眉毛。

他眼睛已瘦得凹了下去,所以就显得特别大。他脸上虽也是面黄肌瘦,满脸病容,但一配上这双眼睛,就显得威风凛凛,令人不敢逼视。

白开心道;.``这小子长得倒实有些奇怪,江湖中有这么样一个怪人,我居然没听说.过,也没有见过,可见我这些年来实在太懒了。''屠娇矫也不禁皱起了眉头,道:``恶赌鬼,你认得这人么?''轩辕三光道:``不认得。''他眼睛只瞪在这怪人后面的一个人身上。

走在这怪人身后的一个人,长得非但不奇怪,而且还很好看,年纪也已过了中年,一张脸却远是保养得很得法。他身上穿著的衣服颜色也配合得很好看,很大方,只不过他脸上然在拚命想装出微笑来。看来还是有些垂头丧气,愁眉不展。

这人赫然竟是江别鹤。

屠娇娇更惊讶,皱眉道.``江别鹤怎会没有跟著魏无牙?反而跟这怪人走到一齐来了?''这时右边的街角忽伏冲出一匹马来。马是红色的,就像是一团火,飞也似的冲入这条街,眼见就要将街旁的一个面摊子撞倒。可是马上人的骑术实在不错,竟在这间不容发的一利那,将马勒住,连一只碗郡没有撞翻。

大家这才看清这马上的人也和马一样,穿著一身火红的衣服,手里还提著根火红的马鞭。健马轻嘶中,她已跃下了马鞍。于是大家又发现她的人原来比她的骑术更美,那双又俏皮,又灵活的大眼睛,简直就美得令人透不过气来。

别人的眼睛都在望著她,她都将这些人全都当做死的一样,根本没有瞧这些人一眼,只是跺著脚道:``喂,快来呀,你骑的马难道是三条腿的么?''这时候街首后才又有匹马奔过来,马上人道:``不是我慢,而是你骑得实在太快了。''语声中,这人也下了马,身手也很矫健,却是个很清秀,很斯文的少年,身上衣服的质料也很高贵。

那红衣少女嘟起了嘴,瞪著眼道:``谁敢说我马骑得太快,我撞过人么?''那少年发现这么多人在看他,脸竟似有些红了讷讷道:``你\ldots\ldots 你不快。是,是我太慢。''红衣少女这才嫣然一笑,道:``这样才乖,姊姊请你吃消夜。''那少年脸更红,简直连头都不敢抬了。大家觉得这位少年实在太斯文,太害臊,就像是个大姑娘,但这位大姑娘实在太刁蛮,太泼辣,简直叫人有些吃不消。

就连那怪人都在注意这少年男女两人了,只有江别鹤瞧见这两人时,却立刻低下了头。因为只有江别鹤认得这两人是谁。这红衣少女就是小仙女张菁:这很斯文,很害羞的少年人,自然就是神拳世家的公子顾人玉了。

小仙女展颜笑道;.``今天真可说是九丫头的好日子,我也很开心,所以我一定要大吃一顿,而且还要喝两杯。''顾人玉像是忍不住轻轻叹了口气。

小仙女立刻又瞪眼道:``你叹什么气?九丫头心上有了别的人,你难道很难受么?''顾人玉赶累陪笑道:``我怎会难受,我\ldots\ldots 我\ldots\ldots{}''他非但脸发红,连脖子都粗了。

小仙女.``噗哧''一笑,道:``你不难受最好,你看,这里居然还有粉蒸肉,还有珍珠丸子,我已经有好几年没有吃过这种小吃了,因为除了湖北外,别地方做的都不好吃。''她吱吱喳喳,又说又笑,刚拉著顾人玉在摊子上坐了下来,忽又站起,瞪著街对面的江别鹤,道;.``你看,这是什么人?''顾人玉随著她目光望了过去,面上也变了颜色,沉声道:``他怎会到了这里。''小仙女冷笑道:``是呀,堂堂的江南大侠,怎会躲到这种小地方来了,难道是已经不敢见人了么,难怪江湖中人都说江大侠已失踪了。''她说话的声音就算聋子都能听得到,街上的人也有知道江南大侠名声的,又都不禁直著眼去瞧江别鹤。只有江别鹤却像是什么都没有听见,低著头往前走。像是恨不得一步就走过这条街似的。

可是小仙女一步就窜到了他面前,冷笑著道:``江别鹤,江大侠,你为什么不开口了?你以前不是能说会道的吗?而且我还记得你的威风不小。''江别鹤非但不说话,连头都不抬。

小仙女厉声道:``江别鹤,你用不著装傻,装傻也没用,不知有多少人正等著找你算一算旧帐,你就跟著我走吧。''江别鹤站在那里,连动都不动,脸上也没有丝毫表情,堂堂的江大侠,竟像是已变成个死人他身旁的那怪人却忽然道:``他不能跟你走!''这人的声音低而嘶哑,嗓子彷佛已撕裂了,他说话的声音,只不过是自那些裂隙里一个字一个字挤出来的。

小仙女骡然见到这样的人,听到这样的声音,也不禁怔了怔脱口道.``他为什么不能跟我走''那怪人道:``只因他要跟我走。''小仙女怒道:``跟你走,你是什么东西!''这一声怒喝叱出,她掌中的鞭子也跟著飞出。这条死的皮鞭到了她手里,就像是忽然变成条活的毒蛇,又像是变成了道闪动的火焰,卷向那怪人的脸。

那怪人的反应却迟钝得很,似乎根本不知道鞭子抽在人脸上会疼的,只是出神地望著这鞭子。

眼看著这鞭子就将在他脸上留下条血痕,谁知鞭梢到了他手里,一条长鞭就忽然断成了十几段,一段段落在地上,小仙女的人也站不稳了,踉跄向后直退,终于倒在顾人玉怀里。

别人只瞧见长鞭寸断,小仙女跌倒,至于那怪人是如何出的手?如何用的力气?谁也没有瞧见。

就连小仙女自己也弄不清这是怎么回事,她只觉一股奇异的力道自长鞭上传了过来,她身子立刻如遭雷电所击。若是换了别的人,骤然遇到如此惊人的武功,就算不被吓得半死,也是万万不敢再出手的。小仙女自出道以来,从没有吃过这么大的亏。

顾人玉见到这怪人的武功,正想悄悄劝她忍囗气,谁知她已跳了起来,双手一分,就拔出了两柄短剑。

只见剑光闪动,如惊虹掣电,就在这一刹那间,小仙女已向那怪人攻出七剑,每一剑都恨不能将他刺个透明窟窿。

只听那怪人轻叱一声,也末看清他有什么动作,小仙女掌中的两口剑,就忽然脱手飞出!宛如两道青色的火花般,在黑暗的天空中闪了闪,就消失不见,竟不知飞到什么地方去了。

再看小仙女,竟又跌到顾人玉怀里,只不过她这次虽然用尽平生力气,也休想再爬得起来。

那怪人沉著脸道:``你是谁家的子弟?怎的不分皂白,就敢对人下这么重的手?江湖中的后辈,怎地越来越不懂规矩了?''小仙女大骂道:``你才是后辈小子!你才不懂规矩,你可知道\ldots\ldots{}''她声音忽然顿住,因为顾人玉忍不住掩住了她的嘴。

小仙女用尽全身的力气,用手肘在他肚子上一撞,顾人玉虽疼得松了手,但她的身子也滑了下去,跌坐在地上。她索性赖在地上,指著顾人玉的鼻子道:``我被人如此欺负,你非但不帮我的忙,还不准我说话,你还能算是男人么!难怪别人要叫你顾小妹了。''顾人玉一张脸涨得通红,吃吃道:``我\ldots\ldots 我\ldots\ldots 我实在\ldots\ldots{}''.``我实在是看错你了,我本来还以为你是个男子汉大丈夫,谁知你却比\ldots\ldots 比豆腐还要软,你实在太令我伤心了。''说到后来眼泪已流了满脸。

顾人玉忽然咬了咬牙,大步向那怪人走了过去,大声道:``阁下武功的确高明,但在下还是要来领教领教。''那怪人沉著脸,也不说话。

顾人玉喝道;.``留神,我要出手了!''他做人虽然有些婆婆妈妈的,但出手倒十分乾净俐落,而且又稳,又狠,又准,又快。

只听.``蓬''的一声,这一拳竟著著实实打在那怪人身上,那怪人也不知怎的,竟没有将这一拳闪开。

小仙女眼泪也不流了,眼睛里也发出了光,只因她早知道顾家神拳的威力,也很了解顾人玉手上有多大的力道。

顾人玉武功虽不花俏,但却很精纯,若被他一拳打实,莫说人吃不消,就算是一条牛,只怕也要被他打扁。

小仙女几乎忍不住要拍起手来,但她立刻又发现那怪人非但没有被打扁,而且连脸色都没有变。顾人玉这祖传的神拳,打在他身上,竟好像是在替他敲腿□背似的,顾人玉自己的身子反而站不住了,摇摇欲倒。

小仙女这才吃了一惊,只听那怪人瞪著顾人玉道:``你是顾老四的什么人!''顾人玉头上直冒冷汗,道;.``前\ldots\ldots 前辈莫非认得家父?''那怪人.``哼''了一声,道:``听说顾老四的家教很严,怎容得你这样的子弟在江湖中招摇?要知越是会武功的人,越该要自己收□,若是一言不合就胡乱出手,那就是盗贼匹夫所为,这道理你爹爹难道未曾教训过你么?''顾人玉被骂得连头都不敢抬,那里还敢说话。小仙女却忍不住大声道;.``你究竟是什么人?凭什么来教训我们?''江别鹤一直木头人般站在一旁,一点也没有吃惊,好像早就知道那怪人一出手就可将小仙女和顾人玉两人击倒。

此刻他忽然笑了笑,道:``你们连他老人家是谁都不知道么?他就是大侠燕南天!''燕南天!这三个字一说出来,小仙女已不敢发横,瞪大了眼,张大了嘴,再也合不拢来。顾人玉更早已翻身拜倒,就连那些从赌场里散出来的地痞流氓们,也有几个听过.``燕南天''这名字,更吓得连大气都不敢喘。

燕南天沉声道:``江别鹤以后永远再也不能欺世盗名,为非作歹了,你们也用不著再找他算帐,因为已有别的人要先找他算帐,那是二十年前的旧帐。''顾人玉汗流如雨,连声道:``是,是\ldots\ldots{}''燕南天道:``只望你们以后也莫要以武凌人,妄动杀手?''顾人玉垂首道:``是。''

燕南天挥了挥手,道:``你们走吧。''

躲在门后面偷看的白开心和屠娇娇,两条腿早已吓得发软,全身的衣服也早已全都湿透。轩辕三光见了燕南天虽然也有些心虚害怕,但却没有他们怕得这么厉害,瞧见他们的模样,轩辕三光忍不住笑了,悠然道;.``你龟儿现在为什么不叫了?听说你们将燕南天在恶人谷中困了二十年,老子本来还不相信,现在看来,只怕真有这回事。''白开心抢著道;.``那是她和大嘴狼他们干的事,与我无关。''轩辕三光笑道;.``既然与你无关,你龟儿为什么怕成这副样子?''白开心道:``你见了他难道不害怕么?''

轩辕三光道:``老子坏事做得没有你多,用不著像你龟儿这么害怕。''白开心忽然咧嘴一笑,道:``常言道,只有强奸的,没有逼赌的,可见逼人赌钱要比强奸更坏,我干的坏事最多也只不过是强奸而已,可是你\ldots\ldots 嘿嘿,你小子等著瞧吧,燕南天若知道你就是恶赌鬼,不打扁你的脑袋才怪。''轩辕三光擦了擦汗,也说不出话来。他们三个人都希望燕南天快些带著江别鹤远远走开,谁知燕南天却要了壶酒,坐在小摊子上自斟自饮起来。

江别鹤垂著手站在一旁,既不敢走,也不敢坐下,别的人也都吓得坐不住了,就连那小摊子老板的手都在发抖。燕南天却旁若无人,一杯杯喝个不停,每喝一杯,就长长叹囗气,彷佛有很重的心事。

轩辕三光皱著眉,喃喃道:``江别鹤这龟儿子怎会和燕南天走到一路的?这倒实是怪事。''他以为这句话绝不会有人回答,谁知屠娇娇却忽然叹了囗气,道:``我现在才想出江别鹤的来历了。''.``他有什么来历?''.

``他一定就是江琴。''.

``江琴又是什么人?''.

``燕南天到恶人谷去,就是为了要找江琴复仇的,因为江琴害死了他的拜把兄弟江枫。''轩辕三光怔了怔,道:``他既然要找江琴复仇,现在为何还不宰了他,反而带著他满街跑呢?''.``因为他要先找到小鱼儿,叫小鱼儿亲手报仇。''.``不错,想必就是这缘故,可是,他若找不到小鱼儿呢?''白开心忽又咧嘴一笑,道;.``他这辈子只怕是再也找不到那小坏蛋了。''轩辕三光耸然道:``为什么?''

白开心张开了嘴,却只笑了笑,再也不说话了,因为屠娇娇已在暗中悄悄的拧住了他的手。

就在这时,突见一个人手里提著壶酒,也走到燕南天正坐在那里吃东西的小摊子上去,而且还在燕南天身旁坐了下来。面摊上吊著盏灯笼,灯光照在这人的脸上,只见他年纪轻轻的,长得倒也眉清目秀,只不过脸色苍白得可怕。

轩辕三光又吃了一惊,道:``这龟儿岂非就是江别鹤的儿子江玉郎么?''白开心道:``一点也不错。''

只见江玉郎就像是没有见到他老子似的,江别鹤也像是根本不认得他,父子两人,谁也没有瞧谁一眼。

轩辕三光皱眉道:``这父子两人究竟在搞什么鬼。''屠娇娇道:``看来他必定是想来救他老子的。''轩辕三光冷笑道;.``就凭这小杂种,只怕还没有这么大的本事。''屠娇娇忽然笑了笑,道:``他本事虽不大,花样却不少,连小鱼儿有时都会上他的当。''轩辕三光瞪著眼睛,冷笑道;.``老子也知道他花样不少,但若要比小鱼儿,他还差得远。''屠娇娇眼珠子一转,不说话了,她已发现这恶赌鬼和小鱼儿的交情不错,否则就绝不会帮小鱼儿说话。

这时江王郎竟已在向燕南天敬酒,而且还陪笑著说话,燕南天显然不知道他就是江别鹤的儿子,也没有给他难看。说了几句话后,燕南天忽然长身而起,大声道:``你真的认得江小鱼。''江王郎也站了起来,陪笑道:``非但认得,而且还可以说是患难之交。''燕南天一把拉住他的肩膀,道:``你\ldots\ldots 你最近见过他么?''.``前两天他还和晚辈在一齐喝酒\ldots\ldots{}''燕南天不等他说话,就抢著问道:``你可知道他现在到那里去了?''江玉郎沉吟著道;.``他的行踪一向很飘忽,但晚辈却也许能找得到他。''燕南天道:``真的?''

江玉郎躬身道:``晚辈就算有天大的胆子,也不敢在前辈面前说谎。''燕南天道:``好,好,好\ldots\ldots{}''他实在太欢喜,竟一连说了十几个.``好''字,那只紧紧握著江玉郎肩膀的手,也忘记松开。

江玉郎虽被他捏得骨头都快断了,但面上却不禁露出微笑。

江别鹤目光闪动,忽然大声道:``这小子来历不明,燕大侠你怎可轻信他说的话。''燕南天怒道:``闭嘴,在我面前,那有你说话之处?''他匆匆撒了把铜钱在摊子上,拉著江玉郎就走,江别鹤只好也垂头丧气的跟著走,但嘴角却正在偷偷的笑。

\hypertarget{ux7b2cux4e00ux767eux5341ux4e94ux7ae0-ux6076ux4ebaux518dux805a}{%
\chapter{第一百十五章
恶人再聚}\label{ux7b2cux4e00ux767eux5341ux4e94ux7ae0-ux6076ux4ebaux518dux805a}}

躲在门后偷看的屠娇娇见燕南天上了江玉郎的当,不由也笑了,喃喃道:``我早已知道燕南天必定要上他的当,我猜的果然不错。''白开心吃吃笑道:``这小鬼果然有两下子,也难为他装得真他妈的像极了,燕南天居然真跟著也走,真是鬼迷了心窍。''屠娇娇笑道:``这下子燕南天非但永远休想找得到小鱼儿,只怕连命也要送在这父子两人的身上。''轩辕三光呆呆的出了会儿神,忽然推开门,就想冲出去,谁知屠娇娇的手早已等在他背后,他刚推开门,屠娇娇就闪电般点了他五,六处穴道,将他的人往肩上一扛,转身从后面的窗子窜了出去。

轩辕三光又惊又怒,怎奈连话都已说不出来。只见屠娇娇从屋子后面绕出了这小镇,天色虽已很亮了,但入山的道路上,并没有人踪。她似乎将吃奶的力气都使了出来,飞也似的窜上山,也不知走了多久,突听一阵铁器敲击声自风中远远传了过来。

李大嘴,哈哈儿,和杜杀正在开山,突见屠娇娇和白开心两人飞掠而回,就像是被鬼追著似的。最奇怪的是,屠娇娇背上还扛著个人。李大嘴他们立刻全都停住了手,迎了上去。

哈哈儿目光转处,失声笑道;.``我当是谁呢,原来是恶赌鬼到了,哈哈,久违久违。''李大嘴大笑道:``恶赌鬼,多年不见,怎地一见面你就爬到屠娇娇身上去了?难道你这赌鬼已变成了色鬼了么?''杜杀却皱眉道:``这是怎么回事?''

屠娇娇先不答话,却将轩辕三光重重往地上一掼,这一掼,便将他穴道全都解了开来。他人还末站起,已大笑道:``原来你们这些龟儿子全都到这里来了,龟山上有了你们这么多龟儿子,倒实在是名副其实。''白开心哈哈一笑,道:``屠娇娇莫名其妙的点了你七.八处穴道,又像条狗似的将你掼在地上,你不找她拚命,反而开起玩笑来了,嘿嘿,看来你这人真在是好欺负得很。''轩辕三光生性豪爽,骤然见到这许多老朋友,已将别的事全都忘了,但此刻被白开心挑拨了几句,他立刻又火冒三丈,跳起来指著屠娇娇的鼻子道:``我问你,你这不男不女的龟儿子为什么要点老子的穴道,难道真当老子是好欺负的么?''屠娇娇道;.``我问你,你方才冲出去是不是想去通风报讯,叫燕南天莫要上江别鹤父子的当。''.``燕南天''这三个字说出,李大嘴.哈哈儿.杜杀全都耸然失色,好像连站都站不稳了。

杜杀失声道;.``燕南天?''

李大嘴道:``难道他\ldots\ldots 他的病已好了么?''

屠娇娇道:``他非但病已好了,而且功夫彷佛此以前更强,我见到他的人时,还没有认出他来,但见他露了一手功夫后,就知道必是燕南天无疑,因为除了燕南天之外,世上再也没有第二个人有那么高的武功。''哈哈儿牙齿打战,非但再也笑不出来,连话也说不出了。

白开心抢著道;.``他已被江别鹤父子骗走,但恶赌鬼却想将他找回来。''这句话还末说完,李大嘴,杜杀.哈哈儿已将轩辕三光团团围住,三个人具是咬牙切齿,满面凶光。杜杀瞪著他一字字道:``你这是什么意思?''轩辕三光别的人不怕,但对杜杀却也有三分畏惧,此刻见到他杀机毕露,显见一伸手就要杀人,轩辕三光心里也不觉有些发毛,勉强笑道:``老子不过是想要他将江别鹤父子宰了而已,并没有别的意思。老子难道还会要燕南天来找你们的麻烦不成?''白开心笑道;.``我问你,你若没有做亏心事,为什么一见到我们就跑呢?''轩辕三光脸色变了变,道;.``这\ldots\ldots 这个\ldots\ldots{}''白开心拍手笑道:``你说呀?你怎地说不出话来了?这不是做贼心虚是什么?''轩辕三光跳了起来,吼道;.``老子又没有掘你祖坟,你龟儿子为什么找老子麻烦。''白开心知道目的已达,无论轩辕三光怎么骂,他都不开腔了。李大嘴.哈哈儿果然俱是满面怒容,杜杀更是面笼寒霜,厉声道:``你方才是不是一见他们就跑。''轩辕三光道:``我,格老子,不错,我是跑了。''轩辕三光挺起了胸膛,大声道:``只因老子已将你们的钱都输光了!''这句话说出来,大家又吃了一惊。

哈哈儿抢著道:``我们的钱是什么钱?''

轩辕三光道:``你们都知道老子是恶赌鬼,却不知老子虽喜欢嬴钱,也喜欢输钱,只要有钱输,实在比赢钱更过瘾,尤其是输给那些没有钱的小赌鬼,看到他们赢钱后那种欢天喜地的模样,那其中的乐趣,你们这些龟儿子只怕永远也想像不到。''他歇了口气,接著又道:``前几个月我替一个朋友将一票银子送回去给江南的大富翁段合肥,虽然因此得罪了江别鹤父子,却跟段合肥斗了半个月蟋蟀,赢了他几十万,我手头有了赌本,就想送出去一些了。''李大嘴冷笑道:``想不到你这恶赌鬼倒实是劫富济贫的侠盗。''轩辕三光道:``但是老子越是想输,那银子就偏偏跟老子作对,总是输不出去。有一天我正在一家菜馆里喝茶,旁边居然有人赌起骰子来了,我一看正中下怀,就和那些龟儿子赌了起来。''李大嘴道:``你又嬴了?''

轩辕三光笑道;.``该当那些龟儿子走运,老子的赌运恰巧在那里走光了,别人掷出个四点,老子都赶不上,竟一连输了几天几夜。''白开心忽然插嘴道;.``输得好。''

轩辕三光道:``那家茶馆在一条小巷子里,老子输了三天后,那巷子里老老少少都嬴了老子不少,只有个糟老头子,虽然每天都到这茶馆里来喝茶,每天都看到老子输,却硬是不动心,硬是不肯下场来赌一手。''他笑了笑,接著道:``他越不肯赌,老子就越找他赌,别人都说这老头子非但不赌钱,而且不抽烟,不喝酒,是个漂标准准的木头人,大家都叫他李老实,还说只要我能引得这李老实跟我赌钱,他们每天就跟我磕一头。''屠娇娇瞟了李大嘴一眼,笑道;.``想不到李家门里还有这么样的老好人,难得难得。''轩辕三光道:``那条巷子里还有个屠寡妇,据说县里已快替她立贞节牌坊了,她虽在巷口摆了个小摊子,但十年来来往往,就没有人看到她笑过,她家里也没有别的人,只有著一条狗,替她看守门户。''李大嘴大笑道:``想不到屠家门里居然还有人肯守寡,难得难得,只不过可惜她还是养了一条狗,\ldots\ldots 哈哈,狗最大的好处就是不会说话。''轩辕三光道;.``赌到第四天,我还剩下三万两银子,我就将银子全都堆到李老实面前,我说我只要说一个字,就能令那屠寡妇笑起来,再说一个字,就能叫她打我一个耳光,我问李老实信不信?''哈哈儿忍不住问道:``他信不信?''轩辕三光道:``屠寡妇从来不笑的,男女授受不亲,寡妇更不能打男人的耳光,李老实自然不信,于是我就跟他打赌,我若输了,就将剩下的银子全都给他,我若瀛了,只要他再陪我赌十把骰子。他望著面前的银子,足足望了半个多时辰,终于还是跟我赌了,他虽然老实,但老实人见到送上门来的银子,也舍不得不要的,只因每个人都认定我这场赌实是有输无嬴,连半分机会都没有。''哈哈儿道:``但你却嬴了。''

轩辕三光道:``只为了要跟他再赌个痛快,我自然非赢不可。''听到这里,连杜杀都不免动了好奇之心,忍不住问道;.``你是怎么样嬴的?''屠娇娇道:``只说一个字就能令寡妇发笑,再说个字就要她翻脸打人\ldots\ldots 这实在连我都被难住了。''李大嘴.白开心,面面相觑,实在也想不出轩辕三光说的那是什么字?怎会有那么大的魅力。

只听轩辕三光悠然道;.``到了下午,那寡妇才摆起她那卖煎饼的摊子,那条狗和她寸步不离,自然也跟在她身旁,于是我就走过去,恭恭敬敬向那条狗磕了头叫了.''爹``那寡妇怔了怔,虽然想板起脸,终于还是忍不住笑了起来。''李大嘴等人听了也都笑了起来。

轩辕三光道;.``别人见到我果然只说了一个字,就令那寡妇发笑,虽然又佩服,又好笑,但还是想不出我怎能令她翻脸打我。''屠娇娇笑道:``老实说,连我都想不出你是有什么法子。''轩辕三光道:``我只不过又跪到她面前,叫了她一声.''妈,她立刻就满脸通红,连脖子都粗了,狠狠打了我一耳光,转身就走。"他话未说完,李大嘴等人已笑弯了腰。

轩辕三光道:``于是李老实只好陪我赌骰子,谁知我手气竟转了,一连嬴了十场,开始时他还赌得很少,但到后来,他也输急了,竟将家里的夜壶棉被都拿出来跟我赌,赌了十场后,他已输得乾乾净净,我就问他,你既然连赌本都没有了,还赌什么?他呆了半晌,忽然咬了牙,把我带到他家里去,他家里已被搬空了,但却还有个小屋子,里面堆著好几口大箱子。''屠娇娇失声道:``大箱子?什么样的大箱子?''轩辕三光道;.``黑黝黝的大箱子,上面积满了尘土,李老实说,这本是别人托他看管的,他从来也没有碰过,但现在,他却顾不得这些了。''他笑著接道:``一个人若是输急了,连老婆儿子都会押上赌桌的,这李老实虽然一生都很老实可靠,但老房子著火,烧得更快。''屠娇娇道;.``他\ldots\ldots 他难道将那些箱子全都输给你了。''轩辕三光道;.``不错,可是我却未想到,那些箱子里竟装著全都是黄金白银,更未想到那些箱子竟是你们的,若非箱子里有你们的记号,我永远也不会想到你们竟会将箱子交给一个老头子保管,哈哈,这法子实在妙极。''他大笑接著道:``但我却正如天上掉下了大元宝,平空落下了几百万,于是我就大赌特赌,到这里,已输得差不多了,剩下的已全都送别人作嫁妆,现在我已又是囊空如洗,你们要我还钱,我是一分也没有,要命倒有一条!''白开心.哈哈儿.杜杀.李大嘴.屠娇娇五人全都听得怔住,面如死灰,如丧考妣一般。

哈哈儿道:``原来\ldots\ldots 原来欧阳丁.欧阳当并没有将箱子藏在龟山,却存在李老实那里,我们还是上了他的当。''哈哈儿忽然将地上的铁锹.铁铲全都抛了出去,大笑道;.``其实我们倒该感激这赌鬼才是。''白开心道:``感激他?''

哈哈儿道:``他若不说,我们就还要在这里作苦工,挖山洞,现在我们反倒可以休息休息了。''杜杀缓缓道:``其实他并没有说错,若非轩辕三光,我们永远也不会知道箱子究竟在那里?反而多费些事,多著些急。''白开心叫了起来,道:``如此说来,你们不准备要他赔了么?''李大嘴笑了笑,道:``他早已说过,要钱没有,要命一条\ldots\ldots{}''白开心道:``但他这身肉也不错,你难道不想尝尝味道么?''李大嘴笑道;.``我若将这赌鬼吃进肚子里,那还得了,他若要我的肠子和胃打起赌来,我怎么吃得消。''他瞪著轩辕三光又道:``你将银子都输光了,难道将箱子也输了么?''轩辕三光道:``没有。''李大嘴眼睛一亮,大喜道:``箱子在那里!''

轩辕三光道:``老子嫌那些箱子太重,早已全都抛进扬子江了。''李大嘴,屠娇娇面面相对,再也说不出话来。

轩辕三光重重啐了一囗,道:``格老子,你龟儿喜欢的是吃人肉,人肉却是银子买不到的,丢了几两银子,你难过什么!''李大嘴叹了口气,道:``这你就不懂了,一个人年纪越大,就越贪财,我虽也知道那玩意儿吃不得,穿不得,也带不进棺材,但我却偏偏越来越喜欢它。''哈哈儿道:``不错,我每天什么都不干,只要让我关起门来数银子我已经觉得很过瘾了。''轩辕三光道:``我看你们这些龟儿子只怕真的已经快进棺材了,一个人若是什么都不喜欢,只喜欢钱的话,他就已经死了大半截。''他又啐了一口,接著道:``但你们既然如此喜欢钱,为什么不再去偷,去抢,那些银子反正是你们这些龟儿偷来抢来的。''李大嘴正色道:``这你又不懂了,恶人也得有恶人的身份,.像我们这样有身份的恶人,若再去杀人越货,岂不叫人笑掉大牙。''轩辕三光怔了半晌,忽然大笑起来,道:``想不到你们这些龟儿连强盗都不敢做了,你们还有什么用?我看你们不如同泡尿自己淹死算了。''屠娇娇道:``放你妈的屁,谁敢说.''十大恶人``没有用?''轩辕三光冷笑道:``二十年前,你们也许可以算得上.''十大恶人``,但在那乌龟洞里躲了二十年之后,你们已只能算是.''五十缩头乌龟``了。''屠娇娇怒道:``你以为你是什么东西,就算在二十年前,你也没有资格称得上.''十大恶人``,别人只不过是将你拿来凑数的。''轩辕三光道:``既然我们都算不上是什么''恶人``,为什么不索性做件好事呢!''李大嘴道:``做什么好事!''

轩辕三光指看地上的花无缺和笼子里的铁心兰道;.``我们为什么不将这三个可怜虫放了,让他们感激一辈子。''李大嘴沉吟看道:``不错,我们被人家恨了一辈子,偶尔也叫几个人感激我们,倒也不错。''轩辕三光道:``杜老大,你的意思怎样?''

杜杀冷冷道:``反正这三个人已离死不远,我杀他们也甚是无趣。''白开心眼珠子直转,忽然道;.``你们既然要作好人,为什么不索性好人做到底。''哈哈儿大笑道:``哈哈,损人不利己难道也做得出什么好事么?''白开心道:``我坏事做了一辈子,如今也想做好事是什么滋味了,否则我死了到阎王爷那里去都不好交代。''轩辕三光道:``你龟儿子究竟玩什么花样?''白开心背看花无缺和铁心兰,笑嘻嘻道:``这两人你爱我,我爱你,已爱了好多年,只是中间多了个小鱼儿,现在小鱼儿既然已翘了辫子,我们为什么不索性将这两人结成夫妇,哈哈,让天下有情人终成眷属,岂非是最大的好事。''哈哈儿拍手笑道:``不错,我们闲了这么多年,现在能为他们办办喜事,好好热闹一场,倒也开心得很。''李大嘴笑道:``我已有二十多年没吃过喜酒了,这想必有趣得很。''屠娇娇却指看白开心笑道:``我就知道这小子没存好心,干的果然还是损人不利己的事。''白开心道:``替别人做媒,正是天大的好事,连阎王知道了,都要添我一记阳寿,你怎么还说这不是好事呢?''屠娇娇笑道;.``你明知这两人现在都很伤心,却偏偏要他们现在成亲,这岂非比杀他们更缺德。''白开心眨著眼道:``他们就算现在很伤心,到成亲后那种妙不可言的滋味,我保险他们绝不会再伤心了。''李大嘴道:``这条狗嘴里真是连一根象牙都吐不出来。''屠娇娇笑道:``这就叫狗改不了吃屎,坏蛋永远做不了好人的。''哈哈儿道:``我不管你们怎么说,反正是非要这两人成亲不可的了,哈哈,我还要亲手替他们换上红衣裳,亲手替他们倒交杯酒。''李大嘴瞟了白夫人一眼,忽又笑道:``这里反正还有一条母大虫,我们索性也替她找个老公吧。''哈哈儿瞧了瞧白夫人,又瞧了瞧白开心,大笑道:``不错,不错,这两人正是天生的一对。''屠娇娇吃吃笑道:``看来这位大嫂子福气不差,也真和姓白的有缘,嫁来嫁去,都是姓白的,连姓都不必改了。''白开心已叫了起来,道:``你们\ldots\ldots 你们\ldots\ldots{}''他嘴里说著话,人已想溜。

但屠娇娇.李大嘴,早已一边一个夹住了他。

屠娇娇笑道:``这是天大的喜事,你为什么还想溜呢?''李大嘴道:``你溜也溜不了的。''

轩辕三光自从听到.``小鱼儿已翘了辫子'',一直都没有说话,此刻眼珠子也转了转,忽然道:``我知道还有两个人要成亲,既是喜事,索性大家合在一起办吧,既省钱,又热闹。''屠娇娇道:``你说的是那慕容九的小丫头和你那黑小子的朋友?''轩辕三光道:``不错。''李大嘴大笑道:``慕容家的人,怎么会和咱们一齐办喜事呢,这赌鬼发疯了。''轩辕三光道:``我们何必跟他们商量,到了那时候,我们就一齐拥进喜堂,将三对新人排在一齐,再吃他们一顿喜酒,他们还能在好日子里跟我们翻脸么?''哈哈儿拍手大笑道:``好主意,好主意,哈哈,我们就跟他来个霸王硬上弓。''李大嘴道:``我真希望他们酒席上有道菜是用人肉做的,到时你们吃你们的山珍海味,我也有人肉吃,那就真的皆大欢喜了。''白开心忽然冷冷道;.``只望那天燕南天也去喝喜酒才好。''这句话说出,大家又全都笑不出了。

只听轩辕三光道:``燕南天绝不会到那里喝喜酒的。''白开心冷笑道:``你怎么知道?你又不是他肚子里的蛔虫。''轩辕三光也不理,他道:``燕南天现在一心只想找小鱼儿,那有功夫去喝喜酒。''白开心道:``你莫忘了,要找人一定会往人多的地方去找,办喜酒的地方人最多,我要是燕南天,也会去凑热闹的。''轩辕三光道:``你龟儿也莫忘了,现在替燕南天带路人的是谁。''白开心怔了怔,不说话了。

屠娇娇笑道:``现在替燕南天带路的是江玉郎,江王郎非但绝不会将燕南天带到慕容家去,也不会将燕南天带到人多的地方,他怕别人揭穿他的把戏。''白开心道:``如此说来,岂非人越多的地方反而越安全。''轩辕三光道:``最安全的地方,就是慕容家那些姑娘们的所在之地。'',屠娇娇笑道:``一点也不错,想不到这赌鬼近来也变得聪明了。''哈哈儿跳了起来,道:``既是如此,我们现在还等什么,赶快走吧,哈哈,我这人天生就喜欢热闹,人越多越好。''

\hypertarget{ux7b2cux4e00ux767eux5341ux516dux7ae0-ux9b3cux7ae5ux590dux51fa}{%
\chapter{第一百十六章
鬼童复出}\label{ux7b2cux4e00ux767eux5341ux516dux7ae0-ux9b3cux7ae5ux590dux51fa}}

李大嘴忽然一拍巴掌,道;.``我们倒忘了一件事。慕容家的人最讲究排场,怎么会在这种穷乡僻壤办喜酒呢?我们总该去打听打听,他们走了没有?准备在那里办喜酒。''屠娇娇道:``就叫这赌鬼去吧,他和她们有交情。''突听窗外有人阴恻恻一笑,道:``活鬼已经去过,赌鬼就不必去了。''轩辕三光大笑道:``格老子,你这半人半鬼的龟儿子还没有被打下十八层地狱么?''阴九幽自窗外露出一张青森森的脸来,嘻嘻笑道:``这世上鬼已够多了又是赌鬼,又是色鬼,再加上穷鬼,酒鬼,讨债鬼,小气鬼\ldots\ldots 世上既有这么多鬼,我怎舍得再到别地方去。''杜杀沉声道:``你是说你已去打听过慕容家的消息了?''阴九幽道;.``不错,她们本来是准备要回去再办喜事的,但后来却改变了主意。''杜杀道:``为何改变主意?''

阴九幽摇著头道:``她们没有说,也没有人敢去问她们。''李大嘴笑道:``女人家决定一件事后,若是不改变主意,倒是件怪事了。''哈哈兄道:``她们为何改变主意,屠娇娇也许知道,哈哈,她至少有一半是女人。''屠娇娇道:``不错,我的确知道。''

哈哈儿反倒怔了怔,道:``你真的知道?你是怎么知道的?''屠娇娇道:``你若肯花些心思,也猜得出来的,只可惜你的心已经给猪油蒙住了。''杜杀道:``你说她们究竟是为何改变主意的?''屠娇娇笑道;.``你想,她们若是真的规规矩矩的办喜事,江湖上有头有脸的人物必定会到齐,大家都想知道这位慕容家的九姑娘究竟是怎么样一位聪明标致的人物,都想瞧瞧她选来选去选到怎么样一位了不起的好姑爷。''她嘻嘻一笑,接著道:``怎奈我们这位慕容九姑娘却已变成了个痴痴呆呆的半疯子,选到的姑爷也是个才貌不扬,还有点疯疯癫癫的人物,这么样的一对夫妻若是被她们的亲戚朋友瞧见,岂非丢尽了慕容家的人么?''李大嘴笑道;.``不错,她们家的亲戚朋友,不是公子哥儿,就是千金小姐,这种人吃饱了饭没事做,就想著看别人的笑话,还有的说不定早就瞧著她们眼红了,她们若丢了这次人,以后在别人面前怎么抬起头来,倒不如省些事了。''屠娇娇道:``所以她们就索性在这小地方为这对见不得人的夫妻成亲,然后再将这对夫妻往别地力一送,叫他们安安份份的过日子,以后别人若是问起来,她们也可以说,不敢惊动罗,新姑爷脾气有些古怪罗,以后再补请喜酒罗\ldots\ldots{}''李大嘴拊掌道:``妙极妙极,这么一来,别人心里就算怀疑,也抓不著她们的把柄了。''屠娇娇道:``话虽如此,但这种人天生的死要面子,还是不会太省事的,她们一定还是要□张一番,请请客,表示她们并非为了想省钱,只不过她们请的一定是些不相干的人,谁也不敢去笑话她们。''阴九幽嘻嘻笑道:``屠娇娇真他妈的不愧是女诸葛,说的一点也不错。''杜杀道:``她们在那里请客?''

阴九幽道:``她们已在江边搭起一两里长的长棚,摆下了流水席,无论谁都可以去吃她们一顿,就连叫化子每人都有两斤肉,一瓶酒。''杜杀道:``什么时候?''

阴九幽道:``就在今天。''

虽然还没有天黑,但长棚内外都已点起了大红灯笼,上面还用金纸剪著双.``喜''字,看起来倒真是喜气洋洋,蛮像那么回事。

长棚里的人,比苍蝇下的蛋还多,有新娘子可看,这些乡下人已经要挤破头了,何况这里还有不花钱的黄酒白酒,大鱼大肉。但有些人并不是完全白吃,居然还用红纸,红布,红绸子做成些喜联喜幛,上面还写著.``天作之合'',.``鸾凤和鸣''一类的吉词,有的居然还有下款,也莫非是张阿大、李洪发一类的名字。慕容家居然还将这些喜联喜幛挂了出来,一眼望去,到处都是红红绿绿的红纸贴在竹子上,被江风次得.``哗啦哗啦''的直响。

江边停著三艘油漆崭新的大官船,舱里舱外不时有穿得花团锦簇般的丫头使女们进进出出。

长棚里喝酒的人,都不时伸长颈子,往这艘官船上去瞧。

有人道:``这家人也真奇怪,无缘无故的请了这么多人来喝喜酒,主人家都躲在船舱里不肯露面,新郎倌也不出来敬我们几杯。''又有人道:``你就马虎些吧,你可知道人家是什么身份,怎会来跟我们这些人喝酒。''那人道;.``看他们这种势派,我还真猜不透他们是干什么的。''另一人道:``听说他们不但是江南首屈一指的大富翁,而且还是武林中响当当的人物,请我们来,只不过是为了想要我们凑凑热闹而已,我们还是多喝酒,少说话的好,莫要说错了话犯了人家的忌讳,那就真是敬酒不吃要吃罚酒了。''大家正在纷纷议论,谈得高兴,忽然一齐闭住了嘴扭过头来望,就好像瞧见了什么怪物似的。

原来这时已有辆马车在长棚外停下,这辆马车的式样已经够奇怪了,从车上下来的人却更奇怪。赶车的是一条很魁伟的大汉,身上穿的虽是件质料很好的新衣服,钮扣却一粒也没有扣上,露出了满胸黑毛!他不笑还好,一笑起来,一张嘴几乎裂到耳边,看来一口就可以吃下两个半斤重的大馒头。接著,车上又走下几个人,有的又矮又胖,有的妖里妖气,还有个人手上竟装著个钢钩,那张脸白里发青叫人一看就害怕。这些人的模样已经是稀奇古怪,天下少有,谁知他们又从车上推推拉拉的拉下三个人来。

这三个人有气无力,面容憔悴,看来已奄奄一息,身上却偏偏穿著红绸绿绸,打扮得和新娘子一样。长棚里几百双眼睛都在盯著他们,他们却大摇大摆,若无其事,忽然一窝蜂的拥进竹棚。

其中一条满脸大胡子的彪形大汉大声道:``格老子,你们这些龟儿子们知不知道主人在那里?老子要找她们。''大多数人都认得这就是那开赌场的怪人,都领教过他们的手段,虽然被叫做龟儿子,也不敢出声。

偏偏有两人是刚从城里来的,还是永什么镖局里的趟子手,总认为自己混得蛮不错的,怎肯受这个气。再加上七八分酒意,两人一齐拍桌子跳起来,吼道:``你这混蛋在骂谁?''.``混蛋''两个字刚说出口,两人已忽然被人夹著脖子提了起来,两人平日以为已练得很不错的功武,竟连一招也使不出。大家都瞧得呆了,只听一个穿著绿衣服的怪人哈哈笑道:``这两个小子居然敢骂轩辕兄是混蛋,胆子倒实不小,轩辕兄若是不教训教训他们,以后别人就全都可以叫你混蛋了。''那大胡子火气本来已够大了,再被这人一挑拨,更是火上加油,两只手一抬,眼看这两人的脑袋就要被撞得稀烂。

幸好这时那圆脸胖子已拉住了他的手,笑道:``哈哈,今天是人家的好日子,你却一来就要杀人,岂非叫做主人的脸上难看?''那张嘴其大无比的人也笑道:``你要杀人,也不该砸坏他们的脑袋,我虽不吃人头,但一个人恼袋若被砸坏了,瞧著都恶心,老母鸡的头若已被砸得稀烂,你也吃不下去的,是么?''那大胡子.``哼''了一声,手一甩,两个人就飞了出去,各个跌在一张桌子上,脑袋恰巧栽入一碗刚端上来的酸辣汤里,烫得鬼叫,桌子上的碗筷杯盏,已被震得跌在地上,砸得粉碎。长棚里立刻大乱,有些小姑娘,老太婆,已吓得鬼叫著往外面逃,有些小孩子更已吓得放声大哭起来。

突听一人道:``是那位朋友在这里撒野,莫非是想给我兄弟难看么?''这人说话的声音也并不十分响亮,但一个字一个字说出来,每个人都听得清清楚楚,而且语声中自有一种慑人的威力,叫人不敢不听话,哭声,叫声,嘈乱声,竟全都被这声音压了下去。

只见一个年轻人站在船头,背负著双手,看来文诌诌的,就好像是个刚入学的秀才,但气度沉稳,站在那里如山停岳峙,明眼人一望而知,此人必是个内外兼修的武林高手!长棚里的人纷纷闪开,让这些怪人走了过去。

那圆脸胖子嘴里打著哈哈,道:``乡下人毛手毛脚,若是礼数欠周,小朋友你原谅则个。''他虽然像是在赔礼,却开口就叫人.``小朋友'',那人面色一沉,似乎要发作,但忽然又似想起了什么,面上露出了惊奇之色,目光在这些人面上一扫,又瞧见了打扮得怪里怪气的花无缺。

这一看更吃惊,失声道:``各位莫非是\ldots\ldots 莫非是\ldots\ldots{}''那圆脸胖子笑道:``小朋友,我们的名字你最好莫要说出来,否则只怕要说脏你的嘴。''那人沉吟了半晌,微一抱拳道:``在下秦剑\ldots\ldots{}''他刚说了四个字,船舱里已又走出几个人来,有也男有女,女的固然是千娇百媚,艳丽中带著华丽,男的也都是风度翩翩的浊世佳公子,他们显然都知道来的是些什么人了,但面上却仍然都带著微笑。他们若是不知道这些人的来历,含笑迎客本是礼数当然,但知道这些人的底细后,居然还能带著微笑,这就很难得了。江湖中人见到.``十大恶人''时,通常不是怒发冲冠,就是咬牙切齿,不是伸手就打,就是掉头就跑的。

哈哈儿先打了哈哈,大笑道;.``你们瞧,人家慕容家的姑爷们多有风度,多有教养,瞧见咱们这几块料,礼貌居然还如此周到。''屠娇娇嘻嘻笑道:``这才叫盛名之下无虚士,否则人家千娇百媚的大姑娘怎么会嫁给他们呢?''李大嘴长身一揖,道:``在下等闻得公子们家有喜事,是以特来致贺,却不如公子们可容得在下等这些山野狂夫登堂入室么?''站在船头的除了三姑爷秦剑外,还有大姑爷.``美玉剑客''陈凤超夫妻,二姑爷南宫柳夫妇,四姑爷.``梅花公子''梅仲良夫妇,五姑爷.``神眼书生''骆明道夫妇,江南武林的精华,可说已大多在此。

他们见到被打扮得奇形怪状的花无缺,面上都不禁露出了惊讶之色,但还是满面笑容,彬彬有礼。

直等李大嘴的话全都说完了,.``美玉剑客''才抱拳笑道;.``各位既肯赏脸,便是在下等的贵客\ldots\ldots{}''慕容双抢著说道:``何况轩辕先生更是我们新姑爷的生死之交呢?各位快请上船吧。''李大嘴也抱拳笑道:``既是如此,在下等就恭敬不如从命了。''这其中只有秦剑和.``梅花公子''面上微带著警戒之色,屠娇娇走过他们面前时,忽然回头一笑,道;.``你放心,咱们今天是专程喝喜酒来的,既不会找麻烦,也不会偷东西,你用不著像防小偷似的防著我们。''轩辕三光大声道:``不错,今天是我黑老弟的大喜之日,若有那个龟儿子敢胡说八道,老子第一个先找他算帐。''白开心冷笑道;.``就凭你,只怕还差著一点,李大嘴吃人的瘾若又发了,你难道还能用脑袋塞住他的嘴不成!''这几人一面说,一面笑,嘻嘻哈哈,骂骂咧咧的全都上了船,竹棚中,人人侧目而视,不知道这几人究竟是什么玩意?这些贵人公子们为何要对他们如此客气?船舱中居然能摆得下好几桌酒,六姑爷.``小白龙''夫妇,七姑爷.``洞庭才子''柳鹤人夫妇,八姑爷.``万花剑''左春生夫妇,以及.``神拳''顾人玉,和.``小仙女''张菁,自然全都在船舱里。

小仙女瞧见他们几个人走进舱,就斜著眼睛瞪他们,但大多数人的目光,却还是都在好奇的望著花无缺。他们实在猜不透.``移花宫''的传人怎么会变得如此模样?但有教养的世家子弟是绝不能过问别人私事的,别人若不说,他们心里就算好奇得要命,也只有装作没有见到。

他们几个人恰好占据了一桌,杜杀高据在首席,坐在主位相陪的是.``美玉剑客''陈凤超和南宫柳。这两人温文尔雅,礼貌周到,坐在这一桌奇形怪状的人中间,更显得品貌出众、风神如玉。若是换了平日,他们和花无缺惺惺相惜,一定要倾心结纳,但此刻他们却连看也不便多看花无缺一眼。

花无缺更是眼观鼻,鼻观心,木头人似的坐在那里,就彷佛是坐在无人的旷野之中,别人是在可怜他也好,是在窃笑也好,也已全不放在心上。酒过三巡,一双新人竟还末露面"。

李大嘴忽然道:``既有喜事,为何无礼乐?''陈凤超沉吟著,陪笑道:``仓卒之间,难以齐备,还望各位恕罪。''李大嘴正色道:``纵然如此,礼亦不可废,何况\ldots\ldots{}''屠娇娇抢著笑道:``何况咱们这里还有两对新人,要沾沾你们的喜气,等著和九姑爷、九姑娘一齐成礼哩。''陈凤超道:``哦?''南宫柳道;.``却不知新人是\ldots\ldots{}''也们虽然慎重而多礼,但此时还是忍不住瞧了瞧花无缺,只见花无缺苍白的睑上,既无悲切之容,亦无欢喜之色。他身旁一个美丽少女的表情却复杂得多,复杂得令人更猜不透这究竟是怎么回事。

哈哈儿道:``哈哈,常言道,好事成双,又道,一二不过三,三对新人一齐成礼,日后这三对夫妇必定三多,多福多寿多子孙。''陈凤超微微一笑,道:``阁下善颂善祷,这一番好意在下更无推却之理,只可惜\ldots\ldots{}''李大嘴皱了皱眉,道:``只可惜什么?''陈凤超淡淡道;.``只可惜舍下九妹吉礼已成,此刻已驾舟归去。''南宫柳接著道:``各位想必也知道,九妹夫妻俱都饱□忧患,是以这一次他们既然想静静的度过此一佳期,在下等自不便反对的。''屠娇娇.李大嘴他们对望了一眼,居然声色不动。

哈哈儿道:``哈哈,若是换了别人这么说,我们一定要以为他这是在瞧不起人,但这话既然是从两位嘴里说出来的,那自然就不同了。''陈凤超道:``多谢。''

屠娇娇嘻嘻笑道:``若是换在平日,各位见到我们这几个人,少不得要替天行道的,因为各位全都是大大的好人,好人遇著恶人,正如冰炭不能相容,是么?''陈凤超微笑不语。

屠娇娇道:``所以,若是换在平日,我们也绝不敢来拜望你们,因为.''慕容``家声势大得吓人,我们实在也惹不起。''陈凤超欠身道:``不敢。''

屠娇娇道:``但今天可就不同了,我们就因为早已算准各位今天绝不会给我们难看的,所以才敢到这里来\ldots\ldots{}''哈哈儿道;.``哈哈,常言道,既来之,则安之,我们既已来了,就少不了得要厚著脸皮赖在这里,好在各位俱是彬彬有礼的君子,今天又是大好的日子,我们就算有些失礼,各位也绝不会将我们赶走的。''另一张桌上的秦剑忽然长身而起,沉声道:``各位究竟有何打算,不妨\ldots\ldots{}''李大嘴大笑著接口道;.``在下等也没什么别的打算,只不过是想借各位这里作喜堂,为这两对新人成亲而已。''秦剑还想说话,陈凤超却拦住了他,微笑道:``各位既肯赏脸,这又是大好的喜事,在下等欢迎唯恐不及,只不过\ldots\ldots 无乐不能成礼。''李大嘴悠然道;.``子日!嫂溺叔援之以手,事急便可从权,何况,乐为礼奏,便无须悦耳,是么?''陈凤超笑道:``阁下通达,非弟能及。''

李大嘴抚掌大笑道;.``既是如此,何患无乐?''他忽然用两根筷子,在碗上敲打起来,哈哈儿也用一双手包著嘴,.``呜哩哇拉''的吹个不停。

屠娇娇笑得直不起腰来,道;.``此乐只应天上有,人间那得几回闻?有此妙乐还不行礼?''她将白夫人和钱心兰一边一个架了起来,白开心瞪著眼,忽然咧嘴一笑,也架起了花无缺。

李大嘴一面敲著碗,一面大声道;.``新人行礼,一拜天地\ldots\ldots{}''慕容家的姊妹们虽然都是秀外慧中的才女,八位姑爷也都是声名久著的俊杰,但实在也没有遇到过这么荒唐这么离奇的事,大家面面相觑,竟没有一人想得出如何应付之策。

就在这时,突听阴九幽阴森森的语声叱道:``什么人?''又听得一人笑道:``我不是人!''这两句话传入耳里,大家不禁全都一惊。

李大嘴他们虽然明知阴九幽必定游魂般在附近,但他遇见的人却是谁呢!.``我不是人''这四个字,是阴九幽自己常说的。

阴九幽显然也怔了怔,才怪笑著道:``你不是人,难道还是鬼?''那人道:``一点也不错。''

阴九幽嵘嵘笑道:``你是鬼?你可知道我是什么?''那人道:``你只不过是.''半人半鬼``,我却是一整个鬼,你还有一半是人,我却完完全全不是人。''听到这里,白开心忍不住拍手大笑道;.``妙极妙极,想不到阴九幽今天真是白日见鬼了。''大家虽然都很惊讶,也不禁都觉得有些好笑。

只听那人大笑道;.``一点也不错,你们全都白日见鬼了,我就是白日鬼!''笑声中,一条人影已自舱外风一般卷了起来。船舱中可说没有一人不是武林中一等一的高手,屠娇娇,白开心,.``万花剑''左春生,.``神眼书生''骆明道,这几人的轻功在江湖中更是赫赫有名。但他们见到这人的轻功,还是不免吃了一惊。

李大嘴他们更知道.``半人半鬼''阴九幽只要缠住一个人,便如附骨之蛆,永远不会让那人脱身的。但这人竟轻轻松松的就自阴九幽身旁掠入船舱来,可见他的轻功竟比身法如幽灵般的阴九幽还高明得多。

他们实在不敢想像这人是谁!因为除了移花宫主和燕南天外,世上有这么高轻功的人实在不多。

但这人并不是燕南天,自然更不会是移花宫主。灯光下,只见他身高不满三尺,竟是个侏懦。别的侏儒长得必定畸形怪状,难看得很,这侏儒却是不同,他的头,手,脚,和身子的发育都很相称,一张脸更是眉清目秀,而且颔下还冒著五柳须,看来居然仙风道骨,很有几分道气。

他身上的打扮,却是非道非俗,穿著件青灰色的短袍,背后还斜插著剑,这柄剑比别人的匕首还短两寸,就像是小孩子的玩具。若是小孩子见到这人,一定会拉起他的手,要他陪自己捉迷藏,若是走江湖卖艺的见到此人,一定要认为是奇货可居,若是贵胄大臣见著此人,一定要将他引见给帝王,作宫廷的弄臣。

但屠娇娇见到此人,却忽然笑不出了,杜杀.李大嘴瞧见她面上变了颜色,心里也忽然想起一个人来。

这时阴九幽也跟著掠进船舱,似乎想要向这人出手,但屠娇娇.李大嘴却赶紧拦住了他,在他耳旁悄悄说了两句话。阴九幽面色也变了变,拍出去的手也立刻缩了回去。

只见这人四下作了个揖,笑嘻嘻道:``不速之客,闯席而来,恕罪恕罪。''陈凤超、南宫柳等人心里自然也很惊讶,但还是很客气的答礼,只有三姑娘慕容珊珊目光闪动,忽然道:``晚辈年纪小时,曾听说过江湖中有位奇侠,形迹如神龙,人所难测,晚辈久已想一睹风采了。''慕容双眼睛一亮,抢著道;.``三妹说的这位奇侠,可是人称\ldots\ldots 人称\ldots\ldots{}''那人哈哈笑道;.``姑娘用不著避讳,只管将.''鬼童子``这名号叫出来就是,我早已听得很习惯了,非但不会生气,而且还觉得这名字蛮不错的哩。''.``鬼童子''这三字说出来,陈凤超,南宫柳等人也不觉都为之耸然失色,他们小时候也曾听人说超过,此人不但轻功绝高,而且据说还是东瀛扶桑岛,伊贺谷,秘宗.``忍术''的唯一传人。

\hypertarget{ux7b2cux4e00ux767eux5341ux4e03ux7ae0-ux72c2ux72eeux94c1ux6218}{%
\chapter{第一百十七章
狂狮铁战}\label{ux7b2cux4e00ux767eux5341ux4e03ux7ae0-ux72c2ux72eeux94c1ux6218}}

据说.``鬼童子''最善于隐迹藏形,他若想来打听你的秘密,就算藏在你的椅子下面,你都休想能发觉到他。但此人五十年前便已成名,近三.四十年来已没有人再听到过他的消息,据说他又已远走扶桑,去领略那里的异国风光去了。又有人说,因为扶桑岛上的人,大多是矮子,所以他住在那里,觉得开心些。此人竟又忽然现身,来意实在难测。

陈凤超躬身道:``晚辈等久慕前辈的大名,今日能一睹前辈风采,实是不胜之喜。''鬼童子笑道:``你嘴里虽然这么说,心里只怕是想问我这老怪物为何到这里来吧?''陈凤超道:``不敢。''鬼童子道;.``其实你不问,我也要说的。''

陈凤超道:``是。''

鬼童子道:``我这次来,是为了两件事,第一件,我听说这位铁姑娘要成亲了,就特地去请了一班礼乐来,我可以保证那些人全都是一等一的好手,他们现在还没有到,铁姑娘就成礼了,岂非令我老头子脸上无光,所以,我只好请铁姑娘千万要等一等。''陈凤超等人暗中似乎都松了口气:``原来这老怪物不是为了我们来的。''李大嘴等人心里却不禁暗暗吃惊:``这老怪物和铁心兰又有什么关系?为何要为他的事担心?''鬼童子向他们嘻嘻一笑,道:``其实我老头子和这位铁姑娘根本就不认得,我只不过是天生的好管闲事而已。''李大嘴心里虽然还是有些怀疑,嘴里并没有问出来。在那.``恶人谷''闷了二十年之后,此番他们重出江湖,行事虽然有些迹近胡闹,但他们毕竟是.``十大恶人'',.``十大恶人''这名字毕竟不是随随便便就可以得来的,真的遇到大事时,他们每个人都很能沉得住气.

``还有一件事,说起来更有趣了。''鬼童子道:``这次我无意中救了一个人,这人据说是个混蛋,但我老头子天生的怪脾气,最喜欢和混蛋交朋友,因为别人都不喜欢跟混蛋交朋友,我若也和别人一样,那么混蛋岂非就很可怜了么?一个人若很可怜,又怎能称做混蛋呢?''这人当真是歪理十八篇,慕容姊妹们听得暗暗好笑。

白开心也笑道:``前辈若喜欢和混蛋交朋友,那是再妙也没有的了,因为这里的混蛋,比别的地方所有的混蛋加起来还多十倍。''他这人若不说两句挑拨雉间、尖酸刻薄的话,不但喉咙发养,而且全身都难过,正如一条狗见到屎时,你若想要它不吃,那实在困难得很。

鬼童子望著他嘻嘻一笑,道:``看来这位就是.''损人不利己``白开心了,果然名不虚传,我老头子这次上船来,就是为了要找你。''白开心吃了一店,道:``找\ldots\ldots 找我?为\ldots\ldots 为什么?我既不吃人,也不赌钱,这些人里,实在没有此我更老实的了。''鬼童子道:``其实也不是我老头子要找你,只不过我那混蛋朋友,跟你还有些手续未清,所以想跟你好好的谈谈。''他忽然高声唤道:``快来吧,你这条没牙老虎,难道真的已不敢见人了么!''这句话说出来,白开心就要开溜,只因他已猜出来的是什么人了,白夫人本来还在羞答答的,故作娇羞,听到这句话,也变了颜色。可是白开心纵然脚底抹了油,这时也跑不了的,他刚一掠而起,却已看到鬼童子的一张脸挡在他的跟前。

这时甲板上.``咚''的一响,已有个人大步走了进来,却不是那老婆被人抢走的白山君是谁。

白开心叹了口气,喃喃道:``这笔糊涂帐,该怎么样才能算得清呢?''李大嘴咧嘴一笑,道:``算不清就慢慢算,反正你们是同靴的兄弟还有什么话不好说呢?''白开心狠狠瞪了他一眼,恨不得找他拚命,可是这时白山君已走到他面前,他赶紧陪笑道;.``咱们都姓白,一笔写不出两个白字,千万莫要听心别人的挑拨离间伤了我们自家兄弟的和气。''李大嘴冷冷道:``一笔写不出两个白字,一只靴子怎么套得下两只脚呢?''白开心跳起来,似乎就要扑过去。

白山君反而拦住了他,居然笑道:``这位兄台说的其实也是实话,我\ldots\ldots{}''白开心叫道:``实话?他这简直是在放屁,我和你老婆并没有什么\ldots\ldots 什么关系,我也并不想娶她,你来了正是再好也没有了。''白山君道:``岂有此理,贱内既已和兄台成亲,此后自然就是兄台的老婆了,小弟虽不才,但也知道朋友妻,不可戏,怎能调戏大嫂哩。''他居然说出这么一番话来,大家全都怔住了。

白开心吃吃道:``你\ldots\ldots 你这是什么意思?你难道不想要回你自己的老婆?''白山君笑道;.``在下万万没有此意,这次在下到这里来,只不过是想和兄台办妥移交的手续而已,此后手续已清,谁也不得再有异议?''白开心怪叫道;.``我抢了你的老婆,你不想跟我拚命!''白山君道:``在下非但全无拚命之意,而且还对兄台感激不尽\ldots\ldots{}''白开心的鼻子都像是已经歪了,失声道;.``你\ldots\ldots 你\ldots\ldots 你感激?\ldots\ldots{}''白山君哈哈笑道:``在下享了她二十年的福,也该让兄台尝尝她的滋味了,她脾气虽然不好,醋性又大,虽然既不会烧饭,也不会理家,但有时偶然也会煮个蛋给兄台吃的,只不过盐稍微多放了些而已?''白开心听得整个人全都呆在那里,嘴里直吐苦水。

白夫人却跳了起来,嗄声道:``你\ldots\ldots 你这死鬼,竟敢说老娘的坏话\ldots\ldots{}''白山君笑嘻嘻道:``大嫂莫要找错对象,在下现在已不是大嫂的丈夫了,这点还求大嫂千万莫要忘记才好。''白夫人也怔了怔,再也说不出话来。

白山君长身一揖,笑道;.``但愿贤伉俪百年好合,白头到老,在下承两位的情,放了在下一条生路,日后必定要为两位立个长生祠,以示永生不忘大德。''他仰天打了两个哈哈,转身走了出去。

大家面面相觑,都有些哭笑不得,谁也想不到天下居然真的会有这么样的人,这么样的事。

过了半晌,只听这位白夫人喃喃道:``他不要我了,他居然不要我了,这是真的么,\ldots\ldots{}''白开心呻吟了一声,道:``若不是真的就好了,只可惜他看来一点也不像假的。''白夫人大叫道;.``这一定不是真的,他一定不是真心如此,我知道\ldots\ldots 我知道他现在一定难受得要发疯,我绝不能就这样让他走。''她一边叫著,一边往外面跑,在饿了三四天之后,白开心也们只让她吃了半个馒头和一小杯水,现在她就将这点力气全都用了出来,就好像生怕有人会在后面拉住她两条腿似的。其实谁也没有拉住她的意思,尤其是白开心。

白开心本来倒也觉得这女人蛮有趣的,最有趣的一点,就因为她是别人的老婆,大多数男人都觉得别人的老婆比较有趣,何况是.``损人不利己''白开心,所以别人要他和这女人成亲,他并没有十分反对。他只希望白山君知道这件事后,会气得大哭大叫,来找他拚命,谁知白山君却将她双手送给了他,就好像将她看成一堆垃圾似的,还生怕送不出去,这下子白开心才真的失望了。他忽然也觉得这女人实在并不比一堆垃圾有趣多少。

这就是大多数男人的毛病,就算是条母猪,假如有两个男人同时抢著要她,那么这母猪全身上下每个地方都会变得漂亮起来,但其中假如有一个男人忽然弃权了,另一个男人立刻就会恍然大悟:``原来她是条母猪,只不过是条母猪。''白开心现在就恨不得这女人赶快跑出去,越快越好,若是一脚踩空,掉在河里,那更是再好也没有了。谁知白夫人刚冲到鬼童子面前,鬼童子一伸手,夹著脖子将她拎了起来。他身材虽然比她矮得多,但也不知怎地,偏偏能将她从地上提起来,而且看来还轻松得很。

他一直将她拎回白开心的身旁,才放下来,白夫人直著眼睛似乎已经被吓呆了己连她自己都弄不懂自己是怎会被这小矮子拎起来的。

她嗫嚅著道:``我要去找我的丈夫都不行么?''鬼童子板著脸道:``你的丈夫就在这里,你还要到那里去找?''白夫人道:``可是\ldots\ldots 我并不想嫁给他,这完全是被别人强迫的。''鬼童子道:``你若不想嫁给他,方才为什么要羞答答的做出一副新娘子的模样来?''白夫人用力揉著眼睛,想揉出眼泪来,可惜她的眼泪并不多,而且很不听话,该来的时候偏偏不来。

鬼童子笑了,忽然拍了拍花无缺的肩膀,他要踮起脚尖来,才能拍得到花无缺的肩膀。

他笑嘻嘻的道:``小伙子,你能娶得到我们的铁大侄女做老婆,实在是你的运气。''花无缺虽然是站著的,但他除了还能站著外,再也没有做别的事的力气,也许他还能说话,可是,到了这种时候,他还能说什么?鬼童子望著他脸上的神色,皱眉道:``无论如何,你总算得到她做老婆了,你还有什么不开心呢?''铁心兰忽然道;.``前辈,我\ldots\ldots 我\ldots\ldots{}''屠娇娇他们并没有点住她的哑穴,因为他们并不怕她说话,假如她说了不该说的话,他们随时都可以阻止她的。

但是现在,有这鬼童子在她面前,他们只好让她说下去,因为谁都不愿被人夹著脖子拎起来的。

这鬼童子就算没有别的功夫,就只这一样功夫,已经够要命的了,因为他们方才看到他拎起白夫人的时候,那么样一伸手,谁也不能保证自己一定能躲得开,他伸手的时候,就像他的手本来就长在白夫人的脖子似的。幸好铁心兰只说了三个字,就说不下去了。

鬼童子却笑道:``我知道你有很多话要问我,但现在不要著急,用不著多久,你什么事都会明白的。''慕容家的姊妹已开始在悄悄交换眼色,似乎正在商量该如何来招待这怪人,慕容家的人从来不愿对客人失礼。

但她们还没有说话,鬼童子已笑著道:``你们用不著招待我喝酒,我向来不喝酒的,因为我个子太小,要喝酒一定喝不过别人,所以就索性不喝了。''陈凤超陪著笑道;.``既是如此,却不如前辈!\ldots\ldots{}''鬼童子道:``你是不是要问我喜欢什么?好,我告诉你,我只喜欢看女人脱光了翻斤斗,你们惹想招待我,就翻几个斤斗给我看好了。''慕容姊妹脸上都变了颜色,秦剑,梅仲良,左春生,已振衣而起,屠娇娇眼睛却发了光,只望他们快打起来。谁知就在这时,江上忽然瓢来一阵乐声,在这清凉的晚风中,听来是那么悠扬那么动人,而且还充满了喜悦之意。无论任何人听到这种乐声,都不会再打起来的。

乐声乍起,四下的各种声音立刻都安静了下去,似乎每个有耳朵的人全都被这乐声沉醉了。

就连.``血手''杜杀的目光都渐渐变得温柔起来,乐声竟能使每个人,都想起了自己一生中最欢乐的时光,最喜悦的事。乐声中,少年夫妻们已情不自禁,依偎到一齐,他们的目光相对,更充满了温柔与幸福。

花无缺的目光也不由自主,向铁心兰望了过去。铁心兰也正在瞧著他。他们心里都已想起他们在一起所经历过的那段时光。在那些日子里,他们虽然有时惊惶,有时恐惧,有时痛苦,有时悲哀,但现在,他们所想起的却只有那些甜蜜的回忆。

鬼童子看著他们,微笑著喃喃道:``你们现在总该相信,我请来的这班吹鼓手,非但是天下第一,而且空前绝后,连唐明皇都没有这种耳福听到的。''乐声越来越近,只见一艘扁舟,浮云般自江上飘了过来,舟上灯光辉煌,高挑著十余盏明灯,灯光映在江上,江水里也多了十余盏明灯,看来又像是一座七宝光幢,乘云而下。

舟上坐著七、八个人,有的在吹箫,有的在抚琴,有的在弹琵琶,有的在奏竽,其中居然还有一个在击鼓。那低沉的鼓声,虽然单调而无变化,但每一声都彷佛击在人们的心上。令人神魂俱醉。

灯光下,可以看出这些人虽然有男有女,但每一个头发都已白了,有的甚至已弯腰驼背,像是已老掉了牙。但等到他们上了船之后,大家才发现他们实在比远看还要老十倍,没有看到他们的人,永远无法想像一个人怎会活得到这么老的,甚至就连看到他们的人也无法想像,这么多老头子.老太婆居然坐在一条很小的船上奏乐,这简直就是件令人无法想像的事。

更令人无法想像的是,这种充满了青春光辉,生命喜悦的乐声,竟是这些已老得一塌糊涂的人奏出来的,这种事若非亲眼瞧见,谁也无法相信。但现在每个人都亲眼瞧见了,只不过谁也没有看清他们是怎么样上船的,这小船来得实在太快。

等到慕容姊妹想迎出去的时候,这些老人忽然已在船头上了,甚至连乐声都没有停顿过片刻。只见击鼓的老人头发已自得像雪,皮肤却黑如焦炭,身上已瘦得只剩下皮肤骨头,他用两条腿夹著一面很大的鼓,这面鼓像是比他的人还要老,看起来重得很,但是他用两条腿一夹,连人带鼓就都轻瓢瓢掠上了船,看来又彷佛是纸扎的,只要一阵小风就能将他吹走。

陈凤超拾先迎了上去,躬身道:``前辈们世外高人,不想今日竟\ldots\ldots{}''他话还没有说出,击鼓的老人忽然一瞪眼睛,道:``你是不是姓曹?''陈凤超怔了怔,道:``晚辈陈凤超。''

他.``陈''字刚说出口来,那击鼓老人忽然怒吼道:``姓陈的也不是好东西。''吼声中,他枯瘦的身子已暴长而起。

鬼童子皱了皱眉,一把拉住了他,道:``你就算恨姓曹的,姓陈的人又有什么关系?''击鼓老人怒道;.``谁说没有关系,若不是陈宫放了曹操,我祖宗怎会死在曹操手下''他这么样一闹,乐声就停止了下来,大家也不知道他胡说八道在说些什么,只有慕容珊珊忽然笑道:.

``如此说来,前辈莫非南海烈士弥衡的后人么?''击鼓老人道:``不错,自蜀汉三国以来,传到我老人家已是第十八代了,所以我老人家就叫弥十八。''陈凤超这才弄明白了,原来这老人竟是弥衡的子孙,弥衡以.``渔阳三捶''击鼓骂曹,被曹操借刀杀人将他害死,现在这弥十八却要将这笔帐算到陈凤超的头上,陈凤超实在有点哭笑不得。

只听慕容珊珊正色道:``既是如此,前辈就不该忘了,陈宫到后来也是死在那奸贼曹阿瞒手里的,所以前辈和姓陈的本该敌忾同仇才是,若是自相残杀,岂非让姓曹的笑话。''弥十八怔了半晌,点头道:``不错,不是你提醒,我老人家倒忘了,你这女娃儿有意思。''突听一人道;.``这里可有姓锺的么?''

这人高瘦顾长,怀抱著一具瑶琴,白开心只当他和姓锺的人有什么过不去,立刻指著李大嘴道:``这人就姓锺。''他以为李大嘴这次一定要倒楣了,因为慕容家的姑娘绝不会帮李大嘴说话的,谁知道这抚琴老人却向李大嘴一揖到地,道:``老朽俞子牙,昔日令祖子期先生,乃先祖平生唯一知音,高山流水传为千古佳话,今日你我相见,如蒙阁下不弃,但请阁下容老朽抚琴一曲。''李大嘴少年时本有才子之誉,否则铁无双也就不会将女儿嫁给他了,伯牙先生和锺子期的故事也自然是知道的,所以白开心说他姓锺,他一点也没有反对,此刻也立刻长揖道:``前辈如有雅兴,在下洗耳恭听。''只见俞子牙端端正正坐了下来,手拨琴弦,□琮一声响,已令人觉得风生两腋,如临仙境。

李大嘴装模怍样的闭起眼睛听了许久,朗声道:``巍巍然如泰山!快哉,妙哉。''俞子牙琴音一变,变得更柔和悠扬。

李大嘴抚掌道:``洋洋然如江河,妙哉,快哉。''愈子牙手划琴弦,戛然而止,长叹道:``不想千古以下,锺氏仍有知音,老朽此曲,从此不为他人奏矣。''屠娇娇早已看出这些老人不是身怀绝技的高手,但她却末想到他们竟如此容易受骗。

她忍不住暗笑忖道:``一个人越老越糊涂,这话看来倒没有说错。这些人实在是老糊涂了。''只见俞子牙竟拉起了李大嘴的手,将那些老头子.老太婆一一为他引见,吹箫的就姓萧,自然是萧弄玉的后人,击筑的就姓高,少不得也和高渐离有些关系,吹笛的会是什么人的后代呢?原来是韩湘子的后人,自然和文起八代之衰的韩愈也有亲戚关系。

慕容姊妹在一旁听得真是几乎要笑破肚子,她们已惭渐觉得这些人都是疯子,而且疯得很有趣。

最妙的是,吹竽的一人竟自命为南郭先生的后代,而且居然叫南郭生,慕容珊珊实在忍不住了,嫣然道:``齐宣王好吹竽之声,必令三百人同次,其中只怕有二百九十九人是比南郭先生吹得好的,前辈吹竽妙绝天下,怎么会是南郭先生的后人呢?''这位南郭先生矮矮胖胖的,看来很和气,所以慕容珊珊才敢开开他玩笑,他果然也没有生气,笑眯眯道:``姑娘只知道先祖滥竽充数,传为千古笑谈,却只知其一,而不知其二。''慕容珊珊道;.``晚辈愿闻其详。''

南郭生道;.``宣王死,□主立,欲令三百人一一吹竽,先祖闻得后,就乘夜而逃,这段故事是人人都知道的,却不知先祖逃走之后,从此奋发图强,临死前已成为当代吹竿的第一高手,而且严戒后人,世世代代都不能不学吹竽,为的就是要洗刷.''南郭吹竽``这段笑话。''他笑了笑,接著道;.``姑娘放眼天下,还有谁吹竽能比姓南郭的更好。''慕容珊珊立刻整容谢道:``晚辈孤陋寡闻,失礼之处,还望前辈恕罪。''其实谁都可以看出南郭先生并不姓南郭,弥十八并不姓□,那位姓韩的老头子更不会是韩湘子的后代。

因为韩湘子一生中根本就没有娶老婆,那里来的儿子,没有儿子,孙子更不会从地下钻出来了。

但这些老人一定要这么说,大家也没有法子不相信。大家虽然也都已看出,这些老人必定都是五、六十年,甚至六、七十年前的江湖名侠,怎奈谁也猜不出他们本来的姓名身份。铁心兰更猜不透这些老人为什么要赶来为自己奏乐,这些人的年纪每一个都可以做她的太祖父了,怎会和她有什么渊源关系?慕容大姑娘温柔端庄,正是.``大言不出,小言不入''的贤妻良母,她始终郡是面带著微笑,静静的坐在那里,此刻忽然悄悄拉她夫婿的衣袖,柔声道:``时候已不早,大家也都很累了\ldots\ldots{}''陈凤超微笑著拍了拍她的手,道:``你的意思我知道。''其实他自然也早就看出今日的局面已越来越复杂,也不愿再和这些稀奇古怪难的邪门外道再纠缠下去,当下抱拳笑道;.``此刻礼乐俱已齐备,还是快些为这两对新人成礼吧,大家也好痛痛快快的喝几杯喜酒。''屠娇娇拍手笑道:``这话对极了。''

哈哈儿道:``哈哈,常言道,春宵一刻值千金,咱们只顾著打岔,却忘了新人们正急著要入洞房哩。''他们也看出这些老人来历诡异,也巴不得早些脱身才好。谁知鬼童子却忽然大声道:``不行,现在还不行,还要等一等。''屠娇娇笑道:``难道前辈们也约了客人来观礼么?''鬼童子道:``不是客人,是主人。''

屠娇娇也不禁怔了怔,道:``主人?主人岂非都在这里么?''鬼童子再也不理她,却向弥十八道:``老么是不是跟你们一齐来的?''弥十八翻了翻白眼,道:``他不跟我们一齐来,跟谁一齐来?''鬼童子道:``他的人呢?''

弥十八道:``他的人在那里,你为何不问他去。''鬼童子道:``我若知道他在那里,还问个屁。''弥十八瞪眼道;."你不知道,我又怎会知道,我又不是他的老子。

鬼童子笑骂道:``你这人简直跟你那老祖宗是一样的臭脾气。''南郭生笑道:``你明知他的臭脾气,为何还要问他,为何不问我呢。''李大嘴在一旁听得暗暗好笑,这几人原来也是越老越天真,斗起嘴来,竟不在自己之下。

陈凤超生怕他们再纠缠下去,幸好南郭生已接著道:``老么本来和我们一齐坐船来的,但他却嫌船走得太慢,所以就跳上岸,要一个人先赶来。''俞子牙道:``这就叫欲速则不达。''

鬼童子笑道:``他这火爆栗子的脾气,只怕到死也改不了。''那吹箫女史插口笑道:``以他近来的脚程,就算绕些远路,此刻也早该到了,就只怕他又犯了老脾气,半路上又和人打了起来。''韩笛子笑道:``若是真打起来,那只怕再等三天三夜也来不及了。''屠娇娇眼珠子一转,忽然道:``前辈们的这位朋友,难道和人一动上手就没完没了的么?''鬼童子叹道:``不打得对方磕头求饶,他死也不肯罢手的。''屠娇娇瞧了李大嘴一眼,道:``莫非是他?''

李大嘴也已想起了一个人,突的失声,道:``前辈们的这位朋友莫非是\ldots\ldots{}''他话还没有说完,突听岸上一人大吼道:``李大嘴,恶赌鬼,你们这些孙子王八蛋在那里,快滚出来吧!''屠娇娇叹了口气,道:``一点也不错,果然是这老疯子。''轩辕三光拼掌大笑道:``这个龟儿子一来,就更热闹了。''一听到那雄狮般的大吼,铁心兰全身就不停的发起抖来,也不知是太惊奇,还是太欢喜。慕容姊妹却在暗暗奇怪,这些老怪物的兄弟又怎会是.``十大恶人''的老朋友呢?她们实在想不通。

只见李大嘴和轩辕三光已跳上船头,大笑著道:``你这老疯子还没有死么?''岸上一人也大笑著道:``你们这些孙子王八蛋还没有死,我怎么舍得先死?''笑声中,一人跳上了船头,这么大的一条船,竟也被他压得歪了一歪,杯中的酒都溅了出来,这人份量之重,也就可想而知了。

但若说他轻功不行,却也未必,他自岸边跃上船头,这一掠之势,至少也有四五丈远近!梅花公子,神眼书生,这些人的轻功在江湖中也可算是顶尖的身手,但自忖能力,末必能一掠四丈。这人的轻功既然不弱,落下来时却偏偏要故意将船震得直晃,也就难怪李大嘴他们要骂他是.``老疯子''了。

大家连看都不必看,已知道来的必定又是个怪人,一看之下,更不禁倒抽了口凉气,这人身材也不太高,最多也只不过有六、七尺,但横著来量,竟也有五尺六七,一个人看来竟是方的,就像是一块大石头。他的头更大得出奇,头砍下来称一称,最少也有三五十斤,满头乱蓬蓬的生著鸡窝般的一头乱发,头发连著胡子,胡子连著头发,也分不清什么是胡子,什么是头发了,鼻子嘴巴,更是连找都找不到。远远望去,这人就像是一块大石块上蹲著一头刺□,又像是一头被什么东西压得变了形的雄狮。

只见这人一跳上船头,就和李大嘴、轩辕三光两人嘻嘻哈哈的纠缠到一齐,三个人加起来已经快二百多岁了,却还是老不正经。陈凤超看得只有苦笑,正不知是该迎出去还是不该迎出去,那怪人忽然一把推开了李大嘴,吼道:``我倒忘了先看看你们这些孙子王八蛋究竟替我女儿找了个什么样的女婿,若是不合我的意,看我不把你们打扁才怪。''他狂吼著跳了起来,屠娇娇迎上去笑道:``我们替你找的这女婿,凭你这老疯子就算打锣也找不到的,包你满意。''铁心兰看到这怪人,眼泪早已忍不住夺眶而出,挣扎著扑了上去,颤声道:``爹爹\ldots\ldots{}''她满心凄苦,满怀幽怨,只唤了这一声,喉头已被塞住,那里还能说得出第二个字来。

花无缺这时也知道.``狂狮''铁战到了,看到铁心兰这样的女儿,他实在想不到她的爹爹竟是这副模样。

铁战拍著她女儿的头,大笑道:``好女儿莫要哭,老爸爸没有死,你该高兴才是,哭什么?''他话还没有说完,已跳到花无缺面前,从头看到脚,又从脚看到头,将花无缺仔仔绌绌瞧了几遍。花无缺似已饿得完全麻木了,动也不动。

铁战点著头道:``看来这小子长得倒还蛮像人样的,只不过\ldots\ldots 怎地连站都站不稳,莫非你们找的竟是个痨病鬼么?''鬼童子笑道:``这不是痨病,他这病只要有新出笼的包子就能站得好。''铁战怔了怔,道:``他这难道是饿病?''

鬼童子笑道:``不错。''

铁战跳了起来,怒吼道:``是谁把我女婿饿成如此模样?''鬼童子道:``除了你那老朋友还有谁。''

铁战霍然一翻身,双手张舞,已抓住了哈哈儿和屠娇娇的衣襟,竟将这两人硬生生提了起来。他武功在.``十大恶人''中算来本非好手,只不过打起架来特别不要命而已,若论真实的功夫,他也末必能就强过屠娇娇。但现在他随手一抓,就将屠娇娇和哈哈儿两个都抓了起来,他们两人非但不能抵抗,竟连闪避都闪避不开。李大嘴等人都不禁骇了一跳,谁也想不到他武功竟有如此精进,但目光一转,只见弥十八,俞子牙等人面上都露出得意之色,不问可知,他武功必定跟这些老怪物学的。哈哈儿只觉脖子都快断了,想打个哈哈,却连气都喘不过来,吃吃道;.``老\ldots\ldots 老朋友有话好说,何必动手妮!''铁战怒道:``什么好说歹说,你自己吃得一身肥肉,为什么将我女婿饿成这副模样。''屠娇娇陪笑道:``铁兄有所不知,若非咱们饿他一饿,他只怕早就跑了。''铁战道;.``跑?为什么要跑?''

屠娇娇道:``铁兄为何不问问他自己。''

铁战果然松了手,却抓起了花无缺的衣襟,吼道:``我问你,你为什么要跑?难道我女儿还配不上你这病鬼么?''铁心兰揪住了她爹爹的手臂,道:``爹爹,快放开他,这不关他的事。''她心里的矛盾和痛苦,又怎能当著这么多人面前说出来。

铁战顿足道:``这究竟是怎么回事?\ldots\ldots 别的事我都不管,我只问你,你愿不愿意嫁给这小子!''铁心兰垂首道:``我\ldots\ldots 我\ldots\ldots{}''铁战怒道:``你现在怎地也变得扭扭捏捏起来了,这还有什么不好说的,愿意就愿意,不愿意就不愿意,只要你点点头,这小子就是你老公了,只要你摇摇头,我就立刻替你将这小子赶走。''铁心兰的头却连动也不能动,她既不能点头,也不能摇头,想起花无缺对她的深情,她怎么能摇头。她知道只要自己摇一摇头,此后只怕永远见不著花无缺了,但想起了那可恨又可爱的小鱼儿\ldots\ldots 却叫她又怎能点头。

这时她的心情,只怕连最善解人意的人也无法了解,又何况是从来不解这种儿女之情的.``狂狮''铁战。他简直快被急疯了,跺脚道:``我不要你开口,但你连头都不会动了么?''铁心兰的头硬是纹风不动。

大家面面相觑,全都瞧得发了呆,慕容姊妹虽然玲珑剔透,但也著实猜不透她心里究竟在打什么主意。这其中了解她心意的只怕唯有花无缺。但他自己也是满心酸楚,他知道铁心兰不肯摇头,只为了不忍让他伤心,但铁心么就算点了头,他难道就不伤心了么?他忍不住黯然道:``我\ldots\ldots{}''谁知他刚说了一个字,铁战就跳起来怒吼道;.``闭嘴,谁要你说话的,只要我女儿愿意,你就得娶她,我女儿若不愿意,你就得渡蛋!''这句话说出来,连慕容姊妹都听得有些哭笑不得,只觉得这么不讲理的老丈人,倒也天下少有。却不知.``狂狮''铁战若是讲理的人,也就不会名列在.``十大恶人''之中了。

萧女史忽然一笑,道:``女人家若是既不肯点头,也不肯摇头,那就是愿意了。''她虽已白发苍苍,满面皱纹,老得掉了牙,但眼神却仍很有风致,想当年必定也是位在情场中打过滚的人物。

铁战一拍大腿,拍手道:``不错,倒底还是萧大姊憧得女儿家的意思\ldots\ldots{}''。

\hypertarget{ux7b2cux4e00ux767eux5341ux516bux7ae0-ux5927ux4f17ux60c5ux4eba}{%
\chapter{第一百十八章
大众情人}\label{ux7b2cux4e00ux767eux5341ux516bux7ae0-ux5927ux4f17ux60c5ux4eba}}

谁知铁心兰却立刻道:``我\ldots\ldots 我不是这意思。''铁战急得直抓头发,道:``你到底是什么意思呢?说呀。''铁心兰垂下头,又变成了哑吧。

这情况莫说铁战快急得发疯,就连别的人也不禁著急起来了。

铁战跳著脚道:``你们这些人难道没有一个人知道她意思的?''轩辕三光笑了笑,道;.``我们知道有个人是知道她意思的。屠娇娇。''最后一个.``娇''字还末说出囗,铁战已又一把拎起了屠娇娇,怒吼道:``你既然知道,为何不说,却害得老子著急。''屠娇娇陪笑道:``你女儿的心意连你都不知道,我怎会知道,这全是恶赌鬼恨我方才得罪了也,所以现在来报仇。''铁战厉声道:``放屁,恶赌鬼一辈子从来不说谎的,我数到『三』字,你若还不说,我就立刻宰了你。''他连.``一''字还没有数,屠娇娇已苦笑道:``好,说就说吧,只不过说出来你更没法子了。''她知道.``狂狮''铁战说得出做得到,到了自己性命交关时她也只有将什么事都说出来了。

铁战道:``只要你说出来,我就有法子。''

鬼童子道;.``就算他没有法子,我们也可以替他想法子。''屠娇娇道:``你女儿本来是很愿意嫁给这位花花公子的,可是,可是\ldots\ldots 她还有个心上人,她既想嫁给花花公子,又想嫁给那人。''萧女史道:``这两人,谁比谁强些呢?''屠娇娇笑了笑道:``两人半斤八两,各有各的好处,我若是她,实在也不知道究竟该要嫁给谁才好。''听到这里,铁心兰心里又是羞惭,又是痛苦,真恨不得立刻死了算了,但想到他们既已提起.``小鱼儿''来,小鱼儿说不定就有了生机,她也只有暗咬著银牙,将眼泪往肚子里流。

只听萧女史叹道:``无论多么强的女人,遇著这件事也没法子,这也难怪铁姑娘如此痛苦,若换她是我,我也\ldots\ldots{}''白开心道;.``她若喜欢两个人,就叫她同时嫁给那两个人好了,左右逢源,岂非再妙也没有。''他狗嘴里果然永远吐不出象牙来,别人都以为.``狂狮''铁战这下子就算不打扁他鼻子,也要打破他脑袋。

谁知道铁战也跳了起来,拍掌大笑道;.``好主意,果然是好主意,一个男人可以娶两个老婆,一个女人为什么不能嫁两个老公?''萧女史叹了口气,喃喃道;.``我是个女人,你却是个疯子。''铁战大笑道:``疯子就疯子,为了我女儿,做做疯子又有何妨。''他大笑著拉起他女儿的手,又道:``还有一个人是谁?只管说出来没关系,全有爹爹我替你作主。''铁心兰的脸早已由赤红转为苍白,只恨不得自己三年前就已死了,那里还能说得出一个字来。甚至连慕容姊妹都在暗暗为她叹息,觉得这女孩实在可怜,居然有这么样一个宝贝父亲。

轩猿三光眼珠子一转,忽又笑道:``格老子,这种事女娃儿家怎么说得出口呢?告诉你,那小子姓江,叫做小鱼儿。''.``小鱼儿''这三个字说出来,慕容姊妹俱都不禁为之动容,小仙女的脸立刻气得通红,屠娇娇他们却在悄悄皱眉头,只有花无缺的眼睛顿时亮了,因为他终于已明白了轩辕三光的用心.

``小鱼儿,小鱼儿,小鱼儿\ldots\ldots{}''铁战将这名字翻来覆去的念了好几遍也皱著眉道;.``这小子怎会叫这种古里古怪的名字。''白开心笑嘻嘻道:``这只因他本来就是个古里古怪的人,无论谁遇著他,至少也要倒楣三年。''铁战咧嘴一笑,道:``你小子少来挑拨离间,只要我女儿欢喜,他就算叫小王八都没关系?''轩辕三光忽又叹了囗气,道:``只可惜我现在也不知道这条小鱼儿在那里?''铁战道:``那倒没关系,只要有这么一个人,我就能找得到。''他用力拍著鬼童子肩头,大笑道:``就算我找不到,你也找得到的,对不对。''轩辕三光道:``不对。他要找别人也许都很容易,但要找这小鱼儿,却难得很,难得很。''铁战又瞪起了眼,道:``为什么?''轩辕三光瞟了屠娇娇他们一眼,道:``只因小鱼儿已被他们藏起来。''铁战跳了起来,瞪著屠娇娇道:``你为什么要将他藏起来,难道你也看上了他?''他像是又要冲过去将屠娇娇拎起来,屠娇娇赶紧陪笑道;.``这赌鬼最近已染上了白开心的毛病,你千万莫要听他的。''轩辕三光笑嘻嘻道;.``你就算没有将他藏起来,至少总知道他在那里的,对不对?''屠娇娇叹了囗气,道;.``你们若一定要找他,我就带你们去,只不过现在只怕已太迟了。''致战根本没有听到她后面两句在说什么,早已跳起来道:``要去现在就去,越快越好。''陈凤超忽也站了起来,道:``不错,这杯喜酒等等再喝也无妨。在下等已久闻.''小鱼儿``的大名,早就想见他一面了。''铁战拍掌大笑道:``如此说来,我这准女婿人缘倒还蛮不错的。''小仙女咬著牙,恨恨道:``他人缘的确不错,据我所知,至少有八百个人全恨不得将他整个人都吞下肚子里去。''幸好这时大家都在抢著往外面走,谁也没有注意她在说什么,只有顾人玉在一旁痴痴的望著她。等到人都走光,顾人玉才轻轻叹了囗气,道:``你也快些去吧。''小仙女道:``你不去?''

顾人玉垂下了头,道:``我\ldots\ldots 我看我还是回家的好。''小仙女瞪起眼望了他半晌,忽然冷笑道:``他破坏了你和九丫头的好事,你还在恨他?''顾人玉黯然一笑,道;.``就算没有他,九妹也不会嫁给我的,我并不是这意思。''小仙女道;.``那你是什么意思?''

顾人玉头垂得更低,讷讷道:``我只不过\ldots\ldots 只不过觉得你\ldots\ldots 你也\ldots\ldots{}''他不但满脸通红,连脖子都粗了。

小仙女瞪了他半晌,忽又笑了,道:``你这呆子,你难道以为我喜欢他!''顾人玉吃吃道:``我前两天听三姊说,女人只有喜欢一个人时,才会恨他,你这么恨他,岂非\ldots\ldots 岂非就是\ldots\ldots{}''小仙女忽然用一只柔软的小手掩住了他的嘴,柔声道:``你这呆子,你难道还不知道我的心?''顾人玉又□又喜,已呆住了。

小仙女道;.``你若以为我喜欢他,我现在就嫁给你,你总该放心了。''她忽然拍手笑道:``对,我们现在成亲,既用不著礼乐,也用不著媒人,等他们回来听到这件事,那时他们脸上的表情一定好看得很。''她越说越开心,突听.``噗通''一声,原来顾人玉竟已连人带椅一齐跌到地上去了。

小仙女吃惊道:``你\ldots\ldots 你怎么了呀?''她刚蹲下去想扶起他,谁知顾人玉忽又从地上跳了起来,大叫道:``我太开心了,太开心了\ldots\ldots 天下还有比我更开心的人吗?''小仙女又惊又笑,吃吃笑道:``想不到顾小妹也会变成个大疯子。''顾人玉大笑著道;.``我现在才知道小鱼儿是天下第一个大好人。''小仙女皱眉道;.``你居然说他是好人,只怕真是疯了。''顾人玉道;.``你想,若不是他,九妹和我们这两对好夫妻是从那里来的。''小仙女红著脸.``噗哧''一笑,却又故意板起脸道:``谁说我和你会是好夫妻,以后我说不定此母老虎还凶,天天打你,骂你,连饭都不给你吃。''顾人玉壮起胆子,拉起了她的手,柔声道;.``只要能和你在一起,不吃饭又有何妨,广东人常说.''有情饮水饱``,却不知我连水都可以不喝的。''小仙女娇声道:``我还以为你是很规矩哩,谁知你也这么不老实。''两人目光相对,心里却充满了柔情蜜意,微风吹入窗户,带来了满窗星光.一船春色,小仙女情不自禁,向顾人玉怀中依偎了过去\ldots\ldots 轩辕三光望著走在前面的一群人,心里暗暗得意,无论如何,他总算为小鱼儿做了一件事。

李大嘴回头瞧了他一眠,也将脚步放缓,走在他身旁,道:``原来你和小鱼儿是好朋友''轩辕三光道:``难道你以为老子只能交你们这些见不得人的龟儿子朋友吗?''李大嘴笑道:``想不到你也学会了用心机,竟连我们几个人都被你骗了。''轩辕三光瞪眼道:``你们这几个龟儿子其实根本就不能算人,小鱼儿是跟著你们长大的,你们却一心只希望他被困死。''李大嘴默然半晌,长长叹了囗气,道:``老实说,我本来也想救他的,可是\ldots\ldots 一听到燕南天已到了这里,我就吓得全没了主意。''轩辕三光道:``你以为小鱼儿会帮燕南天来对付你们。''李大嘴道:``他就算要这么做,也不能怪他的,江枫夫妻虽不是死在我们的手上,可是燕南天\ldots\ldots 唉!''轩辕三光冷笑道:``告诉你,你们全都将小鱼儿看错了,他绝不是反脸无情的人,他若活著一定会在燕南天面前帮你们说情的,他万一死了,你们这些龟儿子才真的倒了大楣。''李大嘴呆了半晌,叹著息道:``但愿他现在还活著才好。''轩辕三光揪住他衣服,变色道;.``他现在难道已死了不成。''李大嘴苦笑道:``我也不知道他现在是死是活,只知道他已在那山腹中被困了七八天,既没有食物,也没有水?\ldots\ldots{}''轩辕三光失色道:``七八天不喝水,就算铁打的人也捱不下去的。''李大嘴道;.``别人也许早就死了,但小鱼儿\ldots\ldots 他说不有定法子的,你永远也猜不到他究竟有多大的本事。''他生怕轩辕三光找他麻烦,赶紧又抢著道:``那位鬼童子的本事也实在不小,我真猜不透他怎会知道我们的行动,竟能及时将铁疯子找来。''他话刚说完,突听身后一人笑道:``若被你猜到了,我老人家还能算是鬼童子么?''笑声中人影一闪,鬼童子已到了他们面前。

李大嘴吃了一惊,陪笑道:``前辈果然是来无影,去无踪,在下实在佩服得五体投地。''鬼童子笑道:``你这两句马屁拍得我很舒服,我就将这件事从头到尾告诉你们吧。''他抢著道:``江湖中人都以为铁战得到了一张藏宝之图,其实他对藏宝一点兴趣也没有,他最大的兴趣,只是在无名岛上。''李大嘴道;.``既然是无名之岛,铁战又怎会知道的呢?''鬼童子道:``这只因有个多事的人,记下了无名岛的方位,而且说,无论谁只要找到这无名岛,就可向岛上的人学武功,回到中土来就可无敌于天下。''他笑著接道:``铁战平生就喜欢打架,见到这封秘件之后,自然大为心动,所以就叫他女儿带著另一份藏宝图将人引开,他自己却悄悄的寻到无名岛上来了。''李大嘴目光闪动,试探著问道:``无名岛上住的却是些什么人呢!''鬼童子道:``岛上住著的都是些早已厌倦红尘的老头子,他们''到了这岛上后,连自己以前的名字都不要了,所以这岛才叫做无名岛。``李大嘴陪笑道:''前辈想必也是岛上的无名英雄了。``鬼童子道:''什么无名英雄,只不过是些老不死罢了,何况,我就算想忘记自己的名字,别人只要一见到我,立刻就会认得出我,不像那些老头子,随便替自己取个名字别人也不知道。``其实李大嘴也早已猜到弥十八.俞子牙这些名字都是杜撰的,此刻虽已证实,却也不说破,只是叹了囗气,道;.''铁战的运气真不错\ldots\ldots{}``鬼童子道:''他在岛上住了三.四年,倒的确学会了不少武功,但若去的是你,此刻只怕早已被我们抛到海里去喂王八了。``李大嘴勉强笑道:''在下虽非好人,但铁战比在也好不了多少,前辈们为何偏偏看上了他呢?``鬼童子沉下脸,道;.''我问你,你打起架来,会不会像他那么样的不要命。``李大嘴道;.''这\ldots\ldots 这只怕要差一点。"

鬼童子道:``我们就看上了他这种不要命的脾气,才觉他孺子可教。''李大嘴只好不说话了,心里却在暗骂:``你们疯子遇见疯子,正是王八看绿豆,对了眼了,自然就一拍即合。''轩辕三光心里本在惦记著小鱼儿的安危,但听了几句后,也不禁动了好奇之心,忍不住道:.

``前辈们既已退隐世外,又怎会重入红尘的呢?''鬼童子道:``这只因铁战跟我们学了三年武功后,有天突然不学了,我们就问他为什么?他居然说我们这些人的武功,就算加起来也比不上燕南天和移花宫主,他学会了也没有用,所以还不如省些力气的。''李大嘴眼睛一亮,道:``如此说来,前辈们这次是想来找燕南天和移花宫主较量较量的。''鬼童子叹了口气道;.``这就叫人老心不老,静极又思动了。''李大嘴心里简直开心得要命,却故意叹息著道;.``依我看,前辈们不如还是快回去算了。''鬼童子瞪眼道;.``为什么?''

李大嘴道;.``别人我不知道,那燕南天的武功却当真是独步古今,空前绝后,前辈们只怕也\ldots\ldots{}''鬼童子果然跳了起来,怒道:``我就不信这个羊上树,倒非要找他此划比划不可。''李大嘴知道话已点到了,见好就收,改口道:``却不知前辈怎会知道铁心兰的婚事呢?''鬼童子又生了半天气,才说道:``我们到了中土后,沿江而行,那几个老不死忽然迷上了武升城里的一个小姑娘,硬说她琵琶弹得妙绝天下,竟赖在那里不肯走了,我生气也没有用,只有一个人四下走走,走到这里,别的人没有遇著,却救了那白老虎。''李大嘴笑道:``看来他的运气也不差。''

鬼童子道:``但那时他却已奄奄一息,我就将他送到山脚下养伤,他的伤还没有好,你们却已到了。''李大嘴苦笑道:``原来前辈也在那里,在下等为何未曾见到前辈呢?''鬼童子冷冷道:``方才我老人家就在你背后,你见到了么?''李大嘴叹了口气,道:``前辈在暗中听到在下等的计划,就立可设法通知铁战,叫他们立刻赶来,所以他们连妙绝天下的琵琶都不听了。''鬼童子笑道:``你这人还算不太蠢,终于弄明白了。''突听铁战大叫道:``你说小鱼儿就在这里?难道他也像孙悟空一样,被如来佛压在山下了么?''轩辕三光一听已到了地头再也顾不得别的,立刻赶了过去,只见铁战又拎起了屠娇娇,怒吼著道:``是你将他弄进去的你就得将他弄出来''屠娇娇苦笑道;.``我那里有那么大的本事。''铁战道:``不是你是谁?轩辕三光大叫道:''格老子,现在还问这些事干什么?小鱼儿已经在里面饿了七.八天了。``铁战失声道;.''七.八天,这姓花的小子只饿了两三天,已有气无力,他若已饿了七.八天,那还有命么?"

\hypertarget{ux7b2cux4e00ux767eux5341ux4e5dux7ae0-ux5e78ux8131ux6b7bux52ab}{%
\chapter{第一百十九章
幸脱死劫}\label{ux7b2cux4e00ux767eux5341ux4e5dux7ae0-ux5e78ux8131ux6b7bux52ab}}

.  ``恶赌鬼''轩猿三光,关心小鱼儿的生死,怕他说话耽误了开山的时间,忙向.``狂狮''铁战道.``幸好这儿人多,人多好做事,也许还来得及。''李大嘴也叫道;.``这儿有开山的家伙,想救小鱼儿的人,就快动手吧。''利斧铁锹本是他藏起的,他自然很快就找到了。

只见人人都在踊跃争先,取斧开山,就连那些养尊处优的少奶奶们竟也不肯落后于,人斧头铁锹没有了,她们就用自己价值不菲的匕首短剑,一时之间,震耳的凿石声已响遍了山巅屠娇娇叹了口气,苦笑道:``我还以为人人都想小鱼儿快些死哩,想不到大家来,却想他活下,小鱼儿呀小鱼儿,如此看来,你就算死也值得了。''白开心也叹了口气,道:``不错,若换了我被困在这山腹伫,只怕连野狗都不会来救我。''李大嘴矢笑道:``想不到你居然也有自知之明。''白开心冷笑道:``你得意个屁,就算这些人能不停的动手,至少也要一半天才能攻入山腹,到那时小鱼儿只怕早已变成咸鱼乾了。''花无缺和铁心兰已忍不住热泪盈眶,他们见到这种情况,心里虽然兴奋,但也知道希望实在渺茫得很。突见白夫人悄悄走过来,手伫提著个油淋淋的包袱,垂著头道:``包袱里有炸鸡和糯米丸子,是我方才偷偷包起来的,你们快吃了吧,吃饱了才有力气动手将小鱼儿救出来。''铁心兰喉头一阵哽咽,嘎声道:``你\ldots\ldots 也想救他?''白夫人揉了揉眼睛,勉强笑道:``我虽然并不清楚他究竟是个怎样的人,但我想\ldots\ldots 他若能活在世上,也许大家全都会快乐得多。''若非亲眼瞧见,武林中只怕再也不会有一个人相信这种事的江湖中最有名的几位世家公子,竟会和声名狼藉的.``十大恶人''们在一起卷起袖子来凿石头,平时连油瓶倒了都不会伸手去扶的慕容姊妹们,此刻竟会用她们吹弹得破的纤纤玉手去挖泥巴。而这一切,竟全是为了一个二十来岁的小伙子,这小伙子居然还是在.``恶人谷''长大的。

突听鼓声响起,如满天风雷大作,又如千军万马,动地而来,大家只觉精神更振奋碎石如雨点般飞起。他们果然创造了奇迹,竟在短短不到半天功夫里,就攻破了十道坚固的石闸,攻入了山腹。花无缺和轩辕三光当先冲了进去,他们的心情虽兴奋,却又不禁在暗中担心,害怕\ldots\ldots 他们只怕发现的是小鱼儿的死尸!花无缺本想呼唤两声,但一颗心似已将跳出腔子,连声音都发不出来。只见那已被劈成两半的石椅上,放著个酒瓶,地上还散落著些破布,线头,花无缺认得那正是从小鱼儿和移花宫主她们穿的衣服上拆下来的。他的脸色立刻变了,手抖得连一块市都捡不起来。

轩辕三光忍不住问道:``这\ldots\ldots 这是他们的衣服?''花无缺茫然点著头道:``嗯。''

轩辕三光一颗心也不禁沉了下去,像小鱼儿他们那样的人,若不是遇著非常的变故,怎会连身上的衣服都会被扯破!他们简直不敢再进一步去找!他们已提不起勇气去面对那残酷的现实。

慕容珊珊忽然道:``这瓶子里是不是酒?''

轩辕三光提起瓶子来嗅了嗅,道:``是。''

慕容珊珊眼睛一亮,喜道:``瓶子里是酒,就有希望了。''轩辕三光道:``为\ldots\ldots 为什么?''

慕容珊珊道:``酒也可以充饥的,他们若有酒喝,就可以多支持几天。''轩辕三光跳起来至少有两丈高,狂喜著大呼道:``小鱼儿,小鱼儿,你在那里,你的好朋友们已全都来救你了!''他狂喜著冲了进去。

空旷的洞穴中,响彻了轩辕三光的回声,但却听不到有人的回应,小鱼儿呢?难道已饿得说不出话来了?地道的入囗并没有封闭,他们看到了魏无牙的尸体,看到了无数只空酒瓶,也看到了那臭不可言,也妙不可言的.``厕所''。

但他们找遍了所有的地方,都找不到一个活人。小鱼儿他们呢?难道他们已化骨扬飞,永远自这世界消失了不成!大家面面相觑,只有站在那伫发呆。过了很久,轩辕三光才笑著道:``格老子,我就知道世上绝没有任何地方关得住小鱼儿,我们还在为他担心,他却早已走了。''李大嘴道:``他没有走。''

轩辕三光怒道:``你这龟儿子就希望他被困死,是么?''李大嘴叹了口气,道:``我也希望他是已逃出了,可是我方才已将这地方全都很仔细的查看了一遍,四面根本就没有出路。''轩辕三光道:``老子也晓得这里没有出路,但一定有法子出去的。''李大嘴道:``他能有什么法子?就算他能破壁而出,多少也会有些痕迹留下来的,除非他会孙悟空的七十二变,变成个苍蝇从那气孔中飞出去。''其实轩辕三光也知道他说的不错,四面山壁都是完整的,根本就没有被打通的痕迹,小鱼儿他也的确没法子出去。但他若没有出去,就应该在这洞穴里。

轩辕三光道:``你龟儿说他们没有出去,那么他们在那里呢?我们为什么连他们一根汗毛都找不到。''李大嘴沉吟著,还没有说话,白开心忽然大声道:``化骨丹!''这三个字说出来,轩辕三光和花无缺背脊上都不禁冒出一股寒气,铁心兰更快急疯了。

李大嘴磴著白开心道:``你的意思是说,魏无牙害死了他们后,又用化骨丹消灭了他们的尸体,''白开心咧嘴一笑,道:``我并没有这么说,这话是你说的。''小鱼儿他们既不可能出去,又没有在这里,自然是因为他们的尸体已被消灭了,这是唯一合理的解释。

就连铁战也不禁摇头叹息,喃喃道:``我本来还想看看他究竟是个什么样的人,为何能令我女儿如此喜欢他?谁知道这小子竟连骨头都没有剩下一根。''他拍著铁心兰的头,道:``这小子既然没有福气娶你,你也不必伤心了,若是觉得一个老公不够,过两天再为你找一个就是。''他不说这些话还好,一说出来,铁心兰连心都碎了,连哭声都没有发出来,就晕了过去。

鬼童子忽然道:``他们可是被魏无牙关在这里的?''李大嘴叹道:``只怕是的。''

鬼童子道;.``那么,魏无牙自己怎会也死在这里了呢?''屠娇娇道:``这也许是因为魏无牙要眼看著他们死,否则就不过瘾。''鬼童子道;.``不错,这很有道理,可是魏无牙既能将他们全都害死,又消灭了他们的尸体,那么魏无牙就不会死了,难道他们的鬼魂远能为自己复仇,将魏无牙杀了不成?''屠娇娇道:``魏无牙是自己服毒的,前辈难道还看不出来么?''鬼童子道;.``他既然将别人全都杀了,自己为何要服毒!''屠娇娇怔了怔,道:``这\ldots\ldots{}''鬼童子笑了笑,缓缓道:``魏无牙算准别人都不敢杀他,所以才敢留在这里看热闹。''李大嘴道:``不错,小鱼儿他们若想出去,就不能杀他,因为他是唯一知道这里秘密的人,但他难道就不怕别人逼他说出秘密么?''鬼童子道:``他自己以为自己藏身之处很隐秘,以为别人必定找不到他,谁知小鱼儿他们的本事比他想像中大得多,还是将他找出来了,他被逼问得受不了时,就只有自己服毒而死,因为他知道只要他一死,别人就都要被困死在这里的,所以他就等于为自己报了仇。''他的猜测居然已和事实相差不远,只因轩辕三光,花无缺,李大嘴他们,多多少少都有些为小鱼儿担心,头脑已无法保持冷静,但鬼童子他们却根本不认得小鱼儿,旁观者清,自然看得清楚些。

轩辕三光不禁喜动颜色,道:``如此说来,魏无牙一定是比小鱼儿他们先死的了。''鬼童子又笑了笑,道:``魏无牙就算有天大的本事,也无法将移花宫主姊妹和小鱼儿三个人一齐杀死的,你说是不是?''轩猿三光拍掌大笑道:``莫说一个魏无牙,就算一百个魏无牙也不行。''白开心道:``常言道饮鸩止渴,一个人若是渴极了的时侯,就算明知酒中有毒,也会喝下去的,你说是不是?''屠娇娇道:``不是。''

白开心瞪眼道:``你知道个屁。''

屠娇娇也不理他,缓援接著道:``酒中绝对没有毒,每个酒瓶我都嗅过了。''轩辕三光展颜大笑道;.``我和你认识了几十年,你总算说了句人话,做了件好事。''白开心悠然道;.``他既不可能逃出去,也不可能死在这襄,那么我问你们,他是到那里去了?''这句话问出来,大家又全都呆住。这件事实在不可思议,无论谁也猜测不出。

天下又有谁知道小鱼儿现在在那里呢?有谁知道他现在是生?是死?是已尸骨无存?还是在好好的活著?每个人心里都有许多疑团,都想问个清楚,但谁也不知道自己该去问谁?只好站在那里发楞。俞子牙,弥十八,萧女史,这些人虽然久已不为世事所动,但这时也都禁不禁在苦苦思索著。因为这件事实在太神秘,他们也动了好奇之心。

轩辕三光最焦急,铁心兰最悲痛,白开心不停的冷笑,哈哈儿却笑不出来,只有杜杀,仍是脸色铁青,也不知心里在想些什么?突听花无缺大声道:``各位的鞋底都是湿的,是不是!''每个人俱都心事重重,又有谁会留意到自己的鞋底?鞋底无论是乾是湿,本都一点关系也没有,但花无缺语声中却充满了兴奋之意,就像是刚发现了一件最重要的事。大家谁也不知道他为何会对这种无足轻重的小事如此关心,可是大家还是不由自主提起脚来瞧了瞧。至少有一半人的鞋底果然是湿的。

轩辕三光的一双草鞋更已完全湿透,忍不住问道:``格老子,鞋底湿了难道也是什么大不了的事么?''白开心笑嘻嘻道:``想不到居然有人将一双鞋子看得比老朋友的生死还重要,妙极妙极。''花无缺根本不理他,仍是满面兴奋之色,道:``此地既然没有水,鞋子怎会被打湿的?魏无牙若想将他们饿死,渴死,此地又怎会有水?''这句话说出来,大家才发现这果然又是件很神秘的事。

轩辕三光道:``但这件事却和小鱼儿的去向有什么关系?''花无缺道:``果然有关系,若是我猜得不错,我已可找出小鱼儿在那里?''轩辕三光大喜道:``快说,他在那里?''

花无缺来不及回答这句话,已又向地道下奔了过去。在这阴湿的洞穴中,那.``厕所''的气味实在令人不敢领教,魏无牙的尸身更令人见了要作呕。若是换了平时,慕容姊妹是再也不肯下去的了,但此时花无缺一走,大家就全都抢著跟了下去。只要能知道小鱼儿的下落,能知道这秘密的真象,这地道下就算真是个大粪坑,他们也忍不住要跟下去的。

地道下果然有水,而且越积越深,此刻几乎已没及他们的足踝,显然有个地方一直在不停的往外面流水。水势虽不大,却也不太小。

轩辕三光道:``格老子真他妈的奇怪,小洞里居然在流水,难道山腹中还有条小河不成?''谁也想不通这水是那里流出来的,只见花无缺俯著身子,很仔细的观察著水势,惭惭又走入了魏无牙那间秘室。这秘室中更是臭不可闻,大家方才见到里面并没有活人,就很快的退了出来,谁也不愿停留在里面。

但此刻,大家已发现秘密的症结就在这秘室里,也就顾不得臭不臭了,全都一拥而入。只听花无缺失声唤道;.``果然不错,就在这时?''他站在那两只已被小鱼儿当厕所的石棺前,满面俱是喜色,但四下仍看不到一个活人。

白开心失笑道;.``你说小鱼儿在这裹里?难道他已撒泡尿自己淹死了么?''他话末说完,突听杜杀怒道;.``那里来的这许多废话,滚出去。''喝声中,白开心已被他打得飞了出去,自众人头上飞过,.``砰''的,跌在地道外,不停的呻吟起来。

但大家并没有去留意这件事,因为此刻大家已发觉水就是自石棺旁一个地洞里往外面冒出来的。地上本来铺著石板,但此刻石板已被撬开,因为这时本来就乱七八糟的堆著些碎石,所以方才才会没有人留意。

轩辕三光满面惊讶之色,道;.``难道说,小鱼儿他们是自这地洞里逃出去的?''花无缺展颜道;.``正是,我们只去注意四面的山壁,所以才认为他们绝不可能已逃出去,却末想到他们是自地下出去的。''轩辕三光拍掌道;.``不错,四面的山壁虽然坚不可摧,但地下却全都是泥土,自然比石头要软得多了。''他瞬又皱起眉头,道:``可是若想从这里挖一条地道通到外回去,那也不容易。''花无缺道:``那自然不容易,只不过这地道并不是他们自己挖的。''轩辕三光道:``不是他们自己挖的,是谁挖的?''花无缺道:``据我所知,大部份的河流虽然都在地面上,但地下有也有一些河流,只因沧海桑田,地势变换,所以这些河流才会被埋藏在地下,只要能找到这种地下河流,凭他们的武功,就不难钻出去。''大家全都不禁听得喜动颜色。轩辕三光跳了起来,大笑道:``格老子,你知道的事真他妈的不少。''花无缺笑了笑道:``我现在也可以想出他们的衣裳是怎会破碎的了。''轩辕三光拍著他肩头:``快说快说,那又是怎么回事?''花无缺道:``小鱼儿并不知道这地下会有被埋藏了的河流,更不会知道它的位置是在那里,因为人虽然是万物之灵,却缺少动物那神秘的本能,譬如说,一条狗可以靠它的嗅觉追踪至千里之外,人是就绝对做不到。人也许并不是没有这种本能,只不过已渐渐退化了,因人并不需要倚靠这种本能来求生存。''轩辕三光大声道:``有道理,有道理!''他现在似乎对花无缺口服心服,无论花无缺说什么也都觉得有道理,其实这道理地却未必真的懂得。

花无缺道:``动物的本能,也并不是完全相同的,譬如说,狗的鼻子特别灵,蝙蝠对声音的反应特别敏锐,候鸟对天气的变化知道得最早,一些自身没有抵抗能力的野兽,对危险往往有种神秘的感觉。''这道理在现在也许已有很多人知道,但在那时却简直比什么.``内功心法''都要深奥玄妙些。

大家都不觉听出了神。

花无缺忽又一笑;.``各位可知道世上最会钻洞的是什么?''慕容珊珊也笑了笑,道:``老鼠。''花无缺道;.``一点也不错,正是老鼠,你无论将老鼠关在什么地方,它都有本事钻洞逃出来的。''轩辕三光失声道;.``魏无牙那龟儿就是个大老鼠,这地方老鼠必定不少。''花无缺道:``小鱼儿必定是找到了几只活老鼠,他想要老鼠替他带路,又怕老鼠跑了,所以就将衣服撕破,搓成绳子绑在老鼠尾巴上,才将老鼠放出去。所以!这地下的河流一定是老鼠找到的,小鱼儿那时也许还不知道老鼠为同要往地下钻?但那时他们已山穷水尽,只有姑且一试了。''轩辕三光大笑道:``我知道小鱼儿是天下第一聪明人,谁知你也并不比他差,看来你们两人倒实该结拜成兄弟才是。''花无缺面上又不禁露出痛苦之色,因为轩辕三光这番话无意中又触及了他的隐痛。现在,小鱼儿既已逃出去了,而且还在移花宫主的掌握中,那么,他还是难免要和小鱼儿一决生死。他们悲惨的命运,彷佛永远也无法改变的。

轩辕三光再也不说什么,也想往那地洞钻下去。

李大嘴道:``你干什么?''

轩猿三光瞪眼道:``干什么?自然是去找小鱼儿!''李大嘴笑道:``他们是无路可走,才钻地洞的,你现在却用不著也跟著钻地洞呀!''轩辕三光道:``老子若不钻地洞,怎知他到什么地方去了?''李大嘴还末说话,突听一人在上面呼道:``三姊,三姊,你们在那里呀?''慕容珊珊皱了皱眉,带著笑道:``是张菁,这小鬼怎地到现在才来。''他也呼唤著,呼声中,小仙女已冲了进来,一张脸红红的,满是兴奋之色,冲过来拉起慕容珊珊的手,喘息笑道:``我见到了一个人\ldots\ldots 我见到了一个人\ldots\ldots{}''慕容珊珊失笑道:``见到一个人也用不著如此大惊小怪呀,我每天都见到几十几百个哩。''.``但这人\ldots\ldots 这人\ldots\ldots{}''她忽然神秘的一笑,转著眼珠子道.``这人是谁,你水远都猜不到的。''慕容珊珊忍不住问道;.``是谁?''她刚问过了,心里忽又一动,也紧张起来,道:``你难道见到了小鱼儿!''这句话问出来,大家全都紧张了,都眼睁睁的望著小仙女。

小仙女笑了笑,道:``不错,就是小鱼儿,你们全都到这里来找他,谁知他却已到了我们的船上去了。''轩辕三光又跳了起来,失声道:``真的。''

小仙女白了他一眼,道;.``酒席一直都没有撤下去,因为要等你们回来吃,谁知到了中午,你们还没有回来,水底下却忽然跳出来几个人,一跳上船,连话也不问,就大吃大喝起来,其中有个人连筷子都来不及用,就是小鱼儿。''轩辕三光大笑道;.``格老子,他只怕已经快饿疯了。''花无缺忍不住道:``除了他之外,还有什么人!''小仙女笑了笑道;.``自然还有移花宫主,我实在想不到她们看来竟那么年轻?她们衣服的料子也很奇怪,从水裹跳出来,居然还没有湿透,小鱼儿已狼狈不堪,但她们两人看来都还是那么高贵,就像是仙女似的。''慕容珊珊笑道:``如此说来,你这外号应该送给她们才是了。''小仙女眨了眨眼睛,又道:``跟她们一齐来的,还有个女孩子,头大大的,一点也不漂亮,却和小鱼儿亲热得很。''这番话说出来,大家不禁又都觉得很奇怪,眼睛不禁都向铁心兰瞟了过去。铁心兰咬著嘴唇,根本不敢抬头。

铁战却大怒道:``这小子竟敢跟别的女人亲热,我女儿难道还比不上那大脑袋的丑八怪?''小仙女笑道:``我本来也在暗暗好笑,小鱼儿选来选去,怎么选上了这么样一个人,但后来我越看越觉得那女孩实在灵极了,一颦一笑,每一个动作,都找不出一点毛病来,就连我见了都要心动。''铁战更是气得暴跳如雷,大叫大喊。慕容珊珊望著小仙女,却觉得有些奇怪。只有女人才能了解女人的心事,小仙女对小鱼儿那种情感,慕容珊珊再了解也不过了。

她以为小仙女看到小鱼儿和别的女人亲热,一定会很不舒服,一定会骂那女人是个丑八怪。

谁知小仙女却将那女人恭维得天上少有,地下无双,慕容珊珊望著她,奇怪她怎的忽然变了的。

却不如小仙女的情感已有了归宿,正是最甜蜜.最幸福的时候,所以对人类也充满了热爱,觉得每个人都不讨厌了。

慕容大姊眼波流动,望著她夫婿柔声道:``船上既然又有贵客来了,我们还是赶快回去吧?''她每件事都先徵求她夫婿的意见,因为她知道他绝不会反对的。

铁战也跳起来,道:``对,我们现在就走,我们要看看那小子有多大的胆子。''萧女史淡淡道:``据说移花宫主驻颜有术,我们也想见识见识。''弥十八道:``我就不信她们的功夫真的已天下无敌。''轩辕三光含笑道:``多日不见,不晓得小鱼儿是否变老成了些。''有的人想去见移花宫主,有的人想去看小鱼儿,也有的人是想去看看那.``大头的美人''究竟是怎么迷上小鱼儿的。大家的理由虽不同,但却都急著想回船去。

只有花无缺,他想见移花宫主和小鱼儿的心虽然此谁都急切,但想到他见到小鱼儿后只怕又难免要拚命,他又希望永远都莫要见到小鱼儿了。

突听小仙女道:``我话还没说完哩,你们莫要急著走呀。''慕容珊珊笑道;.``你少卖关子好不好,快说吧。''小仙女目光闪动,道:``除了移花宫主外,我们船上还有位贵客,这位贵客的名头绝不在移花宫主之下,你们可知道他是谁么?''她话末说完,大家已全都猜出是谁了,因为普天之下,只有一个人的声名能和移花宫主并驾齐驱。大家都不由自主地夫声叫了出来:``燕南天!大侠燕南天!''

\hypertarget{ux7b2cux4e00ux767eux4e8cux5341ux7ae0-ux795eux529fux7eddux5b66}{%
\chapter{第一百二十章
神功绝学}\label{ux7b2cux4e00ux767eux4e8cux5341ux7ae0-ux795eux529fux7eddux5b66}}

听到``燕南天''这名字,屠娇娇.李大嘴等人只恨不得背上生出对翅膀来,快快飞到十万八千里之外。慕容姊妹也不禁俱都为之动容。

弥十八和俞子牙对望一眼,弥十八道:``想不到移花宫主和燕南天都在那里。''俞子牙道:``这真是踏破铁鞋无觅处,得来全不费功夫。''鬼童子道:``却不知移花宫主和燕南天见面时是什么光景,我想那一定有趣得很。''大家想到这当代两大绝顶高手见面时的情况,也不禁心动神驰,只恨自己不能躬临其战而已。

萧女史忍不住问道.``移花宫主她们可认得燕南天么?''小仙女道:``她们好像并不认得,但燕大侠一走上船,大家就似乎都已知道他是什么人了,因为他那种气派,别人学也学不像的。''鬼童子冷冷道:``别人也未必就要学他。''

小仙女笑了笑,道:``奇怪的是,小鱼儿好像也没有见过燕南天,但燕南天一上了船,就瞬也不瞬的盯著他瞧。''轩辕三光道:``小鱼儿呢?''小仙女道:``小鱼儿也盯著他,不知不觉的站了起来,他一步步走过去,嘴里一直不停的说很好,很好,很好\ldots\ldots{}''慕容珊珊.``噗哧''一笑,道:``很好这两个字,你说一遍就够了。''小仙女道:``但燕大侠却一连说了十几遍,眼睛里热泪盈眶,只差没有掉下来,小鱼儿也没有说什么话,只是扑地跪了下去,燕南天就拉起他的手说,你做的事我差不多都已知道了,你并没有丢你父亲的人。''说到这里,她眼睛里也湿湿的,显然当时深受感动。大家以她为中心,随著她往外面走,不知不觉全都听得出了神,甚至不知道已走出了那山洞。

只听小仙女接著道:``移花宫主一直在旁边冷冷的望著他们,过了很久之后,那位大宫主才冷冷道,很好,我们总算见面了。''小仙女道;.``燕大侠又过了很久,才转身望著她,说,二十年前我们就已该见面的,那位大宫主就冷笑著说,你嫌太迟了么?燕大侠就仰天长长叹了囗气。''说到这里,她自己也长长叹了口气。

慕容珊珊忍不住问道:``燕大侠说了什么?''

小仙女叹道:``他似乎要将二十年的辛酸抑郁,全在这口气里叹出来,然后才说,燕某既然还末死,也就不算迟。''轩辕三光等七.八个人忍不住一齐脱口问道:``后来呢?''小仙女道:``这时他们已剑拔弩张,像是随时随刻都要出手,只不过他们的身份不同,不能说打就打而已,我心里正在著急,不知这两位绝顶高手打起来是什么光景,人玉却将我拉到一边要我赶快来通知你们,叫你们赶快回去。''说起顾人玉,她目中就不觉露出了温柔的笑意,接著道:``他说,你们若错过这一场空前绝后的大战,一定会遗憾终生的。''鬼童子叫了起来,道:``何止遗憾终生而已,我以后只怕再也休想睡得著觉了。''轩辕三光道:``只望他们莫要真的打起来才好。''小仙女道:``为什么?''

轩辕三光叹道:``两虎相争,必有一伤,而且说不定两败俱伤,这一战的后果实是不堪想像,我们宁愿见不到这一场大战才好。''花无缺感激的望了他一眼,他知道这一战只要一交上手,就是不死不休的了,那么,无论两人谁胜谁负,他和小鱼儿的冤仇势必要结得更深,只怕也是不死不休,永远也解不开的了。

过了半晌,只听俞子牙也叹息著道:``他两人若是真的两败俱伤,那倒可惜得很。''萧女史笑道;.``你希望他们都等著来和你交手,是么?''俞子牙淡淡道;.``你难道不想试试你那.''娲皇十八变``的新招么?''萧女史轻轻叹了口气,道:``只可惜听他们说话的口气,冤仇似乎结得很深,燕南天既已等了二十年,此番见了面,焉肯甘休。''俞子牙也叹了口气,道:``这两人若是动上了手,世上只怕再也没有人能将他们分开了。''他们回到江岸时,长棚中的桌椅都已撤去,只剩下那些彩纸和喜联,在江风中簌簌的发著抖,想及昨夜的盛况,更显得此时的凄凉,人生本无不散的筵席,早知此时的凄凉,又何必著急于一时的盛衰呢?长棚旁的空地上,此刻却挤著一大堆人,叠叠重重围个圈子,也不知在看什么热闹。燕南天和移花宫主莫非就在圈子里决斗?轩辕三光当先冲了过去,想分开人丛挤进去,但这些人看到他们回来了,早已哄的四下散开。移花宫主并不在里面,更瞧不见燕南天和小鱼儿的影子。

他们的人呢?难道这只不过是小仙女在开玩笑?但小仙女已先叫了起来,这:``咦,他们的人呢?小蛮,他们到那里去了?顾公子呢?''小蛮本是慕容珊珊的贴身丫头,小仙女到了之后,就服侍小仙女了,她明眸善睐,看来必定能说会道。可是小仙女问得实在太快,也太多了。

小蛮先松了口气,方转著眼珠子说道:``姑娘一走了之后,那位燕\ldots\ldots 燕大侠就坐过去和那位小鱼儿少爷喝酒,两人你一杯,我一杯的喝个不停,也说个不停,我只瞧见他们说著说著,忽然大笑了起来,说著说著,又忽然不停的叹著,那位姓苏的姑娘,带著笑替他们斟酒,但只要一扭过头,就不停的悄悄擦眼泪。''小仙女自然也知道他们是正在叙说著这些年来种种悲欢离合,可歌可泣的遭遇,但还是忍不住问道:``他们在说些什么?''小蛮道:``他们说的声音并不大,有些话我根本听不见,有些话我虽然听见了却听不懂。''小仙女笑骂道;.``你呀,瞧你这点出息,加起来还不够半两。''小蛮垂著头道:``我虽然听不见他们说什么,但瞧见他们的模样,也不知为了什么,心里就酸酸的,想掉眼泪。''轩辕三光想到小鱼儿和燕南天的追遇,心里也不禁一阵酸楚,大声道;.``不错,格老子,我虽也没有听到他们在说什么,我也想掉眼泪。''小仙女瞪了他一眼,又向小蛮问道:``他们说话的时候,移花宫主呢?''小蛮道:``移花宫坐在另一张桌子上,既不看他们,也并不著急,她们好像早已知道燕大侠一说完了话,就会来找她们的。''众人对望一眼,心里都不禁暗自唏嘘,因为他们也都已看出,燕甫天这是已决心要和移花宫主决一死战,是以才先将后事向小鱼儿交代。

小蛮道;.``他们好像有说不完的话,尤其那位小鱼儿少爷,更说个不停,我从来也没有见过话说得这么多的男人,简直像是个老太婆了。''轩辕三光叹道:``小娃儿,你不忙的,他这是因为早已看出了燕南天的心意,所以故意多说些话,来拖延时间\ldots\ldots{}''小蛮道:``如此说来,燕大侠必定也看出他的心意了。''轩猿三光道:``哦!''

小蛮道:``因为燕大侠忽然站了起来,拍著小鱼儿的肩头,大笑著说:''你燕大叔素来百戟百胜,你用不著担心的"。

俞子牙冷笑道:``百戟百胜,好大的口气。''

轩辕三光也冷笑道;.``别人说这话,老子一定当他是吹牛,但燕南天这话,却没有人能不服的。''俞子牙并没有再说下去,只.``哼''了一声。

小蛮道:``小鱼儿少爷望著燕大侠,彷佛要说什么,但这时移花宫主已站起来走了出去,燕大侠立刻跟著往外走,他们虽然连一句话也没有说,但也不知怎地,我的心已紧张得几乎要跳出腔子。''她本就口齿伶俐,语声清脆,此刻更知道有很多人都在听她说话,所以说得更为卖力。大家听她说得如此传神,也不禁全都紧张起来,就好像都已亲眼见到那两大绝世的高手,正肃立在江岸,准备做生死的决斗!江风萧萧,大地间也彷佛充满了肃杀之意。

小蛮机伶伶打了个寒噤,缩起脖子,接著道:``但他们走出来之后,也还是没有立刻动手,两个人只是远远的对面站著,你望著我,我望著你。''俞子牙道:``燕南天没有用兵器?''

小蛮道;.``没有,他们两个人都没有。''

俞子牙皱起了眉,喃喃道:``久闻燕南天剑法无双,为何舍长而用短?竟不用剑来交手呢?难道这些年来他已练成了自信足可和移花宫掌法一较上下的拳法不成?''要知移花宫掌法内力,独步天下,所以他不说燕南天也练成一种.``掌法'',而说.``拳法''。

因为他认为世上绝不可能再有一种能和移花宫掌法一较雌雄的掌法了他本身自然也并非以掌法见长的。

只听小蛮道:``他们虽然赤手空拳,但看来却比用什么兵器都凶险,好像只要有一招攻出,立刻就可以分出生死似的。''萧女史望了俞子牙一眼,含笑道:``这小姑娘倒蛮识货的。''小蛮咬著嘴唇向她一笑,才接道:``我看得实在太紧张了,就想求顾公子过去劝他们不要打了,但顾公子却说,他们两人此时虽还没有出手,但精神气力全都已贯注,别人莫说休想能劝得开他们,只要一走过去,恐怕就要被他们的真气震倒。''萧女史有意无意间瞟了小仙女一眼,笑道:``这位顾公子倒也是个识货的。''小蛮道;.``顾公子正在悄悄和我说话,那位小鱼儿少爷不知怎地也听到了,忽然走过来对顾公子说:''你认为真的没有人能劝得开他们了么?``小仙女皱眉道:''这小鬼又想玩什么花样?"

小蛮道;.``顾公子见到他似乎连头都大了,只是不停的点头,那位小鱼儿少爷就又说:''你敢跟我打赌么?``小仙女著急道;.''他是个鬼精灵,顾公子却是老实人,怎么能跟他打赌呢?``小蛮道:''顾公子本来是不愿和他打赌的,但小鱼儿少爷却说\ldots\ldots 说\ldots\ldots{}``小仙女道;.''说什么?``小蛮垂下头,道:''他说:``我早就知道顾小妹不敢跟我打赌的,算了吧!''轩猿三光大笑道:``妙极妙极,想不到小鱼儿连赌鬼诱人上钩的法子都学会了,他这么样一激将,那位顾小妹不赌也要赌了。''小仙女又狠狠瞪了他一眼,小蛮已叹道;.``不错,顾公子果然忍不住和他打赌了。''小仙女连脸都急红了,跺脚道:``他怎么这么沉不住气,他们赌的是什么?''小蛮道:``那位小鱼儿说;.''我只要说一句话,就能令移花宫主住手,燕大叔一个人自然也就打不起来了。``顾公子自然不信。''萧女史道:``莫说顾公子不信,连我都不信,这赌我也要打的。''小蛮叹了口气,道;.``那么你老人家也就输了。''别人只急著想听小鱼儿究竟说的是什么话,能令移花宫主住手,小仙女却只急著想知道顾人玉究竟输了什么东道。小蛮既能做大家小姐的贴身丫□,自然从小就已学会了如何揣摩主人的心意,如何拍主人的马屁。

所以她不说别的,先说道;.``那位小鱼儿少爷说,若是他输了,就随便顾公子要他怎样,若是顾公子输了,他就要顾公子去为他做一件事。''小仙女道:``做\ldots\ldots 做什么事?''

小蛮陪笑道;.``当时他并没有说,后来他说的时候,我却没有听见。''小仙女跺脚道;.``说你没出息,果然没出息,什么你都不知道。''萧女史笑道:``其实她知道的已经不少了。''

轩辕三光道;.``不错,快说那位小鱼儿少爷究竟说了什么样的一句话.''那移花宫主听了他的话,是不是真的立刻住了手?``小蛮道:''小鱼儿只向另一位移花宫主大声说;可惜呀可惜,我和花无缺打起来的时候,你姊姊恐怕已末必能看到了。``萧女史道;.''他说了这句话,移花宫主难道真住手了么?``小蛮道:''立刻就住手了,我也觉得很奇怪,不知是怎么回事?``萧女史讶然道:''她为何一定要看小鱼儿和花无缺的一战呢?难道这一战比她和燕南天的一战还要精彩不成?``俞子牙却皱著眉道:''那燕南天究竟练成了什么惊人的功夫?能令移花宫主住手?``小蛮道:''不是燕大侠令她住手的,是那位小鱼儿少爷。``慕容珊珊道:''傻丫头,少说话。"

萧女史却含笑道:``移花宫主若有必胜的把握,打过了之后,还是能看到小鱼儿和花无缺一战的,她就不会住手了,是么?''小蛮想了想,垂首笑道;.``不错,我真是个傻丫头。''要知花移宫主忽然住手,自然是因为她和燕南天对峙时,已发现燕南天的功力深不可测,她实无制胜的把握。

轩辕三光心里却只惦记著小鱼儿,别的事他根本全都不放在心上,当下大声问道:``现在小鱼儿少爷到那里去了!''小蛮道;.``燕大侠和移花宫主约定,每天清晨日出的时候,都山巅相见,直到移花宫主找到那位花\ldots\ldots 花少爷为止,然后燕大侠就带著小鱼儿少爷走了。''轩辕三光道:``移花宫主呢?''

小蛮道:``她们自然是去找那位花少爷去了,说不定马上就会回来,因为顾少爷已告诉了她们,说花少爷是和大家一齐去的。''小仙女心里却只惦记著顾人玉,抢著道:``那么顾少爷又到那里去了?''小蛮道:``顾少爷输了,已经为小鱼儿少爷去办事了。''小仙女跺脚道;.``那捣蛋鬼还会要他去做什么好事么?他为什么要去呢?''她简直急得眼泪都快要掉了下来。

慕容珊珊望著她,忽然一笑,轻轻道:``,大妹子,恭喜你。''小仙女嘟著嘴道:``人家都快急疯了,你这来恭喜什么?''慕容珊珊笑道:``顾小妹又不是你的什么人,你为何要为他如此著急呀?''小仙女嘴嘟得更高,道:``他又不是没有名字,你们为什么总是要叫他顾小妹?''慕容珊珊吃吃笑道:``顾小妹这名字本是你替他取的,现在你却不许人家这样叫他了,这又是为了什么呀?才一天不见,你们的关系已不同了么?''小仙女低下头,脸已红了,道:``我们\ldots\ldots 我们\ldots\ldots{}''慕容珊珊轻轻拧了拧她的脸,笑骂道:``鬼丫头你还想瞒我们,这顿喜酒你想跑得了么?''慕容双忽然道;.``人家既然已经不打了,你们方才还围在这里看什么?地上难道忽然长出一朵花来了不成?''小蛮笑道;.``地上若是长花就不奇怪了,忽然长出了馒头那才奇怪。''慕容双也不禁怔了怔,道:``馒头?''

只见那片平地上,果然有个小山的土丘凸起,看起来就像是个土馒头似的。

慕容珊珊笑道;.``傻丫头,这又有什么好看的。''小蛮道:``姑奶奶你不知道,这不但奇怪,而且奇怪透了。''她忽然跑过去站在那土丘上,道:``方才移花宫主就是站在这里的,她站上来的时候,这里本来是块平地,可是她站在上面没多久,脚下的地就渐渐凸了起来,这块地面就像是揉著发面,她往上面一站,就蒸出个馒头来了。''大家虽觉她说得好笑,但又不禁觉得很惊讶。俞子牙、□十八等更是耸然动容,忽然一齐掠过去,俯下身去看那土馒头,而且看了又看,就真的像这士丘上忽然长出了花来。

小蛮向慕容珊珊笑了笑,彷佛在说:``你说我是傻丫头,人家这些老头子.老太婆们不是看得很有趣吗?''只见俞子牙他们的脸色越来越□讶,纷纷道:``果然不错\ldots\ldots 但这怎么可能呢?\ldots\ldots 想不到果然有人练成了。''大家也都不禁一齐困了上去,这才发现士丘上还有两只脚印,但脚印却并非凹下去的,反而凸出来一寸多。高手相争时,全身功力凝注,往往会将脚下的泥土踩出脚印来,这倒并非什么奇怪的事。脚印并非下陷反而凸起,就是少见的怪事了。

慕容珊珊目光闪动,道:``移花宫主莫非练成了一种极奇怪的功夫不成?''俞子牙叹道:``不错,她练成的这种功夫虽非空前绝后,至少也可傲视当代了。各位可瞧见这上面的两只脚印了么?''他也知道任何人都不会瞧不见的,所以就自己接著道:``这只因她功力运行时,非但不向外挥发,反而向内收□,无论什么东西触及了她,都会如磁石吸铁般被她吸过去。''慕容珊珊动容道:``如此说来,她的功力永远不会消耗,只有增加,岂非要越用越多?''俞子牙道:``正是如此,她与人交手时,功力越用越多,而对方却势必要渐渐减少,所以就算一个武功和她相若的人和她动手,到后来还是必败无疑。''萧女史抢著道;.``有一种''明玉功``练到第九层时,才会有这种现象,只因她体内的真气,已能形成一种漩涡,无论什么东西触及她,都会被这真气漩涡卷过去,正如泅水的人遇见了水中的漩涡一样。''慕容珊珊道:``如此说来,只要练成这种功夫,岂非一定天下无敌。''萧女史.弥十八、俞子牙等人对望一眼,面上都露出了黯然之色。俞子牙长叹道:``不错,她实已天下无敌,我们都是白来的了。''慕容珊珊道;.``她既已无敌于天下,燕南天自然也不会是她的对手,那么她对燕南天有什么顾忌呢?难道燕南天也练成了这种功夫么?''萧女史道:``不会的,练成这种功夫的人,体内的真气一定会形成漩涡,真气成了漩涡,就一定会有吸力。''俞子牙道:``这就是这种功夫最奇妙之处,但江湖中大多数人都不明白这道理,就因为大家都不知道这种吸力是那里来的,所以就有人认为这是一种邪术。却不知这才是内家正宗的绝顶心法。''慕容珊珊道;.``可是\ldots\ldots 她既然已必无败理,为什么又要忽然住手休战呢?''俞子牙等人的脸色都很沉重,萧女史道;.``这只有一个解释,那就是是燕南天也练成了一种神奇的武功,足以和她的.''明玉功``一争长短。''慕容珊珊道:``世上难道还有别的功夫能和.''明玉功``相抗么?''萧女史道;.``嫁衣神功。这种功夫取的乃是.''为他人作嫁衣裳``之意。''慕容珊珊道;.``既是他人的嫁衣裳,对自己岂非没有用了么?''萧女史道:``不错,只因这种功夫练成之后,真气就会变得如火焰般猛烈,自己非但不能运用,反而要日日夜夜受它的煎熬,那种痛苦实在非人所能忍受,所以她只有将真气内力转注给他人。''她叹了口气,接道:``但若要练成这.''嫁衣神功``,至少也要二十年苦功,又有谁舍得将如此辛苦练成的功力送给别人呢?''俞子牙道;.``所以昔日江湖中有种传说,你若是想害一个人时,才会传授他.''嫁衣神功``的心法,让他受一辈子的苦。''慕容珊珊道:``如此说来燕大侠若是真的练成了.''嫁衣神功``,那么他非但不能和移花宫主动手,只怕早已被折磨死了。''俞子牙道:``嫁衣神功转注给第二人之后,他本身固然已油尽灯枯,第二个人却可受用无穷。''慕容珊珊道:``前辈的意思难道是说,有人练成了『嫁衣神功』,再转注给燕大侠的。''俞子牙道;.``不然,.''嫁衣神功``经过转注之后,其威力也大减,已不能和『明玉功''相提并论了。"慕容珊珊越想越不明白,瞧了大家一眼,但大家却都在等著她再问下去,因为她非但口齿清楚,而且反应很快,问的话都能切中要点,别人既没有插嘴的余地,只有索性让她一个说了。

幸好这时俞子牙已接著道:``要知只有大智大慧的人,才能创立出一种独树一格的武功来,创出这.''嫁衣神功``的人,更是天生奇才,并世无双,这种功夫若真的只能为人作嫁,他又为何要苦心将之创出呢?''大家都不知道他话中真意,只有等他自己说下去。

俞子牙接道;.``世上只知.''嫁衣神功``绝不可练,却不如又本是可以练的,只不过要练这种功夫,另有一种秘诀而已。''慕容珊珊终于有了问话的机会,立刻问道:``什么秘诀?''

\hypertarget{ux7b2cux4e00ux767eux5effux4e00ux7ae0-ux4e92ux76f8ux6b8bux6740}{%
\chapter{第一百廿一章
互相残杀}\label{ux7b2cux4e00ux767eux5effux4e00ux7ae0-ux4e92ux76f8ux6b8bux6740}}

俞子牙将``嫁衣神功''之练法,向众人解说道:``只因这种功夫太过猛烈,所以练到六七成时,就要将练成的功力全都毁去,然后再从头练过。''萧女史笑道;``这正如一个人吃核桃,竟将核桃连壳吞下,结果被梗死了,旁边有人看见,就说核桃是吃不得的,却不知核桃非但可吃,而且很好吃,只不过吃核桃时,要先敲破外面的硬壳而已。''弥十八道;``这就叫,欲用其利,先挫其锋。''俞子牙道:``嫁衣神功经此一挫,再练成后,其真气的锋棱已被挫去,但威力却丝毫末减,练的人等于已将这种功夫练过两次,对这种实力的性能,自然摸得更熟,非但能将之发挥最大的威力,而且可以收发由心,运用如意了,可是,若要将''嫁衣神功``练到六七成,也得要有更多年的苦功,又有谁舍得将多年的苦功毁于一旦呢?''萧女史道:``所以若非有绝大勇气和毅力的人,绝不会练得成这种功夫的。''鬼童子到这时才叹了口气,道:``可见这燕南天的确是位不世的奇才,我们幸好没有找他较量,否则恐怕又要倒楣了。''其实他们只知其一,不知其二。燕南天练这种功夫时,并末有心将之毁去再练的,他性子又强又拗,总认为别人不能做的事,他一定能做。所以他一心只想以本身的力量将``嫁衣神功''征服,谁知他功夫还末练成,就在``恶人谷''遭遇了不幸,全身的功力都被毁去。

这也正是吉人自有天相,屠娇娇,李大嘴他们本想杀了他的,谁知却反而帮了他一个大忙。

他们以七、八人之力来毁燕南天的功力,正如以鞭驯狗,``嫁衣神功''被他们七.八人之力合力围玟后,已锋厉尽折,但这种功力本就是准备练成后再毁的,所以毁去后体内犹有余根,使练的人再练时,便可事半而功倍。

这正如七.八个人合力要将一棵树铲去,他们就连这棵树齐根锯断了,却不知地面下的根却还是存著的。若非如比,燕南天纵然不死,也和废人无异了,又怎能将功力完全恢复后,而且更胜从前。

慕容珊珊感慨了半晌,又忍不住问道:``但各位又怎知道燕大侠已练成''嫁衣神功』呢?``俞子牙道:''你和人交手时,只是全身功力凝集,地面上只怕也会留下你的脚印,但燕南天所站的地方,却连半只脚印也没有留下来,这难道是说他的功力还不及你么!``慕容珊珊笑道:''燕大侠的功力若不及我,移花宫主早已将他置之于死地了。``俞子牙道;''正是如此,就因为燕南天的功力已可完全收发自如,不到运用时,绝不会有一丝外泄,所以他站的地方才会毫无痕迹。``萧女史道:''也就因为他的功力已和他的人结成一体,任何外力都不能将之动摇,所以移花宫主虽已将``明玉功''练至极峰,对他也无法可施。``慕容珊珊叹了口气,道:''听了前辈们这番话,弟子们当真是茅塞顿开。``突听小蛮高声唤道:''顾少爷,顾公子,你快进来吧,有人想你已快想疯了。"大家苒头望去,只见顾人玉果然已走了过来。

小仙女狠狠瞪了小蛮一眼,却又忍不住笑了,若是换了别人,也许还会害羞,但她却不管这么多,居然迎了上去,跺脚道;``你究竟到什么地方去了,怎地也不留一句话。''顾人玉的脸又红了起来,讷讷道:``我\ldots\ldots 我去替小鱼儿做了一件事。''小仙女道:``他还会有什么好事要别人做,你只怕又上了他的当。''顾人玉叹道:``我如今才知道我们以前都误会了他,他实在并不是个坏人。''小仙女眨著眠道:``他是怎样将你打动的?这小鬼的本事倒不小。''顾人玉道:``江别鹤父子想串通了让燕大侠上当的,他们故意装作互不相识,江玉郎才好乘机救他的父亲,再找机会向燕大侠下毒手。''小仙女恨恨道:``我早就知道这父子两人都不是好东西。''顾人玉道:``但燕大侠自从经过恶人谷一役之后,已今非昔比,很快的就看出了他们的阴谋,就用重手法先废了他们的武功,再将他们囚禁在一个山洞里,等小鱼儿亲手去报父母之仇。''小仙女拍掌笑道:``想不到这父子两人也有今天,这真是大快人心。''顾人玉叹道:``但若非小鱼儿,又有谁会知道他们父子是如此奸恶的小人?''小仙女道:``不错,他这一生中,总算做了这么件好事,可是,他又要你去做什么呢?''顾人玉道:``他要我去放了他们。''

小仙女吃惊道:``放了他们?''

顾人玉道:``不错,他非但要我去放了他们,而且还要我替他们安排个可以安身养命的地方,因为他们已变成了废人,已无力求生。''他叹了口气,接著道:``而且,在江湖中闯荡的人,难免没有仇家,若是知道他们武功已失,必定会来寻仇的,他们自然也万万不能回去,所以小鱼儿就要我安排他们到顾家庄去做园丁,这么他们既不至于冻馁而死,也不怕别人会去寻仇了。''小仙女愣然道:``江别鹤害死了他的父母,他自己非但不报复,反而怕别人找他们算帐,这小鬼究竟又在打什么主意?''顾人玉道;``江别鹤虽对不起他的父母,但他却认为这种惩罚已经够了,他认为''冤冤相报血债血还``,并不是一种很明智的思想,江湖中人被这种思想支配,已不如做出了多少愚蠢的事,他决心不再这么做下去。''小仙女道:``父仇不共戴天,他连父仇都不报,难道他能算是人子吗?''顾人玉道;``他认为并不一定要杀死别人才能算报仇,更不想去杀两个已残废无用的人,也许别人会认为他这种想法不对,但他觉得只要自己做得问心无愧,别人对他怎么想,他根本不放在心上。''小仙女道:``你认为\ldots\ldots{}''顾人玉正色道:``我也认为他这种做法是对的,''报仇``这两个字,已不知害了多少人了,江湖中因仇而死的人,每天也不知有多少,若是大家的想法都能和小鱼儿一样,我相信大家过的日子都会平静安乐得多。''他深深注视著小仙女,柔声道:``上天造人,本就不是要人们互相仇杀的,是么?''小仙女道:``那么,他为何不自己去放了他们呢?''顾人玉道:``他怕燕大侠也不赞同他这种想法,是暂时不愿让燕大侠知道。''小仙女道;``原来他还是在用手段,还是在骗人。''顾人玉道;``不错,他的确常常在用手段骗人,但他的居心都是善良的,我想只要是明智的人,就不会觉得他手段用得不对。''小仙女怔了半晌,苦笑道:``他真是个很奇怪的人,实在令人分不清他究竟是个好人,还是个坏人。''俞子牙忽然笑道:``我虽不认得他,也不知道他究竟是好是坏,我只知道江湖中的人若都和他一样,我们就不必远避到海外的荒岛上去了。''轩辕三光拍手道:``格老子,一点也不错,像他这么样的坏人若是多几个,我情愿从此以后再也不摸骰子。''慕容珊珊忽也一笑,道:``那怎么行,以后我们姊妹还想找你再好好赌一场哩。''轩辕三光道;``我只说不摸骰子,并没有说不摸牌九呀。''大家忍不住全都笑了起来,经过这紧张的两昼夜之后,到这时大家总算略为轻松了一些!只有花无缺,心情却更沈重。

他越来越不忍心伤害小鱼儿,他甚至情愿自己被小鱼儿杀死,可是他却不知道,就算他不惜一死,小鱼儿活著却更悲惨。没有一个人在杀死自己的亲兄弟之后,还能安心活著的,他们已注定了要有个悲惨的结局。

这结局看来已是谁都无法改变的了。

混乱之中,谁也没有注意到李大嘴.哈哈儿.杜杀.屠娇娇.阴九幽.白开心,这几人早已半途脱逃。

知道燕南天已出现,就算用刀架在他们脖子上,他们也是万万不敢跟著大家一齐回去的。

那白夫人自然也是寸步不离的跟著白开心。

白开心方才挨了杜杀一耳光,现在半边脸都肿了起来,连嘴都被挤到一边鲜血不时沿著嘴角往外淌。

白夫人忽然悄悄对白开心说道;``你可知道你为什么总是受人欺负吗?''``就因为我遇上了你这扫帚星。''

白夫人也不生气,反而笑了笑,道:``这就是因为他们都有帮手,你却孤单单一个,双拳难敌四手,你既然懂得这道理,为什么不找个帮手呢!''白开心眼睛一亮,立刻拉著白夫人走到旁边,这时他们已走入了乱山之中白开心拉著她躲在一个山坳里,悄悄道:``一言惊醒梦中人,被你这么一说,我倒想起个好帮手来了。''白夫人笑道:``你现在还说我是扫帚星么?''

白开心道;``不是不是,看你这鼻子,我就知道你有帮夫运。''白夫人笑骂道:``少拍马屁,先说说你想出的那个帮手是谁吧!''白开心道;``这些人里面,李大嘴和我早就是冤家对头,现在杜老大也好像站到他那一边去了,他们两人功夫都不错,尤其杜老大更扎手,我本可找哈哈儿对付他们的,但这胖子比泥鳅还滑,我若找他,他说不定一转头就将我给卖了。''白夫人道:``屠娇娇呢?''

白开心道:``这阴阳人也不行,她表面上虽然跟我不错,但平生最怕杜老大,要他和杜老大作对,她死也不肯的。''白夫人笑道;``说不定她和杜老大暗中有一手。''白开心嘻嘻笑道:``这他妈的真一点也不错,所以我算来算去,只有说动阴九幽来搭档,再加上你,有我们三个人,就足够对付他们一帮的了。''白夫人眨著眼道:``你有法子说得动他吗?''白开心道:``本来没法子,现在却有了。''白开心笑著继缤说道;``这人平生最喜欢鬼鬼祟祟的在暗中偷看别人的隐私,尤其喜欢看人家夫妇''办事,因为他自己不能人道,所以只有看别人来过瘾。``白夫人眼珠一转,笑啐道:''你难道想和我在这里『办事』吗?``白开心搂过她,笑道:''你他妈的又说对了,只要我们一开始,用不了多久他就会来的。``白夫人吃吃笑道:''有别人在旁边看著,我就不行了。``白开心笑骂道;''骚婆子,你以为我不懂吗,有别人在旁边偷看,你才更起兴哩!``他重重拧了她一把,道:''动呀!"

白夫人咬著他的耳朵,喘息著道:``重些,好人,拧重些\ldots\ldots 再重些\ldots\ldots 再重些\ldots\ldots 越重越好。''过了半晌,白开心忽然笑道;``阴老九,你要看,索性就出来看个痛快吧?''阴九幽果然在山石后笑道:``好小子,你这老婆真娶对了,她真有两下子。''白夫人喘息著笑道:``你想不想上来试试?''

阴九幽大笑道:``不必不必,只要让我一饱眼福,我已足领盛情了。''白开心道;``不错,你还是乘著这时候多开心吧,若是等燕南天找著你,就来不及了。''提起``燕南天''这名字,阴九幽脸色就变了,冷冷道:``所以你现在才这么样不要命的开心是么?''白开心道:``我们没关系,我可没有害过燕南天,也用不著怕他,可是你\ldots\ldots{}''他嘿嘿一笑,故意不往下说了。

阴九幽铁青著脸呆了半晌,忽也笑道:``你以为我害怕?燕南天此刻只怕已死在移花宫主手里,我怕什么?''白开心大笑道:``不错不错,你实在用不著害怕,燕南天的武功根本就他妈的一文也不值,和移花宫主一动手脑袋就要搬家了。''阴九幽道:``燕南天武功虽不错,但移花宫主\ldots\ldots{}''白开心截口道;``你们只知道燕南天武功已搁下多年,却忘了他说不定已在这些年里练成一种极厉害的功夫,否则他怎敢来找移花宫主呢?难道他真活得不耐烦了么?''阴九幽怔了一怔,脸色更难看。

白开心道:``何况,移花宫主已在那山洞中饿了好几天,人是铁,饭是钢,她们就算有天大的本事,也受不了的,现在就算已吃下了一些东西,但武功至少也要打个七折八扣,她们在这种时侯和燕南天动手\ldots\ldots 依我看只怕是凶多吉少。''阴九幽怔了半晌,道;``就算他不死又有何妨,我惹不了他,难道还躲不了他么?''白开心道;``燕南天若想找一个人麻烦时,我还末听说过有人能跑得了,何况,一个人活到五.六十岁,还要整天提心吊胆,东藏西躲的过日子,那也未免太可怜了。''阴九幽咬著牙,恨恨道;``你在我面前说这种话究竟是什么意思!''白开心悠然道;``我也没有什么别的意思,只不过是想帮你个忙,让燕南天莫要再找你了。''阴九幽动容道:``你有法子?''

白开心闭著眼养了半天神,才缓缓道:``据我所知,向燕南天下手的人并不是你。''阴九幽立刻道;``不错,是李大嘴出的主意,由屠娇娇假扮成死尸\ldots\ldots{}''白开心一拍巴掌,道:``这就对了,只有他们两人,才是真正的罪魁祸首,燕南天只要看到他们两人已死了,气就平了一大半,也就不会再穷凶恶极的找别人算帐了。''阴九幽目光闪动,道;``你的意思是叫我去杀了他们?''白开心道:``你一个人当然不成,但再加上我们夫妻两人,再用点妙计,还怕他们不乖乖的将脑袋送上来?''阴九幽沉吟著,冷冷道:``我看你们这是想为自己出气。''白开心道;``一点也不错,我若不想替自己出气,又何必来帮你的忙?我又不是你老子。''阴九幽反而笑了,喃喃道:``我看这两人也活够了,早点送了他的终,也未尝不是好事。''白开心大喜道:``你他妈的,总算弄明白了,我总算没有找错人。''阴九幽也笑道:``你他妈的眼睛总算没有瞎。''白开心又沉下了脸,叹道:``可是,我们现在若去下手,哈哈儿虽然一定袖手旁观,但杜老大却一定不肯答应的,只要他一伸手管闲事,那就麻烦了?''阴九幽目光闪动,道:``你小子难道想连杜老大也一齐做了!''白开心笑了笑,道;``这就叫:一不做,二不休。''阴九幽冷笑道;``可是以我们三人之力想去斗他们三人,就叫肥猪拱门,一定要送给别人去宰了。''白开心叹道:``你小子真没有学问。连一点兵法也不懂。''阴九幽沈吟了半晌,眼睛又一亮,道:``你的意思莫非是\ldots\ldots{}''白开心道:``乘其不备,攻其弱点,然后再逐个击破。''阴九幽道;``但\ldots\ldots 杜老大又有什么弱点呢?''白开心道:``他的弱点就是自命不凡,好逞英雄,所以我们最好用女人去对付他,因为他总认为女人是弱者。''白夫人忽然一笑,道:``认为女人是弱者的男人,一定要倒楣的。''哈哈儿.屠娇娇.杜杀和李大嘴也在前面停了下来,他们觉得这里的地势很幽僻,可以在这里先休息休息再说。他们知道从今以后,又要开始无休无尽的逃亡了,他们也知道在长期的逃亡之前,必定要先打好主意。但他们现在却连一点主意也没有。

屠娇娇忽然道:``你们看燕南天是否真的会死在移花宫主手里呢?''李大嘴道:``我看他已是凶多吉少的了。''

杜杀冷冷道:``我看倒未必''燕南天的武功,我知道得很清楚。"他望著自己那只断手,目光中现出一种凄凉之意。

屠娇娇道:``燕南天若不死,一定不会放过我们的。我们能逃到那里去呢?难道再回恶人谷?''他们都知道在``恶人谷''里虽可躲得过别人,但却躲不过燕南天的,可是除了恶人谷外,他们又无处可去。一时之间,连这些最多嘴的人也说不出话来。

也不知过了多久,李大嘴皱眉道:``那损人不利己白小子到那里去了?莫非又想打主意害人?''杜杀冷冷道;``他只怕还没有这么大的胆子!''屠娇娇正想说什么,忽然见到白夫人踉跄奔了过来,满面俱是泪痕,仓皇的四下瞧了一眼,就奔到杜杀面前,扑地跪了下去,嗄声道:``杜大哥,求求你\ldots\ldots 求求你救救我吧!''杜杀皱眉道:``救你?什么事?''

白夫人流泪道:``我刚跟他成亲还不到一天,他就想不要我了,而且还要杀了我''我孤苦伶仃,无依无靠,只有求杜大哥替我作主了,我知道杜大哥一向都主持公道的。``杜杀果然怒道:''他既已与你成亲,怎么能再做这种事。``李大嘴立刻接口道:''是呀,他就算不喜欢你,把你休了也就是了,怎么能杀你呢?我早就知道这小子一点良心也没有。``杜杀霍然站起,厉声道:''这小子在那里,你跟我去,看他还敢不敢动你一根手指。``白夫人破涕为笑,道:''我早就知道只有杜大哥是英雄,绝不会眼见一个弱女子受人欺负的。"她挣扎著从地上爬起来,好像连站都站不稳了。

杜杀皱眉道:``你已受了伤?''白夫人又叹了口气,默然道:``他早已将我打得满身都是伤,杜大哥你看。''她忽然解开衣襟,露出了赤裸的身子。

杜杀立刻闭上眼睛,道;``用不著再看了,快穿好衣服跟我走吧\ldots\ldots{}''他话末说完,突觉胸口一凉。一柄利刃已刺入了他的胸膛。

杜杀狂吼一声,断腕上的铁钩已挥了出去。

但白夫人一招得手,就地便滚出了三四丈,她只觉冰凉的铁钩已擦著了她胸前敏感的地方,连脸都骇白了。

这变化实在太突然,李大嘴、屠娇娇、哈哈儿也想不到这女人竟如此大胆,居然敢向杜杀下毒手,只见杜杀反手拔出了胸前的利刃,一股鲜血箭一般喷了出来,他想要再扑上去,但力气已随著鲜血流出。

他一双杀人如麻的手上已沾满了鲜血,他自己的血!李大嘴.屠娇娇双双赶过去,想扶住他,杜杀却甩脱了他们的手,仰天长叹道:``杜某英雄一世,想不到竟死在这淫贱无耻的妇人手里。''屠娇娇咬了咬牙,道:``杜老大,你放心,她也活不了的!''杜杀道:``好,很好\ldots\ldots{}''他忽又凄然一笑,道:``早知如此,我们不如死在燕南天手里了,他毕竟还是个英雄\ldots\ldots{}''``英雄''两字说出,这自命英雄的人已倒了下去"白夫人彷佛直到这时才想起要跑,在地上一滚,翻身掠起。

李大嘴厉声道:``你还想跑了么?''

语声中阴九幽忽然鬼魂般自山石后一掠而出,挡住了白夫人的去路!白夫人话也不说,迎面三掌拍了过去。

但阴九幽只不过一伸手,就已拧住了她的手腕,格格笑道:``今日我们若让你跑了,''十大恶人``还能混么?''白夫人咬牙道:``我已受够了你们这些恶人的欺负,你杀了我吧,反正我已出了一口气。''阴九幽冷笑道:``杀了你,那有如此容易!''

他转过头向李大嘴一笑,道:``听说人肉要往活人身上切片下来吃著才有味,这道好菜我就送给你吧。''李大嘴狞笑道:``我若不切她一千八百刀再让她死,我就不姓李。''白夫人嘶声大笑道:``我远以为你真想替杜老大报仇哩,原来你只不过想吃我的肉而已,来吧,乖儿子只管来吃老娘的奶吧,老娘若皱一皱眉头,就算是你养的。''屠娇娇冷冷道:``这女人自己一定不会有这么大的胆子下毒手,一定是白开心在暗中主使。''白夫人大笑道:``老娘还用得著别人主使?老实告诉你们,白开心那王八蛋也早已死在老娘小肚子上了,正等著你们去收尸哩。''屠娇娇目光闪动,道:``你们先慢动手杀她,我先过去瞧瞧。''李大嘴狞笑道:``你放心,我保险她三天三夜都死不了的。''他拿起那把上面还带著杜杀鲜血的利刃,一步步向白夫人走了过去。

哈哈儿瞧了瞧他,又瞧了瞧已远在十丈外的屠娇娇,咧嘴一笑,道:``白开心那张脸死了后不知是何模样,我还是瞧瞧他去吧。''李大嘴还末走到白夫人面前,她已放声大叫了起来,道:``阴九幽,你若是人,就杀了我吧。莫要让这不是人的东西折磨我,我做鬼也感激你。''阴九幽咯咯笑道;``我是人?谁说我是人?我根本就不是人!''李大嘴大笑道:``原来你也会害怕的,看在你杀了白开心的份上,我就少剐你一百刀吧,但一千七百刀却是再也少不得的。''白夫人嗄声道:``你这畜生,你\ldots\ldots{}''李大嘴一步窜到她面前狞笑道:``我本不知道第一刀该往那里下手,现在才知道了,我要先割下你的舌头,叫你长舌妇的舌头短些。''他手中的刀已划了出去。

谁知就在这时,阴九幽忽然放开了白夫人,两人一左一右,两旁一夹,李大嘴还未弄明白这是怎么回事,左边协下已挨了白夫人一掌,右边协下也挨了阴九幽一拳,口吐鲜血扑倒在地。

李大嘴居然还没有死,呻吟著道;``你\ldots\ldots 你们还要将我弄到那里去?为什么不索性杀了我?''白夫人柔声道:``你要割我一千七百刀,我怎么舍得现在就杀了你呢?''她俯下身,嘴唇似乎还在动著,也不知在李大嘴身旁说了句什么话,李大嘴的眼睛忽然一亮。

忽然间,白夫人双手将李大嘴的身子一托,李大嘴平空飞起三丈,竟一把揪住了阴九幽的头发,将他整个人压在下面。阴九幽做梦也想不到还有这一手,大惊之下,刚想挥拳将李大嘴击开,但白夫人的虎尾银针已刺入了他肋下的血海穴。他立刻身子一麻,动都不能动了。

李大嘴喘息著狞笑道:``你既然知道天下最毒是妇人心,为什么还要相信妇人的话,你害死了我,以为自己会有什么好处?''阴九幽喉咙里格格直响,一句话都末说出,脖子已被李大嘴生生拧断了,于是他剩下的一半``人''也变做``鬼'',而且是个无头鬼。李大嘴望著自己的一双血手,忽然疯狂般大笑起来。

白夫人嫣然道;``李大爷,我让你替自己报了仇,你应该怎么感激我?''李大嘴笑声渐渐停顿,喘著气道:``你究竟想怎么样?''白夫人柔声道:``无论你感不感激我,我却还要帮你一个忙。''李大嘴道:``求求你,莫要再帮我的忙了,我已经受不了。''白夫人笑道:``这忙我是非帮不可的,你们『十大恶人』对我这么好,我怎么能不好好的报答你们呢?''她嫣然微笑著,忽然飞起一脚,将李大嘴踢得晕了过去。

\hypertarget{ux7b2cux4e00ux767eux5effux4e8cux7ae0-ux5154ux6b7bux72d7ux70f9}{%
\chapter{第一百廿二章
兔死狗烹}\label{ux7b2cux4e00ux767eux5effux4e8cux7ae0-ux5154ux6b7bux72d7ux70f9}}

白开心果然已死了。

他活著时就长得不大怎么样,死了后更是难看透顶,就活像个风乾了的黄鼠狼,被人高高吊起在树上。

屠娇娇叹了气,喃喃道:``我早就知道这人不得好死的,''想不到他死得这么惨,我们帮他将白老虎的女人抢过来,反而倒实是帮白老虎的大忙"她嘴里说著话,人已到树下。

突听哈哈儿在后面大呼道:``留神些,这小子说不是在装死的。''他不说这句话还好,一说这句话,屠娇娇自然扭回头瞧他去,她心神一分,白开心的双手已扼住她的脖子,哈哈儿身子一震,呆在那里,似已再也走不动半步。

只听白开心冷冷笑道:``屠娇娇,我和你本没有什么过不去,本来也并不想杀你的,这全是阴老九的主意,你死了变鬼,最好找他去,千万莫要找我。''屠娇娇眼睛翻白,非但说不出话,连听都听不见了,白开心一个斤斗从树上翻了下来,望著哈哈儿笑道:``你看我装死的本事并不比屠娇娇差吧,她一生最会装死害人,只怕再也想不到自己也会死在一个''假死人``的手上。''哈哈儿叹了口气,喃喃道:``天道循环,看来果然是报应不爽,我下辈子投胎,再也不敢害人了。''白开心大笑道:``哈哈儿,你难道也要改邪归正了么?''十大恶人``现在只怕只剩下三.四个了,正要让你来撑场面哩,因为你一个人的份量就可以抵得上别人两三个。''哈哈儿似乎喜出望外,头声道;``你\ldots\ldots 你肯饶了我?''白开心昂起了头,背负起了手道;``也许,只不过我还要考虑考虑。''哈哈儿苦笑著脸道:``求求你,莫要考虑了吧,只要你饶了我,你就是我的重生父母,从今以后你要我往东,我就不敢往西,你要我爬,我就不敢走。''白开心嘻的一笑道:``既然如此,你就爬一圈给我看看。''哈哈儿什么话也不说,竟真的在地上爬了起来。

白开心拍手大笑道;``大家快来看呀,这里有个胖乌龟。''哈哈儿一面咫,一面涎著脸笑道;``胖乌龟,满地爬,白大爷见了拍手笑哈哈,白奶奶一旁赶来了,笑得更像一朵花\ldots\ldots{}''白夫人果然来了,笑得果然像一朵花。

白开心向她挤了挤眼睛,道:``大功告成了么?''白夫人娇笑道:``饶他们奸似鬼,也要吃老娘的洗脚水。''白开心道:``阴老九呢?''

白夫人道:``我们当然不能留下他,否则我们以后\ldots\ldots 以后要好的时侯,他若定要在旁边瞧著,那怎么受得了。''白开心大笑道:``你他妈的说得真对极了,兔子既然全都已死光,还留著那条狗干什么?''白夫人将李大嘴重重往地上一抛,道:``只有这大嘴狼,我知道你舍不得这么快就杀死他的。''白开心跳过去搂著她脖子笑道:``你真是我的心肝小宝贝,肚子里的蛔虫。''白夫人吃吃的笑著道;``这胖乌龟呢?''

白开心道:``这胖乌龟反正我们随时都可以要他命的,何必急著杀他,留下他来,我还可以像逗龟孙子似的逗著他玩,岂不开心。''白夫人眼珠子一转,道:``那么这大嘴狼呢?你想怎么样对付他?''白开心眨著眼道:``你难道又有什么好主意?''白夫人笑道:``他什么人的肉都吃过了,连他老婆儿子都被他吃下肚里,只有一种人的肉还没有吃过,死了岂非遗憾得很,所以我一定要帮他这个忙。''白开心道:``那种人的肉他还没有吃过?''

白夫人道:``吃人的人。''

白开心眼睛一亮,道:``你莫非要他自己吃自己的肉么?''白夫人奸笑道:``你说这主意好不好?''白开心又搂住了她,大笑道:``你真是个活宝贝,从今以后叫我怎么离得开你。''笑声中,只听``格''的一响。

白夫人忽然惨呼一声,身子就像一滩泥似的倒了下去,脖子也软软的垂到一边,眼睛却铜铃般瞪著白开心,她目光中充满了惊骇恐惧,嗄声道:``你\ldots\ldots{}''脖子已被扼断的人,怎么还说得出话来,她虽有许多凶恶狠毒的话要骂,但却只能发出一阵令人毛骨悚然的``丝丝''声,就像是响尾蛇临死前发出的声音。她至死也不相信白开心居然会杀她,正如杜杀和阴九幽至死也不相信她会杀他们一样。

白开心笑嘻嘻道:``你用不著做出这副样子,其实你也早就该知道,兔子既已死光了,我还要你这条母狗干什么?''白夫人瞪著他,眼珠都快凸了出来,无论什么人见到她这么样瞪著自己,晚上只怕永远再也休想睡得著觉了。

但白开心却一点也不在乎,悠然接著道:``何况,我若不杀你,迟早都会被你杀死的,我知道你心里早已将我们这些人全都恨之入骨,所以才会先利用我杀死他们,然后再想法子杀死我,我若不先下手为强,后下手就遭殃了。''白夫人脖子上的青筋一阵跳动,一口气再也咽不上来。

突听李大嘴叹道:``白开心呀白开心,我一直以为你是个呆子,谁知你却比我想像中聪明得多。''白开心狞笑道:``你还没有死?是不是在等著吃自己的肉?''李大嘴勉强笑道:``一点也不错,我早已想尝尝我自己的肉是什么滋味,只可惜没有机会,如今机会到了,我怎能错过。''白开心反倒怔怔,道:``真的?''

李大嘴叹道:``人之将死,其言也善,到现在我为何还要骗你?''白开心眨了眨眼睛,忽又大笑道:``你以为我真会相信你的话?偏偏不给你吃?''李大嘴道:``你不相信最好,快拿刀来吧,但千万莫要割我的手臂,那里的肉最粗。''白开心瞪了他半晌,忽然转向哈哈儿道:``你相不相信他的话?''哈哈儿一直乖乖的趴在地上,此刻忙陪著笑道:``狗改不了吃屎,这大嘴狼没有别人的肉可吃,吃吃自己的肉总也是好的,白老大又何必让他临死还过一次瘾?''白开心抚掌道;``不错不错,我非憋死他不可,他的肉虽长在他身上,我却一定要他眼巴巴的看著乾著急?''李大嘴喘息著道:``我知道阴老九想杀我们,是为了要燕南天以为我们都死了,不再追查,但你要杀我们,对你又有什么好处?''白开心咧嘴一笑,道:``我的名字叫什么你难道都忘了吗?''李大嘴怔了半晌,苦笑著喃喃道:``损人不利己\ldots\ldots 损人不利己\ldots\ldots{}''他的气似也喘不过来了,闭上眼睛,不再说话。

哈哈儿陪笑道:``白老大,你还要看我这只胖乌龟爬么?''白开心挥了挥手,笑道:``起来吧,今天我已看够了。''哈哈儿道:``你\ldots\ldots 你真的已饶了我?''

白开心道:``你放心,只要你乖乖的听话,我绝不会害你,众家兄弟现在已只剩下咱们两个人了,我怎么舍得再杀你,你若死了,天下还有谁肯跟我交朋友?''哈哈儿顿首道;``多谢白老大,多谢白老大。''白开心哈哈大笑,开心得直好像自己已做了皇帝。但他还是``白开心''了一场。

哈哈儿磕到第三个头时,背后忽然飞出三枝乌黑的短箭,``嗖''的射入白开心的胸膛。白开心大喝一声,翻身跌倒,眼睛瞪著哈哈儿,那神情也正和白夫人方才瞪著他时完全一样。

哈哈儿仰天大笑道:``白开心呀白开心,你聪明一世,糊涂一时,我竟会如此怕你,你难道一点也看不出我在作假么?''白开心两只手紧紧握著胸前的箭翎,嗄声道;``我若看得出就不会上你这胖乌龟的当了。''哈哈儿道:``哈哈,但你凭什么认为我会如此怕你?''白开心道:``我以为胖子都怕死,绝对不敢向我出手的,我又以为胖子都不中用,就算你下手我也不怕,但我却忘了\ldots\ldots 忘了\ldots\ldots{}''他脸色发白,嘴唇发黑,眼睛也发花了。

哈哈儿道;``哈哈,你莫非又忘了我的''笑里藏刀三暗器``?你可知道昔日江湖中有多少人死在我这一手绝招之下?''白开心喘息著道:``但你为何要杀我?我们两人在一起搭档,岂非比一个人好得多。''哈哈儿不再望他,却走到屠娇娇面前,柔声道:``娇娇,你还能看得到么?我已为你报仇了!''白开心讶然失声道:``原来你居然是在为她报仇?你难道是她的\ldots\ldots{}''哈哈儿脸上的肉都在簌簌的发抖,彷佛痛苦已极,白开心不用再问,已知道他是屠娇娇的什么人了。

只听哈哈儿黯然道:``这许多年来,你总算对我不错,现在你死了,我心里还真难受得很\ldots\ldots{}''白开心苦笑道:``屠娇娇在恶人谷里熬了二十年,我早就知道她一定熬不住的,一定有个姘头,但我却一直认为她的姘头是杜老大。''他忽又大笑道:``其实我早该知道她的姘头是你,像她这种不男不女的老太婆,除了你这胖乌龟外,她还能勾引上谁?''哈哈儿怒吼著,升起一脚,将他踢得飞了出去。他终于再也说不出损人不利己的刻薄话了。

哈哈儿咬著牙喘息了半晌,突见屠娇娇眼睛竟张开了一线,哈哈儿又惊又喜,立刻蹲了下去:``你还能说话么?''屠娇娇点了黠头,嘴唇动了动,彷佛说了句话。

但她的声音实在太微弱,哈哈儿一个字也听不到,只有将耳朵凑在屠娇娇嘴旁,柔声渲;``你还有什么心事,都对我说吧,我一定替你做到。''屠娇娇呻吟著道;``我们是同命鸳鸯,是不是?''哈哈儿连连点著头道:``不错不错,我们是同命鸳鸯,也是恩爱夫妻。''屠娇娇嘴角泛出最后一丝微笑,道:``所以我死了,你也不能活著。''哈哈儿这一惊真是非同小可,想跳起来却已来不及了。屠娇娇两条手臂已蛇一般缠住了他,一口咬在他咽喉上,哈哈儿拚命挣扎,终于还是挣不动了。只见他脸色渐渐发白,身上的血潮水般流入了屠娇娇的肚子,忽然用尽全身力气,压到屠娇娇身上。只听``格剌格剌''一连串声响,屠娇娇全身的骨头都被压折了,哈哈儿挣扎著站了起来,``哈哈,哈哈,哈哈''仰天大笑了三声,``噗''地倒了下去,终于再也笑不出了。

``李大嘴一直在瞧著,眼睛都已发直。这时他才长长叹了口气,喃喃道:''很好,很好,``十大恶人''终于死光了,三十年前,我就知道这些人必定会自相残杀而死的,老天造我们十个人,本就是要我们以毒攻毒,自相残杀,否则他造一个就够了,何必造出十个来。"他挣扎著想站起来,却又跌倒,于是他就挣扎著往山上爬,似乎想远远躲开这些人的尸身。

山风吹过,远处似有野兽的吼声传来。山坳后灌木丛中,似乎有个很深的洞穴,洞上怪石峥嵘,远远看来就像是一只洪荒怪兽,这洞穴就像是怪石的嘴。李大嘴挣扎著爬了进去。

洞穴里阴森而潮湿,而且还有种令人作恶的臭气。但李大嘴却像是平生也没有到过如此舒服的地方,他长长叹了口气,在地上躺了下来。地上又是泥泞,又是碎石,但李大嘴却像是躺在少女香闺中的软床上,自言自语著道:``李大嘴呀李大嘴,老天能给你这么样一块地方,让你安安静静的等死,已经算对你很不错了,你还有什么好埋怨的?''可是老天并没有让他安安静静的等死。也不知过了多久,洞外忽然响起了一阵脚步声。李大嘴立刻就想跳起来,怎奈他此刻连爬都爬不动了,到了这种时候,一个人反而能听天由命了。

他索性躺著不动,暗道:``我吃了一辈子的人,老天就算要将我喂狗,也是应该的。''只听一人道:``就是这地方,绝不会错的,洞口那块石头我认得。''这人说的虽是很普通的两句话,但话声却是威严沉重,李大嘴虽听不出这声音是谁,但也不知怎他,一颗心竟``怦怦''的跳了起来。

过了半晌,又听得一人道;``大叔,我瞒著你做了件事,你肯原谅我吗?''听到这声音,李大嘴才真的吃了一惊。这人竟是小鱼儿,另一人自然就是燕南天,李大嘴再也想不到自己躲来躲去,竟还是躲不了。

他骇得连气都不敢喘了。

其实他既已去死不远,又还有什么可怕的!但一个人若是做了亏心事,想不害怕都不行。

只听燕南天道;``你瞒著我做了什么事?''小鱼儿道:``我\ldots\ldots 我已瞒著你老人家,叫人来将江别鹤父子放了。''燕南天似也怔了怔,厉声道:``你为什么要这样做?难道你已忘了那血海深仇么?''小鱼儿道:``我没有忘,可是我觉得并不一定要杀死他们才算报仇,我实在不喜欢杀人,别人杀了我亲人,是他们卑鄙恶毒,我若再杀了他们,岂非也变得和他们一样了么?所以我要他们活著来忏悔自己的罪恶,我觉得这样做比杀死他们更有意思得多。''他在燕南天面前侃侃而言,居然毫无畏怯之意。

燕南天沉默了很久,黯然长叹道:``好孩子,好孩子,江枫有你这么样一个儿子,他死在九泉之下也该瞑目了,燕大叔白活了几十年,竟还不及你通达明理。''小鱼儿道:``那么,我和花无缺那一战,可以不打了么?''燕南天声音又变得严厉起来,道;``那万万不行。''小鱼儿道:``为什么不行呢?我和花无缺又没有什么仇恨,为什么要跟他拚命!''燕南天厉声道:``这一战并非为了报仇,而是为了荣誉,男儿汉头可断,血可流,却绝不能做出丢人的事,到了这种时候,你若还想临阵脱逃,又怎么对得起你死去的父母,又怎么对得起我?''小鱼儿叹了口气,也已哑无言了。

燕南天道:``不但你势必要与花无缺一战,我也势必要和移花宫主一战,因为做错了事的人一定要受惩罚,大丈夫有所不为,有所必为,我们就算明知要战死,也绝不能逃避,这道理你明白了么?''小鱼儿黯然道:``我明白了。''

燕南天长叹了一声,柔声道:``我也知道你和花无缺已有了友情,所以不愿和他动手拚命,但一个人活在世上,有时也势必要做一些自己不愿做的事,造化之弄人,命运之安排,无论多么大的英雄豪杰也无可奈何的。''小鱼儿也长叹了一声,忽然道:``大叔,我只想求你一件事。''燕南天道:``你说吧。''

小鱼儿道:``我只求你见到杜杀.李大嘴他们的时候,莫要杀死他们。''燕南天怒道:``这些人早已该死了,你为何又要为他们求情?''小鱼儿道:``一个人做错了事,固然要受惩罚,但他们受的惩罚已够了,他们在''恶人谷``受了二十年活罪后,简直已变成了一群可怜虫,每天都在心惊胆战,东窜西逃,又像是一群丧家的野狗,以后怎么敢再去害人呢?''听到这里,李大嘴忍不住暗暗叹道:``骂得好,实在骂得好,只不过你还是骂得太轻了,我们实在连野狗都不如。''只听燕南天道:``江山易改,本性难移,你怎知他们以后不会再害人了。''小鱼儿道;``他们入谷之前,曾经收藏了一批珠宝,就为了这批珠宝,他们几乎连命都送掉了,大叔你想,他们若还有害人的勇气,是不是尽可再去抢更多的珠宝来?为什么它要寻找这批珠宝呢?''他叹了气,道:``由此可见,他们的胆子早就寒了,已只不过是一些贪财的老头子,那里还有''十大恶人``的雄风,这种人活著已和死了差不多,大叔你又何必再追杀他们,让他们苟延残喘多活两年又有何妨?''听到这里,李大嘴已是热泪盈眶,忍不住长叹道;``小鱼儿,我们果然全都看错你了,我们若能想到你会为我们求情,只怕也不会落到这样的下场。''他话末说完,燕南天和小鱼儿已窜了过来。

小鱼儿失声道:``李大叔,是你!你怎么会变成这样子的?''李大嘴凄然一笑,道:``这只怕就叫做,善恶到头终须报,多行不义必自毙。''小鱼儿道:``别的人呢?''

李大嘴叹道:``死光了,全都死光了。''

小鱼儿讶然道;``是谁杀了他们?''

李大嘴苦笑道:``除了他们自己,还有谁能杀得死他们?''他长叹了一声,道:``燕大侠,我们实在很对不起你,你快杀了我吧。''燕南天见到他时,本是满面怒容,但此刻却已露出怜悯之色,只是摇了摇头,长叹无语。

李大嘴苦笑道:``我知道我这种人已不值得燕大侠出手了,一个人若活到连他的仇人都认为不值得杀的时侯,他活著还有什么意思?''他忽又哈哈一笑,道:``幸好我已活不长了,这倒是我的运气,否则我非撒泡尿自己淹死不可。''燕南天叹息了一声,道:``走吧。''

小鱼儿道:``我现在不能走。''

燕南天皱眉道:``你还要等什么?''

小鱼儿垂头道:``我小的时侯,他对我不错,现在他落到这种地步,我怎么能抛下他,让他一个人在这里等死?''李大嘴大声道;``你用不著可怜我,也用不著报我的恩,我对你根本没什么好处,我将你养大,也只不过是想要你长大出来害人而已。''小鱼儿笑了笑,道:``无论你们是为了什么,但总算将我养大了,现在我活得既然很有意思,就不能忘记你们的恩情。''

\hypertarget{ux7b2cux4e00ux767eux5effux4e09ux7ae0-ux5584ux6076ux4e00ux7ebf}{%
\chapter{第一百廿三章
善恶一线}\label{ux7b2cux4e00ux767eux5effux4e09ux7ae0-ux5584ux6076ux4e00ux7ebf}}

李大嘴听了小鱼儿的话,长叹了一声,喃喃道:``恩情,恩情\ldots\ldots{}''十大恶人``养大的孩子,居然口声声不忘记恩情,看来''十大恶人``早就该改行做别人的保姆才是。''只听一人娇笑道:``不错,我们将来若有了孩子,一定要请你来做奶妈。''原来苏樱也跟在后面来了,只不过一直没有说话。

李大嘴瞪著她,道:``你们有了孩子,你和谁有了孩子。''苏樱瞟了小鱼儿一眼,垂下头抿嘴笑道:``现在虽没有,但将来总会有的。''李大嘴大笑道:``好小子,想不到这条小鱼儿终于还是上了钩,看来你钓鱼的本事倒真不小。''小鱼儿冷冷道:``她自我陶醉的本事更大。''

苏樱嫣然道:``就算我是自我陶醉好不好?无论你说什么,我都听你的。反正我若有了孩子你就是他爸爸。''小鱼儿叹了气,苦著脸道:``我遇见这种人,真是倒了八辈子穷楣了。''李大嘴拍掌大笑道:``想不到小鱼儿终于也遇见克星了,好姑娘,我真佩服你,你真比我们''十大恶人``加起来还有办法。''他笑著笑著,面上又显出痛苦之色,显然又触动了伤处。

燕南天忽然道:``有恩必报,本是男儿本色,你留在这里也好。''小鱼儿道;``你老人家呢?''

燕南天沉吟著,道:``我在山顶等你,算来她们想必已找到花无缺了,你也该赶紧去。''小鱼儿苦笑道:``我既然已答应了你老人家,就算爬,也要爬著去。''燕南天道:``很好!''他说完了这两个字,就大步走了出去。

李大嘴望著他雄伟的背影消失在黑暗中,忍不住长叹道:``这人倒的确乾脆得很,真不愧是条男子汉?''苏樱嫣然笑道;``我觉得你老人家也不愧是条男子汉。''李大嘴怔怔,道:``我?''苏樱道:``十大恶人中,也只有你老人家能算是条男子汉,只可惜你老人家的口味和别人不同,否则只怕已成了燕大侠的好朋友。''李大嘴大笑道:``好,好,好,居然有这么漂亮的美人儿说我是男子汉,我死了也总算不冤了,只可惜看不到你养出来的小小鱼儿而已。''小鱼儿苦笑道:``想不到李大叔也戴不得高帽子的,被人拍了两句马屁,立刻就帮著别人来算计我了。''李大嘴瞪眼道:``算计你?告诉你,你能得到她这样的女人,实在是你天大的运气,我若非已死了一大半,不和你争风才怪。''小鱼儿咧嘴一笑,道;``说不定我的味以后也会变得和李大叔一样,半夜将她吃下肚子里。''李大嘴目中又露出痛苦之色,似乎再也不愿听到别人提起这件事。

小鱼儿是多么聪明的人,察言观色,立刻改口道:``苏樱,你若真想李大叔做你儿子的奶妈,就该赶快替李大叔治好这伤势。''李大嘴怔了怔,道:``你要她为我治伤?''

小鱼儿笑道:``李大叔还不知道么?这丫头除了会自我陶醉之外,替人治病的本事也蛮不错的。''李大嘴忽然大笑道:``我本还以为你真是个聪明人,谁知你却是个笨蛋。''小鱼儿道:``你\ldots\ldots 你难道不愿让她\ldots\ldots{}''李大嘴抢著道:``我问你?你看我几时充过英雄?装过好汉?''他摇了摇头,自己接著道:``没有,从来也没有,我一向是个很怕死的人,若是这伤还能治,我只怕早已跪下来求她了。''苏樱柔声道:``你老人家至少该让我看看。''李大嘴瞪眼道:``看什么?我自己伤得有多重我自己难道不知道?你以为我也是个笨蛋?''小鱼儿和苏樱对望一眼,已知道他这是存心不想再活了,两人交换了个眼色,心里已有了打算。

李大嘴忽又笑道:``你若真认为欠我的情非还不可,倒有个法子报答我。''小鱼儿道:``什么法子?''

李大嘴笑道:``我现在已饿得头都晕了,你想法子请我好好吃一顿吧,听说黄泉路上连家饭馆都没有,若要我一路饿著去见阎王,那滋味可不好受。''小鱼儿怔了半晌,摸著头笑道:``这地方人肉倒真不好找,我看只有请李大叔将就些,从我大腿上弄一块肉去当点心吧。''李大嘴又瞪眼道:``人肉?谁说要你请我吃人肉?''小鱼儿道:``你\ldots\ldots 你不吃人肉?''李大嘴道:``人肉就算真的是天下第一美味,我吃了几十年,也早该吃腻了。''他往地下重重啐了一口,道:``老实说,我现在一想起人肉就想吐。''小鱼儿这才真的怔住了。

李大嘴笑了笑,又道:``你以为我真的很喜欢吃人肉么?老实告诉你,我吃人肉,只不过是为了吓唬人而已。''小鱼儿道:``吓唬人?''

李大嘴道:``你可知道屠娇娇、哈哈儿他们为什么总是对我存著三分畏惧之心?那没有别的原因,只不过因为我吃人!吃人的人总是能令人害怕的。''小鱼儿摸著脑袋,简直有些哭笑不得。

李大嘴忽又叹了口气,道:``一个人活在世上,是为恶?还是为善?那分际实在微妙得很,我之所以成为''十大恶人``也只不过是一念闲事。''他笑著问道:``你们可猜得出我怎会成为''十大恶人``的么?''小鱼儿只有摇头道:``我猜不出。''

李大嘴目光凝注著远方的黑暗,缓缓道:``我从小就好吃,连广东人不敢吃的东西,我都吃过,就是没吃过人肉,总是想尝人肉是什么滋味。''他笑了笑,接著道:``我不去想这件事也倒好了,越想越觉得好奇,有天我杀了个人后,终于还是忍不住将他的肉煮来吃了,觉得味道也不过如此而已,虽然比马肉嫩些,但却比马肉还要酸,非多加葱姜作料不可。''小鱼儿忍不住问道:``人肉的滋味既然并不高明,你为什么还要吃呢?''李大嘴道;``我正在吃人的时候,忽然被个人撞见了,这人本是我的对头,武功比我还高些,但他瞧见我吃人,立刻就吓得面色如土,掉头就走,以后见到我,也立刻落荒而逃,连架都不敢和我打了。''他又笑了笑,道:``我这才知道吃人原来能令人害怕的,自从发现了这道理后,我才忽然变得欢喜吃人起来。''小鱼儿道:``难道你\ldots\ldots 你喜欢别人怕你?''

李大嘴道:``世上的人有许多种类,有的人特别讨人喜欢,有的人特别讨人厌,我既不能讨人欢喜,也不愿令人讨厌,就只有要人害怕。''他笑著接道;``能要别人害怕,倒也蛮不错,所以我也不觉得人肉酸了。''小鱼儿听得目瞪呆,只有苦笑,只有叹息。

他本想问:``你为什么连自己老婆的肉都要吃呢?''但他并没有问出来,因为他已不愿再让李大嘴伤心。

李大嘴道;``这些年来,我总是一个人偷偷去烧些猪肉来解馋,但却不敢被别人看到,就好像和尚偷吃荤一样,越是偷著吃,越觉得好吃。''他大笑著接著道:``但现在我再也不必偷著吃了,你们快好好请我吃一顿红烧蹄膀吧,要肉肥皮厚,咬一口就沿著嘴直流油。''小镇上没有山珍海味,但红烧蹄膀总是少不了的。三斤重的蹄膀,李大嘴竟一口气吃了两个,幸好他们是在客栈里开了间屋子关起门来吃的,否则别人只怕要以为他们是饿死鬼投胎。

吃到一半,小鱼儿将苏樱借故拉了出去,悄悄问道:``你扶他进来的时候,已查过他的伤势了么?''苏樱叹道:``他伤的实在不轻,肋骨就至少断了十根,别的地方还有五处硬伤,若非他身子硬朗,早就被打死了。''小鱼儿道:``我只问你现在还有没有救?''

苏樱道:``若是他肯听我的话,好生调养,我负责可以救他,只怕\ldots\ldots{}''她长长叹了口气,接著道:``他自己若已不想活了,那么就谁也无法救得了他。''小鱼儿咬著嘴唇,道:``我真不懂,他本是个很看得开的人,为什么会忽然想死呢?''苏樱幽幽道:``一个人到了将死的时候,就会回忆起他一生中的所作所为,这种时候还能心安理得,问心无愧的人,世上并不多。''小鱼儿叹道:``不错,他一定是对自己这一生中所做的事很后悔,所以想以死解脱,以死忏悔。''苏樱黯然道;``到了这种时候,一个人若能将生死之事看得很淡,已经很难得了,所以我才说他不愧是条男子汉。''就在这时,突见一个人在小院外的墙角后鬼鬼祟祟的向他们窥望,小鱼儿眼珠子一转,缓缓道:``李大叔对我不错,他变成这样子,我的脾气自然不好,一心只想找个人来出气,现在总算被我找著了。''他嘴里说著话,忽然飞身掠了过去,躲在墙角后的那人显然吃了一□,但却并没有逃走的意思,反而躬身笑道:``我早就知道鱼兄吉人天相,无论遇著什么灾难,都必能逢凶化吉,如今见到贤伉俪果然已安全脱险,实在高兴得很。''小鱼儿矢笑道;``你这兔子什么时候也变得善颂善祷起来了。''原来这人竟是胡药师,小鱼儿想找个人出气的,听到他马屁拍得刮刮响,火气又发不出来了``胡药师道:''自从那日承蒙贤伉俪放给在下一条生路后,在下时时刻刻想找贤伉俪拜谢大恩,今日总算是天从人愿。``小鱼儿道:''既然如此,你见到我们,为何不过来?反而鬼鬼祟祟的躲在这里干什么!``他忽又顿住道:''那位铁萍姑铁姑娘呢?``胡药师似乎怔了怔,讷讷道;''我\ldots\ldots 我不大清楚。``小鱼儿皱眉道;''你们两人本是一齐逃出去的,你不清楚谁清楚!``胡药师垂下头,结结巴巴的陪著笑道:''她\ldots\ldots 她好像也在附近,可是\ldots\ldots 可是\ldots\ldots{}``小鱼儿一把揪住他衣襟,怒道:''你小子究竟在搞什么鬼?快老老实实说出来吧,就凭你也想在我面前玩花样,简直是孔夫子门前卖百家姓。"胡药师脸色都变了,急得更说不出话来。

苏樱柔声道;``有话好说,你何必对人家这么凶呢!''小鱼儿叫了起来,道:``你还说我凶,这小子若是没有做亏心事,怎么怕成这副样子,我看他说不定已将人家那位大姑娘给卖了。''胡药师苦著脸道;``她\ldots\ldots 她只叫我来将两位拖住片刻,究竟是什么事,我也不知道。''小鱼儿瞪大了眼睛,道:``是她叫你来将我们拖住的!''胡药师道;``不错。''

小鱼儿又怒道;``放屁,我不相信,你和铁萍姑八竿子打不到一齐去,为什么要听她的话。''苏樱眨著眼道:``你怎知道他们八竿子打不到一齐去,说不定他们\ldots\ldots{}''小鱼儿忽又大声道:``那么,她为什么要叫他来拖住我们呢?她想瞒著我们干什么!''苏樱咬著嘴唇,缓缓道:``你想,她会不会和李大叔有什么关系?''小鱼儿道:``他们又会有什么关系?''

苏樱道;``李大叔以前的夫人,不也是姓铁么?''小鱼儿心头一跳,忽然想起以前铁萍姑只要一听到``恶人谷'',一听到``李大嘴''这名字,神情就立刻改变了。他又想起铁萍姑曾经向他探问过``恶人谷''的途径,似乎想到恶人谷去,她到恶人谷莫非就是为了去找李大嘴?想到这里,小鱼儿什么话都不再说,跳起来就住院子里跑,还末跑到门外,已听到一阵啜泣声音自他们那屋子里传了出来。

小鱼儿一听就知道这赫然正是铁萍姑的哭声。他立刻冲了进去,只见李大嘴木然坐在椅子上,满面都是凄惨痛苦之色,铁萍姑却已哭倒在他身旁,手里还握著把尖刀,只不过此时她手指已松开,刀已几乎掉落在她手边。

小鱼儿怔住了,失声道:``这是怎么回事?铁姑娘你难道认得李大叔么!''铁萍姑已泣不成声,李大嘴惨笑道;``她认得我的时候,你只怕还末出生哩''小鱼儿讶然道;``哦?难道她是\ldots\ldots 是\ldots\ldots{}''他望了望李大嘴,又望了望铁萍姑,下面的话实在说不出来,因为说出来后连他自己都无法相信。

李大嘴却长长叹息了一声,黯然道;``她就是我的女儿。''小鱼儿这才真的呆住了。

他本想问:``你不是已将自己的女儿和老婆一齐吃了么?''但此时此刻,他又怎么能问得出这种话来。

李大嘴却已看出他的心意,叹道:``普天之下,都以为李大嘴已将自己的老婆和女儿一齐吃了,二十年来,我也从未否认,直到今天\ldots\ldots 唉,今天我已不能不将此事的真象说出来,否则我只怕连做鬼都不甘心。''他语声中竟充满了悲愤之意,像是在承受著很大的冤屈,忍受著满心的悲苦,苏樱悄悄掩上了门,送了杯茶去。

李大嘴道:``铁老英雄爱才如命,将她女儿嫁给了我,希望我能从此洗心革面,我也一直都很感激他老人家的好意,可是\ldots\ldots 可是\ldots\ldots{}''他咬了咬牙?接著道:``可是她女儿却对我恨之入骨,认为我辱没了她,竟在暗中和她的师弟有了不清不白的关系,我知道了这件事后,心里自然是又恨又恼,但念在铁老英雄对我的恩情,我还希望她能从此改过,只要他们不再暗中做那荀且之事,我也不愿将他们这种见不得人的丑事宣扬出去。''他嘴角的肌肉不住颤抖,咬紧了牙齿,接著道:``谁知她非但不听我的良言,反而骂我是个活乌龟,叫我莫要管她的事,我一怒之下,才置之于死地,又将她活活煮来吃了,以泄我心头之恨!''苏樱动容道:``此事既有这么段曲折,你老人家为什么一直不肯说出来呢?''李大嘴道:``这一来是因为我顾念铁老英雄的面子,不忍令他丢脸伤心,二来也是为了我自己的面子。''他惨然一笑,接道:``你们想,江湖中人若知道李大嘴的老婆偷人,我怎么还混得下去,我宁可被人恨之入骨,我也不能让人耻笑于我。''苏樱垂下头,亦自黯然无语,只因他很了解李大嘴这种人的心情,也很同情他的遭遇。

李大嘴道:``我杀了她后,也自知江湖中已无我容身之处,铁无双必定恨不得将我千刀万剐,所以我只好连夜进入恶人谷,可是\ldots\ldots{}''他瞧了铁萍姑一眼,黯然道:``可是我却不愿叫我的女儿在那种地方长大成人,所以我就将她交托给别人,我只希望她能平平安安的长大,平平安安的度过一生。''小鱼儿忍不住问道:``你将她交托给谁了?''李大嘴恨恨道;``我本以为那人是我的朋友,谁知\ldots\ldots 唉,我这种人是永远没有朋友的!''铁萍姑忽然痛哭著道:``那夫妻两人日日夜夜的折磨我,还说我是李大嘴的女儿,是个坏种,所以我很小的时候就逃了出去。''李大嘴凄然道:``你能投身于移花宫,也总算你不幸中的大幸了。''铁萍姑流著泪道:``后来我听人说起李\ldots\ldots 李\ldots\ldots{}''苏樱柔声道:``你听人说起李大叔的故事,就认为你的母亲和姊妹都已被李大叔吃了,你又因为李大叔受了那么多折磨,所以,你一直在心里恨你自己的父亲,认为他不但害了你的母亲,也害了你一生。''铁萍姑已哭成个泪人儿,那里还说得出话来。

李大嘴黯然道;``所以,她今天就算要来杀我,我也不怪她,因为她\ldots\ldots 她\ldots\ldots{}''说著说著他也不禁泪流满面。

小鱼儿忽然大声道:``今天你们父女团聚,误会又已澄清,大家本该高高兴兴的庆祝一番才是,怎会反而哭哭啼啼的呢?''李大嘴忽然一拍桌子,也大声道:``小鱼儿说得是,今天大家都应该开心些,谁也不许再流泪了。''胡药师逡巡著走过去,似乎想替她擦擦眼泪。

谁知铁萍姑又板起了脸,道:``谁要你来,站开些!''胡药师脸红了红,果然又逡巡著站在一边。

小鱼儿和苏樱相见一笑,苏樱道:``看来今天只怕是喜上加喜,要双喜临门了。''李大嘴瞧了瞧胡药师,又瞧了瞧他女儿,道;``这位是\ldots\ldots{}''胡药师红著脸垂首道:``晚辈姓胡,叫胡药师。''李大嘴喃喃道:``胡药师,莫非是十二星相中的''捣药师``么?''胡药师道:``晚辈正是。''李大嘴仰首大笑道:``想不到''十二星相竟做了我的晚辈,看来有个漂亮女儿真是蛮不错的。"铁萍姑虽然红著脸垂下头,却并没有什么恼怒之意。但胡药师却只敢远远的站著偷偷的瞧。

苏樱悄声道:``胆子放大些,没关系,什么事都有我帮你的忙。''小鱼儿拍手大笑道:``看来你那几声贤伉俪叫得实在有用,现在却怎地将拍马屁的本事忘了,还不快跪下来叫岳父。''胡药师红著个脸真的要往下跪了,但铁萍姑的脸一板,他立刻又吓得站了起来,脸都吓得发白。

小鱼儿想到铁萍姑所受的苦难,想到江玉郎对她的负心,此刻也不禁暗暗替她欢喜。

胡药师的年纪虽然大些,但铁萍姑这朵已饱受摧残的鲜花,正需要一个年纪较大的男人细心呵护。年纪大的男人娶了年轻的妻子,总是会爱极生畏的,更绝不会因为铁萍姑不幸的往事而看不起她。

小鱼儿喃喃道;``看来老天爷早已将每个人的因缘都安排好了,而且都安排得那么恰当,根本用不著别人多事操心。''苏樱悄悄笑道:``不错,他老人家既已安排了让我见到你,你想跑也跑不了的。''小鱼儿刚瞪起眼睛,只听李大嘴大笑道:``今天我实在太开心了,我平生从来也没有像今天这么样觉得心安理得,也从没有像今天这么样愉快,我若能死在这种时候,死在这种地方,也总算不枉我活了这一辈子\ldots\ldots{}''只听他语声渐渐微弱,竟真的就此含笑而去。

\hypertarget{ux7b2cux4e00ux767eux5effux56dbux7ae0-ux751fux6b7bux4e24ux96be}{%
\chapter{第一百廿四章
生死两难}\label{ux7b2cux4e00ux767eux5effux56dbux7ae0-ux751fux6b7bux4e24ux96be}}

铁萍姑和胡药师已护送著李大嘴遗体走了。临走的时侯,铁萍姑似乎想对小鱼儿说么,但几次欲言又止,终于什么话都没有说。小鱼儿却知道她是想问问江王郎的下落,而她毕竟还是没有问出来,可见她对江玉郎已死了心。

这实在是好几个月来,小鱼儿最大的快事之一。

临走的时候,胡药师似乎也想对小鱼儿说什么,但他也像铁萍姑一样,欲言又止并末说出,小鱼儿也知道他是想问问白夫人的下落,但他并没有问出来,可见他已将一片痴心转到铁萍姑身上。

这也令小鱼儿觉得很开心。有情人终成眷属,本是人生的最大快意事。

小鱼儿面带著微笑,喃喃道:``无论如何,我还是想不通这两人怎会要好的,这实在是件怪事。''苏樱柔声道:``这一点也不奇怪,他们是在患难中相识的,人的情感,在患难中最易滋生,何况,他们又都是伤心人,同病相怜,也最易生情。''她嫣然一笑,垂著头道;``我和你,岂非也是在患难中才要好的么?''小鱼儿朝她皱了皱鼻子,道:``你和我要好,但我是不是和你要好,还不一定哩。''苏樱笑道:``你莫忘了,这是老天爷的安排呀!''小鱼儿笑道:``你少得意,莫忘了你的情敌还没有出现哩,说不定\ldots\ldots{}''他本想逗逗苏樱的,但是提起铁心兰,就想起了花无缺,他心就像是结了个疙瘩,连话都懒得说了。

苏樱的脸色也沉重了起来,过了半晌,才叹息著道;``看来你和花无缺的这一战,已是无法避免的了。''小鱼儿也叹了气,道;``嗯。''

苏樱道:``你是不是又在想法子拖延。''

小鱼儿道:``嗯。''

他忽又抬起头瞪著苏樱,道;``我心里在想什么,你怎么知道?''苏樱嫣然道:``这就叫心有灵犀一点通。''甜蜜的笑容刚在脸上掠过,她就又皱起了眉道:

``你想出了法子没有?''

小鱼儿懒洋洋的坐了下来,道:``你放心,我总有法子的。''苏樱柔声道:``我也知道你一定有法子,可是,就算你能想出个此以前更好的法子,又有什么用呢?''小鱼儿瞪眼道:``谁说没有用?''

苏樱叹道:``这是就算你还能拖下去,但事情迟早还是要解决的,移花宫主绝不会放过你,你看,他们在那山洞里,对你好像已渐渐和善起来,可是一出了那山洞,她们的态度就立刻变了。''小鱼儿恨恨道:``其实我也早知道她们一定会过河拆桥的。''苏樱道:``所以你迟早还是难免要和花无缺一战,除非\ldots\ldots{}''苏樱温柔的凝注著他,缓缓道:``除非我们现在就走得远远的,找个山明水秀的地方隐居起来,再也不见任何人,再也不理任何人。''小鱼儿沉默了半晌,大声道:``不行,我绝不能逃走,若要我一辈子躲著不敢见人,还不如死了算了,何况,还有燕大叔\ldots\ldots 我已答应了他!''苏樱幽幽叹道:``我也知道你绝不肯这样做的,可是,你和花无缺只要一交上手,就势必要分出死活!是吗?''小鱼儿目光茫然凝注著远方,喃喃道:``不错,我们只要一交上手,就势必要分个你死我活\ldots\ldots{}''他忽然向苏樱一笑,道;``但我们其中只要有一个人死了,事情就可以解决了,是吗?''苏樱的身子忽然起了一阵战栗,头声道:``你\ldots\ldots 你难道能狠下心来杀他?''小鱼儿闭上眼睛,不说话了。

苏樱黯然道:``我知道你们这一战的胜负,和武功的高低并没有什么关系,问题只在谁能狠得下心来,谁就可以战胜\ldots\ldots{}''他忽然紧紧握住小鱼儿的手,颤声道:``我只求你一件事。''小鱼儿笑了笑,道;``你求我娶你作老婆?''

苏樱咬著嘴唇,道;``我只求你答应我,莫要让花无缺杀死你,你无论如何也不能死!''小鱼儿道:``我若非死不可呢?''苏樱身子又一震,道;``那么\ldots\ldots 那么我也只好陪你死\ldots\ldots{}''她目中缓缓流下了两摘眼泪,痴痴的望著小鱼儿道:``但我却不想死,我想和你在一齐好好的活著,活一百年,一千年,我想我们一定会活得非常非常开心的。''小鱼儿望著她,目中也露出了温柔之意!苏樱道;``只要能让你活著,无论叫我做什么都没关系。''小鱼儿道:``若是叫你死呢?''

苏樱道;``若是我死了就能救你,我立刻就去死\ldots\ldots{}''她说得是那么坚决,想也不想就说了出来,但还末说出,小鱼儿就将他拉了过去,柔声道:``你放心,我们都不会死的,我们一定要好好活下去\ldots\ldots{}''他望著窗外的天色,忽又笑道:``我们至少还可以快活一天,为什么要想到死呢?''一天的时间虽短促,但对相爱的人们来说,这一天中的甜蜜,已足以令他们忘去无数痛苦\ldots\ldots 深夜。

四山静寂,每个人都似已睡了,在这群山环抱中的庙宇里,人们往往分外能领略得静寂的乐趣。但对花无缺来说,这静寂的滋味实在不好受。

几乎所有的人都已来到这里,铁战和他们的朋友们,慕容姊妹和她们的夫婿,移花宫主\ldots\ldots 花无缺只奇怪为何听不到他们的声音。他们也许都不愿打扰花无缺,让他能好好的休息,以应付明晨的恶战,但他们为什么不说话呢?他现在只希望有个人陪他说话,但又能去找谁说话呢?他的心事又能向谁倾诉?风吹著窗纸,好像风也在哭泣。

花无缺静静的坐在那里,他在想什么?是在想铁心兰?还是在想小鱼儿?无论他想的是谁,都只有痛苦。

屋子里没有燃灯,桌上还摆著壶他没有喝完的酒,他轻轻叹了口气,正想去拿酒杯,忽然间门轻轻的被推开了,一条致弱的人影幽灵般走了进来。是铁心兰!在黑暗中,她的脸看来是那么苍白,但一双眼睛却亮得可怕,就彷佛有一股火焰正在她心里燃烧著。她的手在颤抖,看来又彷佛十分紧张。这是为了什么?她难道已下了决心要做一件可怕的事!花无缺吃惊的望著她,久久说不出话来。铁心兰轻轻掩上了门,无言地凝注著他。她的眼睛为什么那么亮,亮得那么可怕。

良久良久,花无缺才叹息了一声,道:``你\ldots\ldots 你有什么事?''铁心兰摇了摇头。

花无缺道;``那么你\ldots\ldots 你就不该来的。''铁心兰点了点头。

花无缺似已被她目中的火焰所震慑,一时间也不知该说什么,刚拿起酒壶,又放下,拿起酒杯来喝,却忘了杯中并没有酒。

突听铁心兰道;``我本来一直希望能将你当做自己的兄长,现在才知道错了,因为我对你的情感,已不是兄妹之情,你我又何必再自己骗自己呢?''这些话她自己似已不知说过多少次了,此刻既已下了决心要说,就一口气说了出来,全没有丝毫犹疑。

但花无缺听了她的话,连酒杯都拿不住了。他从末想到铁心兰会在他面前说出这种话来,虽然他对铁心兰的情意,和铁心兰对他的情意,两人都很清楚。可是,他认为这是他们心底的秘密,是永远也不会说出来的,他认为直到他们死,这秘密都要被埋在他们心底深处。

铁心兰凝注著他,目光始终没有移开,幽幽的接著道:``我知道你对我的情感,也绝不是兄妹之情,是吗?''她的眼睛是那么亮,亮得可直照入他心里,花无缺连逃避都无法逃避,只有垂下头道:``可是我\ldots\ldots 我\ldots\ldots{}''铁心兰道:``你不是?还是不敢说?''花无缺长长叹了口气,黯然道;``也许我只是不能说。''铁心兰道:``为什么不能?迟早总是要说的,为什么不早些说出来,也免得彼此痛苦。''她用力咬著颤抖的嘴唇,已咬得泌出了血丝。

花无缺道:``有些事永远不说出来,也许此说出来好。''铁心兰凄然一笑,道;``不错,我本来也不想说出的,可是现在却已到非说不可的时候,因为现在再不说,就永远没有说的时候了。''花无缺的心已绞起,他痛苦的责备自己,为什么还不及铁心兰有勇气?这些话,本该是由他说出来的。

铁心兰道;``我知道你是为了小鱼儿,我本来也觉得我们这样做,就对不起他,可是现在我已经明白了,这种事是勉强不得的,何况,我根本不欠他什么。''花无缺黯然点了点头,道;``你没有错\ldots\ldots{}''铁心兰道;``你也没有错,老天并没有规定谁一定要爱谁的。''花无缺忽然抬起头望著她,他发现她的眸子比海还深,他的身子也开始颤抖,已渐渐无法控制自己。

铁心兰道:``明天,你就要和他作生死的决战了,我考虑了很久很久,决心要将我的心事告诉你,只要你知道我的心意,别的事就全都没有关系了。''花无缺忍不住握起了她的手,颤声道:``我\ldots\ldots 我\ldots\ldots 我很感激你,你本来不必对我这么好的。''铁心兰忽然展颜一笑,道;``我本就应该对你好的,你莫忘了,我们已成了亲,我已是你的妻子。''花无缺痴痴的望著她,她的手已悄悄移到他的脸上,温柔的抚摸著他那已日渐瘦削的颊\ldots\ldots 一滴眼泪,滴在她手上,宛如一粒晶莹的珍珠。

然后,泪珠又碎了\ldots\ldots 风仍在吹著窗纸,但听来已不再像是哭泣了。

花无缺和铁心兰静静的依偎著,这无边的黑暗与静寂,岂非正是上天对情人们的恩赐?爱情是一种奇异的花朵,它并不需要阳光,也不需要雨露,在黑暗中,它反而开放得更美丽。

但窗纸终于渐渐发白,长夜终于已将逝去。

花无缺望著窗外的曙色,黯然无语。他知道他一生中仅有的一段幸福时光,已随著曙色的来临而结东了T光明,虽然带给别人无穷希望,但现在带给他的,却只有痛苦。

花无缺却凄然笑道;``明天早上,太阳依旧会升起,所有的事都不会有任何改变的。''铁心兰道;``可是我们呢?''她忽然紧紧抱著花无缺,柔声道:``无论如何,我们现在总还在一起,比起他来,我们还是幸福的,能活到现在,我们已经没有什么可埋怨的了,是不是?''花无缺心里一阵刺痛,长叹道;``不错,我们实在比他幸福多了,他\ldots\ldots{}''铁心兰道;``他实在是个可怜的人,他这一生中,简直没有享受过丝毫快乐,他没有父母,没有亲人,到处破人冷淡,被人笑骂,他死了之后,只怕也没有几个人会为他流泪,因为大家都知道他是个坏人\ldots\ldots{}''她语声渐渐哽咽,几乎连话都说不下去。

花无缺垂下头望著铁心兰,小鱼儿这一生中本来至少还有铁心兰全心全意爱他的,但现在铁心兰也垂下了头,道;``我\ldots\ldots 我只想求你一件事,不知道你答不答应?''花无缺勉强一笑:``我怎么会不答应?''

铁心兰目光茫然凝注著远方,道:``我觉得他现在若死了,实是死难瞑目,所以\ldots\ldots{}''她忽然收回了目光,深深的凝注著花无缺,一字字道:``我只求你莫要杀死他,无论如何也莫要杀死他?''在这一刹那间,花无缺全身的血液都似已骤然凝结了起来!他想放声呼喊:``你求我莫要杀他,难道你不知道我若不杀他,就要被他杀死!你为了要他活著,难道不惜让我死?你今天晚上到这里,难道只不过是为了要求我做这件事?''但花无缺是永远也不会说这种话的,他宁可自己受到伤害,也不愿伤害别人,更不愿伤害他心爱的人。

他只是苦涩的一笑,道;``你纵然不求我,我也不会杀他的。''铁心兰凝注著他,目中充满了柔情,也充满了同情和悲痛,甚至还带著一种自心底发出的崇敬。但她也没有说什么,只轻轻说了一句;``谢谢你。''太阳还末升起,乳白色的晨雾弥漫了大地和山峦,晨风中带著种令人振奋的草木香气。

小鱼儿深深呼吸了一口气,低头喃喃道;``今天,看来一定是好天,在这种天气里,谁会想死呢?''苏樱依偎在他身边,见到他这副垂头丧气的模样,目中又不禁露出了怜惜之意,轻轻抚摸著他的头发,正想找几句话来安慰他。

突听一人沉声道:``高手相争,心乱必败,你既然明白这道理,就该定下心来,要知这一战关系实在太大,你是只许胜,不许败的。''小鱼儿用不著去看,已知道燕南天来了,只有垂著头道;``是。''燕南天魁伟的身形,在迷蒙的雾色里看就宛如群山之神,自天而降,他目光灼灼,瞪著小鱼儿道:``你的恩怨都已了结了么?''小鱼儿道:``是。''他忽又抬起头来,道:``但还有一个人的大恩,我至今末报。''燕南天道;``谁?''

``就是那位万春流万老伯。''燕南天严厉的目光中露出一丝暖意,道:``你能有这番心意,已不负他对你的恩情了,但雨露滋润万物,并不是希望万物对他报恩的,只要万物生长繁荣,他已经很满意了。''小鱼兄道:``我现在只想知道他老人家在那里?身子是否安好?''``你想见他!''

小鱼儿道:``是。''

燕南天淡淡一笑,道:``很好,他也正在等著想看看你\ldots\ldots{}''小鱼儿大喜道:``他老人家就在附近么?''燕南天道;``他昨天才到的。''

"苏樱也早就想见见这位仁心仁术的一代神医了,只见一个长袍黄冠的道人负手站在一株古松下,羽衣瓢瓢,潇然出尘,神情看来说不出的和平宁静。小鱼儿又惊又喜,早已扑了过去,他本有许许多多话想说的,但一时之间,只觉喉头彷佛被什么东西堵住了,连一句话都说不出来。

万春流宁静的面容上也泛起一阵激动之色,两人一别经年,居然还能在此重见,当真有隔世之悲喜。

燕南天也不禁为之唏嘘良久,忽然道:``已将日出,我得走了。''小鱼儿道:``我\ldots\ldots{}''燕南天道;``你暂时留在这里无妨。''他沉著脸接著道:``只因你心情还末平静,此时还不适于和人交手。''万春流道;``但等得太久也不好,等久了也会心乱的。''燕南天道:``那么我就和他们约定在午时三刻吧!''说到最后一字,他身形已消失在白云飞絮间。

万春流望了望小鱼儿,又望了望苏樱,微笑道:``其实我本也该走开的,但你们以后说话的机会还长,而我\ldots\ldots{}''小鱼儿皱眉道:``你老人家要怎样?''万春流唏嘘叹道:``除了想看看你之外,红尘间也别无我可留恋之处。''小鱼儿默然半晌,忽然向苏樱板著脸道:``两个男人在一齐说话,你难道非要在旁边听著不可?''苏樱眼珠子一转,道:``那么我就到外面去逛逛也好。''万春流望著她走远,微笑道;``脱缰的野马,看来终于上了辔头了。''小鱼儿撇了撇嘴,道:``她一辈子也休想管得住我,只有我管她。若不是她这么听我的话,早就一脚将她踢走了。''万春流笑道;``小鱼儿毕竟还是小鱼儿,尽管心已软了,嘴却还是不肯软的。''小鱼儿道:``谁说我心已软了?''

万春流道:``她若非已对你很有把握,又怎肯对你千依百顺,她若不知道你以后必定会听她的话,现在又怎肯听你的话?''他微笑著接道:``在这方面,女人远比男人聪明,绝不会吃了亏的。''小鱼儿笑道:``我不是来向你老人家求教''女人``的。''万春流道:``我也早已看出你必定有件很秘密的事要来求我,究竟是什么事?你快说吧,反正我对你总是无法拒绝的。''他目中充满了笑意,望著小鱼儿道:``你还记得上次你问我要了包臭药,臭得那些人发晕么,这次你又想开谁的玩笑?''小鱼儿想起那件事,自己也不禁笑了。但他的神情忽又变得严肃,压低了声音,正色道:

``这次我可不是想求你帮我开玩笑了,而是一件性命交关的大事。''万春流也从末见过他说话如此严肃,忍不住问道:``是什么事关系如此重大?''小鱼儿叹了气,道:``我只想\ldots\ldots{}''这两个月以来,苏樱对小鱼儿的了解实在已很深了,女人想要了解她所爱的男人,并不是件太困难的事。平时小鱼儿心里在想什么,要做什么,苏樱总能猜个八九不离!只有这次,她实在猜不透小鱼儿究竟有什么秘密的话要对万春流说。

她本来并不想走得太远的,但想著想著,眼睛忽然一亮,像是忽然下了个很大的决定。于是她就立刻匆匆走上山去。这座山上每个地方,她都很熟悉。

她心里正在想:``移花宫主和花无缺他们已在山上等了两天,他们会住在什么地方呢?\ldots\ldots{}''就在她心里想的时候,她的眼睛已告诉她了。前面山坳后的林木掩映中,露出红墙一角,她知道那就是昔年颇多灵迹,近年来香火寥落的``玄武宫''了。现在,正有几个人从那边走了出来。

这几人年纪都已很老了,但体轻神健,目光灼灼,显然都是一等一的武林高手,其中一人身上还背著一面形状特异而精致的大鼓。还有一个老婆婆牙齿虽已快掉光了,但眼波流动,末语先笑,说起话来居然还带著几分爱娇,想当年必定也是个风流人物。

苏樱并不认得这几人,也想不起当世的武林高手中有谁是随身带著一面大鼓的,她只认得其中一个人。那就是铁心兰。

她发觉铁心兰已没有前几天看来那么憔悴,面上反而似乎有了种奇异的光采,她自然永远不会知道是什么事令铁心兰改变了的。

她不愿被铁心兰瞧见,正想找个地方躲一躲,但铁心兰低垂著头,彷佛心事重重,并没有看到她。

这些人一面说著话,一面走上出去。

铁心兰一行人说的话,苏樱都听不到,只有其中一个满面络腮胡子,生像极威猛的老人,说话的声音特别大。只厅这老人道:``小兰,你还三心二意的干什么,我劝你还是死心塌地的跟著花无缺算了,这小子虽然有些娘娘腔,但勉强总算还能配得上你。''铁心兰垂著头,也不如说了话没有。

那老人又拍著她的肩头笑道:``小鬼,在老头子面前还装什么佯,昨天晚上你到那里去了,你以为做爸爸的真老糊涂了么?''铁心兰还是没有说话,脸却飞红了起来。

那老婆婆就笑著道:``也没有看见做爸爸的居然开女儿的玩笑,我看你真是老糊涂了。''那虬髯老人仰天大笑,彷佛甚是得意。

苏樱又惊又喜,开心得几乎要跳了起来。听他们说的话,铁心兰和花无缺显然又加了几分亲密,而且铁心兰的爹居然也鼓励她嫁花无缺,这实在是苏樱听了最开心的事。

其实天下做父母的全没有什么两样,都希望自己的女儿能嫁个可靠的人,她以后若有个女儿也会希望自己的女儿嫁给``移花宫主''的传人,绝不会希望自己的女儿去嫁给``恶人谷''中长大的孩子。

只听那老人又笑著道:``你既然已决心跟定花无缺了,还愁眉苦脸干什么,等到这场架打完,我就替你们成亲,你也用不著担心夜长梦多了。''那老婆婆也笑道;``未来的老公就要跟人打架,她怎么会不担心呢?若换了是我,只怕早就先想法子去将那\ldots\ldots 那条小鱼弄死了。''那老人哈哈大笑道:``如此说来,谁能娶到你,倒实是得了个贤内助。''老婆婆道:``是呀,只可惜你们都没有这么好的福气。''另一个又高又瘦的老人道:``依我看,花无缺这孩子精气内敛,无论内外功都已登堂入室,显然先天既足,后天又有名师传授,那江小鱼年龄若和他差不多,武功绝对无法练到这种地步,这一战他绝无败理,你们根本就用不著为他担心的。''

\hypertarget{ux7b2cux4e00ux767eux5effux4e94ux7ae0-ux60faux60faux76f8ux60dc}{%
\chapter{第一百廿五章
惺惺相惜}\label{ux7b2cux4e00ux767eux5effux4e94ux7ae0-ux60faux60faux76f8ux60dc}}

但苏樱却开始担心起来,她本来觉得这一战胜负的关键,并不在武功之强弱。而现在,她却越想越觉得这种想法并非绝对正确,小鱼儿的武功若根本就不是花无缺的敌手,那么他就算能狠下心来也没有用,主要的关键还是在花无缺是否能狠下心来向小鱼儿出手。他们两人若是斗智,小鱼儿固然稳操左券,但两人硬碰硬的动起手来,小鱼儿实在连一分把握都没有。她若想小鱼儿胜得这一战,不但要叫小鱼儿狠下心来,还要叫花无缺的心狠不下来。但小鱼儿既能狠下心杀花无缺,花无缺凭什么就不能狠心杀小鱼儿,蝼蚁尚且偷生,何况一个人呢?``花无缺活得好好的,我凭什么认为他会自寻死路呢?他根本就没有理由只为了要让别人活著,就牺牲自己呀。''苏樱叹了口气,忽然发觉自己以前只想了事情的一面,从来也没有设身处地的为花无缺想过。

在她眼中,小鱼儿的性命固然此花无缺重要。但在别人眼中呢?在花无缺自己眼中呢?翻来覆去的想著,越想心情越乱:她自己觉得自己这一辈子心情从来也没有这样乱过。其实她想来想去,所想的只有一句话。要想小鱼儿活著,就得想法子要花无缺死!死人就不能杀人了!苏樱在一棵树后面,等了很久,就看到慕容家的几个姊妹和她们的姑爷陆陆续续的自玄武宫中走了出来。他们的眼睛有些发红,神情也有些委靡不振,显然这两天都没有睡好,江湖中人讲究的本是``四海为家,随遇而安''。但这些养尊处优的少爷小姐们早已不能算是``江湖中人''了。他们就算换了张床也会睡不著的,何况睡在这种冷清清的破庙里。

但他们修饰得仍然很整洁,头发也仍然梳得光可鉴人,甚至连衣服都还是笔挺的,找不出皱纹来。他们也在议论纷纷,说得很起劲,苏樱用不著听,也知道他们谈论的必是小鱼儿和花无缺的一战。这一战不但已轰动一时,而且必定会流传后世。所以他们不惜吃苦受罪,也舍不得离开。

这群人走上山后,苏樱又等了很久,玄武宫里非但再也没有人出来,而且连一点动静也没有了。花无缺是否还留在玄武宫里?移花宫主是否还在陪著他?苏樱咬了咬牙,决定冒一次险。

她想,大战将临,这些人先走出来,也许是要让花无缺安安静静的歇一会儿,所以先上山去等著。现在燕南天既已到了山巅,移花宫主只怕也不会留在这里,她们最少也该让花无缺静静的想一想该如何应战!玄武宫近年香火虽已寥落,但正如一些家道中落的大户人家,虽已穷掉了锅底,气派总算是有的。庙门内的院子里几株古柏高耸入云,阳光虽已升起,但院子里仍是阴森森的瞧不见日色。

苏樱走过静悄悄的院子,走上长阶。大殿中香姻氤氲,``玄武爷''身上的金漆却早已剥落,他座下的龟蛇二将似乎也因为久已不享人间伙食,所以看来有些没精打采的,至于神龛上的长幔更已变得又灰又黄,连本来是什么颜色都分辨不出来了。十来个道士盘膝端坐在那里,垂脸□目,嘴里念念有辞,也不知是在念经,还是在骂人。

苏樱从他们身旁走出去,他们好像根本没有瞧见一样,苏樱本来还想向他们打听消息,但见到他们这样子,也就忍不住了,除了有些脑筋不正常的之外,世上只怕很少有年轻女孩子愿意和道士和尚打交道的。

后院里两排禅房静悄悄的,连一个人影都没后院里两排禅房静悄悄的,连一个人影都没有。花无缺难道也走了么?苏樱正在犹疑著,忽然发现片门后的竹林里还有几间房子,想必就是玄武宫的方丈室。慕容家的姑娘们虽然都是``吃鸡要吃腿,住屋要朝南''的人,但在这出``戏''里,花无缺才是``主角'',主角自然要特别优待。她们就算也想住方丈室,但对花无缺少不得也要让三分。

苏樱立刻走了出去,只见方丈室的门是虚掩著的,正随著风晃来晃去,檐下有只蜘蛛正在结网,屋角的蟋蟀正在``咕咕''的叫著,悟桐树上的叶子一片片飘下来打在窗纸上``噗噗''的响。

屋子里却也静悄悄的没有人声。苏樱轻轻唤道;``花公子。''没有人回应。花无缺莫非已走了?而且走的时候远忘记关上门。

但苏樱既已到了这里,无论如何总得进去瞧瞧。她悄悄推开门,只见这方丈室里的陈设也很简陋,此刻一张自木桌子上摆著两壶酒,几样菜。菜好像根本没有动过,酒却不知已喝了多少。

屋角有张云床,床上的被褥竟乱得很,就彷佛有好几个人在上面睡过觉,而且睡像很不老实。花无缺并没有走,还留在屋子里。

但他的一颗心却似早已飞到十万八千里之外去了。他痴痴的站在窗前,呆呆的出著神,像他耳目这么灵敏的人,苏樱走进来,他居然会不知道。日色透过窗纸,照在他脸上,他的脸比窗纸还白,眼睛里却布满了红丝,神情看来比任何人都委顿。

大战当前,移花宫主为何不想法子让他养足精神呢?难道他们确信他无论在任何情况下都能击败小鱼儿?还是她们根本不关心谁胜谁败?她们的目的只是要小鱼儿和花无缺拚命,别的事就全不放在心上了。苏樱觉得很奇怪,但她并不想知道这究竟是什么原因,因为她知道绝没有任何人会告诉她。

突听花无缺长长叹息了一声,这一声叹息中竟不知包含了多少难以向人倾诉的悲伤和痛苦。

他为了什么如此悲伤,难道是为了小鱼儿?苏樱缓缓走过去,在他身旁唤道:``花公子\ldots\ldots{}''这一次花无缺终于听到了。他缓缓转过头,望著苏樱,他虽在看著苏樱,但目光却似望著很远很远的地方,远得他根本看不到的地方。

苏樱记得他本有一双小鱼儿同样明亮,同样动人的眼睛,可是这双眼睛现在竟变得好像是一双死人的眼睛,完全没有光采,甚至连动都不动,被这么样一双眼睛看著实在不是件好受的事。

苏樱被他看得几乎连冷汗都流了出来,她勉强笑了笑道:``花公子难道已不认得我了吗?''花无缺点了点头,忽然道:``你是不是来求我莫要杀小鱼儿的?''苏樱怔了怔,还末说话,花无缺已大笑了起来。

他笑声是那么奇怪,那么疯狂,苏樱从末想到像他这样的人也会发出如此可怕的笑声来。正常的人绝不会这么样笑的,苏樱几乎已想逃了。

只听花无缺大笑道;``每个人都来求我莫要杀小鱼儿,为升么没有人去求小鱼儿莫要杀我呢?难道我就该死?''苏樱道:``这\ldots\ldots 这恐怕是因为大家都知道小鱼儿绝对杀不死你!''花无缺骤然顿住笑声,道:``他自己呢?他自己知不知道?''``他若知道,就不会让我来了,因为我并不是来求你的。''花无缺道:``不是?''

苏樱道;``不是。''他也瞪著花无缺,一字字道:``我是来杀你的!''这次花无缺也怔住了,瞪了苏樱半晌,突又大笑起来。``你凭什么认为你能杀得了我?你若是真要来杀我,就不该说出来,你若不说出来,也许还有机会。''苏樱道:``我若说出来,就没有机会了么?''

花无缺道;``你的机会只怕很少。''

苏樱笑了笑,道:``我的机会至少比小鱼儿的大得多,否则我就不会来了。''她忽然转过身,倒了两杯酒,道:``我若和你动手,自然连一分机会都没有,但我们是人,不是野兽,野兽只知道用武力来解决一切事,人却不必。''花无缺道;``人用什么法子解决?''

苏樱道:``人的法子至少该比野兽文雅些。''

她转回身,指著桌上的两杯酒道:``这两杯酒是我方才倒出来的。''花无缺道:``我看到了。''

苏樱道;``你只要选一杯喝下去,我们的问题就解决了。''花无缺道:``为什么?''苏樱道:``因为我已在其中一杯酒里下了毒,你选的若是有毒的一杯,就是你死,你选的若是没有毒的一杯,就是我死。''他淡淡一笑,道:``这法子岂非很文雅,也很公平么?''花无缺望著桌上的两杯酒,眼角的肌肉不禁抽搐起来。

苏樱道;``你不敢?''花无缺哑声道:``我为什么一定要选一杯?''苏樱悠然道:``只因为我要和你一决生死,这理由难道还不够么?''花无缺道:``我为什么要和你拚命?''苏樱道;``你为什么要和小鱼儿拚命?你能和他拚命,我为什么不能和你拚命?''花无缺又怔住了。

苏樱冷冷道:``你是不是觉得这样做太没有把握?你是不是只有在明知自己能够战胜对方时才肯和别人决斗?''她冷笑著接道;``但你明知有把握时再和人决斗,那就不叫决斗了,那叫做谋杀!''花无缺脸色惨变,冷汗一粒粒自鼻尖泌了出来。

苏樱冷笑道:``你若实在不敢,我也没法子勉强你,可是\ldots\ldots{}''花无缺咬了咬牙,终于拿起了一杯酒。

苏樱瞪著他,一字字道:``这杯酒无论是否有毒,都是你自己选的,你总该相信这是场公平的决斗,比世上大多数决斗,都公平得多。''花无缺忽然也笑了笑,道:``不错,这的确很公平,我\ldots\ldots{}''突听一人大喝道:``这一点也不公平,这杯酒你千万喝不得!''``砰''的,门被撞开,一个人闯了进来,却正是小鱼儿。

苏樱失声道:``你怎么也来了?''

小鱼儿冷笑道:``我为何来不得?''

他嘴里说著话,已抢过花无缺手里的酒杯,大声道:``我非但要来,而且还要喝这杯酒。''苏樱变色道:``这杯酒喝不得。''

小鱼儿道:``为何喝不得?''

苏樱道;``这\ldots\ldots 这杯酒有毒的。''

小鱼儿冷笑道:``原来你知道这杯酒是有毒的。''苏樱道:``我的酒,我下的毒,我怎会不知道?''小鱼儿怒吼道:``你既然知道,为何要他喝?''苏樱道:``这本就是一场生死的搏斗,总有一人喝这杯酒的,他自己运气不好,选了这一杯,又怎能怪我?''他瞪著花无缺,道:``但我并没有要你选这杯,是么?''花无缺只有点了点头,他纵然不怕死,但想到自己方才已无异到鬼门关前走了一遭,掌心也不觉泌出了冷汗。

小鱼儿望著杯中的酒,冷笑著道:``我知道你没有要他选这杯,但他选那杯也是一样的。''苏樱道:``为什么?''

小鱼儿大吼道:``因为两杯酒中都有毒,这种花样你骗得了别人,却骗不过我,他无论选那杯,喝了都是死,你根本不必喝另一杯的。''苏樱望著他,目中似已将流下泪来。

小鱼儿摇著头道:``花无缺呀花无缺,你的毛病就是太信任女人了!\ldots\ldots{}''苏樱幽幽叹息了一声,喃喃道:``小鱼儿呀小鱼儿,你的毛病就是太不信任女人了。''她忽然端起桌上的另一杯酒,一口喝了下去。

花无缺脸色变了变,嗄声道:``你\ldots\ldots 你错怪了她,这杯毒酒我还是应该喝下去。''小鱼儿道:``为什么!''

花无缺大声道:``这既然是很公平的决斗,我既然败了,死而无怨!''苏樱叹道:``你实在是个君子,我只恨自己为什么要\ldots\ldots{}''小鱼儿忽然又大笑起来,道;``不错,他是君子,我却不是君子,所以我才知道你的花样。''花无缺怒道:``你怎么能如此说她,她已将那杯酒喝下去了!''小鱼儿大笑道:``她自然可以喝下去,因为毒本是她下的,她早已服下了解药,这么简单的花样你难道都不明白么?''花无缺望著他,再也说不出话来。苏樱也望著他,良久良久,才喃喃道:``你实在是个聪明人,实在太聪明了!''他凄然一笑,接著道:``但无论如何,我总是为了你,你实在不该如此对我的。''小鱼儿又吼了起来,道;``你还想我对你怎样?你以为害死了花无缺,我就会感激你吗?''苏樱道:``我自然知道你不会感激我,因为你们都是英雄,英雄是不愿暗算别人的,英雄要杀人,就得自己杀!''说著说著,她目中已流下泪来。但她立刻擦乾了眼泪,接著道:``我只问你,就算我是在用计害人,和你们又有什么不同?''小鱼儿吼道:``当然不同,我们至少比你光明正大些!''苏樱冷笑道;``光明正大?你们明知对方不是你的敌手?还要和他决斗,这难道就很公平?很光明正大吗?难道只有用刀用枪杀人才算公平,才算光明正大.你们为什么不学狗一样去用嘴咬呢?那岂非更光明正大得多。''她指著小鱼儿道;``何况,我杀人至少还有目的,我是为了你,一个女人为了自己所爱的人无论做什么都不丢脸,而你们呢?''她厉声道:``你们马上就要拚命了,不是你杀死他,就是他杀死你,你们又是为了谁?为了什么?你们只不过是在狗咬狗,而且是两条疯狗。''小鱼儿竟被骂得呆住了,一句话也说也说不出来,被人骂得哑口无言,这还是是他平生第一次。花无缺站在那里,更是满头冷汗,涔涔而落。

苏樱嘶声道:``我是个阴险狠毒的女人,你是个大英雄,从此之后,我再也不想高攀你了,你们谁死谁活,也和我完全无关\ldots\ldots{}''她语声渐渐哽咽,终于忍不住失声痛哭,掩面奔出。

她没有回头。一个人的心若已碎了,就永远不会回头了。

悟桐树上的叶子,一片片打在窗纸上,墙角的蟋蟀,还不时在一声声叫著,檐下的蛛网,却已被风吹断了。蛛丝断了,很快还会再结起来,蜘蛛是永远不会灰心的,但情丝若断了,是否也能很快就结起来呢?人是否也有蜘蛛那种不屈不挠的精神?小鱼儿和花无缺面面相对,久久说不出话来。过了很久,花无缺才叹了气,道:``你为何要那么样对她?''小鱼儿又沉默了很久,喃喃道;``看来你和我的确有很多不同的。''花无缺道:``人与人之间,本就没有完全相同的。''小鱼儿道:``她为了我找人拚命!我却骂得她狗血淋头,她要杀你.你却反而帮她说话,这就是我们最大的不同之处。''他苦笑著道:``所以你永远是君子,我却永远只是个\ldots\ldots{}''花无缺打断了他的话,道:``你为何总是要看轻你自己,其实你才是真正的君子,否则你又怎会为了我而伤害她?''他叹息道:``除了你之外,我还想不出还有谁肯为了自己的敌人而伤害自己的情人。''小鱼儿忽然笑了笑,道:``我并不是为了你,而是为了我自己。''花无缺道:``为了你自己?''

小鱼儿道:``不错,为了我自己\ldots\ldots{}''他慢慢的将这句话又重复了一次,目中闪动著一种令人难测的光,这使也看起来像是忽然变成了个很深沉的人.花无缺每次看到他目中露出这种光芒来,就知道很快就会有一个人要倒楣,但这次他的对象是谁?小鱼儿已缓缓接道:``因为找若让你现在就死在别人手上,我不但会遗憾终生,而且恐怕难免会痛苦一辈子。''花无缺动容道;``为什么?''小鱼儿道:``因为\ldots\ldots{}''他的话还没有说出来,突听一人道;``因为他也要亲手杀死你!''这是邀月宫主的声音,但却比以前更冷漠。

她的睑也变了,虽然依旧和以前同样苍白冷酷,但脸上却多了种晶莹柔润的光。她的脸以前若是冰,现在就是玉。

小鱼儿望著她长长叹了气,道:``才两三天不见,你看来居然又年轻了许多,看来天下的美女那该练你那''明玉功``才是。''邀月宫主只是冷冷瞪著他,也不说话。

小鱼儿又叹了口气,道;``自从我将你们救出来之后,你就又不理我了,有时我真想永远被关在那老鼠洞里,那时你多听我的话,对我多客气。''邀月宫主脸色变了变,道;``你的话说完了么?''小鱼儿笑道:``说完了,我只不过是想提醒你一次,若不是我,你就算变得再年轻,不出几天还是要被困死在那老鼠洞里。''从山顶望下去,白云飘渺,长江蜿蜒如带。燕南天孤独的站在山巅最高处,看来是那么寂寞,但他早已学会忍受寂寞,自古以来,无论谁想站在群山最高处,就得先学会如同忍受寂寞,山上并不只他一个人,但每个人都似乎距离他很遥远。山风振起了他衣袂,白云一片片自他眼前飘过。

慕容珊珊忽然长长叹了口气,黯然道:``前不见古人,后不见来者\ldots\ldots 燕大侠虽然绝代英雄但这一生中又几曾享受过什么欢乐?''慕容珊珊叹道:``看来一个人还是平凡些好。''慕容双也叹了口气,悠悠道:``我欲乘风归去,又恐琼楼玉宇高处不胜寒\ldots\ldots{}''突听一人呼道:``来了,来了。''慕容双道:``什么人来了?''她转过身,已瞧见白云缭绕间出现了小鱼儿和花无缺的身影。

山风更急,天色却渐渐黯了。

苏樱茫然走著,也不知走了多远,也不知已走到那里?她只恨不能有一阵霹雳击下,将她整个人都震得四分五裂,一片片被风吹走,吹到天涯海角,吹得越远越好。她又恨不得小鱼儿会忽然赶来,跪在她脚下,求她宽恕,求她原谅,而且发誓以后永远再不离开她。

但小鱼儿并没有来,霹雳也没有击下。杯中的苦酒还满著,她也不知到何时才能喝光。

从铁心兰站著的地方,可以看得到小鱼儿,也可以看得到花无缺,她看到花无缺目光中的痛苦之色,自己的心也碎了。小鱼儿却仍然在笑著,彷佛一点也不担心,他难道早已算准花无缺会杀他?还是他已有对付花无缺的把握?铁心兰咬著嘴唇,咬得出血,血是咸的,心却是苦的,但她的苦心又有谁知道?

\hypertarget{ux7b2cux4e00ux767eux5effux516dux7ae0-ux751fux6b7bux4e00ux640f}{%
\chapter{第一百廿六章
生死一搏}\label{ux7b2cux4e00ux767eux5effux516dux7ae0-ux751fux6b7bux4e00ux640f}}

一阵风吹过,天地间彷佛忽然充满了肃杀之意。

小鱼儿缩了缩脖子,道:``好大的风,好冷,真该多穿两件衣服的?''燕南天皱了皱眉,沉声道:``你难道已觉得有些受不了么?''小鱼儿道:``大叔你放心,我身子还没有那么娇嫩。燕南天默然半晌,缓缓道:''一个内功已有火候的人,虽不能说可以完全寒暑不侵,但至少总不该像常人那么畏寒畏暑。``小鱼儿道:''是。"

燕南天道:``你所练的武功,乃是无数位武林前辈的心血结晶,可说无一招不是武学中的精萃,而且你小时万大叔就已替你打了很多的底子,并没有让你功夫走入邪路,这种种条件加在一齐,所以我才放心让你和花无缺动手,但你功力究竟如何?我并不知道,你很聪明,也很幸运,我唯一只怕你性情太浮,心思太躁,没有将功夫练纯。''小鱼儿垂下头笑了笑,道:``我做别的事虽三心二意,但练武时倒很专心的。''燕南天点了点头,道:``但愿如此就好。''他忽又问道;``你既已和花无缺交过手,可知他的武功如何?''小鱼儿想了想,道:``移花宫能够享这么大的名,武功实在有独得之秘,尤其那种''移花接玉``的功夫,实在令人头痛。''他笑了笑,接著道:``幸好我多少已摸出其中一些诀窍了。''燕南天正色道:``移花接玉只不过是移花宫许多武功之一,移花宫的武功变化繁复,虽冷静却极深契,而且,我看花无缺外表看来虽不如你聪明,其实绝不会比你笨,你的武功博而杂,他的武功精而深,你和他动手时,切莫要和他以招式硬拚,最好先想法子将他的功力耗去几成。''小鱼儿道:``这我也知道,他的根基实在比我打得好,我和他交手,胜算并不多,但我却占了一个很大的便宜。''燕南天厉声道:``武学之道,绝没有便宜可占,你想占人便宜,你就先败了。''小鱼儿肃然道:``是,只不过\ldots\ldots 他武功的深浅,我已全知道,我武功的路数,他却一点也不知道,因为我从来末将真实的武功在人前炫露过。''燕南天目中露出一丝欣慰之色,颔首道;``很好,知己知彼,方能百战百胜。''小鱼儿忽然一笑,道;``燕大叔,我也想问问你老人家一件事。''燕南天道:``你说吧。''

小鱼儿眨著眼睛道:``你老人家若真和邀月宫主动起手来,能有几分胜算,几成把握?''燕南天目光望著远处一朵瓢动的白云,沉默了很久,坚毅的嘴角忽然露出了一丝罕见的微笑。他并没有回答小鱼儿这句话,但小鱼儿已用不著他回答了。小鱼儿面上也不禁露出了会心的微笑。

一直站在旁边没有说话的万春流忽然道;``时候已快到了,你准备好了么?''小鱼儿点了点头,忽又道;``我也还有件事想问万大叔。''万春流笑道;``你问的话我并不见得全能回答的,我知道的事并不比你多。''小鱼儿也笑了笑,道:``但这件事万大叔一定知道。''他忽越然很小心的自攘中取出了个酒杯,道;``这杯子里还有一滴酒,我总怀疑酒里有毒,而且是种无色无味的毒,万大叔你看它究竟是否有毒好么?''万春流接著酒杯,用小指将杯中的余沥沾起了一些,放在鼻子上嗅了嗅,又用舌头轻轻舐了舐,道:``这酒中\ldots\ldots{}''小鱼儿忽又打所了他的话,道;``无论酒中是否有毒,万大叔现在都莫要告诉我。''万春流道:``这又是为了什么?''

小鱼儿叹了口气,道:``因为酒中若真有毒,我就会很生气,酒中若是无毒,我又会觉得很难受,所以万大叔还是等我打完了再告诉我,免得我分心。''万春流虽然觉得很奇怪,还是笑著道:``好,反正你这孩子做的事,总是教人猜不透的。''但小鱼儿却似忘记了一件事。他若是战败,岂非就永远不知道这答案了么?慕容姑娘和他们的姑爷自然也可以同时看到小鱼儿和花无缺两边的情况,他们都觉得有些奇怪。

慕容双道;``你看见了吗?小鱼儿和燕大侠像是有说不完的话要说,但花无缺和移花宫主只是站在那里乾瞪眼。''慕容珊珊道:``不错,看来移花宫主对花无缺这一战的胜负根本一点也不关心,他们师徒间难道连一点感情都没有?''南宫柳叹息了一声,道:``这也许是因为她们觉得花无缺这一战有必胜的把握。''慕容珊珊撇了撇嘴,道:``我看倒未必,花无缺虽然机智武功都不错,但小鱼儿可也不是好惹的,若论动起手来的应变功夫,我看简直没有任何人能比得上他。''慕容双道:``不错,我看花无缺的功力要稍强些,但高手相争,光是功力高并没有太大的作用,主要还是得看当时临机应变,制敌机先。''秦剑道:``据我所知,小鱼儿武学极博,似乎身兼数家之长,这一战至少可有六成胜算。''慕容珊珊道:``我看还不止六成。''

他们对花无缺没有什么好感,所以一心只想小鱼儿得胜,但``狂狮''铁战那边的人就完全不同了。

萧女史正在向铁战道;``你看你女婿这一战有几成把握?''铁战道:``十成。''

萧女史失笑道:"你也莫要太笃定了,我看那小鱼儿并不是好对付的人,何况,他还有燕南天在后面支持他。

铁战道;``那有个屁用,燕南天又不能替他动手的,他就算再聪明,但李大嘴,屠娇娇那几个调教出来的徒弟,强也强得有限。''萧女史道:``哦?我还以为他是燕南天的徒弟哩,早知他武功只不过是你那些恶朋友教出来的,这一战我连看都懒得看了。''突见燕南天长身而起,道:``时候已到了,你去吧。''他这话虽只是对小鱼儿说的,但声如洪钟,响彻了群山。

花无缺也站了起来,向移花宫主躬身道;``师傅还有什么吩咐?''邀月宫主道:``没有了,你去吧,我知道你绝不会令我失望的。''她语音虽平静,心情却也不禁十分激动。

最后的时刻终于到了T这一次,她无论如何也绝不会再让这一战半途中止,这一次,小鱼儿和花无缺必有一人要倒下去。

无论谁想描述她此刻的心情有多么紧张和兴奋,都是多余的,因为她此刻心情之紧张和兴奋,世上根本没有第二个人能想像得到。唯一能知道他的心情的人,自然就是怜星宫主。

她的脸看来比平时更苍白,花无缺转过脸望著她时,她居然避开了花无缺的目光,因为她生怕自己会忍不住将这秘密说出来!她本也不是个富于感情的人,但这雨天,她发觉自己有些变了,因为在那山洞里,她已经历过许多件她平生末曾经历过的事,她从来也未曾想过这件事居然有一天会发生在她身上。

她这一生中从来也不知道一个人面对死亡时是什么滋味,从来也不知道恐惧。她从来也没倚靠过别人,更没有对任何人生出过感激之心。她自然从没有挨过饿,没有喝醉过酒,更绝没有想到自己也会有一天竟倒在一个男人的怀抱中。但这些她活了几十年都没有经历过的事,竟在短短三两天内一齐发生在她身上。而且每件事的印象都是那么鲜明而深刻,她拚命想忘记也忘不了。

这两天她只要一想到小鱼儿,心里就发疼。小鱼儿对她实在不错,而她对小鱼儿呢?这恶毒而残酷的计划,可说全都是她安排的。小鱼儿和花无缺悲惨的命运,只要她说一句话,就可以完全改观,而她竟不能,也不敢将这句话说出来!小鱼儿向燕南天和万春流恭恭敬敬行了个礼,就走了出去,花无缺已在等著他,但他却像是一点也不著急,一一向各人打招呼。

然后向花无缺走了过来。

花无缺望著他在向每个人诀别,心里也不如是什么滋味。因为只有他知道小鱼儿是绝不会死的,他已答应了铁心兰,为了遵守诺言,他已决心牺牲自己,死,并不是容易的事,一个人到了临死的时侯,才知道生命是值得留恋的,但铁心兰的情感,却更令他刻骨铭心,永难舍弃,在两者不可兼得时,他只有舍弃生命,选择爱情。

看到轩辕三光和小仙女他们对小鱼儿所生出的同情和惋惜,花无缺心里更不如是何滋味。现在的他已抱定必死之心,却连一个诀别的对象都没有。

他问自己;``我死了以后,有谁会为我悲伤,为我流泪?''他几乎忍不住要奔到铁心兰面前,和她抱头痛哭一场,可是他并没有这样做,也不能这样做。他只能静静的站在那里,等小鱼儿过来\ldots\ldots 决战已开始!江湖中每天.每时,每刻,都不如有多少人在作生死的决战,但千百年来,只怕再也不会有一次决战比这次更令人伤感的!因为在这一场决战中,两个人都不愿伤害对方,两个人都宁可牺牲自己,这种情况已是江湖中从来末见的了。更令人伤感的是,在这一场决战中死者固然可悲,能活下来的一个人命运却更悲惨。

甚至在决战尚末开始时,甚至远在二十年之前,两个人那已注定只有死路一条。而这两人偏偏竟是亲生的兄弟。在场的人,除了移花宫主之外,无论谁若知道这情况,只怕都难免要伤心落泪,只可惜在两人没有死之前,谁也不会知道这秘密!只有铁心兰的心情,和每个人都不同。花无缺和小鱼儿出手前并没有说话!这也许是因为他们觉得所有的话都早已说完了,现在已没有什么好说的。花无缺也并没有向铁心兰说话,虽然铁心兰的命运已和他联系在一起,无疑已是他生命中最重要的一个。

``开始?''燕南天的叱声方起,两人已猝然动手。但在花无缺出手之前,铁心兰却发现他向她瞧了一眼。

只瞧了一眼``虽然只瞧了一眼,但却已胜过千言万语。铁心兰看到他的目光,已知道他是在向她诀别,在向他允诺,在向她表示他那比山还坚定,比海还深邃的爱情。她已知道他这是在对她说;''我一定不会辜负你,小鱼儿一定不会死,你放心吧。"但铁心兰的心都已碎了。她所要求的,现在固然已得到,但这难道真是她所要求的吗?她难道真希望花无缺死。她望著花无缺,眼泪在流下面颊。

``我也一定不会辜负你的,你放心吧!''她悄悄的后退,退了出去,因为她无论如何也不忍心眼见花无缺为她而死,死在她面前。因为花无缺不但是她的情人,她的夫婿,也是她的朋友,她的兄弟,她的灵魂,她的生命\ldots\ldots{}

\hypertarget{ux7b2cux4e00ux767eux5effux4e03ux7ae0-ux751fux79bbux6b7bux522b}{%
\chapter{第一百廿七章
生离死别}\label{ux7b2cux4e00ux767eux5effux4e03ux7ae0-ux751fux79bbux6b7bux522b}}

白云飘渺。

苏樱倒在树下,痴痴的望著这飘渺的白云,眼泪早已流尽了。因为她的生命和灵魂她的情人和夫婿,此刻正在这飘渺的白云间,在和别人作生死的决斗。她却连这次决斗的结果都不知道,小鱼儿现在究竟是胜?是负?是生?还是死?\ldots\ldots 苏樱揉了揉眼睛,告诉自己;``我为什么还要关心他?他和我还有什么关系?''她想站起来,振作自己,怎奈她不但心已碎了,整个人郡似全都碎了,那里还能站起来。

忽然间,树后有一阵悲惨的哭声传了过来,彷佛有个人已扑倒在这棵树的另一边。这棵树三人合抱,所以她并没有发现树后的苏樱。

苏樱却已听出她就是铁心兰。心中忖道:``铁心兰为何到这里来?为何如此伤心?难道那一场决战已结束,难道小鱼儿和花无缺之间已有个人死了?可是,死的是谁呢?''苏樱挣扎著爬起,绕了过去。

铁心兰猝然一惊,失声道:``你也在这里?''

苏樱紧紧拉著她的手臂,道:``他\ldots\ldots 他已死了?''铁心兰黯然点了点头,又痛哭起来。苏樱只觉头脑一阵晕眩,整个人都似已崩溃。她的人还末倒在地上,也失声痛哭了起来。

两人对面坐在树下,对面痛哭,也不知哭了多久,铁心兰忽然问道:``小鱼儿没有死,你哭什么?''苏樱怔了怔,抽泣著道;``小鱼儿没有死?死的难道是花无缺?''铁心兰道:``嗯。''苏樱又惊又喜,但忽然大声道:``我不信,小鱼儿是绝不会杀花无缺的。''铁心兰道:``不是他杀死了花无缺,而是花无缺杀死了自己。''苏樱道;``他杀死了自己?为什么了,''铁心兰嘴唇都已咬得出血,头声道;``因为\ldots\ldots 因为我求他莫要杀小鱼儿,他答应了我,自己只有死\ldots\ldots{}''苏樱吃惊的张大了眼睛,望著她,就好像从来没有见过她这个人似的,过了很久,才一字字道:``你明知花无缺只有一死,还要求他莫要杀死小鱼儿?''铁心兰全身似已痉孪,痛苦的咬紧了牙。

苏樱道;``花无缺明知如此,还是答应了你?''铁心兰痛苦的目光中露出一丝温柔之色,道:``他本就是世上最伟大的人。''苏樱道:``但你为了小鱼儿,而不惜要这最伟大的人死?想不到你对小鱼儿的情感竟如此深厚\ldots\ldots{}''铁心兰忽然大声道;``但我真心爱著的并不是小鱼儿。''苏樱道:``不是小鱼儿,难道是花无缺?''

铁心所流泪道;``不错,我\ldots\ldots 我爱的是他,全心全意的爱他,你永远不知道我现在爱他有多深,没有人知道我爱他有多深。''苏樱道;``但你却要他死!''

铁心兰抱面痛哭道:``不错,因为我已决心要陪著他一齐死。''苏樱望著铁心兰,像是也怔住了,过了半晌,才长长叹了口气:``你这是为了什么呢?''铁心陌痛哭著道:``因为我爱上了花无缺,花无缺也爱上了我,我觉得我们都对不起小鱼儿,所以我们只有死\ldots\ldots 只有以死才能报答他?''苏樱长叹道:``我还是不懂,虽然我也是女人,却还是不懂你的心意,难怪男人都说女人的心比海底的针更难捉摸了\ldots\ldots{}''突见铁心兰身子一阵抽搐全身似将缩成一团。

苏樱失声道:``你怎么样了?''

铁心兰累闭眼睛,满面俱是痛苦之色,但嘴角却露出了一丝微笑,这微笑看来竟充满了愉快和幸福之意。她一字字道:``现在他已死了,我也要死了我们立刻就要相聚,世上所有丑恶残酷,痛苦的事,再也不能伤害到我们。''苏樱拉著他的手,道;``胡说,你不会死的。''铁心兰凄然笑道;``我已服下世上最毒的毒药,已是非死不可的了。''现在,小鱼儿和花无缺已斗到七百招。两人的武功都宛如长江大河之水,滚滚而来,永无尽时,奇招妙著,更是层出不穷,简直令人目不瑕接,不可思议!但这一战却已显然到了尾声。这并不是说两人内力已竭,而是两人都已不愿再打下去了。他们正如一对孔雀,已开过美丽的屏花。现在,他们已是死而无憾!

萧女史不住摇著头叹息道:``可惜呀,可惜!这两个孩子都是百年难遇的武林奇才,无论谁死了都可惜得很。''弥十八也不禁叹息著点了点头,道:``这就叫造化弄人\ldots\ldots 造化弄人\ldots\ldots{}''别人的心情又何尝不和他们一样,就连燕南天都不禁对花无缺起了怜惜之意,他固然希望小鱼儿能战胜,却也不愿眼见花无缺这样的少年惨遭横死。却不知这两人根本就没有谁能活下去。

只有怜星宫主知道这秘密,她苍白而美丽的面容上,也不禁露出了激动之色,在心里喃喃自语:``我怎能让这两人死?花无缺是我从小带大的孩子,小鱼儿不但救过我的命,而且也保全了我的颜面,我怎么能眼看这两人死在我面前!''她忽然冲了出去。在这一刹那间,她已将二十年前的仇恨全都忘得乾乾净净,只觉心里热血澎湃,不能自已。

她忍不住大声道:``住手,我有话说。''只可惜她的声音已嘶哑,而大家又全都被眼前这一场惊心动魄的大战所吸引,并没有留意到他在说什么。

而邀月宫主却留意到她了。她一句话方出,邀月宫主已掠到她身边,出手如电,拉住了她的手臂,扣住了她的穴道,厉声道:``你有什么话说?''怜星宫主流下泪来,道:``大姊,二十年前的事,已过去很久了,江枫他们虽然对不住你,可是\ldots\ldots 可是他们如今连尸骨都已化为飞灰,大姊,你\ldots\ldots 何必再恨他们呢?''``你难道想饶了他们?''邀月宫主的脸色又白得透明了,道;``你难道想要在此时说出他们的秘密?''怜星宫主道:``我只是想\ldots\ldots{}''她忽然发现邀月宫主的脸色,忍不住机伶伶打了个寒噤。邀月宫主一字字道;``从你七岁的时候,就喜欢跟我捣蛋,无论我喜欢什么,你都要和我争一争,无论我想做什么,你都要想法子破坏!''她的脸色越来越透明,看来就宛如被寒雾笼罩著的白冰。

怜星宫主脸色也变了,颤声道;``你\ldots\ldots 你莫忘了,我毕竟是你的妹妹。''她身形急转,想藉势先甩开邀月宫主的手,但这时已有一阵可怕的寒意自邀月宫主的掌心传了出来,直透入她心底。

怜星宫主骇然道:``你疯了,你想干什么?''

邀月宫主一字字缓缓道:``我并没有疯,只不过,我等了二十年才等到今天,我绝不会再让任何人来破坏它,你也不能\ldots\ldots{}''她每说一字,怜星宫主身上的寒意就加重了一分,等她说完了这句话,怜星宫主全身都已几乎僵硬。她只觉自己就好像赤身被浸入一湖寒水里,而四周的水正在渐渐结成冰,她想挣扎,却已完全没有力气。邀月宫主根本没有看她,只是凝注著小鱼儿和花无缺,嘴角渐渐露出一丝奇异的微笑,缓缓道;``你看,这一战已快结束了,江枫和月奴若知道他们的双生子正在自相残杀,一定会后悔昔日为何要做出那种事的。''怜星宫主嘴唇颤抖著,忽然用尽全身力气,大呼道;``你们莫要再打了,听见了吗?因为你们本是亲生的兄弟!''邀月宫主冷笑著;"并没有阻止她,因为她虽然用尽了力气在呼喊,但别人却只能听到她牙齿打战的声音,根本听不出她在说什么?怜星宫主目中不觉流出了眼泪来,数十年以来,这也许是她第一次流泪,但她流出来的眼泪,也瞬即就凝结成冰。

她知道小鱼儿和花无缺的命运现在才是真的没有谁能改变了,因为现在世上知道这秘密的人已只剩下邀月宫主。而邀月宫主却是永远不会说出这秘密的,除非等到小鱼儿或花无缺倒下去,那时所有的事便已到了结局。这一段错综复杂,纠缠入骨的恩怨,也唯有到那时才会终止。这结局实在太悲惨,怜星宫主已不愿再看下去。事实上,她也已无法看下去。

铁心兰倒在苏樱怀中,喘息著,挣扎著道;``我\ldots\ldots 我们总算是姊妹,现在我想求你一件事,不知道你答不答应!''苏樱温柔的抚摸著她的头发,柔声道:``无论你要我做什么,只管说吧。''铁心兰道;``我死了之后,希望你能将我和花无缺埋葬到一齐,也希望你告诉小鱼儿,我虽然不能嫁给他,但我始终是他的姊姊,他的朋友。''苏樱揉了揉眼睛,道:``我\ldots\ldots 我答应你。''

铁心兰凝注著她,缓缓又道:``我也希望你好好照顾小鱼儿,他虽然是匹野马,但有你在他身旁,他也许会变得好一些的。''苏樱幽幽叹息了一声,道:``他会么?''

铁心兰道:``嗯,因为我很了解他,我知道他真心喜欢的,只有你一个人,至于我\ldots\ldots 他从没有喜欢过我,只不过因为他很好强很好胜\ldots\ldots。''苏樱头声道:``我知道,我全知道,求你莫要再说,无论你要我做什么,我都答应你。''铁心兰嫣然一笑,缓缓阖起了眼睑。她笑得是那么平静,因为他已不再有烦恼,不再有心事。苏樱望著她,却已不禁泪落如雨\ldots\ldots 花无缺的手已渐渐慢了下来。他知道时候已到了,已没有再拖下去的必要。

无论任何事,迟早都有结束的时侯,到了这时侯,他的心情反而特别平静。嫉妒.爱憎.好胜.炫耀\ldots\ldots 这些世俗的情感,忽然之间都已升华,这种情感的升华正是人类至高无上的情操。

他只希望小鱼儿能好好的活著,铁心兰能好好的活著,所有他的朋友和仇敌都好好活著,而且活得愉快。他当心著小鱼儿的出手,等待著机会。

等待著机会死!他准备让小鱼儿``胜''得光光采采,既不希望被任何人看出他是自己送死的,更不希望被小鱼儿自己知道。所以他既不能故意露出破绽,更不能自己撞到小鱼儿掌下去,他要等待小鱼儿施展出一著很奇妙的招式时,再故意``闪避不开''!只见小鱼儿身形施转,左掌斜斜劈下,右掌却隐在身后。花无缺知道他这左掌本是虚招,随在身后的那只右掌才是真正杀手,对方招架他左掌时,他身子已转过,右掌就会忽然自肋下穿出。这一招虚虚实实,连消带打,而且出手的部位奇秘诡异,本可算得上是江湖罕见的绝招杀手。

但小鱼儿却似已打晕了头,竟忘了这一招他方才已使出过一次,花无缺方才避开他这一招时虽曾遇险,可是现在却已对这一招了如指掌。

这正是花无缺的``机会''到了。他手掌自下面反切上去,直切小鱼儿协下,只因他知道等他这一掌切到时,小鱼儿身子已转过,他这一掌就落空,那时他``招式已用老'',等小鱼儿右掌穿出时,他便要立毙在小鱼儿掌下。所以他这一招看来虽也是连消带打的妙著,其实却是送死的招式。

谁知小鱼儿这一次身形转得竟比上次慢了好几倍,等花无缺一掌切到他肋下时,他身子竟还没有转过去,肋下软骨,本是人身要害之一。花无缺本已成竹在胸,故意将这一掌招式用得很老,所以等他发现不妙时,再想收招变式已来不及了。

只听``砰''的一声,小鱼儿已被他打得飞了出去。

四下惊呼声中,燕南天一掠七丈,如大鹏般飞掠了过来。轩辕三光等人也惊呼著赶到小鱼儿面前。只见小鱼儿面如金纸,气若游丝,已是奄奄一息,再一探他的脉搏,亦是若断若级,眼见生机便已将断绝。无论谁都可以看出他是万万活不成的了。

燕南天已不觉急出了满面痛泪,跺脚道;``你\ldots\ldots 你明明可以避开那一招的,你\ldots\ldots 你\ldots\ldots 你\ldots\ldots{}''小鱼儿凄然一笑,挣扎著道:``我本想用这一招故意诱他上当的,谁知\ldots\ldots 谁知他,\ldots\ldots{}''他急剧的咳嗽著,嘴角已泌出了血丝,喘息著又道;``这只因我\ldots\ldots 我太聪明了,反而弄巧成\ldots\ldots 弄巧成拙\ldots\ldots{}''他将``弄巧成拙''这句话一连说了两次,声音越来越微弱,眼睑渐渐阖起,喘息渐渐平静他似乎还想张开眼来,对他所留恋的这世界再瞧最后一眼,但无论他多么努力都已没有用了。他的眼睛再也张不开来。

花无缺木立在那里,心神已完全混乱,眼前却变成了一片空白,什么都不能思想,什么都已看不到。

小鱼儿竟死了!小鱼儿竟被他杀死了!他只希望这件事不是真的,而是一场梦,噩梦!他的眼泪都似已枯竭。

燕南天忽然怒喝一声,反身一掌向花无缺劈下,花无缺却站著动也没有动。

邀月宫主正在检查小鱼儿的脉搏,此刻忽然一掠数丈,将花无缺拉出了燕南天的掌风中。

邀月宫主悠然道:``方才我拉开了无缺,其实却是救了你!只因世上谁都可以杀他,只有你是万万杀不得他的!''燕南天道:``为什么?''

邀月宫主目中闪动著一丝残酷的笑意,道:``你可知道他是谁么?''燕南天忍不住问道:``他是谁?''

邀月宫主忽然疯狂般大笑起来,指著花无缺道:``告诉你,他也是江枫的儿子,他本是小鱼儿的孪生兄弟。''这句话说出,四下立刻骚动起来。燕南天却怔住了,怔了半晌,才怒喝道:``放屁!''邀月宫主大笑著道;``我等了二十年,就是在等今天,等他们兄弟自相残杀而死,我等了二十年,直到今天才能将这秘密说出来,我实在高兴极了,痛快极了?''燕南天狂吼道:``无论你怎么说,我连一个字都不相信?''邀月宫主格格笑道:``我知道你会相信的,一定会相信的,你仔细一想,就会发觉他们两人有多么相似,你再看看他们的眼睛,他们的鼻子\ldots\ldots{}''燕南天双拳紧握,已不觉汗出如浆。

邀月宫主笑著道:``你可知道我为什么要逼他们两人动手?你可知道我为什么一定要花无缺亲手杀死小鱼儿?\ldots\ldots 你们本来一定想不通这道理,是吗?现在你们虽已明白,却已太迟了,太迟了\ldots\ldots{}''这秘密实在太惊人,宛如睛空中忽然劈下的霹雳,震得所有的人全都呆住了,心里虽然激动,却反而连丝毫声音都发不出来。天地间彷佛只剩下了遨月宫主疯狂的笑声。

大家想到花无缺和小鱼儿以前的种种情况,纵然想不信邀月宫主的话,也是万万无法不信了。大家心里也不知是惊讶,是愤怒,还是同情\ldots\ldots 也许这许多情感都有一些,但毕竟还是怜悯和同情多些。

只见花无缺脸色发白,望著地上小鱼儿的尸体,身子渐潮开始发抖,越抖越厉害,到后来抖得连站都站不住了,全身缩成一团。

燕南天望著这一生一死兄弟两人,岩石般的身形竟似也要开始崩溃,在这一刹那间,他才真正变成了个老人。他心里充满了悲哀和痛悔。

``我为什么也要逼著他们两人动手?为什么不阻止他们?''他知道这一切都是为了仇恨!他现在也已知道仇恨并不能为任何人带来光荣,仇恨带来的只有痛苦,只有毁灭!但现在他才知道已太迟了!他甚至已悲痛得连愤怒的力量都失去,非但没有向邀月宫主挑战,甚至连看都没有再看她一眼。

邀月宫主却在看著他们。她目光中的笑意看来是那么残酷,那么恶毒,瞪著花无缺冷冷道;``你自己杀死了你自己的兄弟,你还有什么话说?''花无缺以手掩面,全身都缩到地上。

邀月宫主狞笑著道;``你莫忘了,你身上还有一柄''碧血照丹心``,你现在总该相信这是柄魔剑了吧,无论谁得到它,都只有死!''花无缺霍然抬起头,``碧血照丹心''已在他手上!碧绿色的短剑,在夕阳下散发著妖异的光芒。虽然每个人都知道他要做什么,但却没有任何人能阻止他,无论谁落到这种地步,也都只有死,非死不可!邀月宫主一字字道:``现在你的时侯已到了,你还等什么!''花无缺反手一剑,向自己胸膛刺下!忽然间,一只手伸过来,夺去了花无缺掌中的剑!自花无缺手上夺剑,本不是件容易事,但现在,花无缺已几乎完全崩溃,他抬起头,瞪了这人很久,才顶声道:``你是谁?为什么不让我死!''

\hypertarget{ux7b2cux4e00ux767eux5effux516bux7ae0-ux771fux76f8ux5927ux767d}{%
\chapter{第一百廿八章
真相大白}\label{ux7b2cux4e00ux767eux5effux516bux7ae0-ux771fux76f8ux5927ux767d}}

夺剑的人竟万春流。他叹息了一声,缓缓道:``一个人若是要死,那是谁也拦不住的。''邀月宫主厉声道;``你既然知道,为什么还要来多事!''万春流根本不理她,还是凝注著花无缺,柔声道:``我并不是阻止你,只不过劝你再多等片刻,也许还不到半个时辰,过了半个时辰后,你若还是要死,我保证绝没有任何人来阻止你。''他望著手里的剑,接著又道:``到了那时,无论任何人想死,我非但绝不阻止,而你还会将这柄剑亲自交到她手上。''邀月宫主大笑道:``半个时辰?这半个时辰难道还会有鬼么?孩子,我劝你还是莫要再等了吧,多等一刻,你就多受一刻的痛苦?''``狂狮''铁战忽然大喝道:``就算再多受片刻痛苦又有何妨?你难道连这点勇气都没有?''邀月宫主怒道:``你是什么人?竟敢在我面前多嘴?''铁战大怒道;``我多了嘴又怎样?''

他的喝声更大,邀月宫主脸色又开始透明,一步步向他走了过来,道:``谁多嘴,我就要他死!''萧女史忽也冷冷一笑,站到铁战身旁,道:``我平生什么都不喜欢,就喜欢多嘴?''弥十八叹了口气,道:``我的脾气也正和她一样!''俞子牙道:``还有我!''

刹那之间,这些久已隐迹世外的武林高人,都已站在一排,静静的凝注著邀月宫主,每双眼睛都是清澈如水,明亮如星。

邀月宫主骤然停下脚步,望著各人的眼睛,她只有停下脚步,过了了半晌,才淡淡一笑,道:``我既已等了二十年,又何在乎多等这一时半刻?''除了万春流之外,谁也不知道在这短短半个时辰中,事情会有什么变化?但万春流却似胸有成竹,竟盘膝坐到花无缺身旁,闭目养起神来,燕南天呆了很久,缓缓俯下身,抱起了小鱼儿的尸体。

但万春流却忽然大声道;``放下他!''

燕南天怔了怔,道:``放下他?为什么?''

万春流道:``你现在不必间,反正马上就会知道的。''燕南天默然半晌,刚将小鱼儿的尸体放回地上,突然又似吃了一惊,再拉起小鱼儿的手。只见他面色由青转白,由白转红,忽然放声大呼道:``小鱼儿没有死,没有死\ldots\ldots{}''邀月宫主也一惊,但瞬即冷笑道;``我知道他已死了,我已亲自检查过,你骗我又有什么用?''燕南天大笑道:``我为何要骗你?他方才就算死了,现在也已复活!''这句话说出来,骚动又起,大家心里虽都在希望小鱼儿复活,但却并没有几个人相信燕南天的话。邀月宫主更忍不住大笑起来,指著燕南天道:``这人已疯了,死人又怎会复活?''燕南天仰苜而笑,也不去反驳她的话,大家见到他的神情,心里也不禁泛起一阵悲痛怜惜之心。这一代名侠只怕真的已急疯了。死人又怎会复活?但就在这时,突然一人道;``谁说死人不能复活?我岂非已复活了么?''骤然间,谁也不知道这句话究竟是否小鱼儿自己说出来的,但小鱼儿的``尸体''却已自地上坐了起来!死人竟真的复活了!大家几乎不相信自己的眼睛,怔了半晌,又忍不住欢呼起来,有的人心里已恍然大悟!原来小鱼儿方才只是在装死!但邀月宫主却知道他方才是真的死了,因为她已检查过他的脉搏,知道他呼吸已停,脉搏已断,连心跳都已停止。他怎会复活的?难道真的见了鬼么?邀月宫主瞪著小鱼儿,一步步向后退,面上充满了恐惧之色。

小鱼儿望著她嘻嘻一笑,道;``你怕什么?我活著时你尚且不怕,死了后反而害怕了么?''邀月宫主颤声道:``你\ldots\ldots 你究竟在玩什么花样?''小鱼儿大笑道:``小鱼儿玩的花样你若也猜得到,你就是天下第一聪明人了。''他转向万春流,道:``她什么都说了?''万春流拉起了花无缺,微笑道;``她什么都说过了,这秘密其实只需一句话就可说明!你们本是亲兄弟,而且是孪生的兄弟!''小鱼儿欢呼一声,跳起来抱住了花无缺,大笑道:``我早知道我们绝不会是天生的对头,我们天生就应该是朋友,是兄弟?''他虽然笑著,但眼泪却也不禁流了出来。

花无缺更是已泪流满面,那里还能说得出话,燕南天张开巨臂,将这兄弟两人紧紧拥抱在一齐,仰天道:``二弟,二弟,你\ldots\ldots 你\ldots\ldots{}''他语声哽咽,也唯有流泪而已。

但这却是悲喜的眼泪,大家望著他们三人,一时之间,心里也不知是悲是喜?热泪也不禁夺眶而出。慕容双情不自禁依偎到南宫柳怀里,心里虽是悲喜交集,却又充满了柔情蜜意,再看她的姊妹,亦是成双成对,互相偎依。

萧女史擦著眼睛,忽然道:``无论你们怎样,我却再也不想回去了,这世界毕竟还是可爱的。''邀月宫主木立在那里,根本就没有一个人睬她,没有人看她一眼,她像是已完全被这世界遗弃。

只有万春流却缓缓走到她面前,缓缓道:``水能载舟,亦能覆舟,毒药能害人,亦能救人,其中的巧妙虽各有变化,运用却存乎一心。''他微微一笑,接著道:``若将几种毒草配拣到一齐,就可炼出一种极厉害的麻痹药,刹那间就可令人全身麻痹,呼吸停止,和死人无异,若用这种麻药来害人,自然就可乘人在麻痹时为所欲为,但在下配炼这种麻药,却是为了救人,因为它不但可以止痛,还可要人上当?''说到这里,邀月宫生面上的肌肉已开始抽搐。但万春流还是接著说了下去,道:``小鱼儿还末动手之前,就问我要了这些麻药,他从小和我在一齐,深知这种麻药的用法,所以就想到用它来装死,因为他也知道他一死之后,你一定会将所有的秘密说出来。''他又笑了笑,道:``这孩子实在聪明,所想出的诡计无一不是匪夷所思,令人难测,也就难怪连宫主都会上了他的当了。''他双手将那柄``碧血照丹心''捧到邀月宫主面前,悠然道:``花无缺既已用不著这柄剑了,在下只有将之交回给宫主,宫主说不定会用得著它,是么?''他微笑著转身,再也不回头去瞧一眼。邀月宫主这时只要一挥手,就可将他立毙于剑下!但万春流却知道以邀月宫主此刻的心情,是必定再也不会杀人的了,也许她唯一要杀的人就是她自己!``碧血照丹心''也许的确是柄不祥的魔剑!苏樱早已来了,她来的时候,正是小鱼儿``复活''的时候,但直到这时她才擦乾眼泪,走了过去。小鱼儿忽然发现了她,又惊又喜,道:``你也来了,我知道你一定会回来的。''苏樱面上冷冰冰的毫无表情,道;``我这次来,只因为我已答应过别人,到这里来办一件事。''小鱼儿道:``你答应了谁?来办什么事。''

苏樱道:``我答应了铁心兰,到这里来\ldots\ldots{}''她话末说完,铁战.花无缺已同时失声道;``她的人呢?''苏樱望著花无缺,道:``她只想让你知道,她虽要你为她而死,可是她自己也早就准备陪著你死了,她还是要我将你们两人的尸体葬在一起。''花无缺流泪道;``我\ldots\ldots 我知道她绝不会负我的,我早已知道。她\ldots\ldots 她的人现在那里?''苏樱道:``她早已服下了毒药,准备一死\ldots\ldots{}''铁战狂吼一声,扼住了花无缺的喉咙,大吼道;``都是你这小子害了她,我要你赔命!''花无缺的人早已呆了,既不挣扎,也不反抗,只是喃喃道:``不错,是我害了她\ldots\ldots 是我害了她\ldots\ldots{}''大家本来为他们兄弟高兴,此刻见了花无缺的模样,心情又不禁沉重了起来,总觉得苍天实在不公,为什么总是对多情的人如此残忍。谁知这时小鱼儿却忽然大笑起来。

铁战大怒道:``你这畜生!你笑什么?''

小鱼儿笑道:``莫说铁心兰只不过服下了一点毒药,就算她将世上的毒药全都吞下去,苏姑娘也有法子能将她救治的,苏姑娘,你说对不对。''苏樱狠狠瞪了他一眼,但还是点了点头,向花无缺展颜笑道:``我本来也想让你著急的,可是见了你这副样子,我可不忍了\ldots\ldots 你快去吧,她就在那边的树下,现在只怕已快醒来了。''花无缺大喜道:``多谢\ldots\ldots{}''他甚至等不及将这多谢两个字说完,人已飞掠了出去。

铁战也想跟他一齐走,但萧女史却拉住了他,笑道:``那边的地方很小,你过去就嫌太挤了。''铁战怔了怔,但毕竟还是会过意来,大笑道;``不错不错,太挤了,的确太挤了\ldots\ldots{}''小鱼儿笑嘻嘻的刚想去拉苏樱的手,但苏樱一见到他,脸立刻沉了下去,一甩手扭头就走。

这时邀月宫主竟忽然狂笑起来,狂笑著抱起她妹妹的尸体,狂笑著冲了出去,瞬眼间就消失在苍茫的迷雾中。

但这时小鱼儿谁也顾不得了,大步赶上了苏樱,笑道;``你还在生我的气?''苏樱头也不回,根本不理他。

小鱼儿道:``就算我错怪了你,你也用不著如此生气呀。''苏樱还是不理他。

小鱼儿道:``我已经向你赔不是了,你难道还不消气。''苏樱好像根本没听见他在说什么。

小鱼儿叹了口气,喃喃道:``我本来想求她嫁给我的,她既然如此生气,看来我不说也罢,也免得去碰个大钉子。''苏樱霍然回过头,道:``你\ldots\ldots 你说什么?''

小鱼儿眨了眨眼睛,摊开双手笑道:``我说了什么?我什么也没有说呀。''苏樱忽然扑上去,搂住了他的脖子,咬著他的耳朵,打著他的肩头,跺著脚娇笑道:``你说了,我听见你说了,你要我嫁给你,你还想赖吗。''小鱼儿耳朵被咬疼了,但此刻他全身都充满了幸福之意,这一点疼痛又算得了什么?他一把将苏樱抱了起来,大步就走。

苏樱娇呼道:``你\ldots\ldots 你想干什么呀?''

小鱼儿悄悄道:``这里的人太挤了,我要找个没人的地方去跟你算帐!''苏樱飞红了脸,道;``你\ldots\ldots 你方才说的话,赖不赖?''小鱼儿笑道:``男子汉大丈夫,说出的话还能赖吗?''苏樱``喽咛''一声,累紧勾住了他脖子,在他耳边悄悄道:``不错,这里人实在太多了,你快带我走吧,从今以后,无论你要走到那里去,我都跟著。''慕容双依偎在南宫柳怀里,脸上也是红红的,红著脸笑道:``难道不觉得人太挤了么?''南宫柳温柔的望著她,悄悄道:``你也想回家?''慕容双垂下了头,悄笑道:``何必回家,只要是没有人的地方\ldots\ldots{}''突听慕容珊珊娇笑道:``好呀,老夫老妻的,还在这里肉麻当有趣,也不怕害臊么?''慕容双红著脸,跺脚道:``鬼丫头,谁叫你来听我们悄悄话的''慕容珊珊笑道:``我不管你怎么著急,今天也绝不放你们回去大家全都要留在这里,等著和燕大侠一齐喝杯酒。''慕容双道:``但这里那来的酒?''

慕容珊珊笑骂道;``我看你真是晕了头了,难道没见到轩辕三光方才已拉著铁大侠去买酒了么!''燕大侠大笑道:``不错,今天务请大家都留在这里喝一杯,就算是喝江小鱼和江无缺的喜酒吧!''他将``江无缺''三个字说得特别有力,好像在向大家特别声明,``花无缺''从此之后就是``江无缺''了!萧女史一直在呆呆的出著神,此刻才幽幽的叹息了一声,道;``看到了这些年轻人,我才真有些后悔了。''弥十八道;``后悔什么?''萧女史道:``后悔我以前为什么总是三心二意的,左也不嫁,右也不嫁,否则我现在也不会像这么样孤孤单单的了。''弥十八道:``可是你现在再打定主意找个人也不迟呀。''萧女史叹了气,道:``现在?现在还有谁会要我这老太婆?''弥十八指著自己的鼻子笑道;``你莫忘了,我到现在也还是孤孤单单的光棍一个。''萧女史的睑骡然飞红了起来,像是忽然年轻了几十岁,``拍''的轻轻打了弥十八个耳括子,笑骂道:``瞧你老得牙都快掉了,还敢来打我的主意么?''弥十八嘻嘻笑道:``这就叫老配老,少配少,王八配乌龟,跳蚤配臭虫\ldots\ldots{}''萧女史又是一个耳括子要打过去了,幸好这时铁战和轩辕三光已回来,弥十八赶紧迎了上去:道``你们实的酒呢?''轩辕三光苦著脸道;``格老子,我的钱早已输光了,没想到这老疯子跟我一样,也是个穷光蛋,袋子里连一文钱都没有。''欢乐的时候没有酒,就好像□里没有放盐一样己大家正觉得有些失望,忽然发现黑压压的一群``吱吱喳喳''的爬上山来,仔细一看,却原来是一群猴子。这群猴子有大有小,吵得翻了天,手里却都捧著样东西,竟是些瓶瓶罐罐,破坛子,破茶壶。大家又奇怪,又好笑,正不知这些猴子是为什么来的,鼻子里却已闻到一阵浓烈的酒香。

弥十八赶上去一看,这些瓶瓶罐罐里竟装满了美酒。他忍不住大笑道:``人没将酒买回来,却将酒送来了,看来猴子比我们这些人还强得多。''轩辕三光叹了口气,苦笑著喃喃道:``猴子有时的确比人还聪明些,至少它们不会去赌钱\ldots\ldots{}''这时小鱼儿正在远处的一个山洞里吃吃的笑著,道:``我打赌,他们就算想一万年,也绝对想不出酒是从那里来,是什么酒?''苏樱像条猫似的倦伏在小鱼儿怀里,媚眼如丝,似乎根本懒得说话,只是懒洋洋的问著道:

``那究竟是什么酒?''

小鱼儿道:``那就叫猴儿酒,就是猴子自己酿出来的。''苏樱道:``猴子也会酿酒?''.小鱼儿笑道:``猴子酿的酒,有时比人还好得多,无论酒量多好的人,若是喝多了猴儿酒,至少也得醉三天。''苏樱道:``可是,你究竟是用什么法子要那些猢狲将酒送去的呢?这连我都不懂了。''小鱼儿眨著眼笑道:``江小鱼的妙计,你自然是永远弄不懂的,你若也和我一样聪明,我就不会娶你做老婆了。''苏樱忍不住咬了他一口,嫣然笑道:``小鱼儿呀小鱼儿,你真是个坏东西。''小鱼儿忽然板起脸,道;``我已经是你老公,马上就要做你儿子的爸爸,你怎么还能叫我''小鱼儿``?苏樱娇笑著道:''小鱼儿呀小鱼儿呀,你就算活到八十岁,做了爹爹,人家还是要叫你小鱼儿,因为``小鱼儿''这三个字实在太有名了。"------

(全书完)

\backmatter
\end{document}
