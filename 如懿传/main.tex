\documentclass[12pt,oneside]{book}

\usepackage{mybook}
\usepackage{mybookcover}


\title{如懿传}
\author{流潋紫}

\begin{document}
\bookcover{book_cover.png}

\frontmatter


\addchtoc{目录}
\setcounter{tocdepth}{2}    
\tableofcontents

\mainmatter


\hypertarget{ux5982ux61ffux4f20-ux7b2cux4e00ux518c}{%
\part{如懿传 第一册}\label{ux5982ux61ffux4f20-ux7b2cux4e00ux518c}}

\hypertarget{ux7b2cux4e00ux7ae0-ux7075ux524d}{%
\chapter{第一章 灵前}\label{ux7b2cux4e00ux7ae0-ux7075ux524d}}

云板声连叩不断,哀声四起,仿若云雷闷闷盘旋在头顶,叫人窒闷而敬畏。

国有大丧,天下知。

青樱俯身于众人之间,叩首,起身,俯身,叩首,眼中的泪麻木地流着,仿若永不干涸的泉水,却没有一滴,是真真正正发自内心的悲恸。

对于金棺中这个人,他是生是死,实在引不起青樱过多的悲喜。他,不过是自己夫君的父亲,王朝的先帝,甚至,遗弃了自己表姑母的男人。

想到这里,青樱不觉打了个寒噤,又隐隐有些欢喜。一朝王府成潜龙府邸,自己的夫君君临天下,皆是拜这个男人之死所赐。这样的念头一转,青樱悄然抬眸望向别的妻妾格格①------不,如今都是妃嫔了,只是名分未定而已。

青樱一凛,复又低眉顺眼按着位序跪在福晋身后,身后是与她平起平坐的高晞月,一样的浑身缟素,一样的梨花带雨,不胜哀戚。

忽然,前头微微有些骚动起来,有侍女低声惊呼起来:``主子娘娘晕过去了!''

青樱跪在前头,立时膝行上前,跟着扶住晕过去的富察氏。高晞月也跟着上来,惶急道:``主子娘娘跪了一夜,怕是累着了。快去通报皇上和太后。''

这个时候,太后和皇上都已疲乏,早在别宫安置了。青樱看了晞月一眼,朗声向众人道:``主子娘娘伤心过度,快扶去偏殿休息。素心,你是伺候主子娘娘的人,你去通报一声,说这边有咱们伺候就是了,不必请皇上和太后两宫再漏夜赶来。''

晞月横了青樱一眼,不欲多言。青樱亦懒得和她争辩,先扶住了富察氏,等着眼明手快的小太监抬了软轿来,一齐拥着富察氏进了偏殿。

晞月意欲跟进伺候,青樱身姿一晃,侧身拦住,轻声道:``这里不能没有人主持,太后和太妃们都去歇息了,主子娘娘和我进去,姐姐就是位分最高的侧福晋②。''

晞月眼眸如波,朝着青樱浅浅一漾,温柔的眼眸中闪过一丝不驯,她柔声细语:``妹妹与我都是侧福晋,我怎敢不随侍在主子娘娘身边?''她顿一顿,``而且,主子娘娘醒来,未必喜欢看见妹妹。''

青樱笑而不语,望着她淡然道:``姐姐自然是明白的。''

晞月微微咬一咬唇:``我希望自己永远都能明白。''

她退后两步,复又跪下,朝着先帝的金棺哀哀痛哭,仿似清雨梨花,低下柔枝,无限凄婉。

青樱在转入帘幕之前望了她一眼,亦不觉叹然,怎么会有这样的女人?轻柔得如同一团薄雾轻云,连伤心亦是,美到让人不忍移目。

青樱转到偏殿中,素心和莲心已经将富察氏扶到榻上躺着,一边一个替富察氏擦着脸扑着扇子。青樱连忙吩咐了随侍的太监,叮嘱道:``立刻打了热水来,虽在九月里,别让主子娘娘擦脸着了凉。莲心,你伺候主子娘娘用些温水,仔细别烫着了。''说罢又吩咐自己的侍女,``惢心,你去开了窗透气,那么多人闷着,只怕娘娘更难受。太医已经去请了吧?''

惢心连忙答应:``是。已经打发人悄悄去请了。''

素心闻言,不觉双眉微挑,问道:``主子娘娘身子不适,怎么请个太医还要鬼鬼祟祟的?''

青樱含笑转脸:``姑娘不知道,不是鬼鬼祟祟的。而是方才高姐姐的话说坏了。''

素心颇为不解,更是疑心:``说坏了?''

青樱不欲与她多言,便走前几步看着太监们端了热水进来,惢心侧身在素心身边,温和而不失分寸:``方才月福晋说,主子娘娘是累着了才晕倒的\ldots\ldots{}''

素心还欲再问,富察氏已经悠悠醒转,轻嗽着道:``糊涂!''

莲心一脸欢欣,替富察氏抚着心口道:``主子娘娘要不要再喝些水?哭了一夜也该润润喉咙了。''

富察氏慢慢喝了一口水,便是不适也不愿乱了鬓发,顺手一抚,才慢慢坐直身子,叱道:``糊涂!还不请侧福晋坐下。''

青樱闻得富察氏醒转,早已垂首侍立一边,恭声道:``主子娘娘醒了。''

富察氏笑笑:``主子娘娘?这个称呼只有皇后才受得起,皇上还未行册封礼,这个称呼是不是太早了?''

青樱不卑不亢:``主子娘娘明鉴。皇上已在先帝灵前登基,虽未正式册封皇后,可主子娘娘是皇上结发,自然是名正言顺的皇后。如今再称福晋不妥,直呼皇后却也没有旨意,只好折中先唤了主子娘娘。''青樱见富察氏只是不做声,便行了大礼,``主子娘娘万福金安。''

富察氏也不叫起来,只是悠悠叹息了一声:``这样说来,我还叫你侧福晋,却是委屈你了。''

青樱低着头:``侧福晋与格格受封妃嫔,皆由主子娘娘统领六宫裁决封赏。妾身此时的确还是侧福晋,主子娘娘并未委屈妾身。''

富察氏笑了一笑,细细打量着青樱:``青樱,你就这般滴水不漏,一丝错缝儿也没有么?''

青樱越发低头,柔婉道:``妾身没有过错得以保全,全托赖主子娘娘教导顾全。''

富察氏凝神片刻,温和道:``起来吧。''又问,``素心,是月福晋在外头看着吧?''

素心忙道:``是。''

富察氏扫了殿中一眼,叹了口气:``是青福晋安排的吧?果然事事妥帖。''她见素心有些不服,看向青樱道,``你做得甚好,月福晋说我累了\ldots\ldots 唉,我当为后宫命妇表率,怎可在众人面前累晕了?只怕那些爱兴风作浪的小人,要在后头嚼舌根说我托懒不敬先帝呢。来日太后和皇上面前,我怎么担待得起?''

青樱颔首:``妾身明白,主子娘娘是为先帝爷驾崩伤心过度才晕倒的。高姐姐也只是关心情切,才会失言。''

富察氏微微松了口气:``总算你还明白事理。''她目光在青樱身上悠悠一荡,``只是,你处事一定要如此滴水不漏么?''

青樱低声:``妾身伺候主子,不敢不尽心。''

富察氏似赞非赞:``到底是乌拉那拉氏的后人,细密周到。''

青樱隐隐猜到富察氏所指,只觉后背一凉,越发不敢多言。

富察氏望着她,一言不发。青樱只觉得气闷难过,这样沉默相对,比在潜邸③时妻妾间偶尔或明或暗的争斗更难过。

空气如胶凝一般,莲心适时端上一碗参汤:``主子喝点参汤提提神,太医就快来了。''

富察氏接过参汤,拿银匙慢慢搅着,神色稳如泰山:``如今进了宫,好歹也是一家人,你就不去看看景仁宫那位吗?''

青樱道:``先帝驾崩,太后未有懿旨放景仁宫娘娘出宫行丧礼,妾身自然不得相见。''

富察氏微微一笑,搁下参汤:``有缘,自然会相见的。''

青樱越发不能接口。富察氏何曾见过她如此样子,心中微微得意,脸上气色也好看了些。

二人正沉默着,外头击掌声连绵响起,正是皇帝进来前侍从通报的暗号,提醒着宫人们尽早预备着。

果然皇帝先进来了。富察氏气息一弱,低低唤道:``皇上\ldots\ldots{}''

青樱行礼:``皇上万福金安。''

皇帝也不看她,只抬了抬手,随口道:``起来吧。''

青樱起身退到门外,扬一扬脸,殿中的宫女太监也跟了出来。

皇帝快步走到榻边,按住富察氏的手:``琅,叫你受累了。''

富察氏眼中泪光一闪,柔情愈浓:``是臣妾无能,叫皇上担心了。''

皇帝温声道:``你生了永琏与和敬之后身子一直弱,如今既要主持丧仪,又要看顾后宫诸事,是让你劳累了。''

富察氏有些虚弱,低低道:``晞月和青樱两位妹妹,很能帮着臣妾。''

皇帝拍拍她的手背:``那就好。''皇帝指一指身后,``朕听说你不适,就忍不住来了,正好也催促太医过来,给你仔细瞧瞧。''

富察氏道:``多谢皇上关爱。''

青樱在外头侍立,一时也不敢走远,只想着皇帝的样子,方才惊鸿一瞥,此刻倒是清清楚楚印在了脑子里。

因着居丧,皇帝并未剃发去须,两眼也带着血丝,想是没睡好。想到此节,青樱不觉心疼,悄声向惢心道:``皇上累着了,怕是虚火旺,你去炖些银耳莲子羹,每日送去皇上宫里。记着,要悄悄儿的。''

惢心答应着退下。恰巧皇帝带了人出来,青樱复又行礼:``恭送皇上,皇上万安。''

皇帝瞥了随侍一眼,那些人何等聪明,立刻站在原地不动,如泥胎木偶一般。皇帝上前两步,青樱默然跟上。皇帝方悄然道:``朕是不是难看了?''

青樱想笑,却不敢做声,只得咬唇死死忍住。二人对视一眼,青樱道:``皇上保重。''

皇帝正好也说:``青樱,你保重。''

青樱心中一动,不觉痴痴望着皇帝。皇帝回头看一眼,亦是柔情:``朕还要去前头,你别累着自己。''

青樱道了声``是''。见皇帝走远了,御驾的随侍也紧紧跟上,只觉心头骤暖,慢慢微笑出来。

注释:

①格格:格格原为满语的译音,译成汉语就是小姐、姐姐、姑娘之意。在满语中原来是对女性的一般称谓。而在汉语中出现时则大多表示:一是清朝贵胄之家女儿的称谓,二是皇帝和亲王妾室的称谓,地位较低。

②侧福晋:顺治十七年(1660)规定,亲王、亲王世子及郡王妻封福晋,侧室则称侧福晋。亦用以封蒙古贵族妇女。为了强调正室的嫡妻地位,又称嫡妻为嫡福晋。嫡福晋与侧福晋都由礼部册封,有朝廷定制的冠服,见《大清会典》。侧福晋冠服比嫡福晋降一等。每年一次由宗人府汇奏请封,咨送礼部入册。相比较于侧福晋,又有一种庶福晋的称谓。庶福晋地位比较低,相当于婢妾,不入册,也没有冠服。庶福晋只是别人对她们的客气称呼,是没经过朝廷册封的。

③潜邸:一指皇帝即位前的住所。宋欧阳修《代人辞官状》:``属潜邸之署官,首膺表擢,陪学黉之讲道,无所发明。''清龚自珍《为龙泉寺募造藏经楼启》:``又诏以潜邸之雍和宫为奉佛处,以大臣专领之。''二也借指太子尚未即位。郑振铎《插图本中国文学史》第五十章:``成祖在潜邸时,已为文人们的东道主。''

\hypertarget{ux7b2cux4e8cux7ae0-ux81eaux5904}{%
\chapter{第二章 自处}\label{ux7b2cux4e8cux7ae0-ux81eaux5904}}

外头的月光乌蒙蒙的,暗淡得不见任何光华,青樱低低说:``怕是要下雨了呢。''

惢心关切道:``小主站在廊檐下吧,万一掉下雨珠子来,怕凉着了您。''

正巧素心引着太医出来,太医见了青樱,打了个千儿道:``给小主请安。''

青樱点点头:``起来吧。主子娘娘凤体无恙吧?''

太医忙道:``主子娘娘万安,只是操持丧仪连日辛劳,又兼伤心过度,才会如此。只须养几日,就能好了。''

青樱客气道:``有劳太医了。''

素心道:``太医快请吧,娘娘还等着你的方子和药呢。''

太医诺诺答应了,素心转过脸来,朝着青樱一笑,话也客气了许多:``回小主的话,主子娘娘要在里头歇息了,怕今夜不能再去大殿主持丧仪。主子娘娘说了,一切有劳小主了。''

青樱听她这样说,知是富察氏知晓晞月不堪重用,只管托赖了自己应对,忙道:``请主子娘娘安心养息。''

青樱回到殿中,满殿缟素之下的哭泣声已经微弱了许多,大约跪哭了一日,凭谁也都累了。青樱吩咐殿外的宫女:``几位年长的宗亲福晋怕挨不得熬夜之苦,你们去御膳房将炖好的参汤拿来请福晋们饮些,若还有支持不住的,就请到偏殿歇息,等子时大哭时再请过来。''

宫女们都答应着下去了,晞月在内殿瞧见,脸上便有些不悦。青樱进来,便道:``方才要妹妹替主子娘娘主持一切,实在是辛苦妹妹了。''

晞月也不做声,只淡淡道:``你一句一句妹妹叫得好生顺口,其实论年岁算,我还虚长了你七岁呢。''

青樱知她所指,只是在潜邸之中,她原是位序第一的侧福晋,名分分明,原不在年纪上。当下也不理会,只微微笑道:``是么?''

晞月见她不以为意,不觉隐隐含怒,别过脸去不肯再和她说话。

过了一个时辰,便是大哭的时候了。合宫寂静,人人忍着困意提起了精神,生怕哀哭不力,便落了个``不敬先帝''的罪名。执礼太监高声喊道:``举哀------''众人等着嫔妃们领头跪下,便可放声大哭了。

因着富察氏不在,青樱哀哀哭了起来,正预备第一个跪下去。谁知站在她身侧一步的晞月抢先跪了下去,哀哀恸哭起来。

晞月原本声音柔美,一哭起来愈加清婉悠亮,颇有一唱三叹之效,十分哀戚。连远远站在外头伺候的杂役小太监们,亦不觉心酸起来。

按着在潜邸的位分次序,便该是晞月在青樱之后,谁知晞月横刺里闯到了青樱前头放声举哀,事出突然,众人一时都愣在了那里。

潜邸的格格苏绿筠更是张口结舌,忍不住轻声道:``月福晋,这\ldots\ldots 青福晋的位次,是在您之上啊。''

晞月根本不理会苏氏的话,只纹丝不动,跪着哭泣。

青樱当众受辱,心中暗自生怒,只硬生生忍着不做声。惢心已经变了脸色,正要上前说话,青樱暗暗拦住,看了跟在身后的格格苏绿筠一眼,慢慢跪了下去。

绿筠会意,即刻随着青樱跪下,身后的格格们一个跟着一个,然后是亲贵福晋、诰命夫人、宫女太监,随着晞月举起右手侧耳伏身行礼,齐声哭了起来。

哀痛声声里,青樱盯着晞月举起的纤柔手腕,半露在重重缟素衣袖间的一串翡翠珠缠丝赤金莲花镯在烛火中透着莹然如春水的光泽,刺得她双目发痛。青樱随着礼仪俯下身体,看着自己手腕上一模一样的镯子,死死地咬住了嘴唇。

待到礼毕,已子时过半,晞月先起身环视众人,道了声:``今日暂去歇息,明日行礼,请各位按时到来。''如此,众人依序退去,青樱扶着酸痛的双膝起身,扶了惢心的手,一言不发就往外走。

格格苏绿筠一向胆小怕事,默然撇开侍女的手,紧紧跟了过来。

青樱心中有气,出了殿门连软轿都不坐,脚下越走越快,直走到了长街深处。终于,惢心亦忍不住,唤道:``小主,小主歇歇脚吧。''

青樱缓缓驻足,换了口气,才隐隐觉得脚下酸痛。一回头却见绿筠鬓发微蓬,娇喘吁吁,才知自己情急之下走得太快,连绿筠跟在身后也没发觉。

青樱不觉苦笑,柔声道:``你生下三阿哥才三个多月,这样跟着我疾走,岂不伤了身子?''青樱见她身体姿孱孱,愈加不忍,``是我不好,没察觉你跟着我来了。''

绿筠怯怯:``侧福晋言重了,我的身子不相干。倒是今日\ldots\ldots 高姐姐如此失礼,可怎生是好?''

青樱正要说话,却见潜邸格格金玉妍坐在软轿上翩跹而来。

金玉妍下了软轿,扶着侍女的手走近,笑吟吟道:``怎生是好?这样的大事,总有皇上和主子娘娘知道的时候,何况还有太后呢。侧福晋今日受的委屈,还怕没得报仇么?''

青樱和缓道:``自家姐妹,有什么报仇不报仇的,玉妍妹妹言重了。''

金玉妍福了一福,又与苏绿筠见了平礼,方腻声道:``妹妹也觉得奇怪,高姐姐一向温柔可人,哪怕从前在潜邸中也和侧福晋置气,却也不至如此。难道一进宫中,人人的脾气都见长了么?''

绿筠忙道:``何人脾气见长了?玉妍妹妹得皇上宠爱,可以随口说笑,咱们却不敢。''

玉妍媚眼如丝,轻俏道:``姐姐说到宠爱二字,妹妹就自愧不如了。现放着侧福晋呢,皇上对侧福晋才是万千宠爱。''她故作沉吟,``哎呀!难道高姐姐是想着,进了紫禁城,侧福晋会与景仁宫那位一家团聚,会失幸于皇上和太后,才会如此不敬?''

青樱略略正色:``先帝驾崩,正是国孝家孝于一身的时候,这会子说什么宠爱不宠爱的,是不是错了时候?''

绿筠忙收了神色,恭身站在一旁。玉妍托着腮,笑盈盈道:``侧福晋好气势,只是这样的气势,若是方才能对着高姐姐发一发,也算让高姐姐知道厉害了呢。''玉妍屈膝道,``夜深人困倦,才进宫就有这样的好戏,日后还怕会少么?妹妹先告辞,养足了精神等着看呢。''

玉妍扬长而去,绿筠看她如此,不觉皱了皱眉。

青樱劝道:``罢了。你不是不知道金玉妍的性子,虽说是和你一样的格格位分,在潜邸的资历也不如你,但她是朝鲜宗室的女儿,先帝特赐了皇上的,咱们待她总要客气些,无须和她生气。''

绿筠愁眉不展:``姐姐说得是,我何尝不知道呢?如今皇上为了她的身份好听些,特特又指了上驷院的三保大人做她义父,难怪她更了不得了。''

青樱安慰道:``我知道你与她住一块儿,难免有些不顺心。等皇上册封了六宫,迟早会给你们安置更好的宫殿。你放心,你才生了三阿哥,她总越不过你去的。''

绿筠忧心忡忡地看着青樱:``月福晋在皇上面前最温、柔善解人意,如今一进宫,连她也变了性子,还有什么是不能的?''绿筠望着长街甬道,红墙高耸,直欲压人而下,不觉瑟缩了细柔的肩,``常道紫禁城怨魂幽心,日夜作祟,难道变人心性,就这般厉害么?''

这样乌深的夜,月光隐没,连星子也不见半点。只见殿脊重重叠叠如远山重峦,有倾倒之势,更兼宫中处处点着大丧的白纸灯笼,如鬼火点点,来往皆白衣素裳,当真凄凄如鬼魅之地。

青樱握了握绿筠的手,温和道:``子不语怪力乱神。绿筠你好歹还痴长我几岁,怎么倒来吓我呢?何况高晞月的温柔,那是对着皇上,可从不是对着我们。''

绿筠闻言,亦不觉含笑。

青樱望着这陌生的紫禁城,淡然道:``你我虽都是紫禁城的儿媳,常常入宫请安,可真正住在这里,却也还是头一回。至于这里是否有怨魂幽心,我想,变人心性,总是人比鬼更厉害些吧。''

毕竟劳碌终日,二人言罢也就散去了。

晞月回到宫中,已觉得困倦难当。晞月在和合福仙梨木桌边坐下,立时有宫女端了红枣燕窝上来,恭声道:``小主累了,用点燕窝吧。''

晞月扬了扬脸示意宫女放下,随手拔下头上几支银簪子递到心腹侍婢茉心手中,口中道:``什么劳什子!暗沉沉的,又重,压得我脑仁疼。''说罢摸着自己腕上碧莹莹的翡翠珠缠丝赤金莲花镯,``还好这镯子是主子娘娘赏的,哪怕守丧也不必摘下。否则整天看着这些黯沉颜色,人也没了生气。''

茉心接过簪子放在妆台上,又替晞月将鬓边的白色绢花和珍珠压鬓摘下,笑道:``小主天生丽质,哪怕是簪了乌木簪子,也是艳冠群芳。何况这镯子虽然一样都有,小主戴着就是比青福晋好看。''

晞月瞥她一眼,笑吟吟道:``就会说嘴。艳冠群芳?现放着金玉妍呢,皇上可不是宠爱她芳姿独特?''

茉心笑:``再芳姿独特也不过是个小国贱女,算什么呢?主子娘娘体弱,苏绿筠性子怯懦,剩下的几个格格侍妾都入不得眼,唯一能与小主平起平坐的,不过一个乌拉那拉青樱。只是如今小主已经作了筏子给她瞧了,看她还能得意多久!''

晞月慢慢舀了两口燕窝,轻浅笑道:``从前她总仗着是先帝孝敬皇后和景仁宫皇后的表侄女儿,又是先帝和太后指婚给皇上的,得意过了头。如今太后得势,先帝与孝敬皇后都已作古,景仁宫那位反倒成了她的累赘了。想来太后和皇上也不会再敷衍她。''

茉心替晞月捶着肩道:``可不是么,奴婢瞧主子娘娘也不愿看她。''

晞月叹口气:``从前虽然都是侧福晋,我又比她年长,可是我进府时才是格格,虽然后来封了侧福晋,可旁人眼里到底觉着我不如她,明里暗里叫我受了多少气?同样这个镯子,原是一对的,偏要我和她一人一个,形单影只的,也不如一对在一起好看。''

茉心想着自己小主的前程,也颇痛快:``可不是。小主手腕纤细白皙,最适合戴翡翠了。也是她从前得意罢了,如今给了她个下马威,也算让她知道了。侧福晋有什么要紧,要紧的是在后宫的位分、皇上的宠爱。''

晞月柔婉一笑,嘉许地看了茉心一眼,又不免有些忧心:``我今日在哭灵时这样做,实在冒险。你的消息可确实么?''

茉心笑道:``小主放一百二十个心,是主子娘娘身边的莲心亲口来告诉奴婢的,说是听见皇上与主子娘娘说的。给莲心一万个胆子,她也不敢撒这样的弥天大谎啊!''

晞月闭上秀美狭长的凤眼,笑道:``那就好了。''

\hypertarget{ux7b2cux4e09ux7ae0-ux98ceux96e8}{%
\chapter{第三章 风雨}\label{ux7b2cux4e09ux7ae0-ux98ceux96e8}}

夜深。

殿中富察氏正喝药,莲心伺候在旁,接过富察氏喝完的药碗,又递过清水伺候她漱口。方漱了口,素心便奉上蜜饯,道:``这是新腌制的甜酸杏子,主子尝一个,去去嘴里的苦味儿。''

富察氏吃了一颗,正要合着被子躺下,忽地仿佛听到什么,惊起身来,侧耳凝神道:``是不是永琏在哭?是不是?''

素心忙道:``主子万安,二阿哥在阿哥所①呢,这个时候正睡得香。''

富察氏似有不信,担心道:``真的?永琏认床,怕生,他夜里又爱哭。''

素心道:``就为二阿哥认床,主子不是嘱咐乳母把潜邸时二阿哥睡惯的床挪到了阿哥所么?宫里又足足添了十六个乳母嬷嬷照应,断不会有差池的。''

富察氏松了口气:``那就好。只是那些乳母嬷嬷,都是靠得住的吧?还有,大阿哥也住在阿哥所\ldots\ldots{}''

素心微笑:``主子娘娘的安排,哪次不是妥妥帖帖的?大阿哥虽然也住在阿哥所,但和咱们二阿哥怎么能比?''

富察氏点点头:``大阿哥的生母虽然和我同宗,却这样没福,偏在皇上登基前就过世了,丢下大阿哥孤零零一个。''她婉转看了素心一眼,``你吩咐阿哥所,对大阿哥也要用心看顾,别欺负了这没娘的孩子。''

素心含笑:``奴婢明白,知道怎么做。''

富察氏似乎还不安心,有些辗转反侧。莲心放下水墨青花帐帷,苦口婆心劝道:``主子安置吧,睡不了几个时辰又得起来主持丧仪。今夜您不在,大殿里可不知闹成什么样子了呢。''

富察氏微微一笑,有些疲倦地伏在枕上,一把瀑布似的青丝蜿蜒下柔婉的弧度,如她此刻的语气一般:``是啊。可不知要闹成什么样子呢?尚未册封嫔妃,她们就都按捺不住性子了么?''

莲心淡然道:``由得她们闹去,只要主子娘娘是皇后,凭谁都闹不起来。''

富察氏淡淡一笑:``闹不起来?在潜邸时就一个个乌眼鸡似的,如今只怕闹得更厉害吧。''她翻了个身,朝里头睡了,``只是她们耐不住性子爱闹,就由着她们闹去吧。''

富察氏不再说话,莲心放下帐帘,素心吹熄了灯,只留了一盏亮着,两人悄然退了出去。

青樱回到宫中,只仿若无事人一般。陪嫁侍婢阿箬满脸含笑迎了上来:``小主辛苦了。奴婢已经准备好热水,伺候小主洗漱。''

青樱点点头不说话,抬眼见阿箬样样准备精当,一应服侍的宫女捧着金盆栉巾肃立一旁,静默无声,不觉讶异道:``何必这样大费周章?按着潜邸的规矩简单洗漱便是了。''

阿箬笑盈盈靠近青樱,极力压抑着喜悦之情,一脸隐秘:``自小主入了潜邸,皇上最宠爱的就是您,哪怕是福晋主子也比不上。高小主虽然也是侧福晋,但她起先不过是个格格,后来才被封的侧福晋,如何比得上您尊贵荣耀?''

惢心淡淡看她一眼:``好端端的,你和小主说起这个做什么?''

阿箬笑意愈浓,颇为自得:``大阿哥是富察诸瑛格格生的,诸瑛格格早就弃世而去,那就不提。福晋主子生了二阿哥,将来自然是皇后,但得不得宠却难说。苏小主有了三阿哥,却和高小主一样,是汉军旗出身,那可不行了。''

青樱慢慢拨着鬓角一朵雪白的珠花。银质的护甲触动珠花轻滑有声,指尖却慢慢沁出汗来,连摸着光润的珍珠都觉得艰涩。青樱不动声色:``那又怎样呢?''

阿箬只顾欢喜,根本未察觉青樱的神色:``所以呀,小主一定会被封为仅次于皇后的皇贵妃,位同副后。再不济,总也一定是贵妃之位。若等小主生下皇子,太子之位还指不定是谁的呢\ldots\ldots{}''

青樱望着窗外深沉夜色,紫禁城乌漆漆的夜晚让人觉得陌生而不安,檐下的两盏白灯笼更是在夜风中晃得让人发慌。青樱打断阿箬:``好了。有这嘴上的功夫,不如去倒杯茶来我喝。''

惢心机警:``小主今日哭久了,怕是口渴得厉害。''

阿箬喜滋滋正要离去,青樱忍不住喊住她:``先帝驾崩,你脸上那些喜色给人瞧见,十条命都不够你去抵罪的,还当是在潜邸里么?''

阿箬吓得一哆嗦,赶紧收敛神色,诺诺退下。青樱微微蹙眉:``这样沉不住气\ldots\ldots 惢心,你看着她些,别让她失了分寸惹祸。''

惢心点头:``是。阿箬是直肠子,不懂得收敛形色。''

青樱扫一眼侍奉的宫人,淡淡道:``我不喜欢那么多人伺候,你们下去,惢心伺候就是。''

众人退了出去。

青樱叹口气,抚着头坐下。哭得久了,哪怕没有感情投入,都觉得体乏头痛,无奈道:``在潜邸无论怎样,关起门来就那么点子大,皇上宠我,难免下人奴才们也有些失分寸。如今可不一样了,紫禁城这样大,到处都是眼睛耳朵,再这样由着阿箬,可是要不安生。''

惢心点头道:``奴婢明白,会警醒宫中所有的口舌,不许行差踏错。''

青樱颔首,便由着惢心伺候了浸手,外头小太监道:``启禀小主,海兰小主来了。''

因着海兰抱病,今日并未去大殿行哭礼,青樱见她立在门外,便道:``这样夜了怎么还来?着了风寒更不好了,快进来罢。''

海兰温顺点了头,进来请了安道:``睡了半宿出了身汗,觉得好多了。听见侧福晋回来,特意来请安,否则心中总是不安。''

青樱笑道:``你在我房中住着也有日子了,何必还这样拘束。惢心,扶海兰小主起来坐。''

海兰诚惶诚恐道了``不敢'',小心翼翼觑着青樱道:``听闻,今夜高晞月又给姐姐气受了。''

青樱``哦''一声:``你身上病着,她们还不让你安生,非把这些话传到你耳朵里来。''

海兰慌忙站起:``妾身不敢。''

青樱微笑:``我是怕你又操心,养不好身子。''

海兰谦恭道:``妾身是跟着小主的屋里人,承蒙小主眷顾,才能在潜邸有一席容身之地,如何敢不为小主分担?''

青樱温和道:``你坐下吧,站得急了又头晕。''

海兰这才坐下,谦卑道:``在小主面前,妾身不敢不直言。在潜邸时月福晋虽然难免与小主有些龃龉,但从未如此张扬过。事出突然,怕有什么变故。''她抬眼望青樱一眼,低声道,``幸好,小主隐忍。''

青樱默然片刻,方道:``高晞月忽然性情大变,连金玉妍都会觉得奇怪。可是只有你,会与我说隐忍二字。''

海兰道:``小主聪慧,怎会不知高晞月素日温婉过人,如今分明是要越过小主去。这样公然羞辱小主,本不该纵容她,只是\ldots\ldots{}''

``只是情势未明,而且后宫位分未定,真要责罚她,自然有皇上与皇后。再如何受辱,我都不能发作,坏了先帝丧仪。''

海兰望着青樱,眼中尽是赞许钦佩之意:``小主顾虑周全。''她欲言又止,似有什么话一时说不出口。青樱与她相处不是一两日了,便道:``有什么话,你尽管说就是。这里没有外人。''

海兰绞着绢子,似乎有些不安:``妾身今日本好些了,原想去看望主子娘娘的病情。谁知到了那儿,听娘娘身边的莲心和素心趁着去端药的空儿在说闲话。说月福晋的父亲江南河道总督高斌高大人甚得皇上倚重,皇上是说要给高氏一族抬旗②呢?''

青樱脑中轰然一响,喃喃道:``抬旗?''

海兰脸上的忧色如同一片阴郁的乌云,越来越密:``可不是!妾身虽然低微,但也是秀女出身,这些事知道一星半点。圣祖康熙爷的生母孝康皇太后的佟氏一族就是大清开国以来第一个抬旗的。那可无上荣耀啊!''

青樱郁然道:``的确是无上荣耀。高晞月是汉军旗,一旦抬旗,那就是满军旗了。她原本也就是出身上不如我一些,这一来若是真的,可就大大越过我去了。''

海兰有些忧心:``人人以为小主在潜邸时受尽恩宠,福泽深厚。如今妾身看来,怕却是招祸多于纳福。还请小主万事小心。''她微微黯然,``这些话不中听\ldots\ldots{}''

青樱微微有些动容:``虽然不中听,却是一等一的好话。海兰,多谢你。''

海兰眸中一动,温然道:``小主的大恩,妾身永志不忘。妾身先告辞了。''

青樱看海兰身影隐没于夜色之中,不觉有些沉吟:``惢心,你瞧海兰这个人\ldots\ldots{}''

惢心道:``她在小主身边也有些年,若论恭谨、规矩,再没有比得上她的人了,何况又这样懂事,事事都以小主为先。''

青樱凝神想了想:``仿佛是。可真是这样规矩的人,怎会对宫中大小事宜这样留神?''

惢心不以为意:``正是因为事事留神,才能谨慎不出错呀。''

青樱一笑:``这话虽是说她,你也得好好学着才是。''

惢心道:``是。''

青樱起身走到妆镜前,由惢心伺候着卸妆:``可惜了,这样的性子,这样的品貌,却只被皇上宠幸过两三回,这么些年,也算委屈她了。''

惢心摇头:``小主抬举她了。海兰小主是什么出身?她阿玛额尔吉图是丢了官被革职的员外郎。当年她虽是内务府送来潜邸的秀女,可是这样的身份,不过是在绣房伺候的侍女,若不是皇上偶尔宠幸了她一回,您还求着皇上给了她一个侍妾的名分,才被人称呼一声格格,今日早被皇上丢在脑后了,还不知是什么田地呢。''

青樱从镜中看了惢心一眼:``这样的话,别浑说。眼看着皇上要大封潜邸旧人,海兰是一定会有名分的,你再这样说,便是不敬主上了。''

惢心有些畏惧:``奴婢知道,宫里比不得府里。''

青樱望着窗外深沉如墨的夜色,又念着海兰刚才那番话,慢慢叹了口气。

注释:

①阿哥所:是清宫皇子年幼至成婚前固定住所的俗称,主要有``南三所''、``乾东五所''、``乾西五所''几处。乾东五所在乾清宫之东、千婴门之北,实际上是五座南向的院落,自西向东分别称``东头所''、``东二所''、``东三所''、``东四所''、``东五所''。此区域在明代时就成为皇子的居住之处。乾、嘉、道三朝的多数皇子都居于此。一般来说,皇子成婚封爵之后就要开府,迁出阿哥所,但也有成婚封爵之后仍留在阿哥所居住的。

②抬旗:是清朝政府改变皇后和妃嫔家族的旗籍,以提高其出身的一种制度。不仅包括将包衣汉姓改变为八旗汉军,也包括由八旗汉军改变为八旗满洲乃至由下五旗改变为上三旗。

\hypertarget{ux7b2cux56dbux7ae0-ux76f4ux8a00}{%
\chapter{第四章 直言}\label{ux7b2cux56dbux7ae0-ux76f4ux8a00}}

这日清晨起来,青樱匆匆梳洗完毕,便去富察氏宫中伺候。为了起居便于主持丧仪诸事,富察琅华便一直住在就近的偏殿。

青樱去时天色才放亮,茹心打了帘子迎了青樱进去,笑道:``小主来得好早。主子娘娘才起来呢。''

青樱谦和笑道:``我是该早些伺候主子娘娘起身的。''

里头帘子掀起,伺候洗漱的宫女捧着桎巾鱼贯而出。青樱知道富察氏洗漱已毕,该伺候梳妆了。

茹心朝里轻声道:``主子,青福晋来了。''

只闻得温婉一声:``请进来吧。''

两边侍女双手掀帘,半曲腰身,低眉颔首迎了青樱进去。青樱不觉暗赞,即便是国丧,富察氏这里的规矩也是丝毫不错。

青樱进去时,富察氏正端坐在镜前,由专门的梳头嬷嬷伺候着梳好了发髻。富察氏与皇帝年龄相当,自是端然生姿的华年。简单单一方青玉无缀饰的扁方,显得她格外清淡宜人,如一枝迎风的白木兰,素虽素,却庄静宜人。

青樱请了安,富察氏笑着回头,``起来吧。难得你来得早。''

青樱起身谢过,富察氏指着镜台上一盒盒打开的饰盒,道:``丧中不宜珠饰过多,但太清简了也叫人笑话。你向来眼力好,也来替我选选。''

青樱笑,``主子娘娘什么好东西没见过,不过是考考妾身眼力罢了。''

富察氏微笑不语。青樱拣了一枚点翠银凤含珠的步摇比了比,道:``今日是举哀的最后一日,明日就是正式的登基大典。主子娘娘虽是素装,也得戴些亮眼的首饰。这步摇凤带翠羽,凤凰的眼珠子也是蓝宝珠子,再配上几朵蓝宝的珍珠花儿,最端雅不过,也还素净。''

富察氏向梳头嬷嬷笑道:``还不按青福晋说的做。''

青樱退开一步守着,只在旁伺候着递东西。富察氏看在眼里,也不言语。待到梳妆完毕,才慢慢笑说:``好好的侧福晋,倒为我做起这些微末功夫,可委屈你了。''

青樱忙道:``妾身不敢。''

富察氏对着镜子照了又照,笑道:``你配的珠饰,真真是挑不出错处来。若凡事为人处世,都能无可挑剔,那也算是福慧双修的人了。''富察氏闭目片刻,正色道,``你这个人,终究是委屈了。''

青樱不知富察氏所指,慌忙跪下道:``妾身愚钝,不明娘娘所指,还请娘娘指教。''

富察氏看了她两眼,慢慢说:``你怎么嫁进王府成了侧福晋的,你自己清楚。''

青樱跪在地上,终究不知该如何说起,只好低头不敢做声。

富察氏看她一味低头,慢慢露出笑意,道:``你我姐妹一场,我才这样问你。你这个人,终究是成也萧何,最怕败也萧何。也难怪高氏要处处抢你的风头。''

青樱勉强微笑,``妾身与高姐姐一同伺候皇上,说不上谁抢了谁的风头。妾身若有不如人的,高姐姐合该指教。''

富察氏淡淡笑一声,``指教?从前在王府里,她敢指教你吗?如今时移世易,你又该如何自处呢?''

青樱闻言,不觉冷汗涔涔,轻声道:``主子娘娘\ldots\ldots{}''

富察氏凝视她片刻,又复了往日端雅贤惠的神色,柔声道:``好了。我不过提醒你一句罢了,事情也未必坏到如此地步。''富察氏略略自矜,``到底我也是皇后,皇上的结发嫡妻,若是你安分守己,我也不容高氏再欺负了你去。''

青樱听得如此,只得谢恩,``多谢主子娘娘。主子娘娘一向对我和姐姐一视同仁,我能倚仗的,也只有主子娘娘了。''

富察氏的目光悠悠在她手腕上一荡,看青樱洁白的皓腕上除了一串翡翠珠缠丝赤金莲花镯外,别无其他饰物,不由得暗暗颔首:``你手腕上这串镯子,还是皇上为皇子的时候安南国进贡的珍品,一共只有一对。当时先帝赐给了咱们府里。我想着你和高氏是平起平坐的,便一人一个给了你们。既是让你们彼此间存了亲好之心,也是要你们明白,同为侧福晋,应当不分彼此,不要凡事计较。如今你倒还肯天天戴着。''

这一只镯子,原是安南国极稀罕的贡品。安南本出好翡翠,但如这一对的,真真是罕见。一串碧绿翡翠珠颗颗一样大小,通透温润不说,更难得的是竟然均匀得没有半点杂色,碧幽幽得恍若一汪流动的绿水。若拿到阳光下照着,便会出现一纹一纹水波似的莹白光痕,如同孔雀翎羽一般。因这翡翠珠碧色沉沉,所以特配了赤金缠丝花叶护着珠子周身,每颗翡翠珠的两端各用薄薄的莲花状金箔裹住,更是一分匠心独运。

当年还是四皇子的皇帝得到这对镯子,也是欣喜异常,虽宠爱两位新婚的侧福晋,但还是送给了嫡福晋富察氏。富察氏体念皇帝的心意,收下不过几天,便转赠给了青樱和晞月。

青樱低首,抚着镯子一脸安分随和,``主子娘娘说的是。真是感念娘娘这份心意,所以如娘娘当年嘱咐,时时戴着时时警醒。''

富察氏柔和道:``你是个懂事的。我看高氏也天天戴着,却也未必记得这层意思了。''她顿一顿,``唉,昨夜高氏僭越,我不是不知,只是从今以后你也只得让着她了。''青樱心想着海兰昨夜所言,正要说话,却听富察氏道,``你来之前皇上已经有了口谕,为高氏抬旗,抬的可是镶黄旗,又赐姓高佳氏。大清开国近百年,能得皇上亲口抬旗,获此殊荣的,只有高氏一人,且只有正黄和镶黄两旗是天子亲信,这里面的分量,你可掂量清楚了吧。''

青樱心中悸动,想要说话,却只惊异得口舌麻木,一字也说不出来,只得喏喏含笑。

富察氏回转头在首饰匣里闲闲挑出一双玲珑蓝宝坠耳环,口中道:``从前府中,你的地位自然比高氏矜贵,如今看来,她竟是要跟你比肩了。唉\ldots\ldots 你先跪安吧。''

青樱慢慢走出富察氏殿中,只觉得口干舌燥,仿佛从未如此烦恼过。连当初\ldots\ldots 当初被三阿哥弘时回绝羞辱,也不曾如此。

她脑中想到``弘时''两字,只觉厌烦,用力摆了摆头,扶了惢心的手慢慢出去。

炎夏暑期退散,偶尔一两阵风来,也隐隐有了清凉之气。前头隐约有人说笑着过来,青樱皱了皱眉,正要说话,却见高晞月与金玉妍亲亲热热过来。见了青樱,金玉妍倒还是如常退开半步,屈膝行礼,高晞月却只笑吟吟望着青樱,``妹妹好早啊。''

青樱自知情势不同往日,先与高晞月见了个平礼,方含笑道:``来得早不如来得巧。主子娘娘梳洗完毕,进去正好呢。''

晞月点点头,笑道:``入宫这几日,妹妹都还住得惯吗?''

青樱道:``劳姐姐费心,一切都好。''

晞月颔首,``住得惯就好。我生怕妹妹睡惯了王府的热炕头,不习惯紫禁城高床大枕,半夜醒来孤零零一个,冷不丁吓一跳呢。''

青樱眉心微微一蹙,面上倒还笑着,``高姐姐惯会说笑。皇上为先帝守孝,这些日都在养心殿住着,难不成姐姐还有皇上做伴吗?''

晞月居高临下瞥她一眼。``妹妹千伶百俐,以后可算棋逢敌手了。景仁宫的乌拉那拉皇后,大约会和妹妹一样有空,一同闲话家常呢。''她见青樱神色微微尴尬,走近一步低声道,``夹在皇太后和乌拉那拉皇后之间,妹妹与其有空争宠,不如想想该如何自处是好。''

高晞月向玉妍招了招手,亲热道:``杵在那儿做什么?还不跟我进去!''

玉妍答了声``是'',瞟了青樱一眼,得意地挽上晞月的手,亲亲热热进去了。

有风贴着面刮过。京中九月的风,原来有如此风沙隐隐的凉意,会吹迷了人的眼睛。

惢心待她们进去,扶住青樱的手慢慢往前走,低声愤愤道:``月福晋不过是和您一样的人,受了您的礼也不还礼,她\ldots\ldots{}''

青樱淡淡道:``这样的日子,以后多着呢。我若连这点气都受不住,就白和她相处这几年了。''缓一口气,``何况,她到底年长我几岁,我敬她几分,听她教诲,也是应当的。只要她不过分也就是了。''

惢心欲言又止,青樱看她一眼,``你想说什么?''

惢心低眉顺眼,``小主这样说,也是知道晞月福晋那个人,不是我们让着,她就能不过分的。''

青樱眉毛一挑,沉声道:``知道的事一定要说出来么。讷于言敏于行是你的好处,怎么和阿箬一样心直口快了?''

惢心垂首不语,只伸出手来,``奴婢知错。小主,该去先帝灵前行礼了。''

这一日灵前哭丧,晞月理所当然跪在青樱之前。富察氏一句言语都没有,反而待高氏比寻常更客气。殿中人最擅见风使舵,一时间也改了昨日惊诧之情,待晞月更为恭敬。

过了辰时三刻,太妃们一一入殿,与新帝的嫔妃们分列左右两侧,戚戚举哀。殿中人虽多,然而一眼而去,皆是素服银器,白霜霜的一片哀色。仿佛再有魂灵的一个人,也成了那素色中单薄的一点。不过半个时辰,太后扶着福姑姑的手也过来了。因着连日举哀,太后的神色并太好。太后是先帝的熹贵妃,一向深得宠爱,养尊处优,于保养功夫上也十分尽心,望之如三十许人。如今因着心境哀伤,为着先帝过身伤心得数日水米未进,整个人顿时枯槁了许多。仿佛那红颜盛时,一朝就花叶零丁了。

琅华见太后进殿,忙领着众人行礼如仪。太后微微颔首,``行了。都是为先帝尽心尽孝的时候,也不必那么多规矩了。''

琅华忙应了``是'',起身搀住太后。青樱一向与琅华入宫觐见最多,便也踏出了一步想去扶住太后。哪知晞月往她手肘一撞,一步上前扶住了太后另一只手,婉声道:``太后连日来疲倦了,未免哀思伤身,也应当注意凤体。''

太后颔首,拍拍晞月手背,``你有心了。''

待得太后走近了,青樱才敢抬头看她。从前入宫相见,太后尚且是得宠的贵妃,虽有年轻的宁嫔与谦嫔后来居上,到底也陪伴先帝多年的可心人,总是脂光水腻的精致妆容,不见丝毫放松。如今细细打量去,到底岁月无情,伴着忧伤无声无息地爬过她的皮肤,在她眉梢眼角碾上了细细的痕迹。太后脂粉轻薄的容颜憔悴暗淡,仿佛再好的丝缎,经久了时光,亦染上了轻黄的岁月痕迹,不复光洁平滑。

因着先帝去世,太后的装扮也素淡了许多。服丧的白袍底下露着银底缎子绣白色竹叶的素服,最清淡哀戚的颜色,袖口落着精致绵密的玄色并深青二色丝线捻了银线错丝绣的缠枝佛手花。散缀于缺月形发髻上的玉钿色泽光华,越发衬得一把青丝里藏不住的白发如刺眼的蓬草,一丝丝扎着人的眼睛。

青樱心下恻然,随着太后与琅华跪在灵前,凄凄然哀哭不已。

哭灵的日子虽然乏倦,但真当自己是树在灵前的一支烛台,或是被金丝细绳扎进了饿素白帷幔,时光倒也过得快了许多。

到了午膳时分,因着绿筠诞育三阿哥未久,太后特意准了她回去照看。绿筠感激万分,立刻去了。便由着琅华、晞月和青樱到偏殿侍奉太后用午膳。

太后的午膳本是要回寿康宫中用的。本朝的规矩,新帝不能与先帝嫔妃同居东西六宫。所以先帝过世,匆忙将六宫中一众遗妃都挪去了寿康宫中安置。太后也暂居在寿康宫正殿,并未搬去本应由太后独居的慈宁宫中。而这一日,本是为先帝举哀的最后一日,太后不愿车辇劳动,情愿多些时候为先帝尽哀,便嘱咐了御膳房将午膳挪在了偏殿。

琅华本打算着趁着中午用膳去看看二阿哥,但太后在此,本着孝道,她也尽心侍奉,一丝不错。一时间膳食上来,琅华添饭,晞月布菜,青樱舀汤,伺候的人虽多,但一丝咳嗽声也不闻,静得如无人一般。

太后见琅华服侍在侧,不觉问:``二阿哥还年幼,怎么你不回宫照拂,还要留在这里伺候哀家?''

琅华端然一笑,``太后有所不知,臣妾为了能尽心照拂好后宫诸事,按着祖宗规矩,已经将二阿哥送去阿哥所由嬷嬷照拂了。''

太后微微一惊,似是意外,``怎么?你不自己先照拂他两天,也不怕他住不惯阿哥所?''

琅华眉目恬静,仿佛安然承受,``本朝的家法,一旦生下阿哥公主,若有旨意,低位的嫔妃所出交给高位的嫔妃抚养;若无旨意,则一律交由阿哥所的嬷嬷们照管,以免母子过于情深,既不能安心伺候皇上,也误了再诞育皇嗣的机会。臣妾不敢不以身作则,所以二阿哥和大阿哥都送去了。''

太后凝神片刻,缓声道:``那是难为你了。如此说来,苏氏的三阿哥也不宜留在身边教养了。福珈,吩咐下去,命格格苏氏尽快将三阿哥挪去阿哥所,也好让她专心伺候皇帝。''

福姑姑答应了一声,吩咐下去,又转回太后身边伺候。

太后用膳的规矩,一向是先饮一碗汤。青樱见桌上有一道火腿鲜笋汤,雪白笋片配着鲜红火腿,汤汁金灿,引得人颇有胃口,便用盛了如意头银勺舀了一勺在碗中,又夹了笋片递到太后身前放下。

太后喝了一口,微微颔首,``论到汤饮,没有比上好的金华火腿配了笋片更吊鲜味的了。这汤鲜是鲜,笋片也做得嫩。只是鲜味都在前头了,后头的菜再好,总也觉得食之无味了。''

伺候太后的福姑姑是经年的老嬷嬷了,忙笑道:``太后一向是喜欢这个汤的,但连日来为先帝哀思伤神,本就茶饭无味,如今鲜味一过嘴,后面怕更吃不下了。''

青樱吓了一跳,忙跪下道:``臣妾只惦记着太后素日喜欢,竟未察觉太后当下的胃口,实在是臣妾的过失了。''

晞月看青樱如此,忍不住冷笑一声,只作壁上观。

琅华亦道:``光是汤也罢了。笋片虽鲜嫩,但多食伤胃,于太后是不相宜的。''

太后摆摆手,倦怠道:``算了。你也是一分孝心,是哀家自己没胃口罢了。''太后瞟一眼桌上的膳食,懒懒道,``叫人撤下去吧。哀家看了也没胃口。''

晞月无声冷笑,徐徐道:``妹妹好一分孝心,太后这些日子饮食清减,好容易用些午膳,才喝一口汤就被妹妹败了胃口。今日下午还有好几个时辰的哀仪,妹妹是打算让太后饿着身子熬在那儿吗?''

青樱咬了咬唇,磕了头道:``还请太后恕罪,臣妾一时有失,不想连累了太后凤体。太后要责罚臣妾无怨无悔,但请太后保养身体,多进一些吧。''

太后神思懒懒,并不欲进食。琅华见状,忙舀了一碗熬得极稠的粥来,拿银匙舀了轻轻吹着,递到太后手中,``太后再没胃口,也请为了先帝着想,进一碗粥吧。''

太后扬眸看了一眼,又懒懒闭上眼睛,厌道:``哀家没有胃口。''

福姑姑微微蹙眉,轻声道:``主子娘娘,太后这几日胃口不好,顶多进些熬得极薄的粥水,这么厚稠的粥,太后实在是没胃口吃。''

琅华并不气馁,笑吟吟道:``这种熬粥的米是御田里新进的,粒粒饱满晶莹剔透,吃上去口感微甜,柔软却有嚼劲,最适宜熬得稠稠的,却入口即化。皇上这几日伤心先帝驾崩,又忙着前朝的事情,也是没有胃口。儿臣嘱咐了御膳房做这样的粥,皇上倒能吃几口。''

太后这才点点头,``你是皇帝的结发妻子,是该多多关心皇帝,免他操劳。''她顿一顿,``罢了,皇帝都在努力加餐饭,哀家再伤心,也得用一点了。就尝尝吧。''

琅华喜不自禁,看太后吃了两口,倒还落胃,便也放心些。晞月殷勤布菜,尽拣些清淡小菜,倒也看着太后将小半碗粥都喝了。

琅华方才露了几丝笑意,柔声道:``青樱妹妹的汤是鲜,配着淡粥小菜也能入口了,若是后面的菜还是浓鲜,那才真伤了胃口呢。''

太后回味片刻,``你们有心了。只是哀家喝着,这粥里有股淡淡的姜味,吃下去倒是暖胃,稍稍舒服些。''

琅华意料之外,实在不知,忙看了身后侍候的御膳房太监一眼,便问:``是什么缘故?''

太监打了个千儿,躬身答道:``娘娘的嘱咐是用御田新进的米做粥,但皇上从前儿夜里便有些胃寒。青樱小主知道了,特意吩咐奴才们加了少许嫩姜在粥里,可以温胃暖气。皇上用了一直觉得不错,所以今儿给太后进的粥也是如法炮制。''

太后轻叹一声,``我的儿!这才是用心用足了。''她看了青樱一眼,吩咐道,``在外头跪着,在哀家这里也跪着,也不怕伤了膝盖皇帝心疼,起来吧。''

青樱这才敢谢恩起身。太后扶了扶鬓边的银累丝珍珠凤钗,道:``哀家还想喝点汤,你选一碗给哀家吧。''

青樱不敢再轻举妄动,仔细斟酌了,才选了一碗``紫参雪鸡汤''舀了给太后。太后才看了一眼,眼圈便有些红了,``怎么选了这个汤?''

青樱谨慎道:``紫参提气,雪鸡补身,适宜太后凤体。而且先帝在时,臣妾侍奉先帝与太后用膳,便听先帝嘱咐过此汤适宜太后饮用。如今请太后再饮,只当是请太后顾念先帝苦心,善自保养。''

太后凝神片刻,拈过绢子拭泪道:``先帝在时,是最喜欢这道汤的,总说能提神补气,也常嘱咐哀家喝。如今看着,只是触景伤情罢了。何况先帝才走,这满桌的膳食,多半是荤腥,哀家哪里能入口?罢了吧。''

这几句话虽不是拒绝用膳,但却比方才更严重,青樱只觉得耳后根一阵比一阵烫,烧得头皮发痛,且御膳的汤饮,为怕凉了,都是拿紫铜吊子暖在那儿的。青樱捧着一碗滚烫的汤在手里,起先还觉得指尖又热又痛,如虫咬一般,渐渐失了知觉,捧着汤进也不是退也不是,十分尴尬。

晞月见机,忙殷勤夹了一筷子龙须菜在太后碗里,``这龙须菜还算清口,太后尝一尝,也是吃点素食,略尽对先帝的心吧。''

太后勉强吃了一口,拉过琅华与晞月的手叹道:``哀家也是看在你们的心罢了。其实一饮一食能有多大的讲究?无非是审时度势,别自作聪明罢了!''她瞟了青樱一眼,``好了,还端着那汤做什么?譬如那粥,皇帝适合添些姜,哀家却未必适合。用心是好,但别总拿着对旁人那一套来对如今的人,明白了吗?''

青樱本不知自己错在何处,但听得这句话,才知了原因所在,直如五雷轰顶一般,软软跪下了。

\hypertarget{ux7b2cux4e94ux7ae0-ux76aeux5f71}{%
\chapter{第五章 皮影}\label{ux7b2cux4e94ux7ae0-ux76aeux5f71}}

待到晚来时分,青樱回自己殿中歇息,只觉得精疲力竭,连抬手喝茶的力气也没了。

惢心吩咐了一声,立刻便有小宫女上来,捶肩的捶肩,捏背的捏背。阿箬准备了热水正要给青樱烫手,惢心悄悄摇了摇头,低声道:``换冰水来吧。''

阿箬即刻换了水来,惢心已经从黄花梨的银锁屉子里找了一段清凉膏药出来,伺候着青樱浣了手,用银签子仔细挑了点药膏出来,小心翼翼地抹在青樱十指。

阿箬见青樱的十指个个留着绯红的印子,知道是烫的了,不觉柳眉倒竖,叱道:``惢心,你是跟着小主出去的,怎么小主的手会烫得这么红?你是怎么伺候的!''

惢心急得满脸通红,忙低声道:``阿箬姐姐,这件事说来话长\ldots\ldots{}''

``说来话长\ldots\ldots{}''阿箬轻哼一声,``无非是自己偷懒不当心罢了,这会子还敢回嘴!到底不是跟着小主的家生丫头,不知道心疼小主!''

阿箬是青樱的陪嫁,一向最有脸面,便自恃着是青樱的娘家人,说话做事也格外厉害些。惢心是潜邸里指过去跟着伺候各房福晋格格的,都是从了心字辈,虽然也是体面丫鬟,但毕竟比不上阿箬了,因此阿箬说话,她也不敢过多分辩。

青樱听着心烦不已,只冷冷道:``我没伺候好太后,弄伤了自己,午后已经上过点药了。''阿箬吃了一惊,立刻闭上嘴不敢多言,行动伺候间也轻手轻脚了许多。

青樱涂完了膏药,就着惢心的手喝了一盏茶,缓和了神色,阿箬方上来笑道:``今日是最后一日举哀。明儿个是皇上正式登基的日子,小主也该换点喜庆颜色的打扮了。''

阿箬见青樱点头,愈加笑起来,``奴婢听说前头定了皇上的年号是乾隆,真真是个兴隆旺盛,气象一新的好年号。奴婢们也跟着沾沾喜气,就等着皇上册封小主那一日了。''

青樱默默喝了口茶,``那又如何?''

阿箬喜气洋洋请了一安,``奴婢就等着娘娘册封贵妃的好日子了,这两日别的小主来探望您,她们身边的奴才也都这么说呢。''

青樱似笑非笑,只捧了茶盏凝神道:``你便看准了我有这样的好福气。那么阿箬,若是我只被封做答应,抑或被赶出宫中,你觉得如何呢?''

阿箬大惊失色,张口结舌道:``这\ldots\ldots 这怎么会?''

青樱敛容道:``怎么不会?有你这样红口白舌替我招祸,还敢与别人说这样的是非,我怎会不被你牵连。皇上要册封谁贬黜谁,那全是皇上的心意,你妄揣圣意,我问问你,你有几条命?''

阿箬吓得跪下,``小主,奴婢失言了,奴婢也是关心小主情切。''

青樱冷了冷道:``惢心,带她出去。阿箬言行有失,不许再在殿内伺候。''

阿箬惊慌失措,忙抱住青樱的腿道:``小主,小主,奴婢是您的陪嫁侍女,从小就伺候您,还请您顾惜奴婢的颜面,别赶了奴才去外头伺候。''

青樱摇头道:``你三番五次失言,来日皇上面前,难道我也能替你挡罪吗?''

阿箬哭道:``奴婢伺候小主,一直不敢不当心。小主喜欢多热的水多浓的茶,奴才都牢牢记在心里,一刻都不敢忘。还请小主饶恕奴才这回吧。''

青樱自知自己在潜邸里得意惯了,身边的人难免也跟着不小心,可是如今形势大变,不比往常,这心里的为难气苦,也只有自己知道。偏偏阿箬仗着是自己的陪嫁丫鬟,惯来无甚眉高眼低,自己有心要拿她做个筏子,却也狠不下心来。

半晌,青樱见阿箬兀自吓得伏在地上发抖,拼命哀求,也是从未有过的委屈,立时喝道:``还不出去!要再这样言语没有分寸,立刻叫人拖出去杖责,打死也不为过。''

阿箬闻声,吓得脸也白了,拼命磕头不已,还是惢心机灵,一把扶起了阿箬,赶紧谢了恩让她退下了。

这一来,殿中便安静了许多。伺候青樱的人都是见惯阿箬的身份和得宠的,一见如此,不由得人人噤声。青樱扬一扬脸,惢心立刻会意,打开殿门,青樱慢慢啜一口茶,不疾不徐道:``如今是在宫里,不比在潜邸由得你们任性,胡言乱语,信口开河。但凡我听到一句敢在背后议论主子的话,立刻送去慎刑司①打死,绝不留情。''

她这句话虽无所指,但人人听见无不起了冷汗,齐齐应了声,不敢再多惹半句是非。

青樱扬一扬脸,众人会意,立刻都退了出去。惢心见殿中无人,方伺候了青樱卸妆梳洗。青樱由着她摆弄,自己只坐在妆台前,望着镜中的自己。镜里容颜是看得再熟悉不过了,她才不过十九岁,出自先帝皇后的母族,一路顺风顺水,得了庇护,也难免性子娇些。这一路走路不能不说是安稳,但若论万事真有不足,那也是数年前那一桩旧事了。

出身高贵,青樱知道自己的身份,这一世不论高低,哪怕不是选秀进宫为嫔妃,也是要嫁与皇亲国戚的。最好的出路,当然是成为哪一位皇子的嫡福晋,主持一府事务,延续乌拉那拉氏的荣光。

先帝成年的儿子,只有三阿哥弘时、四阿哥弘历、五阿哥弘昼。当时她要被许配的,是三阿哥弘时。可是弘时偏偏心有所属,并不认可自己做他的福晋。万般无奈之下,正逢上当时尚为熹贵妃的太后为四阿哥求娶,她才如获大赦一般,逃脱了被人指指点点的尴尬,做了四阿哥的侧福晋。

嫁入四阿哥府邸后,日子也还算顺畅。虽然先帝跟前,四阿哥一直不算是最得宠的皇子,她也安下了心思,陪他过着每一日看似平静却得仔细打算着过的日子。幸好家中还安宁,府中比她地位高的,唯有一个嫡福晋富察氏,她一心只念着为四阿哥开枝散叶,巩固地位,也少与她争执。这些年四阿哥虽然收了几个妾室,但待她也算亲厚。她虽然出嫁前性子被家中宠得娇惯些,又有夫君的宠爱,难免骄横些。可是先帝最后那几年,自己的姑母乌拉那拉皇后失宠,她也不敢不收敛了些许。如今先帝驾崩,自己的夫君一朝登上九五之尊的位置,她心中自然欣喜万分,为他骄傲不已。可宫中的生活,才这几日便已经如履薄冰,晞月的凌驾,皇后的冷目,太后的敲打,无一不警醒着她,从前无知无觉的快乐岁月,是一去不复返了。

青樱静静地坐着,看着镜中形单影只的自己。为着先帝驾崩,宫中虽然一切简素,也让她们暂居偏殿,但宫殿到底还是宫殿,富丽堂皇,金堆玉砌,一切都如同繁花拱锦绣,无一不华美炫目。只有她,她是一个人的,对着镜是一个人,影子落在地上还是不成双,如那锦堆里的一根孤蕊。

青樱伸出手,握成一个虚空的圈,才知自己什么都把握不住。她的人生里,从未有过一日如今日这般惶惑无依,仿佛所有的底气,都一朝被抽尽了。

正惶惑间,外头突然吵闹了起来,似乎有人声喧哗,惊破了她孤独的自省。青樱蹙了蹙眉头,还未来得及出声询问,外头守着的阿箬已经推了门进来,惊惶道:``小主,苏格格像是疯了呢,满脸是泪跑到咱们这里来,一定要闹着见小主。天这么晚了\ldots\ldots{}''

阿箬话音未落,却见苏绿筠已经跑了进来。她想是准备歇息了,只穿着家常的玉色薄绸长衫裙,外头罩着浅水绿银纹重莲罩纱,跑得鬓发散乱。这样夜寒露冷的秋夜里,她居然跑得满脸是汗,和着泪水一起混在脸上,全然失了往日的娴静温懦。

青樱乍然变了脸色,大惊失色道:``绿筠,这是在宫里,你是做什么?''

绿筠的脸全然失了血色,苍白如瓷,她仿佛只剩下了哭泣的力气,泪水如泉涌下。良久,她终于``扑通''跪下,倒在青樱身前,放声大哭,``姐姐,姐姐,你救救我!主子娘娘派人带走了永璋!我的永璋,我的三阿哥!他才几个月大,主子娘娘就派人带走了他!''

青樱当下明白,皇后在太后跟前言及自己所亲生二阿哥永琏已经在阿哥所抚养,那么身为小小一个格格所生的三阿哥,更没有留在生母身边养育的理由了。

绿筠哭得头发都散了,被汗水和泪水混合腻在玉白的脸颊上,仿若被横风疾扫过一般。她伏在地上,哀哭道:``姐姐,我求求你,帮我去求求主子娘娘,让她把永璋还给我,还给我!''

青樱忙伸手扶她,哪知绿筠力气这般大,拼命伏在地上磕头不已,``姐姐,我人微言轻,主子娘娘不会理我!可是你不一样,你是出身高贵的侧福晋,以前在潜邸的时候,主子娘娘也只还肯听你几句,你帮我求求她,好不好!''

以前,以前是多久的事了。那是彼此身份地位的约衡,而非真心。

青樱使个眼色,阿箬与惢心一边一个半是扶半是拽地扶了她起来坐定。她见绿筠哭得声嘶力竭,心下亦是酸楚,只得劝她,``永璋是主子娘娘派人带走的,但不是主子娘娘能带得走永璋的,是祖宗规矩要带走永璋!''她顿一顿,``这件事,太后是知道的。''

绿筠登时怔住,双肩瑟瑟颤抖,``哪怕是祖宗规矩,可是永璋还那么小\ldots\ldots{}''

青樱按着她的肩头,柔声道:``永璋是还小。可是你要是在宫里生下的永璋,从他离开母腹的那一刻,他就被抱走了,顶多只许你看一眼。''她缓一缓声气,低声道,``何况主子娘娘禀告了太后,她亲生的二阿哥已经在阿哥所了,她也不敢违背家法。''

绿筠身子一晃几乎就要晕去,青樱忙扶住了她,在她虎口狠狠一掐。她本留着寸长的指甲,这一掐下去绿筠倒是醒了许多,只痴痴怔怔地流下泪来。阿箬赶紧喂了绿筠一口热茶,``小主别这样,真是要吓坏我们小主了!''

青樱按住了她,低柔道:``你这个样子,吓坏了我也就算了。可要吓着了宫里其他人,被她们那些嘴一个接一个地传出去,那成了什么了呢?你不要体面,三阿哥也是要的。''她扬一扬脸,示意惢心取过自己妆台上的玉梳来,一点一点替她篦了头发,挽起发髻,``咱们一进了宫里,就由不得自己了。从前我还是混混沌沌的,到了今日也算明白了。你比我还好些,还有个儿子。不比我,外头看着还不差,其实什么也没有了。你的永璋,养在阿哥所里,有八个嬷嬷精心照顾着,每到初一十五,她们就会把孩子抱来和你见上一个时辰,为的就是怕母子太过亲密,将来外戚干政。这件事,你是求谁都没用了,只能自己受着。''

青樱的手摸到绿筠的脸颊上,脂粉是湿腻的,泪水是灼人的滚烫。绿筠的泪落到手上,青樱才觉出自己双手的凉,竟是一丝温度也没有。这些话,她是劝绿筠的,也是劝自己。事到临头,若是求谁都没用,只有自己受着,咬着牙忍着。

她读过那么多的宫词,寂寞阑干,到了最后,只有这一点顿悟。

绿筠的眼泪啪嗒啪嗒落到衣襟上,转瞬不见。她满眼潸潸,悲泣伤心,``那么以后,难道以后,我就只能这样了。只要生一个孩子,这个孩子就得离开我,是吗?''

青樱为她正好发髻,取过一枚点蓝点翠的银饰珠花,恰到好处地衬出她一贯的柔顺与温和。青樱扬了扬脸,示意惢心绞了一把热帕子过来,重新替绿筠匀脸梳妆。她侧身坐下,轻轻道:``绿筠,不管你以后有多少个孩子。唯有这些孩子,你才能平步青云,在这宫里谋一个安定的位子。如果你真的伤心,你就记着一个人。康熙爷的德妃,先帝的生母孝恭仁皇后,她生先帝的时候,自己身份低微,只能将先帝交给当时的佟贵妃抚养。可是后来她诞育子女众多,最后所生的十四王爷便是留在了自己身边。如今你刚刚在宫里,大家也是一同入宫的,交给谁抚养也不合适,送进阿哥所是最好的。往后,往后你一切平安顺遂,你也能抚育自己的孩子。明白吗?''

绿筠怔怔地坐着,由着宫女们为她上好妆,勉强掩饰住哭得肿泡发红的双眼,泪汪汪道:``姐姐,那我该怎么办?''

青樱拿过绢子,替她拭了拭泪。``忍着。忍到自己有能力抚育自己的孩子。所以,现在你不能出错,不能出一点点错。''她拉着绿筠的手起身,``你现在打扮得整整齐齐的,去皇后宫里,向她谢恩,谢她让阿哥所替你照顾三阿哥。你刚才哭,刚才跑到我宫里,是因为你伤心过了度,一时昏了头。现在你明白过来了,这是恩典,你都受着了。''

绿筠咬着嘴唇,凄惶地摇头,``姐姐,我说不出来。我怕我一说,就会哭。''

青樱安慰似的抚着她单薄的肩,``别哭,想着你的将来,三阿哥的将来,你还有别的孩子。流泪,是为了他们;忍着不哭,也是为了他们。''

绿筠死死忍着泪,点了点头,向外走去。庭院内月光昏黄,树影烙在青砖地上稀薄凌乱,静谧中传来一阵阵枝丫触碰之声,那声音细而密,似无数细小的虫子在啃噬着什么东西似的,钻在耳膜里也是钻心的疼。青樱看着绿筠的影子拖曳在地上,单薄得好像小时候跟着嬷嬷们去看新奇的皮影戏,上头的纸片人们被吊着手脚欢天喜地地舞动,谁也不知道,一举一动,半点不由人罢了。

今时今日的她与绿筠,又有什么不一样呢?

这一夜,琅华本就睡得不深,暂居的偏殿不是睡惯了的安稳的旧床,耳边没有永琏熟悉的儿啼,她怎么也睡不安稳。地翻个身,陪夜睡在地下的侍女茹心便听见了,起来点上蜡烛,倒了盏安神汤递到琅华跟前,体贴道:``都三更了,娘娘怎么还睡不安?''

琅华本无睡意,便支着身子起来,``二阿哥不在身边,我心里总是不安稳。''

茹心塞了个鹅羽软枕在她腰间垫着,温言劝道:``娘娘安心。奴婢早去问过了,三位阿哥都在阿哥所,那些奴才们对咱们的二阿哥最尽心了,生怕有一点照顾不到。那些乳母奶水养得又好又足,轮流喂着二阿哥,嬷嬷们也伺候得精细,一点都不敢疏忽。''

琅华叹了口气,郁然道:``祖宗规矩在那儿,我不能常去看,你一定要替我尽心着。''

茹心忙道:``那是自然了。咱们二阿哥天尊地贵,其他阿哥连他脚趾上的泥都配不上,底下没有一个人敢不尽心尽力的。''她轻笑一声,``今儿三阿哥也被送离了苏格格身边,奴婢才叫高兴呢。凭什么娘娘守着祖宗家法,她偏母子俩一块儿,奴婢就是看不过去。''

琅华就着茹心的手慢慢啜饮着暗红色的安神汤,随口道:``罢了,她也可怜见儿的,明明伤心成那样了,还硬忍着到我跟前来谢恩。听说她哭着跑去乌拉那拉氏那儿了,她也不敢陪着,赶紧送了苏氏出来。''

茹心高兴道:``就得这样!青福晋能帮她,奴婢才不信。她自己都是泥菩萨过江自身难保了,今儿午膳的时候太后都给了她好大的没脸呢。''

琅华微微一笑,``本来乌拉那拉氏是太后为皇上求娶的侧福晋,又是先帝景仁宫皇后的侄女儿,我怎么也要让她三分。如今太后都给了这样的脸色,宫里的人就更有数了。''

茹心扬了扬唇角,甚是欢欣,``宫里除了太后,娘娘是唯一的主子娘娘。你要她们怎么着,她们就只能怎么着,就像那戏台上皮影似的,都得在您的手里。''

琅华抚着胸前一把散着的青丝,凝神片刻道:``是得都在我手里。所以茹心,你明儿就去阿哥所吩咐下去,一定要好好待三阿哥,比待我的永琏更好更精细。吃食由着吃不许约束,冷暖要注意着,一定要好好疼三阿哥,在襁褓里就尽着他玩尽着他乐。咱们皇家的孩子吃不得苦,好好宠着一辈子就是了。''

茹心虽不解其意,但听琅华这样郑重吩咐,忙答应了,取过她手中喝完的安神汤,重又垂下了珠罗帐。

注释:

①慎刑司:清内务府所属机构。初名尚方司,顺治十二年(1655)改尚方院。康熙十六年(1677)改慎刑司。掌上三旗刑名。凡审拟罪案,皆依刑部律例,情节重大者移咨三法司会审定案。太监刑罚,以慎刑司处断为主。

\hypertarget{ux7b2cux516dux7ae0-ux5f03ux5987}{%
\chapter{第六章 弃妇}\label{ux7b2cux516dux7ae0-ux5f03ux5987}}

十三年九月己亥,上即位于太和殿,以明年为乾隆元年。

------《清史稿·高宗本纪》

寿康宫里静悄悄的。太妃们哭了许多日也尽累了,所有的昔年情意恩宠,随着泪水,也都殆尽了。余下的日子,也是活在富贵影里,然后那是数得清的富贵,望不尽的深宫离离,寂寞孤清。

前朝嫔妃们所住的寿康宫,安静得如同活死人墓一般。哪怕是才十几二十岁的先帝遗妃们,也被尘埃覆没了,再没有了一丝活气。

落在偌大的紫禁城内廷外西路的寿康宫,是不同于鲜活的东西六宫的,那是另一重天地,也是住着皇帝的女人们,也是帐帷流苏溢彩,阑干金粉红漆,宫闱里也垂着密密织就的云锦,提到手中沉甸甸绵密密的,照样是上贡的最好锦缎,最最吉祥如意的图案。但那锦缎不是欢喜天地,人月两圆,不是满心期许,空闱等待,而是断了的指望,死了的念想,枯萎尽了的时光,连最顾影自怜的凄清月光,都不稀罕透入半分。

福姑姑端了一盘剥好的柚子进来。才打了帘子进来,便觉得寿康宫内阴暗狭小,不比往日宫内的高大敞亮,连幽幽的檀香在袅袅散开,也觉得这里幽闭,未等散尽就消失了。加上先帝新丧,里头的布置也暗沉沉的只有七八成新,心下便忍不住发酸。她见太后盘腿坐在榻上,碰了一卷书出神,少不得忍了气闷,换了一脸笑容道:``福建进贡的柚子,酸甜凉润,又能去燥火,太后吃着正好。''

太后淡淡笑道:``难为你了,费这么大力气剥了,哀家又吃不上几口。''

福姑姑笑道:``能吃几口也算是这柚子的福气了。''

太后捏了捏手臂,福姑姑会意,立刻上前替她捶着肩膀,轻声道:``今日皇上在太和殿登基,您在大典上陪着,也是累了一天了。不如早点安置,好好歇息。''

太后摸了摸自己的脸颊,``也是,一下子就成了太后了。皇帝登基,哀家的心思也定了。今日看着皇帝似模似样,大典上一丝不错,哀家真是欣慰。只是倒也不觉得困,想是日短夜长,这长夜漫漫的,有的睡呢。''

福姑姑见她如此神色,打量着狭小的正殿,欲言又止,``太后能安心就好,这些日子是委屈了。''

``委屈?''太后取了一片柚子拈在手中,``这片柚子若是被随意扔了出去,那才叫委屈,现在你拿了斗彩蝶纹盘装着它,已经有了安身的地方,怎么还叫委屈?''

福姑姑垂着脸站着,虽是一脸恭顺,却也未免染上了担忧之色,``太后,这柚子原该装在太后所用的斗彩凤纹盘里的,现在将就在这里,一切未能顾全,只能暂时用太妃们用的蝶纹盘将就,可不是委屈了?''

太后将柚子含在嘴里,慢慢吃了,方凝眸道:``福珈,哀家问你,这里是什么地方?''

福姑姑脸上忧色更重,更兼了几分愤愤不平之色,``这儿是寿康宫,太妃太嫔们居住的地方。正经您该住的慈宁宫,又轩亮又富丽,胜过这儿百倍。''

太后脸上一丝笑纹也没有,``是了。太妃太嫔们住的地方,用的是自然是太妃们该用的东西。''

福姑姑听到这一句,不觉抬高了声音,``太后!''太后轻轻``唔''一声,微微抬了抬眼皮,目光清和如平静无澜的古井,``什么?''

福姑姑浑身一凛,恰巧见鎏金蟠花烛台上的烛火被风带得扑了一扑,忙伸手护住,又取了小银剪子剪下一段焦黑蜷曲的烛芯,方才敢回话:``奴婢失言了,太后恕罪。''

太后平静地睁眸,伸手抚着紫檀小桌上暗绿金线绣的团花纹桌锦,淡淡道:``你跟了哀家多年,自然没有什么失言不失言的地方。只是哀家问你,历来后宫的女人熬到太后这个位子的,是凭着什么福气?''

福姑姑低缓了声音,沉吟着小心道:``这福气,不是诞育了新帝,就是先帝的皇后。''

太后的轻叹幽深而低回,如帘外西风,默然穿过暮气渐深的宫阙重重,``福珈,哀家并不是皇帝的亲生额娘,也从未被先帝册封为皇后。哀家所有的福气,不过是有幸抚育了皇帝而已。哀家这个被册封的太后,名不正言不顺,皇帝要不把哀家放在心上,哀家也是没有办法。''

福姑姑眉心一沉,正色道:``先帝在时,就宣称皇上是太后娘娘您亲生的,皇上不认您,难道还要回热河行宫找出宫女李金桂的骨骸奉为太后吗?也不怕天下人诟病?何况先帝虽有皇后,但后来那几年形同虚设,六宫之事全由太后打理。您殚精竭虑,扶着他登上九五至尊的位子,这个太后您若是名不正言不顺,还能有谁?''

太后徐徐抚着手上白银嵌翡翠粒团寿护甲,``这些话就是名正言顺了。可是皇帝心里是不是这么想,是不是念着哀家的抚育之恩,那就难说了。''

福姑姑问:``内务府也来请了好几回了,说慈宁宫已经收拾好了,请您挪宫。可您的意思\ldots\ldots{}''

太后微微一笑,``挪宫总是要挪的,可是得皇帝自己想着,不能哀家嘴里说出来。所以皇帝一日不来请哀家挪宫到慈宁宫。只是内务府请,哀家也懒得动。''

福姑姑皱了皱眉,踌躇道:``先帝驾崩,皇上刚登基,外头的事千头万绪,皇上已经两日没来请安了。哪怕是来了,皇上要不提,难道咱们就僵在这儿?''

太后伸手用护甲挑了挑烛台上垂下的猩红烛泪,``皇帝宫里头的人虽不多,但从潜邸里一个个熬上来的,哪一个不是人精儿似的。总有一个聪明伶俐的,比别人警醒的,知道怎么去做了。哀家没有亲生儿子当皇帝,没有正室的身份,若是再连皇帝的孝心尊重、后宫的权柄一并没有了,那才是什么都没有了。''

新帝登基,青樱也是极欢喜。初到潜邸为新妇的日子,她是有些抱屈的,因为毕竟不是先帝最爱的儿子。然而她却也感激,感激她的夫君拉她出了是非之地。相处的时日久了,她也渐渐发现,她的夫君虽然谨慎小心,但却极有抱负与才华,更具耐心。一点一点地熬着,如冒尖的春笋,渐渐为先帝所注意,渐渐得到先帝的器重。他的努力不是白费的,终于有了今朝的喜悦荣光。那,也是她的喜悦荣光。

晚膳时青樱情不自禁地嘱咐了厨房多做了两道皇帝喜爱的小菜,虽然明知这样的夜里,皇帝是一定不会在后宫用膳的,前朝有着一场接一场的大宴,那是皇帝的欢欣,万民的欢腾。可是她看着那些他素日所喜欢的菜肴,也是欢喜的,好像她的心意陪着他一般,总是在一块儿。

用膳过后也是无事。皇帝的心思都在前朝,还顾不上后宫,顾不上尚无名分的她们。她的欢喜时光,也是寂寞。青樱只能遐想着,想着皇帝在前朝的意气风发,居万人之上。他有抱负,有激情,有对着这片山河热切的向往。她想得出他嘴角淡而隐的笑容底下是有怎样的雄心万丈。

这样痴想着,殿门被轻巧推开,阿箬瘦削的身子闪进来,轻灵得唯见青绿色的裙裾如荷叶轻卷。她在青樱耳边低语几句,青樱神色冷了又冷,强自镇定道:``谁告诉你的?''

阿箬的声音压得极低,语不传六耳,``老主子身边还有一个宫女叫绣儿的,是老主子带进宫的心腹。她偷偷跑来告诉奴婢,说老主子不大好,一定要见您一面。''她见青樱神色沉重如欲雨的天气,急忙劝道,``奴婢多嘴劝小主一句,不去也罢。''

青樱转着手指上的珐琅猫眼晶护甲,那猫眼晶上莹白的流光一漾,像是犹豫不定的一份心思。青樱迟疑着问:``怎么?''

阿箬蹙眉道:``老主子是太后的心腹大患。若是让太后知道,哪怕不是太后,是宫里任何一个人知道,对小主都是弥天大祸,在劫不复。何况老主子对小主您实在算不得好。''她沉吟又沉吟,还是说,``小主自重。''

青樱这位姑母,待青樱实在是算不上好。但,是她给了自己家族的荣华安逸,是她阴差阳错引了自己嫁了今日的郎君。青樱有成千上万个理由不去见她,但是最后,她还是迟疑着起身了。

夜路漫漫,她是第一次走在紫禁城夜色茫茫的长街里。阿箬在前头提着灯,青樱披着一身深莲青镶金丝洒梅花朵儿的斗篷,暗沉沉的颜色本不易让人发现。要真发现了,也不过以为她是看别的嫔妃罢了。

东一长街的尽头,过了景仁门,往石影壁内一转,就是景仁宫。角门边早有宫女候着,见她来了也只是一声不问,开了角门由她进去。阿箬自然是被留在外头了。青樱走进阔朗的院中,看着满壁熟悉的龙凤和玺彩画,眼中不由得一热。

这个地方,是曾经来熟了的。可是如今再来,备感凄凉。住在这儿的曾经最尊贵的女子早已了失了恩宠失了权势,如同阶下囚一般。她有万千个不踏进这里的理由,却还是来了。

因为她们的身上,流着一样的血。

她迟疑片刻,踏着满地月色悄然走进。身后有在地上啄食米粒的鸽子,像是跳跃着的白色幽灵,只顾着贪吃,并不在意她的到来。甚至,连一丝扑棱也没有。或者,比起殿中的人,它们才更像这景仁宫的主人。

青樱推开沉重的雕花红漆大门,宫室里立刻散发出一股久未修葺打扫的尘土气息,呛得她掩住了口鼻。

殿中并没有点过多的烛火,积了油灰的烛台上几个蜡烛头狼狈地燃着,火头摇摇欲坠,好像随时都会灭去。借着一缕清淡月光照进,她辨认片刻,才认出那个坐在凤座上的身影,似足了她的姑母。

她轻声唤道:``姑母。''

那人缓缓站起身来,如一阵阴影逼到她跟前,森森道:``原来你还肯来?''

青樱沉沉点头,``割开肉,掰开骨,我和姑母流着的血都是乌拉那拉氏的。''

那人笑了笑,声音如同夜枭一般嘶哑低沉,``好。不管从前怎么样,有你这句话,我叫你来是对的。''

青樱被她的笑声激起一身战栗,她仔细打量着眼前人,心下密匝匝地刺进无数的酸楚与感慨,低声道:``姑母,您见老了。这些年,叫您受苦了。''

可不是老了?当年乌拉那拉氏虽不算一等一的貌美,也是端然生华的六宫之主。

乌拉那拉氏干脆地笑了一声,冷道:``我虽老了,你还年轻,这才是最要紧的。''

青樱犹豫片刻,还是道:``姑母,今日登基的,是弘历。太后的养子。''

乌拉那拉氏仰天笑了片刻,笑得眼角都沁出泪来。``恭喜啊恭喜,你也算如愿以偿,修得善果了。''她脸上忽然一冷,面色有些凄厉的狰狞,``谁登基谁做皇帝,谁做太后谁做阶下囚,都不必你来说了。今日钮祜禄氏来见过我,她告诉我,新帝会追封我的姐姐,先帝前头的福晋为孝敬皇后,我一生所做的德行,都会记在她身上。钮祜禄氏是成全了先帝的心愿,我姐姐死了,只当她是活着。而我呢,而我呢,不入史册,不附太庙,来日以无名无姓的先帝嫔妃的身份下葬。无声无息,我就成了后宫里一介尘烟,风吹过就散了,半点不留下痕迹。好啊好,好狠毒的钮祜禄氏!这样的狠毒,青樱,你可要好好学着!''

青樱惊得背心寒毛阵阵竖起,整个人定在原地,只觉得冷汗涔涔而下,如细小的虫子慢悠悠爬过,所过之处,又是一阵惊寒。

乌拉那拉氏轻蔑地瞟她一眼,``这般无用,我是白费了心思叫你来了。看来还是如从前一般,心浮气躁,不成大器。''

青樱回过神来,勉强镇定着道:``成不成大器,我能有今日,是姑母的功劳。''

乌拉那拉氏看了青樱一眼,徐徐道:``功劳?当年三阿哥弘时一时糊涂,不肯娶你为福晋,让你受辱,你心中自然不忿。我要你暂忍屈辱,先居格格之位侍奉在侧,以图后算,你也以为受辱,不肯屈就。''

青樱默默片刻,沉声道:``虽然都是妾室,但三阿哥无意于我,只钟情先帝的瑛贵人,才招来弥天大祸。未曾嫁给三阿哥,是我的运气。嫁给四阿哥,我也从未后悔。''

乌拉那拉氏眼皮也不抬,``可是嫁个弘历为侧福晋,你就心满意足了吗?到底,侧福晋也好,格格也好,都只是妾室而已。''

青樱想起弘历,只觉万般郁结都松散开来,只余如蜜清甜。``皇上对我颇为钟爱,三阿哥只视我如无物。情分轻重,青樱自然懂得分辨。''

乌拉那拉氏笑了笑,语气酸涩。``身在帝王家,谈论情分,岂不可笑?''她见青樱只是不以为然的样子,不觉叹了口气,``你这个年纪,自然是不能明白的。也好,不明白总有不明白的好处,自以为安乐,何尝不也是一种安乐呢。只是青樱\ldots\ldots 从今日起,你可再不是王府的侧福晋了,皇宫深苑,又岂是区区一个王府可比?''

青樱想起这几日境遇,不觉也有些蹙眉。乌拉那拉氏打量她神色,淡淡道:``怎么?才进宫,名分尚未定,就波澜顿生了?''

青樱望着乌拉那拉氏,屏息敛神,郑重下拜,``青樱愚昧,还请姑母赐教。''

乌拉那拉氏冷笑,``难得,我这个败军之将,一个为先帝所厌弃至死的弃妇,还有人来请我赐教。''

青樱俯身,``姑母虽然无子无宠,但皇后之位多年不倒。若非因为太后,今日凤座之上或许是您。哪怕您今日困坐深宫,也一定有青樱百般难以企及之处。''

乌拉那拉氏别过头,``当年你姻缘不谐,成为宫中笑柄,难免不记恨我?如今你又是钮祜禄氏的儿媳妇,我又何必要教你?''

青樱沉吟片刻,诚恳望着乌拉那拉氏,``因为姑母与我,都是乌拉那拉氏的女儿。''

乌拉那拉氏望着窗外,深黑的天色下,唯见她黯然面容。乌拉那拉氏声音微哑,``如今,我不是大清的国母,不是先帝的皇后,更不是谁的额娘。我剩下的唯一身份,只是乌拉那拉氏的女儿。''她停一停,沉声说,``当年孝恭仁太后告诉我,乌拉那拉氏的女儿是一定要正位中宫的,如今我一样把这句话告诉你。你,敢不敢?''

心头的惊动乍然崛起,她被惊得后退几步,不免生了几分怯意,低低道:``青樱不敢妄求皇后之位,只求皇上恩爱长久,做个宠妃即可。''

乌拉那拉氏唇角扬起讥笑,``宠妃?除了拥有宠爱,还有什么?宠妃最大的优势不过是得宠,一个女人,得宠过后失宠,只会生不如死。''乌拉那拉氏冷冷扫她两眼,``咱们乌拉那拉氏怎么会有你这样目光短浅之人?''

青樱满脸都觉得烧了起来,讪讪地垂着手立着,不敢说话。

乌拉那拉氏道:``等你红颜迟暮,机心耗尽,你还能凭什么去争宠?姑母问你,宠爱是面子,权势是里子,你要哪一个?''

宠爱与权势,是开在心尖上最惊艳的花,哪一朵,都能艳了浮生,惊了人世。青樱思忖片刻,暗暗下了决心,``青樱贪心,自然希望两者皆得。但若不能,自然是里子最最要紧。''

乌拉那拉氏颔首,``这话还有点出息。人云宫门深似海,立足艰难。何况你又是我的侄女儿,要在后宫立足,只怕更是难上加难。''

青樱被说中心事,愈加低头。片刻,她抬起头来,大声道:``虽然难,但青樱没有退路,只能向前。''

乌拉那拉氏眼中精光一闪,终于露出几分欣慰的神色,缓缓伸出手扶起青樱,``要在后宫立足,恩宠、皇子,固然不可少。但是青樱,你要隐忍,更要狠心。斩草除根,不留后患。干净利落,不留把柄。你要爬得高,不是只高一点点。你高一点点,人人都会妒忌你谋害你;可是当你比别人胜出更多,筹谋更远,那么除了屈服和景仰,她们更会畏惧,不敢再害你。''

青樱有些懵懂,乌拉那拉氏看她一眼,并不理会,继续道:``后宫之中,人人都想有所得,不愿有所失。可是青樱,你要明白,当一个人什么都可以舍弃之时,才是她真正无所畏惧之时。''乌拉那拉氏颇为欷歔,``我的错失,就是太过于在乎后位,在乎先帝的情分,才会落得如此地步。''

青樱若有所悟,``姑母所言是无欲则刚?''

乌拉那拉氏略略点头,冷然道:``我所能教你的,只有这些了。败军之将的残言片语,你觉得有用就听,无用过耳即忘就是。时候不早,你走吧,惹人注目的话,明朝或许就是死期了。''

青樱起身告退,``青樱先走,将来若是方便,还会再来探望姑母。''

乌拉那拉氏漠然道:``不必了,再见也是彼此麻烦。''

青樱无言,``太后没有说如何处置姑母。姑母安心避居一些时日再说吧。''

乌拉那拉氏扬起下颌,骄傲道:``我是堂堂大清门走进的皇后,难道还要听她处置?还是你自己自求多福吧。''

青樱默默拜别,只身出去。快到殿门口时,乌拉那拉氏忽然唤了一声,``青樱。''那声音似乎有些凄厉,青樱心中一颤,立刻转过头去,乌拉那拉氏凄然欲落泪,``乌拉那拉氏已经出了一个弃妇,再不能出第二个弃妇了!你\ldots\ldots{}''

那是一个女人一生的泣血之言啊!

青樱忍着泪,无比郑重,``青樱明白。''

乌拉那拉氏旋即如常般淡然,慢慢走上凤座,端坐其上,静静道:``你要永远记得,你是乌拉那拉氏的女儿。''

青樱鼻中一酸,只觉无限慨然。宝座之上的乌拉那拉氏早已年华枯衰,却依然风姿端华,不减国母风采。青樱情不自禁拜身下去,叩首三次,转头离去。

阿箬候在长街深处,本是焦急得如猫儿挠心一般,见青樱出来,才松了一口气,``小主,你终于出来了。''

青樱忙问:``没人瞧见吧?''

阿箬点头,``没人。''她急急拿披风兜住青樱,扶住青樱的手往前走。

两人急急忙忙走着,也不知道走了多远,才觉得提着的一颗心稍稍放了下来。阿箬才敢问:``老主子突然要见小主,到底是什么事?''

夜风幽幽,吹起飞扬的斗篷,恍若一只凄惶寻着枝头可以栖落的蝶。青樱缓住脚步,远远望见深冷天际寒星微芒,只觉无尽凄然,低低说:``这\ldots\ldots 恐怕是我和姑母的最后一面了。''

阿箬大惊,``老主子她\ldots\ldots{}''

青樱含泪道:``姑母的性子怎肯屈居人下,又是折辱自己的人。宁肯玉碎,也绝不瓦全。''

她望着长街幽狭的墨色天空,极目远望,前朝的太和殿、中和殿、保和殿犹自热闹非凡,五颜六色的烟花绚烂飞起在紫禁城无边无尽的黑沉夜空里,整个夜空几乎被照得亮如白昼,连一轮明月亦黯然失色。不知哪来的一只寒鸦,怕是被绚丽的烟火受了惊,拍着乌沉沉的翅膀,呀呀地飞远了。

青樱忍不住落泪,俯下身体,朝着景仁宫方向深深拜倒。阿箬被她的举动吓了一跳,赶紧搀住她,``小主,地上的砖凉,您小心身子。''青樱扶住她的手霍然起身,再不回顾。

阿箬悄悄看青樱,只见她神色清冷如霜,脸上再无一点泪痕。天际烟花绚烂缤纷的光彩照过重重赤红宫墙,千回百转照映在她脸上,愈显得她肤色如雪,沉静如冰。

\hypertarget{ux7b2cux4e03ux7ae0-ux6c42ux5b58}{%
\chapter{第七章 求存}\label{ux7b2cux4e03ux7ae0-ux6c42ux5b58}}

青樱入殿时,太后正坐在大炕上靠着一个西番莲十香软枕看着书。殿中的灯火有些暗,福姑姑正在添灯,窗台下的五蝠捧寿梨花木桌上供着一个暗油油的银错铜錾莲瓣宝珠纹的熏炉,里头缓缓透出檀香的轻烟,丝丝缕缕,散入幽暗的静谧中。

太后只用一枚碧玺翠珠扁方绾起头发,脑后簪了一对素银簪子,不饰任何珠翠,穿着一身家常的湖青团寿缎袍,袖口滚了两层镶边,皆绣着疏落的几朵雪白合欢,陪着浅绿明翠的丝线配着是花叶,清爽中不失华贵。她背脊挺直,头颈微微后仰,握了一卷书,似乎凝神端详了青樱良久。

青樱福了福身见过太后,方才跪下道:``深夜来见太后,实在惊扰了太后静养,是臣妾的罪过。''

太后的神色在荧荧烛火下显得暧昧而浑浊,她随意翻着书页,缓缓道:``来了总有事,说吧。''

青樱俯身磕了个头,仰起脸看着太后,``请太后恕罪,臣妾方才夜入景仁宫,已经去看过乌拉那拉氏了。''

青樱微一抬眼,看见在旁添灯的福姑姑双手一颤,一枚烛火便歪了歪,烛油差点滴到她手上。太后倒是不动声色,轻轻地``哦''了一声,只停了翻书的手,静静道:``去便去了吧。亲戚一场,骨肉相连,你进了宫,不能不去看看她。起来吧。''

青樱仍是不动,直挺挺地跪着,``臣妾不敢起身。乌拉那拉氏乃是先帝的罪妇,臣妾未等禀告,擅自漏夜看望,实在有罪。''

太后淡淡道:``看都看了,再来请罪,是否多此一举?''

太后声音虽轻,语中的沉疾之意却深沉可闻。有清风悠然从窗隙间透进来,殿外树叶随着风声沙沙作响,不知不觉间秋意已经悄无声息地笼来。

青樱不自觉地耸了耸身子,``不是多此一举。是因为无论今时,还是往后,太后都是后宫之主。''

``后宫之主?''太后轻轻一嗤,撂下手中的书道,``哀家老了,皇帝又有皇后,不是该皇后才是后宫之主吗?''

青樱寥寥相应,``您是皇上的额娘,后宫里毋庸置疑的长辈。''

太后目视四周,轻叹一声,``可惜啊!委屈你来了这里见哀家,这儿是寿康宫,可不是正经太后所居的慈宁宫。''

青樱即刻明白,慈宁宫新翻修过,是后宫的正殿。而寿康宫,一切是简陋了不少。她即刻道:``皇上刚登基,事情千头万绪,难免有顾不到的地方。但总也是因为亲疏有别,外头的事多少臣民的眼睛盯着,一丝也疏忽不得,都是加紧了办的。里头是皇上的亲额娘,稍稍耽误片刻,只要皇上的孝心在的,太后哪里有不宽容的呢?到底是至亲骨肉啊!''

太后的眼睛有些眯着,目光却在荧荧烛火的映照下,含了朦胧而闪烁的笑意,``你这番话,既是维护了皇帝,也是全了哀家的颜面。到底不枉哀家当年选你为皇帝的侧福晋。只是你这番话,不知道是不是皇帝自己的心意呢?''

青樱咬了咬唇,闭目一瞬,很快答道:``皇上忙于朝政,若一时顾不到,那就是后妃们的职责,该提醒着皇上。''

``这就是了。''太后看了青樱两眼,温和道,``虽然你是先帝与哀家钦赐给皇帝的侧福晋,身份贵重,潜邸之时亦是侧福晋中第一,比生了三阿哥的苏氏、后来才从格格晋为侧福晋的高氏都要尊荣。可是如今,却不一样了\ldots\ldots{}''

青樱愈加低头,神色谦卑,``臣妾自知为乌拉那拉氏族人,景仁宫乌拉那拉氏有大罪,臣妾为之蒙羞,若能在皇上身边忝居烹茶添水之位,已是上苍对臣妾厚爱了。''

太后扬一扬脸,不置可否,片刻,方低声说:``福珈,你扶青樱起来说话。''

福姑姑伸手要扶,青樱慌忙伏身于地,``臣妾不敢。臣妾有罪之身,不敢起身答太后的话。''

太后微微叹一口气,柔声道:``青樱,你姑母是你姑母,你是你。虽然你们都是乌拉那拉氏之人,但先帝的孝敬皇后就是皇后,乌拉那拉皇后是罪妇,而你是新帝的爱妃。个中关系,哀家并没有糊涂。''

青樱眼中一热,稍稍安心,``多谢太后垂怜。''

太后微笑,``当年是哀家做主请先帝赐你为皇帝的侧福晋,如今自然也不会因为乌拉那拉皇后而迁怒于你。''她稍稍一停,笑意暗淡了三分,``人死罪孽散,乌拉那拉氏幽禁多年,是不久于人世的人了。哀家活到这个年纪的人了,难道还看不破吗?''

青樱终于敢抬头,再次叩首,热泪盈眶,``多谢太后恕罪。''

太后瞥了青樱一眼,``还不肯起来吗?你初居宫中,哀家就要让你长跪,岂不让那些无端揣测是非之人以为哀家迁怒于你?日后,你又要在宫中如何立足?''

青樱脑中一懵,全然一片雪白。当时脑中一热,只求请罪避嫌,竟未曾想到这一层。青樱呆在当地,只觉太后目光明澈,自己手足无措,只能由着福姑姑扶起自己按在座上。

太后目光一转,只打量着青樱,``新帝潜邸中的那些人,除了你和新后富察氏,还有格格珂里叶特氏,其余都是汉军旗。富察氏和你出身高贵,其他的人就不用说了。可是新帝登基,自然要求满汉一家,所以高氏虽然在潜邸时位分不如你,但是如今在后宫,却不得不多赏她几分脸面了。而且高氏的父亲高斌,也是皇帝所倚重的能臣。''

青樱一怔,心中渐渐有些明白,立刻起身,恭谨道:``臣妾与高姐姐原如姐妹一般,高姐姐贤惠端雅,处处教导臣妾,自然该居臣妾之上。''

太后道:``教你受委屈了。可是有些委屈,你既来了这里,就不得不受。昨日午膳哀家驳你的面子,就是为了这个理儿。以后这样的委屈,即便哀家不给你受,你也少不了的。''

青樱低首含胸,诚恳道:``太后肯教导臣妾,臣妾怎会委屈。''

太后似笑非笑,似有几分不信,只斜靠着软枕,拔下发间的银簪子拨了拨灯芯。

青樱笑一笑,只觉得心里空落落的,此刻大方也不是,客气也不是,左右为难,到底露出了几分小儿女情态,``太后,臣妾明白皇上为难,后宫比不得潜邸。可是皇上应该自己和臣妾说,请太后来安慰臣妾,固然是皇上看重臣妾,可也显得臣妾忒不明理了。''

太后这才笑起来,温煦如春风。``你到底才十八岁。若是太贤惠了,也不像个真人儿了。''太后目光锐利一扫,``你那位罪妇姑母,就是贤惠太过了。''

青樱身体一凛,只觉得悚然。

太后道:``你们小夫妻一心,你肯体谅就最好。自然,新帝在潜邸时一直宠爱你,你另一位姑母也是先帝的孝敬皇后。所以了,哀家与皇帝也不会委屈你。''

青樱心中说不出是感泣还是敬畏,只望着太后,坦诚道:``有太后这句话,臣妾就不算委屈。''青樱福一福身,``臣妾还有一事求告太后,青樱之名,乃臣妾幼年之时所取。臣妾觉得\ldots\ldots 这个名字太不合时宜。''

太后微眯了眼睛,``不合时宜?''

青樱有些窘迫,``是。樱花多粉色,臣妾却是青樱,所以不合时宜。''青樱仔细窥着太后神色,鼓足勇气,``何况\ldots\ldots 臣妾是乌拉那拉氏的女儿,更是爱新觉罗的儿媳,恳请太后亲赐一名,许臣妾割断旧过,祈取新福。''

太后凝神片刻,``你这样想?''

青樱恳切望着太后,``若太后肯赐福\ldots\ldots{}''

太后托腮片刻,沉吟道:``你最盼望什么?''

青樱一愣,不觉脱口道:``情深意重,两心相许。''话未完,脸却烫了。太后微微震惊,颇有些动容,姣好如玉的脸上分不清是喜还是悲。良久,她轻声道:``如懿,好不好?''

``如意?''青樱细细念来,只觉舌尖美好,仿似树树花开,真当是岁月静好。``可是事事如意的意思?''

太后见青樱沉吟,亦微笑,``如意太寻常了。哀家选的是懿德的懿,意为美好安静。《后汉书》说`林虑懿德,非礼不处'。人在影成双,便是最美好如意之事。这世间,一动不如一静,也只有静,才会好。''

青樱欢喜。``多谢太后。''她微微沉吟,``只是臣妾不明白,懿便很好,为何是如懿?''

太后的眉间的沉思若凝伫于碧瓦金顶之上的薄薄云翳,带了几分感慨的意味,``你还年轻,所以不懂这世间完满的美好太难得,所以如懿便很不错。''

青樱心头一凛,恍若醍醐灌顶,瞬间清明。``太后的意思是完满难求,有时候退而求其次便是满足。''她深深叩首,``太后的教诲,臣妾谨记于心。''

太后微微颔首,含了薄薄一缕笑意。``好了。夜深,你也早些回去歇息。今日就是新帝登基之日,为先帝伤心了这些日子,也该缓缓心思迎新帝和你们的大喜了。''

青樱起身告辞。太后见青樱扶了侍女的手出去了,才缓缓露出一分笃定的笑容。福姑姑为太后披上一件素锦袍子,轻声道:``移宫的事儿,太后嘱咐皇后一声就行了,或者月小主如今得皇上的器重爱惜,她去说也行。青樱小主\ldots\ldots 不,是如懿小主的身份,不配说这样的话。''

太后拾起书卷,沉吟道:``你真当她不够聪明吗?从前是家世显赫,被宠坏了的小姐脾气,不知收敛。从乌拉那拉氏被幽禁至今,世态炎凉,还不够打磨她的吗?凭她今日去见了乌拉那拉氏还敢来回哀家,这就是个有主意的丫头了。''

福姑姑迟疑道:``太后是说,她明知宫中人多眼杂,万一将来露了去景仁宫探望的事要遭祸患,所以先来向太后请罪?''

太后道:``宫里除了哀家,还有谁最介意乌拉那拉氏?只要哀家不动气,旁人也就罢了。且她事事撇清,请哀家赐名,又表明心意,只说是爱新觉罗家的儿媳,就是为了消哀家这口气,更是为了求她自己一己存身之地。''

福姑姑叹息道:``昔年乌拉那拉氏那样凌辱太后,这口气一时如何能消得掉?''

``不管消不消得掉,她要求的是安稳。宫里有皇后,又有高月新宠当道,她的日子不好过。若哀家再不放松她些,她就真当是举步维艰了。就因为这样,她才会想方设法去皇帝面前提移宫的事,也会想方设法做好,不容有失。而皇后既有地位,又有皇子;高月有恩宠有美貌,她们什么都不用向哀家求取,自然不会用心用力了。''

福姑姑恍然大悟,``所以太后才会容得下如懿小主。''

太后凝眉一笑,从容道:``容不容得下,就且看她自己的修为了。''

第二日晨起是个晴好天气,富察氏带着一众嫔妃来寿康宫请安。虽然名分尚未确定,但富察氏的皇后是绝无异议的,众妃只按着潜邸里的位分,鱼贯随入。

太后见天朗气清,心情也颇好,便由诸位太妃陪坐,一起闲聊家常。见众人进来,不觉笑道:``从前自己是嫔妃,赶着去向太后太妃们请安。转眼自己就成了太后太妃了,看着人家年轻一辈儿进来,都娇嫩得花朵儿似的。''

月嘴甜,先笑了出声,``太后自己就是开得最艳的牡丹花呢,哪像我们,年轻沉不住气,都是不经看的。''

太妃忍不住笑道:``从前月过来都是最温柔文静的,如今也活泼了。''

月笑着福了福,``从前在王府里待着,少出门少见世面,自然没嘴的葫芦似的。如今在太后跟前,得太后的教诲,还能这么笨笨的吗。''

太妃笑着点头道:``我才问了一句呢,月就这么千伶百俐的了,果然是太后调教得好。''

太后微微颔首,``好了,都赐座吧。''

众人按着位次坐下。正嘘寒问暖了几句,太后身边的贴身太监成公公进来,远远垂手站着阶下不动。

太后扬了扬眉,问:``怎么了?''

成公公上前,打了个千儿道:``回太后娘娘的话,景仁宫娘娘殁了。''

话音未落,如懿心头一颤,捧在手里的茶盏一斜,差点洒了出来。心眼疾手快,赶紧替她捧住了。

月坐在如懿旁边,立时看见了,伸手扶了扶鬓边缠丝镶珠金簪,朗声道:``到底是一家人连着心,才听了一句,青樱妹妹就伤心了呢。''

太后也不理会,只定定神道:``什么时候的事?''

成公公回道:``是昨日半夜,心悸而死。宫女发现送进去的早膳不曾动,才发现出了事。来报的宫女说她身子都僵了,可是眼睛仍睁得老大,死不瞑目呢。''

如懿双手发颤,她不敢动,只敢握紧了绢子死死捏住,以周身的力气抵御着来自死亡的战栗。昨日半夜,那就是自己走后不久。姑母,当真是不行了,她自己明白,所以一定要见自己那一面,将一切都叮嘱了她,托付了她。

太妃摇了摇头,嫌恶道:``大好的日子,真是晦气!''

太后默然片刻。``该怎么做便怎么做吧。皇帝刚登基,这些事不必张扬。''她看一看如懿,``正好如懿你也在。你姑母过世,你也当去景仁宫致礼。''

如懿忙扶着椅子站起身子,强逼着自己站稳了,忍住喉中的哽咽,``臣妾只知寿康宫,不知景仁宫。且乌拉那拉氏虽为臣妾姑母,但更是大清罪人,臣妾不能因私忘公。所以这致礼之事,臣妾恕难从命。''

太后长叹一声,``你倒公私分明。罢了,你是皇帝身边的人,刚到宫里,这不吉的事也不宜去了。''

琅华听到这里,方敢出声:``敢问皇额娘一句,皇额娘怎么唤青樱妹妹叫如懿呢?''

太后微微一笑,``那是哀家昨夜新赐的名字,乌拉那拉氏如懿,凡事以静为好。''

琅华含笑道:``那是太后疼如懿妹妹了。''

太后微微敛容,正色道:``今日是皇帝登基后你们头一日来寿康宫请安。哀家正好也有几句话嘱咐。皇上年轻,宫里妃嫔只有你们几个。今后人多也好,人少也好,哀家眼里见不得脏东西,你们自己好自为之,别做出伤天害理的事来。''

众人一向见太后慈眉善目,甚少这样郑重叮嘱,也不敢怠慢,忙起身恭敬答道:``多谢太后教诲,臣妾们谨记于心。''

如懿一直到踏出了寿康宫,仍觉得自己满心说不出的战栗难过,却不得不死死忍住。举目望去,满园的清秋菊花五色绚烂,锦绣盛开,映着赭红烈烈犹如秋日斜阳般的红枫,大有一种春光重临的美丽。可是这明丽如练的秋色背后,竟是姑母泣血一般的人生之后所余下的苍白的死亡。

明知一别,却不曾想是这样快。然而除了自己,姑母生活了一世的幽深宫苑里,还有谁会为她动容。深宫里的生死,不过如秋日枝头萎落的一片黄叶而已。那会不会,也是自己的一生?

如懿这样想着,忍不住打了个激灵。心吓得赶紧按住她的手,``小主,千万别露了什么神色。''

如懿紧紧地握着心的手,像是要从她的薄而温热的手心获取一点支撑的勇气似的。她轻声吩咐:``回宫。心,我要回宫。''

话音未落,却听月的声音自枫叶烈烈之后转过,即刻到了耳畔:``妹妹好狠的心,得了太后的赐名,连姑母的丧仪都不肯去致礼了,自己撇得倒干净。''

如懿心头如针刺一般,强忍着笑转身,``原来月姐姐这样有心。记得当年姐姐嫁入潜邸时,也是去拜见过姑母的呢。既有姐姐做主,不如姐姐陪我一起去景仁宫行个礼,也当是全了孝心。''说罢,她便伸手去挽月。

月如何肯去,倏地缩回手,冷笑道:``妹妹的亲姑母,自己惦记着就是了。何必扯上我,我是皇家的儿媳,可不是乌拉那拉氏家的女儿。''

如懿含了一缕澹静笑意,``那就是了。我和姐姐何尝不一样,离了母家,就是皇家的儿媳。生在这儿,说句不吉利的,来日弃世,也只能是在这儿。所以别的人别的事,与我们还有什么相干呢?''

月扬了扬小巧的下巴,``也算妹妹你识趣了。只是妹妹要记得,哪怕你撇得再干净,到底你也是姓乌拉那拉氏的,这是谁也改变不了的事。只怕太后听见这个姓氏,就会觉得神憎鬼厌,恨不得消失才好。''

如懿毫不示弱,泠然道:``既然姐姐这么喜欢揣测太后的心思,不如陪妹妹再去一趟寿康宫,问问太后的意思,好吗?''

月好看的远山眉轻微一蹙,冷笑一声。``我此刻要去陪主子娘娘说话,没空陪你闲话。''她扶过侍女的手,``茉心,我们走!''

如懿见她走远,脚下微微一软,花盆底踩在脚心,便有些不稳当。心和阿箬忙扶了她往近旁的澄瑞亭中坐下。如懿倚在碧色栏杆上,以睫毛挡住即将滑落的泪水,缓了缓气息道:``心,你说姑母会不会怪我?''

心替她抚着背心,轻声道:``小主所行,必是景仁宫娘娘所想。否则,小主便是辜负景仁宫娘娘的一片心了。''

如懿闭目片刻,将所有的泪水化作眼底淡薄的蒙眬,静静道:``你说的话,正是我的心意。''

阿箬陪侍在侧,看如懿一言一问只看着心,不觉暗暗咬了咬牙,脸上却不敢露出什么来。

如懿扬了扬手,``你们到亭外伺候,我想静一静。''

阿箬与心忙告了退,走到亭外数十步。阿箬本走在后头,突然往甬道上一挤,心一个不当心,差点被路旁的花枝划了眼睛,忙站住了脚道:``阿箬姐姐。''

阿箬闻声回头,哼道:``自己走路不当心,还要来怪我吗?''

心忙赔笑道:``怎么会呢?我是想说,早上起了露水,甬道上滑,姐姐仔细滑了脚。''

阿箬皱了皱眉头。``自己笨手笨脚的,以为都跟你一样吗?''她横了心一眼,``就会在小主面前抓乖卖巧,明明昨夜是我冒险陪了小主去的景仁宫,小主偏偏每句话都问着你,好像这么危险的差事都是你伺候了。''

心忙欠身笑着道:``正因为我伺候小主不如姐姐亲厚,所以小主才问我呀。姐姐细想,姐姐是小主的贴身人,想什么说什么都是和小主一样的,小主又何必再问。就是我呆呆笨笨的,小主才白问一句罢了。我这么想的,肯定外头那些不知情的,更都是这么想的了。这样小主才能放心呀。''

阿箬这才稍稍消气,抬了抬手上的金绞丝镯子,``你看看这个镯子哪,是小主新赏给我的。别以为你伺候小主的时候多,亲疏有别,到底是不一样的。''

心诺诺答了``是''。两人正守在一旁,忽然见亭中如懿已经站起身子,忙回身过去伺候。

如懿问道:``这个时候,皇上在哪里呢?''

阿箬掰着指头道:``这个时候皇上已经下朝,也过了见大臣的时候,怕是在养心殿看书呢。''

如懿点点头,``去备些点心,我去见过皇上。''

养心殿里皇帝自己的小书房在西暖阁的末间。地方虽不大,却布置得清雅肃穆,窗明几净。里头满架子的书卷整整齐齐地放着,都是皇帝素日爱读的那些。东板墙上疏疏朗朗地挂着十几只壁瓶,有龙纹、高士、八仙、松竹梅、芦雁、折枝花果、雉鸡牡丹等等图样,多选淡雅温润的豆青色,更觉触目清爽。

皇帝身边的大太监王钦替如懿打了帘子进来。想来是刚刚换过家常衣衫,身上是一袭月白色纱缀绣八团夔龙单袍,皇帝闲闲捧一卷书在手,淡金色的澄澈秋阳自雪白的明纸窗外洒落全身,任由光晕染出一身清绝温暖的轮廓,紫铜嵌珐琅的龙纹香炉里燃着琥珀似的龙涎香,整个屋子里弥漫着龙涎香幽宁沉郁的气味,也变得幽幽袅袅,衬着满架书香,倒像是一轴笔法清淡的写意画卷。

皇帝见如懿穿着一身月白缎织彩百花飞蝶袷衬衣①,月白素净的妆花缎面上,以大红、粉红、碧绿、草绿、香黄、驼黄、浅绛、湖蓝、深灰、浅黑、淡白等十余种色线织成点点折枝花卉及虫蝶纹样,虽然素净,却不失华艳。

他仰起身笑道:``你倒巧,都与朕穿了一样的颜色。''

如懿含笑行礼,``没有打扰了皇上读书,就算是巧了。''

皇帝搁下书,朝她招招手,``过来坐。''见如懿在榻边坐了,方才笑道,``朕刚登基,前朝的事没个完,一直不得空去看你们。如今你过来,倒也正好。''他嗅了嗅,见如懿身后的心手里捧着一个红箩小食盒,``带了什么好吃的,好香!''

如懿扬一扬脸,示意心一样样取出来,不过是四样小点心,糖蒸酥酪、松子穰、藕粉桂糖糕和玫瑰山楂馅儿的山药糕。

皇帝笑道:``朕正好有些饿了,陪朕一起用一点。''

如懿取了银筷子出来,递到皇帝手中,笑道:``臣妾本想备四样点心,谁知宫里只备了三样现成的。这一味藕粉桂糖糕还是太后赏赐下来的,说皇上原爱吃这个。这两日不得空去寿康宫,所以赏赐给了臣妾,臣妾就正好借花献佛了。''

皇帝夹了一筷慢慢吃了,``听说皇额娘给你改了个名字?''

``叫如懿。太后说,懿为美好安静。林虑懿德,非礼不处。所以叫如懿。''

皇帝轻嘘一口气。``皇额娘的性子,朕在她身边多年也摸不清楚。她给你改了名儿,又是这个意思,大概是不会难为你了。''他握一握如懿的手腕,``今儿早上,朕听说景仁宫皇后过身了,原想着你该去看看,但怕太后多心,也不便说什么了。''

如懿低眉一瞬,``臣妾知道,臣妾不去。一去,又是是非,臣妾是爱新觉罗家的人,不该给皇上添是非。''

皇帝点点头,亲手递了一块山药糕给她,``这山药糕酸酸甜甜的,你喜欢这个口味。''

如懿谢过,打量着四周道:``皇上喜欢壁瓶,本可四时插花,人作花伴,取其清芬满床,卧之神爽意快之效,只是如今点着龙涎香,反而不用花草好,以免乱了气味。''

皇帝笑吟吟道:``朕也这样想。所以宁可空着,闲来观赏把玩,也是好的。''

如懿立起身,望着其中一尊瓶身道:``这个图案倒好,不比其他的吉祥图案,倒像个什么故事。''

皇帝笑话她,``老莱子彩衣娱亲,这个你也忘了?''

如懿望一眼书架,又见皇帝案上空着,便笑:``皇上素日常看的那本《二十四孝》,怎么如今不在身前了?''

皇帝随口道:``大概是随手放哪里了,回头让王钦去找找。''

如懿似是凝神想着什么,``皇上,臣妾记得《二十四孝》里第一篇是不是闵子骞单衣奉亲?''

皇帝失笑,``你今儿是怎么了?《二十四孝》第一篇是虞舜孝感动天,第二篇才是闵子骞单衣奉亲。''

如懿敛容道:``皇上心存孝道,自然记得清楚明白。《二十四孝》第一篇便是讲虞舜孝感动天,可见世人心中,总是百善孝为先,更以君王作为其中典范,宣扬孝道。皇上才登基,诸事忙乱,来不及走一趟后宫。''她沉吟片刻,``太后,还住在寿康宫里。''

皇帝扬了扬眉毛,``怎么?内务府不是再三请皇额娘去慈宁宫了吗?怎么还住着寿康宫?''

如懿微微一笑,``照臣妾看,不是内务府办事不力,而是太后存心将这个表示孝道的机会留给皇上您了。''

皇帝静了片刻,柔和笑容带一点疏懒意味,``朕也想让皇太后移居慈宁宫。可是\ldots\ldots{}''如懿会意,示意宫人们退下。阁中只留了皇帝与如懿二人,皇帝方低低说:``可朕心里,总还是有道过不去的地方。''他的目光转向窗外,有些痴惘,``朕的亲生额娘\ldots\ldots{}''

如懿巴望地看着皇帝,按住了他的手,轻轻摇了摇头,坚定道:``皇上的亲生额娘,只有太后,就住在寿康宫,等着皇上请她移住慈宁宫。''

皇帝的目光沉静若深水,``皇太后专宠多年,在朝中与宫中都颇有权势,若在正位慈宁宫,朕怕她会不会\ldots\ldots{}''

``会与不会,都不在于进不进慈宁宫,而在于皇上的魄力与才干。皇上心怀天下,胸中有万千韬略,何惧区区一女子。''如懿定定地望着皇帝,``慈宁宫,只是皇太后名正言顺所居住的一个地方。''她反握住皇帝手,以自己手心的冰凉,慰他掌心的潮热,``皇上,委屈了太后的住所,天下臣民会指责您。而把太后送进了慈宁宫,是点醒了天下人,皇上以天下养太后,请她颐养天年。''

皇帝目光微沉,片刻,露了两分笑意,``那朕,就依你所说,尽心孝敬,请太后颐养天年,好生养息。''

注释:

①衬衣:清代女式衬衣为圆领、右衽、捻襟、直身、平袖、无开禊、有五个纽扣的长衣,袖子形式有舒袖(袖长至腕)、半宽袖(短宽袖口加接二层袖头)两类,袖口内再另加饰袖头。是妇女的一般日常便服。以绒绣、纳纱、平金、织花的为多。周身加边饰,晚清时边饰越来越多。常在衬衣外加穿坎肩。秋冬加皮、棉。

\hypertarget{ux7b2cux516bux7ae0-ux540dux5206ux4e0a}{%
\chapter{第八章
名分(上)}\label{ux7b2cux516bux7ae0-ux540dux5206ux4e0a}}

这一日众人皆到皇后的长春宫中请安,富察氏命人赏了一箩红橘下来,含笑道:``皇上念着咱们后宫,江南进贡的红橘一到,就先挑了一箩送来,正好咱们也一起尝尝。''

众人起身谢恩,``多谢皇后娘娘恩典。''

皇后嘱了众人落座,看莲心和素心分了红橘,方慢慢道:``咱们这些姐妹,都是从前潜邸时便一起伺候皇上的,彼此知道性情。如今进了紫禁城做了皇上的人,一则规矩是定要守的,二则也别拘了往日的姐妹之情,彼此还是有说有笑才好。''

月先站了起来,满面恭谨道:``皇后娘娘从前是臣妾们的姐姐和主子,如今更是天下之母。臣妾们不敢不心存恭敬。''

皇后淡然笑道:``月妹妹言重了。本宫比你们虚长几岁,自然在教导之余,更要好好顾全你们。''

月领着众人起来,``谢皇后娘娘隆恩。''

如懿看着皇后与月一唱一和,只低了头慢慢剥着红橘把玩,面上略含了一缕笑,淡淡不语。

皇后对月的应答甚是满意,含笑点了点头,``你们坐着吃些橘子好好聊聊吧,本宫有些乏了,先回寝殿歇息。''

她停一停,环视众人,``皇上已经拟定了你们的位分,也各自安排了宫室与你们居住。如今皇太后已经先移居了慈宁宫。晌午旨意一下来,就各自搬过去住吧。为着这些日子替大行皇帝哭灵,挤在一块儿住也是为难了你们。''

众人闻言一凛,哪有心思再坐,便纷纷告辞了。

果然到了晌午,皇帝册定位分的旨意遍传六宫。如懿站在廊檐下逗着一双蓝羽鹦哥儿,只听着阿箬掰着指头嘟囔道:``立后大典之后,皇后已经挑了长春宫去住。长春长春,真是个好意头,只盼着皇上春恩长在呢。苏格格新添了三阿哥,封了纯嫔,住在钟粹宫。黄格格封了怡贵人,住在景阳宫,她倒挺高兴的。本来嘛,皇上也不是很宠爱她,给个贵人就不错了。金格格只封了嘉贵人,住在太极殿,她又不高兴又不敢说,只抱怨太极殿离皇上的养心殿太远。金格格一直以为自己的朝鲜宗室女的身份便觉得高人一等,眼下也只不过是个贵人,看她还有什么好神气的。''

如懿取过鸟食撒在鹦哥儿跟前,``你说便说,背后议论人家做什么。''

阿箬吐了吐舌头。``奴婢知道了。另外就是海兰格格了,皇上只封了她常在,也没说住哪个宫,大概位分不高,随便跟着哪个主位住着吧。倒是咱们和高福晋那里,还不知是什么旨意。''阿箬说着往门外看了看,不免有些焦灼,``太阳都快落山了,别的小主那儿都住进新殿去了,怎么咱们这儿还没圣旨来呢?''

如懿心里虽有些着急,却不便在阿箬面前流露出来,便拿给鹦鹉取食的小勺子搅着水。阿箬忙道:``小主,咱们的鹦鹉好干净,拿取食的勺子搅了水,它们就不喝那水了。''

如懿正不耐烦,却见心领着传旨太监王钦进来并两位大臣进来。

王钦打了个千儿道:``启禀小主,圣旨下。大学士礼部尚书三泰为正使,内阁学士岱奇为副使,行册封礼。''

如懿忙忙低首跪下,院子里的人也跟着跪在后头。

王钦取过圣旨,朗声念道:``朕惟教始宫闱,式重柔嘉之范,德昭珩佩,聿资翊赞之功。锡以纶言。光兹懿典,尔庶妃那拉氏,持躬淑慎,赋性安和,早著令仪,每恪恭而奉职勤修内则,恒谦顺以居心。兹仰承皇太后慈谕,以册印封尔为娴妃。尔其祗膺巽命,荷庆泽于方来,。懋赞坤仪,衍鸿休于有永。钦哉。''

如懿双手接过圣旨,``臣妾谢皇上隆恩。''

如懿使个眼色,心忙从袖中取过三封红包,一一交到三人手中。

王钦满面堆笑,``多谢娴妃娘娘赏赐,皇上说了,延禧宫就赐给娘娘居住。请娘娘即刻迁往延禧宫。''

如懿心中一沉,勉强笑道:``多谢公公。阿箬,好生送公公和两位大人出去。''

阿箬答应着,王钦拱手道:``奴才还要去皇上那儿复命,娘娘别忘了明日一早换上吉服去长春宫给皇上和皇后娘娘谢恩。''

如懿颔首道:``有劳公公提醒。''

院中众人尚跪在地上,叩头道:``恭喜娴妃娘娘,娘娘万安。''

如懿道:``本宫乏了,等下阿箬会给你们赏钱,你们再把东西收拾了去延禧宫。''

心忙跟着如懿走到内殿。

如懿屏息静气,问道:``月福晋那儿有消息了吗?''

心低声道:``刚得的消息。月福晋封了慧贵妃,皇上的口谕,贵妃之外戚,著出包衣,入于原隶满洲旗分。果然的满门抬镶黄旗,赐姓高佳氏,贵妃也迁往咸福宫居住了。''

如懿冷笑一声,更觉烦恼不堪,``咸福宫?可不是福泽咸聚吗?''

心柔声劝道:``娘娘别烦恼!延禧宫虽然偏僻,虽然\ldots\ldots{}''心想要宽慰如懿,也觉得皇帝恩义悬殊,实在也无从宽慰起。

如懿摇头道:``延禧宫偏僻却不冷清,旁边就是宫人来往的甬道,嘈杂纷扰。且从康熙爷二十五年之后,足有三十多年未再修葺,乃是六宫之中最破败的宫苑。''如懿不安道,``难道太后和皇上,就厌弃我至此吗?''

心道:``皇上和娘娘多年情分,断不会如此。即便是太后\ldots\ldots 太后不也说不怪罪娘娘吗?''

如懿心中烦乱如麻,``口中所言,只怕是说说而已。算了,此时此刻,我也不能争什么,先收拾了东西去延禧宫吧。''

住进延禧宫中,已经是夜来时分。所幸延禧宫虽然靠近宫人进出的甬道,但关上大门,也还清静。宫中虽不是新修葺的,但前后两进院落各五间正殿,又有东西配殿三间,倒也宽敞。如懿本是喜净之人,宫人们仔细打扫之后,反觉得室内古朴,也不是十分简陋。

如懿往延禧宫中看了一圈,庆幸道:``你们打扫得仔细,总算还不是太差。''

阿箬撇嘴道:``娘娘也太知足了。东西六宫之中,哪一个不比延禧宫好。奴婢瞧着承乾宫、翊坤宫,个个都是顶好的,景致又美,离皇上的养心殿又近。住在这儿,不知道皇上多早晚才来一次呢。''

如懿瞥了她一眼,只看着梁上的雕花叹了口气。

心笑着拉住阿箬道:``好姑娘。皇上要愿意来,不会嫌路远;若是不肯来,哪怕住进养心殿后头的围房,也不济事。''

阿箬正要回嘴,如懿淡淡道:``愿意来的总不在乎远近,满肚子的心思未必要挂在嘴上。阿箬,你说是不是?''

阿箬有些气馁,只诺诺地道:``幸好娘娘搬过来之后,皇上也赏赐了好些东西添补宫里的摆设,皇上心里总是有娘娘的。''

如懿颔首道:``皇上今晚宿在长春宫,咱们也早些安置。新换了地方,也不知道会不会睡得香。''

心眼珠一转,笑吟吟道:``就怕娘娘觉着换了地方睡不香,奴婢已经在寝殿点了安神香了。''

如懿赞许地点点头,阿箬却只是暗暗撇嘴,垂了手站到了后头。

主仆三人正准备往寝殿走,外头守着的小太监进来道:``启禀娘娘,海常在来给娘娘请安。''

如懿不觉诧异,``这个时候,怎么海兰还来请安?快请进来吧。''

如懿方走到西暖阁坐下,海兰已经带着侍婢叶心进来了。

如懿含笑道:``怎么这么晚还来请安?可是长夜漫漫睡不着吗?''

海兰倒不似往日一般,只是拘谨。心斟了茶上来,谦恭道:``海常在请用茶。''

海兰也不喝茶,只是盈盈望着如懿不做声。

如懿暗暗纳罕,便笑道:``妹妹有什么话尽管对我说。对了,今日圣旨到的时候还不知道妹妹住在哪个宫里,不知皇后娘娘可安排了?''

海兰眼圈微微一红,低首道:``嫔妾人微言轻,自然是皇后随手安排到哪里就是哪里了。''

如懿奇道:``是什么地方?难道不好吗?''

叶心忍不住道:``皇后娘娘说慧贵妃的咸福宫宽敞华丽,就指了小主去咸福宫。这本也没什么,可是咸福宫那位向来是不容人的,如今抬了旗,那是更不得了了。譬如怡贵人,就是从前伺候皇后娘娘的侍女。可慧贵妃那里,从前有个丫头在她不方便的时候伺候了皇上,就被她想了法子撵出去了。''

如懿柔声打断,``这也是从前的事了。如今她是贵妃,自然要比从前显得温柔些。''

叶心愤愤道:``我们小主好性儿,总被人欺负。到了咸福宫先听了慧贵妃一顿训,又被拨到了一间西晒的屋子里住。''

如懿闻言皱眉,``那哪里是住人的地方,夏天暴晒,冬天冷得冰窖似的,便是一般的奴才也不住那里,不过就是平日里放放不要紧的东西罢了。慧贵妃也不怕皇上看见吗?''

海兰微微啜泣,``皇上素来就少去嫔妾那里,如今在慧贵妃眼皮子底下,那更是不能了。今日慧贵妃还说,若皇上真问起来,便只说嫔妾自己爱住那里,她还劝不住。嫔妾\ldots\ldots 其实皇上哪里会管嫔妾呢。''

如懿心中不忍,``她既这样待你,那你现在这般出来,她可不忌讳?''

海兰泣道:``她有什么可忌讳的?这会儿咸福宫里不知道多热闹呢,人人都趋奉着她封了贵妃,更抬了旗呢。''

如懿沉吟片刻道:``那你如何打算?''

海兰泪汪汪看着如懿,``嫔妾只敢来求娴妃娘娘恩典,希望能与娘娘同住,便心满意足了。''

如懿忙道:``你素来只叫我姐姐,如今还是叫姐姐。口口声声`娘娘嫔妾',倒生分了。''

海兰怯怯点头,``是。''

如懿想了想道:``你要过来住,也不是不行,只消我回禀皇后娘娘\ldots\ldots{}''

如懿一语未完,心上前道:``娘娘,茶凉了,奴婢再替您换一盏。''

如懿正点头,却见心深深望了自己一眼,也是心知肚明,只得暗暗叹了口气道:``你要过来住,也不是不行,只消我回禀皇后娘娘也就是了。只是你知道我如今的情境,一来不能像以前一般开口向皇后求什么,二来我真求了,皇后也未必会答应。只怕还要怪你不安分守己,若是慧贵妃因此迁怒于你,你以后的日子更不好过。''

心替海兰添了茶水,装作无心道:``其实海兰小主在潜邸时就住咱们娘娘旁边的阁子里,若说和咱们一起住延禧宫那也说得过去。这下子硬生生要分开那么远,真不知是什么道理。''

海兰泪眼迷蒙,低头思忖了片刻,才低低道:``原是我糊涂了,怎好叫姐姐为难呢。''

如懿过意不去,``若是在从前,我没有不帮你的道理。可是眼下,你看看我的延禧宫便知,我实在没有开口的余地。且你搬来延禧宫这种偏僻地方,也未必是好事。若是被我牵连失宠于皇上,就更不好了。''

海兰环视延禧宫,也不觉叹了一口气,``姐姐在潜邸时乃是侧福晋中第一人,何曾住过这样委屈的地方。''

如懿拍了拍她的手,``委屈不委屈,不在于一时。你我都好好的,还怕来日会不好吗。''

海兰拿绢子拭去泪痕,展颜道:``姐姐说的是。''她微微含笑,``从前我在潜邸的绣房做侍女时也被人欺负,是姐姐偶尔看见怜惜我,劝我要争气。后来又将我绣的靴子进献皇上,让皇上宠幸我给我名分。姐姐帮我的,我心里都记得。''

如懿温和道:``好了。你有你的忍耐,我也有我的。咱们都忍一忍,总会过去的。''

海兰这才起身,依依道:``时候不早,妹妹先告退了,姐姐早点歇息吧。''

如懿送至廊檐下,心中略略不安,``慧贵妃若真难为你,你还是要告诉我。再不济总能和你分担一些。''

海兰感激道:``多谢姐姐,我都记得了。''

如懿见海兰和叶心出去,庭院中唯见月色满地如清霜,更添了几分清寒萧索之意,不知不觉便叹了一口气。

心取了披风披在如懿肩上,方才跪下道:``娘娘叹气,可是怪奴婢方才劝阻娘娘?''

如懿摇头道:``你做得对。我自身难保,何必牵连了海兰。''

心道:``从前在潜邸时,慧贵妃的性子并不是这样骄横,倒常见她温柔可人,怎么一入宫就成了这样呢。''

如懿望着庭院青砖上摇曳的枝影,心事亦不免杂乱如此,只是耐着性子道:``得意骄横,失意谦卑乃是人之常情。若能在得意时也能谦和受身,温谨待人,才是真正的修为。''

心沉吟道:``皇上一向称赞娘娘慧心兰性,嘉许慧贵妃娴静温婉,怎么到了今日给娘娘的封号是娴,慧贵妃反而是慧?''

如懿紧了紧披风,淡淡道:``皇上做事别有深意,咱们别胡乱揣测了。''

养心殿书房的明纸窗糊得又绵又密,一丝风都透不进来,唯见殿外树影姗姗映在窗栏上,仿佛一幅淡淡水墨萧疏。

皇帝只低头批着折子,王钦悄声在桌上搁下茶水,又替皇帝磨了墨,方低声道:``皇上看了一个时辰的折子啦,喝口茶水歇歇吧。''

皇帝``唔''了一声,头也不抬。王钦又道:``皇上,张廷玉大人来了,就在殿外候着呢。''

皇帝停下笔,朗声道:``快请进来吧。''

王钦听得这一句,就知道皇帝待张廷玉亲厚,忙恭恭敬敬请了张廷玉进来。张廷玉一进殿门,老远便躬身趋前,端端正正行了一礼,``微臣请圣躬安。''

皇帝微笑道:``王钦,快扶张大人起来,赐座。''

王钦扶了张廷玉起身,养心殿副总管李玉已经搬了一张梨花木椅过来,张廷玉方才敢坐下。

皇帝关切道:``廷玉,你已年过花甲,又是三朝老臣,奉先帝遗旨为朕顾命。到朕面前就不必这样行礼了。''

张廷玉一脸谦恭,``皇上恩遇,微臣却不敢失了人臣的礼数。先帝器重,微臣更要勤谨奉上,不敢辜负先帝临终之托。''

皇帝颔首道:``这个时候,你怎么还进宫求见朕?''

张廷玉欠身道:``皇上封慧贵妃,抬旗赐姓是莫大的荣耀,微臣方才正是从慧贵妃母家大学士高斌府第喝了贺酒回来。''

皇帝``哦''了一声,淡淡道:``这是慧贵妃的荣耀,也是高氏一门的荣耀。连你都贺喜,那朝中百官,想是都去了吧。''

张廷玉不假思索道:``皇上皇恩浩荡,高府宾客盈门,应接不暇。''张廷玉觑着皇帝神色,小心翼翼道,``本来鄂尔泰还和微臣玩笑,说这么多人怕是要踏烂了高府的门槛,想来高大学士思虑周详又见多识广,一早命人换了紫檀木的门槛。''

皇帝微微一笑,似乎不以为意,``紫檀木虽然名贵,但也不算稀罕东西。''

张廷玉越发笑容可掬,``微臣也是这么想,只是今日和内务府主事郎大人闲话,郎大人说这两年紫檀短缺,两广与云南皆无所出,只有南洋小国略有所献,漂洋过海过来,所费不下万金。更难得的是高大学士府上所用的紫檀,入水不沉,高大学士深以为傲,约了百官同赏,臣也是大开眼界。''

皇帝笑着饮了口茶水,唤过王钦道:``朕记得,高斌府上所用的紫檀\ldots\ldots{}''皇上似乎思索,只看了王钦一眼。

王钦一愣,还未反应过来,伺候在殿角的小太监李玉已经抢着道:``回皇上的话,高大人府上所用的紫檀是前两日皇上赏的,为着事多,皇上交代了王公公,王公公嘱咐奴才去内务府办的。''

王钦回转神来,忙拍了拍脑袋。``皇上,瞧奴才这记性,居然浑忘了。''王钦忙跪下道,``还请皇上恕罪。''

皇帝并不看他,只道:``你初入宫当差,大行皇帝身后留下的事情多,忘了也是有的。起来吧。''

王钦松了口气,赶紧谢恩爬起来,擦了擦额头的冷汗。

张廷玉微笑道:``原来是皇上赏的,这是天大的恩典,自然该百官同庆。''他略略思忖,``皇后册封以来,臣一直未向皇后请安,心中惭愧。还盼年节下百官进贺时,可以亲自向皇后娘娘问安。''

皇帝道:``那有什么难的?到时朕许你亲自向皇后问安便是。''

张廷玉再度欠身,``臣谢皇上隆恩。皇后娘娘是先帝亲赐皇上的嫡福晋,皇后娘娘出身于名门宦家,世代簪缨。富察氏又为咱们满洲八大姓之一,为大清多建功勋。臣敬慕娘娘仁慈宽厚,才德出众,能得皇上允许亲自向娘娘问安,乃是臣无上荣耀。''

皇帝微微正色,``你的意思朕明白。皇后乃后宫之主,执掌凤印,朕自然敬爱皇后,不会因宠偏私。''

张廷玉肃然道:``臣听闻前明后宫弭乱,宠妾犯上之举屡屡发生,导致后宫风纪无存,影响前朝安定。皇上英明,微臣欣慰之至。''张廷玉望着皇帝案上厚厚一沓奏折,关切道,``先帝在时勤于朝政,每日批折不下七个时辰。皇上得先帝之风,朝政虽然要紧,也请皇上万万保养龙体,切勿伤身。''

皇帝略有感激之色,``廷玉对朕,亦臣亦师。将来朕的皇子,也要请你为师,好生教导。''

张廷玉诚惶诚恐,``微臣多谢皇上垂爱。天色不早,微臣先告退了。''

皇帝道:``李玉,好生送张大人出去。''

李玉忙跟着张廷玉出去了。

皇帝嘴角还是挂着淡淡笑意,十分温和的样子,眼中却殊无笑色,取过毛笔饱蘸了墨汁,口中道:``王钦,你是朕跟前的总管太监,事无大小都要照管清楚,总有疏漏的地方。有些差事,你便多交予李玉去办吧。''

王钦心头一凉,膝盖都有些软了,只支撑着道:``奴才遵旨。''

皇帝埋首寄书,``出去吧,不用在朕跟前了。''

王钦诺诺推出去,脚步声极轻,生怕再惊扰了皇帝。出了养心殿,王钦才发觉脖子后头全是冷汗,脚底一软,坐倒在了汉白玉石阶上。

门口的小太监忙殷勤过来扶道:``总管快起来,秋夜里石头凉,凉着了您就罪过了。''

王钦硬生生甩开小太监的手,远远望见李玉送了张廷玉回来,恨恨骂小太监道:``王八羔子,也敢到我跟前来耍机灵了!''

话未说完,皇帝的声音已经从里头传出来,``去长春宫。''

王钦一骨碌站起来,用尽了嗓子眼里的力气,大声道:``皇上起驾啦------''。

\hypertarget{ux7b2cux4e5dux7ae0-ux540dux5206ux4e0b}{%
\chapter{第九章
名分(下)}\label{ux7b2cux4e5dux7ae0-ux540dux5206ux4e0b}}

太后站在慈宁宫廊下,看着福姑姑指挥着几个宫人将花房送来的数十盆``黄鹤翎''与``紫霞杯''摆放得错落有致。彼时正黄昏时分,流霞满天如散开一匹上好的锦绣,映着这数十盆黄菊与紫菊,亦觉流光溢彩。

福姑姑笑吟吟过来道:``慈宁宫的院子敞亮了许多。若是在寿康宫,这几十盆菊花一摆,脚都没处放了。''她见太后欢喜,愈发道,``也是皇上的孝心,那日携了皇后亲自来请您移宫。如今有什么好的都先尽着您用。连花房开得最好的紫菊,也都送来了您这里。''

太后微笑颔首,扶着福姑姑的手走到阶下,细细欣赏那一盆盆开得如瀑流泻的花朵,``如此,也算哀家没白疼了皇帝。只不过那日虽然是皇帝和皇后来请,可这背后的功劳,哀家知道是谁。''

``太后是说娴妃?''

太后拈起一朵菊花仔细看了片刻,``颜色多正的花儿,和黄金似的,可惜了,还没开出劲儿来。''

福姑姑笑道:``有您爱护调教,要开花不是一闪儿的事?''

``这也急不得。满园子的花,前面的花骨朵开着,后面的也急不来。由着天时地利吧。''太后松开拈花的手指,拍了拍道,``皇上只给她一个妃位,是可惜了。按着在潜邸的位分,怎么也该是贵妃或者皇贵妃。''

福姑姑取了绢子替太后抹了抹手,``有福气的,自然不在这一时上看重位分。往后的时间长着呢。''

太后颔首道:``慧贵妃是会讨人喜欢。有时候跟着皇后来哀家这里请安,规矩也一点不差。''

福姑姑道:``照规矩是该晨昏定省的,但皇后和嫔妃们,也不过三五日才来一次。这\ldots\ldots{}''

``哀家住在这慈宁宫里,便是名正言顺的太后,一日来两次也好,三五日来一次也罢,都不是要紧事。要紧的是哀家的眼睛还看着后宫,太后这个位子原不是管家老婆子,不必事事参与介入,大事上点拨着不错就是了。这样,才是真正的权柄不旁落,也省得讨人嫌。''

福姑姑这才笑道:``太后的用心,奴婢实在不及。''

夜来的长春宫格外静谧,明黄色流云百蝠熟罗帐如流水静静蜿蜒地下,便笼出一个小小天地,由得琅华伏在皇帝肩上,细细拨着皇帝明黄寝衣上的金粒纽子,只是含笑不语。

皇帝本无睡意,便笑,``皇后一向端庄持重,怎么突然对朕这么亲昵起来了?''

琅华轻笑道:``皇上只看见臣妾端庄持重,就不见臣妾也依赖皇上吗?''

皇帝望着帐顶,嘴角含了薄薄一缕笑意,``皇后在后宫一力独断,为朕分忧,朕很高兴。不过见惯皇后的皇后样子,小儿女模样倒是难得了。''

皇后默然片刻,慢慢笑道:``后宫小儿女情肠多了,难免争风吃醋的小心眼儿多些。臣妾若再不持重,岂不失了偏颇,叫人笑话?''她停一停,小心觑着皇帝道,``皇上的意思,是嫌臣妾今早提议让娴妃居住延禧宫是有些失当了。''

皇帝略略含了一丝笑影,松开被琅华倚着的肩膀,``皇后是六宫之主,后宫的事自然应当由皇后决断。皇后的提议,朕自然不会不准的。''

琅华心头微微一惊,不免含了几分委屈,``皇上这样说,真是低估了臣妾了。难道臣妾跟随了皇上这些年,还会如几位贵人一般不懂事,只晓得争风吃醋?臣妾不过是以为,皇上近日抬举慧贵妃,自然是恩宠有加,慧贵妃贤淑安静,也受得起皇上这点眷顾。只是娴妃在潜邸时位分既高,性子又傲,如今被贵妃高了一头,难免气不顺,要与人起争执,不若将她放到安静些的地方,也好静心些。等她心气平伏些许,皇上再好好赏赐她给她些恩典就是了。''

皇帝伸手抚了抚皇后的头发,``皇后思虑周详。''

琅华这才松了口气,伸手揽住皇帝的手臂,笑意盈盈,``臣妾的愚见,怎么比得上皇上的圣明。往日里皇上一向称赞娴妃慧心兰性,而慧贵妃娴静温婉,怎么到了今日给娴妃的封号是娴,贵妃反而是慧?臣妾却不懂了。''

隐隐有风吹进,帐外的仙鹤衔芝紫铜烛台上烛火微微晃了一晃,映着拂动的帐幔,如水波颤颤,明灭不定。皇帝的脸色落着若明若暗的光影,有些飘浮不定,他的笑影淡得如天际薄薄的浮云。``朕也是随手择了两个字罢了。''他低下头看着琅华,``朕嘱咐了内务府,用心布置你的长春宫,你可还满意吗?''

琅华笑意深绽,仿佛烛火上爆出的一朵明艳的烛花,``皇上在后宫的第一夜是留在臣妾宫中,便是对臣妾最大的用心与恩典了。''

皇帝轻轻拍着琅华的肩膀,声音渐渐低微下去,却依依透着眷恋与温柔,``朕的用心,你懂得就好了。你是朕的皇后,又一向贤惠,后宫的事你打理着,朕很放心。''

因出了丧,也立后封妃,嫔妃们也不再一味素服银饰了。海兰一早换了一身如意肩水蓝旗装,只衣襟袖口绣了星星点点素白小花,如她人一般,清新而不点眼。自然,这也是她一贯的态度。

海兰照常来候着如懿起身,又陪她一同用了早膳,才去长春宫中向琅华请安。

琅华气色极好,又精心修饰过容颜,换了芙蓉蜜色绣折枝蝴蝶花氅衣①,头上只用一只鎏金扁方绾住如云乌发,端正的发髻上只点缀了疏疏几点银翠珠钗,并几朵通草花朵而已。虽然简单,倒也大方爽朗。一大早二阿哥也被乳母抱来了,琅华愈加高兴,嫔妃们也少不得热闹起来,说着二阿哥又壮了或是看着聪明伶俐。

唯有嘉贵人金玉妍打量着琅华一身的打扮,笑吟吟不说话。琅华一时察觉,便笑道:``素日里嘉贵人最爱说笑,怎么今日反而只笑不说话了,可是长春宫拘谨你了?''

玉妍忙笑道:``臣妾是看皇后娘娘身上绣的花儿朵儿呢,虽然绣得少,可真真是以清朗为美,看着清爽大气。''

琅华略略正了正衣襟上的珍珠纽子,含笑道:``嘉贵人一向是最爱娇俏打扮的,本宫倒想听你评说评说。''

玉妍斜斜行了一礼,如风摆杨柳一般,细细说来:``臣妾看娘娘身上的满绣折枝花,像是从前大清刚入关的时候,宫眷们最时兴的绣法,往往以旗装绣疏落阔朗的图案为美,用的也是京绣手法,讲究的是大气连绵,富贵吉祥。而时下宫里最时兴的,是用轻柔的缎料,追求轻盈拂动之柔美,往往在袖口衣襟和裙裾上多绣花样,身上则花样轻巧,多用江南的绣法,或用金线薄薄织起,虽然花枝繁密,但追求越柔越好。如今看皇后娘娘的装扮,真是颇有入关时的古风呢。''

众人听玉妍娓娓道来,再看自己身上旗装,虽然颜色花色各异,但比之皇后,果然是轻盈软薄许多。

皇后听她说完,不觉叹道:``同样是穿衣打扮,本宫一直觉得嘉贵人精细,如今看来,果然她是个细心人,能察觉本宫的心意。今早起来,本宫查看内务府的账单,才发觉后宫女眷每年费制衣料之数,竟如斯庞大。本宫身上的衣衫虽然绣花,但花枝疏落,又是宫中婢女、京中普通衣匠都能绣的。而你们所穿,越是轻软,就必得是江南织造苏州织造所进贡的,加上织金泥金的手法昂贵,其中所费,相差悬殊。而且后宫所饰,往往民间追捧,蔚然成风,使得京城之中江南所来的衣料翻倍而涨,连绣工也愈加昂贵。如此长久下去,宫外宫中,奢侈成风,还如何了得。''

琅华一句一句说下去,虽然和颜悦色,但众妃如何不懂其中意思,都垂下头不敢再多言。唯有纯嫔不知就里,赔笑道:``皇后娘娘说的是,只是皇上一向都说,先帝与康熙爷励精图治,国富民强\ldots\ldots{}''

琅华淡淡一笑,取过茶盏定定望向她道:``民间有句老话,叫富不过三代。即便国富民强,后宫也不宜奢华挥霍。否则老祖宗留下的基业,能经得起几代。不过话说回来,纯嫔你刚诞下了三阿哥,皇上看重,自然要靡费些也是情理之中。本宫不过是拿自己说话罢了。''

素心会意,往皇后杯中斟上了茶水道:``可不是呢,昨儿皇后就吩咐了内务府,以后哪怕是长春宫的饰物,也顶多只许用鎏金和珍珠,最好是银器或是绒花通草,赤金和东珠、南珠是一点不许用的呢。''

月闲闲一笑,看着手上的白银镶翠护甲,``皇后娘娘的话,臣妾自然是听着了。不比纯嫔妹妹,有了三阿哥,说话做事的底气,到底是不同了。''

纯嫔虽然单纯,但话至于此,还有什么不明白的。她不觉苍白了脸,腿下一软便跪下了道:``皇后娘娘恕罪,还请娘娘明鉴。臣妾虽然诞下阿哥,但都是皇后娘娘福泽庇佑,臣妾不敢居功自傲,更不敢靡费奢侈。''

琅华淡淡一笑,``好了,别动不动就跪下,倒像本宫格外严苛了你们似的。起来吧。''

纯嫔这才敢起身,怯怯坐下。

玉妍很是得意,扫了一眼众妃,上前一步笑道:``皇后娘娘的话说得极是。只是如今风气已成,别说宫里宫外了,连皇上赏赐给朝鲜的衣料首饰,也无不奢丽精美。臣妾听来往朝鲜的使者说起,朝鲜国中也很是风靡呢。若咱们改了入关时的衣饰,也这般赏赐亲贵女眷或属国,岂不让外人惊异?''

她这一番话,自以为是体贴极了皇后,也能顾全自己喜好。如懿与海兰对视一眼,当下只是笑而不语。

琅华轻轻啜了一口茶水,方徐徐道:``嘉贵人的话自然也是有理的。皇上怎么恩赏外头,那是免不了的。只是在内,咱们深居六宫的,凡事还是简朴为好。''她微微正色,``更要紧的是,如今天下安定,咱们也别忘了祖宗入关平定天下的艰难。咱们身为天下女子的表率,更得时时记着自己的身份,事事不忘列祖列宗才是。''

这番话极有分量了,饶是金玉妍伶牙俐齿,也只得低头称是。

月第一个站起来道:``既然皇后娘娘做出表率,臣妾等定当追随。今日起,不再华服丽饰,一定效仿皇后娘娘,追思祖宗辛苦,简朴度日。''

琅华颔首,轻叹道:``本宫一番良苦用心,你们千万别以为是本宫有心苛责了你们。后宫人多,若人人多花费些,家大业大,总有艰难的时候。''

这时,坐在一旁闷声不语的怡贵人小声道:``奴婢伺候皇后娘娘多年,皇后娘娘一直不事奢华,直到如今,连衣襟上用的珍珠纽子,也不过是内务府最寻常的那种,连上用的珍珠都觉得太过浪费了。''

纯嫔忙赔笑道:``怡贵人从前是贴身伺候皇后娘娘的,自然无事不晓。看来是臣妾们一直太粗心了,不曾好好追随皇后。''

皇后笑盈盈看着怡贵人道:``好了。如今都是皇上正式册封的贵人了,还一口一个奴婢,成什么体统呢。''

怡贵人忙恭恭敬敬道:``臣妾谨遵皇后娘娘吩咐。''

月忽地转首,看了如懿一眼,``娴妃妹妹一直不言不语,难道不服皇后所言,还是另有主张?''

如懿抬了抬眼帘,徐徐道:``所谓言传身教,皇后娘娘身体力行,咱们自然只有听其言随其行的份,何须再多置喙呢。''

海兰亦忙低低道了``是'',又道:``臣妾不敢多言,是怕自己蠢笨失言。所以仔细学着皇后,不敢再多言了。''

如懿微微一笑,``可不是,皇后的意思,就是皇上的意思。咱们好好听着学着,便是受益无穷了。''

月轻笑一声,掩唇道:``娴妃妹妹这句话,倒是意在皇上昨夜留宿长春宫了,好像有些酸意呢。''

如懿淡淡笑道:``我方才说的话,心存和睦的人自然听出帝后一心,后宫和睦的意思;心存酸意的嘛,自然也听出酸意了。''

月秀眉一挑,似有不忿。琅华和悦一笑,``好了。昨夜是皇上眷顾本宫这个皇后的面子罢了,来日方长,你们都精心准备着,皇上自然会一一来看你们。''

众人答了是,如懿举起手腕上的翡翠珠缠丝赤金莲花镯道:``这镯子虽是臣妾入潜邸不久后皇后娘娘亲自赏赐的,但如今宫中节俭,臣妾也不敢再戴了。还请皇后娘娘允准。''

她这般一说,月也忙站了起来。

皇后神色微微一沉,如秋日寒烟中沾上霜寒的脉脉衰草。然而旋即秋阳明艳,那寒意便蒸发得无影无踪。皇后还是那样无可挑剔的笑容,``既是本宫从前赏的,那也无妨。何况你们俩到底一个是贵妃一个是娴妃,不能委屈了。''二人答应了,方才告退。

外头秋色明丽如画卷,绿筠与海兰陪着如懿出来,三人都是默默的。金玉妍与怡贵人走在前头,犹自有些埋怨,``哎呀,从今往后,再不能穿这样江南的软缎子了,我一想着皇后娘娘身上的满绣旗袍,虽然好看,但一点也无飘逸之美,唉\ldots\ldots{}''

怡贵人淡淡笑道:``嘉贵人美貌,自然穿什么都是好看的。再不济,你一向在梳妆打扮上用心,皇上一定会留意的。''

玉妍轻轻``呀''了一声,便道:``怡贵人在皇后身边久了,自然懂得皇后的心思。有皇后娘娘这个榜样,我哪里敢不跟随呢。''

两人说说笑笑,走到前头去了。

如懿安慰地拍拍绿筠的手,``今日的事别往心里去。皇后只是看重祖宗家法,并不是有意指责你。''

绿筠愁眉微笼,``皇后的意思我如何不明白?先头大阿哥的亲娘是皇后族人,虽然殁了,但身份依旧高贵。二阿哥是皇后娘娘亲生的,那更是尊贵无比的嫡子。只有我,身份不尴不尬的,我阿玛不过是笔帖式,要不是我侥幸生养了三阿哥,皇上怎么会给我嫔位。我自知出身不高,平时已经恭谨安分,可是皇后仍然在意\ldots\ldots{}''她再要说下去,已经含了几分泪意。海兰赶紧拿绢子挡在绿筠口边,轻声道:``好姐姐,你对皇后当然是恭谨安分,只是姐姐心思单纯,有什么说什么。这儿是在外头,叫人听见又多是非了。''

绿筠吓得一噤,忙取了绢子赶紧擦去泪痕。四周静寂无声,连陪侍的宫女也只远远地跟在后头。

如懿赞许地看了海兰一眼,柔声道:``好了。有什么事尽管到了我宫里再说。如今,可别再失言了。''

绿筠连连点头,三人便说着话往御花园去了。

彼时秋光初盛,御花园中各色秋菊开得格外艳丽,姹紫嫣红,颇有春光依旧的绚美繁盛。美景当前,三人也少了方才的沉闷。一路绕过斜柳假山,如懿见前头亭中玉妍和怡贵人正坐着闲话,便与绿筠和海兰看着池中红鱼轻跃,自己取乐。

玉妍和怡贵人背对着她们,一时也未察觉,只顾着自己说得热闹。

玉妍笑道:``其实姐姐封为娴妃,我倒觉得皇上选这个`娴'字为封号,真是贴切。''

怡贵人拈了绢子笑:``妹妹说来听听,也好叫我们知道皇上的心意。''

玉妍拔下头上福字白玉鎏金钗,蘸了茶水在石桌上写了个大大的``娴''字,笑吟吟道:``闲字,女旁。皇上登基之后最爱去皇后娘娘和慧贵妃那里,娴妃娘娘好些日子没见到皇上了,可不是一个闲着的女人无所事事吗?''

怡贵人拿绢子捂了嘴笑,倒是怡贵人身边的宫女环心机灵,看见如懿就站在近处,忙低呼一句,``贵人乏了,不如咱们早些回宫歇息吧。''

这样突兀一句,连玉妍也觉着不对,回首看见了如懿一行人。玉妍并不畏惧,索性轻蔑地看着如懿,娇滴滴道:``嫔妾不过是说文解字,有什么说什么,娴妃娘娘可别生气。''

怡贵人瞟了如懿一眼,``娴妃娘娘哪里会生气,一生气可不落实了嘉贵人的话吗?不会不会。''

如懿听着她们奚落,心头有气,只是硬生生忍住。

海兰实在听不下去,大着胆子回嘴道:``娴妃娘娘面前,咱们虽然都是潜邸的姐妹,也不能如此不敬。''

玉妍微眯了双眼,招了招手道:``海常在,快过来说话。''

玉妍的位分比海兰高,她见玉妍召唤,稍稍犹豫,还是不敢不去。待海兰走到近前,玉妍伸手托起海兰的下巴,仔细端详着,``绣房里的侍女,如今做了常在,嗓子眼儿也大起来了。''

海兰窘得满脸通红,只说不出话来。金玉妍越发得趣,银嵌琉璃珠的护甲划过海兰的面庞便是一道幽艳的光。海兰只觉得浑身起了鸡皮疙瘩,颤声道:``嘉贵人,你想做什么?''

玉妍笑吟吟凑近她,``我想\ldots\ldots{}''

话未说完,玉妍的手已被如懿一把撩开。

如懿冷然一笑,将海兰护在身后,``凭着贵人的身份吓唬一个常在算什么本事?你也不过只能在本宫面前作口舌之稽罢了。见到本宫,还不是要屈膝行礼,恭谨问安。''

绿筠忙劝道:``嘉贵人,你若与海常在玩笑,那便罢了吧。她一向胆子小,禁不起玩笑的。''

玉妍轻哼一声,蔑然道:``海兰是什么身份,我肯与她玩笑?''

如懿瞥她一眼,缓缓道:``人在什么身份就该做什么事。若你觉得慧贵妃位分在本宫之上苛责本宫是理所应当,那么本宫要来为难你,也是情理之中你合该承受。''

玉妍嘴角一扬,毫不示弱,``你虽然是妃位位分远在我之上,可你是乌拉那拉氏的后代,我却是朝鲜宗室王女,若论身份,我自然比你高贵许多。虽然我位分一时在你之下,你便以为你坐稳了妃位,我也没有出头之日了吗?''

如懿微微一笑,``你自恃朝鲜宗室王女,却不想想,朝鲜再好,也不过是我大清臣属之国。小国寡民,连国君都要俯首称臣,何况是区区宗室女?你若真要与本宫讨论何为身份何谓高贵,就好好管住自己,做合乎自己身份的言行,才能让人心悦诚服,才是真正的高贵。''

如懿话音未落,却听得身后一声婉转,``本宫当是谁,这样牙尖嘴利不肯饶人的,只有娴妃了。''

如懿微微欠身,冷眼看着她,``昔日在潜邸中,贵妃温顺乖巧,可不是今日这副模样。''

慧贵妃瞥如懿一眼,大是不屑,``此一时彼一时,当日你位序在我之上,我自然不得不尊崇你。而今本宫是贵妃,你只是妃位,尊卑有序如同云泥有别,你自然要时时事事在我之下。若连这个都不知道,你便不用在这后宫里待下去了。''

如懿默然不语,贵妃描得细细的柳眉飞扬而起,``怎么,你不服气?''

如懿笑意淡然。``礼仪已经周全,贵妃连人心也要一手掌控吗?若真要如此,就不是以威仪压人,而是以懿德服人了。''她再度福身,``贵妃娘娘位分在上,我不会不尊。但也请贵妃明白,您的高贵应当来自敬服,而非威慑。''

如懿说罢,径自离去。纯嫔与海兰互视一眼,立刻急急跟上。

玉妍见慧贵妃气得发怔,旋即笑道:``贵妃娘娘别听她饶舌,眼见她以后的日子是不好过了,娘娘何必与她费口舌。娴妃在您之下,将来还怕不能收拾了她吗?''

慧贵妃眉头微松,笑向玉妍道:``有嘉贵人与本宫一心,本宫有什么可担心的呢。''

注释:

①氅衣:氅衣与衬衣款式大同小异,小异是指衬衣无开禊,氅衣则左右开禊高至腋下,开禊的顶端必饰云头;且氅衣的纹饰也更加华丽,边饰的镶滚更为讲究,在领托、袖口、衣领至腋下相交处及侧摆、下摆都镶滚不同色彩、不同工艺、不同质料的花边、花绦、狗牙等等,尤以江南地区,俗以多镶为美。为清宫妇女正式的穿着。

\hypertarget{ux7b2cux5341ux7ae0-ux54f2ux5983}{%
\chapter{第十章 哲妃}\label{ux7b2cux5341ux7ae0-ux54f2ux5983}}

紫禁城中的夜仿佛格外地深沉。如懿记得在王府的时候,院子也是大院子,福晋侍妾们也各有自己的阁子院落,但那夜是浅的,这头望得到那头。站在自己的院中,默默数着,往前几进院落便是弘历的书房了。夜晚乏闷了,出了阁子几步便是旁的妾室的阁院。虽然见面也有龃龉,也有争宠,但那都是眼皮子底下的事。总有几个稍稍要好些的,斟着茶水,用着点心,说说笑笑,便也填了寂寞。连弘历走进谁的阁楼了,那得宠的人的楼台灯火也格外明艳些,心酸醋意都是看得见的,也越发有了盼望。

可是如今,规矩越发大了,宫墙深深,朱红的壁影下,人都成了微小的蝼蚁。长街幽深,哪怕立满了宫人侍婢,也是悄然无声,静得让人生怕。很多次如懿坐在暖阁里,安静地听着更漏滴滴,以为四下里是无人了,一转头,却是一个个泥胎木偶似的站着,殿外有,廊下有,宫苑内外更多的是人,但那都是说不上话的人。一众入宫的嫔妃里,格外要好些的,只有苏绿筠与珂里叶特氏海兰。她们都是性情平和的人。从前她的性子尖锐孤傲,与高晞月一向是彼此看不过的。高晞月身边有黄绮沄和金玉妍,更依附着富察琅,她也只是冷冷地不与她们多言。可如今,苏绿筠沉浸在儿子去了阿哥所不得相见的愁苦里,每常见了也不过是郁郁寡欢。海兰呢,当年一夕承欢就被弘历忘在脑后,受尽了奚落白眼。如懿虽然不喜欢弘历有新宠,但到底也看不过人人都欺负她,偶尔在弘历面前提了一句,才成全了海兰的身份,在府里有了一席栖身之地。为着这个缘故,海兰也喜欢总跟着她,怯怯的,像是在寻找羽翼荫庇的受伤的小鸟,总是楚楚可怜的样子。现下海兰与晞月同住,她也不便总和海兰来往,免得晞月介意,让海兰越发难过。

如此一来,如懿便更觉得寂寞了。像一根空落落燃烧在大殿里的蜡烛,只她一根,孤独地燃烧着,怎么样也只是煎熬烧灼了自己。

皇帝刚刚登基,进后宫的日子并不多。每日敬事房递了牌子上去,三四日才翻一个绿头牌,先是皇后,然后是慧贵妃,仿佛是按着位次来的。如懿盼着数着,以为总该是轮到自己了,皇帝却又久久地没有翻牌子了。

渐渐地,她也晓得这寂寞是无用的了。宫中的日子只会一天比一天长,连重重金色的兽脊,也是镇压着满宫女人的怨思的。

这一夜晚来风急,连延禧宫院中的几色菊花也被吹落了满地花瓣堆积。京城的天气,过了十月中旬,便是一日比一日更冷了。如懿用毕晚膳,换过了燕居的雅青色绸绣枝五瓣梅纹衬衣,浓淡得宜的青色平纹暗花春绸上,只银线纳绣疏疏几枝浅绛色折枝五瓣梅花,每朵梅花的蕊上皆绣着米粒大的粉白米珠,衬着挽起的青丝间碧玺梅花钿映着烛火幽亮一闪。地下新添了几个暖炉,皆装了上等的银屑炭,燃起来颇有松枝清气。

如懿捧了一卷宫词斜倚在暖阁的榻上,听着窗外风声呜咽如诉,眼中便有些倦涩。她迷蒙地闭上眼睛,忽然手中一空,握在手里的书卷似是被谁抽走了。她懒怠睁眼,只轻声道:``阿箬,那书我要看的。''

脸上似是被谁呵了一口气,她一惊,蓦然睁开眼,却见皇帝笑吟吟地俯在身前,晃晃手里的书道:``还说看书,都成了瞌睡猫了。''

如懿忙起身福了福,嗔道:``皇上来了外面也不通传一声,专是来看臣妾的笑话呢。''

皇帝笑着搓了搓手在榻上坐下,取过紫檀小桌上的茶水就要喝。如懿忙拦下道:``这茶都凉了,臣妾给皇上换杯热的吧。''

皇帝摇手道:``罢了。朕本来是去慈宁宫给太后请安的。内务府的人晌午来回话,说明日怕是要大寒,太后年纪大了受不住冷,朕去请安的时候就看看,让内务府的人赶紧暖了地龙,别冻着了太后。这一路过来便冷得受不住,想着你这儿肯定有热茶,便来喝一杯,谁知你还不肯。''

如懿夺过茶盏,唬了脸道:``是不给喝。现下觉得凉的也无妨,等下喝了肚子不舒服,又该埋怨臣妾了。''她回头才见守在屋里的宫人一个也不在,想是皇帝进来,都赶着退下了。如懿朝着窗外唤了一声``阿箬'',阿箬应了一声,便捧热茶进来,倒了一杯在金线青莲茶盏中。

皇帝捧过喝了一口,便问:``是齐云瓜片?''

阿箬娇俏一笑,伶俐地道:``齐云瓜片是六安茶中最好的。这个时候奴婢估摸着皇上刚用了晚膳,天气冷了难免多用荤腥,这茶消垢腻、去积滞是最好的。''

皇帝向着如懿一笑:``千伶百俐的,心思又细,是你调教出来的。''

阿箬笑生两靥:``奴婢能懂什么呢?这话都是小主日常口里颠来倒去说的,惦记着皇上用了什么,用得好不好。奴婢不过是耳熟,随口说出来罢了。''说罢她便欠身退下了。

皇帝握了如懿的手引她一同坐下:``难怪朕会想着你的茶,原来你也念着朕。''

如懿低了头,笑嗔道:``皇上也不过是惦记着茶罢了。明儿臣妾就把这些茶散到各宫里去,也好引皇上每宫里都去坐坐。''

皇帝握住她的手紧了紧:``天一冷就手脚冰凉的,自己不知道自己这个毛病么,也不多披件衣裳。''他见榻上随手丢着一件湖色绣粉白藤萝花琵琶襟袷马褂,便伸手给如懿披上,叹口气继续道,``这话便是赌气了。''他摊开如懿方才看的书,一字一字读道,``十二楼中尽晓妆,望仙楼上望君王。遥窥正殿帘开处,袍袴宫人扫御床①。''

如懿面红耳赤,忙要去夺那书:``不许读了。这词只许看,不许读。''

皇帝将书还到她手里:``是不能读,一读就心酸了。''

如懿奇道:``宫词写的是女人,皇上心酸什么?''

皇帝静静道:``朕在太和殿里坐着上朝,在乾清宫里与大臣们议事,在养心殿书房里批阅奏折。你想着朕,朕难道不想着你么?你在`锁衔金兽连环冷,水滴铜龙昼漏长'的时候,朕也在听着更漏处理着国事;你在`云髻罢梳还对镜,罗衣欲换更添香'的时候,朕在想着你在延禧宫中的日子如何,是不是一切顺心遂意?''

如懿动容,伏在皇帝肩头,感受着他温热的气息。皇帝身上有隐隐的香气,那是帝王家专用的龙涎香。那香气沉郁中带着淡淡的清苦气味,却是细腻的、妥帖的,让人心静。暖阁里竖着一对仙鹤衔芝紫铜灯架,架上的红烛蒙着蝉翼似的乳白宫纱,透出的灯火便落成了十八九的月色,清透如瓷,却昏黄地温暖。皇帝背着光站着,身后便是这样光晕一团,如懿只觉得沉沉的安稳,再没什么不放心的了。

良久,如懿才依偎着皇帝极轻声道:``臣妾初嫁给皇上之时,其实内心忐忑,不知自己托付终身之人会是怎样的男子。可是成婚之后日夕相对,皇上体贴入微,臣妾感激不尽。如今皇上身负乾坤重任,虽然念及后宫之情,却也隐忍以江山为重,臣妾万分钦佩。''

皇帝的声音沉沉入耳:``朕忍的是儿女私情,不过一时而已。而你也要和朕一样,有什么委屈,先忍着。朕知道入宫之后,你的日子不好过,可再不好过,想想朕,也该什么都忍一忍。朕才登基,诸事繁琐,你在后宫,就不要再让朕为难了。''

如懿双眸一瞬,睁开眼道:``皇上可是听说了什么?''

皇帝道:``朕是皇帝,耳朵里落着四面八方的声音,可以入耳,却未必入心。但朕知道,住在这延禧宫是委屈了你,仅仅给你妃位,也是委屈了你。''

如懿道:``延禧宫邻近苍震门,那儿是宫女、太监们出入后宫的唯一门户,出入人员繁杂、关防难以严密,自然是不太好。但宫里哪里没有人?臣妾只当闹中取静罢了。至于位份,有皇上这句话,臣妾什么委屈也没有。''

皇帝微微松开她:``有你这句话,朕就知道自己没有嘱咐错。''他停一停,朝外头唤了一句,``王钦,拿进来吧。''

王钦在外答应了一声,带着两个小太监捧了一幅字进来,笑吟吟地向如懿打了个千儿:``给娴妃娘娘请安。''

如懿含笑颔首:``起来吧。''

王钦答应着,吩咐小太监展开那幅字,却是斗大的四个字------慎赞徽音。

皇帝笑道:``朕亲手为你写的,如何?''

如懿心头一热,便要欠身:``臣妾多谢皇上。''

皇帝忙扶住了她,柔声道:``《诗经》中说`大姒嗣徽音,则百斯男'。徽音即为美誉,这个`慎'字是告诉你,唯有谨慎,才能得美誉。日后宫中度日,朕是先把这四个字送给你。''

如懿明白皇帝语中深意,沉吟着道:``那臣妾便嘱咐内务府的人将皇上的字做成匾额,放在延禧宫正殿,可好?''

皇帝拢一拢她的肩:``你与朕的意思彼此明白,那就最好。''

往下的日子,皇帝依着各人位份在各宫里都歇了一夜,是谓``雨露均沾''。之后皇帝便是随性翻着牌子,细数下来,总是慧贵妃与嘉嫔往养心殿侍寝的日子最多。除了每月朔望,皇帝也喜欢往皇后宫中坐坐,闲话家常。如懿的恩宠不复潜邸之时,倒是随着纯嫔、怡贵人和海常在一般沉寂了下来。

纷纷扬扬下了几场雪之后,紫禁城便入冬了。内务府忙碌着各宫的事宜,渐渐也疏懒了延禧宫的功夫。这日午后,如懿正坐着和海兰描花样子,却听阿箬掀了帘子进来道:``内务府越发会看脸子欺负人了,皇后娘娘今儿赏给各宫的白花丹和海枯藤是做成了香包的,说是宫里湿气重,戴着能祛风湿通络止痛。结果奴婢打开一看,里面塞的白花丹粉末全是次货,想要再跟内务府要,他们说太医院送来的就是这些,没更好的了。奴婢想,慧贵妃那儿他们敢送这样的?连缝的香包都松松散散的,针脚不成个模样\ldots\ldots{}''

海兰停了手,含了一缕忧色:``姐姐这儿都是这样的,我那里就更不必说了。''

如懿抬头看了看阿箬:``既是次的,也比不用好。先搁着吧。''

海兰道:``也是,外头快下雪了,省得来回折腾。这样吧,阿箬,你先都把这些香包送到我那儿去,我替姐姐把针脚都缝一缝,省得用着便散了。''

如懿道:``这些微末功夫,教她们做便罢了,你何必自己这么累。''

海兰静静一笑:``姐姐忘了。我本闲着,最会这些功夫了。就当给我打发时间吧。''

这一日下了一上午的雪点子,皇帝身边的大太监王钦亲自过来了。那王钦本是先帝时的传奏事首领太监,因皇帝为皇子时侍奉殷勤,十分得力,皇帝登基后便留在了身边为养心殿副总管太监。因总管太监的位子一直空缺,他又近身伺候着皇帝,所以宫中连皇后也待他格外客气。

王钦进来时,皇后穿了一身藕荷色缎绣牡丹团寿纹袷衣,外罩着月白底碧青竹纹织金缎紫貂小坎肩,笼着一个画珐琅花鸟手炉,看着素心与莲心折了蜡梅来插瓶。

王钦见了皇后,忙恭恭敬敬地行了一礼,道:``奴才王钦给皇后娘娘请安。''

皇后含笑道:``外头刚下了雪,地上滑,皇上怎么派了你过来?可是有什么要紧事?''说着吩咐了莲心上茶赐座。

王钦诺诺谢恩,方道:``谢皇后娘娘的赏,实在是奴才不敢逾越。话说完了,还等着别的差事呢。''说着道,``皇上吩咐了,明儿是十五,要在娘娘的长春宫用晚膳,也宿在长春宫,请娘娘预备着接驾。''

皇后眉目间微有笑意,脸上却淡淡的:``知道了。夜来霜雪滑脚,你嘱咐着抬轿的小太监们仔细脚下。还有,多打几盏灯笼,替皇上照着路。''

王钦忙道:``娘娘放心,奴才不敢不留心着呢。''

皇后微微颔首,扬了扬脸,道了句``赏''。莲心立马从屉子里取出十两银子悄悄儿放在王钦的手心里。

王钦嘴上千恩万谢着,眼睛却往莲心脸上一瞟,莲心红了脸,忙退到后头去了。王钦又道:``还有一件事。昨儿夜里下了一夜的雪,皇上想起去年潜邸里殁了的大阿哥的生母,道了好几句`可惜'。''

皇后惋惜道:``诸瑛是本宫富察氏的族妹,伺候皇上已久。谁知去岁病了这一场,好好的竟去了,也没享这宫里一日的福。''说罢便拿绢子按了按眼角,继续慢慢说,``诸瑛是大阿哥的生母,当年也只是潜邸里的一位格格,位份不高。如今她虽福薄弃世而去,但皇上也不能不给她一个恩典,定下名份,给个贵人或嫔位,也是看顾大阿哥的面子。''

王钦恭谨道:``皇后娘娘慈心,皇上昨夜便说了,是要追封为哲妃,过两日便行追封礼,还要在宝华殿举行一场大法事,还请皇后娘娘打点着。''

皇后微微一怔,旋即和婉笑道:``还是皇上顾虑周全,先想到了。那你去回禀皇上,哲妃与本宫姐妹一场,又是本宫的族妹,她的追封礼,本宫会命人好好主持的。''

王钦笑道:``是。那奴才先告退。''

皇后眼看着王钦出去了,笑容才慢慢凝在嘴角,似一朵凝结的霜花,隐隐迸着寒气。

皇后眼看着王钦出去了,笑容才慢慢凝在嘴角,似一朵凝结的霜花,隐隐迸着寒气。

素心素知皇后心思,忙端了一盏茶上来,轻声道:``天冷了难免火气大,这江南进宫的白菊还是皇上前儿赏的,说是最清热去火的,娘娘尝尝。''

皇后接过茶盏却并不喝,只是缓缓道:``本宫是皇后,六宫之主,有什么好生气的?''

素心看了皇后一眼,低婉道:``娘娘说的是。其实皇上给哲妃脸面,也是看着皇后娘娘的缘故,要不是哲妃和娘娘同宗,都是富察氏的女儿,哪怕她生了大阿哥,又算什么呢?纯嫔生了三阿哥,皇上不也只给她嫔位么?''

皇后淡淡一笑:``哲妃是与本宫同宗,可她伺候皇上早,和皇上好歹也有些情分,所以也是她先生了大阿哥。''

皇后郁然叹了口气,望着榻上内务府送来的一堆精心绣制的幼儿衣裳:``这件事本宫想起来便有些心酸。当年本宫嫁给皇上为嫡福晋,可是皇上每常只去如懿和晞月的房中多,长久下来,本宫都是恩宠稀薄,膝下无望。本宫还没着急呢,本宫的母家就着急了,硬生生塞了诸瑛进来,说是本宫的族人,她万一得幸生下了孩子,就等于是本宫的孩子。''

素心慨然道:``这件事,娘娘是受委屈了。''

``结果诸瑛一进府,不出几个月就怀上了大阿哥,本宫心里虽然欣慰,却更难过。幸好后来皇天有眼,皇上对本宫越来越眷顾,这才有了二阿哥。''皇后爱惜地抚着那些孩儿衣裳,心酸道,``只是嫡子非长子,本来就是失了本宫的颜面了。''

素心道:``虽然都是富察氏,可哲妃的身份却不能和娘娘比肩的。再怎么样,在潜邸时也不过是个格格。''

皇后摇摇头,双眉微蹙:``她身份如何且不说,皇上如今追封她为妃,就不能不当心了。母凭子贵,子凭母贵是祖宗家法。如今慧贵妃和娴妃都无所出,纯嫔身份略低。除了本宫的二阿哥,就是大阿哥身份最尊了。古来立太子,不是立嫡就是立长。若是永琏是嫡长子,那就更好了。''

素心忙劝解道:``不管怎么样,哲妃都已经没了。大阿哥哪怕再争气,没娘的孩子能翻出什么天来?娘娘可是正宫皇后呢。''

皇后喝了口茶,沉吟道:``凡事但求万全,本宫已经让哲妃福薄了,可不能让大阿哥再福薄。记着,照顾大阿哥的人必须多,万不可亏待了这没娘的孩子。''

素心略略不解:``娘娘?是像厚待三阿哥一样么?''

皇后微微一笑,神色端然:``太后和皇上素来夸本宫是贤后,本宫自然要当得起这两个字。但是三阿哥还小,从襁褓里宠爱着,自然能定了性子。大阿哥年纪却长成了,先头在潜邸的时候皇上还亲自教导过一阵,这个时候才宠着护着,由着他淘气,岂不是背了皇上的心思?福薄的额娘最会生下福薄的孩子,哪怕多多的人照顾着,也是不济事的。人多,才手忙脚乱么。''

素心会意,即刻笑道:``奴婢知道了。''

注释:

①出自薛逢的《宫词》。宫怨是唐诗中屡见的题材。薛逢的这首《宫词》,从望幸着笔,刻画了宫妃企望君王恩幸而不可得的怨恨心理,情致委婉,有其独特风格。全诗为:十二楼中尽晓妆,望仙楼上望君王。锁衔金兽连环冷,水滴铜龙昼漏长。云髻罢梳还对镜,罗衣欲换更添香。遥窥正殿帘开处,袍袴宫人扫御床。

\hypertarget{ux7b2cux5341ux4e00ux7ae0-ux7435ux7436}{%
\chapter{第十一章 琵琶}\label{ux7b2cux5341ux4e00ux7ae0-ux7435ux7436}}

皇后正嘱咐着素心,却听外头传来太监特有的尖细的通传声:``慧贵妃到------''

皇后点一点头:``传吧。''

只见白藤间花绣幔锦帘轻盈一动,外头冷风灌入,盈盈走进来一个美人儿,素心已经先屈膝下去:``慧贵妃万福金安。''

慧贵妃忙笑道:``快起来吧。日常相见的,别那么多规矩。''

说着由侍女茉心卸了披风,慧贵妃才轻盈福了福身:``给皇后娘娘请安,娘娘万福金安。''

皇后忙笑着道:``赐座。本宫也是你的那句话------日常相见的,别那么多规矩。''

慧贵妃谢了恩,往下首的蝠纹梨花木椅上一坐,方才笑道:``刚午睡了起来,想着日长无事,便过来和娘娘说说话,没扰着娘娘吧?''

皇后笑道:``正说着你呢,你就来了。''她打量着慧贵妃,天气虽冷,慧贵妃却早早换上了一袭水粉色厚缎绣兰桂齐芳的棉锦袍,底下露着桃红绣折枝花绫裙,行动间便若桃色花枝漫溢无尽春华。她外头搭着深一色的桃红撒花银鼠窄裉袄,领子和袖口都镶饰青白肷镶福寿字貂皮边,那风毛出得细细的,绒绒地拂在面上,映着漆黑的发髻上一枝双翅平展鎏金凤簪垂下的紫晶流苏,越发显得她小小一张脸粉盈盈似一朵新绽的桃花。

慧贵妃好奇:``皇后说臣妾什么?''

皇后见素心端了茶点上来,方道:``说下了几场雪冷了起来,你原是最怕冷的。果然现在看你,连风毛的衣裳都穿上了。这若到了正月里,那可穿什么好呢?''

慧贵妃捧着手里的紫铜花篮小手炉一刻也不肯松手:``皇后娘娘是知道我的,一向气血虚寒,到了冬日里就冷得受不住。整日里觉得身上寒浸浸的,只好有什么穿什么罢。''

茉心笑道:``皇后娘娘不知道呢。虽说到了十一月就上了地龙,可我们小主还是冷得受不住,手炉是成日捧着的,脚炉也踩着不放呢。''

皇后叹了口气道:``你年轻轻的,也该好好保养着。如今不比在潜邸的时候,什么好太医没有?尽着你瞧的。好好把身子调养好了,也像纯嫔一样给皇上添个阿哥才好。''说到子嗣上,慧贵妃便有些伤感,忙低了头轻轻应了一声。

皇后唤了莲心上前,道:``本宫记得长春宫的库房里有一件吉林将军进贡的玄狐皮,皇上前儿刚赏的,你去取了来。''莲心忙退了下去,皇后见左右都是心腹之人,方肯推心置腹地道,``其实你的年纪比本宫还长些,侍奉皇上的日子又久。说句不见外的话,皇上也是宿你宫里最多,怎么会到了如今也没一点儿动静?你也该好生留意着了。''

慧贵妃眼圈一红,低声道:``皇后这么说,满心里是疼臣妾,臣妾都知道。可太医也一直调理着,还是皇上亲自指定的太医院院判齐大人,不能不说用心看着的。只臣妾自己福薄罢了。''

皇后叹了一声,也是感触:``皇上膝下才三位阿哥,本宫的二阿哥是不消说了。大阿哥和三阿哥的出身都是一般,本宫是有多指望你也能有个阿哥,聪明灵慧不消说,二阿哥也有个伴儿了。那才是真正的亲兄弟哪!''

慧贵妃听了这句话,满心里感激,急忙跪下,含泪道:``皇后娘娘一直眷顾臣妾,臣妾都是知道的。有娘娘这句贴心话,臣妾万死也难报娘娘的垂爱了。''

皇后忙扶起她道:``这样的话就是见外了。本宫与你相处多年,也不过是格外投缘,才把你视若姐妹一般。''她抬首见莲心捧了那件玄狐皮进来,皇后便道,``交给茉心吧,本宫赏给慧贵妃的。''

慧贵妃素知皮货有``一品玄狐,二品貂,三品狐貂''之说,又见那狐皮毛色深黑如墨,唯有顶上一须银毫明灿,整张皮子油光水滑,更兼是吉林将军的贡品,一年也不过一两件,自知是一等一的好货,忙谢恩道:``这样贵重的东西,臣妾怎么敢用?又是皇上赏赐给娘娘的。''

皇后和颜道:``既是皇上赏给本宫的,本宫自然可以做主了。你且收着吧,明儿叫内务府做件保暖的衣裳,自己暖了身子就不枉费了。''

慧贵妃再三谢过,方命茉心仔细收了。皇后一双碧清妙目,往那狐皮上一转,蓦然叹了口气:``其实本宫给你的东西,再好也就是样贡品罢了。左不过今年没玄狐,明年后年也总还有的。哪里比得上旁人,连宫里挂着的一幅匾额,都是皇上御笔亲赐的。''

慧贵妃似是不解,忙问:``什么匾额?''

皇后本要回答,想了想还是摆手:``罢了,什么要紧事呢,本宫也不过随口一说罢了。''

慧贵妃见她宁愿息事宁人,愈加不肯放松:``娘娘是有什么话连臣妾也要瞒着么?''

素心见慧贵妃盏中的茶不冒热气了,忙添了点水,为难道:``娘娘哪里是要瞒着小主?只是怕说了也只是添气罢了,便也懒怠多言。奴婢可是眼里揉不得沙子的。今儿上午内务府来回禀,说皇上御笔写了幅字给娴妃的延禧宫里,娴妃就忙不迭地嘱咐了人做成了金漆匾额挂在了正殿里。其实皇上赏赐谁不赏赐谁,偏她这样抓乖卖巧,生怕人看不见似的硬要挂在正殿里,还一路宣扬着,以为这样就得了恩宠了么?其实奴婢看,哪怕皇上要赐字悬匾,那也是该先在皇后和小主宫里,哪里就轮到她了?''

慧贵妃贝齿轻咬,冷笑一声道:``臣妾还以为这些时日皇上都没召她侍寝过,她便会安分些,原来还是这泼辣货的性格。臣妾倒不信了,皇上御笔而已,一块匾额就这么难了。''她说罢起身,匆匆告辞去了。

皇后望着她的背影,只是淡淡一笑,道:``本宫惦记着二阿哥,你带上本宫亲手缝给二阿哥的那些衣裳,咱们去阿哥所走一趟。''

素心道:``今儿上午内务府不是送来了好些上用的衣裳么?奴婢瞧着都挺好,娘娘总熬着夜给二阿哥做衣裳,自己也仔细凤体才好。''

皇后瞥了眼那堆五颜六色的衣裳,冷冷摇头:``旁人送来的东西,再好本宫也不放心。宁可自己辛苦些,哪怕你们经手也放心些。''

素心闻言一凛,答应了道:``奴婢明白了。''

\hypertarget{ux7b2cux5341ux4e8cux7ae0-ux854aux59ecux4e0a}{%
\chapter{第十二章
蕊姬(上)}\label{ux7b2cux5341ux4e8cux7ae0-ux854aux59ecux4e0a}}

慧贵妃离了长春宫,坐在辇轿上支腮想了片刻,便道:``茉心,你带着这件玄狐皮先回宫。彩珠、彩玥留下,陪着本宫去养心殿看望皇上。''

茉心答应了声``是'',嘱咐彩珠、彩玥好生照看着,便先回去了。

慧贵妃不顾雪后路滑,催促了抬轿的太监两声,紧赶慢赶着便去了养心殿。才到了养心殿门外,王钦见是慧贵妃来了,忙迎上来打着千儿亲手扶了慧贵妃下轿,一叠声道:``贵妃娘娘仔细台阶滑,就着奴才的手儿吧。''

慧贵妃漾起梨涡似的一点笑意:``有劳王公公了。这个时候,皇上在做什么呢?''

王钦赔了十足十的笑意:``贵妃娘娘来得正巧,皇上歇了午觉起来批了奏折,现下正歇着呢。挑了南府乐班的几个歌女,正弹着琵琶呢。''

慧贵妃笑了笑道:``皇上好雅兴,本宫进去怕扰了皇上呢。''

王钦笑道:``这宫里说到音律,谁比得过娘娘?要不是怕雪天路滑,皇上肯定请您来了。''

慧贵妃这才道:``那就劳公公去禀一声吧。''

王钦答应着去了。慧贵妃在廊下立了一会儿,果然听见里头琵琶铮铮,正出神,王钦已出来请她了。

因着皇帝在听曲,她入殿便格外地轻手轻脚,见皇帝斜坐在暖阁里,闭着眼打着拍子。数步外坐着三五琵琶伎,身着羽蓝宫纱,手持琵琶挡住半面,纤纤十指翻飞如莹白的蝶。

慧贵妃见皇帝并未察觉她的到来,便也垂手立在一边静静听着。等到一曲终了,方欠身见过皇帝。

皇帝见了她来,倒是十分高兴,牵过她手一同坐下道:``本想叫你来一同听琵琶,又怕外头天寒地冻的,你本来就畏寒。''皇帝关切道,``朕命齐太医替你调理身体,如今觉得还好么?''

慧贵妃低眉浅笑:``臣妾身子虽然羸弱,但有皇上关怀,觉得还好。所以今日特意来养心殿一趟。''

皇帝握着她的手,眼中微微一沉:``手还是这样凉。王钦,叫人再添两个火盆来,仔细贵妃受寒。''

慧贵妃本来就是弱不胜风的体态,皇帝这般关切,更多了几分女儿娇态:``皇上龙气旺盛,臣妾在旁边,也觉得好多了。''

皇帝眉眼间都是温润的笑意,道:``好好坐着,也就暖过来了。''说罢指着几个琵琶伎道,``方才你在旁边听着,觉得如何?''

慧贵妃娇盈盈道:``如今南府里竟没有好的琵琶国手了么?选这几个来给皇上清赏,也不怕污了皇上的耳朵?''

那几个琵琶伎听了,不由慌了神色,忙跪下请罪。

皇帝扬扬手,示意她们退在一边,微微一笑道:``论起琵琶来,有你这个国手在这儿,朕还听得进别人弹的么?不过是你不在,所以听别人弹几曲打发罢了。''

慧贵妃盈然一笑,愈加显得容光潋滟,一室生春。她随手取过其中一个琵琶伎用过的凤颈琵琶,微微疑道:``怎么现在南府这般阔气了?寻常琵琶伎用的也是这种嵌了象牙的凤颈琵琶么?''

皇帝唇角的笑容微微一滞,那退在一边的琵琶伎便大着胆子道:``奴婢技艺不佳,未免污了皇上清听,所以特别用了最好的琵琶。''

慧贵妃蔑然望了她一眼,见那琵琶伎不过二八年纪,姿容虽不出众,却别有一番清丽滋味,心下便有些不悦:``若没有真本事,哪怕是用南唐大周后的烧槽琵琶,也只是暴殄天物而已。''

那琵琶伎有些怯怯的,低首立在一旁。慧贵妃一眼望去,琵琶伎所用的器乐中,只有这般凤颈琵琶音色最清,便横抱过琵琶,轻轻调了调弦,试准了每一个音,才开始轻拢慢捻,任由音律旋转如珠,自指间错落滑坠,凝成花间叶下清泉潺潺,又如花荫间栖鸟交颈私语,说不尽的缠绵轻婉,恍若窗外严寒一扫而去,只剩了春光长驻,依依不去。

慧贵妃一眼望去,琵琶伎所用的器乐中,只有这般凤颈琵琶音色最清,便横抱过琵琶,轻轻调了调弦,试准了每一个音,才开始轻拢慢捻,任由音律旋转如珠,自指间错落滑坠,凝成花间叶下清泉潺潺,又如花荫间栖鸟交颈私语,说不尽的缠绵轻婉,恍若窗外严寒一扫而去,只剩了春光长驻,依依不去。

一曲而过,皇帝犹自神色沉醉,情不自禁抚掌道:``若论琵琶,宫中真是无人能及晞月你。''

慧贵妃扬了扬纤纤玉手,颇为遗憾道:``可惜了,臣妾手发冷有点涩,又用不惯别人的琵琶,此曲不如往常,让皇上见笑了。''

皇帝颇为赞许:``已经很好了。''他似想起什么,向外唤了王钦入内道,``贵妃说手冷。朕记得吉林将军今年进贡了玄狐皮,统共只有两条,一条朕赐给了皇后。还有一条,就赐给贵妃吧。''他含笑向晞月道,``若论轻暖,这个不知胜了紫貂多少倍,给你最合适了。''

晞月一双剪水秋瞳里盈盈都漾着笑意:``这倒是巧了。方才皇后也赏了臣妾一条玄狐皮,也说是吉林将军进贡的,看来这样好东西,注定是都落在臣妾宫里了。''

皇帝眼中闪过一丝欣慰之色:``皇后贤惠大方,对你甚是不错。如此,这两条都给你就是了。只不过朕的心意比皇后多一分,王钦,你便拿去内务府着人替贵妃裁制了衣裳再送去咸福宫吧。''

王钦答应着又招了招手,引了一班乐伎去了。皇帝不动声色地望了一眼其中一个,只见那羽蓝宫装消失在朱红殿门之后,方低低笑道:``如何?''

晞月嗤地一笑,别过身子道:``什么如何?皇上疼臣妾是假的,疼娴妃才是真的。''

皇帝笑着摇首:``这样的话,也就你说罢了。朕难得才去看娴妃一次,怎么倒是不疼你了?''

晞月露出三分委屈的样子:``臣妾今儿听说,皇上特赐御笔给娴妃,娴妃兴兴头头让内务府做了匾额挂在延禧宫的正殿里。偏臣妾的咸福宫里那块匾额都不知道是谁写的,金粉也不足了。娴妃这样的荣耀,臣妾指望都指望不上。''

皇帝扬了扬唇角,失笑道:``原来你是喜欢那个。朕不过是想娴妃住的延禧宫不如你的咸福宫多了,怕看着寒酸才随手写了一幅字给她。''

晞月牵住皇帝的衣袖盈盈道:``既然是随手,皇上不如也赐给臣妾和皇后一幅。省得满宫里只有娴妃有,臣妾羡慕还来不及。''

皇帝刮一刮她小巧的鼻头:``你有什么羡慕的,朕什么好的没给你?只这一样,你也喜欢?''

晞月半是委屈半是撒娇:``皇上终日忙于朝政,臣妾在后宫日夜盼望,若能见字如见人,也可以稍稍安慰。''

皇帝微微沉吟,顷刻笑道:``好了。这有什么难的?你既惦记皇后,朕赐给你和皇后就是了,也许你们做成匾额,挂在正殿里。这下可满意了么?''

晞月这才娇俏一笑,温顺地伏在皇帝肩头,柔声道:``臣妾就知道,皇上最疼臣妾了。''

皇帝微微沉吟,顷刻笑道:``好了。这有什么难的?你既惦记皇后,朕赐给你和皇后就是了,也许你们做成匾额,挂在正殿里。这下可满意了么?''

晞月这才娇俏一笑,温顺地伏在皇帝肩头,柔声道:``臣妾就知道,皇上最疼臣妾了。''

晚膳过后,皇帝着人送了晞月回去,便留在书房摊开了纸行云流水般写起字来。王钦见皇帝在绵白的洒金大纸上写了十一幅字,便在旁磨着墨汁赔笑道:``皇上对皇后和慧贵妃实在是格外恩典。奴才愚心想着,皇上的字自然都是好的,原来皇上还要在这十一幅里选了最好的赏赐呢。''

皇帝见他满脸堆笑,也不说话,只将毛笔搁在青玉笔山上,含了笑意一张张看过去。皇帝侧首,见侍奉在书房门口的李玉一脸了然笑意,便问:``王钦是这个意思。李玉,你怎么看?''

李玉怔了一怔,回道:``奴才愚笨,以为皇上恩泽遍布六宫。延禧宫已然有了一幅字,这十一幅自然是六宫同沐恩泽了。''

皇帝击掌笑道:``好。算你聪明。''皇帝一幅幅细赏下来,自己也颇得意,一一念道:``咸福宫是滋德合嘉,许慧贵妃福德双修的意头;皇后的长春宫是敬修内则,皇后最敬祖宗家法,这幅字最适合她不过。钟粹宫是淑慎温和,与纯嫔的心性最相宜,也算安慰她亲子不在身边的失意;太极殿是淑容端贤\ldots\ldots{}''

王钦忙凑趣道:``嘉嫔娘娘容色冠后宫。''

皇帝微微颔首:``景阳宫是柔嘉肃静;承乾宫是德成柔顺;永和宫是仪昭淑慎;储秀宫是茂修内治;翊坤宫是有容德大;永寿宫是令仪淑德;景仁宫是德协坤元。''

王钦奇道:``景仁宫也有?''

皇帝道:``景仁宫皇后已经过身,你着内务府好好修整下,以后总要有人住进去的。''

王钦忙答应了,皇帝瞟了眼伺候在旁的李玉,笑道:``方才你机灵,那朕就把这十一幅字送去内务府制成匾额的事,交给你了。''

李玉受宠若惊,只觉得光彩,忙恭声道:``奴才谢皇上的赏。''

皇帝奇道:``这赏干你什么事?''

李玉喜滋滋道:``这赏是皇上给六宫小主娘娘的,奴才有幸接了这个差事,自然是沾了福气的,所以谢皇上的赏。''

皇帝忍不住乐道:``是会说话。朕用剩下的这张洒金纸,就赏给你了。''

李玉喜得忙磕了头,起身才看见王钦脸色阴沉,吓得差点咬了舌头,忙捧着纸退下了。

皇帝似有些倦了,问:``什么时辰了?''

王钦忙道:``到翻牌子的时候了。皇上,敬事房太监已经端了绿头牌来,候在外边了。''

皇帝凝神片刻:``今儿南府来弹琵琶的那个琵琶伎,抱着凤颈琵琶的那个\ldots\ldots{}''

王钦一怔,即刻回过神来:``是南府琵琶部的乐伎,叫蕊姬。''

皇帝按了按眉心,嘴角不自觉地蕴了一分笑意,简短道:``带来。''

王钦只觉得脑袋一蒙,嘴上却不敢迟疑,忙应了赶紧去办。

皇帝凝神片刻:``今儿南府来弹琵琶的那个琵琶伎,抱着凤颈琵琶的那个\ldots\ldots{}''

王钦一怔,即刻回过神来:``是南府琵琶部的乐伎,叫蕊姬。''

皇帝按了按眉心,嘴角不自觉地蕴了一分笑意,简短道:``带来。''

王钦只觉得脑袋一蒙,嘴上却不敢迟疑,忙应了赶紧去办。

长街的积雪已被宫人们清扫得干干净净,缓步走在青石花砖上,两旁堆雪映着红墙碧瓦,越发觉得雪光炫目,犹如白日一般。

如懿扶着海心的手慢慢走着,前头两个小太监掌着羊角宫灯,只见冷风打得宫灯走马灯似地乱晃,四周唯有阴森寒气贴着朱墙呼啸而过,卷起碎雪纷飞。海兰有些害怕,紧紧依偎在如懿身边。

如懿安抚似地拍拍她的手,歉然道:``这么晚了,还要你陪我去宝华殿祈福,实在是难为你了。''

海兰靠在她身边挽着手慢慢走着,眼里却有几丝欢悦:``我一个人待在宫里也闷得慌,贵妃她又\ldots\ldots{}''她欲言又止,``还好能陪姐姐去宝华殿听听喇嘛师父诵经,心里也安静许多。''

如懿道:``佛家教义,本来就是让人心平气和的。我去和大师们一同念念经文,将这些日子抄的《法华经》烧了,也是了了自己的一桩心愿。''

海兰往四下看了看,紧张地道:``姐姐别说,别说了。''

如懿含了一脉坦然笑意:``别怕,只有你明白罢了。亲人不在身边,咱们在世的人也只是尽一点哀思罢了。''

海兰微微点头,触动心事,眉梢便多了几分落雪般的伤感:``海兰父母早亡,只有姐姐在身边,不过姐姐在,我心里也安稳多了。''她说着,将自己单薄的身形更紧地往如懿身边靠了靠,仿佛只有这样,才能抵御冬日里无处不在的侵骨寒意。

如懿懂得地握了握她削薄的手腕,仿佛形影相依一般:``你常来看我是好的,但被贵妃知道,只怕又要刁难你。''

海兰轻声道:``我都惯了。''

两人正低声说着话,忽然听得车轮辘辘碾过青砖,一辆朱漆销金车便从身畔疾驰而过。如懿将海兰拦在身后,自己躲避不及,身上的云白青枝纹雁翎氅便沾了几点车轮溅起的浊泥。

犹有余香散在清冷的空气中,缠绵不肯散去。海兰诧异道:``是送嫔妃去侍寝的凤鸾春恩车!''

如懿顾不得雁翎氅上的污浊,惊异道:``今夜并不曾听说皇上翻了牌子,这凤鸾春恩车走得这样急,是谁在上面?''

海兰嗅了嗅空气中残余的甜香,亦不免惊诧:``好甜郁的香气啊,贵妃都不用这样浓的熏香,会是谁呢?''

二人相视疑惑,只听得宫车辘辘得去得远了,那袅袅余音车过深雪,有两轮深深的印迹便似碾在了心上,挥之不去。

这一日清晨,嫔妃们一早聚在皇后宫中,似是约好了一般,来得格外整齐。殿中一时间莺莺燕燕,珠翠萦绕,连熏香的气味也被脂粉气压得淡了不少。

皇后尚在里头梳妆,并未出来。嫔妃们闲坐着饮茶,莺声燕语,倒也说得极热闹。怡贵人忍不住道:``昨儿夜里吹了一夜的冷风,呜咽呜咽的。也不知是不是妹妹听岔了,怎么觉得好像有凤鸾春恩车经过的声音呢?''

嘉贵人冷笑一声,扶了扶鬓边斜斜坠下的一枚鎏金蝉压发,那垂下的一绺赤晶流苏细细地打在她脂粉均匀的额边,随着她说话一摇一晃,眼前都是那星星点点的赤红星芒。嘉贵人悠悠说道:``不是怡贵人你听岔了,而是谁的耳朵也不差。扫过雪的青砖路结了冰,那车轮声那么响,跟惊雷似的,谁会听不见?''

海兰忍不住道:``别说各位姐姐是听见的,嫔妾打宝华殿回来,正见凤鸾春恩车从长街上过去,是载着人的呢。''

这下连近来一直沉默寡欢的纯嫔都奇怪了,便问:``我明明记得昨夜皇上是没有翻牌子的,凤鸾春恩车会是去接了谁?''说罢她也疑惑,只拿眼瞟着剥着金橘的慧贵妃,``莫不是皇上惦记慧贵妃,虽然没翻牌子,还是接了她去?''

慧贵妃水葱似的手指,慢慢剥了一枚金橘吃了,清冷一笑:``本宫怎会知道是谁在车里?这种有违宫规又秘不告人的事,左右不是本宫便罢了。''

如懿端着茶盏,拿茶盖徐徐撇着浮沫,淡淡道:``不管是谁,大家要真这么好奇,不如去唤了王钦来问,没有他也不知道的道理。''

慧贵妃媚眼微横,轻巧笑了一声:``这样的事只有娴妃敢说,也只有娴妃敢做。不如就劳驾娴妃妹妹,去扯了王钦来问。''

如懿只看着茶盏,正眼也不往慧贵妃身上瞟,淡淡道:``谁最疑心便谁去问罢。金簪子掉在井里头,不看也有人急着捞出来,怎么舍得光埋在里头呢。''

嘉贵人拿绢子按了按鼻翼上的粉,笑道:``也是的,什么好玩意儿,只怕藏也藏不住。等着看就是了。''

众人正说着,只听里头环佩叮当,一阵冷香传至,众人知是皇后出来了,忙噤声起身,恭迎皇后出来。

皇后扶着素心的手,行走间沉稳安闲,自有一股安定神气,镇住了殿中浮躁心神。皇后往正中椅上坐下,吩咐了各人落座,方静声道:``方才听各位妹妹说得热闹,一句半句落在了耳朵里,什么好事情,这么得各位妹妹的趣儿?''

众人面面相觑,到底是嘉贵人沉不住气先开了口:``臣妾们刚才在说笑话儿呢,说昨夜皇上并没有翻牌子,凤鸾春恩车却在长街上走着,不知是什么缘故呢。''

皇后淡淡一笑,那笑意恍若雪野上的日光,轻轻一晃便被凝寒雪光挡去了热气:``能有什么缘故?不过是咱们姐妹的福分,又多了一位妹妹做伴罢了。''

``多了位妹妹?''嘉贵人忍住惊诧之情,勉强笑道,``皇后的意思是\ldots\ldots{}''

``连着天寒,本宫嘱咐你们不必那么早来请安,所以你们有所不知。方才你们来前,皇上已经让敬事房传了口谕,南府白氏,着封为玫答应。本宫也已经拨了永和宫给她住过去。''

慧贵妃攥紧了手中的绢子,忍不住低呼:``南府?那不是------''

如懿心里虽也意外万分,却忍住了,只与海兰互视一眼,暗暗想,难怪这么重的熏香气息,果然是这么一个玉人儿了。

皇后面上波澜不惊,只抬了抬眼皮看了慧贵妃一眼:``照理说贵妃应该是见过的,听说是一个弹琵琶的乐伎。''

慧贵妃眉头微锁,凝神想去,昨日所见的几个乐伎里,唯有一个眉目最清秀,想来想去,再无旁人。她咬了咬牙,忍着道:``是有一个弹凤颈琵琶的,皇上还嫌她们弹得不好\ldots\ldots{}''

纯嫔郁然吁了口气道:``琵琶弹得好不好有什么要紧,得皇上欢心就是了。''

旁人听了这一句还罢了,落在晞月耳中,虽然说者无心,却直如剜心一般,一刀一刀剜得喉咙里都忍不住冒出血来。她死死抓着一枚金橘,直到感觉沁凉的汁液湿润地染在手上,才意识到自己的失态,忙喝了口茶掩饰过去。

嘉贵人柳眉扬起,不觉带了几分戾气:``南府乐伎,那是什么身份?比宫女还不如。宫女晋封还得一级级来,先从无名无品的官女子开始呢,她倒一夕之间成了答应了。''

皇后和蔼道:``乐伎虽然身份不如宫女,但总比辛者库贱奴好多了。康熙爷的良妃,不是还出身辛者库吗?照样生下皇子封妃,一生荣宠。也因着乐伎不是宫女,皇上格外恩赏些,也不算破了规矩。''

嘉贵人眉心微曲,嫌恶似地掸了掸绢子:``乐伎是什么低贱身份?来日在这里与我们平起平坐,是要和我们闲话南府里的哪个戏子有趣呢,还是她穿上哪身乐伎的衣裳弹起琵琶来最勾魂?咱们已经有一个海常在平时陪着说说丝线刺绣了,如今倒来了个更好的。''

海兰听说到她,却也闷闷地不敢说话。皇后脸上一沉,已带了几分秋风落叶的肃然之气:``好了!''

嘉贵人一惊,也不敢多说了。皇后缓和了口气道:``不管怎么说,玫答应都是皇上登基后纳的第一个新人,皇上要喜欢,谁也不许多一句闲言碎语。本宫只有一句话,六宫和睦,才能子嗣兴旺。谁要拈酸吃醋,彼此间算计,本宫断断容不下她!''

众人诺诺答应了。一时间气氛沉闷了下来,倒是纯嫔大着胆子道:``皇后娘娘,臣妾有一个不情之请,实在是\ldots\ldots{}''

皇后温和道:``有什么事,但说无妨。''

纯嫔踌躇片刻,还是道:``娘娘,昨儿夜里刮了一夜的风,臣妾听着怕得很。臣妾的三阿哥还在襁褓之中,一向怕冷畏寒的。臣妾心中挂念,想请皇后娘娘允准,允许臣妾今日去阿哥所多陪陪三阿哥。''

皇后一时也未置言,只是抿了口茶,方微笑道:``今儿本就是十五,你可以去看三阿哥。祖宗规矩,半个时辰也够尽你们母子的情分了。''

慧贵妃笑言:``可不是?除了皇后娘娘,后宫妃嫔每月初一十五可去阿哥所探望,但都不许过了半个时辰。皇后娘娘常去探望几位阿哥和公主,本宫也跟着去过一次,三阿哥受的照顾比皇后亲生的二阿哥和三公主还好呢。饶是这样,皇后娘娘还千叮万嘱了三阿哥年幼娇嫩,要万事小心。有皇后娘娘这么眷顾,纯嫔你还有什么不足的?难道多陪了一会儿,你的三阿哥到了冬天便不知道冷了么?''

纯嫔被她一席话说得哑口无言,只黯然垂下了眼眸。

皇后宽和一笑:``好了。你在意儿子本宫是知道的。只是阿哥所的事,你放心就是。再这样成日记挂着儿子,还怎么好好伺候皇上呢?''

慧贵妃本在最后,正起身要走,见皇后向她微微颔首,便依旧坐在那儿,只剥着金橘吃。

待到众人散尽了,皇后方叹了口气,揉着太阳穴道:``暖阁里有上好的薄荷膏,你替本宫来揉揉。''

慧贵妃答应着跟着皇后进了暖阁。素心取出一个暗花纹美人像小瓷钵来搁在桌上,便悄然退了下去,慧贵妃会意,打开一闻,便有冲鼻清凉的薄荷气味,直如湃入霜雪一般,登时清醒了不少。她用无名指蘸了一点替皇后轻轻揉着,低声道:``不是臣妾小心眼儿,皇上纳了这样一个人,实在\ldots\ldots{}''

皇后轻轻吁了口气:``身份低贱也就罢了,只要性子和顺总是好的。你却不知道她的来历\ldots\ldots{}''

慧贵妃愈加惊疑:``什么来历?''

皇后仿佛无限头痛,泠然道:``本宫只当皇上封了个嫔妃,也没往心里多想。谁知让赵一泰去南府问了底细,才知道那白氏竟是和她有关的。''

慧贵妃大惊失色:``娘娘的意思是\ldots\ldots 娴妃!''她越想越不对,恨声道,``果然呢!臣妾以为皇上不太去她那里,她便安分了。原来自己争宠炫耀不算,暗地里竟安排了人进来,真是阴毒!''

皇后用手指蘸了点薄荷膏在鼻下轻嗅片刻,才觉得通体通泰许多:``不是她阴毒,是咱们整日里以为高枕无忧,疏忽大意了。一个不留神就出来一个玫答应,她若是个好的也罢了\ldots\ldots{}''

慧贵妃切齿道:``南府里出来的,能有几个好的?一个个狐媚惑主,轻佻样儿。臣妾方才想起来,昨日臣妾觉着她们琵琶技艺不佳,随口说了一句,便有一个胆子大的敢当着皇上回臣妾的话。一个两个都是这样胆大包天的,能有什么好的?''

皇后倒吸一口凉气,诧异道:``当着你的面也敢如此,那就真不是个安分的了。''她隐然忧道,``本宫顾着后宫千头万绪的事情,总有顾不到的地方。你是贵妃,一人之下众人之上,你若不替本宫看着点、警醒着点,哪日我们姐妹被人都算计了去都不晓得!娴妃近来无宠,可她才十九岁,来日方长\ldots\ldots{}''

慧贵妃微微失神,按着太阳穴的手也不觉松了下来:``臣妾已经二十五了\ldots\ldots{}''

皇后的手轻轻搭在慧贵妃纤白的手上,低低道:``你二十五,本宫也已经二十五了。''她语气一凛,旋即沉声道,``二十五又如何?只要咱们眼光放得长远,万事顾虑周到,一个人眼睛不够,另一个人帮衬着,总不会有顾不到的地方,也容不得狐媚子媚宠。当日本宫分配殿宇的时候,特特把海兰放在你宫里,你知道是为何么?''

慧贵妃听得皇后语气沉稳,心下也稍稍安慰,忙道:``潜邸之时,除了臣妾与娴妃、嘉贵人,其余人等都不算得宠。皇后娘娘将海兰放在臣妾宫里,是要防着她哪一日又偷偷狐媚了皇上。皇后娘娘放心,皇上快连她是谁都不记得了呢。''

皇后的目光在她脸上轻轻一转,见她只是一副笃定的样子,不觉摇头道:``这虽然是其中一个原因,但却不是最要紧的。海兰向来不得宠,所以对皇上而言,既是一个记不得的人,也很可能会成为一个新鲜人儿。你防着她不错,但更要防的是娴妃与海兰的亲近。''

慧贵妃旋即会意:``娘娘的意思是说,海兰也会成为第二个玫答应。''

皇后沉静道:``那也未必。但凡事不能不多长个心眼。你自己宫里的人,自己留心着吧。''

这边厢延禧宫里也不安静,如懿正站在廊下看着从内务府领来的冬日所用的炭火份例。小太监宝成领着几个人数清了,上来回话道:``娘娘,已经数清了,黑炭一千二百斤,红箩炭三百斤,都已经在外头了。''

如懿点点头,问道:``海常在那儿如何?''

宝成道:``按着常在的位份,没有红箩炭,只有按着每日二十斤的黑炭算。但是奴才方才打内务府过来,听说\ldots\ldots{}''

如懿蹙眉:``说话不用吞吞吐吐,听说什么\ldots\ldots{}''

宝成吓得吐了吐舌头,忙说:``听说海常在宫里总说黑炭不够用,可那份例是定了的,哪有再多。怕是海常在正受着冻呢。''

阿箬替如懿将刚笼上的手炉捧了来,细心地套上一个紫绒炉套才送到如懿手里,轻声道:``外头风大,小主仔细被风扑了脑仁,回头着了风寒。''

如懿笑道:``总关在屋子里闷得慌,这儿避风,倒也不怕。''

阿箬又道:``听宝成说这话,海常在一向是老实的,若不是冻得受不住,怕也不会去跟内务府再要炭了。只不知她宫里统共就那两个人,怎么会不够呢?''

如懿叹息道:``这就是她的难处了。昨儿夜里我和她都在宝华殿诵经祈福,才摸到她的手炉温温的,居然都不热。我还以为是伺候她的叶心和香云不仔细,谁知道问了一句,她眼睛都红了,说是份例的炭根本不够用,她那西晒的屋子本来就冷,平日里烧一个火盆就勉勉强强了,哪里还顾得到手炉脚炉。我这才知道,她的日子竟这样难过。''

阿箬整了整身上一色儿的暗紫色宫装,宽慰道:``这也不能怪小主。贵妃向来和小主不睦,小主自然不便去她的咸福宫看海常在,否则怎会顾不到?要说起来,也是贵妃太不当心了,由着自己宫里人受苦。''

如懿心下难过,忍气道:``按理说海兰只有两个丫头、两个太监,东西自然不会不够。但她告诉我贵妃怕冷,总嫌着宫里不够暖和,内务府送来的炭都是克扣了大半才给她的。她自己也就罢了,连奴才的屋子里都烧得暖烘烘的,也不顾着海兰。''

阿箬倒抽了一口凉气:``那怎么成,再往下正月里、二月里冻得不行,海常在怎么受得住?''

如懿叹了一声:``这何尝不是我的不是,为了避嫌避祸,这样委屈了她。若我仔细些早发觉了,她也不必这样受冻。''她唤过宝成,``你仔细些,悄悄儿送些炭到海常在那儿,别叫人留意着。还得记得只能是黑炭,她的位份不能用红箩炭,那红箩炭烧了的炭灰是银白的,一眼就叫人认出来了,反而不好。黑炭却是看不出多少的。''

宝成应了一声道:``奴才明白。会趁贵妃去请安时隔几天送一次,免得送多了点眼。''

如懿满意微笑:``那就赶紧去吧。还有,内务府拨来的冬衣,你也挑一批好的,悄悄儿送过去。''

阿箬看宝成下去了,便道:``小主待海常在也算有心了,天刚冷的时候就送了好些新棉去,如今又送衣裳。''

如懿颇有触动:``这宫里有几个人是好相与的?海兰也算和我投契了,彼此照应些也是应当的。''她转过脸问阿箬,``方才让你去永和宫送些薄礼给玫答应,可打听到了什么?''

阿箬眼光往四周一转,忙轻声道:``奴婢奉小主之命送了两匹妆花缎过去,谁知道永和宫可热闹了呢,嘉贵人和怡贵人都送了东西去,连慧贵妃也赏了好些东西呢。''

如懿念及什么,便问:``那纯嫔\ldots\ldots{}''

``奴婢去的时候纯嫔宫里还没送东西去呢。''

如懿明白,刚离了皇后宫里,纯嫔一定是紧赶着去了阿哥所看望儿子。即便回来了,也必定伤感儿子不在身边,一时也怕顾不到这些礼数。她便道:``那等下我去钟粹宫看看纯嫔,她也可怜见儿的。''

阿箬又道:``奴婢特意拜见了玫答应。虽然是答应,但永和宫的布置,玫答应的打扮,是比怡贵人还尊贵呢。可见虽然才侍寝了一次,皇上却是极喜欢的。''

话音未落,却听嘉贵人一把婉转嗓音自院外传入:``皇上怎么会不喜欢玫答应?吹拉弹唱的有什么不会?又是人家一手调教出来的好人儿!''

如懿微一扬眸,就见金玉妍穿了一身玫瑰紫百蝶穿花大毛斗篷,扶着侍女丽心的手风摆杨柳似地进来。玉妍见了如懿便躬身福了一福,笑声冷冽如檐下冰:``恭喜娴妃,贺喜娴妃了。''

如懿一怔,旋即笑道:``嘉贵人这句话合该对着永和宫的玫答应说,怎么错到了延禧宫呢?''

嘉贵人冷笑一声:``嫔妾没这样好的本事,调理得出花朵儿一样的人儿吹拉弹唱,歌舞迎人。娘娘一手栽培出了这样得意的人来,怎么不算喜事呢?''

如懿心下含糊,虽不知出了什么事,却听得金玉妍句句话都冲着自己来,便也不假辞色:``嘉贵人一向快人快语,今儿有话也不如直说,本宫洗耳恭听。''

``洗耳恭听?''嘉贵人盈盈一笑,那笑意却似这天气一般,带了犀利的寒气,``娴妃娘娘听琵琶曲儿听得熟了,何必今日早上要和咱们一样糊涂,还议论玫答应的来历呢?''

如懿听她提得``来历''二字,心中越发糊涂。却见金玉妍一脸了然,想是什么都知道,与其自己揣测,还不如听她说来。如懿只得道:``不管嘉贵人说什么,关于玫答应的来历,本宫真是懵然不知。若是嘉贵人觉得不必白来这一趟延禧宫,不如赐教告诉本宫一声,也好教本宫落个明白。''

嘉贵人姣好的长眉轻轻一挑,疑道:``你是真不知还是假不知?''

如懿坦白:``真不知。''

嘉贵人似信非信地挑眉看着她,缓了口气道:``玫答应不是娘娘母家乌拉那拉府邸送进南府的么?''

如懿与阿箬对视一眼,彼此俱是愕然,嘉贵人见她神色不假,也有几分信了:``你真的不知道?''

如懿走到廊下,坦诚道:``这件事本宫也是毫不知情,正打算让阿箬去打听的。妹妹若是知道,不妨直言。''

嘉贵人冷冷看了她一眼:``玫答应是先帝雍正八年,你母家乌拉那拉府邸送进来的人。''

如懿凝神想了一想:``雍正八年本宫才十四岁,如何能得知这些事?''

嘉贵人抚着指上尖尖的护甲:``你不知道,不代表当年的景仁宫皇后不知道。慧贵妃和嫔妾已查问过,当年玫答应入南府,是景仁宫皇后允的。你当年虽不知情,难道后来也一无所知吗?何况玫答应突然得宠,也太奇怪了些。其中关节,也只有娘娘你自己知道了。''

金玉妍言毕,扶了丽心的手径自离去。唯余如懿站在院中,看着檐下冰柱滴答落下冰水来,一滴一滴,敲在她疑惑不定的心上。

\hypertarget{ux7b2cux5341ux4e09ux7ae0-ux854aux59ecux4e0b}{%
\chapter{第十三章
蕊姬(下)}\label{ux7b2cux5341ux4e09ux7ae0-ux854aux59ecux4e0b}}

这一夜是腊月初一,皇帝照例宿在皇后宫中。如懿听着窗外风声凄冷,雪落绵绵,正对着灯想着心事,却见阿箬进来,抖落了一身的雪花,近前道:``小主。''

如懿将自己壶中的茶倒了一碗递给她,又将暖炉给她捧:``先喝杯热茶暖一暖。''

阿箬冻得抖抖索索的,一气把那茶喝尽了,方暖过来道:``都打听清楚了。玫答应的确是出自咱们府里,也是老主子手里进来的人。不过那年先帝选充南府的乐伎,各府里都挑了好的送进来,倒也不止咱们一家。奴婢问过了,玫答应今年十七,是十二岁的时候送进来的。''

火盆里一芒一芒的红箩炭烧得极旺,不时迸出几星通红的火点子。如懿慢慢地拨着指甲,凝神道:``难不成姑母这么早就布置下了人在宫里?只是有这么个人,姑母也不曾向我提过一句呀。''

阿箬拧着辫子道:``奴婢也是这么想。只不过最后那几年老主子自顾不暇,与小主也来往不多,浑忘了也是有的。''

如懿点点头:``也许也是咱们想多了,不过是各府里都送了人进来,咱们恰巧也有一个罢了。落在别人眼里,疑心便生了暗鬼,以为是我唆使了送去皇上那儿的。''

阿箬道:``可不是?什么乱七八糟都往咱们头上栽,小主可别再那么好性子了。什么时候冷不丁给她们一下,她们就知道厉害了。''

如懿一笑:``再厉害也厉害不过你的嘴!''她蹲下身,拿起乌沉沉的火筷子拨着火盆里的炭,底下冒出一阵香气,阿箬嗅了嗅鼻子,喜道:``好香!是烤栗子的味道!''

如懿笑道:``知道你爱吃,你刚出去我就往火盆里扔了好几个栗子,这会儿正好。你自己拿火筷子夹出来,仔细烫手。''

阿箬忙不迭地笑着答应了,取出烤得爆开的栗子,顾不得烫,就剥开吃了起来。

暖阁里灯火通明,隐隐地透着栗子的甜香,主仆俩相视一笑,倒也开怀。

此后连着几日,但凡有侍寝,必是永和宫的玫答应,得宠之深一时风头无两。加之数日鹅毛大雪,出门不便,皇后免了晨昏定省,一时之间众人对这位未曾谋面的玫答应存了无数好奇之心。

好容易五六天后雪止晴霁,终于能出门了。这日的宫中请安,众人便到得格外早。

果然才坐定陪皇后聊了几句,殿外便有太监通传:``玫答应到了。''

听得这一声,本来还在笑语连珠的嫔妃们都静了下来,不自觉地向外看去。

只见殿门豁开,一个身着樱桃红绣栀子花蝶苏缎旗装的女子低着头盈盈走进,她梳着精巧的发髻,发间不用金饰,只以碧玺花朵零星点缀,髻上斜两枝雪色流珠发簪,卷起的鬓边嵌着一粒一粒莹莹的紫瑛珠子。待到走得近了,才看出她的衣裙上绣着一小朵一小朵浅绯的栀子花瓣,伴着银线湖蓝浅翠的蝴蝶,精绣繁巧轻灵如生,仿佛呵口气,便会是花枝展天地,春蝶翻飞于衣裾之上。

慧贵妃见她早不是昔日打扮,冷笑一声:``狐媚!''

因是玫答应一直低着头,虽未看清模样,嘉贵人已然奇道:``咱们冬日的衣衫厚重,怎么她这一身却轻薄,好像不怕冷似的?''

纯嫔坐在她身旁,低低道:``听内务府说江宁织造新贡了一种暖缎,虽然轻薄,却十分暖和。''

嘉贵人郁然叹了口气道:``自从皇上登基,皇后下了命令,不许用纯金的首饰,不许金线织衣,更不许用江南的好料子,说是靡费。如今看她这一身衣裳便是苏缎的料子,只是个答应也用了银线织绣,虽未用金饰,可那碧玺又如何不贵重了?''

纯嫔轻轻摇了摇头,示意她噤声。

玫答应低头欠身,行了一礼:``臣妾永和宫答应白氏参见皇后娘娘、各位小主。皇后娘娘万福金安,各位小主顺心遂意。''

皇后含了一缕妥帖雍容的笑意,和言道:``这便是玫妹妹了,本早应相见的。只是一直大雪,到了今日才得见。起来吧,莲心,赐座。''

玫答应抬起头来,众人见她这般盛装打扮,只以为是个千娇百媚的美人,谁知仰起面来,不过是个白净娇丽的面孔,虽然十分清秀,但也只是中上之姿而已。旁人倒还不觉得怎样,嘉贵人先不由自主地松了口气,只低头拨着自己手腕上的银镶珠翠软手镯,笑吟吟地不说话。

莲心这人在海常在之后添了一张椅子请玫答应坐了,又殷勤端上茶来。

玫答应倒也不羞怯,朗声道:``本该早些来拜见皇后娘娘的,可惜一直天公不作美,到了今日才能来。''

皇后向上挑起的唇勾勒出一朵和婉的笑纹:``来与不来,都只是一份心意。以后朝夕相见,你就知道各位姐妹都是好相处的了。''说罢便由莲心一一指了妃嫔引她见过。

嘉贵人轻声笑道:``不仅咱们是好相处的,皇上也格外疼妹妹啊。妹妹这身料子,轻薄暖和,是江宁进贡的暖缎吧。''

玫答应淡淡笑道:``嘉贵人好眼力。''

嘉贵人唇际欲笑未笑:``不是我好眼力,而是乍一看见妹妹穿得单薄,害怕冻着了妹妹。原来是皇上的一片心意。只是这暖缎难得,连皇后宫里也都没有,我也只是听说了胡乱一猜罢了。''

嘉贵人娓娓道来,众人难免多了一份醋意,玫答应还是那样淡淡的神情:``是吗?皇上只是赏了我衣裳,别的我不多问,也全不知道。''

嫔妃们见她只是这样疏懒的神情,也知道不好相与。倒是慧贵妃说了一句:``皇上登基后皇后娘娘就一直主张后宫简朴。妹妹只是区区一个答应,这身衣服也略奢华了些。''

玫答应懒懒抬了抬眼:``是吗?皇上喜欢嫔妾这样穿而已。''

慧贵妃一时噎住,不觉有些气恼。

皇后看出几分端倪,朗然道:``好了。外头虽然雪停了,但天寒地冻,路滑难行,大家还是早些回去吧。快到年下了,别冻着身子才好。''

众人答应着散了,便各自上了辇轿回宫。

阿箬替如懿围上云白青枝纹雁翎氅,兜好风毛和暖炉,扶了她的手出去。如懿看着满世界冰雪银妆,便道:``别传辇轿了,那么好的雪景,咱们从御花园慢慢走回去。''

阿箬笑道:``也好。好些天没出来了,闷得慌呢。''

二人正要迈步出去,忽听身后一声唤:``娴妃娘娘留步。''如懿转过头去,却见玫答应携了一个小宫女的手盈然上前,笑道:``娴妃娘娘好雅兴,嫔妾正好想去御花园中赏雪,不知娘娘可否愿意与嫔妾同行?''

如懿笑道:``既然妹妹愿意,独行不如结伴了。''

二人慢慢踱步向前,雪后的阳光虽无多少暖意,但与雪光相映更加显得明亮。多日来的积雪更是将御花园映得白光夺目,恍若行走在晶莹琉璃之中。偶尔有树枝上的积雪坠落至地发出轻微的簌簌之声,越发衬得周遭安静得仿佛不在人世。此时积雪初定,间或有几株蜡梅正开得繁盛。那蜡梅素黄粉妆,色如蜜蜡,金黄灿烂一树,加上梅枝间新雪相衬,呼吸间只让人觉得清芬馥郁,冷香透骨。

如懿不觉深吸了一口气,玫答应察觉,便笑:``娴妃娘娘喜欢梅花?''

如懿伸手攀住一挂蜜冻似的花枝轻轻嗅了嗅,沉醉道:``是,尤其是绿梅,清雅宜人,不落凡骨。''

玫答应道:``娘娘见过绿梅?''

如懿颔首:``小时候和阿玛去苏州,在那时见过两次,实在是人间至美之物。''

玫答应淡淡一嗤,唇边露出三分清冷之意:``嫔妾也是因为擅弹月琴,才被人从苏州买来,后来才机缘巧合被送进宫来。''

如懿奇道:``听闻玫答应出身南府琵琶部,不是应该擅弹琵琶吗?''

玫答应幽然凝眸,墨灰色的忧伤从眸底流过:``嫔妾本来擅长的是月琴,只因入了南府,教习师傅说先帝喜欢琵琶,才改学的。''她零丁的叹息转瞬落在寒风里,``哪里不都一样?喜欢什么,中意什么,都由别人说了算,半点由不得自己。''

如懿听她感伤身世,便试探道:``这句话,你是在怪乌拉那拉府当年把你送进南府么?''

玫答应冷然一笑:``送嫔妾也是送,送旁人也是一样,有什么可怪的?不送嫔妾进南府,嫔妾也不过是府里一个乐伎,漂若浮萍罢了。哪里比得上娴妃娘娘金尊玉贵,连喜欢的花都是骨格清奇的稀世绿梅,相形之下,嫔妾不过是风中柳絮,蒲柳命数了。''

``只可惜这绿梅实在难得。凡事太过清奇,终不容于世长久。娴妃,你说是不是?''

如懿闻声抬首,却见慧贵妃携了人站在不远处一树蜡梅下,手中折了两枝蜡梅,盈盈向她笑语。

如懿见了她,便与玫答应屈身行礼道:``给贵妃请安。''

慧贵妃吩咐了``起身'',笑道:``风吹得顺,听见娴妃与玫答应闲聊,倒惹得玫答应自伤身世了。''她笑着向玫答应瞥了一眼,``士别三日当刮目相看,说的就是玫答应啊。''

玫答应微微低首:``再相见,贵妃娘娘雍容华贵,风姿依旧。''

慧贵妃细细打量着她,最后将目光落在她水葱似的纤纤指尖:``这么会说话,南府里应该选你去唱曲儿,只弹琵琶是可惜了。倒还没问过妹妹,叫什么名字呢?''

玫答应不信她不知,却还是答道:``嫔妾姓白,名蕊姬。''

慧贵妃唇角漾着甜美的笑意,眼中的清冷却与这冰雪并无二致:``果然是个好名字,一听生来就是供人赏玩取乐的。''

玫答应眉心一跳,脸上却平静无波:``命里注定的,若能供皇上一时之乐,就是嫔妾无上的福泽了。''

慧贵妃笑意顿敛,冷冷道:``别以为封了个答应,你的荣宠就长久了。你那一手琵琶,皇上闲时听听当麻雀唧喳似的听个笑话儿,还真当自己成了凤凰清啼么?''

玫答应不卑不亢,只蕴了一抹淡淡笑意,悠然望着天际道:``嫔妾自知琵琶不如贵妃娘娘,姿容也不如贵妃娘娘。可是娘娘想过没有,为什么皇上放着娘娘这一手琵琶绝技不听,只喜欢嫔妾这些不入流的微末功夫呢?''

慧贵妃神色一冷,还不及回嘴,玫答应眼波悠悠在她面上一转,恍若无意般望着近处一树怒放的蜡梅,悠然道:``岁月匆匆,不饶人啊!''

慧贵妃脸色大变,只见一张粉面渐次苍白下去,直如枝丫上透白的积雪一般,脚下微微一个踉跄,身边的宫人忙牢牢扶住了。

如懿听得不对,立刻呵斥道:``放肆!贵妃和本宫面前岂容你胡言乱语,肆意犯上!''

玫答应毫不畏惧,笑声落在雪野中恍若檐下风铃一般清脆玎玲:``娴妃娘娘别吃心,娘娘只比嫔妾长了两岁,岁月怎舍得薄待了娘娘?嫔妾说的是谁,那人心里自然清楚!''

如懿本是好意,念在同出于乌拉那拉氏门下,想替她圆了过去。谁知蕊姬毫不领情,越发指着慧贵妃不依不饶。饶是如懿这样的外人,听了亦觉得下不来台去。

慧贵妃才一站稳,听得这一句,脸上腾地红了起来,显是怒到了极点。她的目光如利剑一般,恨不能在玫答应年轻饱满的面孔上狠狠刺出两个血洞来。片刻她口中迸出两个字:``掌嘴!''

那话音掷地有声,不容半句辩驳。慧贵妃身边的首领太监顺成一个抢身,摁住了玫答应的肩就要往下按。偏生那玫答应是南府出身的,身段水蛇儿似的,轻轻一拧便扭开了。顺成一个手快,这下再不留情,往她膝弯里狠狠一踢,玫答应吃痛,一下就跪在了雪地里。顺成一个耳光就要扇上去,玫答应如何肯受辱,喝道:``我是皇上亲封的嫔妃,怎容你一个奴才欺辱?''

顺成稍一犹豫,摁着玫答应肩膀的手却丝毫不肯放松。

如懿看情势不好,忙求道:``贵妃娘娘,蕊姬刚成答应不久,宫中的规矩礼数还没有都懂得,但请贵妃宽恕,饶了她一遭吧。''

慧贵妃冷冷一笑,理也不理如懿,只看着玫答应道:``自己才从奴才堆里爬出来,就嫌弃人家是奴才不配动你了?你是皇上亲封的答应,本宫是皇上亲封的贵妃,云泥之别,你敢冒犯本宫,就活该要受责罚!顺成,给本宫狠狠掌她的嘴!''

话音刚落,玫答应雪白娇嫩的脸颊上便已经狠狠挨了一掌。顺成显是用足了力气打下去,玫答应的左侧脸颊立刻高高肿起,嘴角溢出猩红一抹血痕。她犹自不怕,仰着头道:``旁人说奴才两个字就罢了,贵妃娘娘自己也是包衣奴才出身,和嫔妾有什么两样,又谁比谁高贵了?''

慧贵妃自抬旗为高佳氏之后,平生最恨人提起她是汉军旗包衣出身,生生地比如懿矮了一截。此时又正当着如懿的面,她愈加气得浑身发颤,指着玫答应厉声道:``顺成,她这样不知死活,你也不必留情!给本宫狠狠地打,打到她老实为止!''

这一吩咐,顺成更落了十二分的力气,又狠狠扇了两下。如懿转过头不忍去看,那声音却噼啪响亮入耳,想躲也躲不过去。

突然耳边利落一声``住手'',众人闻言转身,却见浩浩荡荡一行人,前导四人执销金凤首提炉,随侍太监在后执翟扇、掌曲柄五色九凤伞,色彩灼灼,在纷白雪地中格外夺目。皇后身边的赵一泰走在前头喝道:``皇后娘娘驾到!''

众人一个醒神,忙一齐屈身下去,齐声道:``皇后娘娘万福金安。''

皇后的神色并不好看,一时也未叫``起来'',居高临下看着众人:``本宫本想去阿哥所探视几位公主阿哥,谁想才走到这里,就听见你们喧哗吵闹,毫无体统!''她的目光从贵妃、娴妃、玫答应身上从容滑过,带了几分沉肃之意,``这里是宫中御苑,不是你们自家的刑场,容得你们在这儿失了皇家的体统!''

慧贵妃恨恨瞟了玫答应一眼,努力挤出几分笑色,回禀道:``皇后娘娘息怒。娘娘有所不知,玫答应出言狂妄,肆意犯上,不仅讥笑臣妾出身包衣,又讥讽臣妾人老珠黄\ldots\ldots{}''

玫答应毫不示弱,仰起脸露出唇角两道血痕,在她雪白面孔上尤显得凄厉狰狞、``皇后娘娘明鉴,臣妾是说过慧贵妃出身包衣,但就因贵妃出身包衣才有今天的荣宠,这话并没有错。但贵妃娘娘所言`人老珠黄',臣妾绝对没有说过这四个字,只是叹息岁月匆匆罢了。''她转头看了如懿一眼,``皇后娘娘若是不信,大可问一问娴妃娘娘。''

如懿听她辩驳,虽然意指贵妃人老珠黄,但的的确确没有说出``人老珠黄''四个字,只得回道:``方才玫答应的确是出言不敬,但`人老珠黄'四个字,确实是没有说过。''

慧贵妃愈加不忿:``她虽没有说过这四个字,但的的确确就是这个意思。娴妃你如此纵容包庇,要说和玫答应绝无勾连,本宫实在不信!''

如懿心中一惊,再想分辩,想想慧贵妃已然认定,再多言也是无济于事,索性别过脸去不再理会。

皇后脸色一沉,喝道:``好了。各人有各人的意思,一时误会也是有的。''她缓了缓声气,和颜道,``玫答应是新晋嫔妃,自然有礼数不周的地方。你是仅次于本宫的贵妃,管教约束也是应该的。既然掌嘴也掌了,脸也成了这个样子,罢了,都起来吧。''

``玫答应是新晋嫔妃,自然有礼数不周的地方。你是仅次于本宫的贵妃,管教约束也是应该的。既然掌嘴也掌了,脸也成了这个样子,罢了,都起来吧。''

众人忙谢过起身,玫答应倔强道:``皇后娘娘,臣妾的确言语有失,但贵妃娘娘气急败坏便叫掌嘴。臣妾新侍皇上不久,就损伤了容颜,皇上若是问起,臣妾不敢不答。''

皇后看她的目光并不含任何温情:``皇上若是问你,你们各执一词,皇上谁的也不会听。本宫只会秉公直言。你错在言语犯上,贵妃罚你不错,只是罚你的人下手太重罢了。你要再不安分,频频生事,本宫也不会容你!''

皇后甚少以这样的口吻说话,如懿知道利害,忙在后头悄悄拉了拉玫答应的披风。玫答应听得皇后如此语气,一时也不敢再言。

皇后见众人都是默然无声,便向如懿温和道:``娴妃,这件事你未曾过多参与。这样吧,就由你送玫答应回去,好好劝解她几句。''

如懿本不欲接这差事,免得众人都以为她真与蕊姬有何勾连。可偏偏方才有些话没有问完,想想既然身在这嫌疑里,一时也避不开,便也答应了。

慧贵妃见二人去得远了,忍不住愤愤道:``皇后娘娘宽厚仁慈,只是这种小婢子出身寒微,轻狂骄纵,若不好好教导规矩,只怕仗着皇上宠爱要翻了天的。''

皇后冷然瞟了她一眼:``打你也打了,雪地里你也让她跪着了。你还要怎样?真打破了脸跪伤了膝盖,皇上问罪下来,你怎么回话?''

慧贵妃赌气道:``臣妾就实话实说罢了。左右也是玫答应自己先错了。''

皇后看了她一眼,摇头道:``她的确是错了,但你是贵妃,你是居上位者,应该有容人之量,这样发作闹起来,只为了几句言语口角,即便真是玫答应错了,皇上也只会怪你心胸不够开阔。''皇后继续推心置腹道,``好妹妹,不是本宫要说你,她是皇上的新宠,无论如何,你都应该要忍过这一时之气。等到时日长了,皇上冷了下来,你要打要罚,皇上不会心疼,反而还觉得你对。你可明白么?''

慧贵妃这才露出几分懊丧之情:``那臣妾已经把她的脸打成这样了,皇上会怪罪臣妾么?''

皇后微微叹息:`你呀!好了,这件事皇上要真过问,本宫会替你圆过去。另外,本宫会让人从太医院拿些清凉消肿的药膏替你送过去。这件事毕竟她也有错,若她知道其中的利害,也不敢随意去皇上那儿哭诉。''

慧贵妃这才稍稍放心,心悦诚服:``有皇后娘娘做主,臣妾就安心了。''

皇后转头吩咐:``素心,你即刻去太医院送些膏药去永和宫,别耽误了。''

素心答应着去了。慧贵妃感激道:``臣妾谢过皇后。''

皇后含了一分欣慰的笑,道:``好了。你若有空,就陪本宫去阿哥所吧。''

慧贵妃忙扶过皇后的手,两人携着手踏雪而去。

\hypertarget{ux7b2cux5341ux56dbux7ae0-ux98ceux6ce2}{%
\chapter{第十四章 风波}\label{ux7b2cux5341ux56dbux7ae0-ux98ceux6ce2}}

如懿陪着蕊姬一路自御花园返回永和宫。因大雪初停,一路上扫雪的宫人并不少,见了二人同行,忙不迭跪下行礼请安。然而蕊姬因掌掴而受伤的面颊格外惹人注目,即便宫人们再低头行礼时,亦不免拿眼偷瞧,并以彼此的眼色来交换诧异与惊奇之情。蕊姬对此似乎浑不在意,既不借阔大的风帽掩盖掩饰伤口,也不喝止宫人们看似无礼的行径,只是施施然行走,仿佛浑不觉旁人的目光与私语。

回到永和宫中,侍婢们赶忙迎接上来,替如懿和蕊姬接过风帽与斗篷,又换过新的手炉。她们见到蕊姬红肿的脸颊,虽然面色惊疑却不敢相问,想是蕊姬这里规矩极严,自己不说,旁人问都不许问一句。如懿四下里扫了一眼,这才察觉,装饰一新的偌大的永和宫中,侍奉的宫人竟比身为贵人的黄绮沄更多。而殿中所用的炭火,也是身为答应根本用不上的红箩炭,烘得一室洋洋如春。阿箬侍奉在侧,不觉露出几分惊异之色。如懿察觉,旋即道:``阿箬,去问问她们有没有消肿的药膏,若没有,赶紧着人去太医院领。''

阿箬答应着出去了,恰好外头小太监进来通报,说内务府送了新做的匾额来要挂在正殿。蕊姬颔首道:``让他们拿进来吧。''

内务府的执事太监恭恭敬敬捧了匾额进来,却是斗大的金漆大字,写着``仪昭淑慎''四字。

如懿即刻便认了出来,含笑道:``玫答应,这是皇上的御笔呢。''

执事太监笑道:``可不是呢,娴妃娘娘好眼力。''

蕊姬将那四个字轻轻读了一遍,道:``这几个字我倒是都认识,但搁在一块儿就不知是什么意思了。娴妃娘娘,你若知道,还请告诉一声儿。''

如懿微微一笑:``《仪礼》中说`敬尔威仪,淑慎尔德',意思是要求女子和善谨慎,以保仪德。''

蕊姬轻轻一嗤,带了几许轻蔑之色:``那么娴妃,你觉得我配不配得上这四个字?''

如懿从容自若:``皇上是将这匾赐给永和宫的,既然皇上许你住了永和宫,自然是以为你担得起这四个字。''

蕊姬的目光逡巡在匾额之上,只是含了一抹冷淡的笑意:``多少人要看见了都会觉得我不配,可是配不配,这都归了我的。''

执事太监赶着差事,忙请示蕊姬:``请玫小主的意思,是不是即刻挂上去?''

蕊姬点点头:``这样的荣耀,当然不肯藏着掖着,赶紧挂起来吧。''

执事太监响亮地应了一声,便带着几个赭衣的小太监开始动手。执事太监一脸的谄媚:``娴妃娘娘、玫小主,这儿钉起匾额来声音太大,怕吵着二位。不如请两位小主挪动玉步,去旁边暖阁稍事休息,奴才们马上就好。''

蕊姬道:``我听了这些声音就烦,娴妃娘娘跟我往暖阁里间去坐坐吧。''

如懿本不想在她这儿多留,想了想还是陪她进去了。

暖阁的里间倒还安静,如懿见服侍的宫人们并没有跟进来,便问:``脸上的伤肿得厉害,叫下人们煮了鸡蛋给你揉揉。''

蕊姬轻笑一声:``这些下人的功夫,我比她们清楚,娘娘放心就是了。''

如懿闻言微微蹙眉:``眼看着你得宠,听你的话,倒像是很介意自己的出身。''

蕊姬举着护甲轻轻划在黄杨木小几上,冷笑道:``能不介意吗?从我第一次侍寝被封答应,一个个乌眼鸡似的盯着我,动不动就拿我的出身来笑话,恨不能生吞了我。''

如懿正坐着:``人的出身是不能选的,你比别人介意,别人就得意了。''

蕊姬黑冷的眸子在她面上轻轻一刮:``原来出身乌拉那拉氏,也是娴妃娘娘的痛处。''

如懿不意她言辞这般犀利,于是凝了一缕静和的笑意:``若本宫不把这个当痛处,别人也不会让本宫觉得痛。''她目光流转,``倒是你,却是被人认定了和本宫一路人,受了不少委屈。其实本宫也很想知道,到底你为何会一夕得幸,平步青云?''

蕊姬的护甲划在小几上发出``刺啦''的锐声,容色并不好看:``旁人都以为嫔妾出自乌拉那拉府第,是受了娴妃娘娘的指使才得幸于皇上,原来娘娘还疑心嫔妾受了旁人指使。''玫答应冷然道,``嫔妾若有本事受谁的指使就好了。这一辈子都是只由得命,由不得人。原以为娘娘生性有几分傲气,才与娘娘多言几句。既然如此,嫔妾要休息了,请便吧。''

她话音未落,小宫女进来:``小主,皇后娘娘跟前的素心姑姑来了,在外边候着呢。''

蕊姬冷冷道:``她来做什么?''

小宫女道:``回小主的话,说是送太医院的药来。''

蕊姬点头:``那就让她进来吧。''

如懿起身要走,蕊姬便道:``方才说话得罪了,但请娴妃替我看一眼,别是送了什么别的来我也不懂。''

如懿想着到底是皇后嘱咐了自己送她来的,此刻素心来了,若自己不在,只怕又是是非,便又重新坐了下来。

素心进来福了一福道:``娴妃娘娘、玫答应,奴婢奉贵妃娘娘的旨意,特意从太医院取了上好的消肿药膏来给玫答应。''

蕊姬冷笑一声:``慧贵妃好善的心哪!刚打了我就送药来,以为打一巴掌给个甜枣就完了吗?这药我还真不敢用。''

素心不防吃了这句话,捧着药膏进退不得,只好求助似的看着如懿:``娴妃娘娘\ldots\ldots{}''

如懿伸手向她:``让我看看。''入手是一个粉瓷圆钵,钵中盛的是淡淡绿色的半透明膏体,扑鼻便是一股清凉香气,隐隐有蜂蜜、薄荷、丹七的气味。她取过一点轻轻一嗅,的确是寻常所用的消肿良药,并无二致。如懿点头道:``宫中平常所用的消肿药膏,的确是这种。另外,冰敷,用鸡蛋揉,服食山药、薏仁和三七粉,都可以活血消瘀。''

素心这才松了口气:``娴妃娘娘说得不假,红豆薏仁汤的确是可以消肿的。其实贵妃娘娘责罚小主之后自己也很后悔,又被皇后娘娘训斥了一顿,所以忙不迭吩咐奴婢送药来,以免皇上召见小主时小主无法侍奉。小主放心,只要用这个药,三天就会消肿的。''

``三天?''蕊姬嗤笑道,``你能保证这三天皇上都不宣召我?''

素心欠身道:``皇后娘娘说,如有宣召,也请小主顾全大局,切勿动气喧嚷。毕竟贵妃那儿,皇后娘娘已经狠狠训斥过了。若再生枝节,只怕今日的事小主自己也脱不了干系!''

蕊姬微微语塞,旋即语气凛冽:``那就替我谢过皇后和贵妃。只要这张脸没事,这次的事我罢休就是。''

素心微笑道:``这就是了。玫答应新获圣宠,一定希望以后步步顺利,事事遂心。小主这么聪明识大体,一定会的。''

说罢素心便退下去了。如懿稍稍坐过,亦起身告辞离去。

慧贵妃扶着宫女的手顺着长街慢慢走回去,一路看着雪景,神色倒也安宁。正过了建福门的甬道,忽见前面一个绿衣的小太监鬼鬼祟祟领着两个人背着身从咸福宫的角门出来。慧贵妃一怔,立刻吩咐身边的宫女茉心道:``去看看。什么人鬼鬼祟祟地在咸福宫附近晃荡。''

茉心追上去两步,厉声喝道:``谁在那里!见了娘娘怎么也不跪下!还不快转过身来!''

那绿衣太监脚下一迟疑,知道是走不脱了,转身跪下请了个安:``奴才参见慧贵妃,贵妃娘娘万安。''

``万安?''慧贵妃施施然道,``你们见了本宫就跑,本宫还安什么安?抬起头来!''

那绿衣小太监犹豫不决,只得抬起头来。茉心诧异道:``宝成?''

慧贵妃脸色微微一沉:``你是延禧宫的人,跑到本宫的咸福宫来做什么?''

宝成机灵地磕了头道:``都怪这场大雪,奴才走得冻死了,想靠在咸福宫的墙根下取会儿暖再走。谁知见到了娘娘过来,怕娘娘责骂,所以背着身就跑了。''

慧贵妃蹙眉,似是不信:``咸福宫在东边的最末,延禧宫在东边的最前头,你要取个暖也走得太远了吧。''她瞥见宝成按在雪地上的两手洇出乌黑的痕迹来,便抬了抬眼,示意茉心上前看了一眼。茉心会意,往前几步,拉起宝成笑道:``好了,你喜欢往咸福宫跑又怎么了?咸福宫的地气暖,连皇上都爱来,别说你了。''她别过脸,朝慧贵妃点点头。

慧贵妃会意,便换了和缓的笑意:``没事就走吧。记得告诉你们娴妃,有空常来咸福宫走动。''

三宝受了这一场惊吓,正恐瞒不过去,却不想这般轻轻揭过,忙不迭谢了恩走了。慧贵妃见他们走远,盯着地上发黑的六个掌印,鄙夷地笑了笑,``敢在本宫面前装鬼,茉心,去看看是什么?''

茉心蹲下身看了一眼,奇道:``回娘娘的话,那乌黑的东西是炭灰,是黑炭的灰。''

慧贵妃疑道:``黑炭又不是什么上好的东西,难道延禧宫还缺了这个来偷?''她一回神,暗暗咬牙,``不对,她是给海兰的!''

茉心点点头。慧贵妃愈加恼恨,一张粉面紫涨着,``算她珂里叶特氏厉害,本宫用了她一点儿炭,她就敢到处喊冤哭诉去了!弄得旁人来周济,还当本宫怎么苛待了她!''

茉心连忙道:``可不是!皇后娘娘一直说后宫里要节俭,她屋里就那么几个人,能用得了多少,娘娘也是为宫里替她俭省罢了。谁知道海常在这么不惜福!''

慧贵妃洁白的贝齿轻轻一咬,仿若无意道:``她跟延禧宫是一条心,本宫算是看得真真儿的,这吃里扒外的东西\ldots\ldots{}''她抿了抿唇,再没有说下去。

茉心不自禁地闪过一丝寒意,便也低下了头去,忙道:``娘娘,外头冷,咱们赶紧进去吧。''

慧贵妃微微颔首,扶着茉心进了宫。正巧内务府的执事太监从永和宫出来,在咸福宫挂完了匾额,抹了手正要走。回头却见慧贵妃进来,忙堆了一脸的笑意,又是打千儿又是奉承,直哄得慧贵妃万分高兴,嘱咐了宫里的首领太监双喜道:``这么冷的天还要顾着差事,替本宫好好打赏他们。''

执事太监高兴,越发说了许多锦上添花的话,``皇上说了,咸福宫这块匾额是滋德合嘉,许慧贵妃娘娘福德双修的意头。这层意思,听说是皇上斟酌了好久才定的呢。说是给咸福宫的东西,不能轻易下笔了,必得是最好的。''

慧贵妃深有兴致,细细赏着皇帝的御笔,笑若春花,``皇上的御笔难得,这个匾额是独本宫宫里有呢,还是连皇后那里都有?''

内务府执事太监愣了一愣,一时答不上话来。慧贵妃瞟了他一眼,轻笑一声道:``你怕什么?皇后娘娘那里有是应该的,难不成本宫还会吃皇后的醋么?''

那执事太监只好硬着头皮道:``不止皇后娘娘宫里,按皇上的吩咐,东西六宫都有。''

慧贵妃的笑意在一瞬间似被霜冻住,眉目间还是笑意,唇边却已是怒容。她的笑和怒原本都是极美的,此刻却成了一副诡异而娇艳的面孔,越发让人心里起了寒噤,``那么,连永和宫都有么?''

那执事太监连头皮都发麻了,只得战战兢兢答道:``是。''

慧贵妃森然问:``是什么字?''

太执事监道:``是仪昭淑慎。''

慧贵妃神色冰冷,厉声道:``她也配!''

执事太监吓得扑通跪下,忙磕了头道:``玫答应自己也知道不配,还特意去了问了娴妃,结果娴妃说皇上是给永和宫的匾额,她住着永和宫,肯定是她担得起。玫答应这才高兴了。''

晞月脸色变了又变,最后沉成了一汪不见底的深渊,慢慢沉着脸道:``下去吧。''

那执事太监听得这一句,巴不得赶紧走了,立刻带人告退了下去。

慧贵妃走到正殿门前,看着外头天色净朗,阳光微亮,海兰所住的西房里,叶心正端了炭盆出来,将燃尽的黑色炭灰倒在了墙角。

慧贵妃冷冷看着,目光比外头的雪色还冷,``双喜,你给本宫好好盯着海常在那儿,看延禧宫的人多久悄悄来一次。''

双喜看慧贵妃神色不似往常,也知道厉害,忙答应了。

连着几日忙着年下的大节庆,戊寅日,皇帝为皇太后上徽号曰``崇庆皇太后'',加以礼敬。接着又因准噶尔遣使请和,命喀尔喀扎萨克等详议定界事宜,一脸忙碌了好几日。

这一夜雪珠子格楞格愣打着窗,散花碎粉一般下着。如懿坐在暖阁里,惢心拿过火盆拢了拢火,放了几只初冬采下的虎皮松松塔并几根柏枝进去,不过多时,便散出清郁的松柏香气来。阿箬见惢心忙着在里间整理床铺,如懿靠在暖阁的榻上看书,便抱了一床青珠羊羔皮毯子替她盖上,要给踏脚的暖炉重新拢上火,铺了一层暖垫。

阿箬见如懿捧着书有些怔怔的,便问:``小主这两日最喜欢捧着这本《搜神传》看了,怎么今儿倒像没趣了似的。''

如懿笑道:``都是神鬼古怪的东西,看得多了,越发觉着呆在这儿闷闷的。''

阿箬笑嘻嘻道:``可不是?小主从前在老宅的时候,最喜欢偷偷溜出去外头跑马了。如今下了雪这般闷,难怪小主觉得没劲儿。''

如懿闷了一回,便问:``皇上有好几日没召人侍寝了吧?''

阿箬添了茶水,道:``可不是!听说为了准噶尔的事一直忙着,见不完的大臣,批不完的折子。敬事房送去的绿头牌,都是原封不动的退了回来的,说皇上看也没顾上看一眼。''

如懿凝神想了想,``这样也好,就这三四日,用着那药,玫答应的脸也该好全了。''

阿箬轻哼一声,``倒是便宜了慧贵妃!''她稍稍迟疑,还是问,``不过小主,奴婢也是想不通,皇上到底是看上了玫答应什么,要容貌不算拔尖儿的,性子也不算多温顺,出身就更不必提了,竟连婉答应都比不上。婉答应从前好歹还是潜邸里伺候皇上的通房丫鬟呢。''

如懿轻轻瞥了她一眼,叹道:``阿箬,你这个人平时最机灵不过。只一样不好,太喜欢背后议论。这样的话传了出去,旁人听见了,只当我的延禧宫里成日就是坐了一圈爱嚼舌根的。''

阿箬看惢心也在,不免脸上一红,``奴婢也是在小主跟前罢了。若是对着别人,咬断了舌根也不会嚼半句的。''她绞着发稍上的红绳铃儿,``奴婢就是想不通么。''

如懿指着瓶中供着的一束金珠串似的腊梅,问道:``这四时里什么花儿不好,怎么偏折了腊梅来?''

阿箬一愣,``小主说笑呢,不是冬日里没什么别的花,只能折几枝梅花么。''

如懿抿了抿唇道:``是了。别人没有,只有她有,自然是好的。你看咱们宫里这几个人,皇后宁和端庄,贵妃温柔娇丽,纯嫔憨厚安静,嘉贵人是最妩媚不过的,怡贵人和海常在呢,话也不多一句,婉答应更是个没嘴的葫芦。但不论怎么说,咱们这些人都还是有些出身的,也多半顺着皇上。皇上见惯了咱们,偶尔得了一个出身低微却有些性子的,长相也清秀脱俗,怎么会不好好疼着她宠着她。何况宠爱这样出身的人,自己也满足些。''

阿箬怔了片刻,回过神来道:``奴婢听出小主的意思了,男人对着出身低微的女人,宠着她给她尊荣,看她高兴,比宠着那些什么都见过什么都知道的女人,要有成就感得多。''

如懿握着书卷,意兴阑珊,``因为她们曾经获得的太少,所以在得到时会格外雀跃。也显得你的付出会有意义得多。''

阿箬若有所思,``那仅仅因为这样,皇上就会一直宠爱她么?''

炭火噼啪一声发出轻微的爆裂声,越发沁得满室馨香,清气扑鼻。如懿道:``那\ldots\ldots 就是她自己的本事了。''

一时间,两人都沉默了,阿箬低低道:``原来一个男人喜欢一个女人,还有这么多的缘故。''

如懿无声地笑了笑,那笑意倦倦的,像一朵凋在晚风中的花朵。惢心放下帐帷,轻声道:``康熙爷喜欢的良妃出身辛者库,不也一路升至妃位么?其实哪有那么多喜欢不喜欢的缘故,不过是一念之间,盛衰荣辱罢了。''

正说着话,外头三宝急匆匆赶了进来,打了个千儿慌慌张张道:``娘娘,咸福宫出事了,您快去瞧瞧吧。''

\hypertarget{ux7b2cux5341ux4e94ux7ae0-ux51ccux8fb1}{%
\chapter{第十五章 凌辱}\label{ux7b2cux5341ux4e94ux7ae0-ux51ccux8fb1}}

三宝话音刚落,偏偏炭盆里连着爆了好几个炭花儿,连着噼啪几声,倒像是惊着了人一般。

如懿心头一惊,声气倒还缓和,``出了什么事?好好说话。''

阿箬撇撇嘴道:``三宝越来越没样子了,咋咋呼呼的,话也说不清楚。要是慧贵妃出事,我先去放俩鞭炮偷乐子,要是海常在,那也不打紧,慢慢说呗。''

如懿蹙了蹙眉头,``要是慧贵妃,三宝会这么不分轻重么?''

三宝擦了擦额头的汗,马上道:``是海常在出了事儿。两个时辰前慧贵妃宫里闹起来,说贵妃用的红箩炭用完了。可今儿才月半,按理是不会用完的。贵妃怕冷,又不肯用次些的黑炭,一时受了冷,结果发了寒症。''

如懿颇为意外,``寒症?着太医看了么?''

``请了太医了。这事也罢了,但贵妃身边的茉心盘算这用了红箩炭的数目不对,便留心查问宫里。结果在海常在房里倒出来的炭灰里发现了不妥。那黑炭的炭灰是黑的。红箩炭的炭灰是灰白的,所以茉心就闹了起来,指着海常在房里偷盗了贵妃所用的红箩炭。''

如懿盯着三宝,肃然问:``本宫记得当初命你悄悄送炭的时候就吩咐过。贵人以下是不能用红箩炭的,未免麻烦。你可是老老实实每次只送黑炭的?''

三宝忙磕了个头道:``是是是,小主的远见,奴才一次都不敢误了。''

如懿心中着紧,越发担心起海兰来,``那就好。别的本宫不敢说,海兰不是那种僭越的人,她必不敢偷的。阿箬,替我更衣,咱们就去看看。''

如懿霍地站起来,阿箬急得拉住了如懿的袖口,``小主不能去!''她虎着脸,向三宝喝道,``咸福宫就是一滩浑水,贵妃的位份又比小主高,小主哪里能管得上!咱们不去,要去也是该皇后去的事儿!''

如懿静静神,即刻问:``皇后呢?''

三宝向养心殿努了努嘴儿,``今晚皇上翻的是皇后娘娘的牌子。这个时候,皇后娘娘怕在养心殿歇下了。''

如懿倒抽一口冷气,``皇上忙了这么多天的政务,眼下又是皇后侍寝,谁敢去打扰!''她只觉得掌心湿湿的冒起一股寒意,``可要不惊动皇后,宫中贵妃的位份最高,这件事怕是要淹下去了。''

阿箬急忙劝道:``咸福宫出了事情,小主巴巴儿地赶去,即便是到了门口,也帮不上什么呀!''

三宝焦惶惶道:``可是奴才听到消息的时候,说海常在马上要给上刑了,要再不去,若出了什么事\ldots\ldots{}''

如懿大吃一惊,``上刑?上什么刑?''

``杖刑!''三宝见如懿一时没反应过来,忙解释道:``不是用板子责打大腿。而是脱了鞋子,用棍子责打脚心,那可比打在腿上痛多了。''

如懿失声道:``打脚心?''

三宝点头道:``可不是?咱们当奴才的谁不知道,打在腿上只是肉疼,伤不了筋动不了骨。可脚多细嫩啊,几下下去,那都是伤身的。''

如懿定一定神,``除了皇后和贵妃,宫中便是我位份最高,我若不去,海兰要是被上了刑,还不知道要被伤成什么样子?事不宜迟,阿箬,快替我更衣。三宝,去传轿。''

阿箬待要再劝,看如懿着急之下不失决绝,只好答应着去了。

外头下着搓絮似的小雪。如懿坐在暖轿里,抬轿的太监们走得又稳又急,只闻得靴底与石砖摩擦的轻响,飞也似的往咸福宫方向去。

如懿捧着手炉,平时觉得暖暖的,此刻捧在手里,却仿如灼心一般,烫得刺手。她不时地打起帘子往外张望,三宝一路小跑跟着,喘着气道:``小主别急。延禧宫和咸福宫本就隔得远,咱们已经很快了。''

如懿无奈地垂下帘子,正焦心着,却听得三宝在外道:``到了,到了!''

夜来的咸福宫灯火通明,如懿扶着阿箬的手下了暖轿,快步走进院中。只听得太监尖着嗓子通报,``娴妃娘娘到------''

尖细的尾音尚自袅袅飘在空中,如懿人已经到了廊下。只见咸福宫正殿的镂花朱漆填金大门豁然洞开,廊下自台阶左右两列站满了满宫的宫人,一个个噤若寒蝉,只望着廊下一个跪着的宫装女子。

慧贵妃穿着一身锦茜色彩绣花鸟纹对襟长衣,肩上披着一件大镶大滚的紫貂风领玄狐大氅①,人坐在正殿中央的牡丹团刻檀木椅上,旁边七八个暖炉和炭盆众星拱月似的烘着,如懿才一靠近正殿,便觉得暖洋如春,真个人都舒展了过来。可慧贵妃的脸色并不好看,她本是小巧细弱的柳叶身段,大约为着动怒,又过了病气,底下雪里金遍地锦滚花镶狸毛长裙絮絮掠动着,漾起水样的波纹。她照常淡扫娥眉、敷染胭脂,可病中的一张脸雪白雪白的,显得上好的玫瑰丝胭脂也一缕缕地浮在面上,吃不住似的。如懿见她面色不善,忙欠身请安道:``给贵妃娘娘请安,贵妃万福金安。''

慧贵妃坐在椅上一动不动,只冷笑道:``自皇上分封六宫之后娴妃就未曾踏足过咸福宫,怎么今儿什么风连你也惊动了,深夜还闯进本宫宫里来?''

如懿见她左右的太阳穴上都贴了两块乌沉沉的膏药,额上一抹深紫色水獭皮嵌珍珠抹额勒着,真当是憔悴得我见犹怜。

如懿忙低着头道:``听闻贵妃娘娘发了寒症,所以漏夜过来探视。''

慧贵妃扬了扬唇角,``本宫有什么可值得娴妃你劳心的,倒是咸福宫里闹了贼,娴妃你的耳报神快,就紧赶着来看热闹了。''

如懿越发低首,``臣妾不敢。''

身后的海兰嘤嘤低呼一声,``贵妃娘娘,嫔妾\ldots\ldots 嫔妾不是贼!''

慧贵妃陡地敛起笑容,森冷道:``还敢狡辩,人赃俱获了还要嘴硬。双喜,再给本宫狠狠地打!''

如懿方才匆匆进殿,不敢细看海兰。此刻回头,只见海兰被强行剥去了鞋袜跪在廊下冰冷的石砖上,近台阶的砖边结了薄薄的碎冰,一望便生寒意。一双青缎绣喜鹊登梅花盆底鞋被随意抛掷在阶下的雪中,渐渐被落下的小雪浸湿了小半,如她的主人一般全无尊严。

如懿留神去看她的脚,冻得通红的赤足之上有着细密的血珠沁出。海兰见如懿注目,羞愧地极力想缩着足把它藏到裙底下去,茉心一言不发,立刻用手撩起她的裙角,冷冷道:``常在不好好招供,也不老实受刑,别怪奴婢不留情面,掀起您的裙角来。在奴才们面前露足已经够丢脸了,要再让人看见您的小腿,这种丢了脸面的事就是您自作自受了。''

海兰大惊,极力低着头以散落的发丝遮蔽自己因羞愧和愤怒而紫涨的面庞,她忍着痛分辩,``贵妃娘娘恕罪,嫔妾真的没有偷盗娘娘的红箩炭啊!''

如懿忙赔笑道:``贵妃娘娘发了寒症,脸色不太好。病中原不宜动气,不知娘娘到底为什么责罚海常在,而且要动用杖刑责打海常在双足?''

慧贵妃转过脸微微咳嗽了几声,彩玥和彩珠忙上前递茶的递茶,捶肩的捶肩。茉心清了清嗓子道:``海常在偷盗贵妃娘娘所用的红箩炭,犯上僭越,以致娘娘缺了炭火寒症发作,损伤凤体。这样的罪过,还不够受杖刑的么!''

如懿连忙道:``海常在向来安分守己,而且贵人以下是不许用红箩炭的,海常在也不是第一天知道,怎还会如此?''

茉心鄙夷道:``那就要问海常在自己了。奴婢在海常在屋里倒出的炭灰里发现了红箩炭烧过的灰白色炭灰。而且海常在几个奴才那里也问过了,伺候海常在的宫女香云已经招了,是海常在指使她去偷盗的红箩炭。''

如懿看着跪在阶下战战兢兢的香云,起身走到她跟前,``香云,茉心说的是真的么?''

香云脸色煞白,``方才奴婢已经招了,海常在指使奴婢偷盗红箩炭,一是不服气贵妃娘娘用着好东西,二是嫉妒贵妃娘娘得宠于皇上,想害贵妃罢了。''她拼命磕了两个头,乞求道:``贵妃娘娘恕罪,奴婢已经知错了,再也不敢了。''

海兰忍着疼,别过头看着香云道:``香云,你跟了我两三年,我自问待你并不薄\ldots\ldots{}''

香云并不畏惧,迎着海兰的目光,定定道:``小主,不管您待我如何,这种昧着良心的事奴婢是再也不敢了。奴婢也劝您一句,人赃并获,您还是认了吧。''

``有错能改,善莫大焉。所以香云,本宫也不会责罚你。但知错不改,还死不承认,那就要好好责罚了。''慧贵妃不觉微微作色,冷笑道,``这宫里头谁不知道本宫畏寒体弱,是最禁不得冷的。海常在用心这样恶毒!双喜,给本宫再打!''

随着慧贵妃话音利落而下,双喜已经取过一旁的荆棍,道一声``得罪小主'',立刻便要打下去。如懿仔细看去,才发觉那并不是寻常的棍子,而是选取粗大的荆条,未剥皮,也未去刺。两指粗的荆棍上利刺突起,沾了鲜红的血点。想来海兰足上的血珠,便是由此物造成。

双喜二话不说,举起棍子便向着海兰脚心狠狠猛击数下,海兰惨叫一声,几乎没晕倒在地,足上鲜血淋漓,简直惨不忍睹。如懿既惊且忧,她虽知道足心受痛远胜于他处,但看海兰如此吃痛,亦知道不好。情急之下,她只得伸臂拦下双喜手中的荆棍,喝道:``慢着!''

海兰痛得伏在地上,慧贵妃优雅地扬起细长的眼眸,唤道:``茉心------''

如懿赶忙上前扶住了海兰,茉心嗤笑道:``娴妃娘娘来了没关心我们娘娘几句,倒先忙着帮扶海常在,这可真是是非不分了。何况方才海常在受了几下棍子没事,现在怎么弱不禁风了,可不是看人来了,就这般乔张做致么。''

海兰瘫倒在如懿怀里,满脸湿腻腻的冷汗黏住了头发,狼狈之中仍喃喃道:``娴妃姐姐,嫔妾\ldots\ldots 我,没有偷。真的\ldots\ldots{}''她话未说完,人便痛晕了过去。

如懿心疼地抱着海兰,用裙摆遮住她的双足,心中揪痛不已,只得强忍着怒气道:``贵妃娘娘以炭灰和香云的供词便认定海兰偷窃红箩炭逼害娘娘。可娘娘细想,今儿是腊月二十,娘娘的红箩炭是内务府按着每月的份例给的,每日十五斤,一个月便是四百五十斤。海兰若是真的全偷去了害得娘娘无红箩炭可用,那至少也得偷了十天的份额,一共一百五十斤红箩炭。她的宫室就那么点大,能查到哪里去?娘娘一查便知。''

慧贵妃微微变色,朝着茉心扬了扬脸。茉心从如懿怀中一把抢过海兰,顺手端过廊下搁着接檐下冰水的铜盆,哗一声兜头兜脸全泼在了海兰身上。如懿惊怒交加,喝道:``茉心,你做什么?''

茉心笑吟吟道:``海常在痛得晕过去了,不拿水泼醒,怎么问她剩下的红箩炭藏在哪儿啊!''

如懿怒视着她道:``这么冷的天气,你拿冷水泼她,岂不是要了她的命!''

茉心见海兰痛苦地呻吟了一声,笑道:``只要海常在醒了,一切都好说。您看,这不奏效了么?''

如懿连忙取下绢子替她擦拭,阿箬站在一旁也吓呆了,忙不迭取下绢子和如懿一起擦拭。慧贵妃双眼微眯,抬了抬下巴,茉心即刻会意,转身从廊下蓄水的大缸里舀了一盆,不管不顾一泼,将如懿浇得如落汤鸡一般。如懿只觉得一个激灵,浑身上下都已经被冰水浇透了,从骨子缝里直透出寒意来,兼着院中廊下冷风灌入,立时间像被堆在了冰雪中,冷得全身发颤。

茉心``哎呀''一声,忙道:``娴妃娘娘,真是对不住。谁让您离海常在这么近呢?奴婢原以为一盆水下去不能让海常在醒过来,所以加了一盆。这可怎么好\ldots\ldots{}''

慧贵妃微微坐直身子,曼声道:``茉心,你也太不当心了。''她努一努樱唇,``彩珠,彩玥,还不搬几个炭盆过去,替娴妃和海常在暖一暖。''

彩玥和彩珠答应着,却只拣了几个快熄了的炭盆搁在如懿与海兰身边,那火光微弱,实在是无济于事。

如懿死死地握着拳头,以指尖触进手掌的疼痛,提醒着自己要忍耐,将海兰紧紧拥住,希望以彼此的体温来温暖些许。天寒地冻的时节里,浑身湿透的彻骨寒意逼上身来,除了忍耐,还有什么办法?贵妃与妃位不过差了一个位次,地位却是千里之别。晞月,她是正当宠的贵妃。自己呢,不过是一个久未见君面的妃子罢了。她没有别的办法,只能忍耐着,只盼能救出海兰,拉扯她一把。

如懿垂首,冰冷刺骨的水珠滑过她一样冰冷而麻木的面孔,她只觉得头越来越重,声音也有点缥缈:``贵妃娘娘,海常在已经受过责罚,现下全身也湿透了。能否容许我带她去换一身衣裳?否则这样冻下去,她的身子也吃不消的。''

慧贵妃轻咳几声,慵然看着手上的鎏金镶珐琅护甲,微微含了一抹舒展的笑意。然而她眼中却一分笑意也无,那种清冷之光,如她小指上金光闪烁的护甲一点,尖锐而冷清:``方才娴妃有句话说得很好,一百五十斤的红箩炭呢,一下子也烧不完,保不准是藏在哪儿了。既然这样,不能不仔细搜一搜。''她曼声唤道,``双喜!''

双喜答应着凑了上前:``奴才在。''

慧贵妃慵懒道:``去海常在那几间屋子里好好搜一搜,连着海常在的寝殿,仔仔细细,哪儿也别放过。好好查查那些红箩炭放在了哪里,也好叫她们死心。''

如懿听她死死咬着``她们''二字,知道是不得好过了。这一搜也不知要搜到什么时候,自己和海兰冻在这儿,当真是求生不得求死不能。

如懿听她死死咬着``她们''二字,知道是不得好过了。这一搜也不知要搜到什么时候,自己和海兰冻在这儿,当真是求生不得求死不能。

海兰本已幽幽醒转,听得这句话,不禁失色,哭求道:``娘娘要搜查是不错,可嫔妾的寝殿也要搜么?嫔妾\ldots\ldots{}''

如懿矍然变色,怒意浮上眉间,只得强压了怒火道:``贵妃的意思是要搜宫?那不是半点脸面也不给海常在留了!此事若传出去,海常在还如何在后宫立足呢?''

茉心笑滋滋,伸手向海兰身上,作势就要翻开她湿答答的袍子,道:``不仅是海常在的寝殿,哪怕是海常在身上,奴婢也不能不瞧一瞧。''

海兰见她伸手过来,又气又怒,却也不敢反抗,只得拼命缩向如懿怀中。如懿忍无可忍,一手护住海兰,劈面一个耳光打在茉心脸上,怒道:``放肆!小主身上岂是你能乱碰的!''

茉心挨了重重一掌,一时也被打蒙了。她是晞月身边第一得意的侍女,又是侍奉多年的,自认为十分得脸,连晞月的一句重话都未受过,何曾受过这样的委屈?她还尚未从那一巴掌里醒转过来,慧贵妃已经按捺不住,从座椅上霍然站起,三寸长的护甲敲在手炉上叮然作响,在静夜里听来与她的嗓音一般尖锐而令人不适。

慧贵妃厉声道:``来人,给本宫搜检珂里叶特氏的寝殿,箱笼衣物,一律不许放过!娴妃深夜咆哮咸福宫,给本宫跪在院中思过。没本宫的吩咐,不许起身。''

海兰脸色惨然,望一眼如懿,终于伏下身叩头哭泣道:``贵妃娘娘,都是嫔妾的错。嫔妾不是有心偷盗的。''

如懿紧紧攥住她的手,决绝摇头:``没有做下的事,不许乱认!''

海兰满脸是泪,冒在她冰凉的面庞上泛起雪白的热气:``娴妃姐姐,我已经连累了你,不能再害得你浑身湿透了跪在雪地里\ldots\ldots{}''

她凄楚的哭声在落着簌簌细雪的夜里听来格外凄凉。如懿无助地搂着她,感受到身后巨大的拖力要将自己拽到廊下去。阿箬急惶的哭声响在耳边,是在对贵妃哭求:``贵妃娘娘,贵妃娘娘,奴婢求求你,哪怕是要跪,也让我们小主先换身衣裳。她会冻坏的呀,贵妃娘娘!''

慧贵妃站在殿内居高临下看着众人,眼神冻得如檐下能刺穿人心肺的冰凌一般。海兰伏在地上,像一只卑微的蝼蚁,慧贵妃的语气没有任何温度:``茉心,给本宫扒开珂里叶特氏的外裳,一寸一寸仔细地搜查,不许她藏匿了半分!''

茉心响亮地答应了一声,恨恨地咬了咬牙,伸手就上去拉扯。海兰护着自己的衣襟,拼命挣扎着,无助的哭声悲戚地飘在夜空中,像一缕没着落的孤魂一般,又被绵绵的雪子掩埋了下去。

注释:

①大氅:披用的外衣,又称``披风''。无袖、颈部系带,披在肩上用以防风御寒。短者曾称帔,长者又称斗篷,斗篷一般连帽。披风多为一片式结构,多为北方人和儿童在冬季穿用。后也泛指斗篷。中国古代有虚设两袖的长披风。

\hypertarget{ux7b2cux5341ux516dux7ae0-ux541bux5fc3}{%
\chapter{第十六章 君心}\label{ux7b2cux5341ux516dux7ae0-ux541bux5fc3}}

如懿被拽到了阶下跪着,雪子沙沙地打在脸上,像打在冻僵了的肉皮上,起先还觉得疼,渐渐也麻木了。不过片刻,衣襟上结了薄薄的冰凌。她眼见海兰受辱,一时间急怒攻心,仿佛一把野火从心头蹿到了喉咙里,再也忍不住道:``贵妃娘娘,您要责骂海常在或是动手打她,我都无话可回。但海常在到底是皇上的嫔妃,您不能这样羞辱她,尤其是当着奴才们的面。若海常在真被剥了衣衫搜身,您就真是要逼死她了!''

海兰呜呜地哭着,如同一只小小的困兽,做着徒劳而无力的挣扎。她领口的一粒如意扣已被生生拽开,露出生绢色的中衣。慧贵妃只是含了一缕闲适的笑意,好整以暇地看着廊下,如同坐在戏台下看着一出精彩绝伦的戏码。她轻蔑地瞟一眼如懿:``本宫也知道她身上藏不了红箩炭。可是她能偷炭,保不准还偷了什么其他贵重东西。既然做了贼,就别怕没脸,若是想不开,那横竖也是她自己逼死自己的。''

如懿见她丝毫没有转圜的余地,挣扎着便要起身。奈何她是冻透了的人,手脚完全不听使唤,才站起来便禁不住一阵冷风,又被人七手八脚地按了下去。

心中的焦苦直逼舌尖,她只觉得舌头都冻木了,唯有眼中的泪是滚热的,一滴一滴烫在脸孔上,很快也结成了冰滴子。这样的痛苦,就如吹不尽的寒风,没有尽头。

正混乱间,外头忽然有击掌声连连传来,有太监的通报声传进:``皇上驾到------皇后驾到------''

心口几乎就是一松,整个人都软倒在地,于悲戚之中生了一丝欢喜。他来了,他终于来了。

慧贵妃立刻扬了扬脸,示意所有人停下手中的动作。阿箬眼疾手快,忙脱下自己身上的弹花袄子,披在了如懿身上。

门口明黄一色倏然一闪,皇帝已经疾步进来。皇后穿了一身烟霞蓝底色的百子刻丝对襟羽纱袍,虽是夜里歇下了又起来的,鬓发却一丝不乱,疏疏地斜簪着几朵暗红玛瑙圆珠的簪子。虽然急迫,神色却宁静如深水,波澜不惊,连簪子上垂下的缠丝点翠流苏,亦只是随着脚步细巧地晃动,闪烁出银翠的粼粼波光。

慧贵妃领着人在院中接驾。皇帝见了她,忙一把扶住了:``朕一听说你发了寒证,赶紧就过来了。''他握住贵妃的手,焦急道,``怎么样?要不要紧?''

皇后跟在身后,沉静中带了几分关切的焦虑:``皇上一听人禀报说你发了寒证又动气,急得什么似的。本来皇上都睡下了,还是赶紧吩咐了起来,和本宫一起过来了。''

皇帝眉眼间都是急切,道:``太医来看过没有?到底怎么样?''

慧贵妃娇声道:``臣妾谢皇上皇后关爱。臣妾这儿缺了红箩炭,一时顾不上暖着,结果引发了寒证。太医已经来瞧过了,说臣妾因受寒而伤了阳气,以致身寒肢冷,呕吐清水,又使气血凝滞,运行不畅,因而身上疼痛。''她身子一歪,正好倒在皇帝的臂弯里,``此刻臣妾便觉得头晕体乏,膝盖酸疼呢。''皇帝心疼不已,一迭声道:``来人!快扶了贵妃进去坐下。多拿几个手炉暖着。''

慧贵妃就着彩珠的手迈了两步,脚下一个虚浮,差点滑倒。皇帝叹了口气,伸手揽过她道:``朕陪你进去吧。''

皇帝一心着紧在慧贵妃身上,自进来便似没看见如懿一般。如懿和海兰湿淋淋地站在檐下,冷风一阵阵逼上身来,似钢刀一刀一刀刮着。海兰浑身哆嗦着,站也站不稳,被如懿和阿箬搀扶着才能勉强站住脚。皇帝只顾着和贵妃说话,眼光根本都没落到如懿身上。如懿心下酸楚难言,只觉得自己站也不是,坐也不是,恨不得化作一根冰凌子冻在这儿,立时化去便好了。

皇帝经过她俩身旁,微微蹙眉道:``还杵在这儿做什么?去换件暖和衣裳。湿漉漉的,等下别把寒气过给了贵妃。''

皇后温言道:``去吧。都去海兰屋子里换件衣裳再来见驾。''

如懿知道皇帝到底还是怜悯,忙领着海兰退下了。

进了暖阁坐下,皇帝唤过随行的太医:``齐鲁,你是太医院的院判,一直照管着贵妃的身体,你赶紧再替贵妃瞧瞧,别落下什么症候才好。''

齐鲁忙答应着取过诊脉的药包,搭了片刻道:``贵妃娘娘的寒证发得不轻,加之又动了怒气,只怕得好生调养两日。''

皇帝微微松了口气,怜惜道:``往日到了冬天你的身体便格外弱些,今儿又是为了什么,动这样的气?''

慧贵妃眼中有盈盈泪光,别过头去轻轻拭了拭眼角,方哽咽道:``咸福宫不幸,也是臣妾管教无方,竟叫自己宫里人生了偷盗这样见不得人的事。海常在偷了别的也罢了,臣妾不能不顾恤着多年姐妹的情分,送了也就是了。偏偏是臣妾冬日里最不能缺的红箩炭。''

皇帝颇为意外,与皇后对视一眼,问道:``海常在偷那个做什么?''

皇后吁了口气,惋惜道:``怕是满宫里只有海常在和婉答应位分低用不上红箩炭,所以海常在一时糊涂了吧?''

慧贵妃长长的睫毛像小小的羽扇轻盈垂合,眼中似乎有泪光:``每次臣妾奉召侍寝,茉心她们总听见海常在摔摔打打地不乐意。臣妾心想也算了,可是这次想不到她竟这样恶毒,臣妾闻不得黑炭的烟气,一向只用红箩炭取暖,她偷取了臣妾的红箩炭害得臣妾寒证突发\ldots\ldots{}''她说着咳嗽起来,抚着额头道,``臣妾气怒攻心,实在是受不了了,一审之下人赃并获,可海常在还是抵死不认。''

她正暗暗垂泣,如懿已经换过了海兰的衣衫,携了海兰一同进来,嘴上道:``没有做过的事情,叫海常在怎么认?''

如懿领着海兰行了礼,海兰仍是怯怯的,像是一只受足了惊吓的小鸟,浑身颤抖着,缩在如懿后头。

皇后摇头,亦是似信非信的口吻:``看着海常在柔柔弱弱一个人,怎么心思这么毒?''她看着如懿,``娴妃,听说你大闹咸福宫,肆意喧哗,到底怎么了?''

如懿欠身恭谨道:``回禀皇上皇后,臣妾怎敢肆意喧哗,只是看海常在在所谓的`人赃并获'之下,受了足杖,还要被搜身,臣妾实在不能不替海常在分辩几句。而且臣妾若真喧哗,怎会被人泼了一身冰水也不吭声呢?''

皇帝眼角的余光落在她俩身上,漫不经心道:``喝了姜汤才来回话的吧?别带了寒气进来。''

如懿见海兰只是一味缩在自己身后,连头也不敢抬,越发生了怜惜爱护之意,回道:``是。都喝了的,不敢让贵妃娘娘沾了寒气。只是皇上\ldots\ldots{}''她仰起头注视着皇帝冷峻的面庞,``皇上,虽然贵妃在海常在用过的炭灰里找到了红箩炭的灰,也有香云作证,可是\ldots\ldots{}''

皇帝的口气淡淡的,像是说着一件极不要紧的事:``什么可是?朕记得上回天刚冷的时候嘱咐过你一句,说宫里就海常在和婉答应用不上红箩炭,怕黑炭熏着了她们。婉答应位分实在低也罢了,海常在那里要你从自己宫里拨出些给她。朕记得那日也嘱咐了你,这件事不宜声张,免得生是非。你也太老实了,贵妃都气成这样了,你也不肯告诉她一声。''

如懿立刻明白过来皇帝的维护之意,满脸自责道:``都是臣妾的不是,一心想着皇上嘱咐过不许说,所以也特意叮嘱了海兰妹妹。她原是跟臣妾一个心思,不敢说出来惹来是非,没想到还是惹了是非。''

皇帝的眼睛只看着一脸震惊的贵妃,心疼不已:``原是娴妃她们太痴了,不懂转圜。贵妃本就身子弱,哪里禁得起这样气?''他转头吩咐,``王钦,记得嘱咐内务府,以后咸福宫缺什么少什么,一律不用告诉内务府这样麻烦,立刻从养心殿拨了给贵妃用。''

慧贵妃的脸色本是青红交加地难看,听到这一句才缓过来,盈盈道:``多谢皇上关爱。''

皇帝的口吻轻柔如四月风:``好了。既发了寒证,怎么不好好将养着,还要这样折腾?岂不知自己的身体最要紧么?''

慧贵妃犹自有些不服:``虽然皇上吩咐娴妃暗中照顾海常在,可是香云也明明看见海常在偷盗了。海常在她\ldots\ldots{}''

皇帝的语气淡得不着痕迹,口吻却极温和:``这件事说白了也是小事,能有贵妃你的身子要紧么?至于海兰,她既惹你生气,朕便不许她在咸福宫住就是了。''

如懿闻言一喜,赶紧看一眼身后的海兰,她一直苍白的面色上微微浮了一丝绯红,只是紧紧攥着如懿的衣袖,像抓着救命稻草一般。

慧贵妃急道:``偷窃也算了,但犯上都是宫中大罪,皇上就这样轻易饶过了么?还有娴妃,这样莽撞无礼\ldots\ldots{}''

皇帝笑道:``打也打了,罚也罚了。娴妃和海常在一身的冰水也算是责罚过了。今日的事,朕是要赏罚分明,才能解了你的气,平息这件事。''他转头问道,``今儿的事,人证是谁?''

香云怯怯地膝行上前,含了半分笑意道:``是奴婢。''

皇帝眼皮也不抬一下,王钦便道:``是伺候海常在的宫女,叫香云的。''

皇帝这才瞟了她一眼:``模样挺周正的,舌头也灵活。能招出今晚的事,这舌头活灵活现的。''

香云喜道:``多谢皇上夸奖。''

皇帝低下头,把玩着腰间一块镂刻海东青玉佩,漫不经心道:``王钦,带她下去,乱棍打死。''

王钦吓得一抖,赶紧答应了:``是。''他一扬脸,几个小太监会意,立刻拖了香云下去。香云吓得求饶都不会了,像个破布袋似的被人拖了出去。

只听得外面连着数十声惨叫,渐渐微弱了下去,有侍卫进来禀报道:``皇上,香云已经打死了。''

海兰打了个寒噤,如懿只是含了一缕快意的笑意,很快又让它泯在了唇角。

皇帝微微颔首,浑不在意:``拔了舌头悬在宫门上,让满宫里所有的宫人都看看,挑拨是非,谋害主上,是什么下场!''

如懿陡地一凛,目光撞上皇帝深渊静水似的眼波,心头舒然一暖,像是在雪野里迷了路的人远远望见灯火人家,便有了着落。皇帝的目光旋即移开,仿佛对她只是那样的不上心而已。

慧贵妃又惊又怕,浑身止不住地打起冷战,皇帝怜爱地替她紧了紧大氅,柔声道:``别怕!都是下人们的不是,你安心养好身子暖着才要紧。''

慧贵妃在皇帝的安抚下微微放松,咬了咬牙强笑道:``是。这样嚼舌的奴才是留不得的,皇上不发落,臣妾也要杀了她以儆效尤呢。只是拔了舌头血淋淋的,她既然跟这些红箩炭扯上了是非,就拿些热炭填到她嘴里去,好歹留个囫囵的全尸给她。''

皇帝眉目间带着疏懒的笑意,抚了抚她的手:``也好。既然你替她求情,就留个全尸给她。''他目光一沉,环视众人,已是不容置疑的口吻,``贵妃今日做下的典范,后宫里都要谨记,任何一个奴才,都不许挑拨是非,惹起风波。否则不是主子的错,朕只问你们这些舌头和嘴,经不经得起拔舌烫嘴之苦!''

满宫的宫人们吓得魂飞天外,立刻跪下道:``是香云自己生是非,奴才们都不敢的。''

皇帝生了几分倦怠,打了个呵欠道:``好了。夜也深了,你早点歇着。朕和皇后也要回养心殿去了。''

众人忙起身:``恭送皇上,恭送皇后娘娘。''

皇帝携了皇后的手一同出去,在经过如懿与海兰时稍稍驻步,他的目光滑过海兰不带任何温度与情感,仿佛只是看着一粒小小的尘芥,根本不值一顾:``你再住在咸福宫也只是让贵妃生气,换个地方住吧。''

如懿忙道:``皇上,延禧宫还空着\ldots\ldots{}''

皇帝有些不耐烦:``那你好好调教海常在,别再生出这么多事来。''

如懿答应一声,心口松畅,拉了海兰一同跟着出去了

回到延禧宫中已是深夜。安顿了海兰在后殿住下,又请了太医来给她诊治,如懿才回到寝殿里稍稍歇息。虽然早换上了厚实的暖袄,如懿又抱着几个手炉取暖,仍是觉得身上一阵阵发冷,便命小宫女又端了几个火盆进来烧着。小丫头绿痕用松纹银漆盘端了几大碗浓浓的红糖姜汤喂了如懿喝下,又替她加了个貂皮套围得严严的。如懿取过一碗给裹着大袄蹲在火盆边取暖的阿箬:``快酽酽地喝一碗,去去湿冷。''阿箬忙仰头喝了,如懿也喝出了一身的热汗,忍不住打了几个喷嚏,才觉得身上松快了些。

惢心已经陪着太医看过了海兰,此刻又跟过来请许太医给如懿诊脉。许太医取出朱紫色的请脉包垫在如懿手腕下,又搭上一块洁白的绢布,告一声``得罪'',才敢把两指落在如懿的手腕上。

片刻,许太医松了口气道:``娴妃娘娘万幸,素昔身子强健,只是受了一点风寒。微臣会开些发热疏散的方子,只要娘娘连着喝几天药和姜汤,注意保暖,再用生姜和艾叶熬的热水多泡澡,就会好的。但切记切记,这几天不许再见风了。''

如懿取过绢子按了按塞住的鼻子,闷声道:``多谢太医。海常在如何了?''

许太医摇了摇头,似是沉吟不已。

如懿愈觉得不安,便道:``许太医是常来常往,专照顾本宫的,有什么话不妨直说。''

许太医思量再三,沉声道:``受寒和惊吓都是小事,微臣开了安神药给海常在喝下,已经安稳睡了。风寒虽重,调理着也无大碍。要紧的,是海常在的足伤。''

许太医道:``海常在是足心的涌泉穴挨了打受了伤,才会如此虚弱,形同重病。''

如懿奇道:``涌泉穴?''

许太医沉声道:``是。涌泉穴又名地冲穴,乃是肾经的首穴,又是肾经与心经交接的要害。微臣查看过小主的足心,涌泉穴的位置乃是被荆棘重创之地,说明下手之人是特意挑了这个地方的。此穴一旦受损,等于肾经与心经同时受损,便有失眠倦怠、精力不足、晕眩焦躁、头痛心悸等症并发,加之小主受寒,真是险之又险。''

如懿大惊失色,只觉得心头沉沉乱跳,忙问:``太医,可有什么法子医治么?''

许太医沉吟许久,才道:``微臣会仔细掂量着开个方子,使寒气外泄,伤口愈合。也请娘娘吩咐伺候常在的宫人们,每日用热盐水浸泡小主双足的涌泉穴,热水以能适应为度,每日临睡前浸泡半个时辰。另外每日正午用艾灸熏涌泉穴,每日一次,至涌泉穴有热感上行为度,熏好之后敷上用酒炒过的吴茱萸护着。等到伤口好了之后,再每日按摩,但求见效。''

如懿听他细细说了医治之法,知道还是有法子的,也稍稍安心些,眉头也松开了一截:``那就有劳许太医了。绿痕,好好送许太医出去。''

许太医告辞退下,如懿向着后殿方向张望了片刻,惢心忙道:``小主放心,一切都打点好了。海常在服了安神汤药,此刻已经熟睡,想是连番折腾,人也累坏了。您若想看她,还是等明日自己养足了精神再去吧。''

如懿掩不住眉目间的倦怠之色:``好了。我也乏了,准备着安置吧。''

惢心答应着去捧了热汤水来伺候,阿箬拍打着如懿换下来的海兰那身衣裳,满肚子压抑不住的怒气,手上的力气就大了,噼噼啪啪的。如懿听着发烦,蹙眉道:``什么事情,粗手大声的?''

阿箬径自道:``小主身上冷,奴婢心里冷,心里更是有气。慧贵妃是什么人?从前在潜邸的时候是矮了小主一头的\ldots\ldots{}''

如懿心中不快,打断她道:``好了!如今是如今,不要再说从前的事!''

阿箬憋了口气道:``如今竟敢这样折辱小主。小主,你一定得想想法子,不能再这样受委屈了。''

如懿转过身,将手里的汤盏递给蹲在地上拨火的小宫女:``收拾了都下去吧,火盆不必拨了。''

宫人们退了下去,惢心在一旁静静地立着往案上的绿釉狻猊香炉添了一把安神香。那雪色的轻烟便从盖顶的坐狮口中悠悠逸出,温暖沉静的芬芳悄无痕迹地在这寝殿中萦纡袅袅,散出定心安神的宁和飞香。

如懿拨着手炉上的珐琅盖子,轻声道:``阿箬,那么依你的意思,我该怎么办?''

阿箬将拍好的衣裳往花梨木衣架子上一撂,眼睛扑闪扑闪,瞬间亮了起来:``按奴婢的意思,好办!人活一口气,树活一张皮,一定要好好争了这口气回来。''她走近如懿身边,推心置腹道,``小主怕什么?小主什么都不必怕!论家世,乌拉那拉氏是出过中宫皇后的,门楣比富察氏还高,何况她一个包衣抬旗的?论位分,妃位和贵妃就差了那么一阶儿,哪天冷不丁就越过她了。论恩宠,小主从前和她平分春色,只要放出点手腕来好好笼络皇上,皇上也会常来延禧宫了。''

如懿啜了口热茶,慢慢搓着手背暖手,淡淡道:``你的话是不错,什么理儿都占全了。可是你的眼睛太高,只看见了我的长处,却未看见短处。''

阿箬不解:``短处?''

暖炉的热气氤氲地扑上脸来,蒸得室内供着的蜡梅香气勃发,让人有片刻的错觉,恍若置身四月花海,春暖天地。可是,窗外明明是严寒时节,数九寒天。而宫中的际遇,只会比这寒天更寒,怎么也暖不过来。

如懿出神片刻,沉稳道:``一个人的长处和优势,只会锦上添花,让她往高处走得更高些。而她的短处和缺失,却是能拉着她一路跌到深渊再爬不起来的。所以我看人,不看她的长处能带着她走多高,而是看她的短处会让她摔得多重!''

阿箬一时答不上嘴,只得问:``那小主打算一直这么忍下去?''

如懿的手微微一颤,郁然叹了口气:``现在的境况对我并不好,一味去争,只有摔得头破血流。忍一忍过去了,以后的日子便松快些,也觉得没那么难忍了。要是不忍,永远就挤在一条窄道上,那就真的为难了自己。''

阿箬嗫嚅着嘴唇说不出话来。如懿支着额头,轻轻挥手:``今儿晚上你也累了,着了气又受了冷,赶紧去歇下吧。''

阿箬答应着下去了。惢心扶了如懿上床歇下。如懿看着她放下茜紫色连珠缣罗帐,她穿着墨紫色弹花上袄,花纹亦是极淡极淡的玉色旋花纹,底下着次一色暗紫罗裙,这样站在薄薄的帐帘外,仿佛整个人都融了进去,只余一个水墨山水一般暗淡的身影。

如懿淡淡地吁了口气,惢心忙问:``小主,是焐着汤婆子不够暖么?''

如懿拍一拍她的手臂:``方才阿箬说了那么一大篇话,你只在旁边安静听着。但我知道,今儿晚上没有你去养心殿报信,皇上来不了那么快。''

惢心的面色沉静如水:``奴婢候在咸福宫外,看见小主受辱,当然要去禀报。只是\ldots\ldots{}''

``只是什么?''

惢心低低道:``奴婢见着王公公,王公公说既是咸福宫的事,就由咸福宫的主位定夺,就轰了奴婢出来。幸好李玉公公要轮到上夜了,看见了奴婢才去告诉皇上的。否则,事情也被耽搁了。''

如懿沉吟片刻,含笑道:``王钦哪里是个好相与的?他一向只听皇后和贵妃的话。''

惢心的眉眼恭顺地垂着,低声道:``王公公不好相与,是被人定了的。但是李公公\ldots\ldots{}''

如懿眉心一动,笑着拍了拍她的手:``这就是你比阿箬细心的地方了。言语不多,但眼睛都落在了实处。我没有白疼你。''

惢心直直地跪在床前的架子上,眼中微微含了一丝晶莹,道:``奴婢刚进潜邸的时候,不过是被人牙子卖来的小丫环,只值两百个钱,被发配在伙房砍柴,是打死也不作数的贱民。是小主可怜奴婢,把奴婢从伙房的柴火堆里拣出来,一路抬举到了今天这个地位。奴婢没什么可说的,只有尽心尽力护着小主,伺候小主罢了。''

如懿拉着她的手,心头暖暖的,一阵热过一阵:``好,好,不枉我这些年一直这么待你。阿箬机灵,嘴却太快。你心思安静,就替我多长着眼睛,多顾着些吧。''

\hypertarget{ux7b2cux5341ux4e03ux7ae0-ux7389ux9762ux4e0a}{%
\chapter{第十七章
玉面(上)}\label{ux7b2cux5341ux4e03ux7ae0-ux7389ux9762ux4e0a}}

宫中的夜如许深长,如懿从未受过这般折辱委屈,原是乏极了。她原本以为靠着软枕就能沉沉睡去,谁知听着窗外风声凄冷,刮得寝殿外两盏暗红的宫灯风车似的转着,仿佛两只睁大的猩红鬼眼,直愣愣地盯着她不放。如懿看着外头的灯火,心里思绪翻腾不定,仿如千丝万缕都缠在了心上,一丝一丝紧紧地勒着。榻下惢心的呼吸声已经沉稳而均匀,显是睡得熟了。如懿油然便生了一星羡慕之情,若都像惢心一样,无知无觉,能安稳睡到天亮,也是一种福气。她侧过身,将脸埋在丝缎的菀花软枕间,极力闭上了眼睛。也不知过了多久,她睡得其实并不沉稳,半梦半醒的恍惚间,窗外穿行枝丫的风声犹如在耳畔,像是谁在低低地哭泣,幽咽了整整一夜。

醒来时是在后半夜了,如懿觉得烦渴难耐,便唤了一声``惢心'',惢心立刻从榻下的地铺上起身,问道:``小主是要喝水么?''

如懿道了声``是'',惢心披着衣裳起来点上蜡烛,倒了一碗热茶递到她手边,轻声道:``小主慢点喝。''

如懿酽酽地喝了一碗,便说还要,惢心搭了把手在她额头一按,惊呼道:``小主额头有点烫,怕是发烧了呢。''

如懿觉得身上软软的,半点力气也没有,口中腹中都是焦渴着,只得懒懒道:``喝了那么多姜汤,怕还是着了风寒了。''

惢心道:``现下晚了,也不便请太医再过来,明儿先把太医院的方子开上喝一剂。''

如懿抚着头道:``还是老法子,煮了浓浓的姜汤来,我再喝一碗发发汗。''

惢心想了想道:``那奴婢用小银吊子取了来在寝殿里头熬着,随时想喝就喝着。奴婢醒着点神看着就是了。''

两人正说着话,只听得后殿忽然几声惊叫,如懿怔了怔,便问:``什么声音?''

惢心竖着耳朵听着:``怕是风声吧?''

那尖叫声连绵几声,夹杂在风里也显得格外清晰。如懿心头一沉,忙披了大氅起身道:``不对!是海兰!''

夜里惶急起身,如懿只趿了双软底鞋便匆匆赶出来。海兰缩在寝殿的桃花心木滴水大床上,那床原是极阔朗的,越发显得海兰蜷在被子里,缩成了小小一团\ldots\ldots''叶心早吓得跪在了床边,和伺候海兰的一个小太监一起苦苦哀求着,海兰却似什么也听不见一般,只是捂在被子里捂住耳朵发出尖锐而战栗的尖叫。

如懿忙挥了挥手,示意众人噤声,才在床沿上坐下,轻声哄着道:``海兰,是我,是我来了。''

海兰睁大了惶恐的双眼,像是一只刚刚逃脱了死亡与袭击的小小的幼兽,无助地裹着被子,想要把自己缩进看不见的角落里。床上的湖水色秋罗帐子随着她剧烈的颤抖像是被厉风刮过的湖面,无声地漾起起伏不定的波縠。她喃喃地低诉着,带着深受刺激后的低沉与惊悚:``他们打我的脚,他们,他们要搜我身上!姐姐!我受不了,我再也受不了了!''

情绪激烈地波动间,海兰的双足从被子底下露了出来,厚厚地缠着一层层白纱,隐约还有暗红的血点子干涸了凝在上头。如懿轻轻地抚了抚她足上的白纱,挪到床里,隔着被子揽住她,柔声道:``别怕,别怕,这儿是延禧宫了,你就在我身边住着。什么都不用怕,再没人冤枉你了。''

海兰伏在她怀里,呜呜咽咽地抽泣着。那声音低低的,惶惑的,又那样无助,含了无穷无尽的委屈和畏惧,一点一点地往外倾吐着。如懿抱着她,她的眼泪是滚烫的,身体也是滚烫的,可是这滚烫底下,她的心却是和外头冻实了的冰坨子一样,寒到了极点。如懿由着她哭,仿佛海兰的眼泪也是替自己流着,热热地洇在皮肤上,慢慢渗进肌理里去,那样灼热的,好像灼伤了肌肤,就能连带着心里也暖和点似的。

也不知过了多久,海兰才慢慢平伏下来。如懿伸手搭了搭她的额头,柔声道:``额头比我还烫,今儿是冻着了吧?没事儿,太医院的药好得很,喝下去就好了。''她轻轻地拍着海兰的肩膀,像哄着婴儿似的,``药是治病的,别管是你身上的风寒还是脚上的伤,都会好起来。要是心里还害怕,你就想着,这儿是延禧宫,离她的咸福宫远远的。有什么事儿,你说一声我在前殿就听见了。''

海兰呜咽着埋首在她怀里:``姐姐,还好你在。''

如懿替她绾一绾松散的鬓发,语气温沉沉的:``我在这儿呢。''

海兰紧紧地攥着如懿的手腕:``姐姐,我没想到你会来,如果你不来,我一定被她们\ldots\ldots{}''她哽咽着说不下去了。

如懿取下绢子替她擦着额角沁出的汗:``今儿晚上,我本不想来,别说你,我也忌惮她。可是我不能不来,心在嗓子眼儿里跳着,催着我来。从潜邸到如今,多少年来,我也只和你还有纯嫔说得上话。我要不来,或许从此就不知道你在哪儿了。还好,还好事情都过去了。''她看着叶心,``太医开的药还在吗?端来给你们小主喝下去发发汗,再喝一剂安神汤。''

海兰死死攥着如懿的手不肯放,哀哀道:``姐姐,你别走。''

如懿忍着手腕上的疼痛,微笑道:``我不走,我看你睡下了再走,好么?''她接过叶心递来的药,``喝下去,喝下去病就好了。''

海兰顺服地一口一口咽了下去,如懿替她抹了抹嘴角,扶她躺下,替她掖好了被角。海兰安静地蜷缩着,闭上了眼睛。

次日外头落着雪雨,越发冻得人不愿意出去了。屋子里点了沉水香,透着木质淡若轻岫一般的雅淡香气。饶是如此,因着炭盆生得多,尤是闷闷的,唯有几上青花缠枝美人觚里插着几枝新开的淡红色玉蝶梅上,那鲜妍的色彩才让人心头稍稍愉悦。如懿倚在暖阁里养神,正眯着眼睛,忽然见帘下站了一个湖蓝宫装女子,不由得起身招手道:``天寒地冻的,你怎么来了?''

纯嫔笑盈盈侧了侧身,施了一礼,上前坐下道:``原本想去看看海常在,听叶心说昨儿后半夜喝了安神汤还睡着,所以先过来看你。''她看如懿额上围着大红猩猩毡镶碎玉粒子昭君套,披着一身厚厚的多宝丝线密花锦袄,身上还严严实实盖着一床青红舍利皮镶边的红缎锦被,便关切道,``海兰病着,你也没好多少,这些天可不许见风了。''

如懿含笑道:``一早皇后宫里来嘱咐过了,免了我和海兰这些天的晨昏定省,只叫我们歇着。''

纯嫔点头道:``这是应该的。现在可好些了?''

如懿举过茶盏给她看:``眼下都不许我喝茶了,都换成了姜茶。从昨儿起就喝了好多的姜汤了,太医院的药也喝下去发汗了,现在只觉得热得慌。''

纯嫔伸手替她掖了掖锦袄,叹道:``昨儿夜里闹成这样,我早早睡下了竟不知道。今儿一早听说了,我还以为是宫人们乱嚼舌根呢。直到见了嘉贵人才知道是真的。''她念了句佛道,``阿弥陀佛,福祸相倚,还好海兰搬离了咸福宫,也算没白受罪。倒是你,怎么把你也扯进去了呢?''

如懿按了按额头上勒着的昭君套,低声道:``我只问姐姐一句,姐姐相信海兰会偷盗么?''

纯嫔微微吃了一惊,笃定地摇摇头:``皇上不是说那红箩炭是他悄悄儿赏的么?''

如懿伸手拨弄着瓶里供着的那几枝玉蝶梅:``皇上也是为了息事宁人,顺嘴儿安抚过去罢了。我只有那一句话,既说海兰都偷了,那剩余的一百多斤炭海兰能藏到哪儿去?这件事若再查下去,谁都不好看。''

纯嫔眉心微曲,如曲折的春山逸远:``我还以为是皇上心疼你们,所以连那挑拨是非的香云打死了都还塞了一嘴的热炭。今儿早上尸车运出神武门的时候,听守门的侍卫说,香云的嘴都烫烂了,不成个样子。这么看,皇上是给贵妃台阶下了。''

如懿寸把长的指甲掐在梅枝上,汁水细细地沁了出来:``谁知道呢?我只管着自己鼻塞头昏的。''

纯嫔轻轻一嗅:``既然还鼻塞头昏的,就该点点冲鼻醒神的藏香。这沉水香好闻是好闻,却太清淡了。满宫里也只有你喜欢用,旁人是看都不看一眼的。''

如懿看着地下香潭清水里浸着的一块陡峭似山形的黑釉色的木块,静静道:``倒也不只是为了这个味儿。沉香如定石,能沉在水底,故名沉水香。我只是觉得,若是能心若沉水香一般,世事再缭乱,也可以不怕了。''

纯嫔微微出神,盯着如懿的面庞道:``我刚认识你的时候,你并不是这样的性子。''

如懿的笑意淡得若一缕轻烟:``从前事事有人惯着护着,如今可没有了。''

纯嫔似是触动了心事,眉间也多了几许清愁:``你只想着要静下心来,却没想过,慧贵妃如今敢这样嚣张,无非是她有着`欢作沉水香,侬作博山炉'的恩情宠幸。妹妹要是想一改境况,也该好好留心着圣宠,别让贵妃和新人占尽了恩宠。''

如懿明白她意下所指,便问:``这几天皇上似乎都没召见玫答应,是怎么了?''

纯嫔微一凝神,靠近如懿道:``别说是你,我也觉得奇怪。这些天虽说皇上忙于朝政,除了昨夜召幸皇后之外,都没翻过别人的绿牌子。可是我却听说,其实有两日午后皇上是召了玫答应去弹琵琶曲的,可是玫答应却推辞身体不适,并未奉召前去。''

如懿心下也生了一层疑云:``照理说她新得圣宠,应该极力固宠才是,怎么会自己推辞了呢?''

纯嫔摇了摇头:``谁知道呢?我只听说她脸上不大好,难不成那天贵妃让双喜下的手太狠,怎么都好几日了还没见好呢?''她想着忍不住低低笑了一声,``算了。这件事玫答应自己是打落牙齿和血吞,也没闹出贵妃的事来。左右她没在皇上跟前,昨儿咸福宫的又说发了寒证,今儿皇上已经传旨了,午膳和晚膳都留在咸福宫陪着她用,又左赏赐右赏赐的,太医一趟趟地往咸福宫跑。''

如懿心中皱得跟一团揉碎了的纸似的,只勉强笑道:``皇上一向喜欢她,你是知道的。''

纯嫔聊了几句,见扯上了``恩宠''这样的话,也是伤感,便嘱咐了几句让如懿好好调养的话,便也走了。惢心端了药进来服侍如懿喝了,又拿清水漱了口,阿箬便端了几颗酸渍梅子过来给如懿润口。

惢心倒了漱口水进来,道:``小主,方才海常在醒了,烧也退了。''

如懿想了想道:``那就好。如今叶心一个人伺候着不够,内务府拨过来的人也不敢用,再出一个香云这样的可怎么好?''

惢心含笑道:``小主放心。奴婢已经拨了咱们宫里的春熙过去了,那丫头老老实实的,言语也不多,是潜邸里用老了的人了。''

如懿正要说话,阿箬横了惢心一眼,道:``光惦记着别人那里有什么用呀?小主,叫奴婢说,一个香云出在海常在宫里就够让人寒心的了,要是咱们宫里出了这样的奴才,那可就倒了八辈子霉了。''

如懿赞许地看了阿箬一眼,吩咐道:``满宫里的宫人,除了你们两个和三宝,其他的人,哪怕是绿痕这样的,都要仔细留意着。香云平时不言不语的,算是个没嘴儿的葫芦了吧,一被人收了去,就能张嘴咬自己的主子,还不往死里咬不罢休。''她沉下脸,眼中闪过一丝狠意,``这算是前车之鉴,咱们宫里,绝不能出这样的人!''

惢心与阿箬互视一眼,俱是一凛:``奴婢们会仔细防查,断不能这样。''

如懿松了口气,往后殿张望一眼:``我去看看海兰,她精神好些了么?''

惢心忧心忡忡道:``精神是好些了。可人还是那样子,不肯见人,不肯见光。即便是大白天也扯上了厚厚的帘子,将自己裹在被窝里一动不肯动。''

如懿理了理鬓发,起身道:``那我更得去看看了。''

后殿里静静的,安神香在青铜鼎炉里一刻不停地焚着,由镂空的盖中向外丝丝缕缕地吁着乳白的轻烟。朦胧的烟雾袅娜如絮地散开,弥漫在静室之中,像一只安抚人心的手,温柔地拂动着。

海兰的精神好了许多,只是人干巴巴的,头发也蓬着,唯有一双眼睛睁得老大老大,像两个深不见底的黑洞,警觉地望着外头。整个人嵌在重重帘帏中,单薄得就如一抹影子。如懿才进来,海兰便吓得赶紧缩到床角拿被子捂住自己。待看清来人是如懿,方敢露出脸来。如懿心中一阵酸楚。太医的话其实错了,海兰脚上的伤虽重,延及心肾二脉,但她的心志所受的摧残更厉害。昨晚的羞辱,已经彻底损伤了她的尊严与意志。

雨中的竹叶随风摇曳,竹影轻移,淡淡地映在碧罗窗纱上。海兰立刻惊慌地回头,慌不迭地喊:``拉上!把帘子都拉上。''

宫人们忙碌着,海兰睁着惊惶的眼,一把拉了如懿坐下:``姐姐,在这儿,坐在这儿,哪里都别去,外头都是要害咱们的人!''

如懿抚着她的肩,安慰道:``别怕,天已经亮了,事情也过去了。皇上还是心疼咱们的,这么大的事儿,说揭过去就揭过去了,还让你在我宫里住着。这不是你一直盼着的么?''

海兰呆呆地坐着,任由泪水无声而肆意地滑落:``可是姐姐,只要我一起来,我就觉得好多好多的眼睛看着我,看着我赤足受刑,看着我被人诬陷偷窃,看着我险些被人扒了衣裳搜身。那么多奴才的眼睛看着,我\ldots\ldots{}''她浑身战栗着,大口大口地喘息着,神色惊惧而不安。

如懿紧紧搂着她:``妹妹,我知道你是吓着了。可我们在潜邸里住了这些年,如今待在后宫里,过一天,你应该更明白一天。''海兰憔悴的脸孔对着如懿,露出惶惑的神情,如懿继续道,``昨儿的日子过去,今儿你应该活得更明白。活在这儿的人,风刀霜刃,口蜜腹剑,什么没受过,什么使不出来?昨天一盆冷水浇下来的时候,我真是恨极了。可是恨有什么用?我还得抬起脊梁骨来,受完了继续把日子过下去,然后防备着这样的明枪暗箭再过来。''

海兰怔住了,伸手想要替如懿去擦眼泪,才发觉她的眼窝边如此干涸,并无一点泪痕。她的声音低而柔:``姐姐,你要是委屈,就哭一哭吧。''

如懿的嘴角蓄起一点笑意,那笑意越来越深,慢慢攀上她的笑靥,沁到了她的眼底,那笑却是冷冰冰的:``哭?海兰,她们不是就盼着我哭么?我偏不哭,人人当我昨夜在咸福宫受了委屈,我偏不委屈。忍不过的事,咬着牙笑着忍过去,再想别的办法。我哭?我一哭是乐了她们。''

海兰畏惧地耸了耸肩:``姐姐,不,我不行,我做不到!她那样羞辱我,还有香云\ldots\ldots{}''

如懿扶着她坐直身子:``害你的香云已经被乱棍打死,死了还不算完,还让人塞了一嘴热炭烫烂了嘴。至于其他的人,如果你自己都觉得羞耻,那么人人都会把你当笑话羞辱你。你自己打起精神不当回事儿,人家笑话你你便冲着她笑笑,怎么也不当回事,那便谁也不能再笑话你了。''

海兰出了半天的神,睫毛微微发颤:``姐姐,我做不到\ldots\ldots 我\ldots\ldots 我怕做不到\ldots\ldots{}''

如懿站起身,问叶心:``小主今儿的药都吃了么?''

叶心忙道:``都喝下了,一滴不剩。''

如懿沉声道:``海兰,吃了药慢慢医你的病。至于你的心病,医治的法子我已经告诉了你。你若自己不肯用,就当我昨夜拼死护着的,是一个不中用的人。我护了她这回,却护不了下回。''

海兰怔怔地听着,她的影子虚浮在帐上,单薄得好像唱皮影戏吹弹可破的画纸人。如懿待要再劝,三宝蹑手蹑脚进来,低声道:``小主,皇上宣您即刻去养心殿暖阁见驾。''

阿箬满面喜色,笑道:``小主昨儿夜里受足了委屈,皇上一定是宣您去好好安慰几句呢。''她转脸见海兰颓丧地低着头,忙道,``自然还有话让您带给海常在。''

如懿点了点头,便道:``可说是什么事?''

三宝道:``来传旨的小太监面生得很,只说是要紧事,请小主快去。''

如懿只得起身离去,走了两步又嘱咐海兰:``我的话不好听,可良药苦口,你自己掂量着吧。''

外头下着冻雨,地上湿湿滑滑的,连着雨雪不断的天气,长街的砖缝里一溜一溜地冒着湿腻的霉气,连带着朱红色的宫墙亦被湿气染成了一大片一大片泛白的暗红,看着失去了往日被岁月沉淀后的庄严与肃穆,只剩下累卵欲倾般的压抑。

因是皇帝传召,暖轿走得又疾又稳,不过一炷香工夫,便到了养心殿前。惢心正打了伞扶了如懿下轿,却见一旁的白玉台阶下面,跪了湿淋淋一个人。如懿扬一扬脸,惢心忙扶了她过去,仔细一看,却是皇帝跟前伺候的李玉。

如懿微微吃了一惊,忙道:``李玉,这是怎么了?''

李玉见是如懿,抬起被雨淋得全是水滴子的一张脸,苦着脸道:``娴妃娘娘别问了,无非是奴才做错了事挨罚。''

如懿目光一低,却见李玉并非跪在砖石地上,而是跪在敲碎了的瓦片上。她吃了一惊:``到底怎么回事?''

李玉含着泪道:``左不过是王公公罚奴才罢了。这儿冷得很,娘娘快进去吧。''

如懿见旁人也未注意,低声道:``跪这个太伤膝盖,得了空来趟延禧宫,本宫让惢心给你备下药。''如懿还欲再说,却见王钦迎了出来,皮笑肉不笑道:``娴妃娘娘来了,怎么不进去,在这儿跟奴才说话呢。''

如懿恍若不在意似的:``好好儿的,李玉怎么跪在这儿了?''

王钦冷笑道:``伺候得不当心,拿给皇上的茶热了几分,烫了皇上,可不该挨罚么?娴妃娘娘,下贱人的事儿您别操心了,往里请吧。''

如懿才跨进暖阁,却见皇帝与皇后都正襟危坐着,脸上一丝笑容也无。她心头一沉,便福身下去:``皇上万福,皇后万福。''

\hypertarget{ux7b2cux5341ux516bux7ae0-ux7389ux9762ux4e0b}{%
\chapter{第十八章
玉面(下)}\label{ux7b2cux5341ux516bux7ae0-ux7389ux9762ux4e0b}}

暖阁的窗下铺着一张樱桃木雕花围炕,铺着一色青金镶边明黄色万福闪缎坐褥,炕中设一张白檀木刻金丝云腿细牙桌,上头放了些茶点,想是帝后二人本在此闲话家常。因是寻常对坐,皇后只简单绾了个高髻,簪了小朵的攒珠樱桃绢花压鬓,并几支小巧的流苏银簪,身上一件紫棠色芍药长寿纹缂丝袄,被暖阁里地龙的暖气一烘,倒衬得面容微红。皇后见了她请安,便让素心端了小杌子来让她在跟前坐下,方微微扬了扬嘴角:``娴妃,下着冻雨还叫你过来,实在是有件要紧事得问问你。''

皇后正要说话,皇帝慢慢拣了一枚剥好的核桃肉吃了,淡然道:``昨夜的事,你和海常在都好些了吧?''

如懿心中一暖,欠身道:``臣妾本就无碍,海常在倒是受了惊吓,加上足上的伤,还得好生将养着。''

皇帝道:``既然在你宫里,你就费心些照看着吧。嘱咐她宽心些,已经过去的事便不要想了。''

如懿答应着,皇后含了谦和的笑容,向皇帝道:``午后冷清清的,这个时候要是玫答应来弹奏一曲琵琶,倒也清闲。只是她五六日不肯面圣了。''

皇帝的笑意极淡,却似这阁中的静尘,亦带了暖暖的气息:``她总说脸上的伤没好,不宜面圣,由得她去。''

皇后微笑道:``那日贵妃是气性大了些,可玫答应也有不是之处,皇上心里惦记着玫答应,却不纵容她,臣妾很是欣慰。''

皇帝的茶盏里翠莹莹如一方上好的碧玉,他悠然喝了一口:``虽然没见着,心里想着,就如见着了一样。''

如懿入宫后才陪了皇帝一次,久久未见圣驾,虽然心里是存着皇帝的叮嘱的,却难免有那么几丝寂寞。那种寂寞,是欢悦明媚的曲子唱着,却知道下一出的唱词里是男欢女爱的失散,是相思相望不相亲的分离;那种寂寞,是花好月圆的美满里,想得见残月如钩的凄冷;那种寂寞,是灯火辉煌,半壁盛世里的一身孤清的影子;可是再寂寞,那滋味却是温凉温凉的,凉了一阵儿,总还有盼望,有希冀,那便是温热的一层念想。直到昨儿夜里匆匆相见,原本以为皇帝是护着自己的,可是他的眼风却没几次落到自己身上,便是落到了,也像天际上远远飞着的鸽子,落不到绵白的云彩里。

她的目光忽然凝在皇后的衣衫上,那样沉稳而不失艳丽的紫棠色,热闹簇绣的芍药蜂蝶图案,绣着万年青的寿字滚边,映得自己身上一袭梅子青绣乳白色凌霄花的锦衣,是那样暗淡而不合时宜。而凌霄,本就是那样孤清的花朵。

如懿的喉咙里像含着一颗酸透了的梅子,吐不出也咽不下,她脸上挂着勉强的笑意,忍不住问道:``玫答应伺候皇上的日子也不久,怎么皇上这样喜欢她?''

皇帝原本稀微的笑容渐渐多了几分暖色:``正是因为她跟在朕身边的日子不久,却事事遂心,像一个跟朕久了的人似的,什么事儿都想到了,朕才觉得她贴心投意。''

如懿听了这一句,哪怕心底里再酸得如汪着一颗极青极青的梅子,也只能垂下了眼睛。

皇后的笑意凝在唇角,似一朵将谢未谢的花朵,凝了片刻,还是让它张开了花骨朵:``说起这个事儿来,臣妾有句话不知当说不当说。''

皇帝微笑道:``皇后跟朕,有什么不当说的?''

皇后笑容微微一滞:``午膳过后,玫答应来找臣妾,给臣妾看了看她的脸,臣妾一时间不敢定夺,只好带了她过来见皇上。玫答应哭哭啼啼的,现在也不敢进殿来,臣妾想那日玫答应被掌掴的事娴妃是亲眼看着的,又送她回了永和宫,所以急召娴妃过来。也请皇上看一看玫答应的脸吧。''

皇帝颇为意外:``蕊姬来了?人在哪里?''

皇后郁然道:``人在偏殿等着,就是不敢来见皇上。''皇后见皇帝眉心渐渐起了曲折,便道,``素心,你去请玫答应进来,有什么委屈自己来说吧。''

素心出去了片刻,便领了玫答应进来。玫答应如常穿着娇艳的衣裳,只是脸上多了一块素白的纱巾,用两边的鬓花挽住了,将一张清水芙蓉般的秀净面庞遮去了大半。

她眼里含着泪花,依足了规矩行了礼,皇帝未等她行完礼便拉住了道:``这是怎么了?即便是受了两掌,这些日子也该好了啊。''

玫答应撑不住哭起来,娇声娇气道:``横竖是伤在臣妾脸上的,皇上看个乐子,还觉得红肿着挺喜兴的呢。''

如懿听着她与皇帝这样说话,蓦然想起自己初嫁的时候,晨起时对着菱花镜梳妆,也和皇帝这样有一搭没一搭地玩笑着,撒着娇说着贴心话儿,并无尊卑之分。那年岁,真当是一生中最天真无忧的好时候。只是就这么着弹指过去了,到了眼下,见皇帝一面不易,却眼睁睁看着他与新人亲近欢好,一如对着当日的自己。

她想着,便抬眼看了看皇后,皇后只是垂着脸,像庙宇里供奉着的妙严佛像,无喜无悲,宝相庄严。如懿把玩着衣襟上垂下的金丝串雪珠坠子,那珠子质地圆润而坚硬,硌得她手心一阵生疼。她越发觉得风寒没有散尽的晕眩逼上脸来,少不得按了按太阳穴,替自己醒醒神。

玫答应哭着,便将脸上的纱巾霍地扯下,如懿瞥了一眼,差点没吓了一跳。玫答应的脸原本只是挨了掌掴红肿,嘴角见了血,此刻不仅肿成青紫斑驳的一块一块,嘴角的破损也溃烂开来,蔓延到酒窝处,起了一层层雪白的皮屑,像落着一层霜花似的,底下露出鲜红的嫩肉来。

皇帝惊得脸色一变:``你的脸\ldots\ldots{}''他未说下去,与皇后对视一眼,皇后即刻道:``这个样子,断不是掌掴造成的,必是用错了什么东西,或是没有忌口。''

玫答应立刻跪倒在地上,眼波哀哀如夜色中滴落的冷露,哭诉道:``臣妾爱惜容貌,不敢破了面相惹皇上不高兴。得罪了贵妃是臣妾的不是,挨了打臣妾也该受着,但臣妾已经饮食清淡,按时用药了。可是脸却坏得越来越厉害,臣妾心里又慌又怕,不敢面见皇上,只得告诉了皇后娘娘。''

皇后担心道:``臣妾问过伺候玫答应的人,都说她这几日饮食十分注意,连喝水都特意用了能消肿化淤的薏仁水,也不忘拿煮熟的鸡蛋揉着,是够当心了。''

皇帝微一沉吟:``你说你用药了?是哪儿来的药?''

玫答应停了哭泣:``是太医院拿来的,说是贵妃打了臣妾,也愿意息事宁人,所以特意送了药来,略表歉意。''

皇帝目光微冷:``那药你带来了么?''

玫答应从袖中取出一个小小的圆钵,素心忙接了过去,打开一闻,道:``当日是奴婢去太医院领的药,是这个没错。''

皇帝的眼神微有疑惑,皇后便道:``那日臣妾也在,为了后宫和睦,是臣妾劝贵妃送药给玫答应,也是臣妾让素心以贵妃的名义去取的药。''

皇帝眼中闪过一丝赞许的光彩:``皇后有心了,朕有你周全着,后宫才能安稳如斯。''

皇后安然一笑:``皇后的职责,不正是如此么?臣妾只是做好分内之事罢了。''

皇帝便不再言,只问道:``王钦,朕记得刚有太医来替朕请过平安脉,还在么?''

王钦恭声道:``是太医院的赵铭赵太医,此刻还在偏殿替皇上拟冬日进补的方子呢。''

皇帝微微一凝:``着他过来,看看这药有什么名堂。''

王钦立刻去请了赵太医进来,赵太医是个办事极利索的人,请过安一看玫答应脸上的红肿,再闻了闻药膏,沾了一点在手指上捻开了,忙跪下道:``这药是太医院的出处没错,只是被人加了些白花丹,消肿祛淤的好药就成了引发红肿蜕皮的下作药了。''

皇后蹙眉道:``白花丹?怎么这样耳熟?''

赵太医恭谨道:``是。入了冬各宫里都领过白花丹的粉末,配上晒干的海风藤的叶子,是一味祛风湿通络止痛的好药。宫里湿气重,皇后娘娘的恩典,每个宫里都分了不少,做成了香包悬在身上。只有玫答应新近承宠,她的永和宫刚收拾出来,所以是没有的。''

如懿亦道:``是。臣妾的宫里上个月也领了不少。''

皇后连连道:``可不是!臣妾与娴妃身上都挂着这样的香包。''

皇帝避免目光与玫答应的脸相触,只道:``白花丹到底是什么东西?''

赵太医道:``白花丹若与其他药配用,那是一味好药。但若单用,却是一种极霸道的药物,是有毒性的。只要皮肤与白花丹接触,只需一点点,便会红肿脱皮,继则溃破,滋水淋漓,形成溃疡。以后溃疡日久不愈,疮面肉色灰白或暗红,流溢灰黑或带绿色污水,臭秽不堪。疮口愈腐愈深,甚至外肉脱尽,可见胫骨。答应小主的病征,便是这药膏里被掺了白花丹。''

玫答应一听便哭了出来,指着素心道:``皇上,皇上,臣妾不知得罪了什么人,竟叫素心拿了这样的药来害臣妾!''她虽说的是素心,眼睛却瞪着皇后,恨声道,``臣妾自知出身微贱,要是有人容不得臣妾侍奉皇上身侧,臣妾宁可一头碰死在这里,也受不了这些下作的手段!''

皇后神色大变,立刻起身道:``皇上明鉴。药虽然是臣妾让素心去拿的,可若是臣妾做下的这等天理不容的事,臣妾还怎敢带玫答应来养心殿,一定百般阻挠才是啊。''

皇帝啜了一口茶,扶住皇后道:``皇后一向贤惠,朕是有数的。只是素心\ldots\ldots{}''

素心慌得双膝一软,立刻跪倒在地:``皇上明鉴,皇后娘娘明鉴,那日是奴婢亲自取的药,亲自交到玫答应手里,可奴婢不敢往那药里掺和别的东西呀!''她忽地想起什么,撩起袖子道,``那日臣妾取药的时候在太医院被裁药的小剪子误伤了,当时太医们就指点着奴婢用这钵里的药取了一点涂上,说有止血的功效。奴婢当时用了,也没再溃烂哪。''

素心的手腕留着指甲大的一个红色的疤痕,显然是几天前伤的。她急急地辩道:``奴婢不敢撒谎,这事儿太医院好些太医见着的,都可以为奴婢作证。''

赵太医便道:``皇上,皇后娘娘,那日微臣也在太医院,是有这个事。因这种药膏配制不易,那日只有这一瓶了,就从钵里取了一点给素心姑姑用了。''

皇后凝神一想:``当时用了没事,那素心,你一路上过去,有谁碰过这个药膏没有?''

素心斩钉截铁道:``绝没有了,奴婢赶着过去,到了永和宫只有娴妃娘娘陪着,奴婢给了药便走了。''

玫答应绞着帕子,恨得银牙暗咬:``是了。那日素心送了药,娴妃陪臣妾坐了会儿也走了。之后再没旁人来探视过臣妾了。''

皇帝的目光落在如懿的面庞上,带了一丝探询的意味:``娴妃,你待在那里做什么?''

殿内龙涎香幽暗的气味太浓,被暖气一熏,几乎让人透不过气来。如懿面色沉静如璧:``皇后娘娘让臣妾陪玫答应回永和宫,臣妾说了几句话就走了,并没有多留。''

皇后眼波似绵,绵里却藏了银针似的光芒:``那么其实除了娴妃,便没有别人再能碰到那瓶药膏了。永和宫里,也没轮到给这个。娴妃,你能告诉本宫,是怎么回事么?''

如懿跪在寸许长的``松鹤长春''织金厚毯上,只觉得冷汗一重重逼湿了罗衣。她从未这样想过,从那次掌掴开始,到她送玫答应回永和宫以及药膏送来,种种无意的事端,竟会织成一个密密的罗网,将她缠得密不透风,不可脱身。

心中惊悸如惊涛骇浪,她脸上却不肯露出分毫气馁之色,只望着皇帝道:``皇上,臣妾没有做过,更不知道其中原委。''

皇后颇有为难之色,迟疑道:``皇上,玫答应出身乌拉那拉氏府邸,想来娴妃顾念情谊,一定不会做这样的事。''

玫答应转过脸,逼视着如懿,语气咄咄逼人:``嫉妒之心人人有之,嫔妾也知道自从承蒙皇上恩宠,便被人觊觎陷害,却不想这样的人竟是娴妃娘娘!敢问娘娘一句,那日除了你,还有别人有机会在嫔妾的药膏里下白花丹的粉末么?''

如懿平视于她,并不肯有丝毫目光的回避,平静道:``当日本宫一直在你跟前,说了几句话就走,如果你一定认定本宫会当面害你,那本宫无话可说。''

皇帝望着如懿,幽黑的眸中平静无澜:``既然闹出这样大的事情,还伤了玫答应的容颜,朕就不能不彻查。''

皇后歉然道:``嫉妒乃是嫔妃大罪,何况暗中伤人。后宫管教不严,乃是臣妾的罪过。''

皇帝凝眉道:``皇后是有过失,但罪不在你。''他眼底闪过一丝不忍,恰如流星闪过的尾翼,转瞬不见。

皇后思虑片刻,道:``娴妃,无论是不是你做的,总要问一问。去慎刑司吧,有什么话,那里的精奇嬷嬷会问你。''

如懿身上一凛,慎刑司掌管着后宫的刑狱,上至嫔妃,下至宫人,一旦犯错,无一不要在里头脱一层皮才能出来。她忍着身上寒毛竖起的不适,强撑着身体俯身而拜:``事关臣妾清白,臣妾不能不去。只是请皇上相信,臣妾并非这样的人。''

皇帝微微颔首,语意沉沉:``你放心。''

不过三个字,如懿心中一稳,觉得浑身都松了下去。惢心忍不住哭求道:``皇上,即便要问小主的话,也别去慎刑司呀。小主昨晚已经着了风寒,哪里还禁得起这样折腾。皇上!''

皇帝温和道:``若是风寒,朕会让太医去诊治。但规矩是不能破的。''

皇帝话语的尾音尚未散去,只听外头砰的一声响,有人用身体撞破了门冲进来道:``皇上,不是姐姐干的!不是!是臣妾做下的事情,您带臣妾去慎刑司吧!''

\hypertarget{ux7b2cux5341ux4e5dux7ae0-ux4e24ux8d25}{%
\chapter{第十九章 两败}\label{ux7b2cux5341ux4e5dux7ae0-ux4e24ux8d25}}

随着冷风重重灌入,海兰扑到皇帝跟前,死死抱住皇帝的腿道:``皇上,是臣妾嫉妒,臣妾看不惯玫答应得宠,一时起了坏心,是臣妾害她的!不干姐姐的事!''

皇帝皱眉道:``你怎么来了?''

外头小太监怯怯道:``海常在来了好一会儿了。跟着她的叶心说常在见娴妃娘娘久久未回宫,一时担心所以出来了。因为听见皇上在里头问话,所以一直在殿外不敢进来。''

皇后看着海兰的样子,忧心道:``海常在刚受了足伤,身子又不好,你们怎么不拦着?''

那小太监吓得磕了个头:``奴才,奴才实在是拦不住啊!''

皇后秀眉微曲,示意素心拉开海兰,道:``海常在,本宫知道你担心娴妃,但这样的大事,不是谁都能担得起的。你说是你下的白花丹,那本宫问你,你何时去过永和宫,何时下的药?''

海兰微微语塞,立刻仰起脸一脸无惧道:``只要臣妾想下药,何时何地都能下!左右这件事不是娴妃做的!''

皇后神色肃然,严厉道:``海常在,本宫知道你与娴妃姐妹情深,但这种事岂能是你替她背的!''

海兰本伏在地上,听得这一句立刻仰起脸来,梗着脖子倔强道:``不是臣妾要替娴妃姐姐背,只是这件事,一定不会是姐姐做的,但若真要认定是姐姐,那就算是臣妾做的。''

海兰一向怯怯的不太言语,骤然间言辞这样激烈,连皇帝也有几分信了:``那么海兰,你为什么认定不会是娴妃做的?''

海兰一把扯下如懿纽子上佩着的芙蓉流苏香包,她用力过大,将香包上垂着的精致缨络也扯了好几缕下来,颤颤地缠在指尖上。海兰用力解开香包:``因为姐姐香包里根本没有白花丹,她又如何能拿白花丹来下药?''

香包里的东西在她掌心四散开来,唯见几片枯叶与深红色的粉末。赵太医忙取过细看:``皇上,白花丹的粉末为青白色,此物深红,乃是大血藤磨粉而成。''

如懿又惊又疑,只得道:``臣妾记得当日内务府送来的白花丹粉末成色不佳,本说要换的,后来海常在看香包缝得不严实,将延禧宫的都拿去重新缝了一遍。至于里面的白花丹为何不见了\ldots\ldots{}''

海兰戚戚然道:``臣妾知道内务府敷衍娴妃姐姐,送的都是些次的东西。延禧宫地冷偏僻,只怕那些白花丹粉不顶用。正好臣妾宫里有多余的大血藤粉,与白花丹一样都是祛风湿通络止痛的。所以就用上好的大血藤粉换了白花丹。试问姐姐的香包里没有白花丹,又怎能害人?''

玫答应横了海兰一眼,旋即道:``既然大血藤与白花丹功效一样,谁知有毒还是无毒?''

皇帝看一眼赵太医,赵太医立刻道:``皇上,大血藤无毒,绝不会损伤答应小主容颜。''

如懿绷紧的身体终于松懈下来,紧紧握住海兰的手,忍不住热泪盈眶:``海兰,我此身能得分明,都是你了。''

海兰不知哪来的勇气,沉声道:``姐姐不用谢我。要谢就谢内务府藐视姐姐,敷衍姐姐,才使姐姐逃脱一难,免于受苦。''她直挺挺跪着道,``皇上若是不信,大可一一去查。若还有人觉得是姐姐做的,就带臣妾去慎刑司吧。''

皇帝伸手扶起海兰与如懿,温和道:``好了。海兰,从前见你不言不语的,原来如此勇气可嘉。''他的手拂过如懿的手背,有一瞬的停留,``你的委屈,朕都知道。这件事朕会再查,你放心。''

海兰羞得满面通红:``臣妾没什么勇气,只是姐姐怎么拼死护着臣妾的清白,臣妾也怎么护着姐姐就是了。''

皇帝的目光扫过皇后的面庞微微一滞,很快笑道:``这么说,朕没有白白让你住进延禧宫去。倒成全了你们俩好生照应着。''

皇后忙含笑起身,蕴了一分肃杀之意:``这件事,臣妾以为一定要彻查到底。否则无以肃清宫闱,以正纲纪。''

皇帝道:``既然这件事由贵妃而起,也差点蒙蔽了皇后,不如还是交给娴妃去查。后宫琐事众多,又到了年下,皇后安心于其他事务吧。''

皇后身子微微一晃,几乎有些站不住脚,脸上却撑着满满的笑意:``是。从前潜邸的时候,娴妃就很能帮得上忙。''

皇帝又道:``娴妃,不管查出什么来,这件事朕就交给你去处置。''他转头吩咐赵太医,``赵太医,你好好给玫答应治治,该不会落下什么疤痕吧?''

玫答应闻言又要落泪,但见皇帝脸色不好,只得硬生生忍住了。赵太医忙道:``还好下的白花丹分量不多,微臣仔细调治,不过半个月就能好,断断不会留下什么疤痕。''

皇帝道:``那便好。都下去吧。''他见如懿和海兰欠身离去,温言嘱咐,``海常在,你仔细着自己的身子,娴妃也别再着了风寒。''

二人答应着退下了。皇帝见四下再无旁人,也不理皇后将剥好的橘子递过来,只看着别处道:``这件事虽是由贵妃莽撞而起的,玫答应也有些娇气。但你是皇后,事情未查清楚,便对娴妃有了疑心。后宫之事虽多,但只讲究一个公正无疑。你是中宫,心也该摆在中间。''

皇后安静地听着,勉强浮了一丝笑意:``臣妾也是看见玫答应的脸有些吓着了,娴妃又接二连三地扯进是非里去,所以有些着急。''

皇帝口吻愈加冷:``那些是非是娴妃自己要扯进去的么?你是中宫,朕的皇后,这个位子你坐着,便不能急,只能稳。这样朕的后宫才能稳。''皇帝换了温缓些的口气,``眼下宫里才这么几个人,来日人更多了\ldots\ldots{}''

皇后听得这一句,只觉得心口酸得发痛,舌底也涩得转不过来,只得勉力镇定下来道:``是臣妾年轻不够稳重,处事毛躁,以后断断不会了。臣妾会加倍当心的。''

皇帝嗯了一声:``那朕去和贵妃用晚膳,你也早些回去吧。''

皇后答应着出去,外头的冷风如利刃刺进眼中,她都感觉要沁出滚热的血了。片刻,眼中只有发白的雾气,她扬一扬脸,再扬一扬脸,紧紧地攥着手指,忍耐了下去。

如懿和海兰的软轿一前一后回了延禧宫。踏过朱红色的宫门槛的时候,如懿才觉得脚下有点发软。海兰忙搀住了她,从叶心手里接过伞举着。

如懿扶着她站稳了,嗔怪道:``你刚才这样不要命地冲进来,真当是不顾自己了么?''

海兰黯然道:``我只有姐姐了,若是姐姐被她们冤枉了去,我还有什么依靠?何况姐姐昨夜怎么救的我,我以后也一样救姐姐。''

如懿看着她,心底的感动难以言语,只是牢牢握住了她的手,以彼此的温度温暖着对方:``我以为你怕成那样,以后都不敢走出延禧宫了。''

海兰眼中的光彩渐次亮起来:``怕过了昨日,今日还有更怕的。姐姐说得对,我若是一直这样怕下去,别人还没把我怎么样,我自己先掐死了自己。''

如懿稍稍宽慰:``但愿我们以后,只这样扶持着走下去,不要再有昨日和今日这样的事了。''

两人撑着伞走在凄凄冷雨之中,如懿挽紧了她的手臂,彼此的身影依偎得更紧了。仿佛只有这样,才能抵御这深宫中无处不在的寒冷与阴厉。

入了宫中,如懿先陪海兰回了后殿看她足上的伤口上了药,等着天色擦黑了,便见惢心悄悄儿带着李玉进了暖阁。

李玉在门口踌躇了一会儿,如懿向他招手道:``怎么不进来?''

李玉迟疑着:``小主,奴才是怕给您招麻烦。''

如懿停了手里拣艾叶的功夫,笑道:``本宫自己还不够麻烦的么?要是怕麻烦,便不叫你来了。你放心,这个时候王钦跟着皇上在咸福宫伺候,没空理会你了。''

惢心扯了李玉一把,李玉拐着腿便坐下了,如懿让惢心搬了个小杌子过来让李玉坐下,惢心手脚麻利地替李玉卷起裤腿,李玉忙遮了一下,惢心笑道:``好吧,你要害羞就自己动手。''

如懿忍不住笑:``卷起来看看,在本宫这儿怕什么?''李玉臊眉搭眼地卷了裤腿起来,如懿见膝盖上又红又紫一片,夹杂着青肿,跟油彩似的,翻起的皮肉还往外渗着血,不由得变了神色,便问,``跪了多久?''

李玉带了几分伤心委屈:``一个时辰的碎瓦片,瓦片都跪得碎成渣了,又换了铁链子跪了一个时辰。''

如懿带了几分探询的意味打量着他:``就为你伺候皇上一时有不周到的地方?''

李玉惹出了伤心,抽抽搭搭道:``就为了几桩差事,奴才露了几分乖,讨了皇上的喜欢。王副总管就不高兴了,做什么都挑奴才的刺。这不今天被他逮了机会,就狠狠罚了一通。''

如懿叹了口气,伸手从紫檀架子上取下一瓶药粉,小心翼翼地往他伤口上撒了。李玉疼得直龇牙,忙拦着道:``娴妃娘娘,您玉手尊贵,怎么能麻烦您替奴才做这样的事?''

如懿撩开他的手:``这是云南剑川上贡的白药粉,兑着三七和红花细磨的,止血祛淤最好不过了。你要想明天还站起来在御前伺候,当着这份差事,就乖乖坐着上药。''

惢心笑着在李玉额头戳了一下:``瞧你这好福气。我伺候小主这么久,也只一回烫伤的时候小主替我上过药。''

李玉感激得热泪盈眶:``多谢娴妃娘娘。''

如懿叹道:``你不必谢,要不是昨晚惢心通报的时候你替她向皇上传了话,本宫还不知道落到什么田地呢。''

李玉微微正色:``那是因为王副总管不肯,惢心又与奴才是一早相识的。奴才想着,总不能让娘娘在咸福宫遭难。别看皇上平日里不太到延禧宫,心里却是在意的。''

如懿微微失神,旋即道:``这就是你比王钦聪明的地方了。可是王钦资历老,位次高,你的聪明要是随随便便露了出来,不好好藏在心里,就是害了自己了。''

李玉若有所思:``娘娘的意思是\ldots\ldots{}''

如懿取过惢心递来的白纱,替李玉将膝盖包好:``居人之下的时候,聪明劲儿别外露。尤其是上头还是不容人的时候。皇上喜欢你的聪明,别人却未必。回去的时候也别露出怨色来,好好奉承着王钦,毕竟在他手下当差呢。''

李玉拐着腿起来,打了个千儿道:``原是奴才糊涂了,多谢娘娘指点。''

如懿将药瓶塞到他手里:``好生收着药,偷空就上上药。伺候皇上的时候当心点,亮着一百二十个心眼子。''

李玉答应着去了,惢心抿着嘴笑道:``小主终于也肯上心了。''

如懿怔了片刻,慢慢挑拣着艾叶:``能不上心么?连环套这么落下来,差点怎么死的都不知道!王钦是什么人?皇后一早收服了的,只有李玉,聪明,又是你一早结识的可靠人儿。''

惢心低声道:``听说,皇后为了拉拢王钦,打算将身边的莲心给王钦配了对食儿。''

如懿睁大了眼睛:``真的?''

``可不是呢!王钦看上莲心都好久了。只是皇后这么打算着,还没松口。''

如懿出神了一会儿:``皇后也是可怜,万人之上有万人之上的孤寂害怕,就像站在塔尖上,一阵小风都成了大风,吹得人站不稳。''她将手上拣好的艾叶递给惢心,``算了,别想这些事了。把这些艾叶送去给海常在。''

惢心答应着去往海兰处。如懿望着惢心远去的背影心中一阵叹息,这宫里又有谁过得轻巧呢?微末如宫里的奴才,高贵如万人之上的皇后,谁人不是在孤寂害怕中,谨小慎微、如履薄冰。夜色渐要降临,晚归的鸟儿在檐头盘旋着,咕咕作声。``皇上\ldots\ldots 今晚不知翻了谁的绿头牌'',如懿心转此念,一声轻叹转身进房。

皇帝是夜深时分来看的如懿。如懿原本没想到皇帝会过来,已经在寝殿里卸了晚妆,正拿热水兑了玫瑰花拧的汁子浸手。冷不防三宝喜滋滋地从外头进来,一脸捡了元宝的欢喜样子:``小主,皇上来了!皇上\ldots\ldots 您快接驾吧!''

如懿连忙擦净了手,才站起身子,皇帝已经进来了,笑道:``好香的玫瑰花味儿,倒叫朕忘了是在冬天了。''

如懿只穿着一身水玉色的萱草纹寝衣,也不及换衣衫,只得福身下去请安。皇帝忙扶住了她,柔声道:``受了两日的委屈了,还不赶紧坐下。''

如懿凝视着他纹丝不动的衣裾,湖蓝底银白纹饰,是那样熟悉,又带了久未见的陌生。不知怎的,如懿心中蓦然一软,忍了两天的眼泪便潸潸落了下来。众人会意,赶紧退了下去。皇帝伸手沾了她的泪水,低低道:``你不是爱哭的人。这回哭了,是真难为了你。''

四下里寂静无声,唯有沉默的哽咽。大颗大颗的泪珠顺着脸颊滑落在衣襟上,洇出斑驳的泪痕,仿佛夜来霜露,无声地染上了衣裳上的花枝。

皇帝搂过她,静静地按在自己的肩头,欷歔道:``朕以为冷着你一些日子,会对你有好处。至少不会人人的目光都盯着你不放\ldots\ldots{}''他拥得更紧一些,``是朕疏忽了。''

如懿忍一忍泪:``皇上是疏忽了。外头这么冷,夜深了你还过来\ldots\ldots{}''

皇帝握住她的手按在自己心口:``不过来,这里不安稳。''

如懿忍不住低低笑了一声:``那臣妾可以去养心殿。''

话音未落,皇帝已经吻上她的额头,以他的温热来安抚她这几日的惊辱。皇帝的语气低低的,却是那样贴近,就在耳边,也在心上:``朕昨天看你在咸福宫浑身湿透了,朕很想来拉你一把,给你披上衣裳,狠狠责罚那些欺辱你的人。可是如懿,朕不能那样做。因为直到那一刻,朕还以为,朕在人前爱护你,便是害了你。如懿,再出了今日的事,朕却改变了主意。或许朕冷淡了你,所以她们越发以为得了意,以为你失宠,所以敢欺负你,陷害你。你放心,朕不会了,以后不会了。''

如懿依偎着皇帝,感受着他身上陌生而熟悉的气味。那种气味,是让她在覆劫之中尚且觉得安心的来源。她依依道:``臣妾最喜欢皇上说三个字。''

``哪三个字?''

``你放心。有这句话,哪怕臣妾现在身处慎刑司,臣妾也能安心不怕。''

皇帝轻舒一口气:``幸好,你是懂得的。''

如懿挽住皇帝的脖子,额头抵着他的下巴:``臣妾懂得。臣妾初嫁的那一夜,皇上看见臣妾的第一句话,就是一句`你放心'。臣妾这一世的放心,便是从那天开始的。''

皇帝低首吻住她,呢喃道:``你懂得就好。''

如懿是懂得的。但有知心长相重,即便她受了这些日子的寂寞与冷遇,仍能感受如是情意,脉脉蜿蜒于彼此心上。

紫铜蟠花烛台上的烛火一盏一盏亮着,红泪一滴一滴顺势滑落于烛台之上,映着重重紫绡罗帏,浓朱淡紫,混杂了安神香淡淡的香气,幽幽地弥漫开一室的旖旎。

第二日起来是格外好的天气,在一片初阳辉照之中醒来,看着天光放明,冬日里难得一见的朝阳洒下薄薄的金粉似的粲然光芒,透过``六合同春''的雕花长窗的镂空,照出一室淡淡水墨画的深浅。

如懿醒来时皇帝正起身在穿龙袍,王钦和几个宫女忙碌地伺候着。如懿刚仰起身,皇帝忙按住她温声道:``你累着了,好好睡一会儿吧。朕先走。''

如懿脸上一红,嗔着看了皇帝一眼,便缩进了被子里。皇帝刚走,满宫的宫人都喜滋滋地像过节似的,阿箬笑着进来道:``小主,您知道皇上出门前说什么了么?''

如懿瞥她一眼,笑道:``有什么了不得的话,惹得你这样?''

阿箬拖长了语调,学着皇帝的语气道:``皇上说,阿箬,照顾好你们小主,朕晚上再来看她。''

如懿拿被子蒙住脸:``我可什么都听不见,那就是告诉你的,你听着就是了。''

阿箬忍不住笑出了声,往外头去了。

如懿再醒来时已经是巳时一刻了,心里无牵无挂的,睡得倒极安稳。起来梳洗了写了几副春联叫宫人们挂上,便邀了海兰一同过来用午膳。

小厨房的菜向来清爽落胃,海兰又是个不挑拣的,两人说说笑笑,倒吃了好些。正吃着,三宝忽然进来了,垂手站在门边不吭声。如懿知道他是有要紧事,便盛了一碗酸笋鸡丝汤慢慢啜了一口,大概觉得不错,又给海兰递了一碗,才道:``什么事儿?''

三宝的眼睛只盯着地上,道了声``是'',却不挪窝儿。如懿便挥了挥手,示意伺候的人下去:``说吧。''

三宝道:``慎刑司刚来的回话,说太医院有个侍弄药材的小太监去自首了。''

如懿一怔:``自首什么?''

``说是玫答应用的涂脸的药膏里,是他配药的时候不小心沾上了白花丹的粉末在圆钵内壁上,才惹出这么大的祸事。''

海兰端着碗停了喝汤,道:``不对呀,既是沾在圆钵上,怎么素心用了没事,偏玫答应用了有事?''

三宝轻嗤了一声:``那玩意儿说,素心是用了上面的,所以没事。玫答应用得多,便沾上了。''

如懿道:``那慎刑司怎么办?''

三宝道:``已经用刑了,吐来吐去就这两句。所以来请小主的意思。''

海兰放下碗道:``姐姐信么?''

如懿一笑:``那么,你信么?''

海兰坚决地摇了摇头,如懿淡淡一笑:``三宝,去告诉慎刑司,本宫只要他吐完了肚子里的话知道结果可以去回皇上,其余的是他们的差事。''

``可是若逼不出什么了\ldots\ldots{}''

``若是已经吐到底了,就把他打五十大板,打发到辛者库去服役算完。''

三宝答应着下去了。海兰看着她道:``姐姐不细细追查了么?这件事早有预谋,存心是要把姐姐害进去,若是不查\ldots\ldots{}''

如懿气定神闲把汤喝完,摇头道:``查不出来了。''她看海兰不解,便道,``再查下去,那便只有一个,畏罪自杀。慧贵妃可以把事情做绝了,香云打死了,她还要塞上一嘴的炭。我却不能。''

海兰道:``可是事儿闹得那么大,连贵妃和皇后都吃了挂落。''

如懿拨着筷子上细细的银链子:``就是因为贵妃和皇后都吃了挂落,所以不能再查。从你受委屈那晚就该知道,那点红箩炭的事不是查不下去,是皇上不愿意查了。皇上才登基,后宫需要宁静平和,不能惹出那么大的事儿了。皇上的意思既然如此,我又何必追究到底?''

海兰嘴角漾起一抹笑意:``左右这件事是贵妃惹起的,皇后替玫答应说了几句姐姐的嫌疑,皇上也忌讳了。玫答应是受了安慰,可姐姐的委屈也平复了。她们两败俱伤,玫答应无功无过,姐姐反而重新得了皇上的眷顾了。''

如懿笑着拍了她一下:``也学会贫嘴了。既然事情都这样了,再查就伤了脸面,便这样吧。''

夜里皇帝过来时如懿便一五一十对他说了。皇帝换了明黄的寝衣躺下了,听她伏在枕边说完,不觉失笑:``你愿意这样便了了?''

如懿伸手捏了捏皇帝的鼻子,带了一丝顽皮的笑意:``皇上的话,好像不信这是事实似的。''

皇帝微笑着揽过她:``朕有什么信不信的。宫里头一团污秽,后宫更是如此。朕还是皇子的时候,看着先帝的后宫就那么几个人,皇额娘和齐妃她们便斗得那样狠。许多事,再查下去便是无底洞,你肯见好就收,那是最好不过的事。''

如懿笑了笑,安静下来道:``皇上所想,就是臣妾所想了。凡事给别人留有余地,也是给自己留有余地了。倒是玫答应,着实是委屈的。''

皇帝欷歔道:``说到委屈,有谁不委屈的?贵妃觉得她委屈,玫答应也委屈,你和海兰何尝不委屈?朕也十足委屈,前朝的事儿忙不完,后头还跟着不安静。''

如懿伏在皇帝肩上,柔声低低道:``她们不安静她们的,臣妾安静,皇上也不许不安静。''

皇帝笑着轻吻她的额头,西窗下依旧一对红烛高照,灿如星子明光。天地静默间,二人听着檐下化冰的滴水声,自有一分宁静,自心底漫然生出。

\hypertarget{ux7b2cux4e8cux5341ux7ae0-ux6e14ux7fc1}{%
\chapter{第二十章 渔翁}\label{ux7b2cux4e8cux5341ux7ae0-ux6e14ux7fc1}}

如懿得宠的势头便在这次的因祸得福之后渐渐地露了出来,比起贵妃的宠遇深重,如懿自然是不如的,可是皇帝隔上三五天便来看她一回,也是细水长流的恩遇。连带着延禧宫的宫人走到长街上,胸也挺起来了,头也抬高了,再不是以前那低眉低眼的样子。

如懿却不喜欢他们这神色,当着三宝、阿箬和惢心的面再三嘱咐了,要他们叮嘱底下的人,不许有骄色,不许轻狂,更不许仗势欺人与咸福宫发生争执。

叮嘱得多了,别人尚未怎样,阿箬先道:``小主如今这样得宠,何必还怕慧贵妃?再说宫里的人最势利了,老看我们低眉搭脸的,还不知道背后怎么编排呢。''

如懿翻着内务府新送来的冬衣料子,道:``能怎么编排?就因为宫里的人够势利了,你要还自己轻狂,那就是真的眼皮子浅了。得宠不得宠,他们会看不出来?你自己越稳当,别人才越不清楚你的底,越不敢也不能怎样。''

惢心笑着替如懿翻过料子:``这几件大毛的料子原不是份例里的,是内务府额外孝敬了小主的。''她拉过阿箬的手,打开一个包袱道,``这里有两件青哆罗呢羊皮领袍子,一件玫瑰紫的灰鼠皮袄和一条洋红棉绫凤仙裙,是内务府格外孝敬咱们的,我再三问过了小主可以收才收下的。其实那些人的眼睛比刀子还尖呢,什么都看得真真儿的。''

阿箬这才服气,只是抿着嘴笑:``皇上常来,奴婢也替小主高兴嘛。''

如懿道:``越是高兴,越是得不露声色,这才是历练过的人。好了,快年下了,孝敬你们的衣裳都穿上吧,看着也喜兴些。''

阿箬高高兴兴地接过了。过了两日,如懿看阿箬打扮得格外精神,里头穿着青哆罗呢羊皮领袍子和洋红棉绫凤仙裙,外头套着玫瑰紫灰鼠皮袄,头上簪了绯色的绢花和采胜,通身的贵气,竟不亚于宫里位分低的小主了。趁着阿箬在庭院里和三宝清点内务府送来的年货,如懿便问惢心:``我记得内务府额外孝敬你和阿箬的东西,该是你们一人两件的,怎么阿箬一人穿了三件去?我原想着天气冷了,你好歹也该把那件青哆罗呢的袍子穿上了。''

惢心不敢露出委屈的神色,只如常笑道:``阿箬姐姐选了半天,还是件件都喜欢,就都给了她了。''

如懿蹙了蹙眉:``都给了她?那两件青哆罗呢的袍子一模一样的,她要来干什么?''

惢心低了头:``冬日的衣裳,总要替换着的。''

如懿转过脸,透过窗上的霞影纱,正看见阿箬在外头响亮地笑着什么,用手指戳着几个小宫女的脑袋,像是调拨着什么好玩的东西似的。

如懿越发有些不高兴,却不肯露在脸上,便道:``前几日内务府送来一件青绸一斗珠羔皮袄子,我穿着嫌薄,你拿去套在外裳里头穿,倒是挺好。还有一件一起的桃红色软绸裙子,快新年了,穿着鲜艳些。''

惢心眼圈微红,低低道:``奴婢不是小主的家生丫头,小主不必这么心疼奴婢。''

如懿含笑道:``阿箬的性子一向争强好胜,嘴又厉害,你和她住在一块儿,虽然都是大丫头,她明里暗里一定也给了你不少委屈受。就为你什么都没来向我抱怨过,我只要疼你,就是应该的。''

惢心含泪带笑:``那奴婢谢小主的赏。''

如懿笑道:``别谢了,穿上了好看让我觉得高兴,便是最好的了。''

这一日是腊月初八,皇帝留在皇后宫里用了腊八粥,便与皇后在暖阁里说话。皇后将内务府的账簿递过道:``这是这个月后宫的用度,皇上看一眼,臣妾也算有交代了。''

皇帝慢慢翻了几页,吹着茶水含笑道:``皇后厉行节俭,后宫的开支节省了不少,这都是皇后的功劳。只是快年下了,朕见嫔妃们的衣着老是入关时的花色式样,未免在古风之余有些呆板了。''

皇后笑得极为谦和:``皇上说得极是。只是臣妾想着,宫中嫔妃不少,以后还有的是添新人的时候。都是年轻女眷,平日里争奇斗艳是不消说了。皇上初掌大权,前朝尚有许多要动用银两的时候,后宫里能省则省些,也是一点心意。至于皇上以为呆板,臣妾倒以为,大清的祖宗们本是马上得的江山,一刀一枪拼了性命的,后宫的嫔妃尤其不能忘了祖宗的艰难与功德,不该一味追求妆饰华丽,而失了祖宗入关时的俭朴风气。''

皇帝啜了一口茶水,闭目片刻,似乎对茶水的清冽格外满意:``朕才说一句,原来皇后思虑已经这样周详。朕以为,皇后所言,便如这一盏清茶,虽然入口苦涩,回味却有余香。''

皇后恭谨答了句``是'',``若是皇上觉得茶味太清苦,臣妾让人再换一盏八宝茶来。''

皇帝摆摆手:``不必。皇后的意思,朕都明白了。只是朕初立后宫,也就潜邸几个人伺候着,一时裁减了她们的,朕也不忍心。何况她们都还年轻,喜欢娇俏些,只要不过分就是了。皇后且别说,如今快新年了,她们本就穿得厚重,又是沉甸甸的老式绣花,偏偏这些绣花出自宫女之手,也不灵动鲜活,连人也带着沉闷了。本来多些轻灵光鲜的料子,也是一道风景。''

皇后颔首应了,又笑道:``皇上说得极是。只是后宫选嫔妃,与民间娶妾室不同。讲究端正庄严为美,若一个个只晓得打扮,岂不成了狐媚子?妖妖调调的,整日只想着纠缠皇上,也不像皇家的体统呢。''

皇帝正捧着茶盏,听到此节,杯盖不由轻轻一碰,磕在了杯沿上。暖阁中本就安静,冬阳暖暖地隔着明纸窗照进来,连立在阁外伺候的宫人们也成了渺远的身影。青瓷的茶盏本就薄脆,这样一碰,声音清脆入耳,皇后遽然一凛,立刻起身道:``臣妾失言,还请皇上恕罪。''

皇帝静了须臾,伸手向皇后道:``这么多年夫妻了,皇后何必如此。''

皇后就着皇帝手站起来,他的指尖有一缕隔夜的沉水香的气味。皇后心中一动,便能辨出那是延禧宫的香气。皇后稳了稳心神,掩去心中密密渗透的酸楚,一如旧日,微笑相迎。皇帝眷念夫妻之情,一向是常来宫里坐坐的,可是琅分明觉得,那种熟悉已经渐渐淡去。往日那种把握不住的惶惑与无奈一重重迫上身来,她还是觉得不安。

皇后想着,还是恢复了如常淡定的笑容:``臣妾只是为皇上着想。如今新年里,各宫都盼着皇上多去坐坐,譬如怡贵人、海常在和婉答应。''

皇帝凝神片刻,笑道:``朕知道,无非是慧贵妃身子弱,朕多去看了她几次,皇后总不是吃醋吧?''

皇后盈盈望着皇帝的眼睛,直视着他:``臣妾是这样的人么?不过是想六宫雨露均沾而已。''

皇帝扬了扬嘴角算是笑,撇开皇后的手道:``既然如此,朕去看看海兰,皇后就歇着吧。''

皇后看着皇帝出去,脚下跟了两步,不知怎的,满腹心事,便化成唇边一缕轻郁的叹息。

到了正月初一那一天合宫陛见,嫔妃们往慈宁宫参拜完毕,太后一身盛装,逗了几位皇子公主,也显得格外高兴。太后又指着大阿哥道:``旁人还好,三阿哥尤其养得胖嘟嘟的,怎么大阿哥倒见瘦了?''

大阿哥的乳母忙道:``大阿哥年前一个月就一直没胃口,又贪玩,一个没看见就窜到雪地里去了,着了两场风寒。''

太后脸色一沉:``阿哥再小也是主子,只有你们照顾不周的不是,怎么还会是阿哥的不是?下次再让哀家听见这句话,立刻拖出去杖刑!''

那乳母忙讪讪地退下了。皇后见状,忙引了二阿哥和三公主去太后膝下陪着说笑了好一会儿,太后方转圜过来。

嫔妃们告退之后,太后便只留了皇帝和皇后往暖阁说话。

福珈站在暖阁的小几边上,接过小宫女递来的香盒,亲自在银错铜錾莲瓣宝珠纹的熏炉里添了一匙檀香。她看着袅娜的烟雾在重重的锦纱帐间散开,便无声告退了下去。

太后让了帝后坐下,笑道:``听说最近宫里出了不少事,皇后都还应付得过来么?''

皇后安然笑道:``后宫的事,儿臣虽还觉得手生,但一切都还好。''

太后的笑意在唇边微微一凝:``可是哀家怎么听说,皇后忙于应付,差点有所不及?由着她们闹完了咸福宫又闹养心殿,没个安生。''

皇后脸上一红:``臣妾年轻,料理后宫之事还无经验\ldots\ldots{}''

皇帝便道:``你没有经验,皇额娘却有。''他含着笑意看向太后,``皇额娘,后宫的事,还劳您多指点着。有您点拨,皇后又生性宽和贤惠,她会做得更好的。''

太后道:``哀家有心颐养天年,放手什么都不管。可是皇后仿佛心有余而力不足啊。这后宫统共就这么几个人呢,你还安定不下来,真是要好好学着了。''

皇后低着头,一眼望下去,只能看见发髻间几朵零星的绢花闪着,像没开到春天里的花骨朵,怯怯的,有些不知所措:``回皇额娘的话,儿臣明白了。''

太后捻着手里的枷楠香木嵌金寿字数珠,慢悠悠道:``满宫里这么些人,除了宫人就是妃嫔,她们见了哀家,是自称奴婢自称臣妾的。唯独你和皇帝是不一样的,你们在哀家面前是`儿臣',既是孩儿,又是臣下。所以皇后,哀家疼你的心也更多了一分。''

皇后恭谨道:``是。''

太后微微闭眼,仿佛是嗅着殿内檀香沉郁的气味。那香味本是最静心的,可是皇后腔子里的一颗心却扑棱棱跳着,像被束着翅膀飞不起来的鸽子。她抬眼看着太后,她略显年轻却稳如磐石的面孔在袅袅升起的香烟间显得格外朦胧而渺远。好像小时候随着家里人去庙宇里参拜,那高大庄严的佛像,在鲜花簇拥、香烟缭绕之中,总是让人看不清它的模样,因而心生敬畏,不得不虔诚参拜。

皇后一直对太后存了一分散漫之心,只为她知道,当日迁宫的风波,种种起因,不过是因为太后并非皇帝的生身母亲。却从未想到,这样与世无争安居在慈宁宫的深宫老妇,会突然这样警醒,字字如锋刃挑拨着她的神经。呵,她是失策了,她以为自己是六宫之主,却不承想,这个在紫禁城深苑朱壁里浸淫了数十年的妇人,才是真正的六宫之主。

太后的声音不高,却沉沉入耳:``哀家疼你,却也不能不教导你。皇后,你失之急切了。''

皇后身上一凛,只觉得后颈里一凉,分明是有冷汗逼迫而出。这可是冬日啊,滴水成冰的冬日,她居然沁出了汗珠。她只得道:``臣妾恭听皇额娘教诲。''

``你要节俭,哀家只有夸你,不能指摘你。可是皇后,你厉行节俭是不错,但也要顾着后宫和皇上的颜面。康雍盛世近乎百年,国库丰盈,百姓安居乐业。年节下命妇大臣们朝见的时候,不能看着他们心目中住在紫禁城里的高高在上的妃嫔主子们穿得还不如他们。臣民对咱们可以敬畏,可以崇拜,却不能有一丝轻慢之心。就譬如庙里的菩萨,没了金身,没了紫檀座,百姓们还能虔诚拜下去么?他们只会说,寒酸,太寒酸。''

皇后满头冷汗,已经说不出话来了。太后继续道:``再者皇上膝下才这几个皇子,正是要开枝散叶为皇家绵延子嗣传承万代的时候,你让嫔妃们一个个打扮得跟刚入关的女人似的,你让皇帝愿意睁开眼看谁?女人的心思不落在打扮自己上,自然就只盯着别人去了,后宫里也不安宁起来。因小失大,皇后,你实在太不上算!''

皇帝见太后的口吻中带着不容置疑的沉稳,而皇后早已面红耳赤,少不得赔笑说:``皇额娘教训得是,皇后有皇额娘这般耳提面命,应当不会再有差错了。''

太后微笑道:``皇后聪明贤惠,自然是一点就通。可是皇后,你知道你眼下最要紧的是什么?''

皇后已经无力去想,只道:``请皇额娘指教。''

``你膝下已经有了一个公主和一个皇子。但,这是不够的。你还年轻,又是中宫,应该让后宫多些嫡出的孩子,把他们好好抚养长大。你驾驭嫔妃,怎么样都不为过,但有一点,那就是六宫平静,让皇上无后顾之忧。其余的事,放在中宫都算不得什么顶天的大事。''

皇帝道:``那么六宫的事\ldots\ldots{}''

太后沉吟着看了皇帝一眼,慢慢捻着佛珠不语。太后的眼眸明明宁和如水,皇帝却觉得那眼神犹如一束强光,彻头彻尾地照进了自己心里。他明白了太后的意思,斟酌着道:``那么六宫的事,由皇后关照着,每逢旬日,再拣要紧的请示皇额娘,如何?''

太后笑着理了理衣襟上的玉坠子流苏:``皇上的意思,自然是好的。只是慈宁宫清静惯了,皇上不肯让哀家清闲了么?''

皇后立刻明白,恭声道:``是臣妾有不足之处,还请皇额娘多多教导。''

太后笑了一声:``好吧。那就如皇帝和皇后所愿,哀家就劳动劳动这副老骨头吧。''她瞥了皇后一眼,``至于你所行的节俭之策,内务府那边还是照旧,不许奢靡。嫔妃的日常所用也是如常,至于穿着打扮,告诉她们,上用的东西照样可以用,但不许多。一季只许用一次就是了。''

皇后答应着,又听了太后几句吩咐,方才随着皇帝告退了。

福姑姑见皇后与皇帝出去,方才为太后点上一支水烟,道:``太后苦心经营,终于见效了。''

太后长叹一声:``你是觉得哀家不该争这些?''

福珈低首道:``太后思虑周全,奴婢不敢揣测。''

太后举着乌金烟管沉沉磕了几下:``哀家若是不费这点心思,慈宁宫除了点卯似的来请个安,哀家也要成了无人理会的老废物了。哀家成了老废物不要紧,哀家还有一位亲生的柔淑公主,若不靠着哀家,来日和哀家的端淑公主一样被指婚去了准噶尔这样的偏远之地,哀家却连个置喙之地也没有了。而且皇后母家的富察氏,原是满洲八大姓之一,皇后又好强,一旦成了大气候,如何还有哀家的立足之地呢?''

福珈感叹道:``素日皇后虽也常来,但奴婢看她今日这个神情,方是真正服气了。奴婢冷眼瞧着今日来请安的嫔妃,娴妃仿佛比往日得意多了,想是皇上又宠爱了。''

太后微微一笑:``上回咱们用的人用的心思,不就为了这个么?慧贵妃好驾驭,娴妃却是个有气性的。有她在那儿得皇上的欢心,皇后才没工夫盯着中宫的权柄,咱们才腾得出手去!''

福珈会心一笑:``那也因为,太后挑了个可意的人儿,才做得成太后的交代啊!''

皇后回到宫中,已生了满心的气,路上却一丝也不敢露出来。只到了寝殿中关上了大门,只剩了莲心和素心在身边,方冷下脸来道:``自先帝离世后,皇太后一直不问世事,这回的事,你们觉得是谁去太后面前嚼舌根了?''

莲心啐了一口道:``自然有那得了便宜还卖乖的!''

素心看了她一眼:``你也觉得是娴妃\ldots\ldots 只是太后一向不喜欢乌拉那拉氏,怎么肯听她的?''

皇后冷笑道:``娴妃自然嫌隙最大,但别人也不能说没有了。原以为后宫里清静些了,稍不留神对着你笑的都能龇出牙来冷不丁背后咬你一口。''

素心担心道:``那娘娘如何打算?''

``打算?''皇后微微一笑,``太后要宫里别那么俭省,要她们打扮得喜兴些漂亮些,那都无妨。她们奢华她们的,本宫是皇后,是中宫,不能和她们一样狐媚奢华,自然还是老样子。''

莲心笑道:``也是。她们越爱娇争宠,越显得娘娘沉稳大气,不事奢华,才是六宫之主的风范。''

皇后咔地折下连珠瓶中的一枝梅花:``至于皇太后要本宫旬日回话,本宫就回吧。后宫里能有多少了不得的大事?皇太后爱听闲话,本宫就慢慢说给她听。可有一句话,皇太后说的是对的。''

莲心问:``什么?''

``本宫是中宫,中宫只有一儿一女,是太少了。''皇后沉吟道,``二阿哥在咱们眼里是金尊玉贵的苗子,可落在别人眼里,怕是恨不得要折了他才好呢。所以中宫的孩子,自然是越多越稳当。''

素心虽然担心,嘴上却笑道:``中宫权柄外移,未必是好事,也未必是坏事。娘娘有太子在手,便什么都不必怕了。''

皇后淡淡一笑:``是啊,要本宫落得清闲,本宫就清闲片刻吧。再有什么事儿,也不是本宫这个六宫之主的责任了。''

\hypertarget{ux7b2cux4e8cux5341ux4e00ux7ae0-ux6c38ux749c}{%
\chapter{第二十一章
永璜}\label{ux7b2cux4e8cux5341ux4e00ux7ae0-ux6c38ux749c}}

过了新年便是元宵,因是乾隆元年的好日子,每一日都是热热闹闹地过,百戏、杂技、歌舞,没有一日是断的。连清音阁的戏曲,也是流水似的在宫苑的朱墙底下,在水墨青砖的缝隙里,在宫灯微朦的火光里,在曲院亭台的玉阑上四散开去。这才是宫里的日子,天家富贵不只是外人传闻里的锦绣堆砌,金碧辉煌,而是那种戏文曲子里天上人间流水落花缓缓流淌似的沉静。日子一点一点淌过去了,到了明日,还是那样花团锦簇,繁华是凋不尽的,也是望不到头的。

到了二月二``龙抬头''的日子,宫中的地龙收了起来,天气也一日暖似一日了。京城里的开春,未见新绿,总是先带了一点风沙的干冽气味,所以人便成了花,成了叶,宫女们换上了春夏时节浓碧浅绿的宫装,那是鹅黄翠绿的叶,新鲜刮辣的,带着汁水丰盈的气息,越发衬得满宫的嫔妃们成了娇艳的花朵,不,是花朵的蕊,一星儿一星儿柔软的身段,争着最娇的艳。

宫中的琐事虽还是皇后管着,但每逢旬日便拣些要紧的说与太后听。太后若想知道得深些,便自己等内务府总管的回话,一宗宗、一件件理起来,皇后倒是比素日清闲了不少,得了空,除了陪着皇帝,便往阿哥所多走动些。

这一日延禧宫的小厨房里做了些鱼茸荷花糕,拿鲢鱼的脊肉磨细了兑了浆细了的荷花糕,是做给婴儿的吃食。如懿又让惢心收拾了两样时新点心,一并拿去阿哥所给了三阿哥,又道:``年下纯嫔是来得最勤的,她心里除了儿子没别的牵挂。大家常来常往的,你便多送些东西去阿哥所给三阿哥。''

惢心笑道:``说也奇怪了,纯嫔娘娘的三阿哥养得又肥又壮,都三月里了还裹得严严实实的,阿哥所伺候的嬷嬷们连对皇后的二阿哥都没这么上心呢。''

如懿笑道:``三阿哥年纪最小,他们上心也是应该的。你把东西交到三阿哥的嬷嬷手上,看着她喂了三阿哥,看合不合口味。''

惢心答应着去了。才到御花园中,见假山上薜荔藤萝,杜若白芷,在几场春雨过后,藤蔓也泛出青翠的颜色,散发出草木萌发时特有的微微的清香。惢心正贪看着,冷不丁手里的朱漆祥云如意食盒被人撞了一下,她吓得差点没叫出声来,顾不上看是谁,忙护住了食盒打开一看,幸好是点心,没散没撒,倒也不妨。她这才回过神来,看了一眼,却是大阿哥永璜。她忙收敛了神色请了个安道:``大阿哥万福。''

大阿哥随口嗯了一声,抽着鼻子蹭到惢心跟前,盯着点心盒子道:``这是什么?''

惢心忙笑道:``大阿哥,这是延禧宫新做的点心,奴婢送去阿哥所给三阿哥的。对了,今儿是三月三,御膳房给各宫里都送了豌豆黄,大阿哥在阿哥所没看见么?''

大阿哥摇了摇头,一脸不高兴,两只眼睛却盯着点心盒子,目光有些贪婪:``这个是给三阿哥的,我能吃么?''他低低地嘟囔,``三弟什么好吃的都有,吃也吃不完,我却什么也没有。''

惢心有些疑心,脸上却仍笑盈盈的:``大阿哥很想吃这个么?奴婢拿给大阿哥一些吧。''

大阿哥有些胆怯地看着惢心:``这是娴娘娘给三弟的点心,你给了我,不怕娴娘娘责罚你吗?''

惢心微笑:``娴妃娘娘一直疼爱大阿哥,在潜邸时就是这样。大阿哥吃两块点心,怕什么呢。''

惢心说罢打开盒子,取了两块芙蓉糕放到大阿哥手里:``大阿哥快吃吧。''

大阿哥看了惢心一眼,方才敢拿起来,立刻狼吞虎咽吃了,才吃完,又眼睁睁盯着惢心的点心盒子。

惢心不觉生疑,微笑道:``大阿哥还想吃么?糕点吃多了容易撑着,再过半个时辰就是午膳的时候了,阿哥用完膳再吃点心吧。''

大阿哥难过又畏惧地摇摇头,搓着衣角道:``她们总不许我吃饱,才吃了半碗就收了饭菜,我总是饿。''

``她们?她们是谁?''

大阿哥向四周看了看,见没人跟过来,才肯说出来:``就是伺候我的乳母嬷嬷们啊。''

向来年幼的皇子出门,都是由七八个宫人跟着的。惢心看了看并没人跟着大阿哥,便问:``大阿哥,跟着您的人呢?''

大阿哥掰着指头道:``他们都不喜欢跟着我,由着我逛。''

惢心更觉奇怪,也不敢再问,便取出两块奶黄酥交到大阿哥手中:``大阿哥悄悄儿藏着吃吧,可不能说是奴婢给的。奴婢先走了。''

大阿哥小心翼翼地张望着:``那你也不能说我偷偷吃了点心啊,否则我也要挨骂的。''

惢心心头一沉,忙笑问:``奴才们也敢责骂阿哥?''

大阿哥垂下脸点点头,怯怯的似乎不敢多言。惢心知道不好再问,连忙点点头往阿哥所去了。

延禧宫里静悄悄的,阿箬带着宫人们轻手轻脚地换上春日里用的珠绫帘子。如懿站在窗前赏玩内务府新送来的一盆玉石珊瑚花,听得惢心回禀,不觉回头道:``那么你见到大阿哥的时候,他身边并没有奴才们跟着?''

惢心点头道:``大阿哥一个人从假山后面跑出来,身上衣衫都沾了泥灰,定是没有人跟着。''她仔细想了想,``还有,奴婢记得大阿哥的衣领上沾了些油渍,这个时候还没到午膳,阿哥公主们的早膳清淡,不见油腥。这油渍一定是隔夜的。''

如懿思忖片刻:``这么说,阿哥所的嬷嬷们并没有好好照顾大阿哥。''

惢心道:``奴婢一直听人说起,说阿哥所照顾大阿哥的奴才比照顾皇后亲生的二阿哥的人还要足足多上一倍。或许大阿哥顽劣,也未可知。''

珊瑚花冰冷的花瓣硌在手心里,腻腻的有些发滑。如懿道:``是大阿哥顽劣还是奴才们有心怠慢,要仔细查查才知道。但你说大阿哥吃了点心怕挨骂,倒真有奴才欺凌阿哥的可能。今日之事你先别往外说,免得错失。''

惢心点头:``奴婢知道。''

如懿叹口气:``大阿哥也是可怜,才八岁的孩子,额娘死得早,没人看顾着,什么也不周全。''

惢心笑道:``小主担心这个做什么?如今小主得皇上的宠爱,迟早也会有个有福气的小阿哥的。''

如懿的叹息无声地便蔓延出来:``我何尝不想有个阿哥,哪怕是公主也好。虽然皇上眼下还宠着我,但膝下总得有个依靠。只是,总没有动静。''

惢心抿着嘴儿笑道:``小主别急,只要皇上常来,指不定哪天就有了呢。''

如懿有些不好意思,便急着去拧她的嘴:``嘴这样坏,还什么都懂!''

惢心笑着躲开了:``小主小主,奴婢再不说就是了,饶了这遭吧。''

如懿抬头看了看天色:``时候不早了,你去看看小厨房的燕窝可炖好了,若是好了,就陪我把燕窝送去养心殿。''

天色阴沉沉的,看着像快要落点雨珠子下来。那样暗沉的铅云闷在头顶,仿佛那浓墨般的颜色就要滴下来了似的。

到了养心殿前,一溜儿的太监侍卫立在外头,王钦见了如懿的辇轿过来,便迎了上前:``奴才给娴妃娘娘请安,娘娘万福金安。''

如懿含笑道:``王公公快请起。''

王钦满脸堆笑道:``看这天儿快下雨了,娴妃娘娘怎么还过来?''

如懿笑道:``给皇上炖了燕窝,热热的正好呢。''

王钦道:``娴妃娘娘有心。可这个时辰\ldots\ldots 可不巧。''王钦眼睛一瞟,如懿顺着他目光看去,见莲心站在养心殿廊下,便会意道:``皇后娘娘在?''

王钦含笑道:``是。皇后娘娘给皇上送来亲手做的豌豆黄。''

如懿微笑:``皇后娘娘规矩大,陪着皇上说话的时候嫔妃们等闲不能进去。这样吧,还有劳公公通传一声,本宫放下东西请了安便走,若娘娘见怪,本宫自去领受。''

王钦躬身道:``有娘娘这句话,奴才也能安心办事了。''

王钦转身上了台阶,惢心看着他的背影,轻声道:``娘娘,王钦这个人不能不留意着。''

如懿点点头,语不传六耳:``他为皇后做事,咱们有数就成。你和李玉结识得早,得常来往。''

不过片刻,王钦便下来道:``娴妃娘娘,皇上说还有话与皇后娘娘商量,让您把东西交给奴才就成。另外,皇上请小主预备着,夜来接驾。''

如懿看着惢心将燕窝交到王钦手中,含了矜持的笑意:``那就有劳公公了。''

如懿扶了惢心的手慢慢往回走,才到了长街,便见贵妃坐着一乘辇轿从夹道过来,按着规矩,如懿忙侧身站在一旁迎候。只听得太监们的靴声橐橐,踏在石板上吱吱轻响。抬着辇轿的太监步伐齐整,如出一人,转眼便到了跟前。如懿欠身福了一福:``贵妃娘娘万福金安。''

虽然是三月初的天气了,慧贵妃还是穿着二色金花开遍地的锦镶一斗珠的锦袄,那衣裳是用未出生的胎羊皮制成的,因卷毛如一粒粒珠子,故名``一斗珠'',穿在身上十分轻暖柔和。贵妃见了她只是点点头,道:``几日不见你,气色越发好了。''

如懿便道:``贵妃娘娘的气色也比前些日子红润多了。''

慧贵妃抚了抚自己的脸颊,倦倦一笑:``本宫还不是老样子,身上乏。倒劳烦你多伺候皇上了。''

如懿听得这话里有刺,也不欲与她争锋,只是笑笑:``皇上来了也只是惦记着贵妃。''

慧贵妃懒懒一笑:``本宫有什么可惦记的?自己身子不争气罢了,也只是老毛病了。''

如懿知道她一向畏寒体弱,不由得问:``宫里的太医不比外头的,太医院院判齐鲁大人又是一等一的国手,贵妃娘娘的身子应该会很快见好的。''

慧贵妃恹恹地捧着手炉:``我素来不过是那血淤的症候。调养了一冬天,原是好了。谁知道中午贪吃了两块御膳房送来的豌豆黄,就闷闷地滞了胃口,有些克化不动似的,所以刚去御花园遛遛弯消食。''

如懿便笑道:``眼看着快下雨了,贵妃娘娘别着了风,更别沾雨点儿,免得伤身子。''

慧贵妃点点头,一行人迤逦而去。

如懿见她走远了,才道:``她也真是可怜,饶是这般得宠,身子却七灾八难的。''

阿箬撇撇嘴:``该!心术坏了,身子也好不了。''

如懿横她一眼,阿箬立刻噤声,也不敢多话,便和惢心扶着如懿回去了。

慧贵妃回到宫中仍不肯换下厚衣服,只是一味皱眉道:``还说入春了,走进殿里就寒浸浸的,一点暖和气也没有。''

茉心努了努嘴儿,几个小太监忙生了炭盆端进来,茉心倒了一杯热茶送上来,道:``小主尝尝这个,是用大麦和陈皮炒制了泡的茶,闻着倒香,也能开胃消食。是齐太医特意嘱咐给小主用的。''

慧贵妃看了一眼,没好气道:``什么低贱玩意儿做的?如今也拿这个来敷衍本宫了。''

茉心赔笑道:``大麦和陈皮虽然是容易得的东西,但只要对小主的身子有益,有什么吃不得的呢?只要小主的身子稳妥了,早早儿也能有个阿哥,那就四角齐全了。''

慧贵妃捧着茶有些出神,眼角便有些湿润:``如今我是什么也不缺,家世有了,位分有了,皇上的宠爱比从前在潜邸更多,连我父亲也跟着在前朝得重用。''

茉心不免有些得意:``可不是!听说皇上又升大人的官了呢。连宫里人都说,皇上管着整个江山,咱们大人替皇上管着其中的一半呢。''

慧贵妃作势拍了她一下,脸上笑意却更浓:``不许胡说。''她说罢又叹气,``如今唯一缺的,不过是我的肚子连着这些年都没有动静。''她说着便愁云满面,``说到恩宠,满宫里最多的就是我了。可是偏偏总也怀不上,也不知是为什么。''

茉心替慧贵妃轻轻捶着肩膀,道:``小主也别太心急了。您的血淤之症是打娘胎里落下的,这些年您费神费心,也不能好好养着,这病看着也得好好调养才能好。''

慧贵妃急道:``好好调养,好好调养,我都二十六了。再调养下去,岁数也不饶人了,哪里还能有孩子!''

茉心抿唇想了想,压低了声音神秘道:``小主,如果您急着要孩子,奴婢倒听说民间有个法子,叫招弟。''

慧贵妃好奇道:``招弟,是什么?''

``就是民间的富贵人家,有没生养的太太,便抱一个孩子过来养着。养得时日长了,自己的肚子也沾了孩子的旺气,就能有自己的孩子了。最好,还得是个男孩子,这样自己怀胎,就能一举得男。这便叫招弟了。''

慧贵妃悻悻道:``这儿是后宫,别说是这儿,哪怕是潜邸的时候,哪能抱个孩子来养呢?真是越说越不着边际了。''

茉心看了看四下无人,便低声道:``不是不着边际,这边际就在这宫里。小主细想想,皇后生的大公主和哲妃生的二公主都是没福气的孩子,一生下没多久便夭折了。二阿哥、三公主是皇后当珍珠似的养着的,三阿哥更是纯嫔的心肝宝贝儿。可是还有一个孩子,额娘不在了,孤苦伶仃的,正好给小主用来招弟呀!''

慧贵妃目光一亮,喜道:``你是说大阿哥?那倒真是合适。只不过那孩子愣头愣脑的,不像是个机灵的样子。''

茉心笑道:``不机灵最好,横竖咱们只是沾沾他的旺气,领他过来养些日子,等小主有了自己的孩子,再说照顾不过来,把他打发回阿哥所就是了。''

慧贵妃虽然高兴,仍是沉吟:``只是不知道皇上肯不肯\ldots\ldots{}''

``无论肯不肯,家法本来就有将生母卑微的阿哥和公主交给高位的嫔妃抚养的先例。康熙爷良妃出身辛者库,她的八阿哥不就是交给位分高的惠妃抚养的么?再说大阿哥生母没了,更是顺理成章了。''她忽然压低了声音,嫌恶道,``小主还不知道呢?今儿奴婢打御花园过,看见娴妃身边的惢心和大阿哥有说有笑的,小主可得赶紧求求皇上,保不定娴妃也打这样的主意呢。若被娴妃占了先机,她可不得意了?''

慧贵妃警觉,冷笑一声,拨着手腕上的翡翠串道:``我说她今儿怎么关心起我的身子来了,原来就没安着好心。等我先求了皇上,哪怕不为招弟不招弟的话,也不能遂了她的心!''

\hypertarget{ux7b2cux4e8cux5341ux4e8cux7ae0-ux5c01ux8bf0}{%
\chapter{第二十二章
封诰}\label{ux7b2cux4e8cux5341ux4e8cux7ae0-ux5c01ux8bf0}}

傍晚的时候下了一场小雨,到了晚上倒放了晴,半弯朦朦胧胧的毛月亮挂在天际,晕黄得像被眼泪泡过似的,笼了一层湿湿的雾气。如懿忍着困意,拿银簪子拨亮了快要熄下去的烛火,看着淡淡月华透过霞影窗纱漏进来,模模糊糊地洒在地上,像落了一摊清水似的晃悠悠的影子。院中几株桃树吐了一点一点粉红色的花苞,娇怯怯的,不愿冒出头来,却带着整个宫里都沾染了春意将临的喜悦。

阿箬打着呵欠,脸上却带着笑意:``小主再等等,或许今儿折子多,皇上来得晚些。''

如懿点了点头,吩咐道:``打点冷水来,我敷敷脸醒醒神。''

正说着话,却见王钦摆着身子过来了,笑眯眯打了个千儿道:``叫娴妃娘娘久等了。皇上刚从养心殿出来,本来是要过来延禧宫的,奈何慧贵妃身上不爽快,皇上就转道儿去了咸福宫了。这不,让奴才来回禀一声。''

阿箬当下便有些不痛快:``王公公辛苦了,只是要说早该来说一声,怎么闹得这么晚?''

王钦像个笑弥陀似的,一点儿也不恼:``这不皇上宿在了咸福宫,奴才还得去敬事房说一声记档嘛,一来二去的,奴才只有这两条腿,就耽搁了。''

如懿笑意淡淡的:``皇上歇下了就好,只是有劳贵妃侍驾了。夜深了,公公出去慢走。三宝,替王公公掌灯。''

王钦摆摆手:``不敢劳动了,奴才自己走吧。''

阿箬见他出去了,急道:``皇上就这么被慧贵妃拉走了,那可怎么办呢?''

``怎么办?''如懿望着``六合春常在''的雕花长窗,那朱红色的细密格子,一格一格的,把人的心也镂成了细碎的漏子。她微微咬了咬牙:``我什么办法也没有。''

阿箬急得脸都沁红了:``宫里的女人眼瞅着是越来越多了,今儿午后还听说,皇上又晋了玫答应为常在了。您瞧,没皮没脸的南府歌伎都能晋封\ldots\ldots{}''

``住口!''如懿冷不丁一声,阿箬一抬头看见她鼻翼微动,知道是生了气了,忙吓得不敢抱怨,只委屈道:``奴婢是替小主抱屈。小主是什么身份?凭贵妃那妖妖调调、弱不禁风的样子也争着伺候到皇上跟前去,抢了小主的好时候!''

如懿心下烦闷,冷然道:``叫你住口了还有这许多话,玫常在身份再低微,那也是个正经的小主,还有贵妃,她是什么身份,由得你议论来议论去么?出了这延禧宫,要让半个人听到你这样的话,立刻就被拖去慎刑司打死了。''

阿箬又气又委屈,只得垂下了脸,默默垂泪。如懿沉吟半晌,见她还在落泪,也难免有点不忍心,便放缓了语气道:``你是我的陪嫁丫环,事事担心我我怎会不知道?''

阿箬闻声,低低答了句``是''。

如懿柔声道:``你心里不乐意的,正是我心里也不乐意的。可是人这心里的不乐意,放在自己心里还行,一旦说出来,那就成了别人的笑话了。更何况还要嘴上不饶人,把皇上心疼的人也绕进去,那不是自己给自己找麻烦么?''

阿箬眼圈红得像两枚樱桃,抬起头来:``奴婢知道自己性子急,嘴也快。可要不是奴婢是一直跟着小主打小伺候的,有些话也不敢说。这延禧宫里敢说的,也就只有奴婢了。''

如懿本就烦心,见她又自忖着自小伺候自己的情分,更加烦闷,只得忍着道:``好了。你的心意我都知道,先出去擦把脸吧,这儿由惢心伺候着就是了。''

阿箬福了一福出去,走到殿外,正见一轮毛月亮晕乎乎的,更觉得自己一片忠心对着如懿,却总是受斥责,当真是委屈到了家。她忍一忍泪,甩着绢子就下了台阶。一旁候着的太监小福子是跟她一块儿从潜邸伺候过来的,叫了声``阿箬姐姐'',便笑鼻子笑脸凑过来:``小主安置了么?要不要我叫茶水备上,再送点点心进去?''

阿箬没好气道:``要你瞎操心什么,你操心了人家还未必当你是这份心意呢!''

小福子一怔,立刻会意:``小主心情不好,又责骂姐姐了?''

阿箬一听便气道:``什么叫又责骂了?有什么好责骂的!也不看看我是谁,我是打小伺候小主,一路从娘家府第进了潜邸,又伺候进宫里的。小主有什么也不过嘴上一说罢了。''

小福子忙赔笑道:``是是是。可不是说么?咱们这群伺候的奴才里,凭谁也比不上您跟小主亲啊!小主啊也是心烦,嘴上说过了,回头照样疼姐姐的。何况姐姐的阿玛桂铎大人都外放出去做官了,以后前程好着呢,小主更疼姐姐了。''

阿箬这才有些高兴,挺了挺腰板道:``好了。里头有惢心伺候着,我就先去歇歇,你勤谨着点儿,留意着小主要什么。''

小福子点头哈腰答应了,往里头瞅了一眼,悄声道:``怎么又是惢心伺候着?咱们伺候小主的这些人里,就她跟着小主最多,巴儿狗似的。其实论贴心、论懂小主的心思,谁能比得上姐姐您哪!''

阿箬撇撇嘴,不屑道:``谁知道呢?平时闷嘴葫芦似的,现在一个人在小主跟前,还不知道说什么呢。算了,反正咱们也不怕她。一个伺候了小主几年的,能和咱们这些伺候了这么多年的比么?''

小福子连连点头:``那是那是,姐姐的心思,那是谁都比不上的。''他打过灯笼,替阿箬照着路,``姐姐小心点儿,我替您看着路。当心,当心脚下。''

如懿托着腮沉思良久,惢心端了碗八宝甜酪送到跟前,小心翼翼道:``小主老想着事情费神,喝点甜汤润润喉咙吧。''

如懿摆了摆手,惢心看着如懿的脸色,轻声道:``其实阿箬姑娘说得也没错,她就是心太直了,什么都放在了嘴上。她替小主担的心是不错的。''

如懿烦恼地拧着绢子道:``她说得是不错。可是皇上多半的时间在前朝,回了后宫也是在各宫里都走一走,是难免好几天不来延禧宫了。''

惢心凝神想了想:``是啊。宫里女人多了,皇上要一一顾及,其实就是一一冷落了。奴婢的意思\ldots\ldots{}''她悄悄看了如懿一眼,``娘娘是该想个法子,拢住皇上的心才是。''

``拢住皇上的心?''如懿眉心的愁意如同遮住月光的乌云,渐渐浓翳,``皇后是中宫,又有公主和皇子,慧贵妃有身份,纯嫔有三阿哥,再不济嘉贵人也有朝鲜宗女的身份在。我除了皇上眼前的恩宠,还有什么法子呢?自从上次咸福宫的事之后,海兰后怕,其实我也怕,没个依靠,恩宠也是今日在明日走的,不稳当。''

惢心叹口气:``也是。还有太后,太后对小主一直淡淡的\ldots\ldots{}''

如懿眼神一跳,如同被点亮的火苗,熠熠生辉:``太后\ldots\ldots{}''

惢心有些摸不着头脑:``太后怎么了?''

如懿静了片刻,有个念头悄无声息地盘上了她的心头,她便问:``这个时候,皇后会在哪里?''

惢心想了想道:``这个时辰,应该刚去阿哥所看二阿哥,然后就去太后那儿请安了。''

如懿微微一笑:``晨昏定省,皇后是个好儿媳妇。我怎么能不好好追随皇后,向皇上的额娘尽足孝心呢?''

惢心愣住了道:``小主说什么呢?奴婢都不明白。''

如懿默默望着那碗八宝甜酪出神,手指在桌上慢慢比划着:``惢心,你觉得皇上最缺什么?''

惢心掰着指头道:``皇上有公主,有阿哥,有皇后,有嫔妃,也有兄弟姐妹。前朝有张廷玉大人和高斌大人辅佐着,后宫有太后和皇后掌管着。天下太平,皇上没有什么不顺心的,更没有什么缺的。''

如懿的手指定在了那里,沉思道:``不,皇上有一样缺的。''

``什么?''

如懿极力压低了声音:``宫里虽然讳莫如深,但是你应该知道的,皇上并非太后亲生。''

惢心瞪大了眼睛,立刻跑到窗口装作无意瞄了一眼,直到确定门口守着的宫人都站得远远的,方才掩了窗,低声道:``知道。皇上的生母是从前在热河行宫伺候的宫女,叫李金桂。要不是误打误撞受了先帝的宠幸有了皇上,这辈子都是个最低贱不过的宫女。听说生皇上的时候难产死了,先帝都没过问一句。''

``先帝都没过问,旁人更加要践踏了。所以皇上小时候是放在圆明园养大的,他的生母李金桂,至今都无名无分的,埋在哪里都不知道。''

惢心大惊:``小主的意思是\ldots\ldots{}''

如懿拧着绢子打着花结,慢慢道:``皇上嘴上不说,但总得有人提一句。''

惢心大惊失色,慌忙跪下道:``小主不可,这太冒险了。不要说皇上会不会同意,太后那儿就是一道坎儿。她老人家已经对您不咸不淡了,要再招出生母这回事来,太后会容不下您的!''

``如果我说生母,那李金桂自然是要追封圣母皇太后的。太后当然会容不下我,皇上更会嫌我张扬身世,立刻就将我废入冷宫。你放心,我不会冒险就是了。''如懿转首,见惢心一脸担心地看着她,便笑道,``我在这个宫里,并没有任何稳如泰山可以倚仗的东西,我自然会步步留心,绝不轻易冒险。''

三月初五原是如懿的生日。皇帝因着前夜失约,便早早知会了王钦前来通传,说是要陪如懿一同过十九岁的生日。

到了如懿生日的那一天,内务府已经忙碌起来,将延禧宫装点一新,又特意做了新式的菜肴点心让如懿一一品尝。皇帝早早叫人赏下了银丝寿面并一应的赏玩器物。

阿箬陪着如懿站在廊下看着太监们打扫院子,又换上时新花草,不觉喜不自禁道:``皇上心里到底是有小主的。小主的生辰皇上时时惦记着呢。''

如懿只想着自己那桩心事,一时也未说话,只默默出神。

到了晚间时分,天刚刚暗下来,皇帝便来了。尚未行礼,皇帝便先拦住了她,歉然道:``晞月闹了两晚的不舒服,朕陪了陪她,耽搁了你。''

如懿温婉笑道:``贵妃身体不好,皇上陪她是应该的。''

皇帝欷歔道:``她身子不好,还给自己闹心,一直跟朕说想抚养大阿哥,就她那身子骨,大阿哥八九岁正顽皮的时候,何必呢?''

如懿心里一动,一个念头转瞬滑过,不及细想,便泯去了。她与皇帝喝了两盏酒,备下的菜也是时新的爽口小菜,不过是菠菜蛋清、口蘑炖鸡、清炒马兰头、炸酥玉兰花片、浓汤菜心、烤鹿脯、瑶柱虾脍、鸳鸯炸肚、芦笋小炒肉、双百合炊鹌子,并一碗燕窝雪梨爽和荠菜肉丝煨的银丝面。

皇帝吃了两口面,赞道:``这时新荠菜的味道,真是什么都比不上。你哪儿找来的这个?御膳房都还没上呢。''

如懿扑哧笑道:``要吃口新鲜的,哪里能等御膳房?是臣妾托了娘家的人一大早去城外摘的,上午送来的时候还沾着露水呢。''

皇帝笑道:``难为你肯用这份心。''

如懿笑盈盈望着他,柔声道:``臣妾的心思不就是这些了?皇上吃得顺口,睡得香甜,左左右右都和气顺心的,那就好了。''

皇帝笑着揽过她:``你这儿朕虽然不是天天来,但心里记挂着,总觉得想着就能静下来。这些年,你的性子也细腻沉静了许多,不比刚嫁给朕那会儿,闹闹腾腾的。''

如懿笑得垂下了脸,在皇帝肩上轻轻捶了一下,方起身行了一礼道:``今日是臣妾的生日,臣妾有一心愿,不知能否借皇上金口,成全臣妾?''

皇帝笑着扶起她道:``朕与你相伴多年,你想要什么,尽管对朕说。''

如懿并不就着皇帝的手起来,只是垂首道:``不管臣妾的心愿有多不知天高地厚,但请皇上成全。''

皇帝笑盈盈道:``只要你不逼着朕立你为皇后,其余也没什么难的。告诉朕,是不是想晋一晋位分?''

如懿忙低首道:``臣妾如何敢这般不顾尊上予取予求?臣妾的心愿与自己无关,是关系皇上的。''

皇帝挑了挑眉,好奇道:``哦?你说来听听。''

有一瞬的犹豫,如懿咬一咬唇,还是让话语从唇齿间清晰流出:``先帝驾崩遗留下满宫嫔妃,皇上尽数加封,将各位太妃太嫔颐养在寿康宫等处。臣妾想的是,先帝早年去世的嫔妃,有些身份虽然低微,但请皇上顾念她们也曾侍奉先帝,虽然无名无分,也请皇上加以追封,以表孝心。''

皇帝的眉心渐渐拧成川字:``你说的人是\ldots\ldots{}''

如懿微一踌躇,还是说了出来:``是先帝在热河行宫的嫔妃李氏金桂。''

皇帝矍然失色,冷下脸道:``放肆!李氏无名无分,不过是先帝一朝临幸的宫女,如何能得追封!''

如懿俯下身体,恳求道:``李氏对社稷的功劳,皇上一清二楚。只是大清朝立功之人多如过江之鲫,不必事事褒扬。但请皇上看在先帝的面上,哪怕只将李氏追封为太贵人,葬入先帝的妃陵,也算是全她的颜面了。''

皇帝的脸上看不出任何表情,连声音也冷得没有任何温度:``擅自追封先帝的嫔妃,恐怕太后知道了会不高兴。''

``只是追封太贵人或太嫔,名位不需太高,尽的只是一份心意。也好过李氏的陵墓远在热河,荒草斜阳,孤坟寒烟,备受凄凉。''

沉默太长久,几乎能听清彼此呼吸的悠长之声。仿佛连时光也就此凝滞不动,化成一层层不见形的凝胶,逼得如懿的额头沁出一滴滴的冷汗。她伏在地上一动也不敢动,良久,自己额头一滴冷汗落下,落在厚厚的赤锦荔枝红地毯上,转瞬不见踪影。

良久,皇帝终于说了一声:``起来吧。''他淡淡地看着如懿艰难地起身,``今儿是你生辰,早些歇息。朕去后殿看看海兰。''说罢,他头也不回,便朝门外走去。

如懿只觉得身心虚弱,整个人都颓败到底了,看着皇帝离去的颀长背影,情不自禁地唤了一声:``皇上\ldots\ldots{}''

皇帝的脚在迈出门槛的一瞬骤然收住,头也不回地问道:``为什么会向朕提出这样的心愿?''

如懿凄然道:``臣妾的姑母是大逆罪人,不容于先帝,也不被允许有任何名分。所以臣妾不希望另一位亲人也如姑母一般,一辈子无声无息,连该得的东西都没有得到。''

皇帝停了一瞬,径自向外走去。走到门外的一刻,他忽然觉得眼角微凉,像有什么不能见人的东西瑟缩在眼角,不肯再流露分毫。他伸手,才发觉有一滴泪凝在自己指尖,在月色柔白之下,恍若冷露无声。

惢心见皇帝出去,慌慌张张进来道:``小主,小主,皇上怎么走了?''

阿箬也打了帘子,像丢了魂似的跑进来道:``小主,今儿是您的生辰,皇上怎么去了后殿?皇上他\ldots\ldots{}''

如懿失落地摆摆手:``别说了。这里也不用收拾,下去吧。''

阿箬见如懿只留着惢心,却打发自己离开,便有些赌气,撤下帘子便退下了。

惢心着急道:``小主,您是不是还是说了?''

如懿点点头,戚戚道:``该说的,不该说的,我都说了。''

``您这是\ldots\ldots{}''惢心不敢再说下去。

``我知道你要说我失策。可是皇上身为人子,许多事虽然不说,但总是惦记着生母,想要尽一份人子的孝心。今日拼着让皇上责罚,我也要说出这番心意,皇上若能成全,也便是成全了他自己了。''

惢心急急道:``可是今儿是您的生辰,皇上连宴席都没完就走了,显然是生了大气。您实在是不值啊!''

方才点起的成双红烛一明一灭,晃悠悠的,好像随时都会熄去。窗棂开合的间隙,有风直灌而入,带进殿外夜凉疏冷的潮湿,轻易扑熄了紫铜烛台上明炽的烛火。

黑暗如夜凉,悄无声息地弥漫开来。如懿张了张嘴想要出声,可是无尽的孤独与黑暗堵住了她的嘴,让她除了含着温热的泪,发不出任何声音。

惢心忙道:``小主候着,奴婢去点蜡烛。''

如懿任凭眼泪无声地滑落,静静道:``不必了。你出去吧,我自己静一静。''

\hypertarget{ux7b2cux4e8cux5341ux4e09ux7ae0-ux5f97ux5b50ux4e0a}{%
\chapter{第二十三章
得子(上)}\label{ux7b2cux4e8cux5341ux4e09ux7ae0-ux5f97ux5b50ux4e0a}}

这一夜的异变很快成了宫中的笑柄。金玉妍见到海兰的时候还忍不住悄声问她:``昨儿晚上皇上到你那里的时候,是不是很生气?''

海兰忙笑道:``嘉贵人一向是知道我的,我见了皇上连头也不敢抬,哪里还敢看皇上是什么脸色。''

玉妍笑得神秘:``那皇上有没有和你说话解闷儿?你也算不错了,自从住在延禧宫后,皇上去看娴妃,总能有几次顺便去看了你。''

海兰的神色谦卑而谨慎,带了上回受辱后怯怯不安的紧张:``姐姐还不知道我?笨嘴拙舌的,皇上也不大和我说话。不过是和往常一样罢了。''

玉妍似有不信,妩媚清亮的凤眼挑起欲飞:``真的和往常一样?''

海兰的神情看来诚实而可信:``真的。''

玉妍似有些气馁,挽着怡贵人的手无趣地离开了。

回来后海兰如实地向如懿说起今日的见闻,如懿只是比着唐代李昭道的《春山行旅图》低头在檀木绣架绷紧的白绢上绣着一幅一模一样的绣品。

海兰道:``外头都闹成这样了,个个巴不得看姐姐的笑话呢,姐姐怎么还沉得住气在绣这个?''

如懿淡淡笑道:``好容易让如意馆①的人找出了这幅图来,不沉住气绣出来,难道还走到外面去让人看是非么?''

海兰仔细看着画卷道:``这幅设色画悬崖峭壁,石磴曲盘。树间苍藤萦绕,行人策骑登山。盘行雄峻山间,树藤蔽人眼,总让人有一种山重水复疑无路之感。''

如懿伸手抚了抚垂落的鬓发:``画也罢了,我最喜欢的是画卷下面配的诗。''如懿轻声吟道,``苍崖悬磴迷层叠,树色阴浓远近间。云光岚影都无迹,倦顿何妨暂息肩。仰瞑渴饮聊伦逸,巨坡平掌心亦安。''

海兰双眸清明,已含了几分懂得的笑意:``巨坡平掌心亦安。难道姐姐已经有了解决之法?''

如懿绣了几针,便停下手取了丝线比了画卷上的浓绿深翠的颜色,一色一色选过去。海兰笑道:``绣这一片山峰上一棵树,就要用几十种绿色,姐姐也不怕挑花了眼?''

如懿指着院中含苞待放的桃花:``你瞧那花骨朵粉盈盈的,映着湖绿的珠绫帘子,可不像乱花渐欲迷人眼?既然如此,咱们只要平心静气,守着自己才不会迷进去了。''

海兰也不多言语,在铜盆里浣净了双手,取过一枚银针道:``既然如此,妹妹也怕外头乱花迷眼,便陪姐姐一起绣吧。''

沉溺在丝线翻飞的日子是过得沉静而迅疾的。仿佛是绣架上理不清的各色丝线,明绿、翠绿、深碧、鹅黄、朱紫、傅粉、虾青、芙红\ldots\ldots 慢慢地选了在银针的孔眼间穿过,一一绣在了雪白的绢地上,仿佛此身分明,渐渐便也安稳住了心思。

自如懿生辰之后,皇帝足有一月没有踏足延禧宫。六宫的绿头牌照例在指间翻落,咸福宫、永和宫、启祥宫、长春宫、钟粹宫、景阳宫,仿佛皇帝到了哪里,就将春意带到了哪里。唯有延禧宫,即便是庭院的桃花开了几朵,也是瘦怯怯的冷胭脂红,花色不繁,艳亦失色,开在渐渐暖起的春风艳阳里,亦是孤瘦伶仃的。

皇帝骤然冷了延禧宫,如懿和海兰的日子也渐渐不好过起来。一开始是春日里该有的衣裳料子没有送来,她们只得拣旧年的衣裳穿了。幸好皇后还体恤,做主赏了一些,才勉强帮补过去。只是她和海兰的衣裳有了,下人们的也顾全不周,难免有了怨声。渐渐地,御膳房送来的吃食也不算新鲜了。时新的菜肴是没有的,几道主菜都是煮过再煮,今天送了来没吃,明天还是这道菜,煮得油汤浓腻,菜都老了,根本不能吃。如懿不能事事回禀了皇后做主,既惹人笑话,又得罪了御膳房,少不得自己拿出银子来贴补着小厨房的膳食,可也是万事不周全。再渐渐地,连送来的月银也不齐全了。阿箬数了数目不对,便朝内务府的主事太监秦立嚷起来:``凭什么咱们的银子不对,也不许嚷嚷?''

秦立年纪不大,却在内务府当差久了,当下冷笑一声道:``延禧宫里住着两位小主,原本开销就大。年下的时候用这个用那个都是内务府自己掏了腰包贴补的银子。如今都春天了,还不把这笔银子补上么?我都算过了,按着这么个扣月银的法子,延禧宫欠下的数目该要到明年这时候才还清呢。''

阿箬气得浑身打战,指着他的鼻子骂道:``延禧宫什么时候要这要那欠内务府的银子了,欠条呢?款项呢?一一拿出来我瞧!''

秦立晃着脑袋笑道:``哪有主子欠了奴才的钱不还的?还亏了是小主娘娘呢,这么拿奴才的银子不当银子,说出去都让人笑话。''

阿箬看他大摇大摆走了,气得说不出话来。进了暖阁见如懿只顾着绣那幅《春山行旅图》,越发气不打一处来,红了眼眶道:``小主您听听,内务府的人就这么作践我们!''

如懿平静地理好丝线,道:``是委屈你们了。银子不够,将我旧年的一些衣裳送出去换些钱,再不济便是我们辛苦些,多做些绣活儿叫小福子他们送出去换钱罢了。''

阿箬想了想道:``宫中哪里不要用银子?奴婢想着,与其这样艰难,看人脸色,小主不如与母家商量\ldots\ldots{}''

话未说完,如懿脸色已经沉了下来:``宫里的难堪事自己知道就成了,还要告诉娘家人要他们担心么?何况乌拉那拉氏不比从前,他们都还指望着我,我怎么还能让他们放心不下?''

阿箬噎得一句话都说不出来,只得讪讪道:``奴婢想着,到底是至亲骨肉\ldots\ldots{}''

如懿摆手道:``就是因为至亲骨肉,我才不能拖累了他们。''

阿箬无言,只得忍了气下去。如懿拈着银针的手沾了一手的冷汗,一阵阵发涩,索性丢开了绣架去浣手。

彼时正值黄昏,庭院里斜晖脉脉,斜斜照进暖阁里,光线被重重绣帷掩映,更暗淡了几分。那夕阳的余晖是薄薄的金红色,望得久了,并没有那种暖色带来的温意,反而寒浸浸地像是落在秋凉里了。连飞在半空中的燕子,也似被夜寒打湿了翅膀,飞也飞不高。她无端地便想起幼时学过的一首词,前面都浑忘了,只有一句记得清清楚楚:夕阳无语燕归愁,东风临夜冷于秋②。

惢心倒是一声言语都没有,捧过两盏白纱笼的掐丝珐琅桌灯放在绣架旁,安静伺候了道:``小主,奴婢方才整理衣裳,找出几匹旧年的料子,花样是不时兴了,但料子却是极好的,不如先裁了给底下人做了春衫,也免得宫里先闹起来。''

如懿道:``也好。只是我另外交代你的事,你都做了么?''

惢心轻声道:``大阿哥那儿,奴婢知道那些嬷嬷靠不住,所以按小主的吩咐,隔几天就悄悄送些吃食去,避开人给了大阿哥。''

``那就好。我能顾上的也就只有这些了。''如懿拿清水浣了手,无奈道,``原是我鲁莽了,兵行险着,连累了你们。''

惢心淡淡笑道:``在这宫里,起起伏伏也是寻常的。旁人看低了咱们,是他们眼力不够罢了。''

如懿摇头,颇为感慨:``旁人也罢了,偏偏阿箬也这么沉不住气\ldots\ldots{}''

两人正说着话,三宝打了帘子进来道:``小主,奴才刚在外头长街上碰到李玉,他正要去传旨呢,倒是件新鲜事。''

如懿道:``什么?''

三宝道:``皇上不知怎么心血来潮了,说是禀明了皇太后,要替先帝留下的太妃们加以封赏。''

如懿几乎没反应过来,便问:``说仔细些,是什么?''

三宝不想如懿这般有兴致,便细细说道:``皇上前几日去太庙祭祖,回来便伤感得很,对太后说未曾好好尽孝道。太后宽慰了皇上几句,皇上便说,当以天下养太后,又增加了寿康宫太妃太嫔们的月银份例。另外,皇上也想追封先帝已故的嫔妃,一同迁入妃陵,与先帝做伴。''

如懿压在心头数十天的大石骤然间四散如沙,松了开来。她忍不住会心一笑:``先帝驾崩,到了地下自然不能没有人陪着侍奉。妃陵里陪葬的人太少,也不像样子。皇上这样的孝心,皇太后自然没有不答应的。''

三宝笑道:``小主远见,太后也是这样说的。所以先是将先帝已故的敦肃皇贵妃从葬泰陵,然后是从前殁了的几位在圆明园和热河行宫伺候的贵人、常在、答应或是侍奉过先帝的官女子,一律追封了太嫔,也迁往泰陵陪着了。''

如懿的心上泛起无声的喜悦,渐渐地迷了眼睛,成了眼底薄薄的泪花。惢心忙递上绢子,见机道:``小主绣花看累了眼睛,快歇歇吧。三宝,你也下去吧。''

三宝答应着退下了,如懿不由得喜极而泣:``皇上这么做了,他还是这么做了。''眼泪是热的,从眼底落到面颊上,那种温热的湿润,提醒着皇帝的在意与孝心。她的高兴是掺着凄楚与欣慰的。这么多年,皇帝避讳着自己的身世,心里何尝不是也如常人一般记挂着自己的生母?她心里知道,至此,哪怕是身份未明,有了追封,到底是了却了皇帝的一桩心事。这么多年他的心事,也渐渐成了她的心事。哪怕她算计着荣宠,算计着安身立命之道,此刻也是欣慰万分。

惢心笑逐颜开,忍不住带了欣慰的泪:``小主,皇上遂了您的意思。皇上他\ldots\ldots 他很快就要来了。''

然而,皇帝并没有到延禧宫中来。虽然日常朝见总也有见到的时候,皇帝也只是淡淡地和她说几句话,和对其他人并无两样。如懿虽然心焦,却也不知是何故。几次召了李玉来问,饶是聪明如李玉,也是说不上缘故来。如懿心知情急也是无用,只得勉强度日。只是依稀听闻着,皇帝又新纳了一个宫女为答应,已经封了秀答应,住在怡贵人的景阳宫里。即便如此,玫常在却依旧得宠,虽然皇帝有了新人,也半分分不去她的宠爱。这样的事,如懿听在心里,不免有些难过。她也才十九岁,年华正好的时候,旁人是``喜入秋波娇欲溜③'',自己偏是``玉枕春寒郎知否?③''只能眼睁睁看着皇帝的宠爱,谢了荼蘼春事休。平淡的日子里唯一安慰的,是海兰,常来与她做伴,从晨到晚,也不厌倦。再来,便是纯嫔了,虽然她的宠幸也淡薄,但好歹有个阿哥,明里暗里也能帮着如懿些。

再见到皇帝的时候已经是在五月里了,如懿清楚地记得,那一日下着微濛的小雨,雨色青青的,隐隐能闻得雨气中的庭院架上满院的荼蘼香。如懿叹口气,手中的《春山行旅图》绣了大半,自己还在群山掩映中迷惑,春日却是将尽了。

来传旨的是皇帝跟前的李玉,他打了千儿喜滋滋道:``传皇上的口谕,请娴妃娘娘速往皇后宫中见驾。''

如懿忙起身道:``这个时候急急传本宫去,李公公可知道是什么事么?''

李玉忙道:``奴才也不知道。只是王公公和奴才是一同出来的,他去了咸福宫,传了一样的口谕给慧贵妃娘娘。小主,您赶紧着吧,辇轿已经在外头候着了。''

如懿立刻更衣梳妆,出门的时候雨丝一扑上脸,才觉得那雨早无凉意,带着甜沁沁的花香和暑气将来的温热。

到了长春宫中,莲心已经掀了帘子在一边候着,见了如懿便笑道:``娴妃娘娘来了,贵妃娘娘也刚到呢。''

如懿见慧贵妃与皇后一左一右伴在皇帝身边,似在说笑着什么,极为融洽。这样家常热闹的场景,她与皇帝之间却是许久未见了,不觉眼中一热,低头进来一一见过。

皇帝向她招了招手,让她坐下,道:``这么急过来,没淋着雨吧?''

如懿随口答应了。慧贵妃娇俏笑道:``上次在皇上宫里看到一顶遮雨的蓑衣,臣妾可喜欢了,皇上赏了臣妾吧。''

皇帝失笑道:``那是外头得来的,说是民间避雨的器具。还是你父亲高斌找来的玩意儿,谁知他这样偏心,竟没留一件给你。''

慧贵妃撅了樱唇道:``父亲是最偏心了,眼里只有皇上,没有女儿。''她本穿了一身樱色挑银线玉簪花夹衣,外面套着薄薄的淡粉色琵琶襟撒金点小坎肩,显得格外娇艳欲滴。领口上的白玉流苏蝴蝶佩随着她一颦一笑,晃得如白雪珠子一般。

皇帝笑道:``你父亲偏心朕,朕就偏心你了。你既喜欢,便拿去吧,只一样,不许戴了各处逛去。''

慧贵妃含笑谢了,瞥了如懿一眼,得意洋洋地取了一粒香药李子吃了。

皇帝正色道:``今儿这么急着叫你们到皇后宫里来,是有件事与你们商量。''

众人答了``是'',皇帝又道:``今儿朕查问永璜的功课,见他瘦是瘦了些,但换了身新衣裳倒也精神。谁知朕才命他写了几个字,那孩子却不太争气,只盯着朕案上的瓜果心不在焉的。''

皇后微微一凛,忙起身道:``皇上切勿怪罪。永璜年纪还小,读书写字的时候分心也是有的,臣妾一定会让师傅好好管教约束,这样的事定不会再有了。''

皇帝慢慢啜了口茶道:``朕原也这么想着,孩子年幼贪玩总是有的。可是朕看他写字的时候翻出袖口来,手臂上竟带了伤。再三问了,才知道是今天永璜在御花园玩耍的时候在假山上磕的。''他的脸色沉了一沉,旋即平静道,``可是伺候永璜的几十个人,竟没有一个是知道的。''

慧贵妃``哎哟''一声,便道:``那奴才们也太不小心了,既替永璜换衣裳,怎会看不见伤痕?要么是太粗心,要么那衣裳根本就不是他们替永璜换的。''

贵妃说完,皇后便默默横了她一眼,偏偏贵妃尚未察觉,全落到了如懿眼里。如懿不动声色地取了片芙蓉糕慢慢吃了,只见皇帝颔首道:``贵妃这话不错。因为朕发觉,永璜外头的新衣裳是临时套上的,里头的衣裳怕是穿了三四日都没换了,油渍子都发黑了。''

皇后满面愧疚和不安:``都怪臣妾不好。都说永璜是没了额娘的孩子,臣妾格外心疼他些,还特意多拨了一些人去照顾。谁知道人多手杂,反而不好了。皇上放心,等下臣妾亲自去阿哥所好好责罚那些奴才,以儆效尤。''

皇帝冷冷道:``那些奴才朕自会发落。你也不是没用心,是底下人欺负永璜是没娘的孩子罢了。所以朕想来想去,还是得给永璜寻个能照顾他的额娘。''

皇后一怔,尚未反应过来,慧贵妃已经满面含笑:``皇上,臣妾膝下无子,长日寂寞。还请皇上成全臣妾一片盼子之心,将永璜交给臣妾抚养吧。臣妾一定会恪尽为母之责,尽心照料。''

皇帝看了眼如懿,慢慢道:``娴妃可有这样的心思?''

如懿微一寻思,便含笑道:``皇上若放心,臣妾万分欣喜。''

皇后道:``既然贵妃和娴妃都喜欢永璜,皇上的意思是\ldots\ldots{}''皇后沉静一笑,``其实臣妾好歹生养过,若皇上放心的话\ldots\ldots{}''

皇帝叹口气道:``你们都喜欢孩子,这个朕知道。可是也得孩子与你们投缘才好。朕已经让人把永璜带来了,他愿意选谁为养母,谁有这个福气得了朕的大阿哥为子,让永璜自己决定。''

说着便有人带了永璜进来。永璜已经八岁了,身量虽比同龄的孩子高些,却显得瘦伶伶的,面色也有些发黄,总像是没什么精神。如懿见他虽低着头,却有一分这个年纪的孩子所没有的对于世事的了然。

皇帝温和地招手,示意永璜走近,一指众后妃,慈爱地向他道:``永璜,这是你皇额娘、慧娘娘和娴娘娘。你告诉皇阿玛,你喜欢她们谁做你的额娘?''

永璜逐一看她们,片刻道:``皇阿玛,儿子有额娘。儿子的额娘是富察诸瑛,皇阿玛的哲妃。''

皇帝怜爱地抚抚他的头发:``好孩子,你额娘去了,但谁也替不了你的额娘,皇阿玛只想找个人好好照顾你,像你额娘一样疼你。''

永璜懂事地点点头,伸手按了按肚子,贵妃轻笑出声,伸出双手作势要抱他:``永璜,来,来慧娘娘这边!让慧娘娘抱抱你。''

如懿也微笑着,取过一块芙蓉酥道:``好孩子,先吃点东西再过去吧。''

永璜左看看右看看,忽而一笑,取过芙蓉酥扑进如懿怀中,只看着她不说话。

慧贵妃神色一黯,似是无限失落,便有些懒懒的。皇后倒是和颜悦色,展颜对如懿笑道:``恭喜娴妃了,喜得贵子。''

如懿把着永璜的手,喂他吃了芙蓉酥,又赶紧拿水防他呛着,方笑道:``皇上若放心将孩子交给臣妾抚养,就是臣妾的福气了。''

皇帝的目光温煦如春阳:``这种母子的缘分是前世修来的,永璜既选了你,以后你便是他的额娘了。''

慧贵妃犹自有些不服气:``皇上,永璜只是喜欢那块芙蓉酥才过去的。这样不算,您让永璜再选一次,臣妾也拿块糕点在手里。''

皇帝的目光柔和得如潺湲的春水:``好了。你身子不大好,受不住孩子的顽皮。何况你常要陪着朕,娴妃比你清闲许多,永璜由娴妃照料也是好的。''

如懿原本这两个月受足了委屈,听得皇帝这句话,心下一动,仿佛是明白了什么。她仰起头,对上皇帝的目光,不觉也含了温煦清湛的愉悦。

注释:

①如意馆:清朝以绘画供奉于皇室的一个服务性机构。在此处也汇集了全国各地的绘画大师、书法家、瓷器大师,进入如意馆也成为被肯定画艺的一个重要表现。

②出自宋代吴文英《浣溪沙》。全词为:门隔花深旧梦游,夕阳无语燕归愁,玉纤香动小帘钩。落絮无声春坠泪,行云有影月含羞,东风临夜冷于秋。

③出自宋代李祁的《青玉案》。全词为:绿琐窗纱明月透。正清梦,莺啼柳。碧井银瓶鸣玉甃。翔鸾妆详,粲花衫绣,分付春风手。喜入秋波娇欲溜。脉脉青山两眉秀。玉枕春寒郎知否?归来留取,御香襟袖,同饮酴醿酒。

\hypertarget{ux7b2cux4e8cux5341ux56dbux7ae0-ux5f97ux5b50ux4e0b}{%
\chapter{第二十四章
得子(下)}\label{ux7b2cux4e8cux5341ux56dbux7ae0-ux5f97ux5b50ux4e0b}}

慧贵妃陪着皇帝出了长春宫的大门,眼见了皇帝的仪仗迤逦而去,才露出沮丧的神情,悻悻道:``求了皇上这么多次,终于眼见要成事了,谁想便宜了娴妃!''

茉心忙劝道:``小主别生气。''

慧贵妃恼道:``你说皇上两个月不理她了,怎么今儿倒想到了她,还叫她来?''

茉心扶着贵妃的手慢慢走着道:``大概是位分高又没孩子的,只有小主和娴妃了,原是想让她来应应景的,没想到大阿哥那没福气的孩子\ldots\ldots{}''她说着下意识地掩住了口,四下里看了看。

慧贵妃抿了抿唇,低声道:``就是一个没福气的孩子。本宫的位分比娴妃高多了,恩宠也多多了,他偏喜欢去那冷窝儿,那就随他去!''

茉心忙赔笑道:``可不是!就是个没福气妨着额娘的孩子,克死了生母,如今就克着娴妃去吧。小主急什么?您自会生下高贵的孩子,连皇后娘娘的也比不上。''

慧贵妃无限企盼地将手搭在了自己尚且平坦的小腹上,露出几分期许的笑容,步伐放得越发慢了。

皇后看了众人散去,手上微一用力,一双玛瑙缠丝镯敲在紫檀桌上发出清脆欲裂的响声。素心忙笑着捧过一碗燕窝来递到皇后手中,轻声道:``娘娘,这燕窝平肝理气的,您喝一点儿吧。''

皇后接过燕窝伸手欲掼,素心忙拦着喊道:``娘娘仔细烫了手。''

皇后冷笑一声,由着素心接过了燕窝,也不顾燕窝的汤汁淋淋沥沥滴在了手上,便道:``去阿哥所狠狠掌那帮人的嘴!本宫交代的事没一件做得好的,惹出这样的事端来便宜了别人!''

素心忙赔笑道:``是,她们没照顾好大阿哥,娘娘气恼也是有的。只是娘娘别伤了身子。奴婢知道,那些照顾大阿哥的人不是没用心思,只是不敢太急了。谁也没想到大阿哥身子那么好,能熬过那两场风寒的。本想着\ldots\ldots{}''

皇后目光微冷,仿佛含了化不开的冰霜:``来不及了!''

素心的语气低沉而狠戾:``来得及。伺候大阿哥的人是裁了一批,但要紧的奶娘乳母是跟过去的。''

皇后的唇角化出几分薄薄的笑意,似照在冰面上的阳光:``那么素心,你该知道怎么办。''

皇后起身往寝殿走去,唯有裙幅的摆动恍若天际的云霞浮动,余下华光曳然。

永璜跟着如懿到了延禧宫,犹是有些怯怯的。如懿只留了惢心在身边,亲手取了一套干净衣裳替他换上,又打了水仔仔细细擦了脸和手,方才温声怜惜道:``永璜,你已经到了延禧宫,不必再害怕了。''

永璜用力点点头:``只要离开阿哥所,我就不怕了。''

如懿示意惢心取过架子上的白药粉,自己轻轻地替永璜擦在伤口上:``在假山上擦得疼不疼?''

永璜摇摇头:``不疼。''

如懿抚着他的手臂,轻轻地吹着:``傻孩子,怎么会不疼呢?''

永璜露出一丝顽皮的笑意:``我自己撞的,当然不算疼。而且我不说,谁知道我擦伤了呢?''他低下头有些伤感,``嬷嬷们和乳母都不管我。''

如懿柔声道:``就是因为她们不管你,你才要管自己。娴娘娘也是没有办法,才让惢心姑姑给你想了这么个主意。''

永璜乖巧地点点头:``您讲的我都知道。要不是您让惢心姑姑总给我送吃食,她们给我吃得太少了,我每天都饿得胃疼。您是要救我,我心里都明白。''

如懿搂住他,也不觉带了几分伤感的泪意:``好孩子,就因为你明白,我才更心疼你。别的孩子在你这个岁数天天无忧无虑的,偏你要懂得这些,我实在是不忍心。''

永璜伸出小手替她擦了擦欲落的泪,小声地说:``娴娘娘,您别哭,别哭。''

这样温软的小手,碰在脸上有柔软的触感,好像是能抚平一切忧伤的良药。如懿欢喜道:``永璜,有你在,我便高兴多了。''

永璜笑着露出并不整齐的牙齿:``我来这儿,您高兴,我也高兴,所以我是不会选慧娘娘的。''

如懿柔婉笑道:``你若叫不惯我额娘,也可以叫我娴娘娘,反正都一样。你的亲额娘是哲妃,但我会像待亲生孩子一样待你好。''

永璜睁大了乌圆的眼珠看着她,轻轻点了点头:``娴娘娘,我选您是因为您待我好。那么您为什么要选我?''

如懿静静地看着他,这个孤苦伶仃失去母亲庇护的孩子,他的天真顽皮之下有着与年龄不符的思量和远虑。如懿亦不瞒他:``因为我孤零零的没有孩子,永璜孤零零的没有额娘。我们都是孤零零的,所以要彼此靠在一起。就好像冬天的时候,两个不暖和的人靠在一起,就暖和了。''

永璜若有所思地点点头:``我知道,我想暖暖和和的,您也是。所以今天皇阿玛让我选,我便选了您。''他低声道,``从前额娘还在的时候,慧娘娘从来不理我。今天哪怕她要我去,她说喜欢我,我也不喜欢她。''

如懿含笑道:``真是好孩子,我说的你都明白。那么以后便不用怕了,安安心心待在我这儿就是。''

两人正说着话,却听阿箬在外道:``小主,海常在过来了。''

如懿忙让了海兰进来,海兰一进来便笑意盈然,道:``听说姐姐新得了个儿子,我赶紧过来看看,恭喜姐姐了。''

如懿笑道:``是大喜。谁也不承想皇上突然召了我去,原是有这样的福气等着我。''

海兰让叶心抱过两匹青缎道:``我那儿也没什么太好的东西,寻了两匹缎子出来,给大阿哥做件衣裳。''

如懿眨一眨眼,永璜便明白了:``多谢海娘娘。''

海兰笑着道:``真是个懂事的孩子。难怪大家都喜欢你。''

如懿笑吟吟道:``这么喜欢孩子,就该自己赶紧生一个了。''

海兰唇边的笑容骤然凝住了,像是一朵骤然遇到了严霜的花朵。片刻,她黯然道:``我若有了孩子,也不能自己抚养。连纯嫔这样高的位分都逃不脱这些苦楚,我还能怎么样?与其到时母子生离,还不如一个人清静些。''她勉强一笑,``何况皇上如今这个样子,我哪里能指望自己有身孕呢。''

如懿被她无声的感伤蕴染,勉强笑着搂过永璜道:``幸好如今有永璜在,日子也好过些。''

海兰稍稍欣慰:``也是。有个阿哥在身边,论谁也不敢随意欺负你了。''

正说着,外头忽然热闹起来。如懿隔着霞影纱往外一看,却是内务府的主事太监秦立带着一位乳母并十几个太监捧着抱着一堆东西来了。

阿箬在外冷嘲热讽道:``哎哟!哪阵风把秦公公招来了,这么多人和东西,是做什么呀?''

秦立满脸堆笑,恨不得眼缝里也挤出笑意来:``皇上说了,娴妃娘娘有了大阿哥,宫里得多添置些东西。这不,内务府赶紧给挑了上好的东西来了呢。''他说罢便探头,``娴妃娘娘和大阿哥呢,我去请个安。''

阿箬伸手一拦,不客气道:``可不敢让你进,你可是咱们延禧宫的债主,欠着你千儿八百两银子呢。咱们得找个神位把您供起来才好。''

秦立有些难堪,讪讪地赔笑:``阿箬姑娘,那天是我喝醉了说胡话呢,姐姐您别往心里去!''

阿箬叉着腰嚷嚷道:``姐姐,谁是你姐姐?我是你姑奶奶,由着你克扣延禧宫到今天!你去回皇上的话,这些东西咱们不敢收,全当是还给你秦公公的债务!我还要去内务府找总管大人问一问,有没有欠条写着的,我要拿去请皇上瞧瞧。''

秦立吓得脸都白了,连连作揖打躬地告饶:``姑奶奶,好姑奶奶,您饶了我吧。我那是犯浑胡说,您看,这两个月内务府欠了延禧宫的东西,奴才我足足加了倍儿才敢来的。还请姑奶奶笑纳了。''

惢心听着阿箬为难他们,正想出去劝,如懿摆摆手,轻声道:``内务府的人狗眼看人低,由着阿箬闹一闹也好。咱们听着别过分就是。''

海兰笑道:``可不是,这两个月咱们真是委屈够了。''

秦立讨饶了许久,阿箬才消停了些,由着他一一说了拿来的东西,殷勤地在一旁奉承。

秦立道:``原先伺候大阿哥的人都被皇上打发了,这是大阿哥从小的乳母苏嬷嬷,所以留了下来在延禧宫跟着照顾大阿哥。''

旁人听得这一声还好,大阿哥不自觉地打了个激灵,往如懿怀里缩了缩。

如懿即刻明白:``她是你的乳母,却待你不好,是不是?''

永璜低头片刻,眼里噙着泪花道:``我想不明白,别的奴才也罢了,苏嬷嬷跟着我那么久,为什么也这么待我了?饿着我,冻着我。''

如懿低低道:``人心会为了利益变,只有亲情才是不变的。''她拉过永璜的手,``走,我也去看看,你的乳母是个什么人物?''

如懿牵了永璜从暖阁走到正殿坐下,只见一个三十多岁的妇人从人群后走出来,见了永璜便喜笑颜开,伸手扑过来:``我的好阿哥,原来你先来了,叫嬷嬷我好找呢!''

惢心蹙眉道:``你是什么人,当这儿什么地方,见了娴妃娘娘居然这般不尊重。''

那乳母吓了一跳,打量了如懿两眼,忙赔笑道:``娴妃娘娘万福,奴婢是永璜的乳母苏嬷嬷。''

如懿当下皱眉道:``永璜这个名字也是你叫得的吗?没上没下的!''

那乳母怔了一怔,不情不愿改口道:``是,是大阿哥。''

如懿听她改口改得快,便也罢了,淡淡道:``你照顾大阿哥多年,以后还是辛苦你了。''

苏嬷嬷满口笑道:``大阿哥自幼是奴婢奶大的,什么都听奴婢的。日后娴妃娘娘若要管教大阿哥,一切都跟奴婢说就是了。''

如懿知苏嬷嬷是永璜的乳母,自幼带着他的,如今看她这般倨傲,倚老卖老,也不觉含了怒气:``你若能管教大阿哥,就不会连大阿哥衣食不周受了伤都不知道。你仔细告诉本宫,去年冬天大阿哥两次着了风寒,是为什么?又为什么绵延两月都未痊愈?若不是你们这帮奴才懈怠,大阿哥会这般可怜!''

苏嬷嬷倚仗着自己的身份,便倔强道:``大阿哥着了风寒自是他自己贪玩不爱多穿衣裳,又不肯好好吃药。奴婢虽然贴身照顾,但哪里能时时刻刻都照顾到?''

永璜倚在如懿身边,神色凄苦而畏惧,轻轻摇了摇头:``母亲,不是这样的。''

如懿突然一怔:``永璜,你叫我什么?''

永璜的声音虽轻,却极坚定,他重复了一声,望着如懿的眼睛唤道:``母亲。''

如懿心底一软,像是婴儿的手轻软拂过心上,那样暖着心口。她攥紧了永璜的手,为了这一声``母亲'',从未有人唤过她``母亲'',做任何事情,她都能豁得出去。

苏嬷嬷嚷起来:``大阿哥,您虽然是主子,可说话不能这么没良心,您可是喝着奴婢的血吃着奴婢的肉长大的,您可不能睁眼说瞎话!您\ldots\ldots{}''

如懿心思一沉,将手里的茶盏重重一搁,碧绿的茶汤立刻泼了出来,如懿厉声道:``三宝,小福子!把这个藐视主上的刁奴拖出去,立刻给本宫杖打三十,打完赶出宫去!不许她再伺候大阿哥!''

三宝立刻答应了一声,伸手和小福子拖她出去。

如懿又道:``行刑的时候让所有宫人都到院子里给本宫看着,看看背叛主上欺凌主上是什么下场!''

那苏嬷嬷刚被拖出去的时候口中犹自乱嚷,杖板落了几下下去,便只剩下呜呜的讨饶声。如懿拉着永璜的手站在廊下,看着血红的杖板一杖一杖用力落下去,在碰到皮肉筋骨的时候发出沉闷的碰撞声,沉声道:``永璜,别怕!你就看着,看着那些欺负你的人怎么败在你的手下,受他们应受的责罚!''

打到二十杖的时候,苏嬷嬷渐渐没了声气,只剩下低低的呜咽声。血渍染红了她的衣裳,每一杖下去,都溅起鲜红的血点子。永璜看得有些怕,晃了晃如懿的手道:``母亲,还要打么?''

如懿的声音平稳得没有一丝波澜,紧紧拥着永璜道:``永璜,你记着,一个人做了什么因,就要承担什么果。他们欺负你的时候,就该知道这个。所以现在哪怕她受不住被打死了,那也是她自己的恶果。明白了么?''

永璜点点头,乌黑的眸闪过一丝沉稳与坚毅,默默站在如懿身边,一直到行刑完毕。如懿见他们拖了苏嬷嬷出去,地上只留下一摊暗红的血迹,拖出了老远,方才朗声道:``你们都记好了,大阿哥从此之后就是本宫的养子,也是本宫唯一的儿子。谁要敢轻慢了他,就是轻慢了本宫,苏嬷嬷就是个例子!''

众人响亮地答应了一声。秦立守在一旁,一脸畏惧害怕,终于撑不住扑通跪下,求道:``娴妃娘娘饶命,娴妃娘娘饶命!''

如懿冷笑一声:``你的狗命本宫还不想要!要怎么做,你自己看着办!''

秦立吓得一身冷汗伏在地上爬不起来,海兰带了一缕赞许的笑意,低声在她耳边道:``我最喜欢看姐姐这个样子,看着姐姐,我便什么都不怕。''

当晚宫人们便收拾了东配殿出来给大阿哥住下。如懿亲去看了,三间阔朗的屋子明光敞亮,朝向亦好。因着是男孩子住,收拾得格外疏朗。一间卧房,一间书房,一间休息玩耍的地方。每日的膳食若不在读书的书房里用,便是跟着如懿。伺候大阿哥的人全是新挑上来的,如懿一一盘查了底细干净,才许照顾着。如此忙了大半日,无一不妥当。延禧宫上下也因为新得了一个阿哥,皇帝又赏赐不断,知道是时来运转了,高兴得跟过节似的。

晚上如懿陪着永璜用了晚膳,皆是小厨房做的时新菜式,因永璜正在换牙,煮得格外软和些。又因永璜半饥半饱了许久,为了调养胃口,一律只喝煮得极稠的碧粳粥。永璜胃口极好,吃饱了如懿让惢心量了裁衣服的尺寸,便如一个宠溺孩子的母亲一般,亲自给永璜擦洗了,方哄了他睡下。

惢心在旁边拣选着给永璜做衣裳的料子,如懿轻轻拍着永璜,看一匹便挑剔一匹,惢心忍不住笑道:``小主,你给自己选料子都没这么上心。''

如懿怜爱地看着永璜:``原以为自己只想找个依靠。可是他一叫我母亲,我心里就软了,好像他就是我的孩子,我这心里\ldots\ldots{}''

惢心又选了一匹料子递给如懿看,低声道:``为了大阿哥,小主费了好几个月的心思。安排了奴婢私下照顾大阿哥,又将阿哥所的人怎么对待大阿哥的事通过李玉的嘴说给皇上听,带着皇上看见。奴婢原以为皇上是不在乎大阿哥了,才一直不动声色\ldots\ldots{}''

如懿看着永璜熟睡的容颜,低低道:``虽然哲妃不在了,但皇上到底和她有几分情分在,又是亲生的孩子。''

惢心叹口气道:``小主有了大阿哥,也有个安慰。''

如懿侧过身挑了几匹料子:``天快热了,给大阿哥多做几身夏天衣裳换着,要选透气不闷热的。京城的夏天短,一闪儿秋天就到了,秋衣也要备好。还有冬衣,阿哥去年的冬衣都不能要了,弹点新棉花厚厚实实做两身。还有永璜的饮食起居,嬷嬷们是新来的,你要多警醒着点看着,别有什么差错。''

如懿正说着,忽然发觉地上落了一个颀长的影子,转过身去,正见皇帝站在帘下,含了一抹淡若山岚的笑意,深深看着她。

如懿乍然见了皇帝来,方要笑,那笑意却凝成了三分酸楚,连行动也迟缓了。她正要起身,皇帝走过来按住她:``朕刚来的,听你交代惢心的这些话,真像一个慈母。''

如懿有些不好意思:``臣妾没有做母亲的经验,所以唠叨了。让皇上笑话。''

惢心见皇帝进来,便掩上门悄悄告退了。皇帝将她的手指一根一根放到手心里:``这么些日子没来看你,朕知道你委屈了。''

如懿眼中不自禁地便有了酸楚的水汽,低低道:``原来皇上知道。臣妾明白,皇上是埋怨臣妾自作主张、自以为是了。''

皇帝清俊的面容上笼着一层薄薄的笑容,那笑本该是暖的,却带着隐然可见的忧伤,像秋冷寒露里骤然飞落的薄霜:``原以为你那天的话是戳了朕的心了,朕也不想理会。可不知怎么的,想到后来,不知不觉还是这么做了。只有这么做,给李氏一点名分,一点尊荣,哪怕什么都不说破,朕夜里睡着也安稳些。''他望着如懿的眼睛,迟迟的语气如外头雨停后潮湿的水汽,``这些话朕憋了这些天才来告诉你,你是不是觉得朕太傻了?''

如懿懂得地按住他的唇:``是臣妾说了让皇上为难的事,让皇上烦心了。''

皇帝的眼里有深深的情意流转:``可是这样为难的事,只有你会对朕说。除了你,再没有别人。''

如懿颇为歉然:``那日也是臣妾莽撞了。''她心中有无限温柔的情意柔波似的荡漾,``可是臣妾想着,世间万物皇上都有了,千万别留下什么遗憾。圆满中的一点缺失,才会成了大缺失。''

皇帝的眼底有些潮湿,看得久了,里头只能望见如懿清晰的面容:``朕知道你是在替朕补上缺憾。朕一直明白,却不敢来见你。一是如故人所言,大概是近乡情怯。另一桩是因为\ldots\ldots{}''

皇帝尚未说完,如懿盈然一笑,仿佛一朵洁白的栀子疏疏开在暖湿的风里:``因为臣妾清闲,所以可以抚养大阿哥。''

皇帝笑道:``朕的话,原来你记着。朕想着,你也不缺什么,只是子嗣上的事要随缘,朕只能先给你一个养子,暂时补上你的缺憾。''

如懿低着头,半是感慨半是期待:``臣妾也想有个自己的孩子。不过眼下永璜带着,也挺好的。''

皇帝搂住她的肩,看着熟睡中带着笑意的永璜:``这孩子在你这里睡得挺香。''

如懿伸手替永璜掖好被子,痴痴地含了笑,反手握住皇帝的手:``臣妾多少次梦里想着,盼着,等有了咱们的孩子,一家子三个,就这样静静地守在一起。''

皇帝笑着吻了吻她:``会的,你放心。''

红烛烨烨,光晕摇曳在卷绡薄金帐上,照出二人成双的身影。如懿回眸一笑,生出无限情意,仿佛是寻到了一生一世的企盼,紧紧握着皇帝的手,再不愿松开。

\hypertarget{ux7b2cux4e8cux5341ux4e94ux7ae0-ux5c71ux96e8}{%
\chapter{第二十五章
山雨}\label{ux7b2cux4e8cux5341ux4e94ux7ae0-ux5c71ux96e8}}

自从永璜到来,如懿便渐渐品味出日子的不同了。有了个孩子,便有了新的寄托和依靠。从前总盼着君恩长驻,如今一心一意在永璜身上,连向来安静的海兰也愿意常常过来陪着孩子说笑。每日五更天永璜晨起去读书,如懿便一直送他到宫门外。晚膳时分,便候在滴水檐下盼着他回来。每日晚膳后的时分是母子俩最亲近的时候,有时候是海兰陪着一块儿刺绣描花样子,有时候是如懿一个人捧着书卷看书,永璜便有说不完的话,绕在她膝下,将一日的见闻事无巨细都告诉如懿。或者再背上一段太傅新教的文章,向来偏僻清冷的宫苑里,也因为稚子童音而多了许多欢声笑语。

因着永璜,皇帝来延禧宫的时候也比以往多了更多。隔上两三日,即便不在如懿处过夜,也必定是要来陪着一起用晚膳,顺便考问永璜的功课。连久未得幸的海兰,也因为一起抚养着永璜,晋位为贵人。

如懿总是想,即便永璜不是亲生的,但或许这样,便已经是太后所说的``美好如意''了吧。

如此,宫中等人更不敢轻慢了如懿,皆以为她平白无故得了个儿子,连运数也跟着转了。渐渐地,不止后宫诸人,连咸福宫也格外客气起来,饶是背地里慧贵妃对孩子眼红得不行,三番五次往宝华殿求神拜佛祈求子嗣,当面里对如懿也不再如往日般随心所欲了。

这一日永璜下了学便有些闷闷的,不似往日般活泼,如懿当着许多人也不便问他,待到用完了晚膳,便携了永璜往御花园去。

时至盛夏,御花园中凤尾森森,桐荫委地,阔大疏朗的梧桐与幽篁修竹蕴出清凉生静的宁谧。彼时夕阳西下,夜幕低垂,北地春归迟,可是曾经嫣紫粉白繁密欲垂的桐花亦大多开败,凋落在芳草萋萋之上,萎谢了残红作尘。那样红千紫百的繁华也不过是春日里的梦一场,最后何尝不是满地萧条?如懿看着天际升起了一颗一颗明亮的星子,仿佛伸手可得,又那样远,远不可及。能握在手心里的,唯有永璜小小的一双手。

她携了永璜在御苑中,看着清凌凌碧水里鲜翠欲滴的新荷底下悠游往来的绯色金鱼,清波如碧,红鱼悠游。如懿叫永璜折了杨柳在手,将捻得细碎的柳叶抛向池中,引得红鱼争相跃起,相嬉而食。

永璜到底年幼,玩了一阵便高兴起来了,如懿示意跟着的人退下,笑着看他:``永璜,心里舒坦些了么?''

永璜拨弄着柳枝在水里蘸着嬉戏:``母亲,儿子舒坦些了。''

如懿倚着池边的白石栏杆坐下,看着他的眼睛道:``既然舒坦些了,心里的话也可以告诉母亲了。今儿为什么不高兴?''

永璜的目光微微一缩,便看着自己的鞋尖蹭来蹭去:``母亲\ldots\ldots{}''

他欲言又止,似乎在迟疑,如懿温柔地道:``回来的时候新做锦袍上哪里都是干干净净的,只有膝盖的地方落了尘土的痕迹。难道是太傅罚你跪了么?''

永璜难过地点点头,又摇摇头:``母亲,今天永琏来上尚书房了。''

如懿心里微微一惊,嘴上却笑着说:``二阿哥才六岁,那么早就开蒙了么?''

永璜道:``皇额娘也来了。皇额娘说,永琏年纪不小了,要跟着我一起读书了。所以今天尚书房还来了两位新太傅,陈太傅和柏太傅,皇额娘说两位新太傅都是大学士,要我们都要听话。''

如懿微笑:``这是好事呀。明日母亲就陪你去见过新太傅。''

永璜丢下手里的柳枝,委屈道:``可是新太傅们对儿子不好!明明永琏第一天读书,坐不住,可是新太傅们居然罚我,罚我在尚书房的外头跪了半个时辰,连教我的黄太傅都不敢拦着。陈太傅还说下次太子\ldots\ldots{}''

如懿立刻警觉:``什么太子?''

永璜茫然地摇摇头:``母亲,什么叫太子?陈太傅叫了这一声太子,被柏太傅喝止了。''

如懿心中没来由地一紧,脸上还是如常笑道:``母亲也不知道什么是太子。但是好孩子,太傅说的话大多有深意,你别逢人便去问,这话不能问的。你说,陈太傅还说了什么?''

永璜乖巧地点点头,又哭诉道:``陈太傅说下回永琏再不听话,就要把儿子关黑屋子里去败火。''他十分惧怕,``儿子知道什么是败火,去年儿子风寒的时候,苏嬷嬷没叫太医来看,反而把我一个人关在黑屋子里不给吃的。那时候我怕极了!''他紧紧抱住如懿,``母亲,我再不要败火了!''

如懿满心酸楚,却有更深的无奈如重云压着她的心头,她紧紧搂着永璜,柔声道:``好孩子,母亲与你的额娘都是嫔妃的身份,所以你的身份也不如二阿哥贵重。在尚书房读书,难免会受些委屈。''她温和的语气里有不容转圜的坚定,``可是你要记得,你是你皇阿玛的孩子,有母亲照料,不能由着他们欺负你。下回再有这样的事,你便告诉太傅,他们这样罚你,皇阿玛知道么?''

永璜睁大了眼睛道:``母亲,我可以这样说么?''

如懿鼓励似的抱抱他:``你是皇阿玛的长子,照顾幼弟是应当的,但也不能委屈了自己。不管是谁,是你的乳母也好,太傅也好,母亲都不许他们欺负了你去。''

两人正说着话,却见纯嫔忧心忡忡地赶过来,在后头唤了一声:``娴妃娘娘\ldots\ldots{}''

如懿见她神色不似往常,忙将地上的柳枝捡起递到永璜手中,嘱咐他乖乖玩耍。纯嫔匆匆请了个安,便上前挽住如懿的手欲落下泪来。如懿忙低声道:``这是怎么了?''

纯嫔泪眼蒙眬地看了正在逗鱼的永璜一眼:``听说大阿哥今天在尚书房被罚跪了?''

如懿惊异地看她一眼,将她拉远了走到梧桐树底下道:``你怎么知道?''

``在尚书房伺候的小栗子原是我宫里出去的人,本想早点打发他在尚书房伺候,以后我的永璋去尚书房读书也多个人照顾。没承想我刚在甬道上碰到他,却听他说了这么件事。''她悄悄瞥一眼永璜,``大阿哥受委屈了吧?''

如懿叹口气:``咱们都是嫔妃,比不得皇后的嫡亲孩子尊贵,也是有的。''

这句话勾起了纯嫔的伤心事,她眼圈微红,忍不住呜咽道:``大阿哥都这样,那我的永璋以后\ldots\ldots{}''

如懿忙安慰道:``皇后那么疼永璋,照顾他的人是最精细的。连永璜都羡慕呢。''

纯嫔脸上不敢露出哭意来,只得擦了泪,低首附在如懿身边道:``我正是为这事伤心呢。今儿午膳皇上是在我那儿用的,居然说起永璋不太聪明。''她急得六神无主,``我的永璋怎么会不聪明呢?''

如懿微微迟疑,还是道:``我听永璜说,永璋一岁的时候还爬得不太利索。乳母嬷嬷们不是抱着就是背着,从不让落地。如今是不是十四个月了,会走了么?''

纯嫔的眼泪不自禁地落下来:``就是因为不会走路,嬷嬷们老怕他磕着碰着,所以皇上才这么觉得,说永璋学路慢,学话也慢,看着不聪明。这孩子还这么小,若失了他皇阿玛的欢心,可叫我怎么办好?''

星子的微光从树叶的缝隙间簌簌抖落一身稀微的光晕,如懿道:``你几次三番对我说,阿哥所的嬷嬷们对孩子照顾得很精心,如今看来,这精心竟是宠坏了他了。''

纯嫔又是焦灼又是无奈:``这话我怎么敢说,若在皇上面前提一句,岂不是坏了皇后的一番苦心?她对自己的二阿哥和三公主,都没这么上心呢。''

如懿心中一动,骤然生出几分疑义,但这样的话并不能去对纯嫔说,除了加深她的忧心与焦虑,她还能怎样呢?如懿只得劝道:``皇上不过是一时生气才这么说吧,下回再见着皇上,你便说咱们是马背上得的天下,孩子不能多娇惯着,也拉着皇上多去阿哥所看看。有皇上时常过问,或许会好些。再说了,父子亲情是天性,只要多见几次,永璋又那么可爱,皇上会喜欢的。''

纯嫔点点头,她的忧愁深长如练,将自己层层缠裹:``本来想着永璋若是有福气,可以寄养到娘娘膝下,我也能常看看她。如今看来是没有指望了。''

如懿敛容:``这个念头你动也不要动。如今宫里高位而无子女的,唯有慧贵妃,你自然是不肯的。且永璜是阿哥所照顾不周才送来我这里,永璋却无这样的事。你这念头若被人知晓,不止皇后,只怕皇上也要怪你了。''

纯嫔只得噤声,如懿忙道:``赶紧擦了眼泪回去吧,别叫人闲话。''

纯嫔拿绢子按了按眼角:``妹妹如今也有了孩子,有什么话我可得多来问问你,一起拿个主意。''

如懿含笑道:``你且放心,只要不这么哭哭啼啼的,我都答应了你就是。''

纯嫔无可奈何,只得离去。如懿望着她孤独而瘦削的背影,心下亦是生怜。她不过是一个母亲,只想要自己的孩子好好的。可是在这深宫里,偏偏连这也不可得。而自己呢?如果有一天有了自己的孩子,是不是也会如此凄然,欲哭无泪?

眼看着天色也晚了下来,如懿招手唤过永璜,一起慢慢走回宫去。一路上偶尔有鱼儿跃出水面,溅起数点水花。莲叶田田,青萍丛生,早开的睡莲绽了两三朵,粉盈盈的。几只鹭鸶栖在深红浅绿的菖蒲青苇之畔,互相梳理着羽毛。永璜看了什么都欢喜,笑着闹着拉着如懿的手说这说那。如懿嘴里答应着,可心里的疑义难以倾之于口,却如密密的丝线勒在那里,一圈沉闷过一圈。她极力地想撇开那些念头,却好像是这一定会暗下来的天色,那墨汁似的色泽洇在了清水里,无法遮拦地倾散开来。

如懿正凝神想着,却听得假山后头有呜咽的哭声传来,那声音太轻微,叫人一个耳错,只以为是夏虫绵长的唧唧声。如懿不动声色,只作不经意一般,朗声道:``永璜,快回来,别到假山那边去捉蛐蛐儿!''

那边的哭声立刻止住了,如懿示意永璜噤声。不过片刻,却看一个宫女模样的女子从假山绕了出来,如懿撒开永璜的手,永璜立刻会意,只装着跑去捉蛐蛐儿,一下撞在那女子身上。那宫女抬头就要骂,一看如懿跟在永璜身后,忙收敛了气焰请了个双安道:``娴妃娘娘万福金安。''

如懿笑吟吟道:``本宫自然是万福金安。可是莲心,你怎么不安了呢?''惢心手中的风灯照出莲心哭过的面容,``眼睛哭得跟桃儿似的,这是怎么了?''

莲心下意识地摸了摸脸,绷出一个笑容,朗声道:``奴婢伺候皇后娘娘,有什么不安的呢?不过是想家了,偶尔哭一哭罢了。''

如懿情知她不肯说实话,也不愿和她费唇舌,便道:``你伺候皇后娘娘,更当万事小心,别落了一脸泪痕回去。''她微微一笑,``只是话说回来,皇后娘娘那么疼你和素心,自然见了你的眼泪也不会不高兴。''

莲心本仰着脸毫无惧色,听了这一句,不知怎的便低下了脸,带了薄薄阴翳似的黯然,嘴上却犟着说:``皇后娘娘自然是疼我们的。比不得那些刻薄人,连从小跟着的乳母都赶出宫去了。''

这话是指着如懿说的,阿箬立时忍不住了,道:``你说谁?''

莲心盈盈一笑:``我自有我要说的人,阿箬你又急什么?横竖说的不是你,你何必跟着吃这个心?''

阿箬何曾被人说倒过,冷笑一声道:``我自然不吃这个心。只是想着莲心姑娘要大喜了,何必嘴上还不积些福德,免得叫人听了笑话去。横竖你要嫁的好人家,是断不会刻薄了你的。''

莲心脸上登时烧红了一片,却隐隐透着难看的铁青色,恨声道:``你\ldots\ldots{}''

阿箬笑道:``我\ldots\ldots 我自然是没皇后娘娘亲自指婚这般好福气了。先恭喜姐姐、贺喜姐姐了。''

莲心又窘又恼,一跺脚立时跑远了。

阿箬看着她的背影,冷笑连连。如懿便道:``你再这样冷笑,夜枭的笑声都比不上你了,听着怪瘆人的。''

阿箬笑得弯腰:``小主,奴婢是笑莲心呢。您可知道么,今儿上午奴婢去内务府的皮库,想叫他们将今年秋天贡来的好皮子留着些给大阿哥做衣裳,谁知看见内务府的人忙忙碌碌地在旁边的皮库选大毛料子呢。奴婢好奇问了一句,原说夏天找什么大毛料子,谁知他们说是皇后娘娘给莲心备嫁妆呢。''

如懿道:``莲心已经二十四了,本该放出宫去的,偏她是皇后娘娘的家生丫环,也没地方回去。既然要在宫里伺候一辈子,还不如嫁人呢。皇后肯指婚,也是给她面子了。''

阿箬笑着啐了一口,手里的灯笼也跟着晃悠悠地打转:``小主还不知道皇后娘娘给她指了谁吧?''

如懿看了惢心一眼,惢心忙哄着永璜去了。如懿问道:``从前是听说她跟皇上跟前的王钦走得近,皇后也有这么一说,可是这到底是句笑话儿,王钦是个公公,不是个男人,怎么能配了他呢?''

阿箬得意得眉毛都飞起来了,道:``小主别说,还真就是王钦了。内务府的嫁妆都备起来了,说皇上也知道了,就等过了中秋就指婚呢。皇后宫里说了,莲心陪了她那么多年,要跟嫁半个女儿似的呢。''

如懿怔了半天,半晌才回过神道:``好好一个女孩子,真是可惜了。''

阿箬眉飞色舞:``有什么可惜的!满宫里的太监,就数王钦地位最高,多少人想巴结还巴结不上呢。莲心配了他,还便宜了莲心呢!''

如懿不悦地看她一眼:``好了,别说这样的话!宫女配了对食本就可怜了,莲心再不好,你也别当面取笑她了。''

阿箬不情不愿地应了一声,红了半边脸,吭哧吭哧凑到如懿跟前道:``小主,以后你也会给奴婢指个好人家么?''

如懿笑着伸手去刮她的脸:``你放心,去年你阿玛放了外官,我一直听说挺好的。到时候怎么也要给你风风光光地指一个好人家。''

阿箬又是害羞又是高兴:``奴婢能挑什么好人家,全凭小主的恩典罢了。''

如懿道:``外边的人怎么样咱们也不清楚,能挑个御前的侍卫,凭自己挣个好前程就是了。''

阿箬喜不自禁,在如懿身边黏了良久。正好惢心带了永璜过来,阿箬招手笑道:``小主今儿高兴,快求求她,也给你放个好人家指婚,也好抬高了你的门第,省得让人知道你那两百钱的出身!''

如懿嗔怪地拍了阿箬一下,作势要打她的嘴,阿箬笑着躲开了:``奴婢和惢心这么熟,笑话罢了。''

惢心沉静道:``奴婢不比阿箬姐姐好出身,只想一辈子守着小主,哪儿也不去。''

阿箬挑了挑眼角,似有不满,嘟囔一句道:``这么大的恩典在眼前,别假惺惺的!''

惢心替永璜掸干净衣裳,淡淡笑道:``没什么可假惺惺的。阿箬姐姐要嫁个好人家,小主不能没个人伺候,奴婢被卖的时候就忘了家乡在哪里了,正好留下来伺候小主一辈子。''

如懿抚了抚鬓边微凉的鎏金流苏,笑着道:``你有这个心自然是好的,但女孩子不能不嫁人。哪怕是嫁得近些,嫁个侍卫或是太医,也是好的。''

惢心满面赤红,咬了咬唇,只是不说话。

如懿扶着她们的手正要起身离开,忽然看见前头灯火通明,几十盏灯笼晃点着如暗红浅黄的星子,朦胧地亮成一片。

如懿扬了扬脸,惢心立刻跑到前面去,片刻回来道:``小主,是永和宫出事了,皇上正赶过去呢。''

\hypertarget{ux7b2cux4e8cux5341ux516dux7ae0-ux963fux7bac}{%
\chapter{第二十六章
阿箬}\label{ux7b2cux4e8cux5341ux516dux7ae0-ux963fux7bac}}

这一夜永和宫中并不安宁,闹了整整一夜也不知是怎么回事,只见太医去了一拨又一拨,却不见放出来。六宫众人都惊异不已,私下里查问却也问不出什么,只知道永和宫的灯火亮了一夜,却大门紧闭,没有一点声息。

晨起时也不知永和宫中到底出了何事,如懿惦记着要去长春宫请安,早早梳洗了便传了辇轿往外头去。

向例嫔妃出门都是传的辇轿,只是如今初夏早晨尚算清凉,如懿便扶了惢心和阿箬的手慢慢出去,正过了长街,看着初阳澄澈如金,流金般的日光落在琉璃瓦上,仿佛漾着一池金波浮曳。如懿贪看那日色,才走了几步,却见慧贵妃也在前头,忙恭谨立在道边迎候,见她近前,方福了一福。

慧贵妃笑盈盈打量着她道:``几日不见娴妃,气色越发好了。是不是皇上昨儿歇在你那儿,所以人逢喜事精神爽?''

阿箬满面都是甜笑,嘴上却道:``皇上来也是常有的事,这也能算喜事么?''

如懿气恼阿箬嘴这样快,尚未来得及瞪她,慧贵妃只是笑容如常,伸手抚了抚发髻上新簪的一支冷翠色碧玉明珠钗,淡淡道:``也是本宫浑忘了,昨儿皇上仿佛是歇在永和宫。本宫还以为妹妹那儿春色长驻,一日也不落下呢。''

如懿不欲与她逞口舌之快,便只安静地垂下脸,看着自己松花绢子上细细的流苏。

慧贵妃以为她气馁,眼角便多了几分桃花色,正欲再出言讽刺几句,却见斜刺里一顶辇轿横穿出来,差点撞到慧贵妃。她脚下一个踉跄,花盆底一斜,差点摔了出去。幸好彩珠和彩玥扶得快,人虽没事,发髻上的碧玉钗却滑落下来,跌得粉碎。

那顶辇轿撞了人,全作无事一般,往角门一拐便过去了,浑不理撞了什么人,撞得重不重。

彩玥``哎呀''一声,忙蹲下捡起那支碧玉钗,情急道:``这是皇上新赏的,就这么碎了\ldots\ldots{}''

话未说完,彩玥脸上已经重重挨了一掌。慧贵妃气恼道:``看清楚那人是谁没有?''

彩玥捂着脸也不敢哭,倒是茉心道:``背影像是玫常在,但看衣服却不大像呢。''

慧贵妃呵斥道:``只一支玉钗,皇上赏得还少么?小家子气!''说罢,她便丢下如懿匆匆往长春宫去了。

如懿见她离去,不觉含了几分气恼,向阿箬道:``你若再这般逞口舌之快,便不要再和我出来!''

阿箬嘟囔道:``小主怕她做什么?咱们有大阿哥,延禧宫的恩宠也不比贵妃少!''

如懿见她教而不善,气道:``即便如此,你又何苦去惹她?现在大阿哥在我身边,多少人的眼睛看着,你还不肯检点些!''

阿箬还欲再说,终究还是忍耐了下去,扶了如懿的手往长春宫去。

如懿到时嫔妃们都已在了。她跟着慧贵妃进去按着位次坐下,皇后便笑吟吟向贵妃道:``今儿你是怎么了?头发也有些松了,脸色也不大好。''

慧贵妃递一个眼色,茉心忙道:``方才从长街过来,我们小主不知被谁的辇轿横冲直撞出来碰了一下,人差点扭了,连皇上赏的玉钗也跌碎了。''

慧贵妃忙起身道:``如此匆忙来见皇后娘娘,实在是怕误了请安之时,还请皇后娘娘见谅。''

皇后温和道:``这有什么要紧的,倒是你自己没事吧?跟着的人没看清是谁撞的么?''

茉心道:``奴婢看着恍惚是玫常在。''

蕊姬倒也不惊,只是盈然一笑如芙蓉清露:``方才是冒失了,差点撞到贵妃,真是失敬了。''

慧贵妃神色不豫,冷然道:``如今才知道撞着本宫了,方才怎么逃得一阵风儿似的?''

蕊姬盈然一笑,抚着腮边道:``本是想停下来跟贵妃娘娘您致歉的。可是,嫔妾有一桩要紧事不能不先来回禀皇后娘娘,所以只好对不住贵妃娘娘了。至于跌了皇上赏赐的玉钗,您到嫔妾宫里随便挑,喜欢什么您自己拣去,赔您两根三根都不要紧。''

慧贵妃听她如此倨傲,一张秀荷似的粉面不由得含了几分怒意:``昨儿晚上永和宫就闹腾了一夜,今日又来无礼,即便皇上宠着你,也由不得你这个样子!''

蕊姬侧了侧脸,唇角的弧度如一弯新月,起身向皇后恭恭敬敬福了一福:``回皇后娘娘的话,臣妾昨夜腹痛不止,皇上传了太医来看,才知臣妾是有了身孕了,已然两个月了!''

此言一出,四座皆惊。

如懿下意识地按住自己的小腹,不觉生了几分凄楚。她立刻意识到这不是该自己伤心的时候,忙撑住了脸上的笑容,不容它散落下来,也随着众人贺喜道:``恭喜皇上,恭喜皇后,恭喜玫常在。''

皇后倒还镇定,满脸笑意像遮不住漏下的春光:``是么?只是既然有孕,怎会腹痛?''

蕊姬微有得色:``太医说臣妾体质寒凉,胎儿体热,有所冲撞,加之是头胎,所以腹痛。其实也是无妨的,臣妾也是因为这件事要急着回禀皇后娘娘,所以冲撞了贵妃也不敢停留。''她说罢便要屈膝向贵妃行礼,``还请贵妃宽恕嫔妾这遭吧。''

蕊姬虽是要屈膝,动作却极缓慢,贵妃知她的意思,只得让茉心拦住了,道:``才有了身孕便仔细些吧。万一磕了碰了,仔细丢了这福气。''

蕊姬的目光略含挑衅,看着贵妃道:``好容易得的这福气,怎么会丢了?有贵妃娘娘庇佑,嫔妾的福气长着呢。''

皇后连忙道:``你是头胎,得格外仔细着。等下本宫就多拨几个人过去伺候你。缺什么要什么,尽管来和本宫说。十月怀胎,有得辛苦呢。''她蓄了宁和的微笑,看着贵妃与如懿道,``不过这辛苦也是福气,本宫也希望你们两个早有子嗣呢。''

玫常在眼波微曳,看着慧贵妃,曼声道:``是啊,十个月是辛苦呢,嫔妾看着娴妃娘娘照顾大阿哥就费尽心力。不是亲生的尚且如此,若是亲生要当何等艰辛呢。还是慧贵妃福气好,没生养的人,看着也比实际的年龄年轻些,不那么显老。''

慧贵妃气得浑身发颤,几乎即刻就要发作。皇后安抚似的看她一眼,她都没有察觉,素心不动声色地点了点头,递了一碗茶过去,碰了碰贵妃的手肘,示意她安静下来。

皇后环视众人,慢慢道:``有了孩子的固然高兴,没有的也不必着急。皇上待后宫一向仁厚关爱,迟早都会有自己的孩子的。''她顿一顿,缓声道,``对了,本宫今日正好有一桩喜事要告诉你们,也是满宫里的大喜事。''她唤了一声,``莲心。''

莲心本木木地在那儿站了一早上,像个泥胎木雕人儿一般。她听得皇后召唤,几乎是剧烈地颤抖了一下,不由自主跪下了道:``奴婢在。''

皇后指着她,口气温和如春风:``满宫这些丫头里,本宫最疼的就是莲心。如今莲心也大了,本宫想着给她指婚指个好人家,她又不愿意出宫远嫁。跟着本宫忠心耿耿的人,自然不能委屈了她,便和皇上商议了,将莲心指给养心殿副总管大太监王钦,八月十六成亲。''

莲心一个激灵,脸色顿时变得雪白,伏下身哀求道:``皇后娘娘,奴婢\ldots\ldots 奴婢实在不想成婚,只想一直伺候着您。''

皇后笑得极和蔼,仿佛是对着自己的女儿一般温言细语:``本宫知道你的忠心,只是女人总不能不嫁人哪。你是本宫最信任的人,一定要嫁得好才是。王钦才三十出头,会长长久久陪着你的。你的嫁妆,本宫也会加倍厚厚地给你。''皇后语气微微一沉,``王钦中意你许久,这门亲事可是求也求不来的好姻缘。你可别辜负了本宫和皇上对你的疼惜。''

莲心颤巍巍跪在那里,泫然欲泣。素心忙扶了她道:``皇后娘娘慈爱,莲心高兴还来不及呢。她这定是高兴坏了。''说罢便扶了莲心下去。

如懿与海兰互视一眼,皆是默默,只随众人道:``皇后娘娘慈爱悯下。''

慧贵妃更是道:``王钦是皇上跟前的大红人,这门姻缘是配得起莲心的,要换了别人,求也求不得呢,还是皇后娘娘的脸面大。''

皇后笑意不减:``好了。这些都是闲话。''她看着蕊姬道,``如今最要紧的是玫常在的胎。你可得好好养着,万不能掉以轻心。''

蕊姬躬身答应了。众人贺了几声也告退而去。

皇后待殿中安静下来,方看了看素心,淡淡道:``去看看莲心,这样的大喜事,别掉泪珠子,晦气!''

素心忙赔笑劝道:``皇后娘娘放心,莲心只是一时糊涂,还没想明白罢了。''

皇后取了一颗枇杷,剥成倒垂莲花的样子,方慢慢吃了:``她还有什么不明白的!整个长春宫里,不是像你一般过了三十,便是年纪太小入不了眼。幸好王钦喜欢她,再三跟本宫提了,她又是本宫的心腹,本宫才肯抬举她。你要她好好记着,乖乖嫁过去,笼络住了王钦,就等于笼络住了皇上的心思和脚步。本宫断断容不得她坏了本宫的大事!''

素心知道轻重,忙又替皇后剥了一颗枇杷递过去,道:``娘娘的苦心咱们都知道,只是娘娘有阿哥有公主,又有中宫的权柄和皇上的关爱,咱们怕什么呢?''

皇后抬眼看了看碧澄澄空中流金泼洒似的日光,伸手探了探景泰蓝盆里供着的冰山,欷歔道:``本宫何尝不想高枕无忧?可是太后对后宫之事的涉入越来越多,你看玫常在就知道,皇上的嫔妃和子嗣只会越来越多,而本宫只会越来越人老珠黄,色衰爱弛。''她眸中一亮,似是闪过一点黑色的焰火,``所以万事不能不多一层防范。''

素心叹道:``智者必有千虑。娘娘费心了。''

玫常在的身孕是皇帝登基后的第一胎,皇帝虽然早有子女,也显得格外高兴。尽管连着几日操心于江南水事,但皇帝得闲便留在永和宫中嘘寒问暖。

这一夜难得玫常在没再缠着皇帝,皇帝便往延禧宫来,略略问过了永璜的功课,便留在如懿阁中一同用膳。

如懿替皇帝夹了一筷子菜道:``皇上可知道皇后娘娘要为莲心赐婚对食之事?''

皇帝含笑道:``你怎么问起这个了?''

如懿蹙了蹙眉:``臣妾只是觉得,好好的女儿家嫁了太监,实在可惜。''

皇帝道:``皇后这样说,宫中太监宫女多了,又不能都放出去,痴男怨女多了,还不如凑合了赐了对食,也好彼此安慰。皇后是好意,朕便允了。''

如懿听得这样,也不好多说,便倒了一杯酒在皇帝盏中,樱桃色的琼液凝在白玉酒盏中,如同一方上好的红玉,盈盈生辉。

皇帝笑道:``这酒的颜色看着很喜庆。''

如懿看着皇帝神色,亦是欢喜:``皇上心情好,自然看什么都是喜庆的。''

``你觉得朕心情好?''

如懿笑着伸手去抚他的眉毛,一根根浓黑如墨。这样近距离地望着他,连眉毛,也是这样好看的。``脸上全是笑纹儿,藏都藏不住。还有眉毛,眉毛都飞起来了。''她忍着心底的酸涩,轻笑道,``玫常在有了身孕,皇上是真高兴。''

皇帝笑着握一握她的手,只觉得她的手凉得如一块和田玉,握久了,慢慢也生了润意。他朗声道:``后宫里的事再高兴也是小事,前朝出了高兴的事儿,朕心里才真正快活。''

如懿倒了一盏酒敬到皇帝跟前:``皇上心里快活,就是臣妾心里快活。皇上为了治理前朝,日夜操心,所费的心神不是旁人看着就能明白的。所以这一杯,臣妾敬皇上。''

皇帝接过了却不喝,饶有兴致道:``你不问问朕,为什么高兴?''

如懿微微低首:``如同农人耕种,有付出,有收获。这便是高兴。其他的,臣妾身在后宫,不该问,也不能问。''

皇帝接过酒一仰脖子喝了,眼睛里都是晶灿灿的笑影儿,他执着如懿的手,柔声道:``这就是你的好处了。若是慧贵妃,她一定要追着朕问,是什么高兴事儿。''

如懿唇边恬淡的笑意微微一敛:``慧贵妃自然有慧贵妃的好处。可是皇上\ldots\ldots{}''她顿一顿,柔声里带着一分倔然硬气,``皇上,在这儿,咱们不说别人。''

皇帝怔了一怔,不觉一笑:``没看出来,你还有小心眼儿的时候。''

如懿的笑意若映着月亮的水,清亮分明:``皇上的心分成了两半,一半是前朝,一半是后宫。后宫的一半心儿,大半给了太后和公主皇子们,小半儿给了臣妾和诸位姐妹。在这小半里头,皇后占个大头,嫔妃们各自分了皇上的一点儿心,留给臣妾的也不多了。那么这一小瓣心来臣妾这里的时候,皇上若再分给了别人,那臣妾就连芝麻粒儿那么大都占不上了。''

皇帝吁了口气,伸手揽过如懿的肩:``这话你虽是带着笑说的,但是朕知道你心里的委屈和难受。朕还年轻,前朝的事情顾不过来,大臣们都是跟着先帝的老臣了,一个个都有资格摆在那儿。朕若是不亲自一件一件打理好了,哪件落了他们的话柄,都是朕的难堪。为着这个事儿,朕进后宫进得少了,为着孝亲的礼数和正宫的威仪,更要多陪陪太后和皇后。朕有数,朕陪你的时间,是不比在潜邸的时候了。''

如懿倚在皇帝肩头,金线腾云五爪龙纹的花样细密地硌在脸颊上,硌得久了,也觉出一丝粗糙的生硬,她低低道:``臣妾不敢怨,怨了那是不懂得皇上的难处。臣妾也盼着皇上来,私心里,最好是皇上来了就不走了。可是臣妾知道,夫君可以是一人的夫君,但皇上是天下的皇上。所以臣妾盼皇上来,也不敢盼皇上来。''

皇帝静了片刻,抚着如懿的鬓发,定定道:``这是真话了。朕走到后宫里,有皇后这个贤妻,也有慧贵妃的温柔,纯嫔体贴,嘉贵人妩媚,连怡贵人、海贵人和婉答应,也有她们的老实本分。可是唯独一样,你有的,她们谁都没有。''

如懿好奇:``是什么?''

皇帝吻一吻她的额头,静声道:``是一份直爽。这份直爽是对着朕的,从你入潜邸到今天,都没有变过。''

如懿怔了一怔,内心感怀,嘴上却硬着:``直爽算不得后妃之德,不是什么好处。''

皇帝轻叹一声,笑道:``这好处,后妃之中都没有,是夫妻之间的。''

仿佛是心底最柔软的地方被谁的手轻柔拂过,如懿几乎要落下泪来,她低下头,极力忍着泪:``如懿谢皇上,能够这样懂得。''

皇帝动容道:``朕懂得,更珍惜。所以如懿,虽然你不是朕的结发妻子,也不是陪伴朕最久的人,可你的好,都在朕心里。朕也希望你明白,不管这延禧宫朕来得多不多,你总是在朕心里,而不是只在这宫里。''

月光莹白,悠然漫行天际,像冰破处银灿灿流泻而下的一汪清水。远处的风带来花木肆溢张扬的清香。这样好的月色,隔着窗户半开的缝隙望出去,仿佛整个宫苑都凝霜般地冰雪洁白。这样好的月,是要映着这样成双的人的。如懿从未觉得,这紫禁城里的十六月圆,竟也是这般完满无缺。

这样宁和的时光,如懿真觉得自己要眠过去了。若是一眠醒来,还是这般的人月两圆,那该多好。

只是外头的敲门声响了两下,她原本闭着眼不想理会,外头却是又响了两下。如懿叹口气,看看桌上的菜色快凉了,知道是送菜进来的宫女,只得叹道:``进来吧。''

皇帝晓得她的心思,握一握她的手,含着笑并不说话。如懿脸上一红,却听殿门``吱呀''一声轻响,一个身影轻快地闪进来,后头跟着一个端着黄木四方虬纹盘子的小宫女,稳稳当当地走了进来。来人正是阿箬,她轻巧行了一礼,道了``万福'',轻轻颔首,托着盘子的宫女便走上前来。阿箬一道一道将菜式端出来,口中便道:``这道鹌子水晶脍是皇上最喜欢的,小主一早就吩咐了小厨房盯着做好,差半分都做不成这水晶剔透的样子;这道荷花蒸鸭脯是专用了不肥不瘦的鸭脯肉,鸭子爱活水,所以性凉去火,小主特意嘱咐了给皇上备上,解解批折子劳累的火气;这道糖醋鳜鱼酸甜可口,最宜下饭饮酒;还有一道碧糯佳藕口味清甜,是象征着皇上和小主佳偶天成,蜜里调油。''

皇帝笑道:``每道菜都是你们小主的心思,可她自己是不肯说的。从你嘴里说出来,这心思就活灵活现了。''

阿箬作势轻轻拍了一下自己的脸:``是奴婢多嘴了。可咱们娘娘是个实心人儿,惦记着皇上的心存在那儿,说不出来。奴婢要是不替小主说出来,只怕小主的痴心,更没人知道了。''

皇帝笑得轻快,拍了拍如懿的手背道:``其实你也算是个会说话的人了,没想到手下调教出来的丫头,一个赛一个机灵。朕记得,阿箬跟了你好几年了吧。''

如懿颔首道:``阿箬是臣妾的家生丫头,跟着臣妾陪嫁过来的。仗着伺候臣妾久了,那话就不肯安分蹲在舌头底下。''

皇帝倒是颇高兴:``自打住进了宫里,皇后的规矩大,教导得满宫里的奴才一个比一个更会装哑巴,恨不得没了舌头才好。朕倒觉得,都像阿箬这么说说笑笑的才好,你们关起门来过日子,也有趣儿得多。''

如懿听着阿箬被夸奖,心里也颇喜悦,便道:``既然皇上这么抬举你,留下布菜伺候吧。只一样,别得意得没了规矩。''

阿箬福了一福,笑盈盈道:``娘娘的嘱咐,奴婢哪回不记在心里?''说罢,便静静候在一边,伺候着两人用膳。

皇帝夹了一块甜藕慢慢吃了,笑道:``本来朕也不想提前朝的事儿了。可是这会儿看见这块藕,心里又高兴起来。江南水患连年成灾,一到夏天发了洪水毁掉良田万亩,灾民流离失所,这一直是朝廷的心头大患。先帝年年想治水,拨了银子下去筑造堤坝,可那堤坝比豆腐还软,总是防不住洪水。到了朕登基,朕派去江南治理两淮的官员上了折子,说今年的堤坝建得好,发了再大的水都没冲下去,百姓们总算是安乐了一年。尤其是淮阴知县管修的那一段,实实在在是把朝廷派下去的银子都用上了,那堤坝比铁浆浇得还硬实。往年淮阴最容易受灾,今年的知县倒能管事,又能治水,朕好好嘉奖了他一番。''

如懿替皇帝又夹了一筷子藕,侧首笑吟吟看着他:``能为皇上分忧的人,是该好好嘉赏,只不知这淮阴知县,叫什么名字?''

皇帝凝神想了想:``仿佛是叫桂铎,索绰伦氏,镶红旗的包衣出身,倒是极能干的一个人。朕正想着,他能实实在在修好了堤坝,便是个中用的人。朕再看他一阵子,若是经用,便可赏他做个知府。''

皇帝话音未落,却听阿箬利索地跪下磕了个头,激动得泪流满面:``奴婢谢皇上的赏,谢皇上隆恩。''

皇帝奇道:``朕赏朕手底下的官员,你急着谢什么恩呢?''

如懿含笑看着阿箬道:``桂铎是阿箬的阿玛。''

皇帝便也露出几分笑颜:``原来朕夸了半日,人家女儿就在这里。''他便向着阿箬道,``你阿玛在外头替朕尽心,你就好好在后宫伺候着,自己也能熬出个眉目来。''

阿箬喜不自胜,赶紧磕了个头谢恩。如懿见时机恰好,便道:``皇上这个意思,是可以替阿箬指个好人家了,那臣妾先替阿箬谢过皇上。''

皇帝夹了一筷子鳜鱼在如懿碗中:``阿箬有没有这个造化,还得看她自己的。''

阿箬见皇帝取过一旁的热手巾擦了手,忙站起身来,倒了一盏茶递到皇帝跟前:``这是新备下的六安茶,消垢腻去积滞是最好的。皇上尝尝。''

皇帝喝了一口,便含了几分笑意:``论细心周到,娴妃,你这儿是一等一的。''

如懿低眉笑得温文:``细心周到是对心的。皇上感觉到了,这心意也就到了。''

\hypertarget{ux7b2cux4e8cux5341ux4e03ux7ae0-ux5bf9ux98df}{%
\chapter{第二十七章
对食}\label{ux7b2cux4e8cux5341ux4e03ux7ae0-ux5bf9ux98df}}

皇帝站起身,往东暖阁去:``把朕常看的《春秋》拿来,朕去看会儿书,你洗漱完了再和你说话。''

如懿欠身答了``是'',阿箬又伺候着如懿添了一碗汤。西暖阁里烛火通明,越发衬得阿箬一张俏脸欢喜得面若桃花。

如懿笑着望她一眼,低声嗔道:``快把你那喜眉喜眼藏起来,皇上瞧见了,难免要觉得你沉不住气。''

阿箬摸了摸脸,不好意思道:``真藏不住了么?''

如懿笑道:``是呀是呀。不过你可记着,你阿玛只要用心,有的是前程,你也能有个好的将来。但是千万别得意忘形,要都传开了,怕别有用心的人惦记上。''

阿箬忙答应着下去了。

这一晚,皇帝自是宿在如懿这里不提。

到了深夜时分,小太监自是守在寝殿外守夜,阿箬出来看了一圈,见寝殿里都睡下了,便吩咐宫人们灭了几盏宫灯,自行散去歇息。

阿箬回到自己屋里,看着房间的陈设虽是宫女所住,但比绿痕她们所住的好了不止十倍,自是因为自己家中争气,又是如懿的陪嫁缘故。而以后阿玛步步高升,自己的来日更是有得指望了。这样想着,阿箬越发得意,一进门便在铜镜妆台前坐了,慢慢洗了手卸了妆。她自镜中见惢心只专心铺着床被,便瞥着惢心道:``虽然我与你都是伺候小主的宫女,但今日皇上的话你也听见了。从今往后,我与你便更是不同了。''

惢心向来不与她争执,只谦和笑道:``恭喜姐姐了,娘家有这样大的喜事。''

阿箬蘸了点杏花粉扑脸,仔仔细细地揉着道:``这杏花粉就是好,拿杏花汁子兑了珍珠末细研的,扑在脸上可养人了。是我阿玛特意从外头捎给我的。''她眼角带了倨傲的风色,斜眼看着惢心道,``其实阿玛这样巴巴儿地做什么,平日里小主赏我的东西也不少了。''

惢心理着床帐上悬着的流苏与荷包:``小主自然是疼姐姐的了。''

阿箬微微颔首,取下发髻间点缀的几朵嵌珠绢花,倚着手臂道:``小主疼爱,我阿玛也争气,以后你更要有点眼色。咱们虽住在一起,但上下有别。我是旗籍出身,你却是两百钱买回来的。以后这房里的打点,便是你的事了。''

惢心理着杏红流苏的手指微微一颤,旋即道:``知道了。''

阿箬点点头:``出了一身的汗,难受死了,你去打水来给我擦身子吧。还有,拿艾草好好熏熏,别让蚊子半夜咬着我。''

那本是底下小丫头做的事,阿箬虽平时霸道些,也不至于如此使唤她。惢心只觉得手里滑腻腻的,摸着那荷包也冷湿冷湿的。大约真是天热,手上的汗都冒出来了吧。惢心答应着,便也去了。

第二日晨起皇帝便要去早朝,如懿早早服侍了皇帝起身,便提醒小福子去唤了永璜起床预备着去尚书房读书。皇帝正要走,如懿心念一动,含笑道:``皇上的发辫有些乱了,左右离上朝的时辰还早,臣妾替皇上梳梳头吧。''

皇帝微微一笑,坐到镜前道:``从前在潜邸的时候你倒是经常替朕梳头,如今也疏懒了。''

如懿笑道:``臣妾倒想勤谨,只是皇上登基后仪容半分也不松懈,臣妾倒是想着,只那头发不肯给臣妾机会罢了。''

皇帝笑着拧了拧她的脸颊:``越发会玩笑了。''

如懿取过犀角梳子,将皇帝的头发梳得松散了,一点一点仔细地篦着。皇帝看着她蘸取篦发的花水,便问道:``你这篦发的是什么水?不是寻常的刨花水么?''

如懿笑道:``刨花水有什么好的?臣妾不喜欢那味道。这花水里加了薄荷、乌精、苦参、当归、何首乌、干姜、皂角、天麻、桑葚子、榧子、核桃仁、侧柏叶等几味药,收了冬日梅花上的雪水和榆花水兑着,又用茉莉和栀子调香,除了香气宜人淡雅,经常用来蘸了梳头,可以养血温肾,使头发乌黑健旺。''

皇帝笑起来别有温雅之风:``原以为你用东西精细讲究,原来讲究都在这里头。''

如懿为皇帝束好辫发,将辫梢上的明黄缠金丝穗子、翡翠八宝坠角一一结好,才笑道:``女儿家的心思也就弄这点小巧罢了,不比皇上胸中的经纬天地。''

皇帝看着她手中的犀角梳子:``朕记得这把梳子你用了许多年了,你看犀角周身的包浆干净莹润,大约是你女儿家时就用了吧。''

如懿爱惜地抚着梳子:``臣妾喜欢可以长久的东西。''

皇帝握住她的手,满面皆是春色笑影,越发显得丰神高澈:``人家都说是白头到老。朕整日用你的花水梳头,岂不是与你总是黑发到老,不许白头了?''

庭院中开了无数雪白的栀子花,那素华般的荼蘼脂泽如积雪负霜,满盈冰魄凉香。如懿温柔睇他一眼,半是笑半是嗔,那欣喜却化作眼底微盈的泪:``皇上惯会笑话臣妾。''

皇帝含了几许认真的神气,道:``朕只长你七岁,岁月虽长,但慢慢携手同行,总有白发齐眉、相携到老的时候。''

如懿鼻中微酸,眼中的潮热更盛,宫中的女子那样多,就如庭院里无尽的栀子花,前一朵还未谢尽,后一朵的花骨朵早已迫不及待地开了出来。他们的人生还那样长,皇帝不过二十六,自己也才十九。往后的路上还不知有香花几许,蜂萦蝶绕。可是此时此刻,这份真心,已足够让她感动。

心中的感动如云波伏起,她含笑含泪:``到时候臣妾鸡皮鹤发,皇上才不愿意看呢。''

皇帝道:``你是鸡皮鹤发,朕何尝不是?这才是真正的相看两不厌。''

如懿伸手延上皇帝的肩,头紧紧抵在他颈间,聆听着他心脉脉脉地跳动,仿佛是沉沉的承诺。良久,她终于以此心回应:``只要皇上愿意,臣妾会一直陪着皇上走下去。多远,多久,都一直走下去。''

皇帝笑着吻了吻她的脸颊,忽而咬住她的蝴蝶珍珠耳坠:``只说不算。朕要你拿一样东西来应。''

如懿满面羞红,推了皇帝一把:``什么?''

皇帝竖起食指嘘了一声,在她耳畔道:``你看镜子里,朕与你身成双,影也成双。''

如懿望了一眼镜中,泥金的并蒂莲花连理镜,花叶脉脉,皆是成双成对。如懿嗤地一笑:``臣妾想到了,自然会给皇上。''

皇帝不肯轻易放过:``可不许赖。''

如懿点点头,看着天光一分一分亮起:``皇上快起驾吧,别晚了。''

正巧外头敲门声响,是永璜童稚的声音在外头唤道:``母亲。''

如懿忙开了门,正见阿箬和小福子一个拉着永璜,一个替他背着书籍。永璜进来恭恭敬敬请了个安:``给皇阿玛请安,给母亲请安。''

如懿忙扶了他起来,怜惜地替他拢一拢头发:``睡得头发有些蓬了,母亲替你梳一梳再走。''说罢她便取过梳子替永璜梳好了。

永璜眨了眨眼睛,一副阴谋得逞的快乐:``母亲,儿子是故意蓬了头发,这样您就会替我梳了。''

皇帝在一旁看着,也不觉生了爱子之意:``你母亲的手很软,梳头发很舒服是不是?''

永璜用力点了点头,一脸幸福地拉住皇帝的手勾了勾。皇帝心下爱怜,牵过永璜的手道:``皇阿玛要去早朝了。不过还早,你跟着皇阿玛一起,皇阿玛今天先送你去尚书房见见你的师傅,好不好?''

永璜眼里闪过一丝雀跃,很快沉稳道:``儿子多谢皇阿玛。''

皇帝出门前,望着相送的如懿道:``有件事朕先告诉你。玫常在的身孕是朕登基后的第一胎,朕很高兴,所以打算封她为贵人。''他凑近如懿的耳边,语不传六耳,``但朕更盼着你,男孩女孩朕都喜欢。''

如懿面上烧得滚烫,却不敢露出半分神色来,只得极力自持道:``臣妾恭送皇上。''

永璜紧紧攥住皇帝的手走了出去,一路絮絮说着:``皇阿玛,儿子已经能把《论语》都背下来了\ldots\ldots{}''他说着,回头朝如懿挤挤眼睛,跟着皇帝出去了。

阿箬送到了宫门口,复又转进来,笑意满面:``大阿哥可真是聪明,一点就通。能有皇上亲自送去尚书房,以后大阿哥再不会受委屈了。''

如懿兀自微笑,忽然目光落在阿箬身上,逡巡不已。阿箬被如懿看得有些不好意思,不安地摸了摸鬓角和袖口,强自微笑道:``小主这么看着奴婢,是怎么了?''

如懿的目光失去了温和的温度,冷然道:``你这身打扮,都快赶上皇上新封的秀答应了。只是秀答应脸上的坦然倨傲之色也没有你的多。''

阿箬有些讪讪的,摸着袖口密密的樱桃红缠枝绣花,那花色一定是让小宫女拆了缝缝了拆忙活了许久才成的,每一瓣绣花里都点着玉色的蕊,配着双数的翠叶,落在翠粉色的衣料上,十分鲜亮。阿箬的绣花鞋上也绣了满帮的花朵,宫女的鞋原可绣花,但求素净。阿箬却是粉蓝的绣鞋上缀满了胭脂色的撒花朵儿,唯恐人看不见似的,映着一把青丝间点缀着的同色绢花并烧蓝嵌米珠花朵,越发夺目。

如懿蹙眉道:``你进宫时就知道宫训,宫女衣着打扮要朴素,说话行动不许轻浮。尤其是穿衣打扮,得像宝石玉器一样,由里往外透出润泽来。你看你穿粉点翠的,像个彩珠玻璃球一样,只图表面光彩做什么?''

阿箬的脸红成了虾子色,嗫嚅道:``奴婢也是为小主高兴,所以打扮得鲜亮些。''

如懿对镜梳通了头发,由着惢心盘起饱满的发髻,点上几枚翠翘为饰,又选了支简素的白玉珠钗簪上,方道:``你是为我高兴还是因为你阿玛的功劳为自己高兴?你在延禧宫里是最有身份的宫女,和惢心是一样的。只是你得明白,身份不是靠衣饰出格来换取的。''她见阿箬露出几分窘色,只搓着衣角不说话,只得缓和了语气道,``尤其是皇后不喜欢宫中奢华,如今虽然比从前宽松了些,嫔御许用金饰了,但宫女打扮得出格,必是要受责罚的。''

阿箬看如懿神色宽和了些许,才嘟囔着说:``奴婢也是知道自己和旁人不一样了,又是近身伺候小主的,所以才\ldots\ldots{}''

如懿见她如此不知事,不觉懊恼:``除去正月和万寿节外,宫女是不许穿红的。你看看你的衣裳和鞋子,若是被外头人看见,指不定就要挨竹板子。挨竹板子,疼是小事,丢人是大事,让执法的太监把衣服一扒,裤子褪下来,一点情面不留,臊也得臊死。''

阿箬吓了一跳,忙跪下道:``奴婢只是高兴,没想那么多。小主,奴婢\ldots\ldots{}''

如懿拣了一副玉叶金蝉佩正要别上领口,看她那个样子,不觉生烦,呵斥道:``赶紧脱了去,这身衣裳鞋袜,不到年节不许再穿!''

阿箬慌不迭下去了。如懿看了惢心一眼:``她如今有些家世,越发轻狂了。你和她一块儿住着,也提点着她些。''她见惢心只是默然,不觉苦笑,``是了,她那个性子,我的话都未必全听,何况是你呢?你不受她的气就是了。下去吧。''

惢心回到房中,阿箬只穿着中衣,正伏在妆台上哭。衣裳脱了下来横七竖八丢在床上,像一团揉得稀皱的花朵。阿箬听见她进来,忙擦了眼泪赌气道:``惢心,你说实话,我这样穿明明很好看是不是?''

惢心笑道:``是很好看,只是\ldots\ldots{}''

``只是小主觉得我太好看,怕抢了她的风头罢了。方才我送大阿哥去小主寝殿,看见皇上和小主在照镜子,那镜子里落进我半个身影,我也没觉得碍了谁的眼。没想到小主就觉得我碍眼了。''她呜咽着气愤道,``明明我这样打扮了出去的时候问过你,你也不觉得太僭越的。''

惢心露着恰到好处的笑容:``是是是,我是想,姐姐以后不在皇上来的时候这样打扮,就万无一失了。''

阿箬方才破涕为笑,换了衣裳出去了。

如懿趁着无人在旁,便打开压底的描金红木箱子,一层层翻起薄纱堆绣,有一样旧年的物事赫然出现在眼前。那还是她初嫁的时候,新婚才满三月,自然无事不妥当,无事不满意。闲来相伴他读书的时候,嗅着身边沾染了墨香书卷香的空气,一针一针绣下满心的憧憬与幸福。

彼时她才学会刺绣,笨手笨脚的,所以一方打了樱色络子的绢子上,只绣了几朵淡青色的樱花,散落在几颗殷红荔枝之侧,淡淡的红香,浅浅的翠浓,不过是两个名字的映照:青樱,弘历,相依相偎。绣好的时候,她也不敢送出手,怕惹他笑话,终究还是塞了箱底。如今想起来,除了这个,自己所有,除了身体发肤,无一不是他的。唯有那份稚拙的真心,经时未改,长存于此。

她想了想,拿过一个象牙镂空花卉匣封了,唤了三宝进来道:``等皇上下了朝,送去养心殿吧。别叫人看见。''

三宝答应着去了。如懿伏在窗下,看着莹白的栀子花开了一丛又一丛,无声无息地笑了。

日子过得极快,好像树梢上蝉鸣咝咝,荷塘里藕花初放,这一夏便过去了。玫贵人因着身孕而获晋封,一时间炙手可热。人人都想着无论她生男生女,因着这宠爱,皇上也势必对这孩子青眼有加。

永和宫这般热闹,咸福宫也未清静,慧贵妃一心一意地调理着身体,隔三差五便要请太医诊脉调息,又问了许多民间求子之法,总没个安静。这样过了七夕便是中元节,然后秋风一凉,连藕花菱叶也带了盛极而衰的蓬勃气息,像要把整个夏天最后的热情都燃烧殆尽一般,竭尽全力地开放着。

眼看着快到中秋,长春宫也忙碌起了莲心的婚事,虽是宫女太监对食,然而皇后却极重视,事事过问,宫人们无一不赞皇后贤惠恩下,连宫女都这般重视。八月十五的节庆一过,十六那日众人便忙碌了起来。对食是宫人们的大事,意味着此风一开,便有更多的寂寞宫人可以获得恩典,相互慰藉。因着莲心与王钦都在宫中当差,所以在太监们所住的庑房一带选了最东边、离其他太监们又远的一间宽敞屋子做了新房。

这一日黄昏,嫔妃们随着皇后一同在长春宫门外送了莲心。皇后特意给莲心换了一身红装,好好打扮了,慈和道:``虽然你是嫁在宫里,但女儿家出嫁,哪能不穿红的?''

皇后此言一出,众人又是啧啧称赞皇后的恩德。莲心含泪跪在地上,王钦紧跟着她跪下了,千恩万谢道:``多谢皇后娘娘恩典,奴才一定会好好疼莲心的。''

皇后含笑道:``这话就是了。虽然你们不是真夫妻,但以后是要一世做伴的,一定要互相尊重,彼此关爱,才不枉了本宫与皇上的一片心意了。''

莲心似有不舍,紧紧抓着皇后的袍角磕了三个头,泪汪汪的只不撒手。慧贵妃笑道:``莲心果然知礼,民间婚嫁就是这般哭嫁的,哭一哭,旺一旺母家,你就当是旺了皇后了。''

皇后弯下腰,手势虽轻,却一下拨开了莲心的手,温婉笑道:``好好去吧,别忘了本宫对你的期许就是了。''

素心忙笑着道:``恭喜莲心姐姐。以后便是王公公有心照顾了。''

王钦利索地扶过莲心,拉着一步一回头的她,被一群宫女太监簇拥着去了。

如懿自长春宫送嫁回来便满心的不舒服,却无半点睡意。好容易哄了永璜睡着,她便支着腮在烛下翻看一卷纳兰的《饮水词》。

惢心端了一碗红枣银耳汤来,道:``皇上叮嘱了每日早起喝燕窝,临睡前用银耳,小主快喝了吧。否则皇上不知怎么挂心呢。''

如懿头也不抬道:``先放着,我先看会儿书再喝。''

惢心将蜡烛移远了些:``小主看什么这么入神?小心烛火燎了眉毛。''

如懿缓缓吟道:``飞絮飞花何处是?层冰积雪摧残。疏疏一树五更寒。爱他明月好,憔悴也相关。最是繁丝摇落后,转教人忆春山。湔裙梦断续应难。西风多少恨,吹不散眉弯。''她慨然触心,``难为纳兰容若侯门公子,竟是这般相重夫妻之情。绿衣①悼亡,无限哀思。''

惢心舀了舀银耳汤道:``小主,今日是莲心出嫁的好日子,你看这个,好不应景。''

如懿失笑道:``是了。要让贵妃知道,必是以为我在咒莲心呢。''

两人正说笑着,阿箬点了艾草进来放在角落熏着,又换了景泰蓝大瓮里供着的冰。阿箬替如懿抖开纱帐,往帐上悬着的涂金缕花银熏球里添上茉莉素馨等香花,取其天然之气熏这绣被锦帐。花气清雅旖旎,在这寂静空间中萦纡旋绕。忽然静夜里不知何处传来一声尖厉的叫喊,仿佛是谁受了最痛苦的酷刑一般,那叫喊声穿破了寂静的夜空,迅速刺向深夜宁静的宫苑。

如懿一时没反应过来,只以为自己听岔了。正要说话,又一声叫声嘶厉响起,带着凄厉而绵长的尾音,很快如沉进深不见底的大海一般,无声无息了。

三人愣了半晌,阿箬怯怯道:``那声音,好像是从太监庑房那儿传来的。''她迟疑着道,``应该不会错,咱们延禧宫离那儿最近了。''

惢心静静挑亮了灯火,低声道:``这声音像是\ldots\ldots{}''

阿箬眼睛一亮,带着隐秘的笑容:``莲心!''

次日清晨,如懿被照进寝殿的金色光斑照醒,无端便觉得身上沁了一层薄薄的汗意。到了初秋尚有暑意,如懿迷蒙地躺着,看着惢心和绿痕进来卷起低垂的竹帘,又端了新的冰进来,将榻前景泰蓝大瓮里供了一夜渐渐融化的冰都换出去了。她卧在床上,身下的水玉凉簟细密地硌着肌肤。她打着水墨山水的薄绫扇,听着细小的水珠顺着那些巨大的冰雕漉漉沁滑下去,泠泠的一滴轻响。兀地想到昨夜那两声惊破了静寂的凄楚叫喊,仿佛蕴着极大的无助与痛楚。如懿微微一想,便忍不住自惊悸中醒转。

起来梳洗的时候如懿还有些怔怔的蒙昧,惢心一边替她梳头,一边道:``昨天傍晚烧了满天的火烧云,今天起来那太阳红闷闷的,等下怕是要下雨呢。等下了雨,就凉快些了。''

如懿道:``等下去长春宫请安,备着伞吧。''

惢心答应了一声,去外头准备了,便和阿箬陪着如懿往长春宫走。

莲心虽是新妇,一早也在长春宫中伺候了。众人见她穿着平素的宫女衣裳,只是发髻间多了几朵别致绢花,喜盈盈的颜色,神色倒是平静如常。嫔妃们贺了几句``恭喜'',又各自备下了一点赏赐赠她。莲心一一谢过,便安分地随在皇后身边。

皇后含笑饮了口茶,瞥见她手上新戴着的一个玉镯子,便道:``看你这个打扮,想来王钦待你极好。''

莲心脸上一呆,露了几分凄苦之色,很快如常笑道:``托皇后娘娘的洪福,一切都好。''

皇后极高兴:``这便好,也不枉了本宫一番心意了。''她唤过素心,取出一双银鎏金福寿双成簪子捧在锦盒中,``小主们都送了你不少东西,本宫是你的主子,也不能薄待了你。这双簪子便送你吧,希望你和王钦也福寿双安,白头到老。''

莲心身上一个激灵,像是高兴极了,忙屈身谢过。

众人请安过后便一同出来。怡贵人笑盈盈道:``皇后娘娘慈心,对下人们真是好。''

嘉贵人亦道:``莲心不过是个宫女,即便指婚也未必能指到多好的人家,还不如嫁了王钦,也是一世的荣华呢。''

纯嫔带了几分惋惜:``可惜了王钦是个太监,莲心她\ldots\ldots{}''

嘉贵人不屑道:``太监是缺了那么一嘟噜好玩意儿,可是缺了怕什么?莲心嫁到外头,一旦有点好歹,那是贫贱夫妻百事哀。还不如守着宫里的荣华呢。''

纯嫔不好意思地啐了一口,秀答应听她说得直接,红着脸笑得捂住了嘴:``这话也就嘉贵人敢说了,咱们是想也不敢多想。''

玫贵人原走得慢,听到这儿忽然站住了脚道:``各位姐姐难道昨晚没听见什么声音么?''

怡贵人睁大了眼睛,神神秘秘道:``难道\ldots\ldots 玫贵人也听见了?''

玫贵人含了一缕隐秘的笑意:``也不知道我是不是听岔了,恍惚听得太监庑房那儿传来两声女人的叫喊。''

怡贵人连忙拉住了她道:``我也听见了。但我的景阳宫在妹妹的永和宫后头,听得不大清楚,还当是风吹的声音呢。''

玫贵人笑着挥了挥绢子,见众人都全神贯注听着,越发压低了声音道:``我的永和宫在娴妃娘娘的延禧宫后头,照理说延禧宫离太监庑房那儿最近,该是她听得最清楚了。''

阿箬忙兴奋道:``的确是\ldots\ldots{}''

如懿立刻打断道:``的确是我们已经睡熟了,没有听见。''

怡贵人便有些悻悻的:``那个时候还不算太晚,娴妃娘娘不肯说就罢了。''她只打量着阿箬,``阿箬,你伺候娴妃娘娘,肯定睡得晚。你可听见了?''

阿箬含糊地摇了摇头。海兰道:``姐姐们别瞎猜了。即便有什么动静,那太监的喊声,也和女人的声音差不多。''

玫贵人笑道:``太监就是太监,女人就是女人,这点总还是分得出来的。你们想,太监庑房那儿会有什么女人呢?莫不是\ldots\ldots{}''

纯嫔忙念了句佛,叹道:``可不能胡说,这是皇后娘娘莫大的恩典。咱们这么揣测,可是要惹皇后娘娘不高兴的。''

嘉贵人哧哧笑道:``现在已经离了长春宫了。再说了,难道许她喊,就不许我们议论么?我倒想知道个究竟,莲心为什么会喊起来的?''她压低了声音,笑得像一只窃窃的鼠,``即便没见过男人,见个太监,也不必高兴成这样吧?''

玫贵人皱了眉头,拿绢子擦了擦耳朵:``阿弥陀佛,还当是什么叫声呢,夜里听着怪瘆人的!像受了酷刑一般!吓得龙胎都在我腹中抽了两下,差点便要传太医了。''

怡贵人立刻附和道:``玫贵人听得没错,叫得可凄厉了。我还当是夜猫子叫呢。''

嘉贵人不解道:``太监能有什么本事,她便不情愿,还能怕成那样?''

纯嫔听着不堪,便道:``嘉贵人出身朝鲜,便不知道这个了。前明的时候阉宦横行,多少见不得人的脏东西都有呢。''

秀答应忽然诡秘一笑,招了招手示意众人靠近道:``可不是!从前明朝的大太监魏忠贤,便耍尽了那些见不得人的手段,和皇帝的乳母客氏对食。后来还弄死了好几个小宫女呢。''

嘉贵人惊诧道:``这也有死了人的?''

秀答应点头道:``可不是!有些有钱的太监在外头娶了妓女做小老婆的,娶一个弄死一个,连妓女都架不住,何况一般人!''

如懿实在听不下去,脚下步子略快,与海兰拐了弯便进了长街,不与她们再闲谈。她正疾步走着,忽然听得身后一声唤:``娴妃娘娘留步!''转头竟是莲心,捧着一方绢子急急赶上来道,``娴妃娘娘,您的绢子落在长春宫了。皇后娘娘叫奴婢给您送过来。''

注释:

①绿衣:《绿衣》是《诗经》中一首有名的悼亡诗,本诗表达丈夫悼念亡妻的深长感情。诗人目睹亡妻遗物,备生伤感,由此浮想联翩。由衣而联想到治丝,惋惜亡妻治家的能干。

\hypertarget{ux7b2cux4e8cux5341ux516bux7ae0-ux897fux98ceux6068}{%
\chapter{第二十八章
西风恨}\label{ux7b2cux4e8cux5341ux516bux7ae0-ux897fux98ceux6068}}

如懿谢了她,接过。离得近了,方才瞧见她仔细敷好的脂粉底下,一双眼皮微微肿泡着,想是哭过。如懿心中明白,想她素日虽然有几分骄横,如今也是可怜,不觉便生了几分怜惜:``多谢你。看着天色快下雨了,赶紧回去吧。沾了雨可不好。''

阿箬忽然笑了一声,道:``沾点雨怕什么,如今莲心姐姐可与我们不同了,淋了雨都是有人心疼的。''

如懿轻声呵止道:``阿箬,咱们回宫去。''

阿箬走了两步,止住脚转身笑吟吟打量着莲心道:``都说太监会疼人,看莲心姐姐今日的打扮,的确是王公公会疼人了。穿衣打扮都不一样了。''她凑近了低声笑道,``不过还有一件好处,姐姐嫁了王公公,便省了生儿育女的一桩苦处,也省下了为人母亲的烦心事。那是多少人求也求不来的福气。''

莲心气得双唇发颤,雪白的面孔上只见一双充斥了血丝的眼睛黑红交间地瞪着阿箬,又是气愤又是凄楚,显然是气到了极点。良久,她终于吐出一句,那语气冷得像冰锥子一般扎人:``这福气这么好,我就祝愿你,也嫁一个公公对食,白头到老,死生不离。''

阿箬气得眼睛一瞪,很快忍住了笑道:``我哪里能和姐姐比,不过是我们小主抬举,总要将我指婚给御前侍卫的。只好眼看着姐姐和王公公,无儿无女,相伴到老了。''

如懿气得胸口像裹了一团火似的,喝道:``阿箬,你给本宫住嘴!再敢放肆,本宫就要狠狠罚你!''

莲心满眼是泪,只咬着牙狠狠忍着。如懿呵斥声未止,只听后头一个声音森冷道:``什么就要狠狠罚,在宫里这样放肆取笑,立刻就该打死!''

如懿听得声音,知道不好,忙转过身去,只见慧贵妃携了茉心站在拐进长街的朱红门壁边,目光冷厉,盯着如懿,宛如要在她身上剜出两个透明窟窿来。

如懿忙屈身道:``贵妃娘娘万安。''

阿箬也不禁有些慌,忙跟着道:``贵妃娘娘万安,娘娘恕罪。''

慧贵妃冷哼一声,也不看她,语气冷冽如冰:``恕罪?是谁纵得你在宫里放肆喧哗,胡言乱语?还敢在螽斯门①底下说无儿无女这种话,简直是大逆不道!''

如懿立时回过神来,才发觉方才急于避开那些闲话之人,原来是转进了螽斯门。宫中所建螽斯门,意在取螽斯之虫繁殖力强,以祈盼皇室多子多孙,帝祚永延。阿箬在这里说这种``无儿无女''的话自然是大逆不道,更怕是戳着这些日子来一直求子的慧贵妃的心思了。

如懿忙屈身道:``阿箬一时放肆,言语失了轻重,还请贵妃娘娘恕罪。''

阿箬也着实吃了惊吓,忙跪下道:``贵妃娘娘恕罪,奴婢是无心的。''

莲心看了贵妃一眼,低低道:``无心也能说出这般刻薄的话来,奴婢实在是闻所未闻。一切交给贵妃娘娘处置,奴婢先告退了。''

茉心含了一丝讥讽与厌弃:``贵妃娘娘每日晨昏都要来螽斯门祝祷大清子孙昌盛,你也太不要命了!何况莲心的婚事是皇上皇后亲口允的,那是赐婚,是无上荣耀,凭你也敢说三道四,出言嘲讽?等下贵妃娘娘说给皇后听,皇后也必不会饶你。''

阿箬求救似的看了如懿一眼,如懿无奈地摇摇头,实在是恨铁不成钢。阿箬无计可施,只得规规矩矩跪着磕了头道:``奴婢因是与莲心姐姐相熟,才这般玩笑的,娘娘恕罪啊!''

慧贵妃沉默片刻,指着门上匾额向阿箬道:``大清历代祖宗在上,螽斯门乃宫中绵延子嗣最神圣之地,你竟敢在此说出大逆不道的话,本宫不能不在此责罚你,以敬列祖列宗。''

撒金海蓝底的匾额,以满蒙汉三种文字分别书写着``螽斯门''三字。此时天光暗沉,远远有乌云自天际滚滚卷来,唯云层的缝隙间漏出几线金线似的明光,落在匾额的泥金框上,那种炫目的金色,几乎要迷住人的眼睛。

贵妃使了个眼色,双喜立刻会意,一招手带上一个小太监,死死按住了阿箬,茉心拔下头上一支银簪子,没头没脸地往阿箬嘴上戳过去。阿箬吓得面色煞白,拼命躲避,嘴里不住地求饶。茉心戳了几下没戳到,又气又恨,忍不住手上更是加力。

如懿忙拦在阿箬身前道:``住手!阿箬再有差错,也不能这样扎她。''

慧贵妃一把扯开她,轻蔑道:``本宫还没有问你管教不严之罪,你还敢帮她!''

如懿见阿箬躲了两下没躲开,嘴唇上已被扎了一下,汩汩流出殷红的血来,看着甚是吓人。

如懿忙跪下道:``阿箬是有过错,但请贵妃娘娘宽恕,容我带回宫中慢慢管教!''

慧贵妃精心描摹的眉眼露出森冷的寒光,与她娇艳温柔的面庞大不相称:``交给你也只是教而不善。本宫是贵妃之位,就替你管教管教下人。''

如懿眼见阿箬受苦,虽是气她口不择言去伤莲心,可也心疼她唇上的伤,心中愈加焦急难言,只得低头道:``娘娘怎么罚我和阿箬都不敢有怨言。只是宫中的规矩,对宫女许打不许骂,伤人不伤脸。阿箬在宫中还是要当差的,带着伤谁也不好看。还请贵妃娘娘宽宥。''

天际有闷雷远一声近一声传过来,空气黏着如胶,像是谁的手用力挞在胸上,让人透不过气来。贵妃淡淡一笑,眼波却如碎冰一般:``阿箬不要颜面,你不要颜面,本宫却是要的。茉心,你去回皇后娘娘的话,阿箬出言不敬,冒犯祖宗,本宫罚她在螽斯门下思过六个时辰,不到时辰谁也不许放她!''

茉心得意地答应一声,贵妃道:``双喜,留在这儿看着她,本宫先回去歇一歇。''

双喜响亮地答应着,笑眯眯向阿箬道:``姑娘,如今只有我陪着您了。六个时辰,咱们贵妃娘娘已经是大发慈悲了。''

贵妃目光一剜:``至于娴妃,本宫罚你抄写《佛母经》②百遍,今夜之前交到宝华殿焚烧谢罪。''

如懿诺诺答应,见她走远,方才起身。阿箬慌不迭膝行上来,抱住如懿的腿道:``小主救奴婢,小主救救奴婢!''

那长街的青石板砖上都是镂刻了吉祥花纹的,哪里会不疼?跪在那里六个时辰,等于是给膝盖上了刑。如懿又气又恨又心疼,心里跟搅着五味似的复杂,当着双喜的面又不愿露出来,只得撇开她的手,怒其不争道:``你现在知道求我了,我让你闭嘴的时候你怎么就要这么饶舌去取笑人家,挖人家的伤疤!如今你让我去求谁?口不择言伤了贵妃的颜面,羞辱莲心伤的是皇后皇上和王钦的颜面,现下还有谁能来救你!你便老老实实跪着吧!''

不远处隐隐传来贴地旋卷的风声,一股奇特的尘土气息在风里飞散。浓密的雨云汇集过来,乌压压地盖住了天空,每一阵风过,都簌簌卷来不知从何处落下的大片森绿的叶子和残花。落在红墙碧瓦之下,隐隐带了丝阴沉的气味。

雨点子冷不丁地落下来,溅起尘土呛浊的味道,如懿看着更是不忍,只得低声下气向双喜道:``双喜公公,阿箬跪在这儿也罢了,只是眼看着便要下雨,这两把伞便留给您和阿箬吧,免得都淋坏了身子。''

双喜皮笑肉不笑道:``可不敢当。娴妃娘娘,奴才皮糙肉厚的,不怕雨点子淋。可是阿箬嘛,既是受罚,就不必得这样照顾了。难道哪天她那张惹事的嘴拖着她要被送去砍头,您还怕刀太快削了她么?好了,您也请回吧,犯不着和奴才们一块儿堆着。''

惢心低低道:``小主还是回去吧,那百篇的《佛母经》抄不完,只怕贵妃又要怪罪呢。''

乌沉沉的天空中电闪雷鸣,轰轰烈烈的焦雷几乎是贴着头皮滚过,带着水汽的风阵阵袭来,将裙角吹得飞扬如翅。如懿实在是无可奈何,只得摇摇头,撇身离去。

一袭冷风暴烈地叩开窗棂,席卷着泥土草木被雨水暴打的气息肆无忌惮地穿入宫室,忽忽的风吹得窗子啪啪直响,几乎要将四盏蒙着白纱笼的掐丝珐琅桌灯尽数吹灭。如懿赶紧护住案几上已经抄了大半的《佛母经》。惢心忙将窗上的风钩一一挂好,方过来研了墨道:``这雨下到午后了,怎么一点儿也不见小?''

她见如懿只是低眉专注地抄写,又忧声道:``奴婢悄悄去看过阿箬,原想塞两个馒头给她。可是双喜打了伞坐在宫门避雨的檐下看着她,一点都不肯松动。''

如懿笔下一颤,写歪了一个字,只得揉皱了扔下道:``活该!几次三番要她嘴上留心,她偏偏不听,恃强拔尖,嘴上不饶人。''如懿越说越恨,``事事要拔尖也得有拔尖的本事,这样没遮没拦的,活该长个记性!''

惢心不敢再说,只得细心添了水研磨墨汁。如懿心下烦忧,又惦记着慧贵妃的嘱咐,知她不好应付,只得用心仔细抄录,生怕被她挑出一点毛病来。好容易只剩下十几遍了,她又不放心起来,听着雨声哗哗如注,简直如千万条鞭子用力鞭打着大地,抽起无数雪白的水花。她侧耳倾听,叹息道:``都说雷雨易止,这雨怎么越下越大了呢?''

惢心知她心中还是担心阿箬,便道:``也是老天爷爱磋磨人,早起虽热,下了雨却寒凉,阿箬跪在大雨里,回来还不知道是怎么样呢?''

雨水敲打着屋檐瓦当,惊得檐头铁马叮当作响,如懿心下愈加烦躁。她按捺住满心的担忧,吩咐道:``我这儿的《佛母经》快抄完了,你等下赶紧送去咸福宫知会一声,然后去宝华殿焚烧了交差。''

惢心口上答应着,知道如懿的话必定还没完,便拿眼瞧着如懿。果然如懿凝神片刻,唤进三宝道:``阿箬跪了几个时辰了?''

三宝忙道:``四个多快五个时辰了。''

如懿点点头:``你去太医院请许太医过来,就说是我身上不大松快。再嘱咐他备些祛风治寒的发散药物。''

三宝答应着赶紧出去了,如懿又吩咐绿痕:``去多烧些滚烫的热水来,阿箬回来给她泡个热水澡去去寒气。再抱两床厚被子在她屋子里给她捂上。还有,姜汤也要备好。''

绿痕一迭声答应着,惢心含笑道:``小主还是心疼阿箬。''

如懿摇摇头:``她跟了我这些年,自然没有不心疼的。只是,她也太不争气了。''

过了好一阵,如懿将写好的百篇《佛母经》都交到惢心手里:``去吧。回了慧贵妃就去做你的差事。''

惢心叮嘱了绿痕并几个小宫女几声,便告退了出去。

如懿站在廊下,看着惢心擎了伞出去,四周湿而重的水汽带着寒意透过衣裳,像是要把她的身体一同浸润了一般。天色暗沉得宛如深夜,廊下院中数十盏宫灯飘摇在雨中,像是忽远忽近的鬼火,飘忽不定。如懿披衣站着,看着宫苑殿阁的棱角在雨水的冲刷下渐渐变成深色却模糊的薄薄剪影,心中便生出无尽的担忧与惘然。

她正沉思着,只见一个浑身湿透的人豁然闯入宫门,精疲力竭地跪倒在雨水之中。

注释:

①螽斯门:螽斯门是西二长街南门,南向,北与百子门相对。螽斯是一种昆虫,繁殖力强,善鸣。螽斯门的典故源自《诗经·周南·螽斯》,诗中描述了螽斯聚集一方、子孙众多、虫鸣阵阵的景象。皇宫内廷西六宫的街门命名为螽斯,意在祈盼皇室多子多孙,帝祚永延。

②《佛母经》:又名《佛母大孔雀明王经》,内容叙述莎底苾刍为众破樵,为黑蛇所螫,不堪苦痛,阿难向佛求救,佛为他说大孔雀明王神咒而救之。

\hypertarget{ux7b2cux4e8cux5341ux4e5dux7ae0-ux72ecux81eaux51c9}{%
\chapter{第二十九章
独自凉}\label{ux7b2cux4e8cux5341ux4e5dux7ae0-ux72ecux81eaux51c9}}

如懿一怔,旋即辨认出那个如同水里捞出来的身影便是阿箬。如懿连忙让几个小宫女扶她进了自己的房中。绿痕正好烧好了热水进来,忙把水倒进了柏木浴桶中,七手八脚和如懿将她湿透的衣服剥除了,整个人挪进浴桶里去泡着。

阿箬感觉到周围滚烫的水,才呻吟着醒了过来,一见如懿在身边,眼泪立刻落了下来,唤道:``小主。''如懿一壁吩咐绿痕往水中加入活血驱寒的姜片、石菖蒲和黄酒,一壁伸手进水里替她搓着手臂,方道:``不是要六个时辰么?怎么那么快回来了?''

阿箬的脸上已分不清是水还是泪,只哭着道:``说是皇上去皇后娘娘那儿用晚膳,见奴婢跪在那里可怜,便向皇后娘娘提了一句。皇后娘娘才开恩放了奴婢回来。''

如懿道:``先别哭了。赶紧泡热了身子,我给你腿上上点药。跪了那么久腿一定很疼。''她起身回到殿中,默默剔亮了灯芯,听着外头雨疏风骤,不过多久,却见惢心推门进来,她有些诧异:``怎么回来了?''

惢心有些为难,片刻方道:``慧贵妃看了小主抄写的《佛母经》,说小主敷衍了事,写得不仔细,并不是诚心受罚。''

如懿叹口气:``那她要怎样?''

惢心屏息敛气:``慧贵妃说,要小主重新抄录一百遍,明日去长春宫请安前送去咸福宫。''如懿微微凝神,便道:``无妨,我再抄一百遍就是。''

惢心觑着如懿的神色,低低道:``其实,其实慧贵妃压根没翻小主抄的佛经,小主怎么抄她都不会满意的,分明是存心刁难小主。''

如懿淡然一笑:``那不是意料中的事么?她要的何尝是佛经?不过是要看我辛苦劳碌,疲于奔命罢了。''

她说罢再不言语,起身到了案几前,提笔蘸墨,依次抄录了起来:``为着玫贵人的身孕,她已经怄了许多气,我再这般不驯服,便是落了她话柄了。''

惢心踌躇片刻,还是道:``可是贵妃的确是过分了。''

如懿含了一缕微薄的笑意,淡淡道:``阿箬没有分寸,她要管教阿箬。她自己失了分寸,我也会让她知道什么叫在分寸之内。''

惢心看着她提笔立时写就,不觉诧异:``小主不是要抄佛经么?怎么写了一首旁人的诗?''

如懿道:``抄写佛经不过是小巧,这个才是最要紧的。''她附耳低语几句,惢心会意一笑:``奴婢遵命。''

两人正说着话,三宝已经带着许太医过来了。阿箬也换了一身干净衣裳被绿痕扶了颤巍巍地过来。如懿道:``劳烦许太医了,替本宫瞧瞧这位姑娘。''

许太医答应了一声,便替阿箬请了脉,很快道:``姑娘淋了大雨着了风寒,现下有些发热,需得仔细调养。现在最要紧的是防着高热发作,免得烧坏了身体。微臣会开好方子送了药来,请小主宫里的人赶紧替姑娘煎了药吃下去才好。''

``那膝盖上的伤?''

许太医恭谨道:``只是外伤,上点药就不妨事的。''说着从药箱里取了两瓶药粉出来,``内服外敷,好得更快。''

如懿谢过,便吩咐三宝好生送了许太医出去,取过他留下的药,语气平稳无澜:``把裤腿卷起来。''

阿箬卷好裤腿,露出又青又紫的膝盖,最严重的地方硌破了皮肉,沁出鲜红的血丝。如懿微松一口气,替她敷上药粉。阿箬止不住呜咽起来:``小主,奴婢好委屈!''

如懿慢慢在伤口上撒着药粉,淡淡道:``委屈什么?''

阿箬哭道:``慧贵妃这么折磨奴婢,就是为了折损小主的颜面。奴婢受委屈不要紧,可是小主\ldots\ldots{}''

如懿将药瓶往桌上重重一搁:``你受委屈当然不要紧,因为你受的委屈都是自作自受,都是活该!''

阿箬怔了片刻,似乎是不可置信般,放声哭道:``小主以为奴婢是为什么?从前莲心言语冒犯,几次顶撞小主,不阴不阳的,奴婢已经瞧不上她许久了。昨日她指婚荣耀,今日就受折磨,奴婢是替小主高兴,是替小主报仇才奚落了她几句么!''

心口像有一团野火燎原,如懿沉着脸呵斥道:``为我报仇,还是替我挖个坑跳下去?我再三告诫过你,宫里不比外头,由得你这样骄纵任性,满口乱说。这是后宫,一句话说错便是要活活打死的,你有几条舌头去填你自己的命!''

阿箬战战兢兢地看着如懿,哀泣道:``奴婢就算有不是,也是对小主一片忠心呀!''

如懿气得话也不会说了。惢心忙道:``阿箬姐姐,小主就是为了替你求情,才被贵妃娘娘再三为难,抄了一百遍《佛母经》还不够,还要再抄一百遍。''

阿箬怯怯道:``奴婢就是不服气,不服气从前在潜邸的时候小主和她都是侧福晋,如今怎么就要事事踩在小主头上?小主又不是争不过她!''

如懿气得脸都涨红了,手上的护甲敲在紫檀桌上发出沉闷的悠响。她恼怒道:``你凡事只知道争,只知道要出头!却从没想过凡事要适可而止,有进有退!你是想争,偏偏争不过人家,还把自己填了进去!''

阿箬气馁地哭起来,惢心见两下里尴尬,便端过一碗姜汤给阿箬:``姐姐身上不好,快喝了姜汤散一散吧。''

阿箬就着惢心的手正要喝,如懿愈加不乐:``让她自己喝!''阿箬扁了扁嘴不敢再哭,只得自己接过喝了。

如懿严厉道:``等下喝了药好好去睡。这是最后一次,下次还要口不择言,凡事胡乱逞强,我也保不了你。''

阿箬垂着眼睛,无声地啜泣着出去了。

如懿心下烦乱不堪,拽过一管玳瑁紫毫笔便开始抄写佛经。惢心小心翼翼道:``小主也该饿了,不如传晚膳吧!''

如懿头也不抬:``气也气饱了,不必了。''

这一生闷气便是一夜。如懿抄录佛经抄得晚,夜里又听着微凉的雨簌簌一夜,夹杂着雨打芭蕉之声,格外愁人似的,这一夜无论如何便没有睡好。

如懿起来便闷闷的,将昨夜剩下的佛经一并抄录好交给惢心,便道:``去吧。''

惢心见外头雨停了,便先送永璜去了尚书房。绕过尚书房便到了长街,惢心一早便知皇帝昨夜歇在玫贵人处,便特意绕了往永和宫外走。果然见微明的天色下,远远有太监们薄底靴轻快擦着青石砖板的步声传来。一溜宫灯如星子明耀,簇拥着明黄御辇,后头跟着无数仪仗,自悄然寂静的宫墙夹道疾疾走来。

惢心只当是低头走路,打皇帝跟前走过。前头的引导太监便呵斥起来:``谁呢?没看见御驾在此么?''

惢心吓得忙跪下道:``奴婢延禧宫宫女惢心,无心冒犯圣驾,还请皇上恕罪。''

皇帝倒还和气:``这个时候,是刚送了永璜去阿哥所么?''

惢心道:``是。奴婢原本想去永和宫门外迎候皇上。''

皇帝道:``什么事?''

惢心垂着头,恭恭敬敬道:``娴妃娘娘说,今日是八月十八观潮日,皇上曾与小主说起向往海宁观潮胜景,遗憾不能一去。小主特意叫奴婢交一份东西给皇上。''

皇帝点点头,王钦便上前从惢心手中取过,双手捧着奉给皇帝。皇帝打开一看,却见一张玉版纸上,寥寥几行簪花小楷:``八月涛声吼地来,头高数丈触山回。须臾却入海门去,卷起沙堆似雪堆。''那是刘禹锡的《浪淘沙》,写的正是八月十八钱塘江潮壮观之景。

皇帝明如寒星的眼里便有了一丝温暖清澈的笑,这是他曾与如懿说过的,对于钱江狂潮的向往。她却都记得,在这八月十八的清晨,便将满江浪潮一笔一笔写了给他。纸张下部还有一篇《佛母经》,皇帝温和道:``怎么有一篇《佛母经》?''

惢心道:``小主说,钱江潮虽然万马奔腾,气势无可比拟,但难免对民众有所损伤,常常听闻有人被卷落江水。所以小主特意抄写《佛母经》一篇,想借佛母慈悲,眷顾民众。''

皇帝十分喜悦,便道:``如此,朕就收下了。王钦,将娴妃所抄的《佛母经》供在养心殿神龛前,这个月都不必取下来了。''

王钦答应着,惢心侧身跪在甬道边,满面恭敬地看着御驾迤逦而去,才露出了一丝愉悦的笑容。

惢心回到宫中时,如懿已经自长春宫中请了安回来,倚在长窗下挑拣新送来的白菊花苞。那些花苞尚未开放,带着淡淡的青色,仿如凝玉一般。如懿一朵一朵地挑选着,任清幽的香气在指间幽幽弥漫。

惢心笑道:``小主在忙什么?''

如懿盈然一笑,恍若淡淡绽放的白菊盈朵:``挑点白菊花苞做个枕头,给永璜枕着,可以明目清神。''

惢心搬了小杌子坐在如懿身边,帮着一起挑选:``小主怎么突然有这个兴致了?''

``从长春宫请安回来,慧贵妃什么话都没对我说,我就知道,你把事情办好了。''

惢心低眉恭顺道:``是。皇上把小主的《佛母经》供在了养心殿的神龛前,奴婢只在贵妃面前提了一提,贵妃便不做声了。她虽然气恼,但还是让奴婢把佛经都送去宝华殿烧了。''

如懿露出一丝意料之中的微笑,道:``皇上都喜欢的,她还能挑剔么?''

惢心道:``小主没有告诉皇上贵妃刁难您的事,已经是手下留情了。''

``我只是想警醒她,并不欲与她剑拔弩张。还是那句话,适可而止。''她将选好的白菊放进青金色福字软枕中,问道,``昨夜阿箬怎么样?烧得厉害么?''

惢心想了想道:``吃了许太医开的药,前半夜烧得厉害,一直要水喝,后半夜就安静多了。''

如懿凝神片刻,忧然叹了口气:``惢心,这些年我是不是宠坏阿箬了?''

惢心斟酌着词句,慢慢道:``阿箬姐姐是小主的陪嫁,小主疼她也是应该的。''

如懿捻着指尖的白菊慢慢地揉搓着,清香的汁液便沾染上了细白的手指,她沉吟着:``阿箬也到了指婚的年纪了,我想着\ldots\ldots{}''

惢心便露了一个甜甜的笑:``阿箬姐姐好福气。''

如懿叹口气,断然道:``不是我不想留她,只是阿箬的性子,宫里是断断容不得了。不如趁着青春正好,送出宫打发了配人吧。''她想了想,``阿箬到底跟了我这些年,婚事上必得上心,不能造孽。等哪日我额娘入宫,我得托付她去外头打听了,给阿箬安排个好人家。''

惢心有些意外:``小主不是想给阿箬指个御前当差的侍卫么?''

如懿心下愀然,摇头道:``原这么打算,本来能指个在宫中当差的侍卫是最好的,哪怕是个二等虾三等虾①,总有出头之日,也是想让她在我身边长长久久地一起。可是她的性子,若还是跟宫里牵扯关系,终究麻烦。''

惢心会意道:``小主还是替阿箬姐姐打算,若是嫁个准备外放的官员,哪怕去外头苦几年,终究也是正室的名分,少不了一份富贵的。''

如懿微微颔首,赞许地看了惢心一眼:``你说得不错。''

话音未落,只听殿门哐当一响,一个碧色的身影绕过花梨木雕玉兰花碧纱橱,直奔进来道:``小主,小主,求求您别放了奴婢出去,奴婢不想嫁人,不想离开小主!''

如懿不防着阿箬病中起来,竟在外头听着,不觉也吓了一跳,沉下脸道:``越来越没规矩了!''

阿箬含泪跪下,一脸凄楚道:``小主恕罪,奴婢不是有意偷听小主说话的。只是觉得身上好了些,所以起来给小主请安,想来伺候小主。''她原在病中,脸色白得没半分血色,额头上还缠着防风的布条,看着憔悴至深。

如懿有些不忍,便道:``你先起来吧。我也不过是一句顽话,哪里是立刻就要送你出去了,也得好好挑了人家才是。''

阿箬哭得梨花带雨:``奴婢知道,奴婢离开了紫禁城就什么都不是了。如果小主真要放奴婢出去,也请多留奴婢几年,让奴婢可以好好伺候小主。奴婢保证,无论如何,绝不再多嘴多舌给小主惹祸了。''

如懿见她如此诚恳,不觉有几分可怜。毕竟,从十二岁那年开始,阿箬便陪在自己身边,看着自己从骄纵的佐领家的格格成了皇子府邸备受宠爱不知收敛的侧福晋,又成了宫中日渐沉静安敛的嫔御之一。阿箬的骄横,隐隐带了自己从前的几分影子,那样牙尖嘴利,针锋相对,不肯轻易饶人。如懿神思恍惚地想着,那么,她所不喜欢的,到底是如今一样骄矜的阿箬,还是从前那个不知轻重的自己?

这样的念头不过一瞬,便吓到了自己。如此想来,阿箬的错失,也有自己的过错了。那么,她如何还能怪阿箬?

如懿伸出手,怜惜地扶起她:``地上凉,起来吧。''

阿箬哀哀地哭着,求道:``小主不答应,奴婢便再不起来了。''

如懿只得笑道:``宫女出宫的年纪是二十五岁。只要你愿意,便留到二十五岁再走吧。''

阿箬的眼中闪过一丝亮光:``真的?那奴婢多谢小主了。''她慌不迭地又要行礼相谢,如懿挽住她手,温和道:``去吧,好好去养好身子。''

阿箬含了一丝难得的温和谦卑的笑,告退出去。只是在转身的瞬间,她将这缕笑暗暗咬啮成了唇边一个不肯褪去的印子。

紫禁城的秋凉总是显得有些短暂。秋风吹黄了枝头青翠郁郁的叶,便毫不留情地带着它们一同坠落在地,零落成泥碾作尘灰。冬寒伴随这日益光秃的枝丫不动声色地入侵,紫禁城开始进入了漫长的冬季。

空气里永远浸淫着干燥而寡淡的寒冷气息,所以大朵大朵养在清水中的水仙便格外讨人喜欢,香得欲生欲死,散发出湿润而缱绻的气味。宫室内的温度永远要比室外温暖缱绻,仿佛暖洋的春天总未曾离去。但这样的温暖亦是寂寞的,让人离不开又舍不得走远。在这寂寞里,不期而至的冬雪便叫人格外地心生温柔,就连那些棱角分明、生硬硌人的宫墙青砖,那些凌厉如翅的卷翘飞檐,亦少了许多平日的巍峨疏冷,生出几分难得的被雪覆盖后的静谧与安详。

天气渐冷,除了每日必须去的晨昏定省,如懿并不太出门。只是隐隐约约听着永和宫不太安宁,她便也随众去看了几次玫贵人。因是头胎,前三个月玫贵人的反应便格外大,几乎是不思饮食,连太后亦惊动了,每隔三五日必定送了燕窝羹来赏赐。到了三月之后,她渐渐慵懒,胃口却是越来越好,除了御膳房,嫔妃们也各自从小厨房出了些拿手小菜送去,以示嫔御之间的关切,亦是讨好于皇帝。太医每每叮嘱玫贵人要多吃鱼虾贝类,可以生出聪明康健的孩子,她便也欣然接受,每一食必有此物。旁人也还罢了,如懿便吃了些苦头。只因她的延禧宫外离着宫人们进出运送杂物的甬道最近,宫外送进新鲜鱼虾,自苍震门、昭华门而进永和宫,必定要经过她的延禧宫,一时间鱼虾腥味,绵绵不绝。

如懿也不敢多言,只是让宫人们多多焚香,或供着水仙等祛除气味。玫贵人胃口虽好,嘴角却因体热长了燎泡,又跟着牙齿酸痛,皇帝心疼不已,每隔一日必去探望,太医们也跟着往来不绝,简直热闹得沸反盈天。

这一日如懿与海兰、绿筠相约了去探视玫贵人,她正捂着牙嘤嘤哭泣,嘴角上的燎泡起了老大的两个,涂着薄荷粉消肿。她见三人来,便一一诉说如何失眠、多梦、头昏、头痛,时有震颤之症,又抱怨太医无术,偏偏治不好她的病。听得一旁候着的几个太医逼出了一头冷汗,忙擦拭了道:``贵人的种种症状,都是因为怀胎而引起,实在不必焦灼。等到瓜熟蒂落那一天,自然会好的。''

绿筠是生养过的人,便含笑劝道:``怀着孕是浑身不舒服,你又是头胎。方才听你这样说,这些不适多半是体热引起的,那或许是个男胎呢。''

玫贵人这才转怒为喜,笑道:``纯嫔娘娘不骗嫔妾么?''

如懿笑道:``旁人说也罢了。纯嫔是自己生育过阿哥的,必不会错。''

海兰亦道:``我记得纯嫔姐姐怀着三阿哥的时候也总是不舒服,结果孩子反而强健呢。''

众人安慰了玫贵人一番,便也告辞了。出门时纯嫔想着今日是初一,便邀了如懿和海兰一起去阿哥所看三阿哥永璋。如懿想着正好到了时辰去接永璜下学,便推托了。

去尚书房便要抄近路经过御花园,夏日里莲叶田田,青萍丛生的菡萏池只剩下了几脉枯叶残梗,落寞地宁静着。

注释:

①二等虾三等虾:代指二等侍卫三等侍卫。

\hypertarget{ux7b2cux4e09ux5341ux7ae0-ux7578ux73e0}{%
\chapter{第三十章 畸珠}\label{ux7b2cux4e09ux5341ux7ae0-ux7578ux73e0}}

冬日里天黑得早,此时御花园中已经无人走动。如懿才欲带着惢心绕过假山莲池,忽听得咕咚一声巨响,旋即便是水花四溅的声音。

如懿一怔,立即明白过来,失声道:``不好,是有人落水了!''

冬日天色黑蒙蒙的,眼前又枝丫交错,和着半壁假山掩映,遮去了大部分视线。如懿听得动静,心下本是慌乱,忙绕过假山跑到水边。池中扑腾的水花越来越小,却无一点呼救之声,三宝吓了一跳,赶紧喊起来:``救人哪------''

如懿立刻喝道:``喊什么救人,等人来还不如自己救啊!''

三宝咬了咬牙,也顾不得水寒彻骨,霍地往水中一跳,拼命朝着水波扬起处游去。很快三宝从水里捞出个水淋淋的人来,她犹自咳嗽着喘息,如懿心头一松,知道是还有活气,忙唤了惢心一起将她扶到地上平躺。朦胧中只看那女子一身宫女服色,倒颇有身份。惢心举过灯笼一照她的脸,不觉惊道:``小主,是莲心!''

如懿看清了莲心的面孔也是大惊,转念间已经平复下来,看她浑身是水,胸口微弱地起伏着,一时说不出话来。如懿使一个眼色,和惢心拼命地按着她胸口,将腹中的水控出来。

三宝冷得浑身发抖,转身就道:``小主,奴才去请太医!''

如懿喝道:``糊涂!''她静一静,``离这儿最近是养性斋,那儿没人,你赶紧过去生上火盆烤着,然后找附近庑房的太监换身干净衣裳。记着,不许声张!''

三宝立刻答应了小跑过去。

如懿与惢心使劲按了一会儿,只见莲心口中吐出许多清水来,眼睛睁开,眼珠子也慢慢会动了。她呆呆地瞪了半天眼睛,终于迟疑着问:``娴妃\ldots\ldots{}''

如懿松了口气,将自己身上的大氅脱下披在她身上:``会说话就好了。''她看四下无人,便道,``惢心,这里风太大,莲心这个样子不能见人,送她去养性斋。''

惢心答应着,半扶半抱着惢心往养性斋去。养性斋原是御花园西南的两层楼阁,因平素无人居住,只是太监宫女们打扫了供游园的嫔妃们暂时歇脚所用,所以一应布置倒还齐全。三宝已经生好了几个火盆,见她们进来,方才告退出去换衣裳。如懿看莲心坐下了,方道:``惢心,你去宫里找身干净的宫女衣裳给莲心换上,记着别声张。''

惢心连忙掩上门去了。

如懿道:``所以,你就不想活了?''

``这样的日子过一天还不如早死一天,我既然不能自杀,那总能失足落水吧!死有什么可怕的?早死早超生罢了!''

如懿凝视着她:``所以,你新婚那夜,庑房里发出的尖叫声\ldots\ldots{}''

莲心悲切的哭声如同被胡乱撕裂的布帛,发出粗嘎而惊心的锐声:``是!从我被赐婚做他的对食那天起,我的日子就完了。白天是皇后跟前最得脸的大宫女,是副总管太监的对食,看着风光无限,人人讨好。可是到了夜里,只要天一擦黑我就害怕。他简直不是人,他是禽兽!少了一嘟噜东西还要强做男人的禽兽!''

如懿道:``他打你?''

莲心忍着泪,切齿道:``打我?哪个宫女从小不挨打的,我怕什么?''她撩起衣袖,卷得高高的,手肘以下完好无缺,并不妨碍莲心劳作时露出戴着九连银镯并翠玉镯的手腕。可是手肘以上不易露出的地方,或青或紫,伴着十数排深深的牙印,像是有深仇大恨一般,那些牙印直咬进血肉里,带着深褐色的血痂。尚未痊愈的地方,又有新的咬伤。几乎没有一寸皮肤完好。

如懿看得触目惊心:``王钦这样恨你,他何必还要向皇后求娶你?''

莲心冷笑,眼泪在她眼角凝成了冰霜似的寒光:``因为他需要一个女人,一个白天带给他体面的女人,晚上可以任他折磨的女人。''她呵呵冷笑,发出夜枭似的颤音,``他不会亲女人,所以就咬。他没有办法像一个男人那样,就拿针扎我的身体,是身体的每一寸。他极力想做一个男人,补上他所缺失的东西,就拿各种能想到的东西捅我。我求他,我哭,他却愈加高兴!娴妃娘娘,这样的日子,你知道我每天是怎么熬过来的么?''

如懿心里一阵一阵发寒,她不敢去想象,只要一想,就觉得无比恶心,连带着心肝肺脏都一起发抖。可是偏生,莲心就活在那样的日子里,挣扎沉浮,不能托生。莲心看着她捂着胸口,忽然生了一点悲凉的笑意:``娴妃娘娘,您的脸色和您的恶心告诉我,您是在想象我过的苦日子。多谢您,因为我曾经尝试着告诉皇后娘娘,可是她才听了一句就念了阿弥陀佛,要我嫁鸡随鸡嫁狗随狗。还好,您是替我想着的。''

如懿忍耐着腹中强烈的翻江倒海,极力不把那种血腥的画面与莲心连在一起,而是由衷地冒出更大的惊诧:``皇后居然知道?她不肯帮你?''

莲心瑟缩着,眼里只剩下绝望的灰烬:``是。皇后娘娘愿意把我嫁给王钦,也是为了多一层保障,知道皇上的所思所想。如果我不仅做不到这个,还要皇后娘娘出手救我,她怎么肯呢?她是绝对不会为了我和王钦撕破了脸的!''她的泪有无尽的堕落与绝望,仿佛掉到了崖底的人,再无力爬起来,``王钦和皇后娘娘都告诉我,不能自戕,否则会连累家人。可我实在活不下去了,那失足落水总是可以的吧?''

如懿屏住心气,沉声道:``如果王钦不愿意你死,不愿意少了他那点乐子,不管你是自杀还是失足,他都会当你是自杀,拖着你全家一起下地狱。如果猛兽伤人,你以身饲兽之后它还是要吃你的家人,你说应当怎么办?''

莲心眼中微微一亮:``您是说,杀了猛兽,以绝后患?可是我只是个宫女,能有什么办法?''

如懿凝视着她,语意沉着:``任何一个想要求生的人,都会这样想。王钦折磨你,伤害你,他固然无耻,也是看准了你不敢反抗,羞于声张。既然如此,你就假装驯服。因为想要持刀杀兽,你既然力气不够,就可以挖陷阱,下毒药,甚至借别人的手去杀了他。这样和自己撇得干干净净,也不会连累了你,让你受人嘲笑。''

莲心有些胆怯,惶惑道:``娴妃娘娘以为奴婢能做到?''

如懿笑道:``你连死都不怕,还有什么做不到的?只是任何事都要忍耐为先,你若没有耐心,忍不住,那便什么事情都做不成。''

莲心似乎十分惧怕王钦,迟疑良久仍说不出话。正踌躇着,惢心抱着一身干净衣裳进来了:``小主,奴婢已经尽量选了一身和莲心姑姑今日穿着相似的衣裳,请姑姑即刻换上吧。''

如懿看她一眼,示意惢心解下莲心身上披着的大氅。如懿转身离去,缓缓道:``头发已经烤得快干了,是要换上干净衣裳还是任由自己这么湿着再去跳一次莲池,随便你。''

如懿走了几步,正要开门出去,只听莲心跪倒在地,磕了个头,语气决绝如寒铁:``多谢娴妃娘娘的衣衫,奴婢换好了就会出去。''

如懿不动声色地一笑,也不回头,径自走了出去。惢心在身后掩上门,如懿低低道:``去告诉李玉准备着,他的出头之日就要来了。''

尚且等不到李玉的出头之日到来,腊月的一天,玫贵人突然早产了。如懿清晰地记得,那是一个深夜。

她坐在暖阁里,看着月光将糊窗的明纸染成银白的瓦上霜,帷帘淡淡的影子烙在碧纱橱上。阁内只有铜漏重复着单调的响声,一寸一寸蚕食着时光。皇帝正在专心地看着内务府送来的名册,如懿则静静地伏在绷架上一针一针将五彩的丝线化作雪白绢子上玲珑的山水花蝶。暖阁里静极了,只能听到蜡烛芯毕剥的微响和镂空梅花炭盆内红箩炭清脆的燃烧声。

绣得倦了,如懿起身到皇帝身边,笑道:``向例不是生下了孩子内务府才拟了名字来看的么?如今玫贵人还有一个月才生产,尚不知道是男是女,怎么就拟好名字了呢?''

皇帝不自觉便含了一分澹澹的笑色,道:``太医说了,多半是个阿哥。自然,公主也是好的。倒也不是朕心急,是内务府的人会看眼色,觉得朕对登基后的第一个孩子特别期许,所以先拟了名字来看。''

如懿道:``内务府既然知道皇上的期许,那一定是好好起了名字的。''

皇帝揽过她道:``你替朕看看。''皇帝一一念道,``阿哥的名字拟了三个,永字辈从玉旁,永琋、永珹、永珏;公主的封号拟了两个,和宁与和宜,你觉得哪个好?''

如懿笑着推一推皇帝:``这话皇上合该去问玫贵人,怎么来问臣妾呢?''

皇帝笑道:``迟早你也是要做额娘的人,咱们的孩子,朕也让你定名字。''

如懿笑着啐了一口,发髻间的银镂空珐琅蝴蝶压鬓便颤颤地抖动如发丝般幼细的翅:``皇上便拿着玫贵人的身孕来取笑臣妾吧。''

皇帝道:``朕原也想去问问玫贵人的意思。但是她身上一直不大好,总说头晕、嘴里又发了许多燎泡,一直不见好。朕只希望,她能养好身子,平平安安生下孩子来便好了。''

如懿带了几分娇羞,指着其中一个道:``皇上既然对玫贵人的孩子颇具期望希翼,那么永琋便极好。若是个公主,和宁与和宜都很好,再拟个别致的闺名就更好了。''

皇帝抚掌道:``那便听你的,朕也极喜欢永琋这个名字。''

铜漏声滴滴清晰,杯盏中茶烟逐渐凉去,散了氤氲的热气。如懿依偎在皇帝怀中,听着窗外风动松竹的婆娑之声,心下便愈生了几分平和与安宁。

如懿与皇帝并肩倚在窗下,冬夜的星空格外疏朗宁静,寒星带着冰璨似的光芒,遥迢星河,仿佛伸手可摘。如懿低低在皇帝身畔笑道:``在潜邸的时候,有一年皇上带臣妾去京郊的高塔,咱们留到了很晚,一直在看星星。就是这样,不敢高声语,恐惊天上人。''

皇帝吻着她的耳垂,自身后拥她:``如今在宫里,出去不便。但是往后,朕答应你,会带你游遍大江南北。''

如懿依依道:``皇上最喜欢江南的柔蓝烟绿、疏雨桃花。''

皇帝清朗的容颜间满是向往之情:``朕说的,你都记得。小时候听皇阿玛讲佛偈,一口气不来,往何处安身立命?朕想来想去,便是往山水间去。最好的山水,便是在江南。所以朕想去的地方,一定会有你。我们,迟早会去江南的。''他说着,瞥见如懿方才绣了些许的刺绣,``手艺越发精进了,可是那时候为什么送朕那么一方帕子,一看就是你刚学会刺绣的时候绣的。''

如懿的笑意如枝头初绽的白梅,眼中含了几分顽皮之色:``送了那么久,皇上到现在才来问。是不是觉得不好,早就扔了?''

皇帝笑着捏一捏她的鼻子:``是啊,就因为不好,所以得珍藏着。因为以后你的绣功只会越来越好,再不会变成那样子了。''

如懿低低道:``虽然不够完美,但那是最初的心意。青樱,弘历。''

皇帝无声地微笑,似照上清霜的明澈月光,又如暮春时节带着蔷薇暗香的风,暖而轻地起落。

庭院内盛满深冬的清澈月光,恍若积水空明。偶尔有轻风吹皱一片月影,恰如湖上粼粼微波,漾起竹影千点。如懿看着窗外红梅白梅朵朵绽放,冷香沁人,只是默默想着,这样,大约也是一段静好岁月了吧。

她正想着,却听外头响起了一阵急促的步伐,仿佛有低低的人声,如同急急惊破湖面平静的碎石。

如懿微微不悦,扬声道:``谁在外头?''

进来的却是大太监王钦,这么冷的天气,他的额头居然隐约有汗水。如懿看到他的脸便想起莲心身上的伤,满心不舒服地别过头去看着别处。王钦急得声音都变调了:``皇上,永和宫的人来禀报,玫贵人要生了!''

皇帝陡然一惊,脸色都变了:``太医不是说下个月才是产期么?''

王钦连忙道:``伺候的奴才说用晚膳的时候还好好的,还进了一碗太后赏的红枣燕窝羹。用了晚膳正打算出去遛弯儿,结果出门从墙头跳下一只大黑猫,把玫贵人惊着了,一下子就动了胎气。''

皇帝的鼻翼微微张合,显然是动了怒气,喝道:``荒唐!伺候的人那么多,一点也不周全!''

如懿忙劝道:``皇上,现在不是动气的时候。赶紧去看看玫贵人吧。''

皇帝连忙起身,如懿替他披上海龙皮大氅。皇帝拖住她的手道:``你跟朕一块儿去。''

如懿沉静地点头:``臣妾陪着皇上。''

永和宫离延禧宫最近,自延禧宫的后门出去,绕过仁泽门和德阳门的甬道便到了。尚未进永和宫的大门,便已听到女人凄厉的呼叫声,简直如凌迟一般,让人不忍卒闻。

皇帝握着如懿的手立刻沁出了一层薄薄的冷汗,滑腻腻的。如懿握了自己的绢子在皇帝手中,轻声道:``女人生孩子都是这样的。纯嫔那时候也痛得厉害。''

皇帝有些担忧,道:``怎么朕听着玫贵人的叫声特别凄厉一点?''

两人急急进了宫门,宫人们进进出出地忙碌着,一盆一盆的热水和毛巾往里头端。皇上拦住一个人道:``玫贵人如何了?太医呢?太医来了没有?''

那人急得都快哭了:``太医来了好几个,接生嬷嬷也来了,可贵人的肚子还是没动静呢。''

皇帝急道:``没动静就痛成了这样?快去叫个太医出来,朕要问他。''

那人答应着跑进去,很快领了一个太医出来,正是太医院院判齐鲁,齐鲁来不及见过皇帝,皇帝便道:``你都在这儿了,是不是玫贵人不大好?''

齐鲁忙道:``皇上安心。早产一个月不是大事,只是\ldots\ldots 只是胎儿还下不来,微臣要开催产药了。''

皇帝吩咐道:``你赶紧去!好好伺候着玫贵人的胎,朕重重有赏!''

齐鲁忙赶着进去了。不过须臾,皇后也带着人到了。皇后急匆匆问了几句,便吩咐素心道:``多叫几个人进去伺候着,不怕人多,就怕人手不够。''

素心立刻去安排了。皇后低低道:``皇上,臣妾听闻玫贵人是被黑猫惊着了。黑猫晦气,不太吉利。臣妾为了玫贵人能顺利产下孩子,已经请宝华殿的师父诵经祈福,保佑母子平安。''

皇帝微微松一口气,欣慰道:``皇后贤惠,一切辛苦了。''

皇后含了端肃的笑容:``臣妾身为六宫之主,一切都是分内的职责。''

里头的叫声愈加凄惨,恍如割着皮肉的钝刀子,一下又一下,在寂静的夜里,听得人毛骨悚然。伺候着的宫女不断地进出,端出一盆盆染着彻骨腥气的血水。

皇帝的脸色越来越难看,几乎按捺不住,往前走了一步。皇后立刻挽住了皇帝的手臂,语气柔和而不失坚决:``皇上,产房血腥,不宜入内。''

皇帝想了想,还是停住了脚步。

王钦忙劝道:``皇上,外头冷,不如去偏殿等着吧。''皇帝低低``嗯''了一声,攥着如懿的手阔步走进偏殿。只有如懿知道,他那么用力地握着自己的手,以此来抵御那可怕的叫声带来的惊惧。

等待中的时光总是格外焦灼,虽然偏殿内生了十数个火盆,暖洋如春,但掺着偶尔出入带进的冰冷寒气,那一阵冷一阵暖,好像心也跟着忽冷忽热,七上八下。

也不知过了多久,终于听到一声微弱的儿啼。

皇帝遽然站起身,王钦已经满脸堆笑地迎了进来:``皇上,皇上,您听,孩子生下来了。''

皇帝脸上的紧张一扫而空,取而代之的是无限的喜悦。他疾步走到外头,向着从寝殿内赶出来的齐鲁道:``如何?是阿哥么?''

齐鲁说不上话来,只是嗫嚅着不敢抬头,皇帝的笑意微微淡了一些:``是公主也不要紧。''

皇后微微皱眉,侧耳听着道:``怎么哭声那么弱?臣妾的永琏出生时,哭声可响亮了。''

话音未落,只听寝殿里头一声恐惧的尖叫,竟是孩子母亲的声音。

皇帝不知出了何事,便吩咐道:``王钦,去把孩子抱出来给朕看看。''

王钦紧赶着去了,不过片刻,便抱出一个襁褓来,可是王钦却抱着襁褓,站在廊下不敢过来。

皇帝当即变了脸色:``怎么回事?''

王钦面色发青,抖着两腿道:``皇上,玫贵人她昏过去了。她\ldots\ldots{}''

皇帝只管道:``那孩子呢?快给朕看看。''

王钦迟疑着挪到皇帝跟前,却不肯撒手。皇后与如懿对视一眼,隐隐都觉得不好。

王钦扑通跪下了道:``皇上,您不管看到了什么,您都稳稳当当地站着。您还有千秋子孙\ldots\ldots{}''

他话未说完,皇帝已经伸手拨开了襁褓,撒金红软缎小锦被里,露出孩子圆圆的脸,分外可爱。皇帝情不自禁地微笑道:``不是挺好一个孩子么?''他伸手微微抖开襁褓,王钦几乎是吓得一哆嗦,皇帝触目所见,几乎是愣在了当地,碰着襁褓的手似被针扎了似的,立刻收了回来。如懿发觉不对,一眼望去,吓得几乎一个踉跄,连惊叫声也发不出来了。

襁褓中的孩子,四肢瘦小却腹大如斗,整个腹部泛着诡异的青蓝色。更为可怕的是,孩子的身上,竟长着一男一女两副特征。

\hypertarget{ux5982ux61ffux4f20-ux7b2cux4e8cux518c}{%
\part{如懿传 第二册}\label{ux5982ux61ffux4f20-ux7b2cux4e8cux518c}}

\hypertarget{ux7b2cux4e00ux7ae0-ux5ef6ux7978}{%
\chapter{第一章 延祸}\label{ux7b2cux4e00ux7ae0-ux5ef6ux7978}}

四周静得有些骇人,偶尔穿过庭院的风声,像不知名的怪物隐匿在黑暗中发出的低沉的嘶鸣。所有的人都怔在了原地。心头的震撼如惊涛骇浪,冲得如懿微微踉跄一步,下意识地捂住了自己微张的嘴,将那几乎要喷涌而出的惊呼死死扼住。

襁褓中的孩子,四肢瘦小却腹大如斗,整个腹部泛着诡异的青蓝色。更为可怕的是,孩子的身上,竟长着一男一女两副特征。

皇帝吓得双手一颤,几乎是本能地把孩子推了出去。幸而王钦牢牢接住了,他也是一脸惧怕,双手哆嗦着不知该如何处理手中的孩子。皇后一时也看清了,惊得低呼一声,花容失色,大为惊惧,紧紧攥住了皇帝龙袍的袖子。如懿不知道自己的脸色是否亦如皇后一般难看,她只觉得自己的心突突地用力跳着,仿佛承受不住眼前所见似的。她与皇室羁绊多年,虽也知道后宫孕育子嗣往往艰难,孩子多有夭折,可是大清开国百年,从未有过这样的骇事!

那孩子,分明有一张与别的婴儿无异的面孔,小小的潮红的脸上,露出一丝满足的笑容。他的身体在襁褓里蠕动着,并未觉得自己与旁的孩子如此不同。可是他偏偏雌雄未辨,惊世骇俗。

里头隐约响起女人昏迷醒来后疲倦的声音:``孩子,我孩子呢?\textgreater\_\textless''

皇帝的身体剧烈一震,像受了什么无法承受的力量似的,死灰般的面庞上唯有一双惊恐而哀伤的眸子,那双眸子里的哀伤因为触及孩子的面容而如遇见寒雪的青瓦间的冷霜,转瞬被覆盖不见,只余下刺骨寒冷的惊恐与嫌恶。

女人的声音在里头再度响起,带着期盼与希望:``把孩子抱来我看看\ldots\ldots{}''

一片静寂,没有人敢回答。

皇后迅疾反应过来,带着冷冽的决绝。她转首,发髻间一点银凤垂珠的流苏簪闪过一丝寒星般的光芒,划破深蓝至抹黑的天际,转瞬不见。她的语气没有任何柔软与迟疑,决绝道:``皇上,这是孽障,是不祥的妖物,绝不能留!''

皇帝微微一怔,茫然地点点头,皇后旋即看着王钦,一字一字吐出:``你去安排,告诉所有人,玫贵人生下的是个死胎,死胎不祥,立即埋了它!''她说到那个``它''字时,冷漠而不带任何感情,仿佛那个孩子,就是一个不值一顾的小小牲畜,随时可以将他鲜活的生命掐去。

如懿实在有些不忍,低声道:``皇上,这孩子也没有别的问题,只是多了\ldots\ldots 不如请太医看看,看能不能除去其中多余的那个\ldots\ldots{}''

皇帝看着孩子小脸粉红的憨态,一时也有些动摇。皇后立刻转过脸来,照着如懿的脸便是一耳光。那耳光来得太快,几乎叫人反应不过来,如懿硬生生受了这一巴掌,只觉得脸上热辣辣的,胜过了一切痛楚。皇后冷冷看着她,那双眼睛如养在清水寒冰里的一双黑鹅卵石,看着清透乌黑,却有让人浑身一凛的彻骨寒意:``娴妃,你做错什么事说错什么话本宫都不会怪你。但是这一巴掌,你要好好记住,这个孩子是不祥的孽障妖胎。你若再容旁人知道,流传出去伤害圣誉与大清的祥瑞,本宫就是杀了你也不为过。''

脸上的伤痛一点一点逼到肌理深处,痛得久了,没有挨打的另一边脸孔反而有一种奇异的冰冷的触觉,仿佛是滴水檐下的冰柱一点一点化下水来滑在面颊上,冰得寒毛倒竖,凛冽刺骨。她明白那孩子是救不得了,也不敢捂着脸,只得屈膝欠身:``臣妾失言,请皇后娘娘恕罪。''

皇后扬了扬脸示意她起来。皇帝定了定心神,仿佛找到了主心的一缕神魂,极力平静着问:``既然如此,皇后的意思是\ldots\ldots{}''

皇后微微欠身,语气恭和而安稳:``玫贵人不幸,诞下死胎,无福为皇上绵延后嗣,还请皇上节哀。但愿玫贵人来日有福,还能为皇家开枝散叶,再续香火。''皇后瞟了一眼王钦怀中的孩子:``既然是个死胎,就好好处置了吧。王钦,这件事不许再有其他人知道。至于已经知道的人,除了本宫、皇上和娴妃,就是你了。''

王钦悚然一凛,立即答应道:``是。奴才明白了。''

如懿看他转身离去,心下亦明白,这个孩子,断断是活不了了。

皇帝疲倦地摆摆手:``皇后,你和娴妃去安慰一下玫贵人吧,朕累了。''

皇后知道皇帝此时并不愿与玫贵人相见,或许此后,皇帝都不会再想与她相见了,于是便温婉劝道:``皇上累了一晚上,一定也倦了。不如去臣妾宫里稍事休息,臣妾准备了一些五仁参芪汤,原是留着自己喝安神的,皇上赶紧去喝一碗定定神吧。''

皇帝的目光扫过如懿的面庞有些歉意:``那朕先去皇后宫中了。''

如懿亦知,今晚皇帝心里一定不好受,皇后万事稳如泰山,皇帝在她那儿亦是好事。于是她欠身相送:``皇上安心歇息,臣妾会与皇后娘娘好生安慰玫贵人的。''

皇帝点点头,转身离去。皇后看了如懿一眼,伸手轻轻抚上她的面颊,温言问:``痛不痛?''

如懿身体微微一缩,有些难以抑制的畏惧,忙道:``谢皇后娘娘关怀,方才是臣妾失言了。''

皇后叹口气道:``方才那种情况下,这个孩子是断断留不得了。万一皇上起了不舍之心,一时难以决断,往后日日看到那孽障,岂不更加烦心。且事情一旦传出去,这不男不女的妖孽,会让皇室蒙上何等羞辱?还是快刀斩乱麻的好。''

如懿心口堵得慌,像是被谁塞了一把火麻仁一般,喉头又酸又胀,语气却竭力维持着平和从容:``是,臣妾受教,是臣妾糊涂了。''

永和宫寝殿内的哭闹声越来越凄厉,是玫贵人,急着要看她的孩子却无人应对后的焦灼与不安。皇后叹口气:``走吧,如何劝住她,这便是咱们的事了。''

如懿跟着皇后推门进去,布置得精致秀雅的寝殿内颇有琴书静韵,仿佛在那份喧嚣的恩宠之下,蕊姬亦有着一份自己的清新雅致,赢得皇帝的垂眸。可是此时此刻,殿中沉积的百合香气味底下掺着浓郁不退的血腥气和潮腻的来自产妇头顶与这个季节格格不入的大汗淋漓的味道。

皇后与如懿甫一进殿,便见玫贵人惊慌失措地挣开宫人们的扶持,从床上跌爬下来,满面泪痕地扑倒在皇后脚下,泣道:``皇后娘娘,他们不让臣妾见孩子!他们都拦着臣妾!''她的慌张与不安明白无误地铺写在她娟丽清秀的面孔上。``皇后娘娘,您告诉臣妾,孩子是不是不大好?''皇后短暂的沉默让她有些慌不择言,``长得难看些不要紧,只要是全的,全的。皇后娘娘,孩子不会缺了什么吧?''

怎么会缺?分明是多了些许不该有的东西。

皇后伸出双手扶住她,缓缓地道:``玫贵人,你要节哀。''她瞥一眼如懿,如懿会意,只得道:``孩子生下来就是个死胎。皇上吩咐,立刻送孩子\ldots\ldots 回去了。''

玫贵人浑身打了个激灵,像是有惊雷从她头顶毫不留情地碾过,惊得她浑身战栗不已。她瘫软在地,哭号不已:``不会的,不会的!孩子生下来的时候,我还明明听到他的哭声,怎么会是个死胎呢?''

``玫贵人,你当真是听错了。孩子一生下来就是没了气息的,怎么会哭呢?''皇后怜悯地看着她,然后缓缓地目视宫中诸人,``你们当时都在玫贵人身边,告诉玫贵人,孩子是不是生下来就是没有声息的?''

皇后的目光和缓如往日,可是目光所及之处,无人敢不跪下,俯首低眉道:``是,皇后娘娘说得是,还请贵人节哀。''

如懿低低道:``你要是伤心,不如请宝华殿的师父来诵经祈福,也好送孩子早登极乐。''

玫贵人在泪眼蒙眬里醒过神来:``请皇后娘娘好歹告诉臣妾一声,这孩子到底是男是女\ldots\ldots{}''

皇后微微一怔,有些为难地看了如懿一眼,如懿犹豫着道:``是个\ldots\ldots{}''

皇后旋即道:``是个小公主,所以你也别太伤心了。娴妃说得对,是要请宝华殿的师父好好来替小公主诵经超度。''皇后沉声吩咐众人:``这些日子玫贵人要坐月子补养身体,不许她走动见风,只许宝华殿的大师进偏殿祈福诵经,其余任何人都不许来打扰玫贵人休养。''

如懿一听,便知皇后对玫贵人已是形同软禁。她无能为力地看着沉浸在悲痛之中的玫贵人,随着皇后的步伐一起离开。

寒冷的冬夜哈气成冰,如懿远远听着寝殿里传出撕心裂肺的哭声,心底的微凉如同被月光映照的茫茫雪野,凄寒而明亮的冷。她从大氅中伸出手来,接住从无尽的暗色夜空中落下的清冷雪花。这样冷清而小朵的雪花,落在灯火通明的庭院中,伴着玫贵人无助而悲切的哭声,冬夜的寒意,无声无息入骨侵来。

玫贵人骤然丧女,不只合宫惊讶,连太后亦颇为伤心。宫中人心浮动,慧贵妃亦在背后私语,玫贵人是骄奢享福太过,才折了孩子的阳寿。流言如沸,幸而如皇后所言,永和宫不许外人出入,玫贵人才免了惊扰,可以安心休养。但玫贵人伤心如斯,皇帝却也再未踏足永和宫一步探望安慰。太后几度欲问皇帝玫贵人死胎之事,皇帝也不过含糊了几句,便过去了。

这一日已是玫贵人丧女的半月之后,如懿陪皇帝在养心殿暖阁中闲话。皇帝的神色始终有些郁郁,对着窗外雨雪霏霏,兀自沉浸在默然的悲戚中,一遍一遍地抄写着《往生咒》。雨雪天气的黄昏也显得格外暗沉,如懿见皇帝身前的几案上犹搁着一壶残酒,一盏孤杯,数支白烛燃着几簇昏黄的冷焰,每一跳动,都溅起抽搐般的影光。皇帝穿着一身缂金云白狐皮龙袍,那龙袍原是银白的底色,簇了雪白的狐皮滚边,连缂金的绣龙图案亦显得清冷了不少。皇家一向讲究色调清雅富贵,皇帝亦少穿这样的素色。如今这般打扮,也不过是心情的缘故罢了。

空气里残留着冷酒的余香,如懿卷起衣袖,轻轻为皇帝研磨墨汁,轻声道:``皇上要喝酒也先让人温一温,冷酒太伤胃。或者,与人对酌说说话也是好的。''

皇帝并不抬头,淡淡的语调中颇有伤感之意:``自饮自酌,冷酒才有味道。何况殿中熏得那样暖,再喝热酒,就失了意趣。''

如懿静静磨完墨,闻着殿中的龙涎香有点淡了,便让李玉带着人捧了香炉下去,又用紫铜拨子拨开镂空鹤纹铜炉的一角,添入一把紫檀色的苏合香。

皇帝只低头专心抄写,问道:``怎么不用龙涎香了?''

如懿道:``苏合香能通窍辟秽,开郁豁痰,冬日里用最好。''

皇帝搁下笔叹了口气,苦笑道:``通窍辟秽,开郁豁痰?朕知道你是好心,可是朕心气郁结,岂是一把苏合香能解的?''

如懿将皇帝所抄的《往生咒》一一理好,温然道:``皇上抄了这么多《往生咒》供宝华殿诵经超度所用,臣妾就知道皇上心里还是在意那个孩子的。''她小心觑着皇帝的神色:``皇上常到延禧宫看望臣妾,永和宫与延禧宫不过数步之遥,皇上何不去看看玫贵人,稍作安慰?''

皇帝眉心的悲色如同阴阴天色,凝聚不散:``近乡情更怯,更不知该如何安慰彼此?反而是两下里伤心。''他静一静:``幸好玫贵人还不知道那孩子的样子\ldots\ldots{}''

如懿忙道:``皇后娘娘吩咐过,一律不许走漏风声。那日为玫贵人接生的太医与嬷嬷,都已经打发出去了。但凡有可能见过小\ldots\ldots 公主身体的宫人,也都已经拨去了热河行宫,不许再在宫里伺候。''

皇帝微微颔首:``皇后想得很周全。此事不祥,朕连太后也不敢告诉周详。''

如懿点头道:``如今宫里见过那孩子的,只有皇上、皇后、臣妾与王钦。再无第五人了。''

皇帝静默地吁出一口气,正要提笔再写,只听外头两声叩门声响,却是王钦在外道:``皇上,永和宫玫贵人送了东西来请圣上过目,皇上您要不要看一看?''

皇帝犹豫片刻,便搁下笔道:``拿来朕瞧瞧吧。''

王钦答应着推门进来,却是在黄鹂鸣枝多子多福红漆托盘里搁着一叠婴儿衣裳。皇帝一时未解,便问:``这是什么?''

王钦恭声道:``玫贵人说,听闻皇上辛苦手抄《往生咒》化与小公主,所以想把之前亲手做的给小公主穿的衣裳一同焚化,即便小公主在人世间穿不上一遭,到了极乐世界也不会受冻凄寒。''

皇帝的神色间闪过一丝凄楚之色,如懿便道:``皇上,玫贵人忆女心切,您还是成全了她吧。''

皇帝点点头:``朕准了,你告诉她,便留在自己宫里焚化吧。''

王钦又道:``玫贵人说,今晚亥时一刻是半个月前小公主出生的时辰,希望皇上能亲临永和宫,陪玫贵人一同焚化这些衣裳,以尽哀思。''他凑上前几步,翻起盘中的衣裳:``这些衣裳都是玫贵人亲手做的,皇上看看这针线,一定是花了不少工夫的。玫贵人慈母之心,可钦可叹啊!''他随手翻起,直露出盘底上多子多福婴儿嬉戏图来。皇帝眼中一动,本已心软,可是目光触及盘底憨态可掬的婴儿图案,不觉闪过一层蒙眬泪意,那泪意似结了薄薄一层碎冰一般,凝住了层层寒气。

皇帝问:``这个托盘是哪里来的?''

王钦赔笑道:``还能哪儿来的?是永和宫连着衣裳一同送来的。皇上要不信,送衣裳的小贵子还在殿外候着呢。''

皇帝眸中微冷,再也不看那些衣裳:``去告诉玫贵人,她还在月中,朕不宜探望,这些事她这个做额娘的一力完成就是了。''

王钦立时退下。如懿见皇帝面色不善,忙含笑问道:``伺候玫贵人的宫人真是不当心,玫贵人不能平安诞育皇嗣,他们还用这样婴儿嬉戏的图案,玫贵人看见了岂不刺心?''

皇帝颓然坐倒在椅上,长叹道:``朕一看见那些健全的孩子,便会想到玫贵人所生的孩儿,如此畸形可怖,诚如皇后所言,是孽种妖胎。偏偏玫贵人自己懵然不知,她无心所选,却让朕不得不想起那个可怕的孩子。''他握住如懿的手,神色如一个凄惶而无助的孩子:``如懿,你告诉朕,是不是朕无福失德,才会与玫贵人生下这样的孩子?是不是?''

如懿心头一搐,忙安慰道:``怎么会?皇上初登大宝,乃天命所佑。这个孩子,纯属意外而已。''

皇帝的脸贴在如懿温热的手心之上:``就是因为朕初登大宝,所以才更不安。玫贵人的孩子,是朕登基之后的第一个孩子\ldots\ldots{}''

皇帝话音未落,却听有风声伴着殿门悠长的吱呀之声一同扑入。如懿抬首,却见皇后独自站在殿门内,衣袂翩然,颇有正大仙容之姿。

她端然迈进,一步一个沉稳,定定道:``皇上安心。这个孩子的意外,完全是因为玫贵人德行浅薄,不堪承受皇上圣恩。''她行至皇帝身边,俯身将皇帝的手合在自己掌心,语气沉稳而不容置疑:``皇上已经有好几位皇子皇女,个个都聪明康健,唯有玫贵人所生与旁人有异,便可证明万恶之源在于玫贵人而非皇上。皇上大可不必挂怀。''

皇帝神色稍稍弛缓:``皇后所言,不是宽慰朕吧?''

皇后唇边的笑意让人望之心安:``是否是宽慰之词,皇上只要去阿哥所看看各位阿哥与公主,不就知道了。''

如懿知道皇后要借几位年幼的阿哥与公主开解皇上的失落,安慰他丧女之痛外,更不能述之于口的惊骇,或许眼下,这也是让皇上尽早走出颓丧之情的最好良方吧。她默然行礼,缓步退了出去。容色和缓而沉静的皇后身边,连皇帝也露出一丝难得的欣慰之色。她掩上殿门,亦掩上自己此刻的失落与怅惘。

或许,皇后终究是皇后,他可以对着自己倾吐心事,最终却是在皇后那里得到安慰。如懿看着外头寒雨纷纷,夹杂着碎雪纷乱,雨雪寒潮之中的紫禁城,亦如同自己一般失了颜色。

坐在暖轿之中良久,如懿的心事仍是翻覆如潮,不得安定,只觉得暖轿转了一重又一重,仿佛自己一颗不定的心一般,山重水复,千回百转。正苦闷间,忽而听得隐隐约约有哭泣之声传来,如懿掀起帘子,唤道:``惢心,去看看是谁在哭?''

惢心答应着转过甬道过去瞧了瞧,很快过来回禀道:``回小主的话,是永和宫的小贵子躲在角门下哭呢。''

如懿点点头,示意惢心打起伞来,吩咐道:``阿箬,你带着他们先回宫,我自己走回去便是。''

阿箬忙道:``那让他们回去,奴婢留下伺候小主吧。''

如懿道:``不必了。你去替我将案上抄写的经文收好,等下送去永和宫一并焚化,就当是我对玫贵人和孩子的一点心意。''

阿箬转身去了。如懿扶着惢心的手缓步转过甬道,果然见一所偏僻的宫殿外,小贵子正躲在角门边抱着刚才那包婴儿衣裳在抹眼泪。

如懿道:``你家小主还在坐月子,你便这样哭,若她知道了,岂不是让她伤心么?''

小贵子见是如懿,忙磕了个头请安道:``娴妃娘娘万安,奴才不是有心的。''

如懿微微点头道:``你也算个有心的了。要是在自己宫里哭,那真是让玫贵人伤心了。''

小贵子擦着眼泪呜咽道:``我们小主没了孩子半个月了,可是皇上一次也没来探望过。人人都说,皇上是嫌弃小主生了一个死胎,所以再不会宠幸她了。''

如懿心下哀悯:``即便如此,玫贵人也不会坐以待毙的,是不是?''

小贵子忙道:``小主就是怕皇上再也不来了,所以今日特地命奴才送了这些婴儿衣裳来,希望皇上可以惦念昔日之情。''

如懿翻了翻那些衣裳,摇头道:``玫贵人的心思是不错,可是这个装衣裳的托盘,是玫贵人自己选的么?''

小贵子奇道:``不是啊。奴才捧着这包衣裳来,王公公说空手拿着不像样子,所以给了奴才这个托盘装着,还说是有婴儿嬉戏图的,皇上看了也会念及玫贵人。''

``王钦?''如懿旋即明白过来,正色道,``既然这次不成,那便算了。你赶紧回去,记得以后再替你们小主送东西给皇上,再不许有这样的图样花纹了。''

小贵子尚未明白过来,但见如懿语气郑重,也知道是要紧的嘱咐,忙谢了恩赶紧去了。

惢心替如懿打着伞遮蔽雨雪相侵,低声问道:``王钦这般费尽心思,是要绝了玫贵人的宠爱啊!他一个阉人,居然有这样狠毒的心思。''

如懿扶着惢心的手缓步向前:``诚如你所说,他一个阉人,有什么好替自己这般狠毒的?不过是替他人效力而已。''

惢心悄悄望了望四周,低声道:``小主是说\ldots\ldots{}''

如懿缓缓摇头:``这一厢一直腾不出手来,看来王钦,是断断不能留了。''

惢心低低应了声``是'',牢牢扶住如懿的手臂:``雪天路滑,小主当心脚下。''

如懿沉下心气,缓声道:``我自然会当心脚下。否则如今是看旁人摔倒,以后便是自己爬不起来了。''

\hypertarget{ux7b2cux4e8cux7ae0-ux559cux5fe7}{%
\chapter{第二章 喜忧}\label{ux7b2cux4e8cux7ae0-ux559cux5fe7}}

玫贵人的失宠,似乎已成定局。因为生下的是如此不祥的``死胎'',产前的荣宠在她生育之后几乎是消弭殆尽。没有任何安慰,没有一次探视,一向花团锦簇的永和宫就此沉寂,再无一人踏足,连最为贤惠的皇后也退避三舍,不再前往。

为着怕见面伤情,皇后还是不许玫贵人离开永和宫半步,出月之后,连在偏殿祈福的法师也退回了宝华殿,唯有寂寞的风雪回声,相伴同样寂寞而悲伤的玫贵人。

连着好几日是难得的晴好天气,又逢旬日,宫嫔们便也随着帝后一同前往慈宁宫请安。太后见莺莺燕燕坐了满殿,也稍许有了些笑容,支颐含笑道:``前些日子一直雨雪不断,便免了你们往来请安。今日皇帝和皇后有心,带你们一起过来了。''

众人道:``能向太后请安,是臣妾们的荣幸。''

太后含笑道:``昨日福珈陪哀家去御花园走了走,说是欣赏晴日红梅。其实红梅盛开,哪里比得上你们百花齐放,不止哀家,皇帝看了也赏心悦目。皇帝,你说是么?\textless{}''

皇帝赔笑道:``皇额娘说得是。''

太后理了理衣襟上的垂珠流苏,缓缓道:``百花齐放,乍眼看去似乎缺了哪一朵都不明显。可是熟知百花的人便知道,缺了哪一朵都不算是胜春胜景。皇帝,就当哀家人老多言,玫贵人已经出月,怎么还不见她出门向哀家请安?''

皇帝眉目间微有黯然之色,皇后忙含了恭谨的笑意道:``玫贵人伤心失意,是儿臣的意思,要她多多休养的。''

``过于伤心,那便是玫贵人的不是了。''太后叹了口气,随即敛容正色道,``对于嫔妃而言,孩子固然重要,但侍奉君上更为重要。这也是祖宗规矩为何要将你们生下的孩子交给阿哥所或是位高的嫔妃抚养的道理。就是怕你们只一心在孩子身上,疏忽了皇帝。''她瞥了皇帝一眼,好生关切道:``玫贵人无福为皇帝你诞育皇嗣,皇帝你也不要太过伤心。你还年轻,你的后妃们也还年轻,即便是玫贵人,也有再生养的机会,千万不要一时伤心过度,伤了龙体。''

皇帝连忙起身:``儿子多谢皇额娘关怀。''

太后叹口气道:``皇额娘关怀也是嘴上说说的,还是要你自己开解心怀。哀家看你这些日子都清瘦了不少,眼窝底下都是黑的。你这般郁郁寡欢,哀家看着也是焦心。''太后的口吻微有不满:``皇后,听闻这些日子多是你陪伴皇帝,怎么未有好好开解、宽慰圣心?你是六宫之主,宫中琐事固然要紧,但皇帝的一切更是要紧。你可千万不要轻重不分啊!''

这句话说得颇重,皇后微有惶然之色:``皇额娘恕罪,儿臣无能,不能使皇上开怀,所以这些日子也安排各宫嫔妃随侍。娴妃与慧贵妃也多有伴驾,皇额娘若不信,大可命内务府送上记档来查。''

如懿与晞月忙起身道:``恭请皇太后万安,臣妾们的确有奉皇后之命,侍奉皇上左右。''

太后抚着手边一把紫玉如意叹道:``皇帝登基之后虽然立了几个新人,但最得圣心的只有玫贵人。其实生了个死胎又如何,养好了身体很快又会有孩子,皇帝也可安心了。''

皇帝与皇后对视一眼,又看了如懿一眼,便也低下头去。皇后仰面,施施然笑道:``其实儿臣一直安排几位嫔妃随侍皇上,也是这样打算的。''她福下身含笑向太后与皇帝:``恭喜太后,恭喜皇上,继玫贵人之后,怡贵人也已经有孕一个多月了。''

皇帝一惊,旋即大喜,握住皇后的手扶起她道:``皇后所言可是当真?''

皇后的笑意温煦如春风:``孩子千真万确就在怡贵人腹中,臣妾岂敢妄言。而且臣妾查过敬事房的记档,的确是一个多月前承宠受孕的。上天如此安排,必是知道失之桑榆收之东隅,所以特让怡贵人怀上龙胎。''

怡贵人满面红晕,亦起身道:``臣妾深受皇上与皇后福泽,皇后娘娘为怕出错,特意请了三四位太医诊脉,臣妾的确是已经身怀龙裔了。''

如懿只觉得腔子里至喉舌底下,都酸楚极了。可是那种酸楚却全然不顾她的感受,自顾自强行而肆意地蔓延开来,爬入她的五脏六腑。如懿下意识地按着自己的小腹,那里是那样平坦,她还是那样没有福气,没有自己的孩子。或者说,是从未有过。而更难受的,或许是幽闭永和宫中的玫贵人吧,自己的丧女之痛切肤至深,却要眼睁睁看着怡贵人享受有孕之喜,将她曾经的盼望与喜悦一一经历。

皇帝喜不自禁,看向太后道:``皇额娘,皇额娘\ldots\ldots{}''

太后的笑意仍是淡淡的,如月朦胧鸟朦胧顶上一片薄而软的烟云,总有模糊的阴翳,让人探不清那笑容背后真正的意味:``这当然是好事。而且怡贵人从前是侍奉皇后的人,知根知底,没有比这个更好的了。''太后扶着福姑姑的手站起身:``说了一早上的话,哀家也累了,先进去歇息。你们坐一坐,便各自散了吧。''

众人目送太后进了寝殿。

皇后看着怡贵人的肚子,喜悦万分:``后宫顶了天的要紧事,就是为皇家开枝散叶,福泽万年。咱们的千秋万代,不在别的地方,都在你们的肚子上。若都能像怡贵人一样,本宫便是做梦也能笑醒了。''她笑吟吟地转头吩咐:``素心,莲心,今晚收拾下东西,本宫要去宝华殿进香祝祷,答谢神恩。''

皇帝欣慰地拍拍皇后的手,温和道:``有劳皇后了。''

``皇上怎么这样说?''皇后笑嗔,``嫔妃们诞育子嗣,她们固然是孩子的生母,臣妾是孩子们的嫡母,也一样是做母亲的。这份高兴,既是为了她们,也是为了臣妾自己。''

皇帝颇为感慨,眼底闪过一丝润泽:``皇后贤惠。''

皇后环视座下:``臣妾有一事一直想回禀皇上。其实嫔妃之中,慧贵妃与娴妃的位次最高,侍奉皇上也久\ldots\ldots{}''

如懿听见提到自己,不自觉地一凛,看向皇后。她抬头时正撞上慧贵妃的目光,两下里相触一闪,旋即转头,各自露出无比得体的笑容。

皇后含笑望着她们俩,眼中尽是温煦的关切之情:``其实不仅贵妃和娴妃,海贵人和嘉贵人也未生养过。臣妾想,不如请太医院开些催孕坐胎的方子,让各宫嫔妃都喝下,也好早有身孕,宫中也热闹些。''

皇帝欣慰道:``如此,便是皇后有心了。''

如是闲话几句,各人也便散了。皇帝对怡贵人的身孕格外重视,便让皇后亲自送了她回景阳宫,自己回了养心殿。

如懿与晞月踱出慈宁宫外,晞月自嘲地笑笑,难得地没有敌意,寥落道:``怡贵人恩宠一向不多,皇上一个月也不过只去她那里一次,居然也有了身孕。而本宫和娴妃你,居然沦落到要请皇后配制坐胎药才能求子的地步。''

如懿也颇伤怀,小指上的银鎏金嵌米珠护甲硌在掌心是冰冷且不留余地的坚硬。她勉强笑道:``一股子运气不来,皇上来得再多也是我们没有福气。''

晞月黯然一笑:``从前在潜邸的时候,你家世比本宫好,恩宠比本宫多。如今到了宫里,这情景掉了个个儿。本宫哪怕有多不喜欢你,可有一点不得不承认,在子嗣上,本宫和你一样艰难,膝下孤凉。''她话锋一转,忽然道:``本宫和你膝下无子也就罢了,可是玫贵人怀着身孕的时候人人都说她身体康健,即便有点小病小痛,也不过是嘴上溃疡之类的小事罢了。太医也说怀着的是个男胎,怎么生下来成了公主不说,还成了个死胎。死胎便死胎吧,偏偏皇上还存了芥蒂,整整一个月都没去看过她一次!''

如懿淡淡笑着道:``皇上圣意,岂是姐姐与我能揣测的。''

晞月含了一丝隐秘的笑容,挥手示意身后跟着的宫人退下,低低在如懿耳边道:``听说玫贵人的孩子,不只是死胎那么简单。当夜你也在永和宫,难道没发觉什么异样?''

如懿心口微寒,唇角却含了一缕恰如其分的笑意:``能有什么异样,不过是皇上亲眼见过那个孩子,所以伤心罢了。''

``再伤心,时间过去也能冲淡一切,再加上旧情,皇上不至于对玫贵人芥蒂至此。中间一定还有什么别的缘故,是不是?''

晴暖的阳光卷起碎金似的微尘,一丝丝落在身上,亦沾染了那种明亮的光晕,可是如懿分毫也不觉得温暖,那种从身体深处蔓生的凉意,丝丝缕缕,无处不在。她徐徐道:``还能有什么别的缘故,旧爱伤怀,怡贵人又有了身孕,皇上移情之后,玫贵人只会更受冷落了。''

如懿所言非虚。她的延禧宫就在永和宫正前,每每经过,看着门庭冷落,几可罗雀,她便可以想见,里头一寸一寸寂寞孤独的时光,是如何难挨了。

这样的日子,她也并非没有挨过。君恩如水向东流,得宠忧移失宠愁。宫中的女子,这一日复一日,何尝不是这样挨过的。

晞月更走近一步,语不传六耳:``可是本宫怎么听说,皇上命宝华殿的大师在永和宫诵经一月超度祈福,是因为玫贵人生下的孩子,是个妖孽!''

如懿连忙示意噤声,神色平淡而波澜不惊:``贵妃娘娘,宫内不比别处,这样的话可是说不得也传不得的。''

晞月收敛笑容,冷冷一嗤:``这样的话,何止是本宫,满宫里都在传着呢!如今只怕是玫贵人足不出户,迟早也要知道了。''

如懿心头一凛:``满宫里都在传?''

晞月冷笑道:``可不是?以为谁瞒得住谁呢,你若不信,自己去听听便知。''晞月说罢,唤过宫女一同离去了。

宫里的闲言碎语一向就比在阴暗角落里窜来窜去的蛇虫鼠蚁都要多。藏匿在宫苑红墙碧瓦之下的犄角旮旯里,嘈嘈窃窃,鬼鬼祟祟,交头接耳,蠢蠢欲动。像灶房里老鼠的窸窸窣窣,像墙头草左摇右摆,一只耳朵咬了另一只耳朵,好话赖话,一律咬着牙舔着舌头咀嚼着吐进吐出。只有添油加醋,没有短字少句。

这便是后宫的闲话了,没有一日断绝,倒像是无边无际的春草,漫无边际地滋生着。往这闲话的波澜起伏里投下一块惊涛巨石的,是玫贵人的自缢。

永和宫闭绝一个多月的大门再度开启。如懿得知消息的时候,已是午睡醒来饮茶用点心的时分。阿箬来禀告时,如懿惊得险将手中的一盏清茶皆泼了出去,忙忙扶了阿箬和惢心的手往永和宫去。

如懿赶到的时候皇帝和皇后都已经在了。她请了安便在旁边的椅子上坐下。玫贵人被皇后贴身的素心和莲心按住了坐在床上,兀自呜呜哭泣。皇帝气恼之余不免有些心疼,口吻却是十分严厉:``宫中妃嫔自戕是大罪,你有什么想不开的,居然敢在紫禁城内自缢,也不怕添了宫里的晦气!''

玫贵人只穿了一身素白色缀绣银丝折枝迎春的衬衣,外头披着一件石青刻丝灰鼠大氅,那青青翠翠的素白底色,愈显得那脸没有血色,唯有雪白的脖颈上留着深紫一道勒痕,楚楚可怜地昭告天下,她是刚从鬼门关上被人拽了回来。

玫贵人呜呜咽咽地哭着:``臣妾本来就是个晦气的人,还有什么可说的。皇上恕了臣妾,由得臣妾去死便罢了。''

皇帝气得别过头去,皇后亦不免含了怒气:``即便你没有家人需要顾及,也不怕连坐。可是皇上有什么不疼你的,你便这样自轻自贱,轻易毁损自己的性命,岂不是辜负了皇上对你素来的心意?''

玫贵人哭得愈加幽凄:``只有臣妾自己对不住皇上的。臣妾无话可说,也无颜再侍奉皇上!''

皇后看着满地跪着的宫人道:``你们也是,不好好伺候着玫贵人,由得她这样伤心这样闹,本宫要狠狠处置你们才是。''

那些宫人们吓得拼命磕头道:``皇后娘娘恕罪!皇后娘娘恕罪!奴才们也不知是出了什么事,贵人的情绪会这样激动!''其中一个领头的宫女哭着道:``这几日贵人小主一直心绪不定,晚上也惊梦连连,睡得并不好!今儿午后小主本是要午睡的,可是小主并不让奴婢们伺候,全打发了出去。奴婢在外头听着不太放心,又听见凳子落地的声音,怕出了什么事,结果闯进去一看,贵人小主竟把自己挂在梁上了!''

如懿忙问道:``那么你家小主到底是为了什么想不开?可是为了孩子的事?''

那宫女怯怯地摇摇头,又俯首下去。

皇帝气得狠了,连连问:``你有什么想不通的,尽可跟朕和皇后说,再不然,娴妃和你这样近,你也可以告诉她。''

玫贵人哭着道:``皇上不就怕臣妾和别人说话知道些什么吗?所以皇后娘娘也将臣妾关在这永和宫里不许见人。臣妾知道自己人微言轻又命薄如纸,除了把自己吊到梁上,还能有什么办法?''

皇帝将手中的茶盏重重一砸:``荒唐!''

如懿忙接过茶盏吹了吹道:``茶盏太烫,皇上仔细手疼。''

皇帝微微颔首,正要说话,却见寝殿门口杏子红的衣衫翠罗一闪,却是慧贵妃娉娉婷婷立在了那里。她由着宫女伺候脱下斗篷,声音冰冷冷的:``臣妾要是玫贵人,听说了那些闲话,也是要想不开的了。好好的孩子,死了也罢了,还要被人传成是一体双生的妖孽,雌雄不辨。这世上有几个做母亲的能受得了。''

皇帝神色大变,蹙眉道:``你从哪里听来这些无稽之谈,还跑到这里来说?''

慧贵妃倒也不惧,盈盈施了一礼道:``臣妾还用从哪里去听说,满宫里私底下谁不是这样在传呢。''

玫贵人凄厉地尖叫着哭了一声,从床上挣扎着起来,膝行至皇帝跟前,抱着他龙袍一角道:``皇上,请求您告诉臣妾一句实话,臣妾的孩子是不是一个妖孽,是不是连是阿哥还是公主都分不清?所以皇上会厌弃臣妾至此,整整一个多月都不愿来看臣妾一眼!''

皇帝勉强挤了一丝笑容道:``外头的闲话,你别去乱听!朕不来看你,也是为了你安心养好身体!''

玫贵人哀泣道:``臣妾哪里还能养好身体?即便臣妾幽居在永和宫里,也能听见宫墙外头的议论。难怪皇上连那孩子也不让臣妾看一眼便送走了,原来臣妾生的真是个妖孽!''

皇帝有些烦躁,喝道:``王钦!''

王钦紧赶着从外头进来道:``皇上,奴才在。''

皇帝冷冷道:``你去宫中彻查,到底是哪些人在散布谣言,说玫贵人生下的是个妖孽。一旦查到,无论是哪个宫里的,立即送进慎刑司,终身不得出来。''皇帝这话口气虽冷,但目光更是锐利,只逡巡在王钦面孔上,逼得他渗出了一脸冷汗,忙磕了头道:``皇上放心,奴才身边断不会有这样散布谣言的人,更不会有听过这种谣言的人,奴才会即刻去查。''

皇帝轻轻``嗯''一声,道:``玫贵人,旁人有这样的揣测谣言都不要紧,但你是孩子的生身母亲,你若存了这样的疑心,还要为此赴死,岂不是连你自己也在这样揣测自己的孩子了。朕没有别的话,只告诉你,你便再要寻短见,谁也救不了你,更换不回那个孩子!''

皇帝再无二话,起身离去,才走到庭院中,却见慧贵妃紧紧跟了来道:``皇上,臣妾有一言,不知当讲不当讲?''

皇帝道:``有话便说吧。''

慧贵妃施了一礼,便道:``臣妾想着一事,不管玫贵人生下的孩子是什么,即便是个死胎,也是不吉利的。且玫贵人又这样寻死觅活的,怕是冲撞了什么。如今怡贵人有了身孕,又住在永和宫后头,要是受了这不吉利的人与事影响,再涉及腹中胎儿,那便不好了。''

皇帝道:``那你的意思是如何?''

慧贵妃道:``皇上多有子嗣,人人无事,唯有玫贵人的孩子有事,那便是玫贵人的不祥了。与其留这样一个不祥人在宫中,还不如请玫贵人移居宫外别苑,再不要住在紫禁城中了。''

皇帝淡淡``哦''了一声:``只有这样的法子么?朕的本意,是想请几位法师超度之后便可以解了玫贵人的幽禁了。''

慧贵妃摇头,正色道:``臣妾别的不敢多言,不管玫贵人所生的是死胎也好妖孽也好,子嗣为上,若是沾染了她的晦气,宫中再有一个那样的孩子,可如何是好?大清百年国祚祥瑞,难不成就要断送在她手里?''

如懿正跟着皇后出来,听到这句,不觉便上前了一步。皇后按住她的手,缓缓地摇了摇头。如懿心下担忧不已,回头望去,玫贵人还在寝殿深处郁郁哀哭不止。

皇帝依旧是不动声色:``话不要说半截,都吐出来吧。''

``玫贵人不祥,上承天恩居然还会生出那样的孩子,这样阴鸷的祸水,是断断留不得了。臣妾想着,反正玫贵人也是想不开了要自缢,不如成全她,让她陪着那个孩子去了,也算是积了阴德。''慧贵妃扶住皇帝的手臂,小心觑着皇帝的神色,意味深长道,``左右那个孩子是什么样子,皇上是亲眼见过的。这样的孩子,宫中是绝不能有第二个了。''

皇帝的身体轻微一震,像是被她的话语深深触动,旋即陷入更深的沉默之中。

\hypertarget{ux7b2cux4e09ux7ae0-ux6d41ux8a00}{%
\chapter{第三章 流言}\label{ux7b2cux4e09ux7ae0-ux6d41ux8a00}}

皇帝静了片刻,只是看着庭中幽幽红梅,吐着暗红色的花蕊,像是溅开了无数血腥的红点子一般。如懿悄悄看着皇帝的脸色,只觉得什么也瞧不出来,皇帝的神色平静极了,如同秋日里澄净如镜的湖面,犹有暖日的金色余光洒落面上,平添了一分暖调。

皇后按了按如懿的手,悄然上前,柔声道:``慧贵妃的话是急了些,但臣妾心想,这满宫里无论是谁,无论什么事,都比不上大清的国祚要紧。''

如懿一想到``自缢''二字,只觉得浑身发冷,忍不住道:``皇上,玫贵人的孩子纯属意外,既然孩子一生下来就已经死了,那更不会干系旁人,更不会影响大清的国祚。''

慧贵妃笑道:``娴妃这话便是说得太轻巧了。皇上正当盛年,以后多的是孩子。孩子是阿哥还是公主都不要紧,要紧的是聪明齐全,成为对大清有用的人。娴妃如今都未有生育,试想若是受了贱人的祸害,也生下了这样的死胎,娴妃你身为人母,能否接受?到时候便悔之晚矣。''

如懿一听她拿自己做例子,其心恶毒,心底愈加难耐:``天命庇佑,我是不怕的。慧贵妃若要担心,便担心自己的孩子吧。''

慧贵妃眼波一剜,清冷道:``本宫要念及的不仅是自己来日的孩子,还有眼下怡贵人的孩子和日后旁人的孩子。娴妃你为玫贵人求情,是不是敢担保,以后宫中再不会有这样的祸事,还是有了这样的祸事,到时你与玫贵人便一起殉了那孩子,以报大清?\textless{}''

皇帝呵斥道:``好了。站在这儿便这样争执不休,成什么样子?''

如懿与慧贵妃对视一眼,只得屈膝道:``臣妾冒昧了。''

皇后低声道:``皇上,那您的意思是\ldots\ldots{}''

皇帝皱了皱眉,扶住皇后的手道:``怡贵人的孩子就请皇后多多看顾。至于玫贵人,就先挪出永和宫,住到宝华殿前头的雨花阁去,让她邻近佛音,好好清净清净心思。''

慧贵妃犹有不服,道:``皇上,可是她生下了那样的孩子\ldots\ldots{}''

``孩子?''皇帝轻轻一嗤,``是否恩准玫贵人自缢且容后计较。朕倒想知道,宫中到底有哪些胆大妄为的人,敢擅自散布流言,混乱人心。朕断断容不得!''

皇帝这话说得沉肃,众人闻言皆是一凛。皇帝道:``慧贵妃,这里没有你的事情,先跪安吧。''

待到慧贵妃出去,皇帝负手立在庭中,身边再无旁人伺候。如懿见他如此神色,又兼之方才那番话,心下便有些沉郁。皇帝的声音极轻:``那夜在这里,见过那个孩子的,只有朕、皇后、娴妃还有王钦吧。''

皇后婉声道:``是。其余见过孩子的人,当夜都打发出去了,应该来不及在宫里说些什么。''

皇帝长叹一声:``你们都是朕近身的人啊。''

如懿会意,旋即道:``臣妾谨遵皇上吩咐,不敢有一言半语泄露。''

皇帝点点头,又问:``皇后,那日王钦把孩子送去处置,路上不会有人瞧见吧?''

皇后的声音极低,仅仅足以让身边的人听清楚:``出了永和宫的门就扼死了,一路就是个死胎送进小棺椁封好焚化。这件事,臣妾身边的莲心跟着一块儿去办的,绝不会有差错。''

如懿虽知那孩子是必死无疑,却不想是王钦活生生扼死的。不知怎的,她便觉得心口哆嗦着窒闷难言,几乎想要呕吐出来。

皇帝轻轻``嗯''了一声,慢慢踱出庭院。如懿听着满庭风声萧索,肆意而狂暴地穿过枝丫,自己仿佛也成了其中枯靡的一枝,任由逆风侵袭,不得摆脱。

如懿回到殿中,便有些不耐烦。她描了几笔花样子,便烦恼地将笔一搁。冬日所用的杏子红团福撒金锦帘是喜气洋洋、花团锦簇的颜色,落在她眼里却只觉得那金茫茫的颜色格外刺眼。惢心打了帘子捧着茶水进来道:``小主,永和宫的玫贵人是要搬出去了呢。''

如懿点了点头,接过茶水道:``她也可怜见儿的,孩子成了那个样子,挪去雨花阁静静心也是好的。''她抿了一口茶水,问道:``怎么换了茉莉花茶?''

惢心笑道:``茉莉清心宁神,小主一回来就沉着脸,所以奴婢换了这个。''

如懿便道:``阿箬呢?怎么都没有看见阿箬?''

惢心道:``说是去内务府皮库挑些好皮子来做两件冬衣,一去去了这么久,大概是挑皮子耽搁了。小主不是不知道,阿箬选东西算是精细的。''

如懿笑道:``也是,她是见过好东西的,挑东西也严苛。我看她如今的性子安静了好些,不比从前那样浮躁,也放心些。''

惢心道:``可不是呢?上回的事阿箬姐姐算是得了教训了,也亏得小主的调教。''

如懿轻舒了口气道:``她自己知道便好了。''

惢心看着如懿,小心翼翼地问:``那小主为什么又不高兴呢?''

如懿伸出纤细的手指在几案上轻轻划着,理了理自己烦乱的心绪:``宫中流言如沸,不胜其扰。''

``宫中从来都不缺流言,小主何须烦扰?''

云髻上垂落的红瑛流苏沙沙地打着鬓边,每一拂动,便是一层秋雨落叶似的微凉。``如果皇上最忌讳的流言,出处只可能在我、皇后和王钦这三处,你觉得皇上会如何想?''

惢心神色遽变,如蒙了一层白蒙蒙的寒霜一般:``这件事若不查清,只怕皇上会对小主存了极大的疑心。皇上的疑心若是不除,那小主往后的日子便难过了。''

如懿烦心道:``我何尝不知道这个?只是这件事皇上已经在查,但愿很快能水落石出。''

夜来的雨花阁格外幽深寂静。雨花阁本是前明遗留的建筑,一共三层。除了第一层供奉佛像经书外,上面两层均可住人。只是规制陈旧简朴,与东西六宫不可同日而语。玫贵人新移居此地,连侍奉的侍女也少了大半,连着三五日听着后头宝华殿梵音悠长不断,心下更觉凄凉。

可是此身孤苦,一世的荣华与美梦,都随着那个苦命的孩子去了。她也生生被困在了这里,不知何年何月才能得个解脱?

玫贵人伏倒在佛像前,听着窗外风声呜咽如泣如诉,亦不觉落下清泪。只觉此生茫茫,再无可渡之处了。

太后进来之时她尚浑然不觉。倒是福姑姑先唤了一声:``玫贵人,太后往宝华殿参拜,经过雨花阁,还请贵人奉上茶水以侍太后。''

夜来参拜,太后身边只带了福珈,几个随侍的宫人都留在雨花阁外。太后穿着一身简素而不失清贵的宝蓝缎平金绣整枝芭蕉福鹿纹长袍,头上用着一色的寿字如意金饰,不过寥寥数枚,却清简大气。

玫贵人一时未反应过来,忙起身拜见,屏退了众人方郑重其事地三叩首,热泪盈眶道:``不意太后深夜移驾雨花阁,臣妾未能远迎,实在是失礼了。''

太后缓缓地拨着手中的翡翠佛珠,那一汪绿色水莹莹的,在烛光底下如一湖澄净凝翠的碧波,一看便知是上好的贡品。

太后缓声道:``你要还是在永和宫,要来看你也不方便。如今雨花阁住得还惯么?''

玫贵人一时语塞,终究还是摇了摇头。太后温和笑道:``也是。住惯了东西六宫的繁华,哪里受得了雨花阁的孤苦?只是皇帝的意思也对,你总是那样伤心,住在雨花阁听听佛音梵经,也是好的。''

玫贵人闻言,不觉清泪滂然,如止不住的寒雨凄切:``太后,宫中所有人都在传,传臣妾所生的不是死胎,而是个孽障妖胎。臣妾\ldots\ldots 臣妾怎么会生出那样的孩子?''

太后长叹一声:``你的孩子一生下来就被封进棺椁焚化了,是死胎也好孽障也罢,连哀家都无法确证,何况是你。你若多想多思,便是为难了你自己了。''

玫贵人不甘地泣道:``可是,那是臣妾的孩子啊!臣妾十月怀胎含辛茹苦生下的孩子,怎么会是孽障呢?''

太后注视着她,双目沉静如能照透人心:``是不是孽障很要紧么?连皇上都不愿意再多提起,更不愿宫中有任何相关的流言四起,你又何必苦苦执著?毕竟,那已经是死了的孩子了。而你,若再执意如此,虽还活着,却也离死不远了。''

玫贵人浑身剧烈一震,仿佛不可置信一般,瘫软在地:``太后\ldots\ldots{}''

太后慢慢地捻着佛珠,缓缓道:``哀家听闻,慧贵妃已经向皇帝进言,准许你自缢去陪着你的孩子,以免后宫再生下这样不吉的婴孩。皇帝一时心软,未曾答应,若是哪天枕头风吹得更厉害些,他听进去了也未可知。到时候,也不必你寻死上吊,皇帝就成全你了。''

玫贵人吓得花容失色,连连摇头,膝行至太后跟前,匍匐着道:``太后娘娘,太后娘娘,臣妾不是存心要自缢寻死的,只不过臣妾生产之后皇上一直不来看臣妾,臣妾才只好出此下策,引皇上过来。连那些宫女都是臣妾安排好的,臣妾不想死,臣妾不想死!''

太后闭着眼睛,淡淡道:``哀家当然知道你不想死。当日把你从南府捞出来的时候,就发现你是个有心性的,又出身乌拉那拉府邸,一放进后宫准保能让皇后等人费尽心神。皇后专心于后宫纷争,哀家的话在后宫才会有人听、才有用。你要是这么轻易就死了,可就白费了哀家的一片苦心了。''

玫贵人俯首帖耳,再三叩首:``臣妾一入后宫,慧贵妃便极力排挤,视臣妾为娴妃一党,如今还要殉了臣妾。臣妾愚钝,还请太后怜惜,指点迷津。''

太后淡淡一笑:``指点迷津的只有满天神佛,能自渡迷津的就只有自己了。哀家知道你心痛孩子的死,但孩子死了,只要你活着,总还会有机会。你且放心,哀家会告诉钦天监,流年不利,宫中断不能再有白事。但如何走出雨花阁,如何不负哀家所托,就看你自己的了。''

玫贵人俯身拜倒,悲痛的神情中多了一分郑重:``臣妾谨受太后教诲。''

太后扶过福姑姑的手,漫步踱出,她的语气缓而沉:``有件事,哀家一直想不明白,你的胎一直都说不错,孩子也壮健。怎么生出来的会是那个样子,真是可怜了。''

玫贵人伏倒在地,平滑如镜的澄砖地冷而硬地硌在额上,那股冷意直逼进脑仁里去。她抬起头,殿中只余下太后长年所焚的檀香余味,气息幽沉,弥漫一室。

如懿被宣召至养心殿,是在午膳时分。她才用完午膳,由阿箬伺候着浣手洁净,皇帝身边的李玉便急匆匆赶来了:``娴妃娘娘,皇上有旨,请您立即前往养心殿暖阁一趟,闲人勿带。''

如懿听得最后一句,心下便微微一沉,生了几分不豫之情,脸上却还笑着:``皇上这样的旨意,可是出了什么事?''

李玉的神色不似往常,只道:``辇轿已在外头备下,娘娘请吧。''

如懿急急更衣,连阿箬和惢心也未带,便扶着李玉的手出去。直到到了仪门外快要上轿的一瞬,她才听得李玉用极低的声音道:``王钦在皇上面前诉说了一通,奴才也不知是什么事,只知皇后娘娘也到了。''

如懿听得``王钦''与``皇后'',心下更是阴沉难言,只得道:``那就快些去吧,别让皇上等着。''

如懿甫一进殿,便觉得殿中气氛不似往日。皇帝神色沉郁,眼底隐隐含了一分怒气。皇后亦是半坐在榻前的紫檀椅上,并不敢与皇帝同坐在榻上。而王钦垂头丧气地跪在地上,一声也不敢言语。

如懿忙福了福道:``皇上万福金安,皇后娘娘万安。''

皇帝草草抬了抬下巴,示意她起身。如懿忙垂手站在一边,皇帝也不叫``坐下'',只向王钦道:``你把方才跟朕说的,再与皇后和娴妃说一遍。''

王钦忙磕了个头道:``奴才奉皇上之命彻查六宫流言之事,发现宫中的确传言纷纷,论及玫贵人所生的婴孩一体双生,是个妖孽。种种关于婴孩的细节,如同亲见,再加上奴才们嘴贱,添油加醋,便成了说那婴孩如妖物一般。''

皇帝不耐烦道:``说这些做什么!只说你查到的那些!''

王钦吓得一怔,忙道:``奴才查问下来,发现此种流言散布,东六宫远甚于西六宫。''

皇后显然是松了一口气,神色舒缓了不少,拨着珐琅掐丝手炉上的银镏子道:``阿弥陀佛,臣妾居住在长春宫,幸好西六宫流言不多,臣妾也算分明了。''

王钦拿袖子擦了擦汗道:``是。据奴才所知,流言所在,主要盘集在永和宫、延禧宫、景阳宫和钟粹宫一带。''

皇后看王钦说得满头大汗,忙温言道:``东六宫中只有这四宫有嫔妃居住,永和宫又是事发所在,难免流言纷扰。你且说,这些话是哪里传出来的?''

王钦脸色发白,那汗水滴答下来,被殿中的苏合香一熏,气味实在难闻。如懿屏息敛气,只听他说下去。

皇后沉声道:``皇上面前,你还有什么不敢说的么?''

王钦磕了个头,拿眼睛瞟着如懿,道:``宫人们都说,最早有流言传出的,便是延禧宫。''

如懿仿佛被一桶冰水直浇而下,冷得天灵盖阵阵发寒,忙跪下道:``皇上明鉴,当夜永和宫所见所闻,臣妾未曾有一字半句传出。延禧宫中更无人得知,如何能在宫中散布流言!''

王钦急急忙忙道:``奴才不敢妄言,所以特意带了一些散布流言的宫人回来,请皇上细察。''

皇帝冷冷道:``既然查了,那就传吧。''

王钦击掌两下,只听外头窸窸窣窣有人进来,地上的锦毯极厚,几乎是踏步无声,唯有衣袍与地毯相触的摩擦声刮着耳膜一阵阵逼近。大约是四五个宫人,跪在了离皇帝一丈之地,叩头问安,缭乱了一阵。

王钦在宫人们面前便恢复了素日的趾高气扬,冷着脸道:``我问你们什么话,你们据实以答就是了。在皇上面前,都老老实实的,不许有一句妄言胡说。''

众人怯怯答了``是'',王钦又道:``你们几个,在宫里嚼舌根是最厉害的,得了空就在那儿胡说八道,飞短流长。眼下我就问你们,最早的时候,你们是在哪儿听来关于玫贵人的那些不干不净的话的?''

那几个宫人怯怯互视了几眼,又见如懿也在侧,便越发生了胆怯之情,其中一个怯生生道:``时日长久,奴才、奴才们都忘记了。''

如懿见几个宫人看一眼她,便不敢多言,一颗心越发往下沉了沉。她跪在地上,见满地铺着寸许厚的百花戏春图的猩红滚金线织锦云毯,密密匝匝地绣着牡丹含芳、蔷薇凝露、莲花清馨、秋菊迎霜、腊梅傲雪,百鹊千蝶嬉戏其间。那样热闹鲜活的图案,原是一整个春日的欢好,此时看来,却似密密匝匝逼得人透不过气来一般。

``忘记了?''王钦冷笑一声,``方才都还记得,如今便全忘记了。我就知道,不长记性的奴才,除了用刑,再没别的办法。''

皇帝口气亦是森冷:``到了朕跟前还要推诿?王钦,用刑!先夹断了几根手指,便知道要说实话了。''

皇帝话音刚落,其中两个胆小的便没命价地磕着头道:``皇上饶命,皇上饶命!奴才都说了,都说了,奴才最早是经过延禧宫的时候听说的。''

皇后追问道:``最早?最早是什么时候?''

那宫人脸色煞白:``就是玫贵人生产的那一夜。''

皇后神色微变,似是自言自语:``也就是说,皇上刚交代完臣妾和娴妃离开,宫中就流言四起了?''

另几个宫人也忙跟着道:``不错不错。皇上,奴才再不敢胡说八道了,就是在延禧宫一带最早传出来的。''

苏合香的气味原是清宁宜人,此刻嗅在鼻中,只觉得热辣辣的,几乎要熏落了眼泪。如懿深深叩首,凛然道:``皇上明鉴,臣妾的确不曾泄露一字一句。''

皇后有些为难之色:``皇上,以娴妃的为人,想来是不会对外人随意乱说的。只是\ldots\ldots{}''她看着如懿,温婉的眉目间多了几分揣测之色:``娴妃,你是不是那夜受了惊吓,又疲倦过度,一时对谁说过,自己也不记得了?''

鎏金错银福寿无疆的大鼎中,若有若无的苏合香薄烟,丝丝缕缕交错密织,无边无际地扩散开来,仿佛织了一张无形的网,遮天兜地地笼罩下来,让人无处可逃。

如懿只觉内心沉闷凝滞不已,仰面直视着皇帝道:``皇上若肯信臣妾一句,臣妾敢以性命担保,不曾向任何人说过只言片语。''

王钦啧啧道:``这便奇了,人人都说是娴妃的延禧宫传出流言,偏偏娴妃娘娘说只字未漏,难道这些奴才都疯魔了,连哪宫哪苑都分不清楚,信口胡说?或者真如皇后娘娘所言,娴妃娘娘无知无觉中自己说了出去,或是梦话,或是气话,也未可知!''

如懿心中恼怒,盯着王钦道:``你口口声声咬住本宫不放,到底本宫有何居心,一定要害了玫贵人还要损她声誉?更不惜连累皇上与皇室的名声?''

王钦忙摇头道:``娴妃娘娘千万别恼怒,奴才也不过一说罢了。只是娴妃娘娘一直未有生育,出于嫉妒迁怒于玫贵人,一时口快说了出去,恐怕也是有的。''

皇帝默不做声,只是重重一掌击在紫檀几案上,皇后急得捧过皇帝的手仔细察看道:``皇上再生气,也要注意龙体,万勿伤了身子。''

皇帝道:``朕的面前,也不好好说话,只一个个咬住了不放,成什么样子!''

皇后忙起身跪下道:``皇上息怒,哪怕种种证据确凿,人人都指证娴妃,臣妾也不相信是娴妃有意所为。''

皇帝思忖片刻,慢慢道:``朕也相信娴妃,但流言所指,朕不能不查个彻底。''

皇后连忙道:``皇上说得是。只是娴妃侍奉皇上多年,没有功劳也有苦劳。但请皇上先勿责罚。臣妾想,既然此事要彻查,娴妃卷入其中也不适宜,不如请皇上先让娴妃不要出入延禧宫,等到查清,再给娴妃一个清白。''

皇帝沉吟着,殿中苏合香的香烟袅袅飘散荡开,连皇帝的面孔也遮了一层薄薄的雾翳。如懿跪在地下,殿中分明是和暖如春,那空气似乎被春日里的蜂胶凝住,滞塞不堪,闷得她透不过气来。良久,皇帝的声音有如金器冷石般锐利地穿透了一缕缕薄烟,凌空破来:``那么,朕就如皇后所言。''

如懿脚下一软,几乎是失却了起身的力气,只失望而凄切地看着皇帝。皇帝并不闪避她的目光,沉声道:``朕会禁足你一段日子,以求真相。你便先放心住在延禧宫中吧。''他不容如懿再说,唤过殿外的李玉:``李玉,扶娴妃出去。''

如懿只觉得脚下绵软无力,一颗心往下坠了又坠,回望去,皇帝的眼中含了一点锐利的坚定之意,她只得安下心来,缓步出去。待到人少处,她就着李玉的手,仿佛是不动声色,只目视着前方,极偶然的,一个眼波划过李玉的面颊,含了深深的决绝和冷厉。李玉会意地点点头,重又垂下双眸,保持着一如往常的温驯和恭顺。

\hypertarget{ux7b2cux56dbux7ae0-ux6625ux60c5}{%
\chapter{第四章 春情}\label{ux7b2cux56dbux7ae0-ux6625ux60c5}}

如懿禁足的日子,便是从这一个阳光灿烂的晴明午后开始的。朱红色的阔大宫门``吱呀''一声从身后紧紧合上,便是锁链重重锁住的声音。连她自己也不知道,再打开会是什么时候。延禧宫的宫人们慌得眼泪都下来了,忙不迭地跪了一地,却不知该对着谁去跪。海兰在后殿亦被惊动了,惊慌失措地奔过来道:``姐姐,到底出了什么事?为什么要把延禧宫的大门锁起来?''

如懿站在庭院中,缓步拾上台阶,阳光透过落尽了翠叶的光洁树枝斑驳地筛了满地。那样清冷的日光从天空倾泻而下,抬头望时,能看到九重宫阙的琉璃碧瓦在日色下闪耀起冰雪洁白一样的光芒。

那样的光芒,离她真是遥远。

如懿轻声说:``不要怕,我只是被禁足而已。延禧宫的角门还能出入,是为你留的。''

海兰眼底含了稀薄的泪花,不安道:``姐姐,才安静了这些时候,咱们的日子就这么难过么?\textless{}''

如懿望着远处宫阙重重,琉璃瓦上浮光万丈,神色平静得如阳光照耀下的冰雪:``有时候日子安静并不等于难过。你安心就是。''

禁足的时光幽寂而难耐,隔绝了出入,每日所能见的,不过是一方四四方方的小小蓝天。如懿用来打发时光的,不过是让惢心和阿箬把库房里的各色丝线都选出来一一整理。

这是十分费工夫的一件事,每种丝线分门别类,浸在拧了各色鲜花汁子的滚水里煮过。玫瑰汁子配玫瑰红,杜鹃花汁配杜鹃红,芙蓉花汁配芙蓉粉,飞燕花汁煮久了是淡淡的明蓝,栀子花汁配了淡淡杏黄的白色,香蜂花兑了薄荷配蓝紫色,一一都是费尽了心思的。连黄色的要绣作花蕊的丝线,也一一用柠草汁子和番红花汁一起煮过,带了清新之气。而绿色呢,更是麻烦,配着藿香、杜衡、薜荔、菌桂、迷迭香、百里香、山桃草等香草,煮成芬芳的秾翠明艳。

海兰来看她时不免长吁短叹:``姐姐还有心思做这些事,妹妹这些天出去,整日里见王钦在追查那些散布流言的奴才,一个一个都吐了口儿,说是从延禧宫这里听来的。再这样下去,恐怕皇上不只是禁足,而是要对延禧宫上下一一用刑审问了。''

如懿笑吟吟递了一把松石绿的丝线给她:``你细闻闻这个,我放了芳芷、木根、兰茝这三种香草,是不是别有一种草木清香,好像春天已经来了?''

海兰无奈接过,却并不如如懿所言去轻嗅其味,愁容满面道:``姐姐是盼着春天来,妹妹却看着好像这冬天过也过不完似的。''她忧心忡忡:``一旦坐实了流言为姐姐所传播,损害皇室声誉,该如何是好?''

如懿这才抬首道:``王钦找了多少人了?''

``总有十来个了吧。''

如懿轻轻一笑若淡淡的云影:``十来个人,要置我于死地也够了。可是你猜猜,若要置王钦于死地,几个人才够?''

海兰眼底浮起深深的疑惑:``姐姐的意思是\ldots\ldots{}''

如懿看了看窗外浓墨般的天色:``我能有什么意思?对了,这些日子都是谁陪着皇上?''海兰道:``宫中流言纷扰,皇上也很少召见皇后,多半是嘉贵人和慧贵妃伴驾吧。如今怡贵人有孕,宫中妃嫔倒也常去探望怡贵人,听说慧贵妃也去得很勤快呢。''

如懿道:``宫中的嬷嬷们每常说,坐胎药喝下去,也得多沾沾有孕之身的孕气才好呢。慧贵妃盼子心切,一定会去的。''

海兰看着眼前缠绕一团的丝线,烦忧道:``这也罢了,慧贵妃每每特意从景阳宫经过咱们延禧宫,都要伫立良久,感慨姐姐境遇凄寒。于我看来,她不过是幸灾乐祸罢了。''

如懿微微一笑,丝毫不以为意:``她若喜欢,便由着她去吧。左不过她在外面感慨,而我在里头也听不见,就算听见了,只当风吹过就是了。''

海兰见她如此,也只能默然。二人寂静里相对,听着窗外风声簌簌,远远有笑语声传来,海兰叹道:``延禧宫被禁足,永和宫人去楼空,只有景阳宫恩宠不断。风送宫嫔笑语和,大约只有咱们这里这样静,才能听得清楚吧。''

如懿淡淡一笑,手中千丝万缕穿梭不断,只慢条斯理交代惢心道:``这些丝线都是煮过了染上了香气的,你明儿拿到太阳底下去晒过,务必要翻晒多次,等太阳落山后再拿进来煮,得煮好多次,我才能绣出带着香气的《百花春意图》呢。''

惢心答应着,又上来添了几支蜡烛,正静静相对,忽然外头喧哗声大起,夹杂着女人尖叫的声音、宫人的呵斥声和太监含混的话语。

海兰立时警觉起来:``姐姐,你听什么声音?''

惢心侧耳细听片刻,忽而一笑:``仿佛是慧贵妃的声音。''

海兰怔了怔,立时站起身来,却又不知该不该去看看。

如懿淡淡笑道:``我被禁足了,你却没有。海兰,你去外头看看,若是慧贵妃在咱们宫门前出了什么事,可就不好了。''

海兰连忙出去,吩咐守门的侍卫开了大门。如懿披上惢心送来的素色缠枝花灰鼠大氅,紧随在后。守在门前的侍卫看她出来,忙挡住了道:``娴妃娘娘,皇上有旨,您不能出延禧宫的大门。''

如懿淡淡道:``放心!本宫不会教你们为难。本宫只在这儿看着,绝不跨出这扇宫门半步。''

那些侍卫显然是松了口气,躬身站到一旁。外头纷乱异常,有宫人侍卫的脚步声匆匆过来,显然是被方才的声响惊动了。数十盏宫灯将夜来的延禧宫门前照得煌煌如白日,慧贵妃被宫女们簇拥着围在中间,一张莲瓣似的娇美面孔惊怒交加,失了往日的姣好颜色,显是受到了极大的惊吓。

太监侍卫们七手八脚地押着一个服制鲜艳的太监,将他整个脸按在了尘土之中。

慧贵妃鬓发凌乱,云髻松散,几支白玉南红如意珠钗斜斜地坠在耳边,一副将堕未堕的样子。她的厉声呵斥底下有着难掩的震怒与惊恐,喝道:``将这个不知死活的东西立刻拖到皇上跟前去,给本宫交代个清楚!''

如懿悄声问守门的侍卫道:``这样乱糟糟的,究竟出了什么事?''

侍卫道:``回娴妃娘娘的话,那人是皇上跟前副总管太监王钦公公,也不知是喝醉了酒还是怎么,方才慧贵妃带着宫人经过,他便发了狂似的冲上来,言行莽撞,惊扰了贵妃娘娘。''

海兰奇道:``王钦又不是不认识慧贵妃,怎会冒犯贵妃呢?''

侍卫道:``奴才们奉命看守延禧宫,不能走开一步,所以只能干看着。不过王公公的的确确跟疯魔了似的,看见贵妃娘娘就没头没脑地扑了上去。''

如懿见慧贵妃稍稍缓过神,便朗声道:``延禧宫娴妃参见贵妃娘娘,愿贵妃娘娘万福金安。''

海兰见如懿行礼,忙也跟着行礼如仪。

慧贵妃一手护住胸口,一壁恨恨道:``是你?你怎么出来了?''

如懿含笑道:``妹妹没有出来,只是听得外头喧哗,不意是贵妃娘娘在此,所以特意过来一看,娘娘没事吧?''

慧贵妃恼恨道:``本宫有事无事,不必你来关心。''

如懿含着谦恭的笑意,柔声道:``妹妹也不想过多关心,只是此事出在妹妹宫门前,妹妹想不多看一眼也不行了。''

慧贵妃气得发怔,露出森森笑意:``好!好!居然来看本宫这个热闹!本宫也很想知道,王钦突然在延禧宫外冒犯本宫,是不是有人存心指使!''

二人正僵持着,却见不远处明黄一色御辇迤逦而来,双喜忙请了安上前道:``回禀贵妃,皇上正在景阳宫中,奴才已经请了皇上过来了。''

御辇尚未停稳,慧贵妃已满面是泪扑了上去,伏倒在地道:``皇上,皇上,您要为臣妾做主。臣妾自侍奉皇上左右,从未受过这样的羞辱。皇上!''

皇帝的御辇堪堪停稳,见她这个样子,又是怜惜又是着急,便道:``李玉,还不快扶慧贵妃起来。''

慧贵妃犹自啼哭不已,如梨花一枝春带雨,皇帝微微蹙眉道:``好了。那么多人在,你哭哭啼啼成什么样子。有话好好说便是。''

如懿领着海兰向皇帝请了个双安,便道:``皇上,贵妃娘娘伤怀,王钦现在还满嘴嘟嘟囔囔地说着胡话。依臣妾看,不管何事都不宜外扬,不如先拿水泼醒了王钦,再好好问话吧。''

皇帝有几日未见如懿了,此时见她披了一件素色大氅,盈盈站在风中,仿佛不盈一握的样子,口中倒是纹丝不错,句句入理,这几日的芥蒂也稍稍释怀,便道:``长街的风大,你别站在风口上。''

如懿盈盈道:``臣妾多谢皇上关怀。只是此事突然,又出在延禧宫门外。未免张扬,皇上和贵妃若想问什么,不如先移驾延禧宫中。臣妾屏退众人,皇上与贵妃慢慢处置便是。''

皇帝见王钦被人按在地上,满脸通红,似有醉意,也不便再拖去别的地方,便道:``那朕就借你的延禧宫一用。''

如懿答了``是'',侧身让了皇帝与慧贵妃进内,惢心与阿箬、三宝忙不迭地收拾干净了,又奉上茶水。

皇帝在正殿坐了,轻嗅几下道:``如今还在冬月里,怎么你殿中有一股子花草清馨,闻着倒很舒坦。''

如懿淡淡笑道:``臣妾闲来无事,所以配了些花草汁子,让皇上见笑了。''

皇帝颇为意外,扬了扬眉道:``朕禁足了你,你心思倒还闲雅。''

如懿笑意清浅:``臣妾被禁足,是因为皇上要还臣妾一个清白,臣妾只需安心等候便是,心思自然不能不闲雅。''

皇帝的目光清澈如许,深深看了她一眼道:``也罢。你就坐在朕身边,一同听一听吧。''

如懿含笑谢过,吩咐三宝道:``看王钦的样子像是喝醉了,你拿冰水泼醒了他,立刻带进来回话吧。''

因事出突然,贵妃又被惊扰,皇帝也不欲多留人在殿中,只许贵妃随身的侍女茉心、自己的贴身太监李玉在内伺候着。

贵妃一见人少,便忍不住泪如雨下,呜呜咽咽地不肯再多说一个字。皇帝便道:``你一见朕便说受了天大的羞辱,如今又不肯说到底是什么委屈,你叫朕怎么帮你?''

见慧贵妃只是垂泪不已,茉心忍不住膝行上前道:``方才贵妃娘娘从景阳宫看了怡贵人过来,想着娴妃娘娘禁足,心下不忍,所以过来看看,也当尽了姐妹之情。今日贵妃娘娘刚从昭华门过来入了延禧宫前的甬道,谁知王钦从后头苍震门赶了过来,没头没脑地就往贵妃娘娘身上扑,嘴里还说着不干不净的话。''

贵妃伸出衣袖泣道:``王钦简直如疯魔了一般,一上来就撕扯臣妾的衣裳。皇上看臣妾袖口,都被他拉扯破了。''

如懿诧异道:``王钦今日不当值么?怎么从苍震门过来?''

李玉忙躬身道:``是。今夜不是王公公当值,所以他一早便回去歇息了。''

正说着,三宝和小福子拖了半醒半醉的王钦进来。王钦身上全湿透了,显然是被泼了一身冰水,看着比刚才清醒了许多,一张脸却是涨成了猪肝色。

如懿掩鼻道:``王钦并非不认识慧贵妃,素来也礼敬有加,这中间是不是有什么误会?''

皇帝厌弃地看了一眼道:``看他这个样子,像是灌饱了黄汤发酒疯了!''

李玉忙凑上前闻了闻道:``皇上,这气味不像是酒味儿,倒是甜甜的,蜜汁似的味道!''

王钦挣扎着起身,刚向皇帝磕了个头,转脸看见茉心跪在自己身边不远处,嘴角不由得淌下一丝晶亮的涎水,歪着身子向茉心扑去,伸手就要摸她的脸。

茉心大惊失色,也顾不得规矩,一下缩到了慧贵妃身后,拼命尖叫道:``小主救奴婢,小主救救奴婢!''

皇帝忍无可忍,怒喝道:``王钦,你发什么疯!''

皇帝此言一出,李玉一把扯住了王钦,奈何王钦力气颇大,满嘴里哼哼着极力挣扎,看着茉心的眼睛像冒着红色的火焰,贪婪地一寸也不肯挪开。

如懿情急道:``三宝,小福子,快把他拖到廊下按住,不许进来。''

贵妃又惊又羞,悲从中来:``皇上,方才王钦那个狗奴才就是这样看着臣妾扑过来,他\ldots\ldots 他\ldots\ldots{}''

贵妃哽咽着说不下去。皇帝的眼中尽是阴郁的怒火,灼灼即可燎原。李玉忙道:``皇上,王钦这个样子怕是什么都问不出来了。他今日既不当值,便是在自己屋子里,奴才记得他的对食莲心也不当值,估计传莲心来问一问,便知道王钦究竟是发了什么疯了。''

皇帝鼻翼微张,额上的青筋急促地跳动着,极力压抑着怒气道:``你去传莲心,再让人传太医来,看看那个狗奴才到底发了什么癔症才这般胆大妄为!''

李玉躬身退下。如懿见慧贵妃的绢子哭湿了,便将自己的解下递与她跟前,道:``贵妃姐姐别恼,莲心和李玉所住的庑房就在附近,一会儿便到了。姐姐先擦擦眼泪吧。''

皇帝便在眼前,慧贵妃见如懿一脸的似笑非笑,亦不好发作,只得恨恨接过了绢子撂在一边。

沉默等待的须臾,如懿示意阿箬送上茶水,贵妃喝了一口,便皱眉道:``凉丝丝的,什么怪味儿?''

如懿的笑意温婉而柔和:``回贵妃娘娘的话,是薄荷蜂蜜茶,我宫里正好煮了些薄荷汁,兑了蜂蜜拿绿茶泡了,喝下去宁神静气,舒缓郁结,是最适合不过的。''

阿箬的茶正好递到皇帝手边,一时犹豫道:``皇上要不要尝一尝,若是不喜欢,奴婢再换别的来。''

皇帝正气郁难解,随手接过道:``不必麻烦了,娴妃的一番心意,朕喝这个就好。''他的手无意拂过阿箬的手背,阿箬面上一红,忙屈膝告退了。如懿正看着慧贵妃,一时倒未察觉。茶过半盏,只听推门声近,李玉已带了莲心过来了。

莲心一进来便慌慌张张的,心慌意乱地跪下了道:``皇上,贵妃娘娘,娴妃娘娘,王钦是不是发了疯冲撞了人了?但请皇上和各位小主别见怪,饶了他这遭吧。''

慧贵妃秀眉紧蹙:``你这样问,便是知道王钦为何如此癫狂,是不是?''

莲心脸上红一阵白一阵,只是羞愧难当,低下头哭个不止。李玉便道:``皇上,太医也已经来了,在给王钦查看,奴才立即请他进来。''

皇帝微一颔首,李玉已开门召了太医进来,太医亦是大惊失色,磕了头道:``皇上,微臣已经给王公公搭过脉,他不是酒醉,而是服食了过多的阿肌苏丸所致啊!''

慧贵妃微蹙着淡淡烟眉,疑道:``阿肌苏丸是什么?''

太医满面惊惶,不知该不该答,却看皇帝与慧贵妃皆是一脸疑惑,只得硬着头皮道:``此物是外头坊间的秘药,以蛇床子、川芎、淫羊藿所成\ldots\ldots{}''

皇帝立时明白过来,不觉满面铁青,切齿道:``大胆!''

慧贵妃虽不如皇帝醒转得快,却也渐渐明白过来,不觉羞得满面通红,起身便踹了莲心一脚,恨恨道:``王钦吃这种不要脸的东西,必然你们俩是一伙的了。皇后好心赐你们对食,你竟敢如此不知廉耻,淫乱后宫!''

莲心又羞又气,只是不敢言语。如懿忙抬了抬眼示意太医和茉心出去,温言道:``这里已经没有旁人了,你有话就说吧。''

莲心看了看李玉,窘得眼泪直落,还是不肯开口。皇帝道:``留在这儿的李玉是个没嘴没耳朵的,离开了延禧宫的正殿,他便从没听过这件事,也不会对任何人说。你放心说你的就是。''

莲心这才放心,整个人软在地上,呜呜咽咽道:``皇上,皇后娘娘本是好心,希望奴婢终身有靠,所以将奴婢指婚给了王钦做对食。奴婢也是嫁了才知道,原来王钦人模狗样,居然连畜生都不如。他本是个太监阉人,却一心想要做个男人,在奴婢身上作威作福,肆意打骂不说,还偷偷弄来了这些奇淫技巧,一一施加在奴婢身上,害得奴婢生不如死!''

皇帝轻轻咳嗽一声,李玉即刻会意:``奴才立刻带人去王钦的庑房搜查。''说着便匆匆去了。

贵妃一脸嫌恶,拿绢子挡着脸道:``王钦这样不知好歹,你怎么不去告诉皇后,求皇后为你做主?''

莲心哀哀哭道:``奴婢虽然是宫人,但也要脸面。这样的事,怎有脸对外人说去,更不敢辜负了皇后娘娘的恩典,污了娘娘的清听。而且王钦还说,只要奴婢敢吐露半个字,他必定要让奴婢生不如死。''她说着便褪下衣衫,侧身露出肩膀与背心,只见上面满布牙印与指甲的掐痕,直至肌理深处,如被野兽挠抓,伤痕累累,惨不忍睹。

如懿忙取下自己的大氅替她披上,莲心哭得难以自抑:``奴婢白日在皇后娘娘处当差,晚上还要受他如此折磨。光是这样打骂也罢了,后来王钦不知道从哪里搜罗来一些脏药,坚信服食长久之后便会有些男人的效力,每每他自己服食后便要无休无止地折磨奴婢。''莲心动了伤心,索性将嫁与王钦后的苦楚一一诉来。

众人越听越是惊骇,一壁叹息不已。过了一炷香时分,李玉便领了小太监进来。李玉垂手候在一旁,小太监则手捧一个黄杨木盒子站在李玉身侧。

皇帝越听越怒,眉心隐隐有暗火跳簇,道:``那么今日,又是为何?''

莲心哭得差点哽住:``今日王钦不当值,一回到庑房就开始喝这个东西。奴婢正要回房,在窗外看见他这样,便吓坏了。奴婢一时也不敢回去,又不用回长春宫当值,只好在附近徘徊。王钦服食了那些脏东西后四处找不到奴婢,大约是药性发作,发了狂似的跑了出来,奴婢这才敢偷偷回庑房。''

慧贵妃气得满面紫涨,跪倒在皇帝膝下,忍不住泪如雨下:``皇上,皇上,您一定要为臣妾做主。王钦敢在宫内服食这种淫乱之物,冲撞臣妾,简直应该碎尸万段!''

李玉听到此节,方才指着小太监手里的黄杨木盒子道:``皇上,奴才奉旨去王钦房中搜查,一搜便搜到这一大盒污秽东西,奴才实在是闻所未闻,见所未见。奴才不敢擅专,立刻捧来请皇上过目。''说罢,他亲自捧过盒子走到皇帝身边,只对着皇帝一人打开。

皇帝只看了一眼,脸上的肌肉不自觉地搐起,和太阳穴突起的青筋一般,昭示着他发自心底的愤怒。李玉立刻盖上盒子,适时地添上一句:``自从王钦被赐对食之后,总在奴才们面前吹嘘自己有男儿雄风。原来就是凭这些污秽东西!''

皇帝唇齿间吐出的话语如尖锐的冰凌:``召集满宫的内监入慎刑司,看着王钦挑断手筋脚筋,再`贴加官',看哪个不知死活的东西,还敢秽乱后宫!''

所谓``贴加官'',便是由司刑之人将桑皮纸揭起一张,盖在受刑之人脸上,然后嘴里含着一口烧刀子,使劲喷出一阵细雾,桑皮纸受潮发软贴服在脸上,紧接着又盖第二张,如法炮制。直到七张叠完,受刑之人便活活窒息而死。那七张纸叠在一起一揭而下,凹凸分明,犹如戏台上``跳加官''的面具,保留着受刑之人临死的可怖形状。

如懿保持着矜持沉静的容色,略含了一分厌弃与嫌恶,只是在视线与莲心对上时,露出了一分不动声色的笑容。

\hypertarget{ux7b2cux4e94ux7ae0-ux4e09ux96d5}{%
\chapter{第五章 三雕}\label{ux7b2cux4e94ux7ae0-ux4e09ux96d5}}

皇帝看着慧贵妃,有几分漠然的疏远:``好了。朕已经处置了王钦,你也不必哭了。先回宫去吧。''

慧贵妃满腹委屈,想要再说什么,皇帝只是那样淡漠而疏离的口吻,挥挥手道:``朕会再去看你的,你回去吧。''

慧贵妃只得依依告退。如懿看着神色悲戚的莲心道:``皇上,此事王钦有大罪,莲心却只是无辜受害。无论是谁被赐婚给王钦为对食,都逃脱不了这样的命数。还请皇上看在莲心伺候皇后娘娘多年的份上,不要再责罚莲心。''

皇帝微微颔首:``朕知道,朕不会责怪莲心。''他的目光里有浅浅的哀悯,``朕便解了你与王钦的对食,你还是在皇后身边伺候吧。''

如懿悲悯地摇摇头:``皇后娘娘当年也是好心,想让宫中的宦官宫女彼此有个依靠。王钦本也不是个十恶不赦的坏人,只是为何别的宦官从未有这样的事,偏王钦就有呢?想来是他对食之后有了妻室,又感自身残缺,才平白生了这贪色污秽之心。依臣妾看来,王钦固然罪不可赦,对食之风亦不可长。免得宫中再有这样可怖之事。''

皇帝端过茶水慢慢啜了一口:``你的话也有道理,朕回去会再思虑。''他起身道:``天色不早,朕还要去嘉贵人处。你早些歇息吧。''

如懿送皇帝到了廊下,屈膝道:``臣妾身陷流言之祸,乃禁足之身,不宜相送太远。在此恭送皇上了。''

莲心本跟在皇帝身后出去,听得这句,忍不住回头道:``娴妃娘娘所言,是关于玫贵人生子的流言么?''

如懿淡薄的笑意如绽在风里的颤颤梨花:``流言纷扰,本宫亦只能静待水落石出而已。''

莲心``扑通''一声跪下,伏下身爬到如懿脚边,忍不住痛哭道:``娴妃娘娘,请万万宽宥奴婢\ldots\ldots 奴婢的隐瞒之罪。''

如懿一脸疑惑:``你可曾向本宫隐瞒了什么?''

``奴婢\ldots\ldots 奴婢知道玫贵人生子的流言的的确确不是您传出,而是王钦那日做完了差事喝了几口黄汤,自己喝醉了胡说出来的。只是\ldots\ldots 只是奴婢从前深受王钦之苦,所以一直不敢说出来。请娘娘恕罪\ldots\ldots{}''莲心说完便像捣米似的不停地磕头。

皇帝立时停住脚步,转身道:``是王钦?那为何宫人们都说最早是在延禧宫一带传出?\textless{}''

莲心一脸诚挚:``延禧宫是王钦回庑房的必经之路,他那日喝醉了躺在延禧宫外的甬道边满嘴胡说,奴婢找到他时他还烂醉如泥呢。怕正是如此,所以旁人经过听见,还以为是延禧宫传出的流言呢。''

皇帝似是相信了,问道:``此话当真?''

莲心忙磕了头道:``奴婢不敢妄言。皇上圣裁,这件事知道的人不多,皇上皇后自然不会告知奴婢,奴婢与延禧宫也素无往来,若不是王钦胡说让奴婢知道,还有谁会说与奴婢听见?''

皇帝立刻伸手止住李玉:``不必传辇轿,朕今晚留在延禧宫,不去嘉贵人宫中了。''

莲心与李玉知趣,立刻退下。

皇帝目中的愧疚泛起于眼底的清澄之中,握住如懿的手:``如懿,是朕误会你了。''

如懿嫣然一笑,明眸中水波盈动,已微微含了几分清亮的泪意:``那臣妾是不是该唱一曲《六月雪》,以显得自己比窦娥还冤?''

皇帝执着她的手:``朕不怀疑自己,也没有疑心皇后,甚至来不及疑心王钦,他就带了人言之凿凿地过来,让朕只能疑心你。所以朕只能禁足你。''

委屈又如何?怨又如何?如懿再清楚不过,在君恩重临之时,她过多的委屈与哀怨都是春风里的一片枯叶,不合时宜的。

如懿将心底的委屈按捺到底,露出几分浅如初蕾的笑意,那笑意薄薄的,好像春神东君的衣袖轻轻一拂,也能将它轻易吹落:``皇上曾经对臣妾说过,要臣妾放心。哪怕这一次的事皇上没有说,臣妾也会认定皇上会让臣妾放心。所以臣妾也知道,禁足这些日子,臣妾的供应一概不缺。事情的水落石出只是早晚而已。臣妾相信,哪怕真到了所有人所有事都指着臣妾的那一日,皇上也会保护臣妾周全的。''

皇帝轻轻拥住她:``你说的,便是朕想的。若真有那一日,朕也会护着你的周全。''

夜色如同幽暗海洋,一望无尽。浮云散去后,一轮新月愈发明亮起来,满天繁星更似一穹随手散开的碎钻,天上的星月光辉与琼楼玉苑内的灯光交织相映,仿佛是彼此的倒影。璀璨夺目,迷乱人眼。月华洒在皇帝的赭褐色织锦龙袍上,慢慢生出一圈朦胧的光晕来。

如懿伏在皇帝胸前,看着廊下风声萧瑟,吹动枝影委地,她无心去想前因后果,也知道自己不该去想。便索性,露出了一丝如愿以偿的微笑来。

如懿的禁足解了之后,渐渐有了一枝独秀的势头。王钦冒犯慧贵妃被处死后,皇帝不止少去咸福宫,连皇后宫中也甚少踏足了。

这一日如懿正坐在窗下,看着日色晴明如金,不觉笑道:``春天来得真快,这么快桃枝上都有花骨朵儿了。''

惢心捧着晒好的丝线进来,笑得娇俏:``可不是?人人都说春色只在延禧宫呢。若要放宽了说,景阳宫也是。所以人人都指望着东六宫的恩宠呢。''

如懿笑着道:``什么东六宫的恩宠,皇上不过多来咱们这儿几次罢了。你告诉底下人,不许骄矜。''

惢心将晒好的一大把丝线堆到紫檀几案上慢慢理着,抿嘴笑道:``这个奴婢自然知道。只是从前慧贵妃最得宠,如今皇上也不去她那儿了。''

``这次是把香味都染进去了,终于可以用了。''如懿伸手拨了拨丝线,轻轻嗅着指尖的气味,徐徐道,``慧贵妃是聪明反被聪明误。她若真是聪慧,那日被王钦冒犯后就该一言不发,一滴泪也别掉,静候皇上处置。''

惢心托着腮好奇道:``小主为何这样说?但凡女子受辱,可不都要哭闹?''

``是啊。她越是当着皇上的面委屈落泪,皇上听莲心说起王钦如何肆虐之时,便会想起慧贵妃的眼泪,想起她那日差点受了王钦的冒犯。作为一个男人,如何能忍受?''

惢心抿着嘴,藏不住笑意似的:``所以那日小主是选准了贵妃会经过咱们宫门前奚落,才特选了那样的时机。本来奴婢还想着,是皇后娘娘赐婚对食的,这样的事落在皇后身上,叫她身受惊吓,才算痛快呢。''

如懿笑着摇摇头:``皇后不比慧贵妃那样沉不住气,而且这事只有落在慧贵妃身上,才会让皇上迁怒皇后,觉得种种是非都是由皇后赐婚对食而起,皇上才会连着长春宫一起冷落。''

惢心会意一笑,低低道:``只有这样,才能拉下贵妃与皇后,又惩治了王钦,解救了小主自己,一箭三雕。''

如懿冷冷道:``我的初衷从来不只是为了搭把手救莲心,顺带着除了王钦这个隐患,而是要绝了宫中的对食之事。当初流言之祸,皇后表面要救我,请求皇上只是将我禁足,实际上是将我置身于不能自救之地。既然如此,我小惩以戒,既是保全自己,也不能让人将延禧宫践踏到底。''

惢心暗暗点头:``也只有搅清了这趟浑水,皇上才会相信娘娘与流言无干,才算真正安心了。''

如懿慢慢挑拣着丝线比对着颜色,笑道:``你看这一把丝线,光一个红色便有数十上百种色调,若一把抓起来,哪里分得清哪个是胭脂红哪个是珊瑚红。非得放在了雪白的生绢上,才能一目了然。''

惢心会意微笑:``所以小主得留出空当来,让皇上分清了颜色,才好决断。''

如懿微微一笑,缤纷多彩的丝线自指尖如流水蜿蜒滑过,轻巧地挽成一把,悬在紫檀架子上,任它如细泉潺潺垂落。``禁足也好,幽闭也好。外头既然流言纷乱,直指于我,那我便顺水推舟,稍稍回避自然是上上之策。''

``可是小主真的从不担心么?小主被禁足,外头自然就由得他们了,万一小主受了他们的安排算计,坐实了玫贵人诞下妖孽这一流言滋扰宫闱的源头,即便皇上要保全您,也是保不住的。''

如懿纤细的手指微微一挑,拨出一缕鲜艳红色挽在雪白的指间:``他们要安排布置这样的事,光是一两日是不成的。我只要乖乖待在延禧宫中,那么即便他们有事,也不干我的事了。你细想想,我出事必然是他们所害,他们有事却一定与我无关,这样的好事,换了你,你愿不愿意赌一赌?''

惢心抿唇一笑,替如懿捧过一把绿色的丝线慢慢拣选:``奴婢不敢赌,奴婢只安心跟着娘娘就是了。''

如懿描得细细的黛眉飞扬如舒展的翅:``也亏得莲心乖觉,不仅告发了王钦淫乱宫闱,冒犯慧贵妃。还说他总酒后胡言,胡乱吹嘘,流言之事出自他口。何况不论是与不是,皇上心里已经厌弃了这个人,便会认定是他做的。''

惢心微微蹙眉:``玫贵人这件事,知道的人除了皇上、皇后,便是小主和王钦。难道小主从未怀疑过是皇后\ldots\ldots{}''

如懿冷冷一笑,将丝线在手指上细细一勒,森然道:``我何尝没有怀疑过?只是皇后不是我能动得了的人。不管利用流言来害我的人是不是她,我都只能先断其臂膀!''

``但是莲心\ldots\ldots{}''

``莲心一心只想除去王钦,她是皇后的家生丫环,又是陪嫁,有父母族人在,一时间她是不敢背叛皇后的。也好,只要人不犯我,我必不犯人,便先留着她,当做一道防范吧。''

这一日皇帝与皇后携了六宫嫔妃往太后处请安。太后着意安慰了怡贵人一番,便命福珈从里头端了一个垫着大红绣绒的红木漆盘来,上面安放着一枚麒麟送子金锁,捧到怡贵人身前道:``《诗经》有云:麟之趾,振振公子①。哀家就送一枚麒麟金锁给你,希望你早日为皇上添一位阿哥才是。''

怡贵人喜不自禁,忙起身谢过。

皇帝亦颇喜悦,道:``麒麟,含信怀义,步中规矩,彬彬然动则有容仪,更是送子的神兽。皇额娘的礼物,实在是心意独到。''

慧贵妃笑着抚了抚领口的翠玉流苏佩:``太后的心意怡贵人必然是心领了。其实阿哥公主又何妨,只要母子平安,不要像玫贵人一般福薄就是了。''

太后伸手拨着手边几案上新开的簇簇迎春,金英翠萼,枝条舒曼,已带早春暖凉的气息。太后唇边的微笑亦是这般乍暖还凉:``皇后一向不喜奢华,哀家看这些嫔妃们所用的首饰也是银器鎏金为多。哀家赐怡贵人赤金的麒麟锁,皇后不会嫌哀家老糊涂了吧。''

皇后忙起身恭谨道:``皇额娘一片心意,儿臣怎敢这样想呢。何况怡贵人有孕,皇额娘爱护怡贵人,等同是爱护臣妾。''

太后微微一笑:``宫中祥和平安,乃是皇后的德行所致。听说皇后为使后宫嫔妃多有子嗣,让太医院多多熬制了坐胎药每日送到各宫,也是有心了。''她转首向皇帝道:``前几日是二月初二龙抬头的日子,哀家命人夜观天象,祈求祥瑞。不知钦天监可将结果对皇帝说了?''

皇帝扬起几分欢悦之色,道:``钦天监说天象祥和,尤其指北天女宿星尾带小星,连续数月格外明亮,乃是指后宫女子怀有大贵之胎。儿子心里也十分安慰。''

太后笑吟吟道:``女宿星本来形如蝙蝠,主福兆、多吉。而后宫女子怀有身孕的,只有怡贵人而已。看来这一胎也的确是大福之相。''

这样说来,怡贵人更是喜不自胜,慧贵妃不屑地撇了撇嘴,冷着脸不言不语。皇后倒是一脸欣慰道:``如此,臣妾就要向太后和皇上求个恩典了。怡贵人伺候皇上多年,她的位分\ldots\ldots{}''

皇帝爽朗笑道:``等怡贵人生育之后,无论男女,朕一定会给她嫔位,居景阳宫主位,如何?''

太后含笑道:``如此甚好。哀家也希望后宫嫔妃能多有生养,为皇家开枝散叶才好。''

如此寒暄几句,太后又格外叮嘱了怡贵人保胎事宜,便也散了。

才出慈宁宫仪门,皇帝便低低向如懿道:``昨儿江南进贡了些好茶来,朕都赐予你了。趁现在得闲,不如你烹茶给朕品尝,如何?''

如懿低眉浅笑:``臣妾倒不怕皇上不来品茶,只是您已经好些日子没去长春宫了。前几日是二月初一,您本该在皇后宫中过夜的,却也只是去略坐了坐就回了。''

皇帝正要说话,只听皇后疾步上来,请了安道:``皇上万福。''

皇帝笑容一敛,淡淡道:``春寒料峭,皇后还不回自己宫中么?''

皇后颇有为难之色,踌躇片刻,还是道:``皇上,您已经多日没有去臣妾宫中了。臣妾愚昧,不知皇上是不是因为莲心受王钦凌虐之事怪责臣妾?''

莲心跟在皇后身边,忙跪下道:``皇上圣明,奴婢受这些苦楚只是奴婢自己命薄罢了,而且奴婢也不敢告诉皇后怕她担心。王钦出事之后皇后娘娘才知道奴婢吃的苦,十分怜惜自责,还亲自为奴婢上药,奴婢感激不尽。所以王钦的事实属奴婢自己命苦,不干皇后娘娘的事啊!''

皇帝看向皇后的神色多了一丝温意,他和缓道:``皇后你一开始也不过是好心,怜悯宫人孤苦,但却未能知人善察。莲心在你身边多年,你一时失察,不仅连累莲心吃尽苦头,而且宫中歪风也由此而起。朕不能不想到,这是皇后之失。''

皇后站在风口,穿道而过的冷风拂乱了她梳得一丝不乱的精致华髻,几绺墨色青丝拂上她没有血色的面庞,仿若一朵凋零在初秋的冷荷。

皇后躬身福了一福,将眼中微冷的泪光转成自持的冷静:``的确是臣妾失察,臣妾会面壁思过,再三自省。''她屈膝下去:``那么,臣妾恭送皇上了。''

皇帝在如懿处品茗过后,便回了养心殿处理政务。如懿闲来无事,便取过染上香气的丝线一针一针地绣起繁天春色。

阿箬捧着刚燃好的一炉香进来道:``小主失宠的时候也刺绣,如今得宠了忙着陪伴皇上还不够呢,怎么又开始刺绣了?''

如懿微微一笑,取了针线拈好道:``失宠的时候要让自己学会平心静气,得宠的时候亦要告诫自己,不能心浮气躁。刺绣便是如此,一个眼错,便是全局皆毁;一枚针斜,恐怕扎伤的就是自己。所以动心忍性,一步都不可错。''

阿箬若有所思地笑笑,取过一枚烘制好的莲花香饼放进炉中,又覆上云母隔片隔开香饼炭火,滴入一两滴凝露状的蜂蜜:``如今入春了,时气干燥,焚香时滴入蜂蜜,可以清热润燥,小主觉得好不好?''

``如今你的心思越发安静了,做事也更妥帖,自然没有不好的。''如懿浅笑,想了想又道,``怡贵人有孕后喜爱焚檀香,今早说起檀香虽好,但焚香后总觉得气燥体热,她又是个贪吃甜食的。我记得小厨房有去岁备下的槐花蜜,清热凉血是最好不过的。等下你便随我送一瓮去给她吧。''

阿箬笑道:``别的也罢了。那槐花蜜是去岁的时候特意着人去京郊找了一大片槐花林,取雪白洁净的盛开花朵剔干净了,加上适量的嫩桑叶蒸出来的槐花露。奴婢记得槐花最娇气,成百上千棵树上摘下的花儿也经不起那几蒸,最后只得了两小瓮槐花露,再用长白山产的野蜂巢里的蜂蜜炼了,只为小主从前有血热的症候,才这么不怕费事地制了。统共就那么点子,小主还要拿去送人。''

如懿嗔道:``如今怡贵人是皇上的心头肉,连太后都格外高看她些。我也想着,若是怡贵人这一胎安好,皇上也解了上回玫贵人产子的心结,这便是好的。''

阿箬笑道:``旁人怀孕有什么好的。从前怡贵人一点也不得宠,如今有孕皇上便这么抬举了。要是小主也趁着眼下圣眷正隆,赶紧怀上一胎,那才是真正让皇上高兴的呢。还不知道皇上要怎么当眼珠子似的捧着爱也来不及了。''

如懿笑着嗔她一眼:``越发爱胡说了。''

正说着,小宫女绿痕端着汤药进来道:``刚熬好的药,小主快喝了吧。''

如懿轻轻一嗅,蹙眉道:``一闻味道就知道了,就是坐胎药的气味。''

阿箬取过几样酸甜蜜饯放在如懿手边,好声好气道:``这坐胎药是催孕的,再苦咱们也得喝啊。您看,奴婢连雕花金橘和糖渍乳梨都预备下了,小主赶紧喝了吧。''

如懿端过碗仰脸喝下,又用清水漱了口,连忙取过蜜饯含在嘴里缓了一阵,方道:``这坐胎药一碗碗喝下去,连舌头底下都发苦了,真不知道什么时候才会有孕?''

阿箬笑道:``只要皇上常来,那股子运气迟早都会到。小主喝了药,咱们就去景阳宫沾沾孕气吧。听说慧贵妃虽然不满天象说怡贵人是大贵之胎,但为了沾上孕气,也常常去景阳宫呢。''

如懿扶过阿箬的手笑道:``既然如此,你便带上那瓮槐花蜜,陪我去景阳宫看看吧。''

景阳宫便在延禧宫与永和宫之后,如懿看着天色极好,便带了宫人步行过去。因着怡贵人有孕,景阳宫也格外地布置一新,才走到宫墙外,便见朱红宫墙耸立,连琉璃瓦也显得一碧如洗。

如懿仔细看了两眼道:``好喜庆的颜色,这墙是新粉了颜色吧,好似特别鲜艳些。''

迎上来的小太监笑得灿烂:``可不是,皇后嘱咐了,颜色要喜庆,这才吉祥呢。''如懿扶着阿箬的手入了重重朱门,只见雕栏华彩,描赤敷金,鲜华异常。

如懿暗暗点头道:``果然怡贵人有孕,景阳宫也不同往日了。''她转首问小太监:``这个时候,怡贵人在做什么呢?''

小太监道:``贵人身上疲倦,此刻正在暖阁歇着呢。娴妃娘娘请。''

如懿正要迈入正殿,忽听得里头一声惊惧的尖叫,竟是怡贵人的声音。

注释:

①选自《国风·周南·麟之趾》,全文为:``麟之趾,振振公子,于嗟麟兮。麟之定,振振公姓,于嗟麟兮。麟之角,振振公族,于嗟麟兮!''这是一首赞美诸侯公子的诗。

\hypertarget{ux7b2cux516dux7ae0-ux60caux86f0}{%
\chapter{第六章 惊蛰}\label{ux7b2cux516dux7ae0-ux60caux86f0}}

众人面面相觑,一时间尚不知发生了何事。如懿醒转得快,立刻道:``是怡贵人的声音,还不快进去看看!''

如懿一时情急,即刻带了人先赶进去,才进暖阁,却见怡贵人吓得缩在暖阁的紫花梨卷草纹杨妃榻上,身上的锦被蜷成一团,她才唤了一声``怡贵人'',却见怡贵人大惊失色,整张脸白中泛着青灰,指着地上的绣毯呼道:``救我!娴妃娘娘快救我!''

如懿的目光触及地下,吓得几乎倒退几步,宫人们也止不住惊呼起来。原来绣毯之上,一条灰花斑斓的蛇盘绕其上,咝咝地吐着猩红的芯子,在地上摇摆不定。

一个小太监惊呼道:``呀,这是蝮蛇,是有毒的!有毒的呀!''

众人吓得退开十数步远,怡贵人眼看那蛇越游越近,吓得几乎要晕厥过去。如懿心中慌乱不已,眼看那蛇一分分向怡贵人靠近,更是害怕。万一伤及怡贵人腹中的胎儿,皇帝才稍稍平伏的心情又不知要低落成何种模样。

她心下一横,吩咐身边的小太监道:``你们宫里有没有雄黄粉?''

那小太监忙不迭道:``有有有!这是宫里常备着的。''

如懿忙吩咐了他拿了雄黄粉来,照准那条蛇便泼了过去。那条蛇乍然受了雄黄的气味,一时行动有些滞缓,如懿忙伸手取过碧纱橱边一根宫人扫尘灰的掸子,挑起那蛇的身体一撂,照着门口泼了出去,即刻道:``快找人拿大石砸它的七寸,务必砸死为准。''

太监们原本吓得神魂未定,听如懿这样吩咐,忙抱过雄黄粉撒的撒,寻石头砸的砸,不过片刻便将那条蛇处置了。

怡贵人呆呆地看着如懿,片刻才放声大哭,扑入如懿怀中,神色败坏:``娴妃娘娘,娴妃娘娘,多谢您救了嫔妾!''

如懿忙拿锦被裹住了她扶进寝殿躺下,方问道:``到底出了什么事?怎么会忽然有条毒蛇在你暖阁里?\textless{}''

怡贵人神色恍惚道:``嫔妾本觉得困乏,在暖阁里歇息,并没让人伺候在侧。不承想梁上忽然掉下一条蛇来,嫔妾当下便吓得叫起来。''

如懿替她抚着心口,自己也是惊魂初定:``那条蝮蛇是有毒的,若是被它咬伤一口,不只是你,便是你腹中的孩子,后果也是不堪设想。只是好端端的,宫中怎会有毒蛇?''

阿箬替怡贵人端了茶水来道:``贵人喝盏茶压压惊。今儿是惊蛰,想来什么蛇虫鼠蚁都出来了。贵人有孕怕冷,宫中还供着地龙,格外暖和,怕是因为这个招来了蛇也是有的。''

怡贵人接过茶才喝了一口,不由得手中一松,整盏茶都泼在了如懿身上。如懿还顾不得擦,却见怡贵人蜷成了一团,一手死死抓住她手,一手按住了肚子痛呼道:``好痛!我的肚子好痛!''

皇帝与皇后赶来时,太医已经为怡贵人开了安胎的方子。景阳宫中人心惶惶,如懿一时也走不脱,一壁嘱咐了宫人们延医请药,一壁又吩咐太监们在墙根角落里遍撒雄黄与石灰驱蛇。

皇帝步履匆匆地进来,足下之风几乎惊起了静尘,如懿正守在怡贵人床头,见皇帝心急火燎进来,忙起身道:``皇上万福,皇后万福。''

皇帝忙扶了她起身,关切道:``怡贵人如何了?''皇后亦心急不已:``太医已经来过了么?怎会又是遇蛇,又是腹痛,本宫从阿哥所过来,一路上都心悸不已。''

如懿忙道:``俗话说,惊蛰到,蛇出洞。今儿景阳宫里竟不知从何处冒出条毒蛇来,怡贵人骤然受惊牵动胎气,太医开了安胎药服下,怡贵人已小睡片刻,现下应无大碍了。''

皇帝见怡贵人睡中仍有惊惧之色,不免怜惜道:``怡贵人初初有孕,身体百般不适,今日又遇见这样的事,实在是要吓坏她了。''

皇后看了看周遭,担忧道:``皇上,怡贵人身怀贵胎,此番受了这样大的惊吓,实在可怜。臣妾听闻蛇乃至阴至毒之物,突然间侵扰景阳宫,怕是有什么不利。''

皇帝迟疑道:``皇后的意思是?''

皇后满面关切:``皇上,景阳宫靠近玄穹门,地气潮湿,若是往后再招来蛇虫鼠蚁惊扰了龙胎,该如何是好,依臣妾所见,不如让怡贵人迁居别宫居住。''

皇帝诧异道:``迁居别宫?一时间要打扫宫苑出来,想来怡贵人也未必能住得惯。''

皇后道:``东西六宫中有些宫殿一直未有人居住,临时理出来也不便。本来怡贵人也可迁居前头的永和宫,但永和宫大为不吉,自然是住不得的。怡贵人初初有孕,最好是能有人照拂。''她的目光往如懿脸上轻轻一扫:``今日怡贵人之事,幸有娴妃在,才能一切无恙。不如就让怡贵人迁居延禧宫中暂住,等景阳宫肃清一切邪物,再请怡贵人搬回就是了。''

皇帝微微踟蹰,看着如懿道:``延禧宫中已有娴妃和海贵人住着,又有大阿哥,再住进去会不会太挤了?''

正迟疑间,只听怡贵人微微呻吟了一声,悠悠醒转过来,见皇帝在侧,不觉落泪道:``皇上来了,臣妾今日受了这番惊吓,实在是怕见不到皇上了。''

皇帝忙安慰道:``不要胡说。朕还盼你为朕诞下一位阿哥呢。''他沉吟片刻又道:``怡贵人本是皇后房中的人,长春宫也宽敞,不如还是让怡贵人移居皇后宫中吧,有皇后照顾,朕也能安心。''

皇后转脸拭了拭眼角,不觉含了两分悲色:``本来照顾怡贵人是臣妾分内之事。只是臣妾方才从阿哥所来,还未来得及禀报皇上,臣妾的二阿哥着了风寒,身子一直不好。臣妾正想亲自照顾,只怕分身无术,不能照顾好怡贵人,反而有负皇上所托。''

皇帝惊诧地站起身:``永琏病了,要不要紧?''

皇后一提起亲儿,不觉满面悲灼,道:``都怪臣妾疏于照顾,还请皇上允许臣妾将永琏从阿哥所接回,便于臣妾亲自照顾。等永琏痊愈之后,臣妾再送他回阿哥所。至于怡贵人,本来臣妾可以将她托付给慧贵妃。但是皇上也知道,慧贵妃虽然年长,但不比娴妃沉稳有决断,就譬如今日之事,若非有娴妃在,怡贵人的胎恐怕也不能万全了。''

怡贵人牵住皇帝衣袖,感泣道:``回禀皇上,今日幸得娴妃娘娘万事沉着,帮臣妾驱赶毒蛇。可是这个地方\ldots\ldots{}''她环视雕栏画栋的景阳宫,脸上闪过惊恐之色:``臣妾是断断不敢再住了。''

皇帝微一沉吟:``那么\ldots\ldots 如懿,朕只得让怡贵人去你的延禧宫暂住了。''

如懿知道推托不得,便道:``臣妾回去便把正殿的两间东暖阁打扫出来供怡贵人居住,但请怡贵人不要嫌弃简陋才好。''

怡贵人脸露喜色:``怎么会呢,往后可要叨扰娴妃娘娘了。''

皇后亦含笑:``如今宫中皇上最关心的便是娴妃与怡贵人,她们住在一起,皇上去看望倒也更方便了。''

如懿回到宫中便觉得闷闷的,一壁吩咐了宫人收拾出正殿的两间屋子,一壁往海兰殿中去。

海兰闲来无事,只穿着一件家常的月白缂丝凤香菊纹一斗珠长衣,拥着一个小小掐丝珐琅暖炉,正在窗下缝制香包。

如懿挥了挥手示意叶心不必提醒,转过珠帘落帐,笑盈盈道:``天气暖和起来了,怎么还抱着个暖炉,这么怕冷么?''

海兰抬头笑道:``姐姐来了。''她将暖炉递到如懿怀中:``我自己哪里用暖炉呢,是怕姐姐在景阳宫看到了什么心寒惊怕之事,所以特意备下了给姐姐的。''

如懿微微惊愕,替她正一正发髻间一枚将要垂落的攒心嵌珠绢花:``你倒灵通!''

海兰抿嘴一笑:``如今宫里的眼睛都看着景阳宫呢,有什么风吹草动是不知道的。''

如懿微微叹口气:``那么以后,所有的眼睛都要盯到延禧宫来了。''

``一个景阳宫就足以引来毒蛇环伺,那怡贵人移居之后,延禧宫岂不也成了蛇虫鼠蚁纷至沓来之地。''她拉过如懿细看桌上罗列的晒干的香草叶子,``这是薄荷叶、艾叶、半枝莲、薰衣草、天竺葵叶,都有驱虫辟邪之效,妹妹做了这些,希望可以悬挂在延禧宫中,驱邪避灾。''

如懿挥手示意侍奉的宫人们都退下,海兰亲自奉了一盏菊花茶递到如懿手中,如懿无心去饮,只得放下道:``你也觉得怡贵人突然遇蛇,十分蹊跷?''

海兰淡淡一笑,伸手拨了拨桌上的艾叶:``今日虽然是惊蛰,但宫中是什么地方,何况是怡贵人有孕,人人重视,怎会突然有毒蛇出现?又那么巧落在怡贵人休息之处?万一今日不是姐姐沉稳,那么怡贵人一尸两命,便是意料之中了。''

如懿从袖中取出绢子,上面染了一点油彩颜料,递与海兰道:``你看看这油彩有什么奇怪?''

``妹妹出身贫家,所以依稀闻过这种味道,似乎有些蛇莓汁液的气味。''海兰轻轻一嗅,旋即一惊,``民间传闻,蛇虫喜吃蛇莓,故而有蛇莓处常有蛇虫出现一说。''

如懿的叹息轻得恍如云烟:``今日我命景阳宫中遍撒雄黄石灰,谁知至我离去短短两个时辰内,已见十数条毒蛇遁走四窜。此事并非偶然。我虽不知是哪里出了缘故,但想起景阳宫内因怡贵人有孕而特意装饰华彩以表喜庆。这虽然是内务府的惯例,但不知是谁从中做过手脚,才会引来这些脏东西。''

海兰沉吟着道:``我记得景阳宫是怡贵人初初有孕时装饰的,至今已快两个月,等到油彩气味散尽,这种蛇莓汁液的气味才会明显,正好是惊蛰前后百虫出动。想来谋划这件事的人心机极深,才能事先安排丝丝入扣,让人不得怀疑。''

如懿道:``怡贵人要来延禧宫,既是她自己的意思,也是皇后属意。在怡贵人平安生产之前,延禧宫只怕有的小心。妹妹心细如尘,便要依靠你了。''

海兰紧紧握住如懿的手:``姐姐怎样保全妹妹的,妹妹必定一样相待。''如懿心中说不出的感动,只觉得宫苑重重如深海悬冰,有海兰在,亦多了一丝可以依靠的温暖。

二人正相对间,却见叶心叩门而入,端了一盏汤药进来道:``小主,到喝坐胎药的时候了。''

海兰便道:``搁下,你且出去吧。''

如懿摇头苦笑道:``这坐胎药的气味,我一闻到便害怕了。可又不能不喝,只盼望自己也有个孩子。''

海兰轻轻一笑:``我也不喜欢这个气味。好端端的,皇后发一次善心,咱们就要多这桩苦差事。''她说罢,随手将汤药倒进殿中的一盆宝珠山茶内,仿佛毫不在意似的。

如懿惊道:``妹妹这是做什么?''

海兰不以为意:``我又不盼望生子得女,喝这个劳什子做什么,省得苦了舌头。''

如懿颇为惊诧,尽量还是平缓了语气道:``妹妹也不算无宠,何不趁着年轻得个一子半女,也算终身有靠。''

海兰淡然一笑,仿佛真的是不在意:``有孩子未必就是好事了。姐姐且看怡贵人和玫贵人就知道了。玫贵人产子而遭弥天大祸,怡贵人怀着身孕还不知道是被谁所害。妹妹没有这样百计防身的好本事,还是活得安乐些就好。''

``可是\ldots\ldots{}''

海兰笑着用白若葱根似的食指抵住她的唇:``没有可是,我有姐姐可以依靠,便什么都不怕。''

怡贵人移居来之前,如懿和海兰已将延禧宫清扫一新,并在怡贵人所要居住的东暖阁多悬香包驱虫。因为只留了两间房出来给怡贵人居住,如懿心下也颇不安。幸而怡贵人性子平和,也不算是骄矜之人,又见如懿自己住西暖阁,倒把东边让给了她,心下更是感激,只嘱咐把一些贴身东西搬来延禧宫,其余器具,只留在景阳宫中,随时去拿便可。为着让怡贵人静心养胎,如懿特意叮嘱了永璜每日读书只许小声,不许喧哗吵闹。怡贵人倒是很喜欢永璜的样子,每每见到永璜便说,若是有他这么一个懂事孝顺的孩子,便也满足了。如此一来,延禧宫中虽然拥挤些,倒也十分热闹,连皇帝也是每日必来看望一次的。

如此十数日,不觉连慧贵妃亦叹息,她被皇帝冷落了许多时日,虽然每常相见,但却未再让她侍寝,她亦不免感慨,请求将怡贵人挪去她的咸福宫居住,也好得见天颜。皇帝却只是一笑,问她:``那么如果晞月你见到毒蛇,会是吓得惊叫一声自己先跑呢,还是会救怡贵人为先?''

如懿与海兰对怡贵人的胎悉心照顾,一饮一食都细细查看,连太医开的安胎药方,也另请人看过药渣,道是无妨才继续喝下去。这样检验药渣的事,惢心倒是很乐意去做。如懿便笑她:``你去找的太医,可靠么?''

惢心连连点头,眼里有微亮的光芒:``是。他是奴婢家乡的旧识,奴婢进宫后才知道他已经在太医院当了一个小小太医。虽然官职卑微,但奴婢是相信他的医术的。''

如懿笑道:``你是相信他的医术呢,还是相信他这个人?''

惢心红了面庞,只低头不语。如懿已然明白:``看来不必我替你找个婆家,你自己已然有了心上人了。''

惢心又羞又急:``奴婢不敢。''

如懿含笑道:``让他好好在太医院争气,有朝一日,我一定会成全你们。''

惢心感激地望着如懿:``那奴婢先去准备晚膳,皇上已经传过口谕,说要过来与小主一同用膳呢。''

然而这一夜,如懿等到烛火凉透,也不见皇帝前来,出去打探的三宝缩在门边一直不敢进来回话。

如懿慢慢夹了一筷子冷透了的蜜丝山药吃了,那山药本是酥滑软糯,入口即化,又兼浇了蜜丝,格外清甜润舌,可是此刻吃在口中,却只觉得那冷而滑的触感让人捉摸不定,连蜜丝也透出一缕清苦之味。她搁下筷子,只听得银筷头上的细链子玲玲作响,便道:``皇上是不会来了,是什么缘故,你直说便是。''

三宝怯怯道:``皇上从养心殿出来,正要往咱们延禧宫来,谁知看到皇后娘娘跪在螽斯门前祈福,祈求二阿哥身子早点康健,皇上才知道,原来二阿哥的风寒是越来越重了。皇上着急,当下就陪着皇后娘娘去了长春宫,然后\ldots\ldots{}''

``然后就一直在那里,没有再出来。''

三宝点头答了是,如懿舀了口汤慢慢喝了道:``螽斯门是从养心殿到延禧宫的必经之路。皇后娘娘有心求神佛保佑,为何不去宝华殿而去螽斯门这么舍近求远?皇上当然是不会离开长春宫的了。''

三宝眼珠子一转:``舍近求远自然有舍近求远的好处,一箭双雕嘛。''

如懿淡淡一笑,对惢心道:``去把饭菜热一热,我也不必饿着肚子等候了。''

惢心小心翼翼道:``小主\ldots\ldots{}''

如懿微笑:``皇后贵为六宫之首,皇上陪她,是情理之中的事。''

次日清晨,皇帝过来时眼圈下已经一圈墨黑。如懿正在用早膳,见皇帝前来,忙起身道:``没想到皇上会一早过来,并没有准备下精致膳食,还请皇上见谅。''

皇帝笑道:``无妨。你吃什么,朕便也吃什么罢了。''

如懿亲自捧了一碗配了紫姜的清粥过来,又奉上鲜xx子茶和麻酱烧饼,配了几样清爽酱菜,道:``皇上似乎昨夜没睡好,还是吃得清淡提神些才好。''

皇帝的眉宇间隐然有忧色:``永琏病了这些日子,一直不见好,朕看他那个样子,真是心疼。''他握住如懿的手:``如懿,你没有看见永琏的样子,一张小脸瘦得都脱了形。朕看着他都直想掉眼泪。''

如懿甚少见皇帝如此忧虑,心下微微一抽,便道:``皇上放心,二阿哥有皇后娘娘悉心照顾,必然会很快好转。''

皇帝颔首道:``皇后说,若永琏再不见好,便要长跪宝华殿中祈福。''皇帝顿了顿,郑重其事了神色,如懿会意,立刻示意众人退下。

皇帝正色道:``朕已经决意,只要永琏的病好起来,朕就要立他为太子,继承国祚。''

\hypertarget{ux7b2cux4e03ux7ae0-ux4f0fux53d8}{%
\chapter{第七章 伏变}\label{ux7b2cux4e03ux7ae0-ux4f0fux53d8}}

殿中沉水香的气味沉沉入鼻,如懿微微一怔,心里有什么念头还来不及起来,便已把它们死死地按了下去:``永琏是正宫嫡出,皇上立他为太子也是情理之中。''

皇帝饮了一口粥,不觉慨然:``朕自幼便知道自己不是嫡出,庶出的孩子身份到底不同,哪怕如今朕当了皇帝,坐拥天下,午夜梦回的时候仍是觉得心惊委屈。我朝自开国以来,从顺治爷、康熙爷、先帝到朕,都是庶出的儿子。朕真的很想朕的儿子是名正言顺的嫡出之子,身份贵重,无可挑剔。就当是替朕自己,完成一个幼年的愿望。''

如懿听他感慨万千,自能分辨出皇帝言下的失落与怅惘。皇帝是那样敏感的人,生性多思,幼年生涯的种种心酸缺失,即便是如今富有四海也无法弥补的。所以他才那样在意,那样执著,要去完成自己当年的小小心愿。

那么,她又怎肯去拂逆他的心思。她俯下身,伏在皇帝膝头,轻声道:``皇上想做的,那就一定要做到。那是对二阿哥好,也是抚平皇上自己的心意。''

皇帝抚着她新梳起的青丝,缓声道:``如懿,朕知道你疼大阿哥,大阿哥也争气,但他到底不是你亲生。哲妃的位次也不能与皇后相提并论。三阿哥虽然也可爱,但总笨笨的,被养得太过娇气,以后也只能做个富贵闲散宗室了。怡贵人这一胎是公主也好阿哥也好,朕都不想了,只希望他们母子平安就是。''

如懿低低答了声``是'',只是静静伏在他膝头,听着他呼吸声悠然绵长,感触他纷叠的心事如潮。

皇帝低低在她耳边道:``朕知道这样很不公平,朕和你还没有孩子。但朕真的不知道该如何去说,说出朕这么多年的心愿,让你明白。''

如懿翻过皇帝的手,将它贴在面颊上,轻声道:``皇上,臣妾都明白。以后臣妾有了和您的孩子,也只盼他一生富贵平安便是了。''

皇帝眼中有伏波似的动容与感切,仿佛是划过深蓝天际的流星,有那样璀璨的光影:``如懿,谢谢你这样懂得朕。朕也知道,这是在委屈你,可是有时候名分所在,朕也不得不委屈了。''

如懿颔首道:``那皇上要立太子之事,会告诉皇后么?若是皇后知道,一定会非常高兴。\textless{}''

皇帝摇头道:``康熙爷在时,就是因为过早公布了储君,才让诸子起了夺嫡之心。朕会和先帝一样,将太子的名字藏于正大光明的牌匾之后,等朕百年之后,群臣自然会依照这个立定储君。这样也防止太子骄矜,母家专权。所以,朕不打算告诉皇后,如懿,你也不要再和任何人提起。''

如懿望着皇帝的眼睛,颔首道:``皇上说的,臣妾都记着。倒是有一事,臣妾不能不问问皇上。王钦已死,如今伺候皇上的人可还得心应手么?要不要再从内务府选些好的来伺候?''

皇帝夹了一点小菜喝了口粥道:``李玉事事仔细,人也谦和不骄矜,朕打算再看他两个月,就将副总管太监的位子给他。''

如懿柔声道:``李玉人是机灵,也忠心,但他年轻,皇上得好好历练了才能放手重用啊。''

皇帝``嗯''了一声,听见外头人声响起,便道:``外头是什么人?''

如懿探首看了看道:``是御膳房给怡贵人送的新鲜鱼虾,都是一早送来交由小厨房亲手烹制的。''

皇帝道:``太医是说过,有孕之后要多食鱼虾,朕记得那时候玫贵人也很喜欢吃。朕昨日去看怡贵人,发现她这几天总说头昏头痛,夜不安枕,也不知是怎么回事,朕心里十分担忧。''

如懿道:``太医已经来看过,说初初有孕之人的确会如此。而且因为怡贵人夜不安枕,嘴上还发了溃疡,幸而太医已经开了清凉下火的汤药了。臣妾会叮嘱小厨房多用菊花茶和绿豆汤,希望怡贵人服下之后会舒适些许。''

皇帝蹙眉道:``玫贵人有孕之时也是心火旺盛口角溃疡,朕如今看见怡贵人,实在是心有余悸。如今皇后无暇分身,如懿,一切就需你多多照顾了。''

如懿含笑道:``皇上既放心,怡贵人住在延禧宫,便是放心臣妾了的。''

皇帝悠然长嗅:``朕当然放心。就像每每闻着你殿中才有的沉水香,朕便觉得心思宁静分明。''

如懿微微一笑:``那也是皇上恩准,只许臣妾所用罢了。''

饭毕,皇帝便起身往养心殿去。如懿想着太子一事,又念着怡贵人的身体,实在是百感交集。正想着,却见海兰急匆匆过来道:``姐姐,我刚从怡贵人那里过来,像是不大好呢,你快过去看看。''

如懿赶忙起身,一迭声吩咐了去请太医,立刻跟了海兰往东暖阁去。因着怡贵人有身子一直畏寒,虽然入了三月里,她殿中仍供着炭盆暖炉。如懿携了海兰一进去,便觉得那暖意兜头兜脸扑来,不觉生了蒙蒙一层汗意。

怡贵人裹着一条暗紫织花云锦被,整个人乏力地歪在床上,似乎呼吸有些艰难,一张脸也憋成了暗紫色,与那锦被一般无二。殿内焚着檀香,连炭盆里也扔着一把佛手,被暖气一烘,种种香气织在一起,香是香,却让人闻着有些浑浊气闷。

如懿忙吩咐道:``里头的香气太重了,快开了窗给贵人透透气。''

怡贵人紧紧拥着被子,往床里缩道:``娴妃娘娘,别开窗,有人要害我!''

如懿忙笑道:``好妹妹,这是在延禧宫,没人敢害你!''她伸手摸了摸怡贵人的脸,她身上脸上都热热的,出了好大一身汗,她忙取过绢子替怡贵人轻轻擦拭了,温声道:``你别怕,告诉本宫,刚才是不是做噩梦了?''

怡贵人畏惧地缩在床角,惊惶地指着地上道:``好多蛇,好多好多蛇要咬我!''

海兰忙摘下银帐钩上悬着的一个香包,笑道:``你别怕,延禧宫里挂了好多驱蛇的香包,蛇一闻到气味就跑了,你安心住着就是。''

海兰看了看怡贵人,有些担心道:``怡贵人似乎有些发热呢,你们去取些热水来给贵人服下。''她看着怡贵人嘴角的溃疡,似乎又比昨天大了一些,便道:``太医开的清热去火的药都给贵人喝了么?怎么贵人嘴上的口子长得更厉害了。''

伺候怡贵人的环心道:``回海贵人的话,小主昨夜的晚膳贪吃了些鱼虾,那东西是发的,估计因为如此,嘴上的东西才长得大了些。奴婢也劝过,但小主说多食鱼虾可以让腹中的孩子聪明,所以宁可发些溃疡。''

海兰无奈道:``那便罢了。你们还是听我的嘱咐,平日给怡贵人服用的茶水都换成胎菊茶才好。''

正说话间,许太医便到了,如懿忙让了许太医为怡贵人看脉。许太医一径只是摇头:``小主连日来梦魇颇深,是不是?''

怡贵人乏力地点头:``自从上次惊蛰日遇蛇之后,午夜梦回,常自不安。''

许太医会意:``一旦醒来便浑身发热,虚弱无力,心悸难安,更兼因噩梦而浑身颤抖,腹中隐然作痛,可有这样的症状么?''

怡贵人眼中闪过一丝光亮:``太医说的全中了。虽然每日夜来清晨都如此不安,但白日里倒还好些。敢问太医,我为何会如此?''

许太医捋着胡须慢条斯理道:``小主初次有孕,又在怀胎三月之时受惊,导致心悸烦乱,白日有人陪着开解还好,夜来入梦难免会想起。因着多日如此,睡梦不安,小主才会内火上升,嘴角溃烂。微臣可以开些安神的汤药和外敷治疗溃疡的药物,小主只要按时服用应可无虞。''

海兰尚有些不放心:``可是怡贵人有腹痛之状?''

许太医摆手道:``初初有孕之时,的确会有隐隐腹痛,那是腹中孩子在慢慢长大,牵扯到母体的缘故,不打紧的。''

如懿忙问道:``怡贵人身上总一阵阵发热,不要紧么?''

许太医含笑道:``孕中体热,乃是常事。小主不信可以随时在怡贵人身上搭一把,任何时候都一定比各位身上都烫。所以有些女子刚有孕身之时,常以为自己风寒发热,误服汤药,以致没了孩子。其实只要看过大夫,都会无事的。''

如懿不免失笑,亦带了一分感慨:``是啊,要本宫和海兰这样两个未有生育之身来照顾怡贵人,难免有不周到之处,还得多谢许太医提点。''

怡贵人忙道:``有娴妃娘娘在,嫔妾心里已经安稳许多了。若还是留在景阳宫,那才真是后怕呢。''

海兰拍拍她的手道:``前几日我经过景阳宫,看里头已经在重新粉饰了。大约是怕有蛇虫待过,你住着害怕。等一切都装饰好了,你也平安生下了孩子,便可以安心住回景阳宫中做你的主位了。''

怡贵人微微一怔,抚着小腹含笑道:``我哪里敢奢望真能做一宫主位呢。从前在潜邸时我不过是皇后娘娘身边的小小侍女,能有幸侍奉皇上已经是老天爷格外厚待了。现在我只盼着能好好安稳入睡,来日孩子平平安安生下来就好了。''

许太医在旁开好了方子,道:``启禀怡贵人,因贵人有孕在身,微臣不敢开太烈的药,以免损伤胎儿。所以安神汤药也好,外敷治嘴角溃烂的药也好,药性都极为温和,以保贵人和胎儿安好为上,见效会比较慢一些,但请贵人切勿焦急。''

怡贵人的笑意温婉得若三春枝头一朵粉灿灿的樱花:``太医能以我和腹中胎儿为重,我又怎会怪责太医呢。''

如此,如懿和海兰便陪着怡贵人闲聊直至午膳时分。怡贵人甚是热情,索性便拉了如懿和海兰一同用膳。二人推却不得,便也一同坐下了。

因着怡贵人有孕,所有的菜品都是御膳房送了新鲜食料来,然后在延禧宫小厨房由怡贵人自己的厨娘烹制,不可谓不小心。这一日送来的午膳有瓜烧里脊、琵琶大虾、绣球干贝、炒珍珠鸭、奶汁鱼片、桂花鱼条、八宝鸡丁、香油膳糊、红烧鱼骨、鲜蘑菜心、玉笋蕨菜、砂锅煨鹿筋、罗汉酿虾丁、金腿烧鱼圆山鸡汤。

如懿看着琳琅满目一桌菜色,不觉笑道:``难怪妹妹你口角的溃疡好得这样慢,每顿吃那么多鱼虾,饱了口腹之欲,便伤了自己的嘴了。''

怡贵人不好意思道:``娴妃娘娘有所不知,嫔妾原也不喜欢鱼虾腥气,但皇后娘娘有孕的时候一直大量进食,顿顿不离,所以二阿哥如此聪明伶俐。而纯嫔娘娘怀孕的时候总嫌味腥吃得少些,以致三阿哥\ldots\ldots{}''

怡贵人没再说下去,但论起来,这也实在是纯嫔的一桩心病。三阿哥娇生惯养,学走路比旁的孩子慢,学话也是,虽然长得圆头圆脑,十分可爱,但的确是不如大阿哥和二阿哥聪明了。为着这个缘故,皇帝连纯嫔也冷落了不少,一直少去她的钟粹宫,连累了婉答应也更不受宠。而据说本与怡贵人同住景阳宫的秀答应,因为移居到了钟粹宫,也几乎见不到皇帝了。

若是生下这样的孩子,不仅不能母凭子贵,只怕也是一生的拖累吧。

这样想着,彼此也沉闷了不少。倒是怡贵人胃口甚好,一连吃了许多,倒也开怀。

一连安静了几日,皇帝因为挂心永琏的病情,也常逗留在长春宫中,对延禧宫难免有所忽略。如懿既已知皇帝的心事,只管安心照顾好怡贵人,也不再做他想。

这一晚永璜下了学,便留在如懿房中一同用了晚膳。如懿本就雅好笔墨,见永璜的字大有进益,心下也甚欣慰,便亲自看着他习字诵读。

永璜将今日所学都背与如懿听了,忽然生了几分颓丧之意:``母亲,儿子每天都在尚书房用心习读,只盼皇阿玛来查问的时候能讨皇阿玛欢喜。可是,可是,皇阿玛已经多日不来问儿子的功课了。''

如懿笑着抚了抚他的额头道:``那么你就不好好学了么?''

永璜摇头道:``那也不是。不管皇阿玛问不问,儿子都会好好读书的。''

如懿慈爱笑道:``那就是了。不管别人问与不问,你只管做好自己的事便是了。因为你是为自己活着,为自己争气的,不只是为了旁人。''

永璜似乎有些明白,用力地点点头:``儿子知道了。''

如懿微微一笑,牵过他的手道:``不过,自己用心之余,还能讨别人喜欢,自然是更好的。母亲记得前些日子皇阿玛问你在读《史记》了没有?你说已经读了是么?''

永璜道:``是啊,都已学了大半了。''

``那便好。母亲教你一首你皇阿玛的御诗。你好好记下熟读成诵,等到哪一日见到了你皇阿玛背给他听,他一定很欢喜。''

永璜立刻笑道:``那母亲快些教儿子吧。''

如懿握住他的手取过笔,把着他的手一起写下:``鹿走荒郊壮士追,蛙声紫色总男儿。拔山扛鼎兴何暴,齿剑辞骓志不移。天下不闻歌楚些,帐中唯见叹虞兮。故乡三户终何在?千载乌江不洗悲。①''

永璜好奇道:``母亲,这是写谁的诗?''

如懿不觉带了一抹甜蜜笑色:``是你皇阿玛读《项羽纪》后写下的诗,你皇阿玛感叹项羽英雄末路,自刎乌江,所以写下这首诗。你读了《史记》再能熟读你皇阿玛的御诗,他一定会很高兴的。''

永璜郑重地点点头,自己又临了一遍,末了,道:``母亲,儿子跟随你多日,如今才知道原来母亲会写字。儿子的额娘,便是字也不识的。''

如懿轻轻嘘了一声,取过一块湖蓝暗色如意云纹的宁绸料子缝制起来:``有什么本事,别一下子都拿出来。旁人不知道的,或许到了哪一天就是你的傍身之技了。若什么都拿出来让人知道了去,岂不也就让人看穿了。''

永璜的眼珠子机灵一转:``儿子明白了。''他看着如懿手中的料子,问道:``天都黑了,母亲还缝衣裳做什么,仔细看伤了眼睛。''

如懿笑道:``好孩子,你且去背你的诗吧。天气暖起来了,母亲想替你缝制一件薄些的衣裳,那些奴才们手脚太粗,针脚都留在衣裳的背面,怕磨得你不舒服。母亲自己来做,会格外留意,把针脚都塞到夹层里去,让你穿着舒服。''

永璜满脸感激,眼中含了薄薄的泪光:``母亲待儿子这样好\ldots\ldots{}''

如懿的笑容温和而慈爱:``母亲就是该待儿子好的,不是么?乖,快去读你的书吧。''

永璜坐在一旁默默诵读,如懿取过针线慢慢缝制起来,烛光摇曳,纱窗上映着桃花窈窕的枝叶,隐隐闻得见那灼灼其华、其叶蓁蓁的芬芳。

母子二人正温馨相对,忽然间外头喧哗声大作,怡贵人身边的环心面无血色地冲进来,哭着道:``娴妃娘娘,不好了,不好了!我们贵人见大红了!''

如懿陡然一凛,一颗心直直地坠落下去,像是坠进了无底的黑渊里。她听得自己的声音都变了:``怎么会这样?''

环心浑身都在发抖,像筛糠似的,得靠着墙根才能站稳:``奴婢也不知道。用了晚膳之后小主便开始腹痛,因为小主怀孕才四个月,每常也有腹痛之像,还以为不要紧。谁知今晚腹痛来得太急,才发作起来就立刻见了大红。''

``那么太医呢?去请了么?''

环心带着哭音道:``已经去请了,娘娘快去看看吧。''

如懿本能地撂下手中的东西,向外奔了几步,回头才想起永璜还在,忙道:``永璜,不管出了什么事,听见什么动静,你都不许往怡贵人那儿去,明白了么?''

她奔进怡贵人房中时,房内已尽是血腥气。怡贵人整个人蜷缩在床内,已然晕了过去。如懿才要抱过她的身体唤她,一出手褥子上温热一片,她心底瞬即凉透了,仿佛被硬生生塞进了一大块寒冰,冷得她也忍不住发起抖来。她犹疑了片刻,才敢将自己的手从褥子上抬起。

她的整个手掌,都沾满了热而腥的鲜血。

注释:

①出自乾隆御诗《七律读项羽纪》。

\hypertarget{ux7b2cux516bux7ae0-ux524dux4e8b}{%
\chapter{第八章 前事}\label{ux7b2cux516bux7ae0-ux524dux4e8b}}

许太医来时,已然是无力回天了。他和赵太医忙碌得满头大汗淋漓,伸手去掐怡贵人的人中,拿艾叶拼命去熏,又灌入大量的汤药,到最后,只得摊手道:``娴妃娘娘,胎儿已经死在腹中,微臣也没有办法了。''

她一句话也说不出来,只能和海兰依偎在一起,眼睁睁看着怡贵人身下的血越来越多,身体越来越虚弱,连昏迷中辗转的呻吟声也再发不出来。

她茫然地看着,痛楚和惊恸已经将心底最初的惊恐和畏惧湮然吞没。她只能发出无助的喃喃:``怎么会?怎么会?''

虽然她和怡贵人的交情不深,可是这些日子,她几乎每天都陪着怡贵人,看着她的腹部一点点隆起,看着她初为人母的喜悦,连她也情不自禁地期盼,有朝一日,她会亲眼看着这个孩子出世。虽然,她从未有过自己的孩子,可是她可以亲眼看着一个生命的诞生,那种喜悦与企盼,是发自内心深处的。

可是连她自己都不能想到,已然这般小心,怎么还会这样,这样骤然目睹孩子的消逝。听着太医冰冷的话语,那个孩子,已胎死腹中。

太医小心翼翼地过来:``娴妃娘娘,已经没有办法了。微臣要用药打下怡贵人腹中的死胎,免得死胎在母体中留得太久,影响怡贵人的身体。''

她不知道用了多久的力气才逼出这一句话来:``为什么会死?孩子为什么会死?''

太医们吓得面面相觑:``这个\ldots\ldots 微臣也不知道,只能等胎儿拿出来才能计较。\textless{}''

良久,如懿才能挪动自己已然僵硬的身体,她吃力地和海兰互相搀扶着起身,转到门边的时候,她抬头看到了脸色苍白如纸的皇帝。

真的是苍白如纸,他的整张脸,白而透,是那种透着无奈与绝望的锈青色,好像他整个人都那样钝了下去,失去了往日里英挺的活气,只余了单薄的剪影,就那样薄薄地立着。皇帝站在近在咫尺的地方,她看得清他眼底的悲伤与惶惑。可是她什么安慰的话也说不出来,只能静静地与他双手交握,希望以彼此手心仅存的温暖来给予对方一点坚定和支撑下去的勇气。

海兰静默地退下,由着他们悲伤而安静地相对。如懿清晰地看见,他眼底的疼痛清晰凛冽地蔓延开来。皇帝的声音带了丝崩溃般的颤抖:``如懿,你告诉朕,为什么朕的又一个孩子死了?如懿,为什么朕登基后,朕的孩子一个都活不下来?是不是天命在惩罚朕?惩罚朕得到了九五至尊的荣耀,却失去了父子天伦之乐?''

他的话像针刺一样钻进她的耳膜里,即便他贵为天下至尊,却也有这样生离死别不能言说的苦楚。如懿清晰地感到命运的无常如同一柄冰凉而不见锋刃的利刀,你根本不知道它隐藏在何地,只能默默地承受它随时随地都可能的锐利刺入,眼见着自己的血汩汩而出,生生忍住。

如懿沉默地拥住他,将自己心底的无望化作拥抱时的力气,支撑着他随时会倒下的身体。她知道自己的安慰如此无力,可是她还是要说:``皇上,您已经有了三位阿哥,您还会有孩子的。您放心,一定还会有的\ldots\ldots{}''

有晶莹的液体漾得眼前模糊一片,几乎要喷薄而出,她却只能死死忍住,隐忍着不肯掉下。是,若连她都落泪,岂不让他更伤心。她仰起面,感受着夜来的风吹干眼底泪水时那种稀薄的刺痛,檐下的绯色宫灯被风吹得晃转如陀螺,像是磷火一样缥缈不定,更似夺取孩子性命的鬼魂那双不瞑的眼睛,嘲笑似的望着众生。她听着东暖阁里昏迷中的怡贵人断断续续惊痛的呻吟声,心底的无助越来越浓。她只得起身,将西暖阁里数十盏莲花台上的灯烛一一点燃,灼热的光线映得殿内几如白昼,地面上澄金镜砖发出幽黑的光泽,恰如皇帝脸上阴霾不定的锈青色,整个人似乎都被笼罩在深浅不定的阴影之中。

过了半个时辰左右,皇后也匆匆赶到了。她才俯身请安,太医已经捧了一个乌木大盘神色不安地过来。

皇帝吩咐了皇后起身,便问太医:``还能有什么事让你们如此慌张?''

许太医和赵太医互视一眼,慌忙跪下磕了个头道:``皇上容微臣细禀,胎儿已经打下来了,可是\ldots\ldots{}''他犹豫片刻,还是大着胆子说了下去:``可是这胎儿有异,不像是寻常胎死腹中啊!''

皇帝烦躁道:``胎死腹中本来就不寻常,难道还要你们来告诉朕么?''

许太医连忙道:``微臣这些日子以来一直和赵太医轮番伺候怡贵人的胎像,从诊脉来看,胎儿一直没有大碍。可是打下的死胎却\ldots\ldots{}''

皇帝隐隐觉得不好,太阳穴上突突地跳着,脸色愈发难看:``死胎怎么样?''

许太医道:``从母体的脐带到死去的胎儿都周身发青,更可怕的是,胎儿已经成型,能看得出是个男胎,但\ldots\ldots 孩子却显然是中毒猝死的,若是长大分娩而出,按照中毒的情况,也可能是畸胎\ldots\ldots{}''

许太医不敢再说下去,赵太医只得将木盘高高托起:``打下的死胎就在这里,皇上若是不信,可亲眼一观。''

皇帝迅疾地以两指撩起上面黑色的布看了一眼,如懿正好瞥见,只见里面血肉模糊一团,中间那团血肉的确是透着不祥的黑色。

如懿心里一慌,差点没呕吐出来,她弯下腰,抵挡着胸腔里搜心搜肺的酸楚和恐惧。皇帝的身体轻轻一晃,捧在手中的茶盏哐啷砸在了地上,他几乎是狂暴地站起来,怒吼道:``怎么会这样?怎么会?!''

皇后一个支撑不住,差点晕过去,幸好莲心和素心牢牢扶住了。皇后连声道:``不可能!不可能!爱新觉罗家怎么会接二连三出这样的事\ldots\ldots 怎么会\ldots\ldots{}''她忽然醒过神来,喝道:``你们说是中毒?是什么毒?''

赵太医挺起身子道:``若微臣与许太医没有猜错,是中了水银之毒。不知怡贵人以何种方式接触到了水银,不仅透过皮肤沾染,而且有服食的迹象,因为剂量太猛,所以导致胎儿被毒死腹中。而且若是水银慢性中毒,剂量不是如此之大,或许胎儿会长到分娩出母体,但有可能是畸胎或是天性痴傻。''他与许太医对视一眼,朗声道:``微臣还有一个推测,不知当说不当说。''

皇后当机立断:``有什么话你直说便是。''

赵太医道:``怡贵人从有孕便发热、大汗、心悸不安、失眠多梦,又多发溃疡,虽然很像是有孕之身常有的症状,但皇上和皇后不觉得这些症状很像一个人也得过的么?''

如懿心念一转:``你是说\ldots\ldots 玫贵人!''

赵太医道:``娴妃娘娘说得不错。恕微臣大胆推测,玫贵人的死胎或许不是意外,而是如怡贵人一般中了水银之毒,才会如此。''

皇帝大怒:``既然你们发觉怡贵人与玫贵人的症状相似,为何没一早察觉是中了水银之毒?''

两位太医磕头如捣蒜:``微臣说过,水银中毒的情状极慢,症状表现又与初孕的反应极其相似。若不是怡贵人母体不如玫贵人强健,导致未足月便胎死腹中,根本就难以察觉。''

皇后不觉失色:``那么你说的水银,宫中何来此物?''

许太医道:``以朱砂稍稍提炼,极容易便可得到。宫中佛事诸多,宝华殿中有的是朱砂,唾手可得。连太医院配药也是常用,只怕谁都能得到。''

皇帝的双手握紧,青筋直暴:``你们何以敢推断玫贵人的胎也是如此?当时为何没有太医说是水银祸害?''

许太医惶惑道:``微臣没见过玫贵人的死胎,所以不敢妄言。只是以玫贵人和怡贵人的症状来推测。怡贵人的胎儿也是侥幸,因为这种水银的毒是在胎儿幼小时才会明显,有全身连着脐带乌黑的症状。若等怀胎满八月,产出时即便是死胎也不过肚腹泛青而已,症状与其他死胎的差异便不明显了。''

皇后的声音极轻:``皇上,臣妾分明记得,玫贵人的胎是泛青的。''她沉声,如钟磬般郑重,道:``皇上,若玫贵人和怡贵人的胎真的是中毒,那就是说,死胎并非是天意惩戒,而是有人蓄意为之,谋害龙胎,动摇国祚祥瑞。臣妾以六宫之首的身份,请求皇上彻查此事,以告慰两位龙胎的在天之灵。''

皇帝的眼中闪过雪亮的恨意,冷冷道:``查!朕倒要看看,是谁有这样的胆子,敢谋害朕的孩子!''

所有人的注意力都放在了彻查龙胎之死的事情上,没有谁记得,去看一眼尚且昏迷未醒的怡贵人。如懿独自走到暖阁门外,掀起锦帘一角,看着华衾锦堆中昏睡的女子脸色苍白若素,一双纤手在暗紫色锦衾上无声蜷曲,空空的手势,像要努力抓住什么东西。她眼中一酸,忍不住落下泪来,她再清楚不过,怡贵人想要抓住的,再也抓不住了。

因为连着两胎皇嗣出事,连太后亦被惊动,一时间层层关节查下去,雷厉风行,连怡贵人身边侍奉的宫人也一个没有放过,一一盘查。宫中大有草木皆兵之势,风声鹤唳,人人自危。连素日性子最张扬的嘉贵人也避在自己宫中,足不出户。

慎刑司的精奇嬷嬷们最是做事做老了的,慎刑司的七十二样酷刑才用了一两样,便已有人受不住刑昏死过去,有了这样的筏子,再一一问下去便好办得多了。

怡贵人的孩子死后,皇帝也甚少过来安慰探视,即便来了也稍稍坐坐就走了,一心只放在了追查之上。倒是皇后顾念着主仆之情,虽然自己的二阿哥还在病中,倒也过来看望了几次。

怡贵人醒来后一直痴痴呆呆的,茶饭不思,那一双曾经欢喜的眼睛,除了流泪,便再也不会别的了。加之太医说她体内残余未清,每日还要服食定量的红花牛膝汤催落,对于体质孱弱的怡贵人,不啻于是另一重折磨。如懿和海兰一直守着她,防她寻了短见。她却只是向隅而泣,嘶哑着喉咙道:``娴妃娘娘放心,不查出是谁害了嫔妾的孩子,嫔妾是绝不会寻短见的。''说到这句时,她几乎已经咬碎了牙齿:``嫔妾侍奉皇上这么多年才有了一个孩子,他是嫔妾唯一的期盼和希望。到底是谁?是谁这么容不下嫔妾的孩子!''

是谁要害孩子?连如懿自己也想不明白。她只能端过一碗燕窝粥,慢慢地喂着怡贵人,劝慰道:``吃一点东西,才有力气继续等下去,等你想要知道的事。''

一碗燕窝粥喂完的时候,却是皇后身边的赵一泰先来了。

他道:``请娴妃娘娘和海贵人、怡贵人稍作准备,皇后娘娘请三位即刻往长春宫去。''

如懿搁下手中的碗道:``什么事这么着急?怡贵人尚在静养,能不能\ldots\ldots{}''

赵一泰道:``皇后娘娘相请,自然是要事。何况事关怡贵人,还请怡贵人再累也要走一趟。''

话既如此,如懿便命人备下了轿辇,即刻往长春宫中去。待得入殿,皇帝与皇后正坐其上,各宫嫔妃皆已到场,连在雨花阁静修的玫贵人也随坐其中。三人入殿后一一参见,便各自按着位次坐下。皇后见怡贵人病弱难支,不免格外怜惜,道:``赵一泰,拿个鹅羽软垫给怡贵人垫着,让她坐得舒服些。''

怡贵人忙颤巍巍谢过了,皇帝道:``你身上不好,安心坐着便是。''

慧贵妃扬一扬手中的丝绢,慵倦道:``外头春光三月,正当杏娇莺啼之时,皇后娘娘不去御花园遍赏春光,怎么这么急召了臣妾等入长春宫呢?''

皇后一向端庄温和的面庞上不由得浮起几分愁苦之色:``自去冬以来,宫中皇嗣遭厄,悲声连连,本宫与皇上都忧烦不堪,春光再好,也无心细赏。今日急召妹妹们前来,是因为怡贵人胎死腹中之事已有了些眉目,须得找人来问一问。这既是后宫之事,自然应该是后宫人人都听着。''

怡贵人神色一紧,忙问道:``皇后娘娘所说的眉目,是知道害臣妾孩儿的人是谁了么?''

皇后温言道:``怡贵人,少安毋躁。此事关系甚大,本宫与皇上也只是略略知道点眉目罢了。至于事情是否如此,大家都来听一听便是。''

皇帝道:``皇后既然查出了点眉目,有话便说吧。''

皇后看一眼身边的赵一泰,赵一泰击掌两下,便见许太医与赵太医一同进来。

皇后沉声道:``众人都知道怡贵人身罹不幸,龙胎死于腹中,乃是受了水银的毒害。本宫却百思不得其解,怡贵人房中并无水银朱砂,娴妃和海贵人对怡贵人的饮食起居也格外小心,照理说是不会出事的。欲查其事,必寻其源,臣妾让人翻查了怡贵人房中的器物,才发现了这些东西。''

皇后扬一扬脸,莲心捧着一个紫铜盘子,上面放着一对雕银花红烛并一些烧碎了的炭灰。皇帝取过那对红烛看了一看,疑道:``不过是寻常的红烛,怎么了?''

皇后微微摇头,伸手将其中一根拗断了,道:``请皇上细看,这蜡烛有否不同?''

皇帝对着日色一看:``虽然是红烛,但里头掺了一些红色的碎粒,可是内务府如今所用的东西越来越不当心了?居然用这样的红烛。''

皇后又道:``皇上细看这些炭灰。如今也是三月末,宫中只有延禧宫的怡贵人因为怕冷,还用着炭盆。这是她阁中所用的红箩炭烧下来炭灰,颜色灰白。可是细看下去,却有异状。''皇后用护甲轻轻拨弄其间,却见炭灰上沾了些许银色物事,还有一些朱红色的粉末,若不细辨,实在是难以察觉。

皇后抬一抬手,示意莲心端给众人都看看,众人暗暗诧异,却又实在不知道是何物。

皇后道:``这些都是怡贵人宫中所用的东西,请太医瞧一瞧,这蜡烛里头和炭灰里的,是什么好东西?''

赵太医掰开蜡烛,用手指捻了捻细闻,许太医亦翻看了炭灰里头的物事,几乎是异口同声地道:``回禀皇上皇后,这里头的东西都是朱砂。''

赵太医道:``朱砂遇高热会析出水银,水银遇见热便会化作无色无臭之气弥散开来,让人不知不觉中吸入。这炭灰里烧剩下的朱红粉末,定是有人将少许朱砂混入红箩炭中,等到烧尽,也不容易发觉。''

皇后冷笑一声:``这还不算老辣的,皇上且看那红烛,雕了银花装饰,即便烧出朱红和银色的粉末,也会让人以为是烛泪和银花融化后的样子,根本难以察觉。''

慧贵妃秀眉微蹙,啧啧道:``拼上了这样的心思去害怡贵人,哪里还有不成的。这个人还真是心思狠毒。''

皇帝道:``既然如此,那么怡贵人阁中的宫人都会有不适之状,怎么只有怡贵人身体不适?''

玫贵人握着绢子的手瑟瑟发抖,颤声道:``宫人伺候都是轮班入内的,而怡贵人身在其中,几乎每日不离,当然深受其害。''

皇后看了眼皇帝,含了几分不忍与厌憎:``这些都是小巧而已,臣妾听闻太医说起,怡贵人所怀胎儿中毒甚深,显然怡贵人有服食朱砂或水银的迹象。但那东西怎么吃得下去,一定是饮食方面哪里出了问题。''

海兰忙起身,战战兢兢道:``回皇后娘娘的话,怡贵人的饮食一概都是从御膳房送了新鲜的来,由怡贵人贴身的厨娘自己在小厨房中做的。臣妾也与娴妃娘娘每日留心,并无不新鲜的东西送来给怡贵人吃过。''

皇后摇头道:``你们自己都还年轻,哪里晓得这其中的厉害。送来的鱼虾都是欢蹦乱跳的,可是这欢蹦乱跳离下锅也不远了,谁还管它有什么毛病。赵一泰,你来说。''

赵一泰道:``本来皇后娘娘要奴才去御膳房查问,两位贵人在有孕时都喜欢吃什么,这才知道原来两位贵人都很喜欢吃鱼虾。皇后娘娘的原意是要奴才看看这些鱼虾有什么问题,谁知到了御膳房,才发现说供给怡贵人所用的鱼都死了,所以扔了出去。奴才就觉得蹊跷了,给怡贵人所用的鸡鸭鱼虾都是另外养着的,怎么鸡鸭都还好好活着,鱼虾没几日便死完了。所以奴才格外留心,找到了一小袋剩下的鱼食,想看看有什么异样。''

赵一泰转身取过一小袋鱼食捧到皇后跟前。皇后冷眼瞥着道:``这些鱼都是御膳房里养着专供有孕的嫔妃所食的,都是精挑细选过然后专门养在一个小池子,喂的吃食也格外精细。宫里这样重视皇嗣,没想到有些别有用心的人,便在这个上打主意了。''

嘉贵人好奇地望着盆中的鱼:``这些鱼食有什么不同么?''

皇后淡淡道:``有没有不同,叫太医看过了就是了。''

赵太医忙应了声``是'',与许太医头并头看了片刻,神色凛然:``回禀皇后娘娘,这些鱼食里都掺了磨细了的朱砂粉末,喂给鱼虾吃下后,初初几日是不会有异样的。因为朱砂本身只是甘,微寒,有微毒。但等鱼虾吃下养上两天后,这些毒素都化在肉里,一经烹制遇热,毒性愈强。本来少少食用也还无妨,但日积月累下来,等于在生服朱砂和水银,慢慢损害胎儿。其手段老辣之极呀。''

赵一泰又道:``奴才也在御膳房问过,怡贵人与玫贵人有孕后所食鱼虾,的确是由此种鱼食喂养,绝对不会错的。''

嘉贵人吓得忙掩住了口,惊惶地睁大了双眼,下意识地按住了腹部。纯嫔闭着眼连念了几句佛号,摇头不已。慧贵妃嫌恶地看着那些东西,连连道:``好阴毒的手段!''

玫贵人与怡贵人早已一脸悲愤,数度按捺不住,几乎立时就要发作了。

如懿满脸羞愧,忙起身道:``皇上恕罪,皇后娘娘恕罪,臣妾本以为对怡贵人的饮食已经十分仔细,却不承想还是着了如此下作的手段。还请皇上皇后降罪!''

皇后瞟了她一眼,慢条斯理道:``娴妃你的确算是小心了,但再小心,总有百密一疏的时候。至于你要受什么罪,挨什么罚,等下本宫和皇上自会处置。''

\hypertarget{ux7b2cux4e5dux7ae0-ux65e0ux8def}{%
\chapter{第九章 无路}\label{ux7b2cux4e5dux7ae0-ux65e0ux8def}}

玫贵人再忍不住,跪在了地上抱住皇帝的腿道:``皇上,皇上,臣妾怀胎八月,突然早产,却产下那样的孩儿,以致被皇上厌弃。臣妾一直不敢怨天尤人,只以为是自己福薄命舛。如今细细想来,原来便是有人这样暗中布置,谋害臣妾和皇上的孩子。皇上,皇上,咱们的孩子死得好可怜。他一生下来连一句`额娘'都没叫过,连眼睛都没睁开好好看一看,就这样平白无故断送了。皇上啊,哪怕是臣妾在雨花阁再念成千上万遍《往生咒》,孩儿他死得这样冤屈,也不肯往极乐世界去啊!''

玫贵人哭得伤心欲绝,在场之人无不恻然。怡贵人也背转了身,咬着绢子哭泣不止。

赵太医道:``玫贵人且勿伤心。依微臣和许太医看来,这个要害娘娘的人,一开始用药极谨慎,几乎是慢慢入药,所以娘娘才会拖到八月早产生下那样一个孩子。而对怡贵人,那人似乎放心大胆,用药也更猛,所以会害得怡贵人怀胎四月胎死腹中。''

怡贵人终于忍不住痛哭失声:``皇后娘娘既已查到这么多,那么烦请告诉臣妾一声,到底是谁在谋害臣妾的孩子?''

皇后看着神色阴郁不定的皇帝,气定神闲道:``不只你们,本宫也很想知道,后宫有如此阴毒之人留着,丧心病狂,谋害龙胎,到底是想要做什么?所以在请你们所有人到场的时候,本宫已让素心带了人遍查你们所有人的寝宫,想来很快就有消息了。''

皇后话音未落,素心已带了人匆匆进来,福了一福道:``皇后娘娘交代的奴婢都已经做了,果然在其中一位小主的妆台屉子底下找到了一包朱砂,还请皇后娘娘过目。''

皇后将那包朱砂递到皇帝面前:``皇上闻闻,这包朱砂沾上了什么气味?''

皇帝取过轻轻一嗅,目中的瞳孔骤然缩紧,那种厉色,汇成一根尖锐的长针,几能锥人。他失声道:``是沉水香的气味!娴妃,宫里只有你一个用沉水香的!''

如懿心头大惊,眼见皇帝只逼视着自己,情不自禁跪下道:``皇上明鉴,臣妾真的不知情,更不知妆台屉子中何时会有这包朱砂!''

皇后闭目长叹一声:``素心,你实说吧。''

素心道:``皇上所言不错,奴婢便是在延禧宫娴妃娘娘的妆台屉子下找到的这包朱砂。当时娴妃娘娘的侍婢阿箬还左右阻挠,不许奴婢翻查。如此看来,阿箬也是知情的,所以奴婢也带了她来。''

皇后冷冷道:``先不必传阿箬。娴妃,你且看看现在进来的这个人,可是你认识的?''

如懿回首望去,却见素心后面还跟着两个小太监。显然他们是刚从慎刑司出来,脸上还带了些许轻伤,看着倒不甚严重。

如懿摇头道:``臣妾不认识。''

皇后的笑意冷凝在嘴角:``你不认识他们,他们却个个认识你了。这个御膳房的小禄子,是你宫里小福子的哥哥,专管着给有孕嫔妃们养活鱼活虾的。''

如懿沉着道:``臣妾是知道小福子有个哥哥,但臣妾今日也是第一次见他,从前从不相识。''

皇后取过那包鱼食丢在了小禄子跟前道:``说,是谁指使你给那些鱼虾喂朱砂的?''

小禄子偷眼瞟着如懿,嘴上却硬:``奴才不知,奴才实在不知啊!''

``不知?''皇后森冷道,``在慎刑司才一用刑你就招了,此刻还想翻供。本宫也不和你计较,立刻送回慎刑司就是。''

小禄子一听``慎刑司''三字,吓得浑身发抖,连连磕头求饶道:``皇后娘娘饶命,皇后娘娘饶命。是娴妃娘娘吩咐奴才这样做,奴才实在不敢不听啊,她对奴才说,只要奴才敢不乖乖听话,就要寻个由头杀了奴才的弟弟小福子。奴才只有小福子一个弟弟,从小相依为命,实在不敢不听娴妃娘娘的话啊!''

如懿逼视着他道:``小禄子,你好好想想清楚,本宫从未见过你,又怎会拿你弟弟的性命威胁你呢?''

小禄子苦着脸道:``娴妃娘娘,那日在御膳房门外的甬道里,这话分明是您自己说的。您说您还没有身孕,怎么出身低贱的玫贵人和怡贵人都有了,简直让乌拉那拉氏的祖先笑话您!您说一定要出这口气,还说奴才不做,您杀了小福子后一样可以找别人做。奴才万般无奈才答应了的。''

另一个小太监小安子也哭着道:``娴妃娘娘,您当日到内务府找到奴才,要奴才做一些掺了朱砂的蜡烛送到您宫里。奴才送去之后您打赏了奴才三十两银子。奴才只当您是做了自己玩儿的,实在不知道您是去害人呀!''

如懿气得浑身发怔,心口一阵阵发寒,仿佛是掉进了一个深不见底的黑渊里,只觉得四周越来越寒,却不知自己究竟要掉到哪里才算完。

慧贵妃轻笑一声道:``这就难怪了!本宫怎么说呢,从怡贵人惊蛰那日遇蛇开始就觉得奇怪,怎么巧不巧怡贵人遇了蛇就被娴妃你撞见救了呢。怡贵人这就感激涕零去了你的延禧宫同住。这不正好下手,一切方便么?''

如懿恼怒地直视着她道:``慧贵妃慎言。如果说一切是我蓄意所为,那么就该离怡贵人越远越好,才不容易被人发现,怎么还会这么蠢接她来延禧宫同住,好叫人疑心?''

``疑心?''慧贵妃嗤笑,耳边一双明铛垂玉环玲玲作响,``若是和玫贵人一般看起来像个意外,谁会疑心?都只当怡贵人自己命薄留不住孩子罢了。所谓富贵险中求,若是不兵行险招把怡贵人留在身边,哪能又是蜡烛又是炭火又是饮食那么周全。玫贵人不就是你隔得远不方便,所以中毒缓慢,到了八个月才没了孩子。想来你自己腹中空空,看着人家的肚子一个接一个大起来,是越来越不能容忍了吧!''

如懿几乎气结,极力压抑着心口的怒气,冷冷道:``慧贵妃也腹中空空,一定要这样说出自己的心思么?''

慧贵妃平生最恨人说自己膝下无所出,不觉变了脸色,恨声道:``你\ldots\ldots{}''

胶凝的气氛几乎叫人窒息,皇帝微微地眯着眼睛,有一种细碎的冷光似针尖一样在他的眸底凌厉刺出,他隐忍片刻,缓和了气息道:``好了,你们都不要争执。皇后,只有小禄子一个人的证词,怕是不能作数吧。''

皇后轻轻颔首,恭敬道:``皇上所言甚是。臣妾也觉得一面之词不可轻信,所以让素心带了阿箬过来。皇上可还记得,素心说阿箬方才拦着搜查么?那这丫头一定是知情的,依臣妾看,还是要好好查问才是。''她转头看着素心:``阿箬带来了么?''

素心道:``已在殿外候着了。''

如懿看着阿箬神色谦卑地走进来,并无任何紧张不安之态,心中不觉松了一口气。阿箬到底是跟着自己多年的阿箬,没有做过的事,自然不必心慌意乱。她又有什么可担心的呢?或许她的阻拦,也是因为生性里的一分骄傲吧,怎可容许别人轻易侮辱了自己?然而心底的深处,如懿还是有一份深深的不安,到底延禧宫中是谁出了差错,将这一包朱砂放进了自己的妆台屉子里。

旁人不清楚,她自己却是知道的,沉水香的气味颇为清淡,要使这一包朱砂都染上气味,必然是在自己的殿内放了许久了。那么又是谁,能做得这样神不知鬼不觉?

她的心绪繁杂如乱麻。还来不及细细分辨清楚,阿箬已经走到殿中,沉稳跪下了道:``皇上万福,皇后万福,各位小主万福。''

皇后道:``今日也不说这些虚礼。本宫只问你,素心要去搜查延禧宫的时候,你为什么要拦着,还不许搜寝殿。''

阿箬脸上闪过一丝淡淡的哀伤,只是道:``奴婢伺候小主,就要一切为小主打点妥当。''

``打点什么?''

阿箬脸上的悲伤之色愈浓,忽然转首向如懿磕了三个头道:``小主,奴婢伺候您已经八年,这八年来不可谓不尽心尽力。可是小主入宫之后,性情日渐乖戾,每每逼迫奴婢去做一些奴婢自己不愿做的事。奴婢知道,您是奴婢的主子,奴婢只能为您去做。可奴婢做这些事的时候心里并不好受,今日既然事情抖了出来,奴婢也无法了,只能知道什么便说什么。''

如懿越听越觉得不祥:``阿箬,你这样说是什么意思?''

阿箬转头再不看她,只向皇帝和皇后道:``奴婢知道皇上和皇后要问什么,奴婢一并说了就是。自从玫贵人有孕之后,小主时常伤感,喜怒更是无常,常常抱憾虽然抚养了大阿哥却没有自己的孩子。玫贵人有孕后得宠,小主更是恨得眼睛出血。有一日终于叫了奴婢去宝华殿搜罗了一些朱砂回来。''

慧贵妃道:``娴妃突然让你要朱砂,你也不疑心么?''

阿箬摇头道:``奴婢何承想到这个。当时小主也只是说用朱砂抄写经文祈福,可以早些有自己的孩子。有一次小主带奴婢去看望玫贵人的时候,悄悄在玫贵人的炭盆里撒了些朱砂,因为朱砂的颜色与红箩炭相似,颗粒又小,几乎无人察觉。只是每次去,她必定趁人不备这样做。几次之后奴婢就觉得奇怪,几日后小主突然想去御膳房,便带了奴婢在御膳房外的甬道那儿放风,奴婢隐隐约约听见小主吩咐了御膳房的小禄子什么喂朱砂,掺在鱼食里什么,还提到了小福子,小禄子当下便哭着答应了。奴婢吓了一跳,问小主要拿朱砂做什么,小主不许奴婢多问,还让奴婢继续去宝华殿搜罗。''

窗外明明是三月末的好天气,阳光明亮如澄金,照在殿内的翡翠画屏上,流光飞转成金色的华彩流溢。中庭一株高大的辛夷树,深紫色的花蕾如暗沉的火焰燃烧一般,恣肆地怒放着。如懿心里一阵复一阵地惊凉,仿佛成百上千只猫爪使劲抓挠着一般。她的面色一定苍白得很难看,她怎么也不相信阿箬会这样镇定自若地说出这些话来。

阿箬继续道:``自从玫贵人产下死胎之后,小主嘴上虽不说,但奴婢伺候小主多年,看得出来她很高兴的。后来怡贵人又有了身孕,小主和怡贵人并不算太熟,不能像常去看玫贵人一样去景阳宫。可是她总不高兴,说连怡贵人那样侍女出身的都有了孩子,她却偏偏没有。那一天去看怡贵人遇蛇后,小主正好顺水推舟救了怡贵人,本来是想借机可以多去景阳宫,谁知皇上正好让怡贵人住到延禧宫,便遂了小主的心了。怡贵人有孕,皇上每天来看小主的时候都会去看怡贵人,小主气恼不过,下手也特别狠。怡贵人的红箩炭备在廊下,随取随用,都是事先混了朱砂的。连吩咐给小禄子的朱砂,也比往常多了许多。''

惢心气得浑身发抖,怒喝道:``阿箬,小主待你不薄,你受了谁的好处,居然说出这样没良心陷害小主的话来?''

阿箬冷冷看她一眼:``正是因为我还有良心,所以受不住内心的谴责说了出来。哪怕小主待我不薄,我也不能昧了良心。''

惢心气道:``好!好!哪怕你说的不是昧心话,我和你一同伺候小主,怎么你说的这些话我都不知道。细论起来,平日里还是我伺候小主更多些呢。''

阿箬轻蔑道:``你是伺候小主多些不错。但我是小主的陪嫁,有什么事小主自然是先告诉了我,你又能知道什么?而且这样狠毒的事,难道还要人人皆知么?''她目视如懿,毫不畏惧:``小主,这样的事你自己做过自己不知道?难不成奴婢和小禄子都要冤枉你么?''

如懿双目紧闭,忍住眼底汹涌的泪水,睁眸道:``很好,很好,本宫不知道你与谁合谋布了这个局来害本宫,当真是天衣无缝,对答如流。''

阿箬躬身道:``小主若要怪奴婢,奴婢也是无法,自知道此事后,奴婢心里日夜不安,眼见得怡贵人胎死腹中,奴婢夜夜噩梦。当时遵于主仆之情,奴婢不敢说与人知。如今事发,乃是天意,奴婢也只得说了。小主任打任罚,悉听尊便。''

阿箬言毕,忽然看了小禄子一眼。小禄子冲上来道:``娴妃娘娘,奴才知道供了出来对不住您,可是奴才也不想这样平白害了两位皇嗣。奴才我\ldots\ldots 我\ldots\ldots{}''他支吾两声,突然挣起身子,一头撞在了正殿中一只巨大的紫铜八足蟠龙大熏炉上,登时血溅三尺,一命呜呼。嫔妃们吓得尖叫起来。

玫贵人二话不说,冲上来照着如懿的面门便是狠狠两个耳光。她还要再打,却被跟上来的宫女死死拉住了。她口中犹自骂道:``你好狠毒的心,还敢说人冤了你,小禄子能拿他一条命来冤枉你么?你居然狠心到连我腹中的孩子都不肯放过,要他死得这样惨!''

如懿晕头转向,脑中嗡嗡地晕眩着,脸上一阵阵热辣辣的,嘴角有一股热热的液体流了出来,她伸手一抹,才发觉手上猩红一道,原来是玫贵人下手太重,打出了血。可是她居然不觉得痛,只是看着那大熏炉上慢慢滴下的血液,一滴又一滴滑落。撞得头壳破碎的小禄子被人拖了出去。这样温暖的天气里,她居然生出了彻骨的寒意。

死无对证,居然是死无对证!

阿箬脸色惨白,对着如懿道:``小主若是对奴婢今日的话有所不满,奴婢也自知不活,一定跟小禄子一样一头撞死在这里,也算报了小主多年的恩义。''她说完,一头便要撞向那熏炉去。

慧贵妃眼疾手快,一把拉住了道:``已经死了一个,再死一个,岂不是都死无对证了。''她款步向前,向帝后福了一福道:``今日的事后宫诸姐妹都已经听明白了,娴妃谋害皇嗣,人赃并获,已经无从抵赖。臣妾请求皇上皇后还玫贵人和怡贵人一个公道,更还含冤弃世的两位皇嗣一个公道。''

海兰忙跪下,情急道:``皇上,皇后娘娘,臣妾与娴妃娘娘起居一处,深知娘娘并无害人之心,此中缘故,还请皇上皇后明察。''

纯嫔亦道:``皇上,皇后娘娘,臣妾与娴妃相处多年,她的确不会是这样的人,还请皇上皇后明察。''

皇后叹口气道:``后宫出了这样的事,原是臣妾不察之过。人证物证俱在,娴妃是无从抵赖,但娴妃毕竟伺候皇上多年,皇上要如何查办,臣妾听命便是。''

皇帝的眼睛只盯着熏炉上淌下的鲜血,他的声音清冷如寒冰:``阿箬,你是要拿你这条命去填娴妃的罪过了,是么?''

阿箬含泪道:``奴婢自知身受皇恩,阿玛才能在外为朝廷效力,可是忠孝难两全,奴婢只有以死谢罪。''

空气中有胶凝般的滞缓与压抑,庭院中的花香轻而薄地缠上身来,闻得久了,几乎如同捆绑般的窒息。远处不知是不是有蜜蜂在嗡嗡地扑着翅膀,好像那锐利的蜂针也一点一点逼进身体,一阵一阵地发痛。如懿跪在乌金地砖上,膝盖疼得几乎直不起来,她欲分辩,唯觉得自己陷在了一张精心织就的天罗地网之中,口干舌燥无力挣扎,只由得冷汗涔涔而下,濡湿了面庞。

良久,她仰起面,痴痴望着皇帝:``皇上,人证物证皆在,臣妾百辞莫辩。但是皇上,臣妾至死也只有一句话,臣妾不曾做过。''

皇帝并不看她,只是道:``你也知道人证物证,铁证如山。朕再不愿意相信,亦只能相信。''他的脸上有深翳的惨痛与悲伤:``那两个龙胎的死状,朕都是亲眼见过的,一辈子也忘不了。如懿,就算你没有孩子,可是朕一直宠爱你,你还有什么不足,要连尚在母腹中的孩子也不放过。''他仰起脸,将眼中的泪水以愤怒灼干,化作冷厉的口吻:``传朕的口谕,娴妃乌拉那拉氏心狠手辣,着降为贵人,幽禁延禧宫,再不许她出入。''

如懿绝望地瘫倒在地上,眼里蓄满了泪水:``皇上一直对臣妾说要臣妾放心,如今臣妾百口莫辩,只要求皇上能明察秋毫,还臣妾一个清白。''

皇帝并不看她,只道:``怡贵人黄氏即日迁回景阳宫,玫贵人白氏迁回永和宫,一切如旧。至于阿箬\ldots\ldots{}''皇帝脸上生了几分温柔之色:``朕属意你已久,只是一直不得机会对娴贵人说。此次的事你也有身不由己之处,切莫再寻了短见,以后便留在朕身边伺候吧。''

阿箬大喜过望,只是有些畏惧地看了看皇后与慧贵妃。

皇后叹道:``知错能改,善莫大焉。而且此次的事,娴贵人是罪魁祸首,阿箬只是碍于情义一时不得明说罢了。皇上要留她在身边将功抵过,臣妾也觉得是应该的。''

如懿怔怔地望着阿箬含羞带怯的面庞,只觉得天灵盖被人狠狠剖开,贯入彻骨寒冰,冷得她完全无法接受,却只能任由冰冷的冰珠带着棱角锋利地划过她的身体,痛得彻骨,却依然清醒。

阿箬的笑意还未退去,嘉贵人嘴角高傲地扬起,盈然起身道:``皇上,娴贵人谋害龙胎之事做没做过只有她自己有数。只是臣妾\ldots\ldots{}''她按住自己小腹,喜悦道:``臣妾已经有了一个月身孕,实难再与娴妃这样的人共处。皇上幽禁了她,臣妾才敢安心在宫中养胎。''

皇帝所有的悲伤与恼怒在一瞬间被她的笑意化去,他上前一步,紧紧握住了嘉贵人的手道:``你所言可真?''

``臣妾不敢妄言。只是宫里出了这样的事,臣妾不敢说出来而已。''嘉贵人满面得意地笑,牵住皇帝的手,依依道,``皇上,臣妾好怕受人所害,还请皇上允准,许臣妾住在皇上养心殿后的臻祥馆,以借皇上正气驱赶阴邪,护佑龙胎。''

皇帝欢和的笑容里,自然是无不允准。嘉贵人的孩子,恰到好处地驱散了前两个离去的阴霾。只是这样的欢欣喜悦里,没有人会在意如懿的绝望与无助。

她望着窗外艳阳高照,这是三春胜日,她却清晰而分明地觉得,她的春天,已经离得太远了。

\hypertarget{ux7b2cux5341ux7ae0-ux51b7ux82d1}{%
\chapter{第十章 冷苑}\label{ux7b2cux5341ux7ae0-ux51b7ux82d1}}

如懿独自坐在殿中,看着黄铜镜中自己的容颜,居然已经是憔悴如斯。延禧宫中的宫人被撤去了大半,连香炉里的香烟冷了,也没有人再来更换。只剩下一把冰冷的死灰,如同她的心一般,散碎成齑粉,不知哪一阵风来,就散得不见踪影了。

海兰悄无声息地走进来,替她挽好散落的发髻,整了整疏散的珠钗,缓声道:``姐姐切莫心灰意冷,皇上只是降姐姐为贵人,可见心中还是有姐姐的。这件事虽然看似证据确凿,但并非没有一点可疑之处,等到皇上想明白了,就会恢复姐姐位分,放姐姐出去了。''

如懿缓缓地摇头:``没用了。''

的确是没用了。所谓的证人,小禄子已经死了,他的死更像是源于她的逼迫。而唯一活着的最有力的证人,只剩下了阿箬。

海兰正欲说话:``那么阿箬\ldots\ldots{}''

如懿凄然一笑:``你也觉得阿箬劝得回头?今日她在长春宫能够如此犀利冷静地说出那番话,说得那么滴水不漏,我便已经知道,阿箬会是置我于死地的一剂砒霜。你要砒霜变良药,如何可能?而且如今她已经在养心殿行走伺候,谁再要接近她,都不是易事了。''

海兰犹豫道:``可是如今,的确只有阿箬一个证人了。我猜皇上的意思,可能是不想她也和小禄子一样骤死,所以留在养心殿中。''

如懿心灰意冷道:``是什么都好了。这丫头一直心高气傲,我却不知道她还有这样的本事,竟然在我看不见的地方与人勾结做下了这等好事!''

海兰见四下里冷冷清清的,并无旁人伺候在侧,便道:``姐姐以为,阿箬是和谁勾结?''

如懿沉吟着道:``皇后娘娘有皇子和公主,又掌位六宫,按理说并不需除去这两个孩子。''

``但玫贵人盛宠,怡贵人的孩子又被认为是大贵之胎,不能不防。''

``慧贵妃一直与玫贵人不睦,实在有可能是她害的玫贵人。但是怡贵人与慧贵妃一直也没有冲突。''

海兰沉吟道:``可是若以两位龙胎之死打击姐姐,慧贵妃一定做得出。而嘉贵人的恩宠一直与姐姐和慧贵妃相当,哪怕慧贵妃被皇上冷落之后,她都能和姐姐平分春色,今日又恰到好处提出自己怀有身孕,让皇上转怒为喜,恐怕嘉贵人也不简单。''

如懿自嘲地笑笑:``宫中生存,有谁又是简单的?是我自己技不如人,才会受此算计。\textless{}''

海兰急道:``那还有小福子呢,他是小禄子的弟弟,难道什么都不知情?''

如懿道:``慎刑司查问过了,的确是问不出什么。''

她望向院中,中庭的桃花怡然而开,灿烂如凝霞敷锦,散漫开一天一地。一阵风过,连吹来的气息都是甜的。院子里晴丝袅袅,春光骀荡,这样好的时候,她却宫门深闭,只看着黄昏暮色无可阻挡地自远处逼近,无处可逃。

外头有极轻的人语声,那是怡贵人宫中的宫人在搬离延禧宫,她看着海兰道:``你也要搬走了么?''

海兰道:``我求过皇上,暂居延禧宫陪伴姐姐。''

如懿神色凄苦,握住她的手道:``何必?这次不比禁足,你还能出去。陪我住在这里,等于是陪我一起幽禁,葬送了自己。''

海兰眼底都是泪,只是坐在她身前,诚挚道:``妹妹人傻,又愚笨不懂得周旋,即便能出去,也不过任人欺凌罢了,情愿守着姐姐。''

如懿握着海兰冰凉的手,哽咽间一句话也说不出来。忽然,帘下闪过一点响动,如懿转过脸去,却见怡贵人一身素服,头上只别了一支素银如意钗并几点雪白珠花,站在帘下,单薄得几如一枝孱孱在二月冷风中的瘦柳。她脸上的肉几乎都干透了,脸颊深深地凹陷下去,唯有一双干枯的眼,黑得让人生出怕意。

她一步一步缓缓走近,声音轻得仿似一缕幽魂:``娴贵人,看着你跟海兰姐姐这样情好友善,我便想起你照顾我的那段时日,真的是对我很好很好。可是娴贵人,你为何要这样虚情假意,一定不肯放过我的孩子!如果你不喜欢我承宠,你告诉我就是了,为什么要害我的孩子!''

她步步逼近,语中的凄厉之意越来越盛,终于在接近如懿的那一刻,伸出手狠狠扼住了她的脖子。海兰一时不防她如此,立刻伸手去拽,口中大呼道:``来人!快来人!''

不想怡贵人瘦弱至此,力气却极大,海兰根本拉不开。如懿只觉得喉头一阵阵痛得发紧,几乎喘不过气来了。她拼命伸手去掰开怡贵人的手指,好容易和海兰一起用力掰开了她一只手,却见怡贵人一把拔下头上的银钗狠狠向她刺来。那银钗的一头磨得极其锋利,显然怡贵人是有备而来,眼看那银钗的锋尖避无可避,朝着如懿面门直刺而下,海兰伸手一把挡住了钗尖,将自己的手臂横贯其下。

沉闷的一声痛呼,有鲜红的血一瞬间迸开,落在如懿的面上,温热而芬芳。

怡贵人似乎也被那血吓住了,一时行动有些滞缓,便被扑进的宫人们一拥而上拉开了。如懿赶忙握住海兰的手臂细看,只见雪白如藕的臂膊上,一条深深的血痕从手肘到手腕直划而下,鲜血涌出处皮肉翻起,触目惊心。

如懿慌不迭地喊起来:``传太医,快传太医!''

怡贵人被蜂拥的人群拖了出去,口中犹自念念不绝,不住地咒骂哭泣。海兰手臂上不断有鲜红的血液滴落,惢心忙捧了纱布来,如懿急道:``太医不知什么时候过来,我先替你缠上止住血。''

海兰痛得眼中泛起泪光,却极力忍耐着道:``姐姐别怕,一点皮肉伤而已。倒是姐姐你,没被怡贵人吓着吧?''

如懿心疼道:``你都这样了,我能比这个更怕么?''

海兰强笑着安慰道:``没事,一点皮肉伤而已,没有伤及筋骨就好。''

如懿的泪一滴滴落下,洇在纱布上,衬着不断沁出的鲜血,似绽出一小朵一小朵艳色的梅花:``可是伤得这样深,一定会留疤了。''

海兰忍着疼,微笑道:``即便留疤,也比伤了姐姐的性命值得,是不是?''

如懿的喉头隐隐还残留着被怡贵人扼过的痛,然而此刻,却被更深更重的感动填满了。是,这几日来的风波迭起,让她身心俱疲,无力抵抗,可是还有海兰。幸好,还有海兰,容得她在凄苦的宫中有人相依为命,彼此依靠。

怡贵人的死是在三日之后,因为积郁过度,加上腹中孩子的残体没有完全清除,过量催产残余的红花牛膝汤让她的身体再也承受不住,撒手而去。

据说,她死的时候,眼睛都没有闭上,只以布满血丝的双眼,无语望向苍天。

她的死,让原本稍稍平静的后宫再度沸腾起来。

消息传到养心殿的时候,皇帝正在批阅奏折。阿箬换了御前宫女的服饰,虽然不比在延禧宫时华贵,却别有一种在御前伺候的气韵隐隐透出。

阿箬见皇帝只是奋笔疾书,便捧了一小碟点心和茶水进来,不动声色地向李玉努了努嘴。李玉知道她在御前伺候之后颇得皇帝另眼相看,也不知如懿情形到底如何,一时也不敢轻举妄动,便退到了殿外。

阿箬小心翼翼将茶点放在皇帝跟前,便悄无声息地替皇帝研起墨来,她的手势极轻,手腕运力,墨汁磨得浓淡恰到好处,一星也未溅出来。皇帝蘸了蘸墨笑道:``难怪古人说要让闺秀少女来磨墨,红袖添香自然是一种乐趣,但也唯有你们才能用力适度,磨出不涩不枯带光泽的墨汁来。''

阿箬盈盈一笑:``皇上夸奖了。奴婢不过是为娴妃娘娘\ldots\ldots 不,是为娴贵人磨墨久了,熟能生巧而已。''她自悔失言,有些畏惧地看着皇帝:``奴婢失言了。''

皇帝只是一笑:``是么?朕喜欢听你说话,更喜欢你的熟能生巧。''

阿箬羞涩一笑:``奴婢笨笨的,怕说错了话惹皇上不高兴。''

皇帝的眼角带了轻俏的笑意,是薄薄的桃花色,如同窗外的春色一般明媚:``怎么会?你说什么,朕都喜欢。''

阿箬脸上浮起红云,还是忍不住道:``皇上这么说,可是因为爱屋及乌?''

皇帝微微一怔:``什么爱屋及乌?''

阿箬绞着手指,低低道:``皇上爱惜娴贵人,不舍得重责。因为爱惜娴贵人,所以连昔日在她身边伺候的小乌鸦,也就是奴婢,也连着得了些怜惜。''

皇帝的笑意微微淡下去:``当日你仗义执言之后,宫里还会有人把你当做是娴贵人身边的小乌鸦么?你就是你,乌拉那拉氏就是乌拉那拉氏,彼此早不相干了。''

阿箬低首道:``是。那皇上不觉得奴婢是背主弃信之人么?''

皇帝眼底有深邃的墨色,几乎能望到人的心底去:``只要你是仗义执言,不违背本心,没有人会觉得你背主弃信。''

阿箬暗暗地松一口气,朝皇帝露出一个极明丽的笑容。她正盈盈望着皇帝,李玉进来道:``皇上。''

皇帝从他的面上探寻到一丝惊慌的意味,沉声道:``什么事?''

李玉战战兢兢道:``景阳宫来报,怡贵人产后失调,死胎余毒未清,方才已经殁了。''

皇帝的神色变了又变,末了眼角沁出一点泪意,叹息道:``真是可惜了。去告诉皇后,怡贵人追封为怡嫔,一切丧仪按嫔位安置,让皇后好好操办。''

李玉答应着去了,阿箬忙递了茶到皇帝手中道:``怡嫔娘娘真是可怜,孩子没了之后情绪还那么激动,想跑去杀了娴贵人,结果累了自己红颜早逝,真当是可怜。''

皇帝淡淡道:``乌拉那拉氏是咎由自取,还累得海贵人也受了伤。''

阿箬乖巧道:``皇上别生气。幸好现在嘉贵人也有了身孕,在臻祥馆养得好好的,皇上放心就是。''

皇帝嗤地一笑:``你总惦记着别人,那你自己呢?''

阿箬痴痴一笑,别过身去道:``皇上取笑奴婢呢,奴婢有什么好惦记的。''

皇帝取过她捧来的糕点咬了一口:``好甜。''

阿箬忙道:``奴婢记得皇上喜欢吃玫瑰花瓣糖蒸的菱粉糕,所以特意下厨做了一盘,不知皇上喜不喜欢?''

皇帝笑吟吟望住她,一把捉住她的手道:``你还说你不惦记着,连朕喜欢吃什么都记在了心上。''

阿箬羞得满面绯红,忙低下头娇怯怯道:``皇上\ldots\ldots{}''

皇帝在她手上轻轻一吻,笑道:``好甜。''

阿箬越发不好意思,只觉得一颗心怦怦地跳着,几乎有些晕眩。她盼了那么久,渴望了那么久,原来只要稍一用力,就可以伸手攀到了。殿外的花香无孔不入地钻进来,带着甜腻而熏人欲醉的气味,不依不饶地缠上身来。皇帝吻着她的耳畔,低声道:``你阿玛现如今在高斌手下,跟着他颇有出息,不仅治水出色,这个知府也当得有声有色。朕也不想在宫里委屈了你\ldots\ldots 朕打算封你为常在,就住在嘉嫔的启祥宫。封号\ldots\ldots 为慎。''

阿箬受宠若惊,只觉得身上的力气一点一点都被抽去了,只是娇慵无力地瘫在皇上怀中,双手一点一点攀上他的颈,像在寻着最后的依靠似的:``有皇上的眷顾,臣妾一点也不委屈。''

圣旨传遍六宫的时候,便是说因嘉贵人有孕,晋封为嘉嫔。阿箬因在养心殿照顾嘉嫔有功,又能柔顺侍上,封为慎常在。

皇后看着圣旨只是一笑,向陪坐一旁赏花的慧贵妃道:``不承想这个丫头这么有出息。''

慧贵妃微微有些不悦:``祖制宫女册封要从官女子起,她倒好,一步登天了。''

``那不是也要有妹妹抬举么?''皇后折下一朵暗红瑞香花别在衣襟上,``阿箬的阿玛在妹妹的父亲麾下做事,听说颇有才干,他的女儿在宫里能不格外伶俐么?一个眼错没看见,就被皇上调到了御前伺候,指不定怎么伸着胳膊扑棱着翅膀在皇上面前飞呢。祖制也是从前的皇上定的,如今的皇上改一改,也没什么了不得。''

慧贵妃替皇后正了正衣襟上的瑞香花,狠狠掐下一片多余的花叶:``再怎么会扑棱,也不过是一个常在,臣妾不信她还能飞上了天去。真要不识好歹,翅膀是怎么安上去的,就怎么给她卸下来。''

皇后微微一笑,拈过一朵瑞香递到慧贵妃手中,笑道:``古语云瑞香花,始缘一比丘,昼寝磐石上,梦中闻花香酷烈,及觉求得之,谓为花中祥瑞,遂名瑞香。有这样祥瑞的花在手,妹妹已然是胜券在握,不必做无谓的担心了。咱们还是花点心思,将怡嫔的后事料理妥当,也让皇上可以稍稍安慰吧。''

次日面见太后的时候,皇后将怡嫔身前死后所有事一一叙述,无不详尽。太后倚在暖阁的榻上,伸手抚摸着青瓷美人觚里插着的几枝新开的粉紫色丁香花:``皇后看看,福珈替哀家插的这一盆丁香花,如何啊?''

皇后正回禀宫中事宜,突然听得太后这一句,忙赔笑道:``福姑姑伺候太后多年,深知太后心意,这盆丁香花一定很合太后的心意。''

太后微微摇头,淡淡道:``福珈,拿剪子来。''

福珈奉上银剪子,太后剪去多余的几枝,道:``如今看着便清爽多了。''

皇后忙道:``儿臣的眼力远不及皇额娘,所以竟看不出来那几枝花枝多余。''

太后淡淡一笑:``皇后,你知道本宫为什么喜欢这盆丁香花么?芭蕉不展丁香结,同向春风各自愁。丁香花开二色,有紫有白,就好比宫中有人得宠高兴,便有人失宠伤心。这次的事玫贵人痛失胎儿,怡嫔母子俱亡,便连娴贵人也受了责罚幽禁在延禧宫中。可是这边伤心欲绝,那边慎常在就跃上龙门,一朝得宠。嘉嫔也身怀龙种,备受尊崇。但皇后你有没有想过,如此一来,宫中就失却了平衡之道了。''

皇后忙躬身道:``儿臣恭听皇额娘教训。''

太后和颜悦色道:``嘉嫔有喜自然是值得高兴,玫贵人失子也的确让人伤心。娴贵人固然被幽禁,但慧贵妃一直未再得到宠爱,被皇上冷落。这个中的平衡之道,皇后你要好好掂量掂量。''

皇后眼中凌波微动,道:``儿臣会向皇上建议,晋封玫贵人为玫嫔稍作安慰。至于慧贵妃,她位分已高,不宜即刻再进封,儿臣会安排慧贵妃再度侍寝,以免嘉嫔有孕不便伺候,让皇上备感寂寞。''

太后微笑道:``皇后能如此,哀家很是欣慰。''她话锋突然一转:``但是海贵人无错却与娴贵人一同幽禁,而娴贵人罪孽深重,仅仅得此责罚,哀家实在是为两位枉死的皇孙感到可惜。皇后,这些话你便替哀家告诉皇上吧。''

皇后略露为难之色,道:``回禀皇额娘,不是臣妾不敢告诉皇上,但只怕皇上一时心软,顾念旧情\ldots\ldots{}''

太后语气森冷,与外头的明丽春色毫不相符,只道:``皇上固然顾念旧情,但哀家的皇孙也不能白白枉死。那就传哀家的旨意,娴贵人乌拉那拉氏谋害皇嗣,罪无可恕,着废为庶人,终身幽居冷宫。哀家倒要看看,哀家要她生不如死,谁敢拦着!''

皇后微微一凛,忙道:``皇太后懿旨,臣妾遵命。''

皇后去请命时,慎常在正在一旁红袖添香,喜乐娱情。纯嫔与海兰亦守在一旁相伴,众人见了皇后来连忙离了皇帝,恭恭敬敬请了安,半分也不敢骄矜。皇后将太后所言一一回禀,皇帝倒也无一不准,但说到如懿之事时,皇帝冷然一笑:``还是皇额娘有决断。朕顾念着她抚养大阿哥,一时还未下狠心。既然皇额娘这样说,那自然是好的。''他扬声唤道:``李玉,你便按皇后所言,传旨下去。''

皇后道:``那大阿哥\ldots\ldots{}''

皇帝微微蹙眉:``大阿哥便交给纯嫔带着吧。纯嫔生养过孩子,理应会管教些。''纯嫔听了,连忙起身谢过。

皇后连忙道:``是,那臣妾预备下去,明日就将乌拉那拉氏移去冷宫居住。只是\ldots\ldots{}''

阿箬轻轻地为皇帝捶着肩,娇声道:``这样也好。眼不见为净,省得皇上想起了就要生气。''

皇后拈了绢子道:``只是\ldots\ldots 乌拉那拉氏虽然有差错,但皇上念在旧情,关几日就会把妹妹放出来的,让妹妹安心去待几天思过就是。''

皇帝看了皇后一眼,不动声色道:``几天?若无朕的旨意,乌拉那拉氏终身不得出冷宫别院半步。''

皇帝话音刚落,海兰脸色煞白,差点晕了过去。海兰身边的叶心机灵,一把扶住了海兰。

海兰忍不住跪下,膝行上前,磕了个头道:``皇上开恩,请念在姐姐在潜邸时就尽心伺候皇上,不敢有一丝懈怠的份上,还请皇上不要把姐姐赶去冷宫吧。''

纯嫔亦道:``是啊。皇上哪怕要罚月银要责打,都比把乌拉那拉氏一辈子孤零零扔在那儿好啊。''

皇帝看也不看纯嫔,只淡淡道:``跟着朕从潜邸过来的嫔妃不少,若都像乌拉那拉氏一般骄纵恣肆,敢蓄意谋害旁人,朕以后如何管治后宫前朝。你们若再求,就和她一并关进去。到时候永璋没有额娘照管,你也别怪朕狠心。''

纯嫔吓得冷汗涔涔,跪在地上不敢言语。海兰还要再说,纯嫔赶紧拉住了她,摇了摇头。

皇后欠身,淡然道:``皇上三思,如懿妹妹到底陪伴皇上多年,没有功劳也有苦劳。''

皇帝散漫地看皇后一眼,微笑道:``乌拉那拉氏有罪当罚,是皇后向朕提出。如今皇额娘也发了话,皇后却要朕宽恕,皇后贤德是贤德,却未免太出尔反尔,难以服众了。''

皇后神色一惊,连忙屈膝:``臣妾糊涂,还请皇上恕罪。''

皇帝道:``起来。''

皇后这不敢多言,微微敛容正要退下,却听殿外有童声响起,却是在背诵一首诗。

``鹿走荒郊壮士追,蛙声紫色总男儿。拔山扛鼎兴何暴,齿剑辞骓志不移。天下不闻歌楚些,帐中唯见叹虞兮。故乡三户终何在?千载乌江不洗悲。''

那童声反复响起,却只是背诵这首诗。

皇后侧耳细听,道:``仿佛是大阿哥的声音,在背诵皇上的御诗。''

皇帝眉心微微一动,转过脸不悦道:``前些日子永璜背了这首御诗给朕听,朕还夸奖了他几句。如今倒越发懂得取巧了。''

皇后忙道:``小孩子家,哪里有这些心机。皇上切莫错怪了他。''

皇帝听了一会儿,终究不忍道:``传他进来吧。''

永璜倒也乖觉,进来了便磕头道:``给皇阿玛请安,给皇额娘请安,给慎常在请安。''

按照规矩,皇子与公主称呼除皇后与生母之外的庶母皆以``娘娘''相称,如今只呼慎常在的位分,而不唤一句``慎娘娘'',显然并非不懂得规矩,而是不屑如此尊称而已。

皇帝便带了几分不豫之色,道:``越发没有规矩了。''

阿箬强笑道:``臣妾原本就是伺候大阿哥养母的宫女,大阿哥不肯按规矩称呼,也是情有可原。''

皇帝指着永璜便道:``这个样子,和乌拉那拉氏一模一样,朕真是后悔把你交给了她抚养。''

大阿哥忍着泪,倔强道:``儿子受母亲抚养,母亲百般教导只是要儿子学好,从未教坏过儿子。不知皇阿玛此言从何而出。今日儿子背诵的御诗乃是母亲亲口教导,母亲时时刻刻把皇阿玛记在心上,又疼爱儿子,怎么会残害皇阿玛的其他子嗣。其中必有冤情,还请皇阿玛明察。''

皇帝连连冷笑道:``反了!真是反了!连朕的亲生儿子都被她蛊惑,口口声声向着她!''

阿箬忙跪下道:``皇上息怒。大阿哥养在延禧宫的时候,乌拉那拉氏百般笼络讨好,其实并非真心疼爱大阿哥,而是借机邀宠,更是为了她一己私心,想要`招弟'。''

``招弟?''皇后诧异道,``什么是招弟?''

``就是民间传言,收养一个男孩后,自己也会在不久之后有孕诞下一个男孩。''

皇后惊诧道:``你是说,就是因为如此,当日乌拉那拉氏才会与慧贵妃相争,故意要抚养大阿哥?''

皇帝伸手将桌上的茶点挥落,怒道:``这就是你口口声声要求情的母亲,以后你不必跟着她,就由纯嫔来抚养你。朕告诉你,也告诉所有人,都听着,以后朕不许任何人为乌拉那拉氏求情,若有违背,就和她一起去冷宫待着吧。''

\hypertarget{ux7b2cux5341ux4e00ux7ae0-ux5e7dux5c45}{%
\chapter{第十一章 幽居}\label{ux7b2cux5341ux4e00ux7ae0-ux5e7dux5c45}}

永璜回到延禧宫中,见到宫中苍黄昏暗,浑不似一个曾经得宠的主位所住的地方,更想起昔日伺候的阿箬如今在皇帝身边的亲昵模样,纵使心性坚强,也忍不住落下泪来,一头扑入如懿怀中,哭道:``母亲,母亲\ldots\ldots{}''

如懿抱住他好生安慰道:``好孩子,回来了就好。母亲交代你的,你都做好了么?''

永璜哭着道:``儿子不敢辜负母亲,都已经做好了。''

``那你皇阿玛生你的气了么?''

``生了好大的气。还说不许儿子再跟着母亲,要搬去纯娘娘宫中居住,由她抚养儿子。''

如懿心口一松,情不自禁笑出来道:``那就好。纯嫔娘娘位分既高,性子也好,自己又生养过,知道怎么照顾你。你有了好去处,母亲也高兴。''

海兰跟着进来,陪着落泪道:``姐姐何必如此,不让大阿哥求情也罢了,偏还要借着求情去惹恼了皇上,还要皇后和慎常在在旁边看笑话。''

``这个笑话,必得让人看见了才好。''如懿深吸一口气,搂着永璜道,``好孩子,母亲的苦心,你都明白么?''

永璜点头道:``以后儿子不能太露锋芒,更不能太讨皇阿玛喜欢,抢了二弟的风头,以免被人觊觎陷害。''

如懿含泪点头道:``好孩子。以后没有母亲护着你,万万记得要保护好自己,韬光养晦,千万不能显露锋芒。若有什么要紧事,便悄悄儿去找海娘娘,她会护着你的。''

永璜点头道:``所以儿子今天惹了皇阿玛生气,以后看着皇阿玛好像不像以前那么喜欢儿子了,儿子也更安全了。''

如懿连连颔首:``一点就透,真是母亲的好儿子。这样母亲以后即便出不了延禧宫,也能安心了。''

永璜擦干了眼泪道:``可是儿子今日在皇阿玛那里听说,要把母亲移去冷宫,还要废母亲为庶人。''

如懿立时怔在当地,只觉得热泪滚滚而落,刺而痒地扎在肌肤上。

如懿满面是泪,眼中的神采只剩下了乌沉沉的伤心与无奈。``从阿箬被接到皇上身边那刻起,我就知道我的劫数还没完。又说下旨封了慎常在,如此盛宠,再加上旁人的话\ldots\ldots{}''她泣不成声,只觉得心里的惊痛如一副千斤重的磨盘一道接一道碾下,几乎要将一颗已经溃不成军的心磨成齑粉四散在风里,``皇上\ldots\ldots 竟然疑我到这种地步!''

海兰啜泣道:``众口铄金,积毁销骨,何况如今慎常在是皇上枕边的心尖子。皇上一时轻信\ldots\ldots\textless{}''

原以为已经掉到了深渊底下,却没有想到还有一重深渊,如同十八层地狱,要重重堕下,永无超生的可能。原来所谓人生路,不是只有前行与后退,还会如此下坠,坠到连自己也想不到的凄苦之地去。如懿无限凄惘,苦笑道:``一时轻信,也要相信了才好\ldots\ldots 若是不信,终究旁人再多言语也是无用!''

正说话间,却见李玉已经过来传旨,延禧宫中愈加乱作一团,宫人们自伤前程,纷纷哭了起来。李玉不耐烦道:``哭什么哭,小主被贬为庶人,你们自然是不用留在延禧宫伺候了。都给我出去,至于以后的去路,内务府会给你们安排的。''

一时间宫人们都退了出去。海兰趁没有外人在,低声道:``李公公,这件事还有没有办法转圜?''

李玉苦着脸道:``小主,事情已经无法转圜了。皇上金口玉言,谁也不能劝。再加上阿箬\ldots\ldots{}''他作势拍了下自己的脸,低声道:``慎常在几乎是专房之宠,皇上时常要她陪着,旁人要进言也不能啊。''

海兰道:``可是因为大阿哥激怒了皇上的缘故?''

李玉忙道:``那倒不是。小主啊,趁着现在只有奴才在,明天又是奴才送小主入冷宫,一些金银细软,小主好好收拾起来,到了冷宫那种地方,也有要用钱的地方啊!''

他话音未落,却听殿门``吱呀''一声被推开,三宝和惢心哭着进来跪下道:``小主,奴婢和三宝商议过了,奴婢哪里也不去,和三宝跟着去冷宫伺候小主就是了。''

如懿落泪道:``你们可疯了,跟我去那儿做什么?留在外头,还能找个好主子伺候。''

李玉道:``可不是,二位可别糊涂了。''

惢心哭道:``奴婢自知命贱,留在外头也只是被人轻贱,情愿跟着小主。奴婢说过,要一生一世伺候小主的。''

三宝亦道:``奴才也跟着去。''

李玉想了想道:``小主虽然被废为庶人,但冷宫里也不能没有人照顾,带一个去也是可以的。别的不说,以前惢心和阿箬不总是合不来么,留她在外头,只怕委屈更多。''

如懿擦了擦泪道:``那好。冷宫再苦,惢心跟着我总还好些。至于三宝\ldots\ldots{}''她看了戚戚然的海兰一眼:``你便跟在海贵人身边,从此伺候海贵人吧。''

海兰正欲说话,如懿挡住了道:``我知道你要推辞,可你身边只有叶心和春熙,三宝在你身边,也多个照应。''她忍不住热泪潸潸:``从此,我们想要相依为命、守望相助也不能得了\ldots\ldots 你\ldots\ldots 要好好护着自己。''

李玉道:``奴才不能多留。那惢心,你陪小主好好收拾,明日奴才送小主去吧。''他伸手请过永璜:``大阿哥,按着旨意,奴才眼下得把您送去纯嫔娘娘那儿了。''

永璜满脸是泪,只扯着如懿的袖子依依不舍,如懿含泪放开他手,强忍着道:``去吧,好好活着。记得,出了这里就不要再回头看,一定不要。以后也别再和任何人提起母亲,知道么?''

永璜哭着走了出去,果然没有再回头。如懿的泪潸然而下:``真是听话的孩子。''

李玉伤感道:``小主连大阿哥都这么疼爱,奴才实在不相信小主会去害别人的孩子。''

如懿用力按住眼角即将落下的泪:``什么都不必说了。李玉,幸好你还在皇上身边,如果你还记得我曾经扶持过你,那么有朝一日,在保全自己的情况下,能帮上手的时候,一定要帮一把,别让我死在了冷宫里也不得瞑目。''

李玉跪下磕了头道:``奴才永远都会记得,是谁替奴才上了药,是谁暗中拉拔奴才到了今时今日这个位置。''

如懿点头道:``你知道就好。你坐到这个位子不容易,当年王钦是怎么掉下来的,如今你自己也要小心。''

李玉感激得热泪盈眶:``奴才没有别的本事,但会尽一己之力,极力保全小主在冷宫的平安。''

如懿沉默片刻:``那你再帮我一个忙,我想最后见一见皇上。''

李玉一怔,只得点了点头。

如懿再见到皇帝的时候已经是黄昏时分,养心殿还未掌灯,殿内是金红色的淡淡余晖,由着光影由浓转淡。皇帝的语气听不出一点悲喜之情,只是低头练着书法,并不看她一眼:``事情已经定下了,还要来见朕做什么?''

如懿抬头看着皇帝:``臣妾注定是要去冷宫了,只是最后还未能死心,一定要来问一问皇上。皇上,您是否相信世间有公允之道?''

皇帝看着她,仿佛看着一个寻常的陌生人一般,口气却郑重其事:``朕相信。''

如懿望着皇帝,仿佛要从他脸上探出什么究竟一般。然而,她知道,她的路是他给的,她再不能看出什么来了。

心底的相信逐渐被动摇,生了碎刺般的疑惑。但她逼迫着自己,若是连自己都不信了,还能留下什么。

茫然的动摇与悲望之中,如懿伏身三拜,神色哀伤而平静:``为着皇上这句话,臣妾甘愿受罚,长居冷宫。只求皇上福绥安康,岁岁长乐。''

如懿缓缓起身,拂去身上尘灰,澹然若出世之云,转身离去。

皇帝看着她,将写好的字幅揉成一团,随手丢在了地上,缓缓瘫坐在龙椅之上。

宫人们散去后,延禧宫已经冷落一片,封妃的册文、金印、吉服全部被带走,满地狼藉凄冷,让人不忍卒睹。海兰亦被留在后殿,不许再踏入延禧宫正殿半步。

惢心默默陪在如懿身边,将一些贴身衣物和值钱的首饰一同包好,想了想将钱财首饰藏在包袱的最深处,又取过一些糕点收好:``到了冷宫只怕衣食不周,什么都得备下些。''

如懿看着她一点一点收拾,便道:``拿那些点心做什么,备下了明天的,后天也要过那些苦日子。还是收拾些衣衫要紧。''

惢心答应了``是'',便去翻开箱笼,重新收拾衣裳。

正忙碌着,只听殿门被推开的悠长声,如懿不承想此刻还会有人来延禧宫,回过头去,却见是太后身边的成公公,他哑着嗓子道:``太后传召,乌拉那拉氏,随我走一趟吧。''

惢心担心地看着如懿,不知祸福几何。如懿强自定了定心神,事情已经坏到这样的地步,还能如何?

她便道:``我这样去,不会太点眼么?''

成公公努努嘴道:``赶紧换上你宫女的衣服,跟我走吧。''

如懿想了想,便取过惢心的一身宫人装束换上,又梳成宫人们的发髻,仔细看看,走在夜色中应当不算明显了。

去太后宫中的路并不算太远,如懿隐隐想着,这大约是最后一次去慈宁宫了吧。此生此世,她大约都要留在冷宫之中,遥望紫禁城万千灯火金玉绚烂的夜晚。

正想着,成公公已经打起帘子让了她进去。大约是要避开旁人,殿中只有太后和福姑姑两人在。

太后穿着绛色缂金水仙团寿单氅衣,头上与耳上都一色的点翠东珠配翡翠首饰,那碧艳的宝蓝色在灯火的跳跃之下,流转着暗沉不定的光泽,好像太后这个人便是如此,让人觉得暗沉而不可捉摸。太后跪在佛龛前,诚心诵完佛经,又点燃了三支檀香敬上。那香上的三点暗红星火,如同她心里若隐若现的未知的惧怕。

太后扶着福姑姑的手起身,转过脸慢慢打量着她。如懿依足规矩福了一福,请安道:``太后娘娘万福金安。''

太后淡淡道:``到底是乌拉那拉氏的女儿,到了这种境地,居然没有一进来就哭着求哀家饶恕。''

如懿垂手立在一旁,宛如一个宫女应有的姿态:``太后亲口下的懿旨,不容更改,求也无用。''

太后微微一笑:``哀家在想,如果今日被贬为庶人关进冷宫的人是你姑母,她会怎样?''

如懿心头一搐,像是被人冷不防狠狠抽了一鞭:``如懿无用,不能和姑母相提并论。''

太后手上的赤金翡翠点珠护甲恍如一把金色的利刃,轻轻一晃:``你们姑侄俩也真是可怜,居然都落得幽禁终身的命运,你是不是要怪哀家心狠。''

如懿眼中一酸,将眼泪逼在眼底不容它落下:``如懿要怪,只怪自己不谨慎,才会落入旁人圈套。''

太后和颐浅笑,抚了抚手腕上玛瑙连珠镯:``只要是活在宫里的人,但凡不是个神仙,人人都会有不谨慎的时候,人人也都会有百口莫辩的时候。但要紧的是,人在低谷的时候懂得如何自保。不保别的,就只保自己一条命。''

如懿眉心一动,若有所思:``可是冷宫,形同死地,生不如死。''

``是么?''太后不置可否地笑笑,从桌上一盘未动过的糕点里取了一块,小心用绢子拈在手里,抬眼问道,``福珈,哀家要你抱来的猫呢?''

福珈抱了一只寻常的灰猫上前,太后随手将糕点丢在地上道:``给它吃了。''

福珈将糕点喂到灰猫口中,如懿满腹狐疑地看着,直到吃下糕点的灰猫在挣扎之后流血而亡,她的惊惧再也掩藏不住,跪下道:``太后\ldots\ldots{}''

太后扬一扬脸,示意福珈把死去的灰猫拿布裹住扔出去,方才缓缓道:``这是今日一早御膳房要送去给你的糕点,你一旦吃下,就成了畏罪自尽,再也无力回天了。要不是福珈看着可疑替你拦下了送到哀家跟前来,你只怕连自己是怎么死的都不知道。这件事也提醒了哀家,与其让你等在延禧宫中让什么人都能伸手掐死你,还不如把你丢去冷宫,绝了所有人的心思,你也能保住这条命了。''

如懿将信将疑:``如懿的姑母生前冒犯太后,太后为何要保全如懿一条性命?''

``若是只执著于从前的爱恨纠缠,哀家这个太后目光也太短浅了些。''太后取过佛珠缓缓捻着,含了一缕淡薄的笑意,``你自然恨哀家,是哀家要囚禁了你,但终身不得出。不止你,所有人都以为哀家恨极了你姑母,所以迁怒于你。可是你若未被禁足冷宫,还禁得起她们几次折腾?若在冷宫,或许还有一线生机。''

如懿低头默默片刻:``太后说得是。太后纵然是顾虑臣妾,爱惜臣妾性命。可冷宫之中艰辛困苦,暗算之事亦层出不穷。臣妾只能祈求太后庇佑,容许臣妾活到沉冤得雪的那一天。''

太后的笑意仿佛海底的流光一烁:``哀家倒也想,只是六宫之中都是眼睛,哀家何以要偏心你一点。所以哀家只管到你现在为止,等进了冷宫,有没有这个本事躲得过明枪暗箭,学会苟延残喘,就要看你自己的了。''

如懿心中悚然一惊,便道:``是。''

``你要是连这点保着自己福大命大的本事都没有,后宫里埋下的女人成百上千,都为紫禁城的红墙积了血色,也不多你一个。''太后捻着一串紫檀翡翠佛珠,悠悠道,``但是在冷宫里,总比在外头风刀霜剑好过多了。其中的道理,你自己好好掂量掂量。''

如懿思忖片刻,蓦然伏拜:``太后的意思,如懿明白了。只有人人都当如懿是不中用的人了,如懿才能真正平安。''

太后颔首一笑:``无为而治,无欲则刚,你明白了么?你越露出你在乎什么,想要什么,就是把自己最大的弱点暴露人前。所以,无欲无求,别人才会以为你无害。''

如懿心悦诚服,亦有些赧然:``太后所言乃至理名言,可是要到如此境界,如懿实在\ldots\ldots{}''

太后闭目一瞬,很快笑道:``所有的修为,都是历练出来的。你今后有的是时日,慢慢琢磨着吧。''

如懿心中稍稍安定,告辞离去。十二扇楠木雕花嵌寿字镜心屏风后绯色罗裙一闪,漾起明艳如云霞的波縠,却是玫贵人盈盈转出,半跪在太后榻前替她捶着腿道:``太后如此护着乌拉那拉氏庶人,还悉心调教,可真是心疼她。''

太后用护甲挑起珐琅罐里的一点薄荷膏轻轻一嗅,方把罐子交到玫贵人手里,笑道:``不是哀家心疼她,是别人越看重她,用尽了心思对付她,便越是叫哀家知道,她是有分量和那些人分庭抗礼的。后宫之中最要紧的便是平衡之道,如果有谁太盛势了,得尽恩宠与权位,哀家这个太后便没有置喙之地了。''

玫贵人取过薄荷膏一点一点替太后揉着太阳穴:``那太后就应该留下乌拉那拉氏庶人,好跟那些人平分春色啊。''

太后抬眼看她一眼:``怎么?你不觉得是乌拉那拉氏害了你的孩子?''

玫贵人垂下眼睑,将悲伤不露痕迹地藏于眼底,道:``人赃并获,天衣无缝,的确是无可指摘。但,越是这样,反而让人起疑。''

太后微微颔首,叹口气道:``总算有些长进。那你以为是谁?''

玫贵人道:``是谁都不要紧。天网恢恢,疏而不漏,臣妾不必用心去查,若有机会,乌拉那拉氏一定会比臣妾更着紧。臣妾只要一心固宠就是了。''

太后道:``吃一堑长一智,你也算知道些了。后宫之中急于平分春色是没有用的,保得住性命学得会立足才最要紧。''

玫贵人凛然道:``是,臣妾明白了。''

太后轻轻``嗯''一声:``如今慎常在新宠上位,撒娇撒痴。嘉嫔有孕在身,有恃无恐。眼见她留在养心殿的臻祥馆养胎,有皇帝在身边,这一胎必然是无碍了。丢了你和怡嫔的两个孩子,无论嘉嫔这一胎是男是女,她母凭子贵都是毋庸置疑的了。那么你呢?哀家那么辛苦把你从南苑捞出来,又想尽办法保全你。来日如何,全在你自己了。''

玫贵人即刻紧张起来:``是。臣妾一定不会辜负太后期望。''

如懿离开延禧宫那一日,春光如一幅巨大而明艳的绸缎,铺开漫天漫地的晴丝万缕,袅娜如线,看得韶光亦轻贱了岁月。

那漾艳的春光,仿佛一卷上好的精工细描的锦绘,铺陈开花鸟浮艳,刺绣描金的华光,让人几乎睁不开眼睛。

来相送的,唯有海兰和纯嫔,海兰无声地落着泪,被李玉拦着不许上前半步。连纯嫔,亦站得远远的,只能含泪微微点头,以示话别。如懿只以素银扁方挽起长发,穿着无绣无花的薄薄春衫,唯有上面细细的暗纹流转,昭示着她依旧不能离开宫廷寸步。

经过景仁宫的时候,如懿仰起头,看着浮光万丈,金灿炫目。原来辗转浮沉,她的命数,和她的姑母并没有不同。

殊途同归,是不是后宫女人唯一的路?

所谓``冷宫'',便是在翠云馆后一所空置的院落。因为历代失宠犯错的嫔妃都被发落安置在此处终身不得出入,便被宫中人视若冷宫,十分避讳。

幸而历代以来,在寿康、慈宁两宫养老的妃嫔居多,幽闭冷宫终身的女人并不算太多。纵然已经想象过多次,然而走到冷宫前,如懿还是微微意外。她入宫多时,从未走到过这样荒僻而冷清的地方,仿佛从前无人提起,她也从不知道宫里竟有这样的地方。那是一处废旧宫殿模样的房子,不算很大,零零落落十来间屋子错杂其间,像是久无人居住了,宫瓦上蔓生的野草纷杂,连大门上也积了厚厚的尘灰,满目疮痍。她伸手一触,门上的铜钉便扑扑落下一层锈灰来,差点迷了人的眼睛。里头雕栏画栋的描金绘彩尽数脱落,积着厚厚的灰尘和凌乱密集的蛛网。

才一进去,就觉得明亮的天光都被隔绝在了外头。即便是这样晴朗的天气,里头也是阴阴欲雨的昏暗,住得久了,好像身上都会长出暗青色的绿霉来。

李玉领着如懿和惢心走到一间略为整齐的空屋子里,尚未靠近,已有尘灰呛人的气息扑鼻而来,李玉为难道:``小主,奴才已经尽力了。''

如懿了然,感激道:``能找出一个让我和惢心住的屋子已经不容易了。若要再做什么,就太点眼了。好了,你不必在此久留,免得惹人注目。''

李玉点点头,看了看旁边的屋子道:``小主住在这里,千万小心旁边那些人,年纪大了,都成了精怪了。''

惢心看着里外都阴森森的,有些害怕地贴在如懿身边。

外头远远传来礼乐欢喜悠扬的声音,如懿侧耳道:``是什么事?''

李玉犹豫片刻,还是道:``今日是嘉嫔、玫嫔和慎常在行册封礼的日子。听说为着晋封,内务府还要挑出许多宫人来伺候呢。''

如懿将心底的空落按了又按,能如何呢?再热闹,再繁丽,那毕竟是与她无关的人世了。李玉转身离去,如懿看着他的离开将仅存的光明一同带走,只留下无尽的尘灰飞扬和暗沉光影,与她闭锁此间,一生一世。

\hypertarget{ux7b2cux5341ux4e8cux7ae0-ux7a7aux8c37ux4e0a}{%
\chapter{第十二章
空谷(上)}\label{ux7b2cux5341ux4e8cux7ae0-ux7a7aux8c37ux4e0a}}

幽闭的宫苑中,好像日日都下着雨。虽然知道有人一同住着,但总是无声无息,好像待得久了,人也成了鬼魂,没有动静。

如懿和惢心绞了帕子忙碌着打扫,虽然自小养尊处优,不事辛劳,但强逼着自己做起来,也能慢慢做得好。她和惢心忙进忙出,分明是觉得有眼睛在窥探着她们的,但猛然回头去,却又不见人影。

惢心有些害怕:``小主,住在这里的,到底是人还是鬼?''

如懿强自镇定下来,沉声道:``当然是人,这世上哪有鬼?''

惢心有些不安地翻着包袱:``早知道就该多备些蜡烛了,这里不分白天黑夜都黑漆漆的,让人看了害怕。''

到了夜间,两人总算收拾干净了住下。因着每日给的蜡烛只有两根,两个人都当宝贝似的积攒着,加之劳累,天一黑便睡下了。才躺下没多久,只觉得身上的被衾盖着一阵比一阵凉,仿佛是起风了。风自由地穿行在回廊梁柱之间,哗哗地吹起破旧不堪的窗纸,有窗棂吱嘎地摇晃,划出一阵阵几欲刮破耳膜的刺声,啪一下,又一下,仿佛突如其来地敲着人原本就瑟瑟不安的心。

有闪电的光线骤然亮起,残破的纸窗外,分明有人影倏忽晃过。惢心吓得连声尖叫起来:``有鬼------有鬼------''

如懿来不及披衣,点上蜡烛霍然打开门,直冲到外头。脆弱的火光在疾旋的风中微弱地挣扎了几下便灭了。四周黑漆漆的,只有几个破旧的宫灯晃着微弱的火光,和偶尔划过天际的闪电,照亮这破败的庭院。

如懿索性将手中的烛台一扔,金属滚地有刺耳的鸣响。如懿大声道:``不管你们是人是鬼,我既然来了这儿走不了,便是做人也好做鬼也好,也要和你们待在一起。有本事就自己走出来给我瞧瞧,装神弄鬼,难道被遗弃的女人只会做这样的事情么?''

惢心随后冲了出来,披了一件外裳在她身上:``小主,小主,起风了,要下雨了,你小心着凉!''

如懿扯下衣裳甩到她手中,厉声道:``有本事就出来,有什么可吓人的!我若是即刻死在了这里,也比你们这些装神弄鬼只会暗中窥伺的人强!想来吓唬我,便是做了厉鬼,你们见了我也只会躲躲闪闪,避之不及!''

闪电划过处,几张苍老而残破的面容隐约浮现。如懿心生一计,转身去房中取过包袱中的糕点,向面容浮现中一一抛掷而去。很快,有几个年长的妇人从廊柱后转出,纷纷抢过糕点,呵呵笑着,心满意足而去。

如懿稍稍心安,惢心急道:``小主\ldots\ldots{}''

如懿道:``就算是鬼魂,贪于饮食,有什么好害怕?\textless{}''

一声凄厉的冷笑自梁柱后缓缓转出,如懿借着昏黄的宫灯看去,却是一个年迈妇人缓步过来。她的衣着打扮比其余人稍显洁净舒展,只是头发花白,满脸皱纹,老态龙钟,看上去已有六七十岁。

如懿看她沉着走进,并不似旁人贪恋糕点,心知此人一定不寻常,便先拜下道:``晚辈乌拉那拉氏如懿,给前辈请安?''

``前辈?''那老妇人摸一摸自己的脸,森然道,``我很老么?''

如懿见她阴恻恻的,也不免添了一分畏惧,只得坦然道:``既然熬在了这里,即便青春貌美又有什么用?反而年老长寿,才能熬得下去。''

``年老长寿?''那妇人连连冷笑,``熬在这种不见天日的地方,活着还不如死了。''

如懿心中闪过一丝刚硬之气:``话虽这样说,但前辈没有寻死,便知蝼蚁尚且贪生。''

那老妇人虽然年迈,眼中却闪过一丝精光:``是啊,来了冷宫的人没几个熬得住的,你方才看到的那几个便已经疯疯癫癫了,你看不见的那些,都是熬不住自己上吊死了的。冷宫的亡魂不少,你倒不怕?''

如懿黯然道:``迟早也要成为其中一缕亡魂,这样想想,还有什么可怕。''

那妇人不置可否地一笑:``这冷宫,总算来了个异数。''她说罢,缥缈离去。

如懿后退一步,才觉得背心的睡衣已经都被冷汗湿透。如懿长舒了一口气,拍拍惢心的手道:``算是见过了,可以安心睡了。''

惢心畏惧地和如懿贴在一起,如懿笑道:``你便和我一起睡吧。''

一夜风雨大作,起来也是个阴沉天气。惢心跟在如懿身后亦步亦趋,小心翼翼地问:``小主真要去看么?''

如懿换了一身更简朴的衣袍,故意打扮得灰扑扑的:``昨夜她们已经按捺不住来看了我,难道我不去看她们么?''

其实她住的地方与其他人还隔了一座院落,重重曲廊转过去,却听得前面窸窣有声,似有好些人围在那里看着什么。她疾步过去一看,吓得不由得退了一步,原来一座空空的殿阁里,一个女人高高地把自己挂在梁上,只有一双脚摇摇晃晃地,每一动,都散下一点尘灰来。

惢心吓得尖叫一声,指着道:``小主,小主,有人吊死了。''

那些围观的妇人们只是冷漠地望了她们俩一眼,又望了望吊死的女人,毫无惊异地散开了。有人不无羡慕地笑起来:``真好,她去见先帝了。先帝见着了她,一定还会宠幸她的。真是有福了。''

昨夜稍稍整齐的老妇人跟在人群后出来,淡漠地望了惢心一眼:``不必大惊小怪,熬不住自杀的人天天有,你以后住久了就知道了。''

惢心吓得脸色发白,颤抖着说不出话来。那老妇人淡淡道:``你呢?什么时候你也熬不住也把自己挂上去呢?''

如懿觉得自己的身体有点不受控制地发抖,她指着梁上的女人道:``那她怎么办?''

老妇人怪异地笑了笑:``等下会有侍卫来把她拖出去,拖到焚化场烧了,埋了。真好,死了,化了,终于离开了这个鬼地方。''

惢心吃惊道:``这里也有侍卫?''

老妇人鄙夷地看她一眼:``当然。要不然你不是随时随地都可以从这里推门走出去?''惢心惊慌失措地去拍门,惊呼道:``有人么?有人么?里头有人上吊死了!''

良久,有个头儿模样的侍卫懒洋洋地探头进来看了一眼,挥了挥手道:``凌云彻,赵九宵,你们俩去收拾一下。''

分明是个人,倒是像被当做物件,连死后的尊严亦没有,只是被``收拾''一下。如懿见两个大男人伸手就要抱那妇人的尸体下来,忙急道:``你们是两个男人,怎么可以伸手接触前朝嫔妃的尸身这样冒犯不敬?''

凌云彻这才看见如懿,他微微眯起眼睛,似是被她容貌微微惊住,屏息的片刻他旋即收手,在一旁不再触碰。

赵九宵懒懒笑了笑道:``不碰,好哇!那咱们兄弟俩就不干了,劳您自己动手吧。''

如懿被他一激,想到自己来日的下场,亦不觉兔死狐悲,一把拔出他腰间的长刀扔到惢心手里:``惢心,你站到凳子上去砍断绳索,我在下面抱着她。''

惢心有点颤颤的,但见如懿选择抱着尸体,她亦无法可想,只得站到凳子上砍断了挂在梁上的绳索,尸体掉下的冲力极大,如懿一个抱不住,踉跄着连人带尸全摔倒在了地上。她离着那尸身那么近,几乎可以触到尸体上冰凉的死亡气息和那干冷的完全失去了生气的肌肤。

她丢开手,忍不住俯身干呕了几声。

赵九霄像是看着一个有趣的热闹:``既然吓成这样,逞什么强?你既然不许我们兄弟碰,这尸体,我们不抬了!''

如懿仰起脸冷冷看着他道:``要是进了冷宫,我还能出去半步,这具尸身自然不用你们来搬了。何况我只是要你们不许用手直接碰触,并非不让你们抬出去。''

凌云彻奇怪地瞥她一眼:``那你说怎么办?''

如懿转过身,想要在周遭寻到一块裹尸的大布,却左右不见踪影,那老妇人本冷眼旁观,见她如此,转身去隔壁拎了一块硕大的白布来:``这块原是我留着给自己的,如今先给她用吧。只是来日我走之前,你们必得拿自己的衣衫拼缝一块裹尸布送我走。''

如懿感激道:``是。''她和惢心用布裹好尸身,留出两头可以抬的地方,道:``有劳两位了。''

赵九霄见她如此麻烦,本来就心生不忿,懒洋洋地看着天不肯动手。凌云彻看不过去,伸手推了他一把,道:``动手吧,完了还有别的事。''

赵九霄会意,笑嘻嘻道:``只有你还有别的事,我却没有了。''

凌云彻也不理会,伸手抬起尸身的一头,赵九霄便也搭了把手,一起出去了。

如懿这才松了口气,赶紧回到房中拼命洗脸洗手,又换了一身干净衣裳,那种恶心的感觉才没有那么强烈了。那老妇人大剌剌走进她房中,仿佛入了无人之地,自己找了盏干净的茶盏倒了点白水喝了:``既然那么怕,就别去碰。''

如懿洗干净手:``总有一天,我也会那样,是不是?''

那老妇人并不理会,只道:``没想过活着出去?''

如懿犹疑片刻:``前辈在这儿待了多少年?''

那老妇人横她一眼:``前辈?我没有名字么?''

如懿见她性情古怪,忙恭恭敬敬道:``还请您老人家赐教。''

那老妇人掸了掸衣衫:``我是先帝的吉嫔。''她自嘲地一嗤:``可是我一辈子都没吉利过,还留着名位呢,就被关进了这里。''

如懿忙起身道:``晚辈乌拉那拉氏如懿,见过吉太嫔。''

``太嫔?''她黯然一笑,``是啊。先帝过世,我可不是成了太嫔?可惜啊,人家是寿康宫里颐养天年的太嫔,尊贵如天上的凤凰;我是关在这儿苦度年月的太嫔,贱如虫豸。''她忽然警醒,``你说你是乌拉那拉氏?那先帝的皇后乌拉那拉氏是你什么人?''

如懿道:``两位乌拉那拉氏皇后,都是我的姑母。''

``两位?''吉太嫔冷笑道,``一位就够厉害了。不过,再厉害也厉害不过当今太后啊,否则怎么会连你也落到冷宫里来了。不过我到这冷宫八九年了,从未听说有人走出去过,我倒很想看看,乌拉那拉氏家的女儿,能不能走得出去。''

如懿吃惊道:``您才到冷宫八九年,那您今年\ldots\ldots{}''

吉太嫔抚摸着自己的脸,哀伤道:``你以为我七老八十了?我被太后那老妖婆害得进这个鬼地方的那一年是二十六岁,如今也才三十五岁而已。''如懿惊得喉咙里发不出一点声音,只能以不可置信的目光瞪着她。吉太嫔恢复了方才的那种冷漠:``这里的日子,一天是当一年过的,熬不熬得住,就看你自己的了。''

如懿眼看着她出去,满心惊惶也终于化作了不安与忧愁:``惢心,对不住。让你和我一起来了这样的地方。''

惢心有些畏惧,却还镇定:``小主在哪里,奴婢也在哪里。''

如懿再也忍不住满心的伤痛,那种痛绵绵的伤痛,原本只是像虫蚁在慢慢地啃噬,初入冷宫时的种种惊惧之下,她原不觉得有多痛多难熬。可是仿佛是一个被麻木久了的人,此刻她骤然低头,才发觉自己的身体发肤已被这微小的吞噬蛀去了大半,那种震惊与惨痛,让她不忍去看,亦不忍去想。原来,她真的已经失去了那么多,地位、家族、荣耀以及她一直倚仗的他的信赖。都没有了。

可是,她却再没有办法。人在任何境地都有自己眼前的企求,譬如嘉嫔企求生下皇子;慧贵妃企求恩宠一如从前;而阿箬,企求圣眷不衰。她所企求的,只能是学着先活下来,仅仅是活下来。

而门外的凌云彻呢,在把冷宫嫔妃的尸体送去焚化场焚化后,他所愿的,是什么呢?他那样微红的英气的脸庞,疏朗的剑眉亦飞扬起来,站在冷宫和翠云馆偏僻的甬道上,仰首期盼着明媚的少女匆匆向自己奔来,那真是无趣而没有出头之日的冷宫侍卫最美好最乐意所见的场景。

那少女像一只轻盈的蝴蝶扑扇着冷宫前狭长而冷清的石板,虽然只是穿着宫女最寻常不过的青色衣装,她玉蕊琼英一般的娇美面容,依然如一抹最亮的艳色,无可阻挡地撞入了他眼帘。

云彻见她跑近,忙关切道:``嬿婉,跑慢一些,等下跑得累了还要再去当差,更累着自己了。''

嬿婉扶着弱不胜衣的细腰,微微喘着气道:``我就是要跑得快一些,才能多见你一会儿。''她的脸不知因为跑得太急还是羞怯,泛出珊瑚一样的娇润之色,``云彻哥哥,你是不是等了很久?''

云彻忙道:``没有。我只是稍微早一点来,这样就能看着你来。我和九宵说好了,他会替我一会儿。''

嬿婉稍稍放心,笑靥如花道:``那就好,我也和四执库的芬姑姑告了假,说肚子不舒服就出来了。''她看了看周遭,叹口气道:``平日里只有你和赵九霄看着,一定很辛苦吧?每天能做的事情就是守在门口看看天,或者进去替她们搬运尸体。云彻哥哥,为什么我们都那么命苦,没有出头之日?''

云彻道:``你还是想离开四执库?''

嬿婉黯然道:``虽然伺候的是皇上的衣物,但每天只和衣裳打交道,哪一天能够有个好前程。云彻哥哥,我才十四岁,我不想一辈子都在四执库受人呼喝。若是到个好一点的宫里伺候得宠的小主,我也能拉你离开这儿。那么我们\ldots\ldots{}''

云彻摇头道:``何必呢?得宠的小主宫里是非自然多。你不知道昨日进冷宫的那位,还是皇上的娴妃娘娘呢,还不是要在冷宫凄冷终身?何况是小小宫女,一个不小心被主子打死了也是活该,还不如四执库清清静静地安生。''

嬿婉撅起嘴,生了几分委屈之意:``是清静,是安生,可要是过了二十五岁还留在那里,我就要被送出宫了。我虽然是正黄旗包衣出身,但若不是几年前我阿玛犯了事丢了官职,家里门楣虽然低些,也好歹是个格格。可如今我不过是包衣奴才家送进宫的宫女。如果我没有个好去处,没有个好主子替我指婚,那我和你\ldots\ldots 我和你\ldots\ldots{}''她害羞得说不下去,只看着他的眼睛问:``云彻哥哥,你的心意没有变过吧?''

云彻恳切道:``当然没有。虽然我比你早入宫三年,又年长你六岁,但能遇到家乡故知已经很不容易,我和你又\ldots\ldots 情投意合,我的心意绝不会改变。''

嬿婉高兴起来,甜美的笑意再度绽放在唇角:``那就好。昨日是嘉嫔、玫嫔和慎常在行册封礼的日子,过几天内务府马上要挑选宫女去伺候她们,如果我能去伺候嘉嫔娘娘或是慎常在就好了,如今宫中最得宠的就是她们呢。''她按了按袖口:``我已经存了一小笔银子了,到时候只要买通芬姑姑,她愿意荐我去就好了。''她为难地看一眼云彻:``只是我怕银子还不够\ldots\ldots{}''

云彻为难地皱了皱眉,还是道:``你别急,我还有点俸例,再不行的话,我会想想别的办法。''

嬿婉高兴地点点头,眼中闪过一丝倔强的坚韧:``云彻哥哥,宫中我没有别的人,只能依靠你了。''她伸出双手,露出手指上森森的新旧伤痕,凄苦道:``云彻哥哥,我每天都不断地熨衣裳熏衣裳,已经两年了。管事的姑姑们只要一个不高兴,就可以拿滚烫的铁熨子朝我扔过来,拿炭灰泼我。我真的不想一辈子都做一个四执库的宫女,也不想你一辈子都困在冷宫当差。我知道的,你一直想做一个堂堂正正的神武门侍卫,甚至在皇上的御前当差。你放心,只要我们抓住机会,一定不会屈居人下的。''

云彻点点头,小心翼翼地替她呵着手道:``比起我在冷宫这里空有抱负,浪费年华,我更心疼你被人欺凌。你放心,我一定会想办法的。''

嬿婉被他小心地捧着手,心中温暖如绵,好像一万丈的阳光一起倾落,也比不上此刻的温暖和煦。她摸着左手手指上一个色泽黯淡的红宝石戒指,那是红宝石粉研了末做成的,原不值什么钱,却是凌云彻送给她的一片心意。他们原是这紫禁城中贫寒的一对,能有这份心意,已经足够温暖。她柔声道:``有时候再苦再累,看着你送我的这个戒指,就觉得心里舒畅多了。''

云彻的脸微微发红,静了片刻道:``嬿婉,我知道自己没什么银子,只能送你宝石粉的戒指。但我有最好的,一定都会给你,你相信我。''

嬿婉满脸红晕,低下头吻了吻云彻的手指,害羞地回头跑走了。

云彻在嬿婉离开后许久,目光再度触及冷宫深闭而斑驳的大门。他逐渐明白,自己愿意帮助冷宫中那个奇怪而倔强的女人,多半是因为她的脸和美好如菡萏的嬿婉,实在是有三分相似。这样想着,他的一颗心愈发柔软,仿佛被春水浸润透了,暖洋洋地晒着春日艳阳底下。再没有比这更快乐的事了。

如懿洗干净手:``总有一天,我也会那样,是不是?''

那老妇人并不理会,只道:``没想过活着出去?''

如懿犹疑片刻:``前辈在这儿待了多少年?''

那老妇人横她一眼:``前辈?我没有名字么?''

如懿见她性情古怪,忙恭恭敬敬道:``还请您老人家赐教。''

那老妇人掸了掸衣衫:``我是先帝的吉嫔。''她自嘲地一嗤:``可是我一辈子都没吉利过,还留着名位呢,就被关进了这里。''

如懿忙起身道:``晚辈乌拉那拉氏如懿,见过吉太嫔。''

``太嫔?''她黯然一笑,``是啊。先帝过世,我可不是成了太嫔?可惜啊,人家是寿康宫里颐养天年的太嫔,尊贵如天上的凤凰;我是关在这儿苦度年月的太嫔,贱如虫豸。''她忽然警醒,``你说你是乌拉那拉氏?那先帝的皇后乌拉那拉氏是你什么人?''

如懿道:``两位乌拉那拉氏皇后,都是我的姑母。''

``两位?''吉太嫔冷笑道,``一位就够厉害了。不过,再厉害也厉害不过当今太后啊,否则怎么会连你也落到冷宫里来了。不过我到这冷宫八九年了,从未听说有人走出去过,我倒很想看看,乌拉那拉氏家的女儿,能不能走得出去。''

如懿吃惊道:``您才到冷宫八九年,那您今年\ldots\ldots{}''

吉太嫔抚摸着自己的脸,哀伤道:``你以为我七老八十了?我被太后那老妖婆害得进这个鬼地方的那一年是二十六岁,如今也才三十五岁而已。''如懿惊得喉咙里发不出一点声音,只能以不可置信的目光瞪着她。吉太嫔恢复了方才的那种冷漠:``这里的日子,一天是当一年过的,熬不熬得住,就看你自己的了。''

如懿眼看着她出去,满心惊惶也终于化作了不安与忧愁:``惢心,对不住。让你和我一起来了这样的地方。''

惢心有些畏惧,却还镇定:``小主在哪里,奴婢也在哪里。''

如懿再也忍不住满心的伤痛,那种痛绵绵的伤痛,原本只是像虫蚁在慢慢地啃噬,初入冷宫时的种种惊惧之下,她原不觉得有多痛多难熬。可是仿佛是一个被麻木久了的人,此刻她骤然低头,才发觉自己的身体发肤已被这微小的吞噬蛀去了大半,那种震惊与惨痛,让她不忍去看,亦不忍去想。原来,她真的已经失去了那么多,地位、家族、荣耀以及她一直倚仗的他的信赖。都没有了。

可是,她却再没有办法。人在任何境地都有自己眼前的企求,譬如嘉嫔企求生下皇子;慧贵妃企求恩宠一如从前;而阿箬,企求圣眷不衰。她所企求的,只能是学着先活下来,仅仅是活下来。

而门外的凌云彻呢,在把冷宫嫔妃的尸体送去焚化场焚化后,他所愿的,是什么呢?他那样微红的英气的脸庞,疏朗的剑眉亦飞扬起来,站在冷宫和翠云馆偏僻的甬道上,仰首期盼着明媚的少女匆匆向自己奔来,那真是无趣而没有出头之日的冷宫侍卫最美好最乐意所见的场景。

那少女像一只轻盈的蝴蝶扑扇着冷宫前狭长而冷清的石板,虽然只是穿着宫女最寻常不过的青色衣装,她玉蕊琼英一般的娇美面容,依然如一抹最亮的艳色,无可阻挡地撞入了他眼帘。

云彻见她跑近,忙关切道:``嬿婉,跑慢一些,等下跑得累了还要再去当差,更累着自己了。''

嬿婉扶着弱不胜衣的细腰,微微喘着气道:``我就是要跑得快一些,才能多见你一会儿。''她的脸不知因为跑得太急还是羞怯,泛出珊瑚一样的娇润之色,``云彻哥哥,你是不是等了很久?''

云彻忙道:``没有。我只是稍微早一点来,这样就能看着你来。我和九宵说好了,他会替我一会儿。''

嬿婉稍稍放心,笑靥如花道:``那就好,我也和四执库的芬姑姑告了假,说肚子不舒服就出来了。''她看了看周遭,叹口气道:``平日里只有你和赵九霄看着,一定很辛苦吧?每天能做的事情就是守在门口看看天,或者进去替她们搬运尸体。云彻哥哥,为什么我们都那么命苦,没有出头之日?''

云彻道:``你还是想离开四执库?''

嬿婉黯然道:``虽然伺候的是皇上的衣物,但每天只和衣裳打交道,哪一天能够有个好前程。云彻哥哥,我才十四岁,我不想一辈子都在四执库受人呼喝。若是到个好一点的宫里伺候得宠的小主,我也能拉你离开这儿。那么我们\ldots\ldots{}''

云彻摇头道:``何必呢?得宠的小主宫里是非自然多。你不知道昨日进冷宫的那位,还是皇上的娴妃娘娘呢,还不是要在冷宫凄冷终身?何况是小小宫女,一个不小心被主子打死了也是活该,还不如四执库清清静静地安生。''

嬿婉撅起嘴,生了几分委屈之意:``是清静,是安生,可要是过了二十五岁还留在那里,我就要被送出宫了。我虽然是正黄旗包衣出身,但若不是几年前我阿玛犯了事丢了官职,家里门楣虽然低些,也好歹是个格格。可如今我不过是包衣奴才家送进宫的宫女。如果我没有个好去处,没有个好主子替我指婚,那我和你\ldots\ldots 我和你\ldots\ldots{}''她害羞得说不下去,只看着他的眼睛问:``云彻哥哥,你的心意没有变过吧?''

云彻恳切道:``当然没有。虽然我比你早入宫三年,又年长你六岁,但能遇到家乡故知已经很不容易,我和你又\ldots\ldots 情投意合,我的心意绝不会改变。''

嬿婉高兴起来,甜美的笑意再度绽放在唇角:``那就好。昨日是嘉嫔、玫嫔和慎常在行册封礼的日子,过几天内务府马上要挑选宫女去伺候她们,如果我能去伺候嘉嫔娘娘或是慎常在就好了,如今宫中最得宠的就是她们呢。''她按了按袖口:``我已经存了一小笔银子了,到时候只要买通芬姑姑,她愿意荐我去就好了。''她为难地看一眼云彻:``只是我怕银子还不够\ldots\ldots{}''

云彻为难地皱了皱眉,还是道:``你别急,我还有点俸例,再不行的话,我会想想别的办法。''

嬿婉高兴地点点头,眼中闪过一丝倔强的坚韧:``云彻哥哥,宫中我没有别的人,只能依靠你了。''她伸出双手,露出手指上森森的新旧伤痕,凄苦道:``云彻哥哥,我每天都不断地熨衣裳熏衣裳,已经两年了。管事的姑姑们只要一个不高兴,就可以拿滚烫的铁熨子朝我扔过来,拿炭灰泼我。我真的不想一辈子都做一个四执库的宫女,也不想你一辈子都困在冷宫当差。我知道的,你一直想做一个堂堂正正的神武门侍卫,甚至在皇上的御前当差。你放心,只要我们抓住机会,一定不会屈居人下的。''

云彻点点头,小心翼翼地替她呵着手道:``比起我在冷宫这里空有抱负,浪费年华,我更心疼你被人欺凌。你放心,我一定会想办法的。''

嬿婉被他小心地捧着手,心中温暖如绵,好像一万丈的阳光一起倾落,也比不上此刻的温暖和煦。她摸着左手手指上一个色泽黯淡的红宝石戒指,那是红宝石粉研了末做成的,原不值什么钱,却是凌云彻送给她的一片心意。他们原是这紫禁城中贫寒的一对,能有这份心意,已经足够温暖。她柔声道:``有时候再苦再累,看着你送我的这个戒指,就觉得心里舒畅多了。''

云彻的脸微微发红,静了片刻道:``嬿婉,我知道自己没什么银子,只能送你宝石粉的戒指。但我有最好的,一定都会给你,你相信我。''

嬿婉满脸红晕,低下头吻了吻云彻的手指,害羞地回头跑走了。

云彻在嬿婉离开后许久,目光再度触及冷宫深闭而斑驳的大门。他逐渐明白,自己愿意帮助冷宫中那个奇怪而倔强的女人,多半是因为她的脸和美好如菡萏的嬿婉,实在是有三分相似。这样想着,他的一颗心愈发柔软,仿佛被春水浸润透了,暖洋洋地晒着春日艳阳底下。再没有比这更快乐的事了。

\hypertarget{ux7b2cux5341ux4e09ux7ae0-ux7a7aux8c37ux4e0b}{%
\chapter{第十三章
空谷(下)}\label{ux7b2cux5341ux4e09ux7ae0-ux7a7aux8c37ux4e0b}}

云彻回到冷宫门口,往进门的门槛上一靠,有点犯难。方才他回自己住的侍卫庑房里,趁侍卫头领李金柱在睡午觉,翻了翻衣箱底下的俸例荷包,里面不过才七八两碎银子。这点银子,实在是帮不上嬿婉什么忙的。他放好了荷包正要起身,只见李金柱打了个哈欠慢腾腾爬起来道:``小凌,照规矩,该交钱了。''

冷宫的侍卫不过四个人并一个头领,他和赵九宵算是一班,另两个汉军旗出身的张宝铁和包圆算一班,虽然如此,也是要轮值的。张宝铁和包圆交给李金柱的例钱多一些,平时又肯花点钱请他喝酒吃菜,往往便休息得多,不用干什么差事。凌云彻和赵九宵出身包衣奴才,家里贫苦,还要送些钱回去,日子紧巴巴的,孝敬得少了,少不得什么苦活累活都得他们干了。譬如上次去抬尸首,张宝铁和包圆是永远不必干这等又累又脏的活儿的。

云彻想着还要用钱,少不得咬了咬牙,赔笑道:``李头领,我\ldots\ldots 我家里\ldots\ldots{}''

``老规矩,交不出钱就干活儿。接下来守夜都是你的差事。''李金柱爽快地摆摆手,笑道,``知道你和别人不一样,有个相好儿在宫里想着以后要成家。行,存着点就存着点吧。就你和九宵那小子苦哈哈的。''

云彻感激万分地点点头,出去当差了。''\textgreater\_\textless``"

九宵推一推他:``发什么呆?''

云彻怔怔的:``我在想,有没有什么办法弄到一点钱?''

九宵愣了愣,哈哈笑起来:``想钱想疯了吧?冷宫的侍卫是所有侍卫里最穷的,哪里能去弄钱。''

云彻呆呆地望着碧蓝的天空,说不出话来。

九宵摇了摇头道:``别想了。明晚包圆招呼了我们陪李头儿喝酒,他出钱,我们哥儿几个作陪,怎么样?\textless{}''

如懿在夜半时分醒来,隐隐听到角门外幽怨而悲切的哭声,她在最初的畏惧之后分辨片刻,立刻就听出了是海兰的声音。冷宫的侧边有个角门,离她的屋子最近,她悄悄起身靠近,透过门缝望出去,果然见到一身幽蓝暗花素锦袍的海兰。

如懿情急地叩了叩门,低声道:``海兰,海兰。''

海兰从呜咽中探起头来,喜出望外道:``姐姐,姐姐是你么?''

如懿急道:``都夜深了,你们怎么来这里?''

海兰稍稍犹豫:``姐姐,我担心你。所以来看看你。''

如懿借着角门边宫灯微弱的光线,敏锐地发现她脸颊边深红色的红肿,分明是五个指印的模样。她立时紧张起来:``海兰,是不是有人欺负你?''

叶心在近旁放风,低声催促道:``小主,好容易偷溜过来一次,有什么话赶紧说吧?别被人发现了。''

海兰忙止了泪道:``我听人说冷宫苦寒,所以特意包了几件衣裳来给姐姐。''她望着高高的墙头,用旁边的竿子将包袱一挑,扔了进来:``姐姐若缺什么,我会常常送来。''

夜风透过薄薄的衣衫是刺骨的凉。如懿的口吻并不温和:``你以后不许再来这里犯险。还有,告诉我,你的脸怎么回事?''

海兰还未开口,叶心已经忍不住道:``今早我们小主从延禧宫往长春宫去请安,谁知道在西长街上碰到了慎常在,也不知道她发什么疯,看见我们小主低着头就说小主一脸晦气犯她的冲,二话不说伸手就打。''

如懿道:``没有告诉皇后娘娘么?''

叶心气道:``正好遇上皇上,告诉皇上了。谁知道皇上只问慎常在手疼不疼,要不要请太医来上药,根本不过问我们小主,真真是气死奴婢了。也不知道慎常在是怎么了,夜夜侍寝这么承宠,火气还这样大!''

如懿隐隐觉得不对:``如叶心所说,她昨夜刚侍寝,那么那个时间刚离开养心殿,应该很高兴才对。怎么会一早见你就这么大火气?''

海兰落泪道:``我本就是个人人可欺负的。她恃宠而骄,也是寻常。''

如懿想想也是:``从前你心里有了委屈,总喜欢这样来对我说一说。''她心下酸楚:``可是海兰,眼下我不能再宽慰你护着你了,你要自己想办法保护好你自己,不要再受委屈。而且冷宫这样的地方,若是被人发现你偷偷前来,连你也会被连累的。''

她话音未落,忽然听到有人喝道:``是谁在那里?''

陡然间一个声音响起,叶心慌得忙护住海兰,却发现那人正从前面过来,根本无路可退。如懿紧张得一颗心被高高揪起,她反正已经是落在这里的人了,还有什么可怕,倒是海兰,要是被自己连累也来了这里,可怎生是好?

如懿隔着角门的门缝望去,却见正是白天来搬尸身的侍卫之一,便情急道:``侍卫大哥,你千万别声张。她们\ldots\ldots 她们只是来看我的。''

凌云彻提着灯笼打开门锁一看,却见是如懿缩在门边,他狐疑道:``你都被贬进冷宫了,怎么还有人来看你?''

如懿乍然见门打开,海兰站在门外,激动得几乎落下泪来,她指了指地上的包袱道:``这是延禧宫的海贵人,我和她曾经住在一起。她是怕我在冷宫受凉,所以特意来看看。她\ldots\ldots 她不是有心闯到这里来的。''如懿见他衣着寒素,灵机一动,拔下头上的一支银簪交到凌云彻手里:``求求你,千万别声张。千万别!''

凌云彻见如懿一副哀求的凄惶神色,仿佛是在溪边饮水时突然被猛兽惊起的鹿,惶惶不安,而这种不安却并非为了自己,更多的是为了眼前另一个人。他不觉为自己的这个比喻觉得好笑,原来自己竟然是那只猛兽。想到此节,他便有些心软,更兼看到那支银簪,心底更是一动,便硬声道:``给我这支银簪做什么,一拿出去人家还以为我是偷的,还不如银子方便呢。''

如懿心中一动,已然明白眼前这个人不过是贪财罢了。她眉心一松,唇角便有了一点笑意:``那你稍等。''她安慰地拍拍海兰的手,从袖口取出一锭银子交到他手中:``这里是十两,如果你愿意绝口不提今日之事并且护送海贵人出了这里的甬道,我便再给你十两。''

凌云彻眼中微微发光,顿时心念如电:``如果海贵人以后还要给小主你传递什么东西,实在不必这么冒险了,只要交给我转交就是了。至于我这么帮忙\ldots\ldots{}''

他才要说下去,只听那头庑房里有人探出头来唤道:``小凌,你撒泡尿怎么那么久,等着你喝酒呢。''

他忙回头道:``好了好了,就来!''

如懿略略含了几分轻蔑:``你很爱财?''

凌云彻不以为辱:``有贪念的人才肯好好做事。''

如懿松口气:``那你略等,看护好海贵人。''她转身回房中取出五十两银子交到凌云彻手中:``这点银两,够你好好办事了吧?''凌云彻大喜过望,一双眼灼灼发亮,伸手就要去拿,如懿一缩手道:``但你总要告诉我,你叫什么,我才好托付你办事。''

凌云彻倒也坦然:``我是冷宫的侍卫,凌云彻。''

如懿淡淡一笑:``这个名字倒有几分气势。''凌云彻接过银子握在手心,那种冰凉的坚硬给人踏实的感觉,他只觉得心头大石瞬间被移开了大半,连连答应了``是'',又道:``海贵人往后哪怕要过来,提前派个人跟我招呼一声就是了。只是别常来,也别白天来,太点眼了。''他向四周张望道:``赶紧走吧,等下有人出来就不好了。''

如懿看着海兰依依不舍的样子,越加觉得凄然,心疼道:``好好照顾自己。''

海兰贴在她身边轻声道:``姐姐,日后我不能常来,每隔十天若天气好的话,我会在御花园里放起一只蝴蝶风筝,只要你看见,就算我们彼此平安了。''

如懿点头道:``快去快去,无事不要再来。''

海兰被叶心牵着,一步三回头地走了。如懿听着微微松了一口气,将海兰送来的衣裳包袱紧紧抱在胸前,倚靠在墙壁上,无力地坐了下来。风声依旧呼呼的,如泣如诉,仿佛是谁在幽幽地呜咽着。这或许,就是她要习惯的人生了。

冷宫里的日子,过得缓慢而悠长。有时候几乎连她自己都忘记了,她还活在这个地方,一天天过着重复的日子。阴雨的日子里,所有的人像虫豸一样蜷缩在自己的世界里,苟延残喘。天气晴好的日子里,她会看到一个个像幽灵一样冒出来的前朝女人们,干瘪的,枯燥的,疯癫的,安静的,活在自己的世界里的女人。一开始她也会害怕,害怕有人会冲上来抱住她把她当做是接她们出冷宫的先帝,或者在太阳底下袒胸露乳晒着身上虱子的女人。但她渐渐习惯,好像周围的人把冷漠和无动于衷都传染给了她,让她习惯了忍耐、默然、冷眼旁观。就好像她一样习惯着有时候会馊腐的饭菜和经常潮湿晒不干的衣裳和被铺,照样大口大口地吞咽,照样合目而眠。

不为别的,只是她还想活着,活下去。

只是这里实在是太阴冷了,阴冷得几乎能掐出水来,即便她觉得自己渐渐活得像长在墙角的一株霉绿色的青苔,她还是在半年后觉得有些异常,有一种疼痛开始缠绕上她的身体,那就是风湿。虽然海兰常常托凌云彻送来一些治疗风湿的膏药,但在整日的阴冷潮湿之下,这些御药房上好的膏药,也成了杯水车薪。

她无声地忍住疼痛,和惢心缝制着越来越多的护膝和护臂,不仅给自己,也给吉太嫔。这里的每一个女人,都得着这样的病。偶尔,她会抬头望向天空,期待着十天一次的蝴蝶风筝高高飞起。那是海兰在提醒着她,时间的流逝和彼此的平安。当然,偶然凌云彻还是会替她们传递些必需的衣物和所用,因为如懿赏赐给他的银两,足以让嬿婉实现愿望。虽然钱不如预期那么多,不能让她去最得宠的嫔妃宫里,但嬿婉至少离开了四执库,不用再终日和衣裳打交道,受着姑姑的责骂,而是换去了阿哥所伺候皇后的三公主。这虽然算不得最理想的去处,但比起四执库,已经算是一个很好的去处了。

等到秋风渐起的时候,冷宫的日子便越来越难熬了。到了那一日该放风筝的时候,是个阴天,风筝才刚飞起,便又落下了。

如懿心中隐隐不安起来,正盘算着让凌云彻去看一看,才发觉这一日值守的却是另两个侍卫。她心中实在担忧,但又无法,只得忍耐着坐在廊下打着各种各样的络子,寻思着什么时候让凌云彻送出去换点钱来。

而此刻的海兰,心中也如暴风疾雨来临一般,心慌得不行,她的风筝才刚飞起,就被经过御花园的皇后和慎常在、慧贵妃看见。

这些日子以来,皇后的脸色一直不好看。她所亲生的二皇子永琏一直断断续续地病着,春日的时候抱在身边养了一阵已经见好,便即刻送回了阿哥所,但只要天气稍稍反复,便一直发作风寒,让人担心不已。这一层秋凉下来,永琏便再度虚弱了下去。

皇后刚从阿哥所过来,见到发病中的永琏面色紫绀,呼吸急促而微弱,简直如绞心一般,此刻看到一只五彩斑斓的蝴蝶高高飞起,想到自己的孩子竟不能起身放声大笑,尽兴玩一玩,简直气不打一处来。

慧贵妃察言观色,已然喝道:``谁在那里?''

海兰听得声音,心里没来由地一慌,慌慌张张收了风筝线跪下道:``参见皇后娘娘,慧贵妃娘娘。''

跟在皇后身后的慎常在轻蔑地看了她一眼,勉强行了个平礼。

慧贵妃很是不悦,一张芙蓉面如冻了严霜一般,呵斥道:``皇后娘娘担心二阿哥的病情心绪不佳,你竟然还在这里欢天喜地地放风筝。''

皇后一向柔和的面庞犀冷如冰,道:``简直全无心肝!''

慎常在娇声娇气地劝道:``皇后娘娘您别生气了。海贵人一向和冷宫里的乌拉那拉氏交好,不与其他嫔妃来往,性子孤僻是出了名的。她非要在这儿幸灾乐祸一下,放个风筝撒个欢儿,您就由着她去。小人得志,能多久呢?''

海兰慌忙俯下身,卑微地道:``皇后娘娘息怒,皇后娘娘息怒,臣妾并不知道二阿哥病重,只是在此放风筝嬉戏,并非幸灾乐祸!''

慧贵妃``哎呀''一声道:``枉费海贵人还在宫里呢,连外头的诰命夫人都来了好几拨儿入宫看望了,海贵人还真是漠不关心。''

皇后心下愈加恼怒,失了往日的温和沉着,又惊又怒:``本宫与皇上为了二阿哥担忧心烦,她却毫不关心,还在这儿这么兴高采烈,简直是其心可诛。''

慎常在趁着皇后怒气正盛,索性一脚踩在海兰的手上。嫔妃所穿的花盆底鞋的底都是寸许高的桐木,质地异常坚实,这一脚踩下去又格外用力。海兰只觉得钻心疼痛,眼泪都掉了下来。

慧贵妃摇头冷笑道:``此刻才掉眼泪,可知不是关心皇后娘娘的二阿哥了。怎是连牲畜都不如。''

皇后厌弃道:``你那么喜欢在御花园放风筝,就给本宫跪在这儿静心思过。''

``哎呀,这天气怕是要下雨了呢。''慎常在看一看天色,忽然笑道,``娘娘,对待这样不知进退的人,罚跪雨中,好好淋淋雨,脑袋就清醒了。''

海兰再忍不住,抬起头道:``阿箬,你也曾受过淋雨的责罚,己所不欲为何还要施于人?''

慎常在的满头珠翠在愈加阴沉的天光下摇曳出尖冷如利芒的暗光:``我就是这样才足够清醒,那么海贵人,个中滋味,你也该尝尝。''

皇后的语气冷漠而简短道:``那么,就跪在这儿,等着大雨冲刷干净你这样卑劣肮脏的心。''

皇后含怒离开,一脚踩在海兰已经受伤的手背上,整个人差点一滑,幸好被宫女们牢牢扶住了。

皇后嫌恶地看她一眼,道:``手放在不适宜的地方,还不收起来么?''

说罢,皇后便忧心忡忡离去。慎常在和慧贵妃一左一右扶着皇后的手臂前行。慎常在赔笑道:``皇后娘娘切勿生气,小孩子风寒是常有的事,宫中有那么多名医在,请宽心就是。''

皇后担忧不已:``可是太医说永琏的风寒反复发作,已经转成肺热,常常呼吸困难,一不小心就会致命,实在令人担心\ldots\ldots{}''

海兰跪在那里,叶心慌忙去看她的手,手背上已经被坚实的桐木花盆底踩出深紫泛红的两个血印子。海兰痛得死死咬住自己的唇,极力忍耐着,不让屈辱的眼泪落下来。她看着阴翳的云层越来越密,终于积聚成一场罕见的瓢泼秋雨,将自己单薄的身体和着秋日里飘零的残叶一同席卷其中,成为茫茫大雨中漂浮的一点零丁秋萍。

夜来风雨大作,海兰浑身发着高热,再耐不住委屈,撑着伞独自从宫中跑出,奔向冷宫。风雨时节,连侍卫们都躲在了庑房不肯出来,海兰拍响角门,终于惊动了住在近旁的如懿。她门缝里望见如懿撑着伞瑟瑟守在门边,不由得热泪潸然,她哭着诉说了今日的种种屈辱。

皇后、慧贵妃、慎常在,这三个名字,几乎是立刻勾起了如懿心底血肉模糊的沉痛。她咬碎了银牙,恨恨道:``海兰,害我的人总逃不脱是她们三个。如今,可能连你也会被她们践踏至死啊。''

海兰呜咽道:``姐姐,这宫里好冷,可是我只有一个人,连你也不在身边。''

如懿的心伤再度被她勾起,伸手按在破败潮湿的角门上:``海兰,我在这里,每一天都好冷,好像永远没有阳光一样。就像此时此刻,我很想握一握你的手互相温暖,可是却隔着这扇门不能碰到你。''她的声音变得坚定如磐石:``海兰,如果你不想冷死,就好好抱紧自己。不要像我一样,除了恨什么也做不了,像我当初一般除了隐忍便不懂得狠命反击。海兰,不要落到我这样的地步,千万不要!''

海兰举起受伤的手背:``可是姐姐,我怕我的力量不够,不能保护自己。任何人都能践踏我,甚至嫌弃我的存在。''

如懿的声音在呼啸的风雨中听来格外冷硬:``海兰,如果别人嫌弃你,践踏你,你就一定要活得更好。''

海兰的哭泣伤心而无助:``姐姐,可是我知道你活得不好,一点也不好。我也活得一点都不好,怎么办?我要怎么办才能帮你,帮到我自己。''

如懿的脸上分不清是雨水还是泪水,但声音却沉稳而没有一刻迟疑:``海兰,我已经是没有办法的人了,但是你还可以。你活得好一点,或者,我也可以活得好一点。恰如我此刻卑微的祈求,至少有一个太医,可以来治一治我日渐严重的风湿。海兰,靠自己,去争取好一点的生活。''

海兰极力想拭净脸上的泪,却发现她的泪和雨水早已混杂在一起,浇湿了她。她昏昏沉沉的,拖着沉重的双腿,走在茫茫雨帘之中。暴雨如巨大的绳索一下一下用力鞭打着大地,用溅起的硬如石卵的水珠再次暴打不已。

她身上滚烫滚烫的,却觉得自己成了薄薄的一片纸,任由雨水冲淋,除了深寒,还是觉得深寒。紫禁城的秋水这样冰冷,冲刷直下,将无数落叶残花,一同卷落沟渠之中,不知飘零何处。她忽然想,如果自己就此死去,这世间便只有如懿一人会替她伤心吧。那么如懿,便连她这个最后的温暖也失去了。她将如懿的愿望在心中反复掂量。良久,她才恍然发现,原来如懿的愿望,便是她自己的愿望。

曾经很多年前,她能依靠的只有如懿一人。那么今日,她也应该让自己稍稍坚强,变成如懿可以倚靠的后盾。

这样的念头最后在她脑中划过时,她已然走回了延禧宫的门外。叶心和绿痕打着伞守在门边,见她痴痴惘惘地回来,脸上终于有了一点人色,她忙迎上去,带了哭腔道:``小主您白日里淋了好几个时辰的雨发了高热,怎么此刻还要淋雨呢?您的伞呢?小主您说话啊,别吓奴婢啊小主!''

海兰听着叶心的声音在耳边喧哗,再忍不住,身子向后一仰,晕倒在滂沱大雨之中。

\hypertarget{ux7b2cux5341ux56dbux7ae0-ux65e7ux7231}{%
\chapter{第十四章 旧爱}\label{ux7b2cux5341ux56dbux7ae0-ux65e7ux7231}}

海兰的高热是在三天后退去的。她醒来的时候,一缕明媚的秋阳恍如淡淡的金色膏腴从镂空的长窗中斜斜照进,阳光隔着淡烟流水般的喜鹊登梅绣纹轻罗幔缓缓流淌,空气中沉郁的紫檀气味若即若离。

她怔怔地坐在床上,看着窗外的花竹葱茏,阳光温暖,也不过就是一道被凝固了的荒凉寡淡的影子,宫苑蒙尘玉人落灰。延禧宫,真的是空置了太久太久\ldots\ldots{}

叶心端了药进来,见她醒了,喜得热泪盈眶:``小主终于醒了。''

海兰微张着干裂的唇:``这几日辛苦你了,有谁来看过我么?''

叶心稍稍为难,还是说:``纯嫔娘娘和秀答应还有婉答应来看过您。不过秀答应和婉答应只在窗外望了望,只有纯嫔娘娘带着大阿哥送了点东西来,还在您床头坐了会儿。''

海兰微微一笑:``这宫里,也只有纯嫔有心了。只不过,她也是个可怜见儿的罢了。''她想一想,挣扎着坐起身来,抚了抚睡得凌乱的鬓发:``叶心,你去准备些回礼,我要亲自去向纯嫔娘娘致谢。再让绿痕进来替我梳妆,我病了这几天,一定很难看。''

叶心高兴地``哎''了一声答应,也有些意外:``小主平日最不在意打扮,今日怎么也讲究起来了呢。''

海兰似是回答,似是自叹:``一病如新生啊。''

她挽着纯嫔的手在阿哥所一起看着三阿哥的时候,精神已经好了许多。连纯嫔亦赞:``换了颜色衣裳,好好地打扮起来,也真是个美人儿呢,看着也精神了许多。''

海兰笑道:``是啊,老是恹恹的,从春到夏,如今入秋了,真觉得半点精神气儿也没有了。\textless{}''

三阿哥在乳母怀里抱着一个大佛手玩得十分起劲,笑得咯咯的。

纯嫔轻轻嘘了一声,向乳母道:``轻点儿笑,别让隔壁听见了刺心。''

海兰便问:``二阿哥还是老样子么?''

纯嫔苦笑道:``可不是?反反复复的,皇后娘娘的眼泪都快哭出一大缸了。早知道这样子,还不如像本宫的三阿哥一样笨笨的好,虽然不讨他皇阿玛喜欢些,可到底平平安安,壮壮实实。''

海兰低低道:``这话怎么说?''

纯嫔打发了乳母去一旁哄三阿哥抓布老虎玩儿,低声道:``本宫也是听大阿哥说了才知道的。原来自从二阿哥进了尚书房读书,皇后娘娘望子成龙,日夜查问功课,逼得十分紧,为的就是要在皇上面前拔尖出彩。本宫不知道从前如懿是怎么教孩子的,便告诉大阿哥说,千万不要争强好胜和二阿哥比,什么都是输给他才好的。否则呢,可不是自己吃亏了。''

海兰颔首道:``大阿哥听话,会明白娘娘的一片苦心的。''

纯嫔与海兰立在窗下,看着二阿哥房中的太医进进出出,忙作一团。几个宫女站在廊下翻晒着二阿哥的福寿枕被。纯嫔摇头道:``只是可怜了孩子,病着这么受罪。听说二阿哥的风寒转成了肺热,好几次一个不当心就差点缓不过气来了。''

海兰回头看了看玩得正高兴的三阿哥,道:``其实若没有二阿哥,皇上的眼睛里到底也有三阿哥些。纯嫔娘娘,嫔妾一直有个疑惑。当年三阿哥养在您身边时一直聪明伶俐,颇得皇上喜欢。怎么入宫后离了您进了阿哥所,就笨笨的不讨皇上的喜欢了呢。嫔妾随您来了几次,别的不说,嬷嬷们连认东西都不教,难怪三阿哥一味贪玩儿。又整天抱在手里不教好好走路,如今也三岁多了吧,三阿哥走路还是不稳当。''她的声音极低,像一枚绵绵的针,缓缓刺入:``这些嬷嬷乳母们的心是不是向着三阿哥和您,您都清楚么?''

纯嫔的面色渐渐灰败下去:``这念头本宫往常也不过一转,想想宫里的人总是仔细些也罢了。难道妹妹也这样想么?''

海兰低低道:``倒不敢想别的,只是同样是乳母,同样是皇后吩咐下来的,怎么待二阿哥就这么精细严格,待三阿哥就这么宠溺放任?如今小还罢了,若是长大,三阿哥可不止不受皇上器重了。一旦厌弃起来,先帝雍正爷不就把他的三阿哥弘时,咱们皇上的亲哥哥的名字从玉牒上删了,逐出宗谱了么?''

纯嫔向来胆小怕事,但听得儿子的事,哪里能不上心。她一辈子的恩宠也不过如是,唯一的指望全在这个儿子身上,这些话听在耳朵里,几乎是锥心一般,不觉暗暗握紧了双拳,望向一群乳母们的目光,带了芒刺般的怀疑,阴沉难辨。

纯嫔与海兰离开时,皇帝正好带了李玉从二阿哥房中出来。这一年秋来得早,庭院里黄叶落索,寂寥委地。碧澄澄的天空上偶尔有秋雁飞过,亦带了一丝悲鸣。阿哥所死气沉沉的氛围里,一袭紫罗飞花翩莺秀样秋衫的海兰挽着纯嫔盈盈步下台阶,海兰的紫罗色绣蝴蝶兰衣衫下素白色水纹绫波裥裙盈然如秋水,远远望去,便如一树一树浅紫粉白的桐花,清逸悠然。

``是你们俩?''皇帝眼前微微一亮,目光在海兰身上一转,``你难得穿得这样艳。''

海兰含着淡如轻云的笑:``让皇上见笑了。穿得艳点来阿哥所,希望阿哥们看了高兴。''

皇帝笑着虚扶她一把:``你有心了。平日素素的,偶尔鲜艳一点,让人眼前一亮。无论谁看见,都会喜欢的。''

纯嫔亦笑:``可不是,三阿哥可喜欢海贵人了。''

皇帝拍一拍额头,朗然笑道:``朕都忘了,你已经是贵人了。一个人住在延禧宫,可还惯么?''

海兰道:``也惯,也不惯。''

皇帝失笑:``怎么这样说话?''

海兰淡淡一笑:``从前有如懿姐姐就个伴儿,现在一个人,所以不惯。但一个人对着影子久了,也惯了。''

皇帝笑意渐渐淡薄下去,眼里似浮起一层薄影影的霜华,``哦''了一声,道:``朕乏了,你们也乏了,都跪安吧。''

皇帝径自离去,纯嫔嗔怪地看她一眼:``你忘了如懿是皇上下旨发落进冷宫的么?好容易皇上跟你说一回话,你怎么倒提起她惹皇上不高兴呢?''

海兰不以为意道:``皇上半年都没提起如懿姐姐了,既然皇上自己都忘了,嫔妾提一句又怎么了呢?''

纯嫔颇有哀其不争之态:``你呀,再这样下去,那点子恩宠便连本宫也不如了。本宫好歹还有个孩子,你却\ldots\ldots{}''

海兰正色道:``正因为娘娘有孩子,万事都要以孩子为重。''她略略苦笑,那笑意薄薄,似散落在地的凋零的花:``嫔妾这样的人,却是不打紧的。''

纯嫔望了望二阿哥房,听着三阿哥无忧无虑的笑声,神色更加凝重了。

海兰送过了纯嫔,便回到殿中和叶心修剪几枝早起刚送来的芦苇。那芦苇有着蓬松的花絮,远远看去,像浮在半空中的一堆轻雪。海兰道:``我吩咐你去内务府拿的杭绸料子拿了么?''

叶心为难道:``杭绸的料子难得,内务府扣着不放,说是给几位主位娘娘都还不够呢。''

海兰心下不豫,便道:``那也罢了,那些人一贯这样势利的。''

叶心开解道:``也说不准。奴婢去内务府时,听绣房的几位姑姑说,过几日便是重阳节了,皇上特意嘱咐了要给太后缝制一床万寿如意被,听说连上面钉了珍珠的万寿金丝图案床幅是先送去西藏请喇嘛大师开光诵经过的,再从西藏运了过来赶着要在重阳节前绣好图样送给太后的。她们都忙着这事呢,一时顾不上也是有的。''

海兰眉心一动,拨弄着手中轻如柳絮的芦苇:``皇上很着紧这件事么?''

叶心道:``当然了。听说皇上每隔两日便要去绣房亲自看一看,督促进度。''

海兰的笑意慢慢浮起在唇角,似一朵乍然怒放的蔷薇,在暗夜里闪出明艳的丽色。

这一日皇帝往内务府去查看给皇太后的寿辰贺礼,端的是一一精美,皇帝倒也满意,赞许道:``秦立,你做事还算用心。''

内务府总管太监秦立亲自陪在一旁,点头哈腰道:``送给皇太后的万寿如意被已经缝制好大半了,只是上头那凤凰的羽毛怎么配色都不亮,绣娘们都在犯难呢。''

皇帝随口道:``若要艳丽鲜亮,或者多配点颜色,或者捻了金丝,有什么难的?''

秦立一脸犯难:``都绣了给太后看了,太后说俗气,又斥了回来。奴才们啊,想得脑仁都快干了,还是没办法呀。''

皇帝叱道:``糊涂!这点分内的小事都办不好,难怪皇太后生气。给朕去瞧瞧,什么凤凰羽毛便这样难了。''

正说着,一行人已经转到了绣房长窗下。秦立正要通报,皇帝隔着疏朗镂空的长窗,见得绣娘们都围着一个女子,不觉有些好奇,挥了挥手示意不许出声,便站在窗外看着。

那女子柔声道:``太后寿年遐颐,看惯了繁花似锦,加之这被子是盖在身上之物,太过华丽了夜里看起来刺眼,她自然是不喜欢的,更觉俗气。''

有绣娘问道:``那您说怎么办呢?''

那女子的声音清婉如珠落:``这只凤凰气宇昂然,旁边又簇拥百花,颜色更不必太艳,只需用深紫色的蚕丝线八股绞了一股薄银线进去捻成为一股,这样色调柔和又不暗淡,在日光下不夺目,烛火下又微微有温柔光泽。

然后在每一羽凤凰羽毛的边缘用最细小的紫瑛珠和深绿的碧玺珠相间钉珠,紫瑛与深紫色蚕丝线深浅交错,碧玺有宁神之效,更被称为长寿石,颜色压得住百花丝线的繁丽。最后,在凤首处多用蜜蜡珠子,蜜蜡乃是西藏佛宗最喜欢的祈福之物,颜色也稳重大方。这样,想来太后也不会有异议了。''

她言毕,白如玉的手指轻扬起落,如翻飞花间的玉蝴蝶。皇帝看了半日,却见众人围着那女子,只觉得声音耳熟,却想不起是谁,也看不清她的容貌。

不过片刻,那女子便道:``我已经绣了一羽,你们看看,这样可以么?''

她话音未落,皇帝已经款步进来,笑道:``那么朕也可以看看?''

众人听得皇帝的声音,不觉吓了一跳,忙请安道:``皇上万福金安。''

皇帝笑道:``哪里来了这样心思灵敏的绣娘,朕也要看一看,她到底绣了什么新样子,大家都听她的?''

众人忙让了起身,那女子站在人群中间,因着众人都穿着深紫色的宫女服饰,她一身浅浅的月白色的湖绉夹衣,只以宝蓝夹银线纳绣疏疏几朵盛放时的昙花。一时在众人之间显得格外清新夺目,恰如暗簇簇的花瓣别无所奇,那花蕊倒是格外可人了。皇帝细瞧之下,那女子低着头看不清面容,但云鬓堆纵,犹若轻烟密雾,都用飞金巧珍珠带着银镶翠梅花钿儿,只在眉心垂落一点紫水晶穗串儿,如袅袅凌波上一枝芙蓉清曼,似乎是不经意打扮了,却处处有用心处。

皇帝心下的赞赏更多了一分:``朕听着你的声音很耳熟\ldots\ldots{}''

那女子仰起脸来,粉面微晕,含羞带怯:``臣妾卖弄,让皇上见笑了。''

皇帝不禁莞尔:``海兰,是你。''他看着她刚绣完的一尾凤凰羽,果然配色沉稳而不失温沉华美:``朕看了你绣的凤凰羽,不仅太后不会有异议,朕已经要击节赞叹了。你是怎么想出来的?''

海兰温柔的笑意如芙蕖新开:``臣妾想起太后时常握在手中的紫檀嵌碧玺佛珠,所以配了这个颜色。若不是太后最喜欢的,想必不会经常带在身边。''

``人人都看见,你却最有心。''皇帝眼中的温柔与赞许交织愈密,靠近些道,``从前怎么不知你有这样的心思?''

海兰妩然一笑:``心思藏在心里,轻易看不见。''

``那朕今日可巧,居然都见到了。''皇帝目光微微下移,笑道,``怎么身上绣着昙花?''

海兰盈盈道:``因是稍纵即逝的花,开完便谢,想留它长久些,便绣在了身上。''

皇帝颔首道:``如今是过了昙花的季节了。但你要喜欢,下个夏天的时候,朕让人多多地送到你宫里。''

海兰颇有些伤感,摇头道:``花开无人见,再多又有什么意思呢。''

皇帝挽过她的手向外去道:``明年昙花开时,朕一定陪着你。只是今日花开,朕又怎能辜负呢?''他低声细语,带了几分温柔亲昵:``朕记得初见你,是在王府的绣房,你也是这样一身月白色,清丽出尘\ldots\ldots{}''

海兰嫣然含笑,微微侧身,触碰到皇帝的手臂。

秦立看着皇帝携了海兰相笑而去,不觉急了,跟上道:``皇上\ldots\ldots{}''

李玉本跟在皇帝身后,见他如此,呵斥了一声道:``没眼力见儿的,没见皇上要陪海贵人么?不许跟着了。''

如此,待到重阳节夜宴时,海兰已成了与玫嫔和慧贵妃一般得宠的女子,看着满殿歌舞锦绣,对上皇帝含情的眼,露出沉着而清艳的笑容。

待到十月的时候,天气渐渐寒凉下来。延禧宫的桌上随意堆放着内务府送来的杭绸缎子,一匹匹垒在那里,色色花样都齐全。叶心笑吟吟道:``自从小主得宠,内务府巴结得不得了,从前咱们要也要不来的杭绸子,如今多得打赏下人都够了。''

海兰穿着一身全新的玉兰紫繁绣银菀花宫装,头上一色的碧玉珠花,垂落珠翠盈盈,好似一脉青翠的兰叶。她不以为意地笑笑,伸手随便撩拨着道:``这么好的东西,给下人岂不可惜了?''她低声道:``我让你送去冷宫的棉衣,都备下了么?''

叶心笑道:``小主又不放心了!昨晚是您自己选了厚厚的新棉花连夜缝制好的,瞧您眼圈都熬黑了,比做给两位小阿哥的福寿枕被还仔细呢。''

海兰有些不好意思地笑笑,扯着青瓷双耳瓶中的几枝芦花怔怔出神。忽然外头锦帘一闪,却是纯嫔进来了,笑道:``几日不见,妹妹大不相同了。当真是士别三日,当刮目相看了。''

海兰亲热地拉过纯嫔的手坐下道:``娘娘还不晓得嫔妾,不过皇上一时想起来了,半刻的兴致罢了。''

纯嫔微微掩饰着失落,笑得和婉:``跟本宫还这样客气么?这大半个月来,皇上对你,可都赶得上对玫嫔和慧贵妃了。玫嫔和慧贵妃是一向得宠的,而你呢,可是新贵直上啊,宫里多少人羡慕你呢。''

海兰轻轻一嗤:``哪里是新贵呢,不过是偶尔被想起的旧爱罢了。对了娘娘,怎么这个时候过来看嫔妾呢?''

纯嫔目光往四周一旋,海兰会意,便道:``茶点搁在这儿吧,我和纯嫔娘娘说话,你们都不必伺候了。''

众人忙退了出去,殿里安静得如积久的深潭一般。纯嫔见四下里无人,方沉下脸来,攥紧了绢子,恨得眼中含泪,道:``上回妹妹让本宫留意的,本宫一一去探听了。真不想,那帮人竟是这么听皇后的话,害本宫的三阿哥。表面上疼爱三阿哥,实际上什么也不教,什么也不帮着,皇上一旦问起,只说三阿哥贪吃贪睡,其他一无所知,教了认东西也不会。也怪本宫母子傻,皇上就这样疏远了本宫的儿子,自己竟也还蒙在鼓里。''纯嫔说着急切起来:``若到了妹妹所说皇子遭皇上离弃的地步,往后三阿哥还有什么指望!''

海兰惊道:``那日嫔妾也不过疑心罢了,不承想皇后竟真是如此,好歹她也是三阿哥的嫡母啊。''她见纯嫔恨得咬牙切齿,轻轻道:``那娘娘有没有想过法子,让皇后娘娘可以无暇顾及这么害三阿哥,让她也好好心疼心疼自己的儿子。''

纯嫔眼珠微微一动,看着盏中的清茶,缓声道:``本宫倒是想出一口恶气,只是\ldots\ldots{}''她的声音渐次低下去,无可奈何:``只是皇后一向小心,连二阿哥的一应穿戴所用,哪怕是被子枕头,都是亲自缝制的,何况是饮食起居,只怕是密不透风,无从\ldots\ldots{}''

海兰扶了扶发髻上微微摇曳的珠花,那碧玉的质地,硌在手心微微生凉,她淡淡一笑,起身取过一套福寿枕被:``送给三阿哥的一点心意,娘娘可喜欢么?''

纯嫔看了几眼,不觉诧异道:``这不是皇后给二阿哥做的那一套么?''

海兰的笑意隐秘而轻微:``娘娘也觉得很像么?''

纯嫔仔细翻了又翻,看了又看:``真的不是?几乎一模一样,可以乱真。''

海兰晓得温婉无害:``那日在阿哥所院子里看到的,所以试着做了一套。''

``妹妹的手真是好巧!''纯嫔疑惑道,``可是这套枕被的大小,对于三阿哥来说,实在太大了,怕不合用呢。''

海兰望着她的眼睛,几乎要望进她的心里去,推心置腹道:``那么娘娘觉得谁合适,就换上给谁吧。反正都是嫔妾给三阿哥的一番心意,旁人无需知道,也看不出来。''

纯嫔身子一颤,鼻尖微微沁出汗意:``有什么不同?''

``二阿哥得的是风寒肺热,怕凉。这被子和枕头都用杭绸缝制,盖着十分柔软,保护幼儿的体肤,但里头嫔妾用的不全是棉花,而是掺了芦苇絮。盖着看似厚,其实薄,二阿哥的风寒会更重些罢了。让皇后受点教训,以后不要再只疼自己的孩子,不顾别人的孩子。''海兰打量着纯嫔的神色,``娘娘若不敢,只当嫔妾这份心是白费了。嫔妾立刻拿去火堆里烧了,彼此干净。''

纯嫔惊疑的眼神渐渐有了几分动摇,更添了几分憎恨嫌恶,急切道:``只是教训?''

海兰的笑意笃定而沉稳,道:``是。否则咱们能如何?事情若是败了,针脚是嫔妾落的,赖不了别人。若是成功,娘娘也出了这口恶气,不是么?''

纯嫔抓着被子的手越来越紧,实在是万分舍不得从里头推开去,终于道:``好。明日就是十月初一,本宫会去看望三阿哥,把妹妹的心意神不知鬼不觉地带到。''

海兰微笑,切切地握住纯嫔的手,口吻镇定如常:``嫔妾病中只有娘娘一人来探望,也只有娘娘一人把嫔妾放在心上,当做妹妹看待。嫔妾自己是受惯人欺辱的,实在不想娘娘的孩子也是如此。从此,疼爱三阿哥的人,也算上妹妹一份吧。''

纯嫔深深震动,眼底泪水盈然:``皇上不疼爱三阿哥,好妹妹,一切便只有我们了。''

\hypertarget{ux7b2cux5341ux4e94ux7ae0-ux7aefux6167}{%
\chapter{第十五章 端慧}\label{ux7b2cux5341ux4e94ux7ae0-ux7aefux6167}}

因为太医一服服重药用下去,又轮流着悉心陪护,二阿哥的病稍稍见了起色。纯嫔亦在去了阿哥所之后回来道:``本宫趁着宫人们翻晒被子的时候悄悄换过了,按说没有人看见。只是这几日天气稍稍回暖,难道那被子太厚的、就不顶用了?''

海兰笑得稳笃,劝道:``娘娘凡事莫要着急,总有天气冷下来的时候啊。''

纯嫔已经尽力,便也只得静观其变,恨恨道:``总要让皇后也吃点亏才能出本宫心里这口恶气!''

这一夜皇帝宿在海兰宫里,身体的缠绵之后,只余下了彼此相依的力气。云锦帐帷流苏溢彩,零星地绣着暗红银线的吉祥图样,安静地逶迤于地,连帐外的红烛高照,亦只能映进一点微红而朦胧的光线。

皇帝疲倦而惬意地闭着眼睛,轻轻地吸一口气:``海兰,总觉得你这里连枕衾间都有别致香气,旁人那儿再寻不到。''

海兰一把乌黑青丝在皇帝臂间曲出柔和优美的弧度,轻笑道:``皇上去哪儿寻了?皇后?慧贵妃?还是玫嫔?''

皇帝默然叹口气:``皇后一心在永琏身上,昼夜不安。为着这个,朕也很久没留宿在皇后那里了。''

海兰道:``皇后娘娘不是一直求皇上将二阿哥挪到长春宫看治么?皇上不如答应了,两下也好方便些。\textless{}''

皇帝有些欷歔:``皇后是这么求朕。朕想着永琏的病虽好了些,但挪动间容易着凉,太医也觉得不妥,朕便罢了。何况皇后的性子那么好强,春天的时候永琏养在长春宫中,病稍有起色,皇后便催着他读书写字,好好的一个孩子,硬是被逼成那样。''皇帝论到几个皇子,不免有些感慨:``朕的三个儿子,二阿哥管教太严,三阿哥太过放纵,唯有大阿哥勤奋好学,只可惜亲娘去世得早,朕也未能十分顾及。''

海兰伏在皇帝手臂上,皮肉与汗水的黏腻让她有些不习惯,她不动声色地挪了挪,唇边却依旧笑靥如花,仿如小女儿撒娇:``大阿哥不是有养母抚养么?''

皇帝默然叹口气:``纯嫔虽然好,但总比不上\ldots\ldots{}''他下意识地停住口,深吸一口气,轻笑道:``好香。好像是你身上,好像又是帐帷间,到底是什么香气?''

海兰心中微微一震,像是被谁的小手指轻轻挠了挠,隐隐有些明白。她便笑得恬婉,按了按皇帝颈下的软枕道:``是春天刚过的时候收集的荼靡,和菖蒲叶子放在一起搓碎了滚在丝绵里头,这种花枕香气虽淡却悠远留长,让被衾乃至床帐内都弥漫着荼靡的余芬,人在睡梦中都会被花气浸染,以至臣妾在梦中都梦见自己化身成了翩跹花丛中的蝴蝶。''

皇帝在她鼻上一刮,道:``枕里芳蕤薰绣被,今宵帏枕十分香。你心思那么细腻,分明是旧人,却总让朕觉得是新欢,一重又一重惊喜与陌生,好像你与从前都不同了。''

海兰拧着一缕青丝,痴痴地笑着,又有些幽幽:``但愿新欢别又成了旧人,被皇上抛诸脑后。''

``新欢久了,也是旧爱,怎能忘怀。''皇帝笑着搂过她,侧脸枕在玫瑰色的软枕上,轻嗅道,``告诉朕,是谁教你的这个?分明像是江南女儿才有的心思灵巧。''

海兰悄悄地瞥一眼皇帝,见他眉眼间都是沉醉的笑意,便大着胆子试探着道:``是如懿姐姐\ldots\ldots{}''她恍作失言,不再说下去,并以惊惶的神色来窥探皇帝神色的微变,然而皇帝只是转过身去,静静道:``许多事都不能如意\ldots\ldots 海兰,朕累了。''

海兰伸手抚摸着皇帝的肩胛,柔蜜蜜道:``臣妾知道,臣妾都明白。''

皇帝的声音是沉沉的倦意:``嘉嫔只惦记着生皇子,她不喜欢公主;慧贵妃也是一心想在朕身上要到一个孩子;纯嫔只想着孩子而很少念及朕;皇后呢,她的心思也全扑在了永琏身上。朕只有见到你,才觉得松泛一些。因为,你什么都不求。''

海兰从后面抱住他的肩,嘴唇贴在丝质的寝衣上,那种光滑,像女人的肌肤,柔而嫩。不像男人,再饱满的肌体,也总带着情欲的味道。

海兰的声音如在呢喃:``皇上怎么知道海兰什么也不求?''

皇帝已有了蒙眬的睡意,还是答道:``朕要进你的位分,你总是推辞;朕赏赐你珠宝首饰精致玩意儿,你也不过一笑;朕常来,你固然高兴,可是来得少些,你也从不埋怨。朕总觉得你和满宫里的女人们都不一样,你不求什么,或者你求的,朕给不了,甚至不知道\ldots\ldots{}''

说到最末几句,皇帝已经语意含糊。海兰伸手抚摸着他的手臂,想要试着习惯去依靠在他身上,却还是觉得陌生而迟疑。

哪怕是肌肤相亲的一刻,她也觉得,自己的灵魂离身体很远很远,好像只有这样冷眼看着,保持距离,她才是安全的。恰如皇帝所言,她有着与别的女人不同的淡泊,这种淡泊一如她自多年的失宠生涯所知的,帝王的情爱,男人的情爱,从不可靠。因为在你身边时,自然彼此欢悦;要离开,也是顷刻之间的事。这种亲密,既不长远,也非无可取代。

因为这一切的欢悦,在不同的女子身上,总有不同的索取与满足。

而今时今日所拥有的这一切宠爱,都比不上一直在她身边的那个人,那双手。只有那个人,才让她觉得可以依靠,可以安心呼吸,不必辛苦笑颜应对。

这一夜的梦冗长而琐碎,她辗转地梦见许多以前的事,在潜邸绣房劳作的自己,第一次承宠的自己,被冷落和漠视的自己以及此刻被旁人所羡慕的自己。

醒来时天色还乌沉沉的。她悄然起身披上外衣,想喝一盏茶缓解昨夜临睡前过度疲累带来的劳渴。床前的红烛曳着微明的光,烛泪累垂而下,注满了铜制的蟠花烛台,当真是像沾染了女人胭脂的眼泪。

她慢慢地喝下一盏微凉的茶,回首看着床上熟睡的男人,想想自己,大约一辈子也不会为眼前这个面孔俊美的男子流下伤心的胭脂红泪吧。她凝神想着,忍不住伸手抚摸皇帝的脸,平心而论,他的确是个清朗男子,如玉山上行,光彩照人,难怪宫中上至后妃,下至宫女,少有不对他倾心倾意者,便如冷宫中的如懿姐姐,亦是如此吧。只是连她自己也没想过,原以为会以不得宠的嫔妃的身份在深宫度过一生的她,也有这样学会婉转承欢讨他喜欢的时日呵。

正凝神间,忽然有凄厉的哭声剧烈地爆发出来。海兰一个恍惚,还以为是某种夜枭或是野猫凄绝的嘶吼,几乎能撕裂人的耳朵。

可那一声哭,恍如硬生生扯破了紫禁城夜深阑珊的安宁,一声又一声更惨烈的哭声,遥遥地传了过来。

皇帝有些迷茫地醒来,问她:``是什么声音?''

海兰也是一样迷茫,却是李玉在外头急促地敲起门扇。李玉一向是稳当的人,若非十万火急的要事,绝不会在这样的三更时分,以如此急惶而没有分寸的手势,敲响有皇帝留宿的嫔妃寝宫的大门。

海兰忙忙披上氅衣打开殿门,李玉脚下一软,几乎是爬到了皇帝跟前,哭着道:``皇上,皇上\ldots\ldots 出大事了\ldots\ldots{}''

皇帝警觉地坐起身:``外头的哭声是怎么回事?''

李玉伏在地上号啕道:``是阿哥所\ldots\ldots 是阿哥所\ldots\ldots{}''

皇帝有些畏惧地站起身,顿了一顿才下意识地冲到窗前,猛地推开窗望着阿哥所的方向。窗外有冷风凌厉贯入,皇帝不自觉地打了个寒噤。海兰忙抱过大氅替他披上:``皇上保重,别着了风寒。''

皇帝像是在哭泣似的抖动着肩膀,声音里尽是怀疑和不自信:``是不是\ldots\ldots 是三阿哥出了什么事?李玉,是三阿哥对不对?''

李玉跪在地上,痛哭失声:``皇上,您节哀。是二阿哥,二阿哥薨了。''

皇帝不可置信地转过脸来,一步一步跌跌撞撞地走着,几乎是脱力般坐倒在床边,喃喃地问:``怎么会是二阿哥?怎么会?''他像一头悲绝而走投无路的兽,仰天道:``永琏是朕的嫡子,朕的嫡子!朕是上天的儿子,上天是不会把朕的嫡子收走的!他才九岁,他以后要继承朕的帝裔,他\ldots\ldots{}''皇帝被喉中的哽咽呛到,大口喘息着说不出话来。

海兰忙倒了水递到皇帝唇边,替他抚着后背。李玉哭泣着连连磕头道:``皇上,您节哀、您节哀。皇后娘娘已经从长春宫赶过去了,您\ldots\ldots{}''

皇帝来不及拭落眼角的泪,已经怒吼道:``给朕更衣!朕不相信,朕不相信!''

海兰守在一旁,侧耳倾听着那哭声里的悲哀欲绝,脸上也陪皇帝一同露出哀戚的神色,连含在眼中的泪,也随着她的心意沉沉坠落。

可是唯有她知道,唯有她自己知道。那一刻,窃喜与欣慰如何同时蔓延到她的心头,紧紧攫住了她颤抖的灵魂。

乾隆三年,十月十二日巳时,二阿哥永琏卒,年九岁。帝后痛失爱子,伤心欲绝,追封为皇太子,谥曰端慧。

听到消息时,海兰正换好了素色衣衫并银质首饰,坐在暖阁里慢慢地叠着金银元宝和冥纸,闲闲道:``死后哀荣有什么用,不过是活着的人聊以安慰罢了。我却不信,玫嫔和怡嫔死去的孩子在地下见了二阿哥,还会称呼他一句`太子'?''

叶心在旁边帮衬着,悄声道:``小主叠了那么多冥纸,要去哪里烧啊?宫中可不许见这些不吉利的东西的。''

海兰微微翘着银镶碎玉护甲,慢条斯理道:``不是让你告诉如懿小主,我会送冥纸过去陪她一起化了么。''

叶心担忧道:``小主又要去冷宫?''

海兰看她一眼:``怎么了?''

叶心有些担心:``如今宫里是多事之秋\ldots\ldots 又在为端慧太子做法事超度,小主还是不要去比较好。''

海兰轻嗤一声,沉稳道:``我都不怕,你有什么可怕的?''

正说着话,却听暖阁的门豁然被推开,一身素青的纯嫔如同一个影子般迅疾地闪了进来,她一向平和的面孔上有着显而易见的惶惑,六神无主似的。海兰抬了抬脸示意叶心出去,也不起身相迎,只忙着手中的活计道:``如今宫中多事,纯嫔娘娘脸上的害怕惊惶,在嫔妾宫中也罢了,若是在外头被旁人看见,人家还以为是二阿哥的鬼魂追着您的脚跟吓着您了呢?''

纯嫔在她面前坐下,倒了盏茶急急喝下,按着心口道:``你还说这样的话!你知不知道二阿哥是怎么死的?他是在半夜时分呼吸滞住,活活闷死的。而他闷死的原因,是在他鼻中发现了一些芦花和棉絮。''

海兰摇了摇头,怜悯地叹息道:``真是太不小心了。二阿哥的肺热本来就容易缓不过气,这个季节又易起芦花,阿哥所靠近御花园那儿,哪阵风吹来了水塘边的芦苇花絮也不知道。还有那些棉絮,进进出出的宫人太医那么多,入了冬谁的衣裳上没棉絮取暖。这些伺候的宫人们那么不小心,真该全打发了出宫去。''

纯嫔抚着心口,慢慢沉静下来,盯着海兰道:``你应该比谁都清楚,离二阿哥口唇鼻息最近的芦花和棉絮出自哪里。''

海兰嗤地一笑,盈盈道:``当然是娘娘亲手偷天换日的那床福寿枕被啊。''

纯嫔一怔,重重搁下手里的茶碗,气吼吼道:``你现在便撇得一干二净了,那床枕被分明是你做的,看针脚就可以分辨出来,你还敢抵赖!''

海兰轻轻按了按腮边的脂粉,柔声细语道:``娘娘别着急啊,这会子您是替皇后娘娘来向嫔妾兴师问罪的么?针脚会说话么?会认人么?到底除了上回和娘娘一起去阿哥所之外,嫔妾没有再踏足过半步啊。''

纯嫔又气又急又害怕,手指颤颤指着她道:``你\ldots\ldots{}''

海兰温柔地伸出手,握住她发冷的手指轻柔折回掌心,笑道:``嫔妾和娘娘说笑罢了。当务之急娘娘还没想清楚是什么吗?''

纯嫔一愣:``什么?''

海兰收起笑意,一句一句语气稳妥道:``娘娘的当务之急是告诉皇上,阿哥所的嬷嬷和宫人们照顾不周,致使二阿哥早夭,所以请求将三阿哥留在自己身边抚养。娘娘可要知道,要是有人先回过神来打起了三阿哥的主意,您可是防不胜防了。''

纯嫔会意,立刻道:``对对对!本宫还要告诉皇上和皇后,要严惩那些伺候不周的奴才,希望让皇上不要留意到本宫。''

海兰笃定地笑道:``皇上当然不会留意到娘娘了。今日午时焚烧二阿哥的遗物,那套枕被是二阿哥日夜盖着的,也是皇后娘娘亲手缝制的心意,到时候随烈火化去,不是什么都清清静静了。而娘娘有三阿哥在身边亲自抚养,三阿哥来日出人头地,一定会感激娘娘今日为他所付出的一切苦心的。''

纯嫔大为安慰,松弛一笑,马上迟疑而警觉地看着她:``那你\ldots\ldots{}''

海兰恭恭敬敬道:``嫔妾的双手自然不比娘娘的干净。所以娘娘实在不必担心嫔妾会说出去什么,因为嫔妾告诉过娘娘,以后疼爱三阿哥的人,算上嫔妾一个。嫔妾也很希望能沾三阿哥的光,来日能安安稳稳,享享清福呢。''

纯嫔笑道:``若真有那一天,本宫必不负妹妹就是了。''

夜来时分,乌云蔽住明月清辉,连昏暗的星光亦不可见。因着端慧太子崩逝,宫中一律悬挂白色宫灯,连数量也比平日少了一半。紫禁城中除了昏沉的暗色便是凄风苦雨般的啼哭,连平日的金碧辉煌亦成了锈气沉沉的钝色。皇后早已哭昏了好几次,万事不能料理,幸而有皇太后一力主持,事无巨细亲自过问,无一不周到,无一不体面。如此一来,倒是让皇太后在后宫中的威望更高了许多。

这一夜嫔妃们轮流在殿中守丧,因着一切混乱,三阿哥也不独自留在阿哥所了,挪到了纯嫔身边和大阿哥做伴。三公主也暂时跟着慧贵妃起居在一处。嘉嫔怀着身孕不宜在此守丧,行了礼之后便也回宫歇息了。

海兰守在冷宫的角门外,凌云彻早已借口找赵九宵喝酒,哄了他躲了开去,由着海兰和如懿好好说话。海兰找了个背风的角落,慢慢地烧着冥纸,道:``姐姐,你听到宫里的哭声了么?好不好听?我可是从没听过这样好听的声音。''

如懿在里头慢慢化着元宝,火光照亮了她微微浮肿的脸庞,映得满脸红彤彤的:``你办得这样利落,哭声当然好听了。''

海兰嗤嗤地笑着:``好孩子啊,别怪姨娘们心狠,谁让你的额娘这么欺负人呢?有这样的额娘,想保你长命百岁,阎王爷也不肯啊。来,永琏,好孩子,去底下找你那两个未曾谋面的弟弟吧。他们等你呀,等得太久太久了,都寂寞得很哪。''她烧着手里的几个纸制人偶:``来,姨娘再给你烧几个伴儿,让你在地底下别太孤单了。''

如懿苍白的面孔被火光照亮,道:``那套枕被烧了吧?没有人察觉么?''

``没有。就算真有人发觉,姐姐在冷宫里,我一步也没踏进过阿哥所,谁也疑心不到咱们。也算纯嫔争气,我当时便想好了,这件事做得好,是成全了纯嫔和三阿哥的前程;做得败了,是纯嫔这个做额娘的不争气,咱们也没法子了。''

如懿轻轻一笑:``但凡额娘为了儿子,没有不尽心尽力的。''

海兰将一大把冥纸撒进火堆里,暗红色的火舌一舔一舔,贪婪地吞噬着,她慵懒地笑道:``幸好姐姐提点我,告诉我杭绸的空隙比一般的缎子大,也告诉我芦花混在丝绵里会慢慢飞出,永琏的病是最受不了这个的。''

如懿隔着门扇轻轻一笑:``你若不告诉我永琏的病情,我哪里能想到这个。''她将最后一把金银元宝撒落,看着纸灰如黑色的蝶肆意飞扬,自嘲地笑笑:``我是身在冷宫里的人了,坐井观天只能等死罢了。但是海兰,我绝不会让你成为第二个我的。''

海兰静了静神,眼底闪过一丝坚毅决绝之色:``姐姐,只要我想到法子,我一定会让你出来的。我绝不会让你一生一世都陷落在这里,永无出头之日。''

``我这辈子,都不敢做这样的梦了。海兰,我只希望你过得好些。''如懿恍惚地笑笑,轻轻叩动门扇,凑近了,``来,让我告诉你,皇上喜欢些什么,不喜欢些什么。''

海兰微微出神,有些黯然:``姐姐告诉我这些,是想用另一种方式陪在皇上身边,让皇上过得舒心愉悦么?''

如懿惘然地摇了摇头:``不。他已经不信我了\ldots\ldots 他\ldots\ldots{}''

她没有再说下去,因为她听见了急促的脚步声,是凌云彻急着跑过来道:``小主不宜久留,似乎有宫眷从漱芳斋那儿过来呢。''

海兰忙不迭起身:``姐姐,那我下回再来看你。你的风湿\ldots\ldots 我会记在心上的。只是太医院的太医,没一个敢来冷宫,妹妹也是无奈。''

如懿点头道:``你能常常送些御寒的衣物和治风湿的药物来,就很难得了。''

惢心本默默守在一旁,听到此节,不由得黯然叹了口气:``海贵人。内务府有个职位很低微的小太医,叫江与彬。别人若不肯来,你问一问\ldots\ldots 问一问他肯不肯?''

海兰喜道:``这人可靠么?''

惢心迟疑着道:``他若肯来便是可靠,否则奴婢也不能说什么了。''

海兰匆匆离去,如懿隔着门向凌云彻道:``把海贵人烧的纸钱清一清,别露了痕迹。''

海兰跑出了甬道,听见外头渐渐有人声靠近,慌不迭吹熄了手中的灯笼,绕到隐蔽之处。却听几个小宫女四处张望着,低声呼道:``三公主,三公主,你在哪里呀?''

一个女声怒气冲冲道:``本宫叫你们好好看着三公主,结果你们那么多人,偏偏连个小女孩都看不住,简直都是废物。''

一个宫女道:``慧贵妃娘娘息怒。方才三公主说守丧守得累了,想跑来御花园玩玩,结果一个转身,便不见了人影。奴才们该死。''

慧贵妃高昂的语调里含着压抑的怒气:``皇后娘娘将三公主托付给本宫是信任本宫,若是出了什么差池,皇后娘娘已经失去了端慧太子,哪里还受得住?还不快去寻了公主回来!''

海兰趁着人往东边去了,忙迅疾地转过身,消失在茫茫夜色之中。

宫人们正四下寻觅,忽然一个高兴起来,像得了凤凰似的:``公主,你怎么在这儿呢?''

三公主穿着替太子守丧的银色袍服,外头罩着碧青绣银丝牡丹小坎肩,手里正把玩着一片东西出神。慧贵妃循声而来,忙欢喜道:``公主,你怎么待在那儿,快到慧娘娘这儿来。''

三公主低头片刻,将手中的东西递到慧贵妃手中:``慧娘娘,您快瞧瞧,这是什么好玩意儿。''

慧贵妃接过,借着羊角灯笼的光火一看,却是一个烧了一半的纸制人偶,画着五颜六色的花样,想是没烧完就吹了过来,难怪三公主瞧个不住。慧贵妃心下一阵疑惑,知道这东西是烧给地底下的人用的,便问身边的双喜道:``双喜,宫里是不是安排了人在这儿烧冥纸冥器?''

双喜丈二和尚摸不着头脑:``没有哇。这里都快到冷宫了,谁会安排人在这儿烧啊。忌讳哪!''

慧贵妃想了想,取过绢子小心翼翼地包好了那半个人偶,哄着三公主笑道:``来,公主,慧娘娘那儿有新鲜的皮影戏玩意儿,比这个好玩多了,快跟慧娘娘回去吧。''

三公主毕竟小孩子心性,听了高兴便跟着去了。

慧贵妃将袖中的绢子摸了又摸,心下有了计较,只盼着皇后身体好些,再一一商量。只不过皇后痛失爱子,这一病,却缠绵了许久。

\hypertarget{ux7b2cux5341ux516dux7ae0-ux5b3fux5a49}{%
\chapter{第十六章 嬿婉}\label{ux7b2cux5341ux516dux7ae0-ux5b3fux5a49}}

次年正月的某一天里,海兰再度放起那只风筝,这一回,蝴蝶风筝旁已经飞起了另一只小小的童子风筝。

就在前一天,如懿听见宫中喜乐和鞭炮嚣响的声音,她知道,嘉嫔已经顺利诞下了皇四子。这个在乾隆四年正月十四诞下的孩子,成为皇帝登基四年后得到的第一个皇子,也是皇帝失去了嫡子永琏后得到的第一个皇子,几乎是弥补了他那痛失爱子的巨大痛苦和空落。皇帝喜不自胜,亲自为皇子取名为永珹,日日设宴,又赏赐启祥宫上下,连着皇子的生母嘉嫔也春风得意,恩宠不衰。

而长春宫的皇后,却沉浸在失却亲子的痛苦与打击之中,日复一日地病重下去。

四阿哥永珹出世后便被许养在生母嘉嫔身边。这是格外的恩宠与荣昭,落在外人眼中,既是嘉嫔与四阿哥盛宠与荣耀的象征,亦是在向嘉嫔的母族李朝昭告嘉嫔在后宫与皇帝心目中不可动摇的地位。四阿哥出生到满月的欢宴足足持续了一个月,连李朝也特地不远千里派来特使,向朝廷贡贺人参与特产,并且送来了嘉嫔素来爱吃的家乡小食,聊慰她思乡之情。

而与此同时,抚养着两位皇子的纯嫔亦被晋位为纯妃,一时间由默默无闻而至举足轻重,风头颇健。连皇帝亦在闲暇之余,除了逗留嘉嫔宫中之外,往纯妃的钟粹宫亦渐渐去得多了。皇帝为着端慧太子早逝,实在也不放心皇子公主在阿哥所抚养,加之纯妃与嘉嫔每每哭劝,舍不得母子分离,皇帝便也答应了。如此一来,从前热热闹闹的阿哥所也清净了下来,只是形同虚设罢了。阿哥所中除了最低等的洒扫宫人,其余的都分配去了各宫伺候。嬿婉便在此列,分到了纯妃宫中。纯妃又喜她眉目清俊,看着柔婉可人,便专门拨了她去伺候大阿哥茶水点心。

这一日纯妃与海兰在庭中闲坐,赏着冬日微微干枯的枝头用彩纸点缀的花朵,赞赏道:``还是妹妹有心,在枝头点缀些彩纸的花朵,看着也没那么冷清清了。''

海兰凝睇一眼,道:``纯妃姐姐有所不知,这个花本是要用彩绢裁剪了才最好看的。只是如今不能罢了。''

纯妃悄悄向外看了眼,点头道:``这也太糜费了,若是让皇后娘娘知道,又是一顿训诫。\textless{}''

海兰轻声笑了笑,扯着纯妃身上新做的一件玫瑰紫飞金妆缎狐肷氅衣道:``如今皇后娘娘之下便是慧贵妃和纯妃姐姐您了。您又有着两位皇子,地位不同寻常,穿得好些用得好些,旁人自然是奉承的,有谁敢说什么呢。''

纯妃笑着拍了拍她的手,顺势将手上一串玛瑙赤金九环镯推到了她手腕上,亲热道:``若没有妹妹劝本宫为了三阿哥冒险一次,本宫哪里有今日与三阿哥共聚天伦的欢喜,又哪里有封妃的好日子呢。''

海兰悄声笑道:``纯妃姐姐这也值得说,便是见外了。''

两人看着嬿婉陪着大阿哥和三阿哥与几个乳母在廊下嬉闹着玩耍。却见皇帝正好过来,笑着道:``朕走到哪里,都是钟粹宫最热闹,远远便听见笑闹声了,朕听着就觉得高兴。''

纯妃与海兰忙屈膝道:``皇上万福金安。''

皇帝虚扶了二人一把,笑道:``海兰,你也在。''

海兰笑盈盈望着皇帝,目中秋波流转:``皇上喜欢热闹,就不许臣妾也来羡慕一番热闹么?''

纯妃笑道:``海贵人这是羡慕臣妾有个孩子了,说来海贵人若是也能生个皇子便好了。皇上说是不是?''

皇帝的笑意中含着几分欷歔:``朕何尝不是这样想,孩子是越多越好。圣祖康熙爷子嗣繁盛,咱们皇室也能跟着兴旺起来。''

皇帝看着三阿哥跟着大阿哥玩得起劲,便道:``只是热闹是好的。三阿哥如今也四岁了,是该好好认些字,别一味只是贪玩,连带大阿哥也不好好读书了。''

纯妃听皇帝这句话分明是有几分不愉之情了,正要替儿子分辩几句,却见嬿婉盈盈施了一礼,道:``回皇上的话,大阿哥说,三阿哥刚回到纯妃娘娘身边,母子兄弟间难免疏离,所以下了学便陪着三阿哥玩耍,也增兄弟之情。而且三阿哥如今可乖巧呢,大阿哥在屋子里读书温课的时候,三阿哥都跟着身边听着,大阿哥还教三阿哥认字,真是兄友弟恭。''

皇帝喜道:``真的?三阿哥已能认字了么?''

大阿哥牵着三阿哥的手晃了晃,指着钟粹宫正殿内的匾额道:``三弟,那是什么字?''

三阿哥好奇地仰起头来,看了一会儿道:``温和。大哥,是温和。''

纯妃原当三阿哥一字不识,一颗心提得紧紧的,正暗怨大阿哥竟挑了那么难的几个字给儿子认,却不想匾额上``淑慎温和''四字,儿子却能认识两个,也不觉大松了一口气。

``从前大字不识,如今能认两个,已经是不错了。''皇帝含笑,伸手抚一抚大阿哥的脑袋,``好孩子,不愧是朕的大阿哥,能教养幼弟,用心向学。''

大阿哥忙跪下道:``皇阿玛明鉴,不是儿子用心,而是觉得三弟其实资质聪颖,只是以前阿哥所的嬷嬷乳母们太过宠爱才会认字识物太晚,所以想自己多教教三弟,以尽大哥的责任。''

纯妃十分欣慰,亦笑道:``大阿哥纯孝友爱,实在是诸位阿哥的表率。''

大阿哥牵过皇帝的手道:``不过皇阿玛,儿子近日读书有几处不明,可否请皇阿玛指教,教教儿子和三弟。''

皇帝大悦,带着两个儿子便往暖阁里去。他正要抬步,却见嬿婉一脸温柔恭顺,仿佛一朵欲绽未绽的小小迎春,娇嫩而羞怯,却带了一抹独占春光先机的小小得意。

皇帝不觉注目:``你是伺候纯妃的?怎么从前没见过。''

嬿婉的声音清澈如山间泉水,娓娓动人:``奴婢从前是在阿哥所伺候的,如今拨来了纯妃娘娘宫里。蒙娘娘不弃,让奴婢专责伺候大阿哥的茶水点心。''

皇帝见她言语得宜,便道:``朕看你挺机敏聪慧,用心伺候着大阿哥吧。''说罢,便带着两个阿哥入内了。

纯妃见皇帝如此欢喜,不觉大松了一口气,道:``阿弥陀佛,皇天保佑。皇上居然不嫌弃三阿哥了。''

海兰笑着宽慰道:``否极泰来。妹妹就说么,只要三阿哥养在亲额娘身边,那一定会好的。果然有姐姐和大阿哥调教着,三阿哥便讨皇上喜欢了。''

纯妃抚着心口道:``本宫也不承想大阿哥这般机敏,想着替三阿哥露这个脸。真是老天有眼了。''

海兰看了看守候在殿门外一身宫女装束却不失清艳容色的嬿婉,笑道:``纯妃姐姐要赏大阿哥,更要好好赏大阿哥身边这个宫女了。若没有她,皇上今儿还没那么高兴呢。''

纯妃一迭声笑道:``赏,自然要赏。可心,去把御膳房今日送来的糖蒸酥酪赏给这个宫女,叫\ldots\ldots{}''

嬿婉乖觉道:``回娘娘的话,奴婢名叫嬿婉。贱名能入娘娘的尊口召唤,是奴婢的荣幸。''

纯妃愈加眉开眼笑:``可心,便把糖蒸酥酪都赏了嬿婉吧。''

海兰见机忙道:``纯妃姐姐,趁着皇上高兴,您快进去吧,妹妹就先告退了。''

次日海兰往嘉嫔宫中看了四阿哥回来,正携了叶心过御花园,见新开的迎春星星点点闪着鹅黄的星光,掩映在葱茏绿枝之间,果然已经是春临世间了。海兰想着这一冬严寒,本该早些个请江与彬去冷宫给如懿医治风寒的,只是二阿哥早夭,四阿哥出生,宫中的事一桩连着一桩,几乎没有缓过来的余地。如今天气稍稍回暖,也该想办法召这个江与彬入延禧宫问一问,摸摸他的底细。

海兰正想得出神,却听得前头浮碧亭后有人语喁喁,其中一人之声十分熟悉,不觉站住了脚,示意叶心噤声。

一湾碧水如薄薄春绸无声蜿蜒过浮碧亭,潺涴而下。四下里花木日渐萌发出鹅黄翠绿,芳草青郁如茵。隔着丛丛佳木枝丫微叶的空隙,一抹明黄之色意外地撞入眼帘,皇帝只对着身前的青衣宫女道:``朕记得昨日在纯妃宫中见过你,怎么今日你又在御花园中撞进朕的眼睛里。''

那宫女有些怯生生地,道:``皇太后召唤大阿哥去慈宁宫,奴婢伺候完大阿哥送他去了尚书房,便往御花园走回钟粹宫,不是有心要打扰皇上的。''

皇帝笑着托了托她小巧圆润的下颌道:``朕有说过你打扰朕了么?春色撞入眼帘为欢悦欣然之情,朕看你,亦是如此。''

那宫女旋即明白,忙从皇帝的手指底下闪开,含羞带怯,道:``奴婢愚昧,不敢承受皇上如此夸奖。''

皇帝的微笑如拂面的春风,化开含苞的花蕾,催生一树树的花开艳灼:``你叫什么名字?''

``奴婢名叫嬿婉。''

``嬿婉极好,念来口舌生香。是哪个嬿婉?''他忽然眼眸一亮,带了几分调笑的意味,``南朝沈约的《丽人赋》中说,`亭亭似月,嬿婉如春。凝情待价,思尚衣巾'。可是从女旁的嬿婉?''

嬿婉眉目间带了薄薄的绯色,好像天边的云霞凝在她细巧的眉目间,依依不肯离去。她似乎有些畏惧,声音虽柔和,却有些克制的疏远,道:``皇上念的诗真好听,可惜奴婢不懂得。''

皇帝的眼里是蓬勃的笑意,他道:``你不必懂得,因为你便是那个嬿婉如春的丽人。你站在朕面前,便是全部的懂得与明白了。''

皇帝似想起什么,便问:``嬿婉,你姓什么?''

嬿婉似提到不悦之事,却不得不答:``奴婢出身汉军正黄旗包衣,母家姓魏。''

皇帝微微一笑,似是宽慰:``魏这个姓普通,像是委曲求全的鬼心眼儿。但是汉军正黄旗包衣,出身也不算很低。''

有难过的阴翳蔽住了她澄澈而清郁的眼:``虽然是汉军旗上三旗出身,父亲死得早,又没有争气的兄弟,实在不算什么好门第。''

皇帝的手似乎无心从她手背上抚过:``门第好不好,长辈留下的都不算,而是要看你自己能不能争气,争出一副好门第来。''

嬿婉眼中微微一亮,似乎明白。她眼中最初的回避与羞涩慢慢褪去,只剩下笑意盈盈,眉目濯濯,似是明月夜下的春柳依依,清妩动人。她娇怯怯道:``奴婢不过一个弱女子,可以么?''

皇帝一笑:``你要是个男子,那便难些。偏生你是个弱女子,那便简单了。''

嬿婉微微一怔,迷茫而清澈的眼波中似有无尽情思涌过,迷乱如浮絮。皇帝淡淡笑了笑:``其中的意思,你慢慢思量。朕便等着有一日,`欢娱在今夕,嬿婉及良时'。''

皇帝独自离去,唯余一袭青衣春衫的嬿婉,独自立在春风斜阳之中,凝思万千。

嬿婉走到冷宫前的甬道时,已觉得双腿酸软不堪,好像自己已经走了千里万里路,将这一生一世的力气都花在了来时的路上。凌云彻冷不丁见她到来,不觉喜不自禁,忙嘱咐了九宵几句,便赶上前来道:``嬿婉,你怎么来了?''

嬿婉勉强一笑,便道:``我正好没事,就过来看看你。''

云彻心中一暖,伸手握住她的手笑道:``可是想我了?''

嬿婉缩回手,往他身后看了一眼,低声道:``九宵大哥在呢。''

九宵看见二人都望着他,便伸手遮住眼睛,兜住耳朵,吐舌扮了个鬼脸,往远处去了。

云彻关切道:``你现在在纯妃娘娘身边伺候大阿哥,是不是很忙?我看你好些日子不来见我了。''

嬿婉急忙道:``忙\ldots\ldots 是很忙。''

云彻温柔的语调像轻轻流过手背的碧绿春水,带着酥酥的暖意:``大阿哥正在顽皮的年纪,你得学着给自己偷些懒,别太辛苦了。''那声音一向是温柔惯了的,她最受用,入耳也最安心。可是此时此刻,她听来却只觉得遥远而陌生,像浸浴在艳阳底下的人,一脚踩进了冷水里,那水色再如何映人心,也是让人着惊。她心底反反复复念着皇帝那一句:``你要是个男子,那便难些。偏生你是个弱女子,那便简单了''。

那便简单了,那便简单了。这句话不能不让她动摇,汉军旗包衣出身,虽比下五旗高贵些,可还是个包衣。且阿玛犯事丢官,弃下他们一门孤苦。罪臣之后,这是一生一世的禁锢,会随着她的血脉一代一代传延下去,挣脱不得。她看着眼前的云彻,心下更是难过。云彻,他何尝不也是这样卑微的身份,所以入宫多年,也只能是个看守冷宫的侍卫,没有出头之日。她伸手替他掸了掸肩头沾染的蛛网尘灰,心疼道:``只能在这里,没有别的办法么?''

云彻虽然无奈,却也宽慰她:``慢慢来,总会有机会的。''

嬿婉的手轻轻一抖,停在了他肩上:``你是男人,不怕等不到机会。而我到了二十五岁就要出宫,在这之前没有机会,便没有可能了。''

云彻有些糊涂:``什么机会?你在纯妃宫里不好么?''

嬿婉低下头,不敢看他的眼睛。唯觉得鬓边一只紫云绢蝴蝶的绢花,颤颤地在风里颤动着,恨不能张开翅膀立时飞起来。这样振翅飞起的机会,真是稍纵即逝吧,或许今生今世,都没有第二次了。她狠狠心,再狠狠心,终于道:``云彻哥哥,我们不要再见面了。''

云彻似乎被一个闷雷狠狠打在了头顶,嘴唇有些发颤:``你说什么?是不是纯妃娘娘不许底下的宫女和侍卫来往?''

嬿婉不敢看他,只是迅速地退开两步,盯着自己的鞋尖道:``云彻哥哥,我们不要再见面了。你是汉军旗包衣出身,我也是包衣出身,我们若是在一块儿,以后的孩子也不过是包衣,一辈子奴才的命,生生世世都脱不了。你就为自己的前程好好打算吧,别再理会我这个人了,就当不认识我便是了。''

她说完,便逃也似的走了。云彻愣在当地,几乎目瞪口呆,只觉得甬道里无穷无尽的穿堂风如呼啸的利剑,冰冷地贯穿了自己的身体,将血液的温热一分一分地,冷冷冻住。

嬿婉回到钟粹宫的时候,大阿哥已经下了学,正在四处找她,见了她进来便道:``嬿婉,我一向爱吃金针木耳馅的豆腐皮包子,怎么今天点心不是你准备的么?居然拿青菜蘑菇馅的应付我。''

嬿婉郁郁不乐,见大阿哥缠着,只得打起精神道:``好阿哥,今日就将就吃了吧,明日奴婢一定给您准备好金针木耳馅的豆腐皮包子,好么?''

大阿哥缠着嬿婉进了书房。海兰陪着纯妃在暖阁的窗下冷眼看着。

海兰轻声道:``这丫头这么晚才回来,不知上哪儿去动那些见不得人的心思了。''

纯妃含着压抑的怒气:``妹妹方才说的可都是真的?''

海兰秀丽的双眸轻轻扬起,清澈而澄明,蕴着十足十的关切:``纯妃姐姐觉得妹妹编得出这样的谎话么?妹妹想着,皇上如今常来姐姐这儿,怕是已经对那小丫头留上了心思,若再被那小丫头狐媚几下子,宫中可又要添新人了。纯妃姐姐您好不容易才有了今天的地位和荣宠,难道要被这狐媚子分去么?''

纯妃咬了咬唇,苦恼道:``可是皇上要喜欢她,本宫能有什么办法?再说皇后病着,嘉嫔才出月子不能伺候皇上,怡嫔也殁了,后宫里统共就只剩下了这么几个人,皇上要纳一个新人,咱们也没有办法呀。''

``就算皇上要纳新人,也不能出自姐姐宫里。纯妃姐姐您细想想,您已经有了两个皇子,若嬿婉得宠,旁人必定以为是姐姐举荐的。这本是无心事,落在有心人眼里便以为姐姐趁着皇后病重私下勾结,迷惑皇上,要捧高了三阿哥争宠。姐姐倒也罢了,那三阿哥不就成了众矢之的了么?''

纯妃大惊失色:``那怎么行?本宫自己不要紧,但不能害了自己的儿子!''

海兰乌黑的眼眸微微一转,道:``法子自然是有的,而且能彻底绝了皇上的心思。''

纯妃又惊又喜,笑纹里都是舒展的笑意:``妹妹真有把握?''

海兰笑着弹了弹指甲,低声道:``姐姐是第一天认识我么?''她附耳低语几句,纯妃喜上眉梢道:``可心,去传嬿婉过来。''

嬿婉即刻便过来了。她低眉顺眼地请了个安,显得格外恭敬。纯妃本来觉得她清秀可人,眉眼间隐隐有几分亲切,可此时看着她,即便是一身青碧的素色宫装,亦觉得她妖妖调调的,大不成个样子,不觉皱起精心描摹的春柳眉。海兰不动声色地碰了碰她的手肘,取过一枚橙子,用并刀慢慢切着。

纯妃扬了扬绢子,缓缓道:``嬿婉,你伺候大阿哥伺候得很好。本来本宫是想让你留着继续伺候大阿哥的,但今日钦天监过来替大阿哥算流年,本宫拿你的生辰八字和大阿哥的一合,发现不仅和大阿哥犯冲,和皇上也犯冲,这就不大好了。所以本宫思量来思量去,为了皇上和大阿哥,只好委屈你了。从今日起,你就去花房伺候花花草草吧。如此,也不会再有犯冲相克之事了。''

嬿婉本听纯妃夸奖,显是分外器重。想着日后若是在皇帝身边,想来纯妃也不会反对了,却不承想纯妃骤然说出这一篇话来,简直如五雷轰顶一般。那花房本在后宫最偏远之地,除了几个花匠便是宫人,事务繁重,想要出来亦不能了。没想到自己刚有转机的人生,竟然又如此被人摁到了底处,没有翻身的余地。

她听着纯妃口气虽然客气,但却决绝到底,求情必定是无用了。想来想去,只得磕头谢了恩道:``奴婢谢纯妃娘娘恩典。只是大阿哥一时还离不开奴婢,能不能请娘娘稍稍通融,容奴婢和大阿哥交代几日再去。''

海兰慢悠悠道:``既然命数相克,多留又有何益?赶紧去了,免得生出什么意外,那就不是去花房能了的了。''

嬿婉死死咬着嘴唇,忍住眼底泫然欲落的泪水和喉中的酸楚欲裂,磕了个头道:``奴婢遵命,奴婢即刻就去。''

她缓缓站起身,看见海兰将切好的橙子递到纯妃手中,笑脸盈盈:``姐姐尝尝。并刀如水破新橙,便是这种滋味了。''

嬿婉望着那被剖成八瓣的橙子,自己的腔子里几乎要沁出血来。她无望地想着,自己的人生,何尝不是如那只橙子,由着人肆意划破、剖开,半分由不得自己,也从来由不得自己。

\hypertarget{ux7b2cux5341ux4e03ux7ae0-ux76f8ux6170}{%
\chapter{第十七章 相慰}\label{ux7b2cux5341ux4e03ux7ae0-ux76f8ux6170}}

纯妃立时下了令遣她出去,嬿婉再委屈,也不敢在面上露出分毫来,只得赶紧收拾了东西去了。大阿哥见她要走,原也有些依恋,奈何嬿婉不过是个新来照顾他的宫女,虽然好,但身边总有更好的嬷嬷乳母在,他寄养在纯妃宫中,更不大敢出声,只得罢了。

海兰回到宫中,便也有些乏了,自在妆台前慢慢卸了首饰,换了青玉色暗纹梅花衬衣。那衬衣是云呢缎的料子,着身时光滑如少女的肌肤,且在烛光下,自有一种淡淡的烟罗华光,仿佛薄薄的云彩雾蒙蒙地贴上身来。她却格外喜欢袖口上玉白色缠绕了深青的梅花纹样,小小的一朵并小朵,是临水照花的情态,都用极细极细的金线勾勒了轮廓,有一种含蓄而隐约的华贵繁复之美,恰如她此刻的心思,丝丝缕缕地密密缝着,不漏一丝缝隙。

海兰托着腮,凝神望着镜中的自己,骤然也觉得心惊。从前温顺无争的一张面孔,如今也精心描摹起了脂粉,画的是皇帝最喜欢的杨柳细眉,只因他爱着江南的柳色新新,朝暮思念。腮上的胭脂施得极轻薄,先敷上白色的珍珠茉莉粉,再蘸上蔷薇花的胭脂,只为玫瑰色泽太艳,月季又单薄,只有月光下带露的红蔷薇拧了汁子才有这般淡朱的好颜色。胭脂之上还需再压一层薄薄的水粉霜,须得是粉红色的珍珠研磨成粉,才有这样的天然好气色。这胭脂也有个名字,是叫``嫩吴香'',是觅了唐朝的古方子做的,敷在脸上,浑然天成,仿佛吴地女子的轻婉娇媚,未见其人,先闻其香。

这样精致的描摹,自然得到皇帝的圣心常顾,亦是因为她从前实在不太打扮,一旦用起心来,才有这样的惊艳。可是从前的自己,却是铅华不御得天真的。

真的,才是多久的光景呢。如今不说旁人,连自己看着也是另一个人,另一副心肠了。

正凝神间,却从铜镜里瞧见叶心捧了热水进来,要伺候她盥洗。她有些心思恍惚,叶心便道:``小主今日心想事成,还有什么不高兴么?''

海兰摘下护甲将双手泡在热水里,道:``我有什么可心想事成的。''

叶心小心翼翼地替她按摩着手指:``小主不喜欢嬿婉在皇上面前那股子水蛇身段妖媚劲儿,借着纯妃娘娘的手三下五除二便把她料理得一干二净了,小主也可以安枕了。''

海兰秀丽的眉峰微微皱起:``怎么?连你也觉得嬿婉不容轻视么?\textless{}''

叶心仰起脸笑道:``奴婢就不信小主看不出来,除了那股子妖妖调调的娇媚劲儿不像,嬿婉那丫头的脸容,长得倒与冷宫里的如懿小主有两三分相似呢。''

海兰本拿着雪白的热毛巾擦手,听得这一句,将手里的毛巾``啪''地往水里一撂,溅起半尺高的水花来,扑了叶心一脸,她怒声道:``作死的丫头,嘴里越发没轻重了。如懿姐姐虽然在冷宫里,可她是什么身份,岂是你能拿着一个低贱宫女浑比的?下回再让我听见你说这样的话,仔细我立刻打发了你出延禧宫,再不许进来伺候!''

叶心伺候了海兰多年,忠心耿耿,深得海兰信任。海兰又是个极好性子的人,何曾见过她这样气恼的面孔。当下叶心也慌了神,狠狠打了自己两个嘴巴,肿着脸道:``小主别生气,为奴婢气坏了身子不值。都怪奴婢说话没轻重,以后再不敢了。''

海兰这才消了气道:``你永远要记得,不管如懿小主身在何处,从前待我最好的人是她,如今和以后待她最好的人就是我。你若要分出彼此来,就是你自己犯浑作死了!''

叶心吓得大气也不敢出,忙伺候着海兰铺床叠被一应齐整了,又点上了安息香道:``小主,时候不早,早些安置吧。''

海兰拿着犀角梳子慢慢地梳着头发,冷不丁问道:``叶心,你说皇上突然看上了嬿婉,会不会也是觉得嬿婉和姐姐有几分相像?''

叶心吃了方才那一惊,哪里还敢开口,只得诺诺应着,嘴里一味含糊着。海兰知道她是吓怕了,便也叹了口气道:``今儿是我的气性大了些,宫里那么多人和事,哪里有不添烦的。你伺候我这么多年,不要往心里去就是了。''

叶心吓了一跳,脸上虽热,心里头也热了起来,感激道:``小主别这样说,奴婢知道小主自从得宠之后,事情也多了,心里难免难受。''

海兰怅然道:``或许你说得对。我就是不喜欢皇上跟前有一个和姐姐长得相似的人。因为这样,皇上很可能时时惦记着姐姐,也会彻底忘了姐姐。''

叶心答应了``是'',再不敢多嘴。

海兰坐到床上,看着叶心放下了帐帷,便道:``明日皇上要过来用午膳,你早些叫我起来,我好亲自预备些拿手小菜。等午后皇上走了,你记得去太医院找一个叫江与彬的人,带他来见我。''

叶心答应着将帐帷平整垂好,又将地上海兰的绣花米珠软底鞋放得工工整整,方退到自己守夜的地方,躺下睡了。

这一夜睡得并不大安稳,海兰心里装了重重心事,只是辗转反侧。如懿亦犯了风湿,躺在床上浑身酸痛,四肢百骸如同被人强行灌入铅酸一般,被一点一点地腐蚀着。惢心虽然自幼操持身体强健,却也没好到哪里去,只坐在床边,借着一灯如豆的残光,用纱布裹了生姜挤出汁液,一点一点替如懿擦拭关节。

如懿忙扶住她道:``别蹲在那里了,等下仔细腿脚疼,又站不起来。''

惢心咬着牙关一笑:``奴婢熬得住。''

如懿看她的神情,似是隐忍,似是期盼,总有无限情思在眼底流转。她轻声问:``那个江与彬,你与他很熟么?''

惢心微微一怔,脸上带出些许温柔之色,一双眼睛如同被点亮了的烛火:``奴婢与他自幼相识,后来家乡饥荒,各自跑散了,奴婢入了王府,他凭着一点家传的医术入宫做了太医。奴婢其实与他在宫中遇见也是近几年的事情,只是想着,若是同乡也帮不上忙,那就没人肯来帮忙了。''

如懿道:``他的医术很好么?''

惢心微微一笑,继而叹息:``好有什么用?他在太医院中没有关系,没有家世,一向不受人重视,只是个最末流的小太医罢了,只能给宫女侍卫看看病。不过也好,若他都不能来,那就真的谁也不能来了。''

如懿站起身,又拿姜汁替她擦拭手腕和手肘关节,柔声道:``来是他的心意,不来也无需怪他。富贵之中难见真心,你若落得这种地步他还真心待你,此人才值得继续相交。否则,不见也罢。''

惢心道:``小主,奴婢自己来涂吧。您往外起身走一走,涂过姜汁的地方会继续发热才暖得过来。''

如懿走到院中,只见月光不甚分明,雾蒙蒙的似落着一层纱。她蓦然听见一声叹气,那声音便是外头来的,分明是个男人的声音。

如懿听得耳熟,不自觉便隔着疏疏的门缝往外望去,却见凌云彻满脸胡楂,意态萧索,举着把酒壶往嘴里一个劲儿地倒酒。她看了不免暗自摇头。进了冷宫这么久,这个男人也算是朝夕都见得到的难得的正常人了。虽然贪财些,倒也有一颗上进之心。宫里的人,谁不想往上爬呢,倒不和那些与他一起的侍卫一般终日糊涂度日,只是如今,怎么倒也颓丧起来了。

她素性不是个遮遮掩掩的人,索性便道:``人总有不遂心的时候,你却只拿自己的身子玩笑,以后再想要遂心,身子也跟不上了。''

凌云彻本自心烦,所以连一向要好的赵九宵都打发了不在身边,自顾自地喝着闷酒。此时听她这么说了一句,心下愈加不乐,嘴上也不耐烦道:``你是什么人什么身份,自己也不过是晾在泥潭里起不来,还有心思理会别人。''

如懿受了这将近一年的搓磨,心下自宽,也不把这些话放在心上,只在月色下将白日里晾着的衣服又抖了抖平整,道:``虽然身在泥潭里,可总不愿沉沦到底。我要是将心口上的一口气松了,便永远沉沦苦海,无法脱身了。''

``难不成你心里还想走得出这鬼地方?''云彻冷冷笑着,``别痴心妄想了。这个地方你走不出去,我也走不出去的。''

如懿抬头望着月色,淡淡笑了笑:``走不出去又如何?好歹也得活出个人样来。我若稍一松懈,一口气撑不下去,和这里那些疯疯癫癫整日在地上墙角打滚的女人还有什么不同。索性一脖子吊死在那里,尸体也没得善终。''她蹲下身,看着茂盛欲滴的青苔底下四处爬动的蚂蚁:``你见过蝼蚁么?蝼蚁尚且偷生,而且希望偷生得不要那么艰难,所以无论怎样,我都要忍耐下去。''

``忍耐就够了?''他仰天倒着酒喝,冷然道,``还不如痛快一醉,万事皆忘。''

如懿摇头道:``看你这么个喝酒的样子,大约不是为了前程,就是为了女人。偏偏这两样东西,都不是醒来就可以忘记的。反而你越是借酒浇愁,越是没有半分起色。''

``前程?我这种汉军旗下五旗包衣的出身,家里又贫寒,能有什么前程?''他大口大口地吞咽着烈酒,瞪着布满血丝的眼睛,``所以没有人看得起我,所有人都要离开我。''

如懿冷笑连连:``你是汉军旗下五旗的包衣又怎么了?我还是出身满军旗上三旗的大姓乌拉那拉氏,一朝潦倒蒙冤,被人困在这里,终身见不得天日,难道我不比你凄惨可怜么?只是做人自己可怜自己就罢了,要说出这等可怜的话来让人可怜,真真是半分心胸都没有了!''

云彻陡然被人奚落了这几句,又借着酒意冲头,便不管不顾起来:``我能有什么法子?生定了的身世,还有能力往上爬么?你被人冤枉困在冷宫是你没本事。而我呢,一点本事都使不上,便彻底没了希望。连我喜爱的女子也离我而去,嫌我给不了她翻身的机会!我还能怎么样?''

月光朦胧,是个照不亮万千人家的毛月亮。那么昏黄一轮,连心底的心事亦模糊了起来。门外的凌云彻固然是没有指望的,可是她能有什么指望?只不过是含着冤屈,受着悲怨,拼死忍着一口气,不愿彻底沉沦至死而已。是,她是个小女子,都尚且能如此,如何一个七尺男儿,偏偏这般自怨自艾。

如懿忍不住道:``能与你共患难的女子,不得已走了才值得你痛哭大醉!若是只能同富贵不能共患难,还要嫌弃你的出身前程,这种女子,若是早早离开,换了我便要买酒大醉一场额手称幸,以示庆贺。你如今既是喝了酒,要放声大笑庆贺也来得及!''

云彻的酒意兜头兜脑地冲了上来,一股悲怆之意自胸中直冲而上,几乎把胸腔都要迸碎了,他森森冷笑道:``这样子冷心绝情的话,也只有你们女人说得出来。我见过你,你的那张脸,和她竟有几分相像,难怪说出来的话都是这样冷冰冰的没有半分情意!''

如懿听他言语间似是受了那女子极大的委屈,本就很是瞧不上那样薄情寡义的女子。眼下听那醉汉竟拿这样的女子与自己浑比,虽然她如今沦落成冷宫里一个被废的庶人,却也容不得被人这样比了下贱去。如懿本是出来活络活络涂了姜汁的筋骨,想要发热暖暖关节,现下却被气得浑身发热,便也懒得说话,径自回了屋里。

如懿甫一进屋,就见惢心就着微弱的烛光在打着络子。惢心的手巧,丝线落在她手里便在十指间飞舞不定,让人眼花缭乱,不一会儿工夫,便能编出一条好看的花样子汗巾子,有松花结的、福字结的、如意结的、梅花结的,最巧的是戏文里的崔莺莺拜月烧香,她都能活灵活现地打出来,形形色色,颜色也配得好看。最精细的功夫,是在手帕绢子上打出各色花样来,经了她的手,绢子也不是普通的绢子了,配着珍珠穿了络子,或是细巧别致的穿八宝缨络,光是拿在手里,便是一方风景。

彼时尚在闺中,暖阁下的朱漆镂花长窗半开着,凉风吹起低垂的湘妃竹帘,隐约传来数声蝉呜,愈噪复静。有微热的晚风带着迷蒙的栀子花香缓缓散进,那本是最沉静清新的花香,被空气的热气一蒸,也有些醺然欲醉。那是盛夏最末的光景,一阵风过,殿外的蔷薇花四散零落如雨,片片飞红远远地舞过,光影迷离如烟。那时无忧无虑的如懿,便斜签在杨妃榻上,看着窗下的惢心,手指飞舞着打出一只大蝴蝶来。

那样清闲的时光,闺阁的游戏,如今倒成了谋生的技艺了。如懿想着便有些心酸,缓声道:``夜深了,别低头做那些活计,仔细伤了眼睛。''

惢心淡淡一笑,撑着道:``海贵人虽然得宠,也不过是个贵人的份例,皇上赏的那些东西变不了钱,小主的首饰也不能拿去变卖让人落了口实,可是咱们身边的银子,却是越来越少了。''

惢心说的也是实情,初入冷宫的艰难不过是身体发肤受苦,自己虽然是个养尊处优的世家出身,但统共只有她和惢心两个人在这里,身边又是些疯疯癫癫的居多,许多粗活譬如洗衣倒水,一一都得自己学着做起来。只是许多事能忍,譬如送来的饭菜,冬天的时候冷冰冰的没一丝热气还能忍,虽然是放了几天的隔夜饭菜了,倒好歹还不坏。但天一热起来,外头不管不顾送来的馊饭馊菜,夏天的时候远远就能闻到一股酸腐味道,惹得苍蝇嗡嗡乱飞。但冷宫里的人要活着,也要有活着的本事。单看吉太嫔好端端地活了下来,她便知道必定有饿不死的法子。

果然,冷宫外守着的几个侍卫都不是吃素的,打了络子绣了手帕交出去,总能由他们换点银钱回来,虽然总被他们昧下大半,但有他们通融着送饭菜的小太监,送来的饭菜总算是不馊不坏了,冬天的时候最低等的棉絮也总能换回来些。于是,大半的时光,她和惢心都费在了让自己活下去的这些活计上。

次日起来的时候天色便阴阴的不大好,如懿和惢心的风湿便有些犯得厉害,正挣扎着要起来处置一天的活计,却听外面大门``吱呀''一声,扑落了好多灰尘,竟是冷宫的角门被开启的声音。如懿来了这么多时日,从未听见过门锁开启,即便海兰贵为宠妃,也只能和她隔着门扇说说话。如今突然开了门,竟不知道是什么事情。

她听着那角门开启的声音,虽然不大,心里却有了一丝热络一丝畏惧。

谁知道进来的,是什么呢?

如懿坐着还未挪动身子,惢心便先起身去看了。谁知道她才出门外,便是一声又惊又喜的低呼,很快又被压抑住了,立在门边满脸是泪地回过头,那泪雨蒙蒙之中却带了无比欢欣之色:``小主,是他来了。''

昏暗的屋中,借着门口的光线,如懿微眯了双眼,才看到一个太医模样的青年男子提着小药箱进来。惢心又惊又喜地捂着嘴低声啜泣,一句话也说不出来。如懿立刻明白过来,撑着桌子站起身来,缓缓道:``江与彬?''

来人从容不迫,丝毫不以进入这种腌臜地方为辱,彬彬有礼道:``微臣来迟,小主受苦了。''他说完,侧身看着惢心,那一双幽黑眸子,在幽闭的室内看来,亦有暗转的光泽,他轻声道:``惢心,你受苦了。''

这一句话,与方才问候如懿的语气是迥然不同了,那种关切与熟稔,仿佛是与生俱来,更是发自心底的温意。

这样淡淡一句,惢心已经红了眼眶:``没想到你还能来。''

江与彬向如懿请了一安,从药箱里取出请脉的枕包,道:``能来已经不容易了。还是海贵人上下通融了多少关系,才能这样过来。''

如懿道:``其中费了不少关节吧?''

江与彬一笑:``自小主和惢心入了这里,微臣一直想来,可是人微言轻,无计可施。海贵人也因宫中连着出了几件大事,无法立刻来找。如今还好海贵人想了些法子,让微臣在太医院犯了事,被罚来冷宫给废妃太嫔们诊治,希望她们疯得不要太厉害。''

惢心倒了碗白水来给他:``这里没有好东西,你将就着喝吧。''

江与彬笑道:``来了这里,还当是什么锦衣玉食的地方么?你们别太受苦了就好。''他凝神诊了一会儿脉,便道:``小主的身子没有大碍,只是忧思过甚,颇为操劳,肾水有些虚枯。再者风湿是新得的,虽然发得厉害,但根基还不深,慢慢调理是治得过来的。''说罢他又替惢心搭脉:``你的风湿比小主还轻些,大约是素来身体强健的缘故。但切记万万不能逞强,不能在犯风湿时仍强撑着劳作,否则这病便入了骨髓,再难好了。''

说罢,他提笔写了方子念道:``川乌、草乌、独活、细辛、桂枝、伸筋草、透骨草、海桐皮各三钱水煎。''又细心叮嘱:``光服药见效太慢,还得拿桑枝、柳枝、榆枝、桃枝剥了皮,再加追地风、千年健熬水日日熏洗患处,才会好得快。另外,微臣每次来都会给小主和惢心针灸。''

如懿心中感动,谢道:``江太医有心了。''

江与彬满脸愧疚:``有心还来得这样迟,是与彬的错。药开好了微臣会从太医院领来,只是熬药的事得辛苦惢心了。''

如懿感叹道:``有药就很好了。''

江与彬想着惢心笑意温煦:``我虽然来得迟,却总算来了。以后我在,多少能方便些。至于你们的生活起居,''他从药箱中摸出一包银子:``海贵人与我的心意,都在这儿了。''

\hypertarget{ux7b2cux5341ux516bux7ae0-ux86c7ux7978}{%
\chapter{第十八章 蛇祸}\label{ux7b2cux5341ux516bux7ae0-ux86c7ux7978}}

到了三月里的时候,天气渐渐和暖。好似一夜里春风化雨,饱满了柳色青青,桃红灼灼,饱蘸了雨露润泽,洇开了花重宫苑的春天。

时气见好,皇后的病也逐渐有了起色,虽还不能下地,却至少能支撑着坐起身来了。慧贵妃为了宽皇后的心,日日都把三公主带在皇后跟前逗乐尽孝。皇后虽然失了爱子,想着年纪还轻,终究还有一个女儿。皇帝又时时宽慰着,命太医好生调养,指望着再生下一个嫡子来才好。

有了这一分心怀在胸,皇后少不得挣扎起精神来好自调养着。待得精神渐渐好了,有一日慧贵妃便把伺候的人都打发出去,将藏了数月的烧得只剩半片的人偶取了出来,将事情始末一一说个清楚,又有三公主这个皇后亲生女儿的旁证,由不得皇后不信。

皇后人还在病床上,不过穿着一身家常的湖水蓝绣莲紫纹暗银线的绡缎宫装,头上的宝华髻上缀了几点暗纹珠花,脸色苍白中却带了铁青,颤抖着嘴唇道:``你说的都是真的?''

慧贵妃当即跪下,赌咒发誓道:``事情就出在娘娘的端慧太子崩逝后的几天,又是在冷宫附近看到的这个东西。若说不是诅咒,臣妾断断不信!''

皇后不自觉地坐直了身子,如临大敌:``你是疑心她?\textless{}''

慧贵妃道:``冷宫那儿哪里有人去?这个东西只有被风从冷宫里吹出来才是有的。她能那么好心祭拜端慧太子,必定是听到了丧钟哭声,知道了端慧太子早逝,那毒妇不知怎么高兴呢,连太子走了都不肯放过,上了路还要诅咒他。''她神色一凛,姣好的面容间更添了几分戾气:``臣妾想着,这种诅咒怕不是那一日才有的。只怕咱们不知道的时候,就已经偷偷诅咒上了。怪不得从她进了冷宫之后,端慧太子的病就忽好忽坏的,总没个全好的时候,怕就是那疯婆子搞的鬼。''

皇后新丧爱子,听见这些话,简直如椎心泣血一般,如何能听得有人这般诅咒爱子。她细想起来,虽然如懿进冷宫前她的儿子便不大好,可的确是如懿进了冷宫之后,孩子的病情就一直反复,以致突然暴毙,让她这个做母亲的,几乎断了一生的指望、如今想起来,有了这个缘故在里头,几乎是恨得眼睛里要沁出血来,一双手死死攥着锦被,手背上青筋暴起,如同要吞了人一般。

慧贵妃几乎是皇后入府之后即刻随侍在身边的,多年相对下来,何曾见过皇后的神色如此骇人,心下也不觉害怕,忙唤道:``娘娘,皇后娘娘,您可千万别气坏了凤体。''

皇后冷了半晌,才缓过一口气来,慢条斯理道:``本宫哪里是气坏了身体。妹妹分明是送了一贴好药来,催着本宫要逼着自己好起来,再不能像个活死人似的躺在这里,让本宫的孩子白白去了。''

慧贵妃听她虽说得慢,但一字一字狠狠咬着磨出声来,知道皇后心里着实是恨透了,便道:``那皇后娘娘的意思是\ldots\ldots{}''

``如今她在冷宫里,咱们在外头。凡事不要着急,稳稳当当地来就是了。''皇后摆了摆手,慢悠悠弹了弹指甲,道,``那些饮食照样还送进去给她吃的吧?''

慧贵妃道:``她哪里吃得下馊腐的东西,稍稍花点银子通融也是有的。然后咱们顺理成章,把那些东西送进去给她吃。娘娘放心,一点都看不出来的。''

素心捧了碗药进来,皇后点点头道:``搁着吧。''

素心搁下便告退了,慧贵妃虽然对着嫔妃们嚣张肆意,皇后跟前却是无微不至,便亲手端了汤药伺候皇后吃了,又拿了酸梅子给皇后解苦味。

皇后感叹道:``如今真正在本宫面前尽心的,也只有你了。对了,你的身子每常不好,记得多吃温热进补的东西,别耽误了。''

慧贵妃一力谢过,却听外头道:``慎常在来给皇后娘娘请安。''

慧贵妃听得慎常在的名字,便有些不屑之意,坐正了身子略略理了理领扣上的翠玉兰花佩上垂下的碎玉流苏。

皇后看慧贵妃神气不大好,便道:``怎么?很看不上她了?''

慧贵妃只当着皇后一个人的面,便没好气道:``狐媚子下贱,娘娘病了这些日子竟不知道。皇上一个月里头有十来天召幸她的,今儿赏这个,明儿又赏那个,连先头得宠的海贵人和玫嫔都赶不上她的风头呢。''

皇后似笑非笑倚在攒心团枝花软枕上:``那么你呢?皇上可还眷顾你么?''

慧贵妃脸上微微一红:``不过一个月里留在臣妾那儿五六次吧。''

皇后淡淡``哦''了一声道:``那也不算少了。你是宫里的老人儿了,位分又高,只在本宫之下,不必去和那起子位分低的嫔妃计较,没得失了身份。你要记着,她们争的是一时的恩宠,你却要争一辈子的念想。目光且放远些吧。''

慧贵妃得了皇后这一番教训,一时也不敢声张了。听着皇后传唤了慎常在进来,只见锦帘掀起处,一个衣着华丽的丽人盈盈进来,身上一袭洋莲红绣兰桂齐芳五色缎袍,头上是银叶玛瑙花钿,累丝凤的珍珠红宝流苏颤颤垂到耳边,莲步轻移间,便如一团华彩渐渐迫近。

慧贵妃到底按捺不住,轻轻哼了一声,拿绢子按了按鼻翼上的粉,以此抵挡那丽人身上传来的迫人薰香。

慎常在恭恭敬敬地请了个大安,口中道:``皇后娘娘万福金安。臣妾听说娘娘身上大好了,特意过来看望娘娘。''说着又向慧贵妃请安不迭。

皇后含笑吩咐了``起身'',又嘱咐``赐座''。阿箬方才敢坐了。

慧贵妃慢慢转着手上的鸽血红宝石戒指,笑了笑道:``慎妹妹的气色真好,看着白里透红的,跟外头廊下的桃花似的,粉面含春哪。看妹妹这满面春风的样子,想来昨儿皇上是歇在你那里了。''

慎常在听她语气含酸,便讪讪地笑笑:``姐姐说笑了。''

``说笑?''慧贵妃轻嗤一声,``妹妹日常见着皇上,恩情长远,自然是把这恩宠当说笑了。不比咱们,三四日才见皇上一次,高兴都来不及,哪里还敢说笑呢。''

慎常在脸上红一阵白一阵,只垂了脸不去接她的话。

慧贵妃看在眼里,益发以为她是一味地得宠所以不把自己放在眼中,心中更是愀然不乐。慧贵妃的父亲高斌自皇帝登基以来就是前朝最得力的臣子,与三朝老臣张廷玉一起辅佐,如同皇帝的左膀右臂。她在后宫又得宠,哪里受得了这样的气,便打量着慎常在道:``慎常在今日打扮得好颜色好艳丽,不知道的还以为常在不是来看望皇后娘娘病情,安慰娘娘丧子之痛的,倒像是来看热闹凑笑话的。''

慎常在猛地一凛,忙赔着小心道:``皇后娘娘凤体见好,臣妾这么打扮也是来应一应娘娘的好气色。另外一桩\ldots\ldots{}''她转脸对着慧贵妃嫣然一笑:``皇后娘娘盛年体健,又深得皇上眷顾,要再得十位八位皇子也是极容易的事。贵妃娘娘说是么?''

慧贵妃被她这么一说,方知她口齿厉害,果然有皇帝喜欢的地方。当下当着皇后的面也不好再说什么。

皇后和颜悦色地笑道:``你的心意本宫都知道。你做了那么多的事,本宫和贵妃难道还不知道你的心意么?贵妃不过是和你说笑话罢了,也是把你当个亲近人而已。来,你坐近些,好多话贵妃都要和你说呢。''

慧贵妃唇边凝了一点笑涡:``可不是,妹妹如今是皇上心尖子上的人,听说不日还要抬了贵人呢。咱们不指望着妹妹,还能指望谁呢?''

出了长春宫,阿箬扶着宫女新燕的手走得又快又急,一阵风儿似的。新燕知道她是着了恼,越发不敢言语,只得小声劝道:``小主走慢点,走慢点,仔细脚下。''

阿箬走得飞快,骤然停下脚步,鬓边垂落的珍珠红宝串儿沙沙地打着面颊,好像是谁在扇着她的耳光似的。她顺手狠狠一揪,将发髻上累丝凤步摇一把扯了下来掼在新燕手中,恨恨道:``什么劳什子,也来欺负我!''

新燕吓得脸都白了,捧着那累丝凤步摇道:``小主,这可是皇上赏的,您瞧满宫里的小主,嫔位以下哪里能戴红宝呢?都是皇上疼您的心意啊。''

阿箬走得额上微微冒汗,站在红墙底下气咻咻地挥着绢子:``皇上赏我的?皇上赏我的多了去了!''

新燕忙赔着笑道:``可不是。皇上哪一天不赏赐咱们这里,饶是嘉嫔生了皇子,皇上像得了个凤凰似的,也不过这样赏赐罢了,奴婢瞧着许多东西还不如咱们的呢,嘉嫔不知道多眼红。皇上到底还是宠爱小主您的呀!''

阿箬拨着手腕上一串明珠绞丝钏出神,慢慢道:``你也觉得皇上是宠爱我的么?''

新燕喜滋滋道:``可不是,满宫里不是都在说,小主虽然位分低些,但论宠爱,谁都比不上您呢。''

阿箬怔了怔,忽然虎起脸,反手就是一个耳光:``皇上对我宠不宠爱,也是你能议论的么?小心我拔了你的舌头。''

新燕不知她为何发怒,吓得眼泪直在眼眶里打转,一声也不敢哭,只捂着脸低低说:``小主,出来有些时候了,咱们还是回去吧,要不然嘉嫔娘娘又有的排揎了。''

阿箬轻哼一声,不以为然道:``排揎?我若有些好故事告诉她,她更有的排揎呢。''

海兰伏在角门边,一身暗色弹花织锦斗篷将她的身形掩饰得不露痕迹。她悄声道:``江太医来了之后,姐姐的风湿好些了么?''

如懿抚着膝盖道:``好多了。''

海兰低低道:``姐姐好多了,皇后的病也日渐有起色。说来奇怪,病的时候就病得那么厉害,说好了也好得那么快,昨日居然可以下床了。''

``她是心病。有心让自己好起来,总是能好的。''

海兰轻轻``嗯''了一声:``眼下后宫里人不多,皇太后本来打算选秀,可端慧太子刚过世,皇上也无心操办。今日听说皇太后选了几家公卿的格格养在身边,表面上说是鞠养闺秀,伴她老来之乐,想来都是将来为皇上充实后宫准备的。''

如懿轻轻一嗤:``如今皇后不大好,后宫的一大摊子事情都交给了太后,太后自然要尽心尽力的。都选了些什么人?''

海兰掰着指头道:``总有三四个,其中最出挑的便是太常寺少卿陆士隆的女儿陆氏,侍郎永绶的女儿叶赫那拉氏。听说太后喜欢得紧,一直带在自己身边亲自调教呢。''

如懿关切道:``别总想着别人。如今你如何了呢?''

海兰默默道:``我还能如何?老样子罢了,只能牵住皇上的心不走而已。''

如懿蹙眉道:``便这样艰难么?''

海兰犹豫片刻,还是道:``皇上很喜欢阿箬,听说过了端午就要封贵人了。若是有个一男半女,成个主位也不是什么难事。''

如懿一想起阿箬当年红口白牙冤枉自己的事,便觉得刺心无比,恨声道:``她便这样得意么?''

海兰道:``得意自然是得意的。皇上这么宠爱,又是赏赐又是召幸,她阿玛也在外头得意,每年到了治水的时候,总用得上他。可她犹是不足,成日家在宫里打鸡骂狗的,也不知哪里不好了。细想起来,她这样的人总是贪心不足的。''

如懿想了想,忍耐着道:``如今也急不来。你且护着自己要紧,不用替我多筹谋。''

海兰正要说什么,却见凌云彻踢踢踏踏地走过来,不耐烦道:``时辰差不多了,海贵人赶紧走吧。总在这儿磨蹭,耽误了您的大好时光。''

海兰得宠多日,见惯了旁人的奉承,冷宫这儿虽不能进去,但来往亦是自如,何曾听过这样的话,当下就冷下脸来。还是如懿在里头拍了拍门暗示她不要理会,海兰念着往后总有再来的时候,总要靠着凌云彻通融才行,少不得忍着气走了。

如懿见凌云彻这般口气,倒也不恼,只淡淡道:``这么些日子了,还放不下旧事睁开眼睛看看前路么?''

言毕,她便转身进了自己屋子。云彻颓然坐倒在冷宫的角门边,睁眼看着墨黑的天色,眼前浮起嬿婉清丽柔婉的面庞,心中不觉狠狠一搐,像被一把生满了铁锈的钝刀狠狠划过又来回切割着似的。他下意识地去摸怀里的鹿皮酒囊,那里头是他最爱喝的掺了雄黄的白酒,气味又甘又烈,别有一股冲鼻的气息。他拧开盖子正要喝,骤然想起里头的如懿从前说过的话,想想也是无趣,便睁着眼睛打算独自守完前半夜,然后和九宵换了去睡觉。

他模糊地想着,不觉有睡意慢慢袭来。左右冷宫这里没有旁人过来,打个盹儿也是寻常的。他便索性闭上眼睛,由着自己睡去。

凌云彻被惊醒是在夜深时分,他估摸着自己才睡了一两个时辰,脑袋里还昏昏沉沉的,却听得离角门最近的屋子里传来一声又一声压抑而畏惧的低呼声。在冷宫待了这么久,他认得出那声音,是如懿和惢心俩主仆的。他也意识到,这样惊恐的低呼,一定是出了很大的危险。

他迷糊的脑袋骤然醒转过来,几乎是本能地从腰带上解下钥匙开了角门直冲进去。

眼前所见几乎让他目瞪口呆。倾尽他一生的阅历,他也没有看过同时几十条蛇在地下悠游地扭动着躯体,慢慢地往床铺的所在靠近。且不说那腻滑阴森的躯体,咝咝冒出的阴恻恻的声音,光那种腥气,就已让床上两个仅着单衣的女子吓得面目无色,魂飞天外了。

惢心见了他进来,如见了天降神兵一般,几乎是喜极而泣:``凌大哥!快来救我们。''

云彻被这一句``凌大哥''唤得回过神来,几乎是本能在驱使着他背过身转身逃命而去。不错,多年的乡间生活教会他的,便是分辨有毒和无毒的蛇。而这些蛇,分明都是有毒的。趁着现在那些蛇压根儿没注意到他,他如何能不拔腿就跑。

恐惧和惜命的情绪几乎是一下子攫住了他的心口,他转身的一瞬间,忽然听到一声低低的呼喝:``凌云彻!''

他转过脸,看到缩在床铺一角的如懿,分明已经是满脸的惧色了,却还强撑着护在惢心身前,硬撑着一脸的镇定,拿被子死死捂住自己。

两个弱女子,两床薄被,如何能抵挡群蛇的来袭。任意一条蛇只要轻轻咬啮一口,除了死,便再没有别的活路。

可是他,不能硬生生拒绝这样的神情,来自一个女子的神情。他狠一狠心,从怀中掏出鹿皮酒囊,朝着群蛇环伺处用力泼去。那酒中含了些许雄黄,本是蛇最忌讳害怕的。果然所泼之处,那些蛇都纷纷退避,行动也迟缓了好多,连口中的咝咝声也弱了下去。他趁着此时找到落脚之地,拔下腰刀趁着一股勇气胡乱挥去。

床铺上的二人吓得面无人色,只看他左挥一刀右挥一刀,刀锋所及之处,那些蛇都断成两截,心下稍稍安稳起来。谁知凌云彻挥得大意了,一条蛇只被削去尾巴,大半个身体借着刀子的力量飞了过来。如懿挡在惢心跟前,一时不防,却见那蛇冰凉的身体落在了自己手腕上。如懿恶心得浑身都发毛了,才要伸手挥开,却觉得手背上忽然一凉,像是有什么细小而坚硬的东西冰冰凉而尖锐地嵌了进去,还未觉得痛便一阵阵麻上来。

如懿只觉得头晕目眩,胸口一阵阵地憋闷上来,身子一软便歪在了惢心怀里,惢心惊呼道:``小主,小主你怎么了?''便慌慌张张地抬起如懿的手:``小主你的手背怎么都黑了?''

那边厢凌云彻才手忙脚乱处置了蛇,眼看都死透了,却听得惢心没命价慌起来,忙转头去看。他一人应付那些毒蛇,本就出了一身的虚汗,此刻看到如懿面如金纸,心下一慌,那一层本已凉透的虚汗又逼了上来。

如懿虽然身上逐渐失了力气,但脑子里还清楚,便低下头就着伤口一吸。她本是毒性发作虚透了的人,这一吸本吸不出什么。惢心却明白了,忙要探头替她吸去手背上的毒液。云彻立即拦下了,抢在前头附着如懿的手背将毒液一口一口吸了吐出。

惢心看得目瞪口呆,虽然说男女大防,但云彻所为,一切都是在救如懿的性命。她愣了半晌,赶紧倒了茶水来给云彻漱口。云彻吸了半日,见如懿手背上的黑气尽数散去,脸上也只剩了苍白,而不是那种骇人的金色。他松一口气,脚下微微一软,坐在了地上缓过劲,一抬眼竟见如懿脸上微红,眸中带了一点羞涩,侧转身去。

他知道自己是犯了男女大防,但不也是救她的性命么?这样的念头一转,不知怎的,自己脸上也热辣辣起来。他掩饰着拼命漱了口道:``还好,那蛇是被砍了一半的,嘴上没力,咬得也不深,否则大罗神仙在也没用了。不过丫头,你还是得找找有什么解毒的药给她敷上。''

惢心翻箱倒柜找出了上回江与彬留下的一盒子牛黄丸,取了一点给如懿放在嘴里嚼了,又慌道:``还能找什么解毒的?''

云彻看惢心对这些事不通,又慌得手忙脚乱的,便急道:``这些蛇都是蝮蛇,你得找些清热解毒、凉血止血的药来,什么夏枯草、半边莲、生地、川贝、白芷之类有么?''

那都是寻常的药物,惢心连连道:``有,有。''

云彻吩咐了惢心把药嚼碎了敷在如懿伤口上,自己也嚼着服了些,又取一份煮上等会儿让惢心喂如懿喝下,道:``明日我去告诉太医一声,请他再来看看,应该就无妨了。''

惢心千恩万谢道:``还好凌侍卫在,否则今日小主的安危就悬了。本来,本来\ldots\ldots 这吸毒该是奴婢的事。''

云彻点点头道:``本来是该你的事,但你一个小女子,身体自然不如咱们男人。要是你也损伤了,谁照顾你们小主呢。''他自嘲地笑笑:``我就是这么条贱命。''

如懿听他这般自嘲,有心想说什么,嘴唇张合着却无半分力气,缓了半日神,才吐出一句:``多谢。你得去看看太医。''

惢心一壁撒了草灰小心翼翼打扫毒蛇的尸体,一壁接口道:``是要多谢凌侍卫,今日若不是您在\ldots\ldots{}''

云彻看了看地上的蛇尸,仰头看了看屋顶的瓦片,踩着凳子上了桌子,顶起瓦片一看,问道:``天刚黑下来的时候有没有听到什么动静?''

惢心摇头道:``小主和我在外头洗衣服,什么都没听见。''

云彻跳下来道:``房上的瓦片松开了,想必有人往里头的梁上绕了蛇进来。蛇身上血凉,动作迟缓,晚上你们熄了灯火,人身上的热气就凝在一个地方不动,自然会慢慢吸引这些蛇过来。''他抬起头,目光炯炯:``你们到底得罪了什么人?''

\hypertarget{ux7b2cux5341ux4e5dux7ae0-ux6697ux6d8c}{%
\chapter{第十九章 暗涌}\label{ux7b2cux5341ux4e5dux7ae0-ux6697ux6d8c}}

``得罪人?''惢心吃惊道,``咱们都在这儿了,还能得罪什么人?''

如懿躺在床上,吃力道:``就是因为咱们得罪了人,所以都在这儿了。你还不明白么?
\textless{}''

惢心面上一惊,下意识地掩住口,便道:``幸好凌侍卫手上带着雄黄酒,还能抵挡一阵。否则可真是着了人家的算计了。

凌云彻缓过精神来,慢慢道:``我平素爱喝几口雄黄酒,就是因为冷宫这儿湿冷,什么蛇虫鼠蚁没有,喝着带着都是防身罢了。只是这蝮蛇虽然是常见的,

但一下子冒出那么多条来,也着实是出奇。除了故意,要说是意外偶然,也是不可能的。''他拱拱手:``小主自己多保重吧。

惢心急得拉住凌云彻的袖子道:``凌侍卫,要再有这样的事,可怎么办呢?''

云彻淡淡道:``明儿给你们捎点雄黄扔进来,墙角四处都洒一点,自己提防着吧。

他说罢转身便走了。如懿缩在被子里,一阵一阵听得心惊,只睁着眼看着窗外枝丫被风吹得乱舞,像是无数鬼爪子张牙舞爪的挥着过来,越逼越近,越逼越近。她霍地坐起身来,一背脊的虚汗被风一扑,钻心地凉。惢心端了药进来,见她这副模样,也吓了一跳,忙拿衣服给她披上:``小主这是怎么了?别被冷风扑了热身子,又招来什么不好。''

如懿只得道:``方才有点吓着了。''她撩了撩头发道:``药好了么?我身上还难受的紧,好歹拿一点喝喝。''

惢心忙端了药喂到她的唇边,道:``小主先胡乱喝一点罢了。明儿江太医过来,再仔细找他瞧瞧,好好开个方子。''

如懿喝了药,想着毒性还未完全退去,昏昏沉沉地便睡下了。

第二日一早果然江与彬赶着就过来了,如懿心里念着云彻辛苦奔劳的好处,原先看他那一层鄙薄也退了些许。江与彬仔细给她搭了脉,连声道:``幸好昨晚救治得快,否则便是大祸了。等下我得给凌侍卫也去瞧瞧,他可是你们的救命恩人啊!''说着看惢心:``也是我的大恩人!说完他又留了好些清热解毒的草药,一样一样嘱咐了惢心调弄,又多多地留下雄黄之类的药粉,替惢心和如懿撒在了角角落落处。

江与彬问起惢心素日吃风湿药汤的效力,惢心钱钱笑道:

``也不过那样罢了,哪里那么快见效呢。''

江与彬的面上闪过一层疑云:``这一个月来,你们都按时吃药了么?

惢心奇道:``巴巴儿地费了那么多才请了你来治病,怎么会不按时吃药呢?

江与彬道:``方才我搭过小主的脉,蛇毒没有大碍,但是风湿一直还是老样子。按理说你们的风湿不深,我给你们开的药也算药效强力的,虽不能马上见效,但是总能有些起色。''他见如懿手里打着络子做活儿,耳朵却一直听着,索性也不瞒着,道:``微臣这些日子给冷宫的许多嫔妃瞧过病。虽然也有得风湿的,但那都是积年在这里的老人了,阴湿许久,加上年纪渐大,自然容易得风湿。只是小主和惢心年纪还轻,又吃药调理着,屋子也不算是冷宫里最阴湿的地方,为什么风湿会一点也不见起色?''

如懿与惢心面面相觑,也说不出什么来,倒是惢心问道:``会不会中毒?''

江与彬摇头道:``世上没有这样的毒。倒是小主和惢心都是虚寒的体质,倒是真的,其他实在把不出什么。''

正说话间,外头墙下的圆洞里陆续塞进饭菜来,哪些冷宫的嫔妃们一一去领取了。等到人都散去,又送进两份饭菜来,惢心知道是她们的,便出去端了进来,饭菜虽然简陋,倒也不腐坏,不过是两份米饭,一份清炒苦瓜,一份水煮豆腐和一份酱油拌茭白。

江与彬蹙了蹙眉,心疼的看着惢心到:惢心,你们每日就吃这个,一点荤菜都没有?``

惢心摆好筷子,笑道:``我的好太医,这饭菜不馊不坏就不错了,这都费了我和小主好大的功夫花银子才求来的呢。否则吃哪些猪狗不食的饭菜,那里还能熬到你来的这一天。``

如懿笑道:``好了。江太医才说一句话,偏你有那么多话说。前几日是清明节气,有一碗烧田螺肉送进来。逢着年节,总还见点荤腥。''

惢心撇嘴道:``什么荤腥,一股腥味才是。不过就是螺丝、鸭血和蚌肉之类的,素菜也反反复复就这么些。''

江与彬当即变色道:``你说真的?''

如懿见他脸色不好看,即刻放下筷子,疑道:``这些饭菜有什么不对的么?

江与彬肃穆了神色道:``微臣刚说过,小主和惢心都是虚寒体质,这些食物又都是大湿大寒的,小主与惢心一日三餐吃这个,加重了体内的寒气,难怪风湿久久不见起色。原来是在这些地方。

如懿默然,一颗心缓缓、缓缓沉到了底处。原以为昨晚的蛇便己经是杀招,

不承想这里还藏着天长日久的厉害在,却是自己留意万分也留意不到的事情。

惢心恼恨道:`怪道呢,还以为咱们是花了银子通融的,饭菜才和别人不同些。原来是有人做了手脚''

江与彬脸色沉重,道:``若说无心,断不能顿顿都这样。这些东西本是无毒的,也不相克。只是饮食用药,体热的人不能过多温补,虚寒的人切记寒凉。寒凉不是说生食冷食,而是性寒的东西。像小主和惢心的体质,便是碰不得这些的。''

赛心发愁道:``那可怎么办呢?除了这些,咱们也吃不上别的。''

江与彬看着窗外晴和的日头,分明是四月时节春暖花开,在这日头也照不透的地方,却只有凄寒彻骨。偏偏便只有这两个女人熬在这里,叫天天不应,叫地地不灵,年深日久\ldots\ldots{}

他一想到年深日久,他们还在此处,便冷不丁打了个寒噤,仿佛是一阵冷风逼近了骨子里,透心彻凉。

如懿深吸一口气,缓缓摇头道:``没有办法。送这些饭菜的人既然有心,如果看到咱们不吃完,或是悄悄倒在哪里,便知道是起了疑心了,更不知道要用什么法子来谋害我们。与其如此,不如就安他的心,照吃照睡就是了。``她斜睨了江与彬一眼:''至少江太医是不会袖手旁观的。``

江与彬心中暗赞她的沉稳,便道:``微臣会找些温热滋补的药物给小主和惢心慢慢调养,希望能化去食物的湿寒之气。至于其他的事,昨晚已经这样险,若有什么轻举妄动,反而让杀身之祸来的更早。''

江与彬如此嘱咐了一般,惢心便送他到了门外,自也不能远送,只得回来。

如懿看着桌上的饭菜,往日为了活下去,她拼命保重,每顿饭都吃的干干净净。如今看着这些东西,竟像慢毒一般,天长日久积累在自己身上,如何还能下咽。

惢心进来掩了门道:``小主,昨晚的事你疑心是谁?''

如懿一下一下叩着桌脚,极力平缓着自己的情绪,缓缓道:``我还能疑心是谁?不过是想起当年惊蛰的时候,怡殡宫里突然掉下条蛇来。你不觉得事情有些关联么?''

惢心凝眉道:``小主觉得,害咱们的人就是害怡殡的人?那事本来就是一气的。

如懿微微点头,看着廊下丛生的杂草萧萧,黯然道:``只是如今我们哪怕想到了是谁,也没有办法。只能先保住自己的性命,不要不明不白丢在这儿就是了''

主仆俩默默地守着,照旧过活,到了午后时分,却见外头一包东西``啪''地丢进来,如懿正在院中晾晒衣服,拾起一看才知道是凌云彻丢进来的一包雄黄。

她感念他的细心,更兼昨日救命的勇气,也不管他在不在,对着角门边便诚恳道了声``多谢''。

自进了冷宫,如懿满心的怨恨与不甘,更兼对世人冷了心肠,除了海兰与惢心之外,再加上如今一个江与彬,其他人是一个不信,一个不听。无论谁落在她心里,都是带着当初害她的疑影的。

可是经了昨夜那一番事,即使是再冷的心,也不觉生了一份暖意,仿佛一点涓涓的细流,润泽了干枯的心扉,叫她知道,这世上总还有热心肠愿意对人好的人。

或许这一点温暖,足以让她觉得人世苍凉,不那么风寒逼骨了。

如懿这样想着,凌云彻却没那么福气了。这一日傍晚他去领自己和九宵的那顿晚饭,才走到冷宫的甬道口,不知道哪里闯出来几个力大无比的侍卫,把他摁倒在地,只问了一句:``你便是凌云彻?''

云彻才答应了一声,那拳头便不分青红皂白地打了上来。他是宫里混久了的人,知道一定是哪里得罪了人,也不敢分辩,只护住了要害咬着牙一声不吭。那拳头落下来如雨点一般,每一下都是下了狠手的。起初还觉得痛入骨髓,渐渐也麻木了。就像他一直以来的生活,除了忍耐,还是忍耐。因为反抗,只会招来更大的痛苦。

好一会儿,那帮侍卫看他乖乖承受,也不反抗,便也打累了收手。其中一个趾高气扬道:``知道为什么打你么?''

云彻抱着头伏在地上,一时也爬不起来,只道:``小人无知,请大人指教。''

另一人``嘿''了一声道:``原来你还真是个糊涂的!当你有几个胆子呢,连咱们小主的事都敢得罪!还打算英雄救美,哪天连自己怎么死的都不知道呢!''

领头一个抱着肩膀,冷笑道:``咱们小主如今是有皇子的,谁敢不睁开眼睛看看清楚,敢扰了她的好事。真当是不要命了!这次权当你是无知,以后你就牢牢记着,你在冷宫只管是守门的,要是连救命的事也管,便是搭上你自己的性命了。''

说完,几个人一使眼色,便四下散了。

云彻伏在地上,缓了半天的劲才爬了起来,试着动了动手脚,发现还好没伤了筋骨,便慢慢往庑房里走。九宵见他这个样子回来,也吓了一大跳,来不及去问晚上的饭菜如何,忙要拉了他细问。云彻简短应付了几句,便赶紧找出伤药来自己抹了。夜间旁人问起,只说自己不小心得罪了人,便也应付过去了。

次日傍晚时分,赵九宵看他受伤,便帮着去领晚饭。

云彻坐在门口,身上的伤虽没伤及筋骨,却辗转反侧痛了一夜,他没有睡好,便觉得疲倦难耐,心中更含了一包窝囊火气无处发泄,深悔自己那日莽撞进去救人,白白连累自己挨了一顿打。

他正懊恼,只听身后的门上笃笃几声响,有年轻女子轻声唤:``凌云彻。''一包薄薄的东西隔着墙头``哗''地飞落下来,他顺手捡起一看,却是一双鞋垫子,针脚纳得又细又密,显然是新纳的。

云彻心头微微一暖,自从他入宫当差起,便再也没人替他纳过一双鞋垫了,他一笑,牵动嘴角的伤,不觉生了几分懊悔,更兼了一份难以言说的畏惧。他抬起头,看看甬道之上细细窄窄的一痕天空,灰扑扑的,好像随时会变成一条勒死人的绳索,套在自己的脖颈上。他一狠心,随手将鞋垫从墙头抛了进去,以一种拒人于千里之外的口气冷冷道:``自从进了宫就没穿过别人送的鞋垫,怕穿上了走到阎王跟前去。''

里头轻轻笑了一声,忽然笑声止住,换了一种惊疑的口吻:``你的脸怎么了?''

想是里边的人看到了他脸上的伤,他索性也不瞒着,粗声粗气道:``那天是我莽撞了,只想着你们的命,忘了自己也是一条命。''

有片刻的沉默,如懿已经明白过来,虽然明知他看不见,却也是深深一福到底,``抱歉,是我们连累你。''她轻声道,``伤要不要紧?''

云彻听她并未因为自己的呵斥与粗暴而负气而去,转念想见当日救与不救原在自己一念之间,如何能怪旁人,心下便先软了几分,换了稍稍温和的口气:``不要紧,都是皮外伤。''

如懿松了一口气:``那就好,否则我与惢心心里更加过意不去。那么,知道是什么人打的么?''

云彻犹豫片刻,想起领头一个侍卫的话,便道:``他们说了一句,什么有了皇子的小主,其他我便不知道了。''

如懿心头悚然一凛,便道:``你知道得越少越好。''她捡起那包鞋垫道:``这双鞋垫是惢心纳了一个下午的,还望你能收下,也算我们尽一点感激之心。''

云彻想了想道:``如果再加一瓶跌打药给我,就算是谢我了。''

如懿闻言,不觉含笑:``那就谢过凌侍卫了。''

如懿回到房中,嘱咐惢心挑了一瓶最好的跌打药和鞋垫一起送出去,自己只是坐着出神。惢心回来见如懿只是坐在桌前发怔,便道:``小主这是怎么了?''

如懿淡笑道:``我只是听凌云彻方才说起,说打伤他嫌他多管闲事救人的人说起,是有皇子的小主吩咐他们做的。''

``有皇子的小主?''惢心脸色微微一变,``宫中有皇子的小主,只有纯妃和嘉嫔,难道是她们?''

如懿只是沉默不语,惢心越发猜疑道:``纯妃有大阿哥和三阿哥,可是她与我们还算亲厚,嘉嫔虽然不太与咱们来往,言语上又厉害,喜欢落井下石,拔尖抢乖,但比起慧贵妃她们,也算不上有什么深仇大恨。难道会是她?''

如懿摇头,给自己斟了一杯白水,慢慢道:``如果你受了我的指使去害人,会不会当着人家的面提起是谁指使的?哪怕是含含糊糊的影子话都不会落下。''

惢心即刻明白:``小主是说那些人是故意的?''

如懿微微一笑,看着杯中的白水道:``水至清则无鱼。凡事太分明,反而落下疑影,她们非要给我来这一招移祸江东,反而告诉我是哪些人更可疑。''

惢心皱眉叹了一声:``可惜咱们知道归知道,也不能如何防范,只能求菩萨保佑,让她们无心顾忌咱们就是了。''

如懿扬眸浅笑:``这样的事,咱们做不到,海兰却一定做得到。''

因着皇后丧子,皇帝膝下的实则只有三子一女,且三位皇子都是庶出,实在违背皇帝一心立嫡子为太子的心意。这一年暮春,便由海兰提议,因为后宫屡屡失子,有伤阴鸷,为求多子,皇帝与皇后便携了后宫嫔妃,相随去圆明园伴驾。一则散散心,二则也希望借此机遇可以让宫中多些子嗣,三则也暗合了太后的心意,将自己收在身边年龄颇相宜的太常寺少卿陆士隆的女儿陆氏让跟着去了。

果然到了圆明园中不久,陆氏不过十五岁,因着年轻美貌得到圣意垂顾,不久便封了庆常在,在皇帝身边很得恩宠。加着玫嫔旧爱难失,新宠又当道,如此一来,圆明园中愈加热闹,便越发顾不上宫里的情形,如懿也稍稍缓了口气。

只是听着这样新宠旧爱的消息传来时,如懿起初仍布面有些丝丝缕缕的惊痛,一点一滴触及心房,蜿蜒直刺下去,渐渐地,便只剩了酸楚。每每这个时候,便会想起,那年的烟柳蒙蒙时节,与皇帝的初遇。

彼时,她还是高门玉楼里的深宅闺秀,因着表姑母嫁得那样高贵美好,也生出了一点不知天高地厚的心。她知道的,她会嫁到皇室。却极想,与姑母一样,承担起一个家族的荣华,步步踏在紫荆城的朱门锦绣之内。可是偏偏,齐妃的亲生子,皇后抚养的三阿哥弘时,中意的人并不是她。一个错失,眼看着他削爵,去宗籍,逐出玉牒,最后赐死。

一颗心除了惊惶不定,更有一重快意。他是那样看不上她,宁愿去喜欢不该喜欢上的人。于是那样尴尬的时候,遇到了如今的夫君。

当时皇帝仅剩下的两位成年的阿哥里,五阿哥豪放不羁,四阿哥端稳持重之余却不失一段玉树风流。明明是身世普普的皇子,却偏偏更像一个``骑马倚斜阳,满楼红袖招''的偏偏浊世公子。

那一瞬间,便动了心意,忖度着哪怕他是``翠屏金屈曲,醉入花丛宿''的人,便也顾不得自己既一颗芳心了。

在冷宫的侵淫里,或是深宫静院午夜醒转,梦醒衾寒的时候,会忆起很多年前,姑母与当今太后安排着他们见了一次。

姑母含笑轻声唤着``青樱'',她便轻轻巧巧,莲步姗姗,从十二扇泥金仕女簪花屏风后转出来,杏子红透纱绣牡丹含露闪缎长裙缓缓漾起一点涟漪般的微澜,连腰带上垂的一对白玉鹧鸪樱桃佩都微微摇曳,仿佛一朵绽放在暗夜微风里的红蔷薇。

不,她如何不想保持大家闺秀的沉稳笃定,安宁无波,而是,实在是在屏风后一定窥视的害羞,让她晃了晃心思,愿意捧着一颗一瓣一瓣绽放的胭脂色的心,一直一直沉静下来,沉到尘埃的底处去。

那时她也不过是十三四岁,单衫杏子红,双鬟鸦雏色。

一转身,一抬头,眼帘里撞入了以为可以依靠一生的人。那时候的他,不过是一袭月华色淡淡青衣,袖口是极素净的暗色花纹,仔细瞧去是唐棣之华的图纹,腰间只一根明黄色带子,晓谕皇子身份。

她无端地便想起那一句:``唐棣之华,偏其反而。岂不尔思,室是远而。''

怎么会遥远呢?如果是真切的缘分,再远,这个人也会来到你身边。

他是谦谦君子,温润如玉,淡淡含笑间,便是清明天际朗月入怀。可是他即便那样笑着,也难免有一分失势皇子的萧索,萧萧肃肃,若孤松独立山巅之风。

她一贯倨傲的心,莫名地就颤了颤,生了一股相怜之意。

真的,是君须怜我我怜君。他有他身世的不堪,自己也有自己的难为。

然后,亦见过一两次。不过是姑母或者当今太后的安排。

她替太后抄书,他来请安,有时替他磨墨,唤一声``青樱妹妹''。她抬起头来,并没有旁人在,他望住她,也不过,就是相视一笑罢了。

还有一次,是陪着满宫的嫔妃们在清音阁看戏,有一出是他点的,便是《墙头马上》。戏台上的戏子歌舞泣笑,唱的是别人的人生百态。她却被一阕引子惹动了心肠。``妾弄青梅凭短墙,君骑白马傍垂杨。墙头马上遥相顾,一见知君即断肠。''

她忽然便沉了心思,抬起眼。正望见他也含了一缕笑,沉沉望住自己。就是这段,遥遥相顾,一见知君即断肠。仿佛暮春里迟迟未开的花苞,忽然一阵春风至,便张开了重重心瓣,露出一点杏色的蕊。

身边有花朵熏然的陶陶气味,好像一整个春天的,都留在了身边,迟迟不去。

为着这个,她便肯了。肯只是一个侧福晋的地位,肯按下一颗欲比天高的心,肯容忍他的身侧枕边,眼底心间,还有旁人。

那便是一颗初见的痴心了。

而到了如今,他还能如何呢?位分也罢,恩宠也罢,一直引以为依靠的,不过是他口中常说的三个字:你放心。

可原来,到了放心的时候,却彻底没有让她放心过。

还不如海兰,从来不深爱,所以不看,不听,不信,倒安安稳稳,平安富贵了。

如懿一副柔肠百转千回,正凝神间,却见惢心匆匆转进房里道:``小主,海兰小主刚让人从圆明园递来的消息,老爷他------过世了。''

\hypertarget{ux7b2cux4e8cux5341ux7ae0-ux5fc3ux5fd7}{%
\chapter{第二十章 心志}\label{ux7b2cux4e8cux5341ux7ae0-ux5fc3ux5fd7}}

这一惊真当是非同小可。如懿还没将这句话在心里过一过,便觉得一个闷雷在脑中轰炸开来,彻底晕了过去。

良久,也不知过了多久,她悠悠醒转,睁开眼看着窗外清冷的星光,那星子微白的点点寒光,冷得透到了心底。

她的父亲,竟就这样死了?

惢心傍在她床边,啜泣着道:``小主,老爷死的时候府里已经很困窘了。小主是知道的,就着孝敬皇后母家承恩公的恩典,这些年传下来,到咱们这儿已经是内囊都上来了。又因着景仁宫皇后的事,其实很多亲眷都不来往了,田庄上的收成也断断续续的一年不如一年。多少还是倚靠着小主在宫里的位分,日子还能将就着过些。如今\ldots\ldots 如今小主进来这两年,府里的一大家子人不知道多难过呢。如今是树倒猢狲散,听说老爷临终的时候,床前只剩下夫人和小少爷、二小姐三个了。''

热泪流过肌肤有刺痛的感觉,她的魂魄早已飞到了旧日的闺阁,只听着自己的声音空洞地问:``乌拉那拉氏有那么多亲眷,难道都死绝了么?\textless{}''

惢心含着满眶热泪,低低道:``小主难道不知道么?所谓亲眷,都是烈火烹油锦上添花时的热闹。真正到了有难的时候,一个一个逃得比八竿子还远。如今府里只剩下个虚名,老爷死了宫里只赏了二百两银子,里里外外连个丧事都弄不周全,还是海兰小主想尽了办法,送了五百两银子出去,这才勉强像个样子办起来了。''

曾经朱门绣户的乌拉那拉府邸,历代后妃辈出的豪门大族,原来轰轰烈烈之后,也不过是人丁凋零,家财散尽,落得个高楼轰然塌的结局。

她的幼弟不过十岁,她的妹妹更小,才八岁。而母亲已经老了,四十多岁的年纪,身上长年病痛不断,需得延医请药。家中境况好的时候,每常还有太医出入问安,那不仅是医术高明,更是一份荣耀的象征。

非得皇亲国戚,不能如此。

而今呢?而今只怕连请个寻常大夫抓服药都不能了吧?她虽然知道父亲的身体一日不如一日,渐渐颓败,可如今骤然离去,未尝不是世态炎凉刺激着他日渐老弱的心啊。

如懿睁着眼,任由泪水蒙住了眼睛:``阿玛到底是什么病?才会走得这样快?''

惢心道:``听来报信的人说,从去年秋天就不大好,断断续续地痰里带血,到了今日早起一口痰涌上来堵住了喉咙,还来不及请太医,就过去了。听说这之前,也求爷爷告奶奶请了许多大夫,但不是拿不出银子请好大夫,便是人家瞧不上咱们的门第不肯来。所以老爷的病,是拖坏了的。''

如懿挣扎着起身,扑到门外,哭着道:``惢心,我要去见我阿玛,见我阿玛最后一面!''

惢心忙拉住她道:``小主,小主,您别伤心坏了。咱们出不去,咱们一辈子都出不去的呀!''

热泪汹涌而出,像是要刺盲了眼睛。她原是被困在了这里,如同夜莺失去了啼声,鸟儿被折断了翅膀,生生困在了这里。

即便是最困窘痛苦的时候,她都没有这样痛恨过,痛恨过自己身在冷宫,终身不得自由。

她哭得精疲力竭,伏倒在门边,墙根下阴冷的青苔几乎抵着她的脸,湿腻腻的冰冷,融着她的泪:``他老人家便这样去了,我\ldots\ldots 我却连最后一面都见不上,连想要给他磕个头都不能。''

如懿跪在地上,朝着南面家中的方向连连叩头不已:``我阿玛走之前,有没有什么话留下?''

惢心欲言又止:``老爷只有一句话,是说完了这句才咽气的,府里说,一定要落进您的耳根子里。''

``什么话?''

惢心皱紧了眉头,为难着道:``老爷最后一句话是------青樱,你没用!''

额头触地冰冷而坚硬,砰砰地令人发昏。呵!真的是自己没用呵!拖累了自己,拖累了家人,拖累到父亲临死,都不能咽下这口怨气。如懿心头发颤,身子一仰,几欲晕去。

惢心忙扶住了她,抱着她的身子道:``小主,小主您要保重。您若再伤了身子,咱们府里便真是一点指望都没有了。''

如懿的头贴在生冷的泥地上,以此来凉自己的心目。``指望?''她自嘲地失笑,落泪道,``还有指望么?''

从她进冷宫的那一天起,她便知道是没有指望了。一息尚存,百般求生,只是不愿意就此平白死去而已。没有炭火的冬日里,只能拿一床床被子衣物厚厚地盖住自己,恨不能如蛇鼠般冬眠度日。偏偏只能醒着,咬着牙抵御着寒冷,吞下冰冷难咽的食物,苟延残喘。风湿的痛楚在四肢百骸里蔓延的时候,连肢体都仿佛不是自己的了,只好像看着有人切骨磋粉,一点点磋磨着。她都一一忍耐了下来。

可是她却忘记了,以为能求得彼此的平安,却疏忽了因了她的失宠被废,本已没落的家族,更是一切散如烟云。

是她忘了,是她疏忽。家族的荣辱全都系于她一身,她怎可在冷宫继续忍耐下去,没有出头之日?

这一夜,她几乎难以成眠。七月时节雨潇潇,风萧条,雨亦萧条,原本暑热的天气被骤然而至的冷风冷雨裹卷在一起,吹得身上一阵热一阵凉,如同她在沸油与冰屑里翻滚烹炸的一颗心。她听着夜雨敲打青瓦,扑簌扑簌的冷硬声,茫茫漫漫,仿佛是无数低低的哭泣,来自遥远的幽冥世界。

这样翻翻覆覆的两夜,她自己都觉得倦极了,可是偏偏睡不着。外头的雨无尽地下着,仿佛是替她滴着眼泪似的。终于在迷迷瞪瞪之中,她倦极,闭上了眼睛。

却还是不安稳,往事影影绰绰恍惚在眼前。阿玛老实,不过是个佐领,却极疼爱这个长女。额娘的性子虽然厉害些,到底也是妇道人家,每日所研习的,不过是如何做顿好饭菜,让全家欢喜满意。幼妹憨稚,幼弟文气,而她,在管束弟妹之余,不过只懂得针黹刺绣,闺阁游戏罢了。和和睦睦的一家人,欢声笑语还在耳边不曾散去。然而,那一日黄昏,是姑母找她入宫,那时的姑母,雍容华贵,总有着不褪的恬淡笑意,执着她的手语重心长地与她相谈。

乌拉那拉氏虽然出了她这个皇后,但底下的家道已经渐渐日薄西山。

乌拉那拉氏再没有适龄的年轻的女儿,只有你,青樱,年龄合适,又与姑母最亲。

如果没有女眷入宫,或者成为皇亲国戚,乌拉那拉氏的荣耀如何延续?

乌拉那拉氏的男人都不中用,只有女人,只有靠女人了。

那年的自己,还是那样的懵懵懂懂,但姑母执着她的手那样用力,她没得选择,因为她是乌拉那拉氏的女儿。

陡然间,姑母的脸色转成了无限的凄厉,满头华发,发髻间的珠翠只是越发衬出她的衰老与凄苦。她穿着皇后的衣冠,那衣冠却旧得透透的了。

姑母声色俱厉,逼视着她:

``当年孝恭仁太后告诉我,乌拉那拉氏的女儿是一定要正位中宫的,如今我一样把这句话告诉你。你,敢不敢?''

``宠妃?除了拥有宠爱,还有什么?宠妃最大的优势不过是得宠,一个女人,得宠过后失宠,只会生不如死。咱们乌拉那拉氏怎么会有你这样目光短浅之人?''

``等你红颜迟暮,机心耗尽,你还能凭什么去争宠?姑母问你,宠爱是面子,权势是里子,你要哪一个?''

她被逼迫不过,只得道:``青樱贪心,自然希望两者皆得。但若不能,自然是里子最最要紧。这一路虽然难,但青樱没有退路,只能向前。''

姑母终于欣慰:``青樱,你要明白,当一个人什么都可以舍弃之时,才是她真正无所畏惧之时。''

她还有什么可以失去?荣华与权位,夫君的信任,家族的前途,所有的都已失去,她还有什么可以害怕?

有阴冷的风层层逼近,姑母穿着一袭黑衣,披头散发,恍若厉鬼,她气得红了眼睛,大力地扇着自己的耳光。她只隐约记得,姑母死了,已经无名无分地死了很久。

姑母一壁狠狠扇着她的耳光,一壁厉声斥责道:``乌拉那拉氏已经出了一个弃妇,再不能出第二个弃妇了!为什么你还能在冷宫安于做一个弃妇?做一个成为门第之羞的弃妇?你为什么不记得,你是乌拉那拉氏的女儿?你好好活着,并不是为了你一个人,而是整个家族荣辱!''

姑母的耳光打得又狠又准,一下一下激烈地落在她的脸上,亦抽动她已经蒙昧的一颗心。姑母的身后,是老迈的阿玛,老泪纵横,无奈而软弱。

如果是家道中落逼得阿玛早早离世,那么自己,何尝不是罪魁祸首之一?因为她没有本事保全自己,所以只能眼睁睁看着家中人一一衰落,无计可施。

她的冷汗涔涔而下,姑母说得对,她如何配做乌拉那拉氏的女儿?

她自昏聩的睡梦中被自己惊醒,落得满头满身的大汗,靠在粉末簌簌落下的墙壁上大口喘息。

生的感觉如此美妙,哪怕呼吸到口中的空气带着潮湿的霉味,中人欲呕。但,好歹是活着,还要好好地活着。

惢心不安地替她擦拭着,却又不敢惊动旁人,只得低声道:``小主,小主,您是不是梦魇了?''

如懿紧紧攥着惢心的手,哑声道:``不是梦魇,而是我的梦魇应该醒了。''她抬眼看着被水迹霉湿的墙壁,青苔丝生的墙角,永远湿答答潮腻腻的泥土地面,冬冷夏热的屋子。受够了,真的都受够了!

惢心会意地握住她的手,懂得地点点头,只道:``海贵人不在宫里,纸钱什么的不大好弄进来,只好咱们自己随意折一点,尽一尽心意。''

圆明园中连续下了几日的雨,越发多了几分清爽凉意。皇后坐在``天地一家春''的暖阁里,看着廊下的青瓷大缸中新开的几朵碗莲,盈盈巧巧的一朵并一朵,粉润的色泽如桃花宿雨,盈盈欲滴。皇后赏着碗莲,逗着手边铜丝架上的一只彩羽鹦哥儿,问道:``皇上真的让慧贵妃一个人搬进了韶景轩居住?''

赵一泰弓着身子恭声道:``可不是?皇上住在九州清晏的乐安和堂,慧贵妃的韶景轩松柳环绕,景色绝佳不说,与皇上的乐安和堂隔岸相对,最近不过。反而是皇后娘娘与其他小主都住在九州清晏这儿的天地一家春,既拥挤繁闹,又与皇上东西相隔,来往实在是不方便。''

皇后取过一支玉簪,笑吟吟调弄着鹦哥儿:``那按你的意思,本宫该怎么办?''

``皇后娘娘是后宫之主,理应离皇上最近,少不得也得住得清静些。而且您\ldots\ldots{}''赵一泰赔着笑,抬头看了看皇后的脸色,``您也应该尽快添一个小皇子了。否则慧贵妃如今这样得宠,连皇上新宠的庆常在和慎贵人都被撂到了后头呢。您不怕她赶在您前头有了位皇子\ldots\ldots{}''

皇后冷冷剜了他一眼,旋即又是泰然温和的面容:``自从进了圆明园,皇上的几个新宠就一直想尽办法霸着皇上。慧贵妃诗书敏捷,能重新得皇上喜爱是好事,本宫去讨这个嫌做什么?只要皇上不是专宠那几个年轻狐媚的,便也罢了。''她微微挑眉,摸着细白如玉的手腕,冷笑一声道:``只要慧贵妃有生皇子的福气才好呢。''

赵一泰忙道:``娘娘圣明。''

皇后婉然笑道:``不是本宫圣明,太后让咱们进圆明园,就是指望那么多嫔妃能好好侍奉皇上,给皇上添个一男半女,本宫又怎可去干涉?倒不如做一个安静贤惠的皇后,由着她们争风吃醋去便罢了。''

赵一泰接过皇后手中的白玉莲花簪,替皇后端端正正簪在丰盈的宝月髻上,笑道:``奴才明白了。难怪皇后娘娘从不屑与那些小主似的花枝招展,原来便是这个淡极始知花更艳的意思。皇上看腻了她们的弄巧心思,自然会回到皇后身边来的。''

皇后淡淡笑了一声:``你方才说,乌拉那拉如懿的阿玛那布尔死了?''

赵一泰忙道:``是。刚得的消息,因是晦气的事,也不算要紧人物,所以消息递进来慢了些。''

皇后``哦''了一声,扶了扶蝉翼似的鬓角,轻声道:``虽然慢了些,但到底是要紧的事。也是乌拉那拉氏可怜,家族衰败,阿玛又去了。你想办法托人送些纸钱冥器给她,让她烧一些给她阿玛尽尽心。''

赵一泰怔了怔:``可是宫规严令,宫内是不许烧这些东西的\ldots\ldots{}''

皇后的笑意温和,拨了拨那鹦哥儿鲜红的喙:``宫规是宫规,难为她在冷宫里的孝心了。你好好去办吧。''

这一夜月落乌啼,正好逢着七月十五的中元鬼节,又是如懿阿玛的头七之日。天不黑日头就落了,那斜阳带着凄厉的血红色,像是谁把一整桶血都泼在了天上,任由它四溢滑落,渐渐天色亦昏暗下来,那血亦成了枯涸的血痕,黑红黑红地黏在了天边。宫中林木蓊蓊郁郁,无数宫鸦黑羽纷腾,如乌云遮蔽月色,回旋于天际,映着这昏沉天空,像是融入了这无尽的黑暗之中,唯有``啊啊''哀戚鸣声一层层遥遥散落,悸动阴气渐深的宫阙。

到了戌时一刻,远远听得鼓钹齐鸣,佛号喧天,如懿知道是宫中中元节水陆道场放焰口的仪式了。因着太后笃信佛教,宫中分别请来法源寺的僧人、白云观的道人和妙应寺的喇嘛举行法事做道场,表慎终追远,追念故人之意,以平息亡魂,祈求宫中安泰。不仅是宫中嫔妃,连宫人们也可参与。便在昨日,如懿折了一叠纸莲花,趁着凌云彻当值时送给他烧了追念亲人亡魂,云彻倒也十分感激。

往年此时,如懿也会在嫔妃之中放荷花灯表达故人追思。而今时今日,她便只能在院子的廊下偷偷地烧一点纸,寄给九泉之下早逝的父亲。冷宫中的人多半疯疯癫癫,或是早已浑浑噩噩,平日里住得远,自是无人来理会她们。倒是吉太嫔过来取饭食的时候看见,冷笑着几声道:``果然是活腻了,居然偷偷找纸钱来烧。如今太后那老妖婆一个人在宫里,她可最忌讳这些。你可仔细着点。''说罢也不理会,便自顾自走了。

如懿蹲在那堆烧着的纸边,火光暖烘烘地熏在她身上,才觉得暖和了好些,不像父亲刚去那几日,她总觉得冷津津的。

惢心道:``这些纸钱是好不容易送进来的,说是海贵人的意思,给小主略表哀思的。''

如懿点点头:``难为她了,塞在送饭的门洞里送进来的,神不知鬼不觉。''

惢心道:``小主放心吧。嫔妃们都不在宫里,太后肯定去看法事了,没人会察觉的。''

话音未落,只听得外头一声尖利的冷笑道:``真没人察觉么?你们也太胆大妄为,无法无天了!''

如懿骤然听得声音,手中握着的纸霍地全掉进了火堆里,火越发烧得高高的,差点烧到了她的衣角。还来不及反应,冷宫的门霍然开启,只见太后身边的成翰公公领头进来,趾高气扬道:``真是一群不要命的东西,宫中严禁焚香上供烧纸钱这三大样,你们居然还敢躲在后宫里偷偷烧纸钱!真是罪该万死!''

如懿和惢心陡然见了成公公进来,吓得脸色都变了,只懂得跪在一旁,默不吭声。

成公公正呵斥着,只听一把女声慈蔼道:``冷宫是宫中禁地,她们烧纸钱固然是不对,可成翰你在冷宫喧哗,也未免太不懂规矩了。''

成翰听得这一声,忙吓得弯腰守在路边,伸手搭住一只保养得宜、戴着各色珠宝戒指的手,诚惶诚恐道:``冷宫污秽,皇太后仔细足下。''

皇太后扶住他的手缓缓踱进来,淡淡笑道:``想本宫年轻的时候,也不是没有来过冷宫,就当故地重游罢了。''她目光宛然一瞥:``宫中有人向哀家举报,中元鬼节,居然有人敢擅自在后宫烧纸钱违禁,实在是大胆。''

如懿与惢心久未见太后,只觉得她气色越发好了,一袭绿纱绣夔龙牡丹金团寿镶领纱氅衣配着满头赤金与和田玉的钿子,更显得她精神奕奕。

如懿见了太后,那份畏惧之色尚未从脸上褪去,倒先含了满眼热泪,仿佛就是不见人烟的孤魂骤然见了故人,一双眼只落在太后面上,俯首叩了三个响头,道:``奴婢被关在冷宫多时,太后是第一个来看奴婢的人。虽然奴婢明知要受太后责罚,但见太后精神旺健如旧、一切安好,奴婢便愿受任何责罚。''

太后见她如此情真意切,也不免生了几分感慨:``你这孩子,在冷宫里居然还这么惦记着哀家。''

惢心伏在如懿身边,大着胆子道:``回皇太后的话,我家小主虽然身在冷宫,心中却无时无刻不在挂念太后,每日必临窗祝祷,祈求皇太后身体安康,福寿延年。''

太后微微一滞,眼中闪过一丝动容,继而环视着四周道:``哀家还以为你安安分分待在这儿了。既有这份心意,怎么竟然敢违反宫中禁忌,在这儿烧纸钱这么晦气。''

惢心吓得一凛,忙道:``太后息怒,太后息怒。小主的阿玛,乌拉那拉家的那布尔老爷过世,到今日正好的头七了,小主不是有心冒犯宫规的。还请太后体谅小主一片孝心。''

太后的神色看不出一点端倪,仿佛平静的湖面,波澜未惊:``孝心是私,宫规为公。怎能为了私心而枉顾公理。成翰,按照宫规,该当如何处置?''

成翰扬了扬嘴角,皮笑肉不笑道:``擅自烧纸钱,有违宫规,该赏步步红莲之刑。''

太后慢慢拨着手上的赤金嵌和田玉护甲,沉声道:``宫规大如天,那就赏吧!''

所谓步步红莲,乃是取尺把长的铁蒺藜抽到脚心,一顿责打下来,脚心脚背没有一块好肉,筋骨尽现。受刑之人一双脚自此便废了,被扶起行走时骨头触地,踩下血红痕迹,宛若红莲绽放,乃是慎刑司七十二酷刑之一。

如懿一听,不免冷汗涔涔而下,瞬即蔓延到了脖颈处,濡湿了领子。

惢心差点没昏厥过去,忙拼命磕头道:``太后,太后娘娘,求您饶了小主,饶了小主。''

太后微微摇头,淡然道:``凡事一旦做下,必得承担后果。你接受便是吧。''

\hypertarget{ux7b2cux4e8cux5341ux4e00ux7ae0-ux7389ux956f}{%
\chapter{第二十一章
玉镯}\label{ux7b2cux4e8cux5341ux4e00ux7ae0-ux7389ux956f}}

太后一声令下,成翰努了努嘴,便有几个小太监取过铁蒺藜,一边一个按住了如懿和惢心。

如懿满头冷汗,像是无数的小虫子从皮肤的缝隙间一点一点钻出来,慢慢地爬行着,又痛又痒。那几个小太监力气极大,按得她动弹不得。

太后在成翰搬来的紫檀椅子上坐了,慢条斯理道:``哀家也不想动用酷刑。可是如今皇帝和皇后都不在宫是,只剩下哀家一人掌管着偌大的皇宫。若是眼皮子底下出了这样大的事都不顾,旁人多少双眼睛盯着,还以为哀家这个老婆子不中用了呢。少不得你自己做下的事情自己担着了。''

成翰扬了扬下巴,拖着太监特有的尖细嗓音,道:``事有主次,就从乌拉那拉氏起,打到皮肉脱尽为止。''

那铁蒺藜上有数十根寸许长的铁刺,刺尖上闪着锈黑色的光泽,让人不寒而栗。小太监一下正要下去,如懿忙伏在地上道:``太后!太后明鉴!奴婢烧的不是纸钱,不是纸钱啊!''

太后扬一扬脸,福珈便侧身过去,捡起一枚还未来得及烧的纸张展开一看,浑圆的纸片上画着万字不到头的图案,中间却是一句藏传佛教的六字真言。

福珈忙双手捧过给太后一看,果然每一张上都只是六字真言而已。太后微微蹙眉,继而一笑:``怎么是这个东西?\textless{}''

如懿忙磕了头,恭恭谨谨道:``请太后听奴婢一言,圆纸为圆满,与万字不到头的图案相衬,是同一道理。六字真言乃是当年妙应寺的喇嘛大师所授,大师说六字真言是藏传佛教中最尊崇的一句咒语,当初传授时便要奴婢循环往复吟诵,才能功德圆满,消除业障,得大解脱。''

成翰轻哼一声道:``可是今日是鬼节,又是你阿玛那布尔的头七。连伺候你的丫头也说是你的一片孝心。''

如懿不慌不忙,眼中澄澈如镜:``奴婢是一片孝心,但这一片孝心不是对死去的阿玛的,而是对皇太后的诚挚祈祷。奴婢知道今日是中元节,宫中请了雍和宫的喇嘛大师开坛祈祷,心想大师一定会诵读六字真言为太后祈福。奴婢无能,困锁冷宫之中,不能朝夕向太后请安,所以只好趁今日大师入宫祈祷,奴婢也跟随大师功德,念动真言。大师开坛后要将法器经文经幡送上法船焚烧,奴婢自知不能参与,所以只好在这里将亲手所写所诵的真言机械化,只当是放在潜艇上烧了,一尽心意。''

福珈沉吟着道:``回太后的话,奴婢也觉得,若是烧纸钱就该有纸钱的样子,否则烧给了那布尔大人也是无用的。至于七月十五的鬼节,烧这个倒是应景的,无非是没跟着太后和各位太妃太嫔放在法船上烧罢了。''她婉转看了如懿一眼:``倒也不算很违反宫规呢。''

太后的唇角略微浮起一点冷淡的笑意,望着成翰道:``你巴巴儿地跑来告诉哀家说冷宫有人暗烧纸钱违反宫规,如今你可看看,这是什么?''太后的笑容似一朵冰花凝在面上:``还劳动哀家到这种地方来,你可越来越会当差了。''

太后的语气并不严厉,恍若家常闲话一般。成翰却似受不住似的,膝下一软,即刻跪下了道:``奴才无用,奴才妄听人言。''

太后向着福珈微微一笑,神色淡然:``你是妄听人言,不过你是听了谁的话呢?哀家的身边,居然有人不把哀家当主子,而是一心窥伺旁人的心意,想要两面讨好。哀家看他是错了心思。''

福珈低眉垂首,淡淡道:``慈宁宫只有一心侍奉太后的人,没有敢和太后耍心眼的人。成公公,你可是聪明反被聪明误了。''

太后望一望天色,盈然起身:``乌鸦都归巢了,咱们也回去吧。成翰,你就不必走了。''

成翰吓得大惊失色,连连磕头道:``太后,太后饶命!''

太后笑道:``今日是中元节,哀家不会想要谁的命。只是你那么喜欢为人做嫁衣裳,辛苦奔波,那哀家就把步步红莲的刑罚赏赐给你,让你折了双脚,也折不了为旁人尽忠的心。''

太后话音刚落,斜刺里忽然冲出一个人来,举起一把匕首便直刺太后心口。院中地方狭窄,随侍太后的太监宫女都守在门外,成翰吓得早瘫在了地上,身边只有一个福珈,根本是无法防备。

太后吓了一跳,本能地侧身一避,正好避开那劈向心口的一刀。太后毕竟是个养尊处优的女流,更兼有了年纪,躲开了这一刀,下一刀夹着凌厉的风劈面而来,根本是挡无可挡。如懿这一下心慌意乱,若是太后在眼前出了事,那可真真是\ldots\ldots 她下意识地扑了上去,一把推开那近乎疯狂的身影,护在了太后身前。

那人却似疯魔了一般,也不避讳如懿,挥起一刀又扑了上来。如懿死死挡在太后跟前,半分也不退让,眼看着那刀尖已经逼到了下颌,直直地要刺到咽喉里去。太后紧紧攥着她的肩,如懿只觉得自己都要撑不住了,加上雨后地上湿滑,她脚下一滑,整个人斜碰上向后倾去,又避开了几分。

趁着这点空隙,福珈和惢心都赶了上去,拼了死力攥住那人,才拖开了尺许。太后穿着花盆底的高鞋,兀自站立不稳,如懿紧紧扶住了她,连忙问道:``太后,您没事吧?''

太后惊魂未定,一手扶着她的手,一手紧紧按住心口,清白了脸色,道:``如懿,方才那刀尖就在你咽喉底下了。''

如懿大口喘息着,努力平息着胸口的紧张与慌乱,忙欠身道:``太后\ldots\ldots 太后无恙便好。''

趁着福珈和惢心拉住那人的工夫,外头的侍卫们一哄而上,立刻死死按住了那人。太后已经沉稳下来,扶着椅子坐下,喝道:``敢谋刺哀家,哀家倒要看看,到底是冷宫的哪位故人,有这么个好本事!''

福珈应声上去,劈面就是两个耳光,硬生生托起她的下巴来,仔细分辨片刻,道:``回太后的话,真是故人呢。''

太后微眯了双眼,冷笑道:``吉嫔?是你!''

吉太嫔满脸狰狞,声嘶力竭道:``我居然杀不了你!居然还是杀不了你!''

太后清朗一笑,指着天道:``不只你,许多已经上了天下了地府的人都想杀了哀家。可惜呀!''太后抚着身上精心绣制的夔龙牡丹纹样,朗声笑道:``成得了龙的始终是龙,蹦跶得再厉害想要翻龙门的,翻不过还是一条鲤鱼,一辈子困在水里!你从前在外头的时候斗不过哀家,被哀家发落来的冷宫,你以为进了这里反而能斗得过哀家了么?''

吉太嫔的眼底闪过一丝仓皇,态度却依旧强硬:``是吗?刚才要不是有人救你,你早就死在我的刀下了。''

太后仰天一笑,抚着鬓边一朵赤金莲花,轻蔑道:``在冷宫外年轻貌美的时候斗不过哀家,在这里关了这么些年就有指望了么?凭你这点本事,不过就是用蛮力伤人罢了。看来你不管长了多少岁,脑子却一点都没长进!哀家要是折损在你这点微末伎俩里,那才叫天亡哀家也!''

吉太嫔气得脸色发黑,徒然地伸手挠着,却也不过只在泥地上划出几条划痕而已。太后朗然一笑:``福珈,外置了她。别忘了成翰还等在那儿呢。''

福珈答应了一声。太后起身扶住小宫女的手,走了两步回头道:``好好惜命,留待来日吧。''

如懿的身体被惢心紧紧撑着,几乎是要喜极而泣,她的手在衣袖里紧紧攥住惢心的手,两个人手心里全是冷汗,连她自己也不能分辨,是欢喜过后的惊觉,还是劫后余生的痛快。她只知道,唯有握着惢心的手,一个活生生的人的手,她才觉得自己也是活着的。不是冷宫的一块墙皮,一抹青苔。

太后施施然离去,仿佛方才的种种生死惊险,不过是谈笑间一抹云烟。如懿暗暗生出几分羡慕,何时何日,才会有太后这番定力呢?然后未及她细想,福珈已经扬了扬脸,由着几个侍卫将吉太嫔拖进了一间偏殿里。

如懿忙拉住福珈道:``福姑姑,吉太嫔是发了疯了,才会冒犯太后。她只是发疯,不是有意的。''

福珈拍了拍她的手道:``小主,别怪奴婢多嘴。太后的性子便是如此,饶了她一次不死,再敢有第二次,就必死无疑。只怕现在太后心里,正后悔当年留了她一条生路呢。您哪,好好看着,就当太后亲身指点您了。''

她说完,再不发一言,走到偏殿里,看着太后的近身侍卫将吉太嫔用一根粗粗的麻绳吊在了梁上,由着她双脚狂乱地挣扎,喉中发出呜咽的兽般的嘶叫,很快便没有了任何声息。

如懿靠在窗棂上,只觉得冷汗逼透了一层又一层衣衫,依稀恍惚,是她刚到冷宫的时候,那个吊死在悬梁上的不知名的女人。原来熬在这里,不过是这样凄惶地死去,死在自己手里,抑或是旁人手里。

她不知道自己是怎么走回去的,回到空落落的房里,也不顾壶中的水是热是凉,一股脑儿倒在了口中,好像唯有如此,才能安抚自己一颗慌乱的心。外头小太监们责罚成公公的声音渐次低了下去,一开始是惊痛的呼号,哭爹喊娘地求铙,到了最后,只有出气没有进气,彻底没有了动静。

良久,两具肉体被拖出去的声音也彻底消失了。惢心满脸是泪,看着如懿道:``小主,咱们没事了,没事了!''她起身从床底翻出一大包纸钱与冥纸,``还好小主没用这样莫名其妙送进来的东西,否则今天半死不活在那儿受刑的人,就不是成翰,而是咱们了。''

如懿转过脸去,成翰双足留下的血痕在灯笼黯淡的光影下越发显得如朵朵绽放的污泥地上的红莲,一步一血,步步触目惊心。如懿努力地抓着门框,因着被废不戴护甲,手指上留得寸许长的指甲抠在木质的门缝里,有轻微的嘶啦声。她轻声道:``是。差点就中了旁人的计,那么双足残废的人,就是我们自己了。''

惢心静静道:``还是小主警觉。''

如懿蹲下身,取过那包纸钱全部烧了,火光熊熊地染红了她苍白如纸的面颊:``惢心,如果是海兰送东西来,会不通过凌去彻的手自己这样塞进来么?而且送了那么多,好像浑然忘记了上回烧给端慧太子的纸钱还剩下许多。海兰是不会那么粗心大意的。''

惢心犹有余惊:``那小主怎会知道太后会来?''

``有人设了这个局,就是要引出大事来。宫里只剩下太后这个一家之主,冷宫里出了这样违反宫规的事,即便她自己不来,也会让跟前最贴身的人来。那么只要有人来,这个事儿就不怕了,就必定要让人知道,太后身边有为别的主子做事的人。太后岂能容得下身边有这样的耳目,咱们就能脱身了。''

惢心轻轻拍着胸口:``好险好险!奴婢还生怕出了什么差池呢。''

如懿沉下脸,看着微弱下去的火光最终化作了暗黑的灰烬,薄薄地散开,道:``若是不走在刀尖上,如何能走出一条血路来。也是吉太嫔处心积虑报仇,顺手给了咱们这样一个机会。太后既知道了咱们的忠心,又能替她除去不干不净的人,到用人之际,她会想起咱们的。只要有太后惦记,便多了一分出去的指望。''

她站起身将烧完的纸钱灰烬一路洒在成翰双足留下的血迹之上,喃喃道:``阿玛,女儿不孝,只能料理完这些事之后才烧一点纸钱给您。您在九泉之下,一定要保佑女儿,保佑乌拉那拉氏,不要再受凌辱,不要没有出头之日。''她回望着吉太嫔被吊死的偏殿,闭上眼睛:``吉太嫔,我一定不会像你这样胡乱报仇,枉死他人手中的。''

她抬起头,天边墨云依旧,唯有几只昏鸦,啊啊地拍着肩膀,振翅飞走了。

这一阵安稳沉寂,便到了乾隆五年夏末的时候,楚粤苗瑶勾结滋事,皇帝念着苗瑶之事颇为要紧,牵涉亦广,留在圆明园处置到底不便,便下旨回了紫禁城中。而亦如皇帝和太后求子所愿,御驾回銮时,海兰已经怀孕三个多月了。

皇帝继乾隆四年四阿哥永珹出生后,一年之后又再闻喜,怀孕的又是这两年颇为宠爱的海兰,如何能够不喜。加之太医说海兰的身体不够壮健,需得满四月后才能经得起舟车劳顿,皇帝便布置了下来,将延禧宫好好休整一番,再让海兰搬进去住。这一拖,便又得延迟半个月才能回銮了。

海兰有孕,原来也是不动声色,到了三个月胎气稳定才肯告诉皇帝。如此自然是合宫心动,玫嫔与慎贵人犹自尚可,皇帝新宠的庆常在也不过一时的兴致,早被冷落了下来,也没得说什么。最伤心的莫过于慧贵妃,这一年来在圆明园,自是她恩宠最盛,却半点怀孕的动静也没有,只见别人一个个腹中有了骨肉,如何能不伤怀。皇帝虽然也极希望这位得宠十数年的爱妾能有孕身,然而亦是无奈而已。

而这边厢,如懿只盼着上回太后之事可以稍稍助力,却整整一年毫无动静,只是送进来的饭食略有好转,常常一荤一素,不再都是寒湿之物了。因着愁思缠身,因着饮食不思,如懿渐渐地瘦下来。这种瘦是无知无觉的,只是皮肉一分分地薄下去,薄下去,隐隐看得出筋脉的流动。待到夏末秋初的时候,身上因着屋子暑热的痱子褪了下去,手腕却比昔年细了许多,翡翠珠缠丝赤金莲花镯戴在手上,已经能一骨碌地滚到手臂上。她想了想还是取下来搁在了妆台上:``到底是皇后赏的,别摔坏了。''

惢心微敛愁容:``当年皇后娘娘一人赏一串,另一个戴着的人在外头得尽恩宠,小主呢,偏偏被困死在这里。''

正说着,江与彬进来,躬身施礼道:``小主万福,微臣奉旨来给小主请平安脉。''

如懿笑着伸出手腕:``我本以为太医是治病救人的,可是你每每来请平安脉,旁人知道我平安,岂不是给你添堵?''

江与彬淡然一笑,两指隔着纱绢落在如懿手腕上,感觉着她脉搏的跳动:``微臣的责任,只是管照小主的安好,其余的微臣都不必理。''

如懿掰着指头一算,玩笑道:``来得比往日勤,可是冷宫里有什么人牵着你来?''

江与彬看了惢心一眼,面上都有些珊瑚之色。惢心不好意思,便转身去添茶。

江与彬素来是温和的神色:``太后的嘱咐,知道微臣管着冷宫的差事,嘱咐微臣,别让小主七灾八难地难受。''他向着在廊下烧水的惢心微微一笑:``惢心姑娘可以闲些了,除了旧疾,小主一切安了。''

惢心脸上一红,旋即淡然道:``可是奴婢觉得小主瘦了许多。''

``清瘦是福,若过于丰腻,反而引发种种病端。''他笑意澹澹,``后宫最近添了一桩喜事,想来小主听了也会喜悦。''

如懿含笑道:``什么?''

``海贵人在圆明园有了身孕。''

如懿大喜不已,却被更多的担忧覆没:``你要她万事小心。''

江与彬唇角含了一缕笃定的笑意:``海贵人的胎都落在微臣身上,如今快四个月了,胎像已经稳当,别人要做什么,怕也难了。''

如懿按着心口,露出一比欣慰的笑容:``那就好。''她想一想,取过妆台上的翡翠珠缠丝赤金莲花镯:``我身边再没有比这更贵重的东西了,这还是当年皇后赏的,替我送给她,留在身边,当个念想。''

惢心劝道:``小主总有出去的日子,要被皇后知道拿这个送了人,怕是不好。''

如懿凝神片刻,笑道:``这串东西算是跟了我最长久的。只别让人瞧见就好。''

江与彬伸手便要去接,哪知手上一个不稳当,那赤金莲花镯便落在地上。那镯子本是用大颗的翡翠珠子串成,因着翡翠易碎,每颗珠子两头皆用打成莲花形状的赤金片护住,翡翠珠身上绕以藤蔓形状的绞金丝。谁知堪堪落在砖地上,其中两颗便落了个粉碎。

惢心心疼得直念佛,忙蹲下身捡起来道:``可惜可惜,这碎的两颗拆下了,戴在手腕上就会觉得紧了。''

如懿道:``也罢了。反正咱们出不去,碎了也没人看见会怪罪。''

正说着,惢心轻轻``咦''了一声,掰开那珠子碎裂的地方,里头竟掉出一颗小指甲盖大小的黑色珠子。惢心对着光线一瞧,奇道:``有很淡很淡的香味,只不知是什么?''

如懿接过一看,自己也是全然未识。

惢心只撇嘴道:``皇后娘娘也太节俭了,说是赏的翡翠珠子手镯,结果里头大半不是翡翠的,竟是旁的东西,枉咱们还一直宝贝似的戴着。''

如懿道:``这种外邦进贡来的东西,有什么缘故还真不好说。''

江与彬见主仆二人皆是茫然沉吟,便道:``小主若放心,请给微臣一瞧。''

如懿递到他手中,笑道:``女儿家的东西,江太医也都识得么?''

江与彬仔细看了看,放在鼻端嗅了一会儿,又取过惢心掌心那些碎了的翡翠珠片看了,敛容正色道:``女儿家的东西微臣不一定都识得,但这种医家的东西,却是一看就明白了。''

如懿听得这话不大好,心中陡然一沉,便道:``江太医不是外人,有什么话不妨直说。''

江与彬将摔碎的翡翠珠取过拼成完好的形状,道:``小主可以看见,这颗翡翠珠子是事先雕琢好空心的,然后将想塞进去的东西塞好风干,再按着眼子留下穿孔的线,从外面看它就只是一颗翡翠珠,而百其了。''

惢心道:``你这话说得不明不白的。这到底是什么东西?''

江与彬的神色有些难看:``有一种草木叫零陵香,《嘉祐本草》中说零陵香味辛,温,微毒。多用则壅关节,涩荣卫,令血脉不行。气为血之帅,血为气之母。尤其女子,若气血滞缓,便不易有孕。零陵香香气浓烈,可煅烧后研磨成粉,除去异香,再制成稠厚的黑褐色软膏状,可随意挤入物体之中,待到风干硬化,便成了这一件天衣无缝的东西。这翡翠珠两孔之外都封着孔眼更小的金莲花片,又在珠子上缠以金丝,表面看来是为增其华丽美观,其实是保护翡翠珠不摔碎,不让里面的东西露出来。这般的心思,的确是比能工巧匠更厉害上百倍了。''

\hypertarget{ux7b2cux4e8cux5341ux4e8cux7ae0-ux91cdux9633}{%
\chapter{第二十二章
重阳}\label{ux7b2cux4e8cux5341ux4e8cux7ae0-ux91cdux9633}}

如懿怔怔的,唇上的血色慢慢褪了去:``零陵香?所以我一直未能有孕,是么?''

江与彬神色沉重:``气血滞缓,手腕上脉象起伏最厉害。若未见此零陵香丸,微臣也会以为是小主本身体质的缘故。这零陵香日积月累缓缓侵入肌理,牵一发而动全身,不知小主戴了多久了?''

如懿木在当地,觉得嘴唇都不是自己的了,麻木地微微张合:``我嫁与皇上为侧福晋那一年,安南国进贡的贡品,皇上送了富察皇后,皇后再转赠给我和慧贵妃的。算来,也已经十来年了。''

江与彬语中带了沉沉的叹息,道:``这十来年,小主无一日不戴在身边?''\textgreater\_\textless''

如懿只觉得头有千斤重,艰难地点下:``是。福晋所赠,她后来又贵为皇后,这是她所赏赐的最贵重的物品,也一向被皇上视为是妻妾和睦的象征,怎会不戴着?''

江与彬面色极为难看:``零陵香最早出于西南,当地人常用此物或佩戴或煎服,有娠者可断胎气,无娠者久难成孕。此物本就不多见,又藏得如此精巧,难怪小主不知。''

心中像被无数利爪撕挠着,一道道血淋淋的印子淋漓而下。是她蠢,蠢到那样的地步,被人算计了十来年,却懵然其中,迟迟未知。

惢心咬着唇,唇上几乎要沁出血来:``这东西是安南国的贡品,总不会送来的东西就有不妥吧?''

如懿的声音极低,像是虚弱到了极处,自己强撑着自己一般:``你也知道这是安南国的贡品,贡品是给先帝的,最后落到谁的手里谁也未知。安南国的人怎会费这种无的放矢的心思。我却是记得的,当年皇上把这串镯子给富察琅烨,富察琅烨自己留了几日才给我和慧贵妃的。''她心头一滴滴坠着血,那艳红一色,原来十来年日夜期盼,心思枉费。她低低冷笑一声,那声音如清碎的冷冰,划破了自己的腔子,划碎了心肝肠肺,涂然一地。

也好,也好,她混在海兰和纯妃身后,杀了皇后的孩子,皇后也让她的孩子一直来不了人世。后宫倾轧,生死相拼,当真是一报还一报。

如懿死死咬着牙,滚热的泪烫在眼眶里咝咝灼烧着,她拼命仰起脸,忍住,再忍住。已经失去的,何必再为之落泪,眼泪落下来不过是湿了自己,还不如让它流回去,灼伤了心,记得那痛,便不会再心软。

如懿忍住泪,缓缓道:``慧贵妃多年来顺从皇后,一心依附,可怜她竟和我一样,膝下空空。也枉费了她屈居人下,看人颜色。''

江与彬露出几分踌躇之色,还是道:``小主要听微臣一句实施么?\textless{}''

如懿道:``你说就是。''

江与彬叹道:``若细细论起来,慧贵妃可比小主可怜多了。''

``可怜?''如懿叹了一声,死死掐着自己的手指,``活在算计之中,刀锋之上。后宫之中,何人不可怜?''

江与彬有脸色并不大好看,道:``慧贵妃一直身有旧疾,时时离不开太医。一则是因为和小主一样,手上戴着这个东西。另一则,慧贵妃求子心切,曾经召集太医院所有太医为她诊脉。微臣就是那一次为贵妃搭过一次脉,贵妃的脉象是气虚血瘀之症,而且非常严重。''

``严重?''如懿疑道,``不是一直有最好的太医为她调治么?怎么反而不见起色?''

江与彬道:``小主这样想便是了。为什么贵妃一入冬就那么怕冷,夏天又出虚汗,面色淡白,身倦乏力,气少懒言,烦躁易怒,胸肋疼痛如刺,月事也乱不调,每每月事至,则绞痛不已。皆因淤血不去,新血难安,血不归经而发,长此以往,如何会有胎气凝聚?''

如懿微微一滞:``你是太医,才诊了一次脉就发觉了,齐鲁为太医院判,素日为贵妃调理,他会不知?''

江与彬的面上闪过一丝意味深长之色:``小主所言,才是最值得斟酌之处。病症显而易见,却越治越病,当中的缘故\ldots\ldots{}''

如懿矍然变色:``齐鲁没有这么大的胆子!''

江与彬满面恭谨,平静道:``娘娘所言甚是。但是那一回会诊,太医院所有太医却都长了同一条舌头,慧贵妃的病是胎里带来的,如今虽然见好,但根子还在,一时未能清除。而那日所有太医一起开的那张药方,更是一张要紧的药方,但凡按着那个方子服药,表面看着症状会有所减缓,其实就像在寒冰上面泼热水想化了那冰,外面看着冰是化了些,但耐不住慧贵妃的体质便是个大冰窟,再多的水扑上去,一会儿就冷住了,反而冻得更厉害,等到哪一天受不住了,便冻得元气大伤,那便无疑是饮鸩止渴了。''

如懿心头狠狠一抽,一阵爽利的快感过去,亦是凄凉。其实比之皇后,这些年来她与贵妃高晞月的明争狠斗才最是厉害的。一路从潜邸过来,争着荣宠,争着位分,此消彼长,你进我退。虽然此时此刻,她身在冷宫朝不保夕,可是在外备受恩宠的高晞月,也并没有好到哪里去。

那恨意慢慢地积在胸腔里,积得久了,便成了一把利器,钝钝的,带着锈,一下一下割着。从前,是她无用;可是往后,断断不能再无用下去了!

待得皇帝回銮时,海兰已经有四个月的身孕,因着初初回宫忙碌,皇帝之前又连着折损过两个孩子,对海兰的胎便万分看重,身边足足添了一倍的人伺候,动辄便是一群人跟着。之后又正逢着皇帝的万寿节并中秋、重阳三节,节下热闹,海兰也不宜多出宫,越发见不得如懿一次了。

这一日正逢着是重阳,皇帝自登基后便待太后十分亲厚,孝养有加,又兼太后掌着后宫之事,所以这一年的重阳节过得格外热闹。按着宫的规矩,九月重阳的正日,皇帝亲自陪着太后到万岁山登高,以畅秋志。这一日,皇宫上下要一起吃花糕庆祝。那花糕是各宫嫔妃亲自做了进献太后的,自然各出奇招,大致有糙花糕和细花糕两种。糙花糕的皮上粘了一层香菜叶,中间夹上青果、山楂、小枣、核桃仁之类的糙干果;细花糕层数颇多,每层中间夹着较细的蜜饯干果,诸如苹果脯、桃脯、杏脯、乌枣之类,都做成金钱大小,十分精致。到了夜间,太后兴致颇浓,便按着皇帝外赏百官花糕宴的规矩,也在重华宫宴请帝后嫔妃,皇帝生性爱热闹,自然更加凑趣。夜宴以重阳花糕做成九层宝塔状,上缀两小羊以合重阳(羊)之意,与诸人插茱萸,饮菊花酒,欢欣畅饮。

酒过三巡,歌舞之乐也沉沉缓下去,静夜的凉风一重重拂上身来,多了几分蕴静生凉,摇曳得满地黄花灿烂,亦生了几分消瘦憔悴之意。皇帝添了几分沉醉的酒意,望着墨玉般的黑沉天际,一轮昏黄的弯月寂寞地别在黑色幕布上,连星子亦光彩黯然。皇帝唇角带了一抹淡薄而倦怠的笑,道:``年年月月便是歌舞,也实在是无趣得紧了。''

皇后笑道:``那一曲《桃夭》,臣妾记得是皇上最喜欢的。常说妙龄女子素颜红裳,恰如桃之夭夭,灼灼其华,令人赏心悦目。''

皇帝轻轻一嗤,喝尽盏中的酒,道:``宫中宴饭饮常用梨花白,今日饮菊花黄,才有新意。这歌舞朕虽然喜欢,可是看多了也生腻烦。皇后不明白其中的道理么?''

皇后脸上微微一黯,很快还是笑道:``皇上总喜欢别出心裁。''

太后抚了抚鬓边的祖母绿赤金凤缕珠步摇,摇头道:``别出心裁也罢了,若能新颜常在,侍奉君王之侧也是好的。''她看向皇帝道:``皇帝,哀家去岁赐予你的新人陆氏伺候了你才一年,一直还是常在之位,是不是不合皇帝你的心意啊?''

皇帝微微一笑,只是不置可否:``皇额娘垂爱,儿子心领了。''

皇太后微微垂下眼睑,很快朗然笑道:``皇额娘本想你身边有个可心可意的人好好伺候你。若是陆氏不好,就在常在的位分上慢慢熬着吧。身为嫔妃,不能讨皇帝欢心,那就是多余!''

这话说得不轻不重,可是落在在场的嫔妃耳朵里,却是俱然一凛,不觉收敛了神色。太后笑得和颜悦色:``如今是秋日里了,再舞春日桃花盛开时节的《桃夭》,未免不合时宜。皇帝,咱们便换一支歌舞吧。''

皇帝奉起一杯酒:``但凭皇额娘做主。''

太后澹然一笑,抚掌两下,却听丝竹声袅袅响起,幽然一缕如细细一脉清泉潺潺,如泣如诉,慢慢沁入心腑。却见满地各色菊花丛中,悠然扬起一女子纤细翩然的身影,踏着丝竹轻缓而来。那女子玉色纻罗缦衫,淡淡云黄色长裙飘逸如轻云明月,清素衣衫上只绣着朵朵秋菊,也不过寥寥清姿,并不用繁复的绣线堆簇,她堆起的高高云髻上只簪了银色绞丝菊流苏,不细看,还误以为是月光将花影落在了她身上,风吹起她衣衫上的飘带,迤逦轻扬,灼烁生辉,转袖回眸间凉风暗起,身姿空灵。她的嗓音柔缓,伫立在这静好的月色之中,侧身依依念道:``薄雾浓云愁永昼,瑞脑销金兽。佳节又重阳,玉枕纱橱,半夜凉初透。东篱把酒黄昏后,有暗香盈袖。莫道不销魂,帘卷西风,人比黄花瘦!''

那是一阕李清照的《醉花阴》,待她念到最后一个``瘦''字时,余音袅袅飞扬而去,几乎是飞到了遥远的碧海青天,被流去遏住,幽绝缠绵处,不必知音如李清照,也早湿了半幅青衫,为之戚然。她的身子慢慢地低旋下去,低旋下去,成了袅袅的藤蔓轻缠,一直落在了散开的裙裾之间,像是捧出一朵玉色晶莹的花朵,盈然招展,风姿眷眷。

银瓮潋滟浮红颜,翠袖殷勤捧玉钟。原来满目繁华,只为衬得伊人遗世而在。

皇帝忍不住抚掌笑道:``舞低杨柳楼心月,歌尽桃花扇底风。朕原以为歌舞曼妙已经极佳,不承想凌波微步、踏歌吟诗更是清新隽永,只是这样好的才情,这样美的舞姿,不知长相如何,是否曾与朕梦中相逢?''

太后微微一笑,唤道:``皇帝吩咐,还不走近来?''

那女子缓步上前,施了一礼,抬起头来。皇帝触目处,只见那女子神色清冷,却有一番艳绝姿态,修蛾曼睩,貌殊秀韵。

慧贵妃蹙了蹙眉头,似是赞叹,似是嫌恶,冷冷道:``蛾眉玉白,好目曼泽,时睩睩然视,精光腾驰,惊惑人心也。''

皇帝赞许地看她一眼:``这是王逸的《楚辞》注,贵妃好才学。''皇帝的赞叹不过一声,甚是潦草,旋即被那女子吸引。那女子盈盈笑时嘴角微微扬起,似乎是新月般的笑颜,却没有丝毫温度。但若说她是冷淡,偏偏那眼波流转,又觉得她眉目绚然,是在含羞顾盼着你。

皇帝侧首笑道:``皇额娘精心挑选的人,念的是李清照重阳思君的《醉花阴》,果然很合时宜。''

太后眉心微微凝了一丝笑色,缓缓道:``合不合时宜,哀家说了不算,皇帝说了才算。''她凝声道:``这丫头是侍郎永绶之女,满洲镶黄旗人,出身亦算贵重。''

皇帝颔首,柔声道:``上前来吧。''

慧贵妃眉头一锁,旋即含笑娇怯怯道:``皇上,重阳喜日,歌舞娱情助兴才好。念什么诗词,冷冷清清的。''

皇帝恍若未闻,只看着那女子道:``今夜歌舞甚好,为何只念诗词?''

那女子垂着脸,声音却不卑不亢,毫无献媚或畏惧之意:``臣女不喜太过热闹的歌舞,倒觉得古人的诗歌有蕴藉,须细细品味才得意趣。臣女素闻皇上秉圣祖文心之质,善于吟咏,以为会得知音之感。''

皇帝眉梢眼角都是舒展的笑意,问道:``你叫什么名字?''

那女子低垂眼眸,柔声道:``意欢。''她停一停:``是心意欢沉之意。''

皇帝的目光如春日沉醉的晚风,绵绵道:``古人男女相悦,女子对情人的称呼便是欢。这个名字,很有情致。''

意欢有星子般的眼眸,此时眸中如寒夜里明灿的星,骤然亮起,情意宛然,低低道:``是,皇上博学。臣女平生最喜《相见欢》一词。''

``朕与你便是相见欢了。''皇帝的笑如清亮的阳光,无遮无拦洒下,他停一停道:``你姓什么?''

慧贵妃撇嘴道:``这样的名字,多半是个汉军旗的出身姓氏罢了。''

嘉嫔掩口笑道:``还是慧贵妃最明白什么是汉军旗的出身了。''

慧贵妃脸色一冷,转脸不顾。

意欢沉沉道:``叶赫那拉氏。''

皇帝微微一怔,唇边的笑意如遇上了寒雨微凉。皇后已然带了一抹意味深长的笑:``叶赫那拉氏?''

嘉嫔``哎呀''一声,以袖掩口,惊奇道:``叶赫那拉氏?可是被我建州女真所亡的叶赫那拉氏?''她盈盈望住皇帝,娇声道:``皇上,臣妾虽然来自李朝,却也听说当年叶赫部为我太祖努尔哈赤所灭,叶赫部首领金台吉临死前悲愤不已,曾说道叶赫那拉即使只剩下一个女人,也要灭亡建州女真,不知是不是真的?''

慧贵妃见意欢脸上有不豫神色,不觉拈起绢子笑道:``嘉嫔虽然来自李朝,可是对咱们爱新觉罗家的典故还知道不少呢。''

嘉嫔扬了扬唇角,颇有得色道:``可不是?既然身为皇家儿媳,自然事事以皇家为重了。''

皇后含笑颔首:``嘉嫔生下了皇子,果然越发懂事得体了。''

太后不以为意地笑笑:``往日传闻,你们倒是听得有心了。只是叶赫部被我建州女真灭了那么多年了,早已臣服。意欢的阿玛好好地当着皇帝的侍郎,她一个女孩子家,哀家倒不信能成了精了?皇帝,你说呢?''

皇帝微笑着伸手向她,语气柔缓温存:``朕记得,太祖的孝慈高皇后便是叶赫那拉氏,还替太祖生下了太宗,可谓功传千秋啊。''

太后眉毛微微一扬,和缓笑道:``意欢,还不谢恩?''

意欢盈盈下拜:``臣女多谢皇上夸赞。''

皇帝笑道:``朕倒不是夸赞,叶赫那拉氏出身满蒙贵族,却不想将汉人的诗词念得这样婉转动听,真是难得。朕记得宫中通晓治家诗文的,除了慧贵妃,便是\ldots\ldots{}''

他微微一滞,并没有再说下去,只是自斟自饮了一杯,向海兰道:``海贵人,你有着身孕,拣自己爱吃的多吃些吧。''

海兰知道皇帝想起了谁,便作不知一般,笑道:``旁人不说,如今这位意欢妹妹,也是极通读书的。''

意欢眸若秋水,盈盈一荡:``皇上通晓满蒙汉文字诗史,难得在皇上跟前伺候一次,不能做了什么都不懂的人。''

皇帝笑着挽着她的手:``既然你如此有心,你便也留在朕身边,做个贵人陪伴吧。''

皇后先起身举杯道:``皇上自登基以来,册封的嫔妃大多是从答应,官女子做起,如今叶赫那拉氏一举得卦贵人,可见皇上钟爱,臣妾敬皇上一杯,贺皇上新得佳人。''

嫔妃们虽有不甘,亦只得跟随起身,贺道:``恭喜皇上。''

皇帝一饮而尽,嘱咐了叶赫那拉氏伴在身边。那叶赫那拉氏对诸人神色都是冷冷的,唯独对着皇帝时温柔凝睇,一笑如冰上艳阳,冷清中自有艳光四射。

皇后微微使一个眼色,慧贵妃起身娇笑道:``皇上看腻了旧歌舞,咱们这些做旧人的不能不胆战心惊,臣妾只好就想些新鲜法子希望皇上不要厌弃了。''

皇帝笑盈盈望着她,眼底尽是温然的情意:``又胡说了,朕怎会厌弃你?''

慧贵妃嫣然一笑,百媚横生,指一指天上道:``今天新人且歌且舞,咱们地上尽够热闹了,臣妾的父亲从外头送来各色烟花,咱们且看一看天上的热闹吧。''

皇帝颔首道:``烟花不错,只是怎么想起这个来了?''

慧贵妃温柔凝眸,鬓边的一支并蒂海棠花步摇安静垂落,道:``臣妾往日读《少年游》,记得有一句`雨晴云敛,烟花澹荡,遥山凝碧。驱车问征路,赏春风南陌,'可不是就了如今的景么?''

皇帝颔首道:``还是你最解情致,一点小玩意儿,都能答出那么多细腻心思来。''

慧贵妃扬一扬脸,身边的双喜赶紧下去了。不过片刻,只见乌沉沉的墨色天空,忽然划过一道流星般的白光,仿佛一声尖锐的呼啸,五颜六色花旋即绚烂飞起,整个夜空几乎被照得亮如白昼。

慧贵妃一一指着道:``那红的是天女散花,黄的是武松打虎,金猴献果,这几个五彩的是八仙过海、金辉齐鸣、铁树开花、百花齐放。皇上看那个,最别致的杨贵妃观牡丹,还有白蛇仙女、百鸟朝凤、金龙腾飞。''

慧贵妃说一句众人便赞一句,那烟花似颗颗明珠在空中绽放,朵朵变化绚丽,如彩蝶飞舞,纷纷飘然。正喧腾间,只见一朵硕大的烟花绽放在空中,散出满天云霞,金芒似的火星四散飞落开去,远处歌姬们的管弦声以及嫔妃和宫人们的叫好鼓掌声,熙熙攘攘混在一起,将今夜的喧哗热门推到了最高处。

待到烟花尽了,唯剩了满天空的寂寞和宁静,空气里散着淡淡的硝烟味,微微有些呛人。

皇帝回首叶赫那拉氏只是淡淡的神色,便道:``怎么?不喜欢么?''

叶赫那拉氏为皇帝斟了一杯酒,浅浅笑道:``烟花好看是好看,热闹也热闹。只是做人若只是热闹了这一刻,便要回归寂寥,还不如清清静静,做天上一点星子,虽然是微光,却永远明亮。''

皇帝眼中闪过一丝明亮,看向太后道:``果然是皇额娘调教出来的人,见识卓然,与众不同。''

太后眼底精光一闪,和言道:``哀家放她在身边,能调教的不过是规矩罢了。心思,还是她自己的。''

皇帝闭目片刻,含笑道:``叶赫那拉氏的心性,倒是和皇额娘亲生的两位公主一样,让朕想起远嫁的大妹妹端淑长公主了。''

太后神色微微一滞:``端淑长公主在皇帝登基前便已许嫁了蒙古,只剩下柔淑长公主还待字闺中,一直交给庄亲王夫妇教养。哀家也不能常常得见。''

皇帝沉吟片刻道:``那是儿子不孝了,未能顾及皇额娘母女情深。''

太后一凛,旋即笑得柔和:``皇帝何必自责?庄亲王夫妇忠于皇帝,又是皇帝的亲叔叔,必然会替哀家好好教养公主。何况,庄亲王福晋又是出了名的贤德淑女呢。''

``儿子也这样想。皇额娘身边有儿子和这些媳妇,都会孝顺皇额娘的。逢着大年节,公主也会随着庄亲王夫妇进宫,拜见皇额娘,皇额娘一切放心就是。''

皇帝恭谨一笑,转头看着叶赫那拉氏,颇为欣赏,``你说话很能让朕舒心,朕便赐你封号为舒,赐住储秀宫。往后,你便是朕的舒贵人了。''

叶赫那拉氏笑意浅浅,神色平和如镜:``臣妾谢过皇上隆恩。''

皇帝执过她手,相看不厌。却见皇帝身边的小太监进保一脸惶然地急匆匆进来,打了个千儿道:``皇上,不好了,不好了!冷宫走水了!''

\hypertarget{ux7b2cux4e8cux5341ux4e09ux7ae0-ux706bux711a}{%
\chapter{第二十三章
火焚}\label{ux7b2cux4e8cux5341ux4e09ux7ae0-ux706bux711a}}

如懿并没有想到火会突然一下烧起来。一开始,她不过是和冷宫那班妇人一般。站在各自的廊下,看着烟火满天,缭乱夜空。这一夜的风正好是吹向冷宫的方向,把原本遥远而璀璨的烟火在空中带得更近她们一些。真是现世的繁华,虽然越发衬出她们的孤清寒苦,可还是忍不住去看,去向往。

如懿自嘲的笑笑,哪怕被禁闭在此这么长时日,但红尘万丈,浮世虚华,她从未自心底放下过。

第一年的心如死灰,第二年的隐忍后激发的心志,到了第三年,她反而有些和缓。虽然,走出这个困笼的念头日复一日地强烈,可是她明白,一切急不来。

就如冬日里手上脚上的冻疮,夏日里满背的痱子与蚊包,知道必须得过了这个季节,才会好起来。''\textgreater\_\textless"

惢心走过来,嗔着道:``小主,今晚本来是凌云彻和赵九霄当值的,奴婢还想叫他们一起看烟花呢。谁知道那两偷懒的家伙,不知道跑哪去了,连个人影也没有。''

如懿笑道:``每逢佳节倍思亲。也难为他们年年岁岁都守在这儿,由得他们去吧。''

那火苗,就是她在说完这句话的时候''嗤``地燃起来的,毫无预警地,几乎是整个屋顶,都轰的燃烧起来,那火势之快,几乎是窜到哪里哪里就烧了起来。冷宫里阴湿霉冷,那火势却毫不受阻,燃起一股焦霉的味道。惢心大惊,立刻将如懿护在了身后,大呼道:``来人呐!来人呐!失火了!''

满宫里的女人们都着了慌,有几个聪明的,便先抢到了院子里,赶紧去看水缸里有没有积着的水。宫中为防失火,也为乐蓄积天雨,总是在院子里和殿前的廊下放置些铜缸,女人们被这愈演愈烈的大火吓坏了,忙不迭伸手捞起缸中的瓢舀了水一勺一勺泼出去,奈何地上墙上都已着了火,加之许久不曾下雨,缸里本就没多少水。如懿冲到门前,大力拍击着宫门道:``救人啊!救人啊!有人在吗?有人吗?''

她喊了几句,便被滚滚的浓烟呛住了嗓子。凌云彻远远站在庑房门外,和赵九霄、张宝铁、包圆一起垂着手跟在头领李金柱身后。

赵九霄看着火势越来越大,踌躇着道:``头儿!这火烧成这样,咱们真不去救人吗?万一那帮女人全烧死在了里面''

李金柱一脸肃杀,按着腰间的长刀,道:``她们活着的时候就是先帝和当今厌弃的女人,吃这食粮,费着衣着,活的也不体面,倒不如一把火烧死了,一了百了。咱们哥儿也落得清静,不必在这冷宫外受罪熬苦了。''

包圆道:``头儿的意思是''

李金柱瞥了包圆和张宝铁一眼:``冷宫没了,还要咱们这些冷宫的侍卫做什么?自然有更好的去处了。\textless{}''

赵九霄仍是有些害怕:``可是若上头怪罪下来,冷宫失火丧命,也是不小的罪名啊!''

李金柱仰头看着这火势,沉着脸道:``在宫里当差久了,你们好歹也有点眼色,长点见识。你看看这火起来的样子,要不是有人先预备下的,冷宫这地方,能起这么大的火么?你再想想这宫里,有几个人敢烧了冷宫的。便是那样的身份。咱们就得罪不起,若再坏了别人的好事,这脑袋就不在自己脖子上了。''

赵九霄有点怯怯的,听着冷宫里惊惧的哀嚎声越来越凄厉,忙着用袖子堵住了耳朵。不敢再听。凌云彻双手紧紧握着刀把,下意识的往前走了一步,因为他分明听见,有人在唤他的名字,向他呼号求救,他紧紧攥着刀把的手,手背上青筋暴突,那是小主的声音,还是惢心?他一时辩不出来,只知道她们一定是怕极了,才会这样喊着自己的名字求救,他忍不住又走上前一步,李金柱瞥了他一眼:``上次被人打成那样,还不记得教训么?在这宫里坚持着,多一事不如少一事。何况是你惹不起的主儿。''

凌云彻咬了咬牙,跪下道:``头儿,您仔细想想,咱们不能不去救人哪。冷宫里的女人不多,就那十几二十个,没人看得上她们,可真要是死了,头一个罪名便是落在咱们五个人身上,哪怕您说的主儿咱们惹不起。但宫里任何人一个主儿怪罪下来,咱们更惹不起。到时候冷宫一把火,再加上咱们兄弟五个的脑袋,就真的是死无对证了。''

张宝铁看了看凌云彻,再看了看李金柱,有些拿不定主意:``头儿,小凌说的好像也有几分道理。毕竟这事不是上头吩咐下来不要咱们理会的,。那个''

凌云彻恳求道:``头儿,旁人也罢了。最后进来的那个,是孝敏宪皇后的侄女儿,虽然是失宠了皇上不要她了,可到底是皇亲国戚,真出了什么事儿咱们也扛不起啊。''

李金柱显然也是被说动了,却迟疑着不肯再发话。凌云彻听着里头的呼声越来越惨烈,再也忍不住,起身抱了一桶水便冲了出去,赵九霄犹豫片刻,也跟着闯了出去。

张宝铁一惊,张了张嘴:``头儿他们''

李金柱摇头道:``他不听劝,也没办法,只是今晚是他们俩当值,要真出事了,他们是首当其冲,去便去吧。这样也好,万一得罪了哪一边,咱们都不会死绝了。''

凌云彻好不容易打开了冷宫的大门,一闯进去几乎是吓了一大跳,因着廊下堆着草垛,门窗又朽烂了,烧的最厉害。浓烟滚滚中,他绊倒了几个人,衣角头发都着了火了,他吓得半死,赶紧把那桶水洒了点在她们身上,一边咳嗽着呛着烟,一边往里头搜寻如懿和惢心的踪影,他寻了半日,只见如懿和惢心所住的屋子烧的最厉害,大半已经烧毁了,人影也没一个,他心底一慌,难不成当真被烧死在里头了。他有些不甘心,不由得唤道:``小主!惢心!小主!''

有微弱的呻吟从附近传来,凌云彻听得声音熟悉,不觉直闯过去,那一间是素日吉太嫔所住的殿阁,自她死后,便已荒废了。眼下看来,却是那里火势最小。凌云彻抱着最后的一丝希望直冲进去,只见殿门后的角落里,两个浑身湿透的人瑟瑟缩缩躲在那儿,已经被烟呛得快要昏迷了过去。

凌云彻看得了是她二人,心头大喜,正见赵九霄寻了进来,忙招手唤了他过来,一人一个背了出去,才背到冷宫的门边,只见前头灯火通明,两队侍卫驾着水龙匆匆过来,对着冷宫的火便架起水龙直喷上去,凌云彻累得精疲力竭,却忍不住微笑出来,大大地松了口气。

如懿闻干净清醒的空气,脑中稍稍醒转,触目便见云彻焦灼的脸,她心头微微一松,仿佛整个人都落在了实处,情不自禁道:``如懿谢过。''

凌云彻拿着手帕绞了替她擦着被烟熏黑的脸,低低道:``我还以为你的名字就是小主,原来你叫如意,是万事如意么?''

如懿吃力地摇了摇头:``嘉言懿行,是美好的意思。''

凌云彻嗤笑道:``能把你们全须全尾地救出来,就已经很美好了。''

如懿看着昏沉沉的惢心,伸手将她搂在怀里,惑泣道:``多谢你,肯来救我们。''她看着喷起的水龙,犹疑道:``只是这火起的太奇怪了,你贸然过来救我们,会不会连累你?''

凌云彻看着远处忙碌的侍卫们一个个将冷宫的女人搬出来,眉宇间微微松弛:``我也很捏了把汗,不知道该不该救你,但看到皇家的水龙过来,就知道没有救错你们。''他看看周围,低声道:``我和九霄去帮忙,你们好好歇着。''

如懿点点头,看着他离去,仰面深深呼吸片刻。这是她三年来第一次走出冷宫,哪怕她知道片刻后自己还是要回到那困地里去,可是多么难得,外面的星光看着和里头也是不一样的,她深深地吸了口气,紧紧地握住了自己的手。

随着火势消减,她靠在墙边,看着明黄色的九龙仪仗渐渐逼近,一颗心忍不住突突的跳了起来,几乎要蹦出自己的腔子。泪水迷蒙了双眼,她是认得的,那再熟悉不过的九龙明黄仪仗,是他,是他来了。

不只是皇帝,还有皇后,他们远远的站着,看着火苗被水龙压得一分分低下去,方才松了口气,却是皇帝身边的李玉也发觉了她,轻声道:``皇上,那墙根底下靠着的,好像是''

他本觉地没有再说下去,却足以让皇帝注目。皇帝沉吟片刻,还是向她走来。那一刻,如懿说不上是喜是悲,仿佛所有的爱恨与积怨都一一淡去,他依旧是当年的翩翩少年,策马兰台,向她缓缓走来。

泪水模糊了双眼的一刻,她拥着惢心,紧紧卷缩起自己的身子,靠在泥灰簌簌抖落的墙根脚下,想让自己尽量缩成让人看不见的一团物事,哪怕是墙根底下不见天日的苔藓也好。是,她是自惭形愧,他的身边,是风华正茂、懿范天下的皇后,而她,却如此狼狈,落魄可怜。

她拼命低下头,终于,在一步之外的距离,分明的看到他明黄色袍襟下端绣江牙海水纹的图样,那是所谓的``江水万里'',她已经许久许久没有看到过了。

那人如一幢巨大的阴影停留在她的面前,遮挡所有的光线。不远处的一切都淡淡地模糊下去,成了虚幻而遥远的浮影。她隐隐听着皇后焦急的声音在唤:``皇上------''那声音却是让所有人都无动于衷。

透明的火光在他身后,映照在被风鼓起的翩然衣阙上,浮漾起一种遥远而虚浮的光泽。他静默着走上前,如懿亦静默着蜷缩成一团。只有甬道内的风,无知无觉地穿行游荡,簌簌入耳。

他俯下身来,将身上的赤色披风兜在了她身上,手指轻柔地替她拂开脸上湿腻腻的碎发,轻声道:``入秋了,别冻着。''

那样轻柔的口吻,清越宛若天际弯月,仿佛是带着花香的月光,静谧而安详地散开四周难以入鼻的气味,静静弥散。仿佛还是昔年初见的时候,他也用那样的语气唤她:``青樱妹妹。''

她微微点了点头,别过脸去:``别看我,给我留点颜面,别看到我这样狼狈的时候。''

他亦颔首:``无论过了多少年,你在朕心里,还是那个好强的妹妹。''他仰起身,轻声而郑重:``青樱,保重。''

这一刻,他唤她``青樱'',而不是``如懿''。是往年欢好如意的青樱,彼时,他们还年少,心意沉沉而简明。而不是``如懿'',那个在后宫中极为自保,出尽谋算的小小妃嫔,那个受尽委屈,被他发落至冷宫的失宠女子。

青樱,弘历。那是他们最好的一段岁月。

可惜,都已经过去了。

他转身便走,没有丝毫留恋,到了皇后身边,淡淡道:``人员无伤,回去吧。''

皇后口中答应,忧心忡忡地看着他先行而去的背影,回头瞥了眼无比狼狈的如懿,有一丝怨恨深深地掩在了眼底。

这一场大火来得突然,冷宫虽无人烧死,却烧伤了好几个。幸而也算发现得早,但冷宫一半的房屋也被烧毁了,太后和皇帝为着重阳失火,几乎是大发雷霆,然而查来查去,也不过是那日的风势太猛,吹落了烟花所致。慧贵妃急切难耐,又怕皇帝怪罪,在养心殿外跪着脱簪侍罪。皇帝倒也不肯责怪她,安抚了几句便也罢了。

江与彬冷冷嗤笑:``是么?幸而只是烧伤了几个人,没得烧死什么,否则也难以掩盖这件事了。''

如懿笑笑:``敢做这样事情的人,绝对能有本事掩的过去。''

江与彬道:``只不过皇上最近嫌后宫里烦,不大进后宫,进了也不过是去看看海贵人就完了。连新封的舒贵人都没宠幸,一直撂在那儿呢。''

如懿有些迟疑,还是沉吟着道:``皇上不高兴?''

``重阳这样的大节庆出了这样的事,也难怪皇上不高兴。''

如懿缓一缓气息,关切道:``那海兰如何?''

江与彬微微踌躇,斟酌着道:``胎像倒好,只是怀着第一胎,又出了头三个月不思饮食的时候,这些时日一直胃口大开。''

如懿放心地含笑:``吃得下是好事,海兰从前也太瘦了。''

江与彬亦笑:``是好事,就是胖起来快点,微臣总叮嘱海贵人得多走动。否则到时生产便要吃苦。''他往四周看了看:``小主原来的屋子烧了,如今往着吉太嫔从前的屋子,稍稍将就吧。''

如懿倒也淡然:``往哪里不是住着,左右也离不开这里。''

江与彬看见榻上搁着一件赤色披风,用珊瑚和蜜蜡珠子缀着万字不到头的花样,另用金色的丝线绣成玉藻图案,万字不到头的连绵。这是御用的图案,他自然是认得出的。不觉得含笑拱手:``看来冷宫失火,意在小主,反而让小主得了意外之喜。''

如懿扶一扶松散的发髻,道:``你若得空,替我拿出去还给皇上,若是留在这儿,反生了事非。''

江与彬道:``好,不过微臣有一物,是给惢心的。''他打开药箱,取出一包点心:``这是万宝斋的酸梅糕,惢心最喜欢吃的。微臣特意带给她的,以安慰她受火困的惊吓。''

如懿摸着糕点外的包纸,感叹道:``日久见人心,惢心跟着我这样的主子,落魄到这种地步,你对她的心意还是依旧,这是最难得的了。''

江与彬脸色恳切,到:``微臣与惢心都出身贫寒,何必彼此嫌弃呢。纵然她要在冷宫陪着小主一辈子,微臣也是不会变心的。''

如懿起身将皇帝的披风包好,递给江与彬道:``那日冷宫的侍卫为了救咱们这些人,冒着火冲了进来,不知有没有受伤?或者皇上有没有责罚?''

江与彬道:``只是被烟火呛着了,没有事。皇上也看到他们尽力救人了,并没有怪罪,小主的意思是''

如懿看着外头的天光晦暗,忧心道:``我怕他们贸然救人,得罪了人也不知。虽然一时之间皇上没有怪罪,但若被人暗算''

江与彬胸有成竹的笑道:``那也好办。想个法子让他得个病避一避风头就是了。这个微臣会安排。至于惢心,她被烟呛得厉害,一时起不来床,微臣会多备几服药在这儿,小主按时喂她吃下就好。''

如懿颔首道:``你下回来,替我带一包要紧东西来。这东西除了你,旁人弄不到的。''听完如懿这几句低语,江与彬脸色一沉,闪过一丝惶惑,但仍是答应了:``但凭小主吩咐。''

江与彬到了延禧宫请脉的时候,皇帝正与海兰坐在暖阁的榻上。时近黄昏,殿内有些偏暗,只有长窗里透进一缕斜晖,初秋的寒意如清水一脉,缓缓透骨袭来。

江与彬请了个安,皇帝兴致阑珊的,随口吩咐了起来。江与彬请过脉,道了``胎气安稳'',便将如懿托付的那件披风双手恭谨奉上:``微臣刚去了冷宫请脉,如懿小主托微臣将此物转交给皇上,说冷宫不洁,容不下圣物。小主已经清洗干净,请皇上收回。''

皇帝微微出神,倒是李玉机警,赶紧接过了道:``倒是难为如懿小主了,冷宫那种腌襟地方,还能把皇上的衣物清洗得这么干净,都不知道她小心翼翼地洗了多少遍。''

皇帝伸手道:``给朕瞧瞧。''李玉奉上了,皇帝伸手仔细抚摸着,缓缓道:``那是火起那日朕看她全身湿透了,特意给她披上的。她便那么不喜欢么?急急便送了回来。''

海兰梳着家常的发髻,头上点缀着如意云纹的玉饰,一支如意珍珠钗斜斜坠在耳边,清爽而不失温婉。她婉声道:``姐姐的意思,怕是近乡情更怯,触景反伤情。她已经是皇上的弃妃了,怎么还能收着皇上的东西。姐姐她''

皇帝摆手道:``罢了,朕明白。''

李玉忙仔细捧过收下。皇帝便问江与彬:``如懿在那里都好么?''

江与彬忙跪下道:``微臣若说实话,皇上必定怪罪。''

皇帝笑了笑:``是朕问错你了。冷宫那地方自然不好,朕是问她,身体还好么?''

``其他都无碍,就是人熬瘦了好些。整日和那些疯妇在一起,能清醒便是好的了。''

皇帝微微点头:``海贵人举荐你为她安胎,朕一开始是不放心的。太医院比你有资历的人多得多了,你又只在冷宫当差。可海贵人说你做事老道,也不是挑三拣四欺凌主上的人。朕看你伺候海贵人赫尔如懿都尽心,倒也能放心少许了。''

江与彬道:``在微臣眼中,冷宫的小主与海贵人并没有分别,都是微臣要尽心照顾周全的小主。''

正巧敬事房的首领太监徐安捧了绿头牌进来道:``皇上,该到翻牌子的时候了。''

皇帝看着乌黑的紫檀木盘子上一排的绿头牌,轻嗤一声道:``拿下去吧。''

徐安苦着脸道:``皇上,您好些日子没翻牌子了。别的不说,舒贵人眼巴巴地盼着您去呢。''

皇帝斜睨了他一眼,淡淡道:``你的差事越发当的好了。朕召幸谁还得听你的吩咐?''

徐安慌得跪下道:``奴才不敢,奴才不敢。''

海兰忙劝道:``舒贵人是皇上新封的,结果还没召幸就扔在一边了,面子上是不大好看。好歹还有太后呢。''

``朕今日没有兴致。''皇帝摇了摇头,将牌子推开,温和道:``海兰,你好好歇着,朕回养心殿了。''

海兰忙起身送了皇帝出去,眼看着皇帝上了辇轿,方才慢慢走回去。

皇帝坐在辇轿上,看着前后乌泱泱的人群在暮色中沉稳而迅疾的走动,几只鸦雀扑棱着翅膀飞过染着墨色的金红天空,无端便生了积分寂寥之情。他将手探入怀中,取出一方薄薄的丝帕,上头只绣了几颗红荔枝,并几朵淡青色的樱花。他慨然片刻,紧紧地握在手中,像是握着一方失而复得的温暖,再也不肯松开。

\hypertarget{ux7b2cux4e8cux5341ux56dbux7ae0-ux53ccux6bd2}{%
\chapter{第二十四章
双毒}\label{ux7b2cux4e8cux5341ux56dbux7ae0-ux53ccux6bd2}}

海兰的病症,是在怀孕六个月的时候出现的。与怡嫔和玫嫔当时的情况并无二致。一开始,她只是发胖得厉害,因着是头胎,还以为是浮肿,喝了许多去肿的冬瓜汤还是不见起色,才知道是真的胖了起来。第一条粉红色的纹路出现在身上时,她还不以为意,直到第二条第三条第无数条出现在她身上时,她才害怕得哭了起来。然则还来不及哭多久,她便发现了自己更大的不对劲,嘴里的溃疡接二连三地冒出来,时不时地发热、大汗、心悸不安,自己也控制不住似的。并且一夜一夜失眠多梦,她从梦魇里醒来,慌乱之下请来了玫嫔,并在她惊惧失色的面孔上,探询到了一丝可能的意味。

彼时,皇帝的心境已经平复不少,盛宠舒贵人之余很少再顾及到后宫诸人。在听闻海兰的病症之后,皇帝亦是由舒贵人陪同着来到延禧宫。海兰哭得梨花带雨,怯怯地拉住玫嫔的手不放。玫嫔亦是触动了情肠,二人相对垂泪,俱是伤心不已。

皇帝自嘉嫔生育了四阿哥后,以为一切须遂,只盼着海兰能再生下一个阿哥来,更好释怀当年怡嫔与玫嫔腹中之子被害之事,却不想一进延禧宫,太医还是那番旧话。太医神情难看到了极点,道:``回皇上的话,海贵人的确是中了朱砂与水银之毒,种种迹象,与当日玫嫔娘娘与怡嫔娘娘无二。所幸的是,海贵人细心,发现得早,所以一切还无大碍。\textgreater\_\textless''

太医倒也谨慎,令人查了又查,验了又验,回禀道:``皇上,微臣已经检验了海贵人的饮食与所用的蜡烛炭火,此人毒害海贵人龙胎的手法与当年毒害怡嫔与玫嫔两位娘娘的如出一辙。万幸的是,天气刚冷,所用炭火不多,而海贵人又不喜鱼虾,吃得少,所以毒性只入发肤,而未伤及肌理心脉。''

皇帝握住心有余悸的海兰的手不断抚慰:``别怕,别怕,朕已经来了。\textless{}''

玫嫔的神色十分激动,一张脸如同血红色的玫瑰:``是谁?是谁要害我们?''她``扑通''跪下,紧紧攥住皇帝的袍角,哀泣道:``皇上,会不会是乌拉那拉氏?是不是她又要害人了?''

海兰的神志尚且清明,含泪道:``皇上,乌拉那拉氏尚在冷宫,一定不会是她。''

倒是舒贵人提了句:``皇上,臣妾也曾听闻当日乌拉那拉氏毒害怡嫔与玫嫔,祸及龙胎之事,只是她人都在冷宫里了,怎会有人用和她一样的手法再毒害旁人?到底是当日乌拉那拉氏尚有同谋留在宫中,还是乌拉那拉氏是为人所冤,而真正害人的人因着这手法得意,所以一再用来谋害皇嗣?皇上若不查清,只怕玫嫔与怡嫔之后,海贵人还有其他妃嫔都会受人所害。''

舒贵人一向淡淡地不爱与嫔妃们来往,此时娓娓论来,也只是置身事外的清冷语气,恰如她耳边的一双冷绿色的翡翠耳环轻轻摇曳,清醒而夺目。

李玉服侍在皇帝身边,轻声道:``奴才倒记得,当日乌拉那拉氏被人力证以水银和朱砂谋害皇嗣,她拼命喊冤,却是人证物证俱在,反驳不得。如今细细想来,若她真是被冤,那岂不得意了那真正谋害皇嗣之人。奴才想着,真是心惊后怕。''

玫嫔沉吟片刻,睁大了眼道:``皇上,当日臣妾一心以为是乌拉那拉氏谋害了臣妾的孩子。可按着今日海贵人的样子,只怕乌拉那拉氏真被冤枉也未为可知。''她眸中清泪长流,悲戚不已:``皇上乌拉那拉氏被冤也不算第一等要事。可是皇嗣含冤而死,皇上却不能不留意了。''

海兰亦是垂泪不已,她唇角长着溃疡,每一说话便牵起痛楚,带着``咝咝''的吸气声,听着让人发寒:``皇上,当日这事若乌拉那拉氏有同谋,就不会不供出来,落得自己一个人去冷宫的下场,可见必定是另外有人主谋,手法才能如此娴熟。可是\ldots\ldots{}''她迟疑片刻:``臣妾也不能不疑心了,当日所有的人证里,别人也还罢了,最要紧的一个却是皇上的慎贵人,乌拉那拉氏昔日的贴身侍婢阿箬,她的话不能让人不信。所以此事的真伪\ldots\ldots{}''

玫嫔原本就不喜职箬得宠后的轻狂样子,轻哼了一声不语。

舒贵人冷冷道:``慎贵人凭着出卖主子才当的贵人,可见品性也不怎样!要是乌拉那拉氏真的是被冤的,我瞧她便是被真正的主谋收买了也未可知。''

这一语便似惊醒了梦中人一般,玫嫔即刻变色道:``皇上,慎贵人甚是可疑,不能不细察。''

皇帝轻轻``嗯''了一声,仿佛全没把这些话听在耳朵里,只替海兰掖了掖被子,温言道:``你且安心养着,朕把太医院最好的太医都留给你好好调治。别胡思乱想,一切交给朕就是了。''

皇帝潇然起身,向着玫嫔的泪眼温情脉脉道:``已经伤心了那么多年,别再哭伤了眼睛,赶紧回宫去歇着吧。舒贵人,你也跪安吧。''

皇帝说罢,扶了李玉的手出去,一直上了辇轿,到了养心殿书房坐下,一张英挺面容才缓缓放了下来。李玉深知皇帝的脾气,努一努嘴示意众人下去,自己倒了一杯热茶放在皇帝手边,轻声道:``皇上,喝点茶消消气。''

皇帝端起茶冷笑一声:``消气?朕的后宫这么热闹,沸反盈天,连一个孩子都容不下!朕看热闹还来不及呢,哪里来得及生气!''

李玉吓得不敢言语,皇帝一气把茶喝尽了,缓和气息道:``海贵人被人毒害的事,你便替朕传出去,顺道把当年力证如懿的人都提出来,再细细查问。''

李玉答了``是'',又为难道:``可是其中一个,是慎贵人呀!''

皇帝正沉吟,却听外头敬事房太监徐安请求叩见,李玉提醒道:``皇上,是翻牌子的时候了。不过,您若觉得烦心,今日不翻也罢。''

皇帝便道:``那就让他进来吧。''

徐安捧了绿头牌进来,恭恭敬敬跪下道:``恭请皇上翻牌子。''皇帝的手指在墨绿色的牌子上如流水滑过,并无丝毫停滞的痕迹,他似是随口询问:``从前娴妃的牌子\ldots\ldots{}''

徐安忙道:``娴妃被废为庶人,她的绿头牌早就弃了。''

皇帝轻轻``嗯''一声:``那重新做一个绿头牌得多久?''

``很快,很快。''徐安听出点味儿,忙赔着笑,抬起头觑着皇帝的神色,眨巴着眼睛道:``皇上的意思,是要重新做娴妃的绿头牌么?''

皇帝摇头道:``朕不过随口一说罢了。''他的手指停留在``慎贵人''的绿头牌上,轻轻一翻,那``嗒''一声余韵袅袅,晃得李玉眉头一锁,旋即赔笑道:``皇上有日子没见慎贵人了呢。''

皇帝重又坐下,看着外头渐渐暗下来的水墨色天光,懒懒道:``是啊。这些日子都在舒贵人那里,是该六宫里雨露均沾,多去走走了。''

李玉有些不解:``皇上方才让奴才查当年与娴妃娘娘有关的事,那么慎贵人\ldots\ldots{}''

皇帝淡淡道:``奴才是奴才,慎贵人是慎贵人。''他想了想:``慎贵人的阿玛桂铎治水颇有功绩,今秋的洪水又被他挡住了不少。如果南方的官员都会了治水之道,朕该省下多少心思。''

李玉笑道:``皇上不是一早吩咐了慎贵人的阿玛将治水之法整理成书么?今儿一早成书就已经搁在御案上了,想是折子太多,皇上您还没看到呢。''

皇帝眸中微微一亮,旋即微笑道:``朕得空会看的。你去吩咐慎贵人准备接驾吧。''

李玉躬身告退,皇帝从堆积如山的折子底下翻出一本《治水要折》,仔细翻了两页,唇角带起一抹浅笑,无声无息地握在了手里。

连着数日,皇帝都歇在阿箬宫里,一时间连得宠的舒贵人都冷淡了下去,人人都云慎贵人宠遇深厚,长久不衰,是难得一见的福分。而另一边,宫中却开始隐隐有谣言传出,说起皇帝又再提起娴妃,恐要把她恕出冷宫出未可知。

消息传到冷宫的时候,如懿不过置之一笑,从请脉枕上收回自己的手腕,笑道:``真的大家都这样疑心么?''

江与彬微笑道:``宫中本是流言聚散之地,自然会有人在意。''

``那我岂不凄惨?又卷入是非之中?''

江与彬淡然含笑道:``是非何曾离开过小主?越是凄惨之地,越是有生机可寻也未可知。''他将一包药从药匣中取出递给她:``这是包治百病的良药,小主大可一试。''

如懿含笑接过:``那便多谢了,只当借你吉言吧。''

这一日午后,是难得的晴好天气。时近暮秋,也难得有这般秋气爽的日子,天空是剔透欲流的蓝色,晶莹得如一汪上好的透蓝翡翠。惢心从墙洞里取过最后两份菜式不同的饭菜,端过来与如懿同食。

送来的是简单的素食,不沾荤腥,主仆俩虽然吃得习惯了,但这一日送来的菜色是如懿素来不爱吃的苦瓜与豆芽。她夹了几筷便没什么胃口,惢心也吃了两口,摇头道:``都快入冬了,还送这么寒凉的苦瓜和豆芽来,吃着岂不伤身么。''说罢只扒了几口白饭,便要起身将盘子依旧送出墙洞去。

惢心才站起身来,只觉得胸中一阵抽痛,呼吸也滞阻了起来,像是被一块湿毛巾捂住了嘴脸,整个人都透不过气来。她心里一阵慌乱,转回身去,却见如懿一副欲吐而不得的样子,面色青黑如蒙了一层黑纱。

惢心心知不好,一急之下越发说不出话来,还是如懿警醒,虽然痛苦地捏紧了喉头,却借着最后一丝力气,将盘中的碗盏挥落了下去。

凌云彻和赵九宵酒足饭饱,正坐在暖阳底下剔着牙。赵九宵看凌云彻靴子的边缘磨破了一层,衣襟上也被扯破了一道丝儿,不觉笑他:``你的青梅竹马小妹妹这么久不来了,你也像没人管了似的,衣裳破了没人补,鞋子破了没人缝,可怜巴巴的。''

凌云彻蹭了一脚,想起鞋子里垫着的鞋垫是如懿给的,便有些舍不得,缩了脚横他一眼:``可怜巴巴?还不是和你一样。''

赵九宵摇头道:``那可不一样。我不做梦啊。宫里的女人哪里是我能想的,一个个攀了高枝儿就不回头了,比天上的乌鸦心还黑,我可招不起惹不起。''

两人正说话,却听得里头碗盘碎裂的声音哐啷响起,都是吓了一跳,赶紧起身问了两声``什么事'',却无人应答。九宵亦学得不对头,心打开锁道:``你进去瞧瞧,我在这儿守着。''

云彻听得声音是如懿屋里传出来的,一时顾不得避嫌,忙闯了进去,只见地上杯盘狼藉,碗盘碎了一地,到处都是碎瓷碴子。主仆二人都伏在桌上,气喘不定,脸色青黑得吓人。如懿犹有气息,虚弱道:``太医\ldots\ldots 江太医\ldots\ldots 救命!''

云彻吓得脸色发白,也不知她们吃坏了什么,不管三七二十一,先给两人各灌了一大壶温水,用力拍着她们的后背。如懿虚弱地推着他的手,喘着气催促道:``快去!快去!''

消息传到养心殿的时候,皇帝正午睡沉酣。李玉得了水牛,望着里头明黄色帘幔低垂,却是慎贵人陪侍在侧,一时也有些踌躇,不知该不该进去通报。正犹豫间,却见两个延禧宫的宫人也急匆匆赶了过来,道:``李公公,不好了,海贵人出事了。''

这一下李玉也着了慌,顾不得慎贵人在侧,忙推门进去。慎贵人见他毛毛躁躁推门进来,已有几分不悦之情,便冷下脸道:``李玉,你可越发会当差了,皇上睡着呢,你就敢这样闯进来。''

李玉忙道:``回慎贵人的话,延禧宫出了点事儿,让奴才赶紧来回报。''

阿箬原就忌讳海兰与旧主如懿要好,此刻听了,便撇嘴冷笑道:``能有什么不得了的大事,若身上不好,请太医就是了,皇上又不是包治百病的神医。我可实话告诉你,这两夜皇上睡得不是很安稳,好容易午后喝了安神汤睡着了,现在你又来惊扰,我看你却有几个胆子!''

李玉听着帐内的人呼吸均匀,显然睡得安稳,忙磕了个头,神色怯怯而谦卑,口中声音却更大了几分:``慎贵人恕罪,慎贵人恕罪。不是奴才胆子小,实在是事出有因,冷宫里来报,乌拉那拉氏中毒垂危,延禧宫也说海贵人的香料中又被加了水银和朱砂,伤及玉体。宫中屡屡出事,奴才实在不敢不来回报啊。''

阿箬招了招手里的绢子,盈然轻笑一声:``你也太不会分是非轻重了。冷宫里乌拉那拉氏,死了也就死了,值什么呢,只怕说了还脏了皇上的耳朵呢。到于海贵人,传太医就是了。这天下能有什么比皇上更尊贵的,你也犯得上为这点小事来惊扰皇上!''

李玉沉默着擦额头的汗,把头垂得更低,却并无退却的意思。片刻,明黄色五龙穿云绣帐被撩起一角,皇帝的声音无比清明地传来:``李玉,伺候朕起身。''

李玉的唇边扬起一抹淡而稳妥的笑意,嘴里答应了一声,手脚无比利索地动作起来。慎贵人神色微微一变,忙堆了满脸笑意要去帮手,皇帝的手不动声色地一挡,慢慢道:``你跪安吧。这些日子都不必到朕跟前了。''

阿箬慌忙跪下,眼神慌乱:``皇上恕罪,皇上恕罪,臣妾不知做错何事,还请皇上明言。''

皇帝嘴角蕴着一抹冷冽的笑意,眼中寒凉如冰渊:``许多事,你一开始便错了,难道是从今日才开始错的么?''

阿箬只觉得背上一阵阵发毛,仿佛是衣衫上精心刺绣的香色缎密强嫣红月季的针脚一针针戳在背脊上,带着丝丝的糙与针尖的锐,逼向她软和的肉身。不,不,这么多年了,皇帝如何还会知道。果然,皇帝带着不豫的语气道:``冷宫的事好歹也是条人命,何况海贵人怀着的是朕的皇嗣龙裔,你竟也对人命皇嗣这般不放在心上?朕原以为你率真活泼,心思灵敏,却不想你的心底下还藏了这许多冷漠狠毒!''

阿箬被骂得双膝发软,瘫软在地上,心中却漫过一层又一层惊喜,原来,不是为那件事。幸好,不是为那件事。

皇帝由着李玉替他穿上海蓝色金字团福便服,扣好了玉色盘扣,厌弃地看阿箬一眼:``出去吧!''

李玉只是含了一抹恭顺的笑意,目送着阿箬扶着宫女新燕跌跌撞撞地出去,不由得钦佩地望了皇帝一眼。伺候皇上这么些年,他不是不知道皇帝的脾性,也比旁人更清楚,慎贵人这些年的盛宠之下,到底是什么。皇帝这一抹今日才肯流露出来的厌弃,实在是太晚了。

他于是恭谨问:``那么皇上先去哪里?''

皇帝的眉目微微一怔,便道:``自然是延禧宫。''

延禧宫中乱作一团,海兰畏惧地缩在床角,嘤嘤地哭泣着,拒绝触碰一切事物。宫人们跪了一地,皇帝从人群中走进去,一把搂过她,温言道:``到底怎么了?''

叶心跪得最近,便道:``皇上,自从上次的事,我们小主已经足够小心了,饮食上都派人仔细查验过,谁知今儿奴婢想去倒了香炉里的香灰时,发现里头有些异物。奴婢不敢怠慢,请太医看了,才发现了是有人把朱砂混进了小主的安息香里。''

皇帝的神色难看得几欲破裂,冷冷道:``查出来是谁干的么?''

海兰呜咽着伏在皇帝怀里,哭得鬓发凌乱,几枚散落在发丝间的粉色小珠花越发显得她形容憔悴,不忍一睹。

皇帝惊怒交加,安抚地拍着她的肩道:``别怕,朕一定彻查清楚,不会让人再伤害你。''

海兰啜泣着道:``那人存心陷害皇嗣,臣妾宫中已经有所防备,她还敢换着法子下毒,实在是胆大包天。皇上,您告诉臣妾,到底是谁要害咱们的孩子?是谁?''

皇帝柔声道:``还好你身边的侍女发现得早,只是你孕中不宜操心,这件事,朕会交给李玉去细查。''

李玉响亮地答应一声:``是。奴才一定会尽心尽力去查,给皇上和海贵人一个交代。''

皇帝好生安慰了几句,便道:``后宫出了这么多事,朕得去见见皇后。六宫不宁,也是她的过失。''

海兰正要起身相送,皇帝忙按住她道:``你好好歇着,别劳累了自己。朕晚上再来看你。''

宫人们送了皇帝出门,皇帝见已无延禧宫的人跟着,方才低声道:``冷宫里是怎么了?''

李玉忙道:``据太医回禀,是中了砒霜的毒,还好乌拉那拉氏庶人和惢心午膳用得不多,所以中毒不深,除了太医江与彬,奴才还派了两个太医一同去盯着,以防不测。''

皇帝赞许道:``你做得不错。如懿中毒,这边厢海兰就出事,两者几乎是同一时间,看来不会是如懿指使人做的。''他冷笑道:``看来朕才放出点风声,便有人沉不住气了。只是朕没想到,她们竟沉不住气到这地步,居然要杀人灭口。''

李玉看着皇帝的神色,小心翼翼道:``皇上也觉得,这些年\ldots\ldots 她是受委屈了?''

皇帝眼底添了几分焦灼之色,口气倒还沉稳:``朕去瞧瞧她。''

李玉忙道:``冷宫忌讳,皇上金尊玉贵,可去不得。''

皇帝淡淡笑道:``旁人可以去冷宫杀人放火,朕连瞧瞧也去不得么?上回冷宫失火朕也去了,这次不过是再往里走一步,那便怎么了?''

李玉情知劝不住,只得扶了皇帝上轿,向冷宫去了。

\hypertarget{ux7b2cux4e8cux5341ux4e94ux7ae0-ux590dux751f}{%
\chapter{第二十五章
复生}\label{ux7b2cux4e8cux5341ux4e94ux7ae0-ux590dux751f}}

如懿躺在床上,只觉得胸口烦闷难安,呕吐的感觉挥之不去,脑中也一阵阵晕眩,仿佛身体轻飘飘的,堆在一堆浮絮之上,四肢百骸半点力气也无。

江与彬已经灌了如懿和惢心许多浓盐水,催她们呕吐出来,又拿烧焦的馒头研磨成粉给她二人服下吸附毒物。他一个人正手忙脚乱,又来了两个太医院的太医,看来地位在江与彬之上许多,三人商议了用药,才把如懿和惢心从鬼门关扯了回来。

如懿躺着,薄薄的破旧被子盖在身上,像有千斤重似的不能承受。可是,她还有什么承受不住的呢?她怔怔地想着,看着另一张床上面色雪白如纸的惢心,想着自己此时此刻,也是一般的容色吧?幸好,他是不会来这里的,上次失火,她是那么狼狈,在狼藉不堪中见了他一眼,那一眼,她便明甶了自己的在意,明白了自己的舍不得。所以,情愿他不要来。

正胡思乱想着,却听外头脚步声肃然有序响起。如懿在晕眩乏力中看着一抹明黄渐渐逼近,和着泪水模糊了她的双眼。

盼他来,怕他来,他终于还是来了。

皇帝的身影凝在如懿床边,他的声音是那样熟悉而邈远,轻缓柔和:``朕来了。你还好么?''''\textgreater\_\textless``"

好么?这么些年,他不是不知道她身陷在这苦牢里。这个``好''字,她已经不会写,也不懂得写了。如懿并不背过身,只是在默然中以泪眼寂静相对。

她没有别的了,委屈、辛酸、痛苦、悲与冤,都尽数化作了眼底缓缓流淌的累,一如她的心绪,没有激荡,只有沉缓,预料之中期待之外的沉缓。

皇帝似乎被她的泪所感染,亦多了几分沉郁之色,不自禁地想要伸出手握住她的手。如懿望着自己枯瘦得青筋暴现的手背,将它缩回被中,淡淡道:``贱妾鄙薄之身,怎可由万圣之尊触碰?''

皇帝看了看周遭,抑制住自己的神色,道:``娴妃是怎么中的毒?''

江与彬听得皇帝这一声称呼,只觉得心头大石都松懈了下来,他急忙抑制住唇角将要泛起的笑意,沉声道:``娴妃娘娘是中了砒霜之毒,所幸发现得早,娴妃娘娘与惢心姑娘进食也不多,万幸没伤及五脏六腑。''

``没事就好。你们好好替娴妃治着。''皇帝长吁一口气,俯下身,望着如懿一双泪眼,低沉欷歔,``你的性子一直坚毅倔强,却不想也有这样泪水长流的时候。朕与你那么多年,都未见过你那么多泪。''

``性子倔强坚毅,不代表没有委屈冤痛。但即便有,知道申诉无用,也唯有长泪而己。贱妾流泪,不足以入皇上之目。冷宫卑贱之地,也不宜皇上久留。还请皇上尽早离开吧。''

两望的泪眼里,皇帝默然片刻,极力收拢眼中的动容之色,转身向江与彬道:``好好照顾娴妃。''

江与彬躬身道:``是。只是冷宫湿寒,怕不宜养病。\textless{}''

皇帝温然而坚决:``朕知道冷宫不是久留之地。待姻妃能起身了,朕会即可复她位分,带她出冷宫。''

这话是说与江与彬的,亦是对她。

如懿闭上双眸,感受着热泪在眼皮底下的涌动,终于背过身握紧了双手,露出一分淡然的笑意。

六宫之中任何消息都难以被瞒住,人的耳朵和嘴处是最好的传递之物。皇后与慧贵妃站在廊下,望着一蓬新开的绿菊闲话家常,却见赵一泰匆匆进来打了个千儿道:``皇后娘娘万福,慧贵妃万福。''

皇后很看不上他急三火四的样子,扬了扬纤纤玉指,蹙眉道:``这样不稳当,是怎么了?''

赵一泰看了两人―眼:``皇上方才去了冷宫,亲呼乌拉那拉氏为娴妃,说不日便将释放她出冷宫。''

慧贵妃一个踉跄,差点没站稳,声音也不觉高了几分:``乌拉那拉如懿毒害皇嗣,证据确凿,已被废为庶人,怎还会被放出冷宫?皇上还称呼她娴妃?''

皇后脸色白了几分,倒也还镇定:``为何是不日放出冷宫,而非即刻?赵一泰,你把话说淸楚。''

赵一泰稳住了神道:``乌拉那拉氏中了砒霜之毒,一时未能好转,皇上瞩咐待她能起身时再出冷宫。''

皇后挥手示意他下去,转身进了内殿。慧贵妃急急跟进,见无人在侧,忙道:``皇后娘娘,咱们好不容易才把乌拉那拉氏拖进冷宫,如果此刻容她出来,之前的工夫岂不白费了吗?''

皇后平静地目视她片刻,亦缓和着自己突如其来的心绪,慢慢道:``你鬓边的凤钗歪了,扶一扶正吧。''

慧贵妃急切道:``皇后\ldots\ldots{}''

皇后深吸一口气,柔缓道:``仪容端正有肃,是贵妃应有的仪表,任何情况下都不容失了分寸。''

慧贵妃有些羞赧,忙扶正了垂珠凤钗,缓声道:``娘娘,她既然中了砒霜的毒,虽然咱们不知道是谁下的手,但是顺水推舟,总是不难的。''

``你是说\ldots\ldots{}''

慧贵妃含了一缕隐秘的笑容,笃定道:``既然已经中毒,那么再给她追加一点儿,毒发身亡就是了。''

皇后慢慢拨弄着纤白如玉的手指上翠浓的碧玺戒指,摇头道:``来不及了。皇上已经去看过她,也下了旨意,此时再动手,实在是太点眼了。无论得手失手,都把她之前中了砒霜毒的黑锅自己背去了,太得不偿失啊。''

慧贵妃秀眉紧蹙,拧着绢子恨声道:``也不知道是谁下的毒,也不下准点,要了她的命就好了。''

皇后思忖片刻,看着她道:``会不会处慎贵人?''

慧贵妃摇头道:``她没那样的胆子,敢不跟咱们知会一声就去做这样的事。出了事没人替她兜着,她都不知逝死了多少回了。''

皇后淡淡一笑:``当日只想着借她一把力气,谁知道倒成全了她平步青云。''她漫然扬了扬手中的绢子:``也好,留着她在,她也容不下乌拉那拉如懿。''

慧贵妃会心一笑,起身道:``皇后娘娘圣明。''

江与彬的医术颇为精到,不过三四日,如懿和惢心便能起身了,她披衣坐在廊下,看着被略作修缮的屋子,道:``惢心,即刻要走了,何必再收拾:''

惢心微微咳嗽两声,满面含笑道:``奴婢是心里高兴,内务府的太监知道咱们只在这里养几日就要走了,都还巴结着来打理修缮,那是他们知道小主出去后便不一样了。也好,咱们费了这许多心思,终于能够离开这里了。''

如懿靠在廊下破旧的廊柱上,定定道:``出去不过是第一步,要活得好,不再像从前一样任人欺凌宰割,才是最要紧的。否则今日出去,不知哪一日还会被送回来,又有什么意思?''她转过头:``你身子才好,万不要太劳累了。''

惢心出来,笑着替她披上一件外裳,道:``奴婢没事,奴婢为了小主,怎样都是快活的。''

如懿握住她的手道:``惢心,还好万事都有你在我身边。''

``我与小主之间,不说这些。''惢心看着如懿'眼底微有泪光,想了想道:``小主嘱咐奴婢做的靴子奴婢都做好了。''她指着里屋木箱上的---双男靴道,``奴婢见过凌侍卫的靴子,尺码应该是不会错的。奴婢按着小主的吩咐,鞋边上又拷了两层线,这样就不容易破了。''

如懿道:``你的手艺自然是不错的,拿来我瞧瞧。''

惢心即刻捧了过来,如懿仔仔细细看了一遍道:``我也没什么好谢他的,他的鞋磨坏了,就让你做双鞋谢他吧。''

惢心道:``可不是呢?若没有凌侍卫三番四次救咱们,哪有奴稗和小主的今日。''

如懿抚摸着簇新的靴面,心中亦不免触动,感叹道:``虽然他是受了海兰和咱们的银子办事。可许多事,原是在他的本分之外,他还愿意这样帮忙,那便是雪中送炭的情谊了。''

惢心叹息道:``也是,锦上添花易,雪中送炭难。凌侍卫的心意算难得了。''

如懿低头看了看靴子道:``既是送给他的,你在靴筒的里面绣上---朵云纹以作辨别吧,等下黄昏用饭时分,请他瞅着方便过来瞧一瞧就是了。''

惢心答应着,便道:``廊下风冷,小主进去再睡―会儿吧。''

皇帝午睡起来,倒也不像寻常那样便去书房批折子,只是一个人坐在窗下,慢慢地收拾着棋盘上的残子,似是动着什么心思。

李玉不敢让人打扰,亲自捧了茶点上前,道:``皇上,皇后宫里新制的酥酪茶,请您尝尝。''

皇帝头也不抬,便道:``搁着吧。''李玉望了望窗外:``皇上,从您睡下后,慎贵人就一直跪在养心殿外,说前两日服侍不周惹您生气,求您宽恕。''

皇帝将手中的黑子往棋盘上一撂,含了一缕鄙薄的笑意:``她还来求朕宽恕?这些年她做了什么,她自己都没数么?''

李玉低头道:``皇上天意圣裁,奴才哪里能懂得。皇上说慎贵人是什么,她就是什么。''

皇上淡淡一笑:``这些年来她是怎么侍寝的,你是朕的贴身太监,你会一点也不知?''

``皇上不许奴才知道,奴才就不知道。皇上许奴才知道了,奴才也只能心里知道,嘴上可不敢胡说。''李玉将手中的点心一色儿排开,利索道,``这八宝玫瑰花卷是慧贵妃敬献的,奶白枣宝是纯妃敬献的,白果栗子松是玫嫔娘娘的手艺,花盏龙眼是嘉嫔娘娘娘亲自做的,还有一味桃花百合糖渍凉粉和羊脂菠萝冻分别是舒贵人和慎贵人的进献。皇上想尝尝哪一道?''

皇帝看他道:``你不是做事谨慎又不爱言语么?那朕问你,这会子朕觉得看了这些东西都甜腻腻的,你觉得给朕上什么点心好?''

庭下有凉风拂进空落繁丽的大殿,带进殿外菊花的清苦香气。李玉心中一动,便道:``从前娴妃娘娘在的时候,有一道菊花佛手酥是最擅长的。御膳房虽不能做出一模一样,但也可以试试,算是应季的美食了。''

皇帝这才露出几分笑意:``跟在朕身边久了,算你懂事。朕问你,六宫里知道朕要放出娴妃来,可有什么动静?''

``能有什么动静,也不敢动到皇上跟前来。左不过是议论纷纷,流言四起罢了。''

皇帝思付片刻:``这就流言四起了?李玉,朕吩咐你把飒坤宫收拾出来,可怎么样了?''

丨

李玉道:``翊坤宫与皇后娘娘的长春宫并列,紧跟在皇上的养心殿之后。坤为女阴之首,翊为辅佐,除了皇后娘娘大婚所用的坤宁宮,翊坤宫算是最华丽紧要的所在了。皇上吩咐把翊坤宫收拾出来给娴妃娘娘居住,奴才不敢不用心,一应挑的都是最好的东西。''

皇帝颔首道:``翊坤宫尊贵,朕就是要给如懿这份尊贵,好弥补她这些年在冷宫的委屈。对了,如懿一向挑东西最精准,你看看内务府选了哪些东西去布置,都列份单子给朕先过目。''

李玉看着皇帝抿了口茶,躬身道:``皇上心系娴妃娘娘,顾虑周全,奴才万万不及。只是皇上如此看重娴妃娘娘,一心要弥补她的委屈,怎不晋一晋她的位分,更示恩宠。''

皇帝随手取过一块点心尝了,道:``许多事,不在位分上。娴妃家世不够显赫,的确不如慧贵妃。至于后宫这么介意娴妃出冷宫,你便再下一道旨意。娴妃出冷宫之曰,晋封贵人叶赫那拉氏为舒嫔。''

李玉道:``是。奴才遵旨。''皇帝扬脸看了看朱红格栏窗外跪着的慎贵人,凛凛秋风之中,她衣衫单薄,盈然飘飘。皇帝淡淡笑道:``她喜欢跪,便让她跪着吧。''

海兰独自卧在床上,床帐上绣满了多子多福的石榴葡萄纹样,为着吉样如意的好彩头,特意用橘红和深朱的缣丝绕了银线的彩绣,连铜帐钩上悬着的荷包都是和合如意的图样,看着便是洋洋的喜气。叶心端了汤药进来,海兰忍不住掩鼻道:``一股子味儿,真是熏人。''

叶心见没有旁人在,方才劝道:``小主好歹忍一忍喝了吧。这药是去朱砂和水银的余毒的。还好小主中毒不深,太医嘱咐再喝两天就好了。要是余毒未清伤及腹中的小皇子,那可怎么好呢?''

海兰清吁一口气,抚着肚子道:``我知道,左不过都是为了姐姐罢了。''

叶心轻轻地吹着药,叹道:``小主待娴妃娘娘,那真是比亲姐妹还要亲了。''

海兰理了理松散的鬓发,道:``冷宫里不比外头更安全,同样是死,怕姐姐是怎么死的都不知道了。这个宫里,只有她一人真心待我好,我也真心只待姐姐好。''

叶心将药递到海兰唇边,海兰---仰头喝了,皱眉道:``真是苦。''

叶心服侍她漱了口,忙取了酸梅放在她口里,道:``小主这话就是泄气了。小主有皇上的宠爱,眼看着就要生下皇子,有什么可担心的。''

海兰捋着帐上垂落的鸳鸯流苏,神色淡得如一抹寒冰:``皇上?皇上是个男人,一个男人三妻四妾,有什么值得依靠的?我腹中的孩子,也不过是他的孩子之一,能有什么前裎?凡事只能指望这个孩子自己,我还能指望皇上?后宫里朝不保夕,唯一能够依靠的,不过是一场姐妹情谊,才能相伴数十年。其他的,都是浮梦一场,梦过便算了。''

叶心见她盛宠之下却如此灰心冷淡,也知道不好再劝。海兰想了想问:``剩下的那些不干净的东西全清出去了么?不许留下一点痕迹。''

叶心忙道:``全清理干净了。小主放心就是。''

海兰望着外头昏黄的霞光映照在一格格的窗棂上,神色漠然:``等到姐姐在我身边了,我才真正放心。''

暮秋初冬时节的天色容易暗得早,若是逢上晴天,便有极好的晚霞招展,仿佛一匹上好的流霞锦自天际伏曳而下,虾红、宝蓝、云青、米黄,倾倒了一天一地,兀自灿烂,流丽万千。

换作往日,如懿并没有这样好的心情细赏落霞,但是此刻,她有,也愿意。笃定地看着晚霞倾于碧瓦琉璃之上,才能明白,自己将要走回去的地方,是何等繁华似锦,就如这晚霞一般,绚丽之后,只余下无尽的黑暗与凄冷,要她独自面对。

凌云彻借着送饭的机会进来,他比往日更多了几分恭敬,行礼过后才道:``恭喜小主,次日午后便可以出去了。''

如懿回望向她笑:``同喜,你也终于少了我这样一个麻烦。''她取过那双靴子:``我手艺不佳,只好让惢心缝制了一双靴子给你。双脚不受风霜苦侵,才能走的远,走得好。''

凌云彻抚摸着那双样式普通的靴子,不知怎的,竟想起了久未见面的嬿婉。从前,也是嬿婉,只有嬿婉,会这样待她。关心他的一点一滴。如今,嬿婉怕是早就成了枝头婉转滴沥的黄莺儿,飞得越来越高了吧。竟是如懿,拿这个来回报他。

他抑制住心头情绪的起伏,慨然道:``多谢小主。''他望着如懿唇边一点甘甜如露的笑容:``小主仿佛很高兴。''

``今日有期待,所以高兴。明日身在其中,或许发现自己期待的并无预想中的好,便无今日这般高兴了。''

``那小主还是一心想出去?''

如懿嫣然一笑:``留在这里,和你一样隔着一堵墙,数着今日的青苔又长了几寸,墙上的霉灰是否沾染了衣衫吗?困坐这里是死,出去也未免是死,但我还是想争一争,试一试。''

凌云彻听她婉声道来,不知怎的,心下却生了一股豪情壮志,这么些年被人冷眼瞧低,这么些年不得出头,他的心思,何尝不是和如懿一样。不搏一搏,试一试,岂不辜负了自己,辜负了一生?

他捧着那双靴子,心意只在电转间便落定了。他诚恳请求:``若是小主愿意,可否带我离开冷宫,觅一份前程?''

如懿清简的薄薄衣衫被风微微卷起,她微眯了双眼:``你想离开这里?为什么?''

他抬眸,坦然道:``与小主一样,心中不甘,心中有所求。''

如懿淡然一笑,望着天际升起的一抹淡淡月华,怡然吟诵道:``竹院新晴夜,松窗未卧时。共琴为老伴,与月有秋期。玉轸临风久,金波出雾迟。幽音待清晨,唯是我心知。这是白居易《对琴侍月》虽然合了眼前之景,但少了琴音也不够风雅。我却只喜欢`幽音待清晨,唯是我心知'这一句。你救了我许多次,我一直无以为报,许你一个好前程,就当是谢你吧。''

凌云彻心下欢悦,一时也不知说什么,只是深揖到底,默然含笑。

如懿望着满院清亮月光,亦不觉含笑。

次日午后,李玉带着皇帝身边进忠、进保两个小太监一同前来迎候,服侍梳妆更衣的两位姑姑都是皇帝跟前积年的老嬷嬷了,手脚最是利索,也会做事,按着妃位,如懿本该穿金黄色立龙戏珠配八宝寿山江牙立水。立龙之间彩云蛟的朝袍,戴镂金饰宝的约,颈挂朝珠三盘,头戴翎冠。如懿望了那一袭金光灿灿的衣裳,笑道:``本宫是回家去,而非年节庆贺。怎么本宫离开这里,还要欢天喜地大鸣大放才能出去么?''

李玉忙赔笑道:``娴妃娘娘的意思是?''

如懿含笑道:``本宫回去见自己的夫君,何必穿戴成这样隆重辉煌,免得叫人笑话。便是穿家常衣裳就是了。''

李玉会意,即刻吩咐人换了一身新衣裳来,便推到门外由着嬷嬷们替如懿梳妆。梳的是垂云髻,中间以扁方绕成如云蓬松,两端微微垂落至耳边,越发显得饱满而不失小女儿娇态。乌黑的云髻挽成,饰以玉环同心七宝钗,金镶玉步摇,紫鸯花合欢圆珰,飞翅的燕尾上坠着鸳鸯莲纹金蝶白玉压发,玲玲一动间,便有细碎的金玉珠子轻轻摇曳,合着正落在眉心的红珊瑚垂珠,越发添了面颊一抹艳色。

惢心伺候她换上真红色金华紫罗面织锦长袍,在领口别上一枚赤金凤流苏佩。衣襟和袖口都密密绣上缀满细密米珠的``金玉满堂''纹花边。一色的九鸾飞天金丝暗绣折枝花卉图,映着天金丝睹绣折枝花卉图,映着裙角舒展的兰花花饰,以五颗镶金镂空银质扣将琵琶如意纹钮绊住,再配着底下鸳鸯百褶风罗裙,丝滑缎面在阳光下折出光亮,上面的鸳鸯暗纹,也随着光线意思意思透显成痕,几欲展翅飞起。嬷嬷们替她带上乳白色三联东珠耳坠,尾指上套的金护甲上嵌着殷红如血的珊瑚珠子。如懿对镜自照,整个人仿似新雨当中枝烈艳艳的初绽蔷薇,灼艳而夺目。

待到一切停当,惢心蹲下身替她穿上胭脂红缎绣竹蝶纹花盆底鞋。胭脂红的底子上,钉缀着玉石做的万字不到头图案,并着蝙蝠和彩带等纹样,谐寓``万代福寿'';鞋帮上绣制纷繁细巧的竹蝶纹,镶以金线盘成的曲木纹绿边,精巧无比。李玉忙恭恭敬敬伸手,如懿扶着李玉的手站起身来,知道自己要穿着这双鞋,一步一步走到来时的地方去。

\hypertarget{ux7b2cux4e8cux5341ux516dux7ae0-ux5a34ux5983}{%
\chapter{第二十六章
娴妃}\label{ux7b2cux4e8cux5341ux516dux7ae0-ux5a34ux5983}}

如懿打扮稳妥,扶着李玉的手徐徐起身:``这身衣裳是你挑的?选的是鸳鸯纹饰。''

李玉堆了满脸的笑意:``奴才哪里会挑这个,是皇上选的呢。''''\textgreater\_\textless``"

如懿低头,细细看着那精致的鸳鸯暗纹。是呢,``鸳鸯于飞,肃肃其羽。朝游高原,夕宿兰渚。邕邕和鸣,顾眄俦侣''。

鸳鸯,原是相伴终老的爱侣,可是又有几人知道,雌鸟辛苦受难之际,雄鸟便会另觅新欢,做另一对爱侣。那天长地久,合欢月圆,原是世人自己蒙骗自己的。

她无言,只是由着李玉扶着她的手,缓步踱出这住了数年的冷宫。宫门深锁的一刻,她忍不住再度回首,那破朽灰败的回廊屋阁,积满了蛛网与尘灰的角落,终年长着潮湿青苔的墙壁,她都不会忘记。可是此时此刻,再看一眼,是要自己牢牢记住。

再不能回来,再不能落到这样的境地里。

如懿决然转身,扶着李玉的手稳步踏出去。她一直生活在这后宫里,哪怕发落到冷宫,都从未离开过这里。可是走在旧日熟悉的甬道长街上,周遭东西六宫的殿宇辉灿依旧,钦安殿、漱芳斋、重华宫、储秀宫,都跟往日没有半分差别。连地上青砖的花纹,都是熟悉透了的。

她一步一步稳稳踏在上面,似是踏着自己的心潮起伏。她终于,又走了出来。两边的宫人们见她稳然前行,忙一个接一个地跪倒在地,不敢直视。

如懿含了一缕气定神闲,暗自庆幸原来自己已经那么快适应了重出生天的生活。待走到储秀宫门前,却见一个容色极明艳的女子领着侍女站在门外,轻轻向她一福致意:``娴妃娘娘万福金安。''

如懿见她长眉深目,首饰只以绿松石、蜜蜡与珊瑚点缀,明艳不可方物,衣着打扮也格外的明丽华贵,只是十分陌生,便矜持道:``这位是\ldots\ldots{}''

李玉忙道:``储秀宫主位舒嫔叶赫那拉氏见过娴妃娘娘。''\textless"

如懿微微颔首:``舒嫔妹妹有礼了。只是天气冷了,妹妹怎么还守在风口上。''

舒嫔微微一福,神色却是淡淡的:``妹妹今日与娴妃娘娘同喜,所以怎么也要来贺一贺娘娘,迎候娘娘入主翊坤宫。''

原来这一日是如懿出冷宫复位娴妃之日,皇帝亦册封了舒贵人叶赫那拉氏为舒嫔。这一下激起千层浪,倒比如懿出冷宫更引了众人注目。骤然封嫔在后宫是极为罕见之事,金玉妍生育了四阿哥恩宠甚厚,也不过被封为嫔;海兰有孕,也只是贵人。可见这叶赫那拉氏是如何善承圣意了。偏偏她的性子,对着皇帝妩媚婉转,冷热相宜,对着旁人却冷冷地不爱理会,所以与后宫诸人都不甚亲厚。

此刻她迎候在外,特意向如懿请安,也不知是何用意。李玉只得借口天色不早,先陪了如懿回翊坤宫。

翊坤宫为东六宫之一,与皇后富察氏所居的长春宫并驾齐驱,相互辉映。绕过影壁便是极阔朗舒爽的一座庭院,正殿五间与前后走廊都绘制着江南娟秀绮丽的苏式彩画,一笔一画都是皇帝素日所钟爱的江南风韵。台基下陈设铜凤。铜鹤、铜炉各一对,一看便知是新添设的。李玉推开万字锦底五福捧寿的朱门,步步锦支摘窗上垂着银翠色霞影纱。正殿中间设着地平宝座、屏风、香几、宫扇,上悬皇帝御笔``有容德大''匾额。东侧用花梨木透雕喜鹊登梅落地罩,西侧用花梨木透雕藤萝松缠枝落地罩,将正殿与东、西暖阁隔开,越发显得殿内疏朗有致,清雅成趣。

如懿见殿中的摆设虽不奢华,却件件别致典雅,显然是用了一番心思的。李玉忙道:``小主一路过来辛苦,西暖阁中已经备好了茶点,请小主先用吧。''

如懿在正殿中向外张望,发觉李玉安排的都是往日在延禧宫中伺候的旧人,一应都是三宝在外头照应,她便放下心来,往西暖阁中去。转过花梨木透雕藤萝松缠枝落地罩,垂落的明绿色松枝纹落地浅纱被风拂得轻扬起落,一缕淡淡的茶烟袅袅升起,却见一人背向她坐在榻上,缓缓斟了---杯茶在紫檀芭蕉伏鹿的小茶儿上,缓声道:``你回来了?''

那种口吻,仿佛如懿只是去御花园中散了散心,去看了春日的花朵、秋日的黄叶回来。仿佛,她一直在他身边,从未这样被抛弃,从来未曾远离。

隔了三年的岁月,他却还是这样的口吻,转过身看着一步步艰辛走来的她,斜坐在明晃如水的日光下,带着闲和如风的笑意,向她缓缓伸出手来。

如懿有一瞬间的迟疑,不知该不该伸出手回应他。皇帝穿着玉白色长衫,仅以一条明黄吩带系住腰身,越发显得长身玉立,翩翩如风下松。周遭的人都退了下去,四周静得像在碧莹莹的潭底,湖水的觳光轻曳摇荡,让她晕眩着睁不开眼。皇帝在迷蒙的光晕里站起身来,上前轻轻拥住她:``朕知道你受委屈了。''他静一静声:``朕一直知道你受了委屈。朕的如懿,不会做那样的事。''

她的泪在一瞬间无可遏制地落下来。他知道,他居然都知道。心底多年的委屈骤然成了无限的愤恨,如懿用力挣扎开皇帝的怀抱,恨声道:``为什么?皇上明明相信我,还要把我关进冷宫!''

皇帝安抚似的拍着她的背,柔声道:``朕就是因为信你,才要把你放在冷宫里,绝了那些人继续害你的念头。所以朕故意不闻不问,故意对你在冷宫的境况毫不理会,就是希望所有人能淡忘了你,至少保得住你一条性命。可是如懿,到了最后,朕还是发现,冷宫也庇护不了你,唯有在朕身边'你才最安全,最稳妥。''

皇帝的话,似是无理,却也字字入情入理,她没有办法去推敲,去细想。是他送自己进冷宫,也是他拉自己出来。也许他真是害怕,怕自己死在了砒霜下,焚身以火,所以无论如何也要拉她出来,留在他身边。

如懿无声地呜咽着,把泪洇进他的衣衫他的肩。殿外枫叶烈烈,红得蒙住了她的眼睛,那把火,似乎一直要燃烧着,一直烧到她和他的心底去,烧尽所有的疑问与隔阂才好。

皇帝的下颌抵着她的额头,声音柔和得如一匹上好的绸缎:``朕知道你心里有许多的不相信,毕竟这三年你都没在朕身边。你放心,朕会慢慢来,一点一点告诉你。''

皇帝似是明白她的生疏与不惯,略坐了坐便往养心殿去了。如懿被他拥住许久,只觉得如释重负。靠着榻上的鹅羽软垫坐了下来,神思尚且游走在对新居的翊坤宫的熟悉之中,她望着茶水中清亮的天光倒影,一时也不觉有些失神。只听得耳边一声熟悉的轻唤:``姐姐,你终于回来了。''

如懿转过头,见海兰被叶心和绿痕搀扶着立在花梨木透雕藤萝松缠枝落地罩之后,大约是走得急,有些气喘吁吁的,脸上却挂着止不住的笑容,映着满眼喜悦的泪,盈盈望向她。

如懿才站起身,眼里便蓄满了泪,情不自禁地落下来,上前几步握住了她手道:``你有着身子,怎么来了?我正要去瞧你呢。''

``我早来了,见皇上的辇轿在外头,所以一直守着等皇上走了才进来。''海兰握紧了如懿的手丝毫不肯放松,上上下下打量着她道,``姐姐清瘦了不少,是受苦了。都怪我无用。''

``你若还无用,是谁明里暗里照顾了我这些年呢。''心中积蓄多年的感动温然漫上,如懿含泪拉着海兰坐下,``快坐下说话,别累着了。''她边拉着海兰,边吩咐道:``海贵人有孕不能喝茶,上红枣汤来。''

如懿已经三年没见到海兰了,可是见到的时候,仍是不免吓了一跳。虽然她也知道,女人有了身孕会胖起来,但她没布想到,海兰会胖得这么厉害,像吹的球儿似的,原本瘦削的身形变成了从前两个人这般大,一张巴掌大的脸儿也成了十五的银月盘一般,肚子高高地隆起,一旦挪步,就得两三个人搀扶着,像一座小山似的挪动。一身宽大的肉桂色折枝花卉百蝶纹妆花缎长袍也遮不住她发福得厉害的身体,紧紧地绷在身上,裹得她行动越发艰难。

海兰才坐下,似是想起了什么,扶着叶心的手盈盈便要行礼:``嫔妾延禧宫贵人海兰,拜见娴妃娘娘。''

如懿吃了一惊,忙扶住她道:``身子都这么重了,还行什么礼?赶紧坐下吧。''

海兰艰难地起身,微笑逬:``只有给姐姐行过礼了,我才觉得安心,知道姐姐是真的回来了。''

``你还不放心么?我已经活生生站在你眼前了,再不是要和你隔着门板说话,看着你放风筝报平安的人了。''如懿笑中带泪,看着海兰道,``听说你受了朱砂和水银的毒,都好了么?会不会伤及胎儿?知道是谁做的么?''

海兰抚着胸口的气喘,喝了口红枣汤道:``也不知是谁要害我,总之能阴错阳差解了姐姐的困局就好。太医已经看过了,一切无碍。''她低头抚着自己的小腹道:``若是连这点风霜都经不住,那便不是能养在宫里的孩子了,也不能做咱们的孩子。''

如懿微微吃了一惊:``咱们的孩子?''

海兰含笑道:``可不是?纯妃如今抚养着大阿哥和二阿哥,风头极盛,嘉嫔的四阿哥又得皇上钟爱,素日里无事也要去看几次的。看如今的情势,纯妃抚养得大阿哥很好,势必不会再还给姐姐抚养。那么姐姐,你如何能够没有自己的孩子?''

如膝心绪激荡,发髻边的紫鸯花合欢圆珰垂落细密的白玉坠珠,玲玲地打在面颊边,一丝一丝凉。她一直没有自己的孩子,自然明白海兰语中的深意,不觉激动道:``当真么?''

``你我姐妹,只不过差了一层血缘罢了,还有什么要分彼此的么?''海兰微微垂眸,叹泣道,``姐姐可方便么?我给姐姐瞧一样东西。''她看了看垂手侍立在外的叶心和绿痕,并不打算让她们进来帮手,径自牵着如懿的手入了寝殿。

如懿不知她打算做什么,一时也不便唤人,只见她解下风毛围脖,一层层脱去外裳,中衣,解开最后一层小衣,露出浅青色绣水绿牡丹花兜肚。如懿起先只是不明,待看到她后腰与肚腹的肌肤,一时间吓得目瞪口呆,下意识地掩住了口。

海兰原本的肌肤便十分白皙,加之养在深宫多年,日日以花汁萃取的香粉敷体,一身的肌肤都养的细白如玉,触手生腻。可是如今一看,上面布满了深深浅浅粉红色或紫红色的波浪状花纹,简直像个白皮红纹的西瓜一样,可惊可怖,让人触目惊心。

如懿惊道:``怎么会这样?你的身子怎么会成了这样?''

海兰无声地落下泪来,神色倒还平静:``从第五个月的时候开始长出来,太医也不知为何我会胖得这样快,,总说胃口好些对孩子是好事。我总是饿吃得多,人胖的快,身上就长出了这些纹路。''

如懿极力压抑着自己平静下来道:``没事,咱们有江太医,太医院有的是好药,问问他有什么法子或是用什么润体膏,一能能治好这些纹路的。''

海兰凄惶摇头,用小衣遮蔽住自己的身体:``来不及了,姐俎,我已经问过专门侍奉生育的嬷嬷了,治不好的。哪怕日后生完了孩子,也总还会有白色的纹路在。如果他日侍寝,皇上看到我身上这样裂纹,会不会觉得恶心?''

如懿替她一件件穿好衣裳,道:``不会的,不会的。等你生下来孩子,咱们一定还会有别的办法的。''

海兰很快恢复了往日的镇定,将扣子一颗颗扣好,静静道:``这宫里不过是以色事人,所以从那一刻起,我已经知道,我这辈子的恩宠已经完了。我位分低微,孩子生下来未必能养在自己身边。若是送去阿哥所,还不如放在姐姐身边抚养,也就等于是我自己看着他长大了。''

如懿抚着她的手安慰道:``你若放心孩子在我身边,我一定视如己出。''

海兰挽着她的手出去:``姐姐別只管担心我,左不过是我自己的缘故,孩子平安就好。倒是姐姐\ldots\ldots{}''她看了看四周,压低了声音道:``那批砒霜,没给姐姐留下余毒吧?''

如懿含笑道:``有你和江太医把握着分寸,安心就是。若真毒坏了,我哪里还能站在你面前呢。''

海兰眼中闪过一丝沉稳笃定的笑意:``有的时候为了活命,为了反击,只能兵行险招。只要姐姐没事,那就好了。''

如懿送了她回去,见她虽是笑者,心屮却也不免担忧。整个后宫之中,只有海兰真心真意对她,那是日久见人心的情分。可是海兰,虽有了身孕的荣宠,但是未来如何,实在渺不可知。自已能做的,也唯有替她尽力抚育孩子而已了。

这样想着,便也到了晚膳时分,如懿与惢心在冷宫中简衣素食了许久,骤然看到十数道菜色一一上桌,也不免有些慨然。她大病初愈,胃口并不太好,每样菜略略尝了一口,便都赏给了下人,方才留了三宝和惢心嘱咐道:``仔细看着底下的人,断不能再出笫二个阿箬了。''
`

三宝肃然道:``都仔细盘查过了,李玉公公亲自挑的人,已经算小心了。不过奴才还是会仔细留意的。''

惢心亦道:``从前吃过这样的亏了,咱们都会一万个小心的。''

如懿微微颔首,踱步到庭院中,看着清露寒霜,凝在月色金明的瓦檐上,遥望着宫殿楼阁起伏连绵。这样熟悉的气息,细腻的脂粉气中带着各色香料混合的甜香,那是宫中特有的气息,一丝一缕沁入心脾,她深深地吸了几口,终将清冷的寒气缓缓透入肺腑之中,提醒自己要时时保有着这样的清醒。如懿凝神片刻,吩咐道:``惢心,替我更衣。''

如懿换了清简寡淡的装束,通身一袭云紫色如意襟暗纹锦衫,发髻间的珠花也以银饰为主,颇有洗去繁华的素雅之意。她披上夜行的墨绿弹花藻纹披风,扶着惢心的手茕茕独行,直至慈宁宫门前。

前去通传的福珈没有半分惊诧之情,仿佛料定了她会来,只一福到底,道``小主请吧。太后已经备好了茶等您呢。''

如懿翩然入内,数年不见,慈宁宫中的布置越发大气精雅,看似都是极古朴的东西,可是一一细辨去,每一样都是名家至宝,是洗练后的奢华。那才是真正的天家富贵,旁人总说白玉为堂金作马,金堆玉砌繁锦绣,殊不知真正的华贵富丽,是洗褪的金沙隐隐,从不是显露于表面的珠光宝气。亦可见,这些年太后稳居后宫,过得并不错。

如懿深深福了一福,道:``久未向太后娘娘请安了,太后万福金安,福寿延年。''她抬起头,只见太后笑吟吟的,便道:``太后一向喜欢焚檀香,今日怎么不焚了?''

太后微微一笑:``留了上好的茶给你,若用了檀香,反倒冲了茶香的好气味。坐下吧。''

如懿含笑往榻边坐了:``太后知道臣妾今夜必定会来?''

太后抬手端起桌旁放着的定窑茶盅,用盖碗撇去茶叶末子,啜了口茶,袖子落下,露出一段手腕,腕上一只蓝宝石的镯子,蓝得像一汪深沉不见底的海水。她推了一盏给如懿:``是上好的小龙团,原是宋朝的茶叶精品,你尝尝。''她的眼神笃定而温和:``你若不来,岂不辜负了哀家的好茶?''

如懿轻轻啜了一口,恭顺道:``臣妾不敢辜负。''

太后盘腿坐着,胸前一汪琉璃翠的流苏佩长长地坠落,静静蜿蜒而下。那样的颜色,总是让人看了心静。半晌,太后才笑了一声:``皇上没有白心疼你'哀家也没有白心疼你。你到底是熬出来了。''

如懿低首道:``有太后挂怀,臣妾不敢自暴自弃。''

太后点点头道:``你也算乖觉,知道一把火烧得你冷宫里待不下去了,便兵行险招拿自己作筏子。现在满宫里连着皇上都疑心是慧贵妃或是慎贵人给你下的砒霜,连皇后都逃不脱疑影儿,可是哀家却想知道,如果不是自己给自己下毒,哪里还能保得住命等人来救?''

如懿心中一沉,只觉得背心凉透,已然情不自禁地跪下:``太后英明,臣妾也不敢欺瞒太后。''

太后瞟她一眼:``你倒老实。''

如懿俯首低眉:``臣妾敢欺瞒所有人,也不敢欺瞒太后。''

太后蔼然一笑,伸手扶她:``好了,大病初愈的,别动不动就跪。也难为皇帝疑心她们,原是她们做得过了,一而再,再而三不肯放过你,否则也不会逼得皇帝立时把你从冷宫放出来。只是既然出来了,以后,你有什么打算呢?''

殿中漏声淸晰,杯盏中茶烟凉去。如懿立在太后身旁,听着纸窗外冷风吹动松竹婆娑之声,仿佛自己也成了寒风冬夜里摇曳无依的一脉竹叶:``臣妾本无所依靠,唯有凭太后一息怜悯得以苟延宫中。往后一切,还请太后垂怜。''

太后微微颔首:``你既懂事,自然是好的。皇后富察氏出身满族显贵,有老臣张廷玉支持。慧贵妃的父亲高斌在朝中得皇上倚重,是汉臣中的翘楚;慧贵妃一向依附皇后,两人互为援引。哀家不喜欢宫中只有一蓬花开得艳烈,百花盛放才是真正的三春胜景。你若能明白这一点,便也能好好生存了。''

其实如懿也有一瞬的疑惑,太后已经位高权重,为何还要如此在意?念头一转的瞬间,她忽然想起一事,忙屈膝道:``太后所出的端淑长公主已经许嫁蒙古,如今只剩了柔淑长公主养在庄亲王府中,臣妾无能,自居深宫,一定会替两位公主好好孝敬皇太后,侍奉太后颐养天年。''

太后闻得此言,似乎触动心肠,神色也柔和了不少:``你既明白,哀家便收你这一份孝心。''

如懿闻言,亦放心不少,才起身告辞。

回到宫中,如懿也便歇下了。独居翊坤宫的第一夜,她梦到的人居然是自己已经逝去的姑母。她穿戴着皇后衣冠,鬓发花白却风姿不减,只是向她含笑不已。记忆中,那应该是她第一次得到姑母首肯的笑容,哪怕她一直畏惧姑母,可是此刻,亦觉得她的笑如此亲切,带着乌拉那拉氏特有的骄傲,意态清远。''

或许这样骄傲而笃定从容的笑意,也是她此后半生,着意追寻的吧。

\hypertarget{ux7b2cux4e8cux5341ux4e03ux7ae0-ux6069ux5ba0ux4e0a}{%
\chapter{第二十七章
恩宠(上)}\label{ux7b2cux4e8cux5341ux4e03ux7ae0-ux6069ux5ba0ux4e0a}}

如懿回宫的第一夜,皇帝并未留宿在她宫中,只是如常召幸了新卦的舒嫔,倒叫许多人松了一口气。第二日的定省,如懿也不敢疏忽,早早去长春宫中见过了皇后,皇后嘱咐了几句,细问了她饮食起居是否习惯,便也嘱咐众人散了。纯妃见她出来,自然是还高兴的。倒是嘉嫔与慧贵妃一身对她淡淡的,也不亲热。而阿箬,更是对她退避三舍,视而不见。

或许,这样也是好的。''\textgreater\_\textless``"

如懿出冷宫后三日,皇帝倒也常常去见她,只是并未召幸,也不留宿,却让旁人也看不懂这恩宠如何了。这一日恰逢立冬,宫中备下了家宴吃饺子,除了太后畏寒不肯出慈宁宫,宫中的嫔妃倒是齐全了。

所谓家宴吃饺子,原本是因为立冬乃秋季与冬季的交子之时,宫中嫔妃长日无聊,便由各宫都自己做了饺子,凑成一宴,讨皇帝欢心而已。皇帝白是里去京郊察看了农桑,回来听皇后说起,倒也高兴,便在长春宫赐宴。嫔妃们自然是别出心裁,除了寻常的菜馅儿肉馅儿,又做了海鲜馅儿的,酸菜馅儿的。独独皇后和舒嫔最有心思,皇后的饺子是用过冬刚摘下的嫩白菜叶子做的皮儿,为的是京中人人都惯于在冬日囤积白菜过冬,也是勤俭而新鲜的吃食。皇帝对这样的心思自然是赞许不已的。而舒嫔的那一道,中逼着皇帝非咬了那一口,辣得皇帝眼泪都出来了,又好生敬了一杯酒灌足了,方才笑靥频生,道:``这样的饺子吃过了,皇上往后再吃到什么饺子,都不会忘了臣妾的了。''

皇帝笑得不止,击掌道:``皇后,你看也那个矫情样子,比慧贵妃往日如何?\textless{}''

皇后温婉含笑,只是不语。慧贵妃饱含了醋意道:``皇上不就是喜欢舒嫔这样的矫情样子么?何必拿臣妾来比呢。''

到了如懿时,她却中捧出一壶醋来,含笑道:``臣妾比不得各位姐妹的手艺,做不好饺子,特意用红玫瑰花瓣酿了一壶醋来。吃饺子少不得醋,臣妾就当略作点缀吧。''

皇帝薄薄的笑意却温煦异常:``朕若是吃饺子,必少不得醋,否则也是食不甘味。你的东西虽不是最要紧的,却是最不能少的。''

皇后注目含笑道:``你这点点缀,却是怎么也少不得的。娴妃,难怪皇上对你如此牵挂,连在冷宫里都要一意放你出来呢。''

如懿不卑不亢,只是略略含了淡薄的笑意:``有皇后娘娘日夜挂怀,皇上与皇后夫妻一心,自然也是挂怀臣妾的。''她转过头,看着打扮清贵却神色郁郁的慎贵人道:``阿箬,你也是一样的,是不是?''

此时阿箬已是皇帝的妃嫔,如懿仍以旧时称呼相对,显然未曾把她十分放在眼里。慎贵人眼中闪过一丝恼怒,强忍着不敢发作,只是闷头灌了一盅酒。

皇帝望着阿箬,和颜悦色笑道:``慎贵人是该喝酒尽兴。如懿为慎贵人旧主,如懿脱离冤屈,终于让朕知道她不是谋害怡嫔与玫嫔皇嗣之人,沉冤得雪。慎贵人乃是如懿的旧仆,理应同庆。''

皇帝字字句句,呼阿箬为``慎贵人'',对如懿只以名字相唤,亲疏早已十分明显。阿箬最恨旁人提她是如懿的旧婢,早已窘得满面通红,握着酒开盏的手轻轻发颤。皇帝却话锋一转,只笑道:``为表你主仆二人同庆之意,朕便打算封你为慎嫔,你意下如何?''

这样骤然封嫔,比之舒嫔的恩宠万千,出身显赫,更是出人意料。且嫔位是一宫的主位,身份贵重,宫中已有玫嫔,舒嫔与嘉嫔,不是生子,便是家世显要,且获宠多年,仅次于抚养两子的纯妃和在潜邸便为侧福晋的娴妃如懿,地位不可谓不贵重。如此一来,不禁连皇后亦变色,还是嘉嫔忍不住道:``皇上便这般喜欢慎妹妹么?慎妹妹与臣妾住在一起,岂不是启祥宫有了两位主位了?''

皇帝举了酒盏在手,唇边含了一缕俊美笑意``自然。若不喜欢,朕也不会亲自取了`慎'字为慎嫔的封号。''嘉嫔身躯咬了咬唇,隐忍着怨怒,皇帝眼波一转,却轻笑道:``正如嘉嫔你的封号,嘉为美好之意,朕也十分喜欢。所民哪怕慎贵人封了嫔位,启祥宫的主位也只有你一个。''

如此嘉嫔才稍稍平息醋意,却深深剜了阿箬一眼。阿箬逢了这样的恩赏,本该高兴不已,可那高兴也是损兵折将的,她只好撑着站起来,冷汗涔涔地行礼:``臣妾多谢皇上厚爱。''

皇后一袭天水鹅黄的衣裳,耳边一对珊瑚坠子摇曳生辉,笑得极柔和,道:``方才敬事房的人来了,在外候着呢。看来皇上今夜是要陪慎嫔,不必再翻牌子了。''

皇帝握一握皇后的手道:``果然皇后知朕心意。''

皇后向着阿箬温和道:``那么慎嫔,你先回去准备着去养心殿侍寝吧。''

这句话恰到好处地解了阿箬的尴尬,她才起身,嘉嫔便道要回去看四阿哥,也起身告辞了。海兰有着身孕不便,如懿便也陪着她先回去,只留了舒嫔与玫嫔二人随侍在侧,皇帝倒也十分惬意。

如懿扶着海兰正转过长街,却见嘉嫔站在慎嫔跟前,冷笑不已:``不要以为封了嫔位就目中无人,在启祥宫中主位只有一个,就是本宫。哪怕是嫔位,也有高低尊卑之分呢。你索绰伦氏不过是小姓出身,你阿玛再有治水的功绩,也不过是在慧贵妃父亲手下当差,小小知府而已。''

阿箬扶了侍女的手,倒也毫不退怯,只是笑吟吟道:``姐姐是嫔位,我也是嫔位,我年纪比你小,自然该尊您为姐姐。至于别的,大家都是皇上的妾侍,平平起平坐罢了,谁又比谁高贵呢。''

嘉嫔气得神色大变,却也自矜身份:``平起平坐?且不说本宫是皇四子的生母,玫嫔虽然出身南府,好歹生过孩子,奖历怎么也比你高些。舒嫔更不用说,叶赫那拉氏女儿,又是太后亲选赐予皇上的。若要论资排辈,本宫自然是嫔位中第一,玫嫔与舒嫔再次,你不过是屈居末流而已。''

嘉嫔的侍女丽心也是个口舌伶俐的,立刻道:``还没恭喜慎嫔娘娘呢,为着您的旧主娴妃娘娘出了冷宫,皇上才赏您这个嫔位,口口声声还提着您与娴妃娘娘的主仆情分。其实想想也不对,当年是你揭发了娴妃娘娘毒害玫嫔与怡嫔的皇嗣,今日皇上却金口玉言说娴妃娘娘蒙冤。依奴婢看,这封赏嫔位竟是在打您的耳刮子呢。''

阿箬扶了侍女新燕的手,禁不住浑身乱颤,伸手朝着丽心的脸颊便是一掌。她手上戴着纯银的玳瑁护甲,那一掌用力极深,便在丽心白嫩的面颊上留下了两道血痕。

丽心到底有些害怕,纵然满眼里泪水乱转,却中能捂着脸不敢出声。如懿冷眼看着,笑道:``这里风大,要不要先回去?''

海兰抚着肚子道:``这样好看的戏,我肚子里的孩子合该多看看。长大了也不至于吃旁人的亏太多。''

如懿替她正一正风帽,二人相视一笑,便在暗处站定了不动。

嘉嫔看着丽心挨打,却换了和颜悦色的笑容,娇声道:``哎呀,梅香拜把子------都是奴才罢了,何苦自己人打起自己人来了。丽心,好歹人家已经熬成了小主,你便受她这一掌,当受教了,也学学她怎么没日没夜爬了皇上的龙床。''

丽心捂着脸道:``奴婢可不敢背着自己的主子偷偷勾引皇上这么没廉耻,更不敢背弃主子诬陷主子。不管挨了慎嫔娘娘多少巴掌,奴婢都是学不会这些下三滥的本事的。''

嘉嫔连连颔首微笑,骤然伸出手打了阿箬一个耳光。这一掌去得又快又狠,出乎阿箬的意料,她根本招架不住。嘉嫔脸上笑得悠然自得:``这一掌,是教你学乖,尊卑自在人心。别以为得了位分,得了皇上的宠幸,旁人就忘了你是怎么使尽下作手段勾引的皇上,连奴才们都瞧不上呢!''

嘉嫔得意的轻笑声落在风里格外响亮,被宫人们簇拥着一摇三摆扬长而去。阿箬慢慢地抚着脸颊,自嘲似的笑道:``新燕,你瞧,人人都瞧不起我。哪怕我封了嫔位,在她们眼里,我不过是个奴婢罢了,永远只能是个上不了台面的奴婢。''

新燕忙扶着她,好声好气道:``小主别往心里去,嘉嫔不过是仗着自己生了个皇子罢了。她自己也不过是个贡品似的异族贡女罢了,小主可是纯正的满洲血统呢,来日若生下了一儿半女,岂不比她尊贵。本来呢,您还没有子息,皇上就那么宠爱您了。''

阿箬的笑声里带了几许哭腔:``你也觉得皇上是宠爱我的?''

新燕奇道:``小主,您这是怎么了?皇上常常翻您的牌子,赏赐也是最多。哪怕舒嫔新贵得宠,皇上也没忘了您呀。您看,嘉嫔再嚣张刻薄,也不过是妒忌您罢了。''

阿箬神色凄惶,连连点头道:``是啊,她们都是妒忌我,她们都是妒忌本宫。可是是谁把我抬到这种人人妒忌刻薄的地方来的。我承宠这些年,除了皇后和慧贵妃,几乎没看过旁人的好脸色,连慧贵妃,偶尔也是冷嘲热讽的。到底是谁把我拱到这种人人为敌的地方来的?''她的口腔越来越悲怆:``皇上翻我的牌子最多,可是谁知道\ldots\ldots{}''她说到这里,却捂着嘴不敢再出声了,只是畏惧地看着四周,怆然落下泪来。

新燕不解其意,只得道:``小主别伤心了,今儿是您封嫔的大好日子,等下还要侍寝呢。奴婢赶紧陪您回宫,替您拿鸡蛋揉揉脸,别叫皇上看见了,可不好呢。''说着,连搀带扶陪着阿箬走了。

如懿听得有些疑惑,便问:``皇上翻阿箬的牌子最多,难道有什么不对么?''

海兰也是疑虑重重:``这些年阿箬可算是恩宠深厚,皇上对她颇为厚待,屡屡晋封赏赐,能有什么不妥?可是听她今日这话,怕是有些缘故在里头呢。也是,集了一身宠爱,难免招怨。偏她的根基又不够厚,自然谁都能撂脸色给她看了。''

如懿冷冷道:``荣华富贵是她自己求的,自然了,这种羞辱欺凌,也是她自已求得的,还有什么可怨恨的?''她扶住海兰的手:``我看你晚膳用了那么多,不过几个饺子而已,便这么开胃么?可别撑着了,还是传江太医来瞧瞧吧。''

海兰回到宫中饮了一盏消食茶,笑道:``才喝了消食茶,又觉得有些饿了。叶心,你去瞧瞧,小厨房有什么可吃的?''

叶心答应着去了,如懿道:``虽说过了四个月胃口会大好,但你也有六个多月身孕了,怎么还是这样开胃,吃得大多,旁的倒没什么,倒是你身上更见胖了。''

海兰苦笑道:``我还能有什么办法,左右身上是不能见人了,若再不吃一些,怕亏了肚子里的孩子,更不值了。''

正说话间,叶心端了一又能豆腐皮包子并一碗虾仁馄饨上来。海兰才吃完,江与彬便进来请了安道:``娴妃娘娘万福,海贵人万福。''

如懿笑着招手道:``无事也非得叫你来看看,你看海贵人,怀着身孕一天吃许多顿,胃口好得教人害怕,到底是怎么了?''

江与彬搭了脉,看着桌上的空碟子道:``海贵人胃口大开,无妨啊。不过看着,是比前几日又圆润了些。''

正说着,绿痕端了一盏药上来道:``安胎药已经成了,贵人快喝吧。''

海兰端起碗正要喝,江与彬忽然止住,道:``小主是按着微臣开的安胎药方子喝的么?''

海兰立时警觉,放下药碗:``怎么?有什么不妥么?''

``味道似乎不太对?''江与彬立刻接过药碗一嗅,即刻吩咐绿痕:``把剩下的药渣拿来我瞧瞧。''

绿痕知道利害,立刻去了,不过片刻用盘子装了一把药渣。江与彬抓起药渣嗅了又嗅,又拣起一点放在口中仔细嚼了,奇道:``奇怪,味道虽然不对,但居然加的不是害人的药。''

如懿急道:``那到底是什么?''

江与彬道:``微臣断然不会尝错,微臣开的安胎药里被人足足地添了别的东西,可这东西不是坏东西,是开胃的好药,可的确不是微臣方子里有的。''

如懿转念道:``开胃的好药?是不是吃了会胃口奇好,不断进食,然后发胖。一旦发胖\ldots\ldots{}''

江与彬道:``孕中发胖,也是常见的,只是海贵人胖得比常人快,大约是跟这个药有关。孕妇胖得快呢,身上的肌肤承受不住,便容易开裂形成纹路。''

海兰已然明白,眼中哀戚愤恨之色大盛:``而这种纹路,哪怕生产之后,也无法裉去,终身附着身上,让人不忍目睹,是不是?''

江与彬目瞪口呆:``贵人这么说,难道\ldots\ldots{}''

海兰紧紧握住手臂,恨声道:``已然生在身上,无法根除了。''

江与彬凛然道:``贵人放心,微臣一定尽心尽力,替贵人研习药性,力求除去。''

海兰紧紧握拳,含泪道:``你是有心了。只是我的药一直是绿痕照管着的,绿痕是信得过的人,这些开胃的药又是怎么加进去的?''

绿痕慌得赶紧跪下道:``小主明鉴啊小主,奴婢从太医院领了药来就小心谨慎,连着煎药到端到小主跟前,都没有旁人插手过啊。奴婢更不懂得什么药材能开胃,断断不敢擅自加在里头了。''

江与彬沉吟道:``药方是微臣开的,药材是太医院的人抓的,配好之后微臣看过了无妨。但太医院人多手杂,在交到绿痕姑娘手中前被人动了手脚也未可知了。微臣回去之后,必是细察。''

海兰忍着泪,脸色渐渐沉着,沉吟道:``这事细察出来是谁便可,不必声张。''

江与彬满脸疑惑,如懿含着恨意叹息道:``换了我,也决不能相信无端端加了这个药是为了你好。倒是出这个主意的人,借着与人无害的样子行阴毒之事,实在是可怕可恨。只是这事即便张扬了开来,皇上也只会以为那人是无心之失甚至是好意为之,倒成了咱们小人之心了。还是不说也罢。''

海兰双拳紧握,手背上青筋突起,仿佛一条条蜿蜒的青色小蛇,咝咝地吐着芯子:``这样会算计人,真当是厉害!我算是记住了,只当自己吃一堑长一智吧。只是江太医,以后得劳烦你多费心了。''

江与彬赧然道:``娴妃娘娘在冷宫里,微臣难免分心,不能面面俱到。说来,也是微臣失职。往后,微臣一定会格外小心的。另外,待贵人生产之后,微臣也会配好药膏,给贵人涂抹身体,以求消去纹路。''

海兰静静地望着外头漆黑如墨的天色,仿佛是望着自己望也望不见的前路。她眼中泪光一闪,终究是忍住了,轻声道:``姐姐,我只有你和孩子了。''

如懿安慰地拍着她,和她紧紧依靠在一起。她们的影子落在墙上,像一道单薄的剪影,若是哪一阵风吹得大些,便要一同吹去了似的。

阿箬裸露着身体,从被子底下一点点努力地钻上去。黑洞洞地被窝里,她感觉得到皇帝年轻的身体就在她身侧,隔着薄薄的丝绸寝衣,散发着热烈的气息。她熟门熟路地从被窝里探出头来,望着明黄色的宫样帐楣,密密的龙腾祥云绣花,账外的烛火照在上头,混淆着帐上所绘碧金纹饰,华彩如七宝琉璃,璀璨夺目,直刺入心。

她紧紧地拥住皇帝,想要伸的解开他寝衣上第一颗扣子。皇帝一动不动,只是嗤地一笑,带着冷冷的余音,吓得阿箬赶紧缩回了手。

皇帝的口吻平静得没有一丝波澜:``你在做什么?''

她鼓足勇气仰起了脸,望着皇帝如盛开的康棣般炫目的面庞,低低哀求道:``皇上允许奴婢侍寝,奴婢\ldots\ldots 奴婢是来侍奉皇上的。''

皇帝眼底全是薄薄如冰屑的笑意,随手抖来赤色捻金龙纹缎被,散漫看了一眼道:``哦。已经脱得一干二净,是来侍寝了。''

阿箬面红耳赤:``规矩如此,奴婢也是遵照祖制而已。''

皇帝微微一笑:``你也知道你是奴婢。你侍寝三年了,自然学会了如何侍寝,还要按着敬事房那一套来么?''

深赤色的缎被上,以玄黑丝线绣着狰狞的五爪蟠龙,龙爪以金线刺绣而成,尖亮锐利宛如鲜活,似乎一爪一爪都要挠进她的血肉中去。阿箬顾不得害羞。以自己鲜活的肉体贴附在皇帝身上,想用自己的滚烫去温热他,婉声求恳道:``皇上,皇上,求您疼一疼奴婢吧。奴婢侍寝三年,只有第一次\ldots\ldots 第一次您受了奴婢的侍寝。这么久了,就让奴婢再伺候您一次吧!''

皇帝斜靠在自己手臂上,一手漫不经心地拂过她的身体,脸上虽然带着那样疏懒的笑意,目中却只有清寒的冷薄:``是么?朕第一次许你侍寝,是你求仁得仁,一心只想做朕的女人。朕许了你,也是告诉你,你这一辈子,既然侍寝过朕,那么生是紫禁城的人,死也是紫禁城的鬼,老死也出不去半步了。可朕之后每每翻你的牌子,召你侍寝,也赏赐你,给你荣华位分,但再没有碰过你,你却不知道为何么?''

阿箬又窘又羞,愧恨难当,只是无言:``奴婢愚昧。''

皇帝的脸色慢慢冷下来:``既然知道自己只是奴婢,而非臣妾,就不要妄想躺在朕的身边。''

阿箬满脸紫涨,殿中并无她的衣物,只得扯过床上的薄毯,匆匆披上起身。

皇帝淡淡道:``从前怎么伺候朕过夜的,还是老规矩。''

\hypertarget{ux7b2cux4e8cux5341ux516bux7ae0-ux6069ux5ba0ux4e0b}{%
\chapter{第二十八章
恩宠(下)}\label{ux7b2cux4e8cux5341ux516bux7ae0-ux6069ux5ba0ux4e0b}}

阿箬赤着脚,跪倒在塌边。皇帝寝殿本是金砖墁地,那地砖油润如玉,光亮似镜,质地密实,脆若金石,虽然上头铺了厚厚一层锦毯子,仍是禁不住那寒意和坚硬逼迫上膝盖,一点一点触痛了神经。

皇帝闲闲地看着她,漫然道:``朕一直留你在身边,给你这么高的荣宠位分,是有留你的作用。但是你别妄失了分寸,你永远是娴妃的奴婢,朕的奴婢,人前人后,你自要分的清楚。''

起初的时候,这样的言语也让阿箬觉得羞愧欲死,然后这些年下来,每每如是,她也渐渐习惯了,只是麻木的道:``奴婢知道。''

皇帝正欲转身,忽然察觉她脸上的红肿,便问道:``挨了谁的打?\textless{}''

阿箬愣愣地道:``皇上宠爱奴婢,嘉嫔娘娘不忿,打了奴婢。''

皇帝打了个哈欠:``打了就打了,哪有为奴为婢不挨主子的打的。你心甘情愿要得这些恩宠,就要心甘情愿受这些罪。''

皇帝床帐的帷帘内疏疏朗朗地悬挂了三五枚涂金镂花银熏球。那熏球镂刻着繁丽花纹,精雕细镂,缠枝纹样清晰可辨。球内盛有安息香,丝丝缕缕缠扰的香气喷芳吐麝,悠然隐没于画梁锦绣之上,仿佛她的前程,也这般无声无息地弥散殆尽了。阿箬愣了片刻,忽然生出一丝凄微的笑意,终于忍不住道:``皇上,求您给奴婢一个明白。您既然宠幸了奴婢,也给了奴婢外人羡慕的恩宠,为什么您背过身要这么待奴婢?难道您是猫儿,当奴婢是一只卑贱的老鼠逗着玩弄么?皇上!''

皇帝转过身,伸手勾一把她的下巴,嗤嗤笑道:``朕已经成全了你,你还要怎样?记得朕给你的封号是什么吗?慎,就是要你谨小慎微,这么多年你都这样侍寝下来了,怎么今天倒沉不住气了?''

阿箬披着单薄的毯子,浑身颤抖,眼底闪过一丝凄厉的微光,磕了个头道:``皇上,求您给奴婢一个明白,您既然不喜欢奴婢,为什么要这样待奴婢呢?''

皇帝冷冷一笑:``不这么待你,谁知道你又要做出什么事来?你也念着朕的好吧,没朕这样宠着你,你早折在谁手里也不知了。''

阿箬咬了咬牙,苍白着脸道:``是不是因为娴妃娘娘的事,皇上觉得是奴婢冤枉了她?所以要这么折磨奴婢替她出气?''

皇帝的声音渐渐慵懒下去:``出气?谁要出气自己出去,朕懒得理会。''他翻了个身:``好了。朕乏了,有什么话,往后再说吧。''

阿箬跪在那里,看着皇帝沉沉睡去,发出均匀的呼吸声。外头的梆子声一声远一声近地递过来,她瘫软在地上,无声无息地落下泪来。

这样一跪,便是大半夜。接她回去的太监是二更十分到的,按着规矩在皇帝寝殿外击掌三下,低低喊了声``时辰到了'',便由李玉带着人重新将她裹了起来,送入养心殿后的围房穿戴整齐,用一顶小轿抬回她自己宫中。

阿箬受了一夜的折腾,回到自己宫中也是睡意全无。新燕端了一碗安神茶上来道:``小主侍寝,也累了半夜了,快喝了安神茶睡吧。''

阿箬含了泪冷笑道:``侍寝?我倒是真累着了。''她转头打量着宫里的陈设,突然怒道:``本宫已经是皇上亲口所封的慎嫔,为什么本宫宫里的陈设布置还是按着贵人的位分来的?内务府怎么这样惫懒不识好歹?''

新燕为难道:``方才内务府的人已经来过了,说皇上皇后都力图节俭,左右小主还没行册封礼呢,所以嫔位该用的东西也不摆上了。''

``册封礼?''阿箬刻毒一笑,道:``皇上何时说过要给我册封礼?原来不过是让我白担了一个虚名罢了。''她说罢,霍得起身,取过博古架上的琉璃花樽就往下砸,砸完了又把桌上几上能看到的瓶瓶罐罐都砸了个稀烂。新燕这一吓可非同小可,急忙拦下道:``小主,小主,您这是怎么了?今儿可是您刚封嫔位的大喜日子啊,怎么能动气呢?这若传出去,旁人可不知道要怎么议论您呢?''

阿箬发疯般地砸着东西,涕泪横流:``我怕什么?我还怕什么?这样生生被人作践,砸几样东西还不能么?我是慎嫔,我是慎嫔,这几样东西还砸不起么?砸了谁又能拿我怎么样?''说罢,她举起一个青玉佛台便要砸下去。

新燕吓得魂飞魄散,赶紧拦下道:``小主,小主,您可别糊涂了。这个佛台可砸不得呀,那是您封贵人的时候皇上赏的。小主,您要生气就打奴婢几下吧,可千万别砸了这个,更别气伤了自己的身子。''

阿箬满脸是泪,倒在床上哭泣道:``皇上?皇上眼里还有我这个人么?我不过就是件玩意儿,砸了也就砸了,根本就是任人作践的。''

阿箬心酸地哭着,哭得久了,也累了,昏睡了过去。新燕看着满地狼藉,叹了口气,蹑手蹑脚地收拾了起来。

趁着阿箬闹累了没醒,新燕一大早便往慧贵妃宫里走了一趟。慧贵妃正在梳妆,由着宫女蘸了桂花水,一点一点蓖着头发,听新燕说完,便有些纳闷:``昨夜她刚封了嫔位,又被召幸,正是得意的时候,有什么沉不住气的,偏要这样回来闹?''

新燕一无所知,只得摇头道:``奴婢也不知道,只是伺候了慎嫔这几年,只觉得她的脾气越来越暴躁,从前不过是动不动就打骂下人,有时候也问奴婢,皇上是不是真宠爱她?''

``皇上是不是真宠爱她?''慧贵妃疑惑地转过头,``自从娴妃进了冷宫,她的恩宠也算是多的了。如今即便娴妃出来了,她恩宠不衰,还想怎样?''

茉心一边替慧贵妃挽发髻,一边道:``皇上虽然宠她,但到底也看不起她,昨日立冬家宴上,一口一个主仆,分明是瞧不上慎嫔的出身。还说当年的事娴妃是蒙冤的''她忽然闪了一下梳子,扯到了慧贵妃的头发,忙吓得跪下了。

慧贵妃回头,不悦地横了茉心一眼,怒道:``做什么呢?你的爪子越来越不会当差了?''

茉心吓得直打寒噤:``小主恕罪,小主恕罪。奴婢只是想到皇上说娴妃蒙冤,会不会翻查当年的事,牵连到咱们。''

慧贵妃怒了努嘴,示意她起身继续梳好发髻,方懒懒道:``如今娴妃放出来了,皇上自然要找个借口说她蒙冤,否则怎么让人心服呢。再说了,真要细细研究起来,反正当日反口咬定娴妃下毒的人,不是咱们。''

茉心还是有些害怕:``小主说得是,可是慎嫔人不会咬出咱们来么?''

慧贵妃端详着镜中的自己金凤斜簪,云鬓半偏,翠钿疏散,取过一把透雕双凤纹玉梳斜插在脑后青丝上,看了看满意了,才道:``她阿玛到底在本宫父亲手下当差,她有几个胆子连累家人?再说了,她连自己的主子都能背弃,安知不敢冤枉咱们。好了,新燕,你就回去好好伺候着吧,慎嫔有什么动静,记得随时来回报。''

新燕答应着退下了。慧贵妃看了茉心一眼,佩上一对翠绿水滴耳环,容色淡淡道:``你有话要说?''

茉心道:``奴婢只是看不惯慎嫔罢了,一时这样得宠,连小主都越过去了,一时又这样闹脾气,不知检点。''

慧贵妃轻蔑地撇撇嘴:``也难怪她,娴妃出来了,她自然会怕。''

茉心道:``其实奴婢一直都不大放心。当初小主罚她跪在雨地里,后来她怎么肯为咱们所用?且这些年,连皇后娘娘都那么抬举她。''

慧贵妃嫣然一笑,百媚横生:``当初皇后娘娘亲自去笼络她,又将她阿玛调到本宫父亲麾下以作挟制,她才能安分效忠这么多年。不过从一开始,长春宫和咱们的意思都是一样的。阿箬,不过就是颗随时可弃的棋子。因为随时可弃,所以不在乎她如何得宠了。''

茉心满面堆笑道:``小主远见,奴婢实在不及。''

慧贵妃唇角扬起一抹得意的笑意,很快又收敛了,叹息道:``所有的远见,都是皇后娘娘的远见。本宫算什么,即便皇上抬旗,又倚重父亲,可本宫的出身到底摆在那,永远也洗脱不去。''慧贵妃黯然道:``而且本宫承宠多年,你闻闻,殿中的坐胎药气味浓得都散不去了,可本宫还是怀不上一儿半女。''

``可是皇后娘娘亲生的二阿哥也死了,不比小主好多少。''

``二阿哥死了,也被追封为太子。皇后娘娘好歹还生育过,好歹还有三公主,哪像本宫,本宫的肚子是空的,孩子一天都没有来过。''

慧贵妃越说越急,不觉泫然,茉心最怕她想到孩子,一想到便要伤心许久,忙劝道:``小主就是太心急了,所以一直怀不上孩子。只要小主放宽心,皇上又常来,那股子运气一到,自然想什么有什么了。小主,时候不早,咱们也该去向皇后娘娘请安了。小主去长春宫不是一向最勤最准时的么?''

慧贵妃看了看天色,颔首道:``是该走了。皇后再温柔谦和,到底也是满蒙显贵出身,本宫即便位分再高,也不能不依附她,才能在宫中站得更稳,走的更远。''

这一日宫嫔们齐聚皇后宫中请安,皇后看着如懿的手腕,温婉含笑若春水碧波:``本宫记得昔日赏赐给娴妃妹妹一串翡翠珠缠丝赤金莲花镯,怎么这些日子都没见妹妹戴着,可是不称心了么?''

如懿心头一凛,恍若一根尖锐的芒刺被人深深刺入,又呼啸拔出,她维持着面容上清淡适宜的笑容:``莲花镯上赤金丝有些松散了,得空得叫人去绞一绞才好。''

皇后颔首道:``可不是,那原本是一双一对的,本宫独留给了你与慧贵妃。若是让人绞好了,总要时时戴着,才是咱们潜邸姐妹不同寻常的情分。''

慧贵妃笑道:``皇后娘娘厚爱,臣妾日日戴在身上,一丝一毫也不敢松懈相待呢。''

如懿心中冷笑不止,却听皇后道:``皇上兴之所至,突然想到要放娴妃妹妹出冷宫,连本宫这个皇后也是事后才得知。可见这些日子皇上是有多想念妹妹了。''

慧贵妃插嘴道:``只是说来也奇怪,皇上即然这样爱重娴妃,怎么娴妃出来这几日,皇上都没有召你侍寝呢,反而是慎嫔妹妹伺候得多呢。''

如懿只是淡淡含笑,宠辱不惊:``若是以肉身相伴便为情爱珍重,那世人何必还要在意于情意呢?''

纯妃含笑道:``数年不见娴妃,说话倒是越来越有禅意了。''

如懿以温和的目光相迎,道:``纯妃姐姐有所不知,冷宫清静,便于剔透心意。我只是觉得,有皇上牵挂,能得以重见天日已是难得,何必还妄求肉身贴近。''她转眸凝视皇后:``何况即便夫妻日日一处,同床异梦,表面讨人欢喜,私下做着对方不喜不悦之事,又有何意趣呢?''

皇后浑然不以为意:``娴妃这话本宫听着倒很入耳。皇上是一国之君,更是后宫所有人的夫君,只要皇上心里有你们,何必争宠执意,争夺一时的宠幸呢?如娴妃一般淡泊无为,其实才是更有所为呢。''

嘉嫔哧一声笑道:``咱们自然比不得娴妃娘娘的本事,连娴妃娘娘身边昔日伺候的人,都成了精似的厉害,抓着皇上不放呢。''

嘉嫔一向抓尖要强,皇后也不理会,只道要陪三公主习字,便吩咐各人散了。如懿扶了惢心的手才步出长春殿庭院,却听后头一声呼唤,``娴妃娘娘'',转头过去,却见阿箬扶着新燕的手急急上前,拦在她身前道:``娴妃娘娘留步,我有一句话,一定要向娘娘问个明白。''

惢心恭谨地向她福了一福,恪守着奴婢见小主的礼仪。阿箬的脸上闪过一丝凌蔑的得意。如懿不欲与她多费口舌,便问:``什么事?''

阿箬逼近一步:``听说娴妃在冷宫被下毒,皇上前往探望,出冷宫后皇上又见过你一次,你是不是对皇上说了什么?''

如懿抬了抬下吧,骄傲道:``你以为本宫说了什么?''

阿箬的脸有些扭曲,急道:``你是不是告诉皇上,是我给你下的砒霜?你是不是告诉皇上,当年的事是我陷害了你,冤枉了你?''

如懿清朗一笑,迫视着她道:``本宫说了什么很要紧么?本宫见了皇上几次,你侍寝又见了几次,这些年你常常陪在皇上身边,难道见的面说的话不比本宫多么?还需要在意本宫说了什么?皇上宠信你,自然会信你,你有什么好怕的?''

阿箬面色苍白,与她以粉珊瑚和紫晶石堆砌的鲜艳装扮并不相符,她踉跄着退了一步,强自撑着气势道:``我有什么好怕的?我自然什么都不怕。''

如懿的目光从她身上拂过,仿佛她是一团空气一般透明无物:``你能这般自信无愧就好了。人呢,疑心容易生暗鬼,你要坦荡就好,自然不会把你心里的鬼带到皇上心里去。可你要是自己把自己心里的鬼带给皇上了,那就不必旁人说什么,皇上自然也疑上你了。''

说罢,如意正见纯妃出来,向她招着手,便笑吟吟上前,陪着纯妃一同走了。纯妃朗声笑道:``你也是。和她费什么话,忘了当初她怎么害你的么?''

如懿浅浅微笑:``我没忘,她自然更忘不了。''

纯妃亲热地挽过她笑道:``大阿哥一直养在我宫里,可想着你了。你若得空,便去我宫里坐坐吧,也看看我带大阿哥尽心不尽心?''

如懿忙道:``姐姐说这话便是寒碜我了。大阿哥养在姐姐宫里,那便是姐姐的孩子,自然没有不尽心的,我巴巴儿的跑去,算是什么呢。''

纯妃笑道:``只是因为妹妹受了委屈,所以大阿哥暂时养在我宫里。如今妹妹出来了,迟早也是要还到妹妹宫里的。这样,嘉嫔有四阿哥,我有三阿哥,妹妹也有大阿哥,那大家都是一样的了才好呢。''

如懿见她说得半真半假,一时倒也不敢应对,只好笑着道:``纯妃姐姐说哪里话?你到底是生养过三阿哥的,自然比我更会抚养孩子,不像我毛手毛脚的。且姐姐不知道呢,姐姐看方才阿箬对我的口气,我虽出来了,怕也是被人虎视眈眈,自顾不暇呢,哪里还照顾得到大阿哥!''

纯妃大量着她道:``那妹妹的意思是大阿哥便一直养在我宫里了?''

如懿谦和微笑,推心置腹道:``我本不是大阿哥的亲身额娘,如今姐姐养育得大阿哥这样好,我又怎敢腆着脸要了大阿哥去,便是皇上也不肯啊!''

纯妃不动声色地吁出一口气,拍着她的手关切道:``如今妹妹先把身子养好,慎嫔那狐媚子魅惑皇上多年,又目中无人,得空必得好好料理了她,妹妹才能出当年那口恶气呢。''

如懿笑盈盈道:``有姐姐这份心意,我便安心了。''

接连几日下去,阿箬便称病一直不出门了。如懿唤来江与彬一问,方知阿箬气急交加,是真病了。病的缘由无从得知,却总也叫人有点揣测,太医院的药轮番端进去,阿箬也不见得好,见过的人只说,人都干瘦了下去,是病得厉害呢。

如懿得知也不过轻弹指甲,她才刚出冷宫几天,阿箬便自己被自己弄病了,落在他人的口舌里,总以为阿箬是心虚,又禁不住去揣测,是不是给如懿下砒霜,是她的主意。趁着阿箬这样病着,惢心也有些沉不住气,私下里便对如懿道:``小主若是不愿意,这样的腌攢事便交给奴婢去做吧。反正当年害小主的人实打实就是阿箬,咱们就算害她一回,也是有冤报冤,有仇报仇。''

如懿轻轻啜着碧清的茶水,便道:``那么你待怎样?''

惢心咬了咬唇,眼中却毫无畏惧之色:``不过是找江与彬,给她下点好东西罢了。''

如懿取过桌上一枚香砌樱桃,慢慢含了道:``不妥。我听着前几日阿箬的口气,越发觉得皇上待她并不是只像咱们看到的一般。既然皇上并不如表面这般待她好,说了我是蒙冤受屈还要对她位分不降反升,一定是有所道理。这个时候,倒不便咱们下手了。''

惢心见如懿有了主意,也不好再劝。倒是江与彬来请脉时,如懿暗地里嘱咐道:``阿箬的病既然是心病,那么不要治好了她,也不要治坏了她。''

江与彬抬眉一笑,似有千万把握:``小主的吩咐,太医院上下都接到过了。每一位太医都心中有数。''

如懿闭目片刻,闻着殿外幽幽梅香,清寒入鼻:``是皇上?''

``皇上,与皇后。''

如懿的心思却不在阿箬身上,问道:``还有一件事,我一直想不明白,近日我见慧贵妃,看她气色大不如三年前了,慧贵妃与我一样,都得过皇后那串掺了零陵香的手镯,为什么还有人要多此一举给她下那些让她身体病得更重的药,是怕零陵香药力不够么?''

江与彬沉吟道:``或者有人防慧贵妃比防小主更甚。更或者有人与皇后娘娘不谋而合。''

如懿微微沉吟,将锦匣中所藏的碎珠玉镯取出,交到江与彬手中:``你去,找外头靠得住的人,将里头的零陵香丸取出,玉镯我如常戴上,也好让皇后安心哪。''

江与彬收过,眼中满是脉脉情意,看了一眼惢心道:``小主的吩咐,微臣自当尽心力竭。''

如懿点头:``帮过我的人,忠心于我的人,我都不会忘记,自会一一还报。对了,凌云彻''

``小主放心。按着小主的吩咐,已经调出了凌云彻。如今,他已经是戍守坤宁宫的侍卫了。''

本该是帝后大婚所居的坤宁宫,自顺治朝后便成了萨满敬神之地,既尊贵,又清静,果然是个好去处。

如懿仰起头,看着窗外澄碧的天空,暗暗想着,如此,也算是给了凌云彻一个好出路了。自然,往后如何,还是看他自己了。

人人,都只能由着自己走完这条路,无一例外。

\hypertarget{ux7b2cux4e8cux5341ux4e5dux7ae0-ux4e8bux7834}{%
\chapter{第二十九章
事破}\label{ux7b2cux4e8cux5341ux4e5dux7ae0-ux4e8bux7834}}

这一日,冬雪绵绵初至,如懿贪看雪中白梅的景致,便扶了惢心一同出来。冬寒森冷,苑中白梅寂寞地开着。在这清寂少人行的午后,妖娆地绽放勃然的花瓣。惢心笑道:``小主也真是的,旁人踏雪寻梅,都是寻的红梅,小主偏要去看白梅。奴婢倒不信了,白梅隐在白雪之中,只看得清黑压压的枝条,有什么好看的呢。''

如懿披着一件联珠锦青羽大毛斗篷,伸手接住一点纷飞的雪花,道:``白雪红梅自然有艳烈清朗之美,为人赏叹。但白梅隐藏白雪之中,只凭花香逼人与清寒彻骨稍作分别,世间的美,若不细细分辨,轻易得来又有何意味?''

惢心目中闪过一丝顽皮笑色:``奴婢倒觉得,小主是喜欢这种细细分辨的。\textgreater\_\textless''

如懿正了正领口绒绒的毛球,颔首笑道:``很多事若不细辨,便只能看到雪压黑枝,自然不觉得得美,只有走近细观,不被表象所迷惑,才知真美所在。''

她甫一说完,却听一把清婉女声在身后遥遥响起:``娴妃娘娘这番话,倒是深得我心。''

如懿转身,却见白雪琉璃之中,一个穿着挖云鹅黄片金里大红猩猩毡披风的丽人盈盈站在梅树底下,却是舒嫔。她便含笑,客气道:``原来是舒嫔妹妹。''

舒嫔兜下风帽,露出满头玉片与银器的点缀,在冬日寒雪中看来,越发显得高洁冷清,有着冰雪般寂寞高华的神情。也恰如她这个人一般,一眼看去是极艳丽鲜妍的,相处了才知道是那样孤清的性子,恰与这冬雪寒花一般。

舒嫔略略欠身道:``娴妃娘娘若不介意,可以唤我的本名,意欢。我也可以称呼一句姐姐,不必`娘娘'来`娘娘'去,这般俗气。''

如懿见她说话直接,心下更喜欢,便道:``那自然好。''

舒嫔澹然笑道:``后宫人人都在说,皇上放了姐姐出冷宫,却一直很少前去探望,也不曾和姐姐一同用膳,更未曾召姐姐侍寝过一次。宫中诸人都在背后议论纷纷,不知皇上究竟把姐姐置于何地?''

如懿见她毫不掩饰,便也道:``皇上天心如何,岂是我们可揣测的。\textless{}''

近处有大蓬梅花舒枝傲立,枝上承了脉脉积雪,花蕊花瓣越发显得冰清莹洁依然,不为尘泥所染。

舒嫔拨着鬓边一串银丝流苏,徐徐道:``旁人这么认为,我却不是。我一直在想,慎嫔曾经那么得宠,如今病了这些日子,皇上也是不闻不问。而放了姐姐出来竟也示多亲近姐姐,是不是近乡情更怯的缘故。倒觉得,皇上是更看重姐姐呢。''

如懿淡淡一笑:``妹妹方才是从何处来?''

舒嫔道:``陪皇上用了午膳。''她的笑容有点隐秘:``午膳时皇上最爱一道梅花锅子,是以白梅入菜,烹制的清汤浓味。却不想我走到御花园中,却看姐姐也这么巧,独自细赏梅花。''

如懿心头微微一动,像是谁的手冷冷拨动心的琴弦,面上的神色却极淡:``寒冬唯有梅花而已,想要凑巧也太简单了。''

舒嫔笑而不语,只是道:``姐姐不觉得这白雪白梅极美,但那黑黢黢的枝条却实在是太点眼了么?若换作是我,一定用白漆将它全涂没了,那才干净呢。''

一簇梅枝簌簌当风,风吹影动,风资绰绰,好似涟漪。如懿伸手折下一枝白梅在手:``原来妹妹不只快人快语,更是心思果决。只是\ldots\ldots 凡事不急才能好呢。''

舒嫔浅浅微笑,起身离去。

惢心有些担心道:``小主怎么和舒嫔说那么多话?咱们也不知道她的底细。''

``底细?''如懿看着白雪皑皑中她远去的鲜红背影,``舒嫔是太后举荐的人,又自恃清高,不愿与宫嫔妃来往。这样的底细,即使多说几句也是无妨的。''

她回转身,扶着惢心踱出园处,却见凌云彻捧着一束折下的悔花,守在外边不却。

如懿颇为意外:``你如今不是在戍守坤宁宫么?怎么在这里?''

凌云彻行礼如仪:``坤宁宫岁下清供,每日以梅花插瓶,所以都是微臣前来。''他悄悄望一眼如懿,仍是恭声道:``今日听得娴妃娘娘在里头说话,所以特意在园处等候,希望能向娘娘请安。''

如懿含笑凝睇:``梅苑出入只有这一道门,你特地守候,想来不是为了请安那么简单。''

凌云彻有些不好意思:``还是被娘娘看穿了。''

``有话便说吧。''

凌云彻踌躇片刻,思量着道:``花房有一个叫魏燕婉的宫女,她来找微臣\ldots\ldots{}''

如懿轻笑,打量着他道:``自己才有点起色,就有那么多人找上你了么?要是一一帮过去,你能帮得了多少人?''

如懿虽是笑言,凌云彻却不免满面通红,嗫嚅着道:``是。可是她\ldots\ldots{}''

如懿忽然明白:``可是当日让你为她酩酊大醉、意志消沉的人?''

凌云彻被说中心思,只得坦白道:``燕婉是我的同乡,和我一同入宫当差。她虽然心思高些,当日抛下我高飞,可是阴差阳错,最后被贬去了花房当差。花房不分日夜,劳作辛苦,她自己知错,一直不敢来找我。直到今日我在坤宁宫当差,见到她当着花房的差事送来清供的松枝,才知她原来受了这许多苦楚。她的手\ldots\ldots 全是冻疮,因为干的不是伺候人的活儿,所以穿得也单薄寒素。燕婉\ldots\ldots 她是最爱美的。''说着,脸上不觉多了几分怜悯爱惜之意。

如懿打断他道:``她一诉苦,你便忘了往日被她抛弃之苦了?''

凌云彻忙摇头道:``娴妃娘娘明鉴,不是微臣心软。只是\ldots\ldots 只是看她太可怜罢了。燕婉一直痛哭不已,她说她知道当日做错了,所以没有颜面来见我。她\ldots\ldots{}''

``没有颜面来见你,终究也是见了,还说了那么多动人情肠的话。那么,你应承了她什么?又来求本宫?''

凌云彻很是不好意思:``她不是存心让微臣来求娘娘的,只是偌大的深宫之中,微臣能求的,也只有娘娘。微臣只是想,娘娘能不能帮微臣一个忙,把她调离了花房,换个轻松点的差事。''

如懿沉吟片刻:``你真的那么想?''

云彻道:``燕婉也不敢妄求,只求不要满手生满冻疮,她便满足了。''

``听上去,倒也只是个小小心愿,不难满足。''如懿仰起面,呼吸着清冷入肺腑的空气,``只是快到年下了,花房也缺不得人。你把本宫的话带给她,要她安心当差,等开春后,本宫会替她换个好去处的。''

凌云彻忍不住露了几分喜色,打了个千儿道:``那微臣多谢娘娘了。''

如懿忍不住失笑:``看你这么高兴,想来魏燕婉今天说的话,很是力道精准啊。''说罢,也不看他,径自走了。

回到宫中,却见暖阁里供着老大一束绿梅。那淡淡凝玉般的颜色,晶莹剔透,呈半透明妆,而花心又是洁白的。虽不若红梅艳美、白梅清素,但清芬馥郁,尤过寻常梅香。这时房中已被小太监们擦拭得窗明几净,花香与未干的水汽相融,加之殿中炭火洁净,暖气幽幽一烘,越发显得幽雅清新,中人欲醉。

如懿解下斗篷便问:``是谁送来的绿梅,颜色这样好?''

小宫女菱枝仔仔细细地擦拭着供着绿梅的珊瑚釉粉彩花鸟纹瓷瓶道:``小主才出去没多久,皇上便吩咐进保公公送来了。''

如懿凝视了一会儿,笑道:``那你去换个素净点的白瓷瓶来吧。绿梅那么素雅,用个五颜六色的花瓶便太俗气了。''

菱枝不好意思地吐吐舌头:``奴婢只是见这个瓶子喜气,色彩又热闹,所以用了。''

``你要用了这个瓶子插花,好看是好看,却是辜负皇上的一片心意了。''惢心见菱枝出去了,便笑道:``皇上对小主也算是有心的,只是这有心,咱们一时还看不透罢了。''

如懿抚着绿梅笑道:``看不透便先别看,有这么好的绿梅,不细细欣赏,才是浪费了。''

新年过后便是元宵,到了二月里,最兴盛的节日``二月初二龙抬头''了。按着习俗,传说龙头节起源于伏羲氏时代,伏羲``重农桑,务耕田'',每年二月初二``皇娘送饭,御驾亲耕''。到了皇帝当政的时候,也极为重视。这一日便新与皇后去先坛祭祀。回来时皇后兴致颇高,便命人在长春宫中置办了家宴邀请皇帝一同迎春相贺。皇后自爱子早夭之后,一直郁郁寡欢,甚少有展露欢颜的时候,此次主动相邀,皇帝也觉得皇后难得有这样的情致,便也答允了,又让御膳房做了许多皇后爱吃的菜送去。皇帝如此重视,嫔妃们哪有不趋奉之理,于是便由慧贵妃起了个关,遍邀了宫中嫔妃一起为皇后迎春纳福,如此热热闹闹的,竟也成了一个小小的家宴。

皇帝素来爱热闹,自然没有不喜欢的。于是便连位分低微的秀答应,甚至是病中的慎嫔都一一叫来了。皇太后虽未亲至,却也让福珈封了一大屉子的阿胶核桃膏给皇后初益元气,并另赠了两把童子如意,以盼皇后早日再生皇子。

这样的心意,皇后自然是感激涕零。连着皇帝在座,亦不免触动了情肠,柔声到:``皇后放心,以后除了初一十五,逢十逢五的日子朕都会来陪伴皇后,希望皇后能再为朕生下一个白白胖胖的小阿哥。''

如懿坐在西首第一个位子,抿酒入喉间早已字字入耳。皇帝深以自己是庶出为恨,一心盼望得个嫡子,所以虽然有了三阿哥和四阿哥,并且海兰有孕,还是不能弥补他一心的向往。所以失去端慧太子,于一向宠遇不多的皇后而言,可以说是大不幸,亦可谓是幸事。

皇帝赠予皇后的迎春礼是一盒东海明珠,皇后忙起身谢过道:``明珠矜贵,何况是一盒之数,臣妾想到采珠人的辛苦,不敢妄受。''

皇帝握住她的手道:``朕知道你一向节俭惯了,不喜奢华。可这一盒东海明珠再珍贵难得,也比不上皇后你在朕心中的分量。皇后又何必在意这区区一盒之数呢。''

这样的话,皇后哪怕一向注重仪容,也不觉触动了眼底的泪光,她含泪谢过,却看皇帝吩咐李玉将红色的小锦盒送到每位嫔妃手中。慧贵妃与纯妃率先打开,却见里头是一颗与皇后相同的东海明珠。纯妃尚有喜色,慧贵妃却娇嗔道:``皇上好偏心,给皇后娘娘一盒便算了,给咱们的却只有一颗,小气巴巴的。''

皇帝笑道:``给你们的虽然少,但也是朕待你们一样的心意。''

如懿打开锦盒一看,果然光华璀璨,硕大浑圆一颗,胜过烛火明灿。等到慎嫔打开时,她身边的嘉嫔忽然``哎哟''一声,掩口笑道:``咱们的都是东海明珠,慎嫔你这锦盒里的是什么呢?''

话音一落,众人纷纷探头去看,只见鲜红一颗丸药样的东西。慎嫔本就病着,人成了干瘦一把,重重胭脂施在脸上,也是浮艳一酡,虚浮在面上。此时一见此物,脸色更是青灰交加,与面上的胭脂格格不入,人也有些发颤了。

倒是玫嫔先认出了此物,登时神色大变,立刻转头看着皇上道:``皇上!这个脏东西就是当年害死臣妾孩儿的朱砂!''

皇后一脸忧心地看着玫嫔,温和嘱咐:``玫嫔,你别着急,且慢慢听皇上问话。''

慎嫔闻言一凛,立刻跪下,颤声道:``皇上,朱砂有毒,您赐臣妾这个做什么?''她勉强笑道:``是不是放明珠的小公公们错了手,错给了臣妾了。''

皇帝穿着红梅色缂金玉龙青白狐皮龙袍,袖口折着淡金色的织锦衣缘。那样艳丽的色调,穿着他身上丝毫没有脂粉俗艳,反而显得他如冠玉般的容颜愈加光洁明亮,意态清举如风,宛如怀蕴星明之光。他举盏在唇边闲闲啜饭,慢条斯理道:``既然是给你的,自然不会错。朱砂有毒,遇热可出水银。这样好的东西,朕赏赐给我,端然不会有错,也最合你了。''

慎嫔吓得眼珠子也不会动了,勉强笑道:``皇上怎么给臣妾这个?臣妾\ldots\ldots 实在是不懂。''

皇帝忽然将手中的酒盏重重捶落,喝道:``李玉,你来说。''李玉垂手肃然道:``是。奴才按着皇上的吩咐,去查当年与玫嫔和怡嫔两位娘娘皇嗣受损有关之事。当日指证娴妃娘娘的小禄子已经一头撞死,另一个小安子一直发落在慎刑司做苦役,早已初折磨得只剩下半条命。奴才去问了他,才知道当日说娴妃用三十两银子买通他在蜡烛里掺了朱砂的事,是慎嫔娘娘暗中嘱咐他做的。另外小禄子虽然死了,但他的兄弟,从前伺候娴妃娘娘的小福子还活着,只是被送出了宫。奴才出宫一瞧,可了不得,原来小禄子死了之后,他家里还能造起三进的院子,买了良田百亩。而这些银子,都是慎嫔娘娘的阿玛桂铎知府拨的。其余的事,便只能问慎嫔娘娘自己了。''

皇帝嘴角含着冷漠的笑容,声音却是全然不符的温柔:``那么阿箬,朕且问问你,是怎么回事呢?''

阿箬浑身发颤,求救似的看着慧贵妃与皇后。慧贵妃只是一无所知般别过脸去,和嘉嫔悄声议论着什么。

皇帝悠悠道:``当年除了小禄子和小安子,便是你指证娴妃最多,如今,你可有话说么?''

阿箬紧闭的双目骤然睁开,似是想起什么事,膝行到皇帝跟前:``皇上,臣妾冤枉,臣妾冤枉!臣妾和小禄子本无什么来往,他家里买田地建房舍的事,奴婢更是一无所知。至于小安子,臣妾早听说他在慎刑司服役时哑了喉咙,再不能说话了,如何还能说是臣妾指使他的。''

她情急之下喊了出来,哪知话音未落,皇后已经厌弃地闭上了眼睛,搂过三公主和敬在怀里,唤过乳母道:``和敬还小,听不得这些污言秽语,先把她送去太后那里吧?''

如懿扬了扬眉毛,缓声道:``任何人入慎刑司,慎弄司自然有记档。本宫前些日子无意中翻阅过慎刑司的记档,并无任何你或者你宫中人出入的记录。本宫倒是很想知道,慎嫔你是如何得知小安子哑了喉咙再不能说话了。''

阿箬神色剧变,嘶哑着喉咙道:``臣妾、臣妾也是听说。''

如懿饶有兴味道:``那么慎嫔,你是听谁所说,不妨说来听听。''

阿箬怨毒而畏惧地看她一眼:``我也只是听说而已。至于是谁,听过早就忘了。可比不得娴妃心思细腻,连慎刑司的记档都会去查来细看。''

如懿的目光徐徐扫过她的面庞,含笑道:``本宫当然会看,也会去查。因为从本宫被冤枉那一日开始,就从未忘记过要洗雪冤仇。''

阿箬狠狠道:``娴妃娘娘自己做的事自己明白。''

如懿澹然微笑:``这句话说与你自己听,最合适不过。''

皇帝的语气虽淡漠,却隐然含了一层杀意:``那么慎嫔,既然当年你自己亲眼所见娴妃如何加害怡嫔与玫嫔,自然日夜记得,不敢淡忘。那么还是你自己再说与朕听一遍吧,让朕也听听,当年的事到底是如何?''言罢,皇帝转头吩咐李玉:``当年慎嫔还是娴妃的侍女,她的供词你们都是记下了的吧?朕也很想知道,时隔三年,慎嫔是否还能一字不漏,句句道来?''

阿箬急得乱了口齿,拼命磕头道:``皇上、皇上,当年的事太过可怖,臣妾逼着自己不敢再想不敢再记得。奴婢只刻娴妃是如何在蜡烛和饮食里掺的朱砂,至于细枝末节,奴婢实在是不记得了。''

``荒唐!''玫嫔勃然大怒,耳垂上的红玉珠嘀嗒摇晃,``当年你口口声声描述娴妃如何害我和怡嫔腹中的孩子,细枝末节无一不精微,如何今日却都不能一一道来,可见你当日撒谎,所以这些话都没往心里去!''

海兰支着腰慢悠悠道:``当年皇后娘娘派侍女素心带人搜查延禧宫,是阿箬拦着不让搜寝殿才惹得人疑心,后来居然在娴妃寝殿的妆台屉子底下找到一包沾染了沉水香气味的朱砂,才落实了娴妃的罪过。臣妾一直在想,娴妃若真做了这样的事,她既然买通了小禄子和小安子,那么她取朱砂有何难,为何一定要放在自己寝殿的妆台屉子底下?如果那包朱砂娴妃真的是不知情,谁又能随意出入她的寝殿,而且能放了那么久沾染沉水香的气味也不被娴妃发觉呢?''

舒嫔鄙夷道:``那么只能是娴妃的近身侍婢了?''她夹了一筷子菜吃了,看着阿箬道:``看来这样的事,除了当日的慎嫔,也没有旁人可以做到了。''

嘉嫔厌恶地摇头道:``当日言之凿凿,今日慌不择言。皇上,慎嫔实在是可疑呢。''

皇帝眼底的厌弃已经显而易见,他紧握着手中的酒盏,森冷道:``你当年的话当年做的事关系着朕两位皇儿的性命,如果今日你不说实话,便把朕赏你的这颗朱砂生吞下去,朕再吩咐慎刑司的人拿朱砂活埋了你。你自己掂量着办吧!''

阿箬吓得面无人色,一袭粉蓝色缂丝彩绘八团梅兰竹菊袷袍抖得如波澜顿生的湖面一般。如懿望向她的目光漠然如冰霜,丝毫没有怜悯之意,继而向皇帝道:``皇上,臣妾一直在想,阿箬并没有本事找来那么多朱砂,收买那么多人,一一布置得如此详细,布下天罗地网来冤害臣妾。她虽然一直有攀慕皇恩之心,但当时未必有一定要置臣妾于死地之心。臣妾很想知道,到底是谁在幕后指使慎嫔。''

``慎嫔?''皇帝轻笑道,``这么多作孽的事,如果不是旁人指使她做的,就是她自己要谋害皇嗣。她哪里还配做朕的慎嫔,一直以来,她就只是你的侍婢,你要如何处置,都由得你!''

如懿欠身道:``那么恕臣妾冒昧了。以彼之道还施彼身,阿箬若不肯说实话,臣妾便让人用炼制过冒了水银的朱砂一勺一勺给她灌下去,这种东西大量灌入之后会腐蚀她的五脏六腑,从中毒到毒发身亡的过程极其痛苦。但阿箬若招出是谁指使,顶多也只是攀诬之罪,并未涉及谋害皇嗣,臣妾愿意向皇上请求,留她一条性命。''

皇帝谈笑自若,看着皇后道:``阿箬是娴妃的人,自然由娴妃处置。皇后,你说是不是?''

皇后淡淡含笑:``皇上说得不错。只是\ldots\ldots 娴妃的刑罚听着也太可怕了些。''

皇帝淡漠道:``对于这样没心肝的人,这样的惩处,一点也不为过。娴妃,朕答允你便是。''

阿箬自知无望,求救似的看着慧贵妃,唤道:``贵妃娘娘\ldots\ldots{}''

慧贵妃立刻撇清道:``哎呀,你喊本宫做什么!你可别来牵连本宫!娴妃,一切由得你便是了。''

她话音未落,只听地上``咕咚''一声,却是阿箬已经晕了过去。

皇帝见阿箬受不得刺激晕倒在地,便吩咐道:``今日是朕与皇后办的迎春家宴,原不该在这个时候提这件事。只是朕看到皇后,便想起早夭的端慧太子,又想起玫嫔与怡嫔的孩子都胎死腹中,死得不明不白,朕不能不细细查问。''

皇后听他提到二阿哥,亦不免伤感:``皇上与臣妾都为人父母,如何能不伤心?虽然这件事是在臣妾的迎春家宴上提起,但若能得个水落石出,也算是给臣妾最好的贺礼了。如今天色已晚,有什么事皇上也等明日再查问吧,折腾了这么久,还请皇上早点安歇才是。''

皇帝颔首道:``朕原本想陪皇后一起,但今晚也没兴致了。李玉,起驾回养心殿。朕要好好静一静。''

李玉忙道:``请旨。阿箬该如何处置?''

皇帝眼中闪过一丝冷意:``带去养心殿偏殿,着人看着她,不许好寻短见或是旁的什么缘故死了。''

这句话,分明是有深意的。慧贵妃不自觉地缩了缩身子,摸着袖口的苏绣花纹,强迫自己镇定下来。嫔妃们见如此,便出告辞散了。慧贵妃特意落在人后,有些担忧地看着皇后,皇后淡淡道:``不干你的事,你眼巴巴看着本宫做什么?''

慧贵妃怯怯道:``是,可是阿箬若是咬出了咱们\ldots\ldots{}''

``咬出咱们?''皇后轻轻一嗤,闲闲道,``你是贵妃,本宫是皇后,咱们怕什么?''

慧贵妃仍是不放心,上前一步道:``可是皇后娘娘不觉得奇怪么?今日明明是娘娘摆迎春家宴,皇上为何一定要在今日发作,严审此事呢?难不成皇上连娘娘也起疑心了?''

皇后神色一滞,闪过一丝慌乱,很快肃然道:``放肆!皇上只是关心皇嗣,疑心阿箬罢了。在本宫的迎春家宴上提起也是偶然,你不要胡思乱想,更不要想到什么就信口胡说,自乱阵脚。''

慧贵妃极少看到皇后如此疾言厉色,忙低下头不敢言语。

皇后扶着素心的手转到寝殿,卸下衣冠,对着妆台上的合欢铜镜出了会儿神,压低了声音道:``素心,皇上不会真的疑心本宫了吧?''

素心将皇后的大氅挂到黄杨木衣架子上一丝不苟地整理着,口中道:``皇后娘娘安心,皇上不是说了么,也是因为想着咱们早逝的端慧太子的缘故,才这般忍不住。皇上还想着与娘娘再有一个阿哥呢。说到底,皇上总是在意娘娘的,何况,咱们还有三公主。皇上不知道多喜欢三公主呢。''

``本宫生的大公主和哲妃生的二公主都早夭,皇上虽然有几位阿哥,但公主只有这一个,是爱惜得不得了。所谓掌上明珠,也大约如此了。''皇后摘下东珠耳环,叹低头叹息着抚着小腹道,``只是本宫和皇上一样,多么盼望能再生一个嫡出的阿哥,可以替皇上继承江山,延续血脉。''

素心挂好衣裳,替皇后解开发髻,取下一枚枚珠饰通花:``娘娘别急,皇上已经答应了会常来陪伴娘娘,娘娘只要细心调理好身子,很快就会怀上皇子的。''

皇后颔首道:``也是。你记得提醒太医院的齐鲁,好好给本宫调几剂容易受孕的坐胎药。''

素心笑道:``是。说到坐胎药才好笑呢。宫里没有比慧贵妃喝坐胎药喝得更勤快的人了,恨不得当水喝呢。可是越喝身子越坏,娘娘没注意么,这两年慧贵妃的脸色愈加难看了,简直成了纸糊的美人儿。''

皇后道:``本宫有时候也疑心,那串手镯,娴妃和她都有,都怀不上孩子也罢了,怎么难道还能让身子弱下去么?还亏得齐鲁在亲自给她调治呢,居然一点起色也没有。''

``那是她自己没福罢了。哪怕慧贵妃的父亲在前朝那么得皇上倚重,她又在后宫得宠,可生不出孩子,照例是一点用处也没有。永远,只能依附着娘娘而活。''

皇后露出一份安然之色:``皇上不是先帝,不会重汉军旗而轻满军旗,弄得后宫全是汉军旗的妃子。当年先帝的贵妃年氏、齐妃李氏、谦妃刘氏、宁妃武氏、懋嫔宋氏,哪一个不是如此。但话虽如此。本宫也不能不防着汉军旗出身的慧贵妃坐大了。''

素心笑道:``她不敢,也不能。即便她有她父亲这个靠山,娘娘不是也有张廷玉大人这位三朝老臣的支持么。倒是海贵人的胎,奴婢悄悄去问过了。不知什么缘故,是被发觉了还是什么,太医院配药材的小太监文四儿说,如今想要在海贵人的药里加那些开胃的药材,竟是不能了。''

皇后娥眉微蹙:``难道是被发觉了?''她旋即坦然:``那也无妨。左右只是开胃的药,就当小太监们加错了。怀着身孕么,本就该开胃的。何况海贵人胖了那么多,身上该长得东西也都长好了,不吃也没什么。''她忽然止住声,从铜镜中依稀看到了什么,豁然转过头,带了一丝慌乱沉声道:``和敬,你站在那里做什么?跟着你的人呢?''

三公主有些畏惧地站在珠凌帘子之后,慢慢的挪出来,唤了一声:``额娘。''

皇后微微敛容:``告诉你多少次了,要唤我皇额娘,因为我不只是你的额娘,更是皇后。''

三公主已经十岁,出落得十分清丽可人,脸上隐隐带着嫡出长公主才有的傲然,如一朵养在深闺的玫瑰花,不知风霜,兀自娇艳美丽。

她见了皇后,脸上的那些傲气便隐然不见了,只是一个怯怯的小女儿,守着规矩道:``是。儿臣知道了。''她的声音越发低下去:``儿臣不是有意偷听皇额娘和素心姑姑说话,只是想在皇额娘睡前来给皇额娘请个安,独自和您说说话。''

皇后放下心来,气定神闲地换了温和的口气:``那么,你要跟皇额娘说什么?''

``现在没有了。''三公主微微地摇摇头,抬起稚嫩的脸,望着皇后,``皇额娘,你们方才说,给海贵人下什么?''

皇后扬一扬脸,示意素心出去,搂住三公主正色道:``不管皇额娘给谁下了什么东西,对谁做了什么,都是为了你为了皇额娘自己。这个宫里,要害咱们的人太多太多,皇额娘做什么都是为了自保。''她亲了亲三公主的脸,含了泪柔声道:``和敬,你的二哥已经死了。皇额娘没有儿子可以依靠,只有靠自己了。''

三公主大为触动,伸手替皇后擦去泪水,坚定道:``皇额娘,儿臣都明白的。二哥不在了,儿臣虽然是女儿,但也不会没用。儿臣一定会帮着皇额娘的。皇额娘不喜欢谁,儿臣就不喜欢谁。''

皇后脸上笑着,却忍不住心酸不已。她先生下的二阿哥永琏,再有了和敬公主,所以从未曾把这个女儿看得多重要。即便是永琏死后,她不得不借着这个唯一的女儿笼络皇帝的心,也从未这般亲近过。却不想,反倒是这个女儿,那么体贴明白她的心意,真真成了她的小棉袄。

这一夜,想来有许多人都睡不安枕了。如懿听着窗外簌簌的雪声,偶尔有枯枝上的积雪坠落至地发出的``啪嗒''的轻响,间杂着细枝折断的清脆之声,和着殿角铜漏点点。真是悠长的一夜啊。

如懿醒来的时候便见眼下多了一圈乌青,少不得要拿些脂粉掩盖。惢心笑道:``小主也不必遮,今儿各位小主一照面,可不都是这样的眼睛呢。''

如懿轻嗤一声,取过铜黛对镜描眉:``我怕见到皇上时,皇上也是如此呢。''

正说话间,却见李玉进来,恭谨请了个安,道:``娴妃娘娘万福,皇上请您早膳后便往养心殿一趟。''

如懿赶到养心殿时,却是小太监进忠引着她往殿后的耳房去了,道:``皇上正等着小主呢。''

如懿推门入耳房,却见皇帝盘腿坐在榻上,神色沉肃。阿箬换了一件暗沉沉的裙装跪伏在地下,头上的珠饰和身上的贵重首饰被剥了个干净,只剩下几朵通草绒花点缀,早已哭得满脸是泪,见如懿进来,刚想露出厌恶神色,可看一眼皇帝的脸色,忙又收敛了,只和她的是女新燕并肩跪在一块。

皇帝执过如懿的手,通过一个平金珐琅手炉给她,和声道:``一路过来冻着了吧?快暖一暖,来朕身边坐。''

如懿一笑,与皇帝并肩坐下,却听得皇帝对阿箬道:``昨日朕留着你的脸面,没有当下拿水泼醒你逼问你,还许你在耳房住了一晚。如今只有朕和娴妃在,有什么话,尽可说了吧?''

如懿瞥一眼一旁守着的李玉,道:``昨儿本宫吩咐备下的朱砂,她若不说实话,便一点一点要她吞下去。那些朱砂呢?''

李玉指了指耳放角落里的一大盆朱砂:``按娴妃娘娘的吩咐,都已经备下了。''

阿箬自知不能再辩,只得道:``皇上恕罪,当年是奴婢冤枉了娴妃娘娘。''

皇帝端了一盏茶,慢慢吹着浮沫道:``这个朕知道。''

阿箬又道:``是奴婢偷拿了朱砂混到怡嫔娘娘的炭火和蜡烛里,也是奴婢拿了朱砂染好了沉水香的气味,等着素心要搜寝殿时,偷偷塞在妆台屉子底下的小禄子也是受人指使的,但不是娴妃娘娘。''

皇帝有些不耐烦:``这些朕都知道。''

如懿蹙眉道:``该往自己身上揽的都揽的差不多了。本宫还想知道,你混得了怡嫔的东西,却不能常常混进玫嫔宫里去,到底是谁指使你的?''

皇帝啜饮着茶水,低头恍若未闻。阿箬睁大了眼睛惶惑的看着皇帝,皇帝只做未见。如懿缓缓道:``说与不说在你。反正你要把所有的事儿都揽下来,谁也拦不住。本来本宫可以留一条命给你,但是你非要认下谋害皇嗣株连九族的罪过,本宫也由不得你。''

阿箬死死地咬着下唇,唇上几乎都沁出了血,颤抖着喉咙道:``皇后,慧贵妃''

皇帝幽沉乌黑的眸子里闪过一丝疑忌的光,徐徐道:``皇后与贵妃一向仁慈,你想要求她们,也是不能的。还是为你的家人多考虑吧。''

新燕忙在后头道:``小主,小主,您可千万别糊涂了。如今到了这个地步,求谁也不管用了,您做了什么就自己招了吧,别平白连累了旁人。便是奴婢,也只是伺候您而已,许多前事都不知道啊。''

皇帝即刻醒觉:``前事不知?那么现在的事,你又知道多少?譬如朕一直很想知道,是谁给娴妃在冷宫里的饮食下了砒霜?''

阿箬霍地抬头:``皇上,真的不是奴婢!真的!''

皇帝看着新燕道:``你说。''

``奴婢不敢欺瞒皇上,奴婢确实不知。''新燕忙磕了个头,怯怯地看了阿箬一眼,犹疑道:``但奴婢的确听说过,小主深以娴妃娘娘为恨,尤其是那次重阳冷宫失火,皇上见到过娴妃娘娘之后,小主就很怕娴妃娘娘出冷宫,几次在奴婢面前提起,一定要让娴妃娘娘死在冷宫里,没命出来才算完。其他的,奴婢也不知道了。''

阿箬的脸色越来越白,最后成了一张透明的纸,猛地仰起脸来,两眼定在如懿身上,恨不得剜出两个大洞来,道:``娴妃,我是恨毒了你,明明我聪慧伶俐,事事为你着想,你却凡事都压着我,欺辱我!你明明看出皇上喜欢我,却一定要拔除我这个眼中钉把我指婚出去。我得宠对你难道不好么,你也多了一个帮衬。为什么你非要断了我的出头之路呢?''

``皇上喜欢你?''如懿忍不住轻笑,``如今皇上也在这里,你可问问他,喜不喜欢你?若不方便,本宫大可回避!''

如懿说罢便要起身,皇帝伸手拦住她道:``不必了。朕便告诉她实话就是。''

阿箬泪眼蒙蒙,喘息着道:``娴妃,你又何必这般假惺惺!我知道皇上已经不喜欢我了!否则他不会这么待我!''她爬行两步,死死攥住如懿的裙角,冷笑道:``你不是很想知道皇上怎么待我的么?我便告诉你好了。自从第一次侍寝之后,皇上每一次翻我的牌子,都不许我碰他一下,只准我赤身裸体披着一袭薄毯跪在床边的地上,像一个奴婢一样伺候。白天我是小主,受尽皇上的恩赏。可到了皇上身边,一个人的时候,我还是一个低贱的奴婢,连只是侍寝的官女子也不如!可即便是这样,落在旁人眼里,我还是受尽宠爱,所以不得不忍受她们的嫉妒和欺凌!娴妃,你以为你在冷宫的日子难过,我在外头的日子就好过么?每日翻覆在皇上的两极对待下,无所适从,战战兢兢!我怎能不恨?怎能不怕?''

如懿听着她字字诉控,也未成想到她三年的恩宠便是如此不堪,不觉震惊到了极点。良久,倒是皇帝缓缓道:``现在觉得不甘心了么?那么,朕告诉你,都是自找的。你想当朕的宠妃,朕许你了。可是背后的冷暖,你便自己尝去吧。要不是为了留着你这条性命到今日,要不是为了让你尝尝风光之下的痛苦,朕也不必花这份心思了。''他望着如懿,缓缓动情道:``如今,你都该明白了吧?''

阿箬瘫倒在地,不可置信地看着皇帝,满脸怆然,惊呼道:``皇上,你竟这样待臣妾对您的一片心!''

皇帝泰然微笑:``你对朕的心是算计之心,朕为何不能了?''

阿箬怔怔地流下眼泪来:``皇上以为臣妾对您是算计之心,那后宫众人哪一个不是这样?为什么偏偏臣妾就要被皇上如此打压?''

``打压?''皇帝侧身坐在窗下,任由一泊天光将他的身影映出朗朗的俊美轮廓,``朕相信许多人都算计过朕,朕也算计过旁人,但像你一般背主求荣,暗自生杀的,朕倒真是没见过。''

如懿坐在皇帝身侧,只觉得记忆里他的容颜已然陌生,连他说出的话也让人觉得心头冰凉一片,无依无着。她只觉得有些疲累,淡淡道:``那么,所有的事都是你做的么?''

阿箬悲怆至极,茫然地点点头:``都是我,都是我。玫嫔和怡嫔是我害的,娴妃是我想杀的!什么都是我!行了么?''

如懿忽然想起一事:``阿箬,我记得你很怕蛇?''

阿箬沉浸在深深的绝望之中,还是新燕替她答的:``回娴妃娘娘的话,小主是很怕蛇。''

皇帝看如懿神色倦怠,柔声道:``如懿,你是不是累了?你先去暖阁坐坐,朕稍后就来。''说罢,李玉便过来扶了如懿离开。皇帝见她出去了,方盯着阿箬,目光中有深重的迫视之意,问道:``你方才说是皇后和贵妃的主使,是不是真的?''

皇帝回到暖阁时,如懿正在青玉纱绣屏风后等待,她的目光凝注屏风一侧三层五足银香炉镂空间隙中袅袅升起的龙涎香,听着窗外三两丛黄叶凋净的枯枝婆婆娑娑划过窗纸,寒雪化作冷雨窸窣,寂寂敲窗。如懿看着皇帝端肃缓步而入,宽坐榻边,衣裾在身后铺成舒展优雅的弧度。皇帝执过她的手:``手这样冷,是不是心里不舒服?''

如懿点点头,只是默然。皇帝缓声道:``阿箬已经都招了。虽然她要招供的东西朕早就知道了,可是朕不能不委屈你在冷宫这三年。当年的是扑朔迷离,朕若不给后宫诸人一个交代,不知道在你身上还会发生什么可怕的事。朕一直以为,冷宫可以保你平安。''

如懿缓缓抬起眼:``臣妾不知道皇上这些年是这样待阿箬。''

皇帝轻轻搂过她:``如今知道了,会不会觉得朕很可怕?''

皇帝这样坦诚,如懿反倒不知道说什么了,定了半天,方道:``皇上的心胸,不是臣妾可以揣测的。''

他以一漾温和目色坦然相对:``你不能揣测的,朕都会尽数告诉你,因为你是如懿,从来对朕知无不言最最坦诚直率的如懿。而朕还有一句话要告诉你,朕当年留下阿箬,一则是要她放松戒心,也是怕真有主使的人要灭她的口;二来当时治水之事很需要她阿玛出力,旁人也帮不上忙。所以一直拖延到了今日。如懿,你要明白朕,朕首先是前朝的君主,然后才是后宫的君主。''

他的话,坦白到无以复加。如懿忍着内心的惊动,这么多年,她所委屈的,介意的,皇帝都一一告诉了她。她还能说什么呢?皇帝数年来那样对待阿箬,本就是对她的宽慰了。于是她轻声问:``皇上真的相信没有人主使阿箬了么?''

皇帝的目光波澜不兴:``她一个人都认了,你也听见了。再攀扯别人,只会越来越是非不清。所以朕也希望你明白,到阿箬为止,再没有别人了。''

这样的答案,她已经隐约猜到了几分。既然她也想到会是谁,何必要皇帝一个肯定的答案呢。如懿心头微微一松,终于放松了自己,靠在皇帝怀中:``皇上有心了。''

皇帝轻吻她额头:``自你出冷宫,朕一直没有召幸你,很少见你。便是要等这水落石出的一天,你心中疑虑消尽,朕才真正能与你坦然相处,没有隔阂。''

清晨的雪光淡淡如薄雾,映着窗上的明纸,把他们身上扫落的影子交叠在一起。在分开了这些年之后,如懿亦有一丝期望,或许皇帝可以和她这般没有隔阂的相拥,长长久久。

皇帝拥着她道:``如今,你的心中好过些了么?''

如懿微微颔首,含情看向皇帝:``皇上的用心,臣妾都知道了。''

黄帝身姿秀逸,背靠朱栏彩槛、金漆彩绘的背景中,任偶然漏进的清幽的风吹动他的凉衫薄袖,他温然道:``朕很想封你为贵妃,让你不再屈居人下。可是骤然晋封,总还不是万金,朕也不希望后宫太过惊动。但是朕让你住在翊坤宫,翊坤为何,你应该明白。''

坤为天下女子至尊,翊为辅佐襄赞。她知道,皇帝是在暗示她仅次于皇后的地位。她心中微暖,复又一凉,想起阿箬的遭遇,竟有几分凉薄之意。但愿皇帝待她,并无算计之心。

那么,便算是此生长安了。

\hypertarget{ux7b2cux4e09ux5341ux7ae0-ux732bux5211}{%
\chapter{第三十章 猫刑}\label{ux7b2cux4e09ux5341ux7ae0-ux732bux5211}}

如懿回到翊坤宮中,已经是天光敞亮时分。昨夜相拥而眠,红烛摇帐的温存尚未散去,皇帝便着李玉将阿箬送了来。

如懿正对镜理妆,李玉打了个千儿,恭恭敬敬守在一旁,道:``启禀娴妃娘娘,皇上说了,阿箬是您的奴婢,所以还是交还给您,任由您处置,也要以儆效尤,告诫宫中的奴才们,不许再欺凌背主。''

如懿对着镜子佩上一对梅花垂珠耳环,淡淡道:``人呢?\textgreater\_\textless''

``已经在院子里跪着了。只是有一样,阿箬发疯似的辱骂娘娘,皇上已经吩咐奴才给她灌了让她安静的药,所以,她已经不能说话了。''

如懿眉心一跳:``哑了?''

李玉恭恭敬敬道:``是。再不能口出秽语,侮辱娘娘了。''

如懿心头一惊,自然,那是再问不出什么了。只是,这后宫里的一切,原本不是问就能有真切的答案的。想要知道什么,全凭自己,所以,也无所谓了。

惢心替她理好鬂发,轻声在她耳畔道:``小主不是一直要奴婢和三宝留意宫里的人么?如今,倒是个杀鸡儆猴的好机会。''

如懿撂下手中的珐琅胭脂盒,笑道:``你倒是和我想的-样。去吩咐三宝,找个麻袋,寻几只猫来,然后把宫里的人都召集起来,就在院子里看着。''

惢心微微一笑:``是。''

待到三宝预备好,如懿披上一件香色斗纹锦上添花大氅,站在廊下,肃然看着满院黑压压的宫人们,慢斯条理道:``本宫宫中,不怕你伺候人时不够聪明,怕的就是背主求荣,糊涂油蒙了心。一次不忠,百次不用。你们好好当差,本宫自然好好待你们。若是像阿箬一样\ldots\ldots{}''她瞥了眼跪在地上的呜呜咽咽说不出话的阿箬,冷道:``阿箬虽然是本宫的陪嫁侍女,之前伺候了本宫八年。可是她背叛本宫,本宫就容不得她!今日,是给她一个教训,也是给你们一个警戒。''

如懿看了眼三宝,三宝应了一声,一挥手招呼几个小太监取了个巨大的麻袋并几只灰猫来,三宝按着阿箬,让两个小宫女利索地扒下阿箬的外裳,只露出一身中农,喝道:``把她装进去!''

阿箬似是意识到什么,满眼惊恐地看着那几只形态丑陋的灰猫,不背钻进麻袋里去。三宝哪里由得她,兜头拿麻袋一套,收拢了口子,留下只够塞进一只猫的小口子,然后把那些露着锋锐齿爪的灰猫一只只塞进去,拿麻绳扎紧了口袋,回道:``小主,这些是从烧灰场找来的猫,性子野得很,够阿箬姑娘受的了。''

如懿在廊下坐下,细赏着小指上三寸来长的银质嵌碎玉护甲:``那还等什么,让她好好受着吧。\textless{}''

三宝用力啐了一口,举起鞭子朝着胡乱扑腾的麻袋便是狠狠几鞭。那麻袋里如汹涌的巨浪般起伏跳跃,只能听见凄厉的猫叫声和女人含糊不清的呜咽嘶鸣。

阿箬,已经说不出完整的话了,这样不完整的残缺人声,在静静的清晨,听来更让人觉得毛骨悚然。渐渐地,连敞开的宫门外,都聚集了宫人探头探脑,窃窃私语。灰猫凄惨的嘶叫声和着爪牙撕裂皮肉的声音儿乎要撕破人的耳膜,如懿皱着眉听着,吩咐道:``继续!''

三宝往手上吐了两口唾沫,下手更狠,一鞭子一鞭子舞得像一朵花一样眼花缭乱。一开始还有人的喉咙发出的声音,渐渐地,灰白色的麻布袋上渗出越来越多的血迹。如懿颔首道:``可以了。''

三宝打得满脸是汗,应了一声扯开布袋,只见几只灰猫毛发倒竖地眺了出来,龇牙咧嘴地跑了。两个小太监将布袋完全打幵,拖出一个浑身是血的血人儿来,气息奄奄地扔在了地上。如懿瞟了一眼,只见阿箬的中衣被爪子撕成一条一条的,衣裳已经完全被鲜血染透,脸上手上露着的地方更是没有一块好肉。三宝见她痛的晕了过去,随手便是一盆冷水泼上去。阿箬嘤一声醒转过来,身上脸上的血污被水冲去,露出被爪牙撕开翻起的皮肉,一张娇俏容颜,已然尽数毁去。

如懿走上前几步,意欲细看。惢心急忙拦道:``小主小心污秽。''

如懿径自推开惢心的手,缓步走到阿箬身边,俯下身看她---眼,旋即恢复居高临下的姿态,喝逍:``究竟是谁指使你谋害本宫!快说!快说!''

阿箬的喉头发出嘤嘤的呻吟声,挣扎了几下还是无力动弹,索性像一块烂肉似的伏倒在地。如懿露出一丝鄙夷之色,摇头道:``真是可怜,有错当罚,这是你该受的!但你想说出幕后主使之人,却怎么也说不出来,含冤莫白,替人受罪,也当真可怜!''她转头吩咐三宝:``阿箬既被皇上废去位分,自己宫里是住不得了。去冷宫打扫出间屋子来,送她进去。''

阿箬虽然说不出话,一双眼睛却瞪得老大老大,死死盯着如懿,几乎要沁出血来。三宝和几个小太监哪里理会她,径直拖了就走。阿箬喘着粗气,十指用力抓着地面,想要抓住什么可以救命的依靠,然而她早已失尽了力气,只在地上抓出几条深深的暗红血痕,触目惊心。

如懿走回廊下,院中静得如无人一般,几个胆小的宫女太监早已吓得瘫软在地,筛糠似的发抖。

如懿的面色清冷而没有温度:``不要怪本宫心狠,背叛主上的人虽然可以得到一时的富贵,但最后还是没得好下场!你们看看,当年指使怂恿她背叛本宫的人,如今哪里会来救她,急着撇清都来不及呢!''

满宫的宫人们吓得立刻跪下,面如土色:``奴才们不敢背叛小主,心怀二念。''

如水双眸似结了冷冷的薄冰,如懿淡然道:``那就好。否则今日的阿箬,就是来日的你们。''她站起身,似是自然自语:``也难怪阿箬说不了话也要哼哼给本宫听,带着这样的冤屈,谁能不恨呢?''

如此一来,阿箬的事在六宫之内传得沸沸扬扬,人人都说出了冷宫的娴妃心性大变,一改昔日温和隐忍,杀伐决断,手段凌厉,倒让人越发不敢小觑了翊坤宫。

到了晚间时分,惢心正伺候着如懿拿忍冬花水泡了姜汁浸手。紫藤撒花帘子一扬,确实三宝转了进来,悄声禀报道:``小主,冷宫里的人来回话,说阿箬一索子挂在梁上,上吊自尽了。''

如懿头也不抬,只垂着眼帘,看着铜盆中自己---双关节微微肿起的手:``才在冷宫待了一天就受不住了么?惢心,还记得咱们的日子是怎么熬过来的。''

惢心冷道:``有福气的人自然熬得住,没福气的,便是一天也忍不得了。''

如懿接过小宫女递来的软帕,擦净了手方问:``皇上知道了么?怎么说?''

``养心殿的意思,就说是病死了,按着嫔位置办丧仪便是,免得传出去不好听。''三宝停了一停,似乎有些害怕,觑着如懿的神色道,``只是听给阿箬收尸的人说,阿箬穿着红衣红鞋上吊的,穿了一身红去死,那是怨气冲天要带到地府去的呢。''

如懿的眼眸微微一沉,含了寒星似的光芒:``怎么?做人的时候没用,要穿上这一身做鬼来寻仇么?''她虽这样说,却也不免有些畏惧,当下兴致阑珊,也不肯再言了。

这一夜皇帝依旧召了如懿往养心殿侍寝,言谈间却丝亳不过问她对阿箬施用猫刑之事,仿佛那是一件极平常的小事,根本不值一问。为着如懿过来,皇帝的寝殿里每日都供着一束绿梅点染,她便在这清馥甘郁之中,借一盏鎏金琉璃灯的温柔余光,与他轻轻拥抱,以肌肤的贴近与亲昵来宽慰过去的伤痛,落实来日的希冀。

良夜深沉,梦中惊转,却是宫人急急在外敲门,说海兰动了胎气,即刻就要生了。皇帝且惊且喜,立刻披衣起身,与如懿一起往延禧宫去。

才进延禧宫的大门,宫人们早己跪了一地,慌不迭道:``皇上万福金安,娴妃娘娘吉祥安康!''

如懿听得里头海兰的叫声一声比一声凄厉,简直如挖心掏肺一般,便慌得不行,连忙道:``皇上,臣妾心里不安得很,想进去看看妹妹。''

皇帝虽然一脸期盼,但被那声音惊着,又眼看着接生嬷嬷和太医一个个进去了便不再出来,也不安得很,便点头道:``朕不便进去,你去瞧瞧也好。''

如懿巴不得这一声儿,正要往里进去,还是伺候海兰的小太监五福在外拦住了道:``产房血腥不祥,娴妃娘娘进去不得!''

如懿哪里还顾得这些,推开他的手呵斥道:``本宫又没怀着身孕,且延禧宫原是本宫住过的地方,有什么不祥的!再敢胡说八道,立刻拖出去掌嘴!''

五福素知她与海兰的交情,又见过她严惩阿箬的样子,当下也不敢再拦,只得躬身退到一边。如懿推开殿门进去,因海兰有着身孕,殿中都布置成了吉利的红色,漫天漫地的石榴葡萄,瓜瓞绵绵图案,都是多子多福的征兆,混合着殿阁内浓郁的血腥气,越发觉得那红色猩艳得直冲人眼目。

如懿伏到床前,海兰已经是满身大汗淋漓,连着床褥都湿透了,一群接生嬷嬷围着她忙碌,孩子却还是半点没有要下来的意思。

接生嬷嬷急得都要哭了,哭丧着脸对着如懿诉苦道:``催产药都喝了好几剂了,可是可是还贵人生产前太胖,孩子在肚子里养得太大,出来实在是艰难哪!''

太医亦跪在屏风外头,垂头丧气道:``海贵人身子发胖,用不上力气,实在是\ldots\ldots{}''

海兰满脸皆是纵肆的泪痕,斑驳一片。她痛得脸色雪白,拼命摇着头嘶哑着道:``姐姐!我不成了,我实在是不成了!我真真是被人害死了!''

如懿紧紧握住她汗湿的手,那种滑腻的容易从手中逝去的触感着实叫她害怕。她只得压抑住自己惶乱的心神,大声道:``你要自己这么想,放松了力气不肯好好生下孩子,那才是被别人害死了!海兰,我没有孩子,你答应过我,这个孩子生下来会交给我好好抚养!你不能说话不算话!''

海兰痛得心肺都要裂开了,气息阻塞在喉头,一时说不出话来。偏偏接生嬷嬷也不镇定,一直唉声叹气:``孩子直顶在那儿,不肯下来。小主,您使点儿力气呀!''

海兰痛得青筋暴起,像一条条鼓起的小青蛇,要破皮而出。海兰脸容都变形了,大口喘息着道:``姐姐,不是我说话不算话,我真的没力气了,我真的\ldots\ldots{}''

海兰一边说,一边挣扎着用劲,右手紧紧抓着如懿的手腕,如懿感受到她手上渐渐松下去的力气,心里越来越慌,只得在她耳边道:``海兰,你要是现在没力气了,便是遂了她们的心愿了。你听我的话,要是松了这口气,你和孩子都难保,要是拼着这口气,便都保下来了。''海兰的头发全都湿透了,黏在脸上,越发显得一张脸白得没有一丝血色。

空气中浓郁的血腥气混着草药的气味让人觉得窒息。如懿看着她如此辛苦,滚烫的泪在眼底翻腾不已,终于落了下来。她伏在海兰枕边,一字一字定定地道:``海兰,冷宫里那么难熬,因为你撑着我,我也都熬了下来。如今好不容易咱们又能在一块儿了,你若是这么轻易放弃,我一定不会原谅你。''

海兰抓着她的手腕,滑下去一寸,又一村,人也近乎昏死。如懿的泪滴落在海兰面上,似乎是一种深远而沉重的召唤的力量。海兰的牙关咬得死死的,只是吃力地点着头,如懿一迭声地喊道:``来人,来人!她还有意识,快给她灌参汤进去,快!''

叶心很快端来了参汤,如懿急忙接过,示意叶心托起海兰的后颈,一点一点撬开她的牙齿灌进去。海兰能喝下的参汤并不多,几乎是喝一半,流出来一半。如懿看着焦心不已,正见床边搁了一盘切好的参片,只得先取了一片給她噙在口中。或许是参汤起了点效力,海兰抓着如懿手腕的手渐渐有了几分力气,太医们喜出望外,忙道:``娴妃娘娘,海贵人已经有了点意识,要不要再灌些催产药下去?''

如懿如何懂得这些,只得看向接生嬷嬷们,其中一个接生嬷嬷叫起来道:``贵人已经喝了那么多催产药了,孩子还没有动静。太医不妨试试针灸或是别的,若再催产,只怕一时药量过猛,孩子是出来了,可母体要大受损伤呢。何况,太医给小主喝的催产药性子有些猛烈,不是寻常的益母芎归汤呢?''

如懿听着不安,立刻问道:``你们给海贵人吃的是什么催产药。''

为首的是太医院的赵太医,他忙磕头道:``娴妃娘娘,寻常的催产汤药是益母芎归汤,这药以当归、川芎为主,当归养血活血,调经止痛,川芎为血中气药,上至巅顶,旁达肌肤,走而不守,者配合,可加强活血祛淤之力;佐以桃仁、红花、丹参、益母草活血祛淤,合川朴可降气导滞,牛膝引血下行,诸药配合达到养血活血,祛淤催产,引胎下行之功。可海资人胎大难下,又有气虚乏力的症状,所以又加了黄芪三两调治。''

如懿越听越是心惊,不禁矍然变色道:``桃仁、红花和牛膝都是堕胎的猛药,怎么可以用在催产的方子里!''

赵太医忙道:``娴妃娘娘有所不知,催产的药本就该是有活血化瘀之效,桃仁、红花和牛膝都是堕胎的猛药,也是催产的好药。微臣身为太医,这些事断不会弄错的。''

如懿心中不定,回顾四望,却不见江与彬在,忙唤道:``绿痕,江太医呢?''

还是赵太医道:``今日并非江太医当值,深夜宫门下了钥,再唤江太医也不妥当。''

如懿当即知道无望,只得道:``本宫不懂药理,这话你们去回皇上,问问皇上的意思。

赵太医出去片刻,即刻回来道:``皇上说了,母子都要平安,斟酌着用傕产药就是。''

如懿听得``斟酌''二字,便也稍稍放心:``那你们小心剂量,以贵人玉体为重。''

赵太医即刻答应了,吩咐宫女去端了药来,给海兰灌下。催产药加着参汤的效力,海兰渐渐清醒,也有了力气,只是身上的疼痛发作得越加厉害'止不住地惨叫起来。接生嬷嬷们看着几碗催产药灌下,起初也是担忧,但看海兰的胎动渐渐发作,也少不得忙碌起来。

殿中乱作了一团,海兰死死抓着如懿的手腕,几乎失尽了力气,轻声唤道:``姐姐,你还在?''

如懿泪流满面:``我一直都在,你安心生孩子就是。''

海兰再说不出话,拼了命地用起力气来,几乎要将如懿的手腕捏碎了。如懿忍着剧痛,伏在床边不停地替海兰擦着浆出的汗水,熬度着漫长而难耐的时间。良久,也不知过了多久,在凄厉的嘶声过后,终于听得一声响亮的儿啼,却是皇帝的声音先在外头响起来,喜不自胜道:``朕的孩子里,就属这个孩子哭声最洪亮了。''

海兰听着儿啼,露出了一个极为疲倦的笑容,呻吟着说了声``疼'',便虚脱了昏睡过去。如懿惊喜交加,看着-
个带着血丝的孩子被接生嬷嬷从锦被底下抱出,却是个极健康周正的男婴,忍不住欢喜得落下泪来,忙嘱咐乳母去清洗沐浴。如懿看过了孩子,正欲命人给海兰炖补药物,忽然发觉方才嬷嬷掀起锦被时,底下的鲜血似乎多得不可思议。她心下一沉,立刻再度掀起被褥,果然见猩红一片浸湿了被褥,让人不忍卒睹。

一颗心直直地坠下去,如懿立刻拉过一个接生嬷嬷道:``海贵人是睡着了,但似乎不大好。你仔细看看,怎么会那么多血?''

那嬷嬷不看则已,一看之下几乎是吓得魂飞魄散:``娴妃娘娘,大事不好了。贵人服了催产药用力过度,孩子虽然生下了,可孩子太大,贵人的下身,下身都\ldots\ldots{}''

如懿看她惊慌失措的神色,自己虽未生过孩子,却也知道是大不好了。她忙按住心神,问道:``海贵人究竟怎么了?''

那嬷嬷慌得瑟瑟发抖:``贵人的下身,撕裂了!''

如懿一惊之下,只觉得全身酸软,几乎站立不住。她---把抓住嬷嬷的衣襟,厉声道:``赶紧想法子!快!''

嬷嬷急得眼泪都要下来了,又是慌又是怕:``娴妃娘娘,事到如今,只能先撒上止血的白药,然后,然后由咱们几个嬷嬷仔细缝合起来。只是这个活计太难,又难免损伤贵人玉体。即便缝合之后,终究还是不能和从前比了。还请娘娘不要责怪!''

如懿只觉得一颗心涌在喉头突突乱跳,几乎要跳出嗓子眼来。她看着人事不知的海兰,极力强迫自己镇定下来:``现在还论这个做什么,赶紧先治海贵人要紧。''

接生嬷嬷忙不迭地张罗起来。如懿一口气说了这许多,自己也觉得气短胸闷,才恍觉手腕上疼痛不已,仔细一瞧,才发觉是被海兰用力之下,捏得紫胀发青了。叶心忙道:``娘娘稍候,奴婢去拿点消肿的药来给娘娘擦上。''

如激哪里还顾得上这些,忙道:``本宫这点瘀伤不要紧。你去看看皇子沐浴完了么?如果好了就抱来给本宫,本宫去给皇上瞧瞧。你好生看着接生嬷嬷替你们小主缝治,不许再有半点差错了。''

正说着,嬷嬷已经抱了包裹好的孩子出来。如懿忙抱了出去,外头的宫人们一早上赶着喜气洋洋地向皇帝道贺道:``皇上万福,皇上万喜,海贵人一切平安顺遂,生下了一个小阿哥呢。''

皇帝果然高兴,连连吩咐了赏赐延禧宫上下,又抱过了如懿怀中的孩子细看。海兰的孩子比寻常的婴孩大了一圈,一张小脸天圆地方,光滑饱满,十分精神。皇帝欢喜得不得了,抱在怀中爱不释手:``朕的皇子里面,就属五阿哥一出生就长相端方,天庭饱满,连哭声就那么洪亮,真是个有福气的孩子。''

如懿忙笑道:``皇上既觉得五阿哥有福,那就请皇上给五阿哥赐个名字吧。``

皇帝沉吟片刻,朗声道:``《穆天子传》中说,璂琪,玉属也。琪有珍异之意,朕的五阿哥,便叫永琪吧。''皇帝略想了想:``海兰给朕生了这么个好儿子,李玉,传朕的旨意,晋封海贵人为嫔位,为延禧宫主位,封号为\ldots\ldots{}''他朗然一笑:``朕心愉悦,便赐封号为愉,愉嫔如何?''

如懿脸上泛着笑,眼中一酸,忍不住别过脸去:``只可惜愉嫔不能与皇上同愉共悦了。''

皇帝一怔之下,也有些着急:``海兰是不是有什么不好?那么多太医和嬷嬷在,真是无用!''

如懿神色楚楚,屈膝道:``皇上,愉嫔为了给皇上生下五阿哥,被太医灌服了太多催产药,以致下身撕裂,出血不止。怕是好了,以后也会留下不足。''她仰起脸,目视着皇帝:``臣妾恳请皇上,以后不管愉嫔妹妹容颜衰老或是身体老倦,但求皇上不要厌弃她,只记得她是如何拼命为皇上绵延子嗣的。''

皇帝怜惜地看着她,将孩子交到个李玉手中,双手扶起她道:``你放心。朕自然不会。''

如懿就着皇帝的双手起身,隐隐有泪光盈然:``皇上,臣妾还有一亊相求。愉嫔爱子情切,若是可以,还请皇上将孩子留在愉嫔身边,不要送去阿哥所养育了。''

皇帝思忖着道:``愉嫔出身珂里叶特氏,乃是小族,不比嘉嫔母族高贵。这个\ldots\ldots{}''他见如懿满脸期盼,几欲落泪,也不忍拒绝:``那么朕答应你,即便永琪不留在愉嫔身边抚养,朕也会交给你,好让愉嫔时时相见。如何?''

这,也算是最好的打算了吧。如懿忙忙谢过,替皇帝紧了紧身上的海貂龙大氅,温然道:``夜寒如冰,皇上已经得了好消息,赶紧回宫补一补眠吧。臣妾留在这里照顾愉嫔了。''

皇帝微微颔首,吩咐道:``李玉,今晚伺候愉嫔的太医无能,尽数逐出宫去,永不复用。''

李玉正要答应,却听外头的小太监进忠跑进来,白着脸道:``皇上,不好了,不好了!''进忠跑得急,脚下一绊,几乎是滚到了皇帝跟前,张口结舌道:``皇上,慎嫔在冷宫上吊,按着皇上的意思,按嫔位的丧礼置办,对外只说病死。可是方才在火场焚烧慎嫔尸首和棺椁,谁知道那烧出来的火是、是、是蓝色的,不是红色的!''

皇帝乍然听了此言,不免吃了一惊,旋即喝道:``怪力乱神!人都死了,怎么可能烧出蓝色的火来?一定是你们胆小,以讹传讹!''

进忠吓得舌头都打磕绊了:``奴才不敢撒谎,奴才不敢。皇上,火场上的人亲跟见了,都说慎嫔含冤而死,死后发威了!''他说着,忍不住拿眼觑着如懿。

李玉眼尖,伸手左右两个耳光下去,骂道:``用你的贼眼珠子乱瞟哪里?不要命了么!''

夜风吹过光秃的枝丫有霍然的冷声,檐下昏黄的宫灯摇出碎金似的斑驳光影,恍若冷而沉的惶然一梦。

如懿神色如常,仿佛毫不放在心上,牵住皇帝的手沉定道:``自作孽,不可活!总不是臣妾与皇上让阿箬含冤而死。再说阿箬活着也就这点伎俩,死了还能翻出天来么!臣妾一定命人细查,看谁乱做手脚在后宫兴风作浪!''

\hypertarget{ux5982ux61ffux4f20-ux7b2cux4e09ux518c}{%
\part{如懿传 第三册}\label{ux5982ux61ffux4f20-ux7b2cux4e09ux518c}}

\hypertarget{ux7b2cux4e00ux7ae0-ux60c5ux5fc3}{%
\chapter{第一章 情心}\label{ux7b2cux4e00ux7ae0-ux60c5ux5fc3}}

皇帝温沉的手掌有难言的力量,按压着她纷乱而缥缈的思绪。他在她耳畔轻声叮嘱:``如懿,不要动气,不要落了旁人的圈套,心静为上。''这样温暖沉着的言语,听得她心中沉沉一动,不免生了几分依赖之情。

这种依赖,在她初出冷宫承宠的日子里,滋长最甚。一直有噩梦缠绕,那些在冷宫苦度的岁月,内心的惊恸,躯体的痛楚,无一不如蟒蛇将她紧紧纠缠。即便服下安神汤药,昏黑悠长的暗夜里,她仍会断续醒来。

似是察觉她的不安,皇帝陪她的时候,明显多起来。好些时候,她在噩梦中醒来,在烛火微弱的光线下,望着床顶雕刻的富贵华丽的吉祥图案,那些镂刻精致洒朱填金的青凤、莲花、藤萝、佛手、桃子、芍药,有种不知今夕何夕的茫然。然后,她听到他绵长的呼吸声。他的手臂,始终紧紧揽住她微微散着冷汗的身体,将自己的温度绵绵传递。他的手臂健壮而有力,紧紧包围她,即使在熟睡中也不松懈分毫。她昏昏沉沉睡去,又悸动不安醒来,始终被他裹在怀中,肉身相贴。

那一刻,她泪眼迷离。甚至有那么一瞬,她会相信,他一定,一定会陪着自己,共同等待大地黎明的来临。

其实她何必要事事算计,若有人可依靠,事事凭他做主,不也很好。就如阿箬一事,内里再怎么难堪,落在外人眼里,阿箬还是索绰伦氏慎嫔,在宫中谨慎侍奉多年,圣宠不衰,一时暴毙,风光大葬,家中与有荣焉。

皇帝都做得很周全。可是她,却不能不靠着自己。冷宫的蛇可以杀去,火可以扑灭,但是环伺身边蠢蠢欲动的毒物,那些躲在暗地里窥伺自己和海兰的人,如何能不怕?这条命,自己若不顾惜,还有谁会处处回护周全?

如懿静默着任由思绪辗转,皇帝含着温意絮絮述说:``朕知道,海兰为了替朕生下永琪,吃尽了苦头。你与海兰姐妹情深,她的孩子与你的孩子无异。朕明白你们的辛苦,也心疼永琪这个孩子,所以六宫上下,都会因为永琪的降生而得到朕的赏赐。延禧宫更是得足足添上三倍。''

如懿眼底微带了喜色:``皇上疼爱永琪,自然是海兰和臣妾的福气。只是臣妾怕赏赐太厚,反而惹来闲话。毕竟三阿哥和四阿哥降生时,都未曾这样厚赏呢。''

皇帝的眼笑得弯弯的,他的呼吸轻柔地拂在她的耳侧:``海兰为了这个孩子九死一生,差点连命都赔进去了,朕赏得再多也不算什么。六宫里皇后素来节俭,以身作则,宫中一应份例都减半,连金银器物都不甚打造。贵妃跟着皇后的样子,其余人便更不论了。倒是你,这些日子都操心苦辛,朕一直想好好赏你些什么。思来想去,便为你制了一样东西,从有这个主意到命人去做,其间一切,都由朕亲自操持,好容易才得了。本来就要给你的,结果碰上海兰生永琪,便耽搁了。等下闲些朕便叫人送来给你。''

如懿一心悬在未醒的海兰身上,惊悸难定,一时哪里顾得上皇帝要赐些什么,便笑笑也过了:``皇后娘娘主持六宫,素来以节俭为上。皇上为此物煞费心血,臣妾领恩,只不敢太过靡费了。''

皇帝眉目温然:``有皇后在,你们能靡费什么。也唯有嘉嫔爱俏,打扮得格外精细艳丽些。且嘉嫔是朕登基后第一个生下皇子的,又是朝鲜宗女,身份格外不同。所以朕想着,这次给六宫嫔妃的赏赐份例,嘉嫔得添一倍才好。''

这样絮絮半日,皇帝也有些倦,便回宫中歇息。夜寒漏静,永琪在乳母的哺喂后亦沉沉睡去,空气中浓郁的血腥气渐渐变得淡薄,反添了几分新生儿的乳香。如懿守在海兰身侧,拿着蘸了生姜水的热帕子细细替她擦拭着面孔和手臂。海兰过度疲累后昏睡的容颜极度憔悴,泛着不健康的灰青色。她难过得如同吞了一把酸梅子。这次艰难的生育,几乎要走了海兰的命,仅仅是把几个太医赶出宫,又如何抵得过?如懿想了想,还是唤来三宝:``这几日仔细留意着,看看今晚替愉嫔接生的几位太医,私下和什么人接触了。''

三宝知道轻重,立刻答应着去了。叶心上来点了安息香,劝道:``娴妃娘娘,小主的伤接生嬷嬷已经缝好,小主也睡了,您要不要也回宫歇一歇?''

如何能歇呢?在冷宫漫长难度的岁月里,都是海兰醒着神守候着她;如今,也该她守着护着海兰了。如懿沉吟片刻,还是微笑:``叶心,忙了一宿,你也累了。本宫让惢心去熬了止痛的汤药,等愉嫔醒了会给她喝。''

叶心答应着下去了。如懿望着东方渐渐明亮的天色,心中沉郁却又重了几分。

皇帝下了早朝之后便回到养心殿,他新得了皇子高兴,昨夜又替海兰担心,难免有些倦意。他正欲补眠,才进暖阁,却见皇后守着一碗热气腾腾的紫参乳鸽汤,笑吟吟地迎候上来。皇帝见她如此体贴,也是高兴,便由着李玉伺候他除了冠帽,问道:``皇后这么早过来了?''

皇后穿了一身暗红绣百子嬉戏图案刻丝缎袍,配着一色的镶嵌暗红圆珠玛瑙碎玉金累丝钿子,斜斜坠下一道粉白荧光的双喜珊瑚珍珠流苏,越发显得喜气盈盈。她端正地福了一福,满面含笑道:``恭喜皇上新得皇子。''

皇帝闻言欢喜:``皇后也得了喜讯了?''

皇后忙欠身道:``昨夜本该去延禧宫守着愉嫔生产的,可恨奴才们惫懒,见臣妾睡着,也不来叫醒臣妾。臣妾一早起来听闻愉嫔母子平安,当真欢喜,想着皇上肯定也高兴得一夜未睡好,所以特意让小厨房早早炖上了一锅紫参乳鸽汤,给皇上补气提神。''

皇后扬一扬脸,素心立刻捧过汤盅奉上:``皇后娘娘一醒来就嘱咐人备上了,只等皇上下朝来喝。娘娘一番心意,皇上尝一尝吧。''

皇帝掀开青瓷盅盖一嗅,不禁含笑望着皇后,赞许道:``辛苦皇后了。''

料峭冬寒尚未褪去,窗下一溜儿摆着数十盆水仙,那是最名贵的``洛水湘妃'',选取漳州名种,由花房精心培植而出,姿态尤为细窈,蕊心艳黄欲滴,花色白净欲透,颜如明玉,冰肌朵朵娇小,如捧玉一梭,自青瑶碧叶中亭亭净出。此刻那水仙被殿中红箩暖气一蒸,浓香如酒,盈满一室,连汤饮本来的气味都掩了下去,就好像自己对着皇帝的一片心意,总被那么轻易掩去。

想到此节,皇后不觉黯然,却不肯失了半分气度,便勉强笑道:``这水仙开得真好。前些年花房一直进献这些洛水湘妃,皇上总觉得未能臻于至美,如今摆在殿中,想来已经是最好的了。''

皇帝澹然一笑,颇有几分自得之色,轩轩然若朝霞举:``百花之中,朕向来中意水仙,喜爱其凌波之态,若洛水神仙。若是培植不当,岂非损了湘妃意态。''

皇后道:``传说水仙为舜之妻娥皇、女英化身。当年舜南巡驾崩,娥皇与女英双双殉情于湘江。天帝悯其二人对夫君至情至爱,便将二人魂魄化为江边水仙,才得此名。臣妾与皇上一般喜欢此花,便是爱其对夫君忠贞之意。''

皇帝若有所思,望着皇后和声道:``皇后的心意,朕都明白。''他转首看着那凌水花朵,轻声道,``临水照花,朕既是喜爱水仙忠贞之情,亦是深感娥皇、女英对夫君的恭顺无二,若不以夫为天,以君为天,又怎会这般生死不离,一心追随。''他修长的手指爱怜地划过莹润的花瓣,若薄薄的雪凝在他指尖,``且水仙开在冬日,凌寒风姿,才格外难得。''

皇后端然而坐,只觉得热烘烘的融暖夹着浓浓幽香往脸上扑来,几乎要沉醉下去,失去所有的防备。若然真能这般沉醉,却也不失为一桩美事。自成为他正妻的那一日起,负着富察氏全族的荣耀,担着儿女与自己的前程,何曾有一日松懈过。连这夫妻独自相对的时光,也是隐隐绷紧的一丝弦。她何尝不知道,宫中女子多爱花草,唯有那个人,那个让她一直忌惮的女子,也是如眼前人一般,喜爱这凌寒之花。是不是这也算是她与他不可言说的一点相似?

这样的念头不过一瞬,已然勾起心底零碎而杂乱的酸意。那滋味辛辣又苦涩,酸楚得几乎闷住了心肺,逼得她握紧了拳,深深地,深深地吸一口气,提醒自己:嫉妒,并非皇后应该表露的神情。至死,这样的情绪,只能掩埋在心,任凭它咬蚀透骨,亦要保持着外在的雍容得体。

旋然,她眉目温静:``得皇上喜爱,自然是好的。臣妾听闻今冬江南所贡绿梅颇多,娴妃素来喜爱绿梅凌寒独开,想来也是深明皇上惜花之情。''她见皇帝并不接话,只是津津有味地饮着她送来的汤饮,心头微微一暖,蕴了脉脉温柔道,``皇上不仅要为国事辛苦,还要为家事辛劳,臣妾不求别的,但求皇上万事顺心遂意,不要再有烦心之事就好。''

皇帝微有几分动容,口中却渐渐转淡:``皇后这样说,是觉得朕会有什么不顺心遂意的事么?''

殿外朝阳色如金灿,如汪着金色的海浪,一波波涌来,碎碎迷迷,壮阔无比。皇后端庄的脸容便在这样的明灼朝晖下渐渐沉寂下去:``臣妾今早听说慎嫔的棺樽在火场焚化时突然起了蓝色焰火,引得在旁伺候丧仪的宫人们惊慌不已。臣妾又听闻愉嫔昨夜虽然顺利产下皇子,但难产许久,自己的身子大受损伤,不免担心是否因昨夜的不祥而引起,伤了宫中福泽。''

皇帝停下手中汤盅,凝神道:``皇后是六宫之首,有什么话不妨直言。''

皇后的语调沉静而和缓,忖度着道:``臣妾听闻慎嫔虽是在冷宫自裁,但替她收尸的宫人们说,她浑身伤痕,且穿着一身红衣和红鞋死去,怨气深重。臣妾知道慎嫔从前是娴妃的侍女,许多事慎嫔有不当之处。赐死也罢受罚也罢,只是在宫中动用猫刑,还要合宫宫人看着以作训诫,未免太过狠毒,伤了阴骘。''

细白青瓷的汤盏在皇帝修长的指尖徐徐转动,看得久了,那淡青色的细藤花纹似乎会攀缘疾长,蔓延出数不清的枝叶伸展出去,让人辨不清它的方向。皇帝轻哂,颇有玩味之意:``皇后是觉得,愉嫔生育大伤元气,慎嫔棺樽起火古怪,都是因为娴妃私刑太狠的缘故?''

皇后本靠着填满了兰草蕙萝的沙金宝蓝起绒蒲桃锦靠枕,闻言忙欠身道:``臣妾不敢妄言,只是合宫人心浮动,臣妾不能不来禀报皇上。''

皇帝唇边的笑意还是淡淡地定着,眼中却淡漠了下去:``朕说过,皇后是六宫之首。朕曾在年幼时想过,六宫之首若幻化成形,应该是什么样子。朕想了许久,应该便如莲花台上的慈悲观音,心怀天下,意存慈悲,不妄听,不妄语,不行恶事,不打诳语。万事了然心中,凭一颗慧心巧妙处置。皇后以为如何?''

檐下的冰柱被暖阳晒得有些融化,泠泠滴落水珠,晨风吹动檐头铁马在风雨中``叮叮''作响,那深一声浅一声忽缓忽急地交错,仿佛催魂铃一般,吵得人脑仁儿都要崩裂开来。皇后勉强浮起一个笑容:``臣妾妄言了。不过,皇上所说的确是观音的样子,而臣妾虽为皇后,却也只是有七情六欲的凡人。皇上所言的境界,臣妾自愧不如。''

皇帝的侧脸有着清隽的轮廓,被淡金色的朝阳镀上一层光晕。他的乌沉眼眸如寒星般闪着冷郁的光,让人读不出他此刻的心情。``皇后说得对,人就是人,但所达不到的境界,也可以心向往之。''他微微一笑,仿若无意般挑起别的话头,``就好比朕身边伺候的奴才,从前王钦为人糊涂,肆意窥测朕意,连皇后赐婚对食的恩典也辜负,朕已经惩处了。如今有他做例,其他人都本分多了。''

烟罗纱窗滤来翡翠般的明净阳光,西番莲花模样的鎏金熏笼内徐徐飘出几缕乳色清烟。皇后温顺垂首,手指细细理着领口上缀着的珠翠领针。那是银器雕琢的藤萝长春图样,繁密的银绞丝穿着紫色宝石勾勒出精细的春叶紫藤脉络,原是她最喜欢的样式,此刻,却只觉得上头碎碎的珠玉射出细碎如针的炫光,一芒一芒戳得她眼仁儿生疼生疼的。须臾,皇后才觉得那疼痛劲儿缓了过去,露出柔婉容色:``皇上的意思,臣妾懂得。是臣妾失言了。原是早起嘉嫔来请安,提了几句宫中异象。但怪力乱神之语,实不该出自臣妾口中。''

皇帝微微颔首:``这样的话不仅不该出自皇后口中,皇后更应该弹压流言,免得宫中妄语成风,人心自乱。''

皇后恭谨道:``臣妾知道了。回去后自会训示六宫宫人,不许他们再胡言乱语。''

皇帝的笑幽幽暗暗,口气却温和到了极处:``嘉嫔素来口无遮拦,人却是直肠子,有什么话都不瞒着朕。所以她说什么,你听一耳朵便罢了,不必事事过心。''他见皇后的脸容渐渐有雪色,越发笑容可掬,``对了,还有一事,朕要嘱咐皇后。愉嫔生子是喜事,更有皇后替朕料理后宫的苦心。朕想着有子承欢膝下,皇后也可添欣慰。所以,六宫上下同赏半年份例。''

皇后勉强笑着,见皇帝倚窗而坐,这样风姿秀逸的男子,如玉山巍峨,纵然光华万丈,她却只能高山仰止,从来都难以接近,只能由着如是情意,默默淌过。只是此刻,他的欣慰和欢喜也是对着她的,倒并不像是只为添了个皇子,更是多年夫妻的一份安慰和亲近。不知怎的,她心里便软了几分。哪怕多年来时时处处顾着富察氏的恩荣,多年相伴,到底是有几分倾心的,何况又为他生儿育女。远远的儿啼声犹在耳畔,她蓦然念及自己早逝的永琏,心底狠狠一搐,牵动四肢百骸都一同抽痛起来,滴出猩红黏腻的血珠子。她极力将腮边的笑容撑得如十五无缺的月:``是。皇上的庶子,也是臣妾的庶子,都是一样的。只可惜臣妾与皇上膝下都只有一个公主,若是多几个玉雪可爱的女儿,那便更好了。只是说来说去,都怪臣妾无能,保不住皇上与臣妾的永琏。''

这一句``庶子'',骤然挑动了皇帝欢喜中的情肠,有如缕的悲愁蔓延上他微垂的唇角,他情不自禁地握住皇后皓腻的手腕,切切道:``女儿也罢,庶子也罢。皇后,朕与你终究是要有个嫡子的。''

皇后含着朦胧而酸楚的笑意:``皇上,臣妾侍奉您多年,必有许多不是之处。可臣妾一心所念,唯有皇上。臣妾无论如何,也会生下嫡子,以慰皇上心愿。''

皇帝握一握她的手:``皇后,无须说这样的话。''

皇后盈盈睇着皇帝,不觉泫然:``臣妾身为皇后,是不该出此软弱之语。可臣妾上有皇额娘,下有公主,又有母家荣华。可臣妾所能倚仗的,不过是皇上而已。''

皇帝轻嘘一口气,轻抚她肩头:``皇后的心思,朕懂得。皇后亦不要自怨自艾了。''

他懂得么?皇后在心底里轻笑出来,宫里的女子那么多,对着他个个都是笑靥如花,自己的艰难辛酸、如履薄冰,他如何能懂?就如她一般,哪怕相伴多年,很多时候,他的心思,她也是难以捉摸。

一世夫妻,唯有表面的荣光\ldots\ldots{}

皇后这般念着,转身处,终于忍不住低首落下泪来。

\hypertarget{ux7b2cux4e8cux7ae0-ux9b42ux68a6}{%
\chapter{第二章 魂梦}\label{ux7b2cux4e8cux7ae0-ux9b42ux68a6}}

海兰醒来是在黄昏时分。彼时如懿已守了她一日,累得腰肢酸软,不过咬牙挺着罢了。李玉在午后时分便已来过,千珍万重地将一个玛瑙巧雕梅枝双鹊捧珠镶盒交到她手中。那镶盒以大块深红与雪白的双色玛瑙挖成,白玛瑙为底,质地细腻,中间夹杂白色或透明纹路,留出鲜艳的俏色深红玛瑙雕出梅枝,枝干虬曲,花朵盛放,面上嵌青金、珊瑚、绿松、碧玺和水晶,点缀出碧叶红梅雪光明耀之样,两侧以珍珠浮雕衔环铺首,中间一颗拇指大的贝珠包金为纽,一看便知是连城之物。

李玉在她身侧,悄声道:``只为这盒子上的梅花,皇上便画了不下百次,真真是用心。奴才说句不好听的话,娘娘在冷宫的时候,皇上虽然不闻不问,但一人书画的时候,画的梅花比往日里多多了。原可从那些里头挑一幅好的便是了,可皇上还是觉着不够好,又画了好些,叫工匠们细细描摹了,做得不好便废置。饶是这样,这盒子也是出到第三个才好,只可惜了前头那些好玛瑙。啧啧!''

如懿淡淡一笑,不置可否,只是道:``这算是千金换一笑么?''

李玉哪里懂这个,摇头晃脑继续道:``这盒子也罢了,小主快打开看看里头的东西,才叫用心呢!''

如懿见海兰尚未醒来,遂也打开一看,只见两掌大的玛瑙盒子里,罗列着一排排绿梅的花苞,盈盈未开,如绿珠点点。更有一薄薄的红梅胭脂笺,她取过展开,却是皇帝亲笔,写着``疏疏帘幕映娉婷,初试晓妆新①''

那字写得小巧,如懿几乎能想见他落笔时唇角得意的笑纹。她眉心微曲,诧异道:``如今是二月里了,哪里还来这些含苞未放的绿梅?''她轻轻一嗅,``仿佛有脂粉的香气,并不尽是梅花香?''

李玉笑得合不拢嘴,抚掌道:``可不是?先用密陀僧、白檀、蛤粉、冰片各一钱,又以当季开得最盛的白芷、白芨、白莲蕊、白丁香、白茯苓、白蜀葵花、山柰、甘松、鹿角胶、青木香、笃耨香研至绝细,和以珍珠末、蛋清为粉。然后寻最巧手的宫女折来新鲜饱满的绿梅花苞,把这粉小心灌进花苞里,用线扎其花尖,将粉密封于花房之内蒸熟,再藏于玛瑙盒内,静置足月。如此花香沁粉,更能令面容莹似白梅凝雪,乃汉宫第一方。皇上知道小主喜爱绿梅,便称此物为绿梅粉,专供小主一人所用。''

李玉说得畅然尽兴,如懿只听到笃耨香一节,已经暗暗惊动。她出身贵戚,寻常宝物自然入不得她的眼,便是皇帝也每每好与她谈论奇珍。皇帝所用制香粉之法,传自明熹宗懿安皇后张氏的玉簪花粉法,只是玉簪花能存香粉,绿梅花苞却难,且用料更为奢华珍异。那笃耨香出真腊国,乃树之脂也。其色白而透明者名白笃耨,盛夏不融,香气清远,实在万金难得。如今却轻易用来做敷面香粉,珍重之余只觉心惊,若是为旁人所知,不知又要惹来何等闲话是非。

李玉极是乖觉,忙低声道:``用什么东西做这绿梅粉,都是皇上亲自定下的,所以内务府并不曾记档。''

不是不感动的。他记着她喜欢绿梅,惦着她的容颜憔悴,盼着她红颜如昨,为此不惜费尽心思,靡尽珍宝。但是在冷宫那些苟延残喘的日子之后,这些感动也仅仅只是感动而已。身外华物,哪里抵得上腔子里的一口热气,绝境里一双扶持的暖手。

珍重连城,也不过是一座城池的代价而已。

所以,再欢悦,亦有凉薄之意,沁染入心。然而她面上还是笑的,思忖片刻,取过笔饱蘸了墨汁,用一色的红梅胭脂笺一字一字郑重写道:``梅梢弄粉香犹嫩。欲寄江南春信。别后寸肠萦损。说与伊争稳。②''写罢,便依旧封了交予李玉手中:``只许教皇上瞧见。皇上见了,便知本宫心意。''她想一想,又道,``你虽有心帮我,但面上不可露了分毫。王钦之事后,皇上最不喜宫人窥测他心意。你到这个位子不易,一切小心。''

李玉诺诺离去,她方将那绿梅粉并玛瑙盒交予惢心一并送回了翊坤宫中。半倚在榻前,闭目凝神的瞬息里,想起自己所写,原是欧阳修的《桃源忆故人》,她只写了上半阕,却不肯写出那下半阕。只为上半阕的相思,便也是下半阙里她三年冷宫韶华苍苍的哀情。

``小炉独守寒灰烬。忍泪低头画尽。眉上万重新恨。竟日无人问。''她低低呢喃,在暖融融的殿内细细抚摸自己的十指。与旁人不同的是,她的手固然也戴着宝石嵌金的戒指,佩着华丽而尖细的珐琅点翠蓝晶护甲,纤手摇曳的瞬间,那些名贵的珠宝会映出彩虹般的华泽,曳翠销金,教人目眩神迷。可是细细分辨去,哪怕有鹅脂调了珍珠蜜日日浸手,但天气乍暖微寒的时节,旧时冻疮的寒痛热痒,无不提醒着她岁月斧凿后留在她身体上的斑驳痕迹。

唤醒她迷蒙心意的,是海兰初初醒转时低切的呼唤:``姐姐。''如懿如梦初醒,不觉大喜过望,才觉得悬着的一颗心实实归了原位。海兰虚弱地靠在宝石绿榴花喜鹊纹迎枕上,红红翠翠的底子锦华光灿,愈显得她的脸苍白得如一张薄薄的纸。她的神思仍在飘忽:``姐姐,真的是你?''

如懿握住她冰凉的手:``海兰,是我。我在。''

海兰嘘一口气,迷茫道:``姐姐,我以为自己熬不过来了。''

如懿闻言,眼便湿了。她端了止痛汤细细喂海兰服下,又将熬得糯烂的参片鸡汁粥喂了半碗,轻语安慰:``别胡说,我总在这儿。''

海兰问过孩子康健,长松了一口气:``万佛护佑,我终于替自己和姐姐生下了孩子。无论如何,只要孩子长大,咱们的下半生便有了些许依靠了。''

一句话便招落了如懿的泪:``只要你好好儿的,还提什么孩子不孩子。昨夜你九死一生,我只看着,只怕也要将自己填了进去了。''

海兰艰难地笑着,很快冷下脸道:``姐姐不能填进去,我更不能填进去。她们费尽心机,下的药让我变胖,变得丑陋,再不能得皇上宠爱。还让我的孩子难以出生,以致我吃尽了千辛万苦。若不是姐姐在旁陪伴,我一个撑不住,母子俱损,岂不更遂了她们的心愿。''

如懿替她掖好被角,柔声道:``如今你虚着,别想那么多。''

海兰冷笑道:``如何不想那么多!她们步步算计,只恨我自己蠢,后知后觉罢了!此事之恨,有生之年,断不能忘!''

如懿半垂着脸颊,伤感不已:``旁人害你,我自然是恨在心上。可是海兰,我的手也不干净。我的手害死过性命,只是我没有生养孩子,所以今日的事伤在你身上,否则便是这报应落在我身上了。''

海兰吃惊地睁大了眼睛,露出不屑之色:``姐姐居然相信天意报应?如果世上有报应,她们数次残害姐姐,为什么还没有受到老天爷的报应!所谓报应,从无天意,只在人为。今日她们要我和姐姐所受的种种,来日我都要一一还报在她们身上!若老天爷真要怜悯她们,恨我们狠毒,那就全都报应在我珂里叶特氏海兰身上。我只要姐姐和我的孩子万全就是!''

如懿心中震动不已,再多的委屈心酸,有这样的姐妹在身侧,深宫中茕茕独行,亦有何畏惧?她伸出手,紧紧拥住海兰,任由感动的泪水潸潸落下。

用过了晚膳,海兰便又歇下了。海兰的精神并不大好,总是渴睡。还是三宝回来,将火场之事一一告知如懿。

如懿悠悠拨着手上的鎏金红宝石戒指:``如今都认定是本宫逼死了阿箬,所以她死后还要闹鬼作怪,是么?''

三宝擦了擦脸上的汗水道:``可不是!宫中最喜欢这些鬼怪之语,怎么禁也禁不住,何况又是棺身起了蓝火那么诡异!也难怪大家都害怕。奴才方才去火场,几个替阿箬烧尸的太监吓得都说胡话了,满嘴胡言乱语,偷偷给她烧纸钱呢!''

如懿叹道:``冤有头债有主,谁是真正害死她的人,自然她就找谁去,本宫怕什么呢?''

三宝答应了一声:``还有一事,奴才见伺候愉嫔娘娘生产的两位太医,都曾悄悄见过启祥宫嘉嫔小主身边的陪嫁侍女贞淑。奴才记得有次贞淑自己说过,在李朝时她便是医女出身。奴才怀疑,愉嫔小主生产时被猛下催产药的事,只怕和启祥宫有干系。''

有乌云重重的阴沉凝在了如懿眉心。这样的神色不过一瞬,她已然冷笑道:``嘉嫔!本宫与她相处多年,一直以为她只是口舌上尖酸刻薄,爱讨便宜罢了。原来黄雀在后,也不是个省心的!''

三宝目光一凉,低声道:``这才叫日久见人心呢。时间久了,什么飞禽走兽都忍不住要出来了。小主,咱们要不要把那些太医截下来,向皇上告发嘉嫔?''

夜的羽翼缓缓垂落,掩去天际最后一缕蛋青色的光,将无尽的墨色席卷于紫禁城辽阔的天空。那种黑暗的郁积,教人望穿了双眼,也望不到渴盼的一丝明亮的慰藉。窗台上供着的一束腊梅送进一缕若有若无的清幽香气,叫人神清气冽。如懿沉着脸道:``不必了。皇上能治太医的,也不过是一个用药不当之罪。愉嫔胎儿过大,催产药量用得重些也是难免。仅仅是见过嘉嫔身边的宫女,也算不上什么确凿证据。且皇上又格外看重她,只这些话是没用的。''她掐着指甲,感受着指尖触着皮肉的刺痛,冷声道,``要打击一个人,就须彻彻底底,这样不咸不淡一下,费了力气和心思,也没什么大用处。''

如懿守了一会儿,见海兰睡得安稳,永琪也胃口极好,吃饱了乳母的奶水也乖乖睡了,便回到自己宫中去。

夜寒霜重,如懿才下了辇轿,却见一个十几岁的少年在宫门边徘徊不已。几乎是本能一般,她就认出了那人是谁,忙不迭唤道:``永璜!''

那身影惊喜地回首,一下扑进她怀里:``母亲!''

如懿捧起他的脸仔细看了又看:``好孩子!长高了,也壮了,看来纯妃待你很好。来!''她牵过永璜的手便往里走,``外头冷,跟着母亲去里头坐,暖暖身子。母亲叫人给你拿点心吃。''

永璜犹疑片刻,还是摇头道:``儿子在这里站一会儿就好了。''

如懿起疑:``怎么了?''

永璜踌躇着,尽量把自己的身影缩在墙角的阴影里:``儿子\ldots\ldots 纯娘娘不许儿子来翊坤宫。''

如懿当下便明白了,搓着他冻得冰冷的手道:``来很久了么?''

永璜连连点头:``自母亲回宫之后,纯娘娘一直不喜欢儿子来翊坤宫见母亲,所以儿子只能趁着今晚纯娘娘照顾三弟,才偷偷跑出来。''

如懿明白他的为难之处,柔声道:``那你赶快回去吧,出来久了,只怕纯妃宫里寻起来,知道了会不好呢。''永璜依依不舍地点点头,如懿替他整了整衣衫,呵暖了手道,``赶紧去吧,有空母亲会去见你的。再不济,逢年过节总能见上。你如今在纯妃宫里,她又有亲生的三阿哥,你凡事得格外小心顺从,明白了么?''

永璜眼中有晶莹的泪珠:``儿子明白。''

如懿实在是舍不得,心疼道:``这些年母亲不在你身边,你都这么过来了。你一定凡事都做得极好,不必母亲担心。''

永璜含泪道:``母亲在冷宫的时候,儿子一直牵挂不已。如今能看到母亲万事平安,儿子也放心了,只是\ldots\ldots{}''他低低道,``五弟出生,纯娘娘有些不高兴呢。''

如懿婉声道:``她不高兴她的,你只管你的,好好读书,好好争气。''

永璜点点头,终究还是后怕,匆匆带着贴身小太监小乐子跑着去了。一直走到长街尽头的僻静处,永璜才缓下了气息。小乐子忙道:``大阿哥,您慢点儿。恕奴才说一句,今儿您真是犯不上。纯妃娘娘待您好好儿的,你何必还来看望娴妃,若是被纯妃娘娘知道,可不知要惹出多大的是非来。''

永璜平复了气息,冷静道:``纯娘娘固然待我好,但她到底是有亲生阿哥的,我能算什么?再好也不过是个养子。可娴娘娘便不一样了,她如今出了冷宫,皇阿玛一定会待她好。若她再度收养我自然好,若不能,我在她和纯娘娘之间左右逢源,也是保全自己最好的办法。''

小乐子看他成竹在胸,仿佛与平日那个安分寡言的大阿哥判若两人,也不敢再吱声了。

如懿回到宫中,想着世情翻覆,亦不免心事如潮,到了二更天才蒙蒙眬眬睡去。虽然入了二月,京城偏北,地气依然寒冷。殿中用着厚厚的灰鼠帐,被熏笼里的暖气一烘,越发觉得热得有些闷。光线晦暗的室内,紫铜雕琢的仙鹤,衔着一盏绛烛笼纱灯。灯光朦胧暗红,像旧年被潮气沤得败色的棉絮一般,虚弱地晃动。

如懿睡得闷了一身潮腻腻的汗,不觉唤道:``惢心\ldots\ldots{}''

并没有惢心应和的声音,如懿才想起来,今夜并不是惢心守夜当值。应声赶来的是小丫头菱枝,年纪虽小,却也机灵,她忙披衣过来问:``小主可是口渴了?''

如懿掀起帐子,就着她的手喝了两口茶水,抚着心口道:``寝殿里闷得慌,开了窗去!''

菱枝忙道:``这后半夜的风可冷了,小主得当心身子啊。''

如懿摸着汗津津的额头:``瞧本宫满脸的汗,开条窗缝透透气便好。''

菱枝忙答应着走到窗下,才推开窗,只见眼前一道血红的影子倏忽晃了过去,只剩下几个微蓝泛白的小星点散落在空气里,像美丽的萤火,幽幽散开。

菱枝吓得两眼发直,哆嗦着嘴唇喃喃道:``鬼火!鬼火!''

如懿坐在帐内,也不知她瞧见了什么,便有些不耐烦:``菱枝,你说什么?''

菱枝像是吓得傻了,呆呆地转过脸来,似乎是自言自语:``鬼火?冬天怎么会有鬼火?''她忽然尖叫一声,``慎嫔死的时候就是蓝色的火。有鬼!有鬼!有吊死鬼回来了!''她一边喊一边尖叫着捂住了耳朵,缩到了墙角的紫檀花架后头。

如懿听菱枝一声声叫得可怖,也不免慌了手脚,忙趿了鞋子起身,拉扯着菱枝道:``你疯了,开这么大的窗子,是要冻着本宫么?''

菱枝拼命缩着身子,哪里还拉得出来。如懿虽然生气,却也冻得受不住,只好自己伸手,想去合上窗扇。如懿的手才触及窗棂,却有一股冷风猛然灌入,吹得她身上寒毛倒竖,忙紧了紧衣裳,口中道:``这丫头,真是疯魔了!''

如懿的话音还未被风吹散,忽然,一个血红而飘忽的庞大身影从她眼前迅疾飘过。如懿眼看着一张惨白的脸从自己面前打着照面飘过,哪里还说得出话来,身子剧烈一颤,惊叫了一声,直定定晕厥了过去。

①出自宋代词人赵师侠的《朝中措》。全词为:``疏疏帘幕映娉婷,初试晓妆新。玉腕云边缓转,修蛾波上微颦。铅华淡薄,轻匀桃脸,深注樱唇。还似舞鸾窥沼,无情空恼行人。''描写女子妆容之美。

②出自宋代词人欧阳修的《桃源忆故人》,全词为:``梅梢弄粉香犹嫩。欲寄江南春信。别后寸肠萦损。说与伊争稳。小炉独守寒灰烬。忍泪低头画尽。眉上万重新恨。竟日无人问。''此词诉说女子相思之苦,情哀之思。

\hypertarget{ux7b2cux4e09ux7ae0-ux8ff7ux79bb}{%
\chapter{第三章 迷离}\label{ux7b2cux4e09ux7ae0-ux8ff7ux79bb}}

如懿受了这番惊吓,第二日便起不来身了。满嘴嘟囔着胡话,发着高热,虚汗冒了一身又一身。太医来了好几拨儿,都说是惊惧发热。更有一个小丫头菱枝,一夜之间眼也直了,话也不会说了,只会缩在墙角抱着头嘟囔:``吊死鬼回来了!吊死鬼回来了!''

慎嫔棺樽冒蓝火的事才压下去,宫人们私下里难免还有议论,如今听着``吊死鬼''三字,不免让人想起慎嫔便是上吊死的。更加之冷宫一带这两夜常有人听见女子怨恨哭泣之声,越加觉得毛骨悚然。于是,翊坤宫闹鬼之事,便止不住地沸沸扬扬闹了开去,成了宫人们茶余饭后最津津乐道的谈资。

晞月领着绿筠和玉妍去看过如懿受惊之态,不免拿此事说笑了半日。回到宫中,晞月便更有些乏力,正见内务府的几个太监送了安息香并新做的被枕来,便伸出涂了水红蔻丹的手随手翻了翻道:``是什么?''

为首一个太监堆着讨好的笑容,谄媚道:``快开春了,皇后娘娘嘱咐宫里都要换上新鲜颜色的被褥枕帐,所以内务府特挑了一批最好的来给贵妃娘娘。''

晞月见锦被和软枕都绣着她最喜欢的石榴、莲花、竹笙、葫芦、藤蔓、麒麟的图案,不觉露了几分笑容:``这花样倒是极好的!''

那太监赔笑道:``这锦被上的图纹是由葫芦和藤蔓构成吉祥图案,葫芦多籽,借喻为子孙繁衍;`蔓'与`万'谐音万代久长。这个帐子满绣石榴和瓜果,多子多福,瓜瓞绵绵。娘娘您瞧,最要紧的就是这个软枕了,是骑着麒麟的童子戴冠着袍,手持莲花和竹笙,寓意为`连生',又有麒麟送子的意思。''那太监神神秘秘道,``这里头填的全是晒干了的萱草,是`宜男萱寿'的意思,气味清香不说,且和愉嫔与嘉嫔怀阿哥时的软枕是一模一样的。愉嫔与嘉嫔两位小主,就是枕着这个才有福气生下阿哥呢。''

晞月爱不释手,抚着软枕上栩栩如生的童子图样:``嘉嫔是出了名的阔绰,用东西也格外挑剔。她素日也不把愉嫔放在眼里,怎么也会和愉嫔用一样的东西呢?''

那小太监忙凑趣儿上来道:``娘娘您想啊,若不是真有用,嘉嫔哪里肯呢。如今只怕她还想再生一个阿哥呢。''他见晞月眉心微蹙,越发赔笑道,``其实皇上那么宠爱嘉嫔,不过是前头玫嫔和怡嫔小主的孩子都没了,她才那么金贵呢。若娘娘枕着这枕头有了阿哥,那她的四阿哥,给娘娘的阿哥提鞋都不配呢。''

晞月听得满心欢喜:``若不是她有阿哥在皇上跟前得脸,本宫哪里肯敷衍她!''她将软枕郑重交到茉心手中,``即刻就去给本宫换上这对枕头,仔细着点摆放。那灰鼠皮子的枕头帐子,睡得人闷也闷坏了。也把新的换上,讨个好彩头。''她剪水秋瞳喜盈盈地睇一眼那小太监,抿嘴笑道,``若真应承了你们的话,本宫自当好好打赏你们!''

那太监欢欢喜喜答应了,又道:``这安息香是内务府的调香师傅新配的,新加了一味紫苏,有益脾、宣肺、利气之效,于贵妃娘娘凤体最为相宜。还请娘娘笑纳。''说着便也告退了。

晞月便让茉心带着小丫头彩珠、彩玥收拾了被铺床帐,又试着点上了新送来的安息香,果然又甜又润,闻着格外宁神静气。她心下十分喜欢,吩咐道:``也算内务府用心,只是这样宁神静气的香,配着那四扇楠木樱草色刻丝琉璃屏风倒是俗了,也和新换上的颜色床帐不相宜。你们去把库房里那架皇上赏的远山水墨素纱屏风换了来,这才相衬。''

宫女们答应着利索换了。茉心知晓晞月的心意,便在帷帘处疏疏朗朗悬了三五枚镏金镂空铜香球,将安息香添了进去,丝丝缕缕缠绕的香气错落有致,又均匀恬淡,幽然隐没于画梁之上。

因着晞月素性怕冷,又叫添上好几个铜掐丝珐琅四方火盆,直烘得殿中暖洋如春。她眼见着四下也无外人,便低声道:``皇上养心殿外伺候的小张和小林子,别忘了送些银子去打点,这些年一直烦着他们在父亲觐见皇上时提点些消息,可得罪不起。''

茉心答应道:``奴婢都省得。只是有了王钦的事,御前格外严格,有些油盐不进呢。奴婢使了好多法子,李玉和进忠、进保三个,都搭不上。''

晞月烦恼道:``可不是!都叫王钦坏了事!真是可恼!否则,哪里用理会小张和小林子他们!你可仔细些,别教皇上发觉,又恼了!''

茉心乖巧道:``小主安心。今儿小主和纯妃、嘉嫔她们说话也累了,不如早些歇息吧。明儿起来还要去向太后请安呢。小主不是不知道,太后的孤拐脾气,一向不大喜欢嫔妃们晚到,若去得晚了,只怕太后面儿上又要不好看了。''

晞月拨着手里的蓝地缠枝花锦珐琅手炉,轻嗤道:``不好看便不好看吧。父亲当年为端淑公主远嫁进言,本以为太后会格外冷待本宫一些。只是这么些年了,倒也不曾见她对本宫怎样。到底不是皇上的亲额娘,也不敢做什么!便若真有什么,她老人家年寿还有多少,本宫来日方长,只当瞧不见便是了,何苦去理会她!''

茉心赔笑道:``可不是!皇上这么宠爱小主,连皇后娘娘也偏着小主。太后拿这些威势给谁瞧呀,也只能自己给自己添堵罢了。''

晞月由着茉心伺候了洗漱,忽地想起一事:``今日嘉嫔去看了娴妃,回来还向本宫笑话娴妃和阿箬反目,闹得阿箬变了鬼也不肯放过娴妃。可嘉嫔自己又有什么好的了!她最恨阿箬得宠,屡屡压制。后来阿箬封嫔,本宫怎么听说她还打过阿箬?这么看来,不知阿箬会不会也去找她呢?''

茉心笑嘻嘻道:``嘉嫔性子厉害,嘴上更不饶人,阿箬心里指不定怎么恨她呢。''

二人这般说笑,晞月换了一身浅樱红的海棠春睡寝衣,越发衬得青玉边玻璃容镜中的人儿明眸流转,娇靥如花。晞月谈兴颇高:``你没见娴妃今日那样子,自出了冷宫,她的性子也算变厉害了,对阿箬用那么狠的猫刑,逼得她吊死在冷宫里。结果就撞了鬼了,吓成那个样子,真真好笑!''

茉心轻手轻脚地替晞月摘下一双鎏金掐丝点翠转珠凤钗,又取下数枚六叶翡翠青玉点珠钿,双手轻巧一旋便解散了丰厚云髻。她取过象牙篦子,蘸了珐琅挑丝南瓜盒里的香发木樨油,替晞月细细篦着头发,口中笑道:``娴妃呀是自己做了亏心事,难怪阿箬阴魂不散,总缠着她。''

晞月颇有些幸灾乐祸,往足下的红雕漆嵌玉梅花式痰盒啐了一口:``在冷宫的时候,算她大难不死,如今竟也有被厉鬼追着不放的报应。''

茉心笑嘻嘻道:``奴婢听翊坤宫的宫人们说,闹鬼的时候菱枝那丫头看到穿着红衣的影子。阿箬死的时候特意换了红衣红鞋,那是怨气冲天想要死后化为厉鬼呢。如今看来,倒是真的遂了阿箬的心愿了。''

晞月听着便有些害怕:``真有这样的说法?''

茉心凑在她耳边,一脸诡秘:``可不是!奴婢听人说,有些人生前没用,被人冤枉欺负也没办法,只好想要死后来报仇。那样的人死的时候就得穿一身红,这样才能变成厉鬼呢。''

晞月听得惧意横生,按着心口道:``那样的鬼很凶么?''

茉心得意道:``当然了!那是厉鬼里的厉鬼,连萨满法师都镇不住呢,要不娴妃那样刚强的人能被吓成那个样子?小主你听,是不是前头翊坤宫有萨满跳大神的声音,奴婢方才听双喜说,连宝华殿的大师都去诵经镇压了呢,可娴妃还是昏昏沉沉说着胡话,人都没清醒过呢。''

二人正说着,殿阁里的镂花窗扇被风扑开了,``吱呀''一声,吹得殿中的蜡烛忽明忽暗。晞月吓了一跳,赶紧握住茉心的嘴道:``不许胡说!天都晚了,怪怕人的。''

茉心被这阵风一吓,也有些不安,忙噤声伺候晞月睡下了。许是安息香的缘故,晞月很快便入睡了,只是她睡得并不大安稳,翻来覆去窸窣了几回,才渐渐安静。听着晞月的呼吸渐渐均匀,茉心的瞌睡虫一阵阵逼来,将头靠在板壁上迷糊了过去。

也不知过了多久,茉心觉得脸上似乎拂着什么东西,她蒙眬着睁开眼睛,却见寝殿的窗扇不知何时被开了一扇,几点微蓝的火光慢悠悠地飘荡进来。茉心没来由地一慌,伸手去摸自己的脸。借着微弱的烛光,却见到一条红色的拂带悠悠从梁上垂下,正落在她脑袋上方,风一吹,便飘到她脸上来了。偏那拂带上头还湿答答的,像是落着什么东西。茉心心里乱作了一团,不知怎的还是不自禁地伸出手去,摸完瞟了一眼,却见手指上猩红一点。所有的睡意都被惊到了九霄云外,她忍不住叫起来:``血!怎么会有血!''

窗扇外一道红影飘过,恰恰与她打了一个照面,正是一张惨白的流着血泪的脸,吐着幽幽细细的声线道:``是你们害我!''

茉心整个人筛糠似的抖着,丢了魂般背过身去,却看到一脸惊惧的晞月,不知何时已从床上坐起,呆呆地愣在了那里。晞月额头涔涔的全是豆大的汗珠,几缕碎发全被洇得湿透了,黏腻地斜在眼睛上。她哪里顾得去擦,只是颤抖着伸直了手指,惊恐地张大了嘴,发不出一点声音。等到茉心回过神来知道喊人的时候,那个红影早飘飘忽忽不见了。

这一晚咸福宫中合宫大惊,晞月发了疯似的叫人到处去搜,可是除了那条沾血的拂带,哪里找得到半分鬼影。趁着人不防,晞月拉着茉心的手道:``为什么来找我?为什么来找我?她不是该去找娴妃的吗?是娴妃害死她,不是我呀!''

茉心止不住地发抖,依偎在晞月身边,惊惶地看着周围,嘀咕着道:``奴婢看见了,是阿箬,是阿箬没错,她眼睛里流着血,说是咱们害她的。不!她说,是你们害我!''她连连摆手,捂住脸惊悸不已,``不干奴婢的事,不干奴婢的事,阿箬说的你们,不是奴婢呀!''

晞月脸色惨白,颤颤地打了个激灵,尖声道:``不!不!她为什么不去长春宫,不去找皇后,偏来找咱们?''

茉心害怕地抱住自己,嘟囔着道:``皇后娘娘是六宫之首,她的阳气大,什么鬼怪都不敢去找她!所以来找小主您了!''

晞月怕得连眼泪都不会流了,拼命捂住耳朵,激烈地晃着头道:``不会的!不会的!是皇后派素心去招的她,我不过是跟在皇后身边听听罢了。''

茉心吓得哭了起来:``阿箬一定是怪小主当初在长街罚她跪在雨里,后来她虽然归顺了皇后娘娘,可那些事,咱们也脱不了干系!她在娴妃那儿一晃就走了,其实更恨咱们,所以挂了那一条红拂带,还滴着血要找咱们偿命!''她突然发现了什么,跳开老远,指着晞月的寝衣道,``小主,是不是您穿了红色,才招了她来?''

晞月一低头,果见自己穿着一身浅樱红寝衣,惊得几乎晕厥过去,慌忙撕下寝衣用力丢开,扯过锦被死死裹着自己缩在床角落里,喃喃道:``她不该来找我!不该来找我!''她看着周遭烛火幽幽,如初醒时见到的那几点鬼火不散,声嘶力竭地喊了起来,``来人!掌灯!掌灯!''外头的宫人被她惊动,忙将寝殿里的蜡烛都点上,亮得如同白昼一般,晞月才稍稍安静。

连着数日,但凡有咸福宫的宫人夜间出去,总容易听见些不干净的哭声。晞月受了这番惊吓,隔天夜里便去了宝华殿焚香祈福,求了一堆符纸回来。谁知才走到长街上,就见一道红影飘过,更是吓得不轻,再不敢出门。

自此,咸福宫中添了许多太监侍卫戍守。可不管如何防范,总是有星星点点的鬼火在夜半时分浮动。晞月因惊成病,白日里也觉得眼前鬼影幢幢,不分白天黑夜都点着灯,渐渐熬成了症候。连皇帝来看时,也吓得只是哭,连句话也说不完整。皇帝看着固然心疼,请了太医来看,却说是心病,虽然延医请药,却也实在不见起色。

相比之下,如懿倒是渐渐好了些。自从咸福宫闹鬼,翊坤宫就清静起来,惹得一众宫人私下里议论起来,都说那日阿箬的鬼魂原是要去咸福宫的,结果错走了翊坤宫。更有人说,指不定是慧贵妃背后主使害了阿箬,所以更要找慧贵妃报仇雪恨呢。

这样流言纷乱,皇后纵然极力约束,却也耐不得人心惶乱。这一日,皇后携了玉妍与和敬公主去咸福宫看望晞月,才在咸福宫外落了轿,便见福珈姑姑由双喜殷勤陪着,从宫门口送出来拐进了甬道。

皇后微微蹙眉,便道:``福珈姑姑也来了,怕是贵妃真病得有些厉害呢。''

玉妍扬着手里一方宝络绢子,撇着唇道:``太后也算给足了贵妃姐姐面子,若是臣妾病了,还指不定谁来看呢。''

皇后看她一眼:``越发口无遮拦了。你这直肠直肚的毛病,什么时候也该改改了,也不怕忌讳。''

皇后虽是训斥,那口气却并无半分责怪,倒像是随口的玩笑。玉妍娇俏一笑,便扶着皇后的手一同进去了。

才一进殿,却见硕大一幅钟馗捉鬼相迎面挂着,那钟馗本就貌丑,鬼怪又一脸狰狞。和敬陡然瞧见,吓得立时躲到皇后身后去了。皇后正安抚她,又见宫内墙上贴满了萨满教的各式符咒,连床帷上也挂满无数串佛珠,高高的梁上悬挂着好几把桃木剑,满殿里香烟缭绕,熏得人几乎要晕过去。

和敬哪里受得住这样的气味,一时被呛得连连咳嗽,莲心忙扶着她外头去了。

晞月见皇后进来,挣扎着要起身请安,皇后看她病病歪歪的,脸色蜡黄,额头上还缠了一块金铰链嵌黑珠青缎抹额,两边各缀了一颗辟邪的蜜蜡珠子,不觉好气又好笑:``瞧瞧你都干瘦成了什么样儿!太医来瞧过了没有?''

满室香烟迷蒙,晞月躲在紫檀嵌象牙花叠翠玻璃围屏后,犹自瑟瑟发抖。她泫然欲泣:``这本不是太医能治的病,来了也没什么用!''

皇后听着不悦,正欲说话,却见小宫女彩珠端了两盏缠枝花寿字盏来,恭恭敬敬道:``皇后娘娘,嘉嫔小主,这是我们小主喜欢的桑葚茶,是拿春日里的新鲜桑葚用丹参汁和着蜂蜜酿的,酸酸甜甜的,极好呢。''

皇后微微一笑:``若道调弄这些精致的东西,宫里谁也比不上慧贵妃。''说罢便舒袖取了茶盏,尚未送到唇边,已然听得玉妍婉声道:``皇后娘娘,您如今吃着的补药最是性热不过的,这桑葚和丹参都是寒凉之物,怕是会和您的补药相冲呢。''

晞月本自心神难宁,听得这一句,不由得奇道:``臣妾原以为只有皇后娘娘懂得这些药性寒热的东西,怎的嘉嫔也这般精通?''

皇后面色稍沉,停下了手道:``也是。最近本宫吃絮了酸甜的东西,以后再喝也罢。''

玉妍笑得甜腻腻的,只看着皇后道:``贵妃娘娘说笑了,妹妹能懂什么呀。不过是偶尔听皇后娘娘说过几次,记在了心上罢了。''

皇后赞许地看了玉妍一眼,晞月复又沉溺在惊惧之中,哀哀道:``如今皇后娘娘与嘉嫔还有心思记挂这些。臣妾日夜不能安枕,只求那\ldots\ldots{}''她惊惶地看一眼周遭,似是不敢冲撞,低低道,``只求能安稳几日便好了。''

皇后显然不豫,淡淡了容色道:``原想多请几个太医给你瞧瞧,如今看你这样子,倒是不必了。''

晞月颤颤不语,皇后皱了皱眉正要走近,只见茉心端了一盆清水过来,战战兢兢道:``恭请皇后娘娘与嘉嫔小主照一照吧。''

皇后脸色微变,谨慎道:``这是什么?''

茉心眼珠子乱转,看着哪里都一脸害怕:``皇后娘娘不知,如今出入咱们咸福宫的人都要照一照,免得外头不干净的东西附在人身上跟进来。''

皇后一听,遽然变色。玉妍满脸鄙夷,嗤笑道:``怪力乱神!鬼还没来呢,你们倒都自己被自己吓成这个样子了。''

茉心素来跟着晞月,如何受过这般奚落。只是见皇后也不斥责玉妍,只得诺诺退到一边。晞月一双秋水明定的眼眸里全是血丝,戚戚道:``皇后娘娘,臣妾没有一晚是睡得安稳的。她天天都来,天天都来!''

皇后柳眉竖起,正色道:``住口!不许胡言乱语!''言毕,她忽然微微蹙动鼻翼,疑道,``怎的有股血腥气?''

茉心期期艾艾道:``是\ldots\ldots 是狗血!''

皇后一惊,倒退一步:``狗血?''

晞月拼命点头:``是黑狗血。皇后娘娘,黑狗血能驱邪避鬼,臣妾吩咐他们沿着宫殿四周的墙根下都淋了一圈,果然这几天就安静些了。''

皇后向来温和,也不觉含了怒意:``你真是越来越疯魔了!身为贵妃,居然在宫中闹这些不堪的东西,还不如人家娴妃呢!她虽也吓坏了,也不过是请个太医看看,找萨满法师做做法事也就完了。偏你这里这么乌烟瘴气的,成什么体统!难怪皇上不肯来看你,本宫看了也是生气!''

晞月见皇后动怒,眼中含了半日的泪再忍不住,恣肆落了下来:``皇后娘娘,不怪臣妾害怕!实在是臣妾亲眼见过那个女鬼,真的是阿箬啊!这些日子,只要臣妾一闭上眼睛,就看着阿箬一身红衣满脸是血站在臣妾床头向臣妾索命。无论臣妾怎么让人防范,阿箬死的时候那些蓝色火焰还是会飘到臣妾的寝殿里来,臣妾实在是害怕!''

皇后铁青着脸道:``你一定是眼花了,再加上宫人们以讹传讹,才会闹出这样不堪的事来!''皇后正训斥,忽然听得风吹响动,原来是帷帘处垂挂的镏金镂空铜香球相互碰触,发出玎玲之声,其中香烟袅袅传出,更显神秘朦胧。她定下神问:``怎么白日里也点着安息香?''

茉心忙道:``回皇后娘娘,小主惊悚不安,说点着这个闻着舒服些。幸好小主受惊前一日内务府送来了这个,否则现在还不知道怎么好呢?''

皇后娥眉扬起:``是贵妃受惊前一日送来的,这几日一直点着?''茉心连忙点头,皇后脸上的疑色更重,起身走到帷帘下,摘下一个香球轻嗅,旋即拿开道:``贵妃这样心悸多梦,常见鬼神幻影,怕是闻了什么不干净的东西也难说。赵一泰!''

赵一泰忙躬身进来,皇后将香球交到他手中,道:``找个可靠的太医瞧瞧,里头的香料有没有什么不妥。''

赵一泰接了忙退下去,皇后看晞月犹自惊疑不定,便道:``好了,你不用怕。要真说闹鬼,本宫的长春宫怎么平安无事,怕是有人算计你也难说。''

晞月嘤嘤泣道:``若说算计,宫里能算计咱们的,有本事算计咱们的,也就娴妃了。可她自己都受了惊吓不明不白地躺在床上,还能做什么呢。皇后娘娘福气高阳气旺,长春宫百神庇佑,鬼怪自然不敢冒犯,左不过是臣妾这样无能的代人受过罢了。''

皇后的脸色越来越难看,片刻才缓过神色来:``你这么说,便是怪本宫了?''

晞月惊惶难安地抬起头来,慌不择言道:``阿箬来找臣妾做什么?臣妾是罚她跪在大雨中淋了一身病,所以逼急了阿箬投靠了皇后娘娘。许多事,臣妾看在眼里,也搭了一把手,可是臣妾并不是拿主意的那个人。为什么阿箬的鬼魂就抓住了臣妾不放呢?''

皇后眼中闪过一丝震惊,骇道:``放肆!阿箬来找本宫,是素心陪着她,一应都有了人证物证,本宫才听她言语,追查玫嫔与怡嫔之事。这些你都是亲眼看着的。''

素心亦忍不住抱屈:``阿箬是什么人,怎能见到皇后娘娘。她原来找奴婢,奴婢因忌讳她是延禧宫的人,也不理会。还是嘉嫔小主见她急切,才叫奴婢听她分说。这又干皇后娘娘什么事了?要说阿箬来找您,也定是她承宠这些年您总与她不睦的缘故。她死后魂灵有知,才来闹腾呢。''

皇后正色道:``贵妃,从前你偶尔一两句疯话,本宫都不跟你计较。原以为你懂得分寸了,谁知更不知忌讳,胡言乱语!''

缓缓话音未落,只见玉妍身形一闪,伸手朝着晞月就是两个耳光。那耳光来得太突然,只听见清脆两声皮肉相击之声,殿中便只剩下了袅远的静。晞月自侍奉皇帝以来,何曾受过这样的皮肉之苦,一时惊得呆了,不知该如何反应。

皇后颇为意外,盯着玉妍缓缓道:``高氏是贵妃!''

晞月骤然醒转过来,气得面上青红交加,也顾不得身子病弱,挥手便向玉妍扑来,斥道:``李朝贡女,也不瞧自己是什么身份,竟敢对本宫无礼!''

晞月是虚透了的人,哪里经得起这般惊怒挣扎,手指尚未碰到玉妍,自己已力竭斜在榻上,喘息不已。玉妍嫣然一笑,朝着晞月施施然行了一礼,如常般淡然自若:``贵妃娘娘,妹妹再无礼也是为了您好。今儿您可真是病得糊涂了,这样胡乱攀扯的话都说得出来,可不是连满门荣辱都不要了。妹妹虽是李朝贡女,可也懂得轻重高低。您做了这六年的贵妃,原来把生死荣辱看得这样淡,随口就想断送了它。您不可惜,妹妹还替您可惜呢。''她含着谦卑神色,向着皇后低婉道,``皇后娘娘,贵妃怕是病得糊涂了,您可千万别与她一般见识。''

晞月捧着自己的脸,仰面看着神色冷淡的皇后,无声地哽咽起来。

\hypertarget{ux7b2cux56dbux7ae0-ux9065ux9065}{%
\chapter{第四章 遥遥}\label{ux7b2cux56dbux7ae0-ux9065ux9065}}

如懿扶着惢心的手进了咸福宫的院中,只见和敬公主跟着双喜和彩玥正在玩闹。和敬跑着跑着便有些累了,赌气道:``不玩了不玩了!什么老鹰捉小鸡,还不如上回双喜玩那些蛇给我看呢。''

如懿正跨进院中,不觉怔了一怔,与惢心对视一眼,便立住了脚。和敬回过头来,正见如懿,便止了笑,淡淡施了个礼,``娴娘娘万福。''

如懿含笑回礼道:``公主有礼了,本宫看你和双喜玩得正得趣呢。''

和敬撇撇嘴,矜持道:``什么玩不玩的,我是公主,得守着规矩,哪里能整天玩呢。''

如懿见她硬要做出一副大人的样子,也不觉好笑,``可不是,跟这些太监宫女有什么好玩的。昨日本宫还听三宝说呢,外头棋盘街上来了个波斯的玩蛇人,一手蟒蛇玩得可好了。听说那蛇比柱子还粗,可是到了玩蛇人的手里,十分乖巧呢。''

和敬不以为然一笑,``娴娘娘就是见识的少,棋盘街上的东西也能当件事儿来说?要说玩蛇,现成双喜就是个厉害的,何必去说棋盘街上那些不入眼的东西。''

双喜听公主这般说,不觉吓得一噤,连忙摆手道:``奴才那些哪里能看呢?公主是抬举奴才罢了。''

和敬听双喜推辞,有些挂不住脸面,``这会儿倒谦虚了,从前慧娘娘与嘉娘娘都夸你呢。你在火场外头养了好些蛇呢,能引得它们乖乖地游过来游过去,它们可不听你的话?哪天给娴娘娘瞧瞧,也让她不必羡慕外头去了。''说罢,她便走到乳母身边,独自玩去了。

双喜听了这话,恨不得缩到彩玥身后去。如懿浑不在意,``好了。如今贵妃病着,别再说这些怕人的话了。本宫看贵妃病着,也无心顾得到你们呢。对了。贵妃呢?''

彩玥忙道:``小主在里头歇着呢。皇后娘娘正和小主说话。''

如懿便道:``那也罢了,原以为贵妃和本宫得的是一样的病,想过来看看她。彩玥,本宫这里有一本宝华殿大师亲手抄录的佛经,每天念一念倒是很安神。你便替本宫转赠给贵妃吧。''

彩玥忙不迭谢过,``娴妃娘娘真是雪中送炭了,咱们小主得了这个,或许能安心些。''如懿嫣然一笑,深深看了双喜一眼,转身便离去了。

到了夜间,晞月服了安神汤睡了,却眉头紧锁,满口胡乱呢喃,额上冒着豆大的汗珠。茉心守在一旁,着急唤道:``小主,您醒醒,您醒醒!''

晞月自惊梦中醒来,一摸身上,素色寝衣都汗透了。茉心道:``小主,皇后走了之后您便睡得不好,奴婢看您这么辛苦,只得叫醒您了。''

茉心说罢,便递了一碗银耳汤过来,``银耳汤宁神,小主喝一些吧。''

晞月嘴唇上都起了焦皮,勉强喝了一口,抬首见香球照旧挂上了,不觉惊道:``皇后不是说里头的安息香有古怪么?怎么又用上了?''

茉心忙安慰道:``方才是替小主您诊脉的太医送回来的,说安息香无事,可以继续用着。''

晞月点点头,惶恐地抓住茉心道:``我又梦到阿箬了!茉心!我又梦到她了!''

茉心慌兮兮道:``小主,您别说了!奴婢伺候您沐浴更衣吧。身上这么湿着,怕不好受呢。''

晞月吃力地颔首,扬声道:``双喜!叫人备热水!''

进来的却是彩珠,她福了福道:``小主,您有什么吩咐?''

晞月诧异道:``双喜呢?去了哪里?''

彩珠有些为难,不知说还是不说,犹豫了片刻还是道:``双喜被皇上身边的李公公叫走了。说他手脚不干净,趁着去养心殿送东西的时候不知摸走了什么,到现在还没回来呢。''

晞月动气,``双喜被李玉带走了?本宫怎么不知道?''

彩珠道:``小主方才睡着了。李公公说了,不许惊动小主。''

茉心着紧道:``双喜伺候小主这么久了,就算有什么,小主能不能求求皇上,饶了他这次。他可知道咱们不少事情呢。''

晞月一张脸本就熬得干瘦,颧骨高高凸起,此刻更是煞白可怖,她背靠着床喘息着道:``快扶我起来,我去养心殿瞧瞧。''

茉心忙劝道:``可是小主,外头天都黑了呢。怕是\ldots\ldots 怕是\ldots\ldots{}''她的话虽未出口,神色却已提醒了晞月。

晞月吓得浑身一颤,眼珠子骨碌碌望着四周,也顾不得双喜了,忙缩在了床脚,颤声道:``那我,我便明天去吧。''

次日趁着日色明亮,晞月顾不得身子,一早便赶到了养心殿。李玉在滴水檐下迎候着,十分恭谨,``贵妃娘娘且先回去吧。双喜的事,怕是求也不中用了。''晞月如何碰过这样的软钉子,当下不悦道:``双喜犯了什么事?连本宫的话也不中用了?''

李玉笑吟吟的,``回贵妃娘娘的话,双喜手脚不干净,趁着您吩咐来养心殿送东西时,顺走了一块先帝爷用过的玉佩,昨儿奴才一拉他进了慎刑司,才受了十二道刑罚,他便都招了。按着皇上的旨意,已经叫乱棍打死了。''

晞月气得嘴唇哆嗦,``什么玉佩,怎地本宫都不知道?''李玉弯腰陪着笑道:``贵妃娘娘病着,精神不济,自然什么都不用知道,免得伤身。皇上还说了,一切与您不相干,你且回去歇着就是。皇上得空,自然会来看您的。''李玉弯腰陪着笑道:``贵妃娘娘病着,精神不济,自然什么都不用知道,免得伤身。皇上还说了,一切与您不相干,你且回去歇着就是。皇上得空,自然会来看您的。''

晞月迫近两步,急道:``那双喜死前,招了些什么?''李玉皮笑肉不笑,扬了扬拂尘道:``能招什么?做了什么便招了什么罢了。贵妃娘娘,这里风大,您且回去吧。''他定一定神,又笑:``奴才们的事再大也入不得主子的眼,贵妃娘娘不必揪心,再挑好的来伺候就是。就好比\ldots\ldots{}''他一顿,笑得灿烂,``皇上跟前伺候的小张和小林子,今儿一大早也被乱棍打死了。不为别的,就为立个规矩,叫他们不许乱递消息。自然了,这都是奴才的不是,总怪不到皇上身上去。您哪,好自珍重就是。''

晞月听着这话明是劝慰,里头却夹杂着不少自家隐事,一时心神大乱,脸上青一阵白一阵,眼前金星乱冒,勉强扶了宫女的手走了几步,身子一晃,径自晕了过去。如懿听着养心殿外的动静,捧了一盏杏露莲子羹到皇帝跟前,婉声道:``既然贵妃突然晕厥,皇上不妨先让人挪到偏殿休息吧。''

皇帝定定道:``朕不想见她。''他接过杏露莲子羹,看了一眼道:``是杏露莲子羹?好端端的,怎么给朕备了这个。''如懿脉脉睇他一眼,温然含笑,``莲心苦寒,过于伤身,臣妾已经剔干净了,只剩下清火的功效。杏露入口清甜,正好润燥安神。臣妾想,皇上此时的心情,喝这个最好不过。''

皇帝的脸色冷得如一块化不开的寒冰,``该吐的双喜都吐干净了。和高氏有关的,朕都听进去了。再和旁人相关的,双喜语焉不详,也知道的不甚清楚。朕无谓再查下去。''

如懿沉默片刻,轻声道:``宫中传言四起,臣妾重罚过阿箬,固然不能不怕。但高氏也被谣言惊动,畏惧至病,皇上已经觉得她有疑,所以一直不曾好好去看过她。''皇帝冷哼一声,``高氏怕成那样子,朕便知道她和阿箬有见不得人的事。''

如懿立在皇帝身边,似乎这样的切近才能让她安心说出心底的疑虑,``臣妾身在冷宫时被群蛇围伺之事,双喜已然招了是高氏主使的。火场那窝蛇也找了出来。只是臣妾不明白,为什么怡嫔有孕时被蝮蛇惊动胎气之事双喜却至死不招?认了一件难道便不肯认第二件么?''

皇帝嗤之以鼻,``那些奴才素来奸猾,能少认一桩怕也是好的,还以为能少些责罚呢!既然都是蛇,即便不是他做的,哪里能脱得了干系!左右也是一死!''

如懿只得默然不提,又道:``至于朱砂水银毒害龙胎之事,双喜只知道是高氏拉拢了阿箬,参与其中,至于是不是拿主意的人,他也不甚清楚。皇上与臣妾一样,隐隐知道高氏虽然做事狠了些,但未必有这样周全的智谋。''

皇帝静静听着如懿说完,牵了她的手在榻上坐下,温言安抚道:``朕知道事情不查得水落石出,便是委屈了你。可是你要知道,许多事盘根错节,若弄得太清楚,便会到了连朕都无法收拾的地步。朕登基才这些年,不能有任何动摇国本的事出现,免得人心浮动,江山不安。''

如懿低低垂首,伏在皇帝肩上,眼波似绵,丝丝媚然,绵里却藏针:``皇上的心胸里有江山万代,臣妾的心胸里却只有皇上。所以,臣妾听皇上的。只是高氏残害皇嗣,多次意图杀害臣妾,臣妾实在是\ldots\ldots{}''

皇帝的手搭在她肩上,有温热的气息从他掌心隔着薄薄的春衫缓缓透进:``高氏在朕身边多年,总是温柔如水,却不想背后竟是这个样子。朕有生之年,不想再见到这样的毒妇。可是如懿,她的父亲高斌并无大错,又是朕在朝堂上的可用之人。朕不能因为他女儿的过失迁怒于他。所以对着外头,朕不会给高氏任何处罚,她也依旧会是朕唯一的贵妃。''

如懿纤细的手指一点点攀上皇帝的胸口,澹澹儿薄的衣衫下有滚热的心跳,带给她罹乱中些许安定之意:``臣妾不在意名位,只在乎皇上的用心。''

外头春光初绽,如一幅锦绣画卷,初初绽放华彩。皇帝便在这朝阳花影里,轻轻拥住她:``朕能许你的,便是用心了。朕知道你喜欢孩子,愉嫔的身子坏成那样,你的身体既然好些了,明日朕就让人把永琪抱来给你抚养。''

如懿的笑里含了薄薄的喜悦:``多谢皇上体恤。''

皇帝慨叹道:``其实你再喜欢永琪,他到底不是朕和你亲生的。朕一直很想和你有自己的孩子,才当是朕的用心,有了最能着落的地方。''

二月的春光是枝丫上新绽的一点嫩绿的芽,一星一星地翠嫩着,仿佛无数初初萌发的心思,不动声色地滋长。她伏在皇帝心口,听着他沉沉的心跳,似乎安稳地闭上了眼,有了几分感动。这么多年的深宫岁月,她所祈盼的,其实与凡俗妇人并无任何不同。夫君的关爱疼惜,儿女的膝下承欢,如同这世间每一个女子的渴望。若真有不同,或许是她更早地明白,早到也许是在初初嫁为人妇的时候,她便清醒地知道,她从不能拥有自己夫君的全心全意。钟鸣鼎食的王侯府第,朱门绣户的官宅民苑,哪怕只是多了几亩田地的富户农家,也会想着要讨一房妾室。三妻四妾,旧爱新欢,凭着她的家世,无论嫁到何处,都脱不了这样的命数。

虽然她没有孩子,虽然她是那样渴望孩子,可皇帝,到底是以另一种方式成全着她,安慰着她。如懿以轻柔之音相对:``那么,臣妾也用心弹奏一曲,回报皇上,如何?''

皇帝素性雅好器乐,养心殿暖阁中便有上好的宋琴``龙吟'',如懿原是弹得惯了,便取下轻拢慢捻。琴音宛若春雨打破一池春水,渐弹渐高落后琴音渐渐舒缓,愈来愈低好似女子在花树下低声细语,相对言笑。

皇帝闭目须臾,轻声道:``是李之仪的《卜算子》。''

``是。''如懿素手轻扬,衣袖的起伏若碧水三尺,飘飘若许。伴着琴音潺潺,她轻声吟诵:``我住长江头,君住长江尾。日日思君不见君,共饮长江水。此水几时休,此恨何时已。只愿君心似我心,定不负相思意。''

皇帝睁开幽深的眸,怜惜地望住她:``朕与你并无相隔,何来这样日日思君不见君之意?''

悠长的羽睫垂下如扇的浅影,遮掩着绵绵不可言说的心事。如懿低低道:``前头的都不要紧,臣妾只在乎一句。''她微微凝神,正欲言说,皇帝却也同时道:``只愿君心似我心,定不负相思意。''这一瞬的心意相通,让她稍稍有些安慰:``臣妾知道皇上有太多人太多事,臣妾亦不敢妄求贪多,只求这一句便好。''

皇帝的眼中有深深的情意,如同最温暖的泉水,将人都溺了进去:``朕或许宠幸你不是最多,那是因为朕是皇帝,朕也无法做到最多或是最好。但是如懿,朕希望和你长长久久地走下去,那才是朕真正不负了你的相思意。''

琴声袅袅,浮上心头的情意,亦是袅袅。皇帝言毕,铮铮琴音已然奏起。她的双手游移于琴弦之间,修长洁净的指,指节分明的骨,缓缓弹奏吟诵:``车遥遥,马憧憧。君游东山东复东,安得奋飞逐西风。愿我如星君如月,夜夜流光相皎洁。月暂晦,星常明。留明待月复,三五共盈盈。''

唇齿间反复吟诵,寻觅着依稀可知的温情,借以安下自己飘摇不定的一颗心。她投入他怀中,眼中有了温煦的热意:``愿我如星君如月,夜夜流光相皎洁。''

回到殿阁中已经是三更,侍寝后的疲倦尚未消除,如懿泡在浸满玫瑰花的黄杨浴桶中,以温热的水来疏散身体与心思的疲乏。惢心一勺一勺地替她加着热水,如懿闭着眼静静道:``惢心,辛苦你了。''

惢心细长的手指捞起片片殷红的玫瑰花瓣,反复替如懿按着雪白的肩,口中道:``奴婢只是装神弄鬼,哪里比得上小主费心筹谋辛苦。''

如懿将身体浸得更深些,让热水漫到了下颌,才舒然松了口气:``我的辛苦不过是找一个人的软肋。高晞月最在乎身份与恩宠,如今恩宠断绝,身份只成了空衔。她一生心高气傲,却也胆小得紧。自从被你吓了一回,便再没有神志安宁过。''

``小主是找她的软肋,奴婢不过是照着她的软肋打下去罢了。咸福宫寝殿里闹鬼火,那星许磷粉是掺和在蜡烛里头的,每到夜半,蜡烛烧了一半的时候里头的磷粉也会跟着烧起来,不用奴婢去扮鬼,她们也相信是阿箬的鬼魂去过高晞月的寝殿了。还有奴婢扮鬼时那些鬼火,都是烧了一点点磷粉在手炉里藏在奴婢袖子中,用时撒出去就好了。''惢心抿嘴一笑,带了几分得意,``而且奴婢先在咱们自己宫里作怪,只当小主吓病了,那再有什么,人家也疑心不到一样受了惊吓致病的小主身上了。也亏得小主一早就安排三宝在阿箬的棺樽里撒了磷粉生起事端,让所有谣言的矛头都直指咱们宫里,这才反而撇得干净了。''

``欲先取之,必先予之。不把自己扯在浑水里头,反而不好独善其身了。''如懿似是想起什么,``听说皇后曾经以为贵妃宫里的安息香有异,还特意取了些去查过?''

惢心快活极了,脸上是兜不住的笑:``谁会傻到在那些安息香里做手脚,岂不麻烦?奴婢把那些扰乱心志让贵妃睡不安稳的草药细细研磨了缝进她的睡枕里,料谁也不会疑心。谁叫贵妃做了那么多亏心事,夜夜惊梦也是活该!''

如懿赞许地拍了拍她的手背,只是含笑不语。氤氲的水汽扑腾上来,将如懿的脸蒸得嫣红如霞,可她的眉心却渐渐紧锁成个``川''字,她狐疑着道:``惢心,虽说皇上已经处置了双喜,可我心里总有个疑影儿,为什么当日怡嫔有孕时,她所住的景阳宫的油彩里掺着会引蛇的蛇莓汁液?既然双喜会驱蛇,这样做岂不多此一举?''

惢心侧首想了半日:``双喜会驱蛇,若说懂这个,也说得过去。''

如懿伸着三寸长的水葱似的指甲,划着黄杨浴桶,那轻微的触碰声如她不能平复的心境:``我记得怡嫔住在延禧宫安胎时,高晞月为求争宠,曾想让怡嫔也搬去她宫中。若怡嫔被蛇惊动胎气之事是她指使双喜所为,她要怡嫔去她宫中安胎,若有何闪失,岂不是自寻麻烦?''

惢心听得入耳,苦苦寻思:``是有些蹊跷,小主以为当时之事是皇后主使?其实这次的事,小主大可让奴婢再去长春宫吓一吓皇后也好。若能顺势除了皇后\ldots\ldots{}''

如懿转首看了她一眼,摇头道:``皇后是国母,又是先帝亲自挑给皇上的,在皇上心中的地位绝不同于高氏。且皇后不比高氏柔弱胆小,万一吓唬不成,反而让她识破,那便糟了。''

惢心连连顿足,惋惜道:``只可惜这次的事双喜供不出皇后来,否则也还好些。''

温热的水舒散了紧绷的心神,如懿漫然出声:``双喜不过是高氏的奴才,怎么会知道皇后的事。若真要找到能动摇皇后在皇上心中地位的证据,只有真正与皇后密谋过的那个人才说得出来。''

惢心思量着道:``小主的意思,是\ldots\ldots 高晞月?''

如懿撩起一点清水洒在自己的手臂上,朗然道:``是啊。可惜,还不是时候,而且这个时候高晞月所说的话,皇上也必定不会相信。咱们只能等等了。''

惢心不甘道:``那得等到什么时候啊?''

如懿望着殿阁里跳跃的烛光,微笑道:``人之将死,其言也善,才能振聋发聩啊。''

晞月自回咸福宫,病势便越发沉重。原先不过是鬼神乱心,此时又多添了许多人事的惊惧,一来二去,便认真成了大症候。而皇帝,虽然屡屡派人慰问,太医也照旧看着,却再未去看过她一次。情疏迹远,便是如此。

\hypertarget{ux7b2cux4e94ux7ae0-ux4e24ux5fc3}{%
\chapter{第五章 两心}\label{ux7b2cux4e94ux7ae0-ux4e24ux5fc3}}

晞月自回咸福宫,病势便越发沉重。原先不过是鬼神乱心,此时又多添了许多人事的惊惧,一来二去,便认真成了大症候。而皇帝,虽然屡屡派人慰问,太医也照旧看着,却再未去看过她一次。情疏迹远,便是如此。

皇后去看过两次后亦喟然叹息:``既然病成这样,万一病中再说出什么胡话来可怎么好?看着也怪可怜见儿的,若不是满口胡话,本宫倒也肯怜惜她。''

素心笑道:``皇后娘娘就是宅心仁厚。如今皇上都不肯去看她,只是顾着外头的面子,宫里更无人探视,也唯有皇后娘娘肯垂怜。''

皇后叹道:``她追随本宫多年,也不算不尽心。许多事本宫未曾想到的,她先赶着做了。虽然做得不够圆满,但心思总还不错。''

素心思忖着道:``那奴婢会请齐太医好生看着贵妃,给她用些精神气短的药。人病着,就该不必说话,安静养神。另外,奴婢嘱咐彩珠,好好提点她的主子,不要胡言乱语。''她想一想,又禀道,``高夫人一直说想进来看望贵妃娘娘,还有高大人说要送些补品进来问候。''

皇后拨着手上的素银护甲,沉吟道:``即便是本宫病了,也没有母家常来探望的事。对外便说皇上对慧贵妃很好,让他们放心,探望就不必了。至于补品,他们送进来了,你就让送到贵妃床跟前儿,也好提醒着贵妃,她家里是还有人在的。''

素心答应了一声,便道:``皇后娘娘,蜀中新贡了一批颜色锦缎,花样儿可新奇呢,说是比前明的灯笼锦还稀罕!内务府总管已经来回禀过,让咱们长春宫先去选一批最好的用。''

皇后微微低首,看着身上一色半新不旧的双色弹花湖蓝缎袍,正色道:``蜀锦价贵难得,更何况是胜过灯笼锦的。本宫一向不喜欢这些奢靡东西,嘉嫔素爱这些,你悄悄送去启祥宫一些便罢。''她见素心低着头,又道,``你既要去内务府,便告诉他们,快入春了,长春宫该领春日的衣裳了。''

素心忙道:``按着规矩,娘娘的贴身宫人是八身衣裳,余者是四身,奴婢会一应吩咐到的。''

皇后扶了扶鬓边摇摇欲坠的绢质宫花,凝神片刻,道:``做这么些衣裳,谁又穿得了这么多,都是靡费了。告诉内务府,别的宫里也罢了,长春宫宫人的衣裳,一应减半便是。''

素心呆了一呆,很快笑道:``娘娘克己节俭,奴婢不是不知。只是旁的小主好歹有珠花簪钗,娘娘是六宫之主,一应只多用这些通草绢花,实在也是太自苦了些。''

皇后轻叹一声,含了几许郁郁之情:``嫔妃们爱娇俏奢华,本宫有心压制却也不能太过。只能以身作则,才能显出皇后的身份。也好教皇上知道,本宫与那些争奇斗艳之人是不一样的。''

素心勉力抬起下垂的唇角,绷出毫无破绽的笑容:``娘娘用心良苦,已经够为难自己的了。且不说别的,长春宫上下从娘娘开始,到底下的宫人,素来连月例都是减半的。娘娘也别太苦着自己了。''

皇后也不放在心上,只道:``你们都在宫里,没个花钱的去处,月例少些也不妨。且不说别的,外头的名声,可是使银子也不能得的。''

素心诺诺应承了,一脸恭顺地道:``娘娘的嘱咐,奴婢即刻去内务府知会一声。''

皇后看一眼窗台上新供着的迎春花,笑意盈然:``春来花多发,你出去时告诉赵一泰,明日本宫想去坤宁宫好好祭神参拜,也好祈求后宫安宁,贵妃早日康复吧。''

素心出了长春宫,才慢慢沉下脸来,闷闷不乐地沿着长街要拐到内务府去,却见玉妍带着侍婢贞淑,抱了永珹正往长春宫方向来。素心见了玉妍,亲亲热热行了一礼:``嘉嫔小主万安。四阿哥万安。''

玉妍扬一扬绢子,见并无外人,忙亲手扶住了素心:``没外人在,快别闹这些虚文了。''她细细打量着素心神色,``怎么方才瞧你过来像是受了委屈,可是皇后娘娘又要一味节俭拿你们作筏子了?''她放柔了声音,``真是怪可怜的,你额娘的痨病少不得用钱吧。若是还要用山参吊着,你尽管来告诉本宫。''

素心眼圈一红,转过头低叹一声道:``都是奴婢命苦罢了,额娘得了这么个富贵病,光凭奴婢的月例银子,够买几支参请几次大夫的?还好额娘身边有妹妹照顾着,只不过都望着奴婢的月例罢了。本来月例都减半了,如今连季节衣裳都要减半。皇后娘娘是一味慈心得了贤良名声,可苦了咱们底下的人,说是伺候中宫的,穿的戴的竟比那些伺候贵人小主的都不如。若要向娘娘求恳恩典,一回两回也罢了,若是多了,皇后娘娘还当咱们是变着花样儿使钱呢,奴婢更不敢说了。''

玉妍听得连连叹息:``好丫头,难为你一片孝心。''

素心忙按下悲戚之色,强笑道:``都是奴婢不是,又对着小主诉苦。自从奴婢的额娘六年前得了这个病,都不知道用了小主多少山参和银子了,怕奴婢几辈子都还不清。''

玉妍忙牵住素心的手,推心置腹道:``旁人不晓得,你还不清楚本宫的脾气。本宫素来是个眼里容不得沙子的,凡事只讲缘法二字。若是不投本宫的缘法,便是什么宠妃小主,本宫都不理。可你不一样,打从本宫进潜邸,咱们俩便投缘。本宫的母家没什么别的,就是山参多些。至于银子,只要本宫喜欢,用在谁身上不是一样!''

素心见玉妍雪肤花颜,对着自己又这般体谅,心中越发感激,恨不得立时跪下磕头:``奴婢一直伺候着皇后娘娘,可心里也当小主是自己的主子,若能为小主尽心一日,也不枉小主这么厚待奴婢了。''

玉妍忙拉住了她,牵动绿云鬟上的金粟宝钿红纹钗颤起细细的翠玉叶滴珠,沥沥有声。她娇声道:``快别这么着。这些年你对皇后尽忠,也为本宫做了不少。玫嫔与怡嫔的孩子死于非命,若没有你得力查出是娴妃所害让她进了冷宫,皇后娘娘也不能高枕无忧啊!''

素心忙道:``奴婢能知道什么,要不是阿箬来投诚时小主暗中提点要从玫嫔和怡嫔的日常饮食所用上着手去留心,奴婢根本查不出来。只是这样天大的功劳,小主却一直隐瞒不说,也不许奴婢提起,只教皇上以为这些都是皇后娘娘和慧贵妃的功劳,真是委屈小主了。''她顿一顿,颇为埋怨,``前些日子皇后娘娘去看慧贵妃,贵妃还这般胡言乱语,要不是小主一个耳光下去,谁知道她又要胡说些什么呢。说来皇后娘娘也是,许多事都是小主和奴婢办下了,皇后多不知道,希望她日后能理解奴婢的忠心、小主的苦心便好。''

玉妍眼神一跳,摇曳如火焰,很快笑道:``本宫是李朝来的,能在宫中得些福泽,都是因为皇后娘娘的照拂,怎能不为皇后娘娘尽心。只有皇后娘娘稳居中宫,咱们才能安稳啊。切记切记,咱们做奴才嬖妾的,只须悄悄为娘娘打点,切不可露了聪明自招祸患。''玉妍说罢,伸手取下髻后一枚双鹊戏红莲金梳背,上头满满填着玫瑰金宝粟,红莲以红玛瑙琢成,缀以绿松为田田莲叶,青金宝石为波縠,镂金丝双鹊交颈仰首,一看便是名贵之物。她递到素心手中,拿衣袖一掩,笑道:``你的心本宫都知道,宫里人多眼杂,快别这么着了。''

素心热泪盈眶:``这些年若没小主,奴婢早不知到什么田地了。当年皇后娘娘原有心在奴婢与莲心中择一个嫁与王钦,幸好是小主体恤,为奴婢美言,说奴婢是满人,而莲心和王钦都是汉人,对食无妨,奴婢才逃过一劫。奴婢心里都记着。''

玉妍眉眼弯弯,笑语宽慰道:``好了。你这样,叫皇后宫里的人看到也不好,倒误了咱们一场情分。为着避嫌,本宫一向也比不得贵妃,总往你们宫里去,也不能当着皇后娘娘的面对你关照些。时候不早,你赶紧忙你的差事去吧。''

素心连连道谢,眼见着无人,赶紧去了。

这一日天朗风霁,皇后领着合宫嫔妃前往坤宁宫参拜。待到礼毕,逢着旁人不注意,如懿便见到了戍守在宫门外的凌云彻,她含笑道:``事已办妥,你总该放心了吧。虽然你所求的魏嬿婉还在花房当差,但只须往各宫送送花草,不必再辛苦莳弄花草了,这样你还满意吧?''

云彻喜得直搓手:``微臣谢过娴妃娘娘大恩。''

如懿仰起脸,看着碧蓝高远的天空,唇角含了浅浅的笑意:``若要言谢,本宫的性命数次都是你救的,此时只是还报你稍许而已。''

云彻诚挚道:``娘娘所说的一点点,对于嬿婉和微臣而言,已经是大恩了。''

如懿笑时嘴角微微一掀,仿佛是冷淡,却带着热切。她听出了几分意味:``看来那位姑娘已经回心转意了。你高兴得很啊。''

云彻有些不好意思,耳后根都红了一片,亦是感叹:``嬿婉说起来那件事,总是感慨自己的身世,说是身不由己。其实像微臣和嬿婉这种汉军旗出身,想要挣个好前程不让人瞧不起,也实在是难。微臣知道,有些事是难为她了,但是过去,便也过去了。''

如懿微微颔首,明澈眼眸中尽是了然的懂得:``其实说起出身,谁不是一样呢,都得靠着自己。凌云彻,本宫已经替你想过了,只要你愿意,再过几年,你有些出息,她也能攒下点资历,本宫就可以替你们俩指婚,成全你的心意。哪怕是汉军旗包衣奴才的出身,只要夫妻一心,同心向上,又有什么可愁的?''

云彻大喜过望:``娘娘说的可是真的么?''

如懿的唇如柳梢之上的新月,盈盈生辉:``只要你们心意如一,本宫言出必行。''

时光荏苒,海兰身体渐渐养好,只是身上纹路用尽方法也难淡去,不好再侍奉皇帝。因而虽生了皇子,宠眷却大不如前了。幸而永琪乖巧可爱,皇帝爱子,倒不算十分冷落海兰。如今宫中得宠的,也便是如懿、玉妍与意欢了。玉妍因着永珹讨皇帝喜欢,她的性子本就妩媚娇俏,雨露之恩便格外多。到了春来属国来朝之时,皇帝便又晋了她的位分,封了嘉妃。如此一来,竟与如懿和绿筠并列了。

众人虽然知道金玉妍恩深眷重,但三妃之中唯有如懿未曾生养。而晞月病重,如懿也是仅次于皇后而已。但皇后却对玉妍格外另眼相看,对她所生的永珹更是喜爱。玉妍生性最好脸面不过,得皇后这般抬举,如何有不趋奉的,便也常常逗留在长春宫中。

这一日细雨霏霏,因着入了春天气和暖,空气里倒是带着桃花饱蘸雨露后的缠绵而蓬勃的香气,好像整个肃穆沉沉的紫禁城,也被点染成了氤氲的粉色。

如懿刚带着乳母抱了永琪从延禧宫出来,想着海兰身上一直未能痊愈,心下愈是难过,幸好永琪长得壮健,海兰看见了也甚是高兴。

海兰虽然晋封了嫔位,但到底出身低些,孩子只能养在如懿名下,母子分离。于是如懿常常把永琪抱去了给她看,才稍作安慰。即便如此,无人时海兰依旧垂泪:``姐姐,生永琪的时候几乎要了我的性命,这几年怕也不能侍寝。即便侍寝,皇上一看见我身上这些斑纹,怕也嫌恶。幸好永琪养在姐姐膝下,我才能放心些。''

如懿无言可以安慰,只得道:``你也别伤心太过了,终究还有永琪呢。''

海兰虽然伤心,但缓和神色后便生了沉着之意:``我当然不会伤心太过,即便拼着以后再不能侍寝了,只要有姐姐和永琪,咱们总有法子站得更稳。''

宫中的日子悠长而寂寞,唯有海兰这般沉到谷底而不言败的勇气,才能一同并肩抵过岁月粗糙的磨砺。

如懿漫漫想着,回过神时已走到了长街,只见细雨飘零,天地间便如洒下一匹透明的洒银缎子一般,细细软软,无边无际。如懿正嘱咐两位乳母拿伞遮严了永琪防着被雨淋到,侧首却见前路的转角处,凌云彻正撑着一把油纸大伞,小心护着一个双手捧着黄牡丹的宫女。他们的神色都是小心翼翼的,可彼此眉眼间却都是深深的欢喜。仿佛这样走在雨下,便是人生极快乐的事情。凌云彻一心护着那宫女,自己的肩上全都湿了也未察觉,只细心叮嘱她:``仔细脚下,仔细滑。''那宫女回过头,朝着他极明媚地一笑,仿佛那一笑,连雨的湿凉也尽数可以熨去了。

如懿远远注目,不知怎的,心里便生了深深的艳慕。这样的风雨同路,彼此照拂,她从未见过,亦未经历过。即便她与皇帝有并肩行走的时候,也总是有乌泱泱的一堆人跟着,哪里能得这样自在欢喜。

倒让人想起《诗经》里的吟咏,男女相悦,真是这般彼此欢喜。

凝神的瞬间,她忽然想起一个人。

那个人,是活在很遥远很遥远的从前了。那时候,她还只是乌拉那拉皇后的侄女,未出阁的格格青樱,为着能成为皇后的养子,三阿哥弘时的福晋,皇后也曾安排他们见过一次。可是他,却偏偏不喜欢她。

也难怪,那时候的如懿,不过是娇养在深闺不知天高地厚的少女,如何学得会耐下自己的性子讨别人的喜欢呢。

只是,若那时,那时嫁了他,虽然只是平庸的一个青年男子,哪怕有妻妾争宠,但小小的王府之内,日子也会好过许多吧。

连那时的阿箬都偶尔会念叨一句,圣上不可捉摸,不比三爷仁厚。

这样的念头不过一转,她便郁然舒了口气,还有什么可想的呢。乌拉那拉皇后早已作古,连弘时,也早已被先帝革去黄带子,逐出宗室玉牒,病死在外了,更别提阿箬。世事如烟散去,唯有眼前可以把握,她还有什么可想的呢。

待凌云彻他们走近时,如懿已收回了漫天飞扬的神思,只笑吟吟注视着他们。二人忙行礼如仪:``坤宁宫侍卫凌云彻,向娴妃娘娘请安。''

那女子长得清婉灵秀,如一朵芝兰袅袅,映得四周被雨水打成暗红的朱墙,亦瞬间明亮了几分。她轻盈福身:``奴婢花房宫女魏嬿婉,向娴妃娘娘请安,娘娘万福,长宁安康。''

如懿听她婉声请安,那声音如枝头啼莺婉转,瞬时点亮了阴雨时节的晦暗。如懿见她弱态含娇,秋波自流,不觉道:``真的很美。凌云彻,你的眼光极好。''

嬿婉含羞带怯地低下脸去,一如粉荷露垂,杏花烟润,别有娟然风致:``娴妃娘娘赞许,奴婢卑微,不敢领受。''

惢心便笑:``难怪小主那么喜欢嬿婉姑娘,看嬿婉姑娘的眼睛和下巴,和小主长得真是像呢。''

嬿婉有些惶然,忙欠身道:``奴婢卑微,怎敢与娴妃娘娘相较。''

如懿只是笑:``惢心就是这般心直口快,你别理会就是了。''

嬿婉这才敢起身,她手里抱着花,难免有些沉重,抬腰便慢了些许。云彻忙伸手扶了她一把,嬿婉转脸一笑,甚是甜蜜。

如懿将这小儿女情态看在眼中,只作不见,随口问道:``这花像是姚黄,要送去哪里?''

嬿婉忙答道:``这是花房新培植出来的,正是洛阳名种姚黄。奴婢奉命,正要送去长春宫呢。''

如懿看着雨势渐大,有倾盆之象,便道:``皇后娘娘正位中宫,用姚黄装点,最合适不过。正好本宫也要带永琪阿哥去长春宫,你便随本宫同去吧。''

嬿婉清脆答应了一声,便跟在如懿身后一同去了。云彻悄悄在后头道:``外头还在下雨,等下我还是在这边等着你,送你回去。''

跟着如懿的小宫女菱枝见嬿婉走在最后,忙擎了伞跟过去替她遮雨,悄然笑道:``看凌侍卫这样细心,对你真好,你可真有福气。''

嬿婉抱着花,笑笑道:``再好也不过是个侍卫,这辈子也就这样了,还能如何呢。''

菱枝睁大了眼,诧异道:``他对你那么好,还不够么?''

嬿婉郁郁叹口气,笑道:``够是够了,像我这样的出身,还能挑剔些什么呢。这就已经是福气了。''

菱枝不无艳羡道:``可不是呢。易求无价宝,难得有情郎啊。若来日得我们小主的器重,前程远大也未可知啊。''

嬿婉回头看着立在长街口上的云彻,正痴痴地望着自己,点头道:``但愿如此吧。只求不要再是人下人便好了。''

\hypertarget{ux7b2cux516dux7ae0-ux6625ux6a31ux4e0a}{%
\chapter{第六章 春樱(上)}\label{ux7b2cux516dux7ae0-ux6625ux6a31ux4e0a}}

长春宫中布置清雅宜人,毫无奢丽之气,比之一应年轻嫔妃们的宫中更显简素。如此烟雨时节看去,蒙蒙晦暗之中,更不免有些寡淡。幸好皇后素喜时新花卉,廊下满满置了新开的花花草草,姹紫嫣红一片,倒添了不少明媚之色。

如懿扶着惢心的手进了仪门,回头嘱咐乳母:``小心抱着五阿哥,仔细台阶。''

玉妍正站在抄手游廊下赏雨,见了如懿便笑:``虽不是亲生的阿哥,娴妃倒也疼爱得紧呢。''

如懿见是玉妍,便与她行了平礼。玉妍眼睛只看着别处,纤纤十指拨弄着一盆玉版白的牡丹花,笑吟吟地受了如懿一礼。如懿素知她性子,也不愿计较,只是口中淡淡的:``是啊。嘉妃有自己的四阿哥,自然是更心疼了。''

一身艳瑰华衣的玉妍笑意款款,眉目濯濯,微启了红唇道:``自己的孩子么,虽然也心疼,但总得严格些,到底是皇子,太娇纵了不好。倒不比娴妃姐姐自己没生养过,一时疼爱得不知道该怎么去疼爱了,也是有的。''

语中的芒刺显而易见,如懿也不理会,只问立在帘外的莲心:``皇后娘娘呢?''

莲心笑吟吟道:``皇后娘娘正与公主说话呢。娴妃娘娘里头请。''她说罢,便掀了帘子请如懿进去。

皇后的殿中阔朗敞亮,因着皇后不喜奢华,殿内不过错落有致地置着几件金柚木家什,一色的湖蓝夹银纱帐用镶银钩挽起,清爽通透。皇后正与和敬公主说话,见如懿进来,便停了口笑道:``外头下着雨呢,怎么娴妃来了?''

如懿扬一扬脸,乳母们便抱着永琪行礼,口中道:``永琪给皇额娘请安。''

皇后忙和蔼道:``快抱稳了,小心跌着。''她就着乳母的手拨开襁褓看了看永琪,笑道:``永琪真是白胖可爱,看来娴妃养育得极好呢。''又道,``璟瑟,快看看你五弟。''

和敬瞟了一眼,冷冷淡淡道:``是很白胖可爱,但嫔妃养育的孩子就是嫔妃养育的,再怎么养着,都没有端慧太子那般清俊聪明。''

和敬所说的端慧太子,正是她一母同胞的兄弟二阿哥永琏。只可惜永琏早夭,难怪她看了哪个皇子都不喜欢。

皇后听了便有些不悦,沉下脸道:``璟瑟,你有些累了,让嬷嬷带下去吧。''

如懿看和敬下去,方含了谦和的笑色道:``臣妾自己没有生养过,永琪壮健,一来是在愉嫔腹中养得好,更有皇上和皇后娘娘的庇佑。''

皇后斜倚着身子,露出雪白一截手腕,凝脂般的皓雪之色映着一双鎏金凤口衔珠镯,有些暗沉沉的。``论起来也是愉嫔自己,怀着身孕的时候胃口好,生产的时候却吃了大苦头。万幸永琪一切顺遂,否则可要怎么好呢?对了娴妃,你可去看过愉嫔了,她可好些了?''

如懿正要应答,一眼瞥见玉妍走了进来,想起三宝说过给海兰催产的太医私下见过玉妍身边的贞淑,索性笑道:``好是好些了。只是太医说愉嫔生永琪的时候太伤了身体,得好好调养几年呢。不过,当时说让愉嫔催产无碍的是太医,现在出了事儿让好好调养的也是太医。这太医的嘴呀,说是长在自己身上的,可一开一合,谁都能让他说出点什么来。''

玉妍看了皇后一眼,脸上微微一沉,牵动鬓边一串红桃玉串珠流苏轻轻相击,玎玎作声。她轻笑道:``娴妃姐姐这么说,便是不信太医了。也是,我也听说了给愉嫔催产的事,可是这生孩子本就是鬼门关上走了一圈,催产的事哪有以保万全的。倒是可怜那几个太医了,不催产呢只怕愉嫔母子都保不住,催产了呢伤了愉嫔的身体还是要被赶出宫。其实也怪愉嫔自己,怀着身孕的时候管不住自己的嘴,生孩子的时候当然是会伤了自己的身体。''

如懿见玉妍对海兰这般评头论足,心中早就有气,面上的笑意却愈加温然:``说来也怪呢。愉嫔本不是贪嘴的人,怎么一有孕就这样顾前不顾后了。我听说嘉妃怀永珹的时候胃口可节制了呢,倒和愉嫔不一样。''

玉妍远山藏黛的眉得意地扬起,一双笑靥似喜非喜,掩口轻笑道:``这就是同人不同命哪!''

皇后略带嗔怪地看她一眼,语意柔缓得如同绵绵的雨丝:``生孩子的事本就是险事,太医和接生嬷嬷也只能在一旁相助罢了,终究是要靠为娘的自己。幸好愉嫔母子都能平安,其他也罢了。''她看着如懿皓腕三寸,便道,``今日倒是把本宫当年赏你的赤金莲花镯戴上了。本宫看你戴着,倒更想起慧贵妃,她病成这个样子,真是可怜。''

``这串翡翠珠缠丝赤金莲花镯是皇后娘娘赏赐的,前些日子不过是松了去绞一绞,臣妾喜欢得紧,怎么会不戴着呢。倒是皇后娘娘一味节俭,手上鎏金镯子有些暗了,也该去炸一炸才好颜色呢。''如懿面色沉静如水,一丝涟漪也无,只是略略做了惋惜的神态,``至于慧贵妃,如嘉妃所言,这都是命哪。''

三人正嘤嘤呖呖说着,只见莲心领了嬿婉进来道:``皇后娘娘,花房命人送了一盆牡丹花来。''

嬿婉放下了花便退到了一旁恭恭敬敬立着。皇后的眼风只落在牡丹缤纷的艳色之上,向二人赞许道:``是难得的姚黄呢。''

硕大的花盘慵慵如春睡的美人,重重叠叠的花瓣薄如轻盈绢绡,一瓣一瓣簇拥着,极尽瑰丽怒放之姿,花香浮漾,无声无息便濡染了裙裾摇曳。

玉妍见皇后喜欢,一径笑道:``臣妾只觉得颜色好看,却不知姚黄是什么?''

皇后端坐于檀木青凤牡丹椅上,徐徐道:``姚黄和魏紫是洛阳牡丹中最好的两品,素有`绝品万花王'之称。北地天寒,能在这个时节种出姚黄来,也算难得了。''

玉妍正端详着,忽然指着如懿的衣衫道:``哎哟,方才没仔细看,原来娴妃姐姐的袖口上绣着淡黄色的花朵,看着倒像是这姚黄牡丹呢。''

如懿唇角的弧线勾勒出不屑的轻笑,略瞥了一眼,这才发觉相像,便起身道:``臣妾这身衣裳是内务府昨日刚送来的,臣妾看着淡青的衣裳配松黄的花,颜色倒也别致,所以才穿上了,并未留意是不是姚黄牡丹的图案。''

玉妍眼角飞扬,浅笑的唇线带出两朵梨涡:``是么?我想娴妃也是无心的,只是无心也是无心之失啊,牡丹是皇后娘娘才配用的呢。不如娴妃告罪一声,回去把衣裳剪了再不穿,想来皇后娘娘是不会介意的。''

``皇后娘娘当然是不会介意的。因为花中之王后宫之主,本在人心而已。''如懿保持着无可挑剔的恭谨,屈膝道,``臣妾回去之后会脱下这件衣裳送到皇后娘娘宫中,一切但凭皇后娘娘处置。''

皇后微微漾起的笑容缥缈不定,只是深深地看了如懿一眼,转首看着身侧盛开的姚黄:``罢了,你跪安吧。''

如懿神色肃然,默默退下,只是眼中那一点倔强,始终不肯退去。

皇后眼见如懿出去,一张端然生华的面庞慢慢沉下来,仿佛积雨天气时暗垂的铅云,层层压下。片刻,皇后冷然道:``来人,把这盆花撤了,拿去火场烧了。''

听得皇后语气不善,嬿婉赶紧上前,垂着头捧了花蹑手蹑脚出去。

玉妍小心觑着皇后的神色,愤愤道:``这盆姚黄美是美,却送来得不合时宜,也太过耀眼。这样刺目的东西,喧宾夺主,不配养在皇后娘娘宫里。''

皇后扶着头,珐琅嵌玛瑙珠子的护甲横在微微皱起的秀丽眉峰上,才略略遮住她眉心的一丝戾气。皇后凝神片刻,衔着寒意道:``娴妃\ldots\ldots{}''

话音未落,只听殿门前``哐啷''一声,皇后一惊,即刻蹙眉抬头。

素心喝道:``大胆!在娘娘面前竟敢如此惊扰,活得不耐烦了么?''

嬿婉吓得俯首磕头不止,带了哭音惶恐道:``皇后娘娘恕罪,奴婢不是有心的。''

皇后凝眸一看,才知是方才捧着牡丹出去的宫婢,在出殿时被门槛绊了一脚,不留神砸了手中的花。

素心见皇后不悦,上去揪住嬿婉的领子,迫她抬起头来,劈面就是两个耳光:``皇后娘娘与嘉妃小主在此,你也敢这样放肆!当长春宫是什么地方?''

嬿婉嘤嘤哭着分辩:``姑姑恕罪,是奴婢不当心,惊扰了两位娘娘,错了规矩。奴婢再也不敢了,还请姑姑饶恕。''

玉妍轻嗤一声,闲闲抚着鬓角簪着的一朵丹红珠兰:``你那袖口晃着的那俩白的是手么?怎么连爪子也不如?一盆花都拿不稳,那手爪子砍了也不可惜。臣妾原就知道花房里伺候的宫女轻贱,原来还是笨手笨脚的蠢丫头。说起来,终究是规矩没立好,才由着那些轻狂婢子没上没下讨人嫌。''

素心立刻道:``嘉妃小主别生气,奴婢自会给奴才们立好规矩。''她略略扬声,``小顺子,把这个丫头拖下去,重重地掌嘴。看谁还敢在娘娘面前不精心伺候!''

殿外的小太监干脆地答应了一声,上前就来拖那宫婢。

皇后长长的睫毛如寒鸦的飞翅,在眼下染就两片晦暗的青色阴影:``慢着!素心,把她带到本宫跟前来。''

素心不明所以,手上却极快地拖了嬿婉到皇后身前。嬿婉吓得浑身发抖,皇后漫然道:``抬起头来。''

嬿婉惊魂未定,瑟缩着抬起头,腮边犹有两痕晶莹水珠。皇后凝视片刻,缓缓浮起两朵笑靥:``嘉妃,你仔细瞧瞧,她的眼睛和下巴像谁?''

玉妍仔细端详,瞬时浮出厌弃的表情,不屑道:``贱婢,长得就是一脸狐媚样子,合该活活打死才算完!''

嬿婉吓得连话也不敢说,只俯下身磕头不止。

皇后笑着欠身,用护甲轻轻托起她的脸。护甲尖闪着锐利的光泽拂过嬿婉姣好的面容,皇后柔声道:``这样美的一张面孔,要是打死了她也太可惜了!''

玉妍不屑地嗤道:``宫里有一张这样的脸就够烦人了,这婢子长得虽不是一模一样,但细看起来也有三四分像。娘娘要留了这个婢子在长春宫,岂不添烦?''

皇后温和地看着嬿婉:``你叫什么名字?家里是做什么的?''

嬿婉雪白的两颊上浮着通红的指印,眼底全是迷茫惶惑,连声音都颤颤地断断续续:``奴婢魏嬿婉,阿玛曾是正黄旗汉军旗包衣内管领清泰。''

皇后微微颔首:``倒还是好人家的女儿。家人都还在吗?''

嬿婉啜泣着摇头:``阿玛犯了事,已经不在了。''

玉妍不满地看着嬿婉:``再好的人家也不过是狐媚子奴才,连名字都那么妖里妖气,何况如今还是个破落户儿。''

皇后沉吟片刻,眸中闪过一抹亮色:``这名字是小家子了些,本宫给你改个名字。''她沉吟道,``青樱,青樱\ldots\ldots{}''

玉妍一双凤眼斜睨着,满是奚落之色:``跟娴妃一个狐媚样子,就叫樱儿吧,樱花的樱。''

皇后肤色玉华,此刻嫣然一笑,更增端美之态:``还是嘉妃聪慧知趣。素心,你带樱儿下去好好梳洗一番,然后送去嘉妃宫里伺候。''

嬿婉惊魂未定地抬起头来:``奴婢,奴婢\ldots\ldots{}''

皇后和声道:``好了,樱儿。不管你犯了什么错,本宫都把你赐给嘉妃了。''说罢便向玉妍道,``妹妹冰雪聪明,自然知道怎么把一个丫头调教好了。''

素心会意,抿着唇幸灾乐祸地笑:``你福气倒好,还不快谢皇后娘娘恩典。''

嬿婉心知不好,却也不得不毕恭毕敬磕了个头,跟着素心下去了。

玉妍见状,不免有些恼:``皇后娘娘何必对这个贱婢这么好,臣妾也不愿她在跟前,看了就生气\ldots\ldots{}''皇后转脸含笑看着她不语,玉妍恍然省悟,``樱儿樱儿,原来如此\ldots\ldots{}''她一脸喜色,``还是娘娘睿智,有这么个人在,娴妃又是个心高气傲的,不膈应死她!''

皇后微微含笑:``所以,本宫把樱儿赐给你,你可高兴?''

玉妍欢快地施了一礼,恍如一只几欲扑向花丛的蝶,眨了眨眼,那笑容几乎要滴出水来:``臣妾谢皇后娘娘恩典,必不辜负娘娘盛情。''

皇后意态舒然,含笑道:``慧贵妃轻浮急躁,胆子又小,更是个没福气没孩子的。你福气却比她好得多了。本宫喜欢你,喜欢永珹,你也要好好惜福才是。''

玉妍会心地点了点头,谦恭无比:``臣妾出身异族,能有今日,多赖娘娘关照。臣妾愿为娘娘尽心竭力,效犬马之劳。''

皇后含笑示意玉妍往身边的黄花梨琢青鸾座椅上坐了,切切道:``这些年你为本宫做的,本宫心里都有数。当日娴妃进了冷宫,本宫原想着她这一生没了指望,便留她一条性命,就当修一修慈悲。若不是你侍寝时发觉皇上身边放着那块青樱红荔的手帕,连本宫也以为皇上已经不理会她了。''

玉妍哪里沉得住气,气咻咻道:``皇后娘娘心善,潜邸时娴妃深得恩宠,宫里若论出身,也就她和娘娘是大族。她的姑母又是先帝的皇后,咱们不能不格外忌惮些。饶是这样,娴妃进了冷宫,皇后娘娘也不过在饮食上让她吃些苦头,终究没有怎样为难她。要不是因为娴妃在冷宫里还不安分,诅咒二阿哥,咱们也没必要让慧贵妃支使双喜去摆弄那些蛇儿。''

皇后居上座,身子倚在重重石青黄缎的锦茵垫中,背脊挺直,头颈微微后仰,似乎凝神许久:``双喜是慧贵妃的奴才,慧贵妃居然不知他这点本事,还不如你眼明心细,好好用了他这点长处。只是本宫一直也不知道,怡嫔有孕时险些被蛇惊动胎气,那蛇是从何而来?''

玉妍的目睫中有一瞬灼灼的光,唇边的愤愤之色却越发深沉了:``那可真是怡嫔可怜,臣妾听说此事后就说,一定是娴妃安排的,否则怎会那么凑巧是她救了怡嫔,得了皇上的喜欢。也幸好那日有皇后娘娘在,索性把怡嫔推去了娴妃宫里安胎。凭她再如何,总跟咱们无关就是了。''

皇后长叹一声,幽然凄恻:``不是本宫怕事避嫌。那时永琏本就病着,且怡嫔之前已然有玫嫔子嗣有异之事,怡嫔又是本宫房里出来的,若安胎无恙,那是本宫的本分所在,若有丝毫闪失,本宫便是自陷泥淖之中。与其如此,不如推给娴妃,一动不如一静罢了。''

玉妍以温顺驯服之姿徐徐欠身:``皇后娘娘思虑周详。臣妾就是眼里容不得沙子,看了娴妃这样的人就生气。''

皇后微微一笑:``人哪,都是命该如此。''她切切道,``好了。时辰不早,你也回去歇着吧。至于那个不懂事的丫头,由你调教着便是。''

\hypertarget{ux7b2cux4e03ux7ae0-ux6625ux6a31ux4e0b}{%
\chapter{第七章 春樱(下)}\label{ux7b2cux4e03ux7ae0-ux6625ux6a31ux4e0b}}

嬿婉随着宫人们回到启祥宫,正战战兢兢不知该如何是好,却见玉妍慢步进暖阁坐下,吩咐丽心道:``带樱儿换身衣裳再上来。''

丽心忙答应着去了。再回来时,嬿婉已经换了一身启祥宫中低等宫人的服色,梳着最寻常不过的发髻,连头上的绒花点缀也尽数除去,只拿红绳紧紧束着。嬿婉一脸不知所措,丽心拿出一副管事宫女的姿态,傲然喝道:``见了娘娘还不跪下?''

嬿婉吓得双膝一软,忙不迭跪下了道:``奴婢魏樱儿,给嘉妃娘娘请安。''

玉妍斜倚在榻上,滟湖色的软茸妃榻,越发衬得一袭玫瑰紫衣裙的她无比娇艳,仿佛一枝柔软的花蔓,旖旎生姿。玉妍拈了一枚樱桃吃了,轻蔑地笑:``你倒乖觉,这么快就喜欢自己的新名儿了。知道皇后娘娘为什么给你取名叫樱儿么?''

嬿婉怯怯摇头:``奴婢愚昧,奴婢不知。''

玉妍慵懒地直起身子,娇声道:``你呀!今天来送花不是错,送盆姚黄也不是错。偏偏最错的是你的脸,眼睛和下巴长得和娴妃那么像。啧啧啧,你说你,让不让人讨厌呀。''

嬿婉吓得眼都直了,连连叩首道:``奴婢该死,奴婢该死。''

玉妍扑哧一笑:``该死倒也未必,如果你肯挖了自己的眼睛,削了自己的下巴,说不准皇后娘娘心情一好,还是让你回花房当差去。既然你长得那么像她,她从前的名字叫青樱,你便叫樱儿,不是很合适?''

嬿婉直愣愣地跪着,吓得浑身发颤:``娘娘恕罪,娘娘恕罪。''

玉妍饶有趣味地将嬿婉的害怕尽收眼底,顺手在白玉花觚里取了枝红艳艳的芍药花,一瓣一瓣撕碎了把玩,花瓣碎碎扬扬撒了一地。``知道你舍不得你这张狐媚子的脸。也是,你要毁了容,本宫还怎么得趣儿呢。话说回来,你还是得谢谢本宫,要是落在了慧贵妃手里,慧贵妃恨娴妃恨成那样,不拿一炉子热香灰烫烂了你的脸才怪。''

玉妍扬了扬脸,丽心会意,拧住嬿婉的耳朵用力道:``从此你便是启祥宫的人了。这两个耳光是告诉你,好好伺候娘娘,有一点不周到的,便有你受的。''

玉妍娇美的面容上隐着犀利的冷,忽而轻嗅道:``今儿的香点得好,是苏合香吧?''

丽心忙笑道:``是啊。小主回宫前半个时辰便烧上了。''

玉妍葱绿玉白缎的攒珠绣鞋轻轻点地,眼里闪过一丝狡黠:``香倒是好闻,只是放得远了,气味淡淡的。樱儿,''她看着嬿婉,多了一抹促狭的玩味之意,``你把那小香炉捧到本宫身前来。''

嬿婉忙收了眼泪和畏惧,殷勤地捧了紫铜象鼎炉来,才捧到玉妍身边的案几上,便烫得赶紧放下,缩手在背后悄悄搓着。

玉妍不悦地摇头:``谁叫你放下了。放在案几上挡着本宫的视线。你就跪在这儿,拿你自己的手当香案,捧着那香炉伺候本宫吧。''

嬿婉想要分辩什么,抬头见玉妍的神色如这天色一般阴晦,只得忍下了几欲夺眶而出的泪,将香炉高高地顶在了头顶上。玉妍瞥了丽心一眼,娇慵地打了个哈欠:``本宫乏得很,进去眠一眠。记着,以后就让樱儿这么伺候。丽心,你也好好教导着她些。''说罢,玉妍便留了丽心在外看着嬿婉,自己扭着细细柳枝似的腰肢,入寝殿去了。

因着丽心在外,跟着进来伺候的是贞淑。贞淑原是玉妍从李朝跟着来的陪嫁,是最最心腹贴身之人。玉妍不喜自己的陪嫁如寻常宫女般劳碌操持,跌了身份,一向只让她在启祥宫中做些清闲功夫,掌着小库房的钥匙,管着皇帝所赐的贵重物事。此刻贞淑见玉妍只身一人,便默默伺候了她更衣躺下,方才低声问:``小主这么折磨一个小丫头片子,甚没意思。倒让人觉着小主事事都听皇后娘娘的,又沉不住性子。''

玉妍斜靠在软枕上,嗤地一笑,牵动耳边的银流苏玉叶耳坠滑落微凉的战栗:``牙尖嘴利,沉不住性子,又依附皇后?外头的人不是一贯这么看我的么?若是连你也这么看,倒也真是好事。''

贞淑蹙着眉头,不解道:``眼下皇后娘娘膝下无子,又疼咱们四阿哥,难道小主是为着四阿哥有个好前程,才这么打算的?''

玉妍的唇角扯起清冷的弧度,慵懒道:``皇后的永琏没了,难免心里着急,又忌讳纯妃的永璋年长,自然少不了要打我的永珹的主意,一时得个依傍也是好的。只是旁人不知道她,我还不知道么?她拼死也要生个自己的儿子的,眼下左不过是拿永珹留个后着儿罢了。我也只是顺顺她的性子。''她瞥一眼寝殿外,丽心的呵斥声隐隐传进,玉妍娇慵地舒展手臂,懒懒道,``否则我拿那丫头作筏子做什么?无非是皇后因娴妃而迁怒这丫头,又碍着脸面不能发作,借我的手罢了。我多折磨那丫头一分,皇后便以为我厌恶娴妃一分,也多依附她一分罢了。''

贞淑掩口笑道:``奴婢说呢,小主费这个心力做什么,原来还是为了皇后。说来这些日子,皇后娘娘可真笼络小主呢?''

玉妍微启红唇,冷笑声如冰珠落入玉盘,冷而脆地刺耳:``做小伏低了那么多年,她自然信我要比信旁人多些!只是非我族类,其心必异!她们这么看我,我何尝不是这么看她们的?宫里这些人,称呼着姐姐妹妹笑脸相迎,可心里有多污秽,只有她们自己知道。眼下紧紧抱着团儿,可不过就是有利则交,利尽则散,有什么真感情?你且看慧贵妃那草包美人儿,死心塌地依附了皇后这几年,现如今病成这样,皇后理会过没有?至于娴妃,从前不过是拿她当替死鬼,顺道又做了皇后的人情。''

贞淑极是不平:``当初小主是在娴妃和慧贵妃入潜邸的后几日嫁过去的。不过晚了几日,身份就比她们矮了一头。''她忽而得意一笑,``那时她们俩最得宠,慧贵妃又从格格被封为侧福晋,皇上眼里只有她们,哪里顾得上来看小主一眼,连还是福晋的皇后娘娘都被冷落了,咱们更是险些就没了立足之地。还好小主有主意,见安南国送来翡翠珠缠丝赤金莲花镯精巧,才想了偷天换日的主意,从此得了皇后娘娘的欢心。否则这些年步步惊心,哪里那么容易了。''

玉妍的容颜本就艳光四射,此时含了几分戾气,更有着诡异难言的阴柔之美:``如今看来娴妃更不是什么好相与的,越早防着她就越是了。左右在这个宫里,我就自己一个,谁也不信,谁也不靠!''

贞淑沉静道:``小主说得是。咱们熬了这么些年,如今大阿哥没有亲娘,二阿哥福薄走了,三阿哥不得皇上喜欢,怎么轮也该轮到咱们四阿哥了。且这宫里要论起宠眷不衰来,除了前几年的慧贵妃,便是小主了。''

玉妍爱惜地抚着自己的面孔,像是触摸着一件稀世珍宝:``天生了我这么美的一张面孔,可不是白白给浪费的。''她垂着眼睑,浓密的睫毛覆在她凝白如玉的面孔上,似山岚蒙蒙的影子,袅袅沉静。她的语气里含着温柔的怅惘,仿佛在诉说着一个甜蜜的梦境:``我若不是身为宗室之女,凭着这张脸,凭着我的出身,是一定会嫁与我们李朝的世子。世子虽没有皇上这样清俊的面孔,可是他笑起来是那么温柔,那么好看。''她闭着眼,如同沉浸在最美好的梦境中,如乳燕般呢喃,``从我十三岁入宫拜见王后娘娘,第一次见到世子的那一天,我就被他的笑容打动了。我从没见过那么温柔的笑容,他看着我的时候,好像满天的星星都对着我倾倒下来。那一天,我得到了比同行的贵族之女更多的赏赐,甚至在后来的日子里,总有来自宫中的礼物送到我的家中。连我的父亲都暗示我,世子对我很有好感,只要我努力修习女德,终有一日会进入宫廷,成为世子的嫔御。''

贞淑低叹道:``是啊。小主的祖母是王大妃的堂妹,又是出身高贵的金氏,虽然当时世子已经有了世子嫔,可小主入世子宫后成为宠妾,世子继位为王后封为正一品嫔,也是意料之中的。''

玉妍的眼角沁出一滴晶莹的水光:``可是人生的很多事,往往都在意料之外。在决定让我嫁往清朝为皇子妾侍的时候,连我自己也不能相信。我不愿意离开生养了我十数年的故土,不愿意离开我的父亲和母亲,却也不能违抗宫中的旨意,只能每日以泪洗面。直到两日后,我奉命进宫向王后辞行,才见到了世子。我很想问问他,为什么愿意让我嫁往遥远的异国,为什么曾经要那样对着我微笑,难道一切都只是我自作多情?可是在我看到世子的眼睛时,我什么都问不出来了。他的眼睛里满是泪水,他是那样难过。他对我诉说,李朝身为属国一切必须依赖上邦的弱小与痛苦,想要摆脱这种痛苦,就必须让上邦给我们更多。他说,我的美丽不能困在李朝窄小的宫殿里,而要绽放在异国的土地上,去取得属于我们自己的荣光。''她秀美的面孔上闪过一丝挣扎的痛楚,``我看着世子的眼睛,什么话都说不出来。我像着了魔一样,把他的每句话都牢牢地记在了心里,带到了这里。我活着的每一日,睁开眼睛前,都会想着世子说过的这些话。''

贞淑垂下头,难过地道:``小主这些年的辛苦,奴婢都看到了。''

玉妍晶莹美眸霍地瞬开,脸上的伤感如被烈日蒸发的雨水,转瞬找不到任何存在的痕迹。她伸手毫不犹豫地抹去腮边的一滴泪珠,冰冷道:``我背负着李朝的信任和期望,来到这里争取我和母族的荣光。我忍耐着做一个王府的格格,做一个宫里小小的贵人,一点一点讨着皇上的喜欢熬上来,不为了别的,只希望自己不要辜负了世子,不要辜负了我身上流着的李朝高贵的血液。有富察氏一日,我固然不敢奢求皇后尊位,可若我的孩子能成为大清的来日,那么我们李朝就能摆脱从属之国的卑微了。''

贞淑垂首,心悦诚服道:``小主的心志,奴婢都明白。奴婢一定会竭尽全力,忠于小主和李朝。''

从此,嬿婉的日子便没有再好过过。白日里要替启祥宫的宫女们浣洗衣服,一刻不能停歇。到了晚间,便要伺候玉妍洗脚。逢着玉妍不用侍寝的日子,还要跪在玉妍跟前,捧着蜡烛当人肉烛台,由着滚烫的烛油一滴滴烫在手上,烫伤了皮肉,也烫木了一颗心。

偏偏那一日绿筠来玉妍宫中闲话,瞥见嬿婉跪在地上当香案,便很有些看不上,道:``原来这丫头来了你宫里当差了。''嫔妃们之间闲话最多,一来二去,玉妍便知道了皇帝曾对嬿婉青眼有加。玉妍心胸狭窄,如何还会有好脸色给她,原本只是差事苦,吃穿倒也还好,渐渐地连启祥宫的小宫女都敢对她随意打骂,吃饭也只是剩饭剩菜,连想去见一见凌云彻诉苦,也不得半分空闲,不过是拿着一条命,在启祥宫中一日一日煎熬罢了。

自嬿婉进了长春宫,便再无人提起她的去处。凌云彻再三打听,奈何自己只是个在坤宁宫当差的小侍卫,平素不能离开,想要打听东西六宫的消息也使不上力,竟半分也得不到嬿婉的消息。

这一日恰好云彻跟着太监们去浣衣局取坤宁宫侍卫们的衣裳,才遥遥瞥见了嬿婉一眼,想要追上去询问,偏偏浣衣局里都是各宫来领取或浣洗衣裳的宫女,哪里能容许他走近。好不容易辗转打听了,才知道她如今在启祥宫当差。

这一得空,云彻便趁着送坤宁宫萨满法师出宫的机会,转到了启祥宫门外,果然就见到了嬿婉。宫禁森严,启祥宫外的守卫又格外多,他哪里能走到近前去。可是不必走近,他也能看到嬿婉消瘦憔悴的面庞和满是伤痕的双手。嬿婉跟着几个宫女行走,见了云彻,也不敢哭出声,更不敢多看一眼,只是默默流泪,撩起衣裳伸出手臂,露出全是挨了打受了伤的胳膊。正巧前头的宫女回头呼喝几声,伸手便在她肩膀上拧了一把。嬿婉吓得低眉顺眼,赶紧走了。

云彻眼见嬿婉受苦,如何受得了这个。思来想去,趁着十五之日皇后带着嫔妃们入坤宁宫敬香的时机,一咬牙便告诉了如懿身边的惢心。

如懿听得消息时正哄着五阿哥,不觉皱眉道:``你说启祥宫的人叫她什么?''

惢心道:``凌侍卫说,都叫她樱儿。''

``樱儿?''如懿面上浮起一层冷笑,``好端端的怎么就去启祥宫,还要受她们这般凌辱,那便是冲着我来了。既然是冲着我来的,想要袖手旁观也不能。你且让凌云彻安心等一等,金玉妍既然喜欢折磨樱儿,必定不会教她受太重的伤或是死了。等我找一个机会,看看能不能救她一救。''

所谓的机会,很快便等到了。那一日正是五月端午,宫中多以兰草汤沐浴,悬挂艾叶与菖蒲,吃粽子、白肉和咸鸭蛋,饮雄黄酒,佩戴五色丝线做成的五毒香囊,以求吉祥平安。

到了午后,嫔妃们便聚在皇后宫中,接受皇后亲手制作的五毒香囊。

皇后看着素心把香囊一个个交到嫔妃手中,含笑道:``这香囊里放有雄黄、艾叶和各色香药,能驱蚊虫、避邪气。你们自己一人一个,给孩子们也佩戴上,也算是本宫的一点心意。''

绿筠膝下子女最多,忙起身笑道:``每年端午皇后娘娘都亲手制作香囊赠予宫中嫔妃,臣妾们感念皇后娘娘恩德。''

皇后笑道:``纯妃客气。本宫对你们的心意一年也便端午一次,你们若喜欢,好好收着就是。''说罢便吩咐宫人上了五毒饼来。

所谓的``五毒饼'',即以五种毒虫花纹为饰的饼。其实就是在玫瑰饼上做上刻有蛤蟆、蝎子、蜘蛛、蜈蚣、蛇``五毒''形象的印子,盖在酥皮儿上罢了,也是吃个有趣。

玉妍见众人都在,便有心要让如懿没脸,扬声唤道:``樱儿!''

嬿婉怯怯上前,规规矩矩地守在玉妍身后,接过宫人们递来的五毒饼,利索地跪下膝行到玉妍跟前,高高举过盘子道:``恭请娘娘用五毒饼。''

蕊姬奇道:``这是什么规矩?咱们却不知道。''

玉妍含笑道:``玫嫔有所不知,这叫人肉跪盘。樱儿这丫头笨笨的,可有一样好处,什么都能受着。本宫要闻香的时候,她就是捧着香炉的香案;本宫要看书时,她便是举着蜡烛的烛台。还有形形色色的好处,下回一一给各位姐妹们瞧个新鲜。''

意欢冷着脸道:``嘉妃是李朝人,这怕是李朝才有的规矩吧。咱们这儿,可不这样折腾人的。''

玉妍不以为意,取了一块五毒饼吃了:``你瞧她捧得多稳当。奴才生来就是伺候人的,怎么伺候不是伺候呢。''她觑着如懿道,``娴妃,你说是不是?''

如懿的笑容宁和得恍若一面明镜澹澹,却是海兰道:``我记得这丫头从前在纯妃宫里伺候过大阿哥,如今怎么干起这个活儿来?宫里的宫女们好歹都是八旗出身,皇上一向最宽厚待下的,若是知道了,可不大好。''

玉妍扬了扬嘴角算是微笑:``愉嫔也真是小心太过了。宫女们伺候主子又怎么了,也值得说嘴?且樱儿又不在皇上跟前伺候,有什么要紧。''她盯着嬿婉道,``樱儿,本宫可没逼迫你,都是你自愿的吧。''

嬿婉哪里敢说个``不是'',忙道:``樱儿是奴婢,生来就是伺候主子的。''

玉妍指着她嗤笑道:``樱儿啊樱儿,你这张樱桃小口,答起话来倒利落啊。倒和咱们的娴妃平日里说话一个样子。细看起来,和娴妃也有几分相像呢。''

如懿听她直指自己,便也笑道:``就是为了这几分相像,嘉妃就那么喜欢樱儿伺候么?我记得樱儿本来是花房的宫女,叫作嬿婉,怎么到了妹妹身边,名儿也改了,伺候的活儿也改了?''

玉妍放下手中的五毒饼道:``娴妃姐姐这可是多心了。我不过是喜欢她的樱桃小口,所以才叫樱儿罢了。可不是因为姐姐曾经的闺名叫青樱啊。''

如懿淡漠地扬了扬唇角:``这个自然了。太后亲自为我赐名如懿,谁不知道呢。若拿这个来玩笑,可真真是小家子气了。只是方才嘉妃说那丫头长得有几分像我,我便跟妹妹讨个人情,让她跟了我去,如何?''

玉妍``哎呀呀''一迭声唤了起来道:``那怎么行呢!且不说我一时半刻还离不了这丫头,便是给了姐姐,皇上一跨进翊坤宫的宫门,看花了眼拉错了人,可怎么好呢,还是留在我身边稳妥些呢。''

皇后冷眼旁观,含了温和之色道:``不过是个小宫女,娴妃若喜欢,本宫让内务府再挑好的给你。''

如懿与海兰对视一眼,情知无可奈何,便也默然了。

待到从皇后宫中散去,如懿与海兰携了手出来,如懿眉头微蹙,脸上颇有些萧瑟之意,道:``看着金玉妍这般拿樱儿取笑凌辱,不知怎的,心里总有些不好受。''

海兰和婉劝道:``那丫头与姐姐有几分相似,也难怪了。可我还是劝姐姐一句,别想着去救她。一则姐姐开口,嘉妃愈加不肯放,还不如等她腻歪了,自己也觉得无趣,便撒手了;二来\ldots\ldots{}''海兰微微沉吟,``我亲眼见过这丫头在纯妃宫里是怎么在皇上面前抓乖卖俏的,实在不算一个安分守己的人。''

如懿颇为意外:``竟有这样的事?难怪她那时会突然要断了与凌云彻的青梅竹马之情,后来被打发去了花房,才知道要回心转意。原来竟有这样的缘故在里头。''她回头嘱咐惢心,``去告诉凌云彻,我眼下也没有办法。没有人不是熬着的,叫他也心疼心疼自己吧。''

\hypertarget{ux7b2cux516bux7ae0-ux6b7bux8a00ux4e0a}{%
\chapter{第八章 死言(上)}\label{ux7b2cux516bux7ae0-ux6b7bux8a00ux4e0a}}

时间过得极快,仿佛晨起梳妆描眉,黄昏挑灯夜读,枕着天黑,等着天亮,旧的时光便迅疾退去,只剩下的新的日子,新的面孔,唇红齿白的,娇嫩地鲜妍地过去了。乾隆八年,绿筠又生下了她的第二个儿子,皇六子永瑢。如此一来,绿筠便成了宫中生育皇子最多的嫔妃,即便皇帝一向对她的眷顾不过淡淡的,为着孩子的缘故,也热络了不少。连着太后也对绿筠格外另眼相看,对皇孙们也是关爱备至。

这一日皇后亦往绿筠宫中看望,钟粹宫的院落静静的,宫人们皆是垂手侍立,一声不敢言语。为首的太监见了皇后进来,忙道:``皇上来了,在里头陪着小主呢。''

皇后微微颔首:``本宫亦去瞧瞧,不必通传了。''宫女们打起帘子,皇后才踱进殿中,隔着挽起的珠绫帘子,正见乳娘抱着裹在锦绣堆中的初生婴儿,屈下身子坐在床边的小杌子上,小心翼翼地将怀中的孩子递给斜靠在床头的年轻母亲。绿筠尚在月中,丰腴的脸颊不施粉黛,却有着鲜润饱满的红晕。她漆黑的发丝松松地挽成一个家常的垂云髻,疏疏点缀着几枚累丝珍珠点翠花钿,就如它的主人一般婉顺依人。绿筠狭长细美的眼帘温柔地低垂着,唇边满是恬淡和美的微笑。皇帝正与她头并头,一同逗弄孩子可爱的面容,不时喁喁低语,间或,孩子响亮的哭声会断续响起。那是男婴特有的洪亮声音,虽然稚嫩,却有刚健的底蕴。

寝殿中的气息宁静而甜美,是真正一家人的天伦之乐。此时,无论谁走进去,都会显得那样突兀而局外。

皇后的手有些轻微的颤抖,像是深秋的黄叶即将被风带落前薄薄的挣扎。她默然转身,再度提示宫人无须通禀之后,疾步离开。皇后才走到门外,正见永璜进来。永璜见了她便规规矩矩行礼道:``皇额娘万福金安。''皇后亦无心理会,微微颔首便径自走了。

皇后回到长春宫便有些闷闷的,莲心以为她是要午睡了,忙铺好了被铺,点上了安息香便告退出去。皇后见素心仍旧依伴在侧,不觉郁然感伤:``瞧皇上陪纯妃那个样子,好像又回到了本宫刚生永琏的时候。那时候,真是好啊!''

素心忙道:``纯妃怎么能和娘娘比?娘娘生二阿哥的时候就是福晋,纯妃现在也不过是个妃子,还是汉军旗出身,拿她比娘娘,也不怕折了她的福!''

皇后的苦笑带着凄冷的意味:``有什么不能比的?纯妃如今有两个亲生的皇子,一个养子,而本宫膝下孤苦,只剩下一个公主。纯妃的福气,在后头呢。''

素心大是不满:``纯妃的福气还不是因为娘娘宽宏庇佑?说来,娘娘实在不该让她生下这些孩子的。像慧贵妃和娴妃,一笔子干净了多好。''

浓翳的阴郁积蓄在皇后眉间,久久不肯退散:``纯妃家世低,是汉军旗出身,又不大得宠,性格也温顺胆小。比不得娴妃身份高贵,慧贵妃备受恩宠,本宫一定得防着她们。''

素心连连称是,试探着道:``那嘉妃,皇后娘娘这么抬举她?''

皇后的眉头松了一松:``嘉妃是李朝贡女,并非满蒙出身,想要站稳脚跟,只能一心一意依附本宫。再说慧贵妃病着不得力,许多事若有她在,还能分娴妃的恩宠。她又是个心直口快的,没什么心机,还算得用。''她说罢,便有些乏。

素心服侍了她歪着,又替她盖好云丝锦被,道:``娘娘这些年都急于调理身子,想再生一个阿哥,可皇上不知怎么来得更少了,您这么着急也不是个法子。按奴婢看,大阿哥不是纯妃亲生的,又是长子,您大可把他收养在身边,有个依靠后再慢慢生一个自己的阿哥,也不错呀。''

皇后不悦的神色如遮蔽明月的乌云,阴阴翳翳:``本宫一看到永璜,就想起他早死的额娘哲妃当日是怎么赶在本宫前头得了皇上的恩宠,以致本宫嫁入潜邸时,皇上身边已经有了这么个挺着肚子的侍妾。且哲妃死得不明不白,外头多少言语都以为是本宫容不得她。永璜如今大了,万一听了这些闲言碎语,哪里会真正认本宫这个皇额娘,还是远着些好。''

素心半蹲在皇后身边,替她捶捏着手臂道:``皇后娘娘说得是。哲妃过世后,多少闲话都是冲着娘娘的。奴婢真替娘娘不值,明明没影儿的事,怎么都冲着咱们!''

皇后的眉心蹙成黛色的峰峦曲折:``宫里的事,都是疑心生暗云。咱们若有心分辩,不过是越描越黑罢了,便由着她们去。''她的手抚过枕边的三彩香鸭,撩拨着鸭口中袅袅泛起的乳白香烟,``这安息香真好,本宫闻着心里也舒坦多了。''她看一眼素心,``本宫知道你事事为本宫打算,只是本宫若真收养了永璜,他便从庶长子变成了嫡长子,生生尊贵了许多。来日本宫生下了皇子,有这么个嫡长子在,无论立嫡立长都多了一道阻碍,岂不自寻烦恼?''

素心点头道:``那也是。娘娘还是请太医来,好自调养着身体吧。许多事,娘娘其实不必费心,自然有人替您一一想得周到。''

皇后眸中噙着一丝清愁:``慧贵妃虽得宠,但并无多大用处,还好有她替本宫筹谋。这些也罢了,只是论起子嗣,本宫年过三十,会不会再也生不出孩子了?也怪太医无用,大补的汤药整天喝下去,皇上也算常来,却是一点动静也没有。''皇后正说着,忽然觉得鼻中一热,伸手一摸,却见手指上猩红两点,她心头大乱,失声道,``素心,本宫这是怎么了?''

素心急得什么似的:``娘娘,娘娘您流鼻血了。''她向外唤道,``太医,快传太医!''

齐鲁赶来把脉时,也是一味摇头:``娘娘您是太心急了。''

皇后倚在床上,六神不安地问道:``本宫的身体到底如何?''

齐鲁连连摇头:``娘娘凤体本无大碍,微臣已经给您开了催孕的坐胎药,您是否又私下进补大量温热的补品?''

素心忙忙道:``如今入冬,娘娘是心急些,服用了大量的阿胶、人参、冬虫夏草和鹿茸。这些都是大补的好东西,难道有什么不妥么?''

齐鲁叹道:``娘娘一心求子,微臣是知道的,所以开的坐胎药都是最合娘娘体质的,而非像当初给宫中嫔妃所喝的那种,只是普通的安胎药,不论体质的。可娘娘一时之间服下那么多补品,导致气血上扬,所以才会体热流鼻血。若是娘娘再不听微臣劝导,胡乱进补,伤了元气到吐血那一日,便再难补救了。''

皇后撑着身子起来,由着素心替她披上外衣,急道:``齐太医,你是太医院的院判,深得皇上和本宫信任,你告诉本宫一句实话,本宫年过三十,到底还能不能有孩子?''

齐鲁忙躬身道:``年龄不是最要紧的,且微臣一直为皇后娘娘以药物催调,总会有孩子的。只是娘娘素来体质虚弱,又忧思伤身,请娘娘一定要安心,再好好调理一段日子。''

素心亦是苦劝:``娘娘放宽心即是。皇上也和您一样盼着嫡子呢,所以这两年总是来咱们长春宫,有皇上这样的恩眷,何愁没有身孕呢?''

皇后听得颔首,不由得万分郑重地嘱咐:``那一切便托付给齐太医你了。''她闭目片刻,似是十分关切,``那么慧贵妃,近来如何了?''

齐鲁低声道:``老样子,整日昏昏沉沉,偶尔还说几句胡话。左右贵妃的身体,是再不能好了。如今到了冬日里,贵妃那样的体质,皇上不去看望已经伤了心,若少些炭火供应,便又是一重折磨了。''

皇后微微凝眸,睇她一眼,婉然道:``素心,你都记得了?''

素心满面恭谨,道:``娘娘放心,奴婢都会安排好的。''

这一厢皇后急着有身孕,如懿亦是感慨不已,虽然皇后赏赐的莲花镯里,翡翠珠里面的零陵香全被剔干净了,她不过戴个镯子装点样子,可终究是悬心。然而她看着皇帝年过三十,一心一意只求嫡子,便也不好说什么,只由着他一日日往长春宫去。

这一日赵九宵轮休,得了空闲便与凌云彻在侍卫的庑房里喝酒。九宵与云彻最是要好,云彻去坤宁宫领了份闲差,他虽然羡慕,倒也常常来往,和从前一样,喝酒闲话。这日午后他拎着酒和小菜过来,见凌云彻愁眉苦脸的,便捶了他一拳道:``坤宁宫这份差事又清闲钱粮又足,你还整天挂着个脸做什么,还惦念着你的小青梅哪?''

云彻给自己倒了一杯,愁眉紧锁:``自从嬿婉进了启祥宫,我要见她一面也难了。一个月前偶然碰上一次,她一个人抱了那么一大桶衣服去浣衣局洗涮。我才问了一句她就哭,说要赶着去洗完,否则晚饭又没得吃。浣衣局有的是人,她是宫女,为什么要这样为难她?''

赵九宵喝了口酒,摇头道:``宫女也好侍卫也好,哪怕伺候再得宠的主子,也就是个奴才的命。你还想怎么样?嘉妃能好吃好喝供着她?留着条命在就不错了。''

云彻难过道:``宫女也是人,不是畜生。嬿婉不敢和我多说话,就说常常吃不饱穿不暖,连一起伺候的宫女都欺负她,什么粗活儿累活儿都给她干!说不上两句话就只是哭,我看着真是\ldots\ldots{}''

九宵听着可怜:``你看着真是心疼!那你怎么不去求求娴妃娘娘?好歹她在冷宫的时候,咱们也帮衬过她。''

云彻想了想,还是摇头:``上回为了让娴妃娘娘搭嬿婉一把,还害得娴妃娘娘被嘉妃排揎了一场,无端受辱。我哪里还有脸请她帮忙!且娴妃娘娘不比嘉妃有儿子,到底两样些。''

九宵愣了愣:``连娴妃娘娘都没办法,你还能怎么样?我劝你,断了这个心思吧。反正嬿婉也对你起过二心,你实在帮不上,也就算了。''

凌云彻摇头,决然道:``她既然已经回来,我便答应过她,会一生一世照顾她。虽然启祥宫里的日子艰难,我已经托人告诉她,要她一定要熬得住,我一定会想办法的。''

赵九宵看他如此坚决,便举杯道:``那我便祝你心愿得偿吧。只是你小心,别老吃亏在女人手里。''

到了乾隆九年末的时候,宫里又发生了一桩大事,便是卧病许久的晞月病入膏肓了。年复一年的病痛折磨,曾经宠冠六宫的高晞月,已经熬到了油尽灯枯的时候,仿佛一盏点在风中的小小油灯,竭力燃烧着最后的焰火,不知什么时候,就会被风吹去,丝毫不剩。

太医数次禀告之后,皇帝终于道:``既然病得那么厉害,皇后是六宫之主,让皇后去瞧瞧吧。''

而皇后耳聪目明,更兼悉心调理,便推了身体不豫,不肯出门。如懿得知,亦只是含笑向皇帝道:``这么些年不见她了,皇后不肯去,臣妾去见见也好。''

皇帝郁郁不乐,只摩挲着一枚外头新贡的粉色珊瑚扳指。那珊瑚是浓淡相宜的粉色,如婴儿绯红的面孔,极是喜人,因号``婴儿面''。皇帝随手撂给李玉:``这个赏给纯妃正相宜,去吧。''

李玉会意,便领人退下,皇帝方才淡淡道:``她与你不睦已久,你何必巴巴儿赶去。''

如懿剥着水葱似的指甲,漫漫道:``听说这一向咸福宫里不大干净,又有宫女发了疥疮打发出去了,也不知贵妃怎样?她是病透了的人,若再沾上一点半点,皇上也不好对高大人说起。''

皇帝不置可否:``宫里许久无人去看她了,只怕她也不大愿意见你。''

因是去探病,如懿打扮得亦简素,不过是一袭曳地月华裙,不缀珠绣,只有淡淡的珍珠光泽流动,外面罩着紫色旋纹氅衣,衣襟四周刺绣锦纹也是略深一些的暗紫色,再搭一件淡若银白的烟霞色蝴蝶狐毛坎肩,头上松挽宝髻,梳成有流云横空之势,缀几点翠玉莹莹并一枚羊脂白玉凤簪。

如懿缓缓步入咸福宫中,里头一切供应依旧,只是帘子打开的一瞬,并无惯常咸福宫中冬日那种温暖如阳春的暖意扑来。仔细看去,宫中虽然照例供着十几个火盆,但炭都烧尽了,也无人去换,连地龙的热气也不甚足。

如懿身上有些发冷,紧了紧衣裳,暗想,晞月素来的体质最畏寒不过,殿中这样清寒,对于病重孱弱的她,无异于催命一般。

寝殿内,珠帘重重之后还是清约典雅中略带华丽的气息,卧在被褥之中的晞月依旧是养尊处优的唯一的贵妃。可是,却总少了那么点人气,便是这宫里人人赖以生存的皇帝的宠遇。

这些年晞月卧病,皇帝虽然每每派人安慰赏赐,却再未踏足过咸福宫。

如此华艳,却也寂寞如斯啊。

伺候的宫人们见了如懿,忙恭恭敬敬地请安问好,如懿与高晞月相争十数年,两宫中人一向不睦,见了她这般敬畏,倒真是难得之事。看来这些年,咸福宫所受的冷遇苦楚,还真是不少。

如懿一眼望去,便问:``怎么伺候贵妃的人这么少?''

门外伺候的小太监忙赔笑道:``娴妃小主有所不知,宫里有两个宫女发了疹子,也不知是在哪里得的。贵妃小主身子虚弱,怕染上这些脏东西,才叫人领出去了,连着底下同住的人怕不干净,茉心姑姑都吩咐暂时打发出去了。''

说话间,茉心已然迎了上来。如懿道:``你家小主醒着么?''

茉心久不见人来探望,亲自搬了椅子来道:``醒着呢,小主先坐,奴婢着人上茶。''

茶水递上来,便知是旧年的陈茶了,如懿不愿再喝,便道:``殿里这么冷,贵妃的身子怕受不了吧?''

一句话招得茉心眼泪都下来了:``太医总说炭气会熏着小主,不利玉体安康。内务府什么东西都照应着,唯独小主怕冷这一点,怎么也不肯顾及。''

茉心话未说完,背身朝里的晞月挣扎着撑起身体来,凄笑道:``闹了半天,居然是你来看我。''

茉心忙替晞月在身后垫了鹅羽垫子,又给她披上了厚厚的外裳:``小主慢些起身,仔细头晕。''

如懿见晞月双目深凹,憔悴枯槁,瘦得竟脱了形,简直如冬日里的一脉枯竹,轻轻一触就会被碰断。晞月喘着气,整个人嵌在重重帘帏中,单薄得就如一抹影子,仿佛连那披在肩上的外裳都承受不住似的。如懿在她床边坐下,问道:``可觉得好些了?''

晞月僵着面孔,分毫不肯假以辞色:``既然你都来了,自然知道我是好不了了。''她凄然道,``我都到了这个样子,只求见皇上一面,皇上也不肯么?''

如懿笑了一笑:``皇上国事繁忙。''

晞月怅然垂首,似是灰心到了极处:``这种话,你哄哄旁人也就罢了,对我说这个有什么意思。皇上若是忙,怎么还有时间宠爱嘉妃和舒嫔,还和纯妃又有了一个孩子呢?只不过是不愿见我,所以推诿罢了。''

如懿望着她,淡然含笑:``你多年卧病不出宫门,倒是活得越来越通透了。''

晞月仿佛想要笑,可她的脸微微抽搐着,半天也挤不出一个笑容来:``人之将死,还有什么看不穿的。我自知出身汉军旗,比不得你和皇后出身显贵。所以身为侧福晋,享着皇上的恩宠,心里总觉虚得慌。哪怕皇上抬旗封了贵妃,到底也是不一样的。我明白自己的身份,也没有儿女可以依靠,所以一心一意追随皇后,鞍前马后,从不敢有二心。皇后娘娘对我那样笼络,如今也是弃若敝屣,转头去捧着嘉妃了。''她忽而一笑,``当年皇后与我做了那么多事来对付你,要是带去了黄泉也便带去了,你想不想听一听?''

如懿温婉地抿着唇,凝视她片刻:``不想。你若想说,就自己去说给最该知道的人听。对于我,这些都是无用了。''

晞月捂着胸口连连咳嗽,半天才平息下来,疑道:``你不想知道这些?那你巴巴儿地跑来看我做什么?''

如懿轻轻靠近她,语不传六耳:``我告诉你的,自然比你想告诉我的更要紧。''

晞月眼中的疑影越来越重,挥手示意宫人退下:``你有什么话,便直说吧。''

如懿见她枯瘦的手腕上,那一串翡翠珠缠丝赤金莲花镯静静蜿蜒其上。那样翠色生生,如碧水清明,越发显得她手腕枯黄一脉,唯见青色的筋络高高突起。如懿伸出手去,指尖落在晞月干枯的皮肤上,慢慢游弋上她枯瘦的手腕。晞月狐疑而不安地看着她,却不知她想要做什么,眼见得手臂上的皮肤一粒粒起了惊恐的粒子,却也不敢缩回手来,只是颤颤地问:``你到底要做什么?''

如懿笑意轻绽,有怜惜之意:``这么好的肌肤,从前谁看了都想摸一摸,也难怪你得宠这么多年。只是如今,竟也有这一日了。''她说着,便欲摘下晞月手腕上的莲花镯,晞月一惊,忙护住了不解道:``你要做什么?''

如懿也不理会,径自摘下了在手中晃了一晃:``人都这样了,还吝惜一串镯子做什么?''她伸手取过妆台上的小剪子,霍然剪断,取下其中一颗翡翠珠子,猛然往地上一掼。珠玉碎裂处,掉出一颗小指甲盖大小的黑色珠子。如懿用手帕托起,送到晞月鼻端,问道:``香不香?''

晞月看得惊疑不定,直直地盯着那颗黑色珠子道:``这是什么?''

``我和你追随皇上多年,一直未有身孕,都是靠了这样的好东西。''如懿神色微冷若秋霜清寒,``这样好的东西,除了皇后,咱们竟都不识。这可是上好的零陵香啊!产自西南,能让人伤了气血,断了女子生育的零陵香!''

晞月大惊之下气喘连连,她厌恶地推开那样东西,又恨又疑:``你既知道,怎么还一样戴着?''

如懿取下自己的手镯,对着光线道:``我比你的运气稍稍好一点,有次不慎摔碎了翡翠珠子,掉出其中的脏东西来才发现关窍。如今我戴着的手镯,翡翠珠子里头的零陵香丸都是剔干净的了。''她神色凄微,``只是这么久以来我还是没有孩子,安知不是早已被这东西伤尽了根本,已经再不能生育子息了。''

晞月大恸,掩着唇抑制住近乎声嘶的哭声:``为什么?为什么要这样待我?我对她忠心了这么多年,什么事都听她的,什么都想在她前头做了,为什么她要断了我最想要的孩子?''

如懿眼中微有泪光闪烁,冷冷道:``她是皇后,生杀予夺都在她手中。而你,不过是值得被她利用却不能生育的工具而已。当年她把这对镯子分别赐给咱们两人时,这样的念头便已长好了。难为咱们一碗一碗坐胎药喝下去,总怨药石无效,何曾想过,原来早已是不能生了!''

晞月紧紧地攥着胸口稀皱的锦衫,厉声道:``好好好!你既然让我死得明白,我也断然不会辜负你!咱们俩争了半辈子,争恩宠,争名位,不是咱们想争,而是任何人到了这个位子都会争。但到了今日,咱们之间的恩怨慢慢再算!''她的眼里露出狠戾的光芒,如嗜血的母兽,``这辈子我最盼着一个自己的孩子,谁要断了我的念头,便是我不共戴天的仇人!''她仰天长笑,掩去腮边泪痕,沉静不发一言。

如懿轻叹一声,复又微笑:``玉镯的手脚就当是皇后做的。那么你再猜一猜,为什么齐鲁替你治了这么久的病,你的身子却越来越坏?据我所知,你的体质是气虚血淤,可是我让人查过齐鲁开给你的药方,按着那个方子服药,表面看着症状会有所减缓,其实会让你元气大伤。''

晞月死死攥住被角道:``不会!那张方子是太医院所有太医都看过的!''

如懿轻笑道:``那么,是谁能嘱咐齐鲁为你越治越坏,而且太医院上下都为你诊过脉,却是同一条舌头说同一句话呢?我想,那个人一定也不知道皇后也防着你会生下孩子吧。否则,便不必费这样的功夫了。''

晞月瞪大了双眼,目光几能噬人,死死盯着如懿:``你是说\ldots\ldots 你是说?''她凄厉地喊起来,``我要见皇上!我要见皇上!''

如懿安抚地将手放在她的手背上,笑容温柔无比:``我会如你所愿。''

\hypertarget{ux7b2cux4e5dux7ae0-ux6b7bux8a00ux4e0b}{%
\chapter{第九章 死言(下)}\label{ux7b2cux4e5dux7ae0-ux6b7bux8a00ux4e0b}}

如懿回到宫中,便见皇帝坐在窗下,一盏清茶,一卷书帖,一本奏折,候着她回来。她解下披风,坐到皇帝跟前道:``让皇上久等了。''

皇帝淡淡道:``去看慧贵妃而已,怎么去了这么久?''

窗外微明的光线为如懿如花树堆雪般的面容镀上了更为温婉的轮廓,她徐徐替皇帝添上茶,缓声道:``原是想略坐坐就回来的,但是看着咸福宫炭火供应不足,贵妃又病得可怜,所以多说了两句。''

皇帝蹙眉,不以为然道:``何必与她多费口舌?''

如懿露出几分怜悯之意:``贵妃也没有别的什么话好说,昏昏沉沉的,只反反复复惦记着要见皇上一面。''

皇帝眉心拧得越发紧,凝视着茶盏中幽幽热气,冷淡道:``朕不去。''他顿一顿,``你来劝朕,高斌也上书进言,牵挂贵妃,言多年来朕对贵妃的眷顾。唉\ldots\ldots{}''

皇帝的叹息幽幽地钻进心底去,她明白他的不忍、他的为难:``皇上不肯去,是因为人事已变,面目全非么?''

皇帝斜倚窗下,仰面闭目:``如懿,朕一直记得,贵妃在朕面前,是多么温柔腼腆。朕真的不想看见,那么多人让朕看见的、她背着朕的模样。''

如懿深深攒起的眉心有自然的悲怆:``皇上不去,自是因为心疼臣妾,也心疼从前的贵妃。臣妾虽然也恨她,可见她病得只剩下一口气的样子,也真是可怜。臣妾想,这些年皇上到底还顾着慧贵妃在外头的颜面,对她还是眷顾,也是安慰她母族高佳氏。如今她只想再见皇上一次,皇上成全了她,也当是成全了高氏一族吧。''

皇帝的眼底渐渐有纷碎的柔情慢慢积蓄,沉吟良久,他终究长叹:``晞月,她伺候朕也有十多年了。罢了,朕便去瞧瞧她吧。''

皇帝去时,晞月已换上最得宠的年月时心爱的樱桃红洒金蝴蝶牡丹纹氅衣,戴着一色的鎏金翠羽首饰并金镶玉明珠蝶翅步摇。她正襟端坐,脸上以浓厚的脂粉极力掩盖着病色,守候在窗下,引颈企盼皇帝的到来。

皇帝步入寝殿时,她竟先听见了,由侍女们搀扶着,吃力地请下安去,仰起脸对着皇帝露出一个极明媚的笑容。她原是病透了的人,只剩下了一副虚架子,皮肉都松松地垂着,这一笑更显得胭脂虚浮在脸上,如套了一张面具一般。皇帝看着她这样的笑意,想起多年来她娇艳绝伦宠冠六宫的日子,亦有些心酸,便虚扶了她一把:``你既病着,便别劳碌了。''

这话原是寻常,可落在晞月耳中,却是深深刺痛了心肺。她不自觉便落下泪来:``皇上厌弃臣妾至此,多年不肯来见臣妾一次,臣妾原以为自己要抱憾终生而死了。''晞月一落泪,脸上的脂粉便淡了一层,她很快意识到这样流泪会冲刷去脸上的脂粉,匆匆拭去泪痕道,``臣妾深悔当年过失,本不该厚颜求见皇上。但臣妾自知命不久矣,许多话还来不及对皇上说,所以无论如何也要见一见皇上。''

皇帝叹息:``你都病成这个样子了,朕来瞧瞧你也是应该的。你何必还这样费力打扮,穿着这么单薄的衣裳,仔细冻坏了身子。''他嘱咐,``还不赶紧扶贵妃去床上躺着。''

晞月如何肯躺着,挣扎着跪下道:``皇上。臣妾自知是不能了,这件衣裳,是皇上当年赏赐给臣妾的,臣妾很想穿着它再和皇上说说话。''她吃力道,``茉心,你带着人出去,这里有本宫伺候皇上就是了。''

茉心含着眼泪,依依不舍地带着众人退下,紧紧掩上了殿门。晞月跪在皇帝身前,指着桌上的茶点道:``这茶是皇上喜欢的龙井,点心是皇上喜爱的玫瑰酥。皇上都尝一尝,就当是臣妾尽了伺候皇上的心意了。''

皇帝略略尝了尝,容色慢慢淡下来道:``你一定要见朕,有什么话不妨直说吧,也免得自己劳累。''

晞月点点头,从供着茶点的小桌底下的屉子里取出用手绢包着的一样物事,摊开道:``皇上,您还记得这串翡翠珠缠丝赤金莲花镯么?''

皇帝颔首道:``这是你和如懿嫁入潜邸不久,皇后赐给你们俩的,一人一串。朕记得。只是,怎么碎了?''

``是啊,这么珍贵的东西,皇后娘娘自己不用,赏赐给了臣妾和娴妃,臣妾真是感恩戴德。这些年,皇后娘娘对臣妾眷顾有加,臣妾也真心敬畏。真是想不到啊,娘娘在这里头藏了这样好的东西。''晞月从碎玉片里拣出一枚黑色丸药状的珠子,惨然道,``这翡翠珠子里面塞了有破孕、堕胎之效的零陵香,长久佩戴闻嗅,有娠者可断胎气,无娠者久难成孕。臣妾与娴妃一戴就是十数年,连自己怎么没有孩子的都不知道。当真是个糊涂人啊!''

皇帝只瞥了一眼,冷冷道:``朕不相信皇后会做这样的事。''

晞月戚然道:``皇上不信,臣妾也不愿相信。可事实在眼前,东西是皇后亲自赏赐,臣妾也不能不信。''

皇帝的脸瞬时冻住如冷峻冰峰,眉心有幽蓝怒火隐隐窜起:``难怪娴妃与你多年未孕,朕只当时机未到,原来如此!''

晞月缓缓、缓缓笑道:``是啊。臣妾自知荣华富贵来之不易,所以一心侍奉皇上,依附皇后。原以为这样的事一辈子都不会落到臣妾身上,却做梦也想不到,竟被人这样算计了大半生!臣妾自知出身不如娴妃,承蒙皇上厚爱后,一颗心糊涂了,自以为可以凌驾于众人之上,才事事与娴妃不睦。''

皇帝并不看她,别过脸道:``你说的这些,朕都知道。''

晞月雪白的牙齿咬在涂抹得鲜红的唇上,眼中闪过一丝戾色:``这些是皇上知道的,皇上不知道的还多着呢。臣妾自知不保,病中这些年,一直被皇后反复提点不许多言,以保高氏家族。臣妾知道,皇后出身富察氏,她阿玛是察哈尔总管,伯父马齐是三朝重臣。臣妾虽然蒙皇上抬举,但毕竟不如皇后,所以处处以皇后唯命是从,但求保全自身,保全母族荣耀。''

皇帝看着她,眼眸如封镜,不带任何悸动之色:``朕明白你的意思。前朝是前朝,后宫是后宫,朕不会因为你说了什么做了什么牵连你的母族。哪怕有一日你不在了,你的父亲高斌还会是朕的股肱之臣。''

晞月紧绷的面容渐渐有些松动,她大概是累极了,吃力地跪坐在自己的腿上,用手支撑着道:``臣妾所作所为,罪孽深重。所以到了今日,并不敢祈求皇上原谅,有皇上这句话,便是大恩大德了。''她磕了个头,缓缓道,``若有来生,臣妾再不愿被爱恨执着,也不愿再被旁人指使挑唆了。臣妾要从大阿哥生母哲妃之死说起。''

皇帝听得``哲妃''二字,眼中闪过一丝精寒,只是隐忍不发,淡淡道:``你说吧。''

晞月含了一缕快意:``哲妃的死从来不是意外,而是有人嫉妒她比自己先生下了阿哥,又得皇上宠爱。哲妃喜好美食,却不知有些食物本都无毒,但放在一起却是相克,毒性多年累积,哲妃终于一朝暴毙。''

皇帝冷冷扫视着她:``你怎这般清楚?怎么皇后事事都对你说么?''

晞月恨恨道:``皇后娘娘自然不会对臣妾说这个,更不会认。然而哲妃暴毙时皇上正按先帝旨意出巡在外,根本赶不及回来见哲妃最后一面。臣妾也是一时疑心,才让父亲查出此事。皇上且想,这件事谁得益最多,自然是谁做的!当时潜邸之中与哲妃最面合心不合的,唯有皇后而已。长子非嫡子,一直是皇后最尴尬处。臣妾想不出,除了皇后还会有谁要哲妃死呢!这一点皇上您不也疑心么?否则您一直对皇后还算不错,怎的哲妃死后便渐渐疏远了她?''她笑得凄厉,``哲妃死后,皇后也察觉您的疏远,她最怕不知您心意,终日惴惴,所以买通皇上您身边的太监王钦窥探消息,又把莲心嫁给王钦加以笼络。至于阿箬,也是皇后安抚许诺,才要她为我们做事。娴妃入冷宫之后,皇后犹不死心,在娴妃饮食中加入寒凉之物,使得娴妃风湿严重。现在想来,只怕为的就是在重阳节冷宫失火时娴妃逃脱不便,想烧死娴妃。至于娴妃砒霜中毒之事、蛇祸之事,臣妾虽然不知,但多半也是皇后所为了。''她仰起面,``皇上,臣妾所知,大致如此。若还有其他嫔妃皇嗣受害之事,臣妾虽未亲眼所见亲耳所闻,但多半与皇后脱不了干系。所以上天报应,皇后也保不住端慧太子的性命!''

晞月说到最后一句时,语气已是极为凄厉可怖,几近疯魔。皇帝脸色铁青:``你倒是说得清楚细致,可是朕却不信。皇后出身门庭显赫,怎会懂这些下作手段?''

晞月怔了一怔,仿佛也不曾想到这一层。然而转瞬,她便笑得不可遏止:``皇上,一个人想要作恶,有什么手段是学不来懂不得的!''

太阳穴上青筋突突跳起,皇帝的鼻息越来越重,神色间却分明是有些信了,他的手紧紧抓着紫檀木的桌角,镇声道:``你虽然病得快死了,但若有半句虚言,朕还是会让你生不如死。你要明白,皇后是中宫之主,污蔑皇后是什么罪名!''

``臣妾知道。皇后在您心中是一位最合适不过的皇后,她克勤克俭,整肃六宫。她高贵雍容,不争宠夺利。她有高贵的家世,也曾为您生育嫡子。所以哪怕您知道她的不是,也会给自己许多不去追问的理由。因为您害怕,怕她就是让你失望的那个人。''晞月连连冷笑,虚弱地伏在地上,喘息着道,``人之将死,其言也善。臣妾带着这一身的罪孽下到地狱去,还有什么不敢说的。只是皇上细想想,这些事除了皇后得益,还有旁人么?若不是她做的,臣妾想不出还会有谁!今日臣妾全说了出来,也省得走拔舌地狱这一遭,少受一重苦楚了!''

皇帝眸色阴沉,语气寒冷如冰,让人不寒而栗,缓缓吐出两字:``毒妇!''

晞月大口地喘息着,像一口破旧的风箱,呼啦呼啦地抖索。她朗声笑道:``皇上说得对。臣妾自然是毒妇,皇后更是毒妇中的毒妇。可是皇上,您娶了我们两个毒妇,您又何曾好到哪儿去了。皇上与皇后,自然都是天造地设的一对,再般配也没有了。您说是不是?''

皇帝听她出语怨毒,却也不以为意。良久,他脸上的暴怒渐渐消失殆尽,像是沉进了深海的巨石,不见踪影。他只瞟了她一眼,神色冷漠至极:``你的话都吐干净了么?还想说什么?''

晞月见他不怒不愦,一脸漠然,没来由地便觉得害怕。不知怎的,胸中郁积的一口气无处发泄,整个人便颓软了下来。她仿佛是累极了,抚着起伏不定的心口,吃力地一字一字慢慢道:``臣妾实在是不成了。还有一句话,臣妾实在想问问皇上,否则到了地底下,臣妾也死不瞑目。''她从袖中取出一叠药方,抖索着道,``皇上,这是齐鲁和太医院的太医们开给臣妾的药方,臣妾越吃越病,气虚血淤加重,以致不能有孕。如今臣妾想想,您和皇后娘娘真是夫妻同心,都巴不得臣妾怀不上孩子。臣妾自问除了受命于人,对您的心意从未有半分虚假。您让臣妾从潜邸的格格成了侧福晋,又成了您唯一的贵妃,为何还要这样算计臣妾,容不得臣妾生下您的孩子?''

皇帝的眼底闪烁着阴郁的暗火,殿中格外沉静,带着垂死前挣扎不定的气息。片刻,皇帝徐徐笑出声来:``算计?朕自诩聪明,却哪里比得上你们的满心算计。便是朕说未曾做过,怕你也是不信的吧!''

晞月猛地一凛,死死盯着皇帝:``皇上所言可真?''

皇帝伸出手,托起她的下巴,似有无限感慨。他的声音有些沙哑的温柔:``真?什么是真?晞月啊,你待朕有真心,却也算计过朕。朕若不是真的喜欢过你,这么些年对你的宠爱也不是能装出来的。朕记得初见你的时候,你是何等温柔娇羞,即使后来你父亲得势,你在朕面前永远是那么柔婉温顺,所以,哪怕你成了贵妃对着旁人娇纵些,朕也不计较。可你如何会变成后来的狠毒妇人,追慕富贵,永不满足。是朕变了,还是你变了?既然咱们谁的真心也不多,你何必再追问这些?''

晞月薄薄的胸腔剧烈地起伏着,像再也承受不住皇帝的话语,热泪止不住地滚滚而落,仿佛决堤的洪水,将脸上的脂粉冲刷出一道道沟壑。她泣然:``原来皇上就是这样看待臣妾?''

皇帝幽幽道:``朕年少时,只想做一个讨皇阿玛喜欢不被人瞧不起的皇子。后来蒙太后抚养,朕便想平平安安做一个亲王。再后来,先帝的子嗣日益稀少,成年的只剩下了朕与五弟弘昼。朕便想,朕一定要脱颖而出,成为天下之主。人的欲望从来不受约束和控制,只会日益滋长不能消减。朕如今只盼望有嫡子可以继承皇位,其他的孩子,有能生的自然好,若有不能生的,也是无妨。''

晞月听着这些话一字一字入耳,仿佛是一根根钉子钻入耳底,要刺到脑仁儿深处去。皇帝看着她哭残的妆容,缓缓闭上眼睛:``你也累了,好好歇着吧。你身后的事,朕会好好安置,会给你一个好谥号,一个好结果,也不枉你跟着朕这许多年。''

晞月在绝望里抬起婆娑泪眼,痴痴笑着道:``谥号?皇上连谥号都替臣妾想好了?那就容臣妾自己说一句吧。臣妾这一辈子便如一场痴梦,后悔也来不及了,只盼下辈子不要落入帝王家,清清静静嫁了人相夫教子,也做一回贤德良善之人便好了。''

皇帝站起身,负着手徐步踱出:``这是你最后的请求,朕不会不答应。朕便以此`贤'字,作为你下辈子的期许,赐给你做谥号吧。''

泪眼蒙眬中,晞月望着皇帝离去的背影,吃力地瘫在榻边,冷笑中落下泪来:``皇上,即便您不肯认,臣妾还是对您恨不到极处。''她抚摸着皇帝坐过的垫褥、靠过的鹅羽垫子,痴痴笑道,``那么,就让臣妾再小小算计您一回,就这一回吧。''

她伏在地上,剧烈地咳嗽,一直咳到唇角有鲜血涌出。她任凭喉头涌出鲜血,慢慢地抚摸着,只是微笑。茉心听得动静,赶进来一看,吓得几乎魂飞魄散,道:``小主,小主您怎么了?''

晞月睁大了双眼,死死抓住她的衣襟道:``茉心,你是在我身边伺候最久的,我只有一句话嘱咐你。千万,千万别忘了皇后是怎么害我的!''

茉心见她乌水银似的眼珠瞪得几乎要脱出眼眶来,骇得魂飞魄散,啼哭着劝道:``小主都这个样子了,还念着这些做什么?到底自己的身子骨要紧啊!''

晞月的手背上青筋暴突,扭曲得如要蹿起的青蛇,嘶声道:``我是不成了,可你要是还活着一天,还念着我对你的好,你一定要记得皇后是怎么对我的!她以为什么事都吩咐了素心来告诉我,便是我当着她的面问了一二她都装糊涂撇清,我便不知道是她指使的了!原是她害了我这一辈子啊!''

茉心含着泪道:``小主对奴婢的大恩大德,奴婢至死不忘。小主,奴婢赶紧扶您去床上歇着吧。''

晞月竭力伸出手,指着皇帝坐过的垫褥和靠过的鹅羽垫子,嘶哑着喉咙道:``快去,快去烧了。脏东西,留不得。''

\hypertarget{ux7b2cux5341ux7ae0-ux6167ux8d24}{%
\chapter{第十章 慧贤}\label{ux7b2cux5341ux7ae0-ux6167ux8d24}}

皇帝坐在步辇上,看着月色苍茫,想起晞月方才所言,只觉得前事茫茫,亦有花落人亡的两失之感。李玉善察皇帝心思,便道:``今儿皇上也还没翻牌子,此刻是想去哪里坐坐?''

皇帝的眼神不知望着何处,只觉得身体轻渺渺地若一叶鸿毛,倦倦地问:``李玉,朕从前,是不是很宠爱慧贵妃?''

李玉不知皇帝所指,只得赔着笑脸道:``是。可皇上也宠爱舒嫔,宠爱嘉妃,六宫雨露均沾\ldots\ldots{}''

皇帝倏然打断他:``你伺候了朕多年,有没有觉得,朕宠了不该宠的人?''

李玉吓了一跳,也不敢不答,只得道:``能不能得宠是小主们的本事和福分,至于皇上宠不宠,怎么宠,这可没有该不该的!皇上仁厚,后宫这些小主,皇上从没冷落了谁,也不见特别专宠了谁。''他一壁说着,只怕哪里答得不慎,惹得皇上不悦,便越发战战兢兢。

皇帝只是浅浅一哂,流水似的月华泻在他俊逸清癯的面庞上,愈加显得光华琳然,却有着不容亲近的疏冷。皇帝的语气里有着无限寂寥:``或许,朕知道怎么宠她们,却不知如何爱她们,所以落到今日这般田地。''

李玉伺候皇帝多年,深知他心性难以捉摸,更不敢随便言语,只得苦着脸道:``皇上,奴才哪里懂得这些。您和奴才说这些,岂不是对牛弹琴么\ldots\ldots 奴才就是那牛。''他说着,轻轻``哞''了一声。

皇帝忍不住失笑,便吩咐道:``瞧你那猴儿样子。罢了,去翊坤宫吧。''

皇帝进来时如懿正换了玉色湖水纹素罗寝衣,从镜中见皇帝进来,便道:``夜深了,怎么皇上还过来?''

皇帝拉着她的手道:``你这儿让人心静,朕过来坐坐。''他的手指触到如懿手腕上的莲花镯,眼中闪过一丝深恶痛绝之意,伸手便从她手腕上扯了下来抛到门外,道:``这镯子式样旧了,以后再不必戴了。明儿朕让李玉从内务府挑些最好的翠来送你,再让太医给你开几个进补的药方,好好补益补益身体。''

如懿没有任何疑义,温顺道:``是。''她挽着皇帝坐下,``皇上去看过慧贵妃了?''

皇帝支着头坐下:``是。她和朕说了好多话。''

如懿从妆台上取过一点茉莉薄荷水,替皇帝轻轻揉着太阳穴道:``人之将死,其言也善,难免会话多些。''

皇帝握着她的手,抚着她如云散下的青丝万缕,低声道:``如懿,有一天你会不会算计旁人?''

如懿的眸光坦然望向他,``会。若是此人做了臣妾绝不能容忍之事,臣妾会算计。''

``你倒是个直性子,有话也不瞒着朕。''皇帝凝视着她,似乎要看到她的心里去,``那你会不会算计朕?''

如懿心头一颤,有无限的为难委屈夹杂着愧疚之意如绵而韧的蚕丝,一丝丝缠上心来。她对他,并不算坦荡荡,所以这样的话,她答不了,也不知如何去答。良久,她抬起眼,直直地望着皇帝,柔声而坚定:``但愿彼此永无相欺。''

皇帝望了她许久,轻轻拥住她道:``有你这句话,朕便安心了。''他长长地叹口气,``如懿,朕今日见了晞月,听她说了那么多话,朕一直觉得很疑惑。人人都以为朕宠爱晞月,连晞月自己也这么觉得,可是到头来,彼此的真心又有几分?''他抓着如懿的手,按在自己的心口,隔着绵软的衣衫,她分明能感触到衣料经纬交错的痕迹下他沉沉的心跳。皇帝有些迷茫,``如懿,朕知道怎么让一个女人高兴,怎么让一个女人对朕用尽心思讨朕的喜欢,可是朕忽然觉得,不知道该如何去爱一个女人。从没有人告诉朕,也没有人教过朕。父母之爱是朕天生所缺,夫妻之爱却又不知如何爱起。或许因为朕不知道,所以朕有时候所做的那些自以为是对你好的事,却实在不是朕所想的那样。''

如懿看着他的神色,仿佛一个迷路的孩子,极力寻找着想要去的方向,却又那么不知所措。她无言以对,只是紧紧地拥住他,以肉身的贴近,来寻觅温暖的依靠。

许久,皇帝的神色才渐渐安静下来,向外扬声道:``李玉,传朕的旨意。''

李玉忙进来答应了一声,垂着手静静等着。

皇帝沉着道:``贵妃高佳氏诞生望族,佐治后宫,孝敬性成,温恭素著。着晋封皇贵妃,以彰淑德。娴妃、纯妃、愉嫔,奉侍宫闱,慎勤婉顺。娴妃、纯妃着晋封贵妃,愉嫔着晋封为妃,以昭恩眷。''

如懿忙敛衣跪下:``臣妾多谢皇上厚爱。''

皇帝扶住她道:``要你和纯妃同时晋位贵妃,已经是委屈了你。可纯妃为朕诞育了两位皇子,又抚养了永璜,朕不能不多眷顾。''他顿一顿,``愉嫔生育之后一直不能侍寝,朕也不勉强她,至少她生下了永琪,让你和朕都有了安慰。''

如懿微微动情,按着永远平坦的小腹,感伤不已:``是臣妾无能,不能为皇上诞育子嗣。''

皇帝抚着她的肩膀道:``会有的,以后一定会有的。''

星河灿灿,盈盈相语。这样静好的时光,宛如一生都会凝留不去。

两日后,乾隆十年正月二十五日填仓日,皇贵妃高佳氏薨。

众人都说,高佳氏是熬死在咸福宫中,更是盼着皇帝盼了这些年,活活盼死的。当然,这样的话只会在宫闱深处流传,永远也流不到外头去。

在外人眼里,他们所看到的,是高晞月被追封为慧贤皇贵妃。追封的册文亦是极尽溢美之词、哀悼之情:

赞雅化于璇宫,久资淑德;缅遗芳于桂殿,申锡鸿称。既备礼以饰终,弥怀贤而致悼。尔皇贵妃高氏,世阀钟祥,坤闺翊政,服习允谐于图史,徽柔早着于宫廷。职佐盘匜,诚孝之思倍挚,荣分翚翟,肃雝之教尤彰。已晋崇阶,方颁瑞物。芝检徒增其位号,椒涂遂失其仪型。兹以册宝,谥曰慧贤皇贵妃。于戏!象设空悬,彤管之清芬可挹,龙文叠沛,紫庭之矩矱长存。式是嘉声,服兹庥命。

这篇册文,不仅极尽哀情,宣昭皇帝对早逝的慧贤皇贵妃的悲痛哀婉之情,连私下作诗娱情,皇上亦是念念不忘。皇帝将亲笔所书的挽诗《慧贤皇贵妃挽诗叠旧作春怀诗韵》亲自在祭礼上焚烧,以表长怀之意,六宫妃嫔无不艳羡。连皇后亦道:``皇上待皇贵妃情深意长,皇贵妃死前请求皇上以`贤'字为谥,皇上答允。但愿来日,皇上亦将此`贤'字赠予臣妾为谥号,臣妾便死而无憾了。''

皇帝不以为然:``皇后春秋正盛,怎么出此伤感之语?''

皇后悄然注目于皇帝,试探着道:``我朝皇后上谥皆用`孝'字。倘许他日皇上谥为`贤',臣妾敬当终身自励,以符此二字。''

皇帝的神色并不为所动,仿佛是在褒扬,却无任何温容的口气:``皇后好心胸,好志气。''

皇后垂泪道:``皇贵妃去世之后,皇上悲痛不已,再未进过臣妾的长春宫,定是皇上想到臣妾与皇贵妃相知相伴多年,怕触景伤情罢了。''

皇帝漠然一笑置之:``皇后能这样宽慰自己,自然是好的。''

皇后福一福身道:``这些日子皇上除了娴贵妃,很少召旁人侍寝,但请皇上节哀顺变。''

皇帝并不看皇后一眼,只道:``皇后的心思朕心领了。朕也想皇后与慧贤皇贵妃相伴多年,她离世你自然会哀痛不舍,所以不去打扰皇后。至于朕对皇贵妃的哀思,每年皇贵妃去世的填仓日,朕都会写诗哀悼,以表不忘皇贵妃因何逝世。''

皇后面上苍白,身体微微一晃,勉强笑道:``皇上情深意长\ldots\ldots{}''

如懿在侧道:``皇上自然是情深意长,所以今夜只怕还要悼念皇贵妃,对着皇贵妃的画像倾吐衷肠。只怕皇贵妃临终前说不完的话,梦中相见,还要与皇上倾诉呢。''

皇后勉强撑着笑容:``皇贵妃早逝,最牵挂的不过是家中父兄。臣妾恳请皇上,若是眷顾贵妃,也请眷顾其亲眷,让贵妃瞑目于九泉。''

皇帝不置可否,只是凝眸于皇后:``皇贵妃福薄身死,不能追随朕左右,朕哀恸不已。然而其父兄之事,当属朝政,岂干后宫事宜?譬如皇后兄弟犯法,朕当奈何?不过一视同仁而已,那么皇贵妃父兄若不勤谨奉上,朕也不能以念皇贵妃而稍稍矜宥。''

皇后神色愈加难堪。如懿温言道:``皇上内外分明,不以私情而涉朝政。皇后娘娘陪伴皇上多年,自然也清楚。皇上何必以此为例?话说回来,皇上也正是器重皇后娘娘的弟弟傅恒大人的时候呢。''

皇帝如常含笑:``是。皇后无须多心。''

皇后欠身为礼:``傅恒年轻,还缺历练,皇上多磨炼他才好。否则身为公卿之家,凡事懈怠,臣妾也不能容他。''皇后目光一滞,忽然凝视如懿手腕,笑吟吟道,``娴贵妃,本宫赏你的莲花镯呢?怎么不戴了?''

皇帝仿佛不经意似的,道:``那镯子本是和皇贵妃的一对,既然皇贵妃离世,那镯子也戴得旧了,朕让娴贵妃换了。对了,还有一件事,朕想着大阿哥的生母哲妃死得可怜,朕会一并下旨,追封哲妃为哲悯皇贵妃。''

皇后讷讷道:``那,也好\ldots\ldots{}''

皇帝并不容她说完,语气冷漠:``你跪安吧。''

皇帝许人``跪安'',于外臣是礼遇,对内嫔妃,则是不愿她在跟前的意思了。皇后如何不明其中深意,脚下一个踉跄,到底稳稳扶着素心和莲心的手,含悲含怯退下了。

待回到长春宫,莲心便出去打点热水预备皇后洗漱。寂然无人之时,皇后才露出强忍的惊惧之色,拉住素心的手惶然道:``你说,高晞月临死前是不是和皇上说了什么?皇上说哲妃死得可怜,哲妃死得有什么可怜的?当日闲言四起,本宫还特意着人查问了,太医也说了是暴毙而亡,并无疑迹啊。''

素心忙挤出一丝笑容安慰道:``奴婢去问过彩珠,皇贵妃临死前是单独和皇上说过话,但说了什么也无人得知。至于皇上说哲妃死得可怜,大约也是怜惜她年轻轻就走了,没什么旁的意思!''

皇后神色恍惚,唯有一种破碎的伤痛弥漫于面容之上。她紧紧捏着素心的手腕,几乎要捏出青紫的印子来,仿佛唯有如此,才能寻得支撑躯体的力量:``本宫与皇上多年夫妻,可是哲妃死后,皇上渐渐有些疏远本宫,他所思所想,本宫全然不知。太后也一直对本宫有所防范,若非如此,本宫又何必安排成翰在太后身边?皇上对本宫若即若离,本宫永远都不知道自己做得合不合皇上的心意,会不会一个不测便失去所有的一切!本宫永远都在茫然的揣测中惶恐不安。若非如此,本宫也不会急着笼络王钦,逼着莲心嫁给王钦,才能借着王钦窥得皇上的一点点心意。''

素心抚着皇后瘦得脊骨突出的背,柔声劝和:``娘娘一切都是为了皇上,皇上终有一天会明白的!''

皇后潸然落泪,连连摇头:``或许本宫真的是错了,莲心不堪重托,嫁与王钦也是白费,反而断了王钦这条路子。或许当日是你嫁给王钦,周旋圆滑,一切都会好些。只可惜本宫当日一念之差,听了嘉妃说你得力,又见莲心是汉人出身,才做主将莲心嫁了出去。''

素心的眼底闪过一丝怯色,抚着皇后的手不觉加重了力气,勉强笑道:``皇后娘娘别这样说,是奴婢无用,不能替娘娘分忧。''她眼珠一转,笑吟吟道,``娘娘且宽心,皇贵妃为人糊涂,一向敬畏您顺从您。但有一样她是明白的,若是出卖了您,便是出卖了她自己,还会把高佳氏全族给连累进去。她不敢!您且看皇上追谥她为皇贵妃,便知道皇上什么都不知情呢。''

皇后的手按着心口,凄然笑道:``她不敢!但愿她不敢!''她的神色陡然变得凄厉,``即便她敢,本宫也是唯一的皇后,永远是皇上唯一的妻子!谁也别妄想动摇本宫!''

皇帝对皇后的冷落,便是从慧贤皇贵妃死后而起。那三个月,除了必需的典庆,他从未踏足长春宫一步,连皇后亲去西苑太液池北端的先蚕坛行亲蚕礼这样的大事,也只草草过问便罢了。

那种冷落,实在像极了慧贤皇贵妃生前的样子。然而,皇帝这样的冷落也并未引起六宫诸多非议,因为除了皇后宫中,东西六宫他都不曾踏足,身体的抱恙让他无暇顾及六宫嫔妃的雨露之情,只避居养心殿中养病。

这病其实来得很蹊跷,是从慧贤皇贵妃死后半个多月皇帝才开始发作的,一开始不过是肌肤瘙痒,入春后身上渐渐起了许多红疹子,大片大片布及大腿、后背、胸口,很快疹子发成水疱,一个个饱含了脓水,随后连成大片,不忍卒睹。且随着病势沉重,发热之状频频出现,皇帝一开始还觉得难以启齿,不愿告诉太医,病到如此,却也不能说了。

最先发现的人固然是如懿,一开始她还能日夜伺候身侧,为皇帝挑去水疱下的脓水,再以干净棉布吸净,可是皇帝发病后,她的身上很快也起了同样的病症,方知那些红疹是会传染的,且如懿日夜照顾辛苦,发热比皇帝更重,也不便伺候在旁,便挪到了养心殿后殿一同养病。

\hypertarget{ux7b2cux5341ux4e00ux7ae0-ux590dux6069}{%
\chapter{第十一章 复恩}\label{ux7b2cux5341ux4e00ux7ae0-ux590dux6069}}

如此一来,连太后也着了急,一日数次赶来探望,却被齐鲁拦在了皇帝的寝殿外。齐鲁忧心忡忡道:``皇上的病起于疥疮,原是春夏最易发的病症,却不知为何在初春便开始发作起来了。''

太后扶着皇后的手,急道:``到底是什么症候,要不要紧?''

齐鲁忙道:``皇上怕是接触了疥虫,感湿热之邪,舌红、苔黄腻、脉数滑为湿热毒聚之象。湿热毒聚则见脓疱叠起,破流脂水。微臣已经协同太医院同僚一同拟了方子,但之前皇上讳疾忌医,一直隐忍不言,到了今时今日,这病却是有些重了。''

太后遽然变色,严厉道:``这些日子都是谁侍寝的?取敬事房的档来!''

皇后忙恭声回答:``太后,臣妾已经看过记档,除了纯贵妃和舒嫔各伴驾一次,但纯贵妃刚有身孕,之后都是娴贵妃了。''

太后鼻息微重,疾言厉色道:``娴贵妃呢?''

李玉察言观色,忙道:``皇上之前不肯请太医察看,都是娴贵妃在旁照顾,贵妃小主日夜辛劳,如今得了和皇上一样的症候,正在养心殿后殿养着呢。''

太后这才稍稍消气:``算她还伺候周全。只是娴贵妃怎得了和皇上一样的病,莫不是她传给皇上的吧?''

李玉忙道:``皇上发病半个月后娴贵妃才起的症状,应该不像。''

皇后看着齐鲁道:``你方才说皇上的病是由疥虫引起的,疥虫是什么?是不是翊坤宫不大干净,才让皇上得上了这种病?''

齐鲁躬身道:``疥虫是会传染疥疮,也可能是得了疥疮的人用过的东西被皇上接触过,或是皇上直接碰过得了疥疮的人才会得这种症候。至于翊坤宫中是否有这样的东西,按理说只有皇上和娴贵妃得病,那翊坤宫应该是干净的。''

太后沉声道:``好了。既然其他人无事,皇后,咱们先去看皇帝要紧。''

齐鲁忙道:``太后、皇后当心。太后与皇后是万金之体,这病原是会传染的,万万得小心。''说罢提醒小太监给太后和皇后戴上纱制的手套,在口鼻处蒙上纱巾,方由李玉引了进去,又道:``太后娘娘,皇后娘娘,千万别碰皇上碰过的东西,一切奴才来动手即可。''

太后见李玉和太医这般郑重其事,也知道皇帝的病不大好,便沉着脸由着李玉带进去。

寝殿内,一重重通天落地的明黄色赤龙祥云帷帐低低地垂着,将白日笼得如黄昏一般。皇帝睡榻前的紫铜兽炉口中缓缓地吐出白色的袅袅香烟,越发加重了殿内沉郁至静的氛围。偶尔,皇帝发出一两声呻吟,又沉默了下去。

两个侍女跪在皇帝榻前,戴着重重白绡手套,替皇帝轻轻地挠着痒处。太后见皇帝昏睡,示意李玉掀开被子,撩起皇帝的手臂和腿上的衣物,触目所及之处,皆是大片的红色水疱,在昏暗的天光下闪烁着幽异的光泽,更有甚者,一起成了大片红色饱满的突起的疖状物。皇帝含糊不清地呻吟着:``痒\ldots\ldots 痒\ldots\ldots{}''

皇后情难自禁,泪便落了下来。太后到底有些心疼,轻轻唤了几句:``皇帝,皇帝!''

皇帝并没有清醒地回应,只是昏昏沉沉地呢喃:``额娘,额娘,痒\ldots\ldots{}''

太后的面色略沉了沉:``皇后,你听见皇帝说什么?''

皇后知道皇帝的呼唤犯了太后的大忌,这``额娘''二字,指的未必是在慈宁宫颐养天年的皇太后。然而她也知道这话说不得,勉强笑道:``皇上一直尊称您为皇额娘,如今病中虚弱,感念太后亲来看望,所以格外亲热,只称呼为额娘了。''

太后唇边的笑意淡薄得如同远处缥缈的山岚:``难为皇帝的孝心了。''她的口气再不如方才热切,``齐鲁,给皇上和娴贵妃用的是什么药?可有起色?''

齐鲁忙道:``回太后,微臣每日用清热化湿的黄连解毒汤给皇上服用,另用芫花、马齿苋、蒲公英、如意草和白矾熬好的药水擦拭全身。饮食上多用新鲜蔬果,再辅以白鸽煲绿豆、北芪生地煲瘦肉两味汤羹给皇上调治。娴贵妃得的病症晚,虽然发热较多,但不比皇上这样严重,这些药外敷内服,已然见效了。''

太后扶了扶鬓边的瑶池清供鬓花,颔首道:``你是太医院之首,用药谨慎妥当,哀家很放心,就好好为皇上治着吧。一应汤药,你必得亲自看着。''齐鲁答应出去了。太后回转头,见皇后只是无声落泪,不觉皱眉道:``皇后,你是六宫之主,很该知道这时候掉眼泪是没有用处的。若是你哭皇上便能痊愈,哀家便坐下来和你一起哭。''

皇后忙忍了泪道:``是。''

太后皱眉道:``皇上的病不是什么大症候,眼泪珠子这么不值钱地掉下来,晦气不晦气?若是娴贵妃也跟你一样,她还能伺候皇帝伺候到自己也病了?早哭昏过去了。''

皇后见太后这般说,少不得硬生生擦了眼泪:``儿臣但凭皇额娘吩咐。''

太后叹口气道:``你这样温温柔柔的性子,也只得哀家来吩咐了。既然娴贵妃已经病着,宫中其他妃嫔可以轮侍,纯贵妃刚有了身孕,嘉妃要抚养皇子,都不必过来。余者玫嫔、舒嫔是皇帝最爱,可以多多侍奉,愉妃、庆常在、秀答应也可随侍。你是皇后,调度上用心些便是。''

太后一一吩咐完,皇后跪下道:``皇额娘圣明,臣妾原本不该驳皇额娘的话,但是皇上的病会传染,若是六宫轮侍,万一都染上了病症,恐怕一发不可收拾。若是皇额娘觉得儿臣还妥当,儿臣自请照顾皇上,必定日夜侍奉,不离半步。''

太后双眸微睁,眸底清亮:``是么?皇后与皇帝如此恩爱之心,哀家怎忍心分离。便由着皇后吧。只是皇后,你也是人,若到支撑不住时,哀家自会许人来帮你。''说罢,太后便又嘱咐了李玉几句,才往殿外去。

因皇帝病着,寝殿内本就窒闷,太后坐了一路的辇轿,一直到了慈宁宫前,才深吸一口气,揉着额头道:``福珈,哀家觉得心口闷闷的,回头叫太医来瞧瞧。''

福珈正答应着,转头见齐鲁正站在廊下抱柱之后,不觉笑道:``正说着太医呢,可不齐太医就跟来这儿了呢。''

太后闻声望去,见齐鲁依礼请安,却是一脸惶惶之色,不由得皱眉道:``怎么了?皇帝病着,你这一脸慌张不安,也不怕犯了忌讳?''

齐鲁这才回过神来,忙不迭拿袖子擦了脸道:``微臣有罪。微臣有罪。''

这告罪甚是没有来由,太后与福珈对视一眼,旋即明白,便道:``起来吧。哀家正要再细问你皇帝的病情。''

齐鲁上前几步,跟着太后进了暖阁,见左右再无外人伺候,方才缓和些神色。太后扶了福珈的手坐下,稳稳一笑,睨着他道:``三魂丢了两魄,是知道了慧贤皇贵妃临死前狠狠告了你一状吧?''

齐鲁赶紧跪下:``回太后的话,微臣在宫里当差,主子的吩咐无一不尽心尽力做到,实在不敢得罪了谁啊!''

福珈替太后斟了茶摆上,看着齐鲁抿嘴笑道:``齐太医久在宫中,左右逢源,不是不敢得罪了谁,是实在太能分清谁能得罪谁不能得罪了。您怕慧贤皇贵妃知道了您对她做的那些事,教皇上怪您做事不谨慎?那可真真是没有的事。您是皇上最得力的人,皇上有的是要用您的地方,有什么可怕的,您前途无量呢。''

齐鲁慌不迭摆手道:``姑姑的夸奖,微臣愧不敢当。''

太后轻轻一嗤,取过手边一卷佛经信手翻阅,漫不经心道:``你要仔细些,皇帝来日若要怪罪你,不会是因为你替他做的那些事,只会是知道了你也在为哀家做事。''

齐鲁吓得面无人色,叩首道:``太后、皇上、皇后都是微臣的主子,微臣不敢,微臣不敢啊!''

四下里静悄悄的,唯有紫檀小几上的博山炉里缓缓吐出袅袅的轻烟如缕,那种浅浅的乳白色,映得太后的面容慈和无比:``皇后只求生子,皇上看重你的才干,哀家也只取你一点往日的孝心,借你的手让后宫安宁些罢了。皇帝娶的这些人,摆明了就是倚重她们的母族。乌拉那拉氏便罢了,早就是一盘散沙,高氏能由格格而至侧福晋,又一跃而成贵妃,宠擅椒房,也是借了她父亲高斌的力。''太后眼里衔着一丝恨意,``当初哀家的端淑远嫁,一则是为了朝廷安宁不得不嫁,二则何曾少了高斌的极力促成。身为太后,哀家不能不为朝廷考虑,但身为人母,哀家却不能不记得这件事。皇后出身贵重,有张廷玉和马齐在前朝遥相呼应,便是马齐死后,她弟弟傅恒也入朝为官,平步青云。哀家要制衡皇后,原就费些力气。若再有高氏这般对皇后死心塌地之人有了子嗣倚仗,岂不更加费力。''

齐鲁诺诺道:``是是。太后的原意也不想伤了谁的性命,也是慧贤皇贵妃命该如此。''

太后笑得优雅而和蔼,闲闲道:``她的命或许不该如此,只是她父亲送走了哀家的女儿,哀家也不容她女儿这般快活罢了。只不过,这件事哀家才吩咐你去做,便发觉原来皇帝也知她气虚血淤不易有孕,哀家不过是让你顺水推舟,告诉皇帝她已不易有孕,若治愈后再生是非,一则后宫不睦,二则更添高佳氏羽翼,三也勾起哀家思女之心,两宫生分。所以皇帝才会对你所作所为假作不知。你放心,皇帝既然知道你的忠心,便没人能动你分毫。''

齐鲁这才安心些许,想了想又道:``那么舒嫔小主\ldots\ldots{}''

太后垂着眼皮,淡淡打断他道:``各人有各人的缘法,谁吩咐你做什么你便做,旁的不必多理会。''

齐鲁这才告退。福珈见齐鲁出去,便替太后捶着肩,试探着道:``舒嫔小主的事,太后当真不理会么?''

太后凝神想了片刻,叹口气道:``舒嫔是个痴心人儿,一心痴慕皇帝。哀家除了能成全她的痴心,别的什么也成全不了。''

福珈似是不忍,沉吟着道:``可怜了舒嫔一片痴心。不过想想也是,许多时候羁绊越深越不能自拔,若真一颗心都在皇上身上了,便也白费了太后的调教了。''

皇帝如此一病,皇后便在养心殿的寝殿之旁安住下来。皇后自侍奉皇帝,事必躬亲,衣不解带,但凡皇帝有半点不适,她便半蹲在皇帝身前反复擦拭药水,直到瘙痒渐止才肯稍作歇息。而皇帝的病症常在夜深人静时发作,常常不能安眠,皇后便也不眠不休,守候一旁。

如懿身体稍稍好转时,曾往养心殿寝殿探望皇帝,谁知才掀了帘子,李玉已经赶出来,噤声摆手道:``皇后娘娘在里头呢。''

如懿昏昏沉沉,脚下本就虚浮,便靠在惢心怀里道:``只有皇后在么?''

李玉点头道:``皇后娘娘不许六宫前来侍奉,以防病症传染,所以一直是娘娘一个人在。''

如懿了然:``难为皇后的苦心。皇上这一病,倒不能不见她了。''

李玉低眉颔首:``皇后到底是六宫之主。''

如懿伸手撂下帘子,便也不再进去。回到后殿,惢心却有些不安:``皇后娘娘日夜陪伴在侧,见面三分情,小主不得不防啊!''

``防?''如懿淡淡微笑,重又躺好,``皇后能一人侍疾,自然是太后允准的。高晞月已死,皇后也被冷落多时。皇上一直在我宫里,太后自然会不放心。太后不喜欢宫中有人独大,本宫就顺从她的意思罢了。''

惢心替她盖好锦被,低声道:``那小主不怕\ldots\ldots{}''

``怕?高晞月死前的话必定不是白说的,心结已经种下,以后要拔除也难了。我有什么可怕的。''如懿的声音温沉而低柔,``我且养好了身子,比什么都要紧。''

起初,皇帝蒙眬中醒来,见女子衣着清素,以纱巾覆面,总以为是如懿在侧。直到数日后发热渐退,他逐渐清醒,看到伏睡于床边的女子,便挣扎着向李玉道:``娴贵妃累成这样,怎么不扶下去让她休息?''

李玉见皇帝好转,不由得惊喜交加,忙道:``皇上,您不认得了?这是皇后娘娘呀。''

皇帝``哦''了一声,虚弱地道:``皇后怎么来了?''

李玉道:``皇上,自从娴贵妃病倒,一直是皇后娘娘为您侍疾,衣不解带,人也瘦了好些。''

皇帝颇有些动容,咳嗽几声,伸手去拂落皇后面颊上的轻纱。他原是病着的人,下手极轻,却不想皇后立刻坐起,人尚未完全醒转,迷糊着道:``皇上要什么?臣妾在这里。''

皇帝看她如此急切,心下一软,生了绵绵暖意:``皇后,你辛苦了。''他略略点头,``李玉,皇后累了,扶她下去歇息,让别人来照顾吧。''

皇后见皇帝不欲她在眼前,一时情急,忙跪下恳切道:``皇上,臣妾知道您不愿见臣妾,但您病着,臣妾是您的结发妻子,如何能不在床前悉心照料。皇上的病症是会传染的,娴贵妃一时不慎,已经病下了,若是六宫之中再有什么不妥,累及儿女,岂不是臣妾的过错?''

皇帝的口气温和了几许:``皇后,你起来吧,别动不动就跪着。''

皇后见皇帝的语气略有松动,含泪道:``臣妾自知粗陋,皇上不愿见臣妾,所以以纱巾覆面,但求皇上不要厌弃,容臣妾如宫人一般在旁侍奉就好。''

皇帝看了她一眼,含了脉脉的温情,叹息道:``皇后,你瘦了。''

皇后辛苦了多时,听得皇帝语中关切,一时情动,不禁落下泪来:``只要能侍奉皇上痊愈,臣妾怕什么。''

皇帝咳嗽几句,身上又有些发痒,便懒怠言语,侧身又朝里躺下了。皇后忙膝行到皇帝跟前,拿柔软的白巾蘸了药水一点一点替皇帝擦拭,每擦拭一下,便轻轻吹气,为痒处增些清凉之意。皇帝见她做得细致,便也不说话,由着她侍奉。

转眼便到了晚膳时分,皇后出去了一炷香的时辰,方端着膳食进来。因皇帝在病中,一切饮食以清爽为要,不过一碗白粥,一道熘鲜蘑并一个白鸽绿豆汤。皇帝由李玉和进忠扶着坐起来,皇后也不肯假手他人,亲自喂了皇帝用膳。

皇帝尝了两口,抿唇道:``不是御膳房做的?''

素心喜不自胜:``皇上是好多了呢,这个也能尝出来了。这些天皇上的饮食,都是皇后娘娘亲手做的,不敢让旁人插手半分,只怕做得不好呢。''

皇帝眼中有晶润的亮色,一顿饭默默吃完,也无别话。待到饮药时,皇后亦是先每样尝过,再喂到皇帝口中。

皇帝温然道:``太医院开的药,皇后何须如此谨慎?''

皇后眼中一热,垂下眼睑,诚挚无比:``臣妾万事当心,是因为病的是皇上,是臣妾的夫君。''她大着胆子凝视皇帝,恳切道,``皇上这些日子病着,少有言语,臣妾陪在皇上身边,皇上何处不适,想做什么,臣妾一一揣测,倒觉得与皇上从未如此亲近过。''

皇帝沉默片刻,伸手拍一拍皇后的手,温和道:``皇后有心了。''

服完药皇帝便又睡下了。皇后忙碌了大半日,正要歇一歇,却见莲心进来,低低耳语几句,便强撑着身体起来,走到殿外。

廊下里皆是新贡的桐花树,分两边植在青花莲纹的巨缸内。桐花绵绵密密开了满树,绛紫微白,团团如扇。风过处,便有雅香扑鼻。皇后闻得药味久了,顿觉神清气爽。转眸处,月色朦胧之中,却见一个宫装女子跪在殿前,抬起清艳冷然的面庞,朗声道:``皇上卧病,皇后娘娘为何不许臣妾向皇上请安?''

皇后扶着素心的手,和颜悦色道:``舒嫔,皇上的病容易传染,本宫也是担心你们。与其人人都来探视侍奉,哪一个弱些的受了病气,六宫之中还如何能安生。''

意欢不为所动,只是觑着皇后道:``皇后娘娘好生辛劳,独自守着皇上,却忘了您还有公主要照顾,倒不比臣妾这样无儿无女没有牵挂的,侍奉皇上更为方便。''

皇后站在清朗月色下,自有一股凛然不肯相侵之意:``你自是无儿无女,可你还年轻,万一沾染上疥疮伤了你如花似玉的容貌,那以后还怎么侍奉皇上?便是愉妃,本宫都没有让她过来。''

意欢本就长得清冷如霜,肤白胜雪,一笑之下更如冰雪之上绽放的绰艳花朵,艳光迷离。她施施然站起身,风拂她裙袂,飘舞翩跹:``皇后娘娘真是好贤惠,一人侍奉皇上,不辞辛苦,臣妾等人想见一面都不得。这也罢了,只是臣妾为皇上亲手编了福袋,已请宝华殿法师开光,能否请皇后娘娘转交?''

皇后听她这般说话,丝毫不动气,只是笑:``福袋甚好,只是不如等来日舒嫔亲自交给皇上更有心意。夜来露水清寒,恐伤了妹妹。本宫想,皇上病愈后,一定希望见到妹妹你如花容颜,那么妹妹还是回宫好好歇息吧。''说罢,皇后再不顾她,只低声嘱咐,``素心,还是老规矩,不许任何人前来打扰皇上静养。''她想一想,又道,``齐鲁给本宫准备的坐胎药,一定要记得按时给本宫送来喝。''

素心清脆地答应一声:``其实皇上病着,娘娘何必如此着急?''

皇后压低了声音道:``比起之前皇上对本宫不闻不问,如今已是好了许多。若不趁皇上病势好转对本宫有所垂怜之时怀上龙胎,更待何时?''

素心只得默然,便又守在门外。意欢见皇后如此,也无可奈何,只得揉着跪得酸痛的膝盖,悻悻道:``荷惜,陪本宫去宝华殿吧。''

荷惜担心道:``小主,自从皇上卧病,您一直在宝华殿为皇上祈福,不停编织福袋,描画经幡,奴婢真担心您的身子。何况,太后也没有这样交代啊。''

意欢浅浅横她一眼,已然含了几许不悦之色:``本宫关心皇上,何必要太后交代。你若累了,本宫便自己去。''

荷惜忙道:``奴婢不累。只是您这样做,皇上也看不见啊,白白辛苦了自己。''

意欢仰望满天月华,郁然长叹:``皇上看不见又如何?我只是成全我自己的心意罢了。''

\hypertarget{ux7b2cux5341ux4e8cux7ae0-ux6c38ux742e}{%
\chapter{第十二章 永琮}\label{ux7b2cux5341ux4e8cux7ae0-ux6c38ux742e}}

皇帝这一病,缠绵足有百日,待到完全好转,已是六月风荷轻举的时节。而皇后,也因悉心侍疾,复又承恩如初。如懿侍疾致病,皇帝更是疼惜,又偶然听如懿说起意欢日夜在宝华殿祈福的心意,对二人宠爱更甚。咋看之下,六宫中无不和睦,自然是圆满至极了。

到了九月金桂飘香之时,更好的消息便从长春宫中传出,已然三十五岁的皇后,终于再度有娠。这一喜非同小可,自端慧太子早夭之后,帝后盼望嫡子多年,如今骤然有孕,自然喜出望外,宫中连着数日歌舞宴饮不断,遍请王公贵族,举杯相贺。

如此,连承恩最深的如懿与意欢亦是感叹。意欢羡慕不已:``原本就知道借着这次为皇上侍疾,皇后一定会再次得宠,却不想这么快她连孩子都有了。''

如懿抚着平坦的小腹,伤感之中亦衔了一丝深浓如锋刃的恨意,只是不肯露了声色:``想来我已二十八岁了,居然从未有孕,当真是福薄。''她停一停,叹道:``皇后有孕,皇上这么高兴,咱们总要去贺一贺的。''

意欢扬了扬细长清媚的凤眼,冷淡道:``何必去赶这个热闹?皇后有孕与我何干,我既不是真心高兴,自然不必假意去道贺!''

如懿笑语嫣然:``贺的是情面,不是真心。若不去,总落了个嫉妒皇后有孕的嫌疑。''

意欢曲起眉心,嫌道:``姐姐从不在意这些虚情假意的,如今也慎重了。''

如懿的笑容被细雨打湿,生了微凉之意:``浮沉多年,自然懂得随波逐流也是有好处的。''

意欢沉郁片刻:``姐姐也如此,可见是为难了。''

如懿婉声道:``在宫里,不喜欢的人多了,可是总还要相处下去,彼此总得留几分余地。''

意欢沉吟着道:``我是真不喜欢她們\ldots\ldots{}''

如懿忙掩住她口,警觉地看了看四周,郑重摇头道:``含情欲说宫中事,鹦鹉前头不敢言。妹妹心直口快是好性子,但也会伤了自己。慎言,慎言!''

意欢的唇际挂下如天明前虚浮的弯月,半晌才低低道:``知道了。''

如懿含笑看着她道:``幸好皇上是喜欢妹妹这性子的,但再喜欢,宫中也不是只有皇上一个。''她略停了停道:``皇后有孕是喜事,妹妹你终究还年轻,不必着急。只要皇上的恩眷在,一定很快会有自己的孩子的。''

意欢玉白面容泛起一丝红晕,含笑低低道:``承姐姐吉言了。皇上待我情深义重,自从齐太医请脉说我身体虚寒不易有孕,每回侍寝之后皇上总是嘱咐太医院送坐胎药给我,只是吃了这几年,却是半点动静也没有,大概真是我身子孱弱的缘故。''

如懿到底没有生养过,脸皮子薄,如何肯在光天化日下说这些,便也只是含笑:``皇后为了再度得子,吃了多少坐胎药,不也到了今时今日才有好消息么?你且耐心等一等吧。也就是你得皇上恩宠,咱们侍奉皇上这些年,也从没有侍寝后喝坐胎药的恩典呢。''

意欢面上更红,二人笑语几句,也就罢了。偏生这个时候伺候皇帝的进保进来,笑吟吟道:``给娴贵妃娘娘请安,给舒嫔娘娘请安。皇上说了,昨夜是舒嫔娘娘侍寝,为绵延帝裔,特赐舒嫔娘娘坐胎药一碗,请舒嫔娘娘趁热即刻喝了吧。''

如懿``哎哟''一声,忍不住脸红笑道:``一大清早的便喝上这个了。罢了罢了,怕你害臊,我便先走了。''

珊瑚色的红晕迅疾蔓延上意欢的如玉双颊,她赶紧端过药喝得一点儿不剩,才交还到进保手中,拉着如懿道:``好姐姐,你也取笑我做什么,咱们再说说话吧。''

如懿见宫人们都出去了,方笑道:``那有什么难的,宫里谁不盼望孩子,只不知哪种坐胎药更好罢了。你若有心,便把皇上赏你的坐胎药给我留半碗,我若得了孩子,好好谢你便是。''

意欢听得这话,晕红了脸掩袖笑道:``那有什么难的。等下回进保不留心,我偷留出半碗给你便是了。''

如懿奇道:``怎么?皇上还非得让进保看着你喝完?''

意欢娇羞不已:``可不是么?实在是不好意思。''如懿见她如此,笑着打趣几声,便也含糊过去了。

然而那边厢,皇后中年有孕,格外当心,除了饮食一律在小厨房中单做,亦是请了齐鲁并太医院中几个最德高望重的太医一日三次轮流伺候。而此外,为皇后搭脉的齐鲁脸色并不十分好看,只是一味拈须不语。

皇后的心一分一分沉下去,忍不住问道:``齐太医有什么话,不妨直说。''

齐鲁面色凝重,道:``皇后娘娘此次有孕,本是大喜,从脉象来看,十有八九是个皇子。''

皇后大喜过望:``如此,可要多谢齐太医了。素心,看赏。''

素心捧出一匣银子来,齐鲁慌不迭起身避让道:``微臣不敢,微臣不敢。只是皇后娘娘,您的脉象虽好,可是您的脉象\ldots\ldots{}''他迟疑片刻道:``虚滑无力,脉细如丝,怕是\ldots\ldots{}''

皇后一惊,连忙道:``太医有话,不妨直说。''

齐鲁磕了个头道:``微臣该死。恕微臣直言,皇后娘娘已不是有孕的最佳年纪,又因端慧太子之死忧思过度,这些年神思操劳,导致体质虚弱。虽然微臣一直用药为您催孕,但您有孕之前一直日夜侍疾,以致劳累过度,便是有孕的时机不太对,所以\ldots\ldots{}''

皇后心中一阵阵发紧,面色也越发不好看:``所以如何?你只告诉本宫,能不能保住皇子?''

齐鲁犹豫片刻,迟疑着道:``能是能。但皇后娘娘如今怀孕四个月,按微臣的意思,未免母体孱弱以致胎儿不保,微臣\ldots\ldots{}''他咬了咬牙,似下定决心一般,``微臣打算烧艾替娘娘保胎。''

皇后周身一阵阵发冷,只觉得眼前晕眩不已。她是生育过的人,自然知道要烧艾保胎,必是有滑胎之象了。皇后的手心里全是湿腻腻的冷汗,勉强扶着素心的手撑着身体,极力自持道:``既然能保住胎儿,那一切有劳齐太医了。至于皇上那里\ldots\ldots{}''

齐鲁久待宫闱,何等圆滑晓事:``微臣会替娘娘隐瞒,请皇上放心。''

皇后决然摇头道:``不!本宫不是要皇上放心,你一定要让皇上知道,本宫替皇上怀着嫡子有多辛苦多艰难。即便你要烧艾,也必须皇上在侧陪伴本宫。一定要亲眼让皇上看着本宫的辛苦,皇上才会对本宫倍加怜惜。''

这一年的新年,之前有绿筠为皇帝生下和嘉公主璟妍的喜事,更因为皇后的身孕而格外热闹。而皇后自己则避居长春宫中,甚少再参与内延盛事,嫔妃们去探望是,亦每每见到皇后静卧榻上,服用各色安胎汤药,而太医们神色紧张而恭谨,侍立一旁。

这一日太后探望皇后归来,便在慈宁宫焚香静坐。福珈捧了一本《法华经》来供太后诵读,太后读了几段便笑道:``方才看皇后谨慎的样子,看来这个孩子对她而言真的很要紧。''

福珈穿着一身蓝缎地圆纹如意襟坎肩,配着一身象牙色长袍,用铜鎏金素纹扁方挽着头发,清谈得如太后宫中的一抹香烟。她眉目恭顺地道:``中宫无子,等于是无依无靠。皇后已经三十五岁了,能再有身孕,真的很不容易。''

太后颔首道:``当然不容易。哀家私下问过齐鲁,如此烧艾,能否保孩子到足月。齐鲁告诉哀家,能保到九个月都算万幸了。到底比不得纯妃,一看就是个好生养的身段。''

福珈有些担心:``皇后年岁偏长,若孩子再不足月,那便胎里弱了。''

太后凝神片刻,自嘲地笑笑:``说到底皇帝也不是哀家亲生的,皇后更是名义上的儿媳,自有她娘家人疼爱。哀家要关心,也不过是脸面上的情分。你没听皇帝病着的那时候,昏昏沉沉地叫`额娘',你相信皇帝叫的是哀家么?''

福珈犹豫片刻,替太后添上一壶香片道:``再怎么着,皇上的生母都已经死了。皇上这些年都不提这个人,哪怕梦里软弱些,想着一点半点,也不算要紧事。''

太后一下一下拨着鎏金珐琅花鸟手炉上的小蒂子,轻嘘了口气道:``不是自己肚子里出来的孩子,到底不一样,所以哀家也懒得去提点皇后什么。其实她既然要烧艾保胎,又防着旁人,大可不露声色,临到早产时动些手脚,便可除去想除去的人了。只是她一心借着嫡子博皇上怜爱,到底嫩些。''

福珈含笑道:``太后深谋远虑,皇后哪能和太后您比。何况太后不喜欢任何一方独大,那么皇后也好娴贵妃也罢,您都睁一只眼闭一只眼吧。到底咱们将来的指望,是在玫嫔,舒嫔和庆常在身上了。''

太后见桌上有切好的雪梨,便取了一片慢慢吃了:``庆常在和玫嫔也罢了,舒嫔倒真的是很得皇帝的恩宠。''

``太后千挑万选的人,能不好么?''福珈微微迟疑,``可是这几年齐太医每每暗示,奴婢也留意下来,皇上每次让舒嫔侍寝之后都服用坐胎药,说是盼望早得子嗣,可是奴婢觉得那药不大对头啊。''

太后微微一笑:``对头不对头都不要紧,顶多便是皇帝防着她是叶赫那拉氏的出身,再不济便是防着哀家。''

福珈一凛,旋即道:``那倒不像。皇上若要防着太后,大可不收下庆常在和舒嫔,何必费这种麻烦。''

太后的笑淡淡的,仿佛窗外摇曳的花影依依:``咱们这位皇帝,心思可深着呢。否则当年三阿哥弘时是先帝的长子,乌拉那拉皇后的养子,身份这样贵重,怎么就能落败在了咱们皇帝手里呢。''

福珈低眉顺目:``那自然是因为太后您的缘故。''

太后笑着摇了摇头:``哀家啊什么都可以不理会,只理会一桩。''她的神色慢慢沉寂下来,带了一缕无以言及的哀伤,``便是哀家的柔淑,可以不要像她的姐姐一般命运多舛,离京远嫁。要是柔淑能守在哀家身边,好好儿嫁一个疼她的人,那便好了。''

重重销金花衣之下,太后日渐老迈的身量显得单薄而不堪负重。福珈含了一丝安慰,温厚道:``太后放心,一定会的。''两个人紧紧依傍在一起,天光将她们的影子拉得老长老长,好像是悬在窗棂上的薄薄的纸片,摇摇欲坠。

这一日外头风雪初定,皇帝带着如懿和意欢进来,搓着手道:``外头好冷,皇后这儿倒暖和,''

皇后因靠在床上养息,便只是欠身示意:``皇上万福。''

皇帝穿着一身家常的湖蓝团福纹天马皮长袍,外头罩一件竹青色暗花缎琵琶襟熏貂皮马褂,身后的如懿和意欢穿着同色的金红羽缎斗篷,倒像两个出塞的昭君,格外娇俏。

皇后命人奉上茶点,笑道:``皇上今日兴致倒好,怎带着两位妹妹来了?''

皇帝道:``娴贵妃素性喜欢梅花,正好舒嫔也在,朕便陪着她们赏梅去了。''

皇后微微一笑,抚着隆起的肚子安闲道:``娴贵妃喜欢什么,皇上倒一直惦记着。''

如懿盈然含笑:``皇上惦记着臣妾,臣妾也惦记着皇后娘娘。''她唤过惢心,``宫中绿梅难得,这一束是臣妾选了梅苑中最好的送来给娘娘,希望娘娘闻着梅香清冽,可以安心养胎。''她转首笑盈盈对皇帝道:``今日是正月二十五日填仓日,也是慧贤皇贵妃去世一年的日子,臣妾已经命人去咸福宫中供上梅花,略表怀念之情。''

皇后眉心微曲,很快笑道:``慧贤皇贵妃生前与娴贵妃不大和睦,如今看见娴贵妃送去的花,也一定会在九泉之下释然的。''

如懿只是含笑,盈盈望着皇帝道:``臣妾的心意太过绵薄,早起时见皇上在写诗,您只说是悼念慧贤皇贵妃的,如今大家都在,臣妾便求一个恩典,也想听听皇上对慧贤皇贵妃的情意。''

皇帝摆手道:``不过是闲时偶得罢了。朕已经命人抄录出去,送与慧贤皇贵妃的母家了。''

意欢笑意融融,带了几分撒娇的意味,不依不饶:``皇上如此,便是对皇贵妃及其母家最大的恩眷了。想来高斌大人得此诗书,一定也感念皇恩。不如皇上也念给臣妾们听听吧。''

意欢甚少这般爱娇,一扫素日清冷,皇帝见她如此,便道:``光春风物和氤氲,日逢晴鬯三农欣。粔籹菜甲酬节令,礼从其俗古所云。忧民之忧乐民乐,翳予忧乐因民托。底事间情一惘然,自为此念奚堪者。''

如懿侧耳听完,郁然长叹:``底事间情一惘然,自为此念奚堪者。慧贤皇贵妃虽已过世,皇上还是惦念不已啊。''

皇后极力掩饰好眼底的不豫之色,缓缓笑道:``皇上对皇贵妃的心意真是难得。恰好臣妾和皇上想到一处去了,想着皇贵妃身前最喜欢佩戴荷包和香囊,臣妾昨夜缝了一个,今儿中午也让人送去咸福宫供着了。''

素心在旁道:``皇后娘娘连夜缝制,总说是一点姐妹心意,可见悼念之情。''

皇帝略略点头,神色关切:``皇后有心了。只是你有着身孕,针线上的活计,就交给下人们吧。''

素心抿唇笑道:``其他的也罢了,皇后娘娘还亲手做了一个燧囊送给皇上呢。''

皇后嗔怪似的看了素心一眼,有些不好意思道:``臣妾本想赶着新年送给皇上的,可是体力不支,想着今日是填仓日,正月的最后一个节日了,所以特意献给皇上,还请皇上不要嫌弃。''

皇帝从素心手中接过:``是盛装火镰的燧囊?用鹿尾绒毛做的?''

皇后含了几分期盼,望着皇帝道:``去年秋天的时候皇上与臣妾提起关外旧俗,提及祖上刚刚创建帝业之时,衣物装饰都是用鹿尾绒毛搓成线缝在袖口,而不是像如今宫中那样用金线、银线精工细绣而成。臣妾一向主张节俭,觉着宫中用金的玉的自然是好看,可是也奢靡了些。''

皇帝看着手中的燧囊,果然全用鹿毛制成,并无一点缎料,十分朴素,与太祖所用的并无二致,亦感叹道:``如今这样的东西是少见了,难为你记得朕说过的话。''

皇后道:``臣妾想着皇上那日说起时颇有思慕之意,所以特意用鹿尾绒毛搓成线缝制成一个燧囊,希望以此提醒宫中,虽然国库丰裕充盈,天心富庶安康,但后宫不应该养成太过奢靡的风气。越是平安富贵,越该不忘先人创下基业的苦心啊!''

皇帝眼中有赞许,亦闪过一抹感动:``皇后所言甚是,朕会将皇后所制燧囊随身佩戴,以表不忘祖宗辛苦,不忘根本。''

意欢看着皇帝亲手将皇后所做的燧囊佩在身上,淡淡一下:``也是巧了,臣妾本也做了个燧囊,如今看来,是不配送与皇上了。''

皇帝转脸看着她,带了几分疼惜与娇宠:``舒嫔没有旁的,就是气性大。''

意欢听了皇帝这句,从袖中取出一个黄地金花粉彩燧囊。如懿一看,亦不觉暗暗赞叹,那燧囊穿系黄绳,绳上有米珠、珊瑚珠装饰。器内施松石绿釉,外壁周边饰描金卷草、朵花及缠枝花纹。器腹正反两面有长方形开光,开光内粉彩绘西洋人物``进宝阁'',端的是华彩妙丽,映目生辉。

意欢清冷道:``皇上喜欢皇后娘娘的朴素无华,臣妾这个便实在是奢靡太过了,料来是入不了皇上的眼了。''她站起身,见廊下的铜缸里供着水,随手扔了进去道:``既然皇上不会喜欢,臣妾也不送给别人,宁可丢了就是了。''

皇后见她如此,亦不觉瞠目:``即便皇上不用,扔了岂不可惜?皇上,您实在是宠坏了舒嫔。''

意欢见皇后这样说,也无畏惧介怀之色,只是斜坐一旁,冷然不语。

皇帝抚掌笑道:``舒嫔便是这样的性子,不矫揉造作。虽然任性,但也直爽。''皇帝吩咐道:``李玉,去捡回来,替朕放在养心殿的书房里。这样精巧的东西,舒嫔一定费了不少心思,朕闲来细赏也是好的。''

意欢这才缓下脸来:``皇上说细赏的,可不许敷衍臣妾。''

皇后见二人取笑,心里不大好受,也不便多言,便换了姿势倚着,含笑道:``今儿内务府来问臣妾一桩事情,臣妾做不得主,正好问一问皇上。''

皇帝和声道:``你说。''

皇后慢声细语:``三月三上巳节,公主、福晋等内命妇都要入宫拜见。臣妾记得晞月为贵妃时,皇上都是让她接受内命妇拜见的。如今娴贵妃和纯贵妃已在去岁行过册封礼,是名正言顺的贵妃,是否也要入晞月当年一般接受内命妇拜见呢?''

皇帝沉吟片刻,缓声道:``晞月初封即是贵妃,与由妃嫔晋封贵妃者不同。所以,往后也不必让内命妇拜见贵妃了,只拜见你与太后即可。''

皇后眼中闪过一丝欣慰,更多的是一分得意:``那也是应该的,只娴贵妃别在意就好。''

``自然不会。皇上爱重慧贤皇贵妃,宫中人尽皆知,臣妾与纯贵妃又怎会不明事理呢。''如懿翩然起身,``时近黄昏,皇上若得闲,臣妾很想陪皇上去咸福宫做做,略显心意吧。''

皇帝起身,抚过皇后肩头,温声嘱咐:``你好生歇着,明日朕再来看你。''

皇帝行至长春宫外,意欢行了礼道:``皇上,嘉妃有孕三个月了,婉常在邀了臣妾去看她。''说罢便告退离去。

皇帝携了如懿的手并肩同行,良久,他方道:``朕方才不许你和纯贵妃接受命妇拜见,你别多心。''

如懿轻轻颔首,挽住皇帝的手臂道:``皇上,臣妾说过,不会多心。''

皇帝握住她挽着的手,低声道:``高斌是朕在前朝的重臣,哪怕慧贤皇贵妃过世,朕也不能不安抚高氏一族。皇后也是如此,她出身名门,伯父马齐历相三朝,名望夙重,更有老臣张延玉屡屡为皇后进言,朕必须保全皇后的颜面尊荣。''

风扑面,吹着斗篷上柔软的细毛,沙沙地打着面庞,偶尔一两根拂进眼中,酸酸的似要逼出泪来。如懿闭目一瞬,柔声道:``臣妾的家世比不得皇后和皇贵妃,臣妾都明白。''

皇帝的语气温柔沉沉:``这也是朕对着你可以纵情舒意的缘故。''他拢过她,替她挡着身前的寒风,``朕已经想好了,皇后有孕,今年三月的亲蚕礼,由你代替皇后前往西苑太液池北端的先蚕坛进行。''

如懿似有些不能置信:``天子亲耕南郊,皇后亲蚕北郊。臣妾怎能去行亲蚕礼?''

他微笑,目光中渐有和煦的暖意:``采桑亲蚕是天下织妇必须做的,皇后不便,妃子代行也是寻常。朕希望你去,也只有你去。''

心中有一阵暖融蔓延而上,仿佛阳光透过云层暖暖地裹住周身。她不是不明白皇帝对她的爱重,却未曾想到,皇帝对她如此爱重。她无言应答,只是握着他的手,将自己的手放进他的手心里。皇帝在她耳边轻言道:``朕知道你还是对皇后介怀,所以今日提起朕写诗悼念晞月的事。可是皇后有着身孕,下回别再这样气她了。''

如懿扑哧一笑:``皇上硬要这么说,臣妾只当自己这点小心思被皇上看穿了吧。''

\hypertarget{ux7b2cux5341ux4e09ux7ae0-ux62e9ux8def}{%
\chapter{第十三章 择路}\label{ux7b2cux5341ux4e09ux7ae0-ux62e9ux8def}}

一行人去得久了,皇后才缓缓沉下脸来,忧然道:``素心,皇上每到高晞月的忌辰,都要写诗悼念,是不是做给本宫看的?''

素心忙扶住皇后道:``怎么会呢?皇上不是说了,悼诗送去了皇贵妃母家,也是安慰高斌在前朝的辛苦。''

皇后咬着唇道:``可是嘉妃也有了身孕,皇上是不是常去看她?''

``没有没有。嘉妃比皇后娘娘晚一个月身孕,赶不上娘娘的,何况她的孩子怎么和娘娘比。娘娘万安,千万不要多思伤神。''

皇后咬着牙,忽然呻吟一声,捂着小腹道:``素心\ldots\ldots 素心\ldots\ldots 本宫有些不舒服,快去请齐太医进来,快去!''

齐鲁进来,一边搭脉一边摇头:``皇后娘娘又是为何动气?微臣说过,娘娘再不能忧思过虑了,否则,您伤的不只是自己,更是腹中的皇子啊。''

皇后呻吟着,竭力道:``本宫不生气!不生气!你,你快些烧艾,快!''

皇后这般保胎,中宫一直汤药不断。待到入了三月中,皇帝来后宫的时候逐渐少了。入春之后,京中大旱无雨,时日长久。这本是要春播的时候,滴雨未下,春耕无法照旧,到了秋日也会颗粒无收。京中若是收成大减,民心必定不稳。为此,皇帝忧心忡忡,不仅素食一月,更是斋戒沐浴,前往斋宫求雨。

后宫亦在如懿与绿筠携领之下,陪同太后在宝华殿祈福。可是偏偏清明都已经过去,还是晴日高照,一片厚云都没有。

这一日皇帝又在斋宫,如懿与绿筠陪着太后在宝华殿静坐,听着法师们诵经声四起,亦拨动念珠,一同吟诵。天已交子时,太后还未有离去之意,如懿与绿筠虽然困顿,但相互交换一个颜色,亦不敢动弹。

正默念间,赵一泰在门口绊了一脚,几乎是滚进殿内来的,满脸是笑,一迭声道:``恭喜太后,恭喜太后!''

太后倏然睁开眼,还未来得及问什么事,赵一泰一边说一边比画,激动地流下泪来:``太后,太后,中宫喜降麟儿啊!''

太后忙扶了绿筠的手起身,欣喜道:``是么?真的是皇子么?''

绿筠稍稍迟疑:``可是日子不对啊。皇后娘娘的身孕离八个月还有两天呢,怎么现在就生了呢?''

赵一泰道:``一个时辰前娘娘胎动发作,太医说怕是要生了,烧艾也没有用,只能催生。幸好一切平安,皇子立刻就生下来了。''

太后连连道:``去通知了皇上没有?上天庇佑,中宫生下嫡子。哀家赶紧去看看。''她扶过福珈的手,一边走一边叮嘱赵一泰,``皇后是早产,虽然母子平安,但比得悉心照料。''

如懿与绿筠哪敢耽搁,赶紧也跟随了去,才走出宝华殿,忽然听得雷声隐隐,空气中夹带着潮湿的水汽,竟然快要下雨了。

如懿浅笑道:``真是菩萨显灵,今日四月初八是佛祖诞辰,又逢喜雨降临,皇后的孩子,来得真是有福气。''

绿筠伸出手,接住空中偶尔落下的小水滴,似笑非笑道:``是啊。中宫有了嫡子,咱们的孩子终究只是庶子罢了。嫡庶之差,何止天渊之别啊。难怪老天爷都要下雨庆贺呢。''

皇帝对于嫡出的皇七子喜爱异常,亲自取名为永琮。琮为祭地的礼器,又有承兆宗业之意,寄托了皇帝无限厚望。永琮出生当日正逢亢旱之后大沛甘霖,喜雨如注,又值佛祖诞辰的四月初八。这样万事吉祥,皇帝更是大喜过望,挥笔庆贺爱子的诞生,写下《浴佛日复雨因题》:

``九龙喷水梵函传,疑似今思信有焉。已看黍田沾沃若,更欣树壁庆居然。人情静验咸和豫,天意钦承倍惕乾。额手但知丰是瑞,颐祈岁岁结为缘。''

待到皇七子满月之日,皇帝更是亲口嘉许:``此子性成夙慧,歧嶷表异,出自正嫡,聪颖殊常,乃朕诸子中最聪慧灵秀者。''

皇帝早有六子,除端慧太子早夭,诸子一向平分春色。然而七阿哥永琮的殊宠,硬生生将其余几位皇子都比了下去。连三个月后玉妍的八阿哥永璇出生,皇帝亦不过淡淡的,全副心思都用在了永琮身上。只可惜永琮不足八月出生,体质格外虚弱,听不得一点动静响动,早晚便是大哭,又常染风寒,自幼养在襁褓中,便是一半奶水一半汤药地喂养着,不可谓不经心。而皇后因生产艰辛,身子也大不如前,畏热畏寒,经不得半点辛苦劳动。如此,皇帝便把协理六宫的事交给了如懿,由她慢慢料理。

玉妍尚在月中,眼见永璇并不十分得皇帝宠爱,不免郁郁。这一日恰逢八阿哥满月,皇帝不过照着宫例赏赐,玉妍私下便怨道:``七阿哥不过比本宫的八阿哥早出生三个月,皇上就为他大赦天下,本宫的八阿哥还是足月生的呢,哪像七阿哥那么病猫似的,皇上却偏喜欢那病秧子。''

丽心怯怯劝道:``小主别生气了。奴婢听外头的奴才们说,咱们八阿哥是七月十五中元鬼节生的,七阿哥是四月初八佛祖诞辰生的,一佛一鬼,命数差了许多,难怪皇上不喜八阿哥呢。''

玉妍气得脸色铁青:``这样的昏话旁人为了奉承皇后和七阿哥说说也罢了,也值得你放到咱们自己宫里来说。本宫偏不信了,本宫这么健壮的儿子,会活不过那个小病秧子。''

忌讳的话可说不得。''玉妍说完,自己也有些后怕,正见嬿婉蝎蝎螫螫地立在门外要送水进来,便气不打一处来。这些年她本已厌倦了欺辱嬿婉,不过是偶然想起来才打骂一阵,今日在气头上见了她,便喝道:``樱儿,你站在那里做什么?进来!''

嬿婉见玉妍这般,吓得腿脚一缩,却不敢不进去。玉妍更是气恼,伸手把一盆热水推在她身上,没头没脑地打了起来。嬿婉死死地抱着脑袋,想要哭,却再没眼泪落下来。

京中干热,天气越发炎炎难耐。皇帝的意思,本是要去圆明园消暑的,奈何永琮和皇后的身子七病八灾的总没个消停,所以太后吩咐下来,今夏只在宫中避暑,另嘱咐了内务府多多供应冰块风轮,以抵挡京城苦热。

晨起时如懿便觉得眼前金光一片,知是朝阳流火,从宝檐琉瓦上反射了过来,亮得刺目。帘外蝉鸣续续的一声半声,传到殿中更显得静。她半阖上眼,蒙眬间又欲睡去。那声音直叫人昏昏欲睡,却不能再睡。她叹了口气,伸手一摸,旁边的床上是空的,知道皇帝是悄悄上早朝去了,并不肯惊动她。她想着昨夜一晌贪欢,却是有些疲累了,只顾着自己贪睡,脸上便不自觉地烫了起来。

惢心发觉她醒了,忙招手示意侍女们进来伺候洗漱。捧着金盆栉巾的侍女们鱼贯而入,并无一点声息。如懿摸了摸鬓边颈上,果然有些汗津津的,便道:``如今睡过这苇簟有些热,等下换成青竹玉簟吧。都过了中秋,居然还这么热。''

惢心笑生生道:``前儿皇上正赏了一席蕲州产的竹簟,说是小主怕热,睡着最蕴静清凉了,小主正好换上试试。''

如懿不觉含了一缕浅笑:``从前欧阳修说`蕲州织成双水纹,莹净冷滑无尘埃',说的便是蕲州的竹簟了。难为皇上惦记。''

惢心笑得俏皮:``皇上不惦记咱们宫里,还能惦记哪里呢?''

如懿脸上飞红,伸手作势拍了她一下,便道:``八阿哥满月了,这几日天天抱去皇后宫里请安呢。皇后总说要咱们一起去,也沾沾儿孙气,等下用完早膳,咱们早些过去吧。''

惢心伺候着她洗漱完了,便道:``皇后只说七阿哥和八阿哥的岁数相近,只差了三个月,好就个伴儿。皇后娘娘也真看得起嘉妃。''

如懿看她一眼:``别说这种话,我倒想着嬿婉在嘉妃宫里好几年了,一直不能拉拔她出来,如今趁着她带八阿哥忙碌,得想个什么法子带出来才好。''

惢心道:``这件事小主心里也过了好几年了,总替凌云彻和嬿婉想着,也难为他们彼此一片痴心了。''

于是趁着晨凉,如懿便携了惢心和菱枝往皇后宫中去。天气燠闷,走不上几步便微微生了汗意,便是绿荫垂地之处,也是一丝风也没有,只看着万千杨柳的绿丝绦安静垂下,纹丝不动。

园中阒然,只闻蝉语切切,暑光漫热。

如懿披了一件新制的浅妃红双丝绫旗袍,隐隐的花纹绣得繁复却不张扬。只举手投足微见花纹起伏。发髻上亦不过两串鎏金凤衔着的珍珠步摇,在日光下闪烁微粉珠光,投射在她白腻柔婉的脖颈上,到有一种雨洗桃花的简淡嫣然。

如懿正立着,却见前头玉妍过来,面如白玉,黛青画眉,鬓黑光净,愈衬光华满身,浑不似刚出月子的模样。尚未走进,如懿已闻得玉妍满身芳香郁渥,脂粉香泽深透肌理,妍艳无比。玉妍穿着一身耀目的玫瑰红串珠银团绣球夏衣,袖口和领口处打着密密的银线珠络,衣上满满地绣着青莲紫镶银边的玉兰花,碧海蓝镶银线花叶的大朵绣球,配着她头上闪耀烁目的缠丝点翠金饰并一对红翡滴珠凤头钗,整个人金宝锦绣,迷离而惊艳。

如懿看着她,微微笑道:``嘉妃一过来,真是迷得连眼睛都睁不开了。''

玉妍施了一礼:``娴贵妃万安。''乳母亦抱着永璇半蹲下身,口中道:``永璇给娴贵妃请安,娴贵妃万福金安。''

如懿逗了逗永璇,笑道:``满月了,八阿哥长得越发好了。''

玉妍粉面含春,一双凤眼秋水飞扬,恨不得插翅飞上天去:``方才娴贵妃说我迷着您的眼睛了,其实娴贵妃哪里知道我这做额娘的高兴。咱们八阿哥到了。''

说到底,不过讥讽她没有孩子罢了。多年下来,这样的讥讽她也听得惯了,如懿淡淡道:``是啊。七阿哥佛祖诞辰日出生的,八阿哥是中元节,果然都是赶着节庆出生的好兄弟。''

玉妍立时变色,却也不敢发作,只能忍耐着道:``只要能生得出来,便是公主都是好的,何况是阿哥呢。''

如懿笑了笑,悠然转首,果然见嬿婉立在七八个侍女的最后,神色怯怯的,恨不能把自己变成一个隐形人。玉妍嘴角一撇,喝道:``樱儿!''

嬿婉忙怯生生走上来:``奴婢在。''

玉妍伸出雪白的手掌便是一个耳光,没好气道:``蠢笨丫头,天气这么热,也不知道跟在本宫后头扇扇子,一味地好吃懒做,仗着你这副贱格儿,想作死么?''

嬿婉惯了挨了打,也不敢哭,只木着脸拼命替玉妍扇着扇子。

如懿听着她指桑骂槐,脸上的笑影薄薄的:``这些年了,嘉妃还是这么个火爆脾气,动不动就拿丫头撒气。旁的也就罢了,本宫只心疼你那几根水葱儿似的指甲,落在皮肉上仔细伤着。''

玉妍扬着手里的绢子,笑吟吟托着腮道:``原来娴贵妃是心疼我呀!我只当娴贵妃只心疼那些贱皮贱肉的奴才呢,一味地爱和她们投趣儿。''她娇声地笑,那笑声像是薄薄的瓷片,沙沙地刮着人的耳朵。

却听一个声音在后头朗然道:``天气这么闷热,怎么嘉妃在这儿笑得那么高兴?''

玉妍闻声转首,见是皇帝,笑容一下从唇边满出来,绽成一朵丰艳的花。她使一个眼色,丽心她们会意地将嬿婉遮在后头。玉妍迎上前,娇怯怯行了一礼,道:``皇上万福,臣妾在跟娴贵妃说笑话呢。''

皇帝换下了朝服,穿着一身银青色团福纱袍,那袍子本就轻薄如蝉翼,皇帝只在腰间系了一根明黄带子,垂着一块海东青白玉佩,越发显得长身玉立,丰神俊朗。

如懿亦福了一福:``皇上万安,这个时候刚下了朝,是要去看七阿哥?''

皇帝一脸牵挂爱怜:``永琮乖巧可爱,朕一日不见,便有些惦记着。刚巧宝华殿送了些祈福的经幡来,朕叫李玉去打点了,都为永琮求得安康才好。''

玉妍笑得灿若春花,身影轻巧一挤,陪到皇帝身边:``那便最好了,永璇也想着哥哥,臣妾正要陪他去皇后娘娘宫中呢。''

皇帝笑着逗了逗乳母怀中永璇,正要迈步,只听得后面轻轻一声呻吟,便蹙了蹙眉:``什么声音?''

随侍皇帝的进忠眼尖,忙道:``皇上,好像是个宫女挨了打,脸上受不住疼呢。''

玉妍脸上便有些慌张,忙挡着皇帝的视线,笑道:``宫女伺候人哪有不挨打的,臣妾瞧着她就是矫情,在皇上跟前哼唧。''

皇帝看她一眼,漫然道:``朕与皇后一向都宽和待下,从没听说过打人打得宫女都忍不住疼的。进忠,你带上来给朕瞧瞧。''

进忠往跟着的宫人里头一瞧,一眼就看到了脸上带伤的嬿婉,便拉了她上来。嬿婉仿佛一只风雨中饱受惊吓的燕子,瑟缩着身体,显得格外弱质孱孱。

皇帝凝神瞧她,只见嬿婉素净的一张清水面孔,脂粉不施,雅致得好比一朵小小的临风半开的栀子花。她乌鸦鸦的一头好头发,缠着密密的深青色头绳,一身绿湖纱袍,衣裳间一应绣花点缀俱无,却比得肤白净色,容质玉曜。这样简单的打扮,静若碧水,仿佛映着身边的柳色青青,娉婷生色,比得她身边珠光宝气的玉妍无端地俗艳了下去。

皇帝的目光如细细透明的蚕丝,在嬿婉身上黏了片刻。进忠何等乖觉,忙笑道:``娴贵妃娘娘,奴才说句不知轻重的话,这宫女儿倒有福气,长得有几分像小主年轻时的样子呢。只是无论怎么比,却比不上娘娘端贵之姿。''

皇帝听进忠这般说,便向着如懿道:``这丫头是有三分像你年轻时的样子。又穿着青色,活脱脱你刚嫁入潜邸时的模样。偏你那时也爱穿青色,有叫青樱。''

如懿微微一笑,淡淡道:``樱儿是宫女,也喜欢穿青色。''

``樱儿?''皇帝皱眉,``你叫樱儿?''

是良时嬿婉的嬿婉。樱儿是嘉妃娘娘赏的名字,许是因为嘉妃娘娘喜欢樱花呢。''她说到``嘉妃''二字,又是一脸惊恐的模样,越发往后退了一步。

玉妍见她这般不胜娇弱,越发像自己苛待了她似的,不觉又惊又气:``本宫不过是因为你蠢笨不会伺候,才轻轻打了你一下,你平日做出这副样子来做什么?''

如懿本也惊异嬿婉在皇帝面前这般口舌伶俐,见玉妍动怒,便不动声色,只闲闲摇着手中的轻罗菱扇,悠然望着天际。

皇帝细看嬿婉脸上,尚且留着五个通红的指印,知道玉妍下手重了。皇帝素来不喜嫔妃们苛待宫女,便有些不悦:``宫女好歹都是八旗出身,不比太监是汉人。这样动不动就打骂,也失了自己的体面。''他眉心蹙起更深,仿若一条川字虬曲,``你说樱儿是嘉妃给你改的名字?''

嬿婉捂着受伤的半边脸,手臂上的衣袖宽大,一分分滑落,露出带着青紫伤痕的胳膊,她怯生生道:``那是娘娘对奴婢的厚爱。''

皇帝看着嬿婉手臂上的伤痕,多半是旧伤,也有几道新痕,心中愈加有数,直逼得她娇媚的面庞变得如霜雪般泛白,``你明知道青樱是娴贵妃从前的闺名,还让你的宫女改这个名字,穿青色,是在是僭越犯上。''

如懿以扇障面,柔声道:``皇上,或许嘉妃是无心的。''

皇帝嘴角扬起,眼底却殊无笑意:``嘉妃倒真是无心,也厚爱这个丫头。既然嘉妃这么厚爱,朕也厚爱她一回。''他看着嬿婉,严重多了几分温柔神色,``以后不许叫樱儿了,就改回你的本名嬿婉。你读过书,知道良时燕婉?''

嬿婉忙道:``阿玛在时,教过奴婢一点。''

``你阿玛是\ldots\ldots{}''

嬿婉有些羞赧,亦带了几分愧色:``奴婢的阿玛曾是正黄旗汉军旗包衣内管领清泰\ldots\ldots 后来犯了事,奴婢全家都被贬为奴了。''

皇帝点头道:``做官的难免有些起落,到底还算好人家的女儿。朕瞧着你眼熟,你多大了?''

嬿婉越发羞怯,低眉垂首道:``皇上忘了,几年前奴婢是在纯贵妃宫里伺候大阿哥的,那时皇上就和奴婢说过话。奴婢如今已经二十二了。''

如懿听着皇帝这般问,心底隐隐不安,忙笑道:``这样好的年华,指出去配个侍卫也是不错的。''

皇帝笑而不语,片刻道:``如懿,朕瞧她的样子有些像你年轻的时候,便留在朕身边跟你做个伴儿吧。''

如懿蓦地想起凌云彻,心口陡然一沉,勉强笑道:``皇上也是,也不问问嬿婉自己的意思,哪能让臣妾跟您就做主了呢。''

如懿含笑看着嬿婉,亲切和婉到了极处,可眼底的意思却再分明不过。她若不愿意,大可自己退却,求得指婚。然后嬿婉清甜一笑,已经盈盈拜倒:``奴婢自进宫中,一切都是皇上的。但凭皇上做主,奴婢只愿侍奉皇上左右便可。''

如懿心头一阵冰凉,从嬿婉的眼神中,已经探知凌云彻不可挽回的情缘。

皇帝抚掌笑道:``那便好。进忠,传朕的旨意,封宫人魏嬿婉为官女子,赐居永寿宫,今夜侍寝。''他挽过如懿的手,``走,咱们去看皇后和永琮。''

如懿唇边带着笑,在皇帝不经意的时候回头望去,深深地剜了嬿婉一眼,却在绿柳依依之畔无奈地发觉,嬿婉的美,其实是凌云彻一生所无法掌握的。

\hypertarget{ux7b2cux5341ux56dbux7ae0-ux8309ux5fc3}{%
\chapter{第十四章 茉心}\label{ux7b2cux5341ux56dbux7ae0-ux8309ux5fc3}}

凌云彻得知消息之时,一颗心几乎都有迸裂了。他借着戌时三刻交班后的空闲,在长街候到了正扶着侍女春蝉与澜翠预备前往养心殿侍寝的嬿婉。

嬿婉正低声吩咐春蝉:``方才内务府送来的一些赏赐,你得空便挑些好的去打点了养心殿的进忠。我告诉过他,这件事若不成,我便宁可嫁了他做对食。若是成了,便拿一辈子的荣华谢他。这一遭,我总算是赌赢了。''

嬿婉犹有余悸,春蝉一壁答应着,一壁道:``幸好小主赢了,否则可要怎么好?宫里跟太监对食的,有一个莲心也够怕人了。''

``若不这样,进忠怎肯帮我?''嬿婉抚着心口,``万幸!万幸!若是不成,我便只有一头撞死,省得受莲心那般苦楚。''

春蝉忙安慰道:``不枉奴婢和澜翠跟着小主。小主虽然在嘉妃那儿受苦,仍不忘记挂提携花房的奴婢和澜翠。奴婢一定忠心小主,至死不忘。如今小主的前程已经到了,只要今夜侍寝后皇上喜欢,封了答应,那便是真正的小主了。''

二人正密密说着,犹是惊喜交加。嬿婉忽一抬头,见到云彻痴立在长街转角处,心中栗栗一颤,极力维持着沉静的面容,嘱咐侍女们退下稍候。嬿婉已经换了官女子的装束,浅浅的淡橘色无纹锦袍,镶着寸阔的深一色旋波纹缎边,既是吉祥的意思,又是她双十年华的秀美,映着发髻间的星点银饰与脆薄绢花,愈显出尘之美。

嬿婉倒不意外,只坦然望着他:``我要去侍寝了,能与你说话的时间并不多。你想说什么,便一并说了吧。''

云彻一路疾奔而来,胸口塞了无数疑问,然而见了她如此淡然自若的神情,不知怎的,只化作了冰凉一片,寒着自己的心。

片刻,他才能从喉咙里挤出声音来:``是不是有人逼你?''

嬿婉一双明眸清亮无波:``嘉妃与娴贵妃当时都在场,她们都看见的,是我自愿的。''

云彻不信地摇头:``为什么?为什么你要去做别人的妾室?''

嬿婉不可思议地看着他:``我为什么不愿意?做妾室与妻房,在乎嫁的是谁。做皇上的妾室,远比做天下任何人的妻房都尊贵。你难道不明白么?''

云彻如遭重击,怔怔看着她:``你那时在花房受苦,回来说愿意再和我在一起,那些话是不是都是骗我的?''

嬿婉摇头,坦然而诚实:``当然不是。人在任何境遇中都想求得最好的出路。那时嫁与你,便是我最好的前途,自然是最真挚的想法,甚至一直被困在嘉妃宫里当奴婢羞辱的时候,我都一直是想着的。''

云彻郁郁垂首,两颊失去血色,自嘲道:``原来,你不过当我是一条出路!''

嬿婉扬起如繁星微点的眸,在漆黑夜里有冷冽的光:``当然,难不成你会喜欢一块绊脚石么?可惜啊,我如今才明白,我当时的愿望是多么微不足道。我被困在嘉妃宫中被她欺凌羞辱的那几年,我没有一天不盼望着可以被指婚给你,逃出这鬼地方。可我渐渐发现,原来除了我自己,没有人可以救我,没有人可以帮我。既然如此,我为什么不能寻一条更好的出路帮一帮自己呢?''

云彻看着地上她被拉得悠长的影子,惘然地摇头:``嬿婉,你变了。''

``我是包衣内管领家的格格,可我阿玛一朝失势,我们便只能当奴才,只能做人下人。我连选秀的机会都被剥夺,只能做一个最卑贱的宫女,任人欺辱,遭人白眼。这样的日子,我一天都不想过下去了。我只想过得好一点,也做一回人上人,这辈子让我的家人也得些脸面,不用再活得那么卑微。''她的眼底闪过晶亮的泪痕,很快擦了干净,``所以,我从未有错!''

凌云彻无力道:``可你跟我在一起,我也会努力上进,我\ldots\ldots{}''

嬿婉不耐地打断:``你再上进,也不过是个侍卫。咱们的儿孙也不过是个奴才。为什么?我要靠着别人得到一点点微薄的荣耀,而不能凭我自己的力量得到更多。我还年轻,我尚有美貌,如果凭自己的一切能换回更多的荣耀,我为何不肯?上一次,我已经失去过机会,失去过接近皇上的最好机会。这一次已成定局,我再不能、也不会错过了。''

凌云彻看着她,只觉得自己满腔悲伤,却被这小小女子的一言一语,打得只剩下沉沉碎裂般的痛意。

嬿婉沉醉地抚摸着朱红色的宫墙,低低道:``别人侍寝都是坐凤鸾春恩车,你知道我为什么要自己走过去么?''她见云彻只是不语,越发低柔道,``我做了那么多年奴婢,一直用脚用膝盖在行走。我很想在我第一天侍寝的日子,用自己的脚去丈量一下,从永寿宫到养心殿有多远,从一个卑贱的宫婢到来日的宠妃,这条路还有多远。''

云彻听得出她口中的坚决之意,这样美丽而娇柔的嬿婉,是那样熟悉,却已然很陌生很陌生了。

云彻苦苦劝道:``你只想着凭自己的年轻貌美得到一时宠眷,有没有想过有一日失去时有多么痛苦?便是聪慧如娴贵妃,也有冷宫饱受折磨的一日,你便不怕自己的来日走得辛苦崎岖,不能回头?''

嬿婉挽起袖口的绸缎,爱惜地摩挲着道:``我在四执库时,成日里看到那么好的衣缎,却只能辛苦熨烫,自知无福也不配穿在身上。如今你瞧,我穿着多好看。已经穿在身上的衣裳,我如何还能脱下来?便是要死,我也得穿着它们死。''

她的声音极轻婉,仿如往日在他耳畔的呢喃低语,却是如今划下楚河汉界的分明与犀利。他忍住喉头的哽咽,沉声道:``你自己选定的路,自己好好往前走吧。但愿你一路顺畅,永无后悔之日。''

嬿婉幽幽一笑:``只要你不来阻碍我的前路,我一定会走得很远很好。自然了,你还是与我一同长大的云彻哥哥,我永远都会记得。''

她的笑容转瞬即逝,唤过春蝉与澜翠道:``我们去养心殿吧。''她的眸色中带了一丝凛冽的威严,``凌侍卫,你可以退下了。''

云彻茫然地目视于她,仍由痛楚至麻木的躯体半跪而下,一字一字缓缓吐出:``微臣,恭送魏小主。''

他跪在石板上,低头看着石板上镂刻的``春恩常在''的花纹,每一个都是吉祥如意的好口彩,每一个,都是送了嬿婉一路远去的灿烂前程。

他的心口一阵阵绞痛,空得好像被蛀蚀着一般,无知无觉地落下泪来。夏夜的风带着灼热的暑气,一点一点逼住了他,也裹得他失去了力气,完全不能动弹。也不知过了多久,一方淡青色绣着雪白樱花的绢子飘在他眼前。

他见过这方绢子,喃喃道:``娴贵妃娘娘。''

如懿披着淡淡青色竹叶纹的雪絮绛纱披风,盈盈站在月光皎洁中。她的话语并无过多的安慰:``擦掉你的眼泪。你要记住,永远不要为不会回头的人流半滴眼泪,因为太不值得。''

他紧紧地攥着那方绢子,似要以此来发泄自己无可发泄的痛楚。如懿轻声道:``我曾经给过嬿婉机会,希望她能给自己一条别的出路,可她没有。既然这条路是她自己执意选择的,那么,就由着她走下去吧。''

云彻深吸一口气:``是。''

如懿笑容澹澹,带着一分懂得的哀伤:``只是这一次,你不要再像上回一般整天喝酒意志消沉了。那样的傻事,做过一次够了。''

云彻的神志仿佛清醒了许多:``是。为同一个人伤心两次,是不值得。''

如懿赞赏地看他一眼:``这就对了。连嬿婉都知道要为自己争气,何况你一个大男人!你也该为自己好好打算了。''

云彻猛地一凛:``但凭娴贵妃娘娘吩咐。''

如懿轻轻一笑:``御前,如何?''

皇后用完早膳,便着紧去看永琮。永琮还是那样瘦小,睡在乳母怀中,并不太安宁。皇后心疼不已,自己抱着哄了片刻,乳母春娘笑道:``到底七阿哥和额娘最亲,皇后娘娘一抱,他就睡得香了。''

皇后娘娘笑道:``外头给你备了一碗不加盐的肘子,快去喝了。七阿哥喜欢喝你的奶水,这是你的福气。''

春娘答应着下去了。皇后抱着怀中的儿子,怎么都看不够爱不够。正巧素心进来道:``娘娘,方才李玉来传旨,皇上说咱们七阿哥自幼多些病痛,所以打算九月初一与娘娘前往隆兴寺西侧的行宫小住,也好往隆兴寺祈福保佑七阿哥平安。''

皇后喜道:``隆兴寺是千年古刹,寺里供奉的正定大菩萨据说十分灵验,康熙爷在世的时候也多次去参拜呢。皇上真是有心。''

素心亦高兴:``可不是,皇上多疼爱咱们七阿哥,一日不见都舍不得呢。''她想了想,微微皱眉,``还有一事。皇上昨夜临幸了魏官女子,就是嘉妃身边的樱儿,今早起来就晋了答应。''

皇后的笑容瞬间凝住:``樱儿!怎么嘉妃也不得力,一个小丫头也料理不好。''

素心忙赔笑道:``那丫头果然是狐媚东西!嘉妃又有两个阿哥,一时疏忽了也是有的。不过话说回来,到底也只是个答应,能有什么呢!''

皇后稍稍释然:``也是。嘉妃虽然还算得力,但有了两个儿子,也得防着她来日不安分。也好,多个魏嬿婉,她也有得闹心。本宫正好得些空闲,好好养好永琮才是要紧。''

素心诺诺听着,眼波一转,便若无其事陪着皇后一起哄永琮了。

如懿再次看到茉心的时候,已经是乾隆十二年的冬天。这一年京中痘疫四起,秋燥冬暖,略无霜雪,河井枯涸。自九月间起,痘疫流行,自河北蔓延至京郊,又波及京师,十不救五,小儿之殇,日以百计。

宫中因着从前顺治爷福临死于痘疫,连圣祖康熙幼时也得过,所以格外惶恐。皇帝除了忙于前朝痘疫之事,尤其嘱咐阿哥所将各位公主、阿哥都抱到生母或养母宫中养育,小心避痘。宫中供奉了痘神娘娘,为过春节所挂的春联、门神、彩灯全被撤下,同时谕令全国及宫中``毋炒豆、毋点灯、毋泼水'',并颁诏大赦天下。一时之间,宫中人人自危,大为惶恐。

永琮体弱多病,皇后也格外防备,小心谨慎看顾。长春宫中一律不许生人出入,生怕沾染了痘疫。

而茉心,便是在那个时候求见如懿的。彼时如懿正与海兰闲话宫中痘疫之事,连一应的乳母保姆都不甚信任,一切都必得自己亲自过手,她听得惢心小心翼翼提起``茉心''这个名字,不由得含了几分诧异之色:``茉心不是伺候慧贤皇贵妃的贴身丫头么?听说慧贤皇贵妃死前放心不下他,将她指婚给了守顺贞门的一个侍卫,之后便在古董房当差。她忽然要见咱们做什么?''

永琪活泼地笑着,越发逗得海兰笑个不止,拿着拨浪鼓哄了永琪玩,漫不经心道:``如今皇上只宠着魏常在,眼见着年前必定是要封贵人了。咱们得闲不用伴驾,见一见茉心便又怎么了。''

如懿沉默片刻,将永琪抱到乳母怀中,随着惢心起身向外去。见到茉心的时候,是在古董房边一间昏暗的小庑房里,想是她平日当值时所住。茉心一副妇人装束,簪着白绒团花,枯哑的头发用一支素银平簪紧紧压住。她眼睛通红,人也木木的,像是没有活气似的,哪还有半分像从前宠婢模样。

如懿和海兰见茉心这副打扮,知道她是家中出了丧事,便道:``家里怎么了?是不是有为难的地方?''

茉心离她们俩远远的,缩在墙角一隅,戚然叹道:``奴婢的丈夫殁了,奴婢今日是过来替他收拾遗物的。''

如懿叹口气:``惢心,备下五十两银子给茉心,就当给她丈夫操办后事。''

惢心答应一声:``那奴婢回宫去取。''

茉心惨然一笑:``娴贵妃娘娘,难为你还肯给些赏赐,倒不计较奴婢曾是伺候慧贤皇贵妃的人。''

窗外寒气犹冽,庑房里并不如嫔妃所居的宫室一般和暖春洋。如懿远远立在茉心身前,静静听着,心中忽然有一阵短暂的心安。与晞月十数年的争宠怄气,是落在宫墙缝里的尘灰,抠不出,抹不去,只能任它停留成时光柔和的折痕。当这些曾经轻狂的片段从如懿的回忆中慢慢剥离而出时,她不胜欷歔,然而那欷歔也是属于胜利者的活着的绮想。毕竟如今活着的人,是她自己。所以,她凝望茉心的目光疏远而冷淡,却不失一缕悲悯之色:``所谓计较,是对活着的人而言。斯人已逝,前尘往事还有什么放不下的。何况你只是慧贤皇贵妃的侍婢而已,何必再与你有所纠葛?''

``那么奴婢来找娴贵妃,果然没有错。''茉心俯身一拜,``从前奴婢多有不敬,这一拜算是还了。''她微微一笑,叩首道:``只是娴贵妃既然赏赐,五十两银子怎么够?两个人的丧事,要给也是一百两了。''

如懿的眉心细细地拧起,打量着茉心道:``这话怎么说?''

茉心的脸是萎黄的花瓣的颜色,有慢慢颓败的迹象。她惨笑道:``奴婢的丈夫死于痘疫,奴婢服侍了他这些天,恐怕也逃不了了。昨日早上起来,已有呕吐、头疼的症状,今天手臂上发现了两颗红疹子。所以,两位娘娘,奴婢离你们那么远。''

如懿听得``痘疫''二字,心下一阵紧缩,几乎是下意识地退了一步。海兰紧紧依在她身畔,勉强镇静道:``你都得了痘疫,还要见本宫和娴贵妃,是要让我们染上痘疫,好让你替慧贤皇贵妃报仇么?''

茉心眼中闪过一丝雪亮的恨意,摇头道:``奴婢知道,慧贤皇贵妃死不瞑目,最恨的人是谁。慧贤皇贵妃临死前连话都说不出来了,还是死死盯着奴婢,奴婢知道,她是要奴婢不要放过那个佛口蛇心得人!''

如懿凝视她片刻,摇头道:``你都这样了,还想着这些做什么?''

茉心呵呵笑着,干枯的唇微微张阖:``就是因为奴婢到了这个地步了,才终于有了办法。''她笑起来露出森森的白牙,``慧贤皇贵妃死前,奴婢就被指了一个侍卫嫁了,为的就是还能留在宫里好寻个机会。可奴婢身份低微,一点办法都没有。如今她连嫡子都生下来,这一生真是顺心遂意啊!可奴婢一直得得慧贤皇贵妃死前有多恨,奴婢答应过皇贵妃,一定会替她报仇雪恨。''

海兰不以为意地摇头,静静拨弄着手腕上的红玉髓琢花连理镯,如玉髓莹红通透如石榴籽一般,衬出她一双柔荑如凝脂皓玉:``长春宫禁卫森严,你进不去的。''她抬起头来,漫不经心地扫了一眼茉心,``你要本宫帮你?''

茉心点头道:``奴婢既然得了痘疫,法子反而多了。奴婢知道,娘娘和慧贤皇贵妃一样恨她。''

海兰盈然一笑:``你倒真是明白本宫的心思。''

如懿略想了想,背过身去,只留下华服高鬓的身影:``这件事,本宫不做。''海兰忙跟过去,语不传六耳,``姐姐,你忘了她是怎么害你的么?姐姐到如今都没有子息,就是她一手造成的。姐姐若怕脏了手,我来做便是。''

如懿的心忽然一颤,像是猝不及防地被狠狠抽了一鞭,伤口裂开的疼痛上又洒满了雪白的新盐。她握住海兰的手:``我做和你做有什么区别,咱们都别脏了这个手。''

海兰急切道:``姐姐是从冷宫里捞回一条命的人,不能有妇人之仁。''

如懿定定颔首:``不是妇人之仁。你和我都知道,她的这个儿子天生孱弱,活得艰难。再者,说句不怕报应的话,从前没有永琪,下什么手做什么事都没有后顾之忧。但如今\ldots\ldots{}''她摇头,``不是为了别人,只为永琪。我从前不懂,只为恨着一个人,便什么事都肯做。如今我和你都算是人母,这件事,不必做了。''

海兰犹不死心:``姐姐\ldots\ldots{}''

如懿摆一摆手,转身向茉心,决然道:``抱歉,本宫与愉妃都帮不了你。''她见茉心遽然变色,越加宁和道:``本宫知道自己无用,所以有心无力。''

如懿说罢,旋身便挽着海兰的手出来。她殷殷道:``咱们走吧。回去好好儿拿药水洗洗,免得染上痘疫。''

海兰犹不死心,低低道:``姐姐,咱们真的不做?''

如懿沉声道:``若在从前,我绝无二话。戳她的软肋,我心里痛快。可如今\ldots\ldots{}''

海兰的声音有些尖锐:``不只是为了永琪,姐姐也担心地位和尊荣受损,也怕皇上知道吧?从前咱们输得彻底,什么都不怕,如今得到愈多,瞻前顾后也多了。''海兰微微黯然,``姐姐,我真怕有一日,我们的顾虑太多,便只会束手无能了。''

\hypertarget{ux7b2cux5341ux4e94ux7ae0-ux751cux767d}{%
\chapter{第十五章 甜白}\label{ux7b2cux5341ux4e94ux7ae0-ux751cux767d}}

二人静静地站着,风声被两旁耸立的深墙挤得虎虎乱窜,发出呜呜咽咽的鸣声。如懿恻然转首,但见嬿婉携了侍女澜翠缓缓走来,大约是从养心殿出来。

嬿婉见了她们,忙福了福身,剪水双瞳清凌凌的,泛出由衷的欢喜殷切之情:``娴贵妃娘娘万福,愉妃娘娘万福。''

海兰见有人来,便欠身道:``姐姐,快到年下了,宫里事多,我先回去了。''

如懿端正容色,微微颔首。嬿婉走到如懿身前,楚楚的脸庞越加蕴满了自谦的神色:``大冷天的,娴贵妃娘娘怎么立在这儿,仔细着了风寒。''

如懿的客气中带着疏离:``有劳魏常在挂心,本宫正要回去。''说罢,她便径自要离开。嬿婉侧了侧身,却并无让她过去的意思,只道:``娴贵妃娘娘还是那么讨厌嫔妾么?''

如懿淡薄一笑:``常在这话,本宫却不懂了。''

嬿婉挥手示意澜翠走远,道:``娘娘一直以为嫔妾是攀龙附凤不念旧情之人,所以屡屡冷淡嫔妾,却不知嫔妾也有不得已的苦衷。''

``苦衷?''如懿拂了拂被风吹乱的鬓发,她扬起的唇角勾勒出不屑的弧线,长街猎猎的冷风冷不丁地掀起她玉色长袍,配着纽子上系的青碧流苏金累丝缀明珠香囊,越发如云后淡薄的日光,渺渺不可亲近,``你如何一步一步走来,本宫都是亲眼看着的,又何来苦衷二字?''

嬿婉银红色的袍角被风拂起,像一只想飞却飞不高的蝴蝶,颤动着翅膀:``嫔妾听说娴贵妃娘娘出身乌拉那拉氏家族,这个家族,既是荣耀,也是阴霾。想来娘娘当年在冷宫受苦的时候,一定不会忘却自己的家人,所以才奋发而起。嫔妾也是如此,像嫔妾这种出身,所受的种种白眼辛苦,娘娘这样的尊贵之人如何能够体会。但嫔妾不忘家族之心,与娘娘却是一样的。''

如懿默然叹息:``但是你终究辜负了一颗真心。''

嬿婉自嘲地笑笑:``像我们这种人,进了宫中之后,自身的荣耀便与家族的荣耀结为一体,一荣俱荣,一辱俱辱。尤其是嫔妾,既然父母族人不能为嫔妾带来任何荣耀,嫔妾就一定要让自己过得舒心适意。真心这样私己的东西,不能割舍也是要割舍的了。''

如懿紧了紧披风,漠然以对:``你自己选择的路,自己高兴就好。听说皇上打算封你为贵人了,恭喜!''

嬿婉欠了欠身:``但愿以后娘娘不要再鄙夷嫔妾就好。这句恭喜,嫔妾感激不尽。''

如懿径自离开,澜翠走进嬿婉,低声道:``小主何必要理会娴贵妃对您的态度,咱们与她也不想干。''

嬿婉轻笑,明媚的眼睛如同天上细细地月牙儿:``怎么不相干?皇后虽然生下了七阿哥,但身子坏了许多,很多时候都不能侍寝。而娴贵妃有协理六宫之权,我自然得格外小心些。''她看澜翠一眼,``对了,我让你去看看舒嫔一直用的是什么坐胎药,你看了没?''

``拿些舒嫔的坐胎药出来,马上送去太医院,请太医照样子配出一个来给小主服用。''

嬿婉颔首道:``快去!我到现在都没有身孕,哪怕皇上晋封,也不过是个小小贵人,何年何月才能熬到主位?宫里的坐胎药那么多,人人都在喝,只有舒嫔的是皇上亲自赏的,一定特别好!''

澜翠犹豫道:``可舒嫔每次侍寝之后都喝,一直都没怀孕啊。''

嬿婉有些不屑:``那是她福薄。叶赫那拉氏的族人本就不多,没福气延续下去也是有的。''她迟疑片刻,``不过你还是让人看看,是不是上好的坐胎药。''

澜翠答应着去了,嬿婉抚了抚平坦的肚子,饱含希望地长舒了口气。

三日后黄昏时分,李玉来传召如懿前往养心殿一起用晚膳。如懿更衣过后,换上烟霭紫的如意云纹锦袍,清雅的颜色,袖口不过是略深一色的折枝辛夷花纹样,搭着金丝薄烟翠绿缎狐皮坎肩,越发衬得容色多了一分温柔娇艳。

她扶着惢心的手下了软轿,才走到阶下,见云彻穿着养心殿最末等的侍卫服色,两颊冻得通红,一动不动守卫着。

在经过他时,如懿悄然低声:``辛苦。''

云彻微微一笑,甘之如饴:``微臣在御前做了这么久的侍卫,奈何出身寒微,只能如此,辜负娘娘期望了。''

如懿眼中有温情浮漾:``丈夫之志,用十年去实现也不算晚。忍得一时,才能一飞冲天。知道本宫为何一定要调你到御前么?''

``御前机会多,不必其他地方。''

如懿微笑,目光清和:``这只是其一。常常看着自己心爱的女人如何走到另一个男人跟前去,才能真正让你断了念头,磨砺心志。她无情,你更无情,才能无所畏惧。''

云彻懂得:``多些。雪后路滑,娘娘小心足下。''

如懿裹紧身上的孔雀纹大红羽缎披风,缓步入殿。暖桌上已经布好了热气腾腾的金丝菊炖野鸡锅子,如懿闻得香气,先笑道:``好香。''

皇帝起身拉住她手,一脸的亲密无间:``今儿晚膳都是你爱吃的菜,这芝麻青鱼脯制得极好,朕让他们试着做了十来次,只有这一次做出来的一点腥味也没有。菠菜和豆腐制成的金镶白玉版十分清甜,入口即融。尤其这道醉虾,融了虾子本身的鲜嫩,配上醇酒调味的甘芳,所以朕急急催促你来。''

如懿两靥盈盈,眉目澹澹含情:``今儿又不是什么大日子,好好儿的怎么备下了那么多臣妾爱吃的菜?且都是冬日难得的。''

因着从外头进来,她双手冰冷,皇帝捧着她手,轻轻呵气道:``外面可冷吧。今儿是腊月二十三,也算小年。朕想着快到年下了,你协理后宫忙碌了这些天,也给你松泛松泛,''他亦有几分自得,``如今天下富足,库仓串铜钱的草绳都烂了。你喜欢的东西即便难得,朕若想要取来,也不算难事。''

如懿心口暖洋洋的,握着皇帝的手,道:``那臣妾能谢皇上的,就是把这桌菜都吃了。''

如是,帝妃二人相对而坐,也不让人服侍,便自自在在动起筷子来。

皇帝看她贪吃了几口醉虾,甚是喜欢的样子,便高兴道:``虽然贪吃也慢些,到底里头是有酒的。咦?你怎么没喝几口酒就红了?''

如懿笑着摸了摸脸:``新描的眼妆,皇上喜欢么?''她且说且笑,如玉双颊上透出几许红晕,似初露的晚霞弥散,眉眼旁都化为淡淡的芙蓉浅红,更显得明眸灿若星子,顾盼蕴漾。

皇帝伸手轻轻抚摸:``如懿,朕希望你一直这样高兴。''

心跳得有点快,混着红罗轻炭暖融融的气息,将殿中沉水香的气息烘暖出来,徐缓地在空气里面迷漫着。如懿低下头,莞尔一笑,轻轻挠着他的手心,似小鱼轻啄。这般温存,直到有添酒的小太监步入,才稍稍中止。

李玉随后进来道:``皇上,上回您说要在年前晋封魏常在为贵人,叫内务府拟了封号来看,内务府已经拟了三个送来,想请皇上过目。''

皇帝微一颔首,李玉一拍手,内务府的小太监捧着一个红纹木盘子恭谨入内,上面放着洒金纸,分别写着三个大字:令、恪、睦。

皇帝扫了一眼,随口道:``后两个都俗。令,令,美好为令,这个字前人也未用过,便是这个令字吧。''

``令贵人?《诗经》中说`如圭如璋,令闻令望',是赞美如玉般美好之人。''如懿轻声念过,笑盈盈

觑着皇帝,``皇上似乎很喜欢她。''

皇帝静了须臾,眼底的笑意愈来愈浓,几乎笑得眸如弯月,含了几分促狭道:``如懿,你是吃醋么?''

如懿面上微微一红,转首不去看皇帝,故意有些怨怼:``皇上是取笑臣妾么?''

皇帝侧身靠近她,咬着她的耳垂低低道:``\,`如圭如璋,令闻令望'的下一句便是`凤凰于飞,翙翙其羽',乃指两情恩爱,共效于飞之乐。你是觉得朕过于宠爱魏氏了么?''

如懿嘟了一嘟嘴,面色愈红,极力自持道:``臣妾没有这样想,是皇上最爱多心,胡思乱想。''

``好吧,那便是朕胡思乱想。但即便是胡思乱想,也不会是魏氏,而是你。''皇帝捉过她白皙如凝脂的手背轻轻一吻,笑着道:``嬿婉有几分像年轻时的你,但青春虽好,却还失了一段成熟风韵,或许年长些会更好。''

听他娓娓说起那样情长的语句,不是不曾有一分心旌动摇,牵起往日的少年恩爱。然而如懿听完,轻轻啐了一口,便一笑置之:``皇上觉得合心意,那就嘱咐内务府去办吧。''她侧首吩咐侍奉皇帝的毓瑚,``把那甜白釉玉壶春香炉挪远些,里头点了龙涎香,香气太重影响进食。''

毓瑚忙答应着做。二人正说着闲话,只听闻外头细细尖尖的太监的嗓音轻巧道:``皇上,魏常在求见。''

太监的声音一贯尖细如丝,若非听惯,必然觉得扎耳。如懿抿嘴笑道:``说曹操曹操就到,魏常在来得好巧。''

皇帝的眼笑得如弯起的新月牙,闪烁着明亮的璀璨,吩咐道:``唤她进来,正好也在用膳,人多热闹些。''

外头厚厚的明黄重锦团福帘一扬,一个清婉女子莲步姗姗而入,彼时地上铺了厚厚的素红色销金绒毯,她的脚步极轻盈,落在地上寂然无声,牵动碧蓝闪银明霞缎长裙扬起浮波似的涟漪,连着洁白耳垂下挂着的二寸长的金坠子和鬓际的浮花银镀金嵌碧玺珠翠簪上垂落的寸许珍珠流苏微微轻颤,如点点光溢。因着年轻,连用的珠花也是那样明媚柔丽,粉红碧玺是盛开的花朵,红宝粒子是娇盈盈的花蕊,黄玉花苞生生待放,绿色碧玺作五瓣花叶。她的脸如天际的霞色,映着鬓边珠翠珊珊,真恍若一道轻霞柔柔撞入眼帘。

如懿心中微微一颤,无论皇帝如何说嬿婉失了成熟韵致,但青春之美,拱得她若一只骄傲的孔雀,那分清艳是那般肆无忌惮。

皇帝见了嬿婉便含笑,伸手示意她起身:``不必拘礼。外头天寒,你怎么来了?''

嬿婉娇怯怯道:``臣妾炖了一晌午的燕窝,听说皇上和贵妃娘娘正用膳,所以特意奉来给皇上和贵妃娘娘品尝。''

如懿如何不懂她话中之意,蕴了一丝浅浅的笑道:``魏常在的燕窝定是特意备下给皇上的,臣妾沾光了。魏常在来得正好,皇上正说起要给你贵人的位分呢,连封号都拟定了,圣旨一下便是令贵人了。''

嬿婉乍惊乍喜,掩不住唇角满溢的欢愉,连连欠身谢恩不已。皇帝欣赏着她娇媚喜色,亦十分满足。嬿婉脆脆道:``皇上刚有意晋封臣妾,臣妾也备了新制的燕窝,换了新巧的做法进献皇上,真算与皇上心意相通。''她说罢,睇了皇帝一眼,眼波悠悠荡荡,极是轻媚。皇帝看得心醉,嬿婉含了几分羞涩,并不与他目光相触,转首唤道:``澜翠,将我备下的燕窝奉上。''

澜翠喜孜孜从五角红纹食盒里小心翼翼捧出一碗燕窝细粉,柔声道:``臣妾家乡盛产绿豆制成的粉丝,家母额娘托人送了些进宫,原是小家子玩意儿,吃个新鲜罢了。臣妾早起用鸽蛋和金针丝煨了,再配三两燕窝炖制浇上,请皇上和贵妃试个新鲜。''

如懿望了那盏中一眼,细粉原近乎白色,那燕窝更是透明的白,一眼望去,白霜霜堆了满满一盏,几乎要盈了出来。如懿按住心底逸出的一丝诧异,面上淡淡地道:``三两燕窝,所费不少呢。''

澜翠在旁赔笑道:``小主早起便为这道点心费心,还怕皇上吃惯了御膳的菜色,吃说让皇上尝尝心意便是了。只要皇上喜欢,也不怕靡费什么。''

皇帝看了一眼,唇角的笑色越来越浓,几乎忍不住了,他转首看如懿道:``说到制菜,贵妃亦颇为拿手,这道燕窝细粉,贵妃怎么看?''

如懿看着满桌琳琅菜色,含了薄薄的笑色,语音清朗如珠倾落:``魏常在的燕窝细粉素白一碗,颜色倒颇清爽。''她顿一顿,看着喜不自胜的嬿婉,本不欲往下说,然而她想起嬿婉昔日对凌云彻的态度,忽然起了几分恶作剧之心,衔了笑意道:``燕窝贵物,原本不许轻用,如必定要用,先得用天泉滚水泡足,须巧手妇人在光下用银针挑去黑丝和细毛,一丝一缕都不得残余,以免损了滋味。若用嫩鸡、新摘菌子并上好火方三样汤滚之,火方则以金华产最佳,细细煨透后除去杂物,撇去油脂,只余清汤慢炖才是最佳。其次以蘑菇丝、笋尖丝、鲫鱼肚、野鸡嫩片炖汤与燕窝同煮亦可。民间常用肉丝、鸡丝夹杂其中,这是吃鸡丝、肉丝,口味浑杂,并非只吃燕窝之妙。如今常在妹妹用三两燕窝盖足碗面,与细粉混同,一眼望去如满碗白发,反不得其美味了。''

皇帝轻嗤道:``东西用得贵而足,但配制不当,真乃乞儿卖富,反露贫相。''他凝视如懿,笑道:``你善于美味,只是轻易不露真相,如今娓娓道来,可做御厨的师傅了。''

如懿婉然道:``臣妾卖弄了。本该洗手做羹汤侍奉夫君,只是有御厨专美,臣妾的微末技艺,算得什么。只是与魏常在一般,拿心意侍奉皇上罢了。''

皇帝似想起什么,欢喜之色如孩童一般:``朕记得你从前在潜邸时做过一道冬瓜燕窝,滋味甚佳。以去皮冬瓜之柔配燕窝之柔,以燕窝色泽之清入冬瓜之清,重用鸡汁、菌子汁熬足,入口清醇,一试难忘。''他颇为叹惋,``只是如今你不大肯做了。''

如懿摆首,含了一缕黠色:``偶尔一试,才能难忘。若是常常吃到,便也没什么稀罕了。而且臣妾多年不做已经手生,若做得不好,却连皇上记忆中的美味都不保,还是不做也罢。''

如懿的喜色与微嗔都分明落在眉梢眼角,二人一应一答,恍若寻常夫妻。嬿婉侍立在旁,听得如懿字字句句评说,脸早已窘得如煮透的虾子一般红熟。末了皇帝的话,更羞得她成了夹在满桌膳食中的那碗燕窝细粉,一分分尴尬地凉了下午。

还是澜翠悄悄碰了碰她的手臂,示意她赶紧告退。嬿婉竭尽全力挤出一个笑容,道:``皇上与贵妃娘娘用膳,臣妾偶感风寒,还是不陪着了,以免损及皇上与娘娘康健。''殿里暖洋如三春,她只觉得背上黏腻腻的全是汗水,吸住了薄而滑的云丝小衣,闷得透不过气来。皇帝正与如懿说话,只是草草点了点头,也不多理会。

嬿婉匆匆转身,仿佛一刻也待不住了似的,她转得太急,身子撞在了一旁的甜白釉暗花葡萄玉壶春香炉上,炉身一翻,里头的龙涎香洒出大半,殿中立时弥漫了甜腻香气,近乎窒闷。

皇帝不自觉地蹙了蹙眉,睨了嬿婉一眼,旋即向毓瑚道:``方才贵妃嘱咐你把香炉放远些,就是怕香气过于浓郁,影响进食的情绪。怎么你还是如此不当心?''

毓瑚忙跪下请罪,嬿婉听得皇帝有不悦之意,惴惴不安地欠身:``皇上恕罪,是臣妾不当心,碰翻了这白瓷香炉,不干毓瑚姑姑的事。''

皇帝微微瞠目,旋即失笑:``白瓷?这怎是白瓷?''他从容拂袖,细细道来:``这是甜白釉,乃前明永乐窑所产。甜白釉极莹润,白如凝脂,素犹积雪,几能照见人影,触目便有温柔甜净之感,故称甜白。其名贵难得,怎是寻常白瓷可比?''

寥寥数语,几如措手不及的耳光,打得嬿婉几乎站不住。嬿婉的身影微微一颤,好在澜翠在身后紧紧扶住了,她极力自持着颤颤请罪:``臣妾愚昧无知,还请皇上宽宥。''

皇帝摆一摆手,似乎不愿再多言:``依你出身所见,必不知此。罢了,跪安吧。''

皇帝叫臣子``跪安''乃是客气,若是对妃嫔这般说,便是不欲她多留眼前的意思了。嬿婉本是新封贵人之喜,此刻只觉足下无丝毫立锥之地,只得讪讪退出。

如懿望着她仓皇背影,又见宫人退下,方浅笑道:``皇上往日似乎很喜欢魏常在。''

皇帝淡淡含笑:``不过尔尔。只是宫人扰攘,总说魏常在因为像你而得宠,你喜欢么?''

如懿撇一撇嘴:``有什么可喜欢的?臣妾却不信这样的话。''

皇帝大笑:``啊!原来你觉得嬿婉不够美,所以不是因为像你年轻时而得朕欢心。''

如懿轻一旋身,半开玩笑:``因为臣妾不信人与人可相互替代,容貌与性情也不会重复。皇上喜欢魏常在,自然是有她不可取代的好处。''

皇帝笑着拧一拧她的脸:``如懿,那么,你也有你不可取代的好处。''

如懿斜睨他一眼,盈盈双眸几能滴出水来:``臣妾也知道,自己有十足十的坏处,旁人学也学不去。''

皇帝一牵她手,拥入怀中,咬着她耳垂笑道:``那朕来告诉你,你坏在哪儿?''

殿中,一色春意浓。

\hypertarget{ux7b2cux5341ux516dux7ae0-ux742eux788e}{%
\chapter{第十六章 琮碎}\label{ux7b2cux5341ux516dux7ae0-ux742eux788e}}

殿外朔风剧寒,如能蚀骨,嬿婉跌跌撞撞走到玉阶之下,只觉得浑身冷汗肆意,钻骨透心。澜翠慌不迭紧紧扶住了:``小主别在意。您费了半日心意,又冒着严寒送来,这份苦心皇上是知道的。''她见四下无人,低声抱怨道,``都怪娴贵妃,卖弄什么呀,也不过是个家道中落的货色!''

嬿婉死死地掐住澜翠的胳膊,硬着酸涨的脸哑声道:``不许胡说,原是我自己不得脸没见识罢了。娴贵妃家道中落,我不也是个破落户的出身么?''她咬紧了牙关,屏了半日,回首望着灯火通明的养心殿,一字一字着力道,``原本,是皇上给了我一丝希望,他对着我笑,告诉我可以凭自己改变门第命运,我却甜白釉也不识,连燕窝都做得粗俗,可不是自己没脸么?皇上没撤了晋封贵人的旨意,已算留了脸面了。''

澜翠忧心道:``那小主打算怎样?''

嬿婉忽地捏住澜翠的下巴,拧着她的面孔对着自己,哑声道:``澜翠,你仔细瞧,我的脸还在不在?我有没有变老,有没有变难看?''

澜翠见她神色狰厉,吓得一颗心突突乱跳,忙赔着笑道:``小主的脸好好儿的,小主貌美如花,青春正盛。''

嬿婉的手重重地垂落下来,如卸下千斤巨石。她摸着自己的脸凄怆道:

``澜翠,我不是不知道自己为什么得宠。为着皇上一时的兴致,为着一个男人偶然所起的一点欲念,更为着,我的脸,还有几分像娴贵妃年轻时的样子。难道我都不知道么?''

澜翠忙扶着她的身子,柔声道:``小主,娴贵妃位分尊贵,您像她,不算折您的福气。更何况,虽说是三分相像,您却胜过娴贵妃年轻时许多呢。''

嬿婉勉力支起身体,面容渐渐沉静若寒水。她裹紧了身上的青云缎锦毛披风,那声音像从嗓子底处透着心窝迸出来的:``是。能因为像娴贵妃而获宠,自然是我的福气。哪怕我再不懂事,只要这张脸在,只要我不犯下大错,就不会和娴贵妃当年一样,躺进冷宫里去。因为皇上看着我这张年轻的脸,就会想起曾经委屈过娴贵妃的年岁,自然会格外优容。且我还年轻,娴贵妃懂的,我慢慢学着,终有一日也都会懂得。她会的不肯轻易做的,我要什么都做的比她好,那便是最好的打算了。''

殿中晚膳己毕,便有小宫女伺候着捧茶漱口,一众人忙忙碌碌,却是鸦雀无声,丝毫不乱。李玉见一切事毕,方进来道:``皇上,太医院齐鲁大人有要事求见。''

皇帝面色微微一沉,如懿会意:``那臣妾先告退。''

皇帝摆手,笑得轻快:``不必。今夜你留在养心殿。李玉,着人去伺候贵妃沐浴。''

如懿转身离去,才走到后殿,她觉得左耳上空荡荡的,一摸之下才发觉戴着的白玉菡萏耳坠不知去了哪里。她心下微微一沉,只念着这是皇上赏赐的爱物,兼着几分酒意,并未多想便径自往东暖阁去。

才走到东暖阁外,只听见里头齐鲁的声音道:``前日中午,魏常在身边的宫女澜翠来,说要照着这瓶子里的坐胎药配一份,恰巧是微臣在太医院当值,便叫留下了。微臣细看之下,那份坐胎药竟是和皇上赐给舒嫔小主的那份是一模一样的,想是魏常在从舒嫔那儿偷弄去的。魏常在一心想要有孕,所以\ldots\ldots{}''

皇帝的口气有些沉肃:``既然魏常在这么想要,你就照样配一份给她。只告诉她那是上好的坐胎药,是舒嫔没福气才到今日还没怀上。''

齐鲁连连称是:``舒嫔小主问起时,微臣也是说她体质虚寒,不易有孕罢了。''

皇帝淡淡道:``也好。这个药朕本来就只是防着舒嫔是太后的人,又是叶赫那拉氏出身,才不想她轻易有孕。那药是你调制的,你自然知道,哪天停了也还是无碍的。魏常在既然动了这心思,朕反正有了那么多皇子,最要紧是有永琮。旁人能不能生,生儿生女,也无谓得很。''

齐鲁道:``是,皇上仁慈。那微臣这就去办。''

朔风刺寒侵骨,如懿倚在墙上,只觉得全身的力气都被抽空了,一颗心突突地几乎要从胸腔里蹦了出来。她的脑海里一片混沌,只是糊里糊涂地想着。

怎么会这样?居然是这样!

隐隐约约地,她不是第一次知道这样的事,慧贤皇贵妃生前服用的汤药都是加重她病症的,而舒嫔,皇帝更是决绝。也许,皇帝还以为是仁慈的,可不是么?他一定以为,本来一碗汤药就绝育的事情,他却不厌其烦地一次次让她们只是暂且不能受孕而已。

她紧紧按着自己的腹部,心里一阵一阵发凉,这便是帝王家啊!哪怕宠遇再多,恩眷再深,也不过是一念之间的天与地罢了。她脚下一阵阵发软,有些畏缩地蹲下身。正巧凌云彻与人换班经过,见她瑟缩在暖阁后地下,急忙道:

``娘娘,娘娘,你怎么了?''

如懿赶紧捂住自己的嘴,亦示意他捂住,拼命地摇头。云彻连拖带拉将她扶到后殿廊下,低声道:``娘娘可不舒服么?''

如懿强撑着身子起来:``没事,你回去吧。''她挣开他的手,虽然觉得他此时的一句寻常关心,让她在方才巨大的震动与惶惑里觉得有一息的温暖,可她明白,这样失态的自己,是不能让人瞧见的。她茫然地走到后殿,惢心刚想问她是否找到了耳环,见她这般,便知道不能多问了,忙打发了人出去,独自伺候她沐浴。

如懿把整个身体浸在滚热的水里,方只有这样,才能感觉到一丝暖气。沐浴所用之水最是讲究,按着时气用豆蔻花并佛手柑拧了汁子熬煮的,醇厚中不失清新之气,熏得混沌的脑仁渐渐安静下来。如懿静了良久,方才长长地嘘了一口气,茫然地转过脸,木木地问:``惢心,你说会不会有一天,皇上也不许我生下孩子?''

惢心不知出了何事,忙掩住如懿的口道:``小主,您胡说什么呢?''

如懿只觉得脸都僵了,只得揉着发酸的面颊道:``是啊,我正是胡说呢。''

豆蔻花被热水浸泡后氤氲的香气兜头兜脸地包围了如懿,她在那样沉醉的甜美里迟疑地想着,舒嫔该不该知道?或许,舒嫔是爱着皇帝的,才会在皇帝病重不得相见的日子里日日在宝华殿制作福袋祈福,却在皇帝病愈后一言不提自己的辛苦。若她知道,一定会很伤心吧?偏偏,她是那样孤高而骄傲的女子。''

所以,不!一定不能让她知道!哪怕是骗局,也宁可被欺骗的幸福,而不是清醒后钝刀刺身的痛苦。她紧紧地掩住了自己的嘴,将整个人浸了下去。

待到沐浴更衣回到寝殿之时,皇帝亦换好了明黄寝衣在等她。养心殿寝殿高高的房梁上,明黄的锦缎帷帐铺天盖地落落垂下,角落蟠龙金鼎内燃着上等紫檀香,青烟一缕一缕渐渐朝上扩散淡开,整个大殿肃穆而安静。如懿在踏入的一刻已然缓过了神色,温婉如常。

皇帝半垂着眼睑,慵懒道:``有佛手柑的气味,真好闻。''他伸出手向她,似笑非笑,``来,走近些,让朕细细闻闻,仿佛还有豆蔻的甜香。''

如懿静静一笑,走到榻前的双鹤紫铜烛台前,正要吹熄蜡烛,外头慌乱而仓促的脚步骤然响起,拍门声显然已失却了分寸,皇帝蹙眉道:``越来越没规矩!进来回话!''

扑开门滚进来的是皇后身边的赵一泰,他整张脸都扭曲了,大呼小叫地道:``皇上!不好了!不好了!七阿哥的乳母出痘了!七阿哥也紧跟着出痘了!他、他染上痘疫了!''

如懿的心陡然一跳,几乎失去了应有的节拍。积久的怨恨在她身体里如蚁附骨,无声地啃啮着,并随着时光的荡涤愈加深刻。她不是不曾想过,如果当时听了茉心的话,动了手会是如何?然而她心底一闪而过的阴暗的念头,却以这如刺又平顺的姿态破空来到人世。她还来不及细细去分辨心底是怜悯还是意外,皇帝已然霍地起身,撞翻了身边的双鹤紫铜烛台,火苗顺着明黄色碧金盘龙帐霍霍地燃烧起来。

七皇子永琮是在四日后,乾隆十二年的腊月二十九去世的。那是除夕的前一夜,他过早降临世间的身体根本经不起任何看似微小的病痛,何况是痘疫这样来势汹汹的恶疾。即便是在所有太医的拼力救治下,也未能熬到新的一年。

皇后在目睹亲生儿子死于怀中的一刻昏厥过去,且忧伤成疾,再难起身。

皇帝在悲痛中喃喃不绝:``明日就是腊月三十,过了明天,联的永琮就长大一岁了。''他大悲之余,特颁谕旨:``皇七子永琮。毓粹中宫,性成夙慧。甫及两周,岐嶷表异。圣母皇太后因其出自正嫡,聪颖殊常,钟爱最笃。朕亦深望教养成立,可属承祧。今不意以出痘薨逝,深为轸悼。''然而活着的人哀痛再深,如何能换回死去的孩子,一切也不过徒劳而已。

披着离丧之痛,这个新年自然是过得黯淡无比。过了大年初一,皇帝便开始郑重其事为爱子治丧。正月初二,将永琮遗体盛入``金棺''。诸王、大臣、官员及公主、福晋等齐集致哀。初四,将``金棺''移至城外暂安,沿途设亲王仪卫。初六,赐永琮谥号为``悼敏皇子''。十一,行``初祭礼'',用金银纸锭一万、纸钱一万、馔筵三十一席。宗室贵族,内廷命妇齐集祭所行礼。

二十三,行``大祭礼''。乾隆皇帝亲临祭所,奠酒三爵。

丧仪再隆重盛大,也洗不去皇帝的哀恸。嫡子夭折,皇后病重,嫔妃们自然不能不极尽哀仪。如懿协理六宫,费尽心神料理好永琮身后之事,以求极尽哀荣。私下时也不能不动了疑心,去问海兰。海兰却以瞠目之姿显露她同样的意外与震惊,然而她拍手称快:``原来咱们不动手,老天爷也不肯放过她呢!''

这一晚,如懿正前往长春宫探视悲痛欲绝的帝后,却在长春宫外的长街一侧,以惊鸿一瞥的短促,看到了素服银饰的玫嫔,正望着被凄怆的白色包裹的长春宫,悠然噙着一丝诡艳的笑容。不知怎的,如懿便想到了那一日,玫嫔生下那个怪异的孩子那一日。这样艳美的笑容,确是久未在她面上出现过了。

这样寻思间,经不住身边三宝的连连催促:``娘娘,宝华殿的超度事宜还等着您来主持呢。''她摇了摇头,便也走了。

乾隆十三年二月初四,皇帝奉皇太后,欲携后妃,东巡齐地鲁地。秦皇汉武皆有东巡之举,尤以登泰山封禅为盛,皇帝登基十三年,自以为江山安定,民众富庶,放眼四海之内,唯一不足唯有嫡子之事,然而困在宫中,亦不过举目伤心罢了,于是便动了效仿皇祖东巡之意。

自从永琮夭折,皇后大半心气都被挫磨殆尽。在新年后的一个月里,她躺在床上形如幽魂,除了眼泪和绝望,她的眼睛里再也看不到任何明亮的东西。

而太医带来的消息更让她失去可以支撑的意志。

齐鲁在为皇后搭脉后摇头道:``皇后娘娘,当年您一心催孕,太过心急,是在高龄体弱催得皇子,所以皇子早产,天生孱弱。而您也大伤元气,微臣与太医院同僚诊治过,娘娘想再有子息,只怕是不能了。''

听到这番话的时候,皇后的眼里只有一片干涸。淡淡的苦笑在她虚弱而下垂的嘴角边显得格外凄怆,她只是瞪着眼睛看着素色瓜瓞绵绵的帐顶,缓声道:``有劳太医。''

过多的悲伤与绝望终于如蚀木的白蚁渐渐毁坏她的身体。皇后一下子苍老如四十许人,一眼望去与年华犹在的太后并无分别。素心替她一点一点梳着蜿蜒在枕上的青丝,那夜夜丛生的白发如秋草衰蓬一般触目惊心。素心一边替她梳理一边想尽量用黑发遮住白发,然而怎么遮也遮不住。素心一急,忍不住默默流下泪来。皇后侧身躺在床上,看了眼素心手中的头发,居然一点焦灼与哀惋也无,只是淡淡道:``有什么可哭的?我本来就老了。''

这是皇后自册封后第一次自称``我'',素心自皇后名位定正之后,知晓皇后极爱惜矜持身份的``本宫''二字,此刻居然以``我''相称,口气中亦不觉如何惊恸。素心才惊觉,她侍奉多年的女子,心气已经灰败到如何地步。

皇后侧了侧身子,微微又窸窣之声,她的声音听上去疲惫到了极点:``一个无法再生育,传不下子嗣的皇后,老了,死了,又有什么要紧?何况是几缕青丝而已。''

素心含泪相望,双手亦有些颤抖:``皇后娘娘不要焦心,您积福积德,上天垂怜,一定还会有皇子的!''

皇后倚在枕上,神色平静得如一个即将离世之人。她沉默了许久,忽然轻声笑了起来,那笑声在宁静得如同深渊的殿阁里听来有太多的凄绝与幽惶:

``不能够了,我的身子已经不能够了。素心,我的永琏和永琮都保不住,难道都是报应?''

素心跪在皇后床前,拼命摇头道:``皇后娘娘,不是的,不是的。您只是防着该防的人,又没害死了他们,有什么报应不报应的话?''

殿外有微弱的哭声响起,皇后凝神听了片刻:``是谁在哭?怎么早早就替我哭上了。''

素心忙道:``皇后娘娘,是三公主在外头。她一直想进来看您,但以为您睡着,都不敢进来。公主都等了很久了。''

皇后轻叹一口气:``那就让她进来吧。''

和敬公主的步入并没有让皇后有太多的反应,她依旧安静地伏在重重堆锦绣被之中,如同一脉被抽尽了水分的枯叶,抑或,是一尾离水太久的涸泽之鱼。

和敬在进殿后明显收敛了她的哭声和眼泪,极力展露出几分笑意,向着背对她的皇后深深一福到底:``皇额娘万安。''

皇后闭目片刻,口吻淡漠:``你是皇上唯一的嫡出公主,站在长春宫前哭,太失仪了。''

和敬鼻子一酸:``皇额娘,儿臣是担心您。''

皇后的神色冷冰冰的没有温度,以训诫的口吻道:``你是大清的嫡亲公主,任何时刻,都不要忘记自己的身份。再说,你弟弟都死了,哭还有什么用?''

和敬的眼泪哗然如决堤:``皇额娘,永琮和二哥虽然都离皇额娘而去了,可皇额娘还有女儿啊。女儿也会是您的依靠,会给您争气。''

皇后闻言倏然睁开了双眼,吃力地支起身子坐直,上上下下地打量着和敬。和敬从未见皇后用这样的眼光看过自己,不觉悚然,被皇后的目光逼视,渐渐垂下了额头。

皇后冷冷嗤笑:``女儿?女儿有什么用?有了儿子,女儿是锦上添花的点缀;没有儿子,女儿连雪中送炭的那点炭火都比不上。不过聊胜于无罢了。''

皇后虽对女儿的疼惜远不如皇子,但也从未讲过这般刺心之语。和敬心气甚高,何曾听过这样的话,一下就被逼落了眼泪:``皇额娘,您就这样看不起女儿么?''

皇后怆然摇头,伸出手慢慢抚摸着女儿的脸,只是那手势并无多少温情的意味,而是带了一丝丝探索之意:``不是皇额娘看不起女儿,而是看不起自己。像我这样连儿子都保不住的额娘,难怪你皇阿玛伤心归伤心,这些日子也渐渐不来了。''

和敬本是自伤,听得皇后这样的话,不觉激愤地抬起眼睛,握紧了拳头道:``永琮死了还不到一个月,皇阿玛这些日子都流连在纯贵妃与嘉妃宫里。说到底她们不过是个妾侍,凭什么不让皇阿玛来多安慰陪伴您?''

皇后抚了抚自己憔悴得脱了形的面庞,那种干涩而松弛的触感,连自己触手也是心惊。她苦笑道:``你皇阿玛自己不来,旁人也无法。额娘人老珠黄,连个儿子也没有。你皇阿玛当然喜欢有了儿子又长得青春娇俏的女人。你皇阿玛有别的皇子陪伴,很快就会忘了额娘和永琮的。''

和敬忍不住落泪:``皇额娘怎么心气颓丧到这种地步?您是皇后,皇阿玛唯一的正室啊!如果您自己都灰心丧气,您要教女儿怎么办?皇阿玛有嘉妃,有纯贵妃,有娴贵妃,有别的阿哥,可女儿只有您!''她凄然别过脸,``皇额娘病成这个样子,还不知道吧,皇阿玛已经打算东巡,要带着娴贵妃和纯贵妃为首的六宫嫔妃去齐鲁之地,他们会去祭泰山,祭孔庙。这是皇阿玛登基十三年来第一次东巡。您是天下之母,您怎么可以不去?''皇后有一瞬间的茫然,继而是深彻的震惊与疑惑,她看着素心道:``什么东巡,本宫怎么不知道?''

素心有些怯怯的:``其实皇上一直是希望皇后娘娘能去东巡的,只是担心娘娘您悲伤过度,病体未愈,经不得车马劳顿,所以一直没有对您说\ldots\ldots{}''

皇后的眼底有两行清泪涌出:``本宫还没有跟着永琮去了,她们就都当本宫死了么?''

和敬看着皇后的悲怒,不自觉地含了一缕笑:``当然不能!皇额娘能这么问,儿臣真心为皇额娘高兴!''她紧紧握住皇后的双手,跪在皇后身前,``皇额娘,不要紧,哪怕二哥和永琮都不在了,您是皇后,还是不可动摇的皇后。儿臣虽然没用,但好歹是皇阿玛与您唯一的女儿,儿臣一定会紧紧扶着皇额娘您,咱们母女,一定会走得很好很好。您放心!''

皇后所有的意志在这一瞬被和敬眼底的坚毅与不肯服输激得竖硬如铁,她不自禁地伸手抿好蓬乱的鬓发,沉声道:``素心,去传齐太医来,本宫要请他好好看一看了。''

\hypertarget{ux7b2cux5341ux4e03ux7ae0-ux8fdcux5ac1}{%
\chapter{第十七章 远嫁}\label{ux7b2cux5341ux4e03ux7ae0-ux8fdcux5ac1}}

十日之后,皇帝起驾东巡,皇后严妆丽服,从容相随。那样的好气色,连皇帝亦感叹:``本来朕东巡就是想带皇后一同前往散心,可以一起纾解丧子之痛。原以为皇后病卧不起,却不想这么快就见好了。''

皇后含笑雍容:``皇上登基后第一次东巡,臣妾怎可不相伴左右?只是臣妾病体初愈,还得齐太医在侧,随时诊候。''

如懿与绿筠伴随在侧,亦含笑道:``皇后凤体安康,臣妾等也就放心了。''

和敬公主伴随在皇后身侧,倨傲道:``皇额娘母仪天下,自然神佛护佑,你们不过是皇阿玛的妾侍而己,一定要悉心伺候,恪守本分。''

这样的话,听在耳中亦是刺在心上,温和如绿筠,亦不觉变了脸色。如懿笑着在背后按住她的手,含笑如初:``公主孝心,说得极是。''

如此,二月二十四,帝后至山东曲阜谒孔庙。二月二十九,登东岳泰山。

三月初四,游济南览趵突泉。这般游山玩水,舟车劳顿,皇后却时时陪伴在皇帝身侧,须臾不离片刻。沿途臣民官员们偶然窥见,亦不觉感叹帝后鹩鲽情深,形影相随。

然而,唯有素心与和敬公主知道,皇后每天是如何服下剂量极重的提神益气之药,又以大补人参提气,才支撑着她日渐枯竭的身体陪着皇帝言笑晏晏,游历山水。

而年正十七的和敬公主,她的婚事,便是在东巡至济南行宫时议起的。

事情的起初,蒙古博尔济吉特部求娶的只是嫡出公主,而非意指和敬。皇帝的意思,亦只是以太后的亲生女儿,先帝的幼女柔淑长公主下嫁。

但这一提议,几乎是受到了满朝文武的反对,尤其是朝中侍奉过先帝的老臣,反对之声尤为剧烈,皆称``太后长女端淑公主已经嫁准噶尔,幼女再远嫁,于情于理于孝道,都是不合。''

皇帝回到如懿宫中,神色阴阴欲雨。如懿知道皇帝心中不悦,便打发了宫人们都下去,在旁折了雪白香花供在清水中,方问道:``皇上为何不高兴?''

皇帝将手中茶盏重重一放:``朕一直尊养太后,孝敬有加。却不想姑息了太后这般权势,在后宫她事事干预也罢,便是前朝也不肯放开手。''

如懿暗暗一惊,脸上却依旧凝着练达笑色:``后宫不许干政,太后怎会不懂。再说太后的儿子只有皇上一个,但凡太后有权势,那也是皇上以仁孝治天下,尊敬太后的缘故。''

皇帝的脸色稍稍和缓,摩挲着手边莹润如玉的茶盏:``可朝臣们都极力反对朕将太后幼女柔淑长公主远嫁博尔济吉特部。满蒙联姻乃是旧俗,博尔济吉特氏又是我大清历代后妃辈出之地,先祖皇太极与顺治爷的皇后都是出自那里,难道柔淑嫁过去还是委屈了她不成?要朕看,那可是一个极好的归宿。''

如懿沉吟片刻,看着风轮吹过香花缓缓地带来拂面的清馨,柔缓道:``朝臣们只知其一不知其二。以臣妾看来,这对柔淑长公主不是委屈,而是极大的抬举了。''

如懿轻笑,一双美目沉着得辨不出颜色:``太后的长女端淑公主便是远嫁最骁勇善战的准噶尔部,若是柔淑再嫁最富庶尊贵的博尔济吉特部,那么不是蒙古宗亲中最大的两个部落,便可从此紧密联结再无二致了。而皇上治理蒙古之道,一向可提倡花开两朵,平分春色的呀。''

皇帝不觉凛然:``那么,你的意思是\ldots\ldots{}''

如懿乌黑的眸子里有幽幽的柔光闪烁:``既然博尔济吉特部一直是至亲,那么与至亲联结,密不可分,便由自己的女儿嫁去,才是最好最稳当的。''

皇帝郁然道:``纯贵妃的和嘉公主璟妍还小,朕何尝不知道璟瑟是最合适的,可永琮死了才没多久,璟瑟是皇后唯一的孩子,朕怎么再忍心教皇后承受生离之苦。''

如懿的眼波里涟漪潋滟,仿佛是夜色的深沉:``和敬公主是皇后唯一的孩子,又是皇上的长女。但国有重用,公主首先是帝王家臣,然后才是父母之女。皇后一向说嫔妃先是皇上臣子,然后才是侍奉皇上的枕边人。皇后以此教导后宫嫔妃,自然也如此教导公主。''

皇帝颇有几分伤感不舍:``朕有六个儿子,公主却只有璟瑟和璟妍两个。璟瑟自幼承欢膝下,朕自然是有些舍不得。最好她嫁得近些,每日都在眼前。这件事,许朕再想想。''

皇帝这一别,两日都没有到嫔妃宫中来,也不往太后宫中请安,太后自得了要下嫁公主的消息,更兼知是柔淑下嫁的可能最大,急得两天两夜没有合眼。但太后在先帝身边多年,却是极沉得住气的,虽然心急如焚,但对着底下的宫人却是如常和缓坦然,只是暗中叮嘱福珈道:``去告诉舒嫔和玫嫔,养兵千日用兵一时,是该要她们去好好劝皇帝的时候了。那些朝中的老臣虽然看在先帝的颜面上肯为哀家进言,力劝皇帝不要再嫁幼妹,但他们的话哪里比得上枕头风的厉害。''

福珈答应了一声,又道:``可,娴贵妃那边下午来过人,说是请太后一定要知会朝臣们,以力陈柔淑长公主下嫁的益处为由,极力劝谏。''

太后眉眼间隐隐有青色的憔悴之意,支着下颌道:``她居然这样说?也不知是真心假意,别害了哀家唯一的女儿才好。''

福珈低低道:``太后\ldots\ldots{}''

太后蹙眉良久,一支青玉凤钗垂下的玉流苏停在她耳畔纹丝不动。良久,太后的身体微微一震,恍然含笑道:``这个如懿\ldots\ldots 哀家是小瞧她了。福珈,按娴贵妃所言,去叮嘱玫嫔与舒嫔,还有朝中几位老臣。快去!快去!''

玫嫔和舒嫔是太后一手调教出来的人,如何不落力劝谏。果然,两日后皇帝下了口谕,要如懿与绿筠前往先行劝说,要和敬公主接受下嫁博尔济吉特部之议。

彼时绿筠尚未过来,蕊姬伴着如懿闲坐,听闻此事,便冷笑道:``和敬公主是皇后所生,皇后一定常常在公主跟前怨及娘娘和咱们这些人,所以公主才会常常口出狂言,少不得还在皇上面前有不少不中听的话。我倒在想,皇后的孩子一个接着一个不在跟前了,她是怎样的心情!''

如懿轻笑道:``皇后要心疼也是有的,这些日子她日日陪着皇上,夫妻见面的情分,或许本宫与纯贵妃才劝好公主愿意下嫁,她三言两语便能挑回去了。''

蕊姬神秘地摇摇头:``娴贵妃还不知道么,皇后怕是顾不过来了呢。这些日子您看着她气色极好,内里却虚到了极处,每日里悄悄拿药吊着,所以都不敢留皇上在自己宫里呢。''

如懿眉心一动,只是含笑:``还是妹妹聪慧仔细。''说罢,便有小太监通传,说绿筠已然到了门口,邀了她同往公主住处去,蕊姬便也告退不提。

如懿与绿筠结伴到了和敬公主所住殿阁,和敬正坐在窗下看一本长孙皇后所写的《女则》。见了她二人来,也不过抬了抬眼皮,淡淡吩咐宫女:``上茶。''

如懿与绿筠对视一眼,见她如此倨傲,索性开门见山道:``皇上已经想好了,和敬公主尚蒙古科尔沁部博尔济吉特氏辅国公色布腾巴勒珠尔,婚期就在明年三月。草长莺飞,春和景明,果然是公主出嫁的好日子。''

大约这些日子总有些风言风语落进她耳朵里,和敬并无丝毫惊动之意,只端然坐着,捧了一卷书道:``我不嫁。''

如懿微笑不语,绿筠笑吟吟道:``公主还不知吧?这位额驸的来头可不小,他是科尔沁扎亲王满珠习礼的玄孙,满珠习礼是孝庄文太后的四哥,说来爱新觉罗家与科尔沁博尔济吉特部的联姻,当其源远流长。到底也是皇上心疼公主是嫡女,所以舍不得嫁给别人,还是给了最尊贵最至亲的王爷。''

和敬翻了一页书,头也不抬:``虽然博尔济吉特氏出了好几位皇后、太后,可我大清日渐兴盛,蒙古草原依旧是荒蛮落后之辈,我怎能再嫁去边远之地,与牛羊牲畜为伍?''

绿筠与如懿对视一眼,知是谈不下去了。绿筠还不死心,试探着问:``那公主是真不愿意了?''

和敬脸色微微一冷,将手中书卷放下。她原本就是眉目端庄,不怒自威的女子,此刻含气,越发显得神色冷肃。和敬冷冷扫视二人一眼,神色倨傲:

``纯贵妃也好,娴贵妃也好,都不过是皇阿玛的妾室,奉洒扫殷勤之事。我是中宫嫡出,婚嫁大事怎是你们二人可以向我冒昧提及?即便真是要嫁,也该由皇祖母和皇阿玛、皇额娘来向我说才是。再说了,纯贵妃要觉得远嫁甚好,何不让你自己的和嘉公主出嫁?''

绿筠听得这些话,不觉面红耳赤,分辩道:``璟妍才两岁多,如何出嫁\ldots\ldots{}''

如懿保持着不卑不亢的笑意:``公主所言极是。本宫与纯贵妃不是公主生母,此事本不该由我二人开口。但公主口口声声自称为中宫嫡出,岂不知皇后病弱,无暇顾及公主,而皇太后年事己高。皇上自认为男子,所以将这推心置腹之事交给本宫与纯贵妃。''

绿筠缓了尴尬,微笑道:``是呢。这门婚事,皇上也是看重公主的缘故啊。''

和敬眼角飞起,瞟一眼绿筠,语含讥诮:``纯贵妃果然是过来人,满眼的门楣与血统,真真是庶妃的小家子气。我却不是这样只掂量身世的卑贱之人。''

绿筠虽然性子随和,但被她这样讥刺,登时面上挂不住,只别过脸不再说话。

气氛一时凝住,如懿只作不觉,微微笑道:``公主乃皇后亲生,自然胸怀天下,何必把嫡庶你我分得如此清楚。要让无知小人传出去,还以为公主不把庶出的弟妹放在眼中,难免让皇上觉得公主心胸狭窄,好好的疑心了公主了呢。''

和敬无从反驳,深深吸一口气,昂首道:``我是皇后亲生,怎可远嫁蒙古这种不毛之地?''

``蒙古是不毛之地?''如懿宛转瞥她一眼,轻声嗤笑,``公主如此轻蔑蒙古,岂不知皇上有多么重视公主口中的不毛之地。满蒙联姻是先祖传下来的规矩,蒙古铁骑向来就是大清安顿四方的后援劲旅。''如懿凝视和敬公主,神色平静如无风无澜的湖面,``你是公主又如何?是皇后亲生又如何?皇后身为天下之母,也要受皇上约束,受宫规约束,受天下悠悠之口约束。你是公主,享天下之养,自然要为天下倾尽毕生之力。古来公主和亲之事数不胜数,能将一身静胡尘时,多少女子都甘愿舍身,何况只是让公主遵从满蒙姻亲的旧俗呢?''

从未有过的惊恐之色从和敬一贯冷傲的眉梢眼角慢慢渗出,仿佛如冰裂前肆意弥漫的裂痕,终于承受不住那样的重压,碎成满地晶亮的渣滓。不过片刻,和敬凄惶不已,恰如她高高耸起在玉白脖颈边的水绿盘银线立领一般,泛着细碎粼粼的冷色。她不复方才的高傲,只是强撑着道:``父母在,不远游。皇额娘抱病,永琮夭折,这个时候,璟瑟身为长女,理应承欢膝下,洒扫侍奉,以全孝道。''

绿筠笑意温婉,却含了几分犀利:``洒扫侍奉,不是我们这些身为皇上妾室的卑贱之人该做的吗?怎敢劳烦公主干金贵体。''

和敬闻言变色,连连冷笑:``我就知道,你们多嫌了我!眼看皇额娘病重,就个个乌眼鸡似的盯着皇后之位,趁早要先把我赶了出去,你们才安心。''

如懿端然起身,沉静道:``皇后病重?皇后不是好好的嘛!公主岂能为了婚姻之事,空口白舌诅咒生母?而且这婚事,不是为了我们安心,是为了皇后。''

和敬愣了一愣:``怎么会是皇额娘,她怎么舍得我这个唯一的女儿\ldots\ldots{}''

``她舍得!''如懿横了和敬一眼,口气温和而断然,``因为七阿哥早夭,皇后能依靠的,只有公主您一个了。皇后娘娘已经没了儿子,要让中宫之位稳若泰山,必须要有蒙古这个强有力的后盾作为支援,而公主你嫁往蒙古,才是联合蒙古最好的保障。''

绿筠大惊失色,立时不安:``娴贵妃,你和公主说这些做什么?公主她\ldots\ldots{}''

``公主她不懂!公主养在深宫无忧无虑,不知父母苦心,所以本宫要说给公主听。''如懿锐利目光逼向公主,``公主不愿意远嫁,自然有公主的道理。然公主可听过这四个字,叫作`无从选择'?''

和敬茫然:``无从选择?''

``是。无从选择。''如懿朗然道,``皇后身为中宫,无从选择她母仪天下应该背负的责任;皇上执掌天下,无从选择安邦定国的职责;公主天之骄女,更不应该只享受俸禄供养,而忘记了自己身为公主无从选择的人生。住这个皇宫里,卑微如奴才,高贵如您,一辈子都只有四个字:无从选择。''

和敬倒退两步,瘫倒在紫檀椅上,再说不出话来。

如懿的话并没有说错。当和敬公主泪眼婆娑赶到皇后宫中跪求的时候,皇后亦只能抱着女儿垂泪道:``孩子,皇额娘实实已经是不能了。你皇阿玛既然让娴贵妃和纯贵妃去劝你,那便等于告诉你,他的决心只差一道圣旨颁布天下了。''

和敬公主无力地伏在皇后膝上,又是震惊又是害怕,含了一丝祈望之色,垂泪不已:``皇阿玛是有儿臣和璟妍两个女儿,璟妍固然才两岁,又是庶出,身份不配,可皇阿玛还有柔淑长公主这个妹妹,柔淑长公主还比女儿大了两岁,为什么皇阿玛不选柔淑长公主,偏要选女儿呢?''

皇后穿着湖水色绣春兰秋菊缠金线的云锦丝袍,那云锦质地极为柔软,沾上和敬的泪水,倏然便洇灭不见。皇后头上松松地抓着一把翡翠嵌珊瑚米珠飞凤钿子。因是东巡在外,她也格外讲究气度风仪,一应打扮比在宫内时精心许多,便是昂贵的珠饰,偶尔也肯佩戴。如今她妆饰华贵,点染匀称的面宠也因爱女即将远嫁而染上了伤心泪痕;``你皇阿玛要是有办法,也不会想到是你。满蒙联姻是旧俗,尤其是博尔济吉特部。你皇阿玛原也想着是把柔淑长公主嫁过去,但若真这么做,无疑是加强了太后与蒙古各部的联系。''

和敬抬起朦胧的泪眼,无奈道:``皇额娘的意思是,就是因为太后的端淑长公主嫁去了蒙古,所以柔淑长公主不能再嫁?''

皇后的脸上尽是不舍之意,沉吟片刻,强自维持着冷静道:``是。博尔济,吉特部是大清最最重要的姻亲,是大清北方安定的保障。所以要嫁,只能是自己最亲的人。''皇后见身边无人,低沉了声音道,``而且,就因为皇额娘只有你这一个女儿,所以宁可你远嫁,也要嫁得尊贵,嫁得体面。''

和敬再顾不得仪态,苦苦哀求道:``可蒙古那么远,女儿即使想回来省亲,山高水长,又能多久回来一次?皇额娘只有女儿了,要是女儿不在身边,谁与皇额娘彼此扶持呢?''

皇后疲倦而黯淡的眼中闪过一丝精光,紧紧握住和敬的手:``你嫁去蒙古联姻,便是对皇额娘最大的扶持。皇额娘的伯父马齐是两朝重臣,可自从伯父去世,富察氏的声望虽在,但内里实在不比从前了。对皇额娘也好,对富察氏也好,我们都太需要一个强大的后盾来保证现在的地位永无动摇。所以你皇阿玛一说,皇额娘就知道,这是个最好的机会,这样的机会,绝不能给了太后的女儿,必须是在咱们手中。''她的眼底闪过一丝决绝而坚定的冷光,那种冷,带了某些无可回旋的余地,她压住了胸腔中的酸涩,静静道,``所以在你来之前,皇额娘看你皇阿玛有所犹豫的时候,皇额娘已经默许,默许是你远嫁蒙古,也只能是你远嫁蒙古。''

和敬从未见过皇后以这样感触而不容置疑的口吻对自己说话,她便是满心不情愿,也知事情再无一点指望。她半张着嘴,想要说什么,却哽咽得发不出半点声音。从闪烁的泪花里望出去,皇后的面庞显得熟悉而又格外渺远的陌生。和敬心头大恸,哭得花容失色:``原来娴贵妃说的都是真的。她说皇额娘您绝不会反对,这是真的!''

皇后悄然拭去腮边斑斑泪痕,闻言微微惊讶:``娴贵妃当真这样说?''

和敬并不回答,只是痛哭不已:``皇额娘,您真的舍得?真的愿意?''

皇后严妆的面庞一分分退却了血色,苍白的容色如同窗外纷飞的柳絮,点点飞白如冰寒碎雪:``孩子,原也没有什么舍不得的。皇额娘从一出生,就知道自己这个人这条命都是属于富察氏的,皇额娘所做的一切,都是为了富察氏的荣华显赫。而你一出生,从你获得的荣耀开始,一切都是属于大清的。这一点上,你和额娘没有两样。所以,你是大清的公主,这是你最好的归宿。''

和敬终于在母亲平淡而哀伤的语气里明白了自己不可回转的前途,只得俯下身三拜告别,哀哀道:``既然皇额娘与皇阿玛决心已定,女儿也不能说什么了。女儿既然存定了孝心,也是大清与皇额娘母家的期望,那么女儿顺从就是。''

和敬吃力地站起身子,任由眼中的泪水和着唇边淡薄削尖的笑意一同凝住,恍惚失神地一步步摇晃着走出了皇后宫中。

\hypertarget{ux7b2cux5341ux516bux7ae0-ux6bcdux5fc3}{%
\chapter{第十八章 母心}\label{ux7b2cux5341ux516bux7ae0-ux6bcdux5fc3}}

皇后看着女儿步出,仿佛再也支撑不住似的,一下子瘫坐在了紫檀雕花椅上,任由泪水蔓延肆意。素心正端了药走进,见皇后大口大口地喘息着,面如金纸,不觉慌了手脚,忙搁下药盏替皇后抚胸按背。好一顿推揉,皇后才缓过了气息。素心见皇后好些,忙不迭递上药盏,含泪劝道:``皇后娘娘自然也是舍不得公主,其实何不把话都敞亮了说给公主知道呢?这话吐一半含一半,娘娘难受,公主也不能明白您的苦心。''

皇后就着素心的手把一盏药慢慢喝完了,才支起半分力气道:``本宫何曾不想告诉璟瑟,可她到底还小,有些话听不得的,一听只怕更不肯嫁了。''皇后看一眼素心,神色惨然,``这些日子你跟在本宫身边,难道你不知道本宫的身子到底是什么样子么?''

素心一怔,眼底蓄了半日的泪就涌了出来,她自知哭泣不吉,忙擦了泪面笑道:``皇后娘娘福绥绵长,一定会好起来的。''

皇后盯着她看了须臾,不禁苦笑,抚着胸口虚弱道:``你不必哄本宫了,本宫自己知道,要不是齐太医用这么重的药一直吊着,本宫怕是连走出宫门的力气都没有。哪天本宫要是不在了,璟瑟孤零零的,她又是那么高傲的性子,哪怕要嫁人,岂不是也要受那些人的暗亏,落不到一个好人家去。还不如趁着本宫还有一口气,替她安排了好归宿,也卖了太后一个人情,日后可以让太后看在本宫今日保全柔淑长公主的苦心上,可以稍稍善待本宫的女儿。''

素心见皇后连说这几句话都气短力虚,仍是这般殚精竭虑,忍不住落泪道:``皇后娘娘平时嘴上总说最疼两位阿哥,未曾好好待公主,其实您心里不知道多疼公主呢。''

皇后满心凄楚,怆然道:``璟瑟虽然只是个女儿,但到底是本宫怀胎十月所生。本宫不争气,保不住皇子,以后富察氏的基业和昌盛,一半是靠自己的功名,一半便是靠璟瑟了。说来也终究是本宫不好,素日里不曾对璟瑟好好用心,临了却不得不让她远嫁来保全富察氏的荣耀。''她越说越是伤心,气息急促如澎湃的海浪,她死死抓着素心的手,凄厉道,``素心,本宫的儿子保不住,女儿也要远嫁,这到底是不是本宫的报应,是不是本宫错了!可本宫做了这么多,只是防着该防的人,求本宫想求的事,并未曾杀人放火伤天害理,到底是为了什么?为了什么?''皇后如掏心挖肺一般,一双眼突出如核,直直地瞪着素心。

素心听得``杀人放火''四字,脸色煞白如死,忙好声安慰道:``娘娘确不曾做过,您就别多思伤神了,赶紧歇一歇吧。''像是要压抑住此时难掩的心慌一般,素心的指尖一阵阵发凉,哪里扶得住皇后摇摇欲坠的身体,扬声向外喊道,``莲心!快进来!快进来扶娘娘!''

莲心本在门外候着,只顾侧耳听着殿中动静,死死攥紧了手指,任由指甲的尖锐戳进皮肉里,来抵挡皇后一声声追问里勾起的她往日不堪回首的记忆。

直到素心仓皇呼唤,她才强自定了心神,一如往日的谦卑恭谨,匆匆赶进。莲心正要帮着伸手扶住皇后,只见皇后气息微弱,身体陡地一仰,已然晕厥过去。素心吓得魂飞魄散,哪里还顾得上别的,一壁和莲心扶着皇后躺下,一壁吩咐赵一泰去唤了太医来。

太后坐于别馆之内,拿着圣旨反反复复看了许多遍,眼角的笑意越来越浓,仿佛一朵金丝菊花,泼泼绽开无限欢喜欣慰。玫嫔跪在紫檀脚踏边,拿着象牙小槌为太后轻轻敲打小腿,脆生生笑道:``这道圣旨太后看了一个晚上了,还没够么?''

福珈上来添了茶,在旁笑道:``太后悬了多少年的心事,终于能够放下了。''

太后心满意足地喝了口茶:``多亏得玫嫔与舒嫔争气,这几日没少在皇帝跟前吹风。''她抿了抿唇角,``福珈,你往这茶里加了什么,怎么这样甜?''

福珈笑得合不拢嘴:``不就是寻常的白毫银针,哪里搁什么东西了?架不住太后心里甜,所以茶水入口都成了甜的。''

玫嫔正了正鬓边的玫瑰攒珠花钗,笑道:``可不是呢?臣妾也从未见太后这般高兴过呢。''

太后唇边的笑色如同她身上的湖青色金丝云鹤嵌珠袍一般闪耀:``先帝临终前,已经病得万事不能做主了。为保新帝登基后蒙古各部一切稳妥,哀家和敬公主下嫁蒙古之事已然成为定局。三月初七,皇帝下旨和敬公主晋封固伦和敬公主,次年三月尚蒙古科尔沁部博尔济吉特氏辅国公色布腾巴勒珠尔。同时,晋封太后幼女为固伦柔淑长公主,亦于次年三月尚理藩院侍郎宗正。''

福珈笑叹道:``理藩院的侍郎虽然不是什么要紧的官职,但到底也还体面,哪怕额驸是领个闲差,公主能在太后跟前常常尽孝,也是极好的。''

玫嫔抬起妩媚纤长的眼角,轻轻柔柔道:``娴贵妃\ldots\ldots 算是很尽心了。''

太后瞄了她一眼,舒然长叹:``也是。若不是她想到要以退为进,力陈柔淑下嫁蒙古的好处,皇帝未必会听得进去,才反其道而行。这件事,哀家念着娴贵妃的好处。自然了,皇后也是明白事理的。也亏得齐鲁来告诉哀家皇后病重,哀家才能劝得动皇后接受这门婚事。''

玫嫔冷冷一笑:``对皇后来说,是想公主有个婆家的靠山。其实她是最看不穿的,太后娘娘心如明镜,儿女在身边,比什么都要紧得多了。''

太后长叹一声,抚着手腕上的碧玉七宝琉璃镯道:``皇后毕竟还年轻啊。许多事她还不懂得,只怕以后也来不及懂得了。她的病,皇帝心里有数么?''

玫嫔略略思忖道:``齐鲁虽是皇上身边的人,但一向最油滑老道,左右逢源。这次皇后的病虽然一直瞒得密不透风的,怕是皇上也隐约知道些,所以御驾才吩咐了,明日就要准备回銮。''

太后静了片刻,看着小几上的一缕香烟袅袅缥缈,微眯了眼道:``外面虽好,到底不如宫里舒坦。待了一辈子的地方,还是想着要早点回銮。对了,舒嫔原说要和你一起过来的,怎么这个时辰还没过来。''

福珈忙道:``方才舒嫔那儿来过人了,说是预备着侍寝,就不过来了。''

玫嫔嘴边的笑便化成一缕不屑:``侍寝还早呢,这个时候就说不过来了,也敷衍得很。''

太后微微一笑,对这些争风吃醋之事极为了然:``舒嫔跟在哀家身边的时候没有你长,自然不如你的孝心重。好了,时候不早,你也先回去吧。''

玫嫔这才起身告退。福珈看着她出去,低声道:``论起来,玫嫔待太后的孝心,可比舒嫔多呢。''

太后唇角的笑容逐渐淡了下来:``你也看出来了?''

福珈微微沉吟:``奴婢冷眼瞧着,舒嫔待皇上的心是比待太后您重多了,这样的人留在皇上身边,还这么得宠\ldots\ldots{}''

太后笑着弹了弹指甲:``皇帝的风流才情,是招女人喜欢。舒嫔的心在皇帝身上也好,有几分真心才更能成事。皇帝自小不得父母亲情,在夫妻情分上也冷淡些,但他一颗心是知道冷暖的,所以舒嫔的好处他都看在心里,才格外相待些。你且看玫嫔的恩宠,到底是不如舒嫔了。''

福珈还是有些不放心:``那太后不怕\ldots\ldots{}''

``怕?''太后不屑地嗤笑,``皇帝虽宠爱舒嫔,但他对舒嫔做了什么,真当哀家什么都不知道么?舒嫔的性子刚烈,若来日知道了发起疯来,指不定将来会做出什么事情来呢。''

夜色阑珊。

济南的夜,无论怎样望,都是隐隐发蓝的黑,璀璨如钻的星辰,像是洒落了满天的明亮与繁灿。不像京城的夜,怎么望都是近在咫尺的墨黑色,好像随时都会压翻在天灵盖上。

皇后醒来时已是半夜,几名太医跪在素纱捻金线芭蕉屏风外候着,听得皇后醒来的动静,方敢进来请脉。皇后有些迷迷糊糊,睁开眼却见皇帝也在身边,慌忙含笑支撑着起身请安:``皇上万福,皇上怎么在这儿?''她极力掩饰着睡中憔悴支离的容颜,``素心,是什么时辰了?''

素心忙回禀道:``回皇后娘娘,是子时二刻了。''

皇帝忙按住她,柔声道:``别挣扎着起来了,闹得一头的虚汗。''说罢,他取过绢子替皇后擦拭着额头汗珠,``朕本来宣了舒嫔侍寝,但不知怎的,总念着你与璟瑟,想来想去觉得心里头不安,便过来看看你。谁知道你一直昏昏沉沉地睡着,口中念念有词。''皇帝的语气愈加温柔,``怎么了?可梦见了什么?''

皇后忙笑道:``难怪臣妾总觉得和谁在说话,口干舌燥,原是说梦话了。''她仔细想了想,``其实这个梦臣妾已经做过好几次了,皇上也是知道的。''

皇帝想了想,抚着皇后青筋暴起的手背道:``皇后又梦到碧霞元君了?''

皇后苍白的脸上浮起一层薄薄的霞色红晕:``此次东巡以来,臣妾一直梦到碧霞元君在睡梦中召唤臣妾。所以臣妾与皇上祭泰山时,特意往碧霞元君祠许愿。可如今臣妾已经离开泰山了,不知为何,碧霞元君仍是在梦中屡屡召唤。''

皇帝宽慰道:``民间传说碧霞元君神通广大,尤其能使女子生子,母子无恙。朕知道皇后一心还想为朕添个皇子,所以与皇后在泰山诚心拜求,但愿碧霞元君显灵。皇后既然屡屡梦到碧霞元君召唤,看来朕与皇后的心愿都会达成了。''

皇帝既如此说,身边的人哪有不奉承的,连齐鲁也少不得道:``只要皇后娘娘悉心调理,凤体无恙,一定会如愿以偿的。''

皇后明知自己早成了蛀空的腐木,不过外表看着还光鲜罢了,这心愿如何能够得成?只是当着皇帝的面,也只能强颜含笑:``既然如此,皇上不如请钦天监再看看,若是可以,臣妾想再前往碧霞元君祠拜求,希望上天垂怜,实现皇上与臣妾的心愿。''

皇帝略略有些踌躇:``皇后,太医已经为你诊治过,说你身子不适。也是朕不好,这些日子只顾着巡游,让你舟车劳顿。朕已吩咐下去,明日午后御驾回銮,咱们也得回京,议起璟瑟的婚事了。''

皇后心中一酸,怕是皇帝看出了自己病象,不安道:``皇上,臣妾没事。臣妾\ldots\ldots{}''

皇帝替她掖好被子,柔和道:``皇后,你好好躺下歇息。莲心在前厅给朕备了点心,朕去用一些,再进来看你。''说罢,他便领了太医往前厅去。

前厅的案几上放着四色细巧点心,都是山东名产。皇帝无心去动,只黯然道:``皇后的身子,便已经糟糕到这个地步了么?''

齐鲁领着太医们躬身跪在地上,一时也不敢接话,思忖了半天道:``皇后娘娘要强,一心进补提气,原是精神百倍的,但\ldots\ldots{}''他身后一个太医怯怯接口:``但皇后娘娘用心过甚,其实大半是心病\ldots\ldots 微臣们医得了病,却医不得心。''太医们说完,连连磕头请罪:``皇上恕罪,皇上恕罪。''

皇帝的脸上写满了难以名状的沉郁。李玉悄悄道:``皇上,太医们也是尽力了。您还记得东巡离宫前,您原是不想皇后娘娘随行的,因为钦天监在七阿哥夭折后曾奏,`客星见离宫,占属中宫一眚'。当时有一颗时隐时现的`客星'出现在名为离宫的六颗星之中,是为天象大异,钦天监以为这预示中宫将有祸殃临头。''

也好转了许多。这次又有璟瑟下嫁蒙古之事冲喜,你们只要尽力医治,皇后一定会好转的。''他说罢,却见进忠进来道:``皇上,令贵人听说您忧思伤怀,所以特意在殿外等候,想见皇上。''

皇帝不假思索道:``你们都留下好好照顾皇后。李玉,去令贵人阁中。''

嬿婉自封令贵人之后,皇帝虽也宠爱,但比初初承宠时却逊色了几分,自然也是为了当日燕窝细粉与不辨甜白釉之事。嬿婉虽然惴惴,又百般自学以讨皇帝欢心,却也总有些心虚。此刻皇帝宁愿去见她而不留皇后宫中,李玉自然知道其中利害,忙答应着伺候皇帝去了。皇后披衣强自立在屏风后,眼见着皇帝离去,身体一软,靠在了素心怀中,眼泪扑簌簌地滚落下来,失神地絮絮道:``医得了病,医不得心\ldots\ldots 医得了病,医不得心\ldots\ldots{}''

三月初八,皇帝奉皇太后回銮。皇后的病一直忽急忽缓,人也时昏时醒。

虽然还能起身,却消瘦了不少,连早午晚的膳食都不能陪着皇帝一起用。

这一日是三月十一,御驾至德州,弃车登舟,沿运河从水路回京。皇后一路车马风尘,极为吃力,忽然到了水上行舟,眼见两岸轻红蘸绿,迤逦十余里不绝,抹出烟霞般柔丽的色泽,隐隐然有了蒙蒙春意,心下也有几分欢悦,便撑着身体与皇帝和嫔妃们一同用了晚膳。

皇帝见皇后能起身用膳,心下十分安慰,便先打发了嫔妃们离去,特意陪着皇后说了好一会儿话才叫人送了皇后回到青雀舫上,吩咐李玉召如懿至龙舟上,欣赏白日里山东巡抚进献的宋代崔白的名画《双喜图》。

皇帝的龙船之后便是皇太后的翟凤大船,再便是皇后乘坐的青雀舫,其后才是嫔妃们的喜鹊登梅彩船一一跟随。皇太后素喜礼佛,嫔妃们的船尾后专有一船供奉佛像经卷,太后便携了福珈并合船宫人尽数同去焚香祝祷。皇后扶着素心与莲心的手回到青雀舫上,但见两岸月色如画,一时也起了兴致,在船尾伫立,看着夜色中柳色青青,晓风圆月,也颇有几分动人情致,便贪看住了,道:``今儿月色真好,本宫许久没见这样清朗月光了。''

莲心忙劝道:``皇后娘娘,您凤体才稍稍见好,仔细着了风,还是进去吧。''

素心悄悄儿向她摆了摆手,道:``娘娘这才真是大好了。这儿是有些风,不如咱们去取件大氅来给娘娘吧。''她见皇后颔首应允,便恭谨含笑,``娘娘且在这儿立一立,奴婢们速速就来。''

莲心便也顺水推舟道:``也好,那咱们再取些热茶来。''二人说罢,便匆匆去了。

皇后正看着月色清明如许,似一块牛乳色的软纱轻扬滑落,只听得舟后跟随的是苏绿筠的船,船上隐隐有女子说笑声如银铃婉转。她认得这些声音,细细听去,分明是蕊姬、海兰和绿筠。

皇后虽然不比晞月与如懿饱读诗书,可听着这健康而充满欢悦的笑声,不知怎的想起从前自己偶然看过的一首诗:``玉楼天半起笙歌,风送宫嫔笑语和。月殿影开闻夜漏,水晶帘卷近秋河。''

旁人风送笑语,自己却是病烦挣扎,孤凉一身。皇后心底愈加煎熬,正想要出声呵斥,只听见蕊姬的声音格外爽亮,躲也躲不过去似的直直逼来:``东巡前钦天监曾禀报说`客星见离富,占属中富一眚',以为是预示皇后娘娘将巡前钦天监曾禀报说`客星见离宫,占属中宫一眚',以为是预示皇后娘娘将有祸殃临头。如今看来,皇后娘娘病重,原来就是应了这句天象的。''

海兰的声音低低切切的:``皇后病了应着天象便罢了,可我怎么听说是应兆七阿哥的死呢。也真是可怜,这么小小一个孩子,发了痘疫说去就去了。''

绿筠连连念佛道:``阿弥陀佛,还好一场痘疫,只是殁了一个七阿哥,别的阿哥、公主都安然无恙,也算是神佛庇佑了。''

蕊姬看着绿筠,似是关切,亦是怜其不争:``纯贵妃便是太好性儿了。前几日我过来与姐姐说话,却看外头送来的贡缎独姐姐这儿短了两匹,姐姐却不争也不问,由着她们好欺负。后来还是嘉妃看不过,着人拿了自己的补来。''

海兰奇道:``竟有这般事?姐姐孩子多,本该多体恤些,谁知还总短了缺了的。皆是姐姐性子太懦的缘故。''

绿筠有些不好意思:``旁人便罢了,愉妃妹妹还不知道我么?但凡我的阿哥安保无虞,旁事我也懒得理会。再者\ldots\ldots{}''她微微沉吟,``皇后也是可怜,痛失爱子,病中嫁出独女,哪里还顾得到咱们这些小事。罢了罢了。''

蕊姬的笑语带着神秘的意味,道:``可怜?有什么可怜的?两位姐姐没听说过一种说法么?''

绿筠好奇道:``什么?''

玫嫔笑得极爽朗:``就是一报还一报啊!为娘的做了什么孽,便都报应到了孩子身上!二阿哥和七阿哥都是健健康康的好孩子,怎么会一个个都早夭了!追根宄底的事咱们都不知道,许多事咱们也都只是看见了果,没看见因而已。''

绿筠吓得脸色微微发白,忙下意识地站起身来道:``玫嫔,你还年轻,可别这样口无遮拦的,若是皇后娘娘听到了\ldots\ldots{}''

蕊姬撇一撇涂得朱红的唇,垂首拨弄着自己养得水葱似的三寸指甲:``哪里这就听见了?难道皇后不挂念她死了的儿子,没事儿将耳报神竖在咱们这里做什么?''

海兰听她这般说话,忙打了圆场笑道:``玫嫔是爽利人,有什么说什么罢了。''说罢又去按着绿筠,``贵妃姐姐也忒小心了。对了,我正有一事要问姐姐呢,上次姐姐说起哪位太医调理妇科一方极好,玫嫔身上老不大好,每月月信总害她受苦,姐姐若知道好的,也好请来给玫嫔妹妹瞧瞧。''

这话一起,难免玫嫔也经了心不觉红了眼圈,愁道:``自从我那可怜的孩子离了世,我这身子便是作下了病了,近一年来竟是一月不如一月了,如今总不能好好儿伺候皇上,虽说有着嫔位,恩宠到底不如从前了。''她瞥了海兰鬓边簪着的一朵烧蓝溜金蜂点翠蔷薇珠花,不免有些酸溜溜,``纯贵妃姐姐和愉妃姐姐都得了皇上去年七夕亲赏的六对珠花,贵妃姐姐是绣球的,愉妃姐姐是栀子的,这也是该的,谁叫两位姐姐都有阿哥呢。如今竟连比我年轻许多的舒嫔也挣上脸来,得了那真珠兰的珠花,我心里\ldots\ldots{}''

绿筠忙道:``说起来我也不大爱这些花儿朵儿的,也不大戴这些。你若喜欢,我着人取两对送你,如何?''

海兰知蕊姬失落,忙劝道:``你又不是不知道我,这辈子也就这么一个五阿哥罢了,有些赏赐也是皇上偶尔给的脸面。纯贵妃姐姐也是一心在两位阿哥身上。你还年轻,若调理得当,迟早也是有孩子的。''

绿筠子息颇多,听得这样的话难免动了心肠,三人密密说起来闺房私语来,又是一大篇话。

那边厢夜风徐徐之中,皇后却是一字不差,尽数落入耳中,``一报还一报''五个字,几乎如钉子一般实实锥在了她心上,痛得仿佛钻肺剜心一般。尖锐的痛楚排山倒海袭来,皇后一口气转不过来,只觉得无数面孔走马灯似的在眼前转着,直转得天地倒旋,不知身在何处。

皇后只觉得胸腔里一呼一吸格外艰难,正要唤人搀扶,忽然脚下一滑,足下的花盆底全然不受控制一般。船上本就不如平底稳当,皇后身体一个踉跄,还来不及惊呼,便从船尾处``扑通''掉进了冰冷刺骨的河水之中。

\hypertarget{ux7b2cux5341ux4e5dux7ae0-ux7405ux70e8}{%
\chapter{第十九章 琅烨}\label{ux7b2cux5341ux4e5dux7ae0-ux7405ux70e8}}

绿筠正与蕊姬、海兰在船上的阁子里聊得畅快,忽听得有重物落水之声,不觉止了声。海兰疑道:``什么东西落水了,还扑腾着呢?''

蕊姬侧耳听了须臾,不以为然地笑道:``怕是岸上什么东西落水了吧?也是的,夜深路滑的,路上行人落水也是有的。''

绿筠到底有些不放心,一双纤纤素手搭在窗扉上便想开启:``不如开窗看看,别是什么人掉下去了吧。''

蕊姬掸一掸身上极喜庆的桃红锦彩绣八团起花琵琶襟旗装,那衣裙上更是遍绣刺银枝满卉纹样,随着她的动作荡起点点银彩光晕。她笑着按住绿筠的手,漫不经心道:``开什么窗,仔细冷风扑进来伤了身子。''

海兰侧耳听了片刻,把玩着纽子上垂下的绿莹莹翠玉琉璃豆荚珮,笑生生道:``也是。人落水了会不呼救,只顾着扑腾?别是什么猫儿狗儿的,那边好玩儿了。''

三人说笑着,看了看合上的六棱朱漆窗扇,自顾自闲聊去了。

第一个发觉皇后落水的是凌云彻。

凌云彻本是皇帝身前最低等的御前侍卫,因御船比不得养心殿阔朗,而随行侍卫诸多,最低等的侍卫便被安排到了御船的最末护卫。

夹岸四周隐隐有花香浮动,凌云彻闻得出,那是新开的桐花的气味。往日里在家乡的时节,这样并不名贵的花开得夹道都是。桐花万里丹山路,开也烂漫,落也缤纷。他是读过几年私塾的,文字上虽不精深,却也知道些许。

那时春日迟迟,老夫子便摇头晃脑地念:``红千紫百何曾梦?压尾桐花也作尘。''那些散碎的句子,是少年时模糊而温暖的回忆。然而记得清晰的,分明是嬿婉春花般灿烂的明亮笑颜。嬿婉最喜欢的便是桐花。那绛紫柔白的花朵,有漫天铺地的清甜香气,让人几乎要醉倒其中。嬿婉便跳起来去攀折那繁盛花枝,可惜桐花总是长得那么高,她一壁极力去攀,一壁回首笑盈盈道:

``云彻哥哥,你瞧那桐花开得那样高,要是做人也能那么一辈子高高在上,便也好了。''

当日的笑语,如今已然遂愿。今时今日的嬿婉也算是得到她梦寐以求的高高在上了吧。龙舟上的丝竹管弦和鸣声声,水面倒映着夹岸人家的万千灯火,如同花影浮沉,映着这盛世繁华。而嬿婉,便是这繁华锦绣里开得极艳的一朵花。

锦上添花,固然美不胜收。

他这样痴痴地想着,仰首望见天际一轮近乎完满的月。近乎完美,便总有些许残缺。便如自己,也算是嬿婉春风得意后的一抹残影。有沉缓的春风柔暖拂过,玉白月光在粼粼暗金红的波光星点中漾动,连勉强维持的圆满也有了玉碎沉沙的势态,也许这就是他的人生,在失去心爱的女子之后,即便想要奋发图强,也不过是一个小小的最末等的御前侍卫,受尽那些出身贵族的侍卫的冷眼与暗讽。

连样的苍凉孤寂之中,唯有那个人,那个曾与她一同在死寂如坟墓的冷宫里挣扎的女子,偶尔投来的一瞥含笑的眼,激励着他忍耐下去,继续去寻找可以撑起未来的任何微小的契机。

所谓半分残缺的圆满,大概如是。

惊动凌云彻痴念的,是那一声突然的响动。

他分明看见,皇后以极其古怪且不自然的姿态落入水中。

有那么一瞬,几乎是本能一般,他冲上前一步,想要将落水之人救上来。

可毕竟久在宫中,他很快发觉了奇怪之处,尽管皇后的青雀舫与嫔妃所居之船的距离并不近,但皇后的侍女们,都并未随在身侧。

他警觉地止住脚步,不肯再向前。心中惊动的一刻,忽而念及如懿在冷宫的无限苦楚,与眼前落水的女子,无一不隐隐相关。

如懿,她是在自己那样困窘时唯一伸出手的人,他不能不去揣想她的敌意。但若真似如懿所期待的那样,自己的前程来路有所指望,那么此刻,是平生再难一得的时机。

已然不能停驻,向前或退后,都是举步维艰。

河中水花翻腾,隐约是女子的明黄服色,如同月光碎裂的倒影,起伏于河水中央,惊起粼粼波泽,他从未这般为难过,一颗心像是成了一撮烟叶子,被汗湿的手心来来回回地揉搓着。须臾,他的面色渐渐淡然,逐渐成了一种彻骨的冷漠,如同眼前冰冷的河水的泛波。他静静注目,直到看着河中的水花泛起的波澜越来越小。他脸上的肌肉微微一搐,再无半分犹豫,跃身跳入水中。

皇后被救上来时,几乎只剩下一口气。合宫慌乱,随行的太医被急急召往青雀舫诊治,连太后和皇帝亦被惊动,急急赶往守在皇后阁中。

皇帝焦急地踱来踱去,懊恼道:``朕本与娴贵妃在赏画,因觉得风声略显嘈杂,才传了乐班弹奏,谁知丝竹盈耳,竟未听见皇后落水之声。''

太后轻叹一声:``皇后也真是不当心了。''说罢,便又数着手中的佛珠,默默念念有词。素心和莲心都吓坏了,跪在地上瑟瑟发抖。皇帝看着二人的模样便生气,喝道:``李玉,给朕狠狠掌这两个贱婢的嘴。''

李玉答应一声,撩起袖子便开始下手。

皇帝听着皮肉相击的声音噼啪作响,犹不解气,叱道:``身为皇后的贴身侍婢,竟然不时时跟着,才致使皇后落水,杀了也不为过!''

嫔妃们守在下首,眼看二人挨打,更是不敢作声。一屋子莺莺翠翠沉默不语,气氛愈加显得沉闷不已。绿筠听见说皇后是落水,又恰好是在她们闲聊的时候,心下便有些慌,生怕皇帝是知道自己与海兰、蕊姬在一起而没发觉皇后失足落水,便想自己开口分辩几句。海兰在旁侧看她嘴唇一动,知道她要做什么,连忙在身后扯了扯她的衣袖,望着自己的鞋尖恍若无意地摇了摇头。绿筠犹自不安,但见蕊姬只是百无聊赖地拧着绢子玩儿,便也勉强安定下心神。

太后听了一会儿,终于耐不住道:``停手吧。说到底也是皇后让她们去取东西才没跟着的。平日这两个丫头都还算尽心,还要留着伺候皇后的。''

太后这句话多半有安慰皇帝说皇后身体无事的意思。皇帝忍耐着道:``罢了。''

如懿立在绿筠身边,船在水上漂浮,总觉得足下不安稳似的晃动。太后缓声道:``该罚的也罚了,听说救皇后上来的是皇帝身边一个低等的御前侍卫,是么?''

如懿低眉颔首道:``是。当时凌侍卫发现皇后娘娘落水,便下水施救。''

太后点点头,李玉忙道:``那侍卫是皇上御前最末等的蓝翎侍卫,叫凌云彻,汉军旗正红旗包衣出身。此刻刚换了衣裳,在外头候着回话呢。''

太后颔首不语,只看着皇帝。皇帝的心思并不在这个上头,随口道:``既然是蓝翎侍卫,那就传朕的旨意,救护皇后有功,赏白银三百两,升为三等侍卫。不必叫他进来谢恩了。''

如懿淡淡含笑,余光所及之处,见站在最末的嬿婉神色稍不自在,便转过首只看着李玉传旨去了。

齐鲁从皇后殿内出来后,面色便灰扑扑的不太好看,但见皇帝焦灼,忙回道:``皇上,皇后娘娘腹中的水都已经控了出来。经微臣和几位太医诊脉,落水对娘娘凤体影响不深,但看娘娘脉象,乃是急怒攻心,心力交瘁之状,此刻痰气上涌,已经迷了心窍。而且皇后娘娘的神志一直未曾清醒,说着什么`一报还一报'的话,只怕\ldots\ldots 只怕\ldots\ldots{}''

绿筠听得齐鲁的话,不自觉地往里缩了又缩,恨不得融在人群早才好。

皇帝心中猛地一沉,已然知道不好,一时恼道:``只怕什么?''

太后瞥了一眼战战兢兢的齐鲁,长叹一口气:``哀家一把年纪了,还有什么听不得的。你便直说罢了。''

齐鲁道:``皇后娘娘气虚体弱,是油尽灯枯之兆,只怕是在弥留之际了。''他不停地擦着额头的汗,结结巴巴道,``但\ldots\ldots 但\ldots\ldots 皇后娘娘福泽深厚,上天庇佑\ldots\ldots{}''

齐鲁话未说完,和敬公主已经忍耐不住,呜咽着呵斥道:``你胡说什么?皇额娘正值盛年,怎么会油尽灯枯?分明是你们医术不够,才胡言乱语!''

太后看了一眼福珈,福珈忙上去扶住了和敬公主,小声地劝慰着什么。太后见皇帝端着茶盏的手凝在了半空中,微微摇了摇头,伸手替皇帝取过茶盏,温和道:``皇后病得凶险,太医这样说也是情理之中,也唯有齐鲁这样何候多年的人才敢直说。不管皇后境况如何,皇帝,得赶紧通知内务府的人在京中将喜木准备着,哪怕冲一冲也是好的。''

皇帝吃力地闭上眼睛,发白的面孔如被霜雪蒙被。殿阁中静极了,只听到河水蜿蜒潺涴之声,恍若流淌的生命,静静消逝。良久,皇帝才能出声:``一切但凭皇额娘做主。''

太后微微颔首,吩咐道:``齐鲁,好好儿在这儿领人伺候着,有什么动静,赶紧来回禀哀家。''她放柔了声音,``皇帝,你多陪陪皇后吧。''太后挥了挥手,示意嫔妃们出去。嬿婉有些依依不舍,还想跟皇帝说些什么,但见太后目光严厉森寒,也不敢多说什么,只得随着众人退出去了。

嬿婉本就落在人后,徐徐步出船舱,但见凌云彻已守在船头,似是戍卫皇帝。她目不斜视,淡淡道:``恭喜,这么多年,终于迸益了。''

凌云彻并不看她,不卑不亢道:``多谢令贵人。''

嬿婉望着浑浊的河水,仿佛他不存在似的,自言自语道:``拼了性命去救皇后才得一点小小晋升,值得么?''

凌云彻的神色淡得不见丝毫喜怒:``贵人用血肉之躯去换取的,微巨也是一样。既然贵人觉得值得,微臣自然也不会为难。''

嬿婉听出他语中讥诮,不觉莞尔:``原来,你还是在乎的。''说罢,她只报以一丝了然的冷艳笑意,径自离开。

云彻本也不欲多留,方才如懿扶了惢心的手出来,目似无意地剜了他一眼,他便已然会意。眼见嬿婉纤柳似的身姿盈然离去,他只觉得满腔郁塞之情亦如明月出云,稍稍纾解,便觑着空隙,悄悄往如懿船上去了。

如懿甫坐定抿了一口茶水润泽焦枯的唇舌,便见惢心引了凌云彻进来。她漫不经心地瞥他一眼,淡淡笑道:``恭喜了。''

凌云彻见她笑意淡淡落落,分明不似素日一般熟络,心中没来由地一慌,旋即跪下道:``微臣侥幸,得此机遇,实在是意外荣耀。''

如懿何等耳聪目明,眼波微微一沉,宛然间似明月照射下的寒冰千丈:

``你是说,你救了皇后,不是偶然?''

凌云彻俯身,一脸诚恳:``微臣不敢辜负小主劝诫,极力自强。这次机会实在千载难逢,但微臣也从未忘记小主冷宫之苦,小主的敌人,便是微臣的敌人。同仇敌忾之意,微臣时刻牢记,所以皇后落水后片刻,微臣才跳下水去救。''

如懿的面色稍稍见霁,轻拢的云鬓便簪着一支鎏金玉蝶银丝镂翅步摇震颤不已:``谢你有心想着,进退都保全了自己与旁人。''

凌云彻微微思忖:``多谢小主体恤,只是微臣眼见皇后孤身落水,实在不是寻常。''

``你也觉得古怪?''如懿眸中一亮,唤过惢心,``你方才告诉本宫什么,再说给凌侍卫听一遍。''

惢心恭声道:``是。奴婢发觉,皇后失足落水之处,有新刷桐油的痕迹。桐油防水,涂上也无可厚非,但也应该是船只下水前便涂抹好的。咱们出巡改走水路那么久,才突然涂上,岂不奇怪?''

凌云彻一怔,旋即道:``桐油滑腻却无色,涂上后不过许久就会干透,根本无迹可寻。若真是有心,那当真百密而无一疏。''

如懿的思绪有一瞬的飘忽:``原以为只有自己恨透了皇后,原来还有人比本宫更想要她死呢。''

绿筠回到自己船上,过了好一会儿,一颗心犹自惊荡不已。正好可心端了一碗牛乳燕窝来,绿筠立刻接过一气喝下。可心惊异不已:``小主是累着了还是饿了,仔细呛着。''

绿筠慢慢抚着心口,小指上的白银玛瑙粒珐琅护甲闪着幽微的光泽,如她此刻一颗惴惴不安的心。她正犹豫着要不要让可心去请海兰和蕊姬过来说说话,只见深翡花色金丝边帘子一闪,一个穿着百合粉色小金福字锦袍的女子闪身进来,口中道:``皇后娘娘病重,姐姐这儿离皇后娘娘的青雀舫最近,我心里慌得很,还是来姐姐这儿坐着等消息吧。''

绿筠正巴不得海兰来,听得这一句,便往榻上让了让,急惶惶道:``我正等着你来呢。可心,去上壶好茶来。''

海兰奇道:``我是借姐姐的宝地候着消息,若皇后娘娘有什么动静,咱们也好过去。怎么姐姐倒盼起我来了?''

绿筠忙拉住她的手,推心置腹道:``方才齐太医的话你可听见了吧?说皇后娘娘从水里捞上来之后,一直在说什么一报还一报的。我想着皇后娘娘的船就在咱们的船前面,不会是方才我们说的话,那么巧便给她听去了吧?''绿筠心慌意乱,``要是皇后娘娘苏醒,找我们算账可怎么好?都怪玫嫔说话没遮没拦的,还扯着嗓子说这些话,如今可害了我了!''

直到可心送上茶水来,绿筠才按住了惶急的神色,勉强静了片刻。海兰腻白的手指摩挲着细白如玉的瓷盏,仿佛二者浑若一色一般。她含着一缕宁静的笑意,斜签着身子坐着,恍若一枝凝在风中不动的雪白辛夷花。然而海兰面上的宁和之色是秋阳底下的涟漪,微微漾着炫目的光晕,是细细碎碎的不安定,她亦有些疑色:``说来,玫嫔不是说话这般不稳重的人,今日不知是怎么了?''

``怕是玫嫔又想起自己的孩子,浑身不自在。都这些年了,她也真是可怜见儿的。''绿筠见宫人们退下了,复又急道:``愉妃妹妹,你说皇后娘娘要真来寻我的麻烦可怎么办,还是我自己先去跪着请罪?''

海兰见她真着了慌,笃定笑道:``皇后娘娘都那样了,如何会来寻姐妲麻烦?且到底也是玫嫔说话不谨慎,姐姐且安心坐在这里,好好儿看着三位阿哥,做您的贵妃娘娘就是。''

绿筠犹自不解,发髻上一支汉白玉红珠风钗沥沥作响,晃得如风摆杨柳,显是担心不已。海兰轻轻吹着茶水,氤氲的热气拂上面来,那朦胧的淡淡白色,似乎是为她的原本柔和的面庞更添了几许可亲。

海兰温言道:``皇后娘娘是不敢来找姐姐的。她听了咱们这一句`一报还一报',就能吓得失足掉进河里去,被捞上来了还絮絮不止。皇上虽然担心皇后,但听见这些话,只怕皇上心里也在犯嘀咕,皇后娘娘是做了什么见不得人的事,所以才到了这个地步?''

绿筠稍稍松一口气:``真不干咱们的事儿?''

海兰笑道:``真不相干!''

绿筠抚着胸口,笑逐颜开:``阿弥陀佛,那就好!方才吓得我\ldots\ldots{}''她神色忽然一敛,又有些不自在起来,``说到报应,七阿哥死了,皇后又成了这个样子,愉妃妹妹,不知怎的,我总想起那时永琏夭折时的样子\ldots\ldots{}''她的瞳仁碌碌转动,十分不安,``二阿哥的死,到底是咱们\ldots\ldots{}''

海兰脸上的笑意猛然一收,露出几分悲悯的神色:``贵妃姐姐悲天悯人,真是菩萨心肠。二阿哥的死,哪怕咱们再惋惜,也是没有办法。''她清冷的口吻里多了几分无所畏惧的坚毅,``从大公主的夭折,到二阿哥,再到七阿哥,连着皇后娘娘自己,这都是命。姐姐您福德双全,正是您曾经积福,所以三阿哥和六阿哥这样福寿平安。这正是从前你做的,都是好事,没有错事。''

其实自从生下永琪之后,海兰虽然被封为愉妃,但她身体丑陋,已经多年不能侍寝,也不可能再得到皇帝的欢心。也曾在生下永琪后三年,有一次,皇帝一时兴致想到了她召进养心殿侍寝,但是当她被锦被裹着抬入养心殿寝殿后不到一刻,便被送了出来。恩宠于她,已经是再难得到的东西。所以这些年来的海兰,活得太像太像一抹云淡风轻的影子。也便是这样一缕影子般的生存,才让她可以游走于嫔妃之间,从容自得,亦不让人戒备厌烦。

绿筠听得她这样的话,终于松弛下来,握住她的手感泣不已:``好妹妹,幸好你开解我,否则我可真是怕呀!''

\hypertarget{ux7b2cux4e8cux5341ux7ae0-ux85a8ux603f}{%
\chapter{第二十章 薨怿}\label{ux7b2cux4e8cux5341ux7ae0-ux85a8ux603f}}

太医的汤药不断灌入之后,皇后终于在亥时一刻清醒过来。皇后的脸色不复方才绝望般的死白,反而多了一点点珊瑚色的红晕,人也有了力气,可以慢慢说出话来了。

她轻微地咳嗽几声,隔着薄薄的素纱屏风,看见外头一道明黄的影子,知道是皇帝守在外边,她齑粉般碎凉的心头微微一暖,吃力地道:``皇上\ldots\ldots{}''

齐鲁闻言出来:``皇上,皇后娘娘醒了。您\ldots\ldots{}''

皇帝的神色痛苦而疲惫,手边的浓茶喝完又添上,已经好几回了。他听得齐鲁来请,便起身道:``朕去看看皇后。''

皇后的殿阁中有浓重的草药气味,混着一个女人行将就木时身上散发出来的颓败气息。那种气味,好像是深地里开到腐烂的花朵,艳丽的花瓣与丰靡的汁液还在,却已露出黑腐萎靡的迹象。

皇帝陡然升起一股怜悯与悲惜,却亦不自觉地想起,他去看望晞月时,晞月临死前的那副样子。晞月垂死的面孔与皇后的脸渐渐重叠在一起,皇帝蹙了蹙眉头,嘴角蕴了一缕彻寒之意,还是坐在了皇后床前,温沉道:``皇后,你醒了?''

皇后的眼角滑落两行清泪,绵绵无力地滑过她苍白而发皱的面庞,缓缓道:``皇上,臣妾与您结发多年,经此一劫,即便太医不说,臣妾也知道自己寿数无多了。可臣妾不曾想,一睁开眼来还能一眼看到您在身边。皇上\ldots\ldots 臣妾,臣妾真的很高兴。''

皇帝的语气轻柔得如同三月的风,熨帖而暖融:``皇后,不要说这样丧气的话。好好儿歇着,你只是落水后受惊,养一养便会好的。''

皇后想要摇头,但此刻,摇头对她而言业已是十分劳累之事,费了半天力气,她也不过是轻轻地偏了偏头:``皇上,臣妾自己的身子自己知道。臣妾无福,无法为您留住嫡出的阿哥。如今至少璟瑟已经有了好归宿,臣妾请求皇上,不要因为臣妾离世,而让璟瑟守丧三年再出嫁。明年,明年就是个好年头。再不然,就当她早就嫁去了蒙古,明年只是补上婚仪罢了。她已经十七了,从前是舍不得她嫁人,如今却是耽搁不起了。''

皇帝颔首,眼角有微亮的泪光:``璟瑟是朕与皇后唯一的嫡出之女,朕一定会好好疼惜她。皇后安心即是。''他沉吟片刻,似是下定决心,``再不然,朕就破例准许璟瑟出嫁后可另立府邸,与额驸留驻京师。''

皇后眸中一亮,颇有欢欣之意:``臣妾多谢皇上。皇上,可臣妾还有一事相求。臣妾自知无福,上天不肯垂爱,只怕是时日无多了。''她挣扎着想要撑起身子,却也实在是无能为力。皇帝伸手扶住她半边身体,欲要出言相劝,却见她一脸执着,只得道:``皇后有什么话,但说便是。''

皇后依着皇帝的手臂,分明觉得他的手不甚用力,虽是扶着自己,却有着克制的距离和力气。这些年,他与她,名分上是结发夫妻,可这份相守之情,何尝不是如此?这样健硕而温热的身体,却从来不是只属于自己的。皇后油然而生无限凄苦之意,只觉得半生好强之心,尽数化作了一摊灰烬。无数言语挣扎着要从她舌尖蹦将出来,喘息了片刻。方能定住心神:``皇上,臣妾自知不久于世,虽然舍不下与皇上多年情意,但臣妾亦知,天际不可无月,后宫不可无主。''她仰起身,保持着最后一丝皇后的尊严,郑重道,``臣妾以执掌凤印的六宫之主身份,向您举荐继后人选。纯贵妃苏氏诞育皇子,于社稷有功。谨慎侍奉,温厚襄赞,她的德行足以在臣妾身后执掌后宫,继任皇后。''

皇帝眸中一凉,像是秋末最后的清霜,覆上了无垠的旷野。他依旧含着最温和得体的微笑,让人不自觉地生出亲近之意:``皇后多虑了,你会好起来的。''

皇后咬着暗紫的下唇,勉力摇头:``臣妾知道,臣妾是不能了。臣妾的二公主、二阿哥和七阿哥都在下面等着臣妾了。皇上,纯贵妃她\ldots\ldots{}''

皇帝的笑意沉了沉,勉强再度浮起:``皇后,这些事不该是你思量的。皇后不仅是一个称呼,一个身份,更是朕的枕边人。那是朕该量度的事,而不是你。''

皇后的面色逐渐发青,像一块碧色沉沉的玉,却无半点润泽的光华,她笑容凄苦如残叶瑟瑟:``皇上,恕臣妾多嘴一句。纯贵妃、舒嫔,哪怕是您要另选女子为中宫,臣妾都不担心。可有一个人,断断不能。''她眼中闪过残忍而怨毒的光芒,``娴贵妃出身乌拉那拉氏,先帝的景仁宫皇后有多恶毒,您是知道的。这样的女人的后裔,断断不能入主中宫。''

皇帝还是那样平静的口吻,却多了一丝显而易见的冷漠:``皇后,朕讲过,你是多虑。多虑的话朕是不会听的。''

皇后眼中有抑制不住的痛苦,跳跃着几乎要迸出森蓝的火星:``皇上,臣妾自嫁入潜邸,您便只叫臣妾为福晋。臣妾得蒙皇上垂爱,正位中宫,您却也只称呼臣妾为皇后。福晋与皇后,不过是一个身份和名号而已。''她喘息着道,``皇上,您很久没有叫过臣妾的名字,您\ldots\ldots 您记得臣妾的名字么?''

皇帝坐在床沿上,安抚地拍拍皇后的手:``皇后,你身子不好,不要再伤神了。''

皇帝的指尖所经之处,有男子特有的温暖力度,让身体渐渐发冷的皇后,生出无尽的贪恋之意。曾经,曾经这双手亦是自己渴盼的。可从未有过一日,这双手真正属于自己。这一日,它拂过谁红润而娇妍的面颊;那一日,或许又停留在谁饱满而蓬松的青丝之上。皇后这样恍惚地想着,眼中闪过一丝心痛而不甘的光芒,像是划过天际的流星,不过一瞬,就失去了光彩。``皇上,臣妾的名字,名字是\ldots\ldots 琅嬅,是`琅媚福地,女中光华'的意思。''

皇帝点点头,眼里露出几分温情,柔缓道:``你的名字。很像一个皇后。''

``皇上!''皇后枕在床上,忽地仰起身子,激烈地喊了一声。那声音太过仓猝而凌厉,有着玉碎时清脆的破音。

外头即刻有宫女入内,小心唤了声:``皇上,皇后娘娘有何吩咐?''

皇帝温和地摆摆手:``下去吧。皇后只是叫朕一声罢了。''他停一停,又吩咐道,``没朕的传唤,都不许进来扰了朕与皇后说话。''

宫人们恭谨退下,皇后的神色软弱下去,半边削薄的肩靠在苍青色嵌五蝠金线的帐上,整个人恍如一团影子,模糊地印在那里。她的喉间有无声而破碎哽咽:``皇上,为什么臣妾想得到您如妻子一般呼唤一句名字。是这么难?臣妾有时候真的不甘心,也真的害怕。''

皇帝轻轻一嗤,似是不能相信:``害怕?你是富察氏长女,曾经的宝亲王嫡福晋。朕的中宫皇后,你有什么可怕的?所谓不甘心,也不过是你贪婪过甚,不肯满足而已。''

烛光盈然照亮一室的昏沉,却仿佛照不亮她暗郁心境。这一刻,她并不像一个母仪天下的尊贵之女,反而像某种瑟缩墙角不能见到天日的阴湿植物,怯弱而卑微。她的神思不知游离何处,痴痴道:``臣妾自闺中起就被教养要如何做一个正妻。相夫教子。主持家事。能够嫁与皇子,是臣妾的福气。臣妾自知道这个消息起,每一日欢欢喜喜,满怀期盼。哪怕是知道诸瑛先嫁与了皇上为格格,臣妾也不过是稍有忧伤,转头便忘了。可皇上,直到臣妾嫁给您的那一天起,臣妾才知道自己的日子并不好过。您有那么多的宠妾,除了族姐诸瑛,高氏娇柔,有她阿玛辅佐您:乌拉那拉氏骄傲,出身却高贵。二人专宠,连臣妾这个嫡福晋也不得不让她们两分。个中委屈,皇上何曾在意过?您眼里的妻妾争宠,不过是区区小事,而在臣妾眼里,却是攸关荣辱的莫大之事。还好她们彼此争锋不得安宁。但臣妾知道,无论她们谁赢,下一个要争的就是臣妾的福晋之位。还有后来的金氏妩媚,苏氏纯稚,臣妾才发现。原来自己从未真正拥有过一个完整的夫君。可臣妾不能怨,不能恨,更不能诉之于口,失了自习的身份。臣妾真的很想忍,很想做一个好妻子,对得起自己多年教养。可臣妾也不过是个女人,想得到夫君的爱怜,看着您夜夜出入妾室阁中,看她们娇滴滴讨您喜欢,臣妾身为正室,虽然不屑这样讨好,可心里如何能好过!''

皇帝似乎不忍,也不愿听下去,他的口吻淡漠得听不出任何亲近或疏远,仿佛一个不相干的人一般,只道:``皇后多虑了。''

``多虑?''皇后的唇边绽开一丝冷冽而不屑的笑意,仿佛一朵素白而冷艳的花,遥遥地开在冰雪之间,``臣妾并非多虑,而是不得不思虑。您抬举高晞月的家世,抬举她的父亲高斌!您暗中扶持乌拉那拉如懿,哪怕她在冷宫之时,您身边还留着她的那块绢子,从未曾忘记她桩桩件件。臣妾如何能够安稳?皇后之位固然好,可历朝以来,宠妃恃宠凌辱皇后之事比比皆是。您喜欢的女人越来越多,您的孩子也会越来越多。臣妾和臣妾的孩子们,得到的眷顾就越来越少。臣妾如何能不怕,如何能甘心?臣妾\ldots\ldots 臣妾没有一日不是活在这样的畏惧之中不得安生。''

``不得安生?''皇帝冷然相对,以唇际不屑的笑意划出楚河汉界般分明的距离,``你有尊贵的出身,嫡妻的身份,儿女双全,位极中宫。你还有什不得安生的?''

皇后的呼吸渐渐受窒,急促而沉重,那声音如错了点的鼓拍,绝望地敲打着。胸中忽然大恸,他的疏离,原来就是她的绝望。那样前所未有的绝望,盘根错节占据了她行将碎裂的身心。

``皇上,您对臣妾若即若离,臣妾从来也抓不住您的心。臣妾知道您要取笑了,可您想过没有,寻常妇人抓不住夫君的心也罢了,可臣妾是皇后,六宫的人堆到一块儿,臣妾站在峰巅上。臣妾没有什么可以依凭的,若您的心意变化,臣妾所拥有的貌似安稳的一切便会烟消云散。''皇后的哭声哀怨沉沉,她本是虚透了的人,如何经得住这样激烈的情绪,不得不躺在床上仰面大口地喘息着,如同一条离开水太久的行将干枯的鱼,殿阁里静极了,青雀舫偶尔随着水面的波动均匀而和缓地起伏,像遥远的时候母亲轻轻摇晃的摇篮,催得人直欲睡去,直欲睡去。鎏金烛台上的红烛烧得久了,烛泪缓缓垂下,嗒一声,嗒一声,累累如珊瑚珠一般。

皇帝静静侧耳,听着周遭细微的响动,良久,他亦动容:``皇后,你从未对朕说过这么多话,从来也没有。所以竟连朕也不知道,原来你是这样不安稳,这样害怕。只是皇后\ldots\ldots 人的愿望不能太多,太多了,连神灵都不会庇佑。朕自己不是嫡母所生,自小受了不少委屈,所以格外盼望自己的太子能是皇后嫡出。所以朕敬重你,容忍你,也疼惜你所生的两位阿哥。哪怕永琮还在襁褓之中,朕也已经有立储之意,这些你都是知道的。为着阿哥们来日的名声,许多事,朕都睁一眼闭一眼。只作不知。''皇帝忽然放缓了声音,俯下身子,略带神秘之色,在皇后耳边低语如昵喃:``其他的事也罢了,朕听过只当是脏了耳朵,掏干净便是。但过些日子就是哲悯皇贵妃的生辰了,朕一直很想问问你,你的族姐诸瑛,她到底是怎么死的?每逢她生辰死忌,你便没有一点不安么?''

仿佛有惊雷隆隆滚过天灵之上,皇后身体剧烈地一震,睁大了浑浊含泪的颤声道:``皇上。多年来宫中一直传言是臣妾嫉妒诸瑛生下长子,所以害死了她!原来您也是这么想的!''

皇帝俊挺的面庞上疑云深重:``那么阿箬呢,既然阿箬受你安抚指使,那么玫嫔和怡嫔的孩子枉死,自然也是你了,是不是?''

皇后的声线陡然凄厉,高高抛向云际,复又举起右手指天道:``臣妾发誓,臣妾用富察氏全族百年的荣耀和福祉发誓,诸瑛之死,绝非臣妾所为!而玫嫔与怡嫔之子的的确确是娴妃所害,不干臣妾的事!''

皇帝伸出手,轻缓地握住她指天发誓的右手,温和道:``皇后真是病糊涂了,誓言若是有用,朕还要纲纪法度做什么?''

皇后失血的双唇剧烈地颤抖:``臣妾一生所为,无一不是为了保全富察氏尊贵的荣光,为了对得起富察氏列祖列宗用血汗换来的荣光!不到逼不得已,臣妾何必置人于死地,留下威胁富察氏全族的嫌隙?皇上,臣妾爱子私心,是想让永璜自生自灭,也曾故意纵容永璋娇生惯养,可臣妾从未想过要他们死啊!更迫论除去玫嫔、怡嫔之子!她二人出身微贱,便是生下皇子又如何,也断断不会动摇嫡子之位,臣妾费这个心做什么?''

``做什么?''皇帝轻嗤一声,``你自己已经说得明明白白,是为了你心心念念的富察氏一族!如懿的姑母是先帝皇后,你一直忌惮她的出身,也不喜她的性子。除了玫嫉与怡嫔之子,顺带着也除了如懿,岂不合你心意?再者,玫嫔与怡娘出身低贱,那么如懿和慧贤皇贵妃若诞下皇子,你便会觉得是在动摇嫡子之位了吧?哪怕对着一直顺服你的慧贤皇贵妃,你不也赐了她那么珍贵的翡翠珠缠丝赤金莲花镯以防来日么?便是如懿进了冷宫,蛇咬火焚,饮食加害,你不也做得得心应手!''

有片刻死寂,几乎要逼得人发疯。皇后哑声笑了起来,似是用尽了所有的力气,凄然呼道:``是,臣妾是防着身份高贵的宠妃生子,是深恨如懿从前的张扬而在她入冷宫后加以折磨,也曾因为高氏告诉臣妾如懿在冷宫诅咒永琏而欲杀之泄愤。可冷宫失火之事,如懿中毒之事,臣妾真心不知!''她恨到了极处,惶惑地望着四周,枯瘦的手如雪中的残枝紧紧牵缠着床帐上垂落的杏色绞银线流苏。那流苏原是极韧,勒得她的手割出或青或紫的印痕,皇后死死攥着不放,仿佛只有如此,才能撑住自己随时都会倒下的身体似的。她原本温和端庄的杏眼睁得滚圆,几乎要核突暴出,她凄厉地嘶声道:``这些事,是谁害臣妾?是谁要害死臣妾?''

``谁要害死你?''皇帝忍无可忍,鄙夷道:``自作孽,不可活。你便是自己害死了你自己!''

皇后的目光倏地一跳,骤然死死盯在皇帝身上,由炙热而至冰冷,她的神情近乎痴狂:``原来这些事皇上早就知道,却隐忍至今才来问臣妾。这究竟算是您的恩典还是臣妾的冤孽?''

皇帝的神色平静如水,话语的锋利藏在悠然语调中:``这些年的你的所作所为,朕从旁人口中也算略知一二。你私德有亏,但你是朕的皇后。作为一个皇后,你为朕生儿育女,也算节俭自谦,对着嫔妃也未有忌妒尖酸之色,算是御下宽和,不曾让天下臣民有半分议论。朕若揭破你,只会让你成为朕山河岁月里的污点,让皇室成为天下人的笑柄。''就像一袭华美的衣袍,纵使底下虫蛀蚁蚀,破败不堪,他也得保留着外表的金玉绮丽。多年夫妻,恩情固然不会少,但她屡屡进逼,不曾领会他的提点,也终将那些年的恩情积郁成了难以言说的厌烦。只是想起他们共同的孩子时,那样纯真的笑脸,才会让他的情绪稍稍缓和。他知道她本性温和,并不如后来所知的那样凌厉,也知道她会极力维持着这样的温和过下去,只不过来日,终究会渐渐疏远,只剩下礼仪所应有的客气。

皇后静静地听着,所有的情绪在她的克制下渐渐平息,终于回到如常的雍容与宁和。她挣扎再挣扎,终于支撑着俯身拜下,冷然道:``皇上这么顾及皇室颜面,顾及自己的颜面保全臣妾,实在是圣恩滔天。''她仰起脸,目视皇帝,``既是皇上恩惠,那臣妾不能不报,就恕臣妾直言一句。臣妾固然是为了富察氏一族殚精竭虑,您又何尝不是为了自己的心意无所不用?您这样的性子,固然圣明聪敏,但亲近之人,无不为此所伤。事到如今,臣妾做的孽臣妾自己担着。可来目无论谁为继后,有您在一日,只怕下场都不会好过臣妾今日!臣妾就睁着这双眼睛,在天上看着!''

皇帝施施然站起身,全然不以为意,行至紫檀雕牡丹圆桌前,瞥了一眼桌上的茶点,沉声道:``今世之事未有定数,皇后还想着身后的因果么?皇后还是好自保养着,朕与你的日子还长着呢。''

皇帝走到殿阁外,一阵冰凉的水上夜风扑面而来,无声无息地贴附在他的身体,像不曾经意的侵袭。他不自觉地打了个寒噤,心底原本极力压着的恼怒之情,腾地窜起密密的火舌,和着皮肉被舔灼时的焦苦气味,竟有了一缕怜悯之意。这样端正持重的女子,垂垂之际,竟也会如此凄厉哀戚。他从未想过,如她一般的望族之女,也会如自己那些出身寒微的妾室一般,婉转渴盼着他的温柔。

那一瞬,有一个念头,几乎如滚雷般震过他的心头。如果,琅嬅说的是真的;如果,她其实并未做过那么多错事里如果,对如懿和后宫种种挫磨真的仅止于阿箬的无知和刻毒。

那么这个女子,是不是也曾被他错过了许多?

神思蒙昧的瞬间,他突然忆起从前,红烛摇曳成双的那刻,他也曾真心期待过,可以得到一位贤惠温柔的名门闺秀,相伴一生为妻。

琅嬅,固然不是他自己的选择,却也不失为一个很好的选择。他掀起金线绫罗红盖的那一眼相遇,她也曾真心而期待地说过:``妾身愿以富察氏的百年荣光,相随夫君左右,为夫君生儿育女,为贤良妻室。''

或许曾经,他们都曾真心地期盼过,未来的曰子可以风光明媚,永无险途。

却最后,他和她一一失去自己共同的孩子。长女,次子,第七子。唯余下一个璟瑟,如今也要嫁为人妇,不得承欢膝下。

一场数十年的姻缘所得,只能留下这些么?

皇帝用力摇了摇头,似要摆脱这种不悦情绪的困扰,索性迈步朝前走去。李玉早已带人候在外头,见皇帝独自负手出来,觑着皇帝的神色,乖觉地问道:``皇上的脸色不太好看,是为皇后娘娘的病情担心吧?皇上真是情深义重,一直陪着皇后娘娘。''

皇帝并不回答,李玉忙收了话头,恭谨问道:``皇上,夜深了。请旨,去哪儿?''

皇帝扬了扬脸,不假思索道:``去娴贵妃处。''

李玉响亮地答应了一声,扶了皇帝道:``嗻。皇上起驾。''

一行人迤逦而行,不过几步,只听得身后哀声大作,宫人们放声大哭。赵一泰疾奔而出,跪倒在皇后的青雀舫外悲声大呼:``皇后薨逝------''

皇帝怔了怔,有冷风猝不及防地扑进他的眼,扯动他的睫,那样细微的几乎不可察觉的疼痛,如细碎的裂纹,渐渐蔓延开去。他的声音恍然有几分凄切,在深沉的夜色里如碎珠散落:``永琏,永琮,你们在地下别怕,你们的额娘来陪你们了。''

\hypertarget{ux7b2cux4e8cux5341ux4e00ux7ae0-ux6697ux6d8cux4e0a}{%
\chapter{第二十一章
暗涌(上)}\label{ux7b2cux4e8cux5341ux4e00ux7ae0-ux6697ux6d8cux4e0a}}

乾隆十三三月十一日亥时,皇后富察琅嬅薨于德州,年三十七。

皇后薨逝那夜,皇帝一直静静坐在自己的龙舟之内,深深的沉默仿佛巨大的山脊将皇帝压得沉重而无声。如懿闻得消息,早已换过一身素净衣衫,只以素银钗并白色绢花簪鬓。皇帝俊朗的面容在昏黄烛火的映照下,有着虚弱的苍白。想是许久未眠,他的眼微微地肿着,暗红的血丝布满青白色的眼底,如纵横交错的血网。

如懿依在皇帝身边,两个人的影子重叠在一起,仿佛只有一个似的。相对亦是只影寂寥。夜风吹起涌动的水波,拍在船身之上,悠悠荡荡发出沉闷绵长的声音,和着远远传来的哭声,缓而重地拍在心上。

皇帝定定地看着如懿,半晌之后才幽幽地轻叹一口气:``皇后死了,但她至死不认。''

如懿握着他的手,冰凉冰凉的手指,和自己的一样,彼此抵触交缠,却始终暖不过来。她的神情平静至极,徐徐道:``至死不认,也已经是做下了的事情。''

皇帝斜倚在椅上,明明是乍暖微凉的春夜,他的长吁如叹,却是秋色初寒的冷:``皇后拿着富察氏百年的荣耀和福祉发誓,她做过的她认,可冷宫失火之事,玫嫔与怡嫔失子之事,她至死不认。''

如懿的身体微微一颤,牙关紧咬处有讶然之声逸出。她仰起脸问:``富察氏百年的荣耀和福祉?她真的拿这个来发誓?''连她亦是知道的,身在众星拱月的凤位,心心念念着诞育皇子,稳居后位的女子,最在意的,也不过是富察氏的荣耀。然而她的神色旋即冷了下来:``也不过是发誓而已,臣妾不相信誓言。''她沉吟片刻,``皇上,素心与莲心是皇后的心腹随身,许多事咱们如有疑问,如今皇后薨逝,,或许可以从她们口中探知些许。''

皇帝静了片刻,沉声唤了李玉,然而入内的却是进忠,他叩首道:``李公公方才出去了,奴才候着。''

皇帝也不理会,只道:``你在也是一样,去传素心和莲心过来。''

进忠正答应着要转身出去,忽然见外头帘影一动,一个人影闪了进来。恭顺地垂首站在一边,道:``奴才李玉给皇上请安。''他跪伏在地,看了进忠一眼,沉声道,``皇上不必去唤素心了,奴才适才出去,便是听人来报说素心触柱而死,殉了皇后娘娘。''

皇帝与如懿对视一眼,从彼此眼中读到一丝震惊之色,不禁相顾失声:``素心殉主?''

李玉低首道:``是。皇后娘娘薨逝,青雀舫上本有许多事要料理。谁知忙中生乱,莲心遍寻不着素心,只好知会奴才一起寻她。谁知就在上岸的地方有座牌坊,奴才寻着索心时,她已经在牌坊的石柱子上撞死了。''

如懿望着皇帝,从他闪烁的神色里读到一丝再清晰不过的狐疑之情。那狐疑,分明也是长在自己心底的,像一根细细的毛刺,隐隐触动着细微的痛和痒:``皇上,殉主是光明正大之事,素心何必悄悄儿地背着人?''

皇帝凝神片刻,问道:``李玉,你去嘱咐毓瑚,她年长稳重,让她去瞧瞧素心的尸身,商量了叫人如何处置。另则,莲心在哪里?''

李玉一壁答应着,忙回禀道:``莲心不安,已随奴才过来了,正候在外头呢。''

皇帝不假思索,立时道:``让她进来。''

因是皇后跟前儿得脸的宫女,莲心已经换了一身雪白孝服,罩着浅银色弹丝绣暗青往生莲花比甲,黑发用银线挽就,簪着满头白霜霜花朵。她一张容长脸儿极淡漠,细细的眉眼低垂着,眼中虽然含泪,却并无过于悲痛之色。莲心进来行了礼,便规规矩矩跪在地上,也不起身,像是知道有话要答似的。

如懿见莲心这般,便也懒得费口舌,径直道:``皇后娘娘的病不是一日两日了,你和素心同在一处,素心是否早有殉主之意?''

莲心垂首跪在地上,淡淡道:``自奴婢离开王钦又回到皇后娘娘身边伺候之后,虽然还是皇后娘娘的贴身侍婢,但到底不如往日了。有什么事,皇后娘娘和素心也多避着奴婢,只叫奴婢在殿外伺候。倒是皇后娘娘这番病了之后,素心还与奴婢有些话说。''她眸光一扬,少了些低眉顺眼,一字字道,``素心说起皇后娘娘的病状,十分忧心,也曾提到家中仍有病弱老母,希望来日可以出宫侍奉左右。''她轻叹,``素心真是孝顺之人,不比奴婢无依无靠,无家可归。''

皇帝与如懿如何不懂,便是李玉亦惊呼:``素心牵挂家人,怎会突然殉主,想是她知道的事多了,怕获罪才自裁倒说得过去。''

莲心跪在地上,素白的孝服掩得她身姿格外纤弱,可她的话语却是那般掷地有声,铿锵入耳:``李公公这话糊涂了。素心是皇后娘娘的奴婢,她若有罪那皇后娘娘成什么了。若想自裁,也不必惦记着家人了。''

李玉一向在皇帝面前得宠,惯是圆滑的,闻言也有些讪讪。

如懿见皇帝并不作声,只是支着额头,双眸似闭非闭,仿佛只是在听,仿佛亦只是倦了眠一眠。她如何不知其中利害,当下示意李玉出去,方才问出声:``素心是否有罪,皇后娘娘成了什么,本宫与皇上都不甚清楚。只是你在皇后身边多年,许多事,你总该知道些许。''

莲心的目光恍若一渊深潭,乌碧碧的,望得深了也不见底。她俯身叩首,郑重道:``娴贵妃娘娘,奴婢方才已经说过,自回到皇后娘娘身边伺候后,许多事奴婢因未能近身,所以懵然不知。但奴婢到底侍奉了皇后娘娘多年,也算知道皇后娘娘的心性。她虽然难免有私心做些不当之事。但许多事,奴婢觉得她犯不上,也无谓去做。''

如懿目光一震,只觉胸间五味陈杂,酸涩苦辣一齐逼了上来,只在喉头逼仄涌动。她的眼神与莲心短暂相接,不自禁地缓缓摇头,莲心以她眼中的一泊清明的闲定安静,默然承受。烛光微微摇曳,带着几分身不由己的萧瑟,映着她白皙的面庞,却未能染上一层稀薄的红晕。良久,如懿只是轻叹:``难为你肯说这样的话。''

莲心微微一笑:``奴婢知道娴贵妃娘娘未必相信,连奴婢自己都不相信。奴婢活下来的这几年,只要有人有一语提到王钦,奴婢心头就会滴血。连在梦里,奴婢都会梦到那些不堪的日子,夜半惊醒。但诚如奴婢所言,皇后娘娘会因私心而行事不当,但杀人放火的事,她无谓去做,更怕做了会牵连她最重视的富察氏荣耀,还有她日夜期盼的儿子的太子之位。''

这些话,如同铮铮惊雷滚过如懿的心头,一颗心惊得几乎要翻转过来,忍了这么多年,恨了这么多年,到头来若不是自己恨着的那个人,又会是谁?情思恨意于回百转,然而,这一层滋味是无法以言语尽述的。如懿的脸色像初雪一般苍白至透明,是一种脆弱的感觉,仿佛自己成了一片薄而脆的枯叶,转眼便要随着风飘散了似的。信,抑或不信,曾经以肉身和心肠所承受的种种苦楚,抵死之痛,都已经在她的身上留下了不可磨去的烙印。时光的荏苒留给她的,是血肉模糊后疤痕依旧的身心和日渐趋于完美的无可挑剔的笑容。

而这些所受,来自于谁,她一直以为自己是再清楚不过的。可如今,却也是糊涂到了极处。

皇帝见如懿神色恍惚,心中亦是不忍,忙伸手扶住了她道:``夜深了,你再熬着也是苦了自己,赶紧回去歇息吧。''说罢,便吩咐了李玉,殷殷送了如懿出去。

如懿才走到皇帝龙舟尾上,却见风露中宵,一位披着莲青色如意云纹披风的玲珑女子立于舟尾,遥遥望着自己,莹白面容上盈出融融笑意。

如懿原是疲累到了极处,一见她笑盈盈望着自己,不觉心头一暖,疾步上前握住她手道:``海兰,夜来风寒,怎么这个时候还过来?''

因在夜阃,海兰只用一枚羊脂白玉嵌碧玺莲荷扁方松松挽着云髻,燕尾上几朵碧玡瑶珠花点缀,越发显得素雅清简。海兰垂首道:``今日自午膳后便未和姐姐说过话,心里总存着许多事,实在睡不着,便来这里等姐姐了。''

如懿替海兰紧了紧披风上的垂珠深紫缎带,露出她颈间一痕吴棉的浅蓝紫连珠暗花锦纹罗衣,嗔道:``生了永琪后一直畏寒怕风,自己也不仔细些。''她瞥一眼四周,``你若不嫌烦,今夜便在我那里住下,咱们好好儿说说话。''

海兰眼眸一转,正声道:``那是应该的。皇后娘娘薨逝,姐姐怕有许多事要照料,我只陪着姐姐,照应些微末琐事吧。纯贵妃早已守在大行皇后的青雀舫上。''她忽然凝眸,伸手替如懿取过腋下鎏金菡萏花苞纽子上系着的雪青绫销金线滴珠帕子,沾了沾她额头晶莹的汗珠,取笑道,``姐姐怎么了?这会子夜寒,竟出起冷汗来了?''

如懿与她挽了手走得远些,只觉得牙关一阵阵发紧,哑声道:``她拼死不认想要害死咱们,她说不是她做下的\ldots\ldots{}''

海兰骤然停住步子,旋身凝视着如懿。片刻,她樱唇微张,吐出的言语字字雪亮,打断道:``就算不是她做下的事,这些年咱们受的这些苦,都和她脱不了干系!所以,哪怕是她没做,人都死了,算在她头上便又怎的!''她冷笑道,``难不成她做了鬼魂,还要来找咱们分辩不成!我倒盼着她魂魄归来,与我说个明白呢!''

心头如被透明的蚕丝一缕一缕细细牢牢地缠紧,一圈又一圈,几乎透不过气来。如懿喃喃道:``海兰,我不知道自己该不该信。若害咱们的事不是她做的,那会是谁?她已经死了,高晞月也死了,我却不知道还要和谁斗下去,那人又躲在哪里?我们活在这儿,却又和草莽野兽有什么区别,夜防日斗,生死相搏,却永不知下一个对手何时会出现,何时会咬住自己的喉咙。''

``一身绫罗,不过也是享着荣华的困兽,与它们并无区别。''海兰笑色宛然,露出糯白细牙,``姐姐,爱,如果能支撑着人活得更好,那恨,于我们了,她是来不及后悔,咱们是犯不上后悔。''她以澹然的目光相望,唇角衔着一丝清淡笑意,掰着纤纤的指道,``姐姐,前头压着咱们的一个个死绝了,也该轮到我们了。''

如懿只是恍惚地笑着,一双眼藏着幽幽沉沉的心事起伏,茫然不知望向何处。这样清寒的夜里,隐隐约约有春鸟的啼啭夹杂在哭声之中,对着杨柳烟,梨花月,无端惹人悲凉。

海兰上前一步,与她的手紧紧相握:``姐姐,你应该高兴。''

须臾,如懿向上挑起的唇勉力勾勒出一朵笑纹,却清冷得让人觉得凄凉:``海兰\ldots\ldots 我恨了她那么久,如今她死了,我却不觉得高兴。死了阿箬,死了高晞月,死了富察氏,我恨着她们,算计着她们,彼此缠斗了这么多年,可接下来会是谁?我又为什么高兴?总仿佛这样的日子无穷无尽,永远也过不完似的。''

海兰眉目间清净内敛,语调却冷得如万丈寒冰:``旁人的人生可以删繁就简,安稳一世。可咱们一脚踏进了紫禁城,这一辈子就是今日重复昨日的日子,永无尽头。姐姐,你可以不恨,可以不高兴,但你得明白,我们若不努力活着,今日躺在那儿被别人哭的,就是自己。''

簌簌风露拂面,如懿独立于月色波毅银光素涟之下,已无太多喜悦或是悲伤,只是有淡淡的倦,并有寒意。

龙舟殿阁中静得出奇,莲心跪在阴影里,大气也不敢出。皇帝只身长立,凝神俯视不语。莲心的身子俯得越发低了,几乎要匍匐在龙靴边上,那浅金色的靴子,黄漳绒的靴面用夹金线穿着米珠和珊瑚粒,密密匝匝。盯得久了,只觉得自己也成了那靴面上细细一粒,一不留神便会滚落下来,踏成齑粉。

也不知过了多久,皇帝才淡淡道:``你是个聪明人,许多事应该明白。''

莲心恭谨道:``奴婢自然明白,无论奴婢是因为谁而脱离王钦魔掌,但归根究底,能允许奴婢逃离、能放奴婢生路的,这世间只有皇上一人。若无皇上应允,什么都是虚空。''

皇帝颔首:``莲心,这便是你比旁人聪明的地方。可你对皇后也算忠心,回到她身边之后,对她不利的话,你一句不说;对她不利的事,你一件不做。''

莲心的脸容沉静如水:``奴婢终究是皇后娘娘的奴婢,虽然她曾害得奴婢终身受苦,但背主之事奴婢做不出来。皇后娘娘生前奴婢不能出一句恶语。如今身后,皇上但问,奴婢知无不言言无不尽。''

皇帝微微沉吟:``那么,阿箬曾经告诉朕,指使她害娴贵妃、害朕的孩子的人,是皇后和慧贤皇贵妃。''他缓缓论起,将阿箬昔日之言一一述说。

莲心皱眉细想了片刻,扬眉道:``皇上不觉得阿箬说的这些话里,屡屡提到素心,却未曾提到是皇后娘娘么?''

皇帝轻晒,仰首望着阁顶繁复的迷金叠彩,那细腻的金粉填在艳色的朱漆上,炫得几乎要花了眼睛:``素心比你更算是皇后的心腹,她的所作所为,难道不是皇后所指使么?''

莲心一时语塞,她雪白的板缎长袄,裙边绣满浅青并香色缠绕的枝蔓,像一枝没有生气的藤蔓,笔直地僵立在壁间。半晌,她摇头,咬着唇道:``奴婢不知,亦不能答。皇上方才又提起皇后娘娘用冷寒之物毒害冷宫中的娴贵妃,这事奴婢也略听过一二。但奴婢细细想去,皇后娘娘自己素日都不大留心饮食,娘娘离世前几日,太医还曾见素心端了薏米汤饮给娘娘喝。那汤娘娘喝了几日了,反是太医说起薏米清热利水,但颇为寒凉,不宜娘娘饮用。这般想来娘娘其实懵然无知,奴婢也纳罕,为何娘娘对着娴贵妃却又这般懂得了?''

皇帝眸中微寒:``你是说,除了素心和皇后,只怕还有人牵涉其中?素日与皇后往来的,除了慧贤皇贵妃还有谁?''

莲心细细想了半日:``纯贵妃、嘉妃与婉常在也常常来往。皇后喜欢四阿哥,与嘉妃略亲近些。只是嘉妃一向与慧贤皇贵妃只是面子上的和睦,也不大将别人放在眼里,只和纯贵妃亲近些。皇后娘娘一向顾着彼此的颜面,所以慧贤皇贵妃若一人来,便不大叫嘉妃一起。''

皇帝的眼底闪着幽暗的光芒,旋即自己亦摇头,释然道:``嘉妃一向是个口无遮拦的,得罪了人也不仔细,对着朕更是有什么说什么的。她这样的直肠子的人,应该不是她。''

莲心静了片刻,似乎想说什么,想想却也没什么确实的疑迹,便也无言了。

皇帝神色黯然,挥了挥手:``也罢。莲心,你在宫中之事已了,朕会让你出宫安置,好好度日吧。''

莲心一怔,旋然有泪水滑落,郑重三拜,谢恩离去。毓瑚立时进来,端了一盏清茶,悄无声息走到皇帝身边,轻轻唤了一声:``皇上。''

皇帝木然站着,淡淡道:``朕无需人伺候,下去吧。''

毓瑚躬身答了一句,却不退下。他顿了顿,从袖中摸出一枚烧蓝溜金蜂点翠绣球珠花,摊开右手,平伸在皇帝跟前。

那珠花上,分明沾了一丝血痕!

皇帝的身体微微一震,原本空茫的目光骤然缩成一根锐利的银针,几乎能戳穿毓瑚弓腰缩背的身体。他的声音暗哑低涩,像生锈的铁片涩涩地磋磨:``这是朕赏给纯贵妃的!哪儿来的?''

毓瑚到底年长,见惯了御前风雷,便道:``方才奴婢去瞧素心的尸身,想要善后处置,结果在素心拱紧的手心里,发现了这个。''她看一眼皇帝的神色,不动声色道,``素心至死紧紧摇在手里,想是要紧的东西,奴婢不敢错了,也不敢惊动旁人,悄悄取了出来。''

皇帝的神色似是寒霜冻凝:``你做得极好。''他侧一侧脸,毓瑚懂得,将那珠花放在皇帝身后的黄花梨长桌上。她正要离去,皇帝冷冷道:``你也认得是纯贵妃的东西,是不是?''

毓瑚道:``去岁七夕,皇上特为各宫主位所制,说是不要只用主位们素日最爱的花儿朵儿,另外择了的。皇后娘娘用的是佛手花,娴贵妃是玫瑰,纯贵妃是绣球,嘉妃是栀子,愉妃是蔷薇,舒嫔是真珠兰,每人六对,都用烧蓝溜金蜂点翠镶了南珠,作簪鬓之用。奴婢前来见皇上前,特意又找内务府的人查问了一番,并无错漏。''她微微迟疑,还是道,``除此之外,奴婢也未查到什么,只是光凭一朵珠花,做不得数的。''

``一朵珠花!的确做不得数!''皇帝口吻极淡,``眼下纯贵妃在哪里?''

毓瑚顺从地答:``奴婢从皇后娘娘的青雀舫过来,见纯贵妃与嘉妃忙着置办丧仪之事呢。''

皇帝目光一瞬:``嘉妃也在?''

毓瑚道:``是。嘉妃也帮不上什么,一应都是听纯贵妃的安排处置。''

皇帝的声线沙沙的,像是磨着什么铁器似的钝:``嘉妃听纯贵妃的安排处置?纯贵妃倒厉害,朕还没吩咐,她便自己上赶着去安置大行皇后的丧仪了!连嘉妃也得听她的,好不简单!''

毓瑚诺诺应着,陪笑道:``纯贵妃年长,又有三个阿,嘉妃平日纵眼高些,也分得轻重缓急。''

皇帝忽地抿紧了唇,像是拼命压抑着某种涌动的情绪,冷冷道:``纯贵妃,倒是养着朕的大阿哥、三阿哥和六阿哥呢!''

毓瑚哪里敢接这样的话,只得屈膝道:``奴婢失言,奴婢没有诋毁纯贵妃的意思。''

皇帝摆了摆手,和言道:``毓瑚,你是从前和朕的\ldots\ldots{}''他似乎意识到不对,立刻改口道,``你是和李太嫔一同进宫伺候的,年久稳重,又怎会失言?''

毓瑚答应着,见皇帝说罢,沉思着良久无言,便也福了福身告退。皇帝只盯着那枚带血痕的珠花,眼底燃起一簇火苗,渐渐燃成焚心火窟,仿佛要将那珠花烧融殆尽,焚为灰末。

也不知过了多久,月光慢慢移下了金丝术窗棂上蒙着的索丝云绡。那朦胧的流素清光,映上皇帝哀伤而倦意沉沉的脸。他缓缓起身,步至床榻边,颓然倒下:``皇后,要是朕疑心错了你\ldots\ldots{}''他低喃,语意艰涩,``你别怪朕,你别怪\ldots\ldots{}''他无声地抚着榻上一对空落落的明黄云缎挑蝠枕,微一侧首,有透一明的水痕滑落\ldots{}

\hypertarget{ux7b2cux4e8cux5341ux4e8cux7ae0-ux6697ux6d8cux4e2d}{%
\chapter{第二十二章
暗涌(中)}\label{ux7b2cux4e8cux5341ux4e8cux7ae0-ux6697ux6d8cux4e2d}}

皇帝念及皇后相伴多年,悲恸良久,命庄亲王允禄、和亲王弘昼,恭奉皇太后御舟缓程回京,自己则嘱咐了如懿和绿筠在德州料理主持皇后的丧事。

大行皇后薨逝次日,皇帝心中苦绵,忆起两番丧子之痛,哀恸不能自禁,在大行皇后所居的青雀舫上写下了痛悼挽诗:

恩情廿二载,内治十三年。忽作春风梦,偏于旅岸边。

圣慈深忆孝,宫壶尽钦贤。忍诵关雎什,朱琴已断弦。

夏日冬之夜,归于纵有期。半生成永诀,一见定何时?

棉服惊空设,兰帷此尚垂。回思想对坐,忍泪惜娇儿。

愁喜惟予共,寒暄无刻忘。绝伦轶巾帼,遗泽感嫔嫱。

一女悲何恃,双男痛早亡。不堪重忆旧,掷笔黯神伤!

三月十四,皇帝亲自护送大行皇后的梓宫到天津。本留守京中的皇长子永璜连夜策马赶来迎驾。三月十六戌刻,皇后梓宫到京,于长春宫安奉。文武百官及内外命妇缟服跪迎。

皇帝辍朝九日,服缟二十七日;妃嫔、皇子、公主服白布孝服,皇子截发辫,皇子福晋剪发;满朝文武大臣一律百日后才准剃头;停止嫁娶作乐二十七日;国中所有军民,男去冠缨,女去耳环。天下臣民一律为国母故世而服丧。

这样的丧仪,是大清入关以来前所未有的隆重,而这空前的隆重还不止于此。向来后妃及王大臣凡应赐谥者,皆由大学士酌拟合适字样,奏请钦定。而皇帝根本不理会内阁,自行降旨定大行皇后谥号为``孝贤''。更晓谕礼部:``皇后富察氏,正位中宫一十三载。逮事皇考克尽孝诚,上奉圣母深蒙慈爱。覃宽仁以逮下,崇节俭以褆躬。追念懿规,良深痛悼。宜加称谥,昭茂典于千秋;永著徽音,播遗芬于奕稷。从来知妻者莫如夫。朕昨赋皇后挽诗。有圣慈深忆孝,宫壶尽称贤之句。思惟孝贤二字之嘉名,实该皇后一生之淑德。应谥为孝贤皇后。''

皇帝郑重以待,如懿与绿筠在内宫之中更是丝毫不敢放松,带领嫔妃宫人极尽哀仪。终于稍稍得空之时,海兰前来翊坤宫看望如懿,亦看望已经长得聪灵俊秀的儿子永琪。

海兰抱着永琪哄了一会儿,不觉仔细端详如懿连脂粉也遮不住的微微苍白的面色,关切道:``没想到大行皇后过世,皇上对丧仪这么经心,真是难得了。倒是辛苦了姐姐。''

如懿半支着身子斜靠在锦绫缎桃叶纹软枕上,翻看着内务府丧仪用度的簿子,神色疲倦:``皇上这么经心,是真对大行皇后动了悔意了。''

海兰哄永琪喝着手里荷叶盏中的牛乳,笑道:``人走了茶都凉,再后悔又有什么?''

如懿摇摇头:``皇上与大行皇后有过两个嫡子,虽然素日有些隔阂,但情分到底不同些。如今人不在了,自然更念着她的好处了。''

``再有什么好处,也与我们不相干。倒是皇上对姐姐另眼相看,将丧仪的事交给了姐姐和纯贵妃一并处置。我原还以为,纯贵妃有三个皇子,这次大行皇后的丧仪,她还要大权独揽呢。''海兰见惢心半跪在榻上伺候如懿捏着肩膀,面前的桌上还搁着一碗凉了的红参茯苓汤,不觉叹气道:``这几日姐姐劳碌归劳碌,有些正经的大事,也该思量起来了。''

如懿轻轻揉着额头,看着永琪无忧无虑的笑颜,不自觉便松了口气,道:``我知道你说什么。可皇后薨逝,皇上伤心不已,不是筹谋这个事的时候。''

海兰轻声道:``姐姐不筹谋,别人可已经动了这个心思了。''

``这个心思,从大行皇后薨逝那一刻起,宫中就无人不动了。只是这个时候,一动不如一静。''

如懿说着,便端起眼前的红参茯苓汤正要喝,海兰忙伸手拦住,嗔道:``都放凉了,仔细喝了伤胃。''她说罢站起身来,从螺钿圆几上捧过一盏双生莲金丝盏来,``我知道姐姐累着了,这是昨日后半夜就熬着的黄芪玉真汤,拿蜜乳调的,益气补身,又能开胃。''如懿闻言粲然接过手轻轻抿了一口,低声叹道:``难为你的心思了,这些东西容易得,但是熬煮起来最费时不过,又得提前将里头用的黄芪、杏仁、甘草、茴香细细磨碎了。你又心细,不放心旁人动手,这些事必是你自己做的。''如懿端详着她眼底血丝,实在心疼,``我说你进来时眼睛红红的,你还不认。''

海兰微垂着粉白的颈,有些不好意思:``我能为姐姐做的,不过是些微末小事罢了。风口浪尖儿上,姐姐更得仔细自己身子。''她想了想,示意惢心抱了永琪下去,``听说大行皇后临死前,曾举荐纯贵妃为继后。如今纯贵妃趁着这几日领着嫔妃祭拜,格外示好笼络,连嘉妃也巴巴儿地跟着她呢。''

如懿淡淡一笑,撩拨着耳朵上一串银流苏珍珠耳坠:``这是应该的。如今宫中只有我和她两位贵妃,她位分尊荣,儿子也多,又有大行皇后临死前的举荐,难免会动心。''

海兰比着素银缕海棠纹的护甲,有一下没一下地划着掌心:``她的资本,不过是有着两个亲生的皇子,一个养子罢了。''

浅浅的笑影在如懿梨涡内一转便消逝了,她微微黯然:``多好的资本啊!''

海兰轻嗤,并不十分上心:``姐姐也有咱们的永琪。''

如懿看她一眼,比了个噤声的动作,生了几分寥落:``永琪自然是好,可落在旁人眼里,我到底是不能生养的女人。在这宫里,孩子就是恩宠,就是依靠。我却是没有的。''

海兰有些发急:``难道姐姐真的不想么?除了大行皇后和慧贤皇贵妃,姐姐是潜邸里出来的位分最高的人。在潜邸时姐姐是侧福晋,苏绿筠不过是格格,姐姐是满军旗出身,苏绿筠是汉军旗,这到底是不一样的。而且您出身后族,您的两位姑母都是先帝的皇后。''

如懿平静的面容上多了一分忧色:``正因为如此,我才没有担当后位的资历。所谓的家世其实略等于无。无子,无家世,仅仅是出身满军旗,这能算什么。''

海兰沉默片刻,凝眉道:``可姐姐,难道你不想么?不想再居于人下,不想再看旁人的颜色,不想再谨小慎微。你就是六宫之主,往大了说你是国母,往小了说,六宫这些女人再想害你,也不敢明目张胆了。''

如懿凝神须臾,素淡的容颜上闪过一丝凌厉之色:``想,可光靠想有什么用?''

海兰微微露出几分喜色:``那就好。只要姐姐想,那咱们就是一心的。''

如懿轻轻摇头:``想归想,如今却不合适。你不是不知道,大行皇后死后,皇上极为哀痛。大行皇后生前皇上对她并未怎样,可死后皇上却格外情深义重。不管这情深义重是表面还是真心,都表示皇上暂且没有这个想头,咱们还是安静些好。''

海兰拈这绢子一笑,身上银白仙鹤长春素锦服的袖口便闪过一点柔软的光泽:``咱们想安静,可嘉妃那里,却是头一个和纯贵妃走得近呢!也难怪,她再得宠再有儿子,到底是李朝来的,后位也是难指望的,难怪会一反常态去攀着最有指望的纯贵妃了。''

如懿清冷道:``嘉妃一向目中无人,从前只和皇后略亲近些,如今自然更要指着未来的皇后了。由着她去,有些账,我还没好好和她算呢!''

两人正说着话,却见三宝进来禀道:``小主,大阿哥来了,说是来向您请安。''

如懿欢喜,即刻道:``还不赶紧请进来。还有,去备下大阿哥最喜欢的点心。快些!''

海兰掩口笑道:``姐姐到底是抚养过大阿哥的,如今还这么疼爱。这些日子,好像大阿哥也来得勤了。''

正说着话,永璜便进来了,请了安道:``母亲万福,愉娘娘万福。''

海兰起身虚扶了一把,笑道:``大阿哥每每来翊坤宫,还是不忘旧日对娴贵妃的称呼,还是叫母亲呢。''

永璜有些羞涩:``儿子养在纯娘娘名下,在外不得不只称呼一句`娴娘娘',但在内,儿子的心还是同往日一样的。''

如懿忙扶了他起来,吩咐了坐下:``你这孩子,总也不学乖,里里外外称纯贵妃为纯娘娘,一声额娘也不称呼,也不怕她吃心。''

永璜腼腆一笑,看着如懿的眼睛道:``儿子有额娘,也有母亲。纯娘娘自己有儿子,不会怪罪的。''

如懿闻言,心下不由得一软,疼惜道:``这些日子你领着诸位弟弟遵行丧仪,也是累着了吧。其实你的福晋伊拉里氏在去岁为你生下绵德,你应该更顾着府里些。如今却只能以嫡母的丧事为重了。''

永璜谦恭道:``儿子虽然是皇阿玛诸子中第一个有孩子的,但正因如此,儿子才更要恪尽孝道,安慰皇阿玛,时时伴随在侧。''

如懿点头道:``难为你有心。对了,我记得今日是你额娘哲悯皇贵妃的生辰。虽然皇后大丧我不宜去行礼追念,不过姐妹一场,我已叫人去宝华殿为你额娘送了祭品。''

永璜闻得生母之事,不觉双目盈然:``母亲挂念之心,儿子谢过了。只可惜额娘早走,又这般不明不白\ldots\ldots{}''

如懿听他语中颇有不满,即刻打断:``你进宫来,可先去看过纯贵妃了么?要是疏忽了礼仪,她难免会不高兴的。''

永璜忙醒过神来道:``儿子已经去过钟粹宫了,但听宫人们说,纯娘娘往太后宫中去了,怕一时半会儿回不来呢。''

海兰略略惊疑:``纯贵妃这些日子常往太后跟前去么?''

永璜道:``是啊。皇阿玛膝下唯有儿子与三弟永璋最长,得忙着丧仪之事,所以纯娘娘总带了六弟去太后宫中问安,太后也比从前更喜欢六弟和纯娘娘陪着了呢。''

海兰脸色微微一沉,旋即笑道:``中宫薨逝,太后难免郁郁不乐,有纯贵妃这番孝心自然是好的,只是咱们都没想到呢。''

永璜略坐了坐,便起身告辞了。如懿知道他是长子,许多事丧礼上离不开他,因此很得皇帝重用,便也不留他,又嘱咐了道:``你是你皇阿玛的长子,多少眼睛看着你呢,自己仔细些。''

永璜颇有几分自傲:``儿子知道。此刻正是宫内宫外要用儿子这个长子的时候,儿子定当十分尽心。''

如懿见他言语间颇有得色,原本想多叮嘱几句,也说不出来了。倒是他走后,海兰道:``如今看永璜和从前不一样了,常常把长子两个字挂在嘴边呢。''

如懿轻叹道:``也难怪他。谨小慎微了那么多年,皇上一心只想着立嫡,他这个长子从来不受重视。如今能被皇上这样倚重,自然是高兴的。''

海兰带了一点意味深长的笑容:``古来立太子,不是立嫡就是立长,再来就是立贤。皇上所有的儿子里,只有永璜成年,又生了儿子让皇上做了玛父,是占尽天时地利了。''说罢,海兰和如懿看了看时辰,也预备着更衣往长春宫中去守丧。

慈宁宫殿中安静得如一潭碧波沉水,连光影也晃晃悠悠,成了水波涟漪半透明的影子。福珈放下暗银色乌金团寿软帘,悄然躬身走到太后身边。太后闭目静坐:``送走了?''

福珈道:``是。''

太后轻轻笑叹了一声:``从前不大见纯贵妃,总觉得她笨笨的安静不多话,也算是个贤惠人。如今来慈宁宫多了,仔细相处起来,还真有点笨笨的,和她说话是有些累。''

福珈点上了一支翡翠镶金嘴水烟袋送到太后手里,笑道:``宫里都是聪明人,难得有个笨笨的也好。光和聪明人打交道,奴婢这样的蠢人听着费脑子。''

太后嗤地一笑,瞟着她道:``你也觉得这样的人不错?''

福珈道:``太后圣明,什么都在太后预料之中。只是娴贵妃也算是个有孝心的了,这些日子太后反而淡淡的,不太理她。''

太后吸了一口水烟袋,默默片刻道:``大行皇后便是世家大族出身,所以难以把握。娴贵妃的性子是比大行皇后更刚烈的,又透着聪慧劲儿。她又是乌拉那拉氏出身,凭她怎么孝心顺服,一想到从前景仁宫皇后的事,哀家也不愿她成为未来的皇后。''她缓一缓,隐然苦笑,``福珈,哀家是不是终究太小心眼了?''

福珈含笑道:``谁心里没个过不去的坎儿呢?纯贵妃出身虽低些,但是个好性子。最要紧的是纯贵妃子嗣多,哪怕撇开了大阿哥没有生母这回事,再轮下来,按年纪就是她亲生的三阿哥了。有儿子的,到底不一样些。且说了,还是大行皇后临死前亲自向皇上举荐为皇后的。''

太后长叹如幽微的风:``不怪哀家偏心些。说到底,娴贵妃也是吃了没孩子的苦头。看看永琏和永琮夭折后大行皇后的那个样子,你就知道在宫中有个亲生儿子是多么要紧的事。哀家就是吃亏在这点上,所以一把年纪了,还要费心费神,未雨绸缪。''

福珈忙道:``大行皇后过世,皇上只顾着伤心。待得后位定了,太后也可以放一半的心了。''

太后点头道:``但愿如此。皇帝已经够聪明精干了,若皇后还是伶俐透了的人,哀家就有得受累了,还不如乖乖笨笨的就算了。且你以为大行皇后有多真心举荐纯贵妃,不过也是为着这样罢了。''

如懿到了长春宫中,绿筠已经领着命妇们按着班序站好,一切井井有条。一众嫔妃命妇围着绿筠众星捧月似的,绿筠也格外地仪态万方,恰如副后一般。彼时玉妍正怀着她的第三个孩子。自在乾隆十一年七月生下永璇后,如今不过一年多,她又有五个多月的身孕,可见圣眷正隆。可饶是如此,她陪在绿筠身边,脸上仍挂着奉承的笑意,谦恭无比:``幸好一切有纯贵妃打点,才妥妥当当,没什么差池。若换了旁人,定是不成的。''

其中一个命妇道:``嘉妃娘娘说得是。太后不也对纯贵妃娘娘赞不绝口么?且看三阿哥稳重有礼,一看便知是纯贵妃娘娘教导有方。''

玉妍本有着身孕,体态慵憨,闻言便支着腰身笑道:``可不是么?三阿哥是贵妃姐姐亲生的,自然不必说,便是大阿哥,得贵妃姐姐抚养,也是调教得极能干的呀!''

另一常在道:``大阿哥是皇上长子,自然更要有所承担些。也亏得纯贵妃娘娘多年来悉心照顾呢。''

海兰与如懿听着她们嘤嘤呖呖地说话,不过相视一笑,便站到了自己的位子上,向着大行皇后的灵位跪下行敬酒礼。如懿与绿筠并排跪着,绿筠敬完酒,低声向如懿道:``听说方才永璜又去看过妹妹了?''

``略坐坐就走了,哪里谈得上又去看过?''

绿筠似笑非笑:``到底妹妹是抚养过永璜的,难怪永璜老这么惦记着。我就不一样了,呕心沥血抚养了那么多年,知冷着热的,怕人闲话说不疼永璜,比对自己的阿哥还上心。闹了半日,还是不如妹妹。''

如懿的口气极温婉,含了几分谦逊之色,道:``我只抚养了永璜那么点时候,永璜就惦记着,别说姐姐你这么对永璜用心。永璜是个有孝心的,姐姐放心就是。''

绿筠穿着一袭浅银色夹玫瑰金线云锦宫装,裙摆有深一色的银线夹着玄色丝线密密绣着团寿纹样,满头白纷纷珍珠珠流苏如寒光轻漾,在殿中光线掩映之下,更显冷清,恰与她此时疏远与不信任的语调一般:``永璜有没有孝心,果然是娴贵妃知道的更多。我这个做养母的,到底是白心疼了。''她长长地嘘一口气,``只是没有自己的儿子,大行皇后走下来的地方,就别痴心指望着了。不孝有三,无后为大啊。大行皇后不也是因为这个羞愧而死的么?''

如懿回过首,见永璜与永璋并肩而立,领着诸位阿哥在灵前尽孝,端然是长兄风范,十分引人注目。连永璜的福晋伊拉里氏亦十分得体,领着诸位同辈的福晋,进退得宜。

玉妍跪在绿筠身后,听见二人这般低声言语,眼瞅着妃位以下的嫔御们都退得远了,不觉抚着高高隆起的肚子慵慵笑道:``娴贵妃不是好歹还抚养着永琪么?怎么看着旁人的孩子那么眼馋,连纯贵妃的养子您瞧着也是好的。其实您也不怕,不过才过了三十一岁的生辰,便要拼着力气生养一个,也是不难。到底,孩子还是亲生的好啊!''

如懿听玉妍尖酸,便淡淡道:``是啊。不经嘉妃提醒,我总都忘了自己已经年过三十。其实细算起来,咱们姐妹都是差不多的。嘉妃不也三十六岁了么,这样怀着身孕,还要按着规矩行祭礼,真是辛苦了。''

玉妍与绿筠都是康熙五十二年生的人,足足比如懿大了五岁。若要拿年纪来细论,她们自然是论不过如懿的。海兰跟在如懿身后,笑得轻巧和婉:``其实细论起来,咱们的年纪都大过了娴姐姐,只不过娴姐姐的位分比我与嘉妃高,所以咱们都得称呼一声姐姐。宫里嘛,总是先论位分,再论年纪的。''

海兰本就是和声细语的人,说得又在情理之中,玉妍虽然不忿,但也不能驳嘴。正巧意欢敬香上前,听得几人言语,细巧的眉眼斜斜一飞:``其实娴贵妃客气了。论起在潜邸的位分,纯贵妃是格格,娴贵妃是侧福晋,如今虽然都是贵妃了,但到底还是根基有别的。娴贵妃由着纯贵妃称呼一声妹妹,固然是年纪轻些的缘故,但到底位分搁在那儿呢。''

绿筠齿本不及意欢伶俐,如今听她掀起旧事来,只得讪讪不语。还是一同出身潜邸的婉茵打圆场道:``纯贵妃和娴贵妃哪里会计较这个。嫔妾记得刚进紫禁城那会儿,纯贵妃的三阿哥突然要被抱去阿哥所养育,纯贵妃伤心起来,连夜找的第一个人就是娴贵妃呢。两位贵妃这样亲近,一句半句的姐妹称呼,算的了什么呢?''

如懿有一瞬间的恍惚。那样的亲近,是许多年前的事了吧?她和绿筠算不上什么至交密友,但论起来潜邸诸人中,除了海兰,便是与她亲近了。当年困窘尚可彼此相依,如今大家同为贵妃,反而彼此不能相容了么?她看着孝贤皇后乌木漆金的棺樽,这么多年,她害得自己一直没有子息,身体流转的血液里都带着她精心布置的零陵香气息,害得自己做不得一个母亲,一个完整的女人。琅嬅一次次意图逼自己入死地,真的,恨了那么多年,连如懿自己都觉得,这样的恨已经成为了一种深深的习惯,深入骨血。

可此刻,琅嬅穿戴着整齐而华丽的皇后冠服,静静的躺在棺樽之中,接受着天下臣民的哀哭与追忆。

是,高晞月已死,琅嬅已死。那些让她警惕的女人,都成了一抔黄土,红颜枯骨。可她却不能松一口气,新人在不断地出现,旧人们也丝毫不肯放松。皇后死前的暗潮汹涌一派和睦终于随着她的死分崩离析,连胆小如苏绿筠,都可以与她冷嘲热讽,赤眉白眼,来日皇后之位虚位以待,尚不知要生出何种事端?

而她乌拉那拉如懿,她算什么呢?不过是无子、无家世,只能依靠着一息微薄的宠爱而生存的女人。而这宠爱,是多么渺茫,仿佛琅嬅灵前跳动的耀目烛火,一阵轻轻的风,都可以肆意扑灭。

她是太知道``恩宠''了。从阿箬的死,晞月的死,到今时今日死去的琅嬅,无一不是受过皇帝的宠爱,并且仿佛身后还享受着这样的宠爱。

她实在是太懂得了。因为懂得,所以彻骨寒凉。

\hypertarget{ux7b2cux4e8cux5341ux4e09ux7ae0-ux6697ux6d8cux4e0b}{%
\chapter{第二十三章
暗涌(下)}\label{ux7b2cux4e8cux5341ux4e09ux7ae0-ux6697ux6d8cux4e0b}}

趁着祭酒礼歇的一刻,绿筠与如懿听着各宫各处的太监宫人们来报上琐事。海兰跪得久了,只觉得膝头酸麻不已,见别的嫔妃们并无进偏殿歇息的样子,便招了招手示意叶心带上药酒,跟着自己往偏殿去。

叶心扶着她出来,低声道:``小主的膝盖不好,经不得这样长跪呢。''

两人正说话,如懿恰好扶了惢心出来,打算往偏殿更衣,见了海兰便道:``是不是膝盖受不住了。你先去偏殿歇一歇,我叫人端碗八宝甜汤来给你,再涂点药酒。''

海兰摆手道:``生了孩子之后到底是不如从前了。姐姐悄声些,别让人拿住了话柄说我不敬大行皇后。''

海兰这样的话不是没有道理,孝贤皇后死后,皇帝很是哀痛,脾气也喜怒无常,前两日便因指责前朝的几位大臣在丧礼上不够悲痛,便立刻施廷杖打死。如果旁人知道海兰因为跪在孝贤皇后灵前而犯了膝头酸痛,不知又有多少是非呢。

如懿知她言下之意,叹道:``皇上如今的脾气\ldots\ldots 罢了,大行皇后过世,皇上失了结发妻子,到底是伤心的。''

海兰冷笑一声:``生前不见得怎样,如今倒成了恩爱夫妻了。大行皇后若地下有知,会不会嫌自己弃世太晚,不能早些得到这样的尊重恩情?''

如懿看了看四下,比起手指轻嘘一声:``说话越发任性了。''

海兰一脸通透:``我这样的人还怕什么呢?不过是看穿了姐姐看不穿的宠爱罢了。''

如懿正挽着海兰的手要进偏殿,忽然听得里头有窸窣的低语声。二人见有人在,一时也不便进去,正转身要走,却听得依稀是永璜和福晋伊拉里氏在说话。

伊拉里氏温声软语劝道:``爷累了这么几天,喝点参汤提提精神吧,妾身已经准备了热水,爷敷敷脸,精神些。''

永璜似乎很不耐烦:``弄这些劳什子做什么?我得赶紧去皇额娘灵前守着。皇额娘薨逝,弟兄之中唯我居长,这一时半会儿,缺了旁人尚可,我这个长子不在,像什么样子。''

伊拉里氏很是心疼:``爷这辈子就是被长子两个字困住了。您不是铁打的人,但凡多歇一歇又怎么了?一得空还得往娴娘娘那里跑,她只是您曾经的养母,您好歹得顾着纯贵妃的面子啊!''

永璜冷笑道:``纯娘娘的面子我要顾着,母亲那里也不能不走动。说到底,纯娘娘有她亲生的儿子,哪怕抚养了我几年,又算什么?历来皇子所娶的正室福晋多出自满洲八大姓氏,而你只出身伊拉里氏,小姓小族,论起来纯娘娘要是真疼我,怎么会听凭皇阿玛指了我这么个小姓的福晋也不说话?皇子联姻,说来终究是门第姓氏最重要了。''

伊拉里氏赧然道:``都是妾身的不是,帮不上爷什么忙。''

永璜道:``你帮不上忙也罢了,凡事终究是要靠自己的。皇额娘死了,左右我小时候她也不疼我,差点把我害死在阿哥所。她死了也清净,否则她在,我终究没有爬上去的一天。''

伊拉里氏思忖着小心道:``只是皇额娘死了,后位左不过是落在纯贵妃、娴贵妃或者嘉妃身上,爷可要看准了是谁。''

永璜道:``纯娘娘要是当了皇后,我还能有指望么?她的儿子永璋和永瑢就成了嫡子了。嘉妃来路太野,也没什么指望。娴娘娘\ldots\ldots 母亲她到底是吃亏了家世,又没儿子。但我看准的就是她没儿子,没有儿子,才会疼我这个养子。我便不信了,我多多提着与她当年的抚养之情,会比不上永琪那个乳臭未干的小子。即便娴贵妃当不上皇后,只要她多向皇阿玛提着我是长子的事,我也多些胜算了。''

伊拉里氏道:``说来,到底是娴贵妃更疼爷些。''

有片刻静寂,仿佛昔日的温情再度流转其间,然而这样的幻象亦如天际辉丽的彩虹,转瞬消失不见。永璜似是在冷笑:``疼不疼的,谁知道呢?不过是彼此看着还用得上,多多利用罢了。我在这宫里长到这个岁数,难道还不懂这些?什么亲情孝义,都是假的!只有当上太子,大权在握,才是最真的。''

似乎是伊拉里氏唯唯诺诺的应答声,永璜长长地叹了口气:``手头事多,傍晚得闲,我得去宝华殿上香祝祷,今儿是额娘的生辰。''他似是有些哽咽,``我额娘,死得冤屈!''

伊拉里氏道:``爷且忍耐些,别提这个话了。额娘人虽不在,生辰忌日,妾身也该尽孝。听说一早娴娘娘与嘉娘娘都让人送了祭礼去了。''

永璜道:``你我同去太过点眼,免得被人拿住话柄说不敬嫡母。我自己去一遭便好。''

他说完,里头再无声音。片刻,有脚步声逐渐迫近,继而开门声响起。如懿与海兰站在阶下,指着远处的宫殿似乎说着什么。永璜见了她们,便是一脸孝和谦恭的样子,拱手道:``母亲好,愉娘娘好。''他似乎有些紧张,``两位娘娘怎么在这里?''

如懿从容笑道:``本宫正和愉妃说,从长春宫这里望出去对面的琉璃瓦颜色特别亮,在丧仪期间似乎不太合适,得蒙上白布才好。''

永璜松了口气:``那儿子立刻去办。''

他说罢,匆匆离去。

檐外有细雨蒙蒙,三月的紫禁城仿佛融在了暗灰色的烟雨之中,一片哀色凄凄。如懿轻声呢喃,似是问海兰,亦是自问:``海兰,我真心疼过的孩子,怎么会变成了这样?''

海兰对如懿的伤心全然不以为意:``皇家的孩子,以后都会长成这个样子。我倒觉得,这样的永璜更像一个皇子。''她看着如懿,伸手替她挡住被风扑进的蒙蒙银丝,``姐姐很伤心么?''

如懿伸出手,接住细细的雨丝,那种湿润,好像是泪,落于掌心:``永璜,毕竟是我真心疼爱过的孩子。在我没有孩子的日子里。我一直把他当成自己的孩子。''

海兰的声线薄而细韧,仿佛一条拉长的细线,截断细雨如丝的伤感:``姐姐疼爱永琪么?或许有朝一日,永琪也会变成永璜这个样子,不如我们预期中长大。兄友弟恭,父慈子孝,在这宫中不过是个笑话,不过是写进死后功德里的溢美之词。来日永琪会有自己的心思自己的想法,甚至有更多想得到的东西。这世间多的是母子失和,夫妻离心,所以,母子也好,夫妻也罢,这种到头来或许都会疏远的感情,比不上我们姐妹彼此风雨多年的情感。姐姐,或许哪一日,永琪有了自己的亲人,皇上也彻底不再宠爱,那么只有我和你,继续相伴深宫岁月,一如从前。''

海兰的语气里有深深的依赖,然而如懿的心思却在细雨绵绵中飘摇着疑惑不定:``海兰,我从未问过你,为何你对世间的情爱,这么不能相信?''

海兰的眼角闪过一点晶亮的泪光:``姐姐,你知道我的阿玛和额娘是怎么死的么?我额娘与阿玛年轻时也算是恩爱亲密,可有一日我额娘红颜不再,阿玛喜欢上别的女子,我额娘不能忍受,彼此争执之时失手刺死了阿玛,然后悲愤自尽。我自小被寄养在伯父家长大,所以一直认为,再相爱又如何,到最后因爱生恨的太多太多,与其如此,还不如不曾恩爱如许。世间的男欢女爱,不过是皮肉交合,实在是不可依靠的。''

如懿默然,只是轻叹一声:``只是海兰,什么都不相信,会不会太空虚,像找不到依靠?''

海兰轻笑,眼中有深深的依赖:``姐姐,我相信你啊。''她紧紧靠着如懿身侧,``所以姐姐,无论我做什么,你也要相信我。''

如懿温然颔首,一任雨丝凄凄拂上身来:``是,我都相信。''

海兰轻声道:``姐姐,我知道其实你是有些不一样了。从冷宫出来后,你一直很想劝自己不要去多想,只要相信皇上就好。可一个人这样劝自己,她本身就是已经是开始在不相信了。对么?''

如懿闭上眼晴,以此来拒绝眼前的虚空:``海兰,不要再说。''

海兰懂得地点点头:``那我说另一件事。姐姐,纯贵妃志在后位,她的胜算不小,如今又和慈宁宫走得近。姐姐,咱们得想想办法了。''

有冰冷的感觉蜿蜒心上,如懿霍然睁开眼:``她最大的胜算,就是子嗣。''

海兰扬起唇角优美的弧度:``这个我明白。纯贵妃最有利的是什么,我得把她最有利的东西除掉,咱们就安心了。''

如懿颔首,然而微有迟疑:``但,永璜不是她的胜算。哪怕他再不好,别动他。''

海兰笑了笑,伸手仔细拂去她仙鹤衔梅素白银线锦袍上沾上的晶亮雨丝:``姐姐到底还是心疼永璜。''她轻舒一口气,``眼下姐姐在风口浪尖上,凡事不动为妙,一切有我。''

如懿看着帘外细雨阑珊,拂去鬓角雨丝,恍若无心:``如今,皇上最忌讳的可是举丧不哀。咱们去偏殿上了药,赶紧就回去吧。''

如懿回到殿中,绿筠正与玉妍着人派发午后歇息时喝的银耳莲子羹,福晋命妇们仿佛预知绿筠日后可能会有的荣华锦光,亦格外奉承,直如众星捧月一般。相形之下,缓步入内的如懿则显得冷清许多,除了意欢、嬿婉和婉茵,便少有人笑脸相迎了。如懿不知为何众人变数这样快,还是意欢忍不住说了一声:``方才太后来过了,体恤福晋们守灵辛苦,所以亲自送了银耳莲子羹来,并嘉奖纯贵妃守丧辛苦却事事妥帖,有大家之风。又说三阿哥虽未成年,却很能照顾几位幼弟,也十分能干。''

孝贤皇后死后,后宫中本已暗潮汹涌,太后如此褒扬,无疑是在立后的立场上更偏向于绿筠了,众人如何能不见风使舵,处处恭维纯贵妃。

嬿婉与几位答应、常在围着绿筠和玉妍热络地说着什么。嬿婉小心替绿筠拂着衣角的尘灰:``贵妃姐姐仔细脚下,您这么精致的衣袍,沾上尘灰就不好了。''

绿筠不以为意地笑笑,坦然接受她的殷勤,口中道:``这些事交给宫人们打理就是了,令贵人不必如此。''

嬿婉蓄足了满脸笑意,正要搭腔,却听玉妍冷不丁笑了一声,扬着手中的杏子绿百绦绢子道:``纯贵妃姐姐不必担心,令贵人原是我的宫女出身,做这些事最合宜了。''

嬿婉如今也算得宠,听了这话脸色刷一下白了起来,又见众人皆捂着口笑看她,越发臊得无地自容,只得讪讪收手避到人后。

玉妍鄙夷一笑,越发与绿筠聊得热络,一双手蝶舞似得翻飞着:``我这怀的也不知是个阿哥还是公主,我瞧着姐姐的四公主真是好,满心羡慕。太医也说这一胎像是女胎呢\ldots\ldots 我只求啊,若是个阿哥能有姐姐的三阿哥一半争气就好了\ldots\ldots{}''

二人说起孩子来,又是扯不完的话。玉妍又一意奉承着绿筠,哄得绿筠几乎合不拢嘴,亲热地与她牵着手推心置腹。

意欢远远看着,撇了撇樱桃唇道:``一个乐得被巴结,一个嘴上不留德。''

如懿比了个轻嘘的手势,低声笑道:``就你脾气最好!最不是孤拐性子!''

意欢拈了水蓝色打黄莺儿八宝缨络绢子一晃,轻嗤一声:``我知道自己什么孤拐脾气,左右和她们不一样就是了。''说罢荷惜便来请:``小主,该到吃坐胎药的时候了。''

如懿微微诧异:``我记得这些日子皇上并不曾召幸啊,怎么你还吃这个药?''

``如今大约是盼子心切,我求了皇上两次,便按着两日都送来了。''

如懿知道端底,又实在不能说破,勉强含笑道:``无论是坐胎药也好,还是什么,是药三分毒,不吃也罢了。当年慧贤皇贵妃求子心切,也是常常吃坐胎药,却没什么效力。可见什么都是假的,唯有恩宠才是真的。''

意欢的唇角藴了一点甜蜜的笑色:``其实我也知道药石未必有效,但\ldots\ldots{}''她向来冷冽的脸庞上全是甜而柔的红晕,恍若冰雪初融,芙蓉春晓,``但皇上对我好,心疼我,我都是知道的。''她说罢更是含羞,忙扶着荷惜的手走了。

如懿怔在当地,不知自己脸上的表情是喜是悲。她是知道的,唯有她知道,皇帝知道,齐鲁知道。可谁都不会说,不会告诉她。这样的心疼,这样的好,背后是怎样的不堪入目?她唯有闭上眼睛,不可说,不能看,不去想,只当自己是混沌泥潭里的一块污浊,同流合污下去。唯有这样,才是保全了意欢含糊而温柔的一点绮梦。

海兰看她怔在那儿,便牵了永琪过来道:``姐姐,你瞧着舒嫔做什么?''

如懿醒过神来,忙笑道:``没什么,原是有些乏了。''她看海兰牵了永琪过来,便问:``怎么了?要带永琪出去?''

海兰满脸不放心:``方才听永琪有两声咳嗽,我带他去太医院瞧瞧,看要不要喝点枇杷露。''

如懿疼爱地抚了抚永琪的脸,道:``那就快去快回,路上别着了风。''

海兰出了长春宫,便牵着永琪往西长街上走,因居丧不便,只一个亲近的乳母和叶心跟着。才走到储秀宫后头的拐角处,却见永璋也匆匆往太医院方向走过来,她索性立住脚,扬声道:``永琪,现在额娘嘱咐你的话,你可要好好听着了。''

永琪似懂非懂地睁大了眼睛,道:``是。''

海兰朗声道:``永琪,后天你皇额娘的梓宫要奉移景山观德殿暂安,那天是大礼,你可万万记得,一定不能哭,不能伤心,知道么?''

永琪疑惑道:``可娴贵妃额娘嘱咐,是一定要很伤心地哭,否则皇阿玛会生气。''

海兰弯下腰,神神秘秘道:``平时是这样,可到了后天,娴贵妃娘娘也会这样嘱咐你。那天所有的阿哥公主都会去哭丧,谁都会哭得很伤心。只有你一个人镇定自若,一点也不哭,你皇阿玛便会对你另眼相看。因为你是在所有痛哭流涕沉浸于悲哀的人中,唯一保有清醒与理智的一个。''

永琪的眼神有些迷茫:``额娘,为什么?''

海兰郑重道:``因为对于你皇阿玛而言,不仅失去了你皇额娘,也失去了你七弟这个嫡子。所以对他而言,得到几个孝子不是最要紧的,最要紧的是得到一个不为悲喜所左右的未来的太子,你懂么?''

海兰转过头,见到永璋便立在不远处,似乎在侧耳倾听她与永琪的对话。海兰立刻有几分慌张不安,紧紧牵过永琪的手将他掩于身后,有些尴尬地道:``三阿哥,你怎么在这儿?''

永璋不以为意地笑笑,谦恭地行礼:``愉娘娘万安,五弟好。''

永琪亦规规矩矩叫了声``三哥''。永璋摸了摸他的额头,笑道:``儿臣见几位弟弟因为劳累都起了口疮,想着接下来还有奉移梓宫的大事,可不能累坏了身子,所以想去太医院取些金银花来煮水给弟弟们喝。''

海兰不自在地摸着鬓角一朵雪白的海棠花:``三阿哥真是有心。到底是纯贵妃教养出来的好孩子。''

永璋摆手道:``愉娘娘过奖了。那儿臣先行一步。''他侧身,意味深长地看了永琪一眼,含笑离开。

永璋打点完一切,回到绿筠宫中。他一见绿筠,哪里还按得住脾气,便将海兰叮嘱永琪之语悉数告知了绿筠。绿筠冷笑道:``我原当愉妃是个安分的,原来却动了这个心思。本还以为娴贵妃打的是永璜的主意,如今看来,是我们太小瞧她的心胸了。''

永璋迟疑:``那额娘的意思是\ldots\ldots{}''

绿筠爱惜地抚了抚儿子的辫发,替他整好衣衫:``好儿子,永琪还小,能有多大的心思。即便是不哭装出一副大人腔调,也只当他发呆不懂事罢了。你好好学着点,永琪即便不哭,额娘也有本事让他哭了就是。''

永璋松一口气:``多谢额娘替儿子筹谋。''

绿筠心疼道:``你这孩子,跟额娘说起这样见外的话来了。额娘不疼你,还能疼谁。永璜虽然也寄养在额娘膝下,但到底不是亲生的,额娘疼他也是顾着面子罢了。好儿子,除了永璜,阿哥里就数你年纪最长。你是有额娘的,额娘熬到贵妃这个位分上,一切都是为了你,掏心挖肺也是愿意的。你就好好替额娘争口气,得了你皇阿玛的欢心,当上太子就好了。何况,咱们还有大行皇后临死前的一份举荐呢,更要好好用心。''

永璋肃然道:``额娘放心,额娘的心愿就是儿子的心愿。那日儿子还会好好劝慰皇阿玛的。''

绿筠笃定笑道:``这就好了。额娘已经告诉过你,嘉妃便是个聪明人,事事都奉承着额娘。她虽得宠,但到底是李朝贡女,一辈子也指望不上皇后之尊,只要她和咱们一心,你也多一层保障。''她的口气愈加隐秘,``至于永璜,皇上器重他让他主持丧仪,可他到底不经事,你万万留心他一举一动,但凡拿到错处,便好办了。''

永璋顽皮一笑:``额娘舍得?''

绿筠有些难言的伤感:``额娘胆子小,也心软,永璜到底也是额娘的养子。''她顿一顿,深吸一口气,``可为了你,额娘什么都舍得。''

母子两关上殿门,愈加密密筹谋起来。

\hypertarget{ux7b2cux4e8cux5341ux56dbux7ae0-ux56feux7a77}{%
\chapter{第二十四章
图穷}\label{ux7b2cux4e8cux5341ux56dbux7ae0-ux56feux7a77}}

海兰候了永琪从太医院回来,便领着他往养心殿去。才到了阶下,李玉便先迎上来,含笑道:``愉妃娘娘怎么带五阿哥来了?下雨天路滑,您小心脚下。''

海兰含了极谦和的笑,那笑意是温柔的,含了两份怯怯,如被细雨敲打得低垂下花枝的文心兰,柔弱得不盈一握:``永琪有两声咳嗽,但还惦记着皇上,一定要过来请安。本宫拗不过,只好带他来了。''

李玉向着永琪陪了个笑:``五阿哥真是孝心!''他有些为难道:``愉妃娘娘,皇上这几日痛心大行皇后之死,除了纯贵妃和娴贵妃,还有大阿哥和三阿哥,几乎未见其他嫔妃和阿哥。恐怕\ldots\ldots{}''他垂下眼睛不敢说话。

海兰会意,幽然叹道:``皇后仙逝,本宫也伤心。但皇上总得当心龙体才是啊,否则咱们还哪里有主心骨呢。''她摸了摸永琪的头,``罢了,你皇阿玛正忙着,咱们也不便打扰。你去殿外叩个头,把额娘炖的参汤留下便是了。''

永琪乖巧地点了点头,快步走上台阶,在廊下跪倒,磕了头,朗声道:``皇阿玛,儿臣永琪来给皇阿玛磕头。皇额娘仙逝,儿臣和皇阿玛一样伤心,但请皇阿玛顾念龙体,不要让皇额娘在九泉之下担心不安。请皇阿玛喝一点儿臣炖的参汤,养养神吧。儿臣告退。''永琪说完,认认真真地磕了三个头,直磕得砰砰作响,方恭恭敬敬退开了。他才转身走下台阶,只见身后紧闭的朱漆雕花门豁然洞开,皇帝消瘦的身影出现在眼前,伸出手道:``永琪,过来。''

海兰低首,一双翠绿梅花珍珠耳环碧莹莹地扫过雪白的面颊。她露出一丝淡而浅的笑意,恭谨而温顺。永琪赶紧跑到皇帝身边,牵住皇帝的手,甜甜唤了一句:``皇阿玛。''

皇帝连日来见着两个皇子,说的都是规矩之中的话,连安慰都是成人式的,早就不胜其烦。听了这一句呼唤,心中不觉一软,俯下身来道:``你怎么来了?''

永琪垂下脸,似乎有些不安,很快伸出手擦了擦皇帝的脸,道:``皇阿玛,您别伤心了。你要伤心,永琪也会跟着伤心的。''

皇帝脸上闪过一丝温柔与心酸交织的神色,慈爱地揽过永琪的肩膀:``永琪,带了你的参汤进来。''他看了站在廊下微雨独立的海兰,穿着一袭玉白色素缎衫,领口处绣着最简单不过的绿色波纹,下面是墨绿洒银点的百褶长裙,十分素净淡雅,发髻上只戴了一枚银丝盘曲而就的点翠步摇,一根通体莹绿的孔雀石簪配上鬓侧素白菊花,单薄得如同烟雨蒙蒙中一枝随风欲折的花。皇帝虽久未宠幸海兰,也不免动了几分垂怜之意:``愉妃,你来伺候朕用参汤。''

海兰温顺得没有任何多余的表情,走到皇帝身边,掩上殿门。殿中十分幽暗,更兼挂满了素白的布缦,好像一个个服丧的没有表情的面孔,看起来更是有一种难以言喻的死气沉沉。皇帝脸上的胡楂多日未刮了,一张脸瘦削如刀,十分憔悴。

永琪与海兰跟着皇帝进了暖阁,见桌上铺着一幅字,墨汁淋漓,想来是新写的。海兰柔声道:``皇上,殿中这样暗,你要写字,臣妾替你点着灯吧。''

皇帝哑声道:``不必了。大行皇后在时十分节俭,这样的天气,她是断不会点灯费烛火的。''

海兰道了``是''便安静守在一旁:``皇上写的这幅字是给大行皇后的么?''

皇帝颔首:``是给大行皇后的《述悲赋》,一尽朕哀思。''皇帝看着永琪,``你说这参汤是你给朕炖的,那你告诉朕,里头有什么?''

永琪掰着手指头,认真道:``这道参汤叫四参汤。四参者,紫丹参、南沙参、北沙参、玄参也。配黄芪、玉竹、大麦冬、知母、川连、大枣、生甘草,入口甜苦醇厚,有降火宁神、益气补中之效。''

皇帝奇道:``入口甜苦醇厚?你替皇阿玛喝过?''

永琪仰着天真的脸,拼命点头道:``是啊。《二十四孝》中说汉文帝侍奉生母薄太后至孝,汤药非口亲尝弗进。儿臣不敢自比汉文帝,只是敬慕文帝孝心,所以儿臣准备给皇阿玛的参汤,也尝了尝,怕太苦了皇阿玛不愿意喝。''

皇帝颇为欣慰:``好孩子,朕果然没有白疼你。''皇帝由着海兰伺候着盛了一碗参汤出来略喝了两口,``《二十四孝》的故事你已经读得很通了,是个有孝心的孩子。''

永琪坐在皇帝身边,懵懵懂懂道:``皇阿玛,《二十四孝》儿子都明白了,可今天大哥说了一个什么典故,儿子还不大懂,正要打算明天去书房问师傅呢。''

皇帝漫不经心,随口道:``你大哥都忙成这样了,还有心思给你讲典故?说给朕听听。''

海兰忙道:``是啊,有什么不懂的,尽管问你皇阿玛。你皇阿玛学贯古今,有什么不知道的,哪里像额娘,一问三不知的。''

永琪便道:``今日儿臣在长春宫向皇额娘尽哀礼,后来咳嗽了想找水喝,谁知经过偏殿,听见大哥很伤心地说什么明神宗宠爱郑贵妃的儿子朱常洵,不喜欢恭妃的儿子朱常洛,还说什么明朝有忠臣,所以才有国本之争,自己却连朱常洛都不如。儿臣不知道大哥为什么这样伤心,朱常洛又是谁,大哥怎么拿他和自己比呢?不过儿臣还听见大哥跟大嫂说话呢,不敢多听就走了。''

皇帝轩眉一皱:``既是在给你皇额娘尽哀礼,他们夫妻俩又窃窃私语什么?''

永琪掰着手指头,稚声稚气道:``不是窃窃私语。大哥说:皇额娘薨逝,弟兄之中唯我居长,自然要多担当些。儿臣觉得大哥说得没错呀!''

皇帝缄默不语,面孔渐渐发青下去,如青瓦冷霜,望之生寒。永琪有些害怕起来,看了看愉妃,又看了看皇帝,摇了摇皇帝的手道:``皇阿玛,您怎么了?是不是儿臣说错了什么?''

海兰愈发惶恐,忙跪下道:``皇上,永琪年幼无知,若说错了什么,您别怪他。臣妾替永琪向您请罪了。''

皇帝瞟了海兰一眼,口气淡漠如云烟霭霭:``你起身吧。朕知道你不看书,不懂得这些。便是如懿,诗文虽通,这些前明的史书也是不会去看的。永琪还小,这些话只能是听来的。''

海兰诚惶诚恐地起身,拉过永琪在身边。皇帝的手紧紧地握成拳,脸上含了一丝冷漠的笑意,显得格外古怪而可怖:``呵,永璜果然是朕的好儿子,可以自比朱常洛了。那么永璋,是不是也有朱常洵的样子,敢有他不该有的心思了,也是仗着生母的缘故么?''

海兰一脸忧惧,小心翼翼道:``皇上说什么仗着生母?臣妾只知道,纯贵妃是要继立为皇后的呀!''

皇帝意外,不觉瞬目道:``什么?''

海兰睁着无辜而惊惶的眼眸:``皇上还不知么?宫中人人传言,大行皇后临死前向皇上举荐纯贵妃为继后啊!''

皇帝脸色更寒,沉思片刻,含着笑意看着永琪:``原来如此啊。永琪,参汤朕会喝完的,你和愉妃先退下吧。''

海兰忙带着永琪告退了,直到走得很远,永琪才低低道:``额娘,儿子没说漏什么吧?''

``说得很好。真是额娘和娴额娘的好孩子,不枉额娘翻了这些天的书教你。''她仰起脸,一任冰凉的雨丝拂上面颊,露出伤感而隐忍的笑意,``姐姐,我终究没听你的。''

京城三月的风颇有凉意,夹杂着雨后的潮湿,腻腻地缠在身上。永璜只带了一个小太监小乐子,瞅着人不防,悄悄转到宝华殿偏殿来。

小乐子殷勤道:``奴才一应都安排好了,阿哥上了香行了祭礼就好,保准一点儿也不点眼。''

永璜叹口气:``每年都是你安排的,我很放心。只是今年委屈了额娘,正逢孝贤皇后丧礼,也不能好好祭拜。总有一天,我一定会为额娘争气,让她和孝贤皇后一样享有身后荣光。''

二人正说着,便进了院落。偏殿外头静悄悄的,一应侍奉的僧人也散了。永璜正要迈步进去,忽听得里头似有人声,不觉站住了脚细听。

里头一个女子的声音凄惶惶道:``诸瑛姐姐,自你去后妹妹日夜不安,逢你生辰死忌,便是不能亲来拜祭,也必在房内焚香祷告。姐姐走得糊涂,妹妹有口难言,所以夜夜魂梦不安。可如今那人追随姐姐到地下,姐姐再有什么冤屈,问她便是。''

永璜听得这些言语,恍如晴天一道霹雳直贯而下,震得他有些发蒙,他哪里忍得住,直直闯进去道:``你的话不明不白,必得说个清楚。''

那女子吓得一抖,转过脸来却是玉妍失色苍白的面容。身边的贞淑更是花容失色,紧紧依偎着玉妍,颤声道:``大阿哥。''

玉妍勉强笑道:``大阿哥怎么来了?哦哦,今日是你额娘生辰,你又是孝子\ldots\ldots{}''

永璜定下神来:``就是孝子,才听不得嘉娘娘这种糊里糊涂的话。今日既然老天爷要教儿臣得个明白,那儿臣不得不问嘉娘娘了。''

玉妍慌里慌张,连连摆手:``没什么糊涂的,你额娘和孝贤皇后同为富察氏一族\ldots\ldots{}''

``我额娘死得不明不白!方才嘉娘娘说儿臣的额娘走得糊涂。嘉娘娘的意思是\ldots\ldots 儿臣得额娘本不该这么早走?''

玉妍眼波幽幽,忙取了手中的绢子擦拭眼角:``唉\ldots\ldots 多久远的事了,有什么可说的。说了也徒添伤心。大阿哥等下还要去主持丧仪呢,这么气急败坏的可要失礼数的。''她见永璜毫不退让,一壁摇头,似是感伤,``可惜诸瑛姐姐走得早,想起当日姐姐与本宫比邻而居,说说笑笑多热闹。唉\ldots\ldots{}''

贞淑一壁连连使眼色,一壁怯生生劝道:``小主\ldots\ldots{}''

玉妍猛地回过神,懊恼地拍了一下自己的脸:``瞧本宫这张嘴,什么话想到就说了,竟没半些分寸。这半辈子了,竟也改不得一点!''玉妍轻叹一口气,柔声道:``大阿哥和本宫一样,都是个实心人,却不知实心人是最吃亏的。''

永璜低声道:``嘉娘娘心疼儿臣,儿臣心里明白,有些话不妨直说。''

玉妍挺着肚子,眼角微微湿润:``本宫出身李朝,虽然得了妃位,生了皇子,却总被人瞧不起。本宫母家远在千里,我们母子想要寻个依靠也不能啊。''

永璜连忙笑道:``嘉娘娘放心。儿臣是诸子中最长的,一定会看顾好各位弟弟。''

玉妍感触到:``有大阿哥这句话,本宫还有什么不放心的呢。''她忽然屈下膝,行了个大礼道,``但愿大阿哥来日能看顾本宫膝下幼子,不被人轻视,本宫便心满意足了。''

永璜见她如此郑重,慌了神道:``嘉娘娘嘉娘娘,您快请起。''

玉妍执拗,只盯着永璜,泪眼蒙眬道:``有嫡立嫡,无嫡立长。大阿哥若不答应,本宫不敢起身。''

永璜拗不过,只得到:``嘉娘娘所言,儿臣尽力而为便是。''

玉妍这才起身,恢复了殷勤小心的神色,低声道:``慧贤皇贵妃的宫女茉心去世前曾见过本宫,那时她临死,说起你额娘之死乃是孝贤皇后所为。本宫不知道茉心为什么要来告诉本宫,或许她只是想求得一个临终前的心中解脱,或许她觉得本宫曾与你额娘比邻而居,算是有缘。所以大阿哥,作为你对本宫母子未来承诺的保障,本宫愿意将这个秘密告诉你。''

永璜紧紧握住拳头,直握得青筋暴起,几乎要攥出血来。他极力克制着道:``嘉娘娘,虽然在潜邸时的奴才们都传言皇额娘不喜欢我额娘先生下了我,可这话干系重大,断断不能开玩笑\ldots\ldots{}''

玉妍摇头道:``,茉心说完之后,不过几天就出痘疫死了,死无对证。''她叹口气,``当时本宫只当她当时病昏了头胡言乱语。不过大阿哥,就算这事是真的,大行皇后也已经离世了。哪怕她生前再介意您这个长子,也都是过去的事了。这些事您知道就好,其他的便随风而去,只当本宫没说过就是。''

永璜越听越是狐疑,面上如被严霜,迫近了玉妍,万分急切道:``合宫都知嘉娘娘是直性子,最是有什么说什么的。儿臣自幼丧母,无日无夜不思念万分。嘉娘娘早入潜邸,又与额娘比邻而居,若是觉得有什么突然的地方,还请告知一二。''

玉妍被永璜吓得连连倒退,倚在贞淑身上,二人彼此扶着,骇得面无人色,只是一味摇头。贞淑扶着玉妍,跺了跺足,发了狠劲道:``小主,从前咱们满心疑惑,却只碍着那人还活着,什么都不敢说。如今人都走了,咱们还怕什么。便是说了出来,也好过您与哲悯皇贵妃姐妹一场,为她夜夜揪心。''

永璜脸色大变,扑通跪下了道:``儿臣生母早逝,许多不明不白的地方,若嘉娘娘知道也不肯告诉,儿臣来日还有何颜面去见亡母!''他连连磕头不止,``还请嘉娘娘成全!''

玉妍忙弯腰拦住,急得赤眼白眉,为难了片刻,顾不得贞淑拉扯,咬着牙道:``罢了,本宫知道什么便全都告诉你就是了。你额娘素无所爱,只是喜欢美食。本宫原也不在意,也不大吃得惯这儿的东西,她邀本宫同食,本宫也多推却了,一直到你额娘暴毙后许久,本宫自己怀了身孕,才知道饮食上必得十分注意,许多相克之物是不能同食的,否则积毒良久,轻则伤身,重则毙命。后来本宫回想起来,你额娘暴毙后许久,本宫自己怀了身孕,才知道饮食上必得十分注意,许多相克之物是不能同食的,否则积毒良久,轻则伤身,重则毙命。后来本宫回想起来,你额娘素日的饮食之中,甲鱼和苋菜,羊肝和竹笋,麦冬和鲫鱼,诸如种种,都是同食则会积毒的。''

永璜痛苦得脸都扭曲了,低哑嘶声道:``这些东西,是谁给额娘吃的?''

玉妍登时花容失色,咬着绢子不敢言语,贞淑只得劝道:``大阿哥别逼迫小主了。当时潜邸之中,一应事务都由嫡福晋料理啊!''

永璜遽然大恸,撒开手无力地倚在墙上,仰天落泪道:``果然是她!果然是她!''

玉妍慌不迭地看着四周,连连哀恳道:``大阿哥,但求你给本宫一条生路,万万别说出来本宫知道这件事!本宫\ldots\ldots 本宫\ldots\ldots{}''她哪里说得下去,只得扯了贞淑,二人跌跌撞撞走了。

穿过空落落殿堂的风有些冷厉,吹拂起玉妍轻薄的银灰色袍角,似一只怯弱而无助的飞鸟。唯留下永璜立在殿内,任由冷风吹拂上自己热泪而冰冻的眼。

\hypertarget{ux7b2cux4e8cux5341ux4e94ux7ae0-ux7eddux5ff5}{%
\chapter{第二十五章
绝念}\label{ux7b2cux4e8cux5341ux4e94ux7ae0-ux7eddux5ff5}}

三月二十五,孝贤皇后梓宫奉移景山观德殿暂安。皇帝率六宫嫔妃、亲王福晋、宗室大臣同往,并亲自祭酒。皇帝居中,嫔妃以如懿为首,跪于左列,依次至答应。诸皇子跪于右列,以永璜为首,自四阿哥永珹以下,皆由乳母陪伴在侧。

皇帝哀恸之至,亲自临棺诵读刑部尚书汪由敦所写的祭文:``\ldots\ldots 尚忆宫廷相对之日,适当慧贤定谥之初,后忽哽咽以陈词,朕为欷吁而悚听\ldots\ldots 在皇后贻芬图史,洵乎克践前言;乃朕今稽古典章,竟亦如酬夙诺。兴怀及此,悲恸如何\ldots\ldots{}''

汪由敦是本朝出名的文人,下笔文词委婉,感人至深,更兼皇帝临表涕零,娓娓读来,更是动人心肠。在场之人都含了悲痛之色,见皇帝如此伤感,益发哀哀不止。一时间无人不涕泪纵横。永璋原本尚有犹豫,回头见永琪果然呆呆跪着,眼中一点泪意也无,一时间下定决心,生生把含在眼里的泪退了回去,朗声道:``皇阿玛请节哀,勿再哭泣伤身。''

皇帝正在伤心欲绝,听得这一声,骤然转过头去。他这一回头,见永璋殊无悲痛之色。永璋见皇帝注目,心头一喜,道:``皇阿玛节哀,您看大哥镇定自若,毫无悲切,果然气度非凡。''

皇帝眼风扫过,见永璜眼中干涸,神情淡漠,唯在永璋说话时露出厌恶之色,想起海兰言语,不觉沉下了脸。皇帝道:``永璋,你想说什么?''

永璋磕了个头,恭恭敬敬道:``皇阿玛节哀。大行皇后弃世,多日来皇阿玛一直沉浸于悲痛之中,儿臣心疼不已。但愿皇阿玛以龙体为念,切勿悲伤过度。''

皇帝漠然道:``你好孝心!时时处处挂念朕。只是今日是你嫡母丧礼,你两眼只瞧着你大哥举动做什么?难不成你大哥在你心里比嫡母还要紧?''

永璋一怔,连忙道:``儿臣不敢!''

皇帝屏息片刻,两眼如炬:``那么永璜,你又是为什么,对你的嫡母一滴眼泪都没有?''

永璜如何能说得出自己的苦衷,怔了片刻,只得勉强挤出伤心神色:``儿臣想着皇阿玛过于哀伤,儿臣身为长子,还得替皇阿玛操持着大行皇后的丧仪,不敢过于悲痛伤身,以免误了差事。''

皇帝大笑一声,右手颤颤指着两个儿子,一语不发。嫔妃们突然见生了这样的变故,一时也都惊住了,含着泪不敢言语。皇帝回过神来,脸色生硬如铁,朝着两位皇子狠狠扇了两耳光,勃然大怒:``不肖子!大行皇后是你们的嫡母,如今薨逝,你们却不悲不痛,只顾着内斗相争!朕如何会有你们这两个不孝不忠的儿子!''

绿筠吓得低呼一声,赶紧膝行出列,抱住皇帝的腿道:``皇上息怒!皇上息怒!永璜和永璋都是为您着想,不敢过于哀哭,也怕您伤了龙体,并非不孝啊!''她惊慌失措,指着永琪道:``何况也不是永璜和永璋不哭,永琪也没有哭啊!''

皇帝冷冷盯住永琪:``小儿也是这般没心肝么?''

永琪不解世事,睁大看眼睛,一脸无辜:``皇阿玛,儿臣本来很难过。可儿臣方才看三哥不哭只盯着大哥,像皇额娘薨逝与他无关似得。儿臣一时不解,所以不敢哭了。''

绿筠气得浑身乱颤:``你这孩子,小小年纪也敢扯谎,明明是愉妃\ldots\ldots{}''

永琪吓得哇一声哭起来,用手背抹着眼泪道:``皇阿玛,儿臣为皇额娘伤心,但额娘说儿臣不该当着皇阿玛的面哭,会让皇阿玛伤心,所以儿臣不知道该不该哭。儿臣好想皇额娘\ldots\ldots{}''

皇帝听得这一句,冷笑连连:``好个永璋!自己不孝,还带坏了弟弟!果然是兄长里的榜样!''皇帝的脸色冷得如数九寒冰,``纯贵妃,你有永璋和永瑢,朕还把永璜交给你抚养,你倒真替朕教出好儿子来!''

永璜和永璋吓得面无人色,拼命叩首不已:``皇阿玛息怒!皇阿玛恕罪!''

如懿见永璜受责,看皇帝的脸色便知是动了真怒。她膝行上前一步,正要劝解,却发现自己的裙角被海兰用膝盖死死压住。海兰谦卑地低着头,却以眼神制止她再向前一步,如懿还是不能忍耐,唤道:``皇上\ldots\ldots 永璜也是为您和大行皇后的丧仪考虑,并非有心不孝\ldots\ldots{}''

皇帝的鼻翼微微翕张,极怒道:``不是有心就如此!若是有心,岂不要弑父弑君!朕真是后悔,当初没把永璜及早送还到你身边抚养,否则也不至如此!''皇帝指着两个浑身发抖的儿子道:``大阿哥永璜已二十一岁,此次皇后大事,竟然毫不具人子之心,无半点哀慕之忱,实在不孝。以他昏愚之见,必是认定皇后薨逝,弟兄之内以他居长,无嫡立长,日后除他之外无人能肩承社稷重器,才妄生觊觎之心。朕今日就明白告诉,太子之位所关重大,以永璜言行,断不可立之。至于永璋亦不满人意,年已十四岁却全无知识,更无人子之道。朕年幼时如何恪尽孝道,似这般不识大体,朕深愧不止。总之来日,此二人断不可承继大统!''

绿筠惊呼一声,立时晕在了皇帝脚边,不省人事。皇帝毫不理会,犹自气得浑身乱颤。他双拳紧紧握住,却无人看见,他紧握的袖中,死死握住的,正是那一日素心死时手中攥着的那枚烧蓝溜金蜂点翠绣球珠花。

永璜与永璋的师傅与谙达,罚俸,杖责,并未有一丝平息之意。一时之间,满宫之中人人自危,深恐被牵连,曾经门庭若市的钟粹宫,骤然变得门庭冷落,无人探视。

而皇帝又听海兰说起琅嬅临死前举荐绿筠为后之事流传后宫,更认定是绿筠身边的人有意泄露,于是将绿筠身边伺候过的宫人一一查检,略有不顺眼的便打发出宫。

相反,如懿的翊坤宫和玉妍的启祥宫却异常热闹起来。因绿筠抱病,丧仪的后续事宜都落在了如懿的肩上。而引领诸阿哥举丧之事,却由年仅九岁的玉妍之子四阿哥永珹来担当。众人纷纷揣测,永璜和永璋被皇帝厌弃之后,永珹成了最可堪立的皇子。因为永琪的生母海兰虽是妃位却无宠,六阿哥永瑢的生母是受牵连的绿筠,七阿哥永琮夭折,八阿哥永璇亦是玉妍所生。且玉妍自潜邸侍奉皇帝以来,一直宠遇不断,更怀着腹中的孩子,可见皇帝圣眷隆重。这样看来,倒是玉妍更添了几分踏上后位的可能。

为着如此,如懿反而更谨慎,除了日常在宫中处理六宫琐事,几乎极少与嫔妃们来往,便是海兰,也见得少了。这一日海兰来看望永琪,好不容易见上了如懿,几乎要落下泪来:``姐姐这些日子对我避而不见,是在怪我害了永璜么?''

如懿对着棋盘上的黑白子思索不已,冷淡道:``你除去永璋,我无话可说。可永璜,你原不必做得这样绝。''

海兰道:``姐姐都知道了?''

如懿看着棋盘上泾渭分明的黑子与白子,并不看她:``你去对皇上说了什么?你明明知道皇上最恨旁人觊觎太子之位。杀人诛心,你的确很厉害。''

海兰凝神片刻,低低道:``永璜与永璋为太子之位明争暗斗,明眼人都看得出来。我不过让永琪在皇上面前提了明神宗的国本之争,说永璜自比长子朱常洛,埋怨皇上宠爱宠妃之子,皇上便信了。皇上如此多疑,可是我左右不得的。''

``稚子天真,为你所用。你提明神宗的国本之争,是暗指大阿哥自比朱常洛,埋怨身为父亲的皇上不喜爱自己,不肯立长子为太子,又偏爱宠妃所生的三弟,既有夺位之心,又有不孝之怨。更算准了皇上同样也会疑心永璋会仗着生母宠爱生出夺位之心,让永璜忌讳。这样一箭双雕,谋算人心,果然一丝不错。''如懿清冷道:``只是你可知道,永璜自上次遭皇上贬斥,抱病在王府,已经一个月不能起身了。他的福晋多次来求见我,希望我可以去宽解他,可我如此能够宽解?说到底,终究是我害了他。''

海兰分辩道:``我自然不是无意。但姐姐是自己亲耳听见的,如今的永璜这样势利,早不是当年承欢膝下的幼童了。他对姐姐不过是倚仗利用,姐姐又何必对他真心?''

如懿郁然长叹,摩挲着光润如玉的棋子道:``永璜到了如今的地步,固然是因为自小失母的缘故,也是因为他的境遇比别的皇子艰难许多。他错在一意谋算人心。可海兰,我们又何尝不是这样的人。''

海兰语气温婉,甚是推心置腹,神色却是冷然:``按姐姐这么说,宫里都是这样的人这样的心,和我们并无不同,难道个个都是同类?我一心为姐姐,为自己,并不觉得这样是错。''

桌上的一盏清茶淡淡凉去,温润袅袅的茶烟也只剩下触手生凉的意味。如懿缓缓道:``你固然没有错。若我是你,也只会怪永璜轻易上当,不懂克己控制情绪。成王败寇,输的人自然只有认命,没什么好说的。可海兰,他毕竟是我疼过的孩子。''

海兰脸上浮上一层如烟般的失望与哀然:``姐姐,你爱过的男人或许有一日会为了别的女人厌弃你,你疼爱过的孩子有一日会为了自己的追求来利用你。即便是我,也会用可能伤到你的法子来帮你帮自己。姐姐,恕我直言,你太重感情,这会是你最大的软肋。''

如懿默然沉郁:``还好这只是我的软肋,不是你的。''

海兰缓一缓神,脸上那种柔软的气息渐渐散去,那样小巧温柔的面庞,亦能散发出冰冷刺骨的决绝寒意:``姐姐,我不妨直言。真正值得被器重的孩子应该是姐姐和我的永琪。姐姐是永琪名正言顺的养母,以此为依靠,成为皇后指日可待。这就是我的打算。''她含着几许失落,深深拜别,``这是我和姐姐多年第一次生分吧?我知道姐姐还介意,不敢奢求姐姐原谅。但求我所言所行,姐姐都能明白便好。''

惢心看着海兰离去,为凉透的清茶添上热水,道:``小主,愉妃主子的话并没有大错。她的所作所为,若从为了你您来看,是绝对无可挑剔的。''

如懿抚摸着渐渐温热的杯盏,低郁道:``我如何不知道,只是过不去自己心里的这道坎罢了。哪怕亲耳听见永璜算计我,我想到的,始终是那个小小的、在我膝下读书写字的永璜,是我失宠即将被关进冷宫前还去为我求情的永璜。''她眼中有氤氲的潮湿,``我只是伤心,那样的好孩子,终究不见了。''

海兰转身步出翊坤宫四月花香弥漫的时节,原该是最温暖而明媚的。她却只觉得森凉的寒意无处不在地逼来,就仿佛许多年前,她亲眼看着阿玛与额娘双双死去,就像她知道自己被一夕宠幸就被抛诸皇帝脑后,那种对未来的坚信失去后的无助与迷茫。她缓步走上长街,回头看着翊坤宫金字绚烂的匾额,忽然眼底多了一层湿润的白气,遮住了她素来温柔低垂却坚毅的眼。

海兰离开后,随即来拜见的嬿婉并未获得进入翊坤宫的准许。三宝挡在宫门外,和颜悦色道:``娘娘已经歇息了,请贵人改日再来吧。''

嬿婉赔笑道:``我刚看愉妃娘娘离开,贵妃娘娘这么早就歇息了么?''

三宝笑道:``六宫琐事繁杂,娘娘难免劳累,所以愉妃娘娘也不便打扰,先行离开了。''

嬿婉讪讪笑:``那也好,我不打扰贵妃娘娘养神。若娘娘醒来,还请通传一声,说我来请过安。''

三宝笑得谦恭:``那是一定的。请贵人放心。''

嬿婉携了侍女春蝉的手离开,春蝉低声道:``贵人别在意。娴贵妃也不是光不见您,六宫的小主,她都避嫌呢。''她思忖道,``其实嘉妃娘娘也是后位炙手可热的人选,不如咱们去拜见嘉妃娘娘吧。''

嬿婉站住脚,剜了她一眼:``你也觉得嘉妃有登上后位的可能么?''

春蝉素知她与玉妍的心结,仍然道:``奴婢说句不怕小主忌讳的话,嘉妃接连生子,又得皇上宠爱,不能说没有争夺后位的可能。其实无论是娴贵妃或者纯贵妃封后,跟咱们都无干。但若是嘉妃娘娘,小主是知道的,她可不是好相与的脾气,只怕第一个要为难的就是小主您。与其如此,不如咱们先低一低头,当是未雨绸缪吧。''

嬿婉原本含了一腔子怒气,见春蝉这般为她打算,亦动了心思:``你的话我如何不明白。也罢了,去吧。''

嬿婉正转身要往启祥宫,才走了几步,却见前头煊煊赫赫一行人来,软轿上坐着一个衣饰精丽的女子,一身橘灿色凤穿牡丹云罗长衣,衬着满头水玉珠翠,被落于红墙之上阳光一照,几乎要迷了人的眼睛。

嬿婉一时看不清是谁,但见迷离繁丽一团,便知位分一定在自己之上,忙侧身屈膝立于长街粉墙之下,低眉垂首,恭敬迎接。

那行仗在经过她是停驻下来,却听一把尖利的女声带了笑音道:``哟,本宫当是谁站在路边候着呢,原来是令贵妃。''

嬿婉一听声音,心头不觉一缩,便知道是玉妍。她抬起眼,见软轿之上的女子妩媚万千,因着身孕更添了几分慵懒的高贵与丰腴,朝着她似笑非笑。她忙恭声道:``嘉妃娘娘万福金安。''

玉妍摆了摆手,打了个哈欠道:``罢了。''

跟着玉妍身边的丽心俏丽笑道:``看令贵人请安的身段语调,说是贵人的样子,可奴婢瞧着,怎么还是从前伺候娘娘时的身段口吻呢。''

嬿婉平身最恨被人提起是玉妍侍女的往事,那段不堪回首的往事,不仅是刻在心上的羞辱,亦是她最不能提起的伤疤。此刻丽心以这样戏谑的口吻提起,一点也不把她当做嫔妃看待,心下已然含刺。然而她哪里敢露出分毫来,只是一味赔笑:``丽心姑娘说笑了。''

丽心掩了绢子咯咯笑道:``贵人说得对,奴婢是说笑。从前和贵人一同伺候娘娘的时候,咱们可不是这样说笑的么?''

随行的人一同笑了起来,嬿婉面红耳赤,只得低下头,更低下头,不让温柔如手儿的四月风拂上面颊,仿佛挨了一掌,又一掌。

玉妍止了笑,看看她来的方向,便问:``刚去了翊坤宫?可见到娴贵妃了?''

嬿婉只得道:``嫔妾未进宫门,这个时候,娴贵妃怕是午睡呢。''

玉妍抚着肚子笑吟吟道:``这话你也信?怕是哄你呢。着哪里是午睡的时辰,分明是娴贵妃多嫌了你,不愿见你。''她的笑声听起来尖锐地刮着耳膜,``上回你那么巴结纯贵妃,替她去拂衣上的尘埃,如今又掉转头去讨好娴贵妃,她能理你么?换了本宫也看不上你那见风使舵的样子!罢了罢了,你还是乖乖儿\ldots\ldots{}''她正说着,忽然看见玉湖色绣缠枝红萝的鞋尖上落了一点燕子泥,不觉惊叫起来,``哎呀,哪儿来的燕子泥,脏了本宫的新鞋!''

丽心和贞淑忙不迭要替玉妍去擦拭。玉妍眼珠一转,笑道:``哎!你们忙什么?这样的事,可不是令贵人做惯了的。樱儿,你说是不是?''她说完,忙忙掩口,``瞧本宫这记性,有了身孕便忘性大。什么樱儿,如今是令贵人了,是么?''

嬿婉望着她绣工精致的鞋面上一点乌灰的燕子泥,心下便忍不住作呕。她如今养尊处优,又颇得皇帝的恩宠,哪里受过这样的折辱,一时犹豫不前。春蝉忙笑道:``嘉妃娘娘,咱们小主戴着护甲不方便,怕勾破了您这么好苏绣鞋面,不如奴婢来动手吧。我们小主常说,奴婢擦东西可干净了。''

玉妍冷下脸道:``你说令贵人戴了护甲,摘了不就成了。想在本宫跟前伺候,先得掂量掂量自己配不配。''她眼中多了一丝鄙夷的锐色,``令贵人,你不会只愿伺候病歪歪的纯贵妃,而不愿伺候本宫吧?那也好,本宫便向皇上说一声,让你和纯贵妃做伴吧。''

嬿婉浑身一凛,她知道的,玉妍有这个本事,也说得上这样的话。眼见绿筠是失势了,她如何能把自己填进去。于是顺从地摘下护甲,弯下弱柳似的腰身,用真丝绢子一点一点替玉妍擦拭着鞋子。玉妍舒服地歪着身子:``看你那小腰儿细得,说弯就弯下去了。哪里像本宫,大着快七个月的肚子,动也不方便,只好劳驾你了。''

嬿婉死死地咬着舌尖,以此尖锐的疼痛来抵御旁人看她的那种轻视而嘲笑的目光,低声道:``娘娘言重了。''

玉妍打量着她纤纤如春池柳的身量:``话说你承宠的时候也不短了,怎么一直没有身孕呢?到底是沾染了娴贵妃那种不会生儿育女的晦气呢,还是自己本就福薄?熬了这几年,却还是个贵人的位分,本宫看着都替你可怜。''

有滚热的泪一下灼痛了双眼,嬿婉死死忍着,让自己的声音听起来像在笑:``嘉娘娘多子多福,这样的福气,嫔妾怕是不能高攀了。''

玉妍细长的眼眸悠然飞扬,笑容灼得烫人:``你自己明白就好。能伺候在皇上身边已经是你的福气了。别妄求太多,你------不配!''

\hypertarget{ux7b2cux4e8cux5341ux516dux7ae0-ux541bux81e3}{%
\chapter{第二十六章
君臣}\label{ux7b2cux4e8cux5341ux516dux7ae0-ux541bux81e3}}

最后三个字,从金玉妍艳而灼的红唇间如吐着瓜子皮一般轻巧吐出,深深刺在嬿婉心上。争了那么多,求了那么多,原来还是旁人眼中的不配!没有孩子,他便要落到如此境地么?她盯着玉妍隆起的肚子,手指控制不住地发颤。她从未觉得,玉妍高高隆起的肚子是这般惹人生厌。

丽心笑眉笑颜道:``还请令贵人仔细些,别粗手重脚地擦破了小主的鞋。''

玉妍瞥了嬿婉一眼,翘起鞋尖,看的确是擦干净了,方才懒懒道:``好了,退下吧。本宫这苏绣的鞋面可比你的手指还娇嫩呢。''她抬起脚尖,顶了顶嬿婉的下巴,肆无忌惮地笑了起来。

苏绣的鞋面光滑得如新生婴儿的肌肤,几乎吹弹可破。那细密的针脚,鲜艳的配色,一针一线的精巧,硌在他的下巴上,却几乎能蹭出心上的血滴子来,嬿婉攥着绢子站在玉妍面前,不敢动,也不敢退却,渺小的如同一粒尘芥。她忽然觉得,凭着自己所拥有的微薄恩宠,或许哪一日被掩埋在这红砖青瓦之下,也无人问津。

玉妍正得趣,却见李玉带着凌云彻过来,见了她忙打了个千儿道:``嘉妃娘娘万福金安。''

玉妍顺势收回脚,端正了神色笑道:``李公公往哪儿去,这么匆匆忙忙的。''

李玉道:``奴才正要去启祥宫传旨,皇上请娘娘往养心殿共同用晚膳。''

玉妍忙笑道:``有劳公公了,本宫即可就去。''玉妍瞥了嬿婉一眼,轻嗤一声,仿佛厌倦了戏弄老鼠的猫,挥手扬长而去。嬿婉身子一晃,春蝉赶紧扶住了,急切道:``小主,您没事吧?''嬿婉撑着她的手臂站直身子,望着玉妍远去的背影,狠狠掐住了自己的手心。

凌云彻见玉妍走远,忙向李玉道:``公公,我认识去缎库的路,我自己去就可以。公公还是忙着差事去吧。''

李玉微眯了双眼,手笼在衣袖里,笑道:``也好,凌侍卫,皇上记得你救皇后的事,一定要赏你十匹贡缎再做嘉许。你前途无量啊!''

二人拱手而别。嬿婉转过脸,见是凌云彻,知道方才的窘迫都已经落进了他的眼里,越发觉得难堪,恨不得钻进宫墙的缝隙里才好。嬿婉微微横了一眼,春蝉知趣地退开几步,云彻掏出怀中的手帕递给她:``擦一擦吧。''

嬿婉并不去接,云彻微微尴尬,还是笑了笑:``臣下用的东西,小主怎么肯用呢。''

嬿婉将手中的娟子狠狠扔开,抬起绣着白色晓春橘花的袖口用力擦了擦下巴,别过脸道:``我情愿是皇上看见,也不要是你看见。''

云彻默然片刻:``皇上看见是怜惜动情,微臣看见,不过是故人伤情。''

嬿婉哧地一笑,眼里却不由自主冒了几分朦胧的泪气:``我以为你已经忘记了,我们是故人。''

云彻别过脸,清癯的面庞上多了几分英气。是啊,他们都不再是十三四岁的少年,两个渐行渐远的人,如何还有故人心肠。他低声道:``小主要努力忘记的,微臣也会努力忘记。''

嬿婉眼中闪过一丝清亮的明色:``云彻哥哥,要努力忘记的,终究是最难忘记的,是不是?''

有一瞬间的怔仲,连嬿婉自己也不明白,为什么会问出这样的话来。身为宫妃的日子里,她无时无刻不骄傲地提醒着自己,已经是至高无上的君王的女人。她一直不屑提起过往,克制着想起自己所不屑的时光里的人,譬如,云彻。所以她一直避免着与她的相见与交谈。

其实他们自己都知道,彼此是常常能见到的。当她去养心殿承恩的时候,被锦被裹着赤裸的身体从围房抬进养心殿的寝殿时,她会在深沉的黑夜里,看见他守在殿外的模糊的面孔。她甚至猜想,若是在风大的夜里,他是否也能听见自己在皇帝身下甜腻而暧昧的娇笑与呻吟。

但,一重门内,一重门外,便是天渊之别。

而分隔这么多年后,这是她第一次,又换回旧日的称呼,叫他``云彻哥哥'',一如从前。

仿佛有水珠从高处清冷落下,嗒一声,重重敲在心上。无数的往事瞬时汹涌上心头,少年时清纯的嬿婉与此时高贵而娇艳的嬿婉的面庞互相交叠着,许久也不能叠成同一人。

云彻看着她眼底有一丝难掩的怜惜:``嬿婉,这就是千辛万苦求得的路么?''

嬿婉的眼底涌出晶莹的泪水:``这条路固然不好走,也未必见得比从前的路难走许多。我会自己想尽办法,把这条路变得好走一些。''

云彻尽量冷漠了语气,却仍有一丝难掩的温情:``这样与人争,与人斗,还要被人羞辱。嬿婉,我只是觉得你太辛苦。''

``所有的路要往前走,都一样辛苦。''嬿婉的语气低柔如悄然绽放的花瓣,一点一点摇晃着细而软的蕊,``有你这句关怀,我已经很足够。''

她欠身,缓步离去。在数步之后迎上了春蝉伸来搀扶的收,低沉而坚定:``春蝉,无论用什么办法,我一定要怀上一个孩子,一定!''

孝贤皇后薨逝后的日子,虽然琐事不断,却也有条不紊安宁地过了下去。绿筠静心``养病'',几乎是自闭于宫中,日日吃斋念佛惟儿女祝祷,盼望着能平息皇帝的盛怒。宫中唯有玉妍张扬些,却也因为怀着身孕,又不能侍寝,众人都让着她,玫嫔的恩宠渐渐不如从前,唯意欢一枝独秀些。另外,便是海兰、嬿婉、陆缨络、婉茵与秀答应了,除了海兰无须承恩邀宠,其他人也就如常过着。而如懿,除了料理后宫诸事,便一心一意抚养永琪。

相对于后宫的平静,前朝却不太安静。孝贤皇后薨逝的余波不断,先是皇帝发现皇后的册封文书译为满文是,误将``皇妣''译为``先太后'',盛怒之下,将管理翰林院的刑部尚书阿克敦按``大不敬''议罪,斩监候后赦免;刑部满汉尚书、侍郎全堂问罪,革职留任。又因翰林院撰拟皇后祭文,用了``泉台''二字,皇帝认为这两字用于常人尚可,``岂可加之皇后之尊''?连带着三朝重臣,大学士张廷玉等也受到罚俸处分。

工部因办理皇后册宝``制造粗糙'',全堂问罪。光禄寺因置备皇后祭礼所用之饽饽、桌张``俱不洁净鲜明'',光禄司卿、少卿俱降级调用。宗人府也几次受到申饬。随后,外省满族文武官员五十余人因没有具奏折请赴京叩谒皇后梓宫,或降级或消去军工处分。一批官员在皇后丧期内违制剃发,经查究后受到惩处。两江总督尹继善、闽浙总督喀尔吉善、漕运总督蕴著、浙江巡抚顾琮、江西巡抚开泰、河南巡抚硕色等五十三名,均是在先帝在时便受重用的臣子,此次亦再惩处之列。江南河道总督周学建更因擅自剃发,又发现有贪污行为,赐令自尽。甚至因``违制剃发'',连惠贤皇贵妃的父亲大学士高斌特受到严遣,被皇帝在朝堂上当面申饬。

旁人也就罢了,张廷玉乃是三朝重臣,又是一直以来力撑孝贤皇后在后宫地位的老臣之一,此时因孝贤皇后薨逝而获罪,实在是出人意料。更何况惠贤皇贵妃死后,皇帝追念不已,每到皇贵妃去世的填仓日,必定作诗悼念,年年如是。又对惠贤皇贵妃的阿玛都没被顾及,受了这般惩处,实在是皇帝已愤怒到了极点。

所以李玉来请如懿时,脸色都变了,有些不安地擦着额头上因为一路小跑而出的汗:``娴贵妃,高斌大人和张廷玉大人都在养心殿被训斥,皇上发了大脾气,这个时候,怕是只有您能去看看了。''

如懿放下手头正在整理的八宝五色丝线,问道:``皇上怎么又训斥他们了,不是前两日在朝堂上已经训斥过了么?''

李玉忙道:``张大人和高大人原是为上次受责的事前来请罪的,不想皇上见了他们说起要将孝贤皇后东巡时所居的大船青雀舫运回京中保存,高大人原本不敢辩驳,张大人仗着是老臣,先赞许了皇上的伉俪情深,又说此举不妥。''

``不妥?''如懿疑惑道,``青雀舫是孝贤皇后最后所居之地,皇上不过想保留此船,有何不妥么?''

李玉皱了皱眉,比划着道:``船太大了,城门洞狭窄,根本进不了城。皇上就想把城门楼给拆掉。''

如懿大吃一惊,旋即道:``这样的大事,难怪张廷玉要反对了。''

李玉搓着手道:``可不是。所以皇上动怒了,斥责两位大人没心肝!两位大人早了斥责也罢了,皇上气伤了身子可怎么好。''

为着孝贤皇后的丧事,皇上连日来动怒,如懿心下也有些吃紧,便赶紧吩咐了轿辇随着李玉去了。

养心殿中极安静,宫女太监们都伺候在外,一个个鸦雀无声地垂手侍立着,生怕皇帝的雷霆之怒牵扯到他们。如懿扶着李玉的手下了辇轿,示意蕊心和菱枝候在阶下。她才步上汉白玉台阶,便已听得皇上的震怒之声:``孝贤皇后是天下之母,朕为天下之母而拆去一座城墙便又如何了?你们家中夫妻两全,朕的丧妻之痛,你们如何能懂得?全是没心肝的东西,之后满口仁义道德。出去!''

如懿候在殿外,只见两位老臣面面相觑,狼狈不堪地退了出来,见了如懿,便躬身请安:``娴贵妃娘娘万福。''

如懿微微颌首,并不在意他们对于自己的态度不甚恭敬。也是,她与孝贤皇后、惠贤皇贵妃明争暗斗了半辈子,张廷玉一向护持皇后,高斌是皇贵妃的生父,何必要对自己毕恭毕敬。她看着两人的背影,意味声长地笑了笑,尊重与恭敬,原也不在一时。

她缓缓步入殿内彼氏正值午后,四月曛暖的风被紧闭的窗扇隔绝在了外头,阳光亦成了映在窗上的一缕单薄的影子,飘渺无依。皇帝仰起头躺在冰凉的椅子上,一脸疲惫。

如懿笑道:``皇上这样仰面躺着倒好,从来人只看自己脚下的路,却很少望望自己头顶上方是什么。以至乌云盖顶都不知,还在匆匆赶路。''

皇帝的声音里透着淡淡的倦意:``你来了。那朕发脾气,你都听见了。怕不怕人?''

如懿走近他身边:``君子天怒,四海战栗,臣妾当然怕。何止臣妾怕,方才张廷玉与高斌两位大人走出去,战战兢兢,如遭雷击。臣妾想,他们真的是害怕了,也只有他们害怕,朝廷上下才都会敬畏皇上,不再把皇上当成刚刚君临天下的年轻君主。''

皇帝舒一口气,以手抵上额头:``如懿,朕已经三十七岁了。''

如懿从身后搂住皇帝,感慨良多:``是,臣妾已经陪伴皇上十七年了。十七年来,臣妾从未见过皇上如此雷霆之怒。''她从按上取过珐琅描花小钵里的薄荷油,往指尖搓了点蘸上,替皇上轻轻揉着额头,``皇上对着外人发发脾气就罢了,可别真动了怒气伤肝伤身。依臣妾来看,皇上今日做的是高兴的事呢。''

皇帝闭目深吟:``朕怎么高兴了?''

如懿明春一笑:``这些日子来,外人看着皇上肝火甚旺。但皇上处罚的人,或是三朝元老,或是先帝旧臣,或是嫔妃母家。对于尾大不掉,又在前朝倚老卖老掣肘皇上的人,趁这个机会除去,名正言顺,又是皇上情深之举,绝不惹人诟病。''

皇上的嘴角露出几分从容的笑意,伸手攀住她的手道:``如懿,何必这样聪明''

如懿伸开细长的手指与皇帝牢牢交握:``不是臣妾聪明,是臣妾与皇上一心''

皇帝将脸颊紧紧贴在她柔滑手背上:``朕喜欢你说这个词,一心。''

如懿温婉地笑了笑,有一丝感动,亦有一丝疑惑。或许在外人看来,皇帝对皇后这样追念,也是男的的一心了吧。也许所谓的一心,本来就是落在旁人眼里的如花似锦、花团锦簇,而内里却千疮百孔。谁知道呢?

静默了片刻,如懿还是问:``皇上虽然训斥了张廷玉和高斌,但移动青雀舫之事,皇上心中应该已有算盘了吧?''

皇上颌首道:``礼部尚书海望替朕想出了一个运船进城的方法,即搭木架从城墙垛口通过。木架上舍友木轨,木轨上铺满鲜菜叶,使之润滑。届时促使千余名工人推扶拉拽,便可将御舟顺利运进城内,既能保住城楼,又可节省大量人力财力。朕思来想去,孝贤皇后死在宫外,最后一息尚存之地是青雀舫,那么朕将青雀舫移入京城,也可略表哀思。''

她垂首:``皇上对皇后心意真切,臣妾敬服。''

皇帝慢慢拨着手指上的玉扳指:``孝贤皇后薨逝已是无法挽留之事,朕再伤心,也不过是身外之事。只是朕不若借着这次的事好好肃清朝廷,那么那帮老顽固便真以为朕还是刚刚登基的皇帝了。''

如懿浅浅微笑:``朝廷上的事臣妾不懂。臣妾只知道,一朝天子一朝臣,自己手里提拔上来的,才会真正感恩戴德,没有二心。''

皇帝会意一笑:``朕倒是不怕他们有二心,他们也不敢!只是别总以为自己有着可以倚仗的东西,便自居为老臣,朕喜欢听话的臣子,那些喜欢指手画脚的,便可以退下去歇歇了。''

如懿心中一动,想要说些什么,终究觉得不妥,只得换了无意的口气道:``皇上说的是。只是外人也就罢了,永璜和永璋到底是您亲生的孩子,您气过了便也算了。永璜抱病至今,什么人都不敢见,永璋也总是垂头丧气的,怪可怜见儿的。''

皇帝看她一眼,冷然道:``女人的心思就这么温柔细巧,落不得大台面么?或者说,如懿,你一向是最聪明通透的,为什么落到了子女身上,便这般看不清楚。''

如懿一怔。却只能把这惊愕转化为略略郝然的神色:``臣妾不过是个小女子,眼界短浅。偶尔能猜到皇上的心思也不过是侥幸而已,如何真能像皇上一样目光如炬呢?''

皇帝这才释然一笑:``也罢。你一直生活在后宫,所看的世界不过是这紫禁城内的一方天空,难怪许多事被遮了眼睛。''

皇帝的手指扣在紫檀木的桌面上有沉闷的笃笃声:``永璜和永璋的事,固然有他们不孝之处,但朕也明白,他们的不孝,也有孝贤皇后自己的过失在里头,怪不得两个孩子。''

如懿见皇帝的口气有点松动,很为永璜松了口气,忙道:``皇上说的是,孩子们年轻,毛毛躁躁也是有的。''

皇帝口吻陡地凌厉,他站在紧闭的窗扇下,阳光镂在长窗上的印花如同淡淡的水墨痕迹,为皇帝的面孔覆上一层浅浅的阴翳,愈发显得他天威难测:``但朕最介意的,是身为朕的长子与三子,他们居然觊觎太子之位。他们为孝贤皇后守孝以来的种种举止,当朕都看不见么?一个自诩为长子,一个自诩为有生母可以倚仗争宠。这些行径,是当朕死了么?''

如懿见皇帝的口气虽然平静,但底下的森冷意味,如汹涌在河流底下的尖冰,随时可以把人扎得头破血流。她忙伏下身道:``皇上息怒。您正值盛年。阿哥们不敢动这样的心思。尤其是永璜,哲悯皇贵妃去世得早,他一直没有生母教导,能倚仗的只有皇上您,他更不敢有这样的僭越之心。''

皇帝冷哼一声:``再不敢,他也已经动这样的心思。圣祖康熙子嗣众多,长子允禔有夺嫡之意,一直被幽禁而死。前车之鉴,朕如何能不寒心?何况朕的儿子,必须听朕的话,顺从朕的意思。朕伤心的时候他们怎敢不伤心,当着嫔妃亲贵的面与朕不同心同德,朕如何能忍?''

呵,这才是真意了。天家夫妻,皇族父子,说到底也不过是君臣一般,只能顺从。不,连做臣子也有直言犯谏的时候,他们这样的人却也是不能的。只有低眉,只有顺从,只有隐忍。

她们,和他们一样,从来都不是可以有自己主见与意念的一群人。

如懿于是缄默,在缄默之中亦明白,永璜与永璋命运的可悲。或许海兰是对的,她游离于恩宠之外,所以可以看得透彻,一击即中。她推开窗,外头有细细的风推动者金色的阳光涌进,空气里有太甜腻的花香,几乎中人欲醉。那醉,亦是自己醉了自己的。

\hypertarget{ux7b2cux4e8cux5341ux4e03ux7ae0-ux59d0ux59b9}{%
\chapter{第二十七章
姐妹}\label{ux7b2cux4e8cux5341ux4e03ux7ae0-ux59d0ux59b9}}

是夜,如懿宿在养心殿。皇帝睡得极熟,她却辗转无眠,只是一任他牵住自己的手沉沉睡去。呵,真是酣眠。她盯着枕边人熟睡中的面孔,嘴角微微翘起的弧度有温暖而诱惑的姿态,眼角新生的细纹亦不能掩饰他巍峨如玉山的容颜。当真是个俊逸的男子,不为岁月所辜负。

她的手与他紧紧交握,在他熟悉的掌纹里默默感知着彼此年华的逝去。到底,他们都已经变了。他不再是翩翩少年,而是颇具城府的帝王;而自己,已不再是骄纵任性的闺秀,而是善于谋算的宫妃。但,无论如何,他们都还是般配的。因着这般配,才不致彼此离散太久。

如懿出神地想着,忽然觉得有些冷。她伸手抓住锦被紧紧裹住自己的身体,却在那一刹那察觉,如果靠近身边身体温暖的男人,会是更好的选择,然而,他还是选择了自己保护自己,哪怕是在与自己肌肤相亲过的男人身边。

这一种下意识,几乎在瞬间逼出了她一身冷汗。是,或许在她的心底,这个男人未必能保护自己,那么会是谁,谁才能在危险的境地里义无反顾地护住自己。她细细寻思,细细寻觅,唯一能想起的人,居然是凌云彻。

那个小小的侍卫,他有着乌墨天空里明灿如星子的眼睛。哪怕你知道,他也心怀向上的欲望,但他的眼睛,不似她一直看过的那些男人的眼睛,只被欲望的权势蒙住了眼睛。

这样隐秘而不可对人言说的想法,让她在温暖绵绵的被褥里冒着凉浸浸的寒意。骤然,皇帝的呻吟声在睡梦中想起,他温柔的呢喃:``琅嬅,琅嬅\ldots\ldots{}''

如懿仔细分辨片刻,才想起那时孝贤皇后的闺名。在她的记忆里,皇帝从未这样叫过皇后的闺名,他一直是以身份来称呼她,``福晋''或者``皇后''。

她看着皇帝在睡梦里痛苦的摇着头,额上冒出细密的汗珠,终于忍不住推醒了皇帝,轻柔替他擦拭着汗水:``皇上,您怎么了?''

皇帝惊坐起来,有瞬间的茫然,看着帐外微弱的烛光所能照及的一切,气息起伏不定。

如懿柔声问:``皇上,您是不是梦魇了?''

皇帝缓过神来,疲乏地靠在枕上,摇头道:``如懿,朕是梦见了孝贤皇后。她站在朕的床前,满脸泪水地追问朕,日后会有谁取代她入主长春宫。她还直追问朕:皇上皇上,你为什么那么久没叫过臣妾的闺名?你是不是还在怀疑臣妾,怨恨臣妾?''皇帝颓然地低下头,``这样的话,皇后在临终前也问过朕。但朕念着她往日的过错,始终不肯叫她一声`琅嬅',所以她追入朕的梦里,死死缠着朕不放。''

如懿看着皇帝,神色清淡温然,有着让人平静的力量:``人无完人。孝贤皇后虽然有她的错失,但她对皇上的心也是无人能取代的。''

烛影摇动暗红烨烨,皇帝清峻的面容在幽暗的寝殿中并不真切,深邃的眼眸仿佛一潭深不可见的池水。良久,皇帝长舒了一口气,唤进毓瑚道:``你去告诉李玉,传朕的旨意,长春宫是孝贤皇后生前的寝宫,朕要保留孝贤皇后居住时的所有陈设,凡是她使用过的奁具、衣物,一切按原样摆放,再将孝贤皇后生前用过的东珠顶冠和东珠朝珠供奉在长春宫。''他思量片刻,有道,``等等,去吧惠贤皇贵妃的画像也供在那里。还有。每年的腊月二十五和忌辰时,朕都会前往亲临凭吊。长春宫,朕不会再让别的嫔妃居住。''

毓瑚答应着退了下去,如懿默默听着皇帝的种种嘱咐,神色安静如常``皇上这样做,孝贤皇后地下有知,也会安慰。皇上可以安心了。''

皇帝郁然长叹:``朕作了一篇怀念孝贤皇后的《述悲赋》。过几日,朕会亲自抄录送与皇后灵前焚化,希望她在九泉之下与永琏和永琮母子相聚,能够稍稍宽慰吧。''

夜风拂动芙蓉锦帐堆雪似的轻纱,帐上的镂空银线串珠刺绣花纹晶光莹然,床头的赤金九龙帐钩在晃动中轻微作响,连那龙口中含着的明珠亦散出游曳不定的光。皇帝复又躺下,沉沉睡去。如懿望着他,只觉得心底有无数端绪萦绕辗转。最后,亦只能闭上眼,勉力睡去。

这一觉睡得轻浅,如懿醒来时,皇帝正起身准备穿戴了前去上朝。如懿已无睡意,索性起身服侍皇帝穿上龙袍,扣好盘金纽子。皇帝的眼下有淡淡的墨青色,如懿站在他跟前,正好够到他下巴的位置,只觉得他呼吸间暖暖的气息拂上面颊亦有滞缓的意味,轻声道:``皇上昨夜没有睡好,等下回来,臣妾熬着杜仲雪参红枣汤等着皇上。''

皇帝温言道:``这些事就交给下人去做吧。你昨夜也睡得不甚安稳,等下再去眠一眠吧。''

如懿低低应了一声,侍奉着皇帝离开,便也坐着软轿往翊坤宫中去。天色只在东方遥远的天际露出一色浅浅的鱼肚白,而其余的辽阔天幕,不过是乌成一片,教人神鬼难辨。惢心伴在她身边,悄声问:``小主,为何孝贤皇后生前皇上对她不过尔尔,她薨逝之后,皇上反而如此情深,念念不忘?''

如懿淡淡笑道:``有时候人的情深,不仅是做给旁人看的,更是做给自己看的。入戏太深太久,会连自己都深信不疑。''

惢心有些茫然:``小主的话,奴婢不懂。''

如懿长吁一口气:``何必要懂得。你只要知道,你活着的时候他待你好,才是真的好。''她凝神片刻,``惢心,你快三十了吧?总说你二十五岁便让你出宫,可拖着拖着,你都快三十了。九月里是你的生日,便可以放你出宫了。''

惢心笑道:``是。日子过得真快,二十五岁的时候本可离宫,但总觉得离不开小主,如今都快三十了。''

``我刚出冷宫的时候你总说要多陪陪我,如今三十了,可以出宫好好嫁了吧。江与彬是个很不错的人选,我会告诉皇上,把你赐婚给她。''

惢心脸上带着红晕,诚恳道:``可奴婢还想多伺候小主几年。''

如懿微笑:``年纪不等人,一个女人的好年岁就这么几年,别轻易辜负了,再不嫁了你,不知道江与彬背后得多恨本宫呢。不过话说回来,即便你嫁人了,白日里进宫按班序伺候,晚上出宫,也是无妨的,我希望你好好儿出宫,安稳过日子。''

惢心激动得满眼含泪,二人正说话,软轿一停,原来已经到了翊坤宫门口。如懿扶着惢心的手下了软轿,三宝匆匆迎上道:``小主可回来了。延禧宫递来的消息,愉妃小主从昨夜进了太后宫中,一直到现在都没有出来。跟着伺候的人说,愉妃小主在慈宁宫的院落里跪了一夜,太后到现在都不许她起来。''

如懿心下一凉,即刻问:``这消息旁人知道么?''

三宝摇头道:``延禧宫的人都是愉妃小主亲自调教出来的,懂得分寸,只敢把消息递到咱们这里,旁人都不知道。''

如懿略一思忖,往前走了几步:``惢心,我乏了,再去睡一会。''

惢心答应着替她接过解下的云丝银罗披风,道:``是。那奴蜱伺候小主睡着,再去请五阿哥起床,该时候去尚书房了。''

如懿走了两步,微叹一口气,终究忍不住转身:``去慈宁宫!''

如懿赶到慈宁宫外时,天色才蒙蒙亮。熹微的晨光从浓翳的云端洒落,为金碧辉煌的慈宁宫罩上了一层暧昧不定的昏色。如懿伫立片刻,深吸一口气,这个地方,无论她来了多少次,总是有着难以言明的畏惧与敬而远之。

是的,太后曾经救过她,是她的恩人。但对于整个乌拉那拉氏而言,太后又何尝不是一手毁去她们所有荣华与倚仗的仇人呢。

恩仇交织,却不能奈太后何。这才是真正的敬畏。

然而此刻,海兰在里头,虽然不知道是为了什么事,但如懿隐隐觉得不安。太后虽然主持着六宫事宜,但一向并不插手小事,而且她御下也极温和,甚少会有罚跪一夜的厉举。

所以越走进慈宁富,如懿心底的惴惴越重。外头的小宫女们一层层通报进去,迎出来的是福珈,她见了如懿不惊不诧,只是如常平和道:``娘娘略坐坐。太后已经起身,梳妆之后就可见娘娘了。''

太后索性喜爱时鲜花卉,皇帝又极尽孝养,故而慈宁富内广植名贵花木,以博太后一笑。诸如海棠、牡丹、玉兰、迎春等皆为上品,又有``玉堂富贵春''的好意头。花房还特拨十名积年老花匠,专心照料太后最爱的几株合欢花。因此慈宁宫内繁花似锦,永远花开不败。更兼夜露莹透,染上花花草革,更是透出别样的娇艳来。

如懿看了看院子里,除了花草芳菲,唯有两只仙鹤在芭蕉下打盹儿,四下静静的,并无跪着什么人。如懿越发担心,低声问道:``姑姑,愉妃呢?''

福珈笑吟吟垂着手道:``愉妃娘娘是有位分有孩子的,太后怎会要她如此丢了脸面,要跪也不会跪在这里。否则传了出去,愉妃娘娘还怎么做人呢?''

如懿猜不透太后的盘算,便跟着福珈进了暖阁坐下。福珈指着案几上一碟莲心酥并一碗核桃酪道:``这是太后昨夜给娘娘备下的夜宵,娘娘没用上,已经凉了,奴婢叫人撤了,换些早膳点心吧。''

如懿诧异,却只能不动声色含笑道:``姑姑怎知本宫没有用早膳?''

福珈笑道:``奴婢哪里能知道,不过是按着太后的吩咐做事罢了。只不过娘娘昨夜没来,那必定是因为侍寝而不知道。若是侍寝之后即刻回富,那这个时辰知道了会赶来。娘娘一向与愉妃娘娘情同姐妹,不是么?''

如懿暗暗咋舌,太后身边一个姑姑都活成了水晶玻璃通透人儿,何况是太后自己。看着早膳上来,她索性定下神来,用了点奶茶和马蹄饼,又用了一小碗栗子粥。福珈在旁笑眯眯道:``太后临睡前嘱咐了,要是娘娘没有用东西的精神,她便懒得和娘娘多言了。要是娘娘还吃得下,那就还能有心思说话的。''

如懿心头微微发沉,像是坠着什么重物一般,她依然含笑:``福珈姑姑,本宫已经吃饱了,哪怕太后要拉着本宫和愉妃一切受罚,本宫也有力气支撑。只是愉妃\ldots\ldots{}''

福珈如何不懂,笑道:``娘娘放心。太后罚跪便是罚跪,不会饿着愉妃娘娘的。愉妃娘娘若是能,跪着瞌睡也成。''

如此回答,如懿亦只能缄默了。静候了一炷香时分,只听见有珠帘挽起的轻晃声清脆玲玲,如同细雨潺潺。隔着一挂碎玉珠帘,有透澈如水的女子声音传来,仿佛也沾染了碎玉的玲珑通透。太后从帘后漫步而出:``哀家就知道,愉妃罚跪,你迟早会来,因为这件事,少不得有你牵连。''

如懿忙起身行礼,诚惶诚恐:``太后万福金安,富春康宁。''

太后摆手道:``哀家有什么万福的?一下子折了两个皇孙在你们手里,牵连了纯贵妃好让你一人独大。这么好的算盘在哀家眼皮子底下,哀家想闭上眼睛当看不见也不成啊。''

如懿保持者恭谨的微笑:``太后的话,臣妾不明白。''

太后看着宫女们布好早膳退下,笑着从福珈手中取过茶水漱口,然后慢慢舀着一碗燕窝粥喝了几口:``不明白?哀家只须看这件事中谁得益最多,便可以猜测是谁做的。怎么,纯贵妃本与你都是贵妃,如今她抱病不出,你一人独大,还有什么可说的么?不过幸好,纯贵妃子嗣众多。除了永璋不懂事,也罢,皇上本就不喜欢永璋,总还有永瑢和璟妍。儿女双全的人哪,总比哀家着样的有福气,更比你有福气。''

如懿最听不得子嗣之事,心头倏然一刺,仿佛有利针猝不及,逼出细密的血珠。她极力撑着脸上的笑:``太后的福气,自然是谁也比不上的。只是太后所言,无非是觉得臣妾算计了永璜和永璋。''

太后搁下燕窝粥,摆手道:``福珈,这粥太淡了,替哀家去兑点牛乳。''

福珈答应了一声,引着众宫女退下,唯余如懿与太后静静相对。

太后拿绢子擦了擦唇角,随手撂下,转了冰冷脸色:``如今你的心思是越来越厉害了,永璋便罢了,连你抚养过的永璜都可以下手。虎毒尚且不食子啊!''太后面色深郁,忽而一笑,``哀家忘记了,你肚子里何曾出过自己的孩子?养子嘛,自然不必太上心的。''

如懿纵然历练多年,却也耐不住这样的刺心之语,只觉得满脸滚烫,抬起头道:``太后错了,此次的事,哪怕是臣妾算计了两位阿哥,却也顶多是让他们受一顿训斥而已。只能说臣妾算计了开头也算计不到结尾。皇上这样的雷霆震怒,可以断绝两位阿哥的太子之路,连太后抚养皇上多年,都会觉得意外,臣妾又如何能算计得到?''

太后微眯了双眼,神色阴沉不定:``你是说,你与愉妃都无错,是皇帝责罚太重?''

``臣妾不敢这样说。但太后心如明镜,皇上登基十二年,早不是以前凡事问询先帝遗臣的新君了。他有自己的主意与见解,旁人只能顺从,不能违背。即便张廷玉和高斌这样的老臣都如是,何况旁人。''如懿目视太后,意味声长,``或许在皇上眼中'母子之恩'父子之情,夫妻之义,都比不上君臣二字来得要紧呢!''

太后的目光逡巡在她身上:``这是你自己的揣测,还是皇帝告诉你的?''

如懿见太后不再动早膳,便盛了一碗牛骨髓汤,恭恭敬敬递到太后手边:``皇上天心难测,臣妾如何能得知,皇上更不会告诉臣妾什么。只是太后养育皇上多年,对皇上之事无不上心,难道会看不出来么?臣妾若真有什么算计,都也是落了`正巧'二字罢了。若和愉妃有牵扯,那也是偶然。太后是知道的,愉妃生下永琪后就再不能承宠,她没必要争宠算计。''

熹微的天光从重重垂纱帷帘后薄薄透进,太后背着光宽坐榻上,衣裾在足下铺成舒展优雅的弧度。任凭身后是四月锦绣,花香弥漫的浮光万丈,她的面孔却似浸在阴翳之中,连着浑身的金珠玉视、朱罗灿绣,都成了冰冷的死色。太后打量着如懿的神色,片刻,才伸手接过她递来的汤,慢慢啜饮:``你倒是越来越懂得看皇帝了。也算你识趣,自己认了算计永璜和永璋之事。愉妃跪了一晚上,都还不肯招了和你相关呢。''

如懿望着太后,心中隐隐有森然畏惧之情,却还是道:``此事与愉妃无甚关系。而且太后是过来人,遇见这样的事,自然明白,不会去怨算计的人有多可怕,而是可怜被算计的人为何这样容易被算计了。''

太后唇角的笑意越来越深,眼中却是极淡极淡的邈远之色,仿佛她这个人,永远是高不可攀,难以捉摸:``你这样的心思,倒是越来越像你的姑母了。''她瞥一眼帘后,``愉妃跪在哀家的寝殿外头,你自己去看看吧。''

如懿本为海兰担心,听得这一句,忙走到太后寝殿前,见海兰跪在地上,神色虽然苍白且疲惫不堪,倒也不见受了多大的折磨。

海兰一见如懿,忍不住落泪潸潸:``姐姐说的话我都听见了。何必要把事情和我撤清,原本所有的事,都是我做的,姐姐从没有做过。''

如懿示意她噤声,扶着她艰难地站起来,替她揉着膝盖道:``你先坐坐,等下我扶你出去。记得别乱动,跪了一夜,膝盖受不住。''

海兰含泪点点头,乖乖坐下。如懿转到殿外暖阁中,跪下道:``太后怜悯,臣妾心领了。自然事事为了你。但许多事,你搁在心里头就是了,不必痴心妄想。''

如懿静静地听着,目光只落在太后身后那架泥金飞绣敦煌飞仙女散花的紫檀屏风上。那样耀目的泥金玉痕,绚丽的刺绣纷繁,衣饰蹁跹,看得久了,眼前又出现模糊的光晕,好似离了人间。如懿安分地垂首:``一切由皇上和太后定夺,臣妾不敢痴心妄想。''

太后笃定一笑,叹口气道:``这话虽然老实,却也不敬。后宫的事难道哀家做不得主,还要皇上来定夺?''

如懿听到此节,心中的畏惧减了几分,轻笑道:``个中的缘由,太后比臣妾清楚。''

太后收敛笑意,淡淡道:``你便不怕哀家把你算计永璜和永璋的事告诉皇帝?你害了他的亲生儿子,他便容不得你了。''

如懿的神情清淡如同一抹云烟:``若说算计,后富里谁不曾算计过?太后一一告诉了皇上,也便是让他成了孤家寡人。太后舍不得的。''

太后冷冷笑道:``哀家舍不舍得,是哀家说了算。你既然来了,哀家也不能不罚你,可为什么罚你,哀家也不能张扬。不是为了你,是为了皇家的颜面。这件事,哀家便记在心里,你走吧。''

如懿心头一松,忙道:``多谢太后。那么愉妃\ldots\ldots{}''

太后眼皮也不抬:``你都走了,哀家还留她做什么,一起走吧。''

如懿如逢大赦,忙与叶心一起扶了海兰出了慈宁宫。海兰紧紧扶着她的手,一步一步走得极慢极慢。她站在风口上,任由眼泪大滴滑落在天水碧的锦衣上,洇出一朵朵明艳的小花:``我以为姐姐恨我狠毒,再不会理我了。''

如懿凝视着她:``我早说过,你做与我做有什么区别?我不原谅你,便也是不原谅自己。念头是我自己起的,只不过你伸出手做了。做得绝与不绝,原不在你我,而在皇上。''

海兰的轻叹如拂过耳畔的风:``姐姐从冷宫出来的那一年,曾告诉我会变得更决绝狠心,不留余地。可今时今日看来,姐姐还是有所牵绊。我一直想,皇上能做到弃绝父子之情,姐姐为何做不到?''

如懿语气沉沉:``因为我从未走到皇上站过的地方。高处不胜寒,皇上与我们看到的、感受的,自然不一样。''

海兰望着如懿,替她拂了拂被风吹乱的金镶玉步摇上垂落的玉蝶翅萤石珠络:``所以我希望姐姐可以站到和皇上并肩的位置,和皇上一样俯临四方,胸有决断。''

如懿的笑凝在唇际,久久不肯退去:``这是我的愿望,也是乌拉那拉氏的愿望。虽然我知道还有些难,但我会努力做到。''

叶心忙道:``娴贵妃这些日子忙于料理六宫的事,很少和我们小主来往,我们小主虽然不说,但心里不高兴,奴婢是看得出来的。''

海兰嗔着看了叶心一眼,泪中带笑:``其实这些日子我一直想,若是姐姐一直和我生分下去,咱们姐妹会生分到什么地步?''

如懿笑道:``现在还这么想么?''

海兰思忖片刻:``现在我想,若是我们姐妹连这样的事都没有生分,以后还会为了什么事生分呢?''

如懿浅浅笑道:``多思多虑,还不赶紧回宫,治治你的膝盖呢!''

如懿搀着海兰慢慢走在长街上,远处有明黄辇轿渐渐靠近,疾步向慈宁官走来。如懿微微有些诧异,忙蹲下身迎候:``皇上万福金安。''

皇帝脸上有着深深的关切与担忧:``从慈宁宫出来了?太后有没有为难你们?''

如懿不知就里,忙道:``这个时候皇上不是刚下朝么?怎么知道臣妾与愉妃在慈宁宫?''

皇帝道:``太后身边的宫人来传话,说你与愉妃在受责罚,朕刚下朝,便赶来看看。''皇帝执过她手,温言道,``不要紧吧?''

皇帝的日艮底似一潭墨玉色的湖,只有她的倒影微澜不动。如懿心头微微一暖:``皇上放心,已经没事了。''

皇帝微微颔首,柔声道:``你和愉妃先回去,朕正要去向皇额娘请安。''二人退到一边,眼看着皇帝去了,自行回宫不提。

\hypertarget{ux7b2cux4e8cux5341ux516bux7ae0-ux5a9aux597d}{%
\chapter{第二十八章
媚好}\label{ux7b2cux4e8cux5341ux516bux7ae0-ux5a9aux597d}}

皇帝进了慈宁宫,笑吟吟行了一礼:``皇额娘正用早膳呢,正好儿子刚下朝,也还没用早膳,便陪皇额娘一起吧。''

太后招招手,亲热地笑道:``只怕慈宁宫的吃食不合皇帝你的口味。福珈还不替皇上把冠帽摘了,这样沉甸甸的,怎么能好好儿用膳呢。''

福珈替皇上整理了衣冠,又盛了一碗粥递到皇帝手边。皇帝一脸馋相,仿佛还是昔日膝下幼子,夹了一筷子酱菜,兴致勃勃道:``儿子记得小时候胃口不好最喜欢皇额娘这里的白粥小菜,养胃又清淡。皇额娘每天早起都给儿子备着,还总换着酱菜的花样,只怕儿子吃絮了。''

太后欣慰地笑,一脸慈祥:``难为你还记得。''她看皇上吃的欢喜,便替他夹了一块风干鹅块在碗中,``纯贵妃病了这些日子,皇帝去看过她么?哀家也知道她病着,吃不下什么东西,就拣了些皇帝素日喜欢吃的小菜,也赏了她些。''

皇帝喝完一碗粥,又取了块白玉霜方酥在手:``儿子去看过她两次,不过是心病,太医使不上力,朕也使不上力。''

太后微笑着瞥了皇帝一眼:``太医无能,治不好心病,皇帝难道也不行么?''

皇帝唇边都是笑意,仿佛半开玩笑:``儿子要治好她的心病,就得收回那日说过的话,得告诉纯贵妃永璜和永璋还有登上太子之位的可能。儿子还年轻,空口白舌地提起太子不太子的话,实在没意思。''

太后叹口气,替皇帝添了一碗枸杞红枣煲鸡蛋羹,温和道:``慢慢吃那酥,仔细噎着。来,喝点羹汤润一润。''

皇帝快活地一笑:``多谢皇额娘疼惜。''他吩咐道,``毓瑚,朕记得娴贵妃很爱吃这个白玉霜方酥,你取一份送去翊坤宫。''

毓瑚忙答应着端过酥点去了。太后饶有兴致地看着皇帝:``皇帝到很在意娴贵妃啊。''

皇帝生了几分感慨:``潜邸的福晋只剩了如懿一个,多年夫妻,儿子当然在意。''

太后并无再进食的兴致,接过福珈递来的茶水漱了漱口:``皇帝是念旧情的人。裒家冷眼看着,你的许多嫔妃,年轻的时候你待她们不过尔尔,年岁长厂倒更得你的喜爱了。譬如孝贤皇后,皇帝哀思多日,从未消减。但有件事皇帝也不能不思量,后富不可一日无主。否则后位久虚,人心浮动,皇帝在前朝也不能安稳。''

皇帝的笑意如遭了寒雨的绿枝,委垂寒湿:``皇额娘,恕儿子直言。孝贤皇后刚刚去世,儿子实在无心立后。若真要立后,也必得等皇后两年丧期满,就当儿子为她尽一尽为人夫君的心意吧。''

晨光透过浮碧色窗纱洒进来,似凤凰花千丝万缕的浅金绯红的花瓣散散飞进。太后侧身坐在窗下,目光深幽幽的,直望到人心里去。她沉思着道:``皇帝长情,哀家明白。可六富之事不能无入主持,纯贵妃与娴贵妃都是贵妃,可以一起料理。或者,皇帝可以先封一位皇贵妃,位同副后,摄六宫事。''她悠然叹息,``昨日哀家看到?妍与永珞来请安,儿女双全的人,真真是有福气啊。''

皇帝眼底的笑影淡薄得如落在枝叶上浅浅的光影:``若以子嗣论,纯贵妃有永璋、永瑢与璟妍。嘉妃有永珹、永璇。嘉妃腹中这个孩子,太医说了,大约也是个阿哥。纯贵妃性子温和婉转些,嘉妃张扬犀利。但\ldots\ldots{}''

``但你都不属意?''太后闭目须臾,``可娴贵妃的家世,你是知道的''

皇帝的神色极静:``没有家世,便是最好的家世。''

太后一笑:``你是怕有人倚仗家世,外戚专权?这样看来,乌拉那拉氏是比富察氏合适,但纯贵妃的娘家也是小门小户,且纯贵妃有子,娴贵妃无子。宫中,子嗣为上。''

皇帝坦然:``正因无子,才可以对皇嗣一视同仁。''

太后脸色有一瞬的僵冷,很快笑道:``好,好!原来皇帝已经打算这样周全了。原是老太婆操心过头了。只不过先帝在时,有句话叫满汉一家,纯贵妃是汉军旗出身的,你可还记得么?''

皇帝恭谨,欠身道:``皇额娘为儿子操心,儿子都心领了。先帝是说满汉一家,所以纳了许多嫔妃都是汉军旗的。但要紧的当口上,皇后也好,新帝的生母也好,都是满军旗。皇额娘不也是大姓钮祜禄氏么?其实当年皇阿玛在时,疼爱五弟弘昼不必疼爱儿子少,但因为弘昼的生母耿氏乃是汉军旗出身,才失之交臂。皇阿玛的千古思虑,儿子铭记在心。''他顿一顿,深深敛容,``皇额娘,儿子已经不是黄口小儿,也不是无知少年。儿子虽然是您一手调教长大的,但许多事,儿子自己能有决断,可以做主了。''

挂在檐前垂下摇曳的薛荔花蘅芜丝丝缕缕,碧萝藤花染得湿答答的,将殿内的光线遮得幽幻溟濛。气氛有瞬间的冷,太后凝神良久,才勉强挤出一个笑容:``罢了。孩子长大,总有自己的主意。你既然心里选定了乌拉那拉氏,哀家说什么也无用了。你们自己好好过日子吧。但哀家不能不说一句,没有家世没有子嗣的皇后,会当得很辛苦。''

``是。日子是自个儿的,至于辛不辛苦,如人饮水冷暖自知。娴贵妃若不能顺应,便是她自己无能,儿子也无法了。''皇帝说罢起身,``前朝还有事务,儿子先告退了,晚上再来陪皇额娘用膳。''

太后点点头,目送皇帝出去。福珈点了一炉檀香送上来,袅袅的白烟四散,眼前考究而不堂皇的陈设也多一丝柔靡之意。那香烟温润,游龙似的绕住了人,将太后的容颜遮得雾蒙蒙的:``娴贵妃说得对,皇帝果然不是刚登基的皇帝了。皇帝如此桀骜,若是新后再不能把握在手中,哀家在后富的地位岂非形同虚设?''

福珈取过一枚玉搔头,替太后轻轻挠着发际:``太后的阅历,后宫无人能及。娴贵妃也不是个不懂分寸的,何况,皇上不是说了先不立后么,只是皇贵妃而已。太后自然可以慢慢瞧着。''

太后无奈一笑,深吸一口气:``这檀香的气味真好。''

乾隆十三年七月初一,乌拉那拉氏如懿晋为皇贵妃,位同副后,摄六富事:金玉妍晋为贵妃,协理六富;同曰晋舒嫔叶赫那拉氏意欢为舒妃,令贵人魏嬿婉为令嫔,庆常在陆缨络为庆贵人,婉常在陈婉茵为婉贵人,秀答应为秀常在,还有几位平日里伺候皇帝的宫女子,亦进了答应的位分,如揆答应、平答应之流。

而本与如懿同阶的绿筠却依旧只是贵妃,更添了玉妍与她平起平坐。这一来,旁人议论起来,更说是因为在潜邸时如懿便是侧福晋,当时身为福晋的孝贤皇后与侧福晋的慧贤皇贵妃都己过身,论次序也当是如懿了。而更春风得意的是新封的嘉贵妃金玉妍,在晋为贵妃的第八日,产下了皇九子,一举成为三子之母,当真荣耀无比。所以皇帝欣慰喜悦之余,特地允许玉妍接见了来自李朝的贺使与母家的亲眷,并且大为赏赐,一时间风光无限,炙手可热。

然而亦有人是望着启祥宫人人受追捧而不悦的,那便是新封了令嫔的媾婉。虽然封嫔,但她的恩宠却因着如懿晋封、玉妍产子而稀落了下来。且此前燕窝细粉之事,总是蒙了一层不悦与惶然,让她面对皇帝之时一壁暗暗勤学,一壁又生怕说错什么惹了皇帝嗤笑,所以总不如往日灵动活泼,那样得宠。此刻她立在启祥宫外的长街上,看着贺喜的人群川流不息,忧然叹息:``愉妃产子后不能再侍寝,虽然晋封妃位,但形同失宠,难道本宫也要步上她的后尘么?''她凝神良久,直到有成列的侍卫戍卫走过,那磔磔的靴声才惊破了她的沉思。她紧紧按着自己平坦的小腹,咬着唇道:``澜翠,悄悄地去请坤宁宫的赵九霄赵侍卫来一趟,本宫有话要问他。''

九霄其实很久未见嬿婉了。自从凌云彻高升后,便通融了关系,把在冷宫受苦的兄弟赵九霄拨到了坤宁宫,当个安稳闲差。赵九霄自然是感念他兄弟义气。他素日从未进过嫔妃宫殿,在坤宁宫当的又是个闲之又闲的差事,他正和几个侍卫一起喝酒摸骨牌,忽然来了人寻他,又换了太监装束从角门进去,一惊之下不免惴惴。

进了永寿宫,九宵便有些束手束脚,加之穿着不知是哪个小太监的衣裳,紧巴巴的,又有股子太监衣衫上特有的气味,更是浑身别扭。他知道媾婉是有些宠眷的,更见永寿宫布置得颇为奢华,偌大的宫殿之中,静若无人,规矩极大。他小心翼翼地挪着步子,进了殿中,九宵只觉得身上?寒,在外头走了半日的汗意倏然往千百个毛孔里一收,竟有掉进冰窟里的感觉。好一会儿才想起六宫中入夏后便开始用冰,却不知能清凉到这种境地,果然是舒坦极了。但见十二扇阔大屏风上描金漆银,雕花玲珑剔透,琴剑瓶炉皆贴在墙上,四周锦笼纱罩泛着金彩珠光,连地下踩的砖,皆是碧绿暗金的西潘莲凿话。他越发眼花缭乱,不知该往何处落脚。

澜翠很瞧不上他那战战兢兢的小家子气,又是好气,又是好笑,便轻声喝道:``娘娘在上,你的眼珠子往哪里乱转悠呢?''

赵九宵这才抬起眼来,只见暖阁的榻上斜靠着一个堆纱笼绣的美人儿。他认不清那是什么衣料,只觉得散着明艳的光芒,脸上的艳光亦是带着珠玉的华彩。身边一个宫女装束的女子堆红着绣,戴着烧蓝银器首饰,一看便知是有身份的,正替那美人儿打着一把玳瑁柄蹙金薄纱扇子。他很想仔细看看那两位女子的脸,只是阁中景泰蓝大缸中瓮着冰块冒着丝丝的雪白寒气,加之窗上的湘妃竹帘安静地垂落,那女子的脸便有些光晕模糊。半晌,只听得那榻上的女子懒懒打了个哈欠,声音悠悠晃晃道:``澜翠,人来了么?''

九宵紧张得手脚都不知道该往哪里放了,胡乱朝着前头跪下,口中呼道:``令嫔娘娘万福金安,令嫔娘娘万福金安。''

榻上的女子坐直了身子,笑吟吟道:``赵大哥,如今怎么这么客气了?快起来吧。''

九宵不是没听过嬿婉的声音,当年还是宫女的时候,清脆的,娇俏的,总是围绕着一脸喜悦的凌云彻,像只欢快的小黄莺。而如今,这声音如玉旨纶音一般,惊得他拼命磕头道:``令嫔娘娘恕罪,令嫔娘娘恕罪,微臣只是喝了点小酒摸了副牌,不是有意偷懒的!''

嬿婉娇笑一声,亲切中透着几分沉沉的威严:``澜翠,还不扶赵侍卫起来!做人哪里有不忙里偷闲的,何况本宫与赵侍卫是旧识,便是知道了又是什么大事呢。''

澜翠哪里愿意自己的手去碰到他低等太监的服色,便虚扶了一把道:``赵侍卫快起来吧,咱们娘娘还有话问你昵。''

九宵心头大石落地,这才敢抬起头来:``令嫔娘娘有什么尽管问,微臣都会知无不畜言无不尽。''

嬿婉使了个眼色,澜翠搬了张小杌子来给九宵坐下,春婵停下手中的扇子,递上一杯茶,两人便悄然退下了。九宵捧着那杯热茶,见嬿婉只是抚着金丝珐艰护甲含笑不语,便坐也个女,站也不安。片刻,嬿婉才闲闲道:``赵大哥如今和凌侍卫来往还多么?''

赵九宵一愣,才反应过来她问的是凌云彻,便脱臼道:``咱们兄弟,还和以前一样。''

嬿婉轻轻一笑,忽而郁郁:``真是羡慕赵大哥啊!本宫与凌侍卫青梅竹马,如今竟是生疏了呢。想想本富在宫中可以信赖的旧识,也只有赵大哥和凌侍卫了。凌侍卫疏远至此,真是可惜了,他怕是已经恨死了本宫吧?''

九宵摸着脑袋道:``那也不会吧。娘娘侍奉皇上\ldots\ldots 那个\ldots\ldots 云彻他虽然伤心,但也从未说过恨娘娘啊!''

嬿婉满脸忧色,抚着粉红香腮道:``形同陌路,再不过问,和恨本富有什么区别昵?''

九宵愣了愣,正犹豫着该不该说,但见媾婉愁容满面,更见清丽,便忍不住道:``云彻他还是很惦记娘娘的。他受皇贵妃提拔引荐给皇上,也替皇贵妃做事。微臣想,若不是皇贵妃与娘娘有三分相似,云彻也不会替她效力了。''

媾婉听他这般说,心中更有了三分底气,越发笑得亲切:``有赵大哥这句话,本宫也安心了。左右咱们相识一场,别落得个相见不识的地步便好了。''她说罢,也懒得虚留九宵,依旧吩咐了澜翠送了九宵出去,便问,``春婵,这个时候,皇上在养心殿么?''

春婵看了看铜漏,便道:``这个时候皇上怕是娴皇贵妃宫里午睡呢。''

嬿婉点点头,神色郑重了几分,看着湘妃竹帘一棱一棱将郁蓝天空镂成细密的线,微微眯起了双眼:``该预备的都预备下了么?''

春婵道:``都好了。''她看着院子里九宵走出去的身影道,``只是小主,想定了的事,何必还找这么个人来问问,不会多余么?''

``既然要做好一件事,就必须十分有底。''她忧然叹息,``皇上已经有半个多月没来了吧?''

嬿婉默默地转着手指上一枚红宝石银戒指,那戒指本是宝石粉嵌的,并不如何名贵,只是她戴在手上久了,成了习惯,一直也未曾摘下,那还是她剐进宫那时候,手上什么首饰也没有,被一起在四执库当差的宫女们笑话,她向云彻哭诉了,云彻咬着牙攒了好久的月俸,才替她买了这一个。当年爱不释手的饰物,如今戴着,却显得十分寒酸。初初得宠的时候,皇帝赏赐了不少珍贵的首饰,她也曾摘下过,保养得娇嫩如春葱如凝脂的手指,更适合镂刻精美名贵的首饰。可自从那个念头在她心里盘根错节地滋长时,她便又忍不住戴了起来。左右,皇帝是不在乎她戴些什么佩些什么的。嬿婉想了想,从手指上摘下这枚红宝石银戒指,递到春婵手中,下定了决心道:``去吧。''

澜翠将九宵送到了永寿宫门外,半步也不愿再向外多走,转身便要进去。九宵看着澜翠袅娜的背影,心头像有什么东西晃了几晃,起了深深的涟漪,情不自禁道:``姑娘!''

澜翠转过身,带了点不耐烦的笑意,便道:``怎么了?''

九宵笑得嘴都咧开了,收不回来似的:``姑娘,我辛苦你带趟路,还不知道你的高姓芳名叫什么呢?''

澜翠听他说得不伦不类,越加好笑:``本姑娘就是个伺候娘娘的人,什么芳名不芳名的。''说罢甩了甩绢子,吩咐守门的太监道,``外头日头毒,还不关上大门,免得暑气进来!''

那小太监答应了一声:``是,澜翠姑娘。''

九宵站在白花花的太阳底下,浑然不觉得自己已经起了一层油汗,情不自禁地搓着手痴痴笑了。

夜来时分,宫门下了钥,除了偶尔走过的值夜侍卫,静得如在无人之地。夜色浓稠如汁,从天空肆意流淌向紫禁城的每一个角落。深蓝冥黑的天空中星河邀远,沉沉暗淡,夜色迷离得如一层薄薄的轻纱,好似随时能蒙住人的眼睛,叫人失去了方向。半弯皎洁明月里头隐约有些杂色,仿佛是广寒宫桂花古树的枝权错乱,或许嫦娥早已心生悔意,正怀抱玉兔在桂花树下述说着暗偷灵药的悔恨,遥遥无期的寂寥和永不能言说的相思。

云彻跟在春婵身后,不解问:``这么夜了,令嫔娘娘还有何要事吩咐?''

春婵提着灯笼,一脸愁容道:``娘娘本想问问皇上的起居饮食,但李玉公公的嘴有多紧,谁能问得出来。凌大人得皇上信任,娘娘只好求助于您,但请您不要拒绝。''春婵叹口气,担忧不已,``这些话奴婢本不该说,但娘娘一直深受嘉妃欺侮,实在不能不求自保。这个凌侍卫也该是知道的。''

凌云彻静默片刻:``我一个小小侍卫,又能帮得了什么呢?''他说着,扯了扯身上的小太监衣装,浑不舒服地道,``还偏得打扮成这样,鬼鬼祟祟的。''

春婵温静一笑,感激不尽的样子,倒叫人难以拒绝:``只要大人肯来,便是顾念旧识一场,是帮娘娘了。''她说罢,引着云彻继续向前,过了成和右门便看得到永寿富的正门了。

夜已有些深了,皇帝大概已经在平答应的永和宫中歇下。夏夜的署气渐渐被清凉之意逼散,加之甬道上被宫人们泼了井水生凉,在朦朦月色下似水银铺就一般,亮汪汪的。那一瞬,连云彻自己也有些模糊了。他是走在什么地方?这样熟悉的路,却像是要走到一个不能归来的地方去。他心事重重,听着春婵轻巧的脚步声落在镂花青石板上,每一步都引着他往永寿宫越走越近。他深吸一口气,抬头一望,只见宫墙红壁深深,一重重金色的兽脊披着生冷而圆润的棱角,冷冷映着月色,漠然地俯视向他。四下里寂然无声,守卫的侍卫固然不见,连宫门口垂着的灯火都暗暗的无精打采,格外得疏冷凄静。

他微微叹息,想起方才转角经过嘉贵妃的启祥宫,灯火通明,彩致辉煌,无数宫人簇拥,真真是个宠妃所居的地方,可一道之隔的永寿宫却如此冷清。大约嬿婉的日子,当真算不得很好吧。但,他极目远望,隐隐望得见翊坤宫那飞翘的檐角,心里稍稍生了一丝安慰,至少如懿,此刻已经安稳了许多。

他正凝神想着,春婵已经引了他入了庭院。偏殿与后殿当真是一点灯光也无,唯有嬿婉所居的正殿有几星灯火微明。春婵规规矩矩地立到一旁,并无进去的意思,恭谨道:``凌大人请进,娘娘已经在里头等候大人了。''

云彻微一踌躇:``这样似乎不妥吧,还请姑娘陪我进去。''

春婵微微一笑:``娘娘与大人是旧相识,必然有要紧的话商议,奴婢微贱,怎能在旁伺候?何况,里边自有伺候大人的人。''

云彻听得这句,才微微放心,举步入内。他才一进去,春婵已经在身后将殿门紧紧闭上。他颇为意外,再要转身也觉不妥,只得缓步入内。殿中只点了几盏烛火,又笼着莹白的缕纱灯罩,那灯火也是朦朦胧胧、暧昧昏黄的。他试探着唤了一声``令嫔娘娘'',却不曾听见有人回应,隐约中见西次间暖阁灯火更亮些,便又入内几步。

最末梢的暖阁内却是重重绡纱帷坠,是绕指柔的粉红色,温柔得像是女子未经涂染的唇。穿过一扇桃形新漆圆门,数层薄罗纱帐被帐钩挽于两侧,中间垂着淡紫水晶珠帘,微微折射出迷离朦胧的光晕。熏炉内若有若无的香味清幽无比,他虽然常常出入养心殿,闻惯了各种香料,但也说不出那是什么香气,只觉得柔媚入骨,中人欲醉。

\hypertarget{ux7b2cux4e8cux5341ux4e5dux7ae0-ux79c1ux60c5ux4e0a}{%
\chapter{第二十九章
私情(上)}\label{ux7b2cux4e8cux5341ux4e5dux7ae0-ux79c1ux60c5ux4e0a}}

阁中大约是贡着数瓮新起出来的冰雕,将暑意都隔在了外头,只余下一个清凉自在天地来。

云彻见四下无人,心下不安,只得拱手道:``或许令嫔娘娘一时远离,微臣不便久留,先行告退。''

他正要转身离开,只觉得肩上微微一重,似有翩翩的蝶停驻在了肩头。他侧过脸,之间绡纱之后,伸出一只皓白的柔荑来,虽然上方掩盖着明紫绡纱方绢,亦可看清那柔软无骨宛若削葱的纤细手指。隔着一挂水晶珠帘,有透彻如水的女子声音传来,仿佛也沾染了水晶的清透:``云彻哥哥,你便等不得我一等了么?''

云彻脑中一蒙,只得镇声道:``微臣凌云彻,拜见令嫔娘娘。''

嬿婉的笑声轻柔得如攀上枝头的紫藤软蔓:``云彻哥哥,你也太不诚心了。连头也不转过来,怎么拜见呢?''她手指微微一动,像水蛇般绕上了他裸露在外的脖子。云彻不自觉地打了一个激灵,只觉得攀附上自己的那双手指尖冷若寒冰,却柔软如绵,所经之处,便似点燃了小小的火苗,一点一点舔着他的皮肤,让他无端地生出一种原始的渴望来。

嬿婉的气息温柔地拂在他的耳边,轻轻道:``云彻哥哥,你怎么不回头看看我?''那样蛊惑的声音,让他渴望又心生畏惧。记忆中的嬿婉并没有这样柔媚至死的声音,他真的很怕一回头,见到的不是嬿婉,而是一张传说中的诡魅的狐狸面孔。可他不能不转过头去,嬿婉的手已经抚摸到了他的嘴唇,温柔的逡巡着。他不由自主的转过身体,唤道:``令嫔娘娘\ldots\ldots{}''

他的目光在一瞬间看到了嬿婉洁白而裸露的肩头和手臂,像是新剥出的荔枝肉,微微透明,白而冻,却散发着温暖的热气。她身体的其他部分都被一块薄得近乎透明的红绡紧紧围住,勾勒出美好而诱人的曲线。可她的身体,怎美得过她刺客微漾的星眸、丰润的红唇和那欲嗔未嗔的笑容。

他,没有见过这样的嬿婉。从来没有。

一定,是哪里除了错。他狠狠咬了下自己的舌尖。痛,咬得用力,连血液都沁了出来。嬿婉只是一笑,手臂蜿蜒上他的脖子,欲去吻他唇边新沁出的鲜红的血。

疼痛在一瞬间清醒了他的头脑。一定是哪里不对!一定是!

他趁着那一分清醒霍然推开她,挣扎着道:``令嫔娘娘请自重。''

``令嫔娘娘?''嬿婉轻嗤,在他耳边吐气如兰,``哪个娘娘会这样来见你。''她伸出染成粉红色的指尖在云彻掌心悄然回旋,有意无意的挠着,所到之处,便引起肌肤的一阵麻栗,她的身体越发靠近他,``我是你的嬿婉妹妹。''

``嬿婉?''他艰难地抗拒,``嬿婉不会如此。''

她的手指在他的胸口画着圈,透着薄薄的衣衫,那种酥痒是会蔓延的。嬿婉显然是新沐浴过,梨花淡妆,兰麝逸香,浑身都散发着新浴后温热的气息,在这清凉的小世界里格外酥软而蓬勃。嬿婉的身体贴上了他的身体,哪怕隔着衣衫,他也能感受到那玲珑有致的身段,是如何成了一团野火,让他无法克制从喉间浸逸而出一缕近乎渴望的呻吟。嬿婉轻声道:``我如果嫁给你,我们夜夜都会如此。''她轻吻他的耳垂,``云彻哥哥,我是这样思念你,你感受到了么?''

云彻挣扎着挪动身体,他的挪动显然无力而迟缓,弥漫的想起成了一张无形的网,将他控得无处可逃。他的脑海里如同浮絮般轻绵而无处着力,声音亦如此微弱:``不,不\ldots\ldots{}''

``为何要说不?''嬿婉俯身在他之上,几欲吻住他的唇,``难道除我之外,你心里喜欢上了别人?''

嬿婉似笑非笑的看着他,是如此笃定而漫不经心,她认定了的,他心里只有她,再无旁人。可于云彻,却恍然有惊雷贯顶,他没有答案,可那一瞬间,是有一张颇为肖似却神情迥异的面孔出现在了眼前。

是如懿!

居然是如懿!

大约是殿阁中太清凉,大约是气氛太暧昧,大约是他昏了头脑,在这一刻,他想到的居然是如懿。

仿佛有冰水湃入了头脑的缝隙,彻骨寒凉。他霍然站起身来,推开柔情似水的嬿婉:``你对我做了什么?''

嬿婉微微诧异,面颊酲红,唇若施朱,呼吸犹含浅浅柔香:``我能对你做什么?云彻哥哥,这不是你一直以来所想的么,我只如你所愿罢了。''

``不!那是你的意愿,不是我的。''他盯着嬿婉,目光清冽如数九寒冰,``为什么这样?''

``为什么?''嬿婉苦笑,``若不是因为没有孩子,我怎么会落到如此田地?云彻哥哥,我过得并不好。我只是不想再受人欺凌,为什么这样难?''有清泪从她长而密的睫毛间滑落,``我只想要一个孩子,让我后半生有个依靠而已。云彻哥哥,我只希望那个孩子的父亲是你。''

``是我?''云彻愕然而恼怒,``你用这样的方式选择是我?''他别过头,见案几上有一壶茶水,立刻举起倒入口干舌燥的喉舌,以此唤来更多的理智和清明,``你选择的是皇上,不是我!''

``那有什么要紧?''嬿婉红了双眼,``只要你是我孩子的父亲。''

是恼怒还是羞辱,她用这种方式来贬低自己,贬低她。他终于道:``你有皇上!''

嬿婉有些急切:``皇上与我,或许没有子嗣的缘分!而且皇上老了,并不能让我顺利有孕。我已经喝了那么多坐胎药,我\ldots\ldots 我只想要个孩子!你比皇上年轻,强壮,你\ldots\ldots{}''

云彻摇头:``不!如果你有了孩子,会怎么对我?借种生子之后,我便会被你杀人灭口,不留任何痕迹。你要除去我,太简单了。''

嬿婉惊诧地看着他,柔弱而无助:``云彻哥哥,我们多年的情分,你居然这样想我?''

``断得一干二净,不留任何余地,是你一贯的处世之道。''云彻的眼里有一点因愤恨和失望而生的泪光,转瞬干涸,``你找我,不过是我有可利用的地方而已。''他奋力支撑起身体,``令嫔娘娘,但愿你能留住一点我对您最后的善意想象。''他起身,跌跌撞撞离去。

嬿婉望着他离去的背影,颓然坐倒在榻上,眼角的泪光渐渐锋利,成了割人心脉的利刃。春蝉惊惶地闯入:``小主,凌大人怎么走了?他会不会说出去?''

嬿婉疲惫地摇头:``本宫不知!''

春蝉慌不择言:``可借种的事\ldots\ldots 按着咱们原定的想法,只要日后成功,一定得出去凌大人灭口。可现在\ldots\ldots{}''

嬿婉的面色苍白似初春的雪,是冰冷僵死般的残喘,在松弛的尽头散发着无力的七夕:``他走了也好,至少以后不必本宫来杀他了。''

春蝉的手按在了嬿婉的肩头,像是扶持,亦是强逼自己的安慰。可她还是害怕,从骨子里冒出的寒气让她手指发颤。她自言自语道:``他不会,也不敢。对不对?小主。奴婢看得出来,他是在乎您的,他对您有情有义。其实他是个挺好的人,真的!''

嬿婉支着明亮的额头,低眉避过春蝉惊惧的面容,引袖掩去于这短短一瞬间掉下来的清亮泪珠:``他当然是个好人,可以依托终身的人。可春蝉,本宫和你不一样。本宫也曾经是好人家的格格,却入宫做了奴才,还是不甚体面的奴才。本宫再不想吃那些苦了,一辈子都不想再被人欺负。本宫没有办法,所以只能找这个好人,也只能去欺负一个过得不如本宫的好人!''

春蝉甚少见她这般感伤而无助,她吓得一个激灵,全然清醒过来,跪下道:``小主,您别这么说\ldots\ldots 你是有福气的\ldots\ldots{}''

``春蝉,你放心,只要你好好跟着本宫,本宫不会让你只是一个卑贱的奴才。一定不会!''嬿婉静静说完,面上的颓废哀色旋即逝去,她咬着唇狠狠道,``没别人可以帮本宫,那就算了!''她死死按住自己的小腹,含着暴戾的口吻,森冷道,``既然我得不到一个孩子来固宠,那么\ldots\ldots{}''她没有再说下去,只是恢复了如常的冷静,看了春蝉一眼,``那炉香原来那么没用,去倒掉吧。''

云彻走了好一段路,寻到庑房里换回自己的衣裳,又一气灌了许多茶水,才渐渐恢复清明的神志。同住在庑房的侍卫们都睡熟了,浊重的呼吸混着闷热的空气叫人生出无线腻烦。他透着气,慢慢摸着墙根走到外头。甬道里半温半凉的空气让他心生安全,他靠在墙边,由着汗水慢慢浸透了衣裳,缓缓地喘着气,以此来抵御方才暧昧而不堪的记忆。印象中嬿婉美好纯然的脸庞全然破碎,成了无数飞散的雪白碎片,取而代之的是她充满情欲的媚好的眼。他低下头,为此伤感而痛心不已。片刻,他听到响动,抬起头,却见如懿携着惢心并几个宫女从不远处走来。

他心头蓦然一松,起身守候在旁:``皇贵妃娘娘万福金安。''

如懿颇为诧异:``这个时辰,凌大人怎么在此?''

云彻有点窘迫,很快道:``侍卫巡夜,微臣怕她们惫懒,特意过来查看。夜深,娘娘怎么还在外行走?''

惢心笑道:``宫里请了喇嘛大法师在雨花阁诵经,小主刚去雨花阁祈福归来。''

云彻道:``娘娘虔诚,一定会心想事成。''

如懿示意众人退后几步,低声向他道:``凌大人身体不好?脸色怎么这样难看?''

云彻无奈苦笑:``娘娘,微臣只是见到自己不愿见到的改变。想不通旧时的人,旧时的事,怎会面目全非?''

如懿的笑容温暖而沉着:``是人都会变。比起十四岁初入潜邸时的我,如今的我可以说是面目全非。所以不要执念于你过去的所见所闻,能接受的变化便接受,不能接受便由他去。你所能控制的,只有你自己。''她说罢,扶过惢心的手,带着温静神色,缓步离开。

云彻一瞬间的恍惚,这个与嬿婉眉间有着积分相似的女子,这个正当韶华盛放的女子,有着不同于任何女子的沉稳笃定。或许这是她在深宫中失去的,亦是收获的。他望着她,保持着静默的姿态,目送她离开,却清晰地记得,自己在迷糊的一刻,清醒地想起她的脸。

那,才是对于他自己,最撼动心扉的变化。

皇帝的万寿节是八月十三。自过了七月十五中元节,来自密宗的大法师安吉波桑便领着一众弟子入紫禁城,暂住在雨花阁中修行祝祷,为皇室祈福,直到八月十五中秋节。

这是宫中难得的盛事。因为宝华殿主供释迦牟尼佛,而雨花阁则是藏传佛教的佛堂。藏传佛教盛行于川藏,又与和清朝皇室紧密连接的蒙古息息相关,所以宫中笃信藏传佛教之人众多。上至太后,下至宫人,无一不虔诚膜拜。

如懿统摄六宫,对此等大事自然不敢怠慢。一来孝贤皇后去世后,皇帝郁郁寡欢,少于嫔妃亲近。二则自乾隆十二年四川藏族大金川安抚司土司官莎罗奔公开叛乱,朝廷派兵镇压失败,皇帝一怒之下改用岳忠琪分两路进攻大金川,莎罗奔溃败乞降,顶佛经立誓不再叛乱,宫中祈福,也可求国家祥和。三则金玉妍所生的九阿哥身体孱弱。大约是怀着身孕时为孝贤皇后的丧礼操持劳碌,有许多不可避免的礼仪劳顿,所以九阿哥出生快一个月了,总是多病多痛,连哭声也比同龄的孩子微弱许多。整个人瘦瘦小小的,便似一只养不大的老鼠,一点响动都会惊起他不安的哭声。玉妍格外心疼幼子,日日召了太医贴身守护。她原本一心信奉李朝的檀君教,除了必需的例行公事,从不进供奉牟尼佛的宝华殿与供奉藏传教密宗的雨花阁,也不过问宫中一切从佛。如今她爱子心切,也不太顾得,除了每日早晨必将前一日亲手抄写的经文送来请大师诵读,也常常派贴身的是女宫婢前来跟着法师们诵经描画经幡。只是自己绝不进雨花阁敬香礼佛的。

如此,法师们便在雨花阁住了下来,每日日晨昏敬香,虔诚不已。

这一日如懿从雨花阁回来,手了安吉波桑大师所赠的一把藏香并一个青铜香炉,便吩咐菱枝点了起来。如懿问了三宝几句皇帝万寿节的准备,便也让他退下了。

菱枝点了一把放在窗台下,连连道:``好冲的气味,可比沉水香冲多了。''

如懿笑道:``藏香不仅是对上师三宝的供养,并且积聚无量无边的福智二资,对身体、气脉及心神多有裨益。也是安吉波桑大师有心,才赠了本宫一小把。''她转过头见殿中只有菱枝带着小宫女忙碌,便问:``惢心呢?方才没跟着本官去雨花阁,此刻人也不在宫里。''

菱枝抿嘴一笑:``惢心姐姐还能去哪里,估摸着到时辰该请平安脉了,亲自去请江太医了。''

如懿会心一笑,低头轻嗅那藏香,道:``这香味虽有些冲,但后劲清凉醒神,等下留出一份送与太后。''

菱枝正答应着,如懿侧首望向窗外,见江与彬惢心并肩穿过庭院,有风轻柔地卷起她们的衣衫,将袍角卷在一起,江与彬亦从容含笑,体贴地弯下腰,为惢心拂好裙角。

如懿看着他们,仿佛看见昔年的皇帝与自己,如此两情相依,彼此无猜疑。

二人很快进来,如懿笑着道:``再不许你们成婚,便真是我的不是了。''

惢心有些不好意思,转身站在江与彬身后去了。江与彬垂衣拱手,一揖到底:``多谢皇贵妃垂爱。''

如懿由着江与彬请过了平安脉,江与彬道:``娘娘一切安好。''

如懿抚了抚手腕,淡淡笑道:``安好便罢,能不能有子息,也在天意,非我一人主宰。''

江与彬道:``听说皇贵妃近日总在雨花阁祈福,与大法师颇为相熟,娘娘积福积德,一定会有福报的。''

如懿笑道:``说来也怪,我与波桑大师素未谋面,却一见如故。法师年未至四十,但佛学精通,总让人有清风佛面,豁然开朗之感。''

江与彬垂眸笑道:``密宗有通灵一说,想来大法师便是如此。''

如懿略略思忖,抚着塌边一把紫玉多宝如意,慢慢道:``其实你与惢心两情相悦已久,我很该早些把惢心指婚给你。一则是我的私心,身边除了惢心并没有另外可以信任的人。二则宫中多事之秋,也离不开惢心,便一直耽误了你们。本宫已经想好,今年还在孝贤皇后的丧期,明年三月过后,和敬公主出嫁,便把惢心指婚于你。希望你能好好待她。''

江与彬深色激动,跪下道:``有皇贵妃这句话,微臣便是再等上十年也是心甘情愿的。''

如懿笑道:``你等得住四年,惢心可等不住。本宫都已经在想,若你们生下孩子,一定要常常带来,在本宫身边做个半个义子,便算也享了天伦之乐。''

惢心含笑带泪,对着江与彬认真道:``我且告诉你,便是小主赐婚了,每日宫门下钥前,我都会来侍奉小主,天黑才回家。你可不许管我。''

如懿笑得撑不住:``瞧瞧,这还没有嫁人呢,便已经这样霸道了。叫人还以为翊坤宫出去的,都被本宫惯的这样坏性子呢。''

江与彬的笑意纵容而宠溺:``惢心说什么,微臣都听她的。''

如懿微微含笑,仿佛能从江与彬的宠溺与爱意里探知几分往日的时光。但,那终究是往日了。

是夜,如懿便如往常一般在暖各种沐浴梳洗。诵经祈福之后,便为皇帝万寿节的生辰之礼忙碌了很久。孝贤皇后新丧,皇帝的万寿节既不可过于热闹,也不能失了体面,更是要让嫔妃们崭露头角,安慰皇帝。如懿新摄六宫事,不能不格外用心操持。

如懿沐浴完毕,惢心伺候着用大幅丝绸为她包裹全身吸净水分,来保持身体的光滑柔嫩。孝贤皇后在时最爱惜物力,宫中除了启祥宫是特许,一例不许用丝绸沐浴裹体。然而孝贤皇后才过世,自金玉妍起便是大肆索用丝绸,那一阵绿筠与她亲切,便也不太过问,更喜与玉妍讨教容颜常驻的妙方,也开始享受起来。皇帝素来是喜好奢华,如懿有意松一松孝贤皇后在世时六宫节俭之状,便也默许了。由此宫中沐浴后便大量使用丝绸,再不吝惜。

银朱红纱帷垂地无声,如懿用一把水晶钗子挽起半松的云鬓,身上披着一身退红绛绡薄罗衫子,身影如琼枝玉树,掩映其下。身侧的碧水色琉璃缸里满蕴清水,大蓬的粉红雪白亮色晚莲开得如醉如仙。远远有菱歌声和着夜露清亮传来,想是嬿婉宫中,正陪着皇帝取乐。听闻嬿婉新出了主意,命人采来晚开的红莲,又于夜间捕来流萤点点,散于殿阁中,湘簟月华浮,萤傍藕花流,自是合了皇帝一贯雅好风流的心意。

惢心听着那银丝般萦萦不断的曲声,只是笑吟吟向如懿絮絮:``小主今夜披于身上的衫子真好看,红而不娇,像是内务府新制的颜色。''

如懿知她不愿自己听着旁人宫中承宠欢笑,便也有一句没一句地道:``半月前皇上读王建的《题所赁宅牡丹花》,其中一句便是`粉光深紫腻,肉色退红娇',只觉那`退红'二字是极好的,只不知如今能不能制出来,便叫内务府一试。内务府绞尽脑汁只作出这一匹,颜色浓淡相宜,娇而不妖,果然是好的。''

那幽幽的一抹退红,是明婉娇嫩的华光潋滟,有晚来微凉的潮湿,是开到了辉煌极处的花朵,将退未退的一点红,娇媚而安静地开着。

惢心撇嘴笑道:``如今小主新摄六宫事,只弄个退红颜色也罢,便是天水碧那样难的料子,内务府怕也制的欢喜呢。生怕讨好不了小主。''

如懿斜睨她一眼,扑哧一笑,伸手戳了戳她笑得翘起的唇:``你这小妮子,越发爱胡说了。''

如懿任由惢心用轻绵的小扑子将敷身的香粉扑上裸露的肌肤。敷粉本事嫔妃宫女每日睡前必做的功课,日日用大量珍珠粉敷遍身体,来保持肌肤的柔软白滑,如一块上好的白玉,细腻通透。

如懿轻轻一嗅,道:``这敷体的香粉可换过了么?记得孝贤皇后在时,这些东西都是从简,不过是拿应季的茉莉、素馨与金银花花瓣拧的花汁掺在珍珠粉里,如今怎么好像换了气味。''

惢心一壁扑粉一壁道:``小主喜欢白色香花,所以多用茉莉、素馨、栀子花之类,其实若是肌肤好颜色,用玫瑰与桃花沐浴是最好不过的。不过奴婢这些日子去内务府领这些香粉,才发觉已经不太用这些旧东西了。说是皇上偶尔闻(\ldots\ldots 缺)小主用的香粉,是用上好的英粉和着益母草灰用牛乳调制的,又用茯苓、香白芷、杏仁、马珂。白梅肉和云母拿玉锤研磨细了,再兑上珍珠粉用的。这还不是只给咱们宫里的,但凡嫔位以上,都用这个。''

如懿出身名门,见惯了这些豪奢手段,然后听的惢心一一说来,也不觉暗暗咋舌:``孝贤皇后在时最节俭不过,连嫔妃们的衣衫首饰都有定例。如今人方走,大家便物极必反,穷奢极欲起来,也没个管束。只那马珂一例,便是深海里极不易得的海贝,几与珊瑚同价。''

惢心听得连连吐了舌头道:``听闻嘉贵妃还未出月子,便已经每日用桃花拧了汁子擦拭身体,还催命太医院炮制让身形回复少女柔嫩的香膏,用的什么苏合香、白胶香、冰片、珊瑚、白檀,那些稀奇古怪的名字,奴婢记也记不住,珍珠更是非南珠不用。只是皇帝宠她又生了阿哥,没有不允的。''

如懿听的连连蹙眉,片刻方轻笑:``世人总是爱做梦,希望重回少女体态,只是若失了少女身段,还配上一副少女心肠,那便是真真无知了。''

惢心道:``她哪里是无知,是太过自信。以为纯贵妃抱病,又失了大阿哥和三阿哥两个靠山。她便仗着自己生了三个皇子,又新封了贵妃协理六宫,便自以为的得了意了。''

细白的珍珠粉敷及了身体的每一个角落,让本就雪白的肌理泛着更不真实的白色。如懿怅然道:``嘉贵妃自然得意。其实能像她一般急欲保养也是好的,哪里像我,或许没有生养过的人,终究不显老些。''

惢心知如懿一生最痛,便是不能如一个寻常女人般怀孕生子,她正要出言安慰,忽然听的外头砰一声响,很快有脚步声杂沓纷繁,渐渐有呼号兵器之声,骤然大惊,喝道:``什么事?竟敢惊动小主!''

外头是三宝的声音,惊惶呼喝道:``有刺客!有刺客!保护小主要紧!''

这一惊非同小可。如懿本是半裸露着箭头,惢心旋即拿一件素白寝衣将她密密裹住。两人正自不安,恍惚听到外头安静了些许,却是三宝执灯挑帘进来,禀报道:``让小主受惊了。''

如懿因未曾亲见刺客,倒也渐渐镇定下来:``怎么回事?''

三宝道:``方才奴才烧了热水,打算放在暖阁外供娘娘所用。谁知奴才才过院子,却见有一个红袍刺客翻墙进来,奴才吓得摔了脸盆,那人听见动静立刻翻墙走了。谁知便惊动了外头巡守的侍卫,进来查看。''

如懿惊怒交加:``翊坤宫竟敢有刺客闯入,实在是笑话!那结果如何?''

三宝惴惴道:``刺客跑得快,已经不见了。''

``无用!''如懿厉声呵斥,心中忽而有不安的涟漪翻腾而起,``你是说你一发现刺客的行踪喊起来,外头巡守经过的侍卫就听见了?''

三宝答了``是'',如懿愈加疑惑:``从来巡守的侍卫经过都有班次,并不该在这个时刻,怎来的这样快?''

三宝寻思着道:``或许是因为小主晋封了皇贵妃,她们格外殷勤些也是有的。''

如懿心底大为不耐烦,道:``既然殷勤,就不该有刺客闯入。现下又太过殷勤了。''她想了想,``去将今夜之事禀告皇上,再加派宫中人口,彻底搜寻翊坤宫及东西各宫,以免刺客逃窜,惊扰宫中。最要紧的是要护驾。''

三宝答应着赶紧去了,如此喧闹一夜,再查不到刺客踪迹,才安静了下来。

次日一早,皇帝便亲自来探视如懿,安慰她受惊之苦,又大大申饬了宫中守卫,但见合宫无事,便也罢了。

到了午后时分,如懿正在盘查翊坤宫的门禁,却听外头李玉进来,打了个千儿道:``皇贵妃娘娘万福金安。''

如懿见了他便有些诧异:``这个时候皇上应当在午睡,你怎么过来了?''

李玉道:``皇上在启祥宫歇的午觉,也只睡了一会儿,嘉贵妃陪着皇上说了会子话儿。皇上说请娘娘立刻过去呢。至于什么事儿,奴才也不清楚,大约是皇上还在担心娘娘昨夜受惊的事吧。''

如懿便道:``那你等等,本宫更衣便去。''

\hypertarget{ux7b2cux4e09ux5341ux7ae0-ux79c1ux60c5ux4e0b}{%
\chapter{第三十章
私情(下)}\label{ux7b2cux4e09ux5341ux7ae0-ux79c1ux60c5ux4e0b}}

虽然已是八月十一,天气渐渐地凉了下来,但午后总是格外闷热些,如懿坐在轿辇上一路过来,也不免香汗细细,生了一层黏腻。待走到殿中,便觉清凉了不少。

玉妍出身李朝,她的启祥宫也装饰得格外新奇,多以纯白为底,描金绘彩,屏风上所绣的也是李朝一带的山川景色,秀美壮丽。因是在自己宫中,玉妍也是偏于李朝的打扮,李朝女子崇尚白色,所以她穿着浅浅乳白色的绣石榴孔雀平金团寿夏衣,耳上坠着华丽及肩的翠玉琉璃金累丝流苏耳饰,头发梳成低低的平髻,以榴红丝带束起,再用拇指粗的赤金双头并蒂的丹珠修翅长钗簪住,顺滑垂落于脑后,两边鬓发上配着金累丝团福镶红绿宝石和田白玉片,微一侧首,上头的镂花串珠金丝便盈盈颤动,浮漾珠芒璀璨。

相形之下,如懿不过是一袭水天一色海蓝宝蹙银线繁绣长衣,下着水月色云天水意留仙群。云鬓上不过是些寻常的细碎珠花,只在侧首簪了一双赤金丝并蒂海棠花步摇,实在是比不上玉妍的细心雕琢,仪态万千了。

因着畏热,皇帝不过穿着家常的云蓝色银线团福如意纱袍,斜靠在暖阁的榻上。底下的紫檀小几上搁着一碗喝了一半的参鸡汤并一把伽倻琴。想来如懿来钱,皇帝便是听着玉妍弹唱伽倻琴,品着参鸡汤,惬意自在度过午后炎炎。

如懿福身向皇帝问安,玉妍亦起身向她肃了一肃。如懿便客客气气道:``嘉贵妃昨日才出月子,还是不要劳动的好。''

皇帝嘱咐了如懿坐下,脸上犹自挂着淡淡的笑容:``皇贵妃,听说你最近常去雨花阁祈福?''

如懿欠身倒:``是。安吉波桑大师难得入宫一回,臣妾想要诚心祝祷,祈求康宁。''

玉妍伴在皇帝身边,手里轻摇着一叶半透明的玉兰团扇,闲闲道:``臣妾希望九阿哥平安长大,所以每日晨起都会去雨花阁将前一日所抄写的经文请大师诵读,但皇上知道臣妾信奉檀君教,所以未曾亲自入内。说来皇贵妃比臣妾心意更加诚挚,所以晨昏必去,十分虔诚呢。''她莞尔一笑,瞟了如懿一眼,``其实呢,也不是臣妾对九阿哥用心不够。只是臣妾身为嫔妃,想着入夜后不变,大师虽然出家修行,但终究是男子啊。''

皇帝的口吻淡淡的,听不出赞许还是否定:``大师到底是大师,你也别多心。''

玉妍眼眸轻扬,娇声笑道:``臣妾哪里敢多心,不过是随口一说罢了。说来也到底是皇贵妃合波桑大师的眼缘,藏香也好,手串也好,什么都是给皇贵妃的。''

如懿听的她语气不善,便道:``藏香倒是真的,昨日波桑大师刚送了臣妾一把,臣妾闻着气味不错,想留给太后一些。''她想着玉妍笑,``嘉贵妃刚出月子,消息便这般灵通了。倒像是跟着我身后盯着呢。至于手串,我倒是不知了,还请叫贵妃细细分说才好。''

(缺)

得的,认了便也认了。''她击掌两下,换上贴身侍女贞淑。贞淑见了如懿,恭恭敬敬行了一礼,递上一串七宝手串奉于皇帝手中,道:``皇上,昨日奴婢奉小主之命前往雨花阁替九阿哥送经文祝祷,但见安吉波桑大师与皇贵妃举止亲密,窃窃私语。随后波桑大师将一盒藏香、一个青铜香炉交到皇贵妃手中,并将这手串亲自待在皇贵妃手腕上,以作定情之物。''

如懿闻言,遽然变色道:``好个敢擅自窥探主上的奴才,既然亲眼见大师替本宫戴上手串,并未听的言语,如何知道是定情之物?难不成往日宫中发饰赐福,赠予佛珠佩戴,都成了私相授受么?再者,既然是定情之物,为何不在本宫手腕上,却在你受伤?''

如懿的气质如秋水深潭,若非亲近之人,望之便生清冷素寒,又兼之此刻连声诘问,虽然出语从容,但语中凛冽之气,不觉让贞淑颤颤生畏。

玉妍媚眼如丝,轻妩含笑:``皇贵妃何必这般咄咄逼人,贞淑不过是说出她所见而已。至于手串嘛,是臣妾连着这个东西一起拿到的。''她说罢,从袖中取出一枚精巧的玩意儿。

玉妍掌心里是一枚折叠精巧的方胜。方胜折的极精巧,折成萱草的图案,原是取``同心双合,彼此相通''之意。她将方胜递给皇帝过目,皇帝额上青筋微微跳突,闭上眼道:``朕已经看过了,你给皇贵妃自己看便是了。''

玉妍婉声应答,将方胜递到了如懿手中,笑吟吟道:``那手串是与这样东西一起在皇贵妃的翊坤宫外捡到的。宫中巡守的侍卫发觉后惶恐不已,不敢交给皇贵妃,便径自来交予我了。我哪里经过这样的事,也不知是什么东西,更不敢看一眼,立刻封了起来先请了皇上做主。皇贵妃先自己看一看吧。''

如懿抖开方胜,拆开来竟是张薄薄的洒金红梅笺,因她素日喜爱梅花,内务府送入翊坤宫的信笺也以此为多。她心下一凉,之间那洒金红梅笺中间裹着几枚用红丝线穿起的莲子,往下打了一个银丝攒红丝的同心结,却见笺上写着是:``置莲怀袖中,莲心彻底红。忆郎郎不至,仰首望飞鸿。曾虑多情损梵行,入山又恐别倾城。得君手串相赠,已知两下之情。此物凭惢心带与君为证,君若有心,今夜候君于翊坤宫冬暖阁,相知相识,如来与卿,愿君两全。''

那一个个乌黑的字迹避无可避地烙进如懿眼中。她闹钟轰然一震,前几句《西洲曲》原是女子对情郎的执着相思,又有莲子和同心结为证。后面的话,本是情僧六世达赖喇嘛仓央嘉措的诗句化用,若真是妃嫔与喇嘛私通,倒真是恰当之极。而真正让她五内俱寒、如浸冰水的,是那几行柔婉的字迹,分明是她自己的笔迹。

皇帝斜倚榻上,缓缓道:``如懿,你自幼家学,通晓满蒙汗三语,所学的书法师从卫夫人簪花小字,宛然若树,穆若清风。宫中嫔妃通宵诗书的不多,更无其他女子学过卫氏书法,要仿也无从仿起。若是慧贤皇贵妃还在,或许能临摹几许,但慧贤皇贵妃早已乘鹤而去,更无旁人了。''

他的声音甫落,玉妍已经接口:``臣妾一眼认出上面的是皇贵妃的笔记,皇上也认出了。至于这手串,百日里收进,黄昏时分送出,以作信物引刺客\ldots\ldots 哦,应该是奸夫\ldots\ldots{}''玉妍掩口,声音如同薄薄的铁片刺啦作响,``是我失言了,引奸夫入翊坤宫相聚,谁知被人无意中发现惊动,刺客慌不择路逃窜时,落在翊坤宫宫墙之外的。''

如懿将洒金红梅笺递到皇帝身前,勉力镇定下来道:``皇上若以为这些字是臣妾写的,那么臣妾也无可辩驳。因为臣妾一见之下,也会以为这些是出自臣妾手笔。可臣妾的确没有写过这样的字,若有人仿照,却也极可能。''

玉妍横了如懿一眼:``若说仿照,除了自己亲手所写之外,谁能这般惟妙惟肖?也真是抬举了那个人,枉费心机来学皇贵妃的字迹。''

如懿如何肯去理会她,只望着皇帝恳切道:``皇上,请您相信臣妾,臣妾并未做过任何背弃皇上之事。''

皇帝别过脸,慢慢摸着袖口上密密匝匝的刺绣花纹,似是无限心事如洗米的花纹缭乱:``皇贵妃,刺客到来之时,你再做什么?''

如懿道:``臣妾正在敷粉预备安寝,有惢心为证。''

皇帝点点头,看着玉妍道:``玉妍,你去问过雨花阁,当时安吉波桑在做什么?''

玉妍微微得意:``臣妾问过,安吉波桑自称要静修,将自己闭锁在雨花阁二楼,不许僧人出入。而以安吉波桑的修为,要从二楼跃下,一点也不难。''

``这个朕知道。''皇帝鼻翼微张,呼吸略略粗重,``皇贵妃,你沐浴敷粉之后便要安寝,刺客也是算准了时候来的。白日有贞淑见到安吉波桑赠你手串,晚上便出了刺客夜往翊坤宫之事。且有侍卫见到刺客穿着红袍,喇嘛的僧袍便是红色的,加之信笺上的诗句,也实在是太巧了。皇贵妃,你告诉朕,除了巧合之外,朕还能用什么对自己解释这件事?''

如懿听得皇帝的口吻虽然平淡,但语中凛然之意,却似薄薄的刀锋贴着皮肉刮过,生生地逼出一身冷汗涔涔。如懿望着皇帝,眼中的惊惧与惶然渐渐退去,只剩了一重又一重深深的失望:``皇上是不信臣妾了么?既然是臣妾私通僧侣,那么为何没有叮嘱宫人,发觉刺客喊起来的,竟是臣妾宫中的掌势太监三宝?''

玉妍在旁嗤笑道:``偷情之事,如何能说得人人皆知?自然是十分隐秘的。若有无知人喊了起来,也是有的。自从孝贤皇后仙逝,皇上少来六宫走动,皇贵妃便这般热情如火,耐不住寂寞了么!''

皇帝盯着那张信笺,严重直喷出火来:``朕什么都不信,只信铁证如山。''

玉妍道:``皇上,既然信笺上涉及皇贵妃的贴身侍婢惢心,不如先把惢心带去慎刑司审问,以求明白。''

如懿脸色大变,急道:``慎刑司素以刑罚著称,怎能带惢心去那样的地方?''

玉妍笑波流转,望了如懿一眼:``快到皇上的万寿节了,原以为皇贵妃出入雨花阁是为了皇上的万寿节祝祷,却不晓得祷出这桩奇闻来。皇上这个万寿节收了皇贵妃这么份贺礼,真是堵心了啊!''

皇帝冷了半晌,目光中并无半丝温情,缓缓吐出一字:``查!''

如懿不知道自己是怎样走出启祥宫的。外头暑气茫茫,流泻在紫禁城的碧瓦金顶之上,蒸腾起灼热的气息,那暑期仿佛一张黏腻的透明的蛛网,死死覆在自己身上,细密密难以动弹。她本在殿内待了许久,只觉得双膝酸软,手足发凉,满心满肺都是厌恶烦恼之意,一想到惢心,更是难过忧惧,一时发作了出来。她兀自难受,陡然被热气一扑,只觉得胸口烦恶不已,立时便要呕吐出来。

凌云彻本守在廊下,一见如懿如此不适,脸色煞白,人也摇摇欲坠,哪里还顾得上规矩,立时上前扶住她的手臂,急切道:``皇贵妃怎么了?''

如懿只觉得浑身发软,金灿灿的日光照得眼前一片晕眩,唯有手臂处,被一股温热的力量牢牢支撑住。她勉强镇定心神,感激地看他一眼,本能的想要抽出被他扶住的手臂,口中只道:``多谢。''

李玉跟着出来,一看这情形,吓得腿也软了,又不敢声张,赶紧上前替过凌云彻扶住了如懿,慌不迭道:``皇贵妃娘娘,您万安。''他低声关切道,``事情才出,怎么样还不知道呢。娘娘仔细自己身子要紧。''他悄悄瞥了身后一眼,``否则,有些人可更得以了。''

如懿摆摆手,强自撑住身子,按住胸口缓了气息道:``本宫知道。''

凌云彻见如懿这般神色,且殿内的争执大声时也不免有两三句落入二中,便知是出了大事。他本事一介侍卫,许多事做不得主,可此刻见如懿如风中坠叶,飘零不定,不知怎的便生出一股勇气,定定道:``无论何事,皇贵妃且先宽心。微臣若能略尽绵力,一定不辞辛苦。''他神色坚毅若山巅磐石,``皇贵妃安心便是。''

如懿本是失望,又受了委屈忧惧,听得凌云彻这样言语,虽知他人微言轻,但此时此刻自己这般狼狈,却能听到如此慰心之语,满腔抑郁也稍稍弥散,却也无言相对,只是深深望他一眼,从他沉静眼底攫取一点安定的力量。只是,她仍忍不住凄然想,为什么殿中那人,却不能对自己说出这般言语呢?

李玉看了凌云彻一眼,立刻道:``奴才也是一样。''他见如懿虚弱,便道,``娘娘脸色不好,奴才着人去请太医吧?''

李玉刚要唤人,如懿忙拦下,轻声道:``这个时候说本宫不适,谁都会以为本宫乔张做致。罢了,先送本宫回去吧。''

如懿回到宫中时,三宝还带人候在宫门外,只是再不能进殿伺候了。如懿一眼扫去,见人群里头已经不见了惢心,心中便凉了一半。她来不及说更多的话,只得匆匆到:``去找李玉,往慎刑司知会着点。''

三宝眼见着皇帝身边的进忠和进保陪着如懿进了内殿,忙点了点头。

如懿仍居翊坤宫,由四名慎刑司拨来的精奇嬷嬷陪伴,一律饮食起居,都由她们照顾,更不许翊坤宫原本的宫人入内伺候,形同软禁。这般山雨欲来风满楼的仓皇,人人自顾不暇,倒让她想起了当年入冷宫前的情形,也是这般惶惶不安。

如懿坐困愁城,又担心惢心在慎刑司的境况,越发睡不安稳。一早起来,一双眼睛底下便乌青一团,如同附着乌云一般。

到了十三日,皇帝万寿节,便是数月来抱病不出的绿筠亦盛装入席。而如懿自新封皇贵妃之后,理应由她主持万寿节大礼,此时对外也只称皇贵妃抱恙,不能出席盛宴。倒成全了玉妍,着一身水红色金银双花翟凤氅衣,抱着九阿哥陪在皇帝身侧,风光无限。

翊坤宫遇刺之事早已在宫内传的沸沸扬扬,嫔妃们私下里亦有议论。因为同样奇怪的是,早前嫔妃们虔诚礼佛的雨花阁助威法师,也背闭锁阁中。如此一来,更是流言如沸,让人不自觉地去揣测如懿的突遭冷落与雨花阁法师有关,渐渐地私通之说不胫而走,海兰急得几次要去翊坤宫见如懿,也是不得入内。皇帝那儿更是一面都见不到。连得宠的意欢问起皇贵妃一句,皇帝亦是只字不提。末了,看着万寿节上热热闹闹,皇帝伴着玉妍笑语如常,还是太后说了一句:``这便真真是烈火烹油,花团锦簇一场,全是为他人作嫁衣裳了。''

是夜,皇帝并未留宿任何人宫中,只想独自宿在养心殿。太后知道皇帝的心思,便道:``孝贤皇后刚去世,你的万寿节陪着谁都不安静,还是静静对着她,留一份念想吧。''

皇帝黯然道:``是。往年儿子的万寿节,都是孝贤皇后陪在身边,如今她去了,儿子还是希望她魂梦有知,能够如梦相见一回。''

太后正了正发髻上的翡翠西池献寿簪,和声道:``哀家知道皇帝你烦心什么。但雨花阁的法师到底是修行之人,许多事没有问出端倪之前,实在不宜大肆惊动,以免扰了礼佛尊敬之心。若真的有什么,那也只是其中一人修为不足,不干所有人的事。''

玉妍在旁笑道:``臣妾知道,所以雨花阁一切供应如旧,只是为防嫌隙,不许嫔妃宫人们再出入了。拘进慎刑司拷问的,也只有惢心及那夜巡守拾到证据的几个侍卫。''

太后微微不悦,面上的笑意淡了几分,只看着皇帝道:``如今皇帝身边的人越发能干了。哀家和皇帝说话,也敢自己插嘴了。''

玉妍当下便有些讪讪的,皇帝忙道:``嘉贵妃出身李朝,许多事不那么拘束,更率性些。''

太后淡淡``哦''了一声,眸色平淡无波:``原来到底是出身李朝,和咱们不大相同。到底是非我族类啊。''她不顾玉妍窘迫,招手向永瑢道,``纯贵妃,快带着永瑢上来给哀家瞧瞧。抱着怀里的婴儿总是一股奶味,不及永瑢虎头虎脑可爱。''

如此,玉妍也不敢再在太后跟前,借口说去看自己亲自安排的《流霞舞》,便推到一边去了。

待到玉妍再出现时,是在灿灿华灯下,她着一身雪白酒红色泼墨流丽的舞衣,作李朝女子的打扮,带着一众五彩衣裙的舞姬要配长鼓,风情万种的舞了上来。虽然才出月子不久,玉妍的身段已经纤秾合度了,恢复了生产前的柔软。

她堆起的云髻上只簪了金银二色流苏,发髻后系着深红色绣韵文的丝缎飘带。不细看,还误以为是月下流云的影子。风吹起她衣衫上的飘带,迤逦轻扬,宛如轻飘的雾霭环绕周身。流苏与珠络簌簌颤抖,她的舞姿柔缓,伴随着清脆的鼓声,就像这静好的月色流动到了身边。

宴乐正式到了热闹极处,繁鼓轻歌响在了耳畔,是玉妍打着长鼓跳着李朝风情的舞蹈,自然又赢的了雷动般的欢呼。仿佛她还是那一年李朝进贡的芳华少女,以一曲一朝歌曲,轻而易举的映入皇帝年轻的眼眸。

趁着歌舞的空档,海兰哄着永琪往皇帝身前说笑,皇帝亦只是如常。

\hypertarget{ux5982ux61ffux4f20-ux7b2cux56dbux518c}{%
\part{如懿传 第四册}\label{ux5982ux61ffux4f20-ux7b2cux56dbux518c}}

\hypertarget{ux7b2cux4e00ux7ae0-ux7409ux7483ux8106}{%
\chapter{第一章 琉璃脆}\label{ux7b2cux4e00ux7ae0-ux7409ux7483ux8106}}

次日黄昏,御驾前呼后拥,果然到了翊坤宫前。彼时斜阳如金,照在那宫苑重重叠叠的琉璃瓦上,流光如火如霞,刺眼夺目。如懿只觉得这几日望眼欲穿,心中早就焦虑如焚,只是一向自持身份,不肯在人前流露。如此,却又多了一重压抑。

皇帝到来时太监一下一下的击掌声遥遥递来,外面宫人早跪了一地。如懿看着皇帝穿着一袭家常的素金色团龙纱袍徐徐步入,面容越发清晰,如能和心中所思的样子密密重合,不知怎的,便生了一重酸涩之意。

从来,他便一直是自己想象中的模样,却并不曾如她期待一般,信重于她。

如懿这般模糊地想着,皇帝已然步入。如懿屈膝迎了下去:``皇上万福,臣妾多日不见,在此恭请圣安了。''那四名嬷嬷自是亦步亦趋地紧紧跟着,如看管着犯人一般,寸步不肯放松。皇帝知她从冷宫出来后再未受过这般苦楚,何况她又是心性极高的人,这几日被人时时刻刻盯着,怕也是难受到了极处。

这般一想,皇帝心底无端便柔软了几分,也不看旁人,只挥手道:``下去吧。''

那四名嬷嬷即刻退下,殿中越发静谧,只剩了皇帝与如懿二人相对。如懿泪眼盈盈,只是倔强着不肯落泪,一身烟青色无绣丝袍穿着,越发显得如一株凌霜的寒竹,细而硬脆。皇帝蓦然轻叹,只是两相无言。他一眼瞥去,见如懿手边的紫檀小几上搁着一本翻了一半的《菜根谭》,眼底闪过几丝诧异:``这个时候,你倒有心看这个?''

皇帝十指轻翻书页,如同翻着自己忧惶而支离的心情。如懿螓首微垂,低婉的轻叹如薄薄的风:``事有急之不白者,宽之或自明,毋躁急以速其忿(此句的意思是:当事情急切之际难以表白时,不妨先宽缓下来以听其自然,也许事情不久之后就会澄清。不要太急着为自己多方辩解,否则会使对方更加火上浇油)。臣妾看了半本《菜根谭》,唯有这一句颇合己意。''

皇帝凝视她片刻:``所以你不急着向朕申辩,肯安静禁足。''

这一句颇有温厚之意,勾起如懿蓄了满眼的泪。如懿强自撑着道:``痛哭流涕或是苦苦纠缠,不是臣妾的作风。''

皇帝沉默片刻,微微颔首:``所以朕如今才肯来听你说几句。说吧,你有什么可辩的?''

庭前一株株石榴花树,开得团团簇拥,烈烈如焚。她只凝睇着他,执意地问:``臣妾无甚可辩,只问一句,皇上是否肯相信臣妾?''

皇帝并不肯看她。有那么片刻的沉寂,如懿几乎能听见更漏的滴答声,每一声都如千丈碎冰坠落深渊,激起支离破碎的残响。真的,只有那么片刻,仿佛就在那一呼一吸之间,足以让她心底仅余的热情急转直下为荒烟衰草的颓冷。

终于,皇帝的声音渺渺响起:``不是朕肯与不肯,而是朕的眼睛和耳朵能不能让朕的心接受且相信。''

如懿听皇帝这样说,心里更揪紧了几分。``皇上这样问,是不是因为心嘴里什么都问不出来?''她上前一步跪下,急切道,``皇上,到底心受了多重的刑罚?''

皇帝的神情淡漠得如斜阳下一带脉脉的云烟:``方才还拿《菜根谭》的话劝诫自己毋躁急,一提心便急成这样。她不会死的。''

如懿听皇帝的口风,知道是问不出什么了,只是满腹委屈与凄恨纠缠成一团乱麻,逼得她急切不已:``既然罪在私通,皇上可问过安吉波桑大师了?''

皇帝的语气有棱角分明的弧度:``他只道那日自己独居一室,未曾离开,但是并无人可以为他证明。倒是有几个小喇嘛说起,见过你与他多次私下交谈,比寻常嫔妃更亲密。''

如懿沉吟片刻,朗然道:``出家人不打诳语,何况波桑大师是高僧。臣妾与大师交谈,也是视他为佛祖使者,无关男女。''

皇帝瞥她一眼,从袖中掏出那串七宝手串并那枚方胜,霍然扔在她身前的锦花红绒地毯上。那方胜原不过是薄薄的洒金笺,里头又裹着东西,一时受力不住,那莲子便破出来滚了出去。皇帝一时不觉,雪白的靴底踩在莲子之上,发出闷闷的碎裂声响,听得人心神凛凛。那七宝手串仿似一条五彩斑斓的死蛇逶迤在她跟前,吐着僵死的芯子。

皇帝叹道:``既然动了凡俗之念,便是乱了佛法,哪里还记得清规戒律?''他冷哼一声,``圣祖康熙爷在世时便出了仓央嘉措这样的情僧,妄悖佛家至理。如今这一脉俗念竟留在了这些人的血液中,从此只看得见女子,看不见佛祖了么?!''

如懿陡然闻得皇帝冷声,只觉脊背间有细密的汗珠沁出,似多足的细虫,毛刺刺爬过,所经之处,痛痒难耐。她到底还是耐不住性子:``那么皇上打算如何处置波桑大师?''

``朕一生的颜面岂可为蝼蚁之人损伤?一旦查证是真,朕会除去安吉波桑。''皇帝的口气轻描淡写,却含着无可比拟的厌憎,``要处死一个人,不必那么费事。有时跌一跤失足摔死,有时吃错了东西暴毙,有的是办法。''

``这样的办法,会落在安吉波桑身上,也会落在臣妾身上。不是么?''如懿无声地冷笑,``人人都是蝼蚁,无论是被尊崇一时的法师还是皇贵妃,不过是在他人指间辗转求存罢了。''

皇帝摇了摇头:``你不必急着拿自己与他相提并论。''

自那日玉妍将所谓的``证据''七宝手串交给皇帝之后,如懿便只匆匆看过一眼。然而,她亦明白,从那日的所谓``遇刺''开始,到巡守侍卫的经过,再到与她字迹一模一样的私通书信,便是一张精心织就的天罗地网,死死地兜住了她。没有破绽,根本毫无破绽可寻。她有些绝望地看着皇帝,一颗心难过得像被浸在滚水里反复地揉着搓着,勉强浮起,又被死死摁到底处。末了,只是虚弱得无力:``臣妾自问与皇上经历过许多事,皇上还不相信臣妾么?''

皇帝微微犹豫,别过脸道:``朕也很想相信你,可是有人证与物证,朕不能什么都不查就全然相信。且朕要的,不只是让朕信服,更要让所有人都信服,你是清白的。''

如懿盯着皇帝,强忍着心口重重紧皱的郁结,她清静淡漠的眸子依然如旧,仿佛是一泓不见底的深潭,不过轻轻漾了一圈涟漪:``是臣妾糊涂了。臣妾以为凭着多年的情分,相知相许,皇上会相信的。''

那一刻,如懿眸子似有秋水寒星般的冷冽之光,含幽凝怨,乌定定地直直向他心底钻去。那光似乎有某种灼人的力量,刺得他微微发痛。他有些动容,却转首不经意地避开她的目光:``朕不是薄情寡义的人,对你有情分,对后宫诸人都有情分。但是皇贵妃,所谓清白从不是用情分来断定的。''

如懿仰起脸,缓缓地浮上一层稀薄的笑意,恍若月初时分清冷暗淡的月光:``是啊,原来皇上对臣妾的情分,也是对旁人的情分。''

如懿颓然俯下身,死死地抓着那串七宝手串。除了心的抵死不认,她并没有多余的办法来证明自己。雪白而模糊的泪光里,她死死盯着手里的七宝手串,原来所谓情分与信任,是可以被这些身外之物轻易击碎的。她唯有自己,唯有海兰,唯有弥足珍贵的可以信赖的人。而那人,却不是他,不是自己枕畔相守多年之人。

这,算不算一个冷冽的讽刺?

皇帝站起身来:``你若没有话说,朕只能等着慎刑司用完刑罚,心还是说出你未曾私通的供词。受尽刑罚仍不改初衷,朕想,这样的供词,足以服众,足以平息留言。''

如懿眼中的泪冻在眼底,清冷道:``臣妾无奈,也为心痛惜。皇上若肯,请遍查各宫宫女嫔妃,最好是左右手都写字试试,看谁的字与臣妾的最相似。''

皇帝``嗯''一声:``好。朕自会去查。朕也想查知,朕的皇贵妃清白无污。''他向前几步,眼看着就要跨出门槛去了,如懿看着自己指尖的七宝手串,细细摩挲着,触目所及处蓦地惊动了心神,大声道:``皇上!皇上留步!''

皇帝停住脚步,却并不转身,只是冷然道:``话已至此,你还想说什么?''

如懿的一颗心悬在喉头,指间死死攥着那条七宝手串,颤声道:``这几日,皇上可曾细细看过这串手串?''

皇帝的声音里有伤心与厌倦,仿佛蒙蒙的潮湿的雾气,让人觉得窒闷:``这样的污秽东西,朕不想看。''

如懿膝行上前,遏制不住激动之色,扬声道:``皇上,这串手串不对!''

皇帝本欲抬起的右足霍然定住,转身向她道:``什么?''他的话里有热切的不确定的希冀。

如懿立刻将七宝手串递到皇帝跟前,切切道:``皇上,此串手串乃是金、银、琉璃、珊瑚、琥珀、砗磲和玛瑙制成。所谓七宝、因不同经书所记有异,可作七宝圣物的东西有十几种,但密宗七宝中定有西藏盛产的红玉髓而非玛瑙。红玉髓和玛瑙二者颜色与质地相近,看着都是通透嫣红,只是玛瑙更为名贵。大师是密宗高僧,断然不会混淆。''

皇帝的眉头渐渐蹙起,似叠峦山川,曲折难平。他举过那串手串上珠子对着天光细瞧了片刻,重重拍在紫檀螺钿小几上。

李玉一拍脑袋,叫道:``皇上,这手串上用的确实是玛瑙啊。安吉波桑大师是密宗法师,断不会以此相赠,所以说皇贵妃与大师私下往来,绝对是旁人诬害。''

如懿咬了咬唇,扬声利落道:``那么也不必盘查满宫的宫人嫔妃了。宫中嫔妃都出身满蒙汉,通晓佛教常识,断然不会弄错。能弄错的,一定是不懂的外来女子。''

李玉踌躇片刻,搓着手道:``皇上,外来女子怕是只有\ldots\ldots{}''

皇帝扬了扬手中的七宝手串,神色冷漠而锋利:``是了。若是信奉佛理之人,怎敢污蔑僧佛,妄造口孽。也唯有别有信奉之人了!李玉,你去告诉嘉贵妃宫里,每人用左右手各写下密宗七宝常用之物,谁的字像皇贵妃的字迹,立刻带来见朕。''

李玉``嗻''了一声:``皇上,如今小主们总在启祥宫走动,奴才这么雷厉风行去了,怕是不好。''

皇帝想了想:``内务府有一对新进的步摇,朕原要赏给愉妃的,你便送去给嘉贵妃吧。''

李玉答应着,立刻领命去了。

如懿终不肯抬头,只是望着自己素色鞋履上连绵不绝的茉莉花碎纹:``皇上暂肯一顾,许臣妾辩白几句,臣妾感激不尽。''

她俯首,郑重三拜,依足了臣下的规矩。皇帝默默看着她:``你原不必与朕这般生疏。''

原来,他还是明白的。

如懿伏在地上,尘灰弥漫于地的气味,微微有些呛人。她分明听得皇帝的足音出去了,眼底的泪忍了再忍,蒙眬里抬起头来,唯有凌云彻临去一顾,深深颔首。

蓦地,她心底便安宁了不少。

启祥宫宾客盈门,正莺莺燕燕挤了满殿。绿筠本是不大出门的人,也坐在下首,却不似众人一般笑容满面,只是愁绪满怀,含泪垂眸。

玉妍本与绿筠皆为贵妃,此刻却坐在上首,更兼她服色鲜明,一袭红衣如一团烈烈榴花一般,更衬得简衣薄鬓的绿筠似畏畏缩缩,困顿不堪。

玉妍笑吟吟道:``纯贵妃姐姐所请,不是我不愿,实在是无能为力啊。您知道的,宫中一向能说得上话的是皇贵妃。我虽有协理六宫之名,不过是虚名而已。''

绿筠赔笑道:``如今谁不知道皇贵妃自身难保,一切有赖嘉贵妃而已。''

玉妍笑着瞥了一眼绿筠,被蔻丹染得鲜红的指甲点在同样艳红的唇边:``纯贵妃姐姐说这样的话,我可不敢当。''

绿筠急切道:``我知道永璋不争气,读书比不上永珹,甚至连永琪也比不过。可他到底是皇上的儿子。皇上自从在孝贤皇后丧仪上呵斥永璋,也就更瞧不上他了,见面便是叱责。好好儿的孩子,见了皇上如老鼠见了猫似的。嘉贵妃,我知道永珹得皇上欢心,你能在皇上面前说上话,也请你顾及永璋,顾及我做额娘的一点儿心意,为永璋多说几句好话吧。''

玉妍微微正色:``纯贵妃姐姐,你我都是做额娘的人,自然之道孩子争气是得凭自己。我且有三位皇子,如何能顾得过来旁人的孩子呢?没的叫人笑话,说我手太长,去插足你们母子之事。''

绿筠语塞,眼看要落下泪来。玉妍偏还不肯放过,嚼了一枚香药乳梨道:``纯贵妃,说句实话,我只是嫔妃,不是中宫皇后。若有那一日,永璋成了我的庶子,我自然不能不开口。可今日,罢了吧。''

绿筠纵使再好脾气,也按耐不住性子,霍然站起身来。然而,身畔众人只围着玉妍说笑,无人将她放在眼里,一时进也不是,退也不是,无限孤清。

玉妍毫不在意绿筠,只顾着说笑,骤然见了李玉前来,正谈笑风生着,笑纹仍挂在唇边:``李公公怎的一阵风儿似的来了?''

李玉举起手中的青玉钿盒,笑眉笑眼地道:``皇上新得了一对步摇,让奴才给嘉贵妃娘娘送赏赐来。''

为首的庆贵人笑着奉承道:``皇上有好东西只疼嘉贵妃娘娘,今日也让我们开开眼。''

玫嫔冷笑道:``皇上对着嘉贵妃娘娘,有几日不赏的。只怕打开了启祥宫的库房,还不够庆贵人看的。皇上特地命李公公前来,怕还有旁的事要吩咐,咱们何必这么不开眼,非杵在这儿呢?''

庆贵人有些讪讪的。绿筠第一个坐不住,也不告辞,立时去了。当下众人亦识趣,便一一告退。

李玉趋奉上前,打开青玉钿盒,满面堆笑:``皇上新得的步摇,特赐予嘉贵妃娘娘。''

玉妍连声谢了恩,细看道:``这是红玉髓么,还是玛瑙?仿佛是红玉髓吧,二者倒是很想,若不细看,实难分辨。''

李玉道:``而这时相近,但嘉贵妃娘娘好眼力,确是红玉髓。''

玉妍当下便笑:``红玉髓不算名贵之物,皇上怎的想起来做步摇了?''

李玉道:``嘉贵妃娘娘忘了?孝贤皇后在时最不喜奢侈矜贵之物,向来朴素。皇上这几日思念孝贤皇后不已,所以拿红玉髓制了步摇,以表哀思,更表对孝贤皇后俭朴的尊崇。''他微微凑近,``嘉贵妃如今万人之上,可明白其中的道理了?''

玉妍与贞淑互视一眼,强压着满腔狂喜,笑道:``本宫只当皇上知道本宫喜欢红色,所以才赏赐的,不意有如此深意。亏了公公名言。''

李玉拱手含笑:``还有一事,奴才须得禀明嘉贵妃娘娘。娘娘知道,宫中出了皇贵妃私通之事,皇上大为不悦,所以要彻查此事。''

玉妍道:``这是应当的。''

李玉颔首:``娘娘明白就好。如今皇上说事涉法师,又有七宝手串为证,便要各宫都写下密宗七宝常用之物。如今娘娘位分最尊,此时须得从娘娘宫中而始。不知娘娘意下如何?''

李玉每说一句,玉妍的笑容便淡一分。她沉吟片刻,目光徐徐扫过身侧的贞淑,淡然笑道:``皇上既然这么说,本宫自然推脱不得。贞淑,你便去将合宫宫人都唤来吧。''

然而,并没有谁的字格外像如懿的,倒是有一个宫人的字奇丑无比,扭扭曲曲。李玉何等机灵,便立刻提了这人来,正是玉妍身边的宫女贞淑。

贞淑颤巍巍跪在坐塌下,因她是跟玉妍从李朝来的陪嫁,皇帝对她也格外客气些,道:``这些字写的那么难看,可是你的手笔?''

贞淑低着头畏惧道:``是。''

李玉厉声喝道:``那这些年来写家书总是会的吧!李朝的字虽然比满文汉文简单些,倒也不至于换种字就写得跟蚯蚓爬似的吧?!''

贞淑嗫嚅着道:``宫里不许宫女识字写字,奴婢很久不写,也生疏了。''皇帝笑了笑,眼中却如深渊寒冰一般,唤道:``李玉。''

李玉即刻上前来,递上两颗珠子。皇帝道:``那也无妨。这是朕赏你的玛瑙,你选一颗好的带回去串成链子戴着,也算是对你这么多年伺候嘉贵妃的一点儿心意了。''

贞淑不解其意,但见皇帝这么吩咐,惶恐了许久,终于选出其中一颗较红的,欠身道:``奴婢谢皇上赏赐。''

皇帝扬了扬脸,定定道:``李玉,朕方才让你去送给嘉贵妃一对步摇,嘉贵妃怎么说?''

李玉朗声道:``嘉贵妃细问了奴才是红玉髓还是玛瑙,然后谢皇上赏赐的红玉髓步摇。''

皇帝摇头道:``嘉贵妃倒识得清楚。''

皇帝瞥了贞淑一眼,定定道:``朕方才说错了,这两颗不是玛瑙,都是红玉髓而已。但无论是与不是,你要选上那么久,朕便知你不识红玉髓。你不能分辨而物,难怪连密宗七宝不用玛瑙而用红玉髓也不知道。''皇帝沉下脸:``李玉,把贞淑松紧慎刑司,换了惢心出来。告诉慎刑司,对贞淑哪里都能用刑,只不许伤了手,直到她能临摹出和皇贵妃一样的字来。''

李玉忙答应去了,皇帝又唤住他:``送惢心回来,再请最好的太医来,替惢心瞧瞧。''

皇帝这么一说,如懿心中更是一沉,忍不住露出几分焦灼神色来。皇帝温然相对:``如懿,今夜你好好儿歇息,明日是中秋,你是朕的皇贵妃,朕等着你来主持中秋家宴。''说罢,皇帝便起身离去。精奇嬷嬷们也跟随着李玉离开。仿佛不过一瞬,如懿又从地狱回到人世,回到她暂摄六宫的皇贵妃之尊。

云端地狱两重辛苦,虚的一颗心仿佛落不到实在处。如懿来不及细细去分辨这其中的辛酸甘苦,只是一迭声向外道:``三宝,三宝!快去接惢心回来。''

\hypertarget{ux7b2cux4e8cux7ae0-ux5f69ux4e91ux6563}{%
\chapter{第二章 彩云散}\label{ux7b2cux4e8cux7ae0-ux5f69ux4e91ux6563}}

惢心是被放在春藤软围上被抬回来的,她已经根本不能站立。盖在她身上遮掩伤势的白布只有薄薄一层,早被鲜血完全浸透,沥沥滴了一路。江与彬得了消息,一早便来到了翊坤宫,伴着如懿心急如焚,立在宫门口候了良久。惢心的神智尚且清楚,见了如懿,热泪滚滚而落,强撑着道:``小主,小主,慎刑司的人问不出我什么。''

如懿望着地上触目惊心的血红,如何还答得出话来,唯有泪水潸然而落。

才说完这一句,惢心就晕厥了过去。如懿只留了小宫女菱枝和芸枝在旁伺候惢心,检查伤势。惢心身上的衣裳不知道积了多少层血水,混合着伤口的脓液,一层层黏在皮肉上,根本解不开来,轻轻一碰,便让昏迷中的惢心发出痛苦的呻吟。如懿知她必定是受了无数酷刑,一时也不敢乱碰,只得让芸枝端了温水进来,一点一点化开衣服上的血水,再用小银剪子将衣服小心剪开。

见到惢心的身体时,所有人脸色都变了。鞭笞、针戳还有棍棒留下的痕迹让她的身上几乎没有一块好肉。她的十根手指受了针刑,那是用细长的银针从指甲缝里穿进,每一根手指都乌黑青紫,积着淤血。而更可怕的是,她的左腿绵软无力,肿胀得没了腿形,根本碰不得。如懿心痛如绞,只得忍了泪与恨,由着江与彬和几位太医来查验。

等到夜半时分,几位太医才忙完了出来回禀。这些日子的焦灼寒心让如懿困顿不堪,她勉强沐浴梳洗了,换过燕居的绿纱绣枝梅金团銮衬衣,坐在灯下默默挑着灯芯。那一颗烧的乌黑卷曲的灯芯便如她自己的心一般,她不敢去细想自己的内心是为何浮动不定,只担心着惢心,那样忠诚可靠的惢心,居然会为了自己落到这样的地步。

江与彬带着沉重的深色走到她跟前时,她的心便凉津津的,几乎坠到了谷底,那声音仿佛不像是自己的了:``惢心到底如何?''

江与彬含着愠怒的泪光,痛心不已:``从伤痕来看,受过鞭刑、棍刑,伤口被浇过辣椒水,所以化脓的厉害,十指都被穿过针,这些都还能治。可惢心的左腿被上过夹棍,生生夹断了小腿骨,只怕以后便是恢复,她的左腿也不能和常人一样行走了。''江与彬切齿道:``皇上是吩咐了用刑,可她们用刑之重,超出慎刑司所能。微臣问了,是嘉贵妃吩咐格外用重刑的。惢心不过是一个弱女子,竟然被折磨成这样\ldots\ldots{}''

如懿心头像被火舌滋滋地舔着,烫的皮肉焦裂,可她所承受的惊怕,如何抵得上惢心这几个日夜的苦楚。她紧紧地攥着绢子,攥得久了,关节液一阵阵酸痛起来。``他们想折磨的,哪里是惢心?恨不得加诸本宫身上才痛快!''如懿深吸一口气,``你好好儿治着惢心,其余不要多想,要用什么尽管说,没有什么药是难得的,统统都用上去,务求还本宫一个好好儿的惢心。''

江与彬沉声道:``是。微臣什么都不会多想,除了治好惢心,便是要害她的人受一样的苦楚才好。''他仰起脸,``还有一件事,无论惢心以后如何,能不能正常行走,微臣都想求取惢心,照顾她一生一世。''

微红的烛光落在他诚挚的面上,这样深情的男子,不离不弃,亦是世间难得的吧。如懿忽然明白了自己心底更深的害怕,原来她的惊惧与惘然,是明白自己身边可以仰仗终身的男子并不是这样的良人。然而,能如何呢?她亦只能留在这里,留在他身边,继续这样于荣华中颠沛辗转的日子。

如懿在感触中慨然落泪:``惢心性子要强,你肯,她未必肯。她只怕拖累了你。''

江与彬的声音沉沉入耳,叫人心生安稳:``微臣中意一人,不在乎她身躯是否残损。''

如懿微微笑了笑:``你肯,自然是好的。本宫也知道,惢心没有选错人。等本宫回过了皇上,定会给你一个答复。这些日子你便常来翊坤宫照顾惢心吧''

江与彬答应着,躬身离去。如懿望着他的背影,郁然叹了口气,吹熄了蜡烛,任由自己沉浸在孤独的黑暗里。

次日便是中秋团圆夜宴。嫔妃们见如懿照常以皇贵妃身份主持宫仪,前日里趾高气扬的玉妍反而默默无声,一时也不敢多加揣测,只是如常般欢笑饮宴。皇帝似是极高兴,对嫔妃们的欢声笑语殷勤劝酒来者不拒,终致醉倒,斜斜支在青玉案上,如玉山倾颓,伏几醺睡。

筵席上丝竹歌舞的迷媚间,如懿以雍容清远的姿态,含着得体而温煦的笑意冷眼相望,一边吩咐李玉:``好好儿扶皇上回去吧。''她的目光对上嬿婉渴盼的眼,不动声色地嘱咐,``送皇上去令嫔宫中吧。''

嫔妃们一一散去,海兰主持着殿中纸醉金迷的残局,一一收拾。如懿只觉得意懒,仿佛这盛世华章,亦不过是余烬人生的浮华点缀。唯有满月悬于高空,以事不关己的姿态,嘲弄着人间的世事无常。

她轻叹间,望见身边一脉长影。她认得出是谁的影子,便轻声唤:``凌大人。''

一语间,是难言的怅然与感激。凌云彻语意寥寥:``夜凉,皇贵妃不宜立于此地。''

如懿转身看着他,一任裙裾旋成流霞旖旎的盈然。她轻笑如珠:``再冷的地方都待过,这里已经很好。''

这话听在云彻耳中,分明是伤感的。他无言以对,只是道:``皇贵妃受苦了。''

``你眼中本宫的苦,在旁人眼中却是本宫大幸。怕是许多人都在想,瞧,这个女人竟又爬了起来,站得那么稳!''她似笑非笑,倚阑轻叹,``世人只敬仰成功,却无人理会孤寒苦痛。''

云彻坦然:``所以皇贵妃娘娘后福无穷。''

``并非本宫后福无穷。''他深深凝睇,``危局之中,是你偷天换日救了本宫。金玉妍的那串七宝手串并无问题,的确用的是红玉髓,是你和海兰替本宫换了一颗近乎一样的玛瑙上去。金玉妍本性奢靡,也唯有她弄错,才会让人相信。因为只有她不信佛理。''

云彻端方的容颜谦逊之至:``也是愉妃娘娘问起微臣是否见过那串七宝手串,微臣才想到这个。而宫婢大多不识玛瑙与红玉髓的不同,便是嘉贵妃只怕一时也难分辨。皇上既然疑心深重,自然会肯相信。微臣只是想,她既本意要害娘娘,那么以彼之道还施彼身也不算错。''

仿佛一道幽细的微光从阴暗的深邃处蓦然照亮内心深弥的曲折。原来他与海兰一样,无论惊涛骇浪,依旧一叶相随。云彻一语既了,明如寒星的眼闪过一丝心安理得的快意。如懿与他相视一笑,同望朗朗皎月,心内亦有明澈。

到了十六那日,如懿陪着皇帝在养心殿一一赏玩各王府公侯家送来的节礼。皇帝尤其喜欢一个珐琅内绘童子赏春的鼻烟壶,叫人赏赐给了和亲王弘昼。另有一对金凤出云点金滚玉合欢步摇,最是精美不过,皇帝亲手簪在如懿的青丝之上,含笑道:``合欢寓意两情欢好,朕替你簪上,再合适不过。''

如懿亦只是低头浅笑,谢恩而已。真的,所谓两情欢好,只在彼此情义与信任上,若要步步疑心,步步惊心,一丝安稳也难得,又何来合欢情好呢?

此时,李玉捧着一张纸进来道:``皇上,奴才用刑下去,贞淑依旧不肯招供。倒是奴才询问了一些与她亲近的宫人才推得些消息,理出这份供状。又迫使贞淑用左手书写申冤,其中几个字与陷害皇贵妃娘娘的几个字十分相似,全是出自一人之手。''

``她肯动笔,那么再要极力扭曲字迹掩饰也难。难为你这般用心,查得一清二楚。''皇帝瞥了几眼,``用左手写的?倒真和皇贵妃的字迹一模一样。''他递给如懿:``你自己瞧瞧。''

倒真是如出一辙。如懿冷笑:``难为她一个李朝女子,倒和本宫的字这么像。''

李玉道:``是。奴才问过了。贞淑在李朝时就习过书法,又略懂医道,所以才成为嘉贵妃陪嫁。贞淑咬死了什么也不肯招供,是启祥宫的小宫女偶然见她藏了几张皇贵妃的临帖私下练字,奴才才有迹可循。可那些宫人们说,自孝贤皇后逝世后,贞淑便常常背着人研习各种字迹,务求练的一模一样,想来对皇贵妃的字也是了如指掌。''他摇头道,``啧啧,嘉贵妃真是有心。孝贤皇后才刚仙逝,她就动了这样害人的念头了,这心思想的真是长远。除了皇贵妃,还指不定对着谁呢。''

皇帝随手将纸抛掷于地,冷冷道:``贵妃?传旨六宫,嘉贵妃金氏不敬孝贤皇后,骄恣妄为,不睦六宫,降为嫔位,禁足于启祥宫思过。''他想一想,``这样的额娘,不配养育她所生的三位阿哥。李玉,立刻着人领回她的三个阿哥,就交在阿哥所抚养。''

李玉答应着去了。如懿抚摸着发髻上冰冷的金线坠珠流苏,心有戚戚:``金玉妍心思狠毒,皇上只降位为嫔位,臣妾真是可惜了惢心的一条左腿了。''

皇帝静静地看着她,眼波并无一丝起伏:``知道朕为什么明知惢心受了重刑也不过问么?''

如懿泪眼婆娑,心底一片哀凉:``臣妾不知。''

皇帝的声音沉稳而笃定,并无一丝迟疑,朗朗道:``朕的心思很简单,就如同先升你做皇贵妃一般。朕想着的是要许你皇后之位。''

``皇后?''如懿不是不明白,封皇贵妃,摄六宫事,本就是通向后位的必经之路,她以抗拒的姿态面对皇帝的淡然自若,``可惢心,为何惢心要受尽酷刑?''

``朕知道慎刑司刑罚残酷,打残了惢心一条腿是委屈了她。可朕不能不委屈她。因为惢心打死不招,你才是清白的。只有你是清白的,才可以做朕的皇后。''

仿佛被條然抛进冰冻的湖水之中,周身凄寒彻骨。她掩不住心底的冷笑,抬起眼盯着皇帝:``皇上,清者自清,臣妾本来就是清白的!''

皇帝微合的眼眸如秋末清凛的风,冷冷掠过:``如懿啊,你在深宫多年,难道不明白,有时候清白不是由自己证明,而是需要旁人作证的么?清者自清,连莲花出淤泥而不染也需时时有人歌颂明白,何况是红墙之中的波云诡谲。''

皇帝的话固然有直剖心胸的冷酷,但确实有几分道理。然而,她的心仿佛覆着厚厚的冰,寒冷而沉重:``那么如果臣妾没有从那串七宝手串上找出嫌疑,皇上是要处死惢心来力证臣妾清白么?''

皇帝的神情并无半分迟疑:``她不会死。死人是不能用来证明清白的,有时候还会归于畏罪自尽,更让你百口莫辩。只有受尽酷刑而不改口供,那才是真的。''

如懿心中的震惊如裂帛碎石,有震腑之痛:``皇上的意思是\ldots\ldots 要惢心赔上自己手足,成了一个活活的废人,才能让皇上相信臣妾清白。''

皇帝看她如此激动,换了温和的语气,伸手向她道:``如懿,这回的事朕疑心本不深,直到不断有人证咬定你与人私通,朕才下决心彻查此事。朕不仅要自己相信,更是要所有人都相信,要所有人都对你没有异议与微词。''

如懿并没有以手相应,凝视他良久。她下颌微扬,与纤美挺直的脖颈形成清傲的弧度,唇角忽地上挑,拉出道冷冷的月弧:``不,皇上是天下之君,只要您深信不疑,流言不能撼动臣妾。皇上所谓的让所有人相信,其实是最想让自己相信。''她笑色凉薄,凄然落泪,``以一个小小奴婢的残废来换取您的安心,换取您挑选国母的眼光,太合算了。''

皇帝的眼神仿佛铅水凝滞,是沉甸甸的铁灰的冷与硬:``皇贵妃,你何时学会说话这般刻薄,不知轻重?''

有凉风猛烈吹进,宛若一把锋利的尖刀刮过,虽不疼却是冷浸浸的冰凉透心。如懿忍不住轻轻颤抖了一下,真的是自己不知轻重么,还是真相,已经习惯了被温存婉转的表象所覆盖?

她跪坐在厚厚的绒毯上,初秋炫金的阳光从镂花长窗中映照而进,她浑身沐浴在明媚的光影里,然而,金子一样灿烂的阳光并没能给她带来如释重负的心情,相反,在这温暖的阳光里,她竟觉得自己成了华美缎子上一点被火焰烧焦的香灰色,瑟缩暗淡,不合时宜。

那泣声哀婉孤清,若一缕轻烟一线游丝,无力地袅袅漂浮于烛影中,好似吹口气便断了。唯有她自己知道,她曾经是如何忍泪不哭,而此刻,此种悲泣无异于斩断了对于夫君最深重的信任。

皇帝以为她伤心感触到了极致,抑或是他太少见到如懿的泪,终于换喝了口吻,扶她起身:``好了,朕是皇帝,身边的亲人太多,会算计朕的亲人也太多。证据罗列眼前,朕偶尔也会有一丝疑心。但朕终于还是选择相信你,你便不要怨朕,也不能怨朕了。''

如懿怔怔片刻,缓缓道:``是,皇上是没有错的。''

她在皇帝身边多年,不是听不出皇帝的语气里已经是最后的包容和耐心。再有哭诉与不满,都不过是自毁长城。对于聪明人而言,时间是最好的师者,日复一日,将她的聪明调教成智慧。而大部分的智慧,与隐忍和适可而止有关。

皇帝已经年近四十了,即便是保养得宜,眉心也有了岁月经过的浅浅划痕,此刻,那些痕迹随着笑意渐渐疏淡。他爱怜地拍了拍如懿的手:``好了,朕自然是没有错的。''他想了想,或许觉得这样的表示太过于凛冽,``或许朕也会有错,但朕是天子,即便有错,也不是朕的本意。''

这,也许是最委婉的表达了吧。她太明白这个答案底下的凛冽与深寒,亦知是不能揭破的。一旦揭破,便是无可换回的错误。她已经走到了这里,千万辛苦,如履薄冰,断不能再失去了。

于是,如懿含了恰到好处的笑意,有委屈,有柔婉,有近乎于谅解和懂得的情绪:``是,臣妾明白。只是惢心已然废了一条腿,以后在臣妾身边侍奉也不方便。臣妾想,惢心的年纪也大了,太医院的江与彬向臣妾求娶过惢心,不如皇上赏惢心一点儿脸面,将惢心赐婚江太医吧。''

皇帝颔首道:``惢心忠心可嘉,又是潜邸的旧婢,大可指一个朕御前得力的侍卫,譬如凌云彻也好。一介太医,前程上是没什么指望的。''

如懿不意皇帝会突然提起凌云彻,仿佛是谁的指甲重重弹在了心肉上,忙笑道:``江与彬有心,臣妾问了惢心也愿意,算是两情相悦。''

皇帝不以为意:``也好,那朕就成全了他们俩吧。那惢心不在你身边伺候了,你也要挑几个得力的人上来。''

如懿沉默片刻,笑容静若秋水:``臣妾身边比不得嘉贵妃,有那么多得力的人。皇上赏赐了惢心的忠心,那么是否也该赏罚分明?''

皇帝替她擦去眼角的泪痕,道:``贞淑是从李朝跟来的人,即便她受刑不招,朕也不便赐死了她,即刻叫人送回李朝去便是。至于金氏,朕已经下旨降为嫔位,闭宫思过,无事不许到朕跟前来伺候。''

如懿垂下脸,低低道:``皇上赏罚分明,臣妾安心了。''

皇帝沉沉道:``你要安心的不只是这个。从此以后,无人会再质疑你。皇贵妃之后,你的后位之路也会安稳妥当。朕会一直陪着你,走到皇后的宝座之上。''

心底有无声的震动,是,她走到了与后位无限靠近的距离,却也失去了对这个男人发自内心的依靠与信任,却只是更孤寂地感知这种徒劳无功的索然。

如懿欲离开时,已经是月上中天时分。她陪着皇帝用了晚膳,以此温暖家常的情景来告诫自己适应种种变故,又回到了昔日的宁静安详之中。打破这种气氛的是养心殿外传来的已被降为嘉嫔的金玉妍砰砰的磕头声。

没有别的言语,也没有哀切的申诉,更没有伤心欲绝的哭泣,金玉妍只是默默叩首,以额头与金砖地面碰触的沉闷声响,来向皇帝脉脉倾诉。贞淑被赶回李朝,形同告知她失去赖以依靠的母亲,她身边的孤立无援已然显露失宠的败迹。那是最大的危险,远胜于位分的起落,意味着依附在她身上的母族的荣宠也会随之减色。所以她亦明白,自己只能如此,不能哀哭申辩。

殿中静若深水,外头的声响仿佛来自遥远的另一个世界,沉闷而邈远。如懿陪着皇帝临着董其昌的字。自康雍以来,世人多推崇董其昌的书法,皇帝自然也有涉猎。外头响声绵绵不绝,皇帝也不抬头,只问:``谁在外头?''

这话自然不是问如懿的,李玉打开了殿门看了一眼,低声道:``回皇上的话,是嘉嫔。''

皇帝淡淡点头,也不理会。李玉似乎有些动容,忍不住劝道:``皇上,您没看见嘉嫔小主在外头的样子。可怜嘉嫔小主已经三十六岁了,还这样伏地叩首,还当着底下奴才们的面,实在是\ldots\ldots 到底也是三子之母了,得顾及着阿哥们的颜面呀。''

如懿站在皇帝身边,脸色沉静如水,恍若未闻,只悄悄与李玉目光相接。这便是日夜伺候在皇帝身边的人说话的好处了,不动声色地提醒着皇帝,这个心机深重谋夺后位的女子年华已逝又如此不顾身份。

皇帝的脸色果然更难看了几分。如懿轻挽衣袖,不急不缓替皇帝研墨,道:``董其昌云,晋人书取韵,唐人书取法,宋人书取意。此时叩首声扰耳,无论取韵、取法还是取意,都是不能的了。皇上还是暂且停笔,让臣妾为皇上磨出颜色合适的墨汁吧。''

皇帝伸笔饱蘸墨汁,下笔如行云流水,曳曳生姿,丝毫不见滞缓,道:``如懿,你出去,以皇贵妃的身份告诉她,从此刻起,她已经不是嘉嫔,而是嘉贵人。若再吵扰一次,便再降一等,直到被废为庶人为止。''

\hypertarget{ux7b2cux4e09ux7ae0-ux7389ux75d5ux4e0a}{%
\chapter{第三章
玉痕(上)}\label{ux7b2cux4e09ux7ae0-ux7389ux75d5ux4e0a}}

如懿明白皇帝言出必行的性子,便福一福身,缓步走到外头。阔大的廊下,硕大环抱的红柱林立,如巨大的壁垒,将跪伏于地的金玉妍衬得渺小而卑微。玉妍穿着一身月白的素色无纹长袍,袖口与衣襟滚着浅银灰的镶边。她脱簪披发,换下象征嫔妃身份的花盆底,只穿平底软鞋,跪在殿外不断叩首。

在看到玉妍面容的一刻,如懿有微微的惊诧,这个一向妩媚娇艳的女子,却未在此时展露她梨花带雨的更能惹人怜爱的哭容,只是倔强地抿着嘴,重重低下一贯高昂的头颅。

如懿没有多余的表情,只是平静地将皇帝的话复述完毕,方才吩咐道:``送嘉贵人回启祥宫,无事不必再出来了。''

玉妍素白的没有任何脂粉装饰的脸,除了眼角细微的如金鱼尾上柔软摇曳的纹理,依旧那样完美,是几乎没有任何瑕疵的玉璧。甚至连续以额叩地后带来的肿起红色,亦不过为她无神的面孔增加了一点儿明艳的桃色芳菲。唯一美中不足的是,她的声音并不如她的容颜一般诱惑,充满了愤恨与恼怒:``我分得清玛瑙和红玉髓!就算贞淑分不清。那算得什么!这不是真的!是你害我!''

如懿双眸微扬,顺手将鬓边一缕垂覆的红璎玉滴珠流苏掠起,那瞬间流露的神采有几分淡然的鄙夷,隐约又带着倔强的不屑,轻轻一嗤:``在这宫里,真相从来就不重要。许多事,根本无人在意它是真是假,而是在于是否有人相信。其实你和我一样,都是在赌,只赌皇上信还是不信。''她剜了玉妍一眼,目光似森冷的磨着骨片嚓嚓微响的刀,``或者,你也可以告诉皇上,你明明白白知道那七宝手串上本就是用的红玉髓,根本不是玛瑙。那么你猜,皇上会不会想,只有主使之人才会那么明白确凿呢?当然了,这也是你告诉皇上的,那日得了这些东西,你可一眼都不敢看便封起来给皇上了。''

玉妍的身体栗栗颤抖着:``皇上不会这么待我的,我为皇上生了三位皇子!一定是你挑唆的!是你!皇上才会不信我!''她咬着唇,全然不顾雪白的齿落在暗红而柔软的唇上咬出深深的印迹。

如懿冷淡的眉眼仿若这个季节最末的流火炎炎,隐隐带着冷峻与肃杀将来的气息:``是我么,还是你自作自受?就如我分明与波桑大师没有任何瓜田李下之事,但你所做的一切,也不过是想让人信以为真而已!''

有泪水在眼眶里泫然欲落,玉妍用力举袖狠狠擦拭,抹杀了那即将要涌出的泪水滴落的可能,继而以灼灼的目光直视着如懿,仰着脸道:``你想挑唆我和皇上,你想看我伤心难过,我偏不哭,偏不让你如愿!''

任何神情都不足以表示如懿的鄙夷和愤怒,她的眼神冷漠如十二月的霜雪,覆落于玉妍之身:``你自己的所作所为,远胜于一切挑唆!皇上这么做,已是看在你生育皇子的份上格外留情了。''如懿说罢,嫌恶地不欲看她狼狈而狰狞的面容。

玉妍忽地站起身,扑上来欲扇如懿脸孔。她张扬的手高高扬起,凌厉的风贴着皮肉刮过的一瞬,如懿不避不闪,淡然道:``你要打只管打,只是这巴掌一落下来,位分不说,你的三个阿哥必定是不能再接回你身边养育了,你可想清楚了么?''

玉妍举起的手悬在离如懿的面孔只有半寸之地瑟瑟发颤,仿佛找不到着落一般。许久,那白如葱根的手终于重重落在了她自己的脸颊上,响亮的耳光声和着她悲鸣凄幽无尽。``皇上皇上您不能弃绝臣妾,弃绝臣妾母族啊!皇上!皇上!您可以责怪臣妾,惩罚臣妾,但求不要迁怒臣妾的母族,臣妾求您了!''

如懿缓缓摇头,注目她良久:``没有人要弃绝你,是你自己弃绝了你自己,是你为求荣宠不择手段才可能会牵累了你的母族。私通?''她不屑,``你的脑袋除了这些污秽东西,难道生你养你的李朝便没有教给你一点点聪明良善与懂得进退么?''

鄙弃的神色如刻在玉妍面庞上一般不可抹去:``皇贵妃,你以为你是什么良善之人么?你和我都不是善男信女,又何必说这样的套话?你有你想维护的东西,我有我不能不得的东西,既然狭路相逢,我算不过你的心机计谋,便也罢了。但我身为李朝宗室之女,责罚可受,颜面绝不可丢!我才不会哭,不会任由你看我的笑话!''

玉妍一边说,一边有热泪无可抑制地滚滚而下。她一向自恃身份,将自己与李朝的颜面看得极重,如今提及,显然是伤心害怕到了极处。她手忙脚乱地伸手去擦,越是擦泪水越多,将她的袖口染上星星点点的圆晕,彷如灰败的落花,四散弥漫。她极力遏制着喉间可能溢出的悲声凝泣,梗着脖子道:``我不会哭,不会让你看见我哭!不会让你笑我李朝失了颜面!''

``颜面失却与否,只在你自己做了什么。愿赌服输,你承受自己的恶果便是。''如懿俯视于她,凝神片刻,悄然迫近,衔了一丝诡谲的笑意,极轻极轻地道:``金玉妍,你猜一猜,这次,本宫为什么赢得那么快?''

金玉妍睁大了眼,像僵死而不能瞑目一般:``你说什么?''

如懿伸出纤长的两根手指,轻轻一晃:``孝贤皇后也好,慧贤皇贵妃也罢,如果真是她们要害本宫,如今人死尘烟散,也该尘埃落定了。可若她们也是为人挑唆,那么她们一个个死绝了,那个躲在背后的人,也该自己上场了。说到底,皇后之位近在眼前,你终于忍不住了,是不是?''

玉妍吃惊地看着如懿,双肩不由主地一抖,往后缩去。她一贯妩媚轻柔的双眸里隐着尖锐如针芒的冷光,几乎要穿透她的身体。玉妍的牙齿发出咯咯的磨磋声,若不是进忠眼疾手快按住了她,她几乎要忍不住揉身扑上来。玉妍厉声道:``你胡说!你胡说什么!''

当然只是胡说,如懿哪里有半分凭证。唯一所有的,不过是孝贤皇后死前的厉声呼号,和一点点辨无可辨的蛛丝般的痕迹。

如懿懒得与她多费口舌,正漠然相对间,却见安吉波桑大师身着红袍,手持一串橙黄的蜜蜡佛珠,神态祥和,缓缓步上养心殿的台阶。

如懿颔首施礼:``大师安好。''

安吉波桑眉眼间有淡泊清澈的笑意:``皇贵妃积福,一切安好。''

如懿瞥了掩面啜泣的玉妍一眼:``有大师佛法庇佑,邪灵不侵。''

安吉波桑微微一笑:``姜女不尚铅华,似疏梅之映淡月。即便尘埃拂身,亦终归洁净之道。''

如懿会意,眼底闪过一抹明亮的笑影,如湛湛天光。``禅师不落空寂,若碧沼之吐青莲。即便深陷淤泥,亦能不染自身。''她欠身,温言道,``大师为何此刻来养心殿?''

安吉波桑和缓含笑,有拈花看尘的闲雅之态,道:``中秋已过,特来向皇上辞行。''

如懿微微黯然:``宫中污秽,不是大师清修之地。''

安吉波桑微笑道:``修行处虽然苦寒,但自有清净大自在。''他侧过脸,看着玉妍的目光无比悲悯而慈和:``你有一张美丽胜过格桑花的脸,却没有一颗美丽的心。你有你的孩子,有你的家族,有你的未来,为何不体会清净圆明的自在?不要求无相,求虚妄,否则你的罪过会绵延到你的孩子身上,让他们来承受母亲的业报。''

玉妍美丽而狭长的眼睛鄙夷地转过,她娇艳的嘴唇间狠狠往地上啐了一口唾沫,以此来表示她的愤恨与不满。

安吉波桑宽和地微笑,对着如懿道:``皇贵妃,你以后的路还很远,荆棘与险阻还很多,那日你问我什么是禅,其实圆明清净就是禅,不是麻木不仁,不是什么都不知道,外面一切声音动作清清楚楚,而此心明白,了无挂碍,毫无执着,一片祥和。这样,所有的尘埃都侵扰不了你,因为你没有破绽。''

如懿双手合十:``多谢大师提点。''

波桑含笑:``我也只是提点而已。在雨花阁那几日,我已经发现,皇贵妃娘娘虽然来雨花阁参拜,但所求皆为宫中之事,从不为自己,娘娘其实是不信神佛的。''

如懿失笑:``大师目光清明,被您看穿了。本宫向来不信神佛,只信自己可以做到的。''

波桑凝视她须臾:``信神佛的人有心软之处,只信自己的人必然受过谁都不可信的创痛。但皇贵妃娘娘终有一日也会觉得,神佛不在于多么神明灵验,而是让漂泊无助之心有一寄托安慰之处,扶持来日之路而已。''

他待要再说,李玉已经出来,满面笑容道:``大师,皇上在里头等您了,快请吧。''

如懿见安吉波桑进殿,静静看着进忠半押半送了玉妍回去,便也离开了。

并不愿坐辇轿,也不愿侍从随行,连三宝和菱枝也被打发开去,茕茕独行,更适合如懿此时的心境。

五味杂陈。她没有言声,只是默默前行,企图消弭心底汹涌而来的迷茫与怅然若失的惊痛。

也不知过了多久,她才发现有一道身影一直紧随在身后,如同自己的影子一般,不曾离去。她转首,看见提着羊角风灯跟随在后的凌云彻,淡淡问:``跟着本宫做什么?''

凌云彻跟随在如懿身后三尺远:``本来随着进忠公公护送嘉贵人回宫,但见娘娘心情不佳,微臣不能劝解,所以一路随行。''

如懿无心顾他,懒懒道:``那就应该提灯在前,而非跟随在后。''

他眉目间清澈内敛,笑容仿佛天边清淡如许的月光:``娘娘自己看得清前路走在何方,微臣只需伴随身后,为娘娘照亮后头走过的路,不至于回头之时,心下茫然,连退路都难以看清。''

初秋的月光静谧铺满宫院的每一个角落,一丛丛深红的秋海棠开得正盛,绚烂至寂寞。如懿无谓地笑笑:``也好。本宫此刻的心境,不喜有人陪得太近,但一个人走,又太寂寞惶然。你在,总是好的。''

云彻不再多言,只是默默跟随。当翊坤宫门前火红的绢纱宫灯照亮了如懿苍白的容颜时,他方才低声问道:``为什么娘娘脸上的表情一如微臣当年?''

``什么当年?''

``就像微臣已经明白失去了从前的嬿婉。''

如懿感知于他的敏锐,轻声道:``你说的不错,本宫便是如此。本宫得到了一件极要紧的东西,也失去了一件非常要紧的东西。这般得失,对于一个女人而言,其实是得不偿失。''她微笑,``不过,也谢谢你的嬿婉。不管是出于何种原因,她肯在我危困之时向皇上求情,也是难得了。''

云彻微微苦笑,拱手施礼:``微臣只希望,娘娘以后的路平安顺遂,再无荆棘风雨。''

有一瞬的感动犹如江潮汹涌,没顶的一刻,居然只是想着,原来还有人这样关切着自己。她旋即含笑,明白自己此刻的身份:``凌云彻,江与彬已经向本宫求娶惢心。你的年纪不小,如今也有了前程,是否也该娶妻生子,成家立业?本宫可以为你安排,求娶淑女。''

云彻的神情转瞬黯然:``娘娘关心了。微臣一个人很自在,是在不想多了家室负累。''他停一停,``能伴随皇上与娘娘身边,已是微臣的福气。''

如懿微微颔首,仰首看着清明月色,如被霜雪:``自己能觉得是福气,那就真的是福气了。''

惢心到底年轻,仗着素来底子好,皮肉的外伤倒也渐渐好了。只是伤筋动骨一百天,她的左腿伤得厉害,足足养了小半年才能下地。江与彬又担心着冬日里寒气太过,伤了元气,一日三次端了温补药物来给惢心服用,连菱枝亦笑:``还好惢心姑姑有着自己的月例,还有小主的赏赐,否则江太医的俸禄全给姑姑换了补药吃都不够。''

江与彬倒真是尽心,惢心能起身后腿脚一直不利索,她心里难过,背地里不知流了多少眼泪,都是江与彬开解她:``只要人没事,走路慢些又有什么要紧。''

除了江与彬,李玉得空儿亦常来看望惢心,时常默默良久,只站在一边不言不语。如懿偶尔问起,李玉慨然落泪:``奴才与惢心相识多年,看她从一个活泼泼的姑娘家,生生被折磨成了这个样子。''他跪下,动容道:``小主,别让惢心在宫里熬着了。咱们是一辈子出不去的人,惢心,让她出去吧。''

李玉的心意何尝不是自己的心意?便是在望见飞鸟掠过碧蓝的天空时,她也由衷地生出一丝渴慕,如果从未进宫,如果可以出去,那该有多好。

外面的世界,她从未想象过,但总不会如此被长困于红墙之内,于长街深处望着那一痕碧色蓝天,无尽遐想。

如懿与江与彬的心意沉沉坚定。惢心原嫌自己残废了,怕拖累了江与彬,每每只道:``你如今在太医院受器重,要什么好的妻房没有,我年岁渐长,人又残废了,嫁了你也不般配。''便一直不肯松口嫁他。只是日久天长,见江与彬这般痴心,如懿又屡屡劝解,终是答应了。如懿择了一个艳阳天,由皇帝将惢心赐婚与江与彬。

赐婚出嫁那一日,自然是合宫惊动,上至绿筠,下至宫人,一一都来相送。一则自然是顾及皇帝赐婚的荣耀,如懿又是皇贵妃之尊,自然乐得锦上添花;二则惢心是如懿身边多年心腹,更兼慎刑司一事绝不肯出卖主上,人人钦佩她忠义果敢,自然钦慕。所以那一日的热闹,直如格格出阁一般。

如懿反复叮嘱了江与彬要善待惢心,终至哽咽,还是绿筠扶住了道:``皇贵妃是欢喜过头了,好日子怎可哭泣,来来,本宫替惢心盖上盖头。''

绿筠这般赏面儿,自然是因为玉妍落魄,遂了她的心意。海兰与意欢素来与如懿交好,更是足足添了妆奁,欢欢喜喜送了惢心出宫。

终于到了宫门边,如懿再不能出去,唯有李玉赶来陪伴。李玉殷殷道:``我与江与彬。惢心都是旧日相识,起于寒微。如今惢心有个好归宿,我也心安。好好儿过日子,宫里自有我伺候皇贵妃娘娘。还有,京郊有三十亩良田,是我送你们的新婚贺礼,可不许推辞。''

江与彬与惢心再次谢过,携了手出去。李玉目送良久,知道黄昏烟尘四起,才垂着脊梁,缓缓离去。

如懿目视李玉背影,似乎从他过于欢喜与颓然的姿态中,窥得一点儿不能言说的心意。

如此,江与彬置了小小一处宅子,两人安心度日,惢心得闲便来宫中当几日差。如懿也舍不得她多动,便只让她调教着小宫女规矩。如此,翊坤宫中只剩了菱枝和芸枝两个大宫女,如懿亦不愿兴师动众从内务府调度人手,便也这般勉强度日。

嬿婉自为如懿求情后,往来翊坤宫也多了。皇帝对她的宠爱虽是有一日没一日的,但她年轻乖巧,又能察言观色,总是易得圣心。而最得宠的,便是如懿和舒妃。

到了孝贤皇后薨逝一年之际,皇后母族惴惴于宫中无富察氏女子侍奉在侧,便选了一位年方二八的女子送来,那女孩子出于富察氏旁系,相貌清丽可人,丰润如玉。皇帝倒也礼遇,始入宫便封为贵人,赐号``晋'',住在景阳宫。而李朝也因玉妍的失宠,送了几名年轻貌美的李朝女子来,皇帝并未留下,都赏赐了各府亲王。玉妍本以为有了转机,屡屡献上自己所做的吃食和绣品,皇帝也只是收下,却不过问她的情形。如此,玉妍宫中的伽倻琴哀彻永夜,绵绵无绝,只落了嬿婉一句笑话:``真以为琴声能招来人么?连人都不配了,还在那儿徐娘半老自作多情?''

玉妍本就是牙尖嘴利的人,素来同好不多,嬿婉这句笑话,不多时便传得尽人皆知。玉妍羞愤难当,苦于不得与嬿婉争辩,更失了贞淑,无人可倾诉,只得煎熬着苦闷度日。皇帝充耳不闻,疼惜了嬿婉之时,也将潜邸旧人里的婉贵人封了嫔位。即使宫中入了新人,倒也一切和睦安宁。

入春之后,太医院回禀了几次,说玉妍所生的九阿哥一直伤风咳嗽,并不大好。九阿哥身体十分孱弱,自出生之后便听不得大响动,格外瘦小。皇帝虽然担心,但毕竟子嗣众多,又是失宠妃子所生的孩子,也不过是嘱咐了太医和阿哥所多多关照而已。江与彬得到消息,连连冷笑:``虽然说医者父母心,但也要看是谁的孩子。额娘做了孽,孩子便要受罪,不是么?''

那日海兰、嬿婉与婉茵一起来陪如懿说话,暖阁窗下打着一张花梨边漆心罗汉围榻,铺着香色闪银心缎坐褥。榻上设一张楠木嵌螺钿云腿细牙桌,上头搁着用净水湃过的时新瓜果,众人谈起九阿哥,亦不免感叹。

海兰轻嘘一口气:``听说这些日子皇上虽然关心九阿哥身体,但一直没理会嘉贵人,且贞淑被赶回了李朝,她既失了颜面,也失了臂膀,只怕日子更难过呢。''

嬿婉听得专注,那一双眼睛分外地乌澄晶莹。她扑哧一笑,掩口道:``皇上不是说了么,嘉贵人若再胡闹,便要贬她为庶人呢。且她到底是李朝人,没了心腹在身边出谋划策,瞧她怎么扑腾。''她喜滋滋地看着如懿,``皇上金口玉言,可当着皇贵妃的面亲口说的呢。''

如懿不置可否,笑意中却微露厌倦之色:``皇上是金口玉言,但有些话说说也罢了。你我都不是不知,嘉贵人出身李朝,身份不同寻常。''

嬿婉颇为不解:``那又如何?李朝原本依附前明,我大清入关后又依附于大清,一直进献女子为宫中妃嫔。既为妃嫔,就得守宫规。这次不就严惩了嘉贵人么?''

``虽然严惩,但不至于绝情。''如懿神色淡然,亦有一分无奈,``从前李朝依附前明,屡屡有女子入宫为妃。永乐皇帝的恭献贤妃权氏更因资质秾粹,善吹玉箫而宠擅一时。我大清方入关时,李朝曾有`尊王攘夷'之说,便是要尊崇前明而抵触大清。历代先祖笼络多时,才算安稳下来。金玉妍也算是李朝第一个加入大清的宗室王女。所以无论如何,皇上都会顾及李朝颜面。如今打发了她的心腹臂膀,也算是惩戒了。''她颇有意味地看了嬿婉一眼,``再要如何,怕也不能了。''

嬿婉颇有几分失望:``可嘉贵人如此作孽------''

海兰温和一笑,浅浅打断:``作孽之人自有孽果,我等凡俗之人,又何必操心因果报应之事呢。''

嬿婉眸中一动,旋即明白,只衔了一丝温静笑意,乖巧道:``愉妃姐姐说得是,是妹妹愚昧了。''

婉茵生性胆小,一边听着,一边连连念佛道:``当初嘉贵人就不该鬼迷了心窍,污蔑皇贵妃与安吉波桑大师。不为别的,就为了佛法庄严,怎能轻易亵渎呢。皇上心里又是个尊佛重道之人,真是''

海兰睇她一眼,玩笑道:``婉嫔心中真当是有皇上呢。''她见婉茵面泛红晕,也不欲再与她取笑,只看着如懿殿阁中供着的一尊小叶紫檀佛像,双手合十道:``安吉波桑大师曾希望嘉贵人可以体会清净圆明的自在,否则她的罪过会绵延到她的孩子身上,让他们来承受母亲的业报。波桑大师修行高深,这么说想来也有几分道理。如今看来,九阿哥的病痛,岂非嘉贵人的缘故么?''

嬿婉拿绢子绕在指尖捻着玩,笑道:``好好儿的,咱们说这些个不吉利的人不吉利的事做什么?我倒觉得奇怪呢,今年三月初三的亲桑礼,往年孝贤皇后在时,皇上有时是让皇贵妃代行礼仪的,如今孝贤皇后离世,怎么皇上反而不行此礼了呢?''

如懿叹道:``皇上顾念旧情也是有的。毕竟孝贤皇后去世不过一年,和敬公主又刚出嫁,皇上难免伤怀。''

嬿婉便笑:``也是。姐姐已经是皇贵妃,封后指日可待,也不差这些虚礼儿。也许是皇上想念孝贤皇后,这些日子去晋贵人的宫里也多,每每宠幸之后还赏赐了坐胎药,大约是希望能再有一个富察氏的孩子吧。''

海兰摇头道:``其实论起富察氏的孩子,永璜的生母哲悯皇贵妃不也是富察氏么?听说自从去年永璜遭了皇上贬斥之后,一直精神恍惚,总说梦见哲悯皇贵妃对着他哀哀哭泣。这样日夜不安,病得越发厉害。昨日他的福晋伊拉里氏来见皇贵妃,还一直哭哭啼啼。皇上也未曾亲去看望,自然,或许是前朝事多,皇上分不开身。''

如懿掐了手边一枝供着的碧桃花在手心把玩,那明媚的胭脂色衬得素手纤纤,红白各生艳雅。她徐徐道:``永璜如此,纯贵妃的永璋何尝不是。皇上虽然安慰了永璜的病情,也常叫太医去看着,对着永璋也肯说话了。只是父子的情分到底伤了。听说慧贤皇贵妃的父亲高斌,当日因为孝贤皇后的丧礼受了贬斥,到如今都还没缓过来呢。所以以后一言一行,若涉及孝贤皇后,大家也得仔细着才是。''

这样闲话一晌,便有宫人来请如懿往养心殿,说是皇帝自如意馆中取出了画师禹之鼎的名作《月波吹笛图》与她同赏。众人知道皇帝素来爱与如懿品鉴书画,偶尔兴起,还会亲自画了图样让内务府烧制瓷器,便也识趣,一时都散了。嬿婉带着春婵和澜翠回去,想着要给永寿宫里添置些春日里所用的颜色瓷器,便绕过御花园往东五所的古董房去。

正巧前头绿筠携了侍女漫步过来,看她愁眉轻锁,似有不悦之态。嬿婉忙轻轻巧巧请了个安道:``纯贵妃娘娘万福金安。娘娘怎的愁容满面?''

绿筠嘱了她起来,苦笑道:``皇上刚传了永璋去养心殿查问功课,令嫔也知道本宫这个儿子''

嬿婉笑道:``娘娘的阿哥自然是好的。便是学识上弱些,人是最温和敦厚的性子,皇上自然是知道的。德行乃立身之本,皇上也是看着三阿哥品行不差,才对他学业这般上心。''

一席话说得绿筠眉开眼笑,连连道:``难怪皇上疼爱令嫔,果然见微知著,是个知冷知热的人。''

嬿婉忙谢了,又道:``听闻前些日子嘉贵人对娘娘不敬,幸好娘娘也是个宽厚人儿,如今她落魄,娘娘也不曾对她如何。''

可心道:``可不是?嘉贵人担心九阿哥身体,总是在阿哥所外徘徊,想要见九阿哥。但宫规所限,哪里能够呢?而且九阿哥日夜啼哭不安,我们小主可怜孩子,还叫人送了玉瓶去安枕。这般宽宏大量,也唯有小主了。''

绿筠叹息道:``永璋年幼时也不得养在我身边,母子分离之苦,我是知道的,何况九阿哥病着,我何必再去与嘉贵人计较。''

二人这般说着,便也散了。

嬿婉笑道:``这般懦弱性子,难怪身为贵妃还是一事无成,这辈子也便这样了。''

正进了古董房,掌事太监呵斥着宫人们道:``手脚仔细点儿,前儿个不知哪儿来的老鼠撞跌了一个珐琅瓶儿,叫管事的吃了二十鞭子,再毛手毛脚的,仔细你们的皮!''他正数落着,回头见是嬿婉来了,忙堆起笑奉承着。

澜翠也不理会,只管道:``如今都四月里了,我们小主想换些颜色鲜亮些的瓶儿罐儿摆在阁里,也好让皇上来了看着新鲜舒坦。可有什么好东西么?''

嬿婉眼尖,见着博古架上放着一尊白玉花瓶,看着细腻如脂,光滑莹然,便伸出纤纤玉指一晃,笑道:``那个却还不错。''

掌事太监见嬿婉喜欢那个,立刻赔了十足十的笑容道:``哎哟,令嫔娘娘眼力真好。这个玉瓶是嘉贵人生了九阿哥的时候李朝使者送来的。这回纯贵妃听说九阿哥伤风受寒,日夜啼哭,所以让奴才们把这个玉瓶儿送去阿哥所给九阿哥镇着,也是取玉器安神之效了。''

澜翠轻哼一声:``你们也太不识轻重了。九阿哥不过是个贵人生的,咱们小主可是嫔位,看上李朝进献来的东西,是抬举了他们。''

嬿婉横了一眼,澜翠忙吓得不敢作声。嬿婉温然含笑:``小丫头嘴上没个轻重,叫公公笑话永寿宫没规矩了。''

那掌事太监连声道了``不敢'',嬿婉笑吟吟道:``九阿哥乃是皇嗣,皇嗣不安,便是皇上圣心不安。有什么好东西,还是赶紧送去阿哥所吧,别耽搁了。''说罢,她随意拣选了几样瓷器,便也走了。

出了古董房,澜翠犹自不满:``纯贵妃也太会抓乖卖好了,用李朝进献的东西去给九阿哥安神,没费她什么东西,只动动嘴皮子,就给皇上落了个贤惠的印象。''

嬿婉倏然收住脚,伸出手指在她嘴上一戳,沉下脸道:``嘴皮子碰两下就是给本宫出气了么?只长了嘴没长脑子的,不配留在本宫身边伺候。''

澜翠吓得噤若寒蝉,忙跪下道:``小主,奴婢再不敢多嘴了。''

嬿婉轻嘘一口气:``真想给本宫出气,让本宫痛快的话,就去替本宫做一件事。''

澜翠忙道:``但凭小主吩咐就是。''

嬿婉举眸良久,望着幽蓝辽远的天际,轻声道:``方才他们说什么东西撞着珐琅瓶儿了?''

\hypertarget{ux7b2cux56dbux7ae0-ux7389ux75d5ux4e0b}{%
\chapter{第四章
玉痕(下)}\label{ux7b2cux56dbux7ae0-ux7389ux75d5ux4e0b}}

春日的黄昏暗下来早,夜色朦胧如纱,合着最后一道明紫霞光,将阿哥所披拂于沉沙板暗金之色下。窗外的梨花开到盛极,只消一场春雨,便可断送了最后的繁华。偶尔有风吹过,拂动满树雪色芳菲,花影沉沉欲坠。

玉妍在阿哥所外徘徊许久,苦于不得进殿,正巧绿筠经过,她也不理会,别过脸只作不见。

倒是绿筠却不过情面,先唤了一句:``嘉贵人如何在这里?''

玉妍草草行了一礼,倔强道:``纯贵妃娘娘可要指责嫔妾擅自离宫?皇上是责骂嫔妾,让嫔妾无事不得离宫,可嫔妾的九阿哥体弱不安,嫔妾也不能来阿哥所看看么?''

可心不忿道:``嘉贵人也曾经做过贵妃,协理六宫,自然知道祖宗规矩。探望阿哥有时日安排,不是凭谁想进阿哥所就能进的。''

绿筠忙按住可心道:``嘉贵人,伺候九阿哥的嬷嬷是一直跟着你的,想来对九阿哥也会精心照料,你安心就是。''

``奴才嘛,都贱!''玉妍瞟着可心道,``一日不打不骂就要翻天了,离了启祥宫,没有我盯着,哪里还能照顾好孩子。''接着,玉妍冷笑道:``纯贵妃也是有儿女之人,虽然自己的孩子教养不善,也不必这么对旁人的孩子。要知道,若是对孩子关心不够,来日还不知养出什么黑心种子来呢。''

绿筠凡事好性,却最听不得指摘自己孩子的话,一时如何能忍,讥巧道:``嘉贵人这话说的不错!要是为娘的其身不正,的确是要报应在孩子身上。本来这个时候,九阿哥是该养在您身边,不比这般受苦吧!''

玉妍气得面红耳赤,正要辩驳,刚巧古董房的掌事太监送了东西过来,见了绿筠忙趋奉道:``纯贵妃娘娘万福金安,嘉贵人安。''

可心道:``嘉贵人一味只会讥嘲旁人,自己却什么都帮不上。若不是有小主操持,九阿哥只怕连些安枕的玉器都得不上。能指望嘉贵人这位额娘做什么呢?''

玉妍见来人多了,也不便久留,气哼哼道:``别假惺惺的!你的所作所为,真以为我不知么?''说罢,便拂袖而去。

绿筠连连苦笑:``我都知道收敛本性,为了孩子安分守己,嘉贵人这般性子,可怎么收场呢?''

可心道:``人在做,天在看,由着她去吧。小主就该告诉皇上,嘉贵人擅自出宫,顶撞小主。''

绿筠抚了抚鬓角,摇首道:``多一事不如少一事,我何苦与人为难。也是可怜他为人额娘的心肠吧。''说着,便也有可心扶着去了。

古董房的掌事太监便把一应的玉器瓶罐送进了九阿哥房中,在他枕边的紫檀长桌上罗列排好,叮嘱了乳母道:``这是纯贵妃吩咐的,玉器都要放在离九阿哥近的地方,以作宁神安枕之用,可别错了地方。''

乳母们因着玉妍失宠,对九阿哥也没那么上心,嘴里答应着,身上却懒懒的。到了夜间时分,乳母们愈加懈怠,其中一个陈嬷嬷道:``太医说九阿哥喝不下药去,那药太苦,九阿哥一喝便吐,便让我们喝了化作奶水喂给九阿哥。''

另一个李嬷嬷道:``那药比黄连还苦,九阿哥的舌头怕苦喝不下,咱们的舌头难道就不是人的舌头了?我喝了一口就悄悄倒了,阿弥陀佛,喝了一碗蜜都还缓不过劲儿来呢。''

陈嬷嬷笑道:``原来姐姐和我一样。其实不就是伤风,盖严实点就好了,吃那么多药也没用。''正说着,九阿哥又嘤嘤哭起来,陈嬷嬷厌烦道:``早也哭晚也哭,总没个歇着的时候。他没哭累,咱们倒先听累了。''

李嬷嬷摆手道:``罢了罢了,还是看着些吧。嘉贵人那个爆炭脾气,要听见了又以为咱们苛待了九阿哥呢。昨儿上午来见九阿哥瘦了,又责骂了咱们一通。''

陈嬷嬷冷笑道:``她还当自己是嘉贵妃呢,如今可是嘉贵人,差了一个字就是天差地别了。每次来都打鸡骂狗的,我瞧九阿哥就是摊上这么个额娘才落得这个地步。''说着,她打了个呵欠,``晌午哭的我睡不好,我去后头睡一会儿,你先看着。''

李嬷嬷答应了一声,解开衣衫喂九阿哥喝了几口奶,见九阿哥恹恹的没什么胃口,便皱眉道:``喝奶也喝不成个样子。''便抱了在床上,胡乱拍了几下哄他入睡,自己也伏在床边打起了瞌睡。

夜深人静,红烛高照,散发着幽幽的火光。九阿哥哭得累了,终于睡了过去。桌上的玉瓶透着莹润微光,一阵窸窸窣窣的吱吱声,在静夜里听来格外地诡异。忽然,玉瓶晃了几下,咕咚一声歪了过来,滴溜溜在桌上滚了一圈,碰倒了旁边两个青玉双耳花罐。那几个瓶瓶罐罐都打磨得极圆润,一下从一人高的长桌上哐啷摔了下来,砸了个粉碎响亮。

九阿哥骤然听了这巨大的碰摔之声,撕心裂肺地哭了起来。李嬷嬷也被惊醒了,揉了揉眼一看地上一只灰色的老鼠爬过,便举起扫把赶了赶道:``真晦气,好好儿一只老鼠出来撞了东西。''说罢又连连可惜,``这么好的玉瓶儿,就这么摔碎了,可值不少钱呢。''

她略扫了扫,不耐烦地去拍九阿哥哄着,才拍了几下,只见九阿哥面色铁青,翻着白眼,肚子一抽一抽地搐动着,浑身冒着豆大的汗珠,哭声也越来越微弱。她有些着慌,忙不迭唤了陈嬷嬷出来,两人一起看时,九阿哥已经脸都白了,手脚也不会动了,只有出气没有进气。两人对视一眼,慌不迭冲出去喊道:``太医,太医,九阿哥不好了!''

九阿哥是在太医赶到之前停了气息的。待皇帝赶来阿哥所探视的时候,玉妍已经哭成了一个泪人儿,死死抱着九阿哥已经冰凉的尸身不肯撒手。她披头散发地坐在地上,像是睡梦中被惊醒的,脸上脂粉不施,越发显得脸儿黄黄的,凄楚可怜。皇帝见她如此,也难免动了几分怜悯,忙叫进忠和毓瑚扶了玉妍起来。

皇帝向着乳母怒道:``好好儿的,你们是怎么照顾阿哥的?''

跪在地上的太医是院判齐鲁,他忙道:``皇上,九阿哥本就伤风啼哭,心肺脆弱,乍然听了玉瓶跌碎的大响动,饱受惊恐,惊厥而死。''

皇帝看了满地的玉器碎片:``好好儿的玉瓶怎么会跌下来,是不是你们不当心?!''

李嬷嬷吓的慌忙回道:``皇上恕罪,皇上恕罪。这些玉瓶是黄昏的时候古董房送来的,说是纯贵妃叫送来宁神安枕的。奴婢守着九阿哥睡觉,不知怎的,房中溜进了老鼠,撞碎了瓶子才会惊吓到了阿哥。''

陈嬷嬷也拼命磕头道:``皇上,奴婢们不敢撒谎,的确是守着阿哥一步也不敢走开。本来奴婢们还给九阿哥喂了奶,九阿哥睡得香呢。谁也不知道畜生是怎么溜进来做害的。''

齐鲁道:``九阿哥本来就有伤风之症,加上从娘胎里带来的孱弱,听不得大响动。太医院这些日子给九阿哥对症下药,可方才从微臣查验九阿哥来看,这些药九阿哥并没喝多少,病势沉重,加上受惊吓,才会等不到太医来就过身了。''

皇帝惊怒交加,喝道:``为什么九阿哥有风寒却没有吃药?他的药呢,都上哪儿去了?''

陈嬷嬷与李嬷嬷吓的面面相觑:``汤药太苦,小阿哥喝不下去,所以,所以\ldots\ldots{}''

齐鲁道:``阿哥年幼,喝不下药也是有的,乳母可以自己喝下化作乳汁给阿哥,也是一样的。可从九阿哥最后的样子来看,这些药也没到乳母们的嘴里。怕是药太苦,所以乳母们不肯喝吧。''

玉妍听到这里,呆滞的眼神转了两圈,一把将杯中的九阿哥塞给毓瑚,发疯似的冲上来抓着两个乳母又撕又打:``你们这些黑了心肠的女人,平素不好好儿照顾九阿哥,偷懒懈怠!如今到好,生生害死我的九阿哥!''她恨到了极点,下手极凶,如同疯狂的母兽一般死拉抓扯,乳母们也不敢躲避,被她抓的满脸血痕,狼狈不堪。

皇帝实在看不下去,挥了挥手示意拉住了玉妍。陈嬷嬷忍不住道:``嘉贵人这会儿来怪奴婢,奴婢不敢分辨!只是要不是贵人自己存了害人的念头,九阿哥还好好儿地养在您身边,由不得您每次到阿哥所打鸡骂狗的。您的宫里可混不进老鼠去!''

玉妍哭得两眼发直,皇帝冷道:``做错事还敢犟嘴!李玉,这两个贱婢照顾皇子不善,致使夭折,立刻拖出去打断手脚再赐死。''

玉妍见乳母被拖了出去,抱着皇帝的腿哭道:``皇上,皇上!纯贵妃没安好心,她一直疑心是臣妾挑拨了大阿哥和三阿哥失宠于您,所以送了玉瓶来害九阿哥,臣妾的九阿哥死的好冤啊!''

皇帝摆手道:``好了。这玉瓶朕看过了,是李朝送来的贡品,纯贵妃做不了什么手脚。但凡纯贵妃有错,也只是错在太关心你的儿子。朕看方才两个乳母的样子,想来你平时对她们也不好,她们才敢疏忽了九阿哥。别哭成这么个样子,好歹你还有永珹和永璇呢。''

玉妍哭得声嘶力竭,伏倒在地:``皇上,臣妾哪怕有错,但臣妾的爱子之心没有错啊!臣妾跟随您那么多年,一心一意伺候您,为您诞育皇嗣。如今臣妾连幼子都失去了,若没有您在身边,臣妾活着还有什么意思!''她说罢,昏头涨脑地爬起身来,便往墙上撞去。

幸好李玉眼疾手快,一把拉住了。皇帝见她如此,又是生气又是怜悯,便吩咐齐鲁道:``嘉贵人伤心过度,给她服点安神药。''齐鲁答应着,皇帝又道:``李玉,等下好好儿送嘉贵人回宫,再通知内务府,办好九阿哥的身后事。''说罢,他将最后的温情留于手心,抚摸着九阿哥已经冰冷的小脸,眼角闪过一丝泪光,迈着疲倦的步伐出去了。

九阿哥的突然夭折,令玉妍伤心得难以言喻。因着玉妍失宠的缘故,九阿哥一直没有取名,此时皇帝亦是难过,吩咐了九阿哥随葬在端慧皇太子园寝,一切按照郡王身份举丧。而玉妍每次见到皇帝,必要疑心是绿筠暗害的九阿哥,少不得皇帝冷落了绿筠,更少往钟粹宫去。

绿筠诉苦无门,只得拉着如懿泣道:``皇贵妃娘娘必须要替我做主才好。那玉瓶虽是我送的,可谁知道有那畜生爬进去。皇上心疼九阿哥,也不能让我受这不白之冤啊。''

如懿虽然不信绿筠会害九阿哥,但也无从说起,只得好言安慰道:``纯贵妃别伤心,皇上也是心疼九阿哥,怕嘉贵人伤心头上再胡闹生事,所以且冷一冷你,避避嫌疑。''

绿筠且哭且诉:``如今我便知道了。这样没影儿的事皇上都半信半疑,可见从不曾相信我们。我好歹侍奉皇上十数年,为他生儿育女,却连这点信任都得不到,要我日后如何立足?更难怪我连我的孩子都护不住了。''

绿筠语出伤心,何尝又不是如懿的锥心之痛。原来她与旁人也并无二致。

倒是嬿婉从旁劝阻:``纯贵妃看得通透,却也别太难过。皇上对您如此,对贾贵人何尝也不如此。''她长叹不息,``或许除了孝贤皇后,真的无人走得到皇上心里去。''

绿筠闻言愈加悲伤:``那么我这一生,到底是为了什么?儿女不可庇护,恩情不得长久,空有这贵妃位分,却是形单影只。我又为何要来此走一遭呢?''

唇亡齿寒,兔死狐悲。如懿心底的哀凉、疑惑,不过也同绿筠一般。这一生辛苦辗转,苦苦挣扎所求,到底求得了什么呢?

皇帝虽然不喜玉妍陷害如懿之事,但看她为爱子如此伤心,亦不觉怜悯。正逢李朝闻知九阿哥夭折之事,上书表示慰问,皇帝亦不能太不顾李朝的颜面。连如懿亦劝:``看在往日的情分上,还有永珹和永璇,皇上是该去好好儿安慰嘉贵人。''

李玉亦道:``嘉贵人都三十七了,眼看着幼子逝去,以后只怕也不能再诞育皇子,哪能不伤心得发狂。''

彼时江与彬在旁为如懿请平安脉,听完这些之后,看着皇帝离去,方才冷笑:``李公公的话最是滴水不漏,既做了好人,又提醒着皇上嘉贵人的年老色衰。''

如懿微微一笑,低头绣着紫檀绣架上绷着的春意枝头图:``那么告诉本宫,你又做了什么?''

江与彬笑道:``什么都瞒不过皇贵妃。微臣做不了害人的狠心事,只是在九阿哥的伤风药里多加了一味黄连。这样,九阿哥喝不下去,那些受了嘉贵人打骂的乳母也不肯喝,九阿哥的病自然难好了。但是黄连有清热燥湿、泻火解毒的功效,治高热神昏、心烦不寐是最有效的。微臣可没下错药。''

如懿浅笑如烟:``用一味黄连,让嘉贵人也尝尝你和惢心的黄连之苦吧。''

江与彬心疼道:``一想到惢心的腿再不能像常人一般行走,微臣就痛心不已。本来只想让九阿哥受点病痛折磨,没想到他会受了惊吓夭折。''他嗤笑,``大概这就是所谓的报应不爽吧。不过皇上如今肯去启祥宫看她,也算她因祸得福了。''

眼看皇帝的明黄御驾进了启祥宫,嬿婉站在月色底下,体会四月微温的夜风带着木兰的花香愉悦地拂上面颊。天际有阴云掩过,遮了半面弯月,那半月映照在红墙耸立之上,在浮光如锦的琉璃瓦摇碎的粼粼光影中浮沉漾动,渐渐有了支离破碎的势态,映得嬿婉姣好的面庞也有了几分碎玉般的暗影。

澜翠颇为担心道:``皇上这几日日日都去看望嘉贵人,听进忠的口风,皇上只怕要晋她的位分了。小主,咱们会不会是白白为他人作嫁衣裳了?''

嬿婉含着一缕清浅的微笑:``晋位就晋位,探视就探视,左右皇上这些脸面都是给李朝看的,不只给嘉贵人一个。再说了,他都三十七了。女人啊,一过四十就跟开败的花似的,花无百日红,她还能有几天呢。本宫年轻,容得下皇上对她的一时怜悯。''

澜翠道了``是''。嬿婉笑盈盈握住她的手,将手上一串赤金八宝手串顺势推到了她的手腕上。澜翠忙要退下来,急切道:``小主赏赐,奴婢不敢受。''

嬿婉含笑道:``这回的事你做得好,本宫该赏你的。''

澜翠抿嘴笑道:``奴婢不过是抓了一只饿极了的老鼠悄悄塞进玉瓶里。等到夜深人静的时候,那畜生闻到奶香,哪有不急着出来的。那玉瓶子口子细长肚子大,塞进去了便爬不出瓶口,就只能打翻了玉瓶儿逃出来了。''

嬿婉笑道:``所谓大老鼠惊了玉瓶儿,便是如此。你是做得好。这是皇上要怪,也只能怪纯贵妃多事献殷勤罢了。''

次日,皇帝便下了旨意,复玉妍为嫔位。接着又回书李朝,向李朝国主对嘉嫔与皇嗣的关怀略表谢意。

海兰便向如懿笑道:``表面看来皇上是安慰了嘉嫔的丧子之痛,其实明升暗降,倒是便宜了令嫔,与嘉嫔平起平坐呢。''

嬿婉便笑吟吟向如懿道:``妹妹一直受嘉嫔的脸色,哪怕和她是一样的嫔位,可有皇子到底是不同的。''她抚着肚子道,``妹妹承恩这么久,也总是没有身孕,真不知\ldots\ldots{}''

嬿婉说到一半,才想起如懿也一直膝下空空,连忙起身:``皇贵妃娘娘恕罪,妹妹不是有心的。''

如懿淡然微笑:``妹妹不必吃心,你还年轻,迟早会有孩子的。''她看着坐在一旁眼眶微红的意欢,温言道:``舒妃也是,许多事在天意,不只在人为,只要有心,总会有的。''

意欢拭了拭眼角,嘴上却强撑着:``多谢皇贵妃关怀。''

如懿温和道:``其实皇上对舒妃妹妹和晋贵人都格外体贴,也是想你们早早有孕,所以一直赏赐着坐胎药。听说最近连嘉嫔也在向太医院要坐胎药喝了,以期再为皇上添一个皇子。''

嬿婉听得``嘉嫔''二字,脸色便不好看:``一大把年纪了,还不死心,一味折腾着要生皇子做什么?自己不争气,省得再多又有什么用?''她气咻咻说罢,见如懿也不放在心上,忙赔着笑亦试探着道:``皇贵妃娘娘正当盛年,也该喝些坐胎药,以求早日生下皇子。''

如懿含笑道:``年轻的时候,本宫和慧贤皇贵妃都急着没有孩子,眼看着别人的孩子一个个落地了,长大了,哪有不心急的。一碗碗坐胎药喝下去,喝的舌头都不是自己的了。只是后来想明白了,太医院的药再好,毕竟是药三分毒。再说,子嗣之事是命里注定的,所以也不强求了。''

嬿婉看着如懿的神色,见她不像作假,便也笑道:``娘娘说的是。妹妹们受教了。''

意欢亦道:``也是的,这些年喝着这些坐胎药,一开始十分想要得子的心也喝得淡了,总之,听天由命吧。''

除了翊坤宫,嬿婉便有些神色悒悒,春婵知她又在伤心子嗣之事,便道:``小主,今儿是十五,去宝华殿上香最灵验,奴婢陪小主走一趟吧。''

嬿婉有些痴怔:``春婵,你说本宫吃那些坐胎药吃了这么多年,怎么还是一点儿动静也没有。若不然,便停了那些药吧,喝得本宫心都烦了。''

春婵道:``这药是皇上赏赐舒妃的,咱们偷偷弄来已经不易,若是不喝,怕更难有孕了。''

嬿婉思忖片刻,犹豫着道:``也是,那本宫和这只当求个安慰吧。对了,嘉嫔也跟太医院求取坐胎药了,仔细咱们那个方子,别被她学去了。''

春婵连忙道:``那是。太医院的坐胎药,再好也好不过皇上赏赐的。小主这几年吃的那药,都是奴婢取了方子自己熬的,嘉嫔知道不了。''

嬿婉抚着心口,手指上的翡翠嵌珠护甲映得她的下颌碧色莹莹:``不过嘉嫔没了九阿哥伤心成那个样子,本宫可真是痛快!且连消带打又让纯贵妃受了冷落,也算一举两得。''

春婵笑道:``可不是。当初纯贵妃以为要当皇后了,多么得意。后来,她的大阿哥和三阿哥失宠,要说她去害嘉嫔的孩子,人人都信呢。''

二人正笑着,正见凌云彻领了两个侍卫从前头过来。林晕车行礼如仪:``令嫔娘娘万安。''

嬿婉矜持地扬了扬下巴:``凌大人好。''

凌云彻向身后的两个侍卫看了一眼,那两个侍卫自行退开。云彻道:``令嫔娘娘似乎很高兴。''

嬿婉略略不自在:``本宫没有什么可不高兴的。''

云彻沉吟片刻,直视她道:``有件事恕微臣大胆了。九阿哥的死令嫔娘娘可知么?''

嬿婉眉毛一扬:``宫中无人不知。''

他上前一步,低声道:``是否与你有关?''

嬿婉沉下脸:``大胆!东西是纯贵妃叫送去的,你竟敢肆意怀疑本宫?''

云彻带着意味深长的苦笑:``人人都以为这件事和纯贵妃脱不了干系,课微臣的揣测不是怀疑,而是了解。令嫔娘娘,微臣方才去了古董房,听闻九阿哥房中的玉瓶在送去的路上,曾碰到过娘娘身边的澜翠,而澜翠碰过那些玉瓶。微臣想,阿哥所怎么突然进了老鼠,又那么恰好碰倒了玉瓶惊吓了九阿哥?''

嬿婉神色微变,略略惊惶:``那你打算如何?''

云彻不卑不亢道:``若微臣打算如实禀告皇上,由皇上定夺。娘娘以为如何?''

嬿婉惊得倒退一步:``你敢!''

云彻凝神良久,拱手道:``令嫔娘娘,微臣所知,本来仅限于澜翠碰到过古董房的人,至于澜翠有没有碰到玉瓶,连古董房的人自己都只顾说笑,没看清楚。可您的反应却告诉微臣,微臣的揣测是事实了。''

嬿婉惊怒交加:``你敢试探本宫?!''

``令嫔娘娘敢谋害皇嗣,微臣为何不敢试探娘娘?''他起身径直向前。嬿婉慌了手脚,喝道:``凌云彻!''

云彻并不回头,嬿婉紧赶了几步,拦下他道:``云彻哥哥,看在我们多年的情分上------''

云彻打断她,伤感道:``从你骗我进永寿宫那天,我们便已经没有情分了。''

\hypertarget{ux7b2cux4e94ux7ae0-ux7b11ux8bedux95f2}{%
\chapter{第五章 笑语闲}\label{ux7b2cux4e94ux7ae0-ux7b11ux8bedux95f2}}

嬿婉娇美如水仙花的容颜因为紧张和焦灼而微微扭曲,她急急拉住云彻的衣袖,将他拽进近旁甬道,连生饮都变了腔调:``云彻哥哥,我这么做固然是为了自己,可也是为了皇贵妃啊。嘉嫔以私通的罪名诬陷皇贵妃,那几日皇贵妃禁足翊坤宫,蕊心被关进慎刑司拷打,你不也很着急么?我是为了替皇贵妃求情,在养心殿外跪了那么久,你也是亲眼看见的。我只是想救皇贵妃,想替皇贵妃报仇,那有什么错?''她慌不择言,``而且,而且要不是嘉嫔自己存了坏心,她的孩子怎么会那么不禁吓,一吓就死了。这是报应,不是我!''

云彻气恼:``孩子不禁吓,是你的手太狠!''

嬿婉见他难以说动,以不觉动了气:``我的手太狠?这宫里谁的手不狠?谁的手上没沾过脏东西?便是皇贵妃,如今看这万人之上,谁知道她的手曾经作过什么?''

云彻的神色冷若寒冰,亦闪过一丝悲悯:``皇贵妃作过些什么,我不能去指摘。嬿婉,我知道嘉嫔一直欺辱你,可你害了九阿哥,也冤了纯贵妃。你要自保不难,为何要学嘉嫔?你也不怕自己有报应么?''

嬿婉冷笑道:``报应?我还能有什么报应?左右我没有自己的孩子,和皇贵妃是一样的。若这是报应,那也是皇贵妃的报应。''

云彻摇头:``我以为你做这事是攀附皇贵妃的恩宠,向她寻个依靠,原来你对她也不过如此而已,嬿婉,我对你真的是无话可说了。''

嬿婉深吸一口气:``是。你与我无话可说只不过你一定要向皇上揭发这次的事是我做的,我便告诉皇上,是皇贵妃和瑜妃指使我做的。反正嘉嫔死了孩子,纯贵妃被冷落,这样一箭双雕的事,怎么着别人也更相信是皇贵妃和瑜妃所为。''

云彻逼近一步,脸色深寒:``你敢!''

嬿婉索性笑得笃定:``就算是死,我也不能自己死了。你的荣华富贵是皇贵妃给你的,你就看我敢不敢?''

云彻用力甩开她的手:``嬿婉,你真是面目全非。''

嬿婉冰冷的语调中带了几分伤感:``你又何尝不是?从前你只在乎我,现在你不仅在一荣华富贵,也在意皇贵妃了。'

云彻心头微微一颤:``皇贵妃是我的恩人。''

嬿婉迫视着她的眼睛:``但她也是个女人。'她忽然含了几分得意,``不过,只是一个和我长的有些相似,却比我年老的女人。''

云彻以目光坦然接受她的笑意:``皇贵妃的确比你年长,但你知道为什么她比你更得宠?''

嬿婉目光一缩:``我比她年轻,我一定会比她更得宠。'

云彻微微摇头,沉笃道:``我知道她的手未必干净,但她还有自己的底线,而不像你,除了依附献媚,便是阴谋害人。''

他拂袖而去,嬿婉延中国忽然沁出了泪水:``云彻哥哥,我即便再不好,你也别忘了我们的青梅竹马之情。我,我即使变得再多,也从未忘记过。''

云彻微微一怔,神色复杂难言,茕茕离去。

绿筠被冷落一直到了乾隆十五年的春天,而玉妍,亦在这个春天复位嘉妃,但无论如何,恩宠是比不上从前了。而常常陪伴在皇帝身侧的,是一直以来圣眷不断的舒妃意欢。

黄昏时分流霞漫天,余晖金光不减,缠着绵绵的醉紫红铺满长空。晚霞渐渐变为绛紫,空透了一般,烙在万寿长春的支窗上。

如懿进了养心殿书房,见意欢陪伴在测,与皇帝一起翻着一本诗集细赏。她行礼如仪,却也有几分尴尬,只笑道:``皇上万安,臣妾来的不是时候呢。''

意欢起身肃了一肃,面色微红:``皇贵妃最爱说笑了。妹妹不过是陪皇上小坐怡情而已。''

皇帝笑着起身,牵过如懿的手:``这时候怪热的,怎么想着过来了?仔细路上沾了暑气。''

如懿因见意欢在测,脸上一烧,忙袖了手道:``一路上乘着轿辇,并不很热。''

蕊心伴在一旁,吐来了吐舌头笑道:``回皇上的话,我们小主听说这两日天气热,皇上进御膳房的点心都进的不香,所以特意制了些糕点送来给皇上。'

意欢抿嘴笑道;``皇贵妃的手艺妹妹竟未尝过呢?今儿倒是巧了。''她侧首望着蕊心手里的食盒,``皇上素来畏热,御膳房的点心又甜腻的很,仿佛离了糖汁便做不出味道似的,真真无趣。''

皇帝好奇,便伸手去掀食盒:``做了什么,朕瞧瞧。''

如懿卷起绣着连珠葡萄的浅紫袖口,露出一截白藕似的细腕,端了几个素白小碟出来,一一指着道:``这一碟是紫阳湖产的白菱藕,只切成薄片,脆爽甜津,若嫌味薄,也可佐以酸梅汤浇汁。''

意欢似乎颇为中意:``酸梅汤色泽深红,淋在白藕上倒也好看。只是莲藕只取其清甜就已上佳,不用旁的也罢。'

如懿略点头,有道;`这一碟是脂油糕。''

皇帝皱眉,不觉好笑:``朕素日是爱吃这个,但如今天这样热,脂油糕这样油腻的东西怎能下咽?''

如懿睇他一眼,旋又笑道:``臣妾所做和皇上往常吃的不一样。''她盈盈端起,托到皇帝鼻端,眼见皇帝似乎很被香气吸引,忍着得意的欢喜道:``这脂油糕是将仲春盛开的紫藤花剪下,只挑纯正的紫色用,留下开到八分未及开的花苞,只要花瓣,裁蒂去蕊后拿蜂蜜拌了取小坛子封好。那蜜也有讲究,须得是紫藤花蜜,才能气味纯净而不掺杂。等要吃的时候,那纯糯粉伴切成细丁的脂油,再加冰糖捶碎,一层面一层花瓣拌起来放盘中蒸熟,再用冰块煨的微冷,这便成了。''

意欢看着那盘浅紫糕点,很是喜欢:``寻常脂油俗气,藤花清甜解腻,看着晶莹剔透,倒像是春意融融一般。'

如懿听了这赞便道:``舒妃妹妹若喜欢,可得多尝几块。''她才说完,皇帝已经取过银筷夹了一片入口,连连赞道:``清香甜软,的确不错。''说着又眼馋,``还有别的什么?''

如懿的眉眼含着慧黠跳脱,笑着道:``还有一碟软香糕和一盏甘草冰雪冷圆子。这甘草冰雪冷圆子倒也寻常,入口生津罢了。软香糕是用粳米粉兑了薄荷汁做的,入口清爽生凉。''她边说边递给皇上和意欢,不觉生了几分怀念之色,``臣妾幼年随阿玛在苏州小住,最爱这软香糕。别处再也比不上。臣妾随阿玛回京后十余年间再未曾尝到,后来自己按照记忆中的口味试做了几次也不甚佳。今日又做一次,倒还能入口。'

皇帝和意欢尝过,便牵了如懿坐下,感叹道:``你幼年在苏州小住,至今念念不忘。朕每次听你提起,都十分神往。''他抚着如懿的手背,和缓而坚定,``你放心,朕所喜的杭州,你所爱的苏州,便是人间天堂。朕有生之年,一定会带你去苏杭山水间。''

如懿心头微暖,脸色淡淡的透出了几分芙蓉晕红之意,一抹少有的旖旎微笑,点缀于上,竟是奇异动人:``皇上有心,臣妾多谢了。'

皇帝注目片刻,不觉心旌动摇,越发低柔道:``前儿朕嘱咐如意馆的画师郎世宁为你画了像,你可喜欢?朕觉得郎世宁笔法甚佳,不同于朝中画师的拘束古板,只是怕他一向画惯了吉服正容的模样,画不出你此刻的温柔旖旎。''

如懿见意欢抿着唇笑吟吟的听着,越发的窘,眼波横流,睨了皇帝一眼:``郎世宁又不施第一次为臣妾画了,一向也都好。''

如懿叹道:``先祖康熙时的画师禹之鼎,最擅画人物小像,清俊动人。''他笑意温盈,``可惜画像再好,总不及真人风流清朗。你曾说人老画不老,岁月匆匆,铭记一刻也好。朕会命郎世宁为你一一写实,留待日后细细赏玩。''

意欢微微一怔,似是入神想了片刻,不觉艳羡道:``皇贵妃福气真好。皇贵妃说过的,皇上总惦记着。且不说旁的,这一年一度的苏州进贡的绿梅,只有皇贵妃才有呢。''

皇帝意态闲闲,睨了意欢一眼笑道:``舒妃这时吃醋么?四季百花繁盛,皇贵妃却只爱梅花一种,尤其是绿梅。朕最初也疑惑她为何喜欢,后来一见才知,梅花中唯有绿梅色泽纯绿,枝梗亦青色,恍如翠袖笼寒映素肌,特为清妍别致。有好事者比之为九疑仙子萼绿华,倒也合宜。'

意欢俏生生的脸孔一板,取了一片软香糕嚼了道:``臣妾不过叹一句羡慕罢了,皇上便要这般取笑,真是无趣。'

皇帝满眼皆是笑意,只看着如懿牵着她的袖子道:``你瞧,舒妃生气了,你可要怎么赔补才好?''

如懿低低啐了一口,笑着道:``皇上自己惹的祸害,管臣妾何事?岂有让臣妾赔补的道理。''

皇帝笑得前仰后合,指着二人道:``你们俩一个个牙尖嘴利,算是朕说不过你们。罢了罢了,朕只觉得这糕点十分惬意,但得配个什么茶才算是佳。''

蕊心忙道:``皇上说的是。可不是,咱们小主就备下了。''说罢端出一把青玉茶壶,倒出清冽茶汤,道:``这是松阳进贡的银猴茶,小主说了,也不是什么最名贵的茶,但胜在山野清新,颇有雅趣,配着这江南糕点,最是回味甘芳。'

皇帝举杯一抿,便道:``入口鲜醇甘爽,仿佛有点栗子香。''

意欢品了半盏道:``臣妾也听闻银猴茶,只是难得见到罢了。配着今日的点心,果然最相宜。''

皇帝夹了一片白菱藕送到如懿口边:``你忙碌那么久,自己也不尝尝么?''

还不待皇帝说话,意欢轻摇罗扇,似笑非笑嗔道:``是不是只有皇上喜欢的,皇贵妃才会尽力一试?''

如懿见她一双眸子晶光潋滟,也不知她是玩笑还是醋意,只蕴了浅浅笑色道:``换做舒妃妹妹也会这样,是不是?''她眼见意欢的脸色越来越红,仿佛不胜羞涩,只暗自好笑,转头看着皇帝手边的书卷问:``方才皇上和舒妃妹妹在瞧什么书,这样有趣?''

皇帝将手边的书递给如懿,笑道:``是纳兰容若的《饮水词》,算来也是舒妃的娘家人了,都是叶赫那拉氏的文笔。''

意欢素来清冷的脸庞含了一抹温柔笑色,仿佛二月枝头新绽的鹅黄嫩叶。她低下头卷着衣角,轻声道:``臣妾是真喜欢纳兰容若的词,倒不是因为都是叶赫那拉氏的缘故。臣妾进宫前几知道,皇上喜欢纳兰词。''

皇帝看她一眼,甚是温柔。他的手笃笃敲在桌上,激起沉沉的余音袅袅:``朕喜欢的,你都很喜欢。朕也觉得,纳兰的词极好,读来口角噙香。''

意欢纤纤手指翻过浅黄书页,指着其中一篇道:``旁的也就罢了。臣妾细细读来,觉得这一首《采桑子》最好。''她细细吟哦,语调清婉,``而今才道当时错,心绪凄迷。红泪偷垂,满眼春风百事非。情知此后来无计,强说欢期。一别如斯,落尽犁花月又西。'

如懿见意欢临风窗下,着一身碧水色银丝长衫,青翠冷冽如凝于细翠青竹上的白露,她虽是女子,看在眼中亦觉心旌动摇。意欢真是美,难怪这么多年承宠,恩眷不断,皇帝虽不容她生子,却也舍不得丢开。其实如懿也是美的。如懿的美是要在姹紫嫣红的娇艳中才格外出挑,静静的处于明艳之间,便如一支萼华绿梅,或是一方美玉翡翠,沉静的散发温润光华。比之玉妍美的让人觉得不留余地,分分寸寸逼迫于眼前,意欢更像芝兰玉树,盈然出脱于冰雪晶莹之上,让人心醉神迷。

此刻如懿听她语声如大珠小珠落玉盘,十分清越,便道:``纳兰容若的词以真字取胜,写情真挚浓烈,却非如烈火烹油,烧的灰飞烟灭,必得细细读来,以为是淡淡忧伤,回味却是深深黯然。臣妾以为,容若之词比柳永、晏几道的更清淡,却更隽永,算是本朝佳作了。''

意欢听得如懿娓娓道来,不觉颔首:``皇贵妃说到晏几道的词,我却以为有一首堪比容若的《采桑子》。''

如懿抿嘴一笑:``舒妃妹妹且别说,由得我猜一猜。''她沉吟片刻,眼中一亮,``休休莫莫,离多还是因缘恶。有情无奈思量着。月夜佳期,近写青笺约。心心口口长恨昨,分飞容易当时错。后期休似前欢薄。买断青楼,莫放春闲却。可是这一首《醉落魄》''

皇帝抚掌而笑:``不知舒妃说的是不是?朕想的也是这一首。''

意欢素来清冷如冰雪,如今一笑,却似雪上红梅绽放,光艳夺目。她取过桌上切好的两片雪梨,分别递与皇帝和如懿,笑道:``猜得不错,便是这个做嘉奖了。''

皇帝唇边的笑意恬淡如天际薄薄的云:``两日如斯,是该与两位爱妃把酒论诗,闲散度日,总胜过于前朝那些老头子的聒噪了。''

如懿不觉问:``皇上有烦心事?臣妾原本是来禀告这个月六宫用度的。皇上若心烦,臣妾更不敢说了。''

皇帝笑着摆手:``六宫的事,你掌度着便是,不必时时来回禀朕。''

意欢取过一只新橙:``那雪梨太甜腻了,还是吃些酸甜的好。''她抬起果盘边的小银并刀,另一只手扶定新橙轻轻一剖,橙子旋即裂开,露出满盈莹亮水色的深红色果肉,犹有汁水饱满溢出,意欢有条不紊的将新橙切成大小均匀的块搁入雪白的素纹碟中,碧色盈然的织锦袖口下露出一截如玉皓腕,让人注目。

意欢分好橙,望着皇帝盈然有情意流转,笑道:``并刀如水,吴盐胜雪,纤指破新橙。锦幄初温,兽香不断,相对坐调笙。低声问:``像谁行宿,城上已三更。马滑霜浓,不如休去,直是少人行。连宋徽宗都有为了李师师不提政事暂且沉醉的时候,皇上怎么还要提前朝那些不高兴的事?''

如懿知道意欢是在宽解皇帝心绪,但能让她这般费心劝解,想来皇帝是动过真怒的。她当下也不多言,只屏息敛神,取过橙子咬了一片,道:``新橙降火,舒妃有心了。''

皇帝摇头笑道:``朕真能不烦躁便好了。昨日在朝堂上,礼部提起孝贤皇后离世已经三年了,又说立后之事。谁知朕还没言语,张廷玉便向朕道,富察氏乃满洲八大姓之一,在我朝又家世显赫,若要选立继后,当以富察氏出身最佳。他提了这一句也罢了,朝中居然立时有许多人附和,提出要立晋贵人为后。''

意欢微微震惊,与如懿对视一眼,很快垂眸道:``晋贵人入宫不久,出身虽好,资历却浅,只怕难以服众。'

晋贵人年轻貌美,又出身后族,皇帝难免在她宫中多留了几夜,的确也是得宠。但如懿何曾会把这样一个年轻丫头放在眼里,何况皇上名为恩宠之下赏赐的坐胎药便够她松一口气了。

如懿微微沉吟,眸中清亮:``皇上生气的不是晋贵人能否当得起皇后之位,而是张廷玉在朝中一呼百应。''

皇帝的眼眸闪过一丝阴郁:``先帝驾崩时,留下鄂尔泰和张廷玉为辅政大臣,朕一即位,就下令于二人来日配享太庙的待遇。配享太庙是臣属至高无上的荣耀,但因两位都是老臣,辅佐先帝尽心,朕也都肯许他们。现在看来,张廷玉虽不动声色,却极难缠。''

如懿觑着皇帝脸色,轻声道:``张廷玉本家和亲家姚家有二三十人在朝中或地方上做官,若加上门生故旧,势力实在不小。难怪才提了一句要立晋贵人为后,便有那么多人附和。''

``他们附和便附和,朕不肯就是了。朕以潜邸次序论,说起你以侧福晋之位,居孝贤皇后之后,资历又深。再者,还有纯贵妃,嘉妃和瑜妃,有这些潜邸旧人在,晋贵人实在难以服众。又岂有以区区贵人之位,一跃而至皇后的?''

意欢闪过一丝意料之中的笑容:``那么以这些人的心胸,必定要提起孝贤皇后的临终举荐,要荐纯贵妃为后了?'

皇帝冷笑一声:``你倒乖觉,张廷玉所言和你如出一辙。''

意欢秀眉微蹙:``这样的胡话后宫里传来传去,也当是妇人之见了。怎么朝堂上的大臣也这样不堪了?皇后之位取决于皇上,怎是前任皇后选定后任,或是由大臣们商讨皇上的家事呢?若不是张廷玉糊涂,便是他僭越了。''

\hypertarget{ux7b2cux516dux7ae0-ux98ceux6ce2ux5b9aux4e0a}{%
\chapter{第六章
风波定(上)}\label{ux7b2cux516dux7ae0-ux98ceux6ce2ux5b9aux4e0a}}

纱窗隔断的日光只留下淡漠的痕迹,遥远天边的云霞却有炫目的光亮。皇帝捻着一个新橙揉搓着:``糊涂也好,僭越也好,朕怎会容他肆意置喙朕的家事国事,又这般广布党羽,群起进言!这朝廷是朕的,可不是张廷玉的。于是张廷玉便奏告朕,以年老上奏请求告老还乡。折子里有这么一句话,说``以世宗遗诏许配享太庙,乞上一言为券''。''

如懿微微变色:``怎么?张廷玉还怕皇上不许他已经答允的事情,一定要皇上有所保证么?这实在是太无礼了。这么看,他这请求告老还乡的折子,竟有几分试探皇上的意思了。''

皇帝接过意欢递来的橙子吃了一片,缓缓道:``他要试探,朕便成全。只要他安安分分的从朕眼前走开,朕便许他一个安稳到老。朕已让军机大臣汪由敦拟好了折子来看,明日就可发出去了。''

如懿微微松了一口气:``那就好。''她迟疑片刻,还是道:``皇上,臣妾有一事不得不禀告,只要皇上听了不要气急忧心。''

皇帝瞟她一眼,淡淡道:``你说便是了。''

如懿宁静而柔和,含有难得的凝重,和一丝若隐若现的忧虑,她见皇帝脸色松动了些许,才敢婉声劝道:``皇上,永璜的福晋伊拉里氏来回禀,开春之后,永璜身上就很不好,一日不如一日。请皇上若得空,一定要去瞧一瞧。''

皇帝的侧脸棱角分明,平静而至淡漠:``永璜的病情朕也略知一二。无非是他自己心思重,又都是有些不该有的心思。朕已经让最好的太医去瞧了,也吩咐下去,永璜每日要吃山参吊精神,只要他吃得下,便是十斤,朕这个做阿玛的,也给得起。只求他心思安分些,别再做些无妄之念。''

如懿听皇帝口气,仍是对永璜昔年欲为太子之心十分介怀:``那臣妾可否去看望,也好稍稍宽慰\ldots\ldots{}''

皇帝摆手道:``罢了。如今你是皇贵妃,身份贵重。你一去,不知道永璜又要动什么心思。永璜有他养母纯贵妃探视,你便少去这是非之地。''

如懿只得起身应允。正好李玉进来,道:``皇上,张廷玉大人求见。''

皇帝不悦道:``这个时候,他来做什么?''

李玉道:``张廷玉大人喜滋滋的,说知道皇上下旨许他配享太庙,所以特来谢恩。''

这一来,不仅皇帝,连如懿和意欢都变了脸色。皇帝径自起身。走到书房翻了翻奏折,阒然变色:``朕的奏折刚批复完不久,尚未发出,张廷玉怎么会知道?''他横一眼李玉,带了一抹厉色:``李玉!''

李玉吓得忙跪下:``皇上,奴才不敢!''

如懿忙道:``皇上,李玉不敢。内监不得干政,他不敢看皇上的折子。''

``那么,便只有汪由敦了!''皇帝的脸色极难看,``是了。汪由敦出自张廷玉门下,定是他提前给张廷玉透了风,真是大胆!竟敢擅自透露朕的旨意,到底在汪由敦心里,朕是皇帝还是张廷玉是皇帝?朕为天下主,而今在朝大臣因师生而成门户党羽,怎可姑息?''

意欢冷冷道:``皇上自然是皇上,可他这个门生竟忘了天地君亲师,反而将师长凌驾于君主之上,实在是不该!''

皇帝沉下脸:``张廷玉既然来了,朕就见见他。李玉,去传!''

李玉忙不迭去了。如懿与意欢不敢在侧,便也告退离开,才出殿门,便见张廷玉满脸喜色侯在殿外。张廷玉行礼道:``皇贵妃娘娘万福金安。舒妃娘娘万福金安。''

如懿与意欢微微欠身,看他踌躇满志入内。意欢不屑:``自作聪明才自取其辱呢!他以为扶持了一位富察氏的皇后,难不成以后每一位皇后都出自富察氏么?''

如懿悄然一笑:``内外互为援引,一直是后宫与前朝的生存之道。张廷玉即便为三朝老臣,也不能免俗。只是皇上心性极强,岂是轻易可以左右的?''

意欢笑道:``他越是举荐旁人,越是成全了姐姐呢。我便先恭喜姐姐了。''

果然,皇帝勃然大怒,斥责张廷玉道:``太庙配享的都是些功勋卓越的元老,你张廷玉何德何能,有何功绩,可以和那些元老大臣比肩?鄂尔泰还算有平定苗疆的功劳,你张廷玉所擅长的,不过是谨慎自将,传写谕旨,竟也狂妄自大如此?''

一席话骂的张廷玉冷汗淋淋,皇帝犹不解气,下令革去张廷玉的伯爵之位,只以大学士衔告老还乡,又下诏解除汪由敦协办大学士和刑部尚书之职,仍旧让他在行不任上恕罪。自此,再无人敢随意置喙立后之事了。

这一日秋高气爽,明朗天光在紫禁城中无遮无拦的流动,宛如潺潺的河水。静静停滞的凝云,自由盘旋的飞鸟,连绵如重山的殿脊,沉寂的宫阙掩映了平日的喧嚣,让人心意闲闲。如懿闲来无事,便往储秀宫看意欢。如懿才扶着侍女的手进了殿中,便禁不住笑道:``从前进来,你的殿中草药气味最重,如今到淡了许多,只闻得花香清淡了。''

意欢正捧了一束新折的玉色百合插瓶,莲青色的花袖下露出素白的十指尖尖,纤长的深碧色花叶垂在她三寸阔袖上,那袖口滚了三层云霞缎的暗纹边,上头绣着星星点点的橘花,显得分外明艳。意欢的身形高挑,身影最是纤细瘦美,一枚白玉鎏金蝴蝶压发扣在燕尾之上,垂落细长的碎银流苏,被风徐徐浮动,更添了几分难得的柔美。意欢笑盈盈睇她一眼,侧身让如懿坐下,轻轻嘘了一声:``去岁听了皇贵妃的话,如今是想开了。皇上照例还是赏了坐胎药,嫔妃们也都自己找了方子喝。其实有什么呢,我如今也是有一遭没一遭的,惦记着就喝了,没惦记着也便罢了。''

如懿笑道:``你自己想的开便罢了。我如今也不大喝了,左右到了这个年纪了,有没有子嗣都看天意吧。''

意欢笑意幽妍:``是啊,心思都在那上头,成日里夜不快活。倒不如闲下来侍弄侍弄花草,心里也清净些。''

画眉和云雀在廊下啼啭,一唱一和,啼破金屋无人的静寂。如懿笑道:``皇上喜欢在圆明园养这些鸟雀,你也喜欢。''她眼底闪过一丝促狭,伸手刮着意欢的脸颊道:``只是皇上这样宠爱你,前两日内务府新绣的一床满绣合欢鸳鸯连珠帐页独赏了你,可算是娇眠锦衾里,辗转双鸳鸯。既有了鸳鸯,你还要别的鸟儿做什么?''

意欢面颊一红,啐了一口道:``这也是皇贵妃说的话?没半点儿尊重!''她忽然定了乌澄的双眸,盯着如懿道:``皇贵妃这般说,可是拈我的酸呢。''

意欢的话,五分玩笑,五分认真。如懿心头微微一颤,这清光悠长之中,因了她的猝然一问,触动一时情肠。她不愿去思索,由着性子道:``若说不拈酸。都是女子心肠,难免有时小气。

况你初承恩宠的那些日子,也是我最受苦的日子,这样想起来,我能不心酸?只是自你我相识,总觉得心性投契,且在宫里久了,方知寻常人家的拈酸吃醋到了这里竟也是多余,徒增烦恼而已。''

仿若一滴清澈的雨水无意颤起铺满澄阳的湖面,漾起金色的涟漪点点。意欢清冽的眸光微有痴怔:``姐姐说的话,也是我的心思。皇上纵然疼我,但见他宠幸别人,心里也是火烧火燎的,便是对姐姐,有几次也是忍不住。可日子长了,才觉这心思除了磋磨自己受苦,也无旁用。所以我才养这些鸟儿花儿,散散闲心,且在宫里,说话做事都不得不逼着自己小心。有时侯不能对着人说的话,不如对着这些鸟儿说说,也当解了自己的心事。''

意欢自在皇帝身边,便深得圣眷。她有时说话尖锐,待人亦不热络,因着皇帝的宠爱,也无人敢明着计较。这些年,在旁人眼中,她总是能活得纵情恣意的,可在背人处,她也竟有这样的凄清。

如懿温然相望,抚摸着娇艳的花瓣,柔声道:``那是你不爱往别人宫里去走动。侍奉皇上这么多年了,除了我宫里,也难得看你和旁人来往。''

意欢去过小银剪子,细细修剪完花枝,洒了一点儿清水在花叶上,转首道:``我肯与姐姐来往,是性子相投。与其废那些力气和不相干的人来往,我还不如拾掇拾掇自己。''

如懿看着疏朗殿内,布置大气,并不像是寻常女子的闺阁香艳而秾丽,除了满架子诗书,再无多少锦绣装饰。``宫里除了你,再没有谁能把自己拾掇得这样干净舒服了。''

意欢道:``人干净了,心也干净。''

``咱们身在这地方,周遭的污浊血腥自是不必说了。有时侯难免连自己的手也不干净。能求得心有几分干净,也算难得。''如懿莞尔一笑,看她手边搁着一本温庭筠的诗集,道:``那日在皇上跟前,他不过提了句温庭筠的诗好,你便留心了。''

意欢脸上绯红如流霞:``姐姐一直忙着,今日难得有空儿,还替我留心其这些了。我不过是听皇上说起,随手翻翻罢了。''

二人正说着话,忽然三宝跑了进来道:``小主,小主,不好了。''

如懿沉下脸道:``好好回话,这么毛毛燥燥的。''

三宝擦了把汗道:``回娘娘的话,大阿哥府里来传话,大阿哥病重,怕是不好了。''

如懿伙地起身,起得太快,身子不觉晃了一晃,便道:``纯贵妃知道了么?''

三宝道:``大福晋先来禀报的皇贵妃,钟粹宫只怕还不知道。''

如懿忙道:``纯贵妃是大阿哥养母,让菱枝赶紧去钟粹宫通报。你亲自去养心殿告诉皇上,再吩咐备轿,本宫去瞧永璜。''

意欢见如懿担心,亦叹道:``自从孝贤皇后去世,永璜被申斥,终究积郁成疾。好好的一个皇子,唉\ldots\ldots 姐姐路上小心,别太心急了。''

如懿哪里还能和她细细分说,忙出了储秀宫去。才过长康右门的夹道,却见一众年长宫女正立在红墙上,一个个四十上下年纪,都是出宫后无依无靠才继续留在宫中服侍的。一众人等正在听内务府太监的调拨。如懿只看了一眼,云芝道:``回皇贵妃的话,这是内务府新从圆明园拨来的一批宫女,说是做惯了事极老练的,正训了话要拨去各宫呢。''

如懿点点头,也不欲过问。突然,宫女里一个穿着蓝衣的宫女跑了出来,喝道:``赵公公,凭什么你收了她们的银子便拨去东西六宫,咱们几个没钱使银子给你,你便拨咱们去冷宫当差,天下没有这样的道理。''

如懿听得冷宫二字,触动旧事,不觉多看了两眼。那赵公公五大三粗,拉过那宫女拖在地上拽了两圈,抓着她的头发狠狠往墙上搡了一下,喝道:``你们这些圆明园来的宫女,外来的人敢唱内行的戏,猪油蒙了心吧?本公公肯收钱是给你们脸,你给不起就是自己没脸,还敢叫唤?打死了你都没人知道。''

如懿虽然赶着去永璜府邸,亦不觉蹙眉,唤过跟前的小太监小安道:``小安,去把那个赵太监啦过来,说他的专横霸道本宫都知道了,让他自己去慎刑司领五十大棍,从此不必再内务府当差了。''

小安赶紧着上前去了,那赵公公看见如懿来,早吓得腿软了。如懿拿了肯听他啰嗦,留下了小安去内务府知会宫女人选的分配,便要离开。方才挨打的宫女忙膝行到图一跟前道:``多谢皇贵妃娘娘主持公道。''

如懿见她挨了打,神色却十分倔强,一点儿也不害怕,便道:``你倒是个直性子的,只是什么话都喊出来,也不怕自己吃亏么?''

那宫女不卑不亢道:``奴婢自己吃亏不要紧,不能让没钱的姐妹都吃了亏。''

如懿见她被打得灰头土脸的,仔细看相貌却也端庄整齐,落落大方,像是个有主意的,想着蕊心伤了腿之后自己身边也没个得力的人,便道:``你这样的性子是吃亏,可本宫喜欢。等下洗漱干净了去翊坤宫等着,留在本宫宫里当差吧。''说罢,便急匆匆去了。

待赶到永璜府里时,一众的福晋格格们都跪在地下,嘤嘤的哭泣着。绿筠已经先到了,与伊拉里氏陪在床前,她见了如懿进来,少不得擦了擦眼角的泪痕。肃了一肃道:``皇贵妃万安。''

如懿见阁中一片愁云惨雾,忙按住绿筠的手道:``这个时候了,还闹这些虚礼做什么。''说罢便转首急急问向伊拉里氏,``太医看过了么?可怎么说呢?''

伊拉里氏哭得两眼核桃似的,听得如懿问,忙止了泪站起身来,道:``回娘娘的话,太医说永璜梦魇缠身,日夜不安,心气断断续续的,只怕是\ldots\ldots{}''

如懿心中一沉,脸色便有些不好:``别胡说!永璜才二十三岁,怎么会心气断续?''

伊拉里氏说不上两句,呜咽道:``这两年永璜身上总不大好,忧思过虑,像是总转着什么念头,又不肯告诉妾身。好几次从梦里惊醒,总是大哭说自己不孝。前几日是孝贤皇后的忌日,永璜便梦魇的更厉害,说要去找孝贤皇后理论。妾身也吓坏了\ldots\ldots{}''

伊拉里氏话未说完,脸上已挨了重重的一巴掌。绿筠脸色煞白,气急败坏的指着她道:``终究是你没照顾好永璜,还一味胡说八道!永璜最有孝心,他梦魇什么?要去找仙逝的孝贤皇后理论什么?糊涂油蒙了心,红口白舌的来拉扯永璜不孝!依本宫看,永璜身上不好,都是素日里你们这些不知轻重的人挑唆的他没养好身子。''

绿筠素来性子和缓,如今突然发作,如懿自然明白是因为伊拉里氏的话没说好。这样的话若是落到皇帝耳朵里,又惦记起昔年永璜和永璋在灵前不孝的事,更会惹得皇帝不高兴。

如懿忙拉住绿筠劝道:``姐姐别生气。媳妇素来是懂事的,只是一时着急说话不当心罢了。''她盯着伊拉里氏,温声嘱咐道:``这样的话不许再提了。''如懿看着床上昏睡的永璜,见他满头大汗。她看着心疼不已,忙取过绢子替他仔细擦了又擦,心中愈加内疚不已。永璜似是感觉到她的动作,稍稍有些清醒。他动了动身子,忽然睁开了眼,直瞪瞪的望着帐顶,大声道:``额娘,额娘,你别走,您等等儿子,心疼心疼儿子。''

绿筠忙坐到塌边,拉住永璜的手垂泪道:``永璜,永璜,额娘在这里。''如懿听她呼喊哀切,一时触动了心肠,切切唤道:``永璜。''

两人唤了几声,也不见永璜有任何回应。绿筠便有些讪讪道:``什么额娘?怕是咱们都自作多情了,永璜是在唤他的亲额娘哲敏皇贵妃呢。''说罢又叹,``我虽养了他这些年,可这孩子,到底不大肯叫我一声额娘。''

如懿眼底一酸:``永璜到底是个有孝心的孩子。''

正巧太医进来,翻了翻永璜的眼皮,忙灌了一碗汤药下去,磕个头道:``皇贵妃娘娘恕罪,纯贵妃娘娘恕罪,大阿哥怕是回光返照了。有什么话,能说的就赶紧说了吧。''

如懿听了这话悲从中来,转过脸呜咽起来,汤药灌了下去,永璜果然清醒了许多,两眼也渐渐有神,盯着如懿道:``母亲来了。''

绿筠叹口气道:``永璜好歹也曾养在皇贵妃膝下过,我是没用,两个孩子都遭了皇上的训斥,抬不起头来做人。有什么话,皇贵妃陪着说说吧。''她说罢,便扶着几个福晋的手一同出去了。

阁中静静的,恍若一潭幽寂深水,日光细碎的影子落在地上,像是一个幽若的梦。永璜咳嗽了几声,轻轻道:``多谢母亲还惦记这儿子。幼时养育之恩,儿子一直不敢忘记。''

如懿含了泪,抚着他的额头柔声道:``好孩子。母亲也都还记得,你这孩子什么都好,唯独母子情分上亏欠了。虽然有母亲和纯娘娘照料,但若哲敏皇贵妃还在,你也不至于如此。''

永璜大口大口的喘息着,苍白的脸上浮起两团虚弱的酡红,过了好半晌,才缓过一口气:``儿子自知是不能了。这些日子一直梦见额娘对着儿子含泪不语,总像是有许多委屈,却说不出来。前几日孝贤皇后忌日,儿子更梦见孝贤皇后喂额娘吃些什么,额娘吃完就七窍流血。母亲,儿子心里明白,是孝贤皇后害死了额娘。''

如懿看着他颧骨高耸,两眼深深的凹了进去,难过道:``哲敏皇贵妃之死本来就蹊跷,母亲是听过这样的闲话的。可永璜,闲话是不能过心的,一旦过了心,挣不出来,成了你的心魔,你就害死你自己了。''

永璜呜咽的哭着,那样幽咽而绝望的哭泣,像于黑夜中迷失了方向的孩童。``儿子自幼失了额娘,被人欺侮,儿子很想争气,所以也动过利用母亲的念头。可皇阿玛骂儿子对孝贤皇后不孝,儿子是真的孝敬不了。是她害得我在阿哥所受苦,是她害死我的额娘,是她给额娘吃了那么多相克的食物,甲鱼和苋菜,麦冬和鲫鱼\ldots\ldots 诸如种种,就是同食则会积毒的。我额娘就是这样被慢慢毒死的,我怎么能对着她尽孝\ldots\ldots 我\ldots\ldots 我再不要,不要在这污秽之地了!''

如懿抱着永璜,心绪哀痛的须臾,有浓重般的疑惑如同泼洒与素白生绢之上,迅速流泻,扩散晕染。她止不住一颗几乎要跳跃出来的心,紧紧攥住他的手道:``这些食物相克积毒是谁告诉你的?瑜妃告诉过你是孝贤皇后害死你的额娘,可她从来不知道这些细枝末节。告诉母亲,是谁告诉你的?''

永璜一时急切,一口痰涌了上来,咳咳道:``嘉\ldots\ldots 嘉\ldots\ldots{}''

多年来如在迷雾中穿行,终于有隐约窥得的明亮,如懿连连追问:``是金玉妍是不是?是不是?''永璜拼命长大了嘴,极力晃着脑袋想要点头。如懿见他如此,吓得什么都顾不得了,忙唤道:``太医,太医!''

永璜在她怀里挣扎着,如同脱水之鱼,苟延残喘。他的眼神渐渐涣散,终于吃力的闭上了眼睛,回归至永久的安宁。前尘往事纷至沓来,仿佛秋日黄昏时随风涌动的尘埃,轻的几乎没有半分力气,却应萦绕绕缠到身上,闷住了心肺鼻息,竟生出一种彻骨的恍然无力。仿佛还是小时候,永璜不过七八岁,下了学乏了,便是这样靠在如懿的臂弯里,沉沉睡去。

太医扯着袍子三步并作两步赶了进来,摸了摸永璜的鼻息,垂头丧气道:``皇贵妃娘娘节哀,大阿哥已经去了。''

如懿轻缓的摸着永璜的脸,低声道:``好孩子,睡吧,睡吧,你就能见着你的额娘了。''她捂着嘴,压抑着后间的呜咽,终于在沉默中让眼泪肆意的流了下来。

\hypertarget{ux7b2cux4e03ux7ae0-ux98ceux6ce2ux5b9aux4e0b}{%
\chapter{第七章
风波定(下)}\label{ux7b2cux4e03ux7ae0-ux98ceux6ce2ux5b9aux4e0b}}

乾隆十五年庚午三月十五日申时,皇长子永璜薨,追封定亲王,谥曰安。

如懿进养心殿向皇帝禀报永璜的丧仪时,皇帝正横躺在暖阁的榻上。金立屏,软烟绮,枕边螺钿几上供着一尊釉里红缠枝瓶,瓶中斜斜插着一把姿态妖娆的曼陀罗,雪白浅紫的花瓣碎碎流溢下来,蜿蜒成清媚的风姿。

一切陈设一如既往,却毫无生气。

春日明媚清澈的阳光透过细雕花红木格窗。如一片金色的软纱轻扬起落,无声覆盖在他面上,却亦不能遮去分毫憔悴与神伤之色。

皇帝摩挲着手中一枚子母狮和田青玉佩,听得她足音轻悄,只是微微抬了抬沉重的眼皮,嘶哑着喉咙道:``你来了。''皇帝转过脸,露出几日未刮得青青的胡渣,颇有神骨清赢、沉腰潘鬓的支离。

如懿心头一沉,竟泛起些微酸楚的涟漪。原本在永璜府中处理丧仪,皇帝迟迟不肯露面,她虽然只做了永璜几日的养母,心中也不免怨怒,皇帝对这长子竟连最后的颜面也不给。但如今见他这般,如懿亦不由得生出一分哀悯,转了低柔的语声:``皇上放心,一切都料理好了。''

皇帝将手中的子母狮和田青玉佩递到如懿眼前。那是一枚肉质的青玉佩,玉质细腻油润,幽光沉静,刀工古朴流畅,包浆熟美,一大一小两头狮子神态亲昵,依偎在一起,一看便是积古之物。皇帝的言语间凭空透出几许悲凉:``朕找了很久,真的很久,你去主持永璜的丧仪,朕就一直在找,想找出一样诸瑛用过的东西,可以做个念想。可朕一直找不到,还是毓瑚想起来,从库房的锦匣里找到了这个。朕记得很清楚,这是诸瑛的陪嫁。虽然都是富察氏,但她远不比琅嬅,所以这玉也不算十分名贵,可她戴了很久,一直到死才摘来,朕叫人封存起来。''他絮絮地说道,``你看,这对子母狮多亲热,天伦之乐,毫无嫌隙。''

如懿的瞳孔蓦然收紧:``皇上的意思是,天家父子还不如这一对狮子。''

皇帝暼她一眼,并不动怒,只是将那玉佩握在手中,细细抚摸:``这样的话,只有你会说。如懿,你倒真的不怕。''他苦笑,声音像是垫在香炉下的霞色锦缎,星星点点溅着烧糊的焦灰迹子,``朕真的觉得对不住诸瑛。她是朕的第一个女人,若不是那一刻的动心,朕也不会留下她。她是那么天真单纯的女子,看见朕就会笑得那么高兴。''

如懿凄悯道:``可咱们,终究没有善待她的孩子。''

皇帝的眉宇间衔着温默与疲倦,缓缓地道:``朕不是故意不给永璜脸面,不去她的丧仪。''他握住如懿的手,``如懿,朕是真的不敢看,更不敢去面对。永璜病着的那些日子,朕不愿意听到一点儿他病重的消息,也不愿去看他。朕怕他看朕的眼光只剩下了怨恨。朕更怕,怕自己又一次看见朕的孩子走在了朕的前头。''

眼中不可抑制地漫上泪光,酸涩之味亦从腔子里慢慢涌上了喉头。他固然狠心,却原来也是这样难。如懿只得柔声道:``臣妾知道。臣妾把皇上的意思都告诉了永璜府里,所有的阿哥、命妇都去致丧了。''

皇帝挪了挪身子,虚弱地靠在如懿的腿上,颓丧得像个受了伤的孩子。``从乾隆三年端慧太子去世,十二年七阿哥去世,去年九阿哥去世。如今又是朕的大阿哥。朕登基以来,一直敬慕上天,尊崇佛理,为什么朕的儿子一个个先朕而去,让朕落得白发人送黑发人的伤心。朕,到底做错了什么?''

有泪意模糊地盈上羽睫,仿佛暮霭沉沉时分欲落的雨水。如懿低低道:``皇上,人哪,吃五谷杂粮的身子有病,经不住世事的便是心病。这并不是您的错。''

皇帝以手覆额,叹道:``朕知道你说什么,也只有你会告诉朕,永璜的死是心病。自从孝贤皇后死后,朕知道永璜有夺嫡之心,朕便忌讳着他。他是朕的儿子,他刚刚成年,还那么年轻,朕却渐渐开始老了。朕不能不忌讳,不能不疑心\ldots\ldots{}''

心中的触动如潮水上涌,如懿伸出手指,覆住皇帝的口:``皇上,您正当盛年,如日中天\ldots\ldots{}''

皇帝的眼底露出几分颓丧和阴郁:``如日中天之后便是夕阳西下,哪里比得上冉冉升起的太阳?''

皇帝似是在问,却无人也无话可以应答。他沉浸在自己的思绪里:``儿子长成自然欢喜,可长大了,无能让人担心,有野心又让人害怕。如懿,有时候连朕自己也觉得,自己宠爱公主比皇子更甚。因为对女儿,不会又爱又怕。从太祖努尔哈赤以来,长子争权已经成了本朝君王不得不忌惮的事。太祖的长子褚英仗着战功便心胸狭隘,清算功臣,最后被太祖下令绞杀:太宗皇太极的长子豪格觊觎皇位,屡生事端,结果死于多尔衮之手:圣祖康熙爷的长子胤褆因魇咒太子胤礽,谋夺储位,被削爵囚禁:先帝雍正的长子,朕的三个弘时,为逆臣进言,被先帝逐出宗籍。如懿,朕是经历过昔年的弘时之乱的,朕更害怕,自己一手养大的孩子会和列祖列宗的长子们一样,所以朕申饬永璜比对永璋更严厉,但朕的心里还是疼爱永璜的,毕竟朕的这些孩子里,他是陪着朕最久的一个啊!''

如懿眼中一酸,终于有泪含着温热的气息垂垂而落。她哽咽,极力平复着气息,缓缓道来:``皇上,永璜要是明白您的心思,在九泉之下也会有所安慰。臣妾去看过永璜,他临死前念念不忘他的生母哲悯皇贵妃,深悔自己不能尽孝。''

皇帝的声音极轻,如在梦呓:``朕不是对哲悯皇贵妃的死全无疑心。昔年朕不懂得保护她,让她盛年之时便稀里糊涂离世,如今,又是朕的疑心,逼死了她的儿子。''他轻轻握住如懿的手,手心潮湿而微凉,``如懿,朕在万人之上,俯视万千。可这万人之上却也是无人之巅,让朕觉得自己孤零零的,没有人可以陪着朕。''

如懿的手指抚在皇帝发辫之上,发尾上系着一颗墨绿的玉髓珠子并一颗镂空赤金珠。皇帝束发素来只用明黄一色,然而,不知怎的,如懿只觉得那明亮的金色也变得乌沉沉的,让人心头发坠。她柔声道:``皇上不要多思多虑。您是皇上,亦是人夫,人父,有时候走下来片刻,也未必不好。''

皇帝倦怠地摇头:``这个地方,朕一旦走上去,便已经下不来了。朕从前一直以为孝贤皇后太像一个皇后,而不像一个女人,可如今朕却明白了,她也有她的身不由己。如懿,朕的皇后之位一直空缺,朕很想你快点来,来到朕身边,咱们站在一块儿。''

她意外到了极处,也震惊到了极处,不意皇帝会在这个关节上提起立后之事。然而,心底还是有蒙昧的欢喜:``一块儿?''

皇帝重重颔首,软弱而温存:``如懿,告诉朕,这么多年形影相随,无论朕厚待你、冷弃你,你对朕是否有些许真心?''

``真心?''她的欢喜抽离得如此迅疾。终究,还是清醒的吧。哪怕可以拥有与他并肩而立的荣耀与名位,到底还是在乎那一丝真心,``皇上,臣妾一直以为,相信真心的人是不会这般问的。''

皇帝重重叹一口气,捏着她手的掌心潮湿的如被眼泪倾覆:``如懿,朕也很想去相信,时时处处相信,没有半分疑惑,可朕的身边,太多的女子,对朕的心意未必那般真诚。也许,在她们眼里,朕所能带给她们的尊荣与贵宠,甚至朕的这件龙袍,都远远胜过朕这个人。''

``不是的,不是的。''她急急地分辨,仿佛是为了那一缕一直不肯被尘埃泯去的真意,``皇上,自臣妾是青樱,您是皇子时,臣妾相随您左右。臣妾真的希望,臣妾与您,可以是少年时的相伴,白头后的不离。''

她满心满肺的恳切,似是要将多年的心思与委屈一并诉出。皇帝温柔地沉默须臾,紧紧握住她的手,轻声唤她:``青樱。''

如懿微微苦笑,深吸一口气,抖落心底封存多年的疑虑:``皇上,其实臣妾一直很想问,当年臣妾为您兄长弘时所厌弃,不肯娶入府中,让臣妾沦为笑柄。''她仰着脸,深深地望到皇帝眼底,仿佛要从他深不见底的心潭中探知某种真实的情感,``可皇上,为什么在臣妾最尴尬的时候,您会愿意娶臣妾做您的侧福晋,会那样善待臣妾,让别人都知道臣妾嫁的很好,圆满了乌拉那拉氏的颜面?''

皇帝闭着眼睛,伸出手慢慢地抚摸着她的脸颊。他的手那样轻柔,依稀还如当年那样,爱惜地抚过她的面孔,与她一同在镜中看见最年轻饱满的笑颜,人成双,影成双。皇帝轻声道:``如懿,这是你的鼻子,你的眼睛,你的额头。朕那么熟悉,哪怕是闭上眼睛,你的脸都一直在朕的脑海里。那年朕娶你,娶得是失意的你,安慰的却是同样失意的自己。当年弘时被你的姑母乌拉那拉皇后抚养,几乎与嫡子无异,而朕只是庶出之子,伤心人对伤心人,才能最懂得彼此。娶你入府之后,一开始你总是闹小性子,可时日长了,也渐渐沉稳起来。朕自幼拘束,时时克己,有时候看你的小性子,总觉得那是朕做不到的一面。而你逐渐懂事,朕也很欣慰,因为你的懂事,是为你自己,也是为了朕。所以,朕会和你一起走了那么多年,越来越相知相惜。''皇帝睁开眼,有迷蒙的雾气湿漉漉地浮现,``朕这样说,不知道你明不明白?朕与你的感情,若说不是男女之情,那实在冤屈:若说只是男女之情,却也是委屈了它。因为朕对你,早已超出了如此。''

如懿轻叹一声,有无限岁月凝聚的酸涩一同凝在那叹息的尾音里:``臣妾有自知之明,宫中府中佳丽如云,臣妾并不是最美,性子也算不得最好。作为儿媳,臣妾并不是太后所属意的皇后人选。''

皇帝嘘一口气:``朕知道,你的姑母乌拉那拉皇后是太后的死敌,太后虽然为你改名如懿,面子上也还可以,但心里总不是最愿意的。不过,孝贤皇后就是当年太后与先帝为朕所选,后来太后待她也不过尔尔。''他深吸一口气,眸中深沉,有星芒一般的光熠熠闪过,朗然道,``可朕是皇帝,朕才是天下之主!若连立谁为皇后都由不得自己,那朕算什么皇帝!张廷玉已经走了,太后也不是当年能事事调教朕的太后,谁也不能再约束着朕,哪怕有谁不愿意,朕也必要纵情任意一回!''

心里有绵绵的暖意,仿佛少年的时光再度回到她与他的掌心,盛放出连枝并蒂的缠绵。曾经,她是那样爱慕他,仰望他,是他给了自己救赎,让自己不必成为一辈子的失意人,如懿依着皇帝的肩,轻声道:``可皇上,也是您说的,那是无人之巅,太过清寒。''

皇帝的笑意如透过云层的光。``所以,咱们在一块儿。''他长嘘一口气,``朕已经失去了一个长子,两个嫡子。朕希望册立你为皇后之后,朕还是会有自己的嫡子。''

如懿垂下头,语意伤感:``可臣妾已经是三十三岁了,未必能有所生育。''

皇帝伸开手掌,与她的十指一根根交握:``天命顾及,自然会诞育嫡子;天命若不顾,你与朕最喜爱的孩子,就交给你抚养,可以是咱们的嫡子,所以,你不会膝下孤单。''

如懿轻轻颔首,垂下脸和皇帝紧紧贴在一起:``那么,臣妾可不可以更贪心一些,臣妾日夜期许的,不仅是与皇上有夫妻之情,更有知己之谊,骨血之亲。''

``如懿,你是觉得男女欢爱太过缥缈?''

``是。''她心意沉沉,``臣妾所有,不过是与皇上的名分所在。如果可以,臣妾更希望牢牢把握不会轻易碎裂的情分。''

他拥着她,以保护的姿态,颔首允诺:``朕答允你,如懿,朕答允你。''

她与他的感情,其实一开始就并不存粹------是她,为了争一口气,加入宗室,半委屈半期待着嫁做他的侧福晋;是他,借着她与旁人家族的显赫,一步一步走到了九五之尊的地位,才渐渐生出几分真心。这一路走来,明媚欢悦固然不少,可艰难崎岖,也几乎曾要了她的性命,却从未想过,居然也能走到今日。

窗外,有春色如许,遍耀光年。

仿佛所有带着脂粉气的残酷凄烈,种种的破云诡谲、暗潮汹涌,在那一刻都戛然而止,急速归于平静。待回到翊坤宫中,合宫上下已皆知皇帝的立后之意。虽然在皇长子丧中,欢喜不能形于色,可是这么些年的艰难苦辛、辗转流离,终于到了这一步。

海兰早已等在了翊坤宫中,在垂花门下徘徊相候。如懿远远见了她,穿着一袭新崭崭的天水蓝袍子,衣衫上是不同深浅的亮银与暗蓝的颜色,捧出大朵大朵栀子花的影彩,是静默而深沉的真心欢悦。如懿不知怎的,见了海兰,整个人从虚茫茫的震动和喜悦里落定了心意。好似方才那一路,欢喜而恍惚,竟是稀里糊涂回来的。

海兰见了如懿,疾步上前,想要笑,却是落了泪,紧紧执着她的手,哽咽道:``姐姐,终于有这一日了。''

如懿亦是慨然,隐然有泪光涌动:``是。只是赔上了永璜一条命,才成全了我。''

海兰闻言止了泪,正了容声道:``只有到了皇后之位,姐姐才稍稍安全些,所以,不管谁赔了进去,都不可惜。''

夏日天光极长,夕阳的余晖斜斜铺开红河金光,曳满长空。晚霞渐渐变为绛紫与暗蓝交织的宝带,晚霞背后是烧灼了的深红云影,将天际都燃得空透了一般,影影绰绰烙在殿前``光明盛昌''的屏门上,蔓延倒影在青石砖地上,似水墨画上泼斜的花枝。暮色中的二人披着金黄而模糊的光辉,偶尔有乍暖还寒的风拂掠起袍子飞扬的边角,人也成了茫茫暑气中花叶缭乱的微渺的一枝。

如懿的手心有黏腻的微凉汗珠,她悄然紧握海兰的手,低声在她耳边道:``是,我们所走过的路都是必经之路,所做的事都是不可避免之事。哪怕月寒日暖,来煎人寿。但永璜已死,我固然伤心,却也知道一件秘事。原来除了你,金玉妍也对永璜说过哲悯皇贵妃被孝贤皇后所害。''

海兰严重有迷惑的旋影波转,她惊诧道:``金玉妍?''

如懿含着凛冽的警醒:``金玉妍所言,比你细致许多,连哲悯皇贵妃如何被害死的细枝末节都无一不知,且告诉永璜哲悯皇贵妃是吃了哪些相克的食物而死。''她的声音失却这个季节应有的余温,``皇上曾经与我说过,孝贤皇后至死也不认害死哲悯皇贵妃\ldots\ldots 我从前从不相信,如今看来,却真有几分可信了\ldots\ldots{}''

海兰深吸一口气,蹙起了眉头,但随即又以一贯平和无害的微笑抚平了那一丝凌厉的警惕:``若孝贤皇后所言是真,那么唯一能把如何害死哲悯皇贵妃的始末知道得一清二楚的,才是真正下手害死哲悯皇贵妃之人。''她屏息凝神,呼吸渐渐有了明显的起伏,``姐姐记得么?孝贤皇后生前对饮食性寒性热之事几乎一无所知,连自己的一饮一食都不甚注意,还是金玉妍偶尔提醒。虽然阿箬和双喜都说过,是慧贤皇贵妃和孝贤皇后在咱们冷宫的饮食里加了许多寒湿之物,可是背后主使,或许另有其人。且还有许多事,孝贤皇后也是至死不认的。''

如懿眯起眼眸,有一种细碎的光刺在她的眸底幽沉地晃:``如今看来,这个人倒更像是金玉妍呢。只是海兰,她出身李朝,看似不如慧贤皇贵妃和孝贤皇后出身高门华第、身份尊贵,但皇上为了顾着主属两邦之谊,不到绝处,绝不会轻易动她。''

海兰侧了侧首,牵动云鬓上珠影翠微,闪着掠青曳碧的冷光。她拍一拍如懿的手,屏声静气道:``从前不知敌人身在何处,才受了无数暗算。如今知道是谁了,又已经剪除了她的羽翼,只须看得死死的,还怕她能翻出天去么?不怕!天长日久,闲来无事,这些账便一笔笔慢慢算吧。''

如懿的声线里有沉沉的决断与冷冽:``是,是要慢慢算,我们在这宫里多年,唯一学会的,不就是将对方最引以为傲、赖以为生的东西慢慢锉磨殆尽么?下半生还长着呢,咱们还在一块儿,有的是时间,有的是同一份心力。''

她们彼此相握的手指紧紧收拢,关节因为过于郑重和用力而微微泛白。哪怕有更辉煌的荣耀即将披拂于身,她们依然是昔年彼此依靠的姐妹,相伴同行,从未有异。

之后再有嫔妃来贺,如懿一概都谦逊推却了。皇帝在立后的旨意之后,也于同日下旨,在八月初四,也就是立后之后的两天,复金玉妍贵妃之位。这样的安慰,既是因为玉妍的丧子之痛,也是因为立后大典有万国来朝,不能不顾着李朝的颜面。

\hypertarget{ux7b2cux516bux7ae0-ux51e4ux4f4d}{%
\chapter{第八章 凤位}\label{ux7b2cux516bux7ae0-ux51e4ux4f4d}}

立后的典礼一切皆有成例,由礼部和内务府全权主持。繁文缛节自然无须如懿过问,她忽然松了一口气,仿佛回到了出嫁的时候,由旁人一一安排,她便只需安安心心等着披上嫁衣便是。如今也是,只像一个木偶似的,等着一件件衣裳上身量定,看着凤冠制成送到眼前来。皇帝自然是用心的,一切虽然有孝贤皇后的册封礼可援作旧例,皇帝还是吩咐了一样一样精心制作。绫罗绸缎细细裁剪,凤冠霞帔密密铸成,看得多了,一切也都成了璀璨星河中随手一拘,不值一提。

惢心自然是喜不自胜的,拖着一条受伤的腿在宫中帮忙。这个时候,如懿便察觉了新来的宫女的好处。那个宫女,便是容珮。

容珮生着容长脸儿,细细的眉眼扫过去,冷冷淡淡的没有表情,一身素色斜襟宫女装裹着她瘦削笔直的腰身,紧绷绷地利索。容珮出身下五旗,因在底下时受尽了白眼,如今被人捧着也不为所动,谁也不亲近。她的性子极为利落果敢,做起事来亦十分精明,有着泼辣大胆的一面,亦懂得适时沉默。对着内务府一帮做事油惯了的太监,她心细如发,不卑不亢,将封后的种种细碎事宜料理得妥妥当当。但凡有浑水摸鱼不当心的,她提醒一次便罢,若有第二次,巴掌便招呼上去,半点也不容情。

海兰见了几回,不觉笑道:``这丫头性子厉害,一点儿也不把自己当新来的。''

如懿亦笑:``容珮是个能主事的厉害角色,她放得开手。我也能省心些。''

然而海兰亦担心:``容珮突然进来翊坤宫,底细可清楚么?''

如懿颔首:``三宝都细细查摸过她的底细了,孤苦孩子,无根无依,倒也清净。''

这样伺候了些日子,连惢心亦赞:``有容珮伺候娘娘,奴婢也能安心出去了。''

自此,如懿便把容珮视作了心腹臂膀,格外看重。而容珮因着如懿那日相救,也格外地忠心耿耿,除了如懿,旁的人一个不听,也一个不认。

然而,对于这次的立后,也不是人人都心服的。

自从永璜死后,绿筠更是对亲子永璋的前程心有戚戚,不仅日日奉佛念经,渐渐也吃起斋来。若无大事,也不大出门了。可哪怕温厚避世如绿筠,私下无人偶然相见时,亦黯然神伤道:``皇贵妃,你显然出身贵族,但细论起来,你家世破落,又不为太后中意,并不比汉军旗出身的我好多少。若论美貌,你也不是宫中最美最好的,皇上对你也不算椒房专宠,更何况你连一个公主都没有生过,可是到了最后,竟是你成了皇后,是为了什么呢?''

绿筠的迷惑,或许也是许多人不能言说的不解吧。

彼时的如懿,正是盛世芳华,着华丽纯粹的郁金香红锦袍,那样纯色的红,只在双袖和领口微微缀绣金线夹着玉白色的并蒂昙花花纹,袍角长长地拂在霞色云罗缀明珠的鞋面上,泛着浅浅的金银色泽,华丽如艳阳。也只有这样的时候,她才当之无愧地承担着这样热烈而纯粹的颜色,并以淡然之势,逼得那明艳的红亦生生黯淡了几分。

``是为了什么呢?''如懿自嘲地笑笑,``我本是成也家世,败也家世。我没有最耀眼的美貌,没有深重的宠爱,贤名也不如孝贤皇后。至于孩子,我确实比不上你儿女双全,多子多福。我只有这一条命,一口气,什么都是我自己的。可就是因为我什么都没有,我才可以做一个无所畏惧的皇后。''如懿深深凝睇绿筠渐渐被岁月侵蚀后细纹顿生而微微松弛的脸庞,还有经过孝贤皇后灵前痛责之事后那种深入骨髓的会信与颓然,像一层蒙蒙的灰网如影随形紧紧覆盖,她不觉生出几分唇亡齿寒的伤感,``还有,换作我,绝不会如你一般问出,凭什么是谁当皇后这样的话。''

绿筠注视如懿良久,遗下一束灰暗的目光,垂下哀伤的面孔:``这些年我不求别的,只求我的孩子能平安有福地长大。为了这个,多少委屈我也受得。终于,等啊等,居然那些人都死在了我这个不中用的人前头。我便生了痴心妄想,也听信了金玉妍的奉承,以为自己也有资本争一争皇后之位,至少能为我的孩子们争得一个嫡出的身份,争得一个不再被人欺侮的前程。可是,我终究不如你命好。所以,你要怪罪我当初和你争夺后位的心思,我也只能自作自受而已。''

绿筠的痛苦如懿何尝不懂得,也因这懂得而生出一分悲悯。如懿面色宁和,柔和地望着她:``你一切所为,不过是为了你孩子的前程,并非有意害我。因为我膝下无子,所以不会偏袒任何一位皇子,更不会与你计较旧事。''

绿筠眼中一亮,心被温柔地牵动,感泣道:``真的?''

如懿坦然目视她,平静道:``自然。不为别的,只为永璜是我们都抚养过的孩子,更为了曾经在潜邸之时,除了海兰,便是你与我最为亲密。''

绿筠迎着风,落下感动的泪。永璜和永璋的连番打击,早已让绿筠的恩宠不复旧日,连宫人们也避之不及。世态炎凉如此,不过倚仗着往年的资历熬油似的度日罢了。而她,除了尊贵的身份,早已挽留不住什么,甚至,连渐渐逝去的年华都不曾眷顾她。比之同岁的金玉妍,绿筠的衰老过于明显,而玉妍,至少在艳妆之下,还保留着昔年的风华与韶艳。

绿筠离开后,海兰却是在长春宫寻到了如懿的踪迹。

长春宫中一切布置如孝贤皇后所在之时,只是伊人已去,上泉碧落,早已渺渺。

如懿静静立于暖阁之中,宛然如昨日重来。

海兰款步走近:``不承想姐姐在这里。''

如懿淡淡而笑:``皇上常来长春宫坐坐,感怀孝贤皇后。今日,我也来看看故人故地。''

海兰轻嗤:``皇上情深,姐姐大可不必如此。''

如懿螓首微摇:``不!时至今日,我才发觉,当年与孝贤皇后彼此纠葛是多么无知!我们用了彼此一生最好的年华,互相憎恨,互相残害,一刻也不肯放过。到头来,却成全了谁呢?''

海兰垂眸:``左右她是对不起姐姐的。''

``我也对不起她!''如懿瞬然睁眸,``是我,害死了她心爱的孩子!只要我一闭上眼,我就会害怕,会后悔!''

海兰沉吟片刻,方问:``所以今日姐姐由此及彼,肯不顾昔日争夺后位的种种,就这样轻易放过了纯贵妃么?''

如懿凝神片刻,缓缓道:``昔日争夺后位,纯贵妃既是因为爱子心切,也是因为受了孝贤皇后临死举荐的牵累,更有金玉妍的挑唆。''

海兰微微蹙眉:``可她到底是有那份心的。''

如懿衔了一抹澹然笑意,道:``我明白你的意思。可是,我即将正位中宫,许多事,狠辣自然需要,但也须多一些宽和手段,否则逼得太紧了,也是无益,纯贵妃在嫔妃中位分仅次于我,平伏了她,也是平伏了底下一些人,不为别的,只为到底是我牵累了永璜。我一直未曾忘却永璜死在我怀中的模样。''

海兰抿唇而笑,陪伴在如懿身侧:``姐姐说什么,便是什么吧,我只是觉得,姐姐越来越像一个皇后了。''

如懿颦起了纤细的柳叶眉,长长的睫毛如寒鸦欲振的飞翅,在眼下覆就了浅青色的轻烟,戴着金镶珠琥珀双鸯镯的一痕雪腕抚上金丝白玉昙花的袖,轻声道:``越来越像皇后?海兰,你知道这些日子,我最常想到谁?''

海兰立于她身后,穿了一件新制的月白色缕金线暗花长衣,外翠碧玉色银线素绡软烟罗比甲,手中素白绣玉兰执扇有一下没一下地摇着,一双眼睛似睁非睁:``姐姐是想起从前的乌拉那拉皇后了么?''

如懿环视长春宫,静静道:``有这一日,我也算略略对得住死不瞑目的阿玛和苦心的姑母。只是我最常想到的,却是孝贤皇后。''她见海兰浑不在意,继续道,``这些日子我一直在想,身为中宫,孝贤皇后明面上也算无可挑剔,为何皇上却总对她若即若离,似乎总有些戒心,细想起来,自成为正妻,便无一日真正快活过。对着自己的夫君,自己的枕边人,如履薄冰。''

海兰道:``各人有各人的命,姐姐替旁人操心做什么?''

如懿咬一咬唇,还是抵不住舌尖冲口欲出的话语:``海兰,我一直在想,若孝贤皇后只是妾而非正妻,不曾有与皇上并肩而立同治家国的权利,会不会皇上待她,会像待其他女人一般,更多些温存蜜爱?会不会------''

海兰接口道:``会不会姐姐的姑母也会得些更好的结果。''她柔声道,``姐姐的话,便是教我这样冷心冷意的人听了,也心里发慌,总不会姐姐是觉得,即将正位中宫,反而惹了皇上疑忌吧?姐姐,你是欢喜过头了,才会这么胡思乱想。皇上固然一向自负,不愿权槟下移,更不许任何人违逆,但\ldots\ldots 总不至于此吧。''

如懿勉强一笑:``或许我真是多心了。''明灿的日色顺着熠熠生辉的琉璃碧瓦纷洒而下,在她半张面上铺出一层浅灰的暗影,柔情与心颤、光明与阴暗的分割好似天与地的相隔,却又在无尽处重合,分明而模糊。她只是觉得心底有一种无可言喻的阴冷慢慢地滋生,即使被夏日温暖的阳光包围着,那种凄微的寒意仍然从身体的深处开始蔓延,随着血脉的流动一点一点渗透开去。

乾隆十五年八月初二,皇帝正式下诏,命大学士傅恒为正使,大学士史贻直为副使,持节赍册宝,册立皇贵妃乌拉那拉氏如懿为皇后。

册文隆重而华辞并茂:

朕惟乾始必赖乎坤成健顺之功必备,外治恒资于内职,家邦之化斯隆。惟中阃之久虚,宜鸿仪之肇举。皇贵妃那拉氏,秀毓名门,钟祥世德。早从潜邸,含章而懋著芳型。晋锡荣封,受祉而克娴内则。今兹阅三载而届期,成礼式遵慈谕。恭奉皇太后命,以金册金宝立尔为皇后。逮螽斯穋木之仁恩,永绥后福。覃茧馆鞠衣之德教,敬绍前徽。星命有光。鸿庥滋至钦哉。

立后这日清晨,天气并不如何烦热,皇帝执手含笑:``朕选在八月初二,那是你当年嫁入潜邸的日子。八月,也和朕的万寿节,又和中秋团圆同一个月。朕希望与你朝朝暮暮相见,年年岁岁团圆。''

如懿着皇后朝服,正衣冠,趁着立后大典之前前往慈宁宫拜见太后。彼时太后已经换好朝服,佩戴金冠,见她来,只是默然受礼。

如懿伏首三拜,诚恳道:``无论皇额娘是否愿意儿臣成为皇后,但儿臣能有今日,终究得多谢皇额娘指点提拔。''

太后抚着衣襟上金龙妆花,目色平淡宁和:``你虽然不是哀家最中意的皇后人选,但也终究是你,能走到这个位置。''

如懿恭顺低首:``多谢皇额娘夸奖。''

太后平和地摇头:``不是夸奖,是你身上流着乌拉那拉氏的血液,那种骨子里的血性,是谁也及不上的。''太后轻嘘一口气,``便是哀家,当年也未曾真正斗赢你姑母。''

如懿微微惊讶,在她的印象中,太后一向是城府极深、妙算心至的。而姑母,成王败寇,早已成了一抹云烟,为世人淡忘。

如懿沉默须臾,道:``皇额娘,儿臣有意识一直不明,还请明示。''

太后看她一眼,淡淡道:``你说吧。''

如懿直视太后,目光中有太多不解与疑惑:``当年儿臣的姑母贵为中宫,又是孝敬宪皇后的亲妹,圣祖孝恭仁皇后的亲眷,为何会在太后您手下一败涂地,最后惨死冷宫?''

太后微微一笑,眼底是深不可测的寒意:``今日是你的喜日,偏要问这么晦气的话么?''

如懿的笑意静静的,像瑰丽日光下凝然不动的鸳鸯瓦,瑰丽中却让人沉得下心气:``问了晦气的话,是指望自己的来日不会晦气,但请皇额娘成全。''

太后望着殿外浮金万丈,微微眯了双眼,似是沉溺在久远的往事之中,幽幽道:``自作孽,不可活。''

如懿微一沉吟,雪白的齿轻轻咬住:``宫中何人不作孽,为何独独姑母不可活?''

太后望向如懿,细细打量了片刻:``你说这话的时候,很有你姑母不输天下的气度。只可惜\ldots\ldots{}''太后摇摇头,徐徐道,``你姑母就是太在意了。太在意子嗣,太在意后位,更在意君心。其实,皇后就是一个供奉着的神位,什么都是过眼云烟,只要能不出错,不为人所害,终究等得到一生荣华平安。''

如懿迟疑片刻:``那么子嗣、后位、君心,在乎就不对了么?或者,皇额娘不在乎?''

太后从容笑道:``总有人不在乎一些,总有人更在乎一些。更在乎的那些人,露了自己在乎什么,就等于告诉别人自己的致命伤在何处,总让人有机可乘,害了自身。而且,哀家可以再说一次,哀家从未斗赢你的姑母,能斗赢你姑母这位当年的皇后的,只有一个人,那便是先帝,当时的万乘之尊。''

如懿听闻过旧事,抬起明亮的眼眸注目于太后:``是。可是昔年,后宫缭乱,姑母的后位也并不稳当。''

太后的声音是苍老中的冷静,便如秋日冷雨后夫人檐下,郁积着的水珠一滴滴重重坠在光滑的石阶上,激起沉闷的回响:``你错了。历朝历代,即便有宠妃专权,使皇后之位不稳当的,那也只是不稳当而已。从来能动摇后位的,只有皇帝一个。成亦皇帝,败亦皇帝。''

如懿了然于心,扬眸微笑:``所以儿臣一身所系,只在皇上,无关他人。儿臣只要做好皇上的妻子便是了。''

太后亦是笑亦是叹:``能说这话,所以你能坐上后位。但你要明白,你不仅是皇帝的妻子、盟友,也是他的臣子、奴才。即使你是皇后,也是一样。''太后注目片刻,忽而笑得明澈,``从此,你就是万千人之上的皇后,但是,大清的乌拉那拉氏皇后,少有善终啊。''

太后的话,似是诅咒,亦是事实。太祖努尔哈赤的大妃乌拉那拉氏阿巴亥,被太宗皇太极殉葬后,又因顺治爷厌弃其子多尔衮,阿巴亥死后被逐出努尔哈赤的太庙,并追夺一切尊号,下场极为凄凉。而自己的两位姑母,又何尝不凄凉,一个个无子而死,到了自己,自己的来日,又会如何?

她来不及细想,亦没有时间容她细想。喜悦的礼乐声已经响起,迎候她成为这个王朝的女主人,与主宰天下的男子共同成为辽阔天日下并肩而立的身影。

如懿叩首,缓步离开。走出慈宁宫的一刻,她转头回望,日色如金下,慈宁宫的匾额恍如灿灿的金粉挥扬。或许有一日,与太后一样成为慈宁宫的主人,鞠养深宫终老一生,将会是她作为一个皇后最好的归宿吧。

册立之时,钦天监报告吉时已到,午门鸣起钟鼓。皇帝至太和殿后降舆。銮仪卫官赞``鸣鞭'',丹陛大乐队也奏起``庆平之章''的乐声。皮鞭落在宫中的汉白玉石台上格外清脆有力,仿佛整个紫禁城都充满这震撼人心又让人心神眩晕的巨大回声。

如懿站在翊坤宫的仪门外,天气正暑热,微微一动,便易汗流浃背,湿了衣衫。容珮和惢心一直伺候在侧,小心替她正好衣衫,出去汗迹,保持着端正的仪容。其实,比之皇贵妃的服制,皇后的服制又厚重了不少,穿在身上,如同重重金丝枷锁,困住了一身。然而,这身衣衫又是后宫多少女子的向往,一经穿上,便是凌云直上,万人之巅。明亮得发白的日光晒得她微微晕眩,无数金灿灿的光圈逼迫到她眼前,将她绚烂庄重的服色照得如在云端,让人不敢逼视,连身上精工刺绣的飞凤也跃跃欲试,腾云欲飞。

终于走到与自己的男人并肩的一刻,如懿忽然想到了从前的人,同样是继后,她的姑母,在那一刻,是怎样的心情?是否如自己一样,激动中带着丝丝的平静与终于达成心愿的喜悦,感慨万千。

而翊坤宫之侧便是从前孝贤皇后所居的长春宫,比对着翊坤宫的热闹非凡,万众瞩目,用来被皇帝寄托哀思的长春宫显得格外冷清而荒落。或许,连孝贤皇后也未曾想到,最后入主中宫的人,居然会是她,乌拉那拉如懿。

阳光太过明丽眩烈,让如懿在微眯的视线中看见正副册使承命而来,内监依次手捧节、册、宝由中门入宫,将节陈放于中案,册文和宝文陈放于东案,再由引礼女官引如懿在拜位北面立,以册文奉送,如懿行六肃三跪三拜礼。至此,册立皇后礼成。

次日,皇帝在王公和文武大臣的陪同之下,到皇太后宫行礼。礼毕,御太和殿。请王、文武百官各上表行庆贺礼。而如懿也要到皇太后宫行礼,礼毕再至皇帝前行礼。之后,贵妃携妃嫔众人及公主、福晋与内外命妇至翊坤宫内行礼。

而那一日,如懿见到了归宁观礼的和敬公主,一别数年,公主出落成一个明艳照人的妇人,蒙古的水草丰美让她显得丰韵而娇艳,风沙的吹拂让她更添了一丝坚毅凛冽。她扬起美眸望着如懿,那目光无所顾忌地扫视在身上,终于沉沉道:``我没有想到,居然是你成了皇后。知道皇阿玛下旨命我回来观礼之时,我都不能相信,总觉得是纯贵妃也好,嘉妃也好,总轮不到你的。''她的笑意有些古怪,有些鄙夷,``凭什么呢?你配么?''

如懿对着她的视线静静回望:``世间事唯有做不到,少有想不到,何况配与不配,今日本宫与公主,终究也成了名分上的母女。''

和敬骄傲地仰起头:``我皇额娘是嫡后,我是嫡长公主,你不过是继后而已。民间继室入门,见嫡妻牌位要执妾礼,所以,无论如何,你是不能与我皇额娘比肩的。''

如懿笑意蔼蔼,不动声色地将气得脸色发青的容珮掩到身后:``孝贤皇后以`贤'字为谥,本宫自认,无论如何也得不到一个`贤'字为谥了。德行既不能与孝贤皇后比肩,家世亦难望其项背,本宫只有将这后位坐的长久些,恪尽皇后之责,才能稍稍弥补了。''

和敬乍然变色,但闻的周遭贺喜声连绵不绝,她亦不敢多生了是非:``只可惜\ldots\ldots 我皇额娘早逝,幼弟也无福留在人世,才落魄如此,由得你这般落魄户忝居后位。''她重重地咬着唇,衔了冷毒的目光,忽而冷笑声声,``享得住这泼天的富贵,也要受得住来日弥天的大祸。我且看看,看你得意多久?''

如懿望着她年轻的面庞,仔细看着,真实肖似当年的孝贤皇后。她不觉叹了口气,和缓了语调道:``公主,当年孝贤皇后执意将你嫁去蒙古,为的是保有尊荣之余亦可以避开宫中祸端。既然如此,你何不平心静气,好好儿守住自己这一段姻缘。要知道,如今你是蒙古王妃,你的一言一行,系着蒙古安宁与富察氏的荣耀,切记,切记!''

如懿才说罢,便有执礼女官催促她往皇帝身边去,只余下和敬呆立当地,怔怔不言。

日光是一条一条极细淡的金色,如懿仿佛走了很远,终于走到了皇帝身边。皇帝望着她,含着笑意,向她伸出手来,引她至自己身边。

如懿立在皇帝身侧,只觉得自己俯视在万人之上,看着欢呼如山,敬贺之声排山倒海。她有渺茫的错觉,仿佛在浩瀚云端漂浮,相伴终身的人虽在身边,却如一朵若即若离的云,那样不真实。

可是,终也是他,带自己来到这里,不必簇拥在万人中央,举目仰望。如懿的眼角闪过一滴泪,皇帝及时地发现了,轻轻握住她的手,低声道:``别怕,朕在这里。''

如懿温柔颔首,微微抬起脸,感受着日光拂面的轻柔,浅浅地微笑出来。

\hypertarget{ux7b2cux4e5dux7ae0-ux9e33ux76df}{%
\chapter{第九章 鸳盟}\label{ux7b2cux4e5dux7ae0-ux9e33ux76df}}

种种繁文缛节,如懿在兴奋庄正之余,亦觉得疲累不堪。然而那疲累亦是粉了彩绘了金的,像脸上的笑,再酸也不会凋零。

真正的大婚之夜,便是在这一晚。

虽然已是嫁过一次的了,然而皇帝还是郑重其事,洞房便设在了养心殿的寝殿之中。自大婚前一月,皇帝已不在养心殿中召幸嫔妃,仿佛只为静待着大婚之夜。

如懿缓步踏上养心殿熟悉的台阶时,有一瞬的错觉,好像这个地方她是第一次来,如何不是呢?从前侍寝,她亦不过是芸芸众妃之一,被裹在锦缎中,只露出一把青丝婉转,被抬入寝殿,从皇帝的脚边匍匐入内。

比起那时,或许此刻的自己真的是有尊严了太多。如懿静静地想,或许,她所争取的只是这一点生存的尊严吧。当然,这或许是太过奢侈的事。

她缓步走完重重台阶,那样静,连裙角拂过玉台的声音都清晰可闻。仰起脸时,先看到的居然是凌云彻的面孔,他笑意欣慰,屈膝行礼:``皇后娘娘万安。''

这两日一声声入耳皆是皇后娘娘,听得连自己都恍惚了,此刻从她口中唤出,才有了几分真实的意味。如懿含笑:``凌侍卫。''

凌云彻起身相迎:``微臣在此恭迎娘娘千岁。恭喜娘娘如愿以偿。''他微微侧身,``这一路并不好走,幸好,娘娘,走到了。''

如懿盈然一笑:``多谢你,等本宫走到走到这里。''

他拱手,神态萧肃:``微臣会一直陪着娘娘走到娘娘想去的地方。''

如懿颔首,亦不多言,彼此懂得,何须再多言呢,就如她伤心之时,凌云彻只默默身后相随,便是最好的陪伴与宽慰。

如懿行至殿外,见李玉躬身相迎:``皇后娘娘,里头布置妥当,请娘娘举步入内。''

如懿推门而入,素日见惯的寝殿点缀满了让人炫目的红色和金色,连垂落的云锦鲛绡帐也绞了赤金钩帘,缀着樱红流苏。阁中仿佛成了炫彩的海洋,人也成了一点,融入其中,分不清颜色。如懿这才想起,自己已经换下白日皇后吉服,按着皇帝送来的衣衫,穿上了八团龙凤双喜的正红色锦绣长袍。那锦袍用的是极轻薄柔软的联珠对纹锦,触肌微凉,袖口与盘领皆以金线穿雪色小珠密密绣出碧霞云纹西番莲和金云鸾纹小轮花。裙底以捻银丝和水钻做云水潇湘文,显出蔚蓝迷离的变幻之色。两肩、前后胸和前后下摆绣金龙凤合纹八团,以攒枝千叶海棠牡丹簇拥,点缀在每羽花瓣上的事细小而饱满的蔷薇晶与海明珠。除此之外,通身遍饰红双喜、团金万寿字的吉祥纹样,碎珠流苏如星光闪烁,透着繁迷贵气。锦袍下质地轻柔的罗裙,是浑然一体的郁金香色,透明却泛着浅淡的金银色泽,仿佛日出时浅浅的辉光,光艳如流霞。

这并不是寻常的皇后服色,乃是皇帝亲许内务府裁制,仅供这一夜穿着。连佩戴的珠饰也尽显灵龙别致的心思。绿云鬟髻正中是一只九转连珠赤金双鸾镶玉嵌七宝明金步摇,其尾坠有三缕细长的翡翠华题,深碧色的玉辉璀璨,映得人的眉宇间隐有光华流转熠熠。髻边点缀一双流苏长簪,流苏顶端是一羽点翠蝙蝠。蝠嘴里衔着三串流云珍珠红宝石坠角长穗,都以红珊瑚雕琢的双喜间隔,垂落至肩头。髻后是三对小巧的日永琴书簪,皆是以白玉做成,在云鬓间温润有辉。因如懿素喜绿梅,点缀的零星珠花皆以梅花为题,散落其中。而宫中素来爱以鲜花簪发,如懿便在内务府所供的鲜花中弃了牡丹,只用一朵开得全盛的``醉仙芝''玫瑰,如红梅初绽,妩媚娇艳。

那时容佩便笑言:``衣裳上已经有牡丹,再用牡丹便俗了。还是玫瑰大方别致,也告诉别人,花儿又红又香,却有刺,谁也别错了主意。''

是呢,这样步步走来,谁还是无知的清水百合,任人攀折,再美,亦终究是带了刺的。

李玉引着如懿坐下,轻声道:``皇后娘娘安坐,皇上稍后便到。''

如懿安静坐下,描金宽塌上的杏子红苏织龙追凤逐金锦平整地铺着,被幅四周的合欢并蒂莲花文重重叠叠扭合成曼妙连枝,好似红霞云花铺展而开。被子的正中压着一把金玉镶宝石如意和一个通红圆润的苹果。她凭着直觉去摸了摸被子的四角,下面果然放置枣子、花生、桂圆、栗子,取其早生贵子之意。

如懿怔了怔,缓缓有热泪涌至眼底,她知道这样的日子不能哭,忍了又忍,只是没想到,重重地失望复希望之后,皇帝还这样待她,以民间的嫁娶之道,再还她一次新婚之夜。

因为,那时她所缺失的。当年以侧福晋身份入府。到底也是妾室,哪里有红烛高照,对影成双的时刻,那时她的房中,最艳的亦不过是粉色而已,而粉色,终究是上不了台面的侧室之色。

如今,皇帝是补她一次昔日的亏欠,让她再无遗憾。

浸淫在往事的唏嘘中,皇帝不知何时已悄然入内,凝视她道:``想什么这样出神?''

如懿有些不好意思,忙拭了拭眼角道:``皇上万安。''

皇帝温然含笑,眉目澹澹,颇有无限情深:``今夜,朕不是万岁,而是寻常夫君。''他有些愧然,''如懿,朕很想还你一个真正的大婚之夜,但再四问了礼部,皇帝只有登基之后第一次册立皇后,才能在坤宁宫举行大婚,否则便不能了。朕思来想去,祖宗规矩不能改,那么朕便许你一个民间的婚仪,明媒正娶一回。``

如懿直觉的一颗心温暖如春水,绵绵直欲化去:''虽然不是皇上亲自来迎娶臣妾,但能有此刻,臣妾已经心满意足。``

皇帝仔细端详她,温柔道:``寻常的皇后服制太过死板严肃,朕希望给你一夜美满,所以特意嘱咐内府制了这身衣裙,既有皇后服制的规制,也不失华美妩媚。朕希望朕亲自选定的皇后,可以与众不同。''

如懿温柔绵绵,如要化去:``即便只穿一夜,臣妾亦会珍藏。''

皇帝牵着她手并肩坐下,击掌两下,福珈和毓瑚便满面堆笑的进来,把皇帝的右衣襟压在如懿的左衣襟上。毓瑚端上备好的红玉酒盏,``请皇上皇后饮交杯之酒。''

如懿与皇帝相视一笑,取过酒盏互换饮下。许是喝得急了,如懿唇边滑落一滴轻绵酒水,皇帝以手擦去,温柔一笑。

福珈喜滋滋端过一盘子孙饽饽,屈膝道:``请皇上皇后用子孙饽饽。''

如懿取过银筷夹起吃了一口,连忙皱眉道:``哎呀,是生的!''

福珈笑得满脸皱纹都散开了:``千金难换皇后这句话呀!''

如懿这才回过味来,不觉脸上绯红,皇帝已笑得痴了,便也吃了一口道:``皇后说是生的,那自然是生的。''

福珈道:``交杯酒已经喝过,子孙饽饽也已经吃了,请皇上皇后听一听合婚歌吧。''她说罢,打开寝殿的长窗,窗外庭院中立着的四位年长的亲王福晋唱起了合婚歌。合婚歌共分三节,每唱一节后,左首的年长福晋即割肉一片掷向天,注酒一盅倾于地,以供神享,祝愿帝后和和美美。

终于曲终人亦散去,寝殿中亦安静了下来。

皇帝的眼中有如许情深,似要将如懿刻进自己的眼眸最深处:``如懿,这两天,朕虽然亲自下旨册封你为皇后,可也只有此时此刻,I与朕宁静相对,朕才觉得,你是真的成为朕的皇后了。''

如懿温婉侧首:``臣妾与皇上一样,如在梦中,此刻才觉美梦成真。''

皇帝轻轻握住如懿的手,低头吻了一吻,那掌心的暖意,便这样分分寸寸的蔓延上心来,一脉一脉暖了肌肤,融了心意。

皇帝执着她的手,声音低而沉稳,仿若青山唯一,岿然不动:``如懿,朕能许你天下女子中最至高无上的地位,却不能许你一心一意的夫妻安稳。哪怕从前,此刻,还是以后,朕都不能许你。这是朕对不住你的地方,亦是朕最不能给你的。''

如懿微微低下头,鎏金百合大鼎里有飘渺的香烟淡若薄雾,袅袅逸出。她从未曾发觉,那样轻的烟雾,也会有淡淡水墨般的影子,笼上人荫翳的心间。

这样的话,从前她不是不知,一路妻妾成群过来,她不能,也不敢期许什么。哪怕午夜梦回,孤身转醒的那一刻,曾经这样盼望过,也不敢当了真。可如今听他亲口这样说出来,哪怕是情理之中,意料之内,也生了几分失落。

她依偎在皇帝胸前,轻声道:''皇上说的,臣妾都明白,臣妾所祈求的,从来不是位份与尊荣。``

皇帝轻轻颔首,下颌抵在她光洁的眉心,仿佛叹息:''可是如懿,不管皇额娘是否反对,朕都会立你为皇后。或许皇后之位也不是最要紧的,朕能给你的,是朕心里的一份真心意。或许这份心意抵不上荣华富贵,权倾后宫来的实在,可是这是唯一能由着朕自己,不被人左右的东西。``

如懿心头震动,仿佛看着陌生人一般看着眼前这个相守相伴了十数年的男子,她不是不知道他的多疑他的反复,也不是不知道他身边从来有无数的姹紫嫣红。可是她深深的觉得,哪怕是陪在他身边最长久的时刻,也比不上着一颗内心的百感交集,倾尽真心。

他不过是弘历,她也只是青樱,是红尘万丈里最平凡不过的一对男女。没有雄心万丈,没有坐拥天下,更没有勾心斗角你死我活。只有一个男人和一个女人,这一刻的真心相许。

如意微微含泪,仅仅伏在他胸口,听着他心跳沉沉入耳,只是想,倾这一生,有这一刻,便也足够了。她这般凝神,伸手缓缓解下衣袍下一个金线绣芙蓉鸳鸯荷包。

她轻轻解开荷包,一样一样取出其间物什,呢喃低语:''这是臣妾嫁给皇上那日戴过的一双耳坠,这是皇上第一次写给臣妾的家书,这是臣妾在潜邸第一次生辰时皇上所赠的玉佩\ldots\ldots''她一一数了七八样,无一不爱惜珍重。

皇帝拈起一个薄薄的胭脂红纸包抖开,里头是两束发丝,一粗一细,各自用细巧红绳分别扎好,并排放着,显然是属于两个不同的人。皇帝的眼里忽然沁出星子般的光,冲口而出:``朕记得这个。这是你出嫁那夜,朕与你各自剪下一缕发丝作存,以待来日白首之时再见。你竟然还存着。''

浅笑的唇线牵动一弧梨涡浮现于如懿面上:``臣妾一直仔细保存,便是进冷宫前,亦交由海蓝保管。幸好,一直以来都未曾错失。''她有些不好意思,引过华彩映红的袍袖掩在唇际,``只是那年,臣妾嫁与皇上为侧福晋,所以这两束发丝可放在一处已是皇上格外垂怜,切不可行结发之仪。''

皇帝慨然微叹:``那年大婚,与朕能结发的唯有嫡妻,所以朕与琅华是结发之仪。''

这样美好的夜里,谈起故去的人,总有几分伤感。皇帝很快撇开这些情绪的浮缕,和声道:``不过今夜,你终于是朕的妻子了。''

一双明眸水光潋滟,如懿将手心之物`珍重存起,期许而感慨:``臣妾左思右想,皇上为了今日费劲心思博臣妾欢愉之心,臣妾所有皆是为皇上所赐,无以为报,只能将旧年岁月里值得珍惜之物一一保存妥帖,以表臣妾之心。''

皇帝的眼里是满满的感动:``谁说你无以为报?这两根头发不能结也罢了''他手指轻滑,滑至她发髻后拨出细细一缕,取过紫檀台上的小银剪子,又缕出自己辫梢一缕一并剪下,对着灼灼明火用一根红绳仔细结好,放入胭脂红纸中一并叠好,``那是从前的不够完美,这是今夜结发往后,一并存起''。

如懿怔怔地看着,有泪水轻轻溢上眼睫,她只是一味垂首,摇头道:``皇上不可,少年结缡,原配夫妻才可结发,臣妾不是。

皇帝将温柔眸光深深凝住:``朕知道你不是原配,结发之礼不是相宜,所以只取结发为夫妻,恩爱两不疑``之意。''

莫名的情绪泛着巨大的甜蜜,和那甜蜜里的一丝酸楚,她无言,只能感受着泪水的润与热,与她的心潮一般,温柔的汹涌,喃喃细语:''结发与君知,相要以终老,满人不可轻易剪发,皇上是为了臣妾,臣妾都知道。``

他且行且笑:''是了。满人头发珍贵,若无决绝之事,不可断发,否则形同悖逆。可今夜朕与你,是欢喜之事。``他缓身行至攒枝金线合欢花粟玉枕边,俯身取出一个浮雕象牙锦匣,打开莲瓣宝珠金钮,里头薄薄一方丝帕,只绣了几只殷红荔枝,并几朵淡青色的樱花。他叹道:''青樱,弘历,并存于此,便是你最好的回报。''他亲吻她眉心,温柔的如同栖落花瓣的蝶,``你出冷宫之后,朕告诉过你,希望和你长长久久的走下去。如懿,如今你是朕的妻子,生同寝,死同穴,会一直一直永永远远和朕在一起了。''

她无言已对,唯有以感动的朦胧泪眼相望,还报情深,低低吟道:``愿我如星君如月,夜夜流光相皎洁。皇上说过的话,臣妾都记得。''她垂首,略有几分无奈,却终究仰望着他,切切道:``臣妾知道,往昔来日,臣妾择不尽皇上身边的人。臣妾所求,唯有一句。''

皇帝拥着她问道:``什么?''

她郑重而恳切:``臣妾不敢求皇上一心,但求此生长久,不相欺,不相负。不管去到何处,皇上总是信臣妾的,便如臣妾信皇上一般。''

皇帝亦沉沉慨然:``如懿,此生长久,不相欺,不相负。君无戏言,这个君,既是天子君王,也是你枕畔夫君。''

如懿有说不出的感动,一颗心向北浪潮裹袭着,退却又卷近,唯有巨大的喜悦与温情将她密密匝匝包裹,让她去释怀,去原谅,去遗忘。

皇帝的吻落下来,那是一对经年夫妻的轻车熟路,彼此熟知。她以温柔的低吟浅唱相应,看着红罗帐软肆意覆落,轻轻地闭上了眼睛。

唯余龙凤花烛,虹影双双,照彻一室旖旎。

殿中的烛火越来越暗,终于只剩了一双花烛如双如对的影子,守夜的太监在廊下打开了蒲团和被铺守着,李玉打了个哈欠道:``皇上和皇后都睡下了,你们也都散了吧。''便有小太监将檐下悬挂的水红绢纱灯摘下了一半,守在养心殿外的是为也散去了两列。凌云彻亦在其中。

李玉拱手道:``这一日辛苦了。凌大人早些回去歇息吧。''

凌云彻道:``哪里比得上李公公的辛劳,皇上大婚,一刻也离不开您上上下下打点着。''二人寒暄罢,便也各自散了。

八月初的天气,即便是夜深,也有些许残留的署意。这几日的喧闹下来,此刻只觉得紫禁城中安宁的恍若无人之境。凌云彻说不出自己此刻的心情是喜是愁,倒像是汪着一腔子冰冷的月光倒在了心里,似乎是分明的照着什么,却又是稀里糊涂的。

他这样想着,脚也不知迈去了哪里,并非是自己平日休息起居的侍卫房,抬头一看,却是到了坤宁宫。他想了想,左右赵九霄也在这里当差,便进去他所住的庑房。赵九霄见了他来十分欢喜,二人倒了一杯酒,拨了几个菜,相对而饮。赵九霄拿胳膊撞了撞他,道:``你在皇上跟前挺得器重的,今儿又是皇上大喜的日子,你怎么不高兴?是不是看着皇上娶亲,自己也想娶亲了?''

凌云彻笑道:``你自己这样想罢,别扯上我。''

赵九霄搓着手道:``你还别说,我倒真为了一个姑娘朝思暮想呢!''

凌云彻好奇:``谁?是宫里的宫女吗?''

赵九霄凑近了道:``就是令嫔娘娘宫里的澜翠,那模样那身段儿,我\ldots\ldots{}''

凌云彻横了他一眼,道:``别人也就罢了,要是永寿宫,想都别想。''

赵九霄啧啧道:'你这个人也太小心眼儿了。人望高处走嘛,也不能都说她不对,你就这么嫉恨令嫔娘娘?``

凌云彻冷冷不言,赵九霄也无趣了:''弄了半天,你不高兴也不是为了令嫔娘娘?我还当皇上立后,你是心疼她被冷落了呢。``

凌云彻喝了几大杯酒,那是关外的烧刀子,入口烫喉,一阵阵热到肠子里,却也容易上头。他有些昏昏沉沉:''皇后?你以为立了皇后就好么?从前的孝贤皇后出身名门,还不是活的战战兢兢?我是心疼,心疼坐到这个位子上的人会受苦。``

赵九霄也有些晕了,往他胸口戳了一拳,道:''谁的婆娘谁心疼!你心疼个什么劲儿?这个年纪了,也不成个家,孤零零的什么意思?``

凌云彻按着自己的胸口:`我也不知道,孤零零的是为了什么;我更不知道,她是什么时候在我心里落了个影儿。这么个只能远不能近的影儿。她伤心的时候我只能远远的看着她。可是她的伤心,我都明白。如今见她好,我自然高兴,可是高兴了还是担心她来日还会遇到什么。''

赵九霄吃了筷牛肉,伏在桌上昏昏沉沉道:``你看,你看,你还想着令嫔娘娘不是?''

凌云彻苦笑了一刻,仰起头,把酒浇入了喉中。任由酒气杀烈,弥漫心间。

福珈回到慈宁宫时已是夜深,她悄然入内,却见阁内灯火通明,太后托腮凝神,双眼微闭,听得她来,太后只是轻声询问:``回来了?''

福珈吃了一惊,忙道:``太后怎么还不安置?时辰不早了。''

太后淡淡一笑,睁开眼道:``知道,只是喧闹了这两日,总觉得喜悦声还聒噪在耳边,嗡嗡的,让人不想睡。''

福珈忙道:``那奴婢去点安神香吧。''

太后摆了摆手,直起身,道:``人老了就是心事多,不容易睡着。你陪哀家说说话。''

福珈应了声``是'',在太后膝边坐下。太后出神片刻,似是自言自语:``养心殿那儿都好了?''

福珈嘴角不觉多了一丝笑意:'都好了,这个时辰,怕已经安置了。洞房花烛,皇上对皇后真是有心了。``

太后颔首道:''皇帝肯用心,真是难得。``她的目光落在远处空茫茫的一点,隐隐多了一丝沉溺的微笑,''肯被人这样用心相待,又能用心待之,真好,乌拉那拉如懿到底是有福的。''

如懿睡在皇帝身侧,一夜都做着繁迷的梦。梦里,有皇帝的执手相看两不厌,有琅华的泪眼哀怨,亦有云彻与海蓝的相伴在侧。但是梦见最多的,居然是姑母唇边不退的微笑。姑母穿着与自己一样的皇后冠服,神色悲喜交加,更是欣慰。那声音似远忽近,是姑母的叮嘱:``乌拉那拉氏不可出废后!如懿,乌拉那拉氏不能再有弃妇了。''

她终于松一口气,原来只与自己有数面之缘的姑母,是那样深刻的活在自己的记忆里,又深远的影响着今时今日的自己。

她从梦中醒来,隐隐觉得夜凉如水,似游弋浮动在身侧。皇帝仍在熟睡,眉心带着舒展的笑意,大约是个好梦。她披衣坐起,才发觉寝殿的窗扇不知何时已微微开了一隙,凉风徐徐穿入。她正要起身关窗,忽然周身的血液一凉,竟呆住了。

案几上所供的龙凤花烛,不知何时,那支凤烛上的火焰依然湮灭,只余一卷烧焦了的烛心,映着累累烛泪,似一只流泪至盲的眼睛。``

心中的恐惧骤然冰裂灌入,不是没有听说过,龙凤花烛要在大婚之夜亮至天明,若有一只先灭,便是夫妻中有一人早亡,或是半路分折恩爱断绝。民间传闻虽然有些无稽,谁能保证夫妻能白首到老,又同年同月逝去,只是这样夜半熄灭一支,却也实在不吉。

她回头见皇帝犹自沉睡,忙关上了窗扇,又仔细检查一遍无碍,重新点燃了凤烛。做完这一切,她才觉得自己的双手有些发抖。

原来她还是怕的,是那样怕,怕夫妻恩情中道断绝。如懿回到皇帝身边,紧紧依在他身侧,仿佛只有他的温热才能提醒着自己一切的美好才刚刚开始。

\hypertarget{ux7b2cux5341ux7ae0-ux7a7fux8033}{%
\chapter{第十章 穿耳}\label{ux7b2cux5341ux7ae0-ux7a7fux8033}}

这样思虑,再度入梦便有些艰难。蒙蒙眬眬中,便已天色微明。皇帝照例要去早朝,嘱咐她起身后再休息片刻。如懿想着今日是嫔妃陛见的日子,也随着皇帝起身,一同穿戴整齐,含笑送了皇帝出门,亦回自己宫中去。

金玉妍自九阿哥夭折后脾气越发不大好。皇帝看在她丧子之痛,着意安慰,又再立后次日重新复她贵妃之位以示恩遇,沉寂多时之后,她也终算扬眉了。

这一日是立后之后嫔妃第一次合宫拜见。如懿不愿摆足新后的架子,便按着时辰在翊坤宫与嫔妃们相见,倒是众人矜守身份,越发早便候在了宫中。

因着是正日,如懿换了一身正红色龙凤勾莲暗花纱氅衣,发髻上多以纯金为饰,夹杂红宝,喜庆中不失华贵雍容。

彼时嘉贵妃玉妍与纯贵妃绿筠分列左右首的位置,绿筠下首为愉妃海兰、令嫔嬿婉、婉嫔婉茵、庆贵人缨络、秀常在,玉妍之下为舒妃意欢、玫嫔蕊姬、晋贵人、平常在、揆常在及几个末位的答应。为免妨皇后正红之色,嫔妃们多穿湖蓝、罗翠、银珠、淡粉、霞紫,颜色明丽,绣色繁复娇艳,却不敢有一人与如懿的穿戴相近,便是嫔妃中位列第一的苏绿筠,也不过是一身桔色七宝绣芍药玉堂春色氅衣,配着翠绿银丝嵌宝石福寿绵长佃子,有陪同着喜悦的得体,也是谦逊的退让。

嫔妃之中,唯有新复位的玉妍一身胭脂红缀绣八团簇牡丹氅衣,青云华髻上缀着点满满翠镶珊瑚金菱花并一对祥云镶金串珠石榴石凤尾簪,明艳华贵,直逼如懿。

如懿心中不悦,却也不看她,只对着绿筠和颜悦色:``本宫新得了乌木红珊瑚笔架一座,白玉笔领一双,想着永瑢正学书法,等下你带去便好。''

绿筠见如懿关爱自己儿子,最是欢喜不过,忙起身谢道:``皇后娘娘新喜,还顾念着臣妾的孩子,臣妾真是感激不尽。``说罢便向着玉妍道:''嘉贵妃复位,又贺皇后娘娘正宫中位之喜,难得打扮得这样娇艳,咱们看着也欢喜。''

嬿婉温婉道:``臣妾等侍奉皇后娘娘,穿的再好看也不是为了自己,只是薄皇后娘娘一笑罢了。能让皇后娘娘高兴,也不枉嘉贵妃穿了这么一身颜色衣裳。好赖都是讨主子娘娘欢喜罢了。''

玉妍的笑冷艳幽异:``令嫔一心想着讨好主子娘娘,本宫倒是巧合,只不过惦记着皇上说过,喜欢本宫穿红色而已。''

嬿婉有些窘迫,掩饰着取了一枚樱桃吃了,倒是海兰笑道:``皇上与皇后娘娘本是夫妻一体,嘉贵妃记得皇上,便是记得皇后娘娘了。''

玉妍见如懿端坐其上,慢慢合着青花洞石花卉茶盅的盖子,热气氤氲蒙上她姣美的脸:``皇后是新后,翊坤宫却是旧殿。臣妾记得当时皇上把翊坤宫上次给还是娴妃的皇后娘娘居住,便是取翊为辅佐之意,请娘娘辅佐坤宁,原是副使的意思,怎么如今成了中宫之主,娘娘住的还是辅佐之殿呢?''

这话问得极犀利。如懿想起封后之前,皇帝原也提起过换个宫殿居住,但东西六宫中,只有长春宫、威福宫、承乾宫和景仁宫不曾有人居住。长春宫供奉着孝贤皇后的遗物;威福宫乃是慧贤皇贵妃的旧居,慧贤皇贵妃死后便空置着;景仁宫,如懿只消稍稍一想,便会想起她可怜的姑母,幽怨而死的姑母,如何再肯居住。皇帝倒也说起,承乾宫意为上承乾坤,历来为后宫最受宠的女子所居住,顺治帝的孝献皇后董鄂氏便是,但年久失修,总得修一修才能让如懿居住。只是,这样的话何必要对她金玉妍解释。

如懿便只是浅笑不语,不去理会。嬿婉抿起唇角轻笑,纤细的手抬起粉彩绣荷叶田田的袍袖掩在唇际,带着一丝讥诮的眸光潋滟,拨着耳上翠绿的水玉滴坠子,柔柔道:``皇后便是皇后,名正言顺的六宫之主,不拘住在哪里。都是皇上的正妻,咱们的主子娘娘。''

玉妍笑意幽微,微微侧首,满头珠翠,便曳过星灿似的光芒,晃着人的眼:``主子娘娘倒都是主子娘娘,但正妻嘛''她的身体微微前倾,对着绿筠道:``纯贵妃出身汉军旗,自然知道民间有这么个说法吧?续弦是不是?还是填房,继妻?''她甩起手里的打乌金络子杏色手绢,笑道:``到底是续娶的妻子,是和嫡妻不一样的吧?''

这话,确是刻薄了。绿筠一时也不敢接话,只是转头讪讪和意欢说了句什么,掩饰了过去。

有那么一瞬间的沉吟,如懿想起了她的姑母,幽怨绝望而死的景仁宫皇后,或许,她生前也是一样在意吧?在意她的身份,永远是次于人后的继后,如懿忽然微笑出来,坦然而笃定。其实,有什么要紧?真的,在这个位置的唯一的人,才是最重要的人,之前之后,都只是虚妄而已。

如懿侧脸,召唤容珮:``去将本宫备下给纯贵妃与嘉贵妃的耳环呈上来。''

容珮答应了一声,立刻从小宫女手中接过了一个水曲木镂牡丹穿风长盘,上面搁着两只粉红色织锦缎圆盒。她利落打开,按着位序先送到绿筠面前,那是一对玛瑙穿明珠玉珏耳环,颜色大方又不失明亮,极适合绿筠的年纪与身份。绿筠忙起身谢过:``多谢皇后娘娘赏赐。''

如懿淡淡含笑:``等下还有三把玉如意,你带回去给三阿哥、六阿哥和四公主,也是本宫的一点儿心意。''

绿筠再次谢过,神色恭谨。容珮又将另一对耳环送到玉妍面前,如懿温然含笑:``这一对耳环与纯贵妃那对不同,专是为你选的。嘉贵妃应该会喜欢吧?''

玉妍只瞟了一眼,矍然变色,如懿恍若未见,如常道:``给嘉贵妃的这一对是红玉髓的耳环,配着七宝中所用的松石和珊瑚点缀,在最末垂下拇指大的雕花金珠,颜色明丽,很适合嘉贵妃这样亮烈妩媚的性子,只是,红玉髓到底不如玛瑙名贵,那也是没办法的,纯贵妃到底资历深厚,儿女双全,自然是在嘉贵妃之上了。''

这话,既是褒奖绿筠众妃之首的超然地位,稳了她永璜和永璋被贬斥后惶惑不安的心思,亦是提点着玉妍当日一图用七宝手串暗害她的事。前因后果,她都记得分明。

玉妍果然有些失色,脸色微微发白,并无意愿去接那对耳环。

如懿的脸色稍稍沉下,如秋日阴翳下的湖面:``怎么?嘉贵妃不愿接受本宫的心意么?''

绿筠到底还乖觉,忙摘下自己耳垂上的碧玺琉璃叶水晶耳坠,将如懿赏赐的耳环戴上,起身道:``皇后娘娘赏赐,臣妾铭记于心,此刻便戴上,以表对娘娘尊敬。''

如懿满意地颔首,平静目视玉妍,玉妍勉强道:``谢过皇后,臣妾回去自会戴上。''

嬿婉轻笑,脆生生道:``这是咱们第一日拜见皇后娘娘,嘉贵妃若有心,此刻戴上便是了,何必分回去不回去?再说了,怎么回去不都是在皇后娘娘所辖的六宫里。''

意欢素来不喜玉妍,侧目道:``嘉贵妃不喜欢便是不喜欢,何必伪作托词,可见为人不实。''

婉茵亦劝:``嘉贵妃,皇后娘娘赏赐的耳环极好看,也便只有你和纯贵妃有,咱们羡慕都羡慕不来呢。''

玉妍只得伸手掂了掂耳坠,勉强道:``皇后娘娘可真实诚,这么大的金珠子,想必是实心的吧,臣妾戴着只怕耳朵疼呢,昔年孝贤皇后在时,最忌奢侈华丽,这么华贵的耳坠,臣妾实在不敢受。''

这一来,已经戴上耳环的绿筠不免尴尬,还是海兰笑道:``孝贤皇后节俭,那是因为皇上才登基,万事草创。如今皇上是太平富贵天子,富有四海,便是贵妃戴一双华贵些的耳环怎么了,只怕皇上瞧见了更欢喜呢。''

玉妍仔细看那耳坠,穿孔的针原是银针做的,头上比寻常的耳坠弯针尖些,针身却粗了两倍不只,便道:``这耳针这么粗,臣妾耳洞细小,怕是穿不过的。''

如懿不欲与她多言,扬了扬下巴,容珮会意,便道:``戴耳坠原不是嘉贵妃娘娘的事,穿不穿的进是奴婢的本事,肯不肯让奴婢穿便是嘉贵妃自己的心意。''

如懿微微斜过身子,拨弄着身旁一大捧新折的深红芙蓉,笑吟吟道:``嘉贵妃自然知道本宫为何要赏你红玉髓耳坠。本宫的心思,你明白就好,若是说穿了,你这个贵妃之位复位男的,别再轻易丢了。''

玉妍满脸恼怒,到底也不敢发作,只得低下了头对着容珮厉色道:``仔细你的爪子,别弄伤了本宫。''

容珮答应一声,摘下玉妍原本的耳环,不管三七二十一,对着她的耳孔便硬生生扎了下去,那耳针尖锐,触到皮肉一阵刺痛,很快被粗粗的针身阻住,怎么也穿不进去。容珮才不理会,硬生生还是往里穿,好像那不是人的皮肉耳洞似的。玉妍起先还稍稍隐忍,后来实在吃痛,转头喝道:``不是教你仔细些了么?你那手爪子是什么做的,还不快给本宫松下来!''

容珮面无表情,手上却不肯松劲儿,只板着脸道:``不是奴婢不当心,是奴婢的手不当心,认不得人。当初嘉贵妃把惢心姑姑送进慎刑司,自己可没做什么,可慎刑司那些奴才不就是嘉贵妃您的手爪子么,您的手爪子遂不遂您的心奴婢不知道,可现在奴婢的手爪子不听自己使唤了,非要钻您的耳朵,您说怎么办呢?''

玉妍又惊又怒,痛得脸孔微微扭曲:``皇后娘娘!你就这么纵容你的奴婢欺凌臣妾么?''

如懿含笑不语,似乎只是看着一场有趣的笑剧,吩咐道:``惢心,给各位小主添些茶点。你的腿脚不好,慢慢走吧,不必着急。''

玉妍见如懿如此,愈加惊恼:``惢心的腿坏了,是慎刑司的人下手太重,皇上也已经贬斥过臣妾。如今臣妾复位,那是皇上不计较了。皇上都不计较,皇后还敢计较么?''

如懿看着她,和煦如春风:``皇上不计较是皇上仁慈,本宫不计较是与皇上同心一体,所以,本宫眼下是赏赐你,而不是惩罚你,你可别会错了意。''

容珮冷着脸道:``嘉贵妃,耳针已经穿进去了,您要再这么挣扎乱动,可别怪自己不当心伤了自己的耳朵。再说了,您规规矩矩一些,奴婢立刻就穿过去了,您也少受些罪不是?''

玉妍恨得双眼通红:``皇后娘娘,您是拿着赏赐来报自己的私仇!臣妾不服!''

如懿笑得从容淡然:``你从来都是不服的,也不是这一日两日了。而且,本宫大可明明白白告诉你,不是本宫要报自己的私仇,而是你承担自己做过的事!所以对你,赏也是罚,罚也是赏!''

嬿婉伸着柔若无骨的指,缓缓地剥着一枚枇杷:``皇后娘娘已经是足够宽宏大量了。身为嫔妃,对着皇后娘娘你呀你的,敬语也不用,还敢撩了皇后娘娘的颜色。说白了,嘉贵妃再尊贵,再远道而来,还不是和咱们一样,都是妾罢了。我倒是听说,在李朝遵守儒法,妾室永远是正室的奴婢,妾室所生的孩子永远是正室孩子的奴婢。怎么到了这儿,嘉贵妃就忘了训导,尊卑不分了呢?若是皇上知道,大约也会很后悔那么早就复位您的贵妃之位了。这么不懂事,可不是辜负了皇上的一片苦心么?''

玉妍听得``皇上''二字,到底也不敢再多争辩,只得红了眼睛,死死咬牙忍住。容珮下手毫不容情,仿佛那只是一块切下来挂在钩子上的五花肉,不知疼痛、不知冷热的,举了耳针就拼命钻。玉妍痛得流下泪来,她真觉得这对耳垂不是自己的了。这么多年来养尊处优,每夜每夜用雪白的萃取了花汁的珍珠粉扑着身子的每一寸,把每一分肌理都养得嫩如羊脂,如何能受得起这般折腾。可是,她望向身边的每一个人,便是最胆小善良的婉茵,也只是低垂了脸不敢看她。而其他人,都是那样冷漠,只顾着自己说说笑笑,偶尔看她一眼,亦像是在看一个笑话。

玉妍狠狠地咬住了唇,原来在这深宫里,她位分再高,皇子再多,终究也不过是一个异类而已。

也不知过了多久,容珮终于替玉妍穿上了耳坠,那赤纯的的金珠子闪耀无比,带着她耳垂上滴下的血珠子,越发夺目。容珮的指尖亦沾着腥红的血点子,她毫不在乎的神情让人忘记了那是新鲜的人血,而觉得是胭脂或是别的什么。倒是玉妍雪白的耳垂上,那过于重的耳坠撕扯着她破裂的耳洞,流下两道鲜红的痕迹,滴答滴答,融进了新后宫中厚密的地毯。

有须臾的安静,所有人被这一刻悲怒而绮艳的画面怔住。

如懿面对玉妍的怒意与不甘,亦只沉着微笑。她忽然想起遥远的记忆里,她偶然去景仁宫看望自己的皇后姑母,在调理完嫔妃之后,踌躇满志的姑母对她漫不经心地说:``皇后最要紧的是无为而治,你可以什么都想做,但若什么都亲手做,便落了下乘了。要紧的,是借别人的手,做自己想做的事。''

如懿知道,此时此刻的自己早已违背了姑母的这一条禁忌。但,她是痛快的。此刻的痛快最要紧,何况作为新任的皇后,自己从妃妾的地位一步步艰难上来,她懂得要如何宽严并济,所以平抚了苏绿筠,弹压了金玉妍。

如懿笑意吟吟地打量着玉妍带血的艳丽耳垂,那种鲜红的颜色,让她纾解了些许惢心残废的心痛和自己被诬私通的屈辱。她含笑道:``真好看!不过,痛么?''

玉妍分明是恨极了,却失了方才那种嚣张凌厉,有些怯怯道:``当然痛。''

如懿笑着弹了弹金镶玉的护甲:``痛就好。痛过,才记得教训!起来坐吧。''

玉妍身边的丽心吓得发怔,听得如懿吩咐才回过神来,畏怯地扶了玉妍起身坐下。

意欢瞟了眼丽心,语气冷若秋霜:``你可得好好儿伺候嘉贵妃,别和贞淑似的,一个不慎被送回了李朝、贞淑有李朝可回,你可没有!''

丽心吓得战战兢兢,哪里还敢作声。

容珮见玉妍脸色还存了几分怒意,便板着面孔冷冷道:``嘉贵妃的眼泪珠子太珍贵,要流别流在奴婢面前,在奴婢眼里,那和屋檐上底下的脏水没分别!但您若要把您的泪珠子甩到皇上跟前去,奴婢便也当着各位小主的面回清楚了。皇后娘娘给的是赏赐,是奴婢给您戴上的,要有伤着碰着,您尽管冲着奴婢来,奴婢没有一句二话。但若您要把脏水往皇后娘娘身泼,那么您就歇了这份心吧,所有的小主都看着呢,您是自己也愿意承受的。不为别的,只为您自己做了亏心事,那是该受着的。''

众嫔妃何等会察言观色,忙随着为首的绿筠起身道:``是。臣妾们眼见耳闻,绝非皇后娘娘之责。''

如懿和颜悦色,笑对众人:``容珮,把本宫备下的礼物赏给各宫吧。''

如是,嫔妃们又陪着如懿说笑了一会儿,便也散了。

到了晚间时分,皇帝早早便过来陪如懿用膳。如懿站在回廊下,遥遥望见了皇帝便笑:``皇上来得好早,便是怪臣妾还没有备好晚膳呢。''

惢心俏皮道:``可不是!皇上来得急,皇后娘娘亲自给备下的云片火腿煨紫鸡才滚了一遭,还喝不得呢。''

皇帝挽过如懿的手,极是亲密无间:``别行礼了,动静又是一身汗。''他朝着惢心笑道:``不拘吃什么,朕批完了折子,只是想早些来陪皇后坐坐。''

如懿笑道:``皇上说不拘吃什么就好,有刚凉下的冰糖百合莲子羹,皇上可要尝尝么?''

皇帝眼底的清澈几乎能映出如懿含笑的仿佛正在盛放的莲一般的面容:``自然好,百合百合,百年合欢,是好意头。''

如懿婉然睨他一眼:``一碗羹而已,能得皇上这样的念想,已是它的福气了。''

惢心顷刻便端了百合莲子羹来,又奉上一碗冰碗给如懿。那冰碗是宫中解暑的佳品,用鲜藕切片,鲜菱角去皮切成小丁块,莲子水泡后去掉皮和莲心,加清水蒸熟,再放入切好的蜜瓜、鲜桃和西瓜置于荷叶之上,放入冰块冰镇待用。这般清甜,如懿亦十分喜欢。

如懿才舀了一口,皇帝便伸手过来抢了她手中银勺:``欸,看你吃得香甜,原来和朕的不一样。''说着便就着如懿用过的银勺吃了一口,叹道,:``好甜!''

如懿奇道:``臣妾并不十分喜甜,所以这冰碗里不会加许多糖啊''

皇帝便道:``不信,你自己再尝尝。''

如懿又尝了一口,道:``皇上果然诳臣妾呢。''

皇帝忍不住笑了,凑到她耳边低低道:``是朕自己心里觉得甜。''

如懿笑着嗔了皇帝一眼,啐道:``皇上惯会油嘴滑舌。''

皇帝眉梢眼角皆是笑意:``油嘴滑舌?也要看那个人值不值得朕油嘴滑舌啊。''他陪着如懿用完点心,话锋骤然一转,``对了,方才嘉贵妃来养心殿见朕,哭哭啼啼的,耳垂也弄伤了。是怎么了?''

长长的睫毛如寒鸦的飞翅,如懿羽睫低垂,暗自冷笑,金玉妍果然是耐不住性子去了。她抬起眼,看着皇帝的眼睛笑意盈盈道:``是是非非,皇上也已经听嘉贵妃自己哭诉了一遍,臣妾便是不饶舌了。''

皇帝慢慢舀了一颗莲子在银勺里:``她说的话自然是维护她自己的,朕想听听你的说辞。''

如懿不假思索道:``后宫是归臣妾的,更是归皇上的。臣妾不会蓄意惹是生非。''

皇帝粲然一笑,眉毛一根根舒展开来:``有你这句话,朕便放心了。其实你不说朕也知道。嘉贵妃刚刚复位,难免有些桀骜,从哪里争口气来恢复自己往日的尊荣,挣回些面子。你初登后位,若不稍加弹压,往后也的确难以压制''

如懿低眉颔首,十分温婉:``皇上说得是,嘉贵妃出身李朝,本该格外优容。可是前两日臣妾见到和敬公主,深觉公主有句话讲得极是。''

皇帝饶有兴味,笑道:``和敬嫁为人妇,如今也不再任性。她说出什么话来,叫朕听听。''

如懿拨着手里的钥匙,轻轻笑道:``公主说,享得住泼天的富贵,也要受得住来日弥天的大祸。''

皇帝轩眉一挑,显是不豫:``前两日是朕的立后大典,她说这般话,是何用心?''

如懿知他不悦,浅浅笑道:``公主这句话放诸六宫皆准,臣妾觉得倒也不差。皇上开恩垂爱,嘉贵妃便更应谨言慎行,不要再犯昔日之错。''

皇帝摆手,温言道:``嘉贵妃之事你已经处置了便好。和敬她到底已经出嫁,你也不必多理会。对了,再过几日便是朕的万寿节。朕想来想去,有一样东西要送与你。''

描绘得精致的远山黛眉轻逸扬起,如懿笑道:``这便奇了。皇上的生辰,该是臣妾送上贺礼才是,怎么皇上却倒过来了?''

皇帝握住她的手,眼中有绵密情意:``朕今日往漱芳斋过,想起你在冷宫居住数年,苦不堪言,而同住的女子,多半也是先帝遗妃。所以,朕已经下了旨意,将这些女子尽数遣往热河行宫,择一处僻静之处养老,不要再活得这般苦不堪言。''

有轻微的震动涌过心泉,好像是冰封的泉面地下有温热的泉水潺潺涌动,如懿似乎不敢相信,轻声道:``皇上的意思是''

``朕不想宫中再有冷宫了。''皇帝执着如懿的手郑重道,``没有冷宫,是朕要宫中夫妻一心,再无情绝相弃之时''

心中的温热终于破冰而出,如懿回望着皇帝,笑意温柔:``皇上情意深重,六宫同沐恩泽。''

殿中清凉如许,如懿只觉得心中温暖。只是在那温暖之中,亦有一丝不合时宜的惆怅涌过。其实,冷宫也不过是一座宫殿,若有朝一日皇恩断绝,哪怕身处富贵锦绣之地,何尝不是身在冷宫,凄苦无依呢?

只是这样的话,太过不吉。她不会问,亦不肯问。只静默地伏在皇帝肩头,劝住自己安享这一刻的沉静与温柔。

\hypertarget{ux7b2cux5341ux4e00ux7ae0-ux6bcdux5bb6}{%
\chapter{第十一章 母家}\label{ux7b2cux5341ux4e00ux7ae0-ux6bcdux5bb6}}

封后之后,如懿的父亲那尔布被追尊为一等承恩公,母亲亦成为承恩功夫人,在如懿册封为后的第五日,入宫探望。

一家团聚,如懿自然是喜不自胜。从前为贵妃、皇贵妃之时,母亲也不是没来探望过,但那时谨言慎行、战战兢兢,到底比不上此刻的舒展畅意。

如此一家子絮絮而言,母亲说得最多的一句,便是``乌拉那拉氏中兴,你阿玛在九泉之下亦可瞑目了''。这样的话在喜庆时节听来格外招人落泪,如懿适时地阻止了母亲的喜极而泣,再论起来,便是小妹的嫁龄已经到,求婚的人家都踏破了门槛。

如懿沉吟道:``从前无人问津,如今踏破门槛,不过是因为女儿这皇后之位。可见世人多势利!''

母亲便道:``若论势利也总是有的。额娘冷眼瞧着,来求婚的人家里头,有皇上的亲弟弟和亲王来求娶侧福晋的,还有便是平郡王来求娶福晋,赵国公为他家公子------''

母亲的话尚未说完,如懿便连连摆手:``额娘别再说这个,皇上嘴上不说,心里却是最忌讳与皇室或重臣多沾染的。咱们和皇家的牵扯还不够么?若要女儿说,在从前相熟不嫌弃咱们落魄的人家里选一个文士公子,便是最安稳了。武将要出征沙场,文士才子便好,还得是不求谋取功名的,安安稳稳一生便了。''

母亲迟疑片刻,摇头道:``咱们这样的人家,好容易兴旺了,便嫁与这样的人,便是你妹妹甘心,我也不能甘心呀!''

如懿道:``额娘万勿糊涂。富贵浮云,有女儿一个在里头便是了,妹妹便清清静静嫁给有情人的好,连弟弟,以后也是承袭爵位便好,不要沾染到官场里头来。''

如此郑重其事地嘱咐,母亲终于也应允了。

母亲离去时已是黄昏时分,晨昏定省的时刻快到,嬿婉候在翊坤宫外,看着如懿亲自将母亲搀扶到门外,不觉微湿了眼眶,低低道:``春婵,也不知本宫的额娘在家如何了,有心要见一见,可本宫到底不算是得宠的嫔妃,家中又无人在朝为官,想见一面也不能够。''

春婵好生安慰道:``小主想见家人又有什么难的,您与皇后娘娘常有来往,请皇后娘娘的恩典便是了。''

嬿婉迟疑:``也不知皇后娘娘肯不肯?''

春婵笑道:``嘉贵妃的事小主是出了力的,皇后娘娘自然会疼小主呢。而且,皇后娘娘刚被册封,自然是肯施恩惠下的。''

嬿婉想了想,果然去求了如懿,如懿亦允准了,慨叹道:``你家人原在盛京,本宫让人早些准备下去,好接你家人入宫探视。''

嬿婉的母亲和弟弟便是在十来日后入宫的,那一日晨起,嬿婉便吩咐备下了母亲和弟弟喜爱的点心,又将永寿宫里里外外都打扫了一遍,更换了重罗新衣,打扮得格外珠翠琳琅,只候着家里人到来。

果然,到了午后时分,如懿身边的三宝已经带着嬿婉的母亲和弟弟入内,打了个千儿便告退了。

嬿婉多年未见母弟,一时情动,忍不住落泪,伏在母亲怀中道:``额娘,弟弟,你们总算来了。''

魏夫人仔仔细细打量着永寿宫的布置,又推开怀中的女儿上上下下看了一遍,方郑重了神色问道:``小主可有喜了么?''

嬿婉满心感泣,冷不防母亲问出这句来,不觉怔住。还是澜翠乖觉,忙道:``魏夫人和公子一路上辛苦了,赶紧进暖阁坐吧,小主都备下了两位最喜爱的点心呢。''

魏夫人不过四十多岁,穿着一身烟灰红的丝绸袍子,打扮得倒也精神。而嬿婉的弟弟虽然身子壮健,但一身锦袍穿在身上怎么看着都别扭,只一双眼睛滴溜溜打量着周围,没个定性。魏夫人虽然看着有些显老,但一双眼睛十分精刮,像刀片子似的往澜翠身上一扫,道:``你是伺候令嫔的?''

澜翠忙答了``是'',魏夫人才肯伸出手,由着她搀扶进去了。

到了暖阁中坐下,澜翠和春婵忙将茶点一样一样恭敬奉上,便垂手退在一边。魏夫人尝了几样,看嬿婉的弟弟佐禄只管自己狼吞虎咽,也不理会,倒是澜翠递上了一盏牛乳茶过去,道:``公子,喝口茶润润吧,仔细噎着。''

佐禄不过十六七岁,看着澜翠生得娇丽,伺候又殷勤,忍不住在她手背上摸了一把,涎着脸笑道:``好滑。''

澜翠自幼在宫里当差,哪里见过这般不懂规矩的人,一时便有些着恼,只是不敢露出来,只得悻悻退到后头,委屈得满脸通红。

嬿婉脸上挂不住,忙喝道:``这是宫里,你当是哪儿呢?''

佐禄便垂下脸,抓了一块点心咬着,轻轻哼了一声。

魏夫人什么都落在了眼里,便沉下脸道:``左不过是伺候你的奴才,也就是伺候你弟弟的奴才,摸一把便摸一把,能少了块肉怎的。''嬿婉一向视澜翠与春婵作左膀右臂,听母亲这般说,只怕澜翠脸皮薄生了恼意,再要笼络也难了,便嘱咐道:``澜翠,你出去伺候。''

魏夫人立刻拦下,也不顾澜翠窘迫,张嘴便道:``出去做什么?当奴才的,这些话难道也听不得了?''她见嬿婉紫涨了脸,也不顾及,只盯着嬿婉的肚子道:``方才我看小主你吃那些甜食吃得津津有味,偏不爱吃那些酸梅辣姜丝儿,怕是肚子里还没有货搁着吧?''

嬿婉听她母亲说得粗俗,原有十分好强之心,此刻也被挫磨得没了,急得眼圈发红道:``额娘,这命里时候还没到的事,女儿急也急不来啊。''

魏夫人嘴角一垂,冷下脸道:``急不来?还是你自己没用拢不住皇上的心?别怪你兄弟眼皮子浅,连伺候你的奴才的手都要摸一把。话说回来,还是你不争气的缘故,要是多得宠些,生了个阿哥,也可以多给咱们家里些嚼用,多给你兄弟娶几个媳妇儿,也不会落得他今天这个样子了。''

佐禄听母亲训斥姐姐,吸了吸鼻子,哼道:``不会下蛋的母鸡!''

嬿婉自侍奉皇帝身侧,虽然明里暗里有许多委屈,但到底是养尊处优的嫔妃,再未受过母弟这么粗鲁的奚落。如瑾母女重逢,又听见幼年时听惯了的冷言冷语,禁不住落下泪来:``旁人怎么说是旁人的事,怎么额娘和弟弟也这么说我?这些年我有什么好的都给了家里,满心的委屈你们只看不见,好容易来了宫里一趟,人家都欢欢喜喜的,偏你们要来戳我的痛处!''

魏夫人一不高兴,神色更加难看:``人家欢喜是因为人家高兴,我们有什么可高兴的?你伺候了皇上这么些年,怎么到了今天还是个嫔位?嫔位也就罢了,这肚子怎么还是一点儿动静也没有?你这个年纪,我们庄上多少人都拖儿带女一大群了。''

春婵听不过,只得赔笑道:``夫人别在意,小主一直吃着坐胎药呢,小主心里也急啊!再说了,孩子跟恩宠也没什么关系,愉妃有五阿哥,皇上还不是不大理会她,便是皇后娘娘,也还没有子嗣呢,可皇上还不是照样封了她为皇后。''

魏夫人浑不理会,横了春婵一眼:``人家的福气是生在骨子里的,咱们姑娘的福气是要自己去争取来的,她要有皇后娘娘这个本事,一个孩子也没有便封了皇后,我还有什么可说的。我记得我们姑娘这个嫔位总有两年没动了吧,伺候皇上也四五年了,眼见着年纪是越来越大了,我这个当娘的能不着急么?都说进了宫是掉在金银堆里了,福气是堆在眼前的,怎么偏咱们就不是呢?''她看着嬿婉道:``你看,额娘来了,坐了这么久,皇上那边连个使唤的人也没派来看看,可见你的恩宠是一日不如一日了吧。''

春婵听魏夫人说的话句句戳心,实在是太不管不顾,便她是个宫女也听不下去了,忙将嬿婉准备的绫罗绸缎、金银首饰一一捧上来给魏夫人看了,殷勤道:``这些绸缎都是江南织造进贡的,宫里没几个小主轮得上有。这些首饰有小主自己的,也有皇后娘娘知道了夫人要来特意赏赐的,夫人都带回家去吧。来一趟不容易,小主的孝心都到跟前了呢,''

魏夫人看一样便念一句佛,眼见得东西精致,脸色也和缓了许多:``还是皇后娘娘慈悲。''她看完,神神秘秘对着嬿婉道:``听说皇后娘娘跟你长得有几分相像,真的假的?怎么她成了皇后,你连个妃子也没攀上呢?要不,皇后娘娘赏赐了这许多,我也带了你弟弟去给皇后娘娘谢个恩?''

嬿婉听得这一句,急得眉毛都竖了起来,哪肯母亲去翊坤宫丢丑。还是春婵机敏,笑吟吟劝道:``这个时候,皇后娘娘怕是在处理六宫的事宜呢,不见人的。''如此,魏夫人才肯罢休。

好容易时辰到了,小太监来催着离宫,魏夫人抱着一堆东西,气都缓不过来了,还是连连转头嘱咐:``赶紧怀上个孩子,否则你阿玛死了也不肯闭眼睛,要从九泉之下来找你的。''

魏夫人一走,嬿婉还来不及关上殿门,便落下泪来:``旁人的家人入宫探望,都是一家子欢喜团圆的,怎么偏本宫就这么难堪。原以为可以聚一聚,最后还是打了自己的脸。''她拉过澜翠的手,``还连累了你被本宫那不争气的兄弟欺负。''

澜翠见嬿婉伤心,哪里还敢委屈,只得道:``小主待奴婢好,奴婢都是知道的,奴婢不敢委屈。''

春婵叹气道:``奴婢们委屈,哪里比得上小主的委屈。自己的额娘兄弟都这么逼着,心里更不好受了。其实,夫人的话也是好心,就是逼得急了,慢慢来,小主总会有孩子的。便是恩宠,小主还年轻,怕什么呢。''

嬿婉紧紧攥了手中的绢子,在伤感中沉声道:``可不是呢。娘家没有依靠的人,一切便只能靠自己了。''

册后大典的半个月后,皇帝便陪着新后如懿展谒祖陵,祭告列祖列宗,西巡嵩洛,又至五台山进香,游历名山大川。

而除了皇后之外,所带的亦不过是纯贵妃、嘉贵妃、舒妃、令嫔而已。宫中之事,则一应留给了愉妃海兰料理。

细细算来,那一定是一生中难得的与皇帝独处的时光。他与她一起看西山红叶绚烂,一起看蝶落纷飞,暮霭沉沉。在无数个清晨,晨光熹微时,哪怕只是无言并立,静看朝阳将热烈无声披拂。虽然也有嫔妃陪伴在侧,但亦只是陪侍。每一夜,都是皇帝与如懿宁静相对,相拥而眠,想想亦是奢侈。然而,这奢侈叫人欢喜,因为她是名正言顺的皇后,皇帝理当与她出双入对,形影不离。

后宫的日子宁和而悠逸,而前朝的风波却自老臣张廷玉再度受到皇帝斥责而始,震荡着整个九月时节。

自皇长子永璜离世,初祭刚过,张廷玉不顾自己是永璜老师的身份,就急着匆匆地向皇帝奏请回乡。皇帝不禁动怒,斥责道:``试想你曾侍朕讲读,又曾为皇长子师傅,如今皇长子离世不久,你便告老还乡,乃漠然无情至此,尚有人心么?''

张廷玉遭此严斥,惶惶不安。之后,皇帝命令九卿讨论张廷玉是否有资格配享太庙,并定议具奏。九卿大臣如何看不出皇帝的心意,一致以为应该罢免张廷玉配享太庙。皇帝便以此为依据,修改先帝遗诏,罢除了张廷玉死后配享太庙的待遇。自此,朝中张廷玉的势力,便被瓦解大半。

如懿这新后的位置,因着孝贤皇后去世时慧贤皇贵妃母家被贬斥,而孝贤皇后的伯父马齐早在乾隆四年去世,最大的支持者张廷玉也就此回了桐城老家。据说地方大官为了避嫌,无一人出面迎接,只有一位侄子率几位家人把他接进了老宅之中。

前朝自此风平浪静,连西藏郡王珠尔墨特那木扎勒的叛乱亦很快被岳钟琪率兵入藏平定,成为云淡风轻之事。皇帝可谓是踌躇满志。而为了安抚张廷玉所支支持的富察氏,皇帝亦遥封晋贵人为晋嫔,以示恩遇隆宠,亦安了孝贤皇后母家之心。

这样的日子让如懿过得心安理得,而很快地,后宫中便也有了一桩突如其来的喜事。

这一年十一月的一夜,皇帝正在行宫书房中察看岳钟琪平定西藏的折子,如懿陪伴在侧红袖添香;嬿婉则轻抚月琴,将新学得彝家小曲轻巧拨动,慢慢奏来;而意欢则临灯对花,伏在案上,将皇帝的御诗一首首工整抄录。

嬿婉停了手中的弹奏,笑意吟吟道:``舒妃姐姐诶,其实皇上的御诗已经收录成册,你又何必那么辛苦,再一首首抄录呢?''

意欢头也不抬,只专注道:``手抄便是心念,自然是不一样的。''

如懿轻笑道:``舒妃可以把皇上的每一首御诗都熟读成诵,也是她喜欢极了的缘故。''

皇帝合上折子,抬首笑道:``皇后不说,朕却不知道。''

如懿含笑:``若事事做了都只为皇上知道,那便是有意为之,而非真心了。''

皇帝看向意欢的眼神里满盈几分怜惜与赞许:``舒妃,对着灯火写字久了眼睛累,你歇一歇吧,把朕的桑菊茶拿一盏去喝,可以明目清神的。''

意欢略答应一声,才站起身,不觉有些晕眩,身子微微一晃,幸好扶住了身前的紫檀梅花枝长案,才没有摔下去。

如懿忙扶了她坐下,担心道:``这是怎么了?''

皇帝立刻起身过来,伸手拂过她的额,关切道:``好好儿的怎么头晕了?''

荷惜伺候在意欢身边,担忧不已:``这几日小主一直头晕不适,昨日贪新鲜吃了半个贡梨,结果吐了半夜。''

嬿婉怔了一怔,不自禁地道:``该不会是有喜了吧?''

皇帝不假思索,立刻道:``当然不会!''

意欢对皇帝的斩钉截铁颇有些意外,讪讪地垂下脸。如懿微微一怔,才反应过来皇帝是答得太急了,便若无其事地问:``月事可准确么?有没有传太医来看过?''

意欢满脸晕红,有些不好意思:``臣妾的月事一直不准,两三个月未有信期也是常事。''

荷惜掰着指头道:``可不是。左右小主也已经两个多月未曾有月信了。''她忽然欢喜起来,``奴婢听说有喜的人就会头晕不适,小主看着却像呢。''

嬿婉看着荷惜的喜悦,心中像坠着一块铅块似的,扯着五脏六腑都不情愿地发沉。她脱口道:``这样的话不许乱说。咱们这儿谁都没生养过,万一别是病了硬当成身孕,耽搁了就不好了,还是请太医来瞧瞧。''

这一语提醒了众人,皇帝沉声道:``李玉,急召齐鲁前来,替舒妃瞧瞧。''

李玉当下回道:``正巧呢。这个时候齐太医要来给皇上请平安脉,这会儿正候在外头。''

说罢,李玉便引了齐鲁进来,为舒妃请过脉后,齐鲁的神色便有些惊疑不定,只是一味沉吟。皇帝显然有些焦灼:``舒妃不适,到底是怎么回事?''

齐鲁忙起身,毕恭毕敬道:``恭喜皇上,贺喜皇上,舒妃小主的脉象是喜脉,已经有两个月了呢。''齐鲁虽是道贺,口中却无格外欢喜的口吻,只是以惴惴不安的目光去探询皇帝的反应。

行宫的殿外种了成片的翠竹,如今寒夜里贴着风声吹过,像是无数的浪涛涌起,沙沙地打在心头。

如懿心中一沉,不自觉地便去瞧着皇帝的脸色。皇帝的唇边有一抹薄薄的笑意,带着一丝矜持,简短道:``甚好。''

这句话过于简短,如懿难以去窥测皇帝背后真正的喜忧。只是此时此刻,她能露出的,亦只有正宫雍容宽和的笑意:``是啊,恭喜皇上和舒妃了。''

意欢久久怔在原地,一时还不能相信,听如懿这般恭喜,这才回过神来。想要笑,一滴清泪却先涌了出来。她轻声道:``盼了这么些年''话未完,自己亦哽咽了,只得掩了绢子,且喜且泪。

皇帝不意她高兴至此,亦有些不忍与震动,柔声道:``别哭,别哭。这是喜事。你若这样激动,反而伤了身子。''

如懿见嬿婉痴痴地有些不自在,知道她是感伤自己久久无子之事,便对着意欢道:``从前木兰秋狩,舒妃你总能陪着皇上去跑一圈,如今可在不能了吧。好好儿养着身子要紧。''她看一眼嬿婉,向皇帝道:``皇上,这些日子舒妃得好好儿养着,怕是不能总侍奉在侧了,令嫔,一切便多劳烦你了。''

嬿婉低低答了声``是'',脸色稍微和缓了些许,便道:``舒妃姐姐要好好儿保养着身子呢,头一胎得格外当心才好。''她小心翼翼地伸出手,抚着舒妃的肚子,满脸艳羡,``还是姐姐的福气好,妹妹便也沾一沾喜气吧!''

意欢低头含羞一笑,按住嬿婉的手在自己尚且平坦的小腹上:``多谢妹妹,但愿妹妹也早日心愿得偿。''

皇帝神色平静,语气温和得如四月里和暖的风:``舒妃,你既有孕,那朕赏你的坐胎药以后便不要喝了。''他一顿,``许是你一直喝得勤,苍天眷顾,终于遂了心愿。''

意欢小心地侧身坐下,珍重地抚着小腹:``说来惭愧,臣妾喝了那么些年坐胎药,总以为没了指望,所以这一两年都是有一顿没一顿地喝着。这次出宫以来,皇上一直无须臣妾陪伴,这身孕怕还是在宫里的时候便结下的。仿佛臣妾是有好几次耽搁着没喝了,谁知竟有了!''

齐鲁忙赔笑道:``那坐胎药本是强壮了底子有助于怀孕的。小主的体质虚寒,再加下以前一直一心求子,心情紧张,反而不易受孕。如今底子调理得壮健了,心思又松快,哪怕少喝一次半次,也是不打紧的。但若无前些年那么多坐胎药喝下去调理,也不能说有孕便有孕了。''

意欢连连颔首,恳切道:``齐太医说得是,只是这般说来,宫中还是纯贵妃与嘉贵妃的身子最好,所以才子嗣连绵。''

齐鲁道:``纯贵妃一向身子壮健,而嘉贵妃出身李朝,自小人参滋补,体质格外温厚,所以有所不同。''

意欢笑靥微生,信任地望向齐鲁道:``那本宫以后的调理补养,都得问问齐太医了。''

齐鲁诺诺答应。皇帝温声嘱咐道:``齐鲁是太医院的国手,资历又深,你若喜欢,朕便指了他来照顾你便是。''

意欢眉眼盈盈,如一汪含情春水,有无限情深感动:``臣妾多谢皇上。''

皇帝嘱咐了几句,如懿亦道:``幸好御驾很快就要回宫了,但还有几日在路上。皇上,臣妾还是陪舒妃回她阁中看看,她有了身孕,不要疏漏了什么才好。''

嬿婉亦道:``那臣妾也一起陪舒妃姐姐回去。''

皇帝颔首道:``那一切便有劳皇后了。''

\hypertarget{ux7b2cux5341ux4e8cux7ae0-ux60caux5b55}{%
\chapter{第十二章 惊孕}\label{ux7b2cux5341ux4e8cux7ae0-ux60caux5b55}}

三人告退离去,皇帝的脸色慢慢沉下来,寒冽如冰:``齐鲁,怎么回事?''

齐鲁听皇帝说完,不觉神色惶恐:``舒妃娘娘突然有孕,而坐胎药也没有按时喝下,那必定是坐胎药上出了缘故。皇上,因您怜惜舒妃娘娘,所以那坐胎药并非是绝育的药,而是每次临幸后喝下,才可保无虞,漏个两次三次也无妨。只是听舒妃娘娘的口气,大约是有一年两年这么喝得断断续续了,药力有失也是有的,才会一朝疏漏,怀上了龙胎。''

皇帝微微一惊:``你的意思是,舒妃或许知道了那坐胎药不妥当?''

齐鲁想了想,摇头道:``未必。若是真知道了,大可一口不喝,怎会断断续续地喝?怕是舒妃娘娘对子嗣之事不再指望,所以没有按时喝下坐胎药,反而意外得子。''他忙磕了个头,诚惶诚恐道:``微臣请旨,舒妃娘娘的身孕该如何处置?''

皇帝脱口道:``你以为该如何处置?''

齐鲁不想皇帝有此反问,只得冒着冷汗答道:``若皇上不想舒妃娘娘继续有孕,那微臣有的是神不知鬼不觉的法子落胎。左右舒妃是初胎,保不住也是极有可能的。''他沉声道:``宫里,有的是一时不慎。''

皇帝有些迟疑,喃喃道:``一时不慎?''

齐鲁颔首,伏在地上道:``是。或者皇上慈悲,怜惜舒妃和负重胎儿也罢。''

皇帝怔怔良久,搓着拇指上一颗硕大的琥珀扳指,沉吟不语。许久,皇帝才低低道:``舒妃她是皇额娘的人,她也是叶赫那拉氏的女儿她她只是个女人,一个对朕颇有情意的女人。''

齐鲁见皇帝语气松动,立刻道:``皇上说得是。舒妃娘娘腹中的孩子,也有一半的可能是公主。即便是皇子,到底年幼,也只是稚子可爱而已。''

``稚子可爱,稚子也无辜!''皇帝长叹一声,``罢了!她既然有福气有孕,朕又何必亲手伤了自己的骨血!留下这孩子,是朕悲悯苍生,为免伤了阴鹜。至于这孩子以后养不养的大,会不会像朕的端慧太子和七阿哥一般天不假年,那便是他自己的福气了。你便好好儿替舒妃保着胎吧。''

齐鲁得了皇帝这一句吩咐,如逢大赦一般:``那么,令嫔娘娘和宫里的晋嫔娘娘也还喝着那坐胎药呢,是否如旧还给两位小主喝?''

皇帝的手指笃笃地敲着乌木书桌,思忖着道:``令嫔么,喝不喝原是由她自己的性子,朕可从来没给她喝过,是她自己要心太强了,反而折了自己。至于晋嫔''皇帝一摆手,冷冷道,``她还是没有孩子的好,免得富察氏的人又动什么不该有的心思。左右你想个法子,让她永无后顾之忧便是。''

齐鲁道:``用药是好,但就怕次数频繁了太过显眼。''

皇帝犹豫再三,便道:``也是。那就朕来。''

齐鲁听皇帝一一吩咐停当,擦着满头冷汗唯唯诺诺退却了。

从意欢阁中出来已经是皓月正当空的时分了。如懿吩咐了侍女们换了柔软的被褥,每日奉上温和滋补的汤饮,又叮嘱了不要轻易挪动,要善自保养。

如懿守在意欢身侧,见她行动格外小心翼翼,便笑道:``你也忒糊涂了,自己有了身子竟也不知道。''

意欢且喜且叹:``总以为臣妾身子孱弱,是不能有的。哪里想到有今日呢。''如懿见她手边的鸡翅木小几上搁着一盘脆炸辣子,掩袖更笑:``这么爱吃辣?也不觉得自己口味变了。''

嬿婉忙笑道:``酸儿辣女,说不定舒妃姐姐也会喜欢吃酸的了呢。''

意欢红晕满面:``男女都好。我一贯爱吃辣,总觉得痛快,所以口味也无甚变化。''

如懿伸出手去刮她的脸:``你呀!只顾着自己痛快淋漓,以后也少吃些,辛辣总是刺激腹中胎儿的。''

意欢殷殷听着,一壁低下雪白柔婉的颈,唏嘘道:``从未想过,竟也有今天。''

嬿婉赔笑道:``其实依照舒妃姐姐的盛宠,怀上龙胎也是迟早的事。''

意欢略略沉吟,重重摇头:``不是的,不是。男欢女爱,终究只是肌肤相亲。圣宠再盛,也不过是君恩流水,归于虚空。只有孩子,是我与他的骨血融合而成。从此天地间,有了我与皇上不可分割的联结。只有这样,才不枉我来这一场。''

如懿听得怔怔,心底的酸涩与欢喜,执着与期盼,意欢果然是自己的知己。她何尝不是只希望有一个小小的人儿,由他和她而来,在苍茫天地间,证明他们的情分不是虚妄。这般想着,不觉握住了意欢的手,彼此无言,也皆明白到了极处。

如此,知道意欢有些倦怠,如懿才回自己宫中去。

嬿婉伴在如懿身边,侍奉的宫人都离了一丈远跟着。如懿看着嬿婉犹自残留了一丝笑意的脸,婉声道:``是不是笑得脸颊都酸了?''

嬿婉摸了摸自己的脸,低低道:``看着舒妃姐姐如愿以偿,是为她高兴,但心里还是忍不住发酸。''

如懿喜欢她这样不加掩饰的口吻:``心里再酸,脸上也别露出来。再好的姐妹,你脸上酸了一酸,也难免有让人吃心的时候。记着,待在这宫里,该笑的时候,再想哭也得笑;该哭的时候,再高兴也得哭出来。如果连自己的悲喜都不能掌控,那就不是宫中的生存之道了。''

嬿婉眼波流转,低柔若叹息:``娘娘一晚上都很是高兴,嘱咐了舒妃姐姐那么多有孕的保养之道,其实娘娘心里也不好受吧?''

如懿伸出手,接住细细一脉枝头垂落的清凉夜露:``诚如你所言,是为舒妃高兴,也是为自己伤感。懂得那么多有孕的保养之道,却都不能用在自己身上。''

嬿婉一语勾中心思,不觉泪光盈然:``皇后娘娘,不瞒您,舒妃喝什么坐胎药,臣妾也一样喝了。这么多年,却是一点儿动静也没有。可见是无福了。''

如懿虽然明白个中原委,但如何能够说破,只得婉转劝慰道:``舒妃有孕,到底也是意料之外。她侍奉皇上也八九年了,谁能想到呢?你也是太想得子了,或许如舒妃一般,停一停药,或许就能有了也未可知啊!''

嬿婉语气幽微如诉:``但愿吧!但愿臣妾能如舒妃姐姐一般,得上苍垂怜照顾。''

如懿替她拂了拂鬓边被夜风吹乱的一绺银丝紫金流苏,和婉道:``本宫虽然被册封为皇后,一时得皇上宠爱,但到底也是三十三岁的人了。纯贵妃与嘉贵妃的年纪犹在本宫之上,玫嫔也是三十来岁的人了。年轻的嫔妃里,你是拔尖儿的。凡事不要急,放宽了心,自然会好的。''

如在冰天雪地中忽得一碗热汤在手,嬿婉心头一暖,眼中噙了晶莹的泪:``多谢皇后娘娘眷顾。''

嬿婉的殿中烛火幽微,那昏暗的光线自然比不上舒妃宫中的灯火通明、敞亮欢喜。嬿婉的面前摆了十几碗乌沉沉的汤药,那气味熏得人脑中发沉。嬿婉脸上似笑非笑,似哭非哭,像发了狠一般,带着几欲癫狂的神情,一碗碗往喉咙里灌着墨汁般的汤药。

春婵看着胆战心惊,在她喝了七八碗之后不得不拦下道:``小主,别喝了!别喝了!您这样猛喝,这到底是药啊,就是补汤也吃不消这么喝啊!''

嬿婉夺过春婵拦下的药盏,又喝了一碗,恨恨道:``舒妃和本宫一样喝着坐胎药,她都怀上了,为什么本宫还不能怀上!我偏不信!哪怕本宫的恩宠不如她,多喝几碗药也补得上了!''

她话未说完,喉头忽然一涌,喝下的药汤全吐了出来,一口一口呕在衣衫上,滑下浑浊的水迹。

春婵心疼道:``小主,您别这样,太伤自己的身子了!您还年轻,来日方长啊!''

嬿婉痴痴哭道:``来日方长?本宫还有什么来日?恩宠不如旧年,连本宫的额娘都嫌弃本宫生不出孩子!一个没有孩子的女人,算是什么!''

春婵吓得赶紧去捂嬿婉的嘴,压低了声音道:``小主小声些,皇后娘娘听见算什么呢!''

嬿婉吓得愣了愣,禁不住泪水横流,捂着唇极力压抑着哭声。她看着春婵替自己擦拭着身上呕吐下来的汤药,忽然手忙脚乱又去抓桌上的汤碗,近乎魔怔地道:``不行,不行!吐了那么多,怎么还有用呢?本宫再喝几碗,得补回来!一定得补回来!''

春婵吓得赶紧跪下劝道:``小主您别这样!这坐胎药也不一定管用。您看舒妃小主不就说么,她也是有一顿没一顿地喝着,忽然就有了!''她凝神片刻,还是忍不住道,``小主,您不觉得奇怪么?当初舒妃小主每次喝每次喝也没怀上,怎么有一顿没一顿的时候就怀上了。难不成她是不喝了才怀上的?或者您不喝这坐胎药了,也能怀上也说不准!''

嬿婉当即翻脸,喝道:``你胡说什么?这药方子给宫里的太医们都看了,都是坐胎助孕的好药!''

春婵迟疑着道:``奴婢也说不上来,宫里的药宫里的药也不好说。小主不如停一停这药,把药渣包起来送出去叫人瞧瞧,看是什么东西!''

嬿婉柳眉竖起,连声音都变了:``你是疑心这药不对?''

春婵忙道:``对与不对,奴婢也不知道。只是咱们多个心眼儿吧!谁让舒妃是断断续续喝着药才有孕的呢,奴婢听了心里直犯嘀咕。''

嬿婉被她一说,也有些狐疑起来:``那好。这件事本宫便交给你办,办好了本宫重重有赏。''

春婵磕了个头道:``奴婢不敢求小主的赏,只是替小主安安心罢了。奴婢的姑母就在京中,等回去奴婢就托她去给外头的大夫瞧瞧。这些日子小主先别喝这坐胎药就是了。''

嬿婉沉静片刻:``好!本宫就先不喝了。''

春婵忙道:``是啊。小主总急着想有了身孕可以固宠,其实换过来想想,咱们先争了恩宠再有孩子也不迟啊!左右宫里头的嫔妃一直是舒妃最得宠,如今她有了身孕也好,正好腾出空儿来给小主机会啊!''

嬿婉的神色稍稍恢复过来,她掰着指头,素白手指上的鎏金玛瑙双喜护甲在灯光下划出一道道流丽的光彩:``宫里的女人里头,皇后、纯贵妃、嘉贵妃、愉妃和婉嫔都已经年过三十,再得宠也不过如此了。年轻的里头也就是舒妃和晋嫔得脸些罢了。舒妃这个时候有孕,倒实在是个好机会。''

春婵笑道:``如此,小主可以宽心了。那么奴婢去端碗黑米牛乳羹来,小主喝了安神睡下吧。''

御驾是在九日后回到宫中的。意欢直如众星捧月一般被送回了储秀宫,而晋嫔亦在来看望意欢时被如懿发觉了她手上那串翡翠珠缠丝赤金莲花镯。嬿婉一时瞧见,便道``眼熟'',晋嫔半是含笑半是得意道:``是皇上赏赐给臣妾的晋封之礼,说是从前慧贤皇贵妃的爱物。''

嬿婉闻言不免有些嫉妒:``慧贤皇贵妃当年多得宠,咱们也是知道些的。瞧皇上多心疼你。''

那东西实在是太眼熟了,如懿看着眼皮微微发跳,一颗心又恨又乱,面上却笑得波澜不惊:``这镯子还是当年在潜邸的时候孝贤皇后赏下的,本宫和慧贤皇贵妃各有一串,如今千回百转,孝贤皇后赏的东西,最后还是回到了自家人的手里。''

众人笑了一会儿,便也只是羡慕,围着晋嫔夸赞了几句,便也散了。

这一日陪在如懿身边的恰是进宫当值的惢心,背着人便有些不忍,垂着脸容道:``晋嫔小主年轻轻的,竟这样被蒙在鼓里,若断了一辈子的生育,不也可怜。''

有隐约的怒意浮上眉间,如懿冷下脸道:``你没听见是皇上赏的?慧贤皇贵妃死前是什么都和皇上说了的,皇上既还赏这个,是铁了心不许晋嫔有孕。左右是富察氏作的孽落在了富察氏自己身上,有什么可说的!''

惢心默然点头:``也是!当年孝贤皇后一时错了念头,如今流毒自家,可见做人,真当是要顾着后头的。''

檐下秋风幽幽拂面,寂寞而无声。半晌,如懿缓了心境,徐徐道:``若告诉了晋嫔,反而惹她一辈子伤心,还是不知道的好,只当是自己没福罢了。''

太后得到意欢有孕的消息时正站在廊下逗着一双红嘴绿鹦哥儿,她拈了一支赤金长簪在手,调弄那鸟儿唱出一串嘀呖啼啭,在那明快的清脆声声里且喜且疑:``过了这么些年了,哀家都以为舒妃能恩宠不衰便不错了。皇帝不许她生育,连自作聪明的令嫔都吃了暗亏,怎么如今却突然有了?''

福珈含笑道:``或许皇上宠爱了舒妃这么多年,也放下了心,不忌讳她叶赫那拉氏的出身了。''

太后松了一口气,微微颔首:``这也可能。到底舒妃得宠多年,终究人非草木,皇帝感念她痴心也是有的。''

福珈亦是怜惜:``太后说得是。也难为了舒妃小主一片情深,这些年纵然暗中为太后探知皇上心意,为长公主之事进言,可对皇上也是情真意切。如今求子得子,也真是福报!''

太后停下手中长簪,瞟一眼福珈,淡淡道:``所谓一赏一罚,皆是帝王雨露恩泽。所以生与不生,都是皇帝许给宫中女子的恩典,只能受着罢了。不告诉她明白,有时也比告诉更留了情面。糊涂啊,未必不是福气。何况对咱们来说,舒妃有孕自然多一重安稳,可若一直未孕,也不算坏事。''

福珈幽幽道:``奴婢明白。舒妃对皇上情深,有孕自然是地位更稳,无孕也少了她与皇上之间的羁绊,所以太后一直恍若不知,袖手未理。''

太后不置可否,只道:``对了,舒妃有孕,皇帝是何态度?''

福珈笑道:``皇上说舒妃小主是头胎,叫好生保养着,很是上心呢。''

太后一脸慈祥和悦:``皇帝是这个意思就好。那你也仔细着些,好生照顾舒妃的身子。记着,别太落了痕迹,反而惹皇帝疑心。''

福珈笑容满面答应着:``以后是不能落了痕迹,可眼下有孕,也是该好好儿赏赐的。''

太后笑道:``可不是,人老了多虑便是哀家这样的。那你即刻去小库房寻两株上好的玉珊瑚送去给舒妃安枕。还有,哀家记得上回李朝遣使者来朝时有几株上好的雪参是给哀家的,也挑最好的送去。告诉舒妃好好儿安胎,一切有哀家。''

福珈应道:``是。可是太医院刚来回话,说晋嫔小主身子不大好,太后要不要赏些什么安慰她,到底也是富察氏出来的人。''

太后漫不经心地给手边的鸟儿添了点儿水,,听着它们叫得嘀呖婉转,惊破了晨梦依稀:``晋嫔的病来得蹊跷,这里怕是有咱们不知道的缘故1,还是别多理会,你就去看一眼,送点哀家上回吃絮了的阿胶核桃膏去就是了。''她想了想,``舒妃有孕,,玫嫔的宠遇一般,身子也不大好了,哀家手头也没什么新人备着。''

福珈想了半日,为难地道:``庆贵人年轻,容颜也好,可以稍稍调教。''

太后点头道:``也罢。总不能皇帝身边没一个得宠的是咱们的人,你便去安排吧。''

这边厢意欢初初有孕,宫中往来探视不断,极是热闹,连玉妍也生了妒意,不免嘀咕道:``不就是怀个孩子么,好像谁没怀过似的,眼皮子这样浅!''然而,她这样的话只敢在背后说说,自上次被当众穿耳之后,她也安分了些许,又见皇帝不偏帮着自己,只好愈加收敛。

而嬿婉这边厢,春婵的手脚很快,将药托相熟的采办小太监送出去给了姑母,只说按药拟个方子,让瞧瞧是怎么用的。她姑母受了重托,倒也很快带回了消息。

嬿婉望着方子上的白纸黑字,眼睛里几乎要滴出血来。她震惊不已,紧紧攥着手道:``不会的!怎么会?怎么会!''

春婵吓了一跳,忙凑到嬿婉跟前拿起那张方子看,上面却是落笔郑重的几行字:``避孕去胎,此方极佳,事后服用,可保一时之效。''

阳光从明纸长窗照进,映得嬿婉的面孔如昨夜初下的雪珠一般苍白寒冷。嬿婉的手在剧烈地发抖,连着满头银翠珠花亦沥沥作响。春婵知道她是惊怒到了极点,忙递了盏热茶捧到她手里头道:``不管看到什么听到什么,小主千万别这个样子。''

嬿婉的手哪里捧得住那白粉地油红开光菊石茶盏,眼看着茶水险些泼出来,她放下了茶盏颤声道:``你姑母都找了些什么大夫瞧的?别是什么大夫随便看了看就拿到本宫面前来应付。''

春婵满脸谨慎道:``小主千叮咛万嘱咐的事,奴婢和姑母怎敢随意,都是找京城里的名医看的。姑母不放心,还看了三四家呢。您瞧,看过的大夫都在上头写了名字,是有据可查的。小主,咱们是真的吃了亏了!''

嬿婉摊开掌心,只见如玉洁白的手心上已被养得寸把长的指甲掐出了三四个血印子,嬿婉浑然不觉得疼,沉痛道:``是吃了大亏了!偏偏这亏还是自己找来的!''她沉沉落下泪来,又狠狠抹去,``把避孕药当坐胎药吃了这些年,难怪没有孩子!''

春婵见她气痛得有些痴了,忙劝解道:``小主,咱们立刻停了这药就没事了。方子上说得明明白白,这药是每次侍寝后吃才见效的,舒妃小主停了几次就怀上了,咱们也可以的,小主还年轻,一切都来得及。''

嬿婉的眼中闪过一丝冷厉:``可是这药是皇上赏给舒妃,后来又一模一样赏给晋嫔的。咱们还问过了那么多太医,他们都说是坐胎的好药,他们''

春婵忙看了看四周,见并无人在,只得低声道:``说明皇上有心不想让舒妃和晋嫔有孕,而小主只是误打误撞,皇上并非不想让小主有孕的!''

嬿婉惊怕不已:``那皇上为什么不许她们有孕,皇上明明是很宠爱舒妃和晋嫔的''

春婵也有些惶惑,只得道:``皇上不许,总有皇上的道理。譬如舒妃是叶赫那拉氏的出身,皇上总有些忌讳''

嬿婉脸上的惊慌渐渐淡去,抓住春婵的手道:``会不会是舒妃已经察觉了不妥,所以才停了那药,这才有了身孕?''她秀丽的脸庞上有狠辣的厉色刻入,``她知道了,却不告诉我?''

春婵忙道:``小主,小主,咱们喝那药是悄悄儿的,舒妃不知道,倒是皇后跟前您提过两句的。''

嬿婉雪白的牙森森咬在没有血色的唇上:``是了。皇后屡次在本宫和舒妃面前提起要少喝些坐胎药,要听天由命,要随缘。这件事,怕不只是皇上的主意,皇后也是知道的。''

春婵惊道:``小主一向与皇后娘娘交好,皇后娘娘知道,竟然都不告诉您?或者舒妃小主也是听了她的劝才停了药的,她只告诉舒妃,却不告诉您?您可是为了皇后娘娘下了好大的力气整治嘉贵妃的呀。皇后娘娘的心也太狠了!''

嬿婉死死地咬着嘴唇,却不肯作声,任由眼泪大滴大滴地滚落下来,湮没了她痛惜而沉郁的脸庞。

\hypertarget{ux7b2cux5341ux4e09ux7ae0-ux87bdux65af}{%
\chapter{第十三章 螽斯}\label{ux7b2cux5341ux4e09ux7ae0-ux87bdux65af}}

这一日是意欢怀孕满三月之喜,因为胎象稳固,太后也颇喜悦,便在储秀宫中办了一场小小的家宴以作庆贺。

席间言笑晏晏,便是皇帝也早早自来朝归来,陪伴意欢,太后颇为喜悦,酒过三巡,便问道:``近些日子时气不大好,皇帝要留心调节衣食才是。''

皇帝坐于意欢身侧,忙陪笑道:``请皇额娘放心,儿子一定随时注意。''他转脸对着意欢,关切道:``你如今有了身子,增衣添裳更要当心。''

意欢满面红晕,只痴痴望着皇帝,含羞一笑,一一谢过。

太后的韶华日渐消磨于波云诡谲的周旋中,彷佛是紫禁城中红墙巍巍,碧瓦巍峨,却被风霜侵蚀太久,隐隐有了苍黄而沉重的气息。然而,岁月的浸润,深宫颐养的日子却又赋予她另一种庄静宁和的气度,不怒自威的神色下游如玉般光润的和婉,声音亦是柔软的。和蔼的:``看舒妃盼了那么多年终于有了身孕,哀家也高兴。只是舒妃如今不能陪侍皇帝,皇帝可要仔细。''

皇帝极为恭敬:``是。巡幸归来,前朝的事情多。儿子多半在养心殿安置了。''

太后夹了一筷子凤尾鱼翅吃了,慢悠悠道:``皇帝来回养心殿,都会经过蟲斯门吧?''

皇帝不意太后有此问,便笑道:``是,儿子来回后宫,时常经过蟲斯门。''

太后停了手里的银累丝祥云筷子,庄重道:``皇帝知道蟲斯门的来历么?''

皇帝神色悠然,缓缓吟道:``蟲斯羽,诀诀兮,宜尔子孙,振振兮。''他停一停,环视殿内,将众妃仰慕的神色尽收眼底,有几分得意,``蟲斯门的典故源自《诗经·周南·蟲斯》,儿子都记得的。''

如懿伴在皇帝身侧,微微地偏过头,精致的红翡六叶宫花,玲珑的花枝东菱玉钿,随着她语调的起伏悠悠地晃:``皇上博学,此诗是说蟲斯聚集一方,子孙众多。''她与皇帝相视一笑,又面向太后道:``内廷西六宫的麟趾门相对应而取吉瑞之意,便也是意在祈盼皇室多子多孙,帝祚永延。''

太后微微眯眼,颌首道:``皇帝与皇后博学通识,琴瑟和鸣,哀家看在眼里真是高兴。先帝在时,常与哀家说起蟲斯门的典故。说蟲斯门原来是明朝的旧名,祖先进关以后,更改明宫旧名,想扫除旧日之气,却在看到蟲斯门时心有所触,说这个名字甚好,是让咱们子孙后代繁盛的意思,所以就留了下来。也是,雄蟲斯一振动翅膀叫起来,雌蟲斯便蜂拥而至,每个都给它生下九十九个孩子,当真兴旺繁盛!''

原先渺然的心便在此刻沉沉坠下,如懿如何不明白太后所指,只得不安地起身,毕恭毕敬地垂手而听。皇帝的面色也渐渐郑重,在底下悄悄握了握如懿的手,起身笑道:``皇额娘的教诲,儿子都明白。正因为皇额娘对上缅怀祖先,对下垂念子孙位,儿子才能有今日儿女满膝下的盛景啊。''

皇帝此言,绿筠、玉研、意欢、海兰等有所生育的嫔妃都起身,端正向太后敬酒道:``祖宗福泽,太后垂爱,臣妾等才能为大清绵延子嗣。''

太后脸上含着淡淡笑意,却未举杯接受众人的敬酒。皇帝眼神一扫,其余的嫔妃都止了笑容,战战兢兢站起身来,一脸敬畏与不安:``臣妾等未能为皇家开枝散叶,臣妾等有愧。''

太后仍是不言,只是以眼角的余光缓缓从如懿面上扫过。如懿只觉得心底一阵酸涩,仿佛谁的手狠狠绞着她的心一般,痛得连耳根后都一阵阵滚烫起来,不由得面红耳赤。她行至太后跟前,跪下道:``臣妾身为皇后,未能为皇上诞有一子半女,臣妾忝居后位,实在有愧。''

太后并不看她,脸上早已没了笑容,只是淡淡道:``皇后出身大家,知书识礼,对于蟲斯门的见解甚佳。但,不能只限于言而无行动。''她的目光从如懿平坦的腹部扫过,忧然垂眸,``太祖努尔哈赤的孝慈高皇后、孝烈武皇后皆有所出;太宗的孝庄文皇后诞育世祖福临,孝端文皇后亦有公主;康熙爷的皇后更不必说;先帝的孝敬宪皇后,你的姑母到底也是生养过的;便是连皇帝过世的孝贤皇后也生了二子二女。哀家说的这些人里,缺了谁,你可知么?''

如懿心口剧烈一缩,却不敢露出丝毫神色来,只得以更加谦卑的姿态道:``皇额娘所言历代祖先中,唯有世祖福临的两位蒙古皇后,废后静妃和孝惠章皇后博尔济吉特氏没有生育,无子无女而终。''

太后眉眼微垂,一脸沉肃道:``两位博尔济吉特氏皇后,一被废,一失宠,命运不济才会如此。可是皇后,你深得皇帝宠爱,可是不应该啊!''

脸上仿佛挨了重重一掌,如懿只觉得脸上烧得滚烫,像一盆沸水扑面而来。她只能忍耐,挤出笑道:``皇额娘教诲得是,是臣妾自己福薄。''

海兰看着如懿委屈,心头不知怎的便生了股勇气,切切道:``太后,皇后娘娘多年照顾永琪,尽心尽力,永琪也会孝顺皇后娘娘的。''

太后一嗤,冷然不屑道:``是么?''

皇帝上前一步,将酒敬到太后跟前,连连赔笑道:``儿子明白,儿子知罪了。这些年让皇额娘操心,是儿子不该,只是皇后未有所出,也是儿子陪伴皇后不多之过,还请皇额娘体谅。而且儿子有其他妃嫔诞育子嗣,如今舒妃也见喜,皇额娘不必为儿子的子嗣担心。''

太后的长叹恍若秋叶纷然坠落:``皇帝,你以为哀家只是为你的子嗣操心么?皇后无子,六宫不安。哀家到底是为了谁呢?''

皇帝忙道:``皇额娘自然是关心皇后了,但皇后是中宫,无论谁有子,皇后都是嫡母,也是一样的。''

有温暖的感动如春风沉醉,如懿不自觉地望了皇帝一眼,满心的屈辱与尴尬才稍稍减了几分。到底,他是顾着自己的。

意欢见彼此僵持,忙欠身含笑:``太后关心皇后娘娘,众人皆知。只是臣妾也是侍奉皇上多年才有身孕,皇后娘娘也会有这般后福的。''

许是看在意欢有孕的面上,太后到底还是笑了笑,略略举杯道:``好了,你们都起来吧。哀家也是看着舒妃的身孕才提几句罢了。皇后,你也不要放在心上,只是有空儿时,变多去蟲斯门下站一站,想想祖先的苦心吧。''

如懿诺诺答应,硬撑着发酸的双膝撑起身子,转眼看见玉研讥诮的笑色,心头更是沉重。她默默回到座位,才惊觉额上、背上已逼出了薄薄的汗。仿佛激烈挣扎扑腾过,面上却不得不支起笑颜,一脸云淡风轻,以此敷衍着皇帝关切的神色。到底,这一顿饭也是食之无味了。

自储秀宫归来时已经是月上中天了。如懿回到宫中,卸了晚妆,看着象牙明花镂春和景明的铜镜中微醺的自己,不觉抚了抚脸道:``今儿真是喝多了,脸这样红。''

容珮替如懿解散了头发拿篦子细细地篦着道:``娘娘今儿是为舒妃高兴,也是为皇上高兴,所以喝了这些酒,得梳梳头发散发散才好。''

容珮说罢,便一下一下更用心为如懿篦发,又让菱枝和芸枝在如懿床头的莲花鎏金香球里安放进玉华醒醉香。那是一种专用于帮助醉酒的人摆脱醺意的香饼,翊坤宫的宫女们会在阳春盛时采摘下牡丹的花蕊,与荼蘼花放在一起,浇入清酒充分地浸润牡丹花蕊和荼蘼花瓣,然后在阴凉处放置一夜,再用杵捣,将花蕊与花瓣一起捣成花泥,把花泥捻成小饼,外刷一层龙脑粉,以它散发出的天然花香,让人在睡梦中轻轻地摆脱醉酒的不适。

如懿素来雅好香料,尤其是以鲜花制成的香饵,此刻闻得殿中清馨郁郁,不觉道:``舒妃有孕,本宫自然是高兴的。只是\ldots\ldots{}''她沉吟着道,``前儿内务府说送来了几坛子玫瑰和桂花酿的清酿,说是跟蜜汁儿似得,拿来给本宫尝一尝吧。''

容珮知道她心中伤感与委屈,便劝道:``娘娘,那酒入口虽甜,后劲儿却有些足,娘娘今日已经饮过酒了,还是不喝了吧?''

如懿笑:``喝酒最讲究兴致。兴之所至,为何不能略尝?你快去吧!''

容珮经不得她催促,只好去取了来:``那娘娘少喝一些,免得酒醉伤身。''

如懿斟了一杯在手,望着盈白杯盏中乳金色的液体,笑吟吟道:``伤身啊,总比伤心好多了!''

容珮知她心意,见她印了一杯,便又在添上一杯:``娘娘今日是伤感了。''她的声音更低,同情而不服,``今儿这么多人,太后也是委屈您了。''

如懿仰起脸将酒倒进喉中,擦了擦唇边流下的酒液,哧哧笑道:``不是太后委屈本宫,是本宫自己不争气。太后让本宫去蟲斯门下站着,本宫一点儿也不觉得那是惩罚!若是能有一个自己的孩子,让本宫在蟲斯门下站成一块石头,本宫也愿意!''她眼巴巴地望着容珮,眼里闪过蒙眬的晶亮,``真的,本宫都愿意!舒妃入宫这么多年,喝了这么多年的坐胎药,如今多听了几回,便也怀上了。到底是上苍眷顾,不曾断了她的念想。可是本宫呢?本宫已经三十三岁了,三十三岁的女人,从来没有过自己的孩子,那算什么女人?!''

容珮难过道:``娘娘,你还年轻!不信,您照照镜子,看起来和舒妃。庆贵人她们也差不多呢。''

如懿带着几分醉意,摸着自己的脸,凄然含泪:``是么?没有生养过的女人,看起来或许年轻些。可是年轻有什么用?!这么些年,本宫做梦都盼着有自己的孩子。''她拉着容珮的手往自己的小腹上按,``你摸摸看,本宫的肚子扁的,它从来没有鼓起来过。容珮,本宫是真心不喜欢嘉贵妃,可是也打心眼儿里羡慕她。她的肚子一次又一次鼓起来,鼓得多好看,像个石榴似的饱满。她们都说怀了孕的女人不经看,可是本宫眼里,那是最好看的!''

容珮眼里沁出了泪水:``娘娘,从奴婢第一次看到您,奴婢就打心眼儿里服您。宫里那么多小主娘娘,可您的眼睛和别人不一样,人家的眼睛是流着眼泪珠子的,您的眼睛在愁苦也是忍着泪的。奴婢佩服您这样的硬气,也担心您这样的硬气。不爱哭的人都是伤了心的了。奴婢的额娘也是,她生了那么多孩子,还是挨我阿玛的打。我阿玛打她就像打沙袋似的,一点儿都不懂的心疼。最后奴婢的额娘是一边生着孩子一边挨着我那醉鬼阿玛的打死去的。那时候奴婢就想,做人就的硬气些,凭什么受那样人的挫磨。可是娘娘,现在奴婢看您哭,奴婢还是心疼,奴婢求求老天爷,让一个孩子来您的肚子里吧!''

如懿伏在桌上,俏色莲蓬绣成的八宝瑞兽桌布扎在脸上硬硬地发刺,她伸着手茫然地摩挲着:``还有纯贵妃,这辈子她的恩宠是淡了,可是她什么都不比怕,儿女双全,来日还能含饴弄孙。公里活得最自在最安稳的人就是她。''

容珮从未见过如懿这般伤心,只得替她披上了一件绛红色的廿金大氅:``娘娘,您是皇后,不管谁的孩子,您都是嫡母,她们的子孙,也都是您的子孙。''

如懿凄然摇首:``容珮,那是不一样的,人家流的是一样的血,是骨肉至深。而你呢,不过是神庙上的一座神像,受着香火受着敬拜,却都是敷衍着的。''

容珮实在无法,只得道:``娘娘,好歹您还有五阿哥啊,五阿哥多争气,被您调教的文武双全,小小年纪已经学会了满蒙汉三语,皇上不知道多喜欢他呢!来日五阿哥若是得皇上器重,您固然是母后皇太后,愉妃娘娘是圣母皇太后,一家子在一块儿也极好呢。''

如懿带着眼泪的脸在明艳灼灼的烛光下显出一种苍白的娇美,如同夜间一朵白色的优昙,独自含着清露绽放:``永琪自然是个孝顺的好孩子。可是容珮,每一次盼望之后,本宫都恨极了。恨极了自己当年那么蠢钝,被人算计多年也不自知,恨极了孝贤皇后的心思歹毒。所以,本宫一点儿都不后悔,旁人是怎样害得本宫绝了子嗣的希望,本宫便也要绝了她所有的希望。可是容珮,再怎么样,本宫的孩子都来不了了!''

迷蒙的泪眼里,翊坤宫是这般热闹,新封的皇后,金粉细细描绘的人生,怎么看都是姹紫嫣红,一路韶华繁盛下去。可是只有如懿自己知道,那些恩爱荣华之后,她是如何孤独。夜静人散之后,宫里只剩下她。阔大的紫檀莲花雕花床上铺着一对馥香花团纹鸳鸯软枕,上面是金红和银绿两床苏织华丝凤栖梧桐被,皇帝在时,那自然是如双如对的合欢欣意。可是皇帝不在的日子,她便清楚地意识到,那才是她未来真正的日子。她会老,会失宠,会有``红颜未老恩先断,斜倚熏笼坐到明''的日子。那种日子的寂寞里,她连一点儿可以依靠可以寄托的骨血都没有。只能嗅着陈旧而金贵的古旧器皿发出陈年的郁郁的暗香,淡淡地,像沉浸在水里发黄的旧蚕丝,一丝一缕地裹缠着自己,直到老,直到死。

那就是她的未来,一个皇后的未来,和一个答应,一个常在,没有任何区别。

容珮自知是劝不得了。她只能任由如懿发泄着她从未肯这般宣之于口的哀伤与疼痛,任由酒液一杯杯倾入愁肠,代替一切的话语与动作安慰着她。

过了片刻,芸枝进来低声道:``容姐姐,令嫔小主来了,想求见皇后娘娘了。''

容珮有些为难地看着醉得不省人事的如懿,轻声道:``娘娘酒醉,怕是不能见人了,这样吧,你去好生回了令嫔小主,请她先回去吧。''

芸枝答应着到了外头,见了嬿婉道:``令嫔小主,皇后娘娘方才从储秀宫回来,此刻醉满了,怕是不能见小主了。''

嬿婉想着暖阁的方向望了一眼,道:``方才看娘娘从储秀宫回来有些薄醉,所以特意回宫拿了些醒酒汤来,怎么此刻就醉倒了呢?''

芸枝笑道:``娘娘回来还喝了些酒呢。今儿酒兴真是好!''

嬿婉心中一突,很快笑道:``是啊。舒妃有喜,娘娘与舒妃交好,自然是高兴了,所以酒兴才好!''

正说着,却见菱枝端了一碗醒酒汤走到殿外,容珮开了门道:``娘娘醉得厉害,吐得身上都是,快去端热水来,醒酒汤我来喂娘娘喝下吧!''

菱枝忙着答应了。嬿婉一时瞧见,不觉道:``皇后娘娘醉得真厉害,本宫便不妨碍你们伺候了,好好儿照顾着吧。''

芸枝恭恭敬敬送了嬿婉出去。春婵候在仪门外,见嬿婉这么快出来,不觉诧异道:``小主这么快出来,皇后娘娘睡下了么?''

澜翠本跟着嬿婉进去,嘴快道:``什么睡下,是喝醉了。''

春婵打趣道:``哎呦!贵妃醉酒也罢了,怎么皇后也醉酒呢!''

嬿婉嘴角衔了一缕冷笑,道:``贵妃醉酒也好,皇后醉酒也好,不过都是伤心罢了。本宫还以为皇后多雍容大度呢,巴巴儿地提醒了舒妃坐胎药的事,原来还是过不了女人那一关,也是个妒忌小心眼而罢了。''

春婵笑道:``小主说的是,女人就是女人,哪怕是皇后也不能免俗。''

嬿婉长睫毛轻扬,点漆双眸幽幽一转:``所以啊,来日哪怕舒妃的胎出了什么事儿,也是小心眼儿的人的罪过,跟咱们是不相干的。''

春婵会心一笑,扶着嬿婉悠然回宫。

乾隆十六年,前朝安静,西藏的骚乱也早已平定,皇帝以西北无忧,便更重视江南河务海防与官方戎政。正月,皇帝以了解民间疾苦为由,奉母游览,第一次南巡江浙。

起初,倒颇有几位朝中官员觐见,以为南巡江浙,行程千里,惊动沿途官员百姓,趋奉迎接,未免靡费。皇帝便有几分不悦:``如今你们都称天下安定富庶,这安定富庶朕都是在奏折上看到的,未曾眼见。圣祖康熙爷也曾南巡,下江南与官民同乐,了解民生疾苦。朕为圣祖子孙,理当效仿。''

如此,再不敢有人谏言。待回到宫中,皇帝见如懿已经候在养心殿暖阁等候他下朝,那笑意便不觉从唇边溢出,照的眉眼都熠熠生辉。

如懿忍不住笑:``皇上虽然喜爱江南风景,但也不必如此喜形于色啊。''

皇帝握住她手,附近她耳边轻声道:``你幼时曾去过苏州,每每与朕说起,都十分向往可以再去。朕当日只是幌子,并不能擅自带你离京。如今,朕便与你一同实现心愿。去咱们最想去的地方走一走。''他眼底有明亮的光,像星子在墨蓝夜空里闪出钻石般璀璨的星芒,``朕大云你,不仅是这次,往后咱们还有许多时日,朕会一直陪着你去山水之间。''

心底的暖色仿佛敷锦凝绣的桃花,迎着春风一树一树绽放到极致,那样轻盈而芬芳,充斥着她的一颗心。她依在皇帝胸前,依依婉然道:``只要是皇上想去的地方,臣妾一定伴随身侧,绝不轻离。''

窗外仍有薄薄的飞雪如柳絮轻扬,而他与她的眸光相融间,唯有无限欢喜与安宁。

按着皇太后的意思,因是巡幸江南烟柔之地,随行的嫔妃除了皇后,便以汉军旗出身的纯贵妃、玫嫔、令嫔、婉嫔庆贵人和李朝出身的嘉贵妃陪伴。

皇帝对太后的安排甚是满意,便将六宫中事都托了愉妃海兰照应。临行前,如懿又去探望了意欢,彼时意欢已经有五个多月的身孕了,逐渐隆起的腹部显得她格外有一种初为人母的圆润美满。如懿含笑抚着她的肚子道:``一切可都还好么?''

身下浅碧色的玉兰花样坐褥软似棉堆,意欢爱惜地将手搭在腹部:``一切都还好。只是总觉得像是在梦里似得,不太真切。''

如懿忍不住取笑:``肚子都这么大了,孩子也会踢你了,还总是如在梦中么?''

窗外的雪光透过明纸映得满殿亮堂,意欢满面红晕的脸有着难言的柔美,似有无限情深:``娘娘知道么?臣妾第一次见到皇上的时候,是在入宫的前一年,皇上祭陵回来,街上挤满了围观的百姓,臣妾便跟着阿玛也在茶楼上看热闹,隔了那么远的距离,臣妾居然能看清皇上的脸。在此之前,臣妾作为备选的秀女也曾熟读皇上的御诗,可是臣妾从未想过,这个人会有这样好看的一张脸。从那时开始,这个人便扎在了臣妾心理,知道皇上那年不选秀的时候,臣妾哭得很伤心,却也没想到会被太后选中入宫侍奉。跟着太后的日子里,太后待臣妾很好,他告诉臣妾皇上喜欢翰墨,喜欢诗词,喜欢画画。咱们满人马背上得天下,可是皇帝精通琴棋书画风雅典趣,几乎没有什么是他不会的。有时候皇上来慈宁宫,臣妾便躲在屏风后悄悄瞧他一眼,那时臣妾真是高兴,原来我一生为人,熟读诗书,都是为了要走到这个人身边去。''

如懿见她痴痴地欢喜,隐隐却有了莫名的忧愁盘旋在心间,她只得笑道:``妹妹如今又有了孩子,是该高兴。''

意欢眼底有明亮的光彩,仿佛满天银河也倾不出她心中的喜悦与幸福:``臣妾一直觉得,能在皇上身边是最大的福气。因为这福气太大,所以折损了臣妾的子嗣。皇后娘娘,这话臣妾对谁说她们都不会明白,但是娘娘一定会懂得,满宫里这么些人,她们看着皇上的眼神,她们的笑,都是赤裸裸的欲望。只有皇后娘娘和臣妾一样,您看皇上的眼神,和臣妾是一样的。''

果真一样么?她在心底惆怅的想,其实连她自己也怀疑,当初所谓的真心,经过岁月的粗糙挫磨,还剩了几许?看到的越多,听到的越多,她质疑和不信任的也越来越多,那样纯粹的爱慕,或许是她珍惜意欢愿意与之相交的最大缘由,那是因为,她看见得意欢,恍然也是已然失去的曾经的自己。可那样的自己,那样的意欢,又能得到些什么?

这样的念头在她的脑中肆意穿行,直到荷惜担心的上前劝道:``小主一直害喜得厉害,到了如今,闻见些什么气味不好还是呕的厉害。这会子说了这许多话,等下又要难受了。''

如懿强按下自己纷繁的念想,关切道:``你是头胎,难免怀着身孕吃力些,不过本宫也听人说,越是害喜得厉害,腹中的孩子往后便越聪明。你大可安心就是。''说罢又嘱咐了伺候的荷惜,那些东西不能碰不能闻,连茶水也要格外当心。

荷惜笑道:``皇后娘娘嘱咐了许多次了,奴婢一定会当心的。''

如懿叹道:``不是本宫不放心,本该留着江与彬伺候你的,可是他如今在太医院颇有资历,也得皇上信任,要跟着南巡一路伺候,所以你这里要格外小心留意。''

意欢颔首道:``皇后娘娘对臣妾这一胎的关切,臣妾铭感于心,好在愉妃姐姐是个细心的,有她在,皇后娘娘也可以放心了。''

如懿含笑道:``可不是,本宫就是看你有孕了欢喜,所以左也放不下右也放不下的。不过话说回来,本宫此次跟着皇上南巡,永琪年幼不能带在身边,海兰又要照顾永琪,又要料理后宫中事,只怕也是吃力,凡是你自己多小心。''

意欢且笑且忧,小心翼翼地护着小腹:``且不说前朝如何,就是当今,从怡嫔、玫嫔的孩子的事儿,还有愉妃姐姐生产时的凶险,臣妾还不知道警惕么?这个孩子是臣妾与皇上多年情意的见证,臣妾必定好好儿爱护,不许任何人任何机会伤他分毫!''

\hypertarget{ux7b2cux5341ux56dbux7ae0-ux821e}{%
\chapter{第十四章 舞}\label{ux7b2cux5341ux56dbux7ae0-ux821e}}

这一年正月十三,皇帝奉皇太后离京,经直隶、山东至江苏清口。二月初八,渡黄河阅天妃闸、高家堰,皇帝下诏准许兴修高家堰的里坝等处,然后由运河乘船南下,经扬州、镇江。丹阳、常州至苏州。三月,御驾到达杭州,观敷文书院,登观潮楼阅兵,遍游西湖名胜。

毕竟西湖六月中,风光不与四时同。何况是江南三月,柳绿烟蓝,动若莲步轻移,婀娜多姿;静如少女独处,袅袅婷婷,姹紫嫣红,浓淡相宜,就那样偎依在西湖的周围,晕染着。守望着西湖一湾碧水。

皇帝对江南向往已久,终于一偿夙愿,守着晴也是景,雨也是景,烟雾蒙蒙又是一景的西湖,沉醉其间,如溺醇酒,不能自拔。

除了与文官诗酒相和,如懿亦陪着皇帝尝了新摘的雨后龙井,鲜美的西湖莼菜和宋嫂醋鱼,还有藕粉甜汤、桂花蜜糕。虽然年年有岁贡,但新鲜所得比之宫中份例,自然更受一筹。闲暇之时,苏堤春晓、柳浪闻莺、雷峰夕照、双峰插云、南屏晚钟、三潭印月,都留下皇帝纵情浏览的足迹。

然而,人后皇帝亦感叹,虽然是春来万物生,自然有``桃红复含宿雨,柳绿更带朝烟。酌酒会临泉水,抱琴好倚长松''之美,但断桥残雪不能访见,曲院风荷亦是新叶青青,未见满池红艳擎出了。

这一夜本是宫中夜宴,皇帝陪着太后与诸位王公、嫔妃临酒西湖之上。亲贵们自然是携带福晋,相随而行;后妃们亦是华衫彩服,珠坠摇曳,更不时有阵阵娇声软语传开。人们挨次而入,列上珍馐佳肴,白玉瑞兽口高足杯中盛着碧莹莹的醇香琼浆,更要添一枝明艳似得,陪行的官员将侍奉的女子都换成年方二八的少女,软于烟罗。嫔妃们虽然出身汉军旗,却也不得不稍逊江南女子的柔媚了。

皇帝叹道:``皇额娘属意曲院美景,只是风荷未开,唯有绿叶初见,不能不引以为憾了。''

太后笑吟吟道:``哀家承皇帝的孝心,才得六十天灵还能一睹江南风光。爱家知道皇帝最爱苏堤春晓,可惜在咱们不能在杭州留到夏日,所以也难见曲院风荷美景了,只是哀家想,既然来了,荷叶都见着了,怎么也得瞧一瞧荷花再走啊。''

说罢,太后轻轻击掌,却见原本宁静的湖面上缓缓飘过碧绿的荷叶与粉红荷花。那荷叶也罢了,大如青盏,卷如珠贝,小如银钱,想是用色色青绿生绢裁剪而成,与湖上的真荷叶掺杂其间,一时难辨真假。而那一箭箭荷花直直刺出水面,深红浅白,如胭脂,如粉黛,如雪花,荷叶田田,菡萏妖娆,清波照红湛碧。偶尔有淡淡烟波浮过,映着夹岸的水灯觳波,便是天上夭桃,云中娇杏,也难以比拟那种水上繁春凝伫,潋滟彩幻。

其中两朵荷花格外大,几油斑人许高,在烟波微澜之后渐渐张开粉艳的花瓣。花蕊之上,有两个穿着羽黄绢衣的女子端坐其中,恰如荷蕊灿灿一点。二人翩翩若飞鸿轻扬,一个缓弹琵琶,一个轻唱软曲。

灯火通明的湖面渐渐安静下来,在极轻极细的香风中,琵琶声淙淙,有轻柔舒缓的女子歌声传来,唱出令人沉醉的音律:

西湖烟水茫茫,百顷风潭,十里荷香。宜雨宜晴,宜西施淡抹浓妆。尾尾相衔画舫,尽欢声无日不笙簧。春暖花香,岁稔时康。真乃上有天堂,下有苏杭。

那女子的歌声虽不算有凤凰泣露之美,但隔着春水波清韵,一咏三叹,格外入耳,更兼那琵琶声幽丽入骨,缠绵不尽,只觉得骨酥神迷,醉倒其间。直到有水鸟掠过湖面,又倏忽飞入茫茫夜气,才有人醒转过来,先击节赞赏。

皇帝亦不觉赞叹,侧身向如懿道:``词应景,曲亦好,琵琶也相映成趣。这些也就罢了,只这曲子选的格外有心。''

如懿低首笑道:``素来个赞西湖的词曲多是汉人所作,只这一首《仙吕·太常引》乃是女真人所写,且情词独到,毫不逊色于他作。''

皇帝不觉含笑:``皇后一向好汉家词曲,也读过奥敦周卿?''

如懿轻轻侧首,牵动耳边珠络玲珑:``臣妾不是只知道`墙头马上遥相顾,一见知君即断肠',元曲名家如奥敦周卿,还是知道一些的。''

皇帝伸出手,在袖底握一握她被夜风吹得微凉的手:``朕与你初见未久,在宫中一起看的第一出戏便是这白朴的《墙头马上》。''他的笑意温柔而深邃,如破云凌空的旖旎月色,``朕从未忘记。''

如懿含羞亦含笑,与他十指交握。比之年轻嫔妃的独出心裁,事事剔透,她是一国之母,不能轻歌,亦无从曼舞,只能在不动声色处,拨撩起皇帝的点滴情意,保全此身长安。

太后转首笑道:``皇帝是在与皇后品评么?如何?''

皇帝笑着举杯相敬,道:``皇额娘又为儿子准备了新人么?''

太后笑着摇首,招手唤荷花中二女走近:``皇帝看看,可是新人么?''她的目光在如懿面上逡巡而过,仿佛不经意一般,``宫中新人太多,只怕皇后要埋怨哀家不顾她这个皇后的辛劳了。''

如懿心头一突,却笑得得体:``有皇额娘在,儿臣怎么会辛劳呢?''

太后不置可否地一笑,只是看着近前的两名女子,弹琵琶的是玫嫔,而唱歌的竟是入宫多年却一直不甚得宠的庆贵人。

玉研举起自己手中的酒盏,抿嘴笑道:``旧瓶装新酒,原来是这个意思。''

皇帝颇有几分惊喜之意:``缨络,怎么是你?''

绿筠亦笑:``玫嫔的琵琶咱们都知道的,除了先前的慧贤皇贵妃,便数玫嫔了,但是庆贵人的歌声这样好,咱们姐妹倒也是第一次听闻呢。''

众人的目光都只瞧着庆贵人,唯独玫嫔立在如懿身旁。如懿无意中扫她一眼,却见她脸色不大好,便是在娇艳的脂粉也挡不住面上的蜡黄气息。她正暗暗诧异,却听太后和缓问道:``庆贵人,你是哪一年伺候皇帝的?''

庆贵人依依望着皇帝,目中隐约有幽怨之色,道:``乾隆四年。''

太后叹息一声:``是啊,都十二年了呢,哀家记得,你刚侍奉皇帝那年是十五岁。''

庆贵人垂下娇怯怯的脸庞:``是。太后好记性。''

``哀家记得,你刚伺候皇帝的时候,并不会唱歌。''

庆贵人害羞带怯望了皇帝一眼,很有几分眉弯秋月、羞晕彩霞的风采:``臣妾自知不才,所以微末技艺,也是这十二年中慢慢学会,闲来打发时光的。还请皇上和太后不要见笑。''

庆贵人这几句话说的楚楚可怜,皇帝听得此处,不觉生了几分怜惜:``这些年是朕少少冷落了你,以致你长守空闺,孤灯寂寞,只能自吟自唱打发时光,以后必不会了。''

玉研媚眼横流,笑吟吟道:``皇上待咱们姐妹,总是新欢旧爱都不辜负的。''

婉嫔亦打趣:``嘉贵妃难不成还说自己是新欢么?自然是最难忘的旧爱了。''

如此闲话一响,太后略觉得湖上风大,便先回去。只留了嫔妃们陪伴皇帝笑语。

彼时皓月当空,湖上波光粼粼,有三五宫裳乐伎坐于湖上扁舟之中,或素手抚琴,或朱唇启笛。笛声顺着和煦的微风飘来,细长有如山泉溪水,醇和好似玉露琼浆,丝丝绵绵宛若缠萦的轻烟柔波,在耳畔萦绕不绝,湖边彩灯画带,悉数投影在微凉如绸的湖水中,让人仿似身处灿灿星河之中。

皇帝与身侧的庆贵人絮絮低语,也不知是谁先来惊唤起来:``是下雪了么?''

此时正当三月时节,南地温暖,何曾见三月飘雪。然而,众人抬起头来,却果然见有细碎白点缓缓洒落,尽数落在了湖上,恍惚不清。

有站在湖岸近处的宮眷伸手揽住,唤起来道:``不是雪花,是白色的梅花呢!''

如懿惊喜:``人间三月芳菲盛,怎么此时还会有梅花?''

和亲王弘昼素来好风雅,便道:``皇嫂有所不知,孤山与灵峰的寒梅开得晚,或许还有晚梅可寻。再不然,附近的深山里也还有呢。''他转首惊叹:``寒梅若雪,此人倒有点心思。''

如懿微微不悦:``梅花清雅,乃高洁之物,只这般轻易抛撒,若为搏一时之兴,实在是可惜了。''

玉研托腮欣赏,手指上累累的宝石戒指发出炫目的光。只见一叶墨色扁舟不知何时已经驶到了漫天如虹的绸缎之下,一名着莹白色薄缦纱杉的女子俏立当中,举着一枝盛开的红梅和韵轻盈起舞。她的衣衫上遍绣银线梅花,上面缀满银丝米珠,盈盈一动,便有无限浅浅的银光流转,仿若星芒萦绕周身。画舫上的彩灯将湖面映得透亮,连夜空也有几分透亮,照得那女子眉目如画,顾盼生情,更兼大片月光倾泻如瀑,玉人容色柔美,如浸润星月光灿中,温柔甜软,人咫尺可探。更有身后青衫乐姬相衬,几乎要让人以为身处蓬莱仙岛之境。

婉嫔低声惊道:``这不是令嫔么?''

玉研看了片刻,手上绕着绢子,撇嘴冷笑道:``今儿晚上可真是乏味,除了歌便是舞,咱们宫里的女人既便是卯足了心思争宠,也得会点儿别的吧。老跟个歌舞乐伎似的,自贬了身价,有什么趣儿。''

绿筠笑着瞥了眼玉研,慢悠悠说道:``嘉贵妃也别总说别人,你忘了自己刚入潜邸那会儿,什么长鼓舞啊扁鼓舞啊扇舞啊剑舞啊,又会锤短萧又会弹伽倻琴,一天一个花样儿,皇上宠你宠的不得了,如今也惯会说嘴了,也不许别人学一点儿你的样儿么?''

玉研嗤笑道:``那也得舞得起弹得出才好啊。我出身李朝,学的也是李朝的歌舞,到底还能让皇上喜欢个新鲜。可如今庆贵人和令嫔她们不过是东施效颦罢了,有什么好看的。''

绿筠叹了口气,有些自怨自艾:``东施效颦也得看是谁效啊,像我和嘉贵妃都是半老徐娘了,哪里比得上十几二十来岁的妹妹们年轻水嫩呢。''

玉研笑道:``那也难说,有时候女人的韵味,非得年级长一点而才能出来。岂不知半老徐娘还风韵犹存呢。姐姐忘了,我生四阿哥那会儿是二十六岁,愉妃生五阿哥也是二十六了,舒妃如今头胎也是二十六了。姐姐生三阿哥是二十三岁,那还算是早的。咱们皇上啊,或许就是觉得十几岁的丫头们嫩瓜秧子似的,伺候的不精细。且看庆贵人就知道了,从前十几岁的时候跟着皇上也不得宠,倒是如今开了点儿眉眼了。所以啊,姐姐别整天念叨着人老珠黄,除了把自己念叨得絮烦了,其他真没什么好处。''

如懿笑道:``有嘉贵妃这句话,本宫也宽心多了,原来越老,好处越在后头了。''

玉研犹自在哪儿絮絮,只见湖上景致一变,四艘青舫小舟遍盛鲜花围了过来,舫上一页页窗扇打开,连起来竟是一幅幅西湖四时图,嬿婉曼步舞在那绸带之间,衣袂飘飘,宛若凌波微步,跌宕生姿。最后轻妙一个旋身,往最末的舫上一靠,身姿纤柔,竟融进了西湖冬雪寒梅图中。

高台之上掌声四起,惊赞之声不绝于耳,歌舞乐姬在众人的赞叹中逐一退场。

皇帝抚掌叹道:``舞也罢了,最难得的是匠心独运,白衣红梅,轻轻一靠,便融入画中。''他轻含了一缕薄笑,``如今令嫔也进益了,不是当日只知燕窝细粉,连白瓷和田百优也不分的少女了。''

如懿闻言而知意,当下亦点头:``在皇上身边多年,耳濡目染,自然长进,此刻令嫔白衣胜雪,手中红梅艳烈,果然是用心思了。''

玉研轻哼一声:``这样的好心思怕也是皇后娘娘的安排吧。''

如懿懒得顾及,只淡漠道:``心思若是用在讨皇上喜欢也罢了,若是一味地旁门左道,可真是白费了一番心思了。''

玉研见皇帝笑意吟吟,目光只凝在舫中寻找蜿蜒的身影。也不觉有些讪讪。

皇帝眼中有无限惊艳赞叹之意,扬声道:``令嫔,再不出来,真要化作雪中红梅了么?''

须臾,嬿婉从冬雪寒梅图中盈然而出,捧着手中一束红梅,却先奉到如懿身前,盈然一笑若春桃轻绽:``臣妾知道皇后娘娘素爱绿梅,原想去寻些绿梅来奉与皇后娘娘的,只是绿梅难得。虽是红梅,却也请皇后娘娘笑纳吧。''

如懿凝眸嬿婉手中所捧,乃是江南盛产的杏梅,花头甚丰,叶重数层,繁密斑斓如红杏一般,大似酒晕染上玉色肌肤。如懿一时未伸手去接,只是笑得意味深长:``这些日子不见妹妹,原来是在忙这些呢。''

嬿婉眼波流漾:``臣妾能懂什么,不过是花点儿心思博皇上和皇后一笑罢了。''

如懿见她将红梅捧在手中,进退有些难堪,也不欲把这些心思露在人前,便颔首示意容珮接过。

皇帝笑着招手,示意她在身边坐下:``庆贵人与玫嫔弹琴唱曲,确实有心,你却能融情于景,借着西湖三月落一点儿白雪之意。''

嬿婉低眉浅笑:``臣妾曾听皇后娘娘读张岱之文,向往雪湖之美,虽不能够逼真,也多一分意境罢了。''

皇帝笑着在她的鼻尖一刮:``意境二字最好,朕最喜欢。''

话音尚未散去,敬事房总管太监徐安上前道:``皇上,该翻牌子了。''

皇帝执着嬿婉的手,笑语亲昵:``不必翻了,便是令妃吧。''

这一言,举座皆惊,还是徐安反映的快,忙躬身道:``是。恭喜令妃娘娘。''

皇帝与嬿婉笑意盈盈,眉眼生春。如懿如何不知趣,借着不胜酒力,便带着嫔妃们先告辞了。

玉研十分不满,想着绿筠轻哼道:``说句不好听的,咱们当年都是生了皇子才封的妃位,她凭什么,便也一跃封妃了?''

绿筠扬了扬绢子道:``那有什么?舒妃当年不也没生孩子便封妃了么?''

玉研轻嗤一声道:``那可不一样!舒妃是满军旗贵族的出身,又得太后亲自举荐,得了皇上多年宠爱。令妃是汉军旗下五旗的出身,怎能和她比呢?''

绿筠郁郁失色,道:``比不比的,都是人家的恩宠。太后今晚替玫嫔和庆贵人费了这一番心意,却是螳螂捕蝉黄雀在后,便宜了令妃呢。''

这话落在如懿耳中,便更是不能悦耳。她转过脸,沉声吩咐道:``嘉贵妃,你在宫中有位分有资历,有些话,人微言轻的人说说便也罢了,若是从你的嘴里出来,便是自个儿不尊重了。若是落在奴才们的耳朵里,知道主子们也这样背后议论,更不成个体统。''

绿筠听得这话知道不好,忙笑道:``皇后娘娘,四公主第一回跟了臣妾出来,怕是要惦记臣妾了,臣妾先回去了。''

如懿温言道:``也好,三公主出嫁,四公主是皇上心尖儿上的女儿,你仔细照顾着便是。''

玉研受了一夜的气,俞加有些悻悻,离去时,她犹是忍不住:``皇后娘娘,今夜令妃的精彩若是您的安排,臣妾无话可说;若不是您的安排,她这样伶俐,可是伶俐过头了。即便您的瘦是五指山,也拢不住这样的孙猴子吧!''

玉研的话如同芒刺,密密锥在心上。如懿回首,见皇帝与嬿婉举止亲昵,宛若一对密好情人,细语呢喃,将一应的烟花璀璨,歌舞升平都拂到了身后,只成了成双影儿后头的盛世点缀。

她有些伤怀地轻笑,皇帝原是这盛世华章里最得天独厚可以随心所欲之人,他所喜欢的,别人正好讨了他的喜欢,又有何不可呢?她所能做的,也不过是个旁观者而已。

待回到殿中,如懿便有些闷闷的,容珮支开了伺候的小宫女,亲自替如懿换了一件家常的深红凌暗花夔龙盘牡丹衬衣,拿玉轮替她轻轻摩挲着手背的经络。``皇后娘娘,今晚嘉贵妃的话是不中听,但不中听的话也有入耳的道理。按说令妃小主一直和翊坤宫来往亲密,她若想多得些宠爱,皇后娘娘也不会不成全了她,怎么忽然有了这样自作主张的心思却不让咱们知道你?奴婢倒以为,嘉贵妃的心思有多深,咱们到底是碰到过有些数的,但令妃小主的心思,却是不知深浅的哪!''她想一想,``不过令妃小主再怎么样,跳完了舞还是先把红梅奉给了娘娘,可见她还是顾忌娘娘的,有顾忌,就不怕她太出格。''

如懿闭着眼缓缓道:``可那顾忌若是表面上的,她也太会做人了些。''

如懿若有所思,正把玩着一个金腰线荷花茶盏轻吟,只见底下的小太监瑞穗跑了进来,瑞穗儿原是来往京城替海兰和如懿传递宫中消息的,如懿见了他便问:``这么急匆匆的,可是宫中出了什么事?愉妃和舒妃都还好么?''

瑞穗儿忙道:``回皇后娘娘,自从御驾离京,从二月里起,五阿哥便断断续续地伤风咳嗽,一直不见好,愉妃娘娘都快急坏了,这才不得已想问问,能不能拨了江太医回京照顾。''

如懿为难道:``皇上的圣驾一直是齐鲁齐太医照顾的,这一向齐太医身上也不大好,一应请平安脉之类的起居照顾,都托付了江太医,一时三刻怕是不能够呢,''她到底还是着急,``五阿哥得病到底要不要紧?''

瑞穗儿道:``要紧倒不要紧,只是这伤风缠绵未愈,愉妃娘娘到底心疼,还有\ldots\ldots{}''

如懿心中一紧:``还有什么?''

瑞穗儿道:``还有便是舒妃娘娘,原先害喜吐得厉害,一吐完就胃疼吃不下东西,人见天儿就瘦下去了,那太医调了药,胃是不疼了,如今月份大了便水肿,手上脚上肿得晶晶亮的,又得调了泻水的药。小主有孕之后太医一直说小主肾气虚,这些日子掉头发掉的厉害,一把一把往下落,愉妃娘娘也是担心的不行,找了太医再去看,可是除了肾气弱也没别的了。''

``那孩子呢?孩子有没有事?''

瑞穗儿忙张了笑脸道:``娘娘安心,一切都好。''

\hypertarget{ux7b2cux5341ux4e94ux7ae0-ux7ea2ux8273ux51ddux9999}{%
\chapter{第十五章
红艳凝香}\label{ux7b2cux5341ux4e94ux7ae0-ux7ea2ux8273ux51ddux9999}}

如懿抚着胸口,想来想去还是不放心:``海兰一向精细,照顾着永琪怎么会出错?偏偏永琪一病,舒妃也身上不安,虽然怀了身孕的女人肾气弱是常事,可是掉头发也厉害了些。''

瑞穗儿道:``那奴才回去一定提醒着,多请几个太医瞧瞧。''

如懿叮嘱道:``舒妃这一胎不容易,仔细这点儿。''

这般怀着心事睡去,也不大安稳,如懿昏昏沉沉的睡着,一会儿梦见嬿婉长袖翩翩,一会儿梦见永琪烧的通红的小脸与海兰焦灼的神情,一会儿是大把大把的黑色头发散落,还有意欢惊慌的面孔。

如懿吃力地辗转着身子,忽然背后一凉,惊醒了过来,才发现冷汗湿透了罗衫寝衣,容珮便睡在地下,听的动静,忙起身秉烛,照亮了如懿不安的面庞。

容珮仔细替如懿擦着汗,又端来了茶水:``娘娘可是梦靥了?''

如懿喝了几口茶水润泽了干涸的心肺:``老是梦见心里头不安的事,尤其是舒妃和永琪。''

容珮劝道:``娘娘别着急,女人怀了孕脱发是在寻常不过的,从前奴婢的额娘怀着奴婢的妹妹时也这样。至于五阿哥,亲娘照顾着,不会坏到哪里去。''

如懿犹豫片刻,霍然坐起身,惊起手腕上赤金桌子玎玲作响:``不行!不管怎么样,还是得让江与彬回去一趟!''

如懿如实向皇帝说起永琪与舒妃的事,彼时玉研、嬿婉与缨络亦陪伴在侧,皇帝听着亦十分焦急,立即唤了江与彬来,嘱咐了他回去。江与彬立时赶回京去,一刻也不敢耽搁。为着怕水路缓慢,还特意快马加鞭,只夜里赶到驿站休息。如此,如懿才放心了小半。

待得御驾离开杭州之时,皇帝已晋陆缨络为庆嫔,与嬿婉平分春色,二人都颇得恩幸。

自杭州离去之时,皇帝仍叹惋不已:``未能抛得杭州去,一半勾留是此湖。''又道,``晴湖不如雨湖,雨湖不如月湖,月湖不如雪湖。''深以不能如张岱一般湖心亭看雪而憾。

如懿含笑:``那日令妃妹妹一舞,若雪中红梅,还不能让皇上一窥西湖雪夜之美么?''

皇帝笑道:``小女子取巧而已,怎可与漫天雪景相媲美。''

这个自然是难不倒如懿的。她擅长绣工,待回到回京之时,一副《湖心亭看雪》图比早已奉于皇帝的养心殿内,足以让他时时回味雪中西湖之美了。

离开杭州,御驾便从江宁绕道祭祀明太祖陵,且在太祖陵前阅兵扬威。皇帝为解太后枯闷,亲自陪着皇太后到江宁制造机房观织,又命江宁织造赶制皇太后六十寿辰所用的布料,以讨皇太后的欢心。

淮扬风情,江宁原是六朝古都,彼时金陵王气已收,更添了几许秦淮柔媚,引得皇帝驻足了好些日子。

这一日午膳刚毕,皇帝由江宁一带的官员陪着赏玩了玄武湖与莫愁湖,便留了一众嫔妃在行宫中歇息。

嬿婉得了江宁织造私下奉送来的几十匹名贵锦缎,心中正自高兴,偏那织造府遣来的小侍女口齿伶俐,一匹匹指了道:``这是鸾章锦,纹如鸾翔;这是云昆锦,纹似云从山岳中出;这是列明锦,纹似罗列灯烛;这是蒲桃锦,纹似蒲桃花,富贵吉祥;这是散花绫,纹皆花朵,多多不同。还有这最名贵的杂珠锦,纹以贯珠配,须得最好的织娘用最细最亮的米珠按着纹路纹,又华贵,上身又轻盈配给令妃娘娘是最合适了。这些都是咱们大人的一番心意,还请娘娘笑纳,便是咱们大人的荣光了。''

一席话说得嬿婉心花怒放,抓了一大把金瓜子放在她手里,好好儿打发了出去,又让春婵挑了好几匹最名贵的杂珠锦,亲自送去如懿殿中。

彼时风光晴丽,行宫又驻在栖霞山上,风景秀美乃是一绝。嬿婉坐在步辇上,闲闲地看着手腕上的九连赤金龙须镯,道:``这镯子的颜色不大鲜亮了,得空儿拿去炸一炸。''想想又蹙眉,``罢了,炸过了也是旧的了,匣子里多得是这些镯子,也不是什么稀罕玩意儿。``她随手递给春婵:``赏你戴了吧。''

春婵千恩万谢地接过了戴上。嬿婉掠过水红色的宫纱云袖,倚在步辇的靠上抚弄似葱管似的指甲:``等下晚膳去问问御膳房,有什么新鲜的吃食么。前几日中午夸了一句他们的鸭子做的好,便顿顿都是鸭子了,有神酱烧鸭、八宝鸭、盐水鸭、煨板鸭、水浸鸭,弄得宫里一股鸭子味儿,吃什么都是一样的。''

春婵笑道:``那还不是因为小主一句话,他们就跟得了玉旨纶音了似的,哥哥巴结着咱们。虽然庆嫔小主也得宠,却不能像小主这般一言九鼎了,便是这江宁织造私下孝敬的东西,咱们也比别的宫里足足多上三倍呢。''

嬿婉得意一笑:``知道了就行了,别怪在嘴上。''

春婵应了``是'',又道:``小主如今这么得宠,为何还那么殷勤去皇后娘娘哪里?连最好的杂珠锦都不自己留着,反而给了皇帝。''

嬿婉轻嗤一笑:本宫上次非得那一番心意,原是借了太后抬举庆嫔和玫嫔的力。否则哪有这么顺利,只是即便这样也好,到底借了太后的东风,事先皇后也不知,只怕两宫心理多有些嘀咕,所以本宫显得殷勤小心,别得意过了头落了错处才好。

春婵笑道:``虽然是借了东风,可到底也是小主青春貌美,否则您看玫嫔,到底是人老珠黄,太后怎么安排也是不得力了。''

嬿婉细长的手指轻轻抚在腮边,娇滴滴问道:``春婵,人人都说本宫和皇后长得像,你觉得像么?''

春婵听他她语气一切如常,却不敢不多一份小心:``是有几分相似,但是小主比皇后娘娘年轻貌美多了。''

嬿婉撇下手,拧着手里的桃花色双莺结儿绢子,淡淡道:``皇上喜欢皇后,本宫这张脸便也得了便宜。只要想要比皇后更得宠,就要看她日日如何得宠,还有,便是将皇后的短处,变成本宫自己的长处。''

春婵微微诧异:``皇后也有短处么?''

嬿婉的唇扬起优美的弧度:``是人总会有短处。如今情爱欢好,短处也看出了长处;那一日情分浅了,短处就更成了容不下的错处,本宫只有将皇后没有的做得更好,才能屹立不倒啊!''

嬿婉笑语盈盈,正说得得趣,砖头看见凌云彻领着侍卫走过,向她欠身道:``令妃娘娘金福万安。''

嬿婉的脸色便有些不自在,略略点头示意:``凌大人有礼。这个时候,凌大人怎么不陪着皇上在外呢?''

凌云彻简短道:``李公公怕皇上在外人手不够,特意派微臣回宫多调派些。''他拱手又道,``自杭州以来,一直未曾恭贺小主晋封之喜。''

嬿婉此刻只觉得扬眉吐气,眼角亦绽开一点儿粉色的笑意:``凌大人有心了,能得凌大人这一生道贺,真是比什么都难得。''

凌云彻的脸上比武多余的表情:``恭喜小主是因为小主得偿所愿,以后许多不必要的聪明心思和计谋都可以收起来了。''

嬿婉的脸色倏地一变,如遭霜冻,可是那么多人在,她如何能发作,只得极力维持着矜持的笑容:``聪明是长在骨子里的,去也去不掉。至于计谋嘛,本宫可听不懂大人在说什么。''她的脸色愈加冷淡,``本宫还要去看望皇后娘娘,就不妨碍大人的公务了。''

凌云彻施礼离去。嬿婉发狠似得扭着手里的绢子,沉声道:``看见凌云彻本宫便想起昔日的不痛快,他日日在皇上跟前当差,难保那一日不会说出去什么。''她眼里闪过一丝厌恶与忌惮,``方全之策,还是除了他在皇上眼前为妙。''

春婵笑吟吟道:``小主的智谋足以决胜于千里之外,还怕眼前一个小小的侍卫么?自然是轻而易举之事了。''

嬿婉来到如懿殿中,彼时如懿正香梦沉酣,躺在暖阁的长榻上静静沉眠。嬿婉算着如懿午睡也快醒了,便候在一边,取过如懿在绣的一幅《湖心亭看雪》图绣了起来。不过一炷香时分,如懿便醒转了过来,见她在侧,不觉有些诧异:``令妃怎么来了?''

嬿婉忙搁下手中的绣针,起身道:``臣妾是想来给皇后娘娘请安的,不防娘娘正在午睡,便在一旁候着娘娘。''她指着绣架上的《湖心亭看雪》图笑道,``娘娘怎么成日在绣这个?这图看着不难,但都是用银白,雪白,玉白各色丝线融成雪景颜色,看久了可怕伤眼睛呢。''

如懿就这芸枝的手起身漱了口浣了手,方道:``左右不过是打发时间罢了,长日无聊,绣着玩儿的。''

嬿婉笑生两靥:``皇上每日都要来看娘娘,娘娘都说长日无聊,咱们还怎么说呢?''

如懿取过菱枝端来的莲子羹慢慢喝了一盏,方看了她一眼道:``令妃如今最得恩宠,自然是不会说长日无聊这样的话的。''嬿婉待要说什么,如懿先笑了起来,``来,给本宫瞧瞧,本宫睡着不备的时候,妹妹做了些什么。''

嬿婉蓦然一凛,指着绣布笑道:``臣妾能做什么,不过是皇后娘娘绣了什么,臣妾跟在后面绣什么罢了。''她双眸清灵如水,看来似有无限诚恳,``皇后娘娘既是臣妾的姐姐,又是臣妾的主子,臣妾自然是亦步亦趋,跟随娘娘罢了。''

如懿微微一笑:``好了,坐着说话也累。菱枝,将本宫的莲子羹端来给令妃一碗。''

嬿婉起身谢过:``臣妾新得了一些杂珠锦,臣妾想着此物名贵,不敢擅专,所以特意奉送给娘娘,也只有娘娘才配得起这样华贵的锦缎。''

如懿瞧了一眼春婵捧进的缎子,不以为意道:``妹妹有心了。容珮,收下吧。''

嬿婉见如懿如常,才松了一口气,拣了些江宁的风土人情,陪着如懿一一述说起来。二人正说着话,却见瑞穗儿打了个千儿进来。

如懿本不想瑞穗儿当着嬿婉的面说话。但看瑞穗儿一脸神色匆匆,心下便有了些不安,问道:``出什么事了?''

瑞穗儿道:``回皇后娘娘,江太医自奉了皇上的旨意一路赶着回京北上。可是到了山东境内,不知是劳累还是饮食不慎的缘故,一行人一直拉肚子,两条腿直打晃,根本无法走路。''

如懿惊异不已:``江太医自己就是太医,难道医不好自己么?''

瑞穗儿擦着额头上的汗道:``江太医是想医治自己来着,可是病得太厉害,跟去的人也未能幸免。那地界又偏僻的很,缺医少药的,驿站的驿丞赶出去买药就得一天,一来二去到底耽搁了。''

容珮疑道:``这就奇怪了,怎么早不病晚不病,偏在那些个穷乡僻壤给误了。''

嬿婉将唇角一缕笑意及时抿了下去,急道:``真是可怜见儿的。皇上要他回去便是看着五阿哥和舒妃姐姐的,这别的能耽搁,皇嗣的事可耽搁不得呀!''她看着如懿,``姐姐,不如再派个人去瞧瞧江太医吧。''

如懿沉思片刻,道:``远水救不了近火,江太医能救人,必能自救。且看他自己的。''她又问瑞穗儿:``五阿哥和舒妃如何了?''

瑞穗儿道:``都好,五阿哥病象有缓,舒妃小主除了掉点儿头发,也没什么别的不适了。''

如懿稍稍放心,嬿婉宽慰道:``左右山东离京城也不太远了。江太医这些人一病顶多耽误个十天半个月,既然五阿哥和舒妃姐姐不要紧,娘娘且放宽了心就好。''嬿婉唤过春婵:``听说咱们行宫所在的栖霞山上有座栖霞庙,千年古刹,十分灵验。等下你便陪本宫去栖霞庙好好儿为五阿哥和舒妃姐姐祈福。''

春婵忙答应了道:``这些日子小主总为五阿哥和舒妃小主悬心。与其如此,还是去拜一拜,求了菩萨保佑,也好安心。''

如懿道:``怎么?你们小主总计挂着五阿哥和舒妃么?''

春婵道:``回皇后娘娘的话,小主嘴上不说,心里却总记挂。在杭州时,便托了奴婢去各个有名的寺庙里替五阿哥挂了寄名符儿,替五阿哥求取平安呢。''

嬿婉满脸诚挚:``皇后娘娘,臣妾自己没有孩子,看着皇后娘娘抚养五阿哥,心里也是疼爱得紧。臣妾一向与愉妃姐姐和皇后娘娘交好,只盼望五阿哥平安康健才好。''

如懿见她说得动容,口气也和缓不少:``你还年轻,迟早会有自己的孩子的。''

嬿婉黯然地垂下眼眸,伸手拨弄着几上新贡的一盆蔷薇花,暗红色的枝叶带着柔靡的气味从她身旁萦绕散开。``早有多早,迟有多迟,不过都是心里虚盼着罢了,娘娘也不必安慰了。''她轻叹一口气,``便是眼前的恩宠,皇后娘娘或许觉得臣妾是费尽心机争来的,可是臣妾想争的,不过是一个日后可以相依为命彼此依靠的孩子,并不是贪求荣华富贵。''

如懿别过脸,轻叹一声:``好好儿喝莲子羹吧,莲子莲子,有个愿心在,总是好的。''

嬿婉寒暄之后,便也离开了。她走出殿阁,正见容珮带了两个小宫女开了库房的门,将杂珠锦搬了进去,不过是门缝开合的一瞬,嬿婉已被库房中成堆的杂珠锦惊住。正巧一个小宫女退了出来,嬿婉便笑道:``原来皇后娘娘有这许多杂珠锦了,本宫还送来,可是白白占了你们的地方了。''

那小宫女拍着手笑道:``江宁织造原也要送来的,可是皇后娘娘说,皇上已经私下赏了这么多,连最名贵的鲛文万金锦皇上也全赏了娘娘,便叫江宁织造不必费事了。''

所谓的鲛文万金锦,原是汉成帝殊宠的飞燕与合德二姐妹的爱物,早些年皇帝偶然读《飞燕外传》所知,吩咐江宁与江南二织造竞相复原此锦,不想江宁织造真是做了出来,且皇帝全数赏给了皇后,她竟一点儿也不知。

嬿婉慢慢地走出如懿的庭院,嘴角忽而多了一丝冷凝的笑意,原来她所以为的荣宠万千,与如懿的皇后之尊相比,竟是如此不堪一击。她心里忽然闪过一丝旋电般的念头,何时她亦能享有这样的尊荣之宠,临天下凤位,便是好了。

那念头不过一瞬,便连她自己也惊着了,不自觉出了一身冷汗,站在甬道的风口上,身上一阵阵发冷。

春婵忙道:``小主,左右您的心意也到了,咱们要给皇后娘娘看的,不就是这一份心意嘛。其他的,皇后有多少好东西,关咱们什么事呢。''

嬿婉淡淡地笑了笑,那笑像个阴天的毛太阳似得挂在唇边,春婵看了有些害怕,没话找话地道:``小主别担心,有澜翠在宫里,一切都好着呢。''

嬿婉淡淡一笑:``这个本宫自然知道,她要是个不能干的,本宫也不留她了。''

二人正说着,眼看着玉研坐在鸾轿上,穿了一袭鎏色纱银缎长衣,明艳照人地过去了。

嬿婉沉下脸来道:``这些日子,出了本宫和庆嫔还有皇后,便是嘉贵妃陪伴皇上最多了吧?''

春婵啐道:``可不是?一把年纪了,还打扮得这么妖娆调调的,奴婢就是看不惯她!''

嬿婉轻轻一笑:``你真看不惯她吗?''

春婵疑惑地看了嬿婉一眼,垂下了头。

春夜里格外安静,这一夜皇帝翻得是玉研的牌子。长夜得闲,如懿便捧了一卷《小山词》在窗下静静坐着,窗外偶尔有落花的声音轻缓而过,像是谁的低吟浅唱,如懿侧首问道:``容珮,是什么花落了?''

容珮推开朱漆长窗,望了一眼笑道:``娘娘的耳力真好,是窗外的玉兰呢。''

如懿道:``哪里是本宫耳力好,长夜如斯,寂静而已。''她轻声吟道,``千千万蕊,不叶而花,当其盛时,可称玉树。这样干干净净的花,凋零了真是可惜。''

容珮笑道:``说起玉兰花,昨儿奴婢还碰到凌大人,他也说这样的花儿落在污浊的腻里可惜。''

如懿笑道:``他这么个男人,也这么怜花惜草,伤春悲秋的?''

容珮认真道:``是啊,所以凌大人说,还不如做个玉兰羹炸个玉兰片什么的,吃进肚子里也尽干净了。''

如懿掌不住笑道:``原来说了半天,到底还是副男人的心肠,罢了罢了。''

容珮道:``男人家心肠豁达,笑一笑就过去了,倒是今日令妃小主来,她说的一番话,娘娘可信么?''

如懿淡淡道:``信与不信,她既要说,本宫就听着,彼此留着一点儿脸面也就是了。''

容珮松了一口气:``奴婢就是怕娘娘被轻易说动了。''

如懿淡然一笑:``凡事只看她做了什么,只凭说什么,本宫是不信的。''

二人正说着,却见三宝慌慌张张进来道:``皇后娘娘,凌大人出事了!''

如懿一怔,放下手中的书卷道:``怎么了?''

三宝急慌慌道:``皇上寝宫传来的消息,今晚本是嘉贵妃侍寝,谁知道围房里送嘉贵妃进去的宫女嚷了起来,说才一会儿工夫,收拾嘉贵妃的衣衫时就发现嘉贵妃的肚兜小衣不见了,这才闹了起来。''

``那她的肚兜去了哪里?''

三宝不安道:``是在当值的侍卫们休息的庑房里的凌大人的衣物里夹着的。''

如懿下意识地脱口而出:``不会!''

三宝忙道:``皇后娘娘,这会不会的谁也说不清啊!毕竟,毕竟\ldots\ldots{}''他吞吞吐吐道,``凌大人一直没有成婚,或许是私下恋慕嘉贵妃的缘故,也是有的。''

如懿不悦道:``旁人胡说八道就算了,你是翊坤宫里出来的人,在呢么也跟着胡乱揣测,不言不实!''

三宝吓得发昏,立刻道:``皇后娘娘恕罪,皇后娘娘恕罪!奴才也是把在皇上寝宫那边的话如实说给娘娘听而已。不管怎么样,皇上发了好大的脾气,嘉贵妃还一直缠着皇上处死凌大人,凌大人现在已经受了刑了,李公公递来消息,问怎么办。''

如懿立刻起身:``容珮,替本宫更衣备轿,即刻去皇上哪儿!''

\hypertarget{ux7b2cux5341ux516dux7ae0-ux65cbux6ce2}{%
\chapter{第十六章 旋波}\label{ux7b2cux5341ux516dux7ae0-ux65cbux6ce2}}

如懿赶到时,凌云彻已经挨了满身的鞭子,衣衫破得不堪入目,连帮着他的庑房廊柱下的石砖上都沾上了斑斑血迹。然而,执刑的太监犹未收手,一鞭,一鞭下去,又快又狠,直打得血沫飞溅,皮肉绽开。凌云彻倒也硬气,硬生生忍着,不肯发出一丝呻吟。

如懿脚步一滞,想要近前去看,还是觉得不妥。她扬了扬脸,容珮会意,朝着那执刑的太监摆了摆手,低低道:``皇后娘娘要进去向皇上回话,先停一停手。''

进得寝殿中,烛火下流动着水样的光泽,明明灭灭,樱红色的流苏款款漾漾,一摇一摇地拖出皇帝与玉研细细长长的影子,皇帝在寝衣外披了一件湖蓝团墨外裳,脸色铁青。玉研半坐在榻边,散着一把青丝,身上一袭梅艳色缂丝八团春花秋月衬衣,几颗鎏金錾花扣疏疏地开着,露出雪白的一抹脖颈,正伏在皇帝手臂上哭得梨花带雨。

如懿见她打扮得如此艳,不觉粗了蹙眉,只对着皇帝行礼如仪。

皇帝满脸不悦,并无招呼如懿的心思,便道:``起来吧,夜深,皇后怎么来了?''

如懿和婉道:``臣妾本要睡了,听得皇上寝殿恼了起来,便赶过来瞧瞧。''她含了几分谦卑与自责,``后宫不宁,说来到底是臣妾无能的缘故,还请皇上降罪。''

皇帝摆摆手,气恼道:``不干你的事,到底是朕身边的人手脚不干净,做出这等见不得人的事来。''他问李玉:``人在外头,打得怎么样了?''

李玉探头向外看了看道:``打的没声气儿了,执刑的太监手都酸了呢。''

玉研晃着皇帝的胳膊,恨声道:``皇上!一定要活活打死他,才能泄了臣妾心头之恨!''

如懿轻声道:``李玉,说是不见了嘉贵妃的肚兜,给本宫瞧瞧,是什么肚兜?''

李玉忙答应着奉了上来,如懿看了一眼,却是一个包花盘金鸳鸯戏水的茜香罗肚兜,上面扎着鸳鸯戏莲的花样,红莲绿叶,五色鸳鸯,四周滚连续暗金色并蒂玫瑰花边纹,周匝压青丝绣金珠边儿,十分香艳。

如懿故意蹙眉道:``这是嘉贵妃的东西么?怎么瞧着便是几个小常在她们十几岁的年纪也不用这样艳的东西呀。''

玉研轻哼一声,撇了撇嘴,转脸对着皇帝笑色满掬:``皇上说臣妾皮肤白,穿这样的颜色好看,是不是?''

那原是闺房私语,这样骤然当着如懿的面说了出来,皇帝也有些不好意思,掩饰着咳嗽了一声,道:``什么年纪了,说话还没轻没重的。''

玉研娇声道:``皇上在臣妾眼里,从来都是翩翩少年,那臣妾在皇上身边,自然也是永远不论年纪的。''

如懿听着不堪入耳,便转脸问:``李玉,这东西怎么会落到凌侍卫手里?''

李玉道:``回皇后娘娘的话,嫔妃侍寝,都是在围房里用锦被裹了送进皇上寝殿的,哪怕是在行宫,规矩也是不改的,嘉贵妃进了寝殿,围房的宫女便开始收拾换下来的衣物了,谁知这么一会儿功夫,便不见贵妃娘娘的肚兜。''

如懿目光一亮:``那怎么会跟凌侍卫有关?''

``凌侍卫今夜就守在围房外,且嘉贵妃进殿后,侍卫便轮了一班。凌大人回过庑房喝茶,又换去了皇上殿前守卫。之后进忠带人搜查侍卫们休息的庑房,才在凌侍卫的替换衣物里发现了嘉贵妃的东西。''

如懿用两指拈起那肚兜对着灯火晃了晃,笑道:``李玉,你告诉本宫,什么人会偷肚兜啊?''

李玉满脸通红:``这个\ldots\ldots 这个\ldots\ldots{}''

玉研翻了个白眼,叱道:``必是浪荡之徒做的下作事情!''

如懿瞥着玉研笑道:``也是啊!嘉贵妃保养得宜,青春不老,别说皇上喜欢,是个男人也动心啊。干得出这样的事的,总得是思慕嘉贵妃的人才是吧?''

玉研嫌弃地扬了扬绢子,靠得皇帝更近些,可怜巴巴地道:``皇上,臣妾可什么都不知道。''

玉研粉面低垂,一身艳梅色八团折枝西番莲花样的纱袄衣裙,灯光下愈加容光夺魄,却比平日倍添妩媚别致,如懿蹙眉道:``也真是奇怪了,若是巴巴儿地偷了这不能见人的东西,就该贴身藏着才是啊。怎么放到侍卫庑房那种人多手杂的地方去?也不怕人随手就翻出来,还是故意等着人翻出来呢?''

皇帝道:``皇后的意思,此事有蹊跷?''

店内安静极了,瑶瑶听见远处不知名的虫儿有气无力地鸣叫着。鎏金八方烛台上的红烛还在滋滋燃烧着,流下的丝丝缕缕的红泪,似凌云彻身上滴落的血迹,静静淌下。如懿欠身,神色分明:``出了这样的事,嘉贵妃生气也是情理之中,只是臣妾在想,凌侍卫自伺候皇上以来,一直忠心耿耿,孝贤皇后落水之时他亦不顾性命去救,多年来颇得皇上信任。而嘉贵妃侍寝的次数多得是,为什么偏偏在行宫便出了事,若是有凌侍卫真的觊觎嘉贵妃,在宫里下手偷嘉贵妃的肚兜岂不是更隐蔽些么?若这件事有人存心陷害,只怕皇上一怒之下杀了凌侍卫不要紧,身边缺少了一个忠心得力的人了。''

皇帝乜了如懿一眼,淡淡道:``你是在替凌云彻求情?''

如懿深深垂下眼,以谦和恭敬的姿态深吸一口气,道:``是,这件事虽然蹊跷,但人赃俱获,皇上要怎么罚凌侍卫都不为过,要是能出了嘉贵妃一口恶气,更是值当!只是有一桩,如今是在行宫,不比在宫里。这儿地方小闲人多,今夜为此事打死了侍卫的事传出去,怕也不好听。依臣妾的意思,未免冤死了凌侍卫,还是死罪当免,活罪当罚!''

皇帝略略凝神,亦觉得困倦。他抚慰似得拍了拍玉研香肩:``也罢,那边打发凌云彻去木兰围场做个打扫的苦役,以后再不许回京就是。''

玉研还欲再说什么,如懿及时打断了她:``连肚兜都会被人盯上,说白了不过是嘉贵妃自己言行上还不够检点,本该是位分尊贵得人尊重的年纪了,偏偏还弄得满身小姑娘的玩意儿。若真传出去,也是嘉贵妃自己的名声了。皇上,今夜既然闹出这么大的事,就不宜再由嘉贵妃侍寝,以免皇上再想起这烦心事,''如懿肃了脸容,一派中宫威仪,``嘉贵妃也宜后宫反省静思,以免日后再惹出这样的麻烦。''

皇帝不耐烦地摆了摆手,道:``嘉贵妃,你跪安吧。进保,去接令妃过来。''

进保答应着退下了。如懿亦告退离去。到了门外,如懿见是李玉亲自送出来,便低声道:``多谢你传话过来。''

李玉忙道:``凌侍卫对皇后娘娘有救命之恩,奴才是知道的,且奴才是皇后娘娘在宫里的一只眼睛,凌侍卫便是另一只,奴才可不愿看着旁人生生剜了娘娘的眼珠子去,免得剜了这一只,到时候就来剜奴才了。''

如懿点头道:``你是个乖觉的。好好儿给凌侍卫上点儿药,择日送去木兰围场,一切便靠你打点了。''

李玉答了``是''恭恭敬敬送了如懿出去。

透破厚厚的云层洒落的微弱月光,在宫巷一片迷蒙的黑暗之中浮荡着,像是一层薄纱摇曳,落下迷蒙的湿润。夜风拂面微凉,如懿心头却不松快,只是陈着脸,默默前行。

容珮扶着如懿,低声道:``娘娘以为,今夜的事是不是有人在背后算计娘娘?''

如懿摇了摇头:``事情来得太突然,且本宫是举荐过凌云彻,但他并未明里暗里帮本宫做事,所以算不得是本宫的心腹,又有谁要算计呢?''容珮疑心道:``莫不是嘉贵妃\ldots\ldots{}''

``嘉贵妃和凌云彻无冤无仇,不会托了自己下水去害他,且扯进了肚兜这样香艳私密的东西,他不怕丢了自己的脸面么?''

容珮细想:``要说算计嘉贵妃,宫里算上跟嘉贵妃不睦的,纯贵妃是一个,令妃也是一个,便是婉嫔,也与嘉贵妃不大合得来。''

如懿凝神道:``跟嘉贵妃和睦的人不多,可是本宫看来,那人的目的不只是要拉了嘉贵妃下水,私偷嫔妃肚兜这样的事,更是要对凌云彻斩草除根,所以,谁最忌惮凌云彻在宫里,便是谁了。''

容珮想了半日,低声道:``奴婢听蕊心姑姑说起过,从前凌大人和令妃娘娘\ldots\ldots{}''

如懿转过脸,低声喝止:``住嘴!这件事不许再提。''

容珮道:``是。奴婢可以不提。但这宫里能和凌大人沾上点儿忌讳的人就只有令妃娘娘了。这\ldots\ldots{}''

如懿长叹一声:``无论怎样,先送些上好的金疮药去给凌云彻治伤,否则天气热起来,他那一身伤要化了脓也是要命的事,然后悄悄松了凌云彻去木兰围场安置好,在得空儿问问他,可曾得罪了什么人。''

容珮见如懿如此郑重,忙答应了不敢再提。

凌云彻的伤养了三五日,便被催着押送去了木兰围场。木兰围场原是皇家林苑,里头千里松林,乃是皇家每年狩猎之处。但除了这一年一回的热闹,平时只有与野兽松风为伍,更何况是罚做苦役,不仅受尽苦楚,更是断送了前程。

如懿自然是不能去送的,只得命容珮收拾了几瓶金疮药供他路上涂抹,又折下一枝无患子相送,以一语凭寄:长恨此身非我有,何时忘却营营?

容珮叹道:``娘娘是以此物提醒凌大人,希望他无忧无虑。''

如懿道:``无患子抗风耐旱,又耐阴耐寒。本宫是希望凌侍卫无论身在何处,都耐得住一时苦辛,图谋后路。再告诉他,走得不体面,若想回来,就必得堂堂正正,体体面面。''

、容珮依言前去相送,回来只道:``凌大人走了,只有一句话,娘娘的嘱咐他都知道,请娘娘小心令妃便是。''

如懿的笑意顿时凝在嘴角,冷冷道:``果然是她!''

然而,如懿一时也未有什么动作,令妃照样是万千宠爱,陪伴君侧。而寒的,只是如懿一颗素来提防的心,又愈加凉了几许。

四月过江宁后,御驾便沿运河北上,从陆路到泰安,又到泰山岳庙敬香。五月初四方才回到宫中。

回京后第一件事,如懿便是去了储秀宫看望了意欢。彼时海兰亦带着永琪在意欢身边陪着说话,海兰素来装扮简素,身上是七成新的藕丝穿暗花流云纹蹙银线杀衫,云鬓上略微点缀些六角蓝银珠花,唯有侧鬓上那支双尾攒珠通玉凤钗以示妃子之尊,海兰行动间确有几分临水拂风之姿,楚楚动人。然而,却是永无恩宠之身了。

时在五月,殿中帘帷低垂,层层叠叠如影纱一般,将殿中遮得暗沉沉的。意欢穿着一袭粉红色纱绣海棠春睡纹氅衣,斜斜地靠在床上,爱怜地抚摸着永琪的手,絮絮地嘱咐着什么。江与彬便跪坐一侧,替意欢搭脉请安。

见了如懿来,意欢便是一喜,继而羞赧,背过身去,低低缀泣道:``臣妾今日这个样子,岂敢再让皇后和皇上瞧见。''

如懿微笑着劝慰道:``皇上还在养心殿忙着处理政务,是本宫先来看你,大家同为女人,你何必在乎这些。''

海兰勉强笑道:``这些日子,舒妃妹妹也只肯见臣妾罢了。''她环顾四周,``连殿里都这么暗沉沉的,半点儿光也不肯透进来。''

如懿懂得地点点头,搂过永琪:``永琪病了这些日子,脸也小了一圈,叫皇额娘好好儿瞧瞧。''

海兰心疼道:``可不是,总是断断续续的,幸好二十多日前江太医终于赶回来了,可算治好了。''

如懿蹙眉:``不晓得什么缘故?''

海兰摇头:``小孩子家的病,左右是晚上踢了被子什么的受了凉,乳母们一时没看严。''

如懿沉吟道:``那几个乳母便不能用了,立即打发出去。''

海兰微微点头:``打发出去前得好好儿问问,别是什么人派来害我们永琪的。''她疑惑,``可若真是害永琪,偏又害得那么不在点子上,只是让臣妾揪心,分不得身罢了。''

江与彬请完了脉,如懿问:``不要紧么?''

江与彬温和道:``就是脱发,其他也无碍。''

意欢缓过劲儿来,终于肯侧转身来。她前额的头发掉了好些,发际线拢得老高老高,只有头上笼着的发髻还异常饱满乌黑,许是觉得额头太高太阔了不好看,又剪了好些刘海儿下来。偏偏她的头发掉得稀稀拉拉的,像枯草般发黄,遮住了前头遮不住后头,越发显得欲盖弥彰。女子素来以``淡扫蛾眉朝画师,同心华髻结青丝''为美,头发少了,难免使她容貌折损。

如懿忙道:``发髻还厚重,可是江太医调理了之后见好了些?''

意欢难过道:``发髻是掺了假发的,若是散下来,臣妾自己的头发已经掉了大半,根本不能看了。吃了多少黑芝麻和核桃,一点儿效果也没有。''

论容貌,意欢乃是宫中嫔妃的翘楚,与金玉研可算是花开并蒂,一清冷一妩媚,恰如白莲红薇。偏偏意欢的性子与玉研爱惜美貌瑜命不同,她拥有清如上弦月的美貌,却从不以为自己美。但女子始终是女子,在如何疏淡容貌,如今青丝凋零,倒也真的是难过,如懿只得安慰道:``你现如今怀着孩子呢,肾气虚弱也是有的。等生下了孩子月子里好好儿调理,便能好了。''她爱惜且艳羡地抚着意欢高高隆起的肚子,又问:``孩子都还好么?''

意欢这才破涕为笑,欣慰道:``幸亏孩子一切都好。''

海兰抱着永琪慨叹道:``只要孩子好。做母亲的稍稍委屈些,便又怎样呢?花无百日红,青春貌美终究都是虚空,有个孩子才是实实在在的要紧呢。''

意欢怀着深沉的喜悦:``是啊,这是我和皇上的孩子呢,真好。''

海兰这话是肺腑之言,意欢也是由衷的欢喜。如懿怕惹起彼此的伤感,便问:``你又不爱出去,也不喜见人,老这样闷着对自己和孩子都不好,这些日子都在做什么呢?''

意欢脸上闪过一点儿羞赧的笑色,像是任春风把殿外千瓣凤凰花的粉色吹到了她略显苍白的面颊上,她招招手,示意荷惜将梨花木书桌上厚厚一沓纸全拿了过来,递给如懿,道:``皇后娘娘瞧瞧,臣妾把皇上自幼以来所写的所有御制诗都抄录了下来,若有一个字不工整便都弃了,只留下这些抄的最好的。臣妾想好了,要用这些手抄的御制诗制成一本诗集,也不必和外头那些臭墨子文臣一般讨好奉承了编成诗集,便是自己随手翻来看看,可不是好?''

海兰笑道:``还是舒妃妹妹有心了,皇上一直雅好诗文,咱们却没想出这么个妙事儿来。''

如懿笑道:``若是人人都想到,便没什么稀罕的了。这心意就是难得才好啊!什么时候见了皇上,本宫必得告诉皇上这件妙事才好。''

意欢红了脸,忙拦下道:``皇后娘娘别急,事情才做了一半儿呢,等全好了再告诉皇上也不迟。''

从意欢宫中走出来时,海兰望着庭院中晴丝袅袅一线,穿过大片灿烂的凤凰花落下晴明不定的光晕,半是含笑半是慨叹:``舒妃妹妹实在是个痴心人儿。''

如懿被她一语,想起了自己初嫁皇帝时的时光,那样的日子是被春雨润透了的桃红明绿,如这大片大片洵烂的凤凰花,美得让人无法相信。原来自己也曾经这样绽放过。

诚然,封后之后,皇帝待她是好的,恩宠有加,也颇为礼遇。但那宠爱与礼遇比起新婚燕尔的时光,到底是不同了,像画笔染就的珊红,再怎么艳,都不是鲜活的。

如懿笑了笑,便有些怅惘:``痴心也有痴心的好处,一点点满足就那样高兴。''

海兰深以为然:``是。娘娘看咱们一个个怀着孩子,都是为了荣宠,为了自己的将来,只有舒妃,她和咱们是不一样的。看着冷冷清清一个人儿,对皇上的心却那么热。''

如懿道:``这样也好。否则活着只营营役役的,有什么趣儿呢?''

海兰长叹一声:``但愿舒妃有福气些,别痴心太过了。人啊,痴心太过,便是伤心了。''

二人说着,便走到了长街上。在外许久,突然走在宫内长长的甬道上,看着高高的红墙隔出一线天似的蓝色天空,便觉得无比憋气,好像活在一个囚笼里似的。可是这球笼里,终究是有人快乐的。

如懿这样想着,却见前头的转角处裙裾一闪,似乎是玫嫔的身影,却没有一个宫女跟着,如懿道:``海兰,本宫是不是眼花了,前面过去的是玫嫔么?怎么鬼鬼祟祟的?''

海兰笑着啐道:``宫里的女人,活得像鹦哥儿,像老鼠,像金鱼,那个动起心思来不是鬼鬼祟祟的?''她低声道,``皇后娘娘不知道么?玫嫔的身子坏了。''

如懿想起在杭州的时候,她那样费尽心思和庆嫔一起讨皇帝的欢心,最后还是受了冷落,及不上令妃和庆嫔的千宠万爱。而且,她的脸色那样不好,想着便疑云顿生。如懿问道:``是怎么坏了?''

海兰叹口气:``臣妾也是偶然看她吃药才知道的。许是那年生下了那个死孩子之后便坏了,玫嫔这些年总不能有自己的孩子,听伺候它的宫人说起来,常常是大半年都没有月信,以来便是一两个月,身子都做弄坏了。''

如懿惊道:``有这样的事?江与彬也不曾和本宫提起?''

海兰摆摆手,也动了恻隐之心:``这有什么可提的?女人的身体,熬不住就坏了呗。也是常事。况且她这些年不如从前得宠了,年纪到了,也没个孩子,更没什么家世,就这样熬着呗。''

如懿想起玫嫔的身世和那个只见过一眼便离开了人世的孩子,心下仿佛被秋风打着,沙沙地酸楚。她想说什么,微微张了唇,也唯有一声幽凉叹息而已。

\hypertarget{ux7b2cux5341ux4e03ux7ae0-ux73abux51cbux4e0a}{%
\chapter{第十七章
玫凋(上)}\label{ux7b2cux5341ux4e03ux7ae0-ux73abux51cbux4e0a}}

人后不防时,如懿便召来了江与彬问起意欢的身体。

江与彬说起来便很是忧虑,道:``舒妃娘娘有身孕后一直有呕吐害喜得症状,呕吐之后便有胃疼,这原也常见。为了止胃疼,医治舒妃娘娘的太医用的是朱砂莲,算是对症下药。朱砂莲是一味十分难得的药材,可见太医是用了心思的,这朱砂莲磨水饮服见效最快,却也伤肾。且舒妃娘娘越到怀孕后几个月,水肿越是厉害,微臣看了药渣中有关木通和甘遂两味药,那都是泻水除湿热的好药,可却和朱砂莲一样用量要十分精准,否则多一点点也是伤肾的。舒妃娘娘常年所服的坐胎药,喝酒了本来会使肾气虚弱,长此以往,也算是积下的旧病了。有孕在身本就耗费肾气,只需一点点药,就能使得肾虚脱发,容颜毁损,一时间想要补回来,却也是难。''

如懿听了他这一大篇话,心想一点点沉下去:``你的意思,替舒妃诊治的太医是有人指使?''

江与彬思虑再三,谨慎道:``这个不好说,用的都是好药,不是毒药,但凡是药总有两面,中药讲求君臣互补之道,但是在烹煮时若有一点儿不当,哪怕是三碗水该煎成一碗被建成了两碗,或是煎药的时间长或短了,都必然会影响药性。''

如懿沉吟道:``那舒妃的头发若要涨回来,得要多久?''

江与彬掰着指头想了想:``少则两三年,多则五六年。''

如懿无奈,只得问:``那对孩子会不会有影响?''

江与彬道:``一定会。母体肾气虚弱,胎儿又怎会强健?所以十阿哥在腹中一直体弱,怕是得费好大的力气保养。只是,若生下来了,能得好好儿调养,也是能见好的。''

如懿扶着额头,头痛道:``原以为是昔年坐胎药之故,却原来左防右防的,还是落了错失。''

江与彬道:``坐胎药伤的是根本,但到底不是绝育的药,只是每次侍寝后用过,不算十分厉害。女子怀胎十月,肾气关联胎儿,原本就疲累,未曾补益反而损伤,的确是雪上加霜,掏空了底子。再加上微臣在山东境内腹痛腹泻,耽搁了半个多月才好,也实在是误了医治舒妃娘娘最好的时候。''

如懿眉心暗了下去:``你也觉得你在山东的病不太寻常?''

江与彬颔首:``微臣细细想来,似乎是有人不愿意微臣即刻赶回宫中,而愉妃娘娘因为五阿哥的身子不好,一时顾不上舒妃娘娘,那些汤药上若说有什么不谨慎,便该是那个时候了。''

如懿闭上眼睛,暗暗颔首:``本宫知道了。''她微微睁开双眼,``对了,听愉妃说起玫嫔的身子不大好,是怎么了?''

江与彬道:``玫嫔小主从那时怀胎生子之后便伤了身体,这些年虽也调养,但一来是伤心过度,二来身子也的确坏了,微臣与太医们能做的,不过是努力尽人事罢了。''

如懿心头一悚,惊异道:``玫嫔的身子竟已经坏到这般地步了么?''

江与彬悲悯道:``是。玫嫔小主底子里已经败如破絮,从前脸色还好,如今连面色也不成了。微臣说句不好听的,怕也就是这一两年间的事了。只是玫嫔要强,一直不肯说罢了。''

思绪静默的片刻里,忽然想起玫嫔从前娇艳清丽的时候,一手琵琶声淙淙,生生便夺了高晞月的宠爱。从前,她亦是满庭芳中占尽雨露的那一只,到头来昙花一现,这一生最美好的时光,便那样匆匆过去了,留着的,不过是一个惨败的身体和一颗困顿不堪的心。

如懿虽然感叹,却无伤春悲秋的余地,第二日起来,整装更衣,正要见来请安的合宫嫔妃,骤然闻得外头重物倒地的闷声,确实忙乱的惊呼:``庆嫔!庆嫔!你怎么了?''

如懿霍然站起,疾步走到殿外,却见庆嫔昏厥再地,不省人事。她定了定神,伸手一探庆嫔鼻息,即刻道:``立刻扶庆嫔回宫,请齐太医去瞧,众人不得打扰。''

众人领命而去,忙抬了庆嫔出去。

如懿立刻吩咐:``三宝,先去回禀皇上,再去查查怎么回事。''

到了午后时分,江与彬提了食盒进来,笑吟吟道:``惢心在家无事,做了些玫瑰糕,特来送与皇后娘娘品尝。''

如懿惦记着庆嫔之事,便道:``你来得正好。正要请你回太医院去,瞧瞧庆嫔素来的药方。''

如懿正细述经过,正巧三宝进来了,低低道:``皇后娘娘,庆嫔小主的事儿明白了。''

接二连三的事端,如懿依然能做到闻言不惊了,便只道:``有什么便说吧。''

三宝道:``庆嫔小主喝下了牛膝草乌汤,如今下红不止,全身发冷抽搐,怕是不大好呢。''

江与彬惊道:``草乌味苦辛,大热,有大毒,且有追风活血之效,而牛膝有活血通经、引血下行的功效。牛膝若在平时喝倒还无妨,只是庆嫔小主这几日月事在身,她本就有淋漓不止的血崩之症,数月来都在调理,怎经得起喝牛膝汤?''

如懿的入鬓长眉蜷曲如珠,盯着江与彬道:``你确定?''

江与彬连连道:``是,是!为庆嫔小主调理的方子就在太医院,且这几日都在为她送去调理血崩的固本止崩汤。这一喝牛膝草乌汤,不仅会血崩不止,下红如注,更是有毒的啊!''

如懿沉声道:``三宝,有太医去诊治了么?''

三宝道:``事情来得突然,庆嫔宫中已经请了太医了,同住的晋嫔小主也已经请了皇上去了。''

如懿本欲站起身,想想还是坐下,嫌恶道:``这样有毒的东西,总不会是庆嫔自己要喝的吧?说吧,是谁做的?''

三宝微微有些为难,还是道:``是玫嫔小主送去的。''

如懿扬了扬眉毛:``这可奇了,玫嫔和庆嫔不是一向挺要好的么?''

三宝道:``是要好。所以玫嫔小主一送去,说是替她调理身子的药,很容易托外头弄来的,比太医院那些不温不火的药好,庆嫔小主一听,不疑有他,就喝了下去,谁知道才喝了半个时辰就出事了。''

如懿不假思索道:``那便只问玫嫔就是了。''

三宝躬身道:``事儿一出,玫嫔小主已经被拘起来了,皇上一问,玫嫔就自己招了,说是嫉妒庆嫔有宠,所以一时糊涂做了这件事,可奴才瞧着,她那一言一行,倒像是早料到了,一点儿也不怕似的。''

有一抹疑云不自觉地浮出心头,如懿淡淡笑道:``可怜见儿的,做了这样的事,还有不怕的。''她说罢亦怜悯,``算了,出了这样的事也可怜。容珮你陪本宫去瞧瞧庆嫔吧。''

待到景阳宫里,庆嫔尚在昏迷中,如懿看着帮着擦身的嬷嬷将一盆盆血水端了出去,心下亦有些惊怕。暖阁里有淡淡的血腥气,太后坐在上首,沉着脸默默抽着水烟。皇帝一脸不快,闷闷地坐着,晋嫔窃窃地陪在一旁,一声也不敢言语。宫人们更是大气儿不敢出一声。

如懿见了太后与皇帝,亦受了晋嫔的礼,忙道:``好端端的怎么出了这样的事。庆嫔不要紧吧?''

晋嫔显然是受了惊吓,忙道:``回皇后娘娘的话,庆嫔身上的草乌毒是止住了,但还是下红不止,太医还在里面救治。''

太后敲着乌银嘴的翡翠杆水烟袋,气恼道:``玫嫔侍奉皇上这么多年,一向都是个有分寸的。如今是失心疯还是怎么了,竟做出这种丧心病狂的事来?''

皇帝的语气里除了厌恶便是冷漠:``皇额娘说玫嫔是丧心病狂,那就是丧心病狂。儿子已经吩咐下去,这样狠毒的女人,是不必留着了。''

太后一凛,发上垂落的祖母绿飞金珠珞垂在面颊两侧,珠玉相碰,泛起一阵细碎的响声,落在空阔的殿阁里,泛起冷催的余音袅袅。``皇帝的意思是\ldots\ldots{}''太后和缓了口气,``玫嫔是糊涂了,但她毕竟伺候皇帝你多年,又有过一个孩子\ldots\ldots{}''

皇帝显然不愿听到这件陈年旧事,摇头道:``那个孩子不吉利,皇额娘还是不要提了。''

太后被噎了一下,只得和声道:``阿弥陀佛!哀家老了,听不得这些生生死死的事。但玫嫔毕竟伺候了你十几年,没功劳也有苦劳,且庆嫔到底也没伤了性命。若是太医能救得过来,皇帝对玫嫔要打要罚都可以,只别伤了性命,留她在身边哪怕当个宫女使唤也好。''她斜眼看着进来的如懿:``皇后,你说是不是?''

皇帝显然是恨极了玫嫔,太后却要留她继续在皇帝身边,这样的烫手山芋,如懿如何能接,旋即赔笑道:``有皇额娘和皇上在,臣妾哪里能置喙。且臣妾以为,眼下凡事都好说,还是先问问庆嫔的身子如何吧。''

太后有些不悦:``平日里见皇后都有主意,今日怎么倒畏畏缩缩起来,没个六宫之主的样子。''

如懿低眉顺眼地垂首,恰好齐鲁出来,道:``皇上,庆嫔小主的血已经止住了。只是此番大出血太伤身,怕要许久才能补回来。''

太后双手合十,欣慰道:``阿弥陀佛,人没事就好。''

齐鲁微微一滞:``姓名是无虞,但伤了母体,以后要有孕怕是难了。''

太后嘴角的笑容霎时冻住,在布恩那个展开。皇帝一脸痛心地道:``皇额娘听听,那贱人自己不能为皇家生下平安康健的皇子,还要害得庆嫔也绝了后嗣。其心恶毒,其心可诛!''

福珈有些不忍心,叹道:``皇上,按着庆嫔这么得宠,是迟早会有孩子的。但今年是太后的六十大寿,就当是为太后积福,还是留玫嫔一条命吧。''

皇帝的眉眼间并无一丝动容之色:``按着从前的规矩,玫嫔这样的人不死也得打入冷宫。''皇帝脸色稍稍柔和些,``只是朕答应过皇后,后宫之中再无冷宫,所以玫嫔只能一死。且她自己也已经招认了,真无话可说,想来皇额娘也无话可说吧。''

太后的目光有一丝疑虑闪过,逡巡在皇帝面上。片刻,太后冷淡了神色道:``既然皇帝心意已决,那哀家也没什么好说的。就当是玫嫔咎由自取,不配得皇帝的宠爱吧,及早处死便也罢了。''她摇头道,``景阳宫的风水可真不好,昔年怡嫔死了,庆嫔又这么没福。''太后伸过手起身:``福珈,陪哀家回宫。''

如懿见太后离去,便在皇帝身边坐下:``皇上别太难过。''

皇帝倒真无几多难过的神色,只是厌烦不已:``朕没事。''

如懿温声道:``那,皇上打算怎么处置玫嫔?''

皇帝显然不想多提玫嫔,便简短道:``还能如何处置?不过是一杯鸠酒了事。''

如懿颔首道:``臣妾明白了,那臣妾立刻吩咐人去办。''她想一想,``只是如今天色已晚,皇上再生气,也容玫嫔活到明日。免得有什么惊动了外头,传出不好听的话来。''

皇帝勉强颔首:``也好。一切交给皇后,朕不想再听到与此人有关的任何事。''

如懿婉顺答应了,亦知皇帝此刻不愿有人多陪着,便嘱咐了李玉,陪着皇帝回了养心殿。才出了景阳宫,容珮好奇道:``皇后娘娘,玫嫔犯了这么大的事儿,是必死无疑的。难道拖延一日,便有什么转机么?''

``没有任何转机,玫嫔必死无疑。''如懿轻叹一声,``翻了这么不可理喻没头没尾的事儿,也只有死路一条。只是宫里不明不白死了的人太多了,本宫虽不能阻止,但总的替她做些事,了她一个久未能完的心愿。''

如懿望着遥远的天际,那昏暗的颜色如同沉沉的铅块重重逼仄而下,她踌躇片刻,低声道:``叫三宝打发人出去,吩咐惢心替本宫做件事。''

到了第二日,惢心一早便匆匆忙忙进了宫,如懿正嘱咐了三宝去备下鸠酒,见了惢心连眼皮也不抬,只淡淡道:``事情办妥了?''

惢心忙道:``一切妥当。娘娘昨日吩咐了出来,奴婢连夜准备了祭礼和元宝蜡烛去了乱葬岗,只是年头太久,那地方不太好找。还是娘娘细心,吩咐三宝找来知会奴婢的人,是当年经过手的人,这才找到了,奴婢就赶在子时前带了风水先生寻了个宝地安葬下去,又做了场法事,希望他\ldots\ldots 在地下可以安宁了。''

如懿眉心一松,安宁道:``虽然本宫只见过那孩子一眼,但到底心里不安,如今这事虽然犯忌讳,但做了也到底安心些。你便悄悄去玫嫔宫里,告诉她这件事情,等下本宫遣人送了鸠酒去,也好让她安心上路。''

惢心答应着去了,不过一炷香时分,便匆匆回来道:``皇后娘娘,玫嫔小主知道自己必定一死,所以恳求死前见一见娘娘。''

彼时如懿正倚在窗下,细细翻看着内务府的记账。闻言,她半垂着羽睫轻轻一颤,却也不抬,只淡淡问:``事情已经了了,本宫遂了她的心愿,难道她还有什么非说不可的话么?''

惢心沉吟着道:``玫嫔小主只求见娘娘,只怕知道要走了,有什么话要说吧。''她说罢又央求,``皇后娘娘,奴婢看着玫嫔小主怪可怜见儿的,您就许她一回吧,她只想在临走前见见娘娘,说几句话。她是要死的人了,娘娘\ldots\ldots{}''

如懿念着与玫嫔同在宫中多年,惢心又苦苦央告,便点了点头,道:``等晚些本宫便去看她。''

永和宫中安静如常,玫嫔所居的正殿平静得一如往日,连侍奉的宫人也神色如常,唯有来迎驾的平常在和揆常在的面上露出的惶惶不安或幸灾乐祸的神色,才暗示着永和宫中不同于往日的波澜。

如懿也不看她们的嘴脸,只淡淡道:``不干你们的事,不必掺和进去。''

平常在看着三宝手里端着的木盘,上头孤零零落着一个钧釉灵芝执壶并一个桃心忍冬纹的钧釉杯,不由的有些害怕,垂着脸畏惧地看着如懿,揆常在答应了一声,努了努嘴堆了笑道:``皇后娘娘,那贱人一回来就待在自己房里没脸出来呢,也真是的,怎么做下这种脏事儿。说来贱人也不安分,还让自己的贴身侍女请了您来的吧,还是想求情饶她那条贱命么?''

揆常在是五王爷弘昼的侧福晋送进宫来的美人儿,桃花蘸水的脸容长得妖妖调调的,素来不大合如懿的眼缘,眼下张口闭口又是一个``贱''字,听得如懿越发不悦,听得如懿越发不悦,如懿皱了皱眉,横她一眼:``她做的什么事儿,用得着你的嘴去说么?''

如懿素来不大言笑,揆常在听得这句,更是诺诺称是。平常在扯了扯揆常在的袖子,揆忙缩到一边,再不敢说话了。如懿懒得与她费口舌,瞥了惢心一眼,吩咐道:``你去瞧瞧。''说罢,便往内殿去了。

外头的太监们伺候着推开正殿的殿门,如懿踏入的一瞬,有沉闷的风扑上面孔,恍惚片刻,仿佛是许多年前,她也来过这里,陪着皇帝的还是新宠的蕊姬。十几年后,宫中的陈设还是一如往常,只是浓墨重彩的金粉黯淡了些许,雕梁画栋的彩绘亦褪了些颜色。缥缈的暮气沉沉缠绕其间,好像住在这宫里的人一样,年华老去,红颜残褪,也不过是弹指一挥间的事。

江湖子弟江湖老,深宫红颜深宫凋。其实,是一样的。

晚来的天气有些微凉,殿内因此有一种垂死的气息。尽管灯火如常点着,但如懿依旧觉得眼前是一片深深幽暗,唯有妆台上几朵行将凋零的暗红色雏菊闪烁着稀薄的红影,像是拼死绽放着最后的艳丽。

如懿依稀记得,那朵采胜是昔年玫嫔得宠时候皇帝赏赐给她的首饰中的一件,她格外喜欢,所以常常佩戴。那意头也好,是年年岁岁花面交相映,更是朱颜不辞明镜,两情长悦相惜之意。

如懿在后头望着她静静梳妆的样子,心下一酸,温言道:``皇上并没有废去你的位分,好好儿打扮着吧,真好看。''

玫嫔从镜中望见是她,便缓缓侧首过来:``皇后娘娘来了。''她并不起身,亦不行礼,只是以眸光相迎,却自有一股娴静宜雅,裙带翩然间有着如水般的温柔。

如懿也不在意礼数,只是伸出手折下一小朵雏菊簪在她的鬓边,柔声道:``好好儿的,怎么对庆嫔做了这样的事?在宫里活了十几年,难道活腻了么?''

玫嫔轻轻点头,洁白如天鹅的脖颈垂成优美的弧度。``每天这样活着,真是活腻了。''她看着如懿,定定道:``皇后娘娘不知道吧?我和庆嫔,还有舒妃,都是太后的人。''

如懿的惊异亦只是死水微澜:``哦?''

玫嫔取过蔻丹,细细地涂着自己养的如水葱似的指甲,妩然一笑:``是啊,天下女人中最尊贵的老佛爷,皇太后,皇上的额娘,也要在后宫安置自己的人,是不是很好笑?''

如懿的神色倒是平静:``人有所求,必有所为。没什么好笑的。''

玫嫔嫣然一嗤:``也是,哪怕是万人之上的皇太后,也有害怕的时候啊,安置着我们这些人在皇上身边,该窥探的时候窥探,该进言的时候进言,该献媚的时候献媚,太后和长公主才能以保万全无虞啊!''

如懿奇道:``既然你和庆嫔是一起的人,你为什么还要害庆嫔?''

玫嫔看着自己玫瑰红的指甲,露出几分得意:``太后自己的人给自己人下了毒药,绝了子嗣,伤了身子,好不好玩儿?''她慵懒一笑,似一朵开得半残的花又露出几瓣红艳凝香,越发有种妖异得近乎诡艳的美,``反正众人都以为在曲院风荷那一夜,庆嫔占尽风光,我却是为他人作嫁衣裳,做了陪衬,那便随便吧,反正我是看穿了,说我嫉妒便是嫉妒好了,什么都不打紧。''

如懿轻颦浅蹙,凝视她片刻:``你若真嫉妒庆嫔,就应该下足了草乌毒死她,何必只是多加了那么多牛膝让她血崩不止,伤了本元,生不了孩子呢?你既是太后调教出来的人,就该知道斩草除根才是最好的办法。这半吊子的手法,除了叫人以为你无能,没有别的。''

\hypertarget{ux7b2cux5341ux516bux7ae0-ux73abux51cbux4e0b}{%
\chapter{第十八章
玫凋(下)}\label{ux7b2cux5341ux516bux7ae0-ux73abux51cbux4e0b}}

``我无能?''玫嫔抹得艳红的唇衬得粉霜厚重的苍白的脸上有种幽诡凄艳的美,她郁郁自叹,幽幽飘忽,``是啊!一辈子为人驱使,为人利用,是无能,不过,话说回来,有点儿利用价值的人总比没有好吧。这样想想,我也不算是无能到底。''她微微欠身,``皇后娘娘,请您来不为别的,只为在宫里十几年,临了快死了,想来想去欠了人情的,只有你一个。''

``你要谢本宫替你好好安葬了你的孩子?''如懿凄微一笑,``本宫这一世都注定了是没有孩子的人,替你的孩子做了旁人忌讳的事,就当了了当年见过他的一面之缘。''

玫嫔的眸中盈起一点儿悲艳的晶莹:``我知道。我的孩子生下来就是一个怪物,可是多谢你,愿意为我的孩子做这些事。''

``他不是怪物,是个很好看的孩子。''如懿的声音极柔和,像是抚慰着一个无助的孩子。``他很清秀,像你。''

一阵斜风卷过,如懿不觉生了一层恻恻的寒意,伸手掩上扑棱的窗。玫嫔痴痴地坐着,不能动弹,不能言语,唯有眼中的泪越蓄越满,终于从长长的睫毛落下一滴泪珠,清澈如同朝露,转瞬消失不见。片刻,她极力镇定了情绪:``谢谢你,唯有你会告诉我,他是个好看的孩子。不过,无论旁人怎么说,在我心里,他永远是最好的孩子。''

如懿懂得地凝视着她:``你的孩子进不了宗谱玉牒,死了只能无声无息地去乱葬岗。本宫曾经想做这件事,但终究不敢。如今选了风水宝地重新安葬,又好好儿超度了孩子,就当是送你一程,让你们母子地下相见,再不用生死相离了。''

玫嫔长长地舒了一口气,那面上细细一层泪痕水珠瞬间凝成寒霜蒙蒙,绽出冷雪般的笑意:``是啊!我这个做额娘的,到了地下,终于可以有脸见我的孩子了。他刚走的那些年,我可真是怕啊。怕他在地下孤单单的。都没个兄弟可以和他就伴儿。你猜猜,这个时候,我的孩子是会和孝贤皇后的二阿哥永琏在一起呢,还是更喜欢和他年纪相近的七阿哥永琮?''

如懿见她这般冷毒而笃定的笑容,蓦地想起一事,心中狠狠一搐:``永琮?''她情不自禁地迫近玫嫔,``永琮好好儿地得了痘疫,跟你扯不开干系的,是不是?''

像是挨了重重一记鞭子,玫嫔霍地抬起头:``自然了!孝贤皇后害死了我的孩子,我拿她儿子的一条命来赔,一命抵一命,公平得很!''

如懿极力压着心口澎湃的潮涌,不动声色地问:``七阿哥是怎么死的?''

极度的欣慰和满足洋溢在玫嫔的面容上,恰如她吉服上所绣的瑞枝花,不真实的繁复花枝,色泽明如玉,开得恣意而绚丽,是真实的欢喜。她拨弄着胸前垂下的细米珠流苏,缓缓道:``皇后娘娘,不是只有你见过茉心,我也见了,她求不到你,便来求了我。''

如懿一怔:``茉心求过你?''她的眉头因为疑惑而微微蹙起,``你不过是小小嫔位,不易接近孝贤皇后的长春宫,也未必有能力做这些事,茉心怎会来求你?''

玫嫔语气一滞,也不答,只顾着自己道:``我为什么会生出那样的孩子,我的孩子是怎么死的,我都蒙在鼓里呢。那时候,你被指着害了我和怡嫔的孩子,其实我的心里终没有信了十分!但是只有你进了冷宫,皇上才会看见我的可怜。看见我和我的孩子的苦,看见我们母子俩不是妖孽!所以我打了你,我指着你朝皇上哭诉!没办法,我坐南府里出来,好容易走到了那一日,我得救我自己!不能再掉回南府里过那种孤苦下贱的日子!''她含了几分歉然,``皇后娘娘,对不住!''

如懿也未放在心上,缓和道:``本宫知道,那个时候,人人都认定是本宫害了你们,你怒气冲心也好,自保也好,做也做了,但是本宫出了冷宫之后,你并未为难过本宫。''

玫嫔颔首道:``是了。老天有眼,我日思夜想,终于知道了仇人是谁,该怎么报仇!我一点儿犹豫都没有,立即让人将春娘送去浣洗的贴身衣物偷偷拿去给茉心穿了几日再送回来。茉心穿着那些衣裳的时候,身上的痘都发成脓包了。她还怕不足,特特儿刺破了脓包涂了上去。我再让人用夹子夹了取回来混进春娘的衣物里,真好啊!春娘毫无察觉地穿着,每天都抱着永琮喂奶,神不知鬼不觉地,春娘染了痘疫,永琮也染上了。''她轻嘘一声,晃着水葱似的指甲,森森地笑得前仰后合,``可怜的孩子啊,就这样断送在她狠心的额娘手里了。''她痛快地笑着,眼里闪过恶毒而愉悦的光,``孝贤皇后活着的时候害得你和愉妃那么惨,你们怕是也恨毒了她,茉心求你们,你们居然不答应,白白把这么好的时机给了我。''

如懿张着自己素白的手掌:``因为本宫的手沾过不该沾的血,因为本宫发觉,有些事,看似是孝贤皇后所为,其实未必是她所为,许多蹊跷处,本宫自己也不明白。''

玫嫔狠狠白了如懿一眼:``不是她,还会有谁要这么防着我们的孩子?一命抵一命,我心里痛快极了!''

阁中静谧异常,四目相股,彼此都明白对方眸中刻着的是怎样的繁情复绪。

如懿如在梦呓之中:``如今,心里痛快了么?''

玫嫔抚着心口,紧紧攥着垂落的雪珠碎玉流苏珞子,畅然道:``很痛快!但是更痛!我的孩子,就这么白白被人算计了,死得那样惨!甚至,富察氏都比我幸运多了,至少她是看着她的儿子死的,而我,连我的孩子长什么样子都不知道!''

玫嫔狂热的痛楚无声无息地勾起如懿昔年的隐痛,那个曾经害过自己的人,那个或许还隐隐躲在烟云深处伸出利爪的人,还有那个被自己与海兰,绿筠静静掩去的幼小的生命。她的手,比起玫嫔,又何尝干净过。有时候,人静处,瞧着自己保养得宜的雪白细嫩的手,半透明的粉红的指甲,会骤然心惊,恍惚看见了指甲缝里残留的暗红发乌的血迹和零碎的皮肉,那股血腥气,无论如何都是洗不去的了。她不得不涂上艳色的蔻丹,套着尖锐而优雅的护甲,以宝石和金器冷淡的光艳,以护甲冰冷的坚硬,来树起自己看似的坚不可催,呼吸的悠缓间,她沉声道:``惢姬,都已经过去了,至少你的丧子之痛,那人已经感同身受,甚至亲眼看着自己的孩子死去,她的惨烈不下于你!''

玫嫔原本清秀而憔悴的脸因为强烈的恨意而狰狞扭曲:``还好我见到了茉心,否则我这个没用的额娘就什么都做不了,至死也被蒙在鼓里!''

如懿静了静心神,轻声问:``本宫听说,茉心痘疫发作,是跪在地上朝着咸福宫的方向死的。''

玫嫔微微颔首:``我吩咐人把她送去烧了,也算了她一片忠心!她紧紧攥着手,直到指节都泛白了,``那些日子,听着长春宫的哭声,我真是高兴啊!我从没听过比那更好听的声音,一报还一报,这是皇后的报应啊!''她的嘴角衔着怨毒的快意,一字一字仿佛锋利的刀片,沙沙划过皮肤,划进血肉,泛出暗红的沫子,``我原以为,这辈子连我的孩子是怎么死的都不知道了,可那一刻,害死她儿子的那一刻,我真高兴!我苦命的孩子,额娘终于替你报仇了,额娘这辈子都没这么高兴过。''她眼中的泪水越来越多,汹涌而出,如决堤的河水,肆意流淌,``可是,我的孩子,额娘却连你是什么样子都没见过,来日到了地下,咱们母子怎么相见呢?额娘多怕,多怕见不到你,认不出你。''

心底有潮湿而柔软的地方被轻轻触动,像是孩子软软的手柔柔拂动,牵起最深处的酸楚,如懿柔声道:``母子血浓于水,他会认得你的。''

玫嫔的眼神近乎疯狂,充斥着浓浓的慈爱与悲决,呜咽着道:``也许吧,孩子,别人嫌弃你,额娘不会,额娘疼你,额娘爱你。''她向虚空里伸出颤抖的枯瘦的手,仿佛抱着她失去已久的孩子,露出甜蜜而温柔的笑容,``我的好孩子,不管别人怎么看你,你都是额娘最爱的好孩子。''

如懿看着她,好像生吞了一个青涩的梅子一般,酸得舌尖都发苦了,在这华丽的宫殿里,她们固然貌美如花,争奇斗艳,固然心狠手辣,如地狱的阿修罗,可心底,总有那么一丝难以言说的温柔,抑或坚持,抑或疯狂,如懿不自禁地弯下腰枝,伸手扶住她:``惢姬,你又何必如此?''

玫嫔仿佛在酣梦中醒来,怔怔落下两滴清泪,落在香色锦衣之上,洇出一朵朵枯萎而焦黄的花朵。``是啊!我何必如此,只是不能不如此罢了。''她抬起脸,死死地盯着如懿,``你想想知道为什么?你敢知道?''

如懿静静相望:``从本宫踏进这里开始,不管你说了什么,她们都会以为你什么都对本宫说了。''

玫嫔的眼睛睁得极大,青灰色的面孔因为过于激动而洇出病态的潮红,衬着盛妆胭脂柔丽如霞光的红晕,一双占漆黑眸烧着余烬的火光,灼灼逼人。她颓然一笑:``你说得不错。所以不管我说什么,都只是为了还皇后娘娘今日为我和我孩子所做的一切。''

心头闷闷一震,仿佛有微凉的露水沁进骨缝,让如懿隐隐感知即将到来的迷雾深深后的森寒,她的点头有些艰涩:``有什么便说吧。''

玫嫔仰着脸,神色坚毅而清冷,嘴角的笑意却是冷冽的妩媚与不屑:``皇后娘娘,你猜,我为什么要害庆嫔?是谁指使的我?''

屏息凝神片刻,如懿凝视着她略带嘲讽的面容,淡淡道:``固然不是太后,但旁人也指使不了你。你什么也不缺,什么也不怕。''她不知怎的,忽然想起了意欢,骤然惊道,``难道是\ldots\ldots{}''

玫嫔哧哧地笑着,那声音是透明而坚韧的丝线,扯着尖细的尾音,绷着如懿因极度震惊而混乱的脑仁,雪白的牙齿切切咬在玫嫔暗紫的唇上:``你猜到了,但你不敢说是不是?你不敢说,便是猜准了哈!''她止了笑,厉声道,``太后固然老谋深算,但皇上也不是一个真正足以托付的枕边人,一个男人,能把在深宫里浸淫多年的女人都给算计了,让太后吃了亏都说不出来,只能怨自己选错了人在皇上身边,这样的手段,你说厉害不厉害?皇上的心思一告诉我,我便吸人五体投地,心悦诚服,我便知道太后赢不了皇上。罢了,左右我的身子也坏透了,不过就是这几年的命,从我的防卫镄后,从我报了仇之后,我已经没有活着的心劲儿了,一个黑锅背下来,能换来家里人几辈子的荣华富贵,便也值得了。''她逼视着如懿,``皇后娘娘,我的话,您都明白了么?''

如懿的背抵在墙上,仿佛不如此,便不能的的抵御玫嫔这些言语所带来的刮骨的冷寒一般:``是皇上借你的手?''

玫嫔冷笑道:``借谁的手不是手?是皇上可怜我,临死了还给我这么个机会,左右我在太后跟前也是个不得宠的弃子,能被皇上用一遭便是一遭吧。一颗棋子,能为人所利用,才是它的价值所在,否则它就不该留在这世上,不是么?''

如懿的牙根都要颤抖,她控制不住,控制不住自己冲口而出的话语:``皇上是什么时候知道的?''

``从曲院风荷那一夜,或者更早,为柔淑长公主劝婚的时候。''她瞥如懿一眼,``皇后娘娘,我记得那时您也为柔淑长公主进言了吧。仔细着皇上也疑心上了您。''她轻笑道,``咱们这位皇上啊,疑心比谁都重,却什么也不爱说出来,只自己琢磨着,他以为自己琢磨上什么了,不管你说什么,也都认定自己是琢磨对的了。皇后娘娘,陪着这样一个良人,您的日子不大好过吧?''

如懿心底有些难过,那难过像吃着一个带了虫子的果子,想咽咽不进,想吐吐不出,只得忍耐着道:``好不好过,本宫都是皇后。''

玫嫔唇边挂着淡淡的笑意,眼里却有着深深的希翼。``皇后娘娘,告诉您这些话,便处是报了当年您的恩情了。您的日子比我长,只怕受的苦也不会比我眼下少,好好儿过着吧。''她的眼中渐渐平静如死水,``皇上打算怎么赐死我?白绫吊了脖子会成个吐着舌头死的鬼儿,往身上插一刀会有个洞眼,皇后娘娘,我想体体面面齐齐整整地下去见我的孩子,不想吓着他。''

如懿的眼底有点潮潮的湿润,她别过脸道:``鸩酒已经替你准备好了,是皇上御赐的,你不会走得太难过。''她击掌两下,三宝捧了酒进来。

玫嫔笑了笑,起身道:``皇后,我这样打扮好看么?''

心头的酸楚一阵阵泛起涌动的涟漪,如懿还是勉力点头:``很好看,你的孩子见了你。会很骄傲他有一个这么美的额娘。''

玫嫔绷紧的神色松弛下来,温婉的点点头,接过鸩酒一饮而尽,并无一丝犹豫,她走到床边,安静地躺下,闭上眼,含着笑,仿佛期待着有一个美梦。药性发作得很快,她的身份剧烈地抽搐了几下,嘴角流下一抹黑色的血液,终于回复沉睡般的平静。

那是如懿最后一次凝视玫嫔的美丽,恰如晚霞的艳沉里含露的蔷薇,凝住了最后一刻芳华。这些年,玫嫔并非宠冠后宫,可年轻的日子里,总有过那样的好时候,露湿晴花春殿香,月明歌吹在昭阳。笑是甜的,情是暖的,那样迷醉,总以为一生一世都是那样的好时光,永远也过不完似的。

只是,终究年华会老,容貌会朽,情爱会转淡薄,成了旧恨飘零同落叶,春风空绕万年枝。

如懿摘下手钏上系着的素色绫绢,轻柔地替她抹去唇角的血液:``好好儿去吧。你最爱的孩子在下面等着你,和你再续母子情分。''

春风吹过,如懿觉得脸上湿湿的,又有些发凉,风吹得满殿漫漫深深的珠绣纱帷轻拂如缭绕的雾,让人茫然不知所在。

紧闭的门扇戛然而开,有风乍然旋起,是惢心闪身进来,她戚然望着锦榻上玫嫔恬静的容貌,轻声道:``娘娘,玫嫔小主去了?''

如懿微微颔首,夜风扑着裙裾缠丝明丽的一角,宛如春日繁花间蝴蝶的翅,扇动她的思绪更加烦乱,她按下心神,问道:``方才揆常在说玫嫔遣了自己的贴身侍女出去,是去了哪里?''

惢心眼波微流,低声道:``奴婢去查了,玫嫔遣了她的贴身侍女去过启祥宫,但启祥宫的人并未见她,连宫门都不曾开。奴婢想着,玫嫔与启祥宫素无来往,怎么巴巴地派人去了,问了那宫女,她也说不出什么头尾,只说玫嫔着她向嘉贵妃磕个头,若是见不着,在启祥宫外磕个头便走就是了。''

惢心答得行云流水,想是细细查问过了。如懿微眯着眼,有一种细碎的光凝成疑虑的波彀,在她的眼眸里流过:``你告诉了玫嫔为她孩子超度善后之事,她要见本宫言谢,那也算情理之中。可去启祥宫这便奇怪了,没头没尾的,去做什么呢?''

惢心揣度着道:``奴婢想着,玫嫔小主是个恩怨分明的人,娘娘替她了结了她孩子的事,她自然要谢娘娘。且说来玫嫔小主也够委屈的,一辈子的苦楚说不得言不得,不能说出口一句,怕许多事许多话,一辈子也要烂在自个儿肚子里,带到地下去了。''

惢心说者无心,如懿的太阳穴突突地跳着,像是被一根银针挑动了最痛楚的神经,她哑声道:``是金玉妍!一定是金玉妍!孝贤皇后的七阿哥莫名染上痘疫离世,玫嫔说是她自己做下的,可是她只是一个嫔位,哪里有能力做到这样左右逢源,天衣无缝!只怕,是因为她想着临死前谢了所有该谢的人,就像她一定要见本宫一般。所以\ldots\ldots 所以\ldots\ldots{}''

惢心一步上前,紧紧扶住被怒火与恨意烧得灼痛的如懿,隐忍着道:``皇后娘娘,如果孝贤皇后临死前的话是真的,许多事她没做过,那么如今的事,真的很可能是嘉贵妃指使,若是连孝贤皇后的七阿哥都能死得无声无息,那这个女人的阴毒,实在是在咱们意料之外。''她越说越痛,情不自禁俯下身抚摸着自己伤残的腿脚,切齿道:``皇后娘娘,她能害了奴婢和您一次,就能害咱们许多次。''

如懿紧紧地攥着手指,骨节发出咯咯的脆硬声,似重重叩在心上,她的声音并不如内心沸腾的火,显得格外平静而森:``惢心,无处防范是最可怕的事,只要知道了是谁,有了防范,便不必再怕。''

惢心垂着头,懊丧道:``只可惜,嘉贵妃有李朝的身份,轻易动她不得,只是,不能除去这样的人,日日在身边,真是芒刺在背。''

如懿摇了摇头,将无奈躁郁之情深深摁入情绪的最底处,轻吁道:``即便我贵为皇后,许多事也不能如愿以偿,眼下能做的,也唯有如此而已。''

她在踏出殿门的一刻,最后望向玫嫔沉浸在死亡中显得平和的脸容,有一瞬间的恍然与迷茫;若有来日,自己的下场,会不会比玫嫔好一点点?还是一样,终身限于利用和被利用的旋涡之中,沉沦到底?

\hypertarget{ux7b2cux5341ux4e5dux7ae0-ux521dux8001}{%
\chapter{第十九章 初老}\label{ux7b2cux5341ux4e5dux7ae0-ux521dux8001}}

玫嫔的丧礼办得极为草草,没有追封,没有丧仪,没有哀乐,更没有葬入妃陵的嘉遇,白布一裹便送还了母家。皇帝不过问,太后亦当没有这个人,仿佛宫里从来就没有过玫嫔,连嫔妃的言谈之间,也自觉地掩过了这个人存在的痕迹。

倒是数十日后,与如懿一起时,皇帝才淡淡问起:``那日送鸩酒,听说皇后亲自去了,玫嫔对你说了什么?''

如懿坐在曝光晴明底下,拈着一枚白玉棋子,专心于棋盘之上,不以为意道:``姐妹一场,终究得去送一送,玫嫔倒是说了几句,但都是疯话,不值得臣妾入耳,更不值得皇上入耳。''

皇帝含了若有若无的笑意:``疯话也是人话,说给朕听听。''

如懿支着腮,思忖片刻,郑重其事地下了一枚子,方才松了口气道:``玫嫔想知道,当年她死去的孩子长得什么模样?''

静室内幽幽泛着微凉,角落里放着一尊鎏金龙鼎炉,毓瑚捻着尺余长的细金箸,熟练拨弄中炉内浅银色的细灰,又撒落一把龙涎香,香料燃烧,不时发出轻微的``噼啪''之声,越发衬得四周的空气安静若一潭碧水,皇帝道:``只是这样?''

如懿扬起眼眸,平视着皇帝:``对于一个母亲来说,没能见到自己的孩子一面,是最大的缺撼,足以抱憾终身。''

墨玉的棋子落下时有袅袅余音,皇帝嘘一口气:``你告诉她了?''

如懿的目光微有悲悯:``这是她最后的心愿。''

皇帝微凉的手指像带着微湿的水汽,抚过她的手背:``皇后慈悲。''

如懿有难以言说的心绪,细细辩来,居然是一种畏惧:``是皇上慈悲,玫嫔自裁,皇上并未牵连她家人。''

皇帝的口气淡得如一抹云烟:``她也是一时糊涂。''

隐忍已久的哀凉如涌动于薄冰之下的冷水,无法静止。如懿只觉得齿冷,那种凉薄的心境,如山巅经年不散的浓雾,阴翳成无法穿破的困境,她终于忍不住道:``是。与其一世再这么糊涂下去,还不如自己了断了自己,由得自己一个痛快。''

如此寥寥几语,两人亦是相对默然了。殿中紫檀架上的青瓷阔口瓶中供着一丛丛茶蘼,雪白的一大蓬一大蓬,团团如轻绵的云,散着如蜜般清甜的雅香,垂落翠色的阴凉。置身花叶之侧,相顾无言久了,人也成了花气芬氲里薄薄的一片,疑被芳影静静埋没。幸好,意欢诞育的消息及时地拯救了彼此略显难堪的静默。李玉喜滋滋地叩门而入:``皇上大喜,皇后娘娘大喜,舒妃小主生了,是个阿哥!''

皇帝喜悦表情后有一瞬的失望:``是个阿哥?''

如懿及时地捕捉到了这一微妙的变化,笑道:``皇上跟前如今只有一个四公主,一定盼着舒妃生一个和她一般玲珑剔透的公主吧?其实阿哥也好公主也好,不都是皇上的骨血么?''

皇帝笑笑道:``甚好,按着规矩赏赐下去吧,叮嘱舒妃好好儿养着。朕和皇后晚上再去瞧她。''

李玉答应着,满面堆笑地下去了。

如懿轻声道:``皇上不高兴?''

棋盘上密密麻麻落满黑子白子,皇帝懒懒地伸手抚过:``没有。皇后多思了,只是有了那么多阿哥,又添上一个,没有从前那般欢喜罢了。''

彼时如懿与皇帝尚未踏足储秀宫,太后已经由福珈陪着去看了新生的十阿哥,欢喜之余更赏下了无数补品。其中更有一支千年紫参,用香色的宫缎精致地裹在外头,上面刺绣着童子送春来的烦琐花样,足有小儿手臂粗细,就连参须也是纤长饱满的------自然是紫参中的极品了。恰好嫔妃们都在,连见惯了人参的玉妍亦连连啧叹:``太后娘娘的东西,随便拿一件出来便是咱们没见过的稀罕物儿。''

福珈笑道:``可不是!这也算咱们太后压箱底的宝贝之一了,还是旧年间马齐大人在世的时候孝敬的。太后一直也舍不得,如今留着给舒妃小主了。''

意欢自然是感谢不已:``太后,臣妾年轻,哪里吃得了这样的好东西。''

太后笑叹着慈爱道:``自孝贤皇后去世后,皇帝一直郁郁不乐。你诞下皇子,这样让皇帝高兴的事,哀家自然疼你,且你生这个孩子受了多少的辛苦,临了生了,肚子里孩子的胞衣又下不来,硬生生让接生嬷嬷剥下来的,又受了一番苦楚,哀家疼你,更是疼皇帝和皇孙。''

意欢抱着怀中粉色的婴儿,仿佛看不够似的:``只要孩子安好,臣妾怎么样都是值当的。''

嫔妃们见太后如此看重,愈加奉承得紧,储秀宫中一片笑语连绵。

待回到自己宫中,嬿婉才沉下脸来,拿着玉轮慢慢地摩挲着脸颊,一手举着一面铜鎏花小圆镜仔细端详着,不耐烦道:``陪着在那儿笑啊笑的,笑得脸都酸了,也不知道有没有长出细纹来。''

澜翠正蹲在地上替嬿婉垂着腿,忙笑着道:``怎么会呢?小主年轻貌美,哪像舒妃在坐蓐,眼浮面肿,口歪鼻斜的。''

嬿婉丢下手里的小镜子,懒懒道:``舒妃哪里有你说的那么丑,本宫看她除了头发少些,也没什么大碍啊!''

澜翠不敢接嘴,却是春婵进来道:``小主,田嬷嬷来了。''

嬿婉神情一变,忙敛容正色道:``请她进来。''

田嬷嬷是个半老的婆子,穿了一身下人的服色,打扮得倒也干净,一看就是在宫里伺候久了的嬷嬷。十分世故老练,只是一笑起来,那笑容便能腻死个人。

嬿婉见她进来,倒也不急着说话,由着澜翠给田嬷嬷搬了张小杌子坐下,自已慢慢喝下了一碗冰豆香薷饮,才闲闲道:``如今天热了,不喝点子解暑消闷的东西,心里总是闷得慌。''

田嬷嬷忙同赔着笑脸道:``令妃娘娘说得是,这过日子谁没点儿闷着憋屈着的时候呀,奴婢这不就给您送痛快来了么?''

嬿婉的表情有些不大舒服:``舒妃不知道?''

田嬷嬷信心满怀:``这个自然,女人生下孩子之后,总得一刻钟到半个时辰的工夫,这胞衣才会娩出来。奴婢便假称舒妃小主的胞衣脱不下来,时辰未到就硬生生探手到宫体里给她硬扯了下来。''她得意地摆弄着右手道,``这一扯呀,手法可轻可重,奴婢的手一重,便是伤着宫体了,舒妃小主生下了十阿哥是她的福气,可再要生育,那便是再也不能了。''她说罢,眼巴巴地瞧着嬿婉,谄媚地笑,``这一切神不知鬼不觉的。小主的吩咐,奴婢做得还好么?''

嬿婉强忍着恶心与害怕,点点头:``做得是不错。可接生的嬷嬷不只你一个,还有太医在,你是怎么做到神不知鬼不觉的?''

田嬷嬷得意道:``人虽多,但奴婢是积年的老嬷嬷了,论起接生来,谁的资格也比不过奴婢。奴婢说的话,他们都得听着,都信。且太医到底是男人,虽然伺候在旁,却不敢乱看的,小主放心就是。''

嬿婉这才笑了笑,示意澜翠取出了银票给她:``三百两银票,你收好了。''

田嬷嬷笑得合不拢嘴,忙不迭将银票仔细叠好收进怀里。

嬿婉惋惜地摇摇头,撩拨着冻青釉双耳壶扁瓶中一束盛开的雪白茶蘼,那香花的甜气幽幽缠绕在她纤纤素手之间,如她的神情一般,``只是舒妃到底有神气,十阿哥平平安安,全须全尾地生下来了。''

``不能不生下来,那么多太医和嬷嬷在,又有太后万全的嘱咐,小主便容她一回吧。''田嬷嬷笑得有十足的把握,``只是生下来了,养不养得大还是一说呢。舒妃小主有孕的时候肾气太弱,生的若是个公主还好,可是个阿哥,那就难了。''

嬿婉眼中微微一亮,不动声色道:``真的难?''

``真的难!''田嬷嬷会心一笑,``那奴婢不扰小主歇息,先告退了。''

嬿婉凝视着田嬷嬷离去的背影,冷冷地笑了笑,任由微红的烛光照耀着她恬美容颜。

日子平静地过去,仿佛是随手牵同的大片锦缎,华美绚烂又乏善可陈。

玫嫔蕊姬与庆嫔缨络的事仿佛也一页黄纸,揭过去也便揭过去了。太后依旧是慈宁宫中颐养天年的太后,皇帝依旧是人前的孝子皇帝,连庆嫔身体见好后都依旧得宠,一切仿佛都未曾改变。唯一美中不足的是,意欢这一生生育到底伤了元气,头发也没长回来多少。皇帝虽然常常去看望意欢和新生的十阿哥,并且嘱咐了太医仔细治疗脱发之首,但甚少再传她侍寝。意欢将何首乌汤一碗碗地喝下去,效果也是若有若无的,幸好她一门心思都在孩子身上,得闲便整理皇帝的御诗打发时日,倒也不甚在意。

而十阿哥仿佛一只病弱的小猫,一点点风凉雨寒都能惹起他的不适,扯去意欢所有的心血精力,但,这也不过是漫长年岁里小小的波澜而已。日子就这样平静祥和地过着,仿佛也能过到天荒地老去

然而,打破这平静的,是平常而又不平常的一夜。

作为一个陪着同一个男人从少年同眠到中年的女人,如懿是难以忘却这特殊的一次的。

养心殿中小小一双红烛的火光跳跃着,照得双眼发涩。风凉雨软,吹得帐幕微微掀起,那灯光便又忽忽闪闪,这是一个寻常不过的秋天的夜晚,窗外天色阴沉,半点月光也没有,连星星都被银线般的雨丝淹没了,细雨绵延不绝地落在殿前的花树上,从树叶黄灿的枝条上溅起碎玉般凌冽的声音。

皇帝在她身上吃力地起伏着,分明已经汗流浃背了,却还是徒劳。如懿敏锐地发现了皇帝眼睛里深深的恐惧和迷乱,像一张布满毒丝的蛛网,先蒙住了他,然后蒙住了自己。

如懿的手指像春水一样在皇帝身上淙淙流淌,抚摸过他的面颊,他的耳垂,他的胸膛,她极力镇静着自已的心神,以此来面对皇帝从未有过的突如其来的失败。

皇帝的声音像漏着风,失去了一贯的沉稳笃定,变得软弱而胆怯:``如懿,如懿。''好似这样,便能唤回一点儿自信与精神似的。

如懿用明黄色赤线腾龙滑丝锦被遮住自己的身体,凝视着窗上一小块被雨淋湿的旋罗绢的窗纱,那种半干半湿的痕迹像某种开到糜烂的植物,散发着香气熏人而行将枯萎的气味,她的心绪烦躁而恐惧,有个念头秘不可示地转过,年过四十的皇帝,开始出现衰老的迹象。

皇帝绷紧的身体不受控制地松弛下去,成了一摊软绵绵的滑腻的肉,养尊处优多年,皮肉是光滑滑而富有弹性的,夹杂着力不从心后汗水黏腻的气味。她情不自禁地哀伤起来,对着这个比自己大了七岁的男子,可是,这样的情绪她又怎敢流露。终于,克制住心神,极尽所能地柔声道:``皇上日理万机,是太累了。''她替他掩好被子,``皇上,先睡一会儿歇一歇吧。''

皇帝把身体翻转过来,仰面朝着空茫无迹里的一点儿,嘴唇颤动着,摇着头说:``不是不是,我不相信。''

皇帝一向自重身份,对尊卑之分极为看重,很少在旁人面前自称是``我'',便是如懿陪伴他多年,在登基后的日子里,也极少极少听他这样自称。

他静了静,向外呼喝道:``李玉,李玉!朕的参汤呢?''

这样的呼喊含着某种暴戾的气息,李玉不知就里,忙端着参汤上来。皇帝一口气喝了,将珐琅戗金盖碗狠狠砸了出去,喝道:``滚出去!''

李玉吓得连滚带爬出去,皇帝还未等他将沉重的殿门合上,便再度翻上了如懿的身体,低低喝道:``再来!''

这证据是不容置疑的命令。皇帝的手势很用力,像发了狠劲在宣泄着什么似的。半透明的霞影纱帐下,被子上的腾龙仿佛是活的,缠绕着一个女人饱满的躯体,如懿忍着身上传来的痛楚,用力地咬着嘴唇,把那种声音变得更像是一种隐忍的不能克制的呻吟。她无法感受到欢悦的来临,只能死死盯着帐顶,微弱的烛火照在那帐上,上头所绘碧金纹饰,便泛起如七宝琉璃般的华彩。

那样的璀璨夺目在夜里看来像是锐利的芒刺,直刺入心似的。如懿一根一根数着穗子的数目,来抵挡无计可施的迷茫。良久,皇帝的精神气也没被那一碗参汤唤回来,他瘫下疲软的身体,虚弱而敷衍地亲了亲如懿的耳垂:``你来。''

如懿是懂得这句话的含意的,所以当她的唇吻上了皇帝的身体时,只觉得一把绯色的火影颤抖着在自己的血液里焚烧起来,恍如野火,把浓浓的夜色焚成了情欲的艳娆。

然而,是徒劳的,这把火终究没烧到皇帝的体内,最后,连皇帝自己也不耐烦了,推开了她,侧转了身。寝殿里很静,连平缓而迟钝的呼吸声都清晰可闻。皇帝不知是不是睡着了,他身上滚烫的气息逐渐散去,只剩下了冷汗流淌过的迹子,湿嗒嗒地腻。如懿摸索着悄无声息地换上了寝衣,裹着被子蜷缩成一团,偌大的床帐里,溢着一晕一晕昏黄的光,那寂寞和空虚也是一晕一晕地荡涤着,逐渐湮没了帐内的全部空隙。

如懿听着外头淅淅沥沥的雨声,倚在枕上暗自神伤。窗外的纱绣宫灯在夜来的风雨中飘摇不定,而庭院里的枯得有些蜷曲发黄的芭蕉和满地堆积的黄花上响起一片沙沙之声,这样的雨夜里,许多曾经茂盛的植物都在静静等待腐烂。

如懿黯然地想,原来好时光就是这样逝去的。不仅是精力,亦是肉体的颓靡,而她,竟然也和他这样慢慢地步入了不可预知的衰老,一步步走向白头,她这样念着,转过身,从背后拥住皇帝,很想对他倾诉,他会老,她亦会老。男欢女爱的欢愉终有一日会在他们身上逝去,那并不要紧,所谓的相濡以沫,并非只是以体液彼此温润,如果可以,绛纱帐内的十指相扣,并枕而眠,一夜倾谈,更能于身体痴缠的浅薄处,透出彼此相依为命的深情。

只是这样的话,她如何敢说,尤其是皇帝良久后寥落的一声:``如懿,朕是不是老了?''

她只得愈紧地拥住他,温言道:``不,皇上只是为国家大事操心,太累了。只要慢慢养着,你的精神会回来的。''

的确,皇帝这些日子是忙而累的。自从七月河南阳武十三堡黄河决口之后,皇帝便重新起用备受贬斥的慧贤皇贵妃的父亲高斌赴河南办阳武河工。这似乎意味着高氏家族的复恩之兆,高斌自然是尽心竭力去办这一桩河南阳武黄河决口合龙的辛苦差事。

前朝的事错综复杂,如懿虽然不喜高斌的复起,但也习惯了不轻易表达,皇帝倦倦地追问了一句:``是么?朕只是累了而已么?''

如懿用力颔首道:``自然,嘉贵妃不是又怀上身孕了么?皇上怎么会老呢?''

皇帝虚软地点了点头,如意绞金丝帐帷层层叠叠地垂落下来,把两个孤清的身影隔绝在芸芸众生之外,他们所拥有的,除了那高处不胜寒的唏嘘,还有世人都会有的,对于苍老逼近后的深深惶恐。

玉妍的再度有孕是在意欢诞下十阿哥不久之后,这个喜讯足以让复位后受过惩罚曾经一度惴惴不安的她再度趾高气扬起来。然而,再如何得意,对如懿亦不会再有一毫放松。

也是,对于一个入宫便恩宠不断的女子,在三十八岁的时候再度有孕,的确是让人万分欣喜的,这足以安慰了玉妍痛丧九阿哥的哀伤与难过,更意味着她在皇帝跟前长久的恩宠不哀。这一点,足以羡煞宫中所有的女子。

那一日,酷暑炎炎的天气下,玉妍兴致恹恹地看着嫔妃们一一向如懿请安,一手搭在腹部,似笑非笑地看着如懿,许久不肯起身。

如懿久在宫中,怎肯为这一点儿小事向她发作,遂也只是微笑:``若嘉贵妃伺候皇上伺候得手足酸软,本宫也不勉强嘉贵妃了。''

玉妍迎着她的目光站起身,慢悠悠抚着平坦的小腹,骄傲地抬起脸:``让皇后娘娘费心了。臣妾只是又有了身孕,所以起身才有些迟缓\ldots\ldots{}''她说着,便用势欲呕,赶紧有宫女七手八脚地替她端茶的端茶,抚胸的抚胸,忙作一团。

绿筠很有些看不上玉妍的矫情样子,拿绢子掩了掩鼻子,向着海兰轻声不屑道:``瞧她那样子,像谁没生过孩子似的。''

海兰贝齿轻露,微微一笑:``这个年纪还能有,当然不容易。''她说得轻婉,但咬在``这个年纪''四字上,让两个女人都忍不住哧哧地笑了起来。

玉妍并不理会她们,只是微斜了凤眼,瞟着嬿婉道:``其实本宫的雨露之恩哪时比得上令妃妹妹呢,只是令妃妹妹的肚子有点儿不大争气啊。''

这下庆嫔亦有些不悦:``令妃姐姐还年轻,不怕没有孩子。''

玉妍轻蔑地笑了笑,傲然道:``是么?''

如懿感受酷暑的烈日照透宫殿后那种薄薄的云翳似的微凉,她含着淡如浮云的笑意,徐徐道:``嘉贵妃不是第一次做额娘的人了,也不当心些,有话慢慢说就是了。''

玉妍娇俏一笑,直视着如懿,以倨傲的姿态相对:``臣妾一次次有身孕,让皇后娘娘费心,实在是过意不去。说来,皇后娘娘自己都没有孩子,还要了及臣妾的龙胎,恐怕真是费心不少了。''

玉妍手上的赤金红宝珠子护甲太过耀眼,在阳光下流转出针芒样的刺眼光芒,如她的话语一般让人觉得不悦。

如懿太阳穴的青筋倏地一跳,眼里闪过一丝黯然,容珮便笑道:``皇后娘娘抚养着五阿哥,又是所有阿哥公主的嫡母,自然是把每一位皇嗣都照顾得妥妥贴贴的。除了皇后娘娘,还有谁有,谁配操持这份心呢?只要嘉贵妃自己当心,龙胎在您肚子里自然是安安稳稳的。''

玉妍的眼风在容珮脸上凌厉一转,笑着抚了半月髻上的赤金流珠累丝簪:``可不是,皇后娘娘是所有皇嗣的嫡母,为了公平照顾,不偏不倚,哪怕委屈自已些暂时没有孩子,也是应当的,到底臣妾见识短浅,不及娘娘宅心仁厚,思虑深远。''

玉妍嘴上这样说,手却搭在自己腹部,露出无限得意之姿。如懿微微黯然,脸上却维持着一个皇后应有的威仪与和蔼,平视着前方,将自己无声的痛苦,默默地掩饰在平静之下。

玉妍得意扬扬地离开之后,如懿不无伤感地道:``平时总说嘉贵妃嘴上刻薄,人也轻佻,可是她的福气就这般好,伺候皇上这么些年,就一次接一次地怀上了龙胎,不管是男是女,那总是人为母亲的福气啊。''

容珮咬着唇,低声道:``会生孩子罢了,有什么了不起的。有娘娘在,她还能翻出天去。''

如懿愈加黯然。或许,昨夜皇帝意外的失败,更是昭示了她终身不可有孕的悲剧。她这样沉默着,脑海里盘旋着玉妍趾高气扬的笑声,忽然有些难掩地恶心。

但这样的情绪,是会让向来敏感的皇帝误会的,她只能极力忍耐着,无趣地想,这才九月初,怎么秋凉这么早就来了呢?

\hypertarget{ux7b2cux4e8cux5341ux7ae0-ux79bbux9699}{%
\chapter{第二十章 离隙}\label{ux7b2cux4e8cux5341ux7ae0-ux79bbux9699}}

这一夜半梦半醒,睡得便不大安稳。四更时分,皇帝起身,如懿便也醒了。皇帝一早便犯了起床气,脸色阴沉沉的,如同眼睛底下那一片憔悴的青晕一般,宫人们们伺候得格外小心翼翼,还是免不了受了几声呵斥。如懿想着是睡不着了,便起身亲自侍奉皇上更衣洗漱。一切停当之后,李玉便击掌两下,唤了进中端了一碗银耳羹进来。

这一碗银耳羹是皇帝每日早起必饮的,只为清甜入口,延年益寿。做法也不过是以冰糖清炖,熬得绵软,入口即化。

这一日也是如此。才用完银耳羹,离上朝还有一些时候,皇帝仍有些闷闷的。如懿见皇帝梳好的辫子有些毛了,想着皇帝不看见便好,一旦看见,那梳头的太监少不得是一顿打死。恰巧李玉也瞧见了,只不敢出声,急得满脸冒汗。

如懿灵机一动,便道:``皇上,臣妾好久没替您篦头发了。时辰还早,臣妾替您篦一篦,发散发散吧。'

皇帝夜来没睡好,也有些昏乏,便道:``用薄荷松针水篦一篦就好。''

皇帝对吃穿用度一想惊喜,所用的篦子亦是用象牙雕琢成松鹤延年的图案,而握手处却是一块老坑细糯翡翠做成,触而温润,十分趁手。如懿解开皇帝的辫子,蘸了点薄荷松针水,不动声色替皇帝梳理着头发。

然而在一切行将完成时,她却彻底愣住了。

皇帝乌黑浓密的发丝间,有一根银白的发丝赫然跃出,生生的刺着如懿的双眼。她反反复复地想着,皇帝才四十一岁啊,居然也有白头发了。

她下意识便是要掩饰过去。拔是不能拔的,否则皇帝一定会发现。但若是不拔,迟早也会被皇帝发现。这么一瞬间的迟疑,皇帝便已经敏锐的发现了,立刻问:``什么?''

如懿知道掩饰不过去了,索性拔下了那根白发,轻描淡写地道:``臣妾在想,臣妾的阿玛三十岁时便由白发了,皇上怎么如今才长第一根。''

这句话大大缓和了皇帝紧张的面色,他接过如懿手中的白发看了一眼,紧紧握在手心里道:``这是朕的第一根白发。''

如懿见皇帝并未大发雷霆,心头大石便放下一半:``圣祖康熙爷在世时很喜欢喝乌桑葚茶,臣妾也想嘱咐太医院做一些,皇上愿意讲究臣妾一起尝尝么?''

皇帝看她一眼,神色稍稍松驰:``皇后喜欢的话,朕陪皇后。''

如懿恍若若无其事般替皇帝结好了辫发,皇帝低低道:``再没有了吧?''

皇帝的语气是微凉的潮湿,如懿点点头,温柔道:``哪里来这样多,一根而已。臣妾倒想着,若臣妾与皇上都有了白发,那也算是白头到老了呢。''

皇帝笑了笑,静默着叹了口气,闭上了眼睛。

也难怪,皇帝素来极重养生之道,每日晨起必得先饮一碗银耳羹,早朝回来便在庭院中打一套五行拳舒筋散骨,午睡后照例是一碗浓浓的枸杞黑豆茶,晚膳后必含了参片养神片刻,到了睡前又是一碗宁神燕窝安眠。这些规矩,如懿跟了皇帝多年,也学了大半。除了不懂打拳,早晚也是如是保养。此外,皇帝连一饮一食都格外注意,喝酒不必多饮,更不曾醉,顶多喝一些太医院和御膳一起调制的龟龄酒喝松龄太平春酒,可活血安神,益气健身。而壮阳气的鹿肉更是膳食上最常见的东西,除此,便是十分清淡的新鲜时蔬了。

皇帝这般精心保养,最恨自己见老。此时见到自己华发暗生,又想起昨夜的失败,如何能不气恼伤感。如懿虽然有心开解,却也只能无言,这样静默着,她便又觉得有点恶心,只好极力忍耐着道:``皇上,时候不早,臣妾恭送您早朝。''

接下来一连数日,如懿便再难见到皇帝了,一查敬事房的记档,才知这些日子皇帝得空儿便在几个年轻的妃嫔那里,不是饮酒作乐,便是歌舞清赏。而去得最多的,便是嬿婉宫中。

容佩神神秘秘道:``最近嘉贵妃忙着替腹中的龙胎挑选乳娘,听说令妃宫中也悄悄挑了几个呢。

如懿正对镜敷着脂粉,闻言不觉停了手,疑惑道;``平白无故的,她要挑选乳娘做什么?''

容佩见四下并无其它人,压低了声音道:``听说皇上这几日都歇在令妃宫中,每日令妃都命奶娘挤了人乳,兑了奶茶给皇上喝。''

如懿入耳便不舒服,一个恶心,胸口有难言的窒闷,不禁弯了腰呕出了几口清水。

容佩吓得赶紧给她递了绢子擦拭:``皇后娘娘,您这是怎么了?这几日您的面色都不好看呢。''

如懿摇头道:``本宫是听着太恶心了。''

容佩忙道:``娘娘这几日老觉得胸闷不适,奴婢还是去请个太医来看看吧。''

如懿摇头道:``蕊心刚生了孩子正在坐月子呢,江与彬从两个月前便忙着照顾蕊心,本宫就干脆打发他回去休息三个月再回宫当差。除了他,本宫也不放心别人来请脉。也就是恶心一下,不打紧的。''

容佩犹豫地猜:``娘娘不会是有喜了吧?奴婢看娘娘这两个月月信未至,而且嘉贵妃也有喜了,就是这么恶心啊恶心的。''

如懿不以为然:``本宫这一世要真能有孩子便好了,只怕梦也梦不到。那月信\ldots\ldots 本宫一向是有的没有的,也惯了。''她撇开话,只管又问:``那些人乳皇上都喝了么?''

容佩有些不敢说了:``为了能延年益寿,青春常驻,皇上当然喝啊。令妃也陪着喝,还兑了珍珠粉,每天都不落下。''

如懿只觉得胸腔里翻江倒海似的,只差没再吐出来。她想起前几日绿筠看她的眼神,是那样的暧昧而揣测,只是心照不宣地彼此暗示,皇帝的身体起了异样。

而太医院得来的消息更让人震惊,除了大量进服补益强身的药物之外,皇帝已经开始每日饮用新鲜的鹿血酒了。

如懿是知道鹿血的功效的,鹿血主阳痿,益精血,止腰痛,大补虚损,和酒之后效力更佳。御苑中便养着百十头马鹿和梅花鹿,随时供宫中刺鹿头角间血,和酒生饮。先帝晚年沉迷丹药之时,亦大量地补服过鹿血,甚至在年轻时,因为在热河行宫误饮鹿血,才在神智昏聩之中仓促临幸了皇帝相貌粗陋的生母李金桂,并深以为耻,以致皇帝年幼时一直郁郁不得重视。

容佩忧心忡忡道:``皇上服用这么多鹿血酒,本就阳气太盛,若再频频临幸,只怕是上身哪!''

这样的话,宫中也只有如懿和太后劝得。然而皇帝却未必喜欢太后知道。如懿想劝,却又无从开口,沉吟许久才道:``容佩,去炖一碗绿豆莲心汤来。''

容佩讶异道:``皇后娘娘,已经入秋,不是喝绿豆莲心汤的时候啊!''

如懿拂袖起身,道:``本宫何尝不知道是不合时宜。但也只能不合时宜一回了。''

如懿进了永寿宫的庭院时,宫人们一个个如临大敌,战战兢兢。伺候嬿婉的太监王蟾端着一个空空如也的黄杨木方盘从内殿出来,见了如懿刚要喊出声,容佩眼疾手快,``啪''一个耳光上去,低声道:``皇后娘娘面前,少胡乱动你的舌头。''

容佩看了看他端着的盘子上犹有几滴血迹,伸出手来蘸了蘸一嗅,向如懿回禀道:``是鹿血酒。''她转脸问王蟾:``送了几碗进去?有一句不实的,立即拖出去打死!''

王蟾知道怕了,老老实实道:``四碗。''

里头隐隐约约有女子响亮的调笑声传出来,在白日里听着显得格外放诞而妖调。如懿听了一刻钟工夫,里头的声音渐渐安静了下来,方才平静着声气道:``谁在里头,请出来吧。''

王蟾慌慌张张的进去了,不过多久,门``吱呀''一声开了,几个艳妆女子鱼贯而出。

如懿原以为永寿宫中只有嬿婉,却不想出来的是平常在、揆常在、秀常在、晋嫔,一个个都在,又毛躁了鬓发,钗环松散。尤其是晋嫔,一颗织金缎玉片扣还送送地解开着,她自己却未发觉。

如懿见她们如此,可以想见寝殿之内皇帝一碗碗鹿血酒喝下去是如何的胡天胡地。她的脸色越发难看了起来,几乎是要破裂一般,冷冷喝道:``跪下。''

年轻的女子哪里经得起这样的脸色和言语。平常在、揆常在、秀贵人三个先跪了下来,晋嫔虽然有些不情愿,但也不敢一个人站着,只好也跟着跪了下来。

如懿不屑与她们说话,只冷着脸道:``好好想想,自己的错处在哪里?'

其余三人涨红了脸色低首不语,眼看窘得都要哭出来了。倒是晋嫔扭着绢子嘟囔道;``什么错处,不过是侍奉皇上罢了。'

如懿扬了扬唇角算是笑,眼中却清冽如寒冰:``孝贤皇后在世的时候最讲规矩,约束后宫。要知道她身死之后她的族人富察氏的女子这般不知检点侍奉皇上,那可真是在九泉之下都蒙羞了。''

晋嫔仗着这些日子得宠,气鼓鼓道:``臣妾伺候皇上,皇上也愿意臣妾伺候,有什么蒙羞不蒙羞的?皇后娘娘别是自己不能再皇上跟前侍奉讨皇上喜欢,便把气撒在臣妾身上吧?''

如懿似笑非笑道:``果然是富察氏家出来的,牙尖嘴利。''她扬了扬脸,容佩会意,上千揪住晋嫔的衣领子一扯,笑嘻嘻道:``晋嫔小主,光天化日的,您散着领口和皇后娘娘说话,您不觉得羞耻,皇后娘娘还替您觉得羞耻呢,这要传出去或是被人瞧见了,您富察氏家大族的颜面还要不要呢?''

晋嫔一低头,不觉含羞带气,手忙脚乱的地头扣上了纽子。

如懿扫了四人一眼,望着王蟾道:``怎么?就她们几个,永寿宫的主位呢?''

正问着话,嬿婉穿着一袭家常的桃花色直径地纳纱绣金丝风流散花氅衣,一壁急急地系着水色芙蓉领子,忙跪下了满面通红道:``不知皇后娘娘凤驾来临,臣妾未能远迎,还请皇后娘娘恕罪。''

如懿看了看她,发髻显然是匆匆挽起的,还有几缕碎发散在一边,几朵金雀珠花松松的坠着,犹自有些娇喘细细。

如懿心中有气,压低了声音道:``皇上呢?''

嬿婉一脸楚楚:``皇上刚睡下了,臣妾在旁伺候,不敢打扰。''

如懿问:``喝了四碗鹿血酒就睡了?''

嬿婉听她直截了当挑破,更不好意思,只得硬着头皮道:``是。''

如懿慢步上前,以护甲的尖锐拨起她的下巴,直视着她的眼睛道:``鹿血酒喝了是要发散的,你都不让皇上发散出来就睡下了,是成心要皇上难受么?''

嬿婉嗫嚅着唇,眼泪在眼眶里滴溜溜转折,半晌,声如细蚊:``已经发散了。''

``发散了?''如懿脸色骤然一变,又是心痛又是气急,``凭你们五个?''

嬿婉一脸无辜的望着如懿道:``皇后娘娘,臣妾也想劝皇上注意龙体,可是劝不住啊。皇上一定要累了,才肯睡过去。''

如懿逼视着她,沉肃道:``这些天,皇上都在永寿宫里,都是这样才肯睡下的?''

嬿婉窘得满脸紫涨,只恨不得找个地洞钻下去,看了看其余几人,道:``是。''

如懿的目光冷厉如剑:``这几个人中就属你位份最高,又是永寿宫的主位,偌大的永寿宫都归你处置。你若劝不住,大可来告诉本宫和太后。你存心不说便是居心不良,有意纵着皇上的性子来。''如懿唤过三宝:``三宝,去穿内务府的人过来记档。十六年十月初二未时二刻,令妃,晋嫔,秀贵人,平常在,揆常在于永寿宫侍寝。''

嬿婉登时脸色大变,面上红了又白,哀求道:``皇后娘娘留些脸面吧,皇上说了,今儿的事不记档。''

``不记档?''如懿的神色淡淡的,望着游廊雕梁上龙腾凤逐的描金蓝彩,并不看她们,``那若是你们几个之中谁有了身孕,那算怎么回事?没有记档的事情可是说不清的。''

嬿婉惨白了脸道:``就当是臣妾替晋嫔她们几个求求皇后娘娘了。这不是臣妾们几个的脸面,是皇上的脸面。''

如懿冷笑道:``皇上的脸面?皇上的脸面都被你们丢在永寿宫了。''

晋嫔犹自不服:``皇上就是要咱们几个伺候,那便怎么了?令妃娘娘有什么可怕的呢?我们是皇上的女人,伺候皇上是光明正大的。''

嬿婉急得狠狠瞪了她一眼,呵斥道:``你懂什么?''

如懿的目光扫视着她们,疾言厉色道:``晋嫔是不懂,但其中的厉害,令妃你是懂的吧。太后一旦查问起来,看了记档问皇上为何会有五女相陪,且是青天白日的这么不爱惜自己,你们这五条性命还要不要?淫乱后宫,迷惑皇上的罪名,是连你们母家的族人都要一起担待的。''

话音未落,只听见永寿宫正殿的大门霍然打开,一个气恼的声音道:``是朕要她们伺候的,一切都由朕担着。'

如懿见皇帝扬声出来,身上穿着一件蓝色江绸平金银缠枝菊金龙纹便袍,想试方才的话皇帝都听到了,便索性道:``皇上万福金安,臣妾恭请圣安。''

皇帝不耐烦道:``朕有什么安不安的,连个午觉都睡不安稳,听着你们吵吵闹闹,不成个体统。''

他这话虽然是对着众人说的,然而,目光只落在如懿身上。晋嫔立刻看懂了皇帝眼色,揉着膝盖娇声道:``皇上,臣妾跪得膝盖都疼了,臣妾能起来么?''

皇帝皱眉道:``大白天的,一排跪在滴水檐下成什么样子,回自己宫里去。''

晋嫔得意的扭着腰身站起来,朝着如懿横了一眼。如懿也不愿再众人面前再僵持着,便由着她们离开。晋嫔等人走得,嬿婉却走不得。

皇帝瞥了嬿婉一眼:``你还跪在这儿做什么?不是给朕炖了茯苓地黄大补汤么,还不叫人端了来?''

如懿使了个眼色,容佩端着绿豆莲心汤来。如懿尽力温婉了声线道:``皇上若是渴了,臣妾熬了绿豆莲心汤来,正好解渴。''

皇帝不悦的看了一眼:``又不是大伏天,送这么不合时宜的东西来作什么?''

如懿婉声道:``皇上这些日子连着进补鹿血,那东西的性子是热的。臣妾怕皇上烈性的东西喝的多了,所以特意送了性凉解热的绿豆莲心汤来,请皇上一尝。''

皇帝的目光倏然冷了下来:``皇后什么时候学会拐着弯子骂人了?'

如懿忙屈膝垂首:``皇上,臣妾不敢。'

``不敢?''皇帝冷哼一声,``你晚上扫朕的兴致,白天也来扫朕的兴致。你就这么容不得朕舒心一会儿吗?''

这句话仿佛一个突如其来的耳光,打得如懿晕头转向。她怔了半天,只觉得眼底一阵阵滚热,分明有什么东西要汹涌而出。她用尽了全身的力气咬住了唇,仰起脸死死忍住眼底那阵热流,以清冷相对,道:``是臣妾扫了皇上的兴致么?''

皇帝正被几个年轻貌美的嫔妃奉承得惯了,如何受得了这一句,不觉得冷笑连连:``皇后没扫朕的兴致,难道是令妃晋嫔她们扫了朕的兴致么?朕倒觉得,在她们面前,朕也年轻了许多,不像对着皇后,不温不火惯了。''

如懿只觉得自己的一颗心在芒刺堆里滚来扎去,扎得到处都痛,偏偏又拔不出来,却又实在忍不得这样的罪名和指责,只能低首道:``皇上的兴致若要一碗碗的鹿血酒喝大补汤吊着,臣妾也不敢劝皇上要爱惜身子这样的话了。臣妾立刻去奉先殿跪着,向列祖列宗请求宽恕便是。''

皇帝登时恼羞成怒,喝道:``你去奉先殿?就凭你是皇后么?''

如懿镇声道:``是!皇上封了臣妾为皇后,臣妾便不能不言。''

皇帝在懊丧中口不择言:``且不说你是继后,便是孝贤皇后这位嫡后在这里也不能扭了朕的性子!且你能去奉先殿做什么?去奉先殿告诉列祖列宗身为朕的皇后却不能绵延子嗣,为爱新觉罗氏生下嫡子嫡孙吗?皇后无能,无皇嗣可诞,朕为江山万代计,宠幸几个嫔妃又怎么了?''

是啊,她原本就是继后,哪怕是他亲自封了自己为后,心里到底也是这般瞧不起的。如懿满脸血红,一股气血直冲脑门儿:``臣妾无子是臣妾无能,但皇上不爱惜自己的龙体,便是对不起列祖列宗和天下苍生。''她接过容佩手里的汤盏捧过头顶,极力忍着眼泪道:``臣妾不敢有什么劝谏的话,所有臣妾要说的都在这碗汤里了。''

皇帝登时勃然大怒,拂袖而去,一盏绿豆莲心汤砸得粉碎,连着汤水淋淋沥沥洒了如懿满头满身。那碎瓷片飞溅起来,直刮到如懿手背上,刮出一道鲜红的血口子,瞬间有鲜血涌了出来。

嬿婉吓得花容失色,指着如懿的手背道:``血,皇后娘娘,有血。'''

如懿猛地擦去手背上的血液,浑身狼狈,却不肯放柔了语气,道:``臣妾这点子血,比起皇上的精血实在算不上什么,皇上生气,要打要罚臣妾无怨无悔,但皇上不爱惜自己,臣妾哪怕是覥着脸也要跪在这儿求皇上明白的。''

皇帝又气又恼,狠狠推了她一把:``你要跪便跪在这儿,少去奉先殿丢人现眼!''他转身吩咐:``令妃,跟朕进去,朕要你伺候着。''

如懿进退不得,直直跪在殿门前,看着嬿婉携着皇帝的手亲亲热热的进去。

容佩吓得脸色发青,忙陪着如懿跪下,低声道:``娘娘,您这是何苦呢?''

如懿望着那紧闭的门扇,镂花朱漆填金的大门,上面雕刻着栩栩如生的云幅八宝团纹,团花以芍药为心,五蝠衔银锭,灵芝,如意,菊花,珊瑚分布于四周,本是极热闹的华彩,却像是缭乱纷飞的蝙蝠翅膀上的刚刺,一扑一扑,触目惊心。

``何苦?''她怔怔地落下泪来,``皇上的龙体\ldots\ldots 难道是本宫的错吗?夫君不爱惜自己的身体,作为妻子不能劝一劝么?即便他是高高在上的君主,本宫是臣子,亦不能劝一劝么?''

容佩无言以对,只得踌躇着道:``出了这样的事皇上也不高兴,也在气恼性子头上,皇上他\ldots\ldots 不找自己亲近的人撒气找谁呢?''

如懿用力抹去腮边的泪:``所以,本宫就要忍受皇上当着妾室的面这样羞辱么?''

容佩扶住了如懿,忍耐着抹去眼角的酸涩。

\hypertarget{ux7b2cux4e8cux5341ux4e00ux7ae0-ux89c1ux559c}{%
\chapter{第二十一章
见喜}\label{ux7b2cux4e8cux5341ux4e00ux7ae0-ux89c1ux559c}}

嬿婉陪着皇帝进了寝殿,一下一下替皇帝揉着心口道:``皇上别生气了,皇后娘娘也只是气臣妾们伺候了您,所以才会一时口不择言的。''

皇帝闭着眼睛道:``你伺候朕也不是一天两天了,皇后一向挺喜欢你,今日是发了什么失心疯,一定要这么不依不饶?''

嬿婉伏在皇帝肩头,柔声道:``皇后娘娘也是关心皇上,皇上一碗碗的鹿血酒喝下去,别说皇后娘娘,臣妾看着都怕。''

皇帝暧昧地看她一眼,沿着她的手腕慢慢地摸下去:``怕?你有什么可怕的?''

嬿婉无限娇柔地一笑,咬着皇帝的耳垂道:``臣妾就是怕嘛,怕吃不消您。''

皇帝满脸的阴郁顿时消散,搂过她道:``朕原以为你和皇后容貌有些相像,可是仔细辨起来,你们俩的性子却完全不同。皇后是刚烈脾气,宁死不折;你却是绕指柔情,追魂蚀骨。''

嬿婉哧哧笑着,故意笑得大声,然后压低了声音娇滴滴道:``皇后娘娘的样子臣妾可是学不来。皇后娘娘如今的脾气这么刚烈,就是因为她一心只以为是您的妻子,是大清国的皇后,却忘了她和臣妾一样,都先是您的臣子您的奴才,然后才是伺候您的枕边人哪。''

皇上笑着在她脸上抚了一把:''你倒是懂事``

眉梢眼角缓然生出一段妩媚风情,嬿婉柔到了极处,几乎要化了去,嘤咛一声道:''不是臣妾懂事,是臣妾时时刻刻都记着,臣妾就是伺候您的,只要您高兴,臣妾做什么都愿意的。``

皇帝低低地在她耳边笑了声,说了一句什么,便道:''这样你也愿意么?``

嬿婉粉脸通红,娇羞地在皇帝胸膛捶了一下:''臣妾说了,为了皇上,臣妾什么都愿意。``

也不知跪了多久,秋末的毛太阳晒在身上轻绵绵的,好像带着刺,痒丝丝的,如懿望着门上云幅八宝团花纹,明明是五只一格的蝙蝠扑棱着翅膀,她的眼前一片花白,越数越多。五只\ldots\ldots 六只,十只\ldots\ldots{}

如懿轻轻地呻吟一声:``容佩\ldots\ldots 这些蝙蝠怎么多了\ldots\ldots{}''

她的话未说完,忽然身子一软,发晕倒了下去。容佩吓得魂飞魄散,死死抱住如懿惊呼道:``皇后娘娘!皇后娘娘!您怎么了?您别吓奴婢呀?''

如懿醒来时已经在自己的翊坤宫里。床前床后围了一圈的人,一个个笑脸盈盈的,连天青色暗织芍药春睡纱帐不知何时也换成了了海棠红和合童子牡丹长春的图案。那样喜庆的红色,绣着金银丝穿嫩黄蜜蜡珠子的图案,牡丹是金边锦红的,长春花也是热热闹闹簇拥着的淡粉色,密密的让她生厌。如懿只觉得身体轻飘飘地没个落处,头是晕乏的,眼是酸涩的,身上也使不上力气。她心下极是不耐烦,半闭着眼睛转过身去:``都笑什么,下去!''

却是皇帝的声音在耳边,喜气盈盈道:``如懿,你有身孕了!''

这句话不啻一个惊雷响在耳边,如懿急忙坐起身来。一起来才发觉自己起的急了,只怕是伤着了哪里,于是半僵着身体,瞪大了眼睛看着皇帝,犹自不信:``皇上说什么?''

然而皇帝是那样欢喜,方才在永寿宫的雷霆之怒全然化作了春风晴日。他握着如懿的手,有些愧疚:``如懿,你方才在永寿宫外晕了过去。朕赶紧抱了你回来让齐鲁一瞧,你已经有了两个多月的身孕了。''

嬿婉陪在皇帝身后,满面的笑中有些畏惧:``皇上一听说娘娘发晕,急得什么似的,丢下了臣妾就抱着娘娘冲出了永寿宫。''

容佩忙挤上前来替如懿在身后垫了几个垫子,把令妃挤到了身后,道:``娘娘仔细凤体,慢慢起身。''

如懿脑中有一瞬的空白,什么也反应不过来,仿佛是在空茫的大海上飘荡着。怎么会有孩子呢?怎么会有孩子呢?

如懿慌慌张张地抚着肚子,肚子是平坦的,怎么就会有孩子在里头了呢?可若不是有了孩子,皇帝怎么会这样高兴?她急忙唤道:``江与彬呢?''

齐鲁忙膝行向前道:``皇后娘娘安心,江太医还在家中呢。微臣已经跟皇后娘娘搭过脉了,确实是有了身孕无疑。但皇后娘娘之前未有生育,这是第一胎,一定一定要格外小心。''

皇帝的心情极好,朗声道:``齐鲁,朕便把皇后的身孕全权都交与你了。若是有一点儿错失\ldots\ldots{}''

齐鲁赶紧趴下了身体道:``微臣不敢,若有闪失,微臣便不敢要这条老命了。''

皇帝笑道:``那就好。皇后一向是由江太医请平安脉,你便和他一起照顾着,以求万全。''

如懿的神色还是有些疲乏,并不愿十分搭理皇帝,连笑也是一抹淡淡山岚。还是李玉乖觉:``皇后娘娘可是乏了?奴才立刻让齐太医去熬上好的安胎药,娘娘好好儿歇一会儿吧。''

嬿婉忙堆了一脸柔绵的笑容,道:``那臣妾伺候皇上先回永寿宫吧。晚膳备好了,是皇上最喜欢的炙鹿肉呢。''

如懿的眼光飘渺拂过嬿婉的脸,皇帝清了清嗓子,道:``这些日子都是鹿肉啊野鸡啊,朕都吃絮了,不去了。''

嬿婉还欲陪着皇帝,有些眷恋不舍。皇帝也不看她,摆手道:``你先跪安吧,朕想陪陪皇后。''

嬿婉只得讪讪告辞。众人散去后,皇帝对着如懿作小服低:``如懿,朕今日是在永寿宫喝酒昏了头了。''

如懿侧身朝着里头,淡淡道:``皇上是喝多了酒,臣妾会让容佩熬好了醒酒汤给皇上的。请皇上恕罪,臣妾怀着身孕,怕酒气过给了孩子,还请皇上去暖阁歇着吧。''

如懿眼里浮起些许内疚,像浮于春水之上逐渐融化的碎冰:``如懿,你别生气,会伤者你腹中咱们的孩子的。''

如懿心中一酸,抚着肚子发怔。是啊,若不是这个孩子,今日她又会到什么田地呢?明明不是她的错,他却能轻而易举的将所有的错处都落在她身上,在妾室们面前这样折辱她。

她眼中酸极,像小时候那手剥完了青梅又揉了揉眼睛,几乎逼得她想落下泪来。可是落泪又能如何呢?她在永寿宫前落了再多伤心痛惜的泪也无济于事,若不是这个孩子,她的伤心担忧,不过也都是白费而已。

她望着帐上浮动的幽影,轻声道:``若不是臣妾突然有了这个孩子,皇上也不会对臣妾这样说话吧?''

皇帝略略有几分尴尬:`如懿,朕不喜欢你这样。``

如懿长叹一声:'臣妾让皇上不喜欢的地方太多了。臣妾不过是继后,人微言轻,行事莽撞,难免让皇上不喜欢。''

皇上轻吁道:``皇后,你真要为朕一句醉话计较到这种地步吗?`

如懿侧过身子,未语,泪先涌出:'臣妾怎刚计较皇上,臣妾只是计较自己。皇上不爱惜自己的身体,无非是臣妾无能而已,臣妾还有何面目见皇上呢?''

皇帝的神色有几分伤感,仿佛凝于秋日红叶之上的清霜:``如懿,朕是皇帝,也是男人。所有男人到了这个年纪,都会有力不从心的时候。朕着急,也生气,那是对着自己的。人啊,气急交加的时候,说什么话,做什么事,都是糊涂了的。你若在这个时候计较朕的糊涂,朕也无话可说。今日的事,朕是纵情任性了些,但几个年轻嫔妃在侧,朕一时兴致上来,她们也没劝\ldots\ldots{}''他有些尴尬,说不下去,``总之,朕不再那样了就是。''

如懿垂下的眼眸微微一扬:`那臣妾不为别的,只为皇上说的这一句,皇上一时兴致上来,她们也没劝。臣妾就不得不给令妃和晋嫔一个教训,``

皇帝沉吟片刻,笑道:''只要你高兴,你腹中的孩子高兴,朕没什么可说的。``

如懿故意盯着他:''皇上不心疼?``

皇帝笑,一字一字咬重了道;'自然。你是朕的正妻,责罚妾室,朕有什么可心疼的。''

如懿爽然道:``那么,臣妾就请皇上允准,自今日起至臣妾平安诞下孩子满月之后,令妃、晋嫔全数罚奉,秀贵人、平常在、揆常在罚奉一半,如何?`

皇帝笑着抚上如懿的小腹,亲昵道:``朕都由得你。''

如懿半笑着唏嘘道:``有什么由不由得臣妾的,只要皇上爱惜龙体,保养自身,臣妾便什么话都没了。''

殿中有清明的日光摇曳浮沉,初秋的静好时光便渐渐弥漫开来。这一切似乎是那样完满,自然,也只能一味它是完满的。

海蓝和意欢结伴来看望如懿时,如懿正倚在长窗的九枝梅花塌上,盖着一床麒麟同春的水红锦被,看着菱枝领着小宫女们在庭院里收拾花草。

各宫嫔妃都来贺喜过,连太后也亲在来安慰了。如懿应付的多了,也有些疲乏。用过午膳,也许是有孕的缘故,总是懒怠动弹。宫人们虽都在外头忙活,但个个屏气吸声的,一丁儿点声音都没有,生怕惊扰了她静养。于是,翊坤宫中也就静得如千年的古刹,带着淡淡的香烟缭绕的气息,静而安稳。

如懿戴着银嵌宝石碧玉琢蝴蝶纹细钿子,里头是烟霞色配浅紫瓣兰刺绣的衬衣,身上披着玫瑰紫刺金边的氅衣,春意融融的颜色,偏又有一分说不出的华贵,长长娥衣摆拖曳在松茸色地毯上,仿佛是被夕阳染了色的春溪一般蜿蜒流淌。

暖阁内的纱窗上糊着``杏花沾雨''的霞影纱,在寂寞的秋末时节看来,外头枯凉的景色也被笼罩在一层浅淡的杏雨蒙蒙,温润而舒展。

海蓝比意欢早跨进一步,欲笑,泪却先漫上了睫毛。她在如懿身边坐下,执了如懿的手含泪笑道:'想不到,原来还有今日。''

意欢忙笑道:``瑜妃姐姐高兴过头了。这是喜事,不能哭啊!''她虽这样说,眼眶也不觉湿润了:``皇后娘娘别嫌咱们来得最晚。一大早人来人往的,人多了都是应酬的话,咱们反而不能说说体己话了。''

如懿忙挽了意欢的手坐下;''多谢你们,沾了你们的福气。``

海蓝忙拭了泪道:''皇后娘娘,等了这么多年\ldots\ldots``

是啊,等了这么多年,梦了这么多年,无数次在梦里都梦见了抱着自己的孩子的那种喜悦,可最后,却是一场空梦。梦醒后泪湿罗衫,却不想,还有今日。

意欢接口道:''只要等到了,多晚都不算晚。``她不免感触,''皇后娘娘等到了,臣妾不也等到了么?一定会是个健健康康的孩子。''

意欢穿着湘妃竹绿的软缎滚银线长衣,袖口略略点缀了几朵黄蕊白瓣的水仙。发髻上也只是以简单的和田玉点缀,雕琢着盛开的水仙花。那是她最喜爱的花朵,也极衬她的气质,那样的凌波之态,轻盈亮洁,便如她一般,临水照花,自开自落的芬芳。她从袖中取出一个一盘花籽香荷包,打开抖出一串双喜珊瑚十八子手串,那珊瑚珠一串十八颗,白玉结珠,系珊瑚杵,翡翠双喜背云,十分精巧可爱。

意欢含笑道:``这还是臣妾入宫的时候家中的陪嫁,想来想去,送给皇后娘娘最合适了。''

海兰笑着看她:``你轻易可不送礼,一出手就是这样的好东西。''

如懿推却道:``既是你的陪嫁,好好儿收着吧。等十阿哥娶妻的时候,传给你的媳妇儿。''

意欢从来对嬿婉也只是淡淡的,如今更多了几分鄙夷之色,失笑道:``那里等的到那时候,臣妾也不过是什么人送什么东西罢了。虽说令妃每常和咱们也有来往,可她若怀孕,臣妾才不送她这个。''

海兰从藕荷色缎彩绣折枝藤萝纹衣的纽子上解下闪色销金绢子扬了扬,嫌恶地道:''好端端的,提她做什么?''

意欢轻轻啐了一口,冷然道:``要不是她这么狐媚皇上,今日娘娘在永寿宫也不会受这么大的罪过。若是不小心伤了腹中的孩子可怎么好?''

说起这个来,海兰亦是叹气:``皇上年过不惑,怎么越来越由着性子来了呢?''她看着如懿道:''娘娘有时便是太在意皇上了。许多事松一松,也不至于到今日这般剑拔弩张针锋相对的时候,平日让令妃和晋嫔她们看了笑话。``她犹疑着道,''其实皇上要多喝几口鹿血酒要寻些乐子,便也由着他吧.''

意欢咬了咬贝齿,轻声而坚决道:``臣妾说句不知死活的话,今日若是臣妾在皇后娘娘这个位置,也必是要争一争的。''

海蓝瞪大了眼道:``你是指太后会责怪皇后娘娘不能进言?''

意欢摇摇头,微红了眼圈:``不只是太后,便为夫妻二字,这些话便只能由皇后娘娘来说。''

海兰沉默片刻,叹息道:``说句看不破的话,你们呀,便是太在意夫妻二字了。无论民间宫中,不过恩爱时是夫妻,冷漠时是路人,不,却连路人也不如,还是个仇人呢。凡事太在意了,总归没意思。''

一席话,说得众人都沉默了。海兰只得勉强笑道:``臣妾好好儿地又说这个做什么?左右该罚的也都罚了,臣妾过来的时候,还听见晋嫔在自己宫里哭呢。也是,做出这般迷惑圣心的事来,真是丢了她富察氏的脸面!''

她唤过叶心,捧上一个朱漆描金万福如意盘子,垫着青紫色缎面,内中放着二十来个颜色大小各不同的肚兜,有玉堂富贵、福寿三多、瑞鹊衔花、鸳鸯莲鹭、锦上添花、群仙贺寿,还坠着攒心梅花,蝉通天意、双色连环、柳叶合心的串珠络子,簇在一堆花团锦簇,甚是好看。

如懿拣了一个玉堂富贵的同心方胜杏黄肚兜,讶异道:``哪里来这么些肚兜,本宫瞧这宝照大花锦是皇上刚登基时内务府最喜欢用的布料,如今皇上用的都没有这么精细的东西了,你一时怎么找出来的?''

海兰抿着嘴儿笑道:``只许娘娘盼着,也不许臣妾替娘娘想个盼头么?从臣妾伺候皇上开始,就替娘娘攒着了。一年只攒一个,用当年最好的料子,挑最好的时日里最好的时辰。臣妾就想着,到了哪一年,臣妾绣第几个肚兜的时候,娘娘就能有身孕了。不只不觉,也攒了这些年了。''

如懿心中感动,比之皇帝的喜怒无常,情意寡淡,反而是姐妹之间多年相依的绵长情意更为稳笃而融洽。或许怀着这个孩子,也唯有海兰和意欢,是真心替她高兴的。她爱惜地抚着这些肚兜:``海兰,也只有你有这样的心意。''她吩咐道:``容佩,好好儿收起来,等以后孩子长大了,都一一穿上吧.''

海兰眉眼盈盈,全是笑意,道:''其实皇上赏的哪里会少,臣妾不过是一点儿心意罢了。娘娘只看舒妃妹妹就知道了,自从生下十阿哥,皇上没个三五日就要赏赐呢!``

意欢虽然带着澹澹的笑意,眼角眉稍却添了几分薄雾似的惆怅。她不自觉的伸手摸了摸自己的发髻,虽然是用了假发,但那把青丝还是看起来薄薄脆脆的,让她昔日容颜失色了不少。''东西是赏了,可人却少见了。从前总以为多年相随的情分,到头来也不过是以色侍人罢了。若不是这个孩子,只怕臣妾早已经闭锁深宫,再不的见君颜了。``

此话亦勾起海兰的愁意:''不过有个孩子总是好的。红颜易逝,谁又能保得住一辈子的花容月貌呢?不过是上半辈子靠着君恩怜惜,下半辈子依仗着孩子罢了。比起婉嫔无宠无子,咱们已经算是好的了。``

如懿怅然道:''你们说的何尝不是。没有孩子,哪怕本宫位居皇后至尊,也是如风中残烛,岌岌可危。``

海兰与意欢相对默然,彼此伤感。半晌,意欢才笑了笑道:''瞧咱们,明明是来给皇后娘娘贺喜的,有什么可不高兴的。只盼着娘娘放宽心,平平安安生下个小阿哥才好呢,也好给五阿哥和十阿哥做个伴儿啊!``

如懿亦笑:''可不是。五阿哥虽然养在本宫膝下,但本宫如今有孕,怕也顾不上。还是海兰自己带回去照顾方便些吧。``

海兰接了永琪在身边,自然是欢喜的,于是聊起养儿的话来,细细碎碎又是一大篇,直到晚膳时分,才各自回宫去。

翊坤宫中一团喜庆,中宫有喜,那是最大的喜事。皇帝择了良辰吉日祭告奉先殿,连太后也颇为欣慰:''自从孝贤皇后夭折两字,中宫新立,也是该添位皇子了。``

而几家欢喜几家愁。永寿宫中却是一片寂静,半点儿声响也不敢出。

嬿婉忍着气闷坐在榻上,一团木樨血燕羹在手边已经搁的没半点儿热气了。春蝉小心翼翼劝道:''怒气伤肝,小主还是宽宽心,喝了这碗血燕羹吧。``

嬿婉恼恨道:''喝了这碗还有下一碗吗?停了本宫这么久的月俸,以后眼看着连碗银耳羹都喝不上了,还血燕呢?``她想象更加气恼,''偏偏本宫的额娘不知好歹,又来跟本宫伸手要钱。钱钱钱,哪里变出这么多钱来,难不成还要去变卖皇上给的赏赐吗?``

春蝉半跪着替嬿婉捏着小腿道:''瘦死的骆驼比马大,何况皇上喜爱小主,明里暗里的赏赐下来,小主还在乎这点月俸吗?``

嬿婉愁眉不展,道:''月俸虽小,也是银子。在宫中哪里不要赏人的,否则使唤得动谁?银子流水价出去,本宫本来就没有个富贵娘家,一切都指望着皇上的赏赐和月俸,如今少了这一桩进项,到底难些。``

春蝉帮着出主意道:''那也没什么。有时候织造府和内务府送来孝敬的料子堆了半库房呢,咱们也穿不了那么多,有的是送出去变卖的法子。左右也不过这一年,等皇后娘娘出了月子合宫大赏的时候,多少爷熬出来了。``

嬿婉听到这个就有气,顺手端起那碗木樨血燕羹便要往地下砸,恨道:`舒妃生了阿哥,皇后也有孕!为什么只有本宫没有?明明本宫最年轻,明明本宫最得宠!为什么?为什么本宫偏没有?''

春蝉吓得立刻跪在地上,死死拦住嬿婉的手道:``小主,小主,奴婢宁可您把奴婢当成个实心肉凳子,狠狠砸在了奴婢头上,也不能有那么大动静啊!''

嬿婉怔了一怔,手悬在半空中,汤汁淋淋漓漓地洒了春蝉半身,到底也没砸在地上。春蝉瞅着她发怔的瞬间,也顾不得擦拭自己,忙接过了汤羹搁下道:''小主细想想,若被外人听见,皇后娘娘有孕这么高兴的时候您却不高兴了,那要生出多大的是非啊。好容易您才的了皇上那么多的宠爱呢。皇后娘娘这个时候有孕也好,她不便伺候皇上,您便死死抓住皇上的心吧。有皇上的恩宠,您什么都不必怕。``

嬿婉缓缓地坐下身,解下手边的翠蓝绡金绫绢子递给她:''好好擦一擦吧。本宫架子上有套新做的银红织金缎子对襟配蓝缎子裙儿,原是要打发给娘家表妹的,便赏给你穿了。``

春蝉千恩万谢地答应了,越发殷勤伺候不停。

\hypertarget{ux7b2cux4e8cux5341ux4e8cux7ae0-ux6b22ux7231}{%
\chapter{第二十二章
欢爱}\label{ux7b2cux4e8cux5341ux4e8cux7ae0-ux6b22ux7231}}

然而,如懿的有孕,并未让嬿婉有意料之中的继得君恩。皇帝仿佛是含了对如懿的愧意,除了每日取陪如懿或是玉妍用膳,平日里便只歇在绿筠和庆嫔处。连太后也不禁感叹:``日久见人心,伺候皇帝的人还是要沉稳些的好,便足见庆嫔的可贵了。那日永寿宫那样胡闹,到底也不见庆嫔厮混了进去。''

这番话,便是对嬿婉等人婉转的申斥了。如此,皇帝亦不肯轻易往这几个人宫中去,只耐着性子保养身体,到底也冷落了下来。

在得知如懿的身孕不久之后,皇帝便开始了一次隆而重之的选秀。三年一次的选秀时祖宗成例。可是皇帝登基后一直励精图治,将心思放在前朝。且又有从宫女或各府选取妙龄女子为嫔妃的途径,所以一直未曾好好选秀过一次。如今乍然提出,只说是以太后六旬万寿之名选取秀女侍奉宫中,太后与如懿虽然惊愕,也知是祖宗规矩。且自皇帝冷落了嬿婉等人,如懿和玉妍也有孕不便伺候皇帝,宫中只几个老人儿侍奉也很不成样子,便也只能由着皇帝的性子张罗起来。

因着如懿有孕不能操劳,太后又安于享受六十大寿的喜庆,所以便由内务府和礼部操办,皇帝亲自选定了人选。

容佩私下里对如懿道:``选秀本该是皇后娘娘主持的事,皇上却连露面都不允,可是恼了皇后娘娘上回送绿豆莲心汤之事?''

如懿扶着腰肢慢慢在庭院中踱步,抚着一枝开得茂盛的金桂道:``事无完全,你若以为皇上是有心冷落,削了本宫的皇后颜面,那便是如此。你若以为皇上只是体贴本宫有孕,那便也是皇上的一番苦心了。''

太后寿辰之前,皇帝选了巡抚鄂舜之女西林觉罗氏为禧常在,都统纳亲之女巴林氏为颖贵人,拜唐阿佛音之女林氏为恭常在,德穆齐塞音察克之女拜尔果斯氏为恪常在。

许是因为宫中汉军旗女子不少,皇帝此次所选多为满蒙亲贵之女。如懿在皇帝处看到入选秀女的名单时,不觉笑道:``这是皇帝第一次选秀,怎么费了这么大的劲儿,只选了4个出来?''

皇帝笑道:``这便够了。选了4个,四角齐全就好。''

如懿换了个舒服的姿势坐着,轻笑道:``那想必个个都是才貌双全的美人儿了。只是臣妾想着,皇帝今春刚南巡回来,会多选几个汉军旗的女孩子呢。''

皇帝将内务府定好的封号给了如懿看,道:``西林觉罗氏是满军旗,林氏虽然是汉军旗的,但她阿玛拜唐阿佛音是蒙军旗的,拜尔果斯氏和巴林氏也都是蒙军旗的。皇后看看,宫室该如何安排?''

如懿思忖着道:``自从先帝的乌拉那拉皇后过身之后,景仁宫一直空着,倒也可惜。还是慧贤皇贵妃的咸福宫。臣妾想着,不如让恭常在和禧在住景仁宫,颖贵人和恪常在住咸福宫。''

皇帝道:``那也好。即日着人打扫出来吧。尤其颖贵人和恪常在是蒙古亲贵之女,布置上要格外有些蒙古的风味。''

如懿笑盈盈颔首:``是,皇上不久才刚在前朝平定西藏郡王珠尔默特那木札勒叛乱之事,如今准格尔内讧,正在蠢蠢欲动,这样的人选,倒是对满蒙尤其是蒙古各部极好的安抚。''

皇帝搁下笔,意味深长地看了如懿一眼,口气温和关切而不容置疑:``皇后有着身孕,才三个月吧,还是不宜多思,尤其是前朝的闲话,也不要多听。''

如懿心头徒地一跳,忙欠身道:``臣妾也只是随口说起选秀的家事,若惹皇上不悦,是臣妾的过失。''

皇帝笑了笑,那笑影却未曾弥漫到眼睛里,只是道:``皇后有孕辛苦,还是早点儿回宫休息吧。朕去瞧瞧庆斌。''说罢,起身便传轿出去。

如懿看着皇帝的身影,不觉百感交集,抚着小腹,神色黯然。这便是君恩了,虽则有了身孕,虽永寿宫那场风波,到底是伤了里子了。''

借着这样的由头,十一月太后的六旬万寿,皇帝亦是办的热热闹闹,风光无限,除了循例的歌舞献寿,奉上珍宝之外,更在太后的徽号``崇庆慈宣''之后又加四字``康惠敦和'',便尊称``崇庆慈宣康惠敦和''皇太后。

然而,如懿亦知,这样的尊荣背后,更是因为太后的长女端淑长公主嫁在了准格尔内讧颇有牵制之效,皇帝才会如此歌舞升平。但太每每关心起端淑之事,皇帝便笑着挡回去:``妹妹一切安好,又有公主之尊,皇额娘什么都不必担心。''

到了十二月里,新人入宫,皇帝颇为垂卒,侍寝也常常是这四人。其中颖贵人长得杏眼樱口,脸若粉雪,年轻娇憨又带了几分草原的泼辣爽利,格外得皇帝的喜欢,近新年时便封了颖嫔,可谓一枝独秀。如此,嬿婉日渐被冷落,日子也越发难过了。

年下时天气寒冷,接连下了几场雪,皇帝索性除了养心殿,便只宿在咸福宫力,嬿婉益发不得见皇帝,不觉也着急起来。然而颖嫔出得恩宠,却也有些手段,和恪常在将皇帝围得水泄不通,嬿婉如何能见到,去了咸福宫几次,反而被颖嫔瞧见受了好些闲话。``令妃放心,皇上在我这儿好好的,怎么也不会贪喝鹿血酒了。''

颖嫔风头正盛,嬿婉也只得悻悻的回来了。这一来,嬿婉气急交加,少不得吩咐春蝉唤了田嬷嬷过来说话。

田嬷嬷倒也还殷勤,见了面便说笑:``小主这个时候唤奴婢过来,可是看上了嘉贵妃身上的胞衣?算着嘉贵妃可也快生了呢。''

嬿婉一时也不接话,只往桌上一指。那里原放着一匣子银子,嬿婉扬了扬脸,澜翠又添上一小盆珠宝,看得田嬷嬷的眼睛都直了。

嬿婉笑道:``听说田嬷嬷的独生儿子要捐前程了,这些东西正好帮得上忙吧!''

田嬷嬷收回了直要黏到那些珠宝上的目光,会心一笑,道:``小主要什么,直说吧。奴婢一定尽力而为。''

嬿婉含笑抿了口茶;``嘉贵妃的胞衣本宫不在意,要就要最好的。皇后身上那张,如何?''田嬷嬷愣了愣,像被针扎了似的赶紧缩回几欲抚上那些银子的手,咋舌到:``小主的意思是,像对着淑妃那样如法炮制?''

嬿婉抚了抚鬓边一对金蔓枝攒心碧玺珠花,慢条斯理到:``皇后娘娘生产,嬷嬷资历最深,一定会去接生的。一回生二回熟,嬷嬷熟能生巧,一定能再次做的神不知鬼不觉。''

田嬷嬷脸都不敢抬起来:``小主,那可是皇后娘娘!''

``一样是女人,有什么不同的?对着舒妃你敢下手,对着皇后就不敢了?''

嬿婉莞尔一笑,``本宫也没叫你杀了皇后腹中的孩子,只是希望皇后不要再生育罢了。皇后娘娘三十多岁了,生了一胎再不能生,也不奇怪啊!没人会疑心你的。''她伸出纤细的这么一剥,撕下胞衣,扯伤了宫体,一了百了。''

田嬷嬷吓得脸都变了,腿脚一软就跪在嬿婉跟前,哀求道:``令妃娘娘,可不敢啊!那不是旁人,是皇后娘娘!''

嬿婉扬了扬青黛色的柳眉,不屑道:``舒妃也是宠妃,你怎么敢?''

田嬷嬷伏在地上拼命磕头:``舒妃小主是叶赫那拉氏的,不比皇后娘娘是中宫国母。而且皇后娘娘是头胎的嫡出,皇上这么郑重,还去奉先殿祈福祷告了。连太后平日里那么不待见皇后娘娘,也嘘寒问暖,关怀备至。这个节骨眼上,便是杀了奴婢也不敢啊1''

嬿婉见她磕的额头也青了,怕旁人见了要问,忙止住道:``好了1`

田嬷嬷吓得忙跪直了身体,直瞪瞪得看着嬿婉。嬿婉烦恼地摆了摆手;'罢了。本宫不过随口问了一句,你不愿便算了。澜翠,好好儿送田嬷嬷出去。`

澜翠答应着半搀半扶拖了田嬷嬷出去,春蝉见嬿婉一脸阴郁,便递了茶上前低声道:''其实要田嬷嬷做也不难,就拿她上回害舒妃的事要挟她,谅她也不敢不对皇后下手。``

嬿婉托腮凝神,道:``田嬷嬷是个派得上用场的人,逼急了她,以后一拍两散,对谁都没有好处。本宫没有娘家,宫里能用的人有一个算一个,都得用上。''

春蝉愤愤,亦为难道:``皇后娘娘害的小主没有自己的孩子,她和舒妃却一个个都怀上了咱们难道一点法子都没有吗?''

嬿婉望着窗外黑漆漆的夜色,恨恨道:``本宫也不敢弄死了皇上的孩子,只是要她们尝尝和本宫一样生不出孩子的痛苦罢了。''清冷的月光洒落在她有些憔悴的泛着鸭蛋青的脸上,``哎,要是皇上肯来,本宫也不比那么难过了。要紧的,还是君恩阿!''

然而,天际唯有一抹云翳,淡淡遮蔽了那抹淡月的痕迹。清冷的永寿宫,仿佛连一点儿月光的照拂也不能得了。

如懿怀到第六个月时,额娘便进宫来陪伴了,如懿是知道皇帝的恩典,亦是替皇帝陪伴着已经数月不能侍寝的自己。

太后派遣了福珈姑姑来看时亦笑:``到底皇后娘娘好福气。先头孝贤皇后在世时,也只有在潜邸生二阿哥时娘家的额娘进来陪过,到底也不是入了宫里这般郑重其事呢。''上了年纪的人,论起生儿育女的事来又是一大篇话,福珈姑姑又是个极健谈的,一口一个``承恩公夫人,''直哄得如懿的母亲十分开怀。

待到人后,母亲问起女儿生男生女来,如懿亦是一脸淡然:``太医说起来,仿佛是个公主。''

母亲便怔了一怔,犹自不敢相信:``是哪位太医说的,准不准?''

如懿倒不甚放在心上:``皇上也问起过女儿,但侍奉女儿的太医齐鲁和江与彬,一个是老练国手,一个是后起之秀,都是在太医院里数一数二的。''

母亲的脸色便有些不好看,半晌叹了口气道:``也好,先开花后结果,总能生出皇子的。''

其实有孕五月时,皇帝每每看着如懿渐渐隆起的肚子,便慨叹:``若是位嫡子\ldots\ldots{}''他见如懿笑容淡淡的,便笑着道:``当然,公主也是好的。''

如懿便笑吟吟地缝着一件水蓝色的婴儿衣衫:``也是,皇帝膝下只有两位公主,和敬公主又嫁去了蒙古,臣妾也想添一个公主呢。女儿多贴心啊。''

背转身无人之际,如懿便盯着江与彬道:``胎象如何?''

江与彬含笑躬身:``一切安稳。''

如懿掂量着问:``男胎女胎?''

江与彬拱手贺道脉象强劲有力,皇上会心想事成,有一位嫡子。''

如懿松一口气:``本宫相信你说的事实话。齐鲁老成谨慎,他不敢对本宫论男女,也不敢对皇上说。''

江与彬笑言:``自然不敢。说了之后,万一不对,可是死罪。''

如懿笑着瞟他一眼:``你却敢说?''

``那是因为皇后娘娘不会杀了微臣。''

如懿扑哧一笑,继而正色,捻了一片酸梅糕吃了:``男胎也好。可本宫不想让皇上高兴得太早,也不想让旁人不高兴得太早。''

江与彬懂得:``胎象的事,除了请脉的人,旁人都不知道。他们若要揣测娘娘腹中孩子是男是女,只能看娘娘的饮食。''

如懿举着酸梅糕笑:``酸儿辣女?''

``民间传闻,有一定的道理。''

如懿微微一笑:``本宫嗜酸,如今可要多多吃辣了。''

于是小厨房流水价端上的彩色,色色以辣为主,辛辣的气味便在翊坤宫中弥漫开来,让所有进进出出的鼻子都闻见了。''

便有好事之人开始揣测:``皇后娘娘那么爱吃辣,别是位公主吧?''

有人便附和:``可不是?酸儿辣女。嘉贵妃怀德每一胎,都是爱吃酸的。今儿午膳还吃了一大盘她家乡的渍酸菜和一碗酸汤鱼了。''

``还是嘉贵妃好福气,胎胎都是皇子。皇后娘娘年纪大了,好容易怀上一胎,却是个公主,白费力气了。''

``皇上做梦都盼着是位嫡子,要是公主,可不知要多失望呢。''

``啧啧,那嘉贵妃不是更得宠了?''

这样的传言,在乾隆十七年二月初七,玉妍剩下十一阿哥永瑆之后更是甚嚣尘上。连宫人们望向如懿的眼神也不觉多了一丝怜悯,似乎在慨叹这位大龄初孕的皇后生不出皇子的悲剧命运。

且不说嬿婉和玉妍,连皇帝新宠的颖嫔亦在背后笑:``好容易怀上了孩子,不过是个公主,有什么趣儿。听说内务府又送了几匹粉紫嫣红的料子去给皇后腹中的孩子做衣裳呢。''

如懿闻得流言纷纷,也不过一笑。临近生产,容佩领着合宫宫人愈加警觉,只是那警觉不是明面上的劳师动众,而是暗地里事无巨细的查看。如懿入口的一饮一食均是用银针仔细检查过,再叫江与彬细看了才能入口。连生产时的银剪子,白软布,乃至一应器皿及衣衫被褥,都反复严查,生怕有一丝错漏,直熬的容佩两眼发绿,看谁都是森森的。

而如懿,便好整以暇的看着钦天监博士张镇息在翊坤宫后殿东门边选了``刨喜坑''的``吉位'',来作为掩埋来日生产后孩子胎盘和脐带的吉地。三名太监刨好``喜坑'',两名嬷嬷在喜坑前念喜歌,撒放一些筷子、红绸子和金银八宝,取意``快生吉祥''

如懿陪着母亲和太后笑吟吟看着,满心期待与喜悦,享受着初为人母的骄傲与忐忑。

次日,内务府送来精奇嬷嬷、灯火嬷嬷、水上嬷嬷各十名,如懿亲自挑选了两名身份最高,儿女双全的嬷嬷备用1。另有四名经验丰富的接生嬷嬷,从三月初一起,在翊神宫``上夜守喜'',太医院也有六名御医轮流值班,以备不时之需。

如懿只敢把酸杏子藏在锦被底下,偷偷吃一个,吃一个,酸的直冒眼泪。

容佩笑吟吟道:``这是昌平进贡的酸杏,奴婢偷偷拿了的,好吃么?''

如懿笑道:``晚膳吃了那么多辣,辣的胃里直冒火儿,现下吃了杏子才舒服些.''

容佩悄悄道``奴婢藏了好些呢。娘娘要吃就告诉奴婢,晚上是奴婢守夜,尽着娘娘吃,没人知道。``说罢又慨叹,``您是皇后娘娘,怀了皇子也不敢随便叫人知道,奴婢看着真是辛苦''。

``树大招风,当年孝庄皇后怀着皇子的时候,多少眼睛盯着呢。本宫比不得孝庄皇后有家室,凡是只能自己小心。''如懿扶着隆起的肚子道,``如今在肚子里还算是安稳的,若生下来,还不知得如何小心呢。''

容佩一脸郑重:``娘娘放心,奴婢拼死也会护着娘娘和皇子的。''

在众人或嗤笑或疑惑的目光中,乾隆十七年四月二十五日寅时,如懿在阵痛了一天一夜之后,终于诞下了一位皇子。

寝殿内放着光可鉴人的小巧樱桃木摇篮,明黄色的上等云缎精心包裹着孩子娇嫩柔软的身体,孩子乌黑的胎发间凑出两个圆圆的旋涡,粉白一团的小脸泛着可人的娇红,十分糯软可爱。

彼时皇帝正守在奉先殿内,闻知消息后欣喜若狂,向列祖列宗敬香后,即刻感到翊坤宫。

海蓝早已陪侯在如懿身侧,皇帝看过了新生的皇子,见了如懿便亲手替她擦拭汗水,喂了宁神汤药,笑道:``此子是朕膝下唯一嫡子,可续基页,便叫永基可好?''

如懿吃力的点点头,看着乳母抱了孩子在身侧,含笑欣慰不已。

海蓝笑道:``臣妾生下永琪的时候,皇上便说,琪基,玉属也。永琪与永基,果然是对好兄弟呢。''

永基的出生倒是很好的缓和了帝后之间自永寿宫风波后的若即若离。如懿有时会想,难怪男人和女人之间一定要有孩子,孩子就是相连相通的骨血。原本只是肌肤相亲的两个人,再黏腻再换号,也不过是皮相的紧贴,肉体的依附。可有了孩子,彼此的血液就有了一个共通的凝处,打不开分不散的。

而皇帝亦对永基十分爱护,特许如懿养在了自己宫里,并不曾送到阿哥所去。因为有乳母照护,又有母亲在身边悉心照拂,如懿很快便恢复了过来。

\hypertarget{ux7b2cux4e8cux5341ux4e09ux7ae0-ux5f97ux610f}{%
\chapter{第二十三章
得意}\label{ux7b2cux4e8cux5341ux4e09ux7ae0-ux5f97ux610f}}

待到八月时,如懿已能陪着皇帝木兰秋狩,策马扬鞭了。她便在那一年,以自己春风得意的眼,再度撞上了凌云彻落魄的面容。

彼时凌云彻已到木兰围场待了很长的一段时日,木兰围场是一处水草丰美,禽兽繁衍的草原,虽然皇帝每年都要率王公大臣、八旗精兵到这里举行秋狩,但过了这一阵热闹,这里除了浩瀚林海、广袤草原,平日里便极少有人来往,只得与落叶山风、禽兽野兽为伴了。

这于凌云彻无疑是一重极大的痛苦,而更让他难以忍受的,是背着这样香艳而猥琐,屈辱的罪名离开了宫廷,所以当如懿在围场随扈的苦役之中看见凌云彻消瘦而胡子拉碴的面庞时,亦不觉惊了目,惊了心。

彼时人多,皇帝携了和亲王弘昼,十九岁的三阿哥永璋,十四岁的四阿哥永珹,十二岁的五阿哥永琪,还有一众亲贵大臣,正准备逐鹿围场,行一场尽兴的秋狩,如懿便和几位阿哥的生母跟随在后,望着众人策马而去的方向,露出期待的笑容。

绿筠笑色满目,道:``没想到五阿哥年纪最小,跑起马来一点儿都不输给两个哥哥呢。''

海兰腼腆道:``小孩子家的,哥哥们让着他罢了。''

玉妍亦不肯示弱:``是么?怎么我瞧着四阿哥跑得最快呀!''

绿筠素知玉妍心性,便也只是一笑置之:``四阿哥跟着嘉贵妃吃了那么多李朝的山参进补,体格能不好么?等下怕是老虎也打得死了。要好好儿在皇上面前显露一手呢。''

玉妍扬一扬手中春蝶般招展的娟子,掩口笑道:``能显什么身手呢?大阿哥和二阿哥不在了,三阿哥这位长子这么显眼,哪里轮得到咱们的四阿哥呢?''

绿筠闻言便有些不悦,自从孝贤皇后丧礼时三阿哥被申饬,一直是绿筠的一块心并不是。且皇帝渐有年事,对立太子一说抑或是立长一说十分忌讳,大阿哥永璜便是死在这个忌讳上,谁又敢再提呢。

绿筠的脸色冷了又冷,即刻向着如懿,一脸恭顺道:``嘉贵妃是越发爱说笑了,都是皇上绷着她。咱们的孩子再好,也渤是臣下的料子,哪里比得上皇后娘娘的十二阿哥呢。且不说十二阿哥在襁褓之中,便是五阿哥也是极好的呢。''

如懿与海兰对视一眼,亦不作声。这些年如何用心教导永琪,如何悉心培育,且在人前韬光养晦,积蓄十数年的功夫,岂可一朝轻露?便也是含笑道:``这个时候不看狩猎,说这些没影子的话做什么呢?''

皇帝猎兴最盛,跟随的侍卫和亲贵们心下明白,便故意越跑越慢,扯开了一段距离,前头尽数是围场上放养的各色禽兽,以鹿、麋、羊、兔、獐为多,更有几头蓄养的半大豹子混杂其中,以助兴致。

那些温驯的牲畜如何能入皇帝的眼,唯有那金色的奔窜的半大豹子,才让皇帝热血沸腾,他正策马疾追,横刺里一匹不知马的马匹疾奔而过,鬃发油亮,身形高大,马色如霜纨一般,直如一道雪白闪电横刺而过。相形之下,连御马也被比得温驯而矮小。

皇帝眸中大亮,兴奋道:``哪儿来的野马?真乃千里驹!''他手中马鞭一扬,重重道:``此马良骏,看朕怎么收服它!''

皇帝素来爱马,又深感御马温驯不够雄峻,眼见此良驹,怎不心花怒放,众人深知皇帝脾气,亦不敢再追!

策马奔过红松洼,丘陵连绵起伏,皇帝原本有尽让侍众们跟着一段距离,奈何那野马性烈,奔跑飞快,皇帝一时急起来,也顾不得后头,加紧扬鞭而去。

很快奔至一茂密林中,落叶厚积,道路逐渐狭小,跑得再快的马也不知不觉放慢了脚步,缓步悠悠。北方高大的树木林叶厚密,蔽住了大部分阳光,只偶有几点斑驳的亮点洒落,像金色的铜钱,晃悠悠亮得灼目。四周逐渐安静,身后的马蹄声,旌旗招展声,呼呼的风声都远离了许多,唯有渐渐阴郁潮湿的空气与干燥的夏末的风混合,夹杂着藤萝灌木积久腐败的气息,不时刺激着鼻端。

四下渺然,一时难觅野马踪影。皇帝有些悻悻,正欲转身,只见左前方灌木丛中有一皮色雪白的小东西在隐隐窜动,皇帝一眼瞥见是只野兔,却不愿轻易放过,立刻搭箭而上。然而,在他的箭啸声未曾响起之时,另一声更低沉的箭羽刺破空气的声响死死钻入了他的耳际。

皇帝一惊之下本能地矮下身子,紧紧伏在马背上,一支绿幽幽的暗箭恰好掠过皇帝的金翎头盔。``咔''的一声轻脆的响,似乎是什么东西断了。

是有人在施放冷箭!

皇帝尚未回过神来,另一声箭响再度响起。皇帝正要策马往前,只见前头灌木丛中仰起一张野马的脸。那是一张受到惊吓后激起突变的脸,它面孔扭曲,前蹄高高扬起,朝着正前方的皇帝当胸踢来。皇帝有一瞬间的犹疑,若是向前,难免受到惊马的伤害,便是拔箭射杀也来不及;而后头逼来的利箭,已经让他无从躲避,更不得退后。

只那么一瞬,皇帝便觉得一股劲风袭来,有人将自己从马上扑了下来,在地上滚了两下,避过了那随后追来的一支冷箭。皇帝在惊魂未定中看清了救自己的那张脸,熟悉,却一时想不到名字,只得脱口而出道:``是你!''

凌云彻护住皇帝,道:``微臣凌云彻护驾来迟,还请皇上恕罪。''

这一巨大的响动,显然是刺激到了前方灌木丛中的那匹发性的野马,未经驯化的马匹身上腥臭的风渐渐逼近。

若是寻常,那是不必怕的。比之凌云彻的赤手空拳,皇帝有弓箭在手。然而,在转身的瞬间,皇帝才发现落马之时背囊散开,弓虽在手,但箭却四散落了一地,连最近的一支也离了两三尺远。而那高高踢起的铁蹄,几乎已要落在自己三步之前!

凌云彻有一瞬的绝望,难道一番苦心,真要葬送在野马蹄下?他的意志只软弱了片刻,念及再凶猛也不过是匹野马而已,立刻冷静而坚决道:``微臣会护着皇上!''

他的话音未落,只见斜刺里一个人影贴着草皮滚过,大喊了一声:``皇阿玛'',便挡在了身前。同时,一支长箭在身后放出,正中前方野马的额头中心,直贯入脑,只听一声狂啸,那野马剧痛之下惊跳数步,终于随着额头一缕浓血的流出,倒地而亡。

皇帝长长地松了一口气,只觉得冷汗淋漓,湿透了衣裳。片刻,他终于回过神来,才发现五子永琪张开双臂,死死挡在那野马奔袭过来的方向,而四子永珹背着箭囊赶了过来,伏地道:``儿臣救驾来迟,皇阿玛没事吧?''

皇帝从箭翎的颜色上分辨出那是永珹的箭,不觉惊喜交加,紧紧揽住永珹肩头道:``好儿子!是朕的好儿子!''

永珹激动得满面通红,连连谢过皇帝的夸赞。而永琪只是若无其事地站起来,松了松手脚,默默地站在兄长之后。

还是凌云彻先问:``五阿哥没有受伤吧?''

永琪摇了摇头:``皇阿玛没事就好。''

皇帝笑了笑,显然那笑不如对着永珹般亲热而赞许,只是随口问:``方才你先过来抢到朕身前,怎么不先射野马,反而只促手待着?''

永琪淡然自若道:``儿臣方才的距离,拔箭已经来不及了。而且,儿臣听师傅说过,猛兽伤人,往往得一而止。儿臣护在皇阿玛身前,那野马伤了儿臣,便不会再伤害皇阿玛了。''

年方十二的孩子,这番话说来十分诚恳。皇帝不觉动容,抚摸他的额头:``你是个有孝心的孩子!''

皇帝余悸未消地摘下自己的金翎头盔,发现那金色的尾翎已经被箭矢射断。他示意永珹小心捡起那两支冷箭,仔细看过,冷下脸凝道:``有没有毒?''

永珹仔细查验了道:``无毒。''

皇帝的目光在冰寒如铁中夹杂了一丝不易发觉的恐惧与阴鸷:``谁在施放冷箭?谁想害朕?''

永琪低眉顺目,沉声道:``想害皇阿玛的人,最终都不会得逞的。''

皇帝朝四面的山坡树林眺望着,沉默良久道:``忠于朕的人都来救朕了!害朕的人,此时一定躲得最远!''他沉下声,以委以重任的口吻吩咐永珹:``永珹,带人搜遍围场,朕就要看谁有这样的胆子,竟敢谋害天子!''

十四岁少年的脸上闪过一丝兴奋的红晕,大声道:``是!''

而永琪,只是依偎在父亲身边,扶住了他的手,紧紧护卫他左右。

皇帝走了几步,回过头看凌云彻:``朕记得你本来在朕身边当差的,为什么走的?''

凌云彻有些羞赧,低头道:``微臣被冤偷了嘉贵妃的肚兜,因此被遣来围场做苦役。''

皇帝点点头:``朕从前不信你被冤,现下信了。因为觊觎朕的女人的人,是不会拼死来救朕的。跟朕回去吧,在围场吹风是浪费了你!''

林间的风夹杂着八月初北地的秋意,带给皮肤低凉的温度,却没有心底衍生的滚热更畅快。凌云彻将一缕狂喜死死压了下去,恭声道:``微臣谨遵皇上旨意。''

木兰围场的猎猎风声无法告知暗害者的身份,亦彻底败坏了皇帝狩猎的兴致。唯一可知的,不过是那野马奔驰至林间,是有母马发情时的体液蹭于草木之上,才引得野马发狂而至。而那冷箭,却是早有弓箭安放在隐蔽的林梢,以银丝牵动,一触即发。林场官员连连告饶,实在不知是有人安放弓箭本欲射马才阴差阳错危及帝君,还是真有人悉心安排这一场阴谋。但有人擅闯皇家猎场布置这一切,却是毋庸置疑,皇帝又惊又怒,派了傅恒细细追查,然而,仓促之下,这一场风波终究以冷箭施放者的无迹可寻而告终。

自此皇帝心性更伤,偶有几次惊梦,总道梦见当日冷箭呼啸而过的情景,却不知暗害者谁,唯有利刃在背之感,如懿只得紧紧抱住了皇帝的肩,以此安慰这一场莫名惊险后的震怒与不安。

待消息传到宫中,饶是太后久经风波,亦惊得失了颜色,扶着福珈的手臂久久无言。

福珈温声道:``太后安心,奴婢细细查问过,皇上一切安好,太后可以放心。奴婢也着人传话过去,以表太后对皇上关爱之意,只是这件事\ldots\ldots 太后是否要彻查。''

太后思忖片刻,断然道:``不可!这件事皇帝自己会查,且风口浪尖上,人人都怕惹事,警惕最高,也难查出原委。如今风声鹤唳,皇帝最是疑心的时候,哀家若贸然过问,反倒惹皇帝不快。''

福珈心疼,亦有些怨:``太后也是关心皇上,倒怕着皇上多心似的。反而疏远了。''

太后抚着手中一把青金石嵌珊瑚如意,那触手的微凉总是让人在安逸中生出一缕警醒。恰如这皇家的母慈子孝,都是明面上的繁华煊赫,底下却是那不能轻触的冷硬隔膜。须臾,她郁郁叹道:``毕竟不是亲生,总有嫌隙,皇帝自小是个有主意的人,年长后更恨掣肘。哀家凡事能婉劝绝不硬迫。且你看他如今遴选妃嫔是何等谨慎,便知咱们的前事皇帝是有所知觉了,哀家只求女儿安稳,余者就当自己是个只懂享受的老婆子吧。''

自木兰围场回宫,风波余影渐淡去,却生出一种煊煊的热闹,除了凌云彻成为御前二等侍卫,深得皇帝信任之外,利益最多的便是玉妍的四阿哥永珹。首先是皇帝对玉妍的频频临幸,继而是对永珹学业和骑射的格外关照,每三日必要过问。这一年皇帝的万寿节,李朝使者来贺,皇帝便命永珹应待。而永珹亦十分争气,颇得使者赞许。而最令后宫与朝野震动的是,在重阳之后,皇帝便封了永珹为贝勒。

这不啻是巨石入水,引得众人侧目。因为已经成年娶亲的三阿哥永璋尚未封爵,反而是这位尚未成年的四弟拔了头筹。而对五阿哥永琪,皇帝虽然倍加怜爱,诸多赏赐,但却无对待永珹这般器重,所以永琪也不免黯然失色了。

凌云彻回言之后,比之从前更加谨言慎行,更因少了世家子弟的纨绔习气,皇帝十分倚重。

这一日皇帝正因木兰秋狩之事欲责罚围场诸人,正巧三阿哥永璋前来请安,听见皇帝龙颜震怒,欲牵连众多,便劝了一句道:``儿臣以为此次秋狩之事查不出元凶,也是因为围场服役之人过多,一时难以彻查。皇阿玛若都责罚了,谁还能继续为皇阿玛查人呢?''

这话本也在情理之中,然而,皇帝经此一事,疑心更胜从前,当下拍案怒道:``你是朕诸子中最长,本应是你救驾才对!一来围场之事有疏漏,你这个长子有托管不力之嫌;二来救驾来迟则属不孝不忠,能力庸常,不及两个弟弟;三来事后粗漏,不能为君父分忧,反而为一已美名,轻饶轻恕,不以君父安危为念!朕要你这样的儿子,又有何用?''

皇帝这般雷霆震怒,将永璋骂得汗湿重衣,满头冷汗,只得诺诺告退。

皇帝随后便问随侍在旁的凌云彻道:``你瞧瞧永璋这般请求轻恕木兰围场之人,那日冷箭之事会否与他有关?''

凌云彻恭谨道:``三阿哥是皇上的亲子。''

皇帝摇头,呼吸粗重:``天家父子,不比寻常人家。可为父子,可为君臣,亦可为伊雠!圣祖康熙爷晚年九子夺嫡之事,朕想来就惊心不已。''

凌云彻道:``皇上年富力强,没有谁敢,也没有能力敢谋害皇上!''

皇帝听得此言,稍稍宽慰:``那木兰围场诸人,你觉得当不当罚?''

凌云彻恭顺着垂着眼眸,感受着孔雀花翎在脑后那种轻飘又沉着的质感,想起在木兰围场那些望着月忍着屈辱受人白眼的日子,道:``有错当罚,有功当赏。皇上赏罚分明,胸中自有定夺,微臣又怎敢妄言。''

皇帝笑着画下朱批,赞许道:``甚好。''

这句话不知是皇帝赞许自己的举措还是夸奖凌云彻的慎言。凌云彻正暗自揣摩,皇帝忽而笑道:``你已年过三十,尚未成家,也不像个样子。''他随手一指,唤过御前一个青衣小宫女道:``茂倩,你也二十五了,快要出宫,朕就将你赐给凌侍卫为妻,如何?''

那宫女一怔,旋即跪下,眉开眼笑道:``奴婢谢过皇上。''

凌云彻愣在当地,脑中一片空白,全不知该如何反应,直到李玉在旁推他的手臂,笑眯眯道:``瞧凌大人,这是欢喜傻了吧?快谢恩哪!''

他这才回过神来,看见皇帝已经有些不耐烦的笑意,茫然跪下身行礼,来接受这突如其来的恩典。

至此,永璋的失宠便已成定局,而永琪得了如懿与海兰的嘱咐,只潜心学业,若非皇帝召唤,亦不多往皇帝跟前去。

这一日,凌云彻自养心殿送永琪回翊坤宫,便顺道来向如懿请安。如懿正在廊下看着侍女调弄桂花蜜。她静静立于飞檐之下,裙裾拂过地,淡淡紫色如木兰花开,夕阳流丽蕴彩的光就在她身后,铺陈开一天一地的华丽,更映得她风华如雪,澹澹而开。

如懿见了他便含笑:``士别三日,当刮目相待。''

凌云彻屈膝拱手,正色道:``皇后娘娘曾要微臣堂堂正正地走回来,微臣不敢辜负皇后娘娘的期望。''

如懿端详他片刻:``被北边的风吹得脸更黑了。但,能这样风光地回来就好。本宫更得多谢你,救了皇上。''

云彻见她欢悦之色,不觉低下头道:``这是微臣的本分。''

``有功也不忘本,才能在皇上跟前处得长远,你很好。''她笑道,``你在皇上跟前如此得脸,也是该娶亲成家了,皇上亲自赐婚,这是无上的荣耀,旁人求也求不来呢。''

凌云彻心头一抖,忽然一颗心便飘到了木兰围场的那些日子,孤清的寒夜里,常常想起的,居然是如懿含笑的清婉脸庞。

那是唯一的念想,连着她的嘱咐,一路引着他不惜一切也要走回紫禁城,堂堂正正地走回来。

这样的念头不过在脑中转了一瞬,他便按捺了下去,淡淡道:``微臣知道自己要什么,不是女人。''

如懿的眸光幽然垂落,略带惋惜地看着他:``还是因为她伤害过你的缘故么?''

云彻别过脸,抿紧了薄薄的唇:``微臣不想再记得。''

如懿的笑意愈加清婉,仿佛天边明丽的霞光映照:``不想记得也好。皇上御前的宫女出身尊贵,都是满军旗的女儿,你有这样的妻子,对你的出身和门楣也有益。对了,你家里有谁帮你操办喜事么?''

云彻有些失神,道:``父母已在几年前亡故,无人安排。''他微微苦笑,``微臣终于能回到紫禁城中,不负娘娘所望,但皇上赐婚这样的意外之喜,也实在是太意外了。''

如懿意味深长地目视于他:``无论是否意外,皇上的恩赐是不容许你有一丝不悦和推脱的,茂倩是御前的人,你须得好好儿待她。''她温然含笑,``至于你家中无人,江与彬与惢心就在京中,本宫让他们为你打点,助你一臂之力。''

云彻勉力微笑,振作精神答应:``多谢皇后娘娘美意。''他看着如懿身边的乳母怀中抱着的婴儿,心中有了一丝伤感的欣喜,``虽然微臣身在围场,但也听说娘娘喜获麟儿,微臣在此贺过。''

如懿颔首道:``有心了。''

云彻懂得地道:``彼此过得好才是最有心。''他还想再说什么,皇帝身边的李玉已经来传旨,皇帝会来陪着如懿用晚膳。他即刻意识到自己的存在不合时宜,就好像翊坤宫所有描画的鸳鸯龙凤都是成双成对,比翼交颈,花纹都以莲花与合欢为主。

合昏尚知时,鸳鸯不独宿。他如何不明白这个道理,连自己,很快不也要如此么?他只得躬身,恭恭敬敬告退离去。

\hypertarget{ux7b2cux4e8cux5341ux56dbux7ae0-ux7aefux6dd1}{%
\chapter{第二十四章
端淑}\label{ux7b2cux4e8cux5341ux56dbux7ae0-ux7aefux6dd1}}

从翊坤宫出来之后,凌云彻便见到了嬿婉,嬿婉茕茕走在暮色四合的长街上,夹道高耸的红墙被夕阳染上一种垂死之人面孔上才有的红晕,黯淡而无一丝生气。而一身华服的嬿婉,似乎也失却了他离开那时的因为恩宠而带来的光艳,像一个华丽的布偶,没有生气。

在与他目光相触之后,嬿婉眸中有明显的惊异和畏惧:``你回来了?''

云彻有礼地躬身:``有负小主的期望,微臣还是回来了。''

嬿婉很快掩饰了自己不应有的情绪:``那就好。听说你高升了,也由皇上赐婚,即将娶亲,恭喜。''

云彻直截了当道:``小主还是那么喜欢说违心的话,做违心的事。''

嬿婉不悦地皱眉:``即便你得皇上宠幸,就可以这样和本宫说话么?害你的人是嘉贵妃,有什么话冲着她说去,别来赖本宫。''

云彻澹然一笑,了然道:``嘉贵妃凭什么要害微臣?宫中谁容不下微臣,微臣明白。''

他走近一步,嬿婉显然对他这样的举动很是不安,诧异地退了一步,道:``你要做什么?你\ldots\ldots{}''她眼中有深深的戒备,``若有证据,你大可去告诉皇上!''

``所谓证据,有时只在一个眼神,一种了解。''凌云彻哑声道:``你不必害怕,我与我都已非从前的自己,只要相安无事就能各保平安。但,你也别想再害我。''他深深地看了嬿婉一眼,如同最彻底的告别,``这些话,便是从前所有的情分所在了。你再敢害我,我也有的是把柄。''

嬿婉靠在墙上,怔怔地看他离开,似乎在思索着他语中的深意。良久,终于自嘲地笑笑:``可不是?一个不得宠的女人,帮得了谁,双害得了谁?''她含了一缕怨恨之意,望着斜阳渐渐坠入西山,浓墨般的天色随即吞噬了她孤清的身影与面容。

从木兰围场回来后数月,如懿很快发觉自己又有了身孕。也许是生子之后皇帝的眷顾有加。也许是江与彬调息多年后身体的复苏。乾隆十七年秋天的时候,如懿再度怀上了身孕。而云彻,也在这个秋天迎娶了茂倩过门。娶亲后的他似乎愈加忙碌,除了该当值的日子,也总是替别的侍卫轮守,一心一意侍奉在皇帝身边,也更得皇帝倚重。

中宫接连有喜是合宫欢悦之事。有了永璂的出生,这一胎是男是女似乎都无关紧要了。如懿而言,再添一个皇子固然是锦上添花;但若有个女儿,才真真是儿女双全的贴心温暖。

而彼时,意欢的爱子十阿哥却渐渐不大好了。

也许是从娘胎里带出来的肾气虚弱的病症,随着十阿哥的日渐长大,并未有所好转,反而渐渐成了扼住他生命的一道绳索,并且越勒越紧,仿佛再一抽紧,便能要了他的性命去。

那段时间的储秀宫总是隐隐透着一股阴云笼罩的气息,哪怕太后和如懿已经遣了太医院最好的太医守在储秀宫延医问药,但意欢隐隐约约的哭声,似乎暗示着阴霾不会散去。

入春之后,为了让十阿哥养息得更好,也为了如懿能好好儿养胎,皇帝便携带太后与嫔妃们去了圆明园暂住怡情。

圆明园从圣祖康熙手中便有所兴建,到了先帝雍正时着手大力修建,依山傍水,景致极佳。到了皇帝手中,因着皇帝素性雅好园林景致,又依仗着天下太平,国富力强,便精心修建。园中亭台楼阁,山石树木;将江南秀丽景致与北地燕歌气息融于一园。

春风开紫殿,天乐下朱楼。莺歌闻太液,风凤吹绕瀛洲。迟日明歌席,新花艳舞衣。烟花宜落日,丝管醉春风。比之宫内的拘束,在圆明园,便是这样随心如流水的日子。

皇帝喜欢湖上清风拂绕的惬意,照例是住在了九州清晏,如懿便住在东边离皇帝最邻近的天地一家春,紧依着王陵春色。颖嫔恩宠深厚,皇帝喜欢她在身边,便将西边的露香斋给了她住。绿筠上了年纪,海兰恩宠淡薄,便择了最古朴有村野之趣的杏花春馆,带着儿女为乐。玉妍住了天然图画的五福堂,庭前修篁万竿,与双桐相映,风枝露俏,绿满襟袖,倒也清静。尤其四阿哥永珹甚得皇帝钟爱,对他读书之事颇为上心,便亲自指了这样清雅宜人的地方给他读书,亦方便日常相见。

庆嫔和几位新入宫的常在分住在茹古涵今的茂育斋和竹香斋,茹古涵今四周嘉树丛卉,生香蓊葧,缭以曲垣,邃馆明窗,亦别有一番情致。意欢为求十阿哥安静养病,便住了稍远的春雨舒和。如懿因忌讳着嬿婉,便让她住着最远的武陵春色的绾春轩,与同样失宠的晋嫔的翠扶楼相近,太后喜好清静,长春仙馆屋宇深邃,重檐羊槛,逶迤相接,庭径有梧有石,最合她心意,其余嫔妃,便闲散在于其间,彼此倒也惬意。

如懿的产期是在七月初,她除了素日去看望意欢和十阿哥,时时加以安慰,便也只安心养胎而已,后宫里的日子不过如此,有再大的波澜,亦不过激荡在死水里的。不过一时便安静了。而真正的不安,是在前朝。

因着如懿生下了嫡子永璂,皇帝圣心大悦,五月之时,再度大赦天下,减秋审、朝审缓决三次以上罪。这本是天下太平的好事,然而,国中这般安宁,准噶尔却又渐渐不安静起来了。

昔年准噶尔首领噶尔丹策零死后,留有三子。长子多尔札,困是庶出不得立位:次子纳木札因母贵而嗣汗位;幼子策妄达什,为大策零敦多布拥护,纳木札尔的姐夫萨奇伯勒克相助多尔札灭了纳木札尔,遂使多尔札取得汗位,但他的登位遭到准噶尔贵族反对,朝廷为平息准噶尔的乱象,便于当年下安胎太后亲女端淑长公主为多尔札之妻,以示朝廷的安稳之意,多年来,多尔札一直狂妄自傲,耽于酒色,又为防兵变再现,杀了幼弟策妄达什,十分不得人心,准噶尔贵族们忍耐不得,只好转而拥立准噶尔另一亲贵达瓦亲。达瓦亲是巴图尔珲台吉之后,大策零敦多布之孙,趁着准噶尔部人心浮动,趁机率兵绕道入伊犁,趁多尔札不备,将其趋而斩之,抚定部落,自此,达瓦齐自立。

这一来。朝野惊动,连太后亦不得不过问了。

只因准噶尔台吉多尔札乃太后长女端淑固伦长公主的夫君,虽然这些年多尔札多有内宠,性格又极为强悍骄傲,夫妻感情淡淡的,并不算十分融洽,甚至公主下嫁多年,连一儿半女也未有出。但毕竟夫妻一场,维系着朝廷与准噶尔的安稳。达瓦齐这一拥兵自立,准噶尔部大乱,端淑长公主也不得不亲笔家书传入宫宫,请求皇帝干预,为夫君平反报仇,平定准噶尔内乱。

然而,端淑长公主的家书才到宫中,准噶尔便传来消息,达瓦齐要求迎娶端淑长公主为下威,这一言不啻一石激起千层浪,爱新觉罗氏虽然是由关外兴起,兄娶弟媳,子承父妾之事数不胜数。哪怕是刚刚入关初定中原之时,这样的事也屡有发生,当年便有孝庄皇太后下嫁摄政王多尔衮的流言,便是顺治帝亦娶了弟弟博果尔的遗孀董鄂氏为皇贵妃。

但大清入主中原百年,渐渐为孔孟之道所洗礼,亦要顺应民心,尊崇礼仪。所以顺治之后,再无此等乱伦娶亲之事,连亲贵之中丧夫再嫁之事亦少。而准噶尔为蒙古部落,一向将这些事看得习以为常,所以提出娶再嫁之女也是寻常。

这般棘手的事,皇帝自然每日都在勤政殿与大臣们议政,更抽不得身往后宫半步。

这一日午后,如懿正在西窗下酣眠,窗外枝头的夏蝉咝咝吟唱,催得人睡意更沉。九扇风轮辘辘转动,将殿中供着的雪白素馨花吹得满室芬芳。容珮进来在耳边低声道:``皇后娘娘,太后娘娘急着要见您呢。''

这一语,便足以惊醒了如懿,她立刻起身传轿,换了一身家常中略带郑重的碧色缎织暗花竹叶氅衣,只用几颗珍珠纽子点缀,下身穿一条曳地的荷叶色绛碧绫长裙,莲步轻移,亦不过是素色姗姗。她佩戴金累点翠嵌翡翠花簪钿子,在时近六月的闷热天气里,多了一抹清淡爽宜,一副乖巧勤谨的家媳模样。她想了想,还是道:``给皇上炖的湘莲燕窝雪梨爽好了么?''

容珮道:``已经炖好凉下了,等下便可以给皇上送去。这些日子里皇上心火旺,勤政殿寻边回话说,皇上喝着这个正好呢。''

如懿正了正衣襟上和田白玉竹节领扣,点头道:``备下一份,本宫送去长春仙馆。''

长春仙馆空旷深邃,有重重翠色梧桐掩映,浓荫匝地,十分清凉。庭前廊下又放置数百盆茉莉、素馨、剑兰、朱槿、红蕉,红红翠翠,十分宜人。偶尔有凉风过,便是满殿清芬。如懿入殿时,太后穿了一身黑地折枝花卉绣耀眼松鹤春茂纹大襟纱氅衣,想是无心梳妆,头发松松地挽起,佩着点翠嵌福寿绵长钿子,菘蓝宝绿的点翠原本极为明艳,此时映着太后忧心忡忡的面庞,亦压得那明蓝隐隐仿佛成了灰沉沉的烧墨。

太后的幼女淑长公主便陪坐在太后膝下垂泪,一身宝石青织银丝牡丹团花长衣,棠色长裙婉顺曳下,宛如流云。柔淑戴着乳白色玉珰耳坠,一枚玉簪从轻轻的如雾云髻中轻轻斜出,金凤钗衔了一串长长的珠珞,更添了她几分婉约动人。而此时,她的温婉笑靥亦似被梅雨时节的雨水泡足了,唯有泪水潸潸滑落,将那宝石青的衣衫沾染成了雨后淋漓的暗青。

如懿见此情景,便晓得不好。彼时她已有了八个月的身孕,行动起坐十分不便,太后早免了她见面的礼数。然而,眼下这个样子,如懿只得规规矩矩屈膝道:``皇额娘万安,长公主万安。''

柔淑虽然伤心,忙也起身回礼:``皇嫂万安。''

太后摇着手中的金华紫纶罗团扇,那是一柄羊脂白玉制成的团扇,上覆金华紫纶罗为面,暗金配着亮紫,格外夺目华贵。而彼时太后穿着黑色地纱氅衣,那上面的缠枝花卉是暗绿、宝蓝、金棕、米灰的颜色,配着灼热耀目的金松鹤纹和手中的团扇,却撞得那华丽夺目的团扇颜色亦被压了下去,带着一种欲腾未腾的压抑,屏着一股闷气似的。

太后瞥如懿一眼,扑了扑团扇道:``皇帝忙于朝政,三五日不进长春仙馆了。国事为重,哀家这个老婆子自然说不得什么。但是皇后,''她指了指向边的柔淑道,``柔淑是嫁出去的女儿泼出去的水了,哀家见不得儿子,只能和女儿说说话排解心意。但是儿媳,哀家总还是有的吧?''

如懿闻言,立刻郑重跪下,诚惶诚恐道:``皇额娘言重了,儿臣在宫中,无一日不敢不侍奉在皇额娘身边。若有不周之处,还请皇额娘恕罪。''

太后凝视她片刻,叹口气道:``容珮,看你主子可怜见儿的,月份这么大了还动不动就跪,不知道的还当哀家这个婆母怎么苛待她了呢,快扶起来吧。''

如懿地着腰身,起身便有些艰难,忙赔笑道:``儿臣年轻不懂事,一切还得皇额娘调教,但儿臣敬爱皇额娘之心半点不敢有失,儿臣知道这几日天热烦躁,特意给皇额娘炖了湘莲燕窝雪梨爽,已经配着冰块凉好了,请皇额娘宽宽心,略尝一尝吧。''

如懿说罢,容珮便从雕花提梁食盒昌取出了一盅汤羹,外砂全用冰块瓮着。容珮打开来,但见汤色雪白透明,雪梨炖得极酥软,配着大颗湘莲并丝丝缕缕的燕窝,让人顿生清凉之意。

柔淑长公主勉强笑道:``这汤羹很清爽,儿臣看着也有胃口。皇额娘便尝一尝吧。好歹是皇嫂的一份心意。''

太后扫了一眼,颔首道:``难为皇后的一片心了。哀家没有儿子在跟前,也只得你们两个还略有孝心。只是哀家即便没有胃口,也没心思。这些日子心里火烧炎燎的。没个安静的时候,只怕再好的东西也喝不下了。''

如懿明白太后话中所指,只得赔笑道:``皇额娘担心端淑长公主,儿臣和皇上心里也是一样的。这日子皇上在勤政殿里与大臣们议事,忙得连膳食都是端进去用的,不就是为了准噶尔的事么?''

太后一扬团扇,羊脂玉柄上垂下的流苏便簌簌如颤动的流水。太后双眉紧蹙,扬声道:``皇帝忙着议事,哀家本无话可说。可若是议准噶尔的事,哀家听了便要生气。这有什么可议的?!哀家成日只坐在宫里坐井观天,也知道达瓦齐拥兵造反,杀害台吉多尔札,乃是乱臣贼子,怎的皇帝不早早下旨平定内乱,以安准噶尔!''

如懿听着太后字字犀利,如何敢应对,只得赔笑道:``皇额娘所言极是。但儿臣身在内宫,如何敢置喙朝廷政事,且多日未见皇上,皇额娘所言儿臣更无从说起啊!''

这话说得不软不硬,即将自己撇清,又提醒太后内宫不得干政,太后眸光微转,取过手边一碗浮了碎冰的蜜煎荔枝浆饮了一口,略略润唇。

那荔枝浆原是用生荔枝剥了榨出其浆,然后蜜煮之,再加冰块取其甜润冰凉之意,然而,此时此刻却丝毫未能消减太后的盛怒。太后冷笑道:``皇后说得好!内宫不得干政!那哀家不与你说政事,你是国母,又是皇后,家事总是说得的吧?''

如懿忙欠身,恭顺道:``皇额娘畅所欲言,儿臣洗耳恭听。''

太后重重放下手中的荔枝浆,沉声道:``大清开国以来,从无公主丧夫再嫁这富。若不幸丧偶,或独居公主府,或回宫安养,再嫁之事闻所未闻,更遑论要嫁与自己的杀夫仇人!皇帝为公主兄长,不怜妹妹远嫁蒙古之苦,还要商议她亡夫之事,有何可议?派兵平定准噶尔,杀达瓦齐,迎回端淑安养宫中便是!''

如懿端然含笑道:``皇额娘说得在理。皇上心中哪有不眷顾端淑长公主的,自幼一起长大,情分固然不同,何况是一母同胞的兄妹。''她的笑意有些意味深长的隽永,``且皇额娘有心如此,皇上是您亲子,母子连心,又怎会不听皇额娘的话?''

这一语,便是挑破了种种无奈,太后纵然位极天下群女之首,但皇帝实际并非她亲生,许多事她虽有意,又能奈何?

太后语塞片刻,柔淑长公主温声细语道:``儿臣记得皇兄东巡齐鲁也好,巡幸江南也好,但凡过孔庙,必亲自行礼,异常郑重。皇嫂说是么?''未等如懿反应过来,柔淑再度宁和微笑,``可见孔孟礼仪,已深入皇兄之心,大约不是做个样子给人瞧瞧的吧。既然如此,皇兄又遣亲妹再嫁,又是嫁与杀夫仇人,若为天下知,岂不令人嗤笑我大清国君行事做作,表里不一?''

同在宫中多年,柔淑长公主给她的印象一直如她的封号一般,温柔婉约,宁静如碧。便是嫁为人妻之后,亦从不自恃太后亲女的身份而盛气凌人,仿佛一枝临水照花的柔弱迎春,有洁净的姿态和婉顺的弧度。而记忆中的端淑,却是傲骨凛然,如一支凛然绽放于寒雪中的红梅。却不想柔淑也有这般犀利的时候,她不觉含笑,原来太后的女儿,都是这般不可轻视的。

如懿温然欠身:``皇上敬慕孔孟之心,长公主与本宫皆是了然,只是国事为上,本宫虽然在意姑嫂之情,但许多事许多话,碍于身份,都无法进言。''

柔淑含着温柔的笑意,轻摇手中的素色纨扇:``皇嫂与旁人是不同的。皇嫂贵为皇后,又诞育嫡子,且此刻怀有身孕,所以即便您说什么,皇兄都不会在意。''她的目光中含了一缕寸薄的悲悯与怅然,``皇兄忙于国事,我只是公主,皇额娘也不能干预国事,只是想皇兄能于百忙之中相见,让皇额娘亲自与皇兄共叙天伦,不知如此,皇嫂可愿意否?''

如懿垂眸凝神,须臾,低低道:``其实皇额娘苦心多年,也是知道儿臣的话未必管用,如今的情形,便是孝贤皇后在世也怕是难以置喙,若是舒妃和庆嫔\ldots\ldots{}''

太后眸光微微一颤,含了一缕凄悯的苦笑,道:``不中用了!嫔妃不过只是嫔妃,而你是皇后。''太后有一瞬的茫然,``这些日子,哀家多次让福珈去请皇帝,皇帝却只托言政事忙碌,未肯一顾,哀家是怕,皇帝是有心要让端淑再安胎了。''她眼中盈然有泪,``端淑是哀家长女,先前是嫁蒙古,是为国事。哀家虽然不舍,也不能阻止,但如今端淑丧夫,哀家如何忍心让她嫁于弑夫之人,终身为流言蜚语所苦。''她别过头,极力忍住泪,``哀家,只是想让自己的女儿回到身边安度余生,皇后,你能够懂得么?''

柔淑在旁轻声道:``无他,皇嫂只所孔孟之礼与皇额娘的话带到即可。我与皇额娘不勉强皇嫂做力所不能及的事。''她双眸微微一瞬,极其明亮,``不为别的,只为皇嫂还能看在皇额娘拉了你一把出冷宫的分上。''

有片刻的沉默,殿中置有数个巨大银公盆,堆满冬天存于冰库的积雪,此刻积雪融化之声静静入耳,滴答一声,又是一声,竟似无限心潮就此浮动。

太后的声息略微平静:``若你念着你姑母乌拉那拉氏的仇,自然不必帮哀家,但哀家对你,亦算不薄。''她闭目长叹,``如何取舍,你自己看着办吧。''

如何取舍?一直走到勤政殿东侧的芳碧丛时,如懿犹自沉吟。脚步的沉缓,一进一退皆是犹豫的心肠。

太后固然是自己的恩人,却也是整个乌拉那拉氏的仇人。若非太后,自己固然走不到今日万人之上的荣耀,安为国母?但同样若非太后,初入宫闱那些年,她怎会走得如此辛苦,举步维艰?

\hypertarget{ux7b2cux4e8cux5341ux4e94ux7ae0-ux5973ux54c0}{%
\chapter{第二十五章
女哀}\label{ux7b2cux4e8cux5341ux4e94ux7ae0-ux5973ux54c0}}

芳碧丛是皇帝夏日避暑理政之地。皇帝素爱江南园林以石做``瘦、漏、透''之美,庭中便置太湖石层峦奇岫,林立错落,引水至顶倾泻而下。玉瀑飞空,翠竹掩映。风吹时,便有凤尾森森、龙吟细细绵凉爽宜人,穿过曲折的抄手游廊,一路是绿绿的阔大芭蕉,被小太监们用清水新洗过,绿得要滴出水来一般,如懿伸手轻佛,仿佛还闻得到青叶末子的香。园中深处还养着几只丹顶鹤,在石间花丛中剔翎摆翅,悠然自乐,檐下的精致雀笼里亦挂着一排各色珍奇鸟儿,不时发出清脆悦耳的悠悠鸣声。

李玉正领着小太监们用粘竿粘了树上恣意鸣叫的暗里是蝉儿,见了如懿,忙迎了上来,轻声道:``皇后娘娘怎么来了?您小心身子。''

如懿轻婉一笑,望着殿内道:``皇上还在议事么?''

李玉悄悄儿道:``几位大人半个时辰前走的,皇上刚刚睡下,这几日,皇上是累着了,眼睛都熬红了。''

如懿思忖片刻道:``那本宫不便进去了?''

李玉抿嘴笑得乖觉:``旁人便罢了,您自然不会。皇上这些日子虽忙,却总惦记着您和您腹中的孩子呢,还一直说不得空儿去看看十二阿哥。''

或许是``孩子''二字挑动了如懿犹豫不定的神经,她终于敛衣整肃,缓声道:``那引本宫去见见皇上吧。''

从芳碧丛出来之时,已经是暮色沉沉的时分,她与皇帝说了什么,自然只有她自己与皇帝知,但是她明白,她说的话,还是打动了皇帝。

夕阳西坠,碎金色的余晖像是红金的颜料一样浓墨重彩地流淌。暮霭中微黄的云彩时卷时舒,幻化出变幻莫测的形状,让人生出一种随波逐流的无力,有清风在琼楼玉宇间流动,微皱的湖面上泛出金光粼粼的波纹,好似幽幽明灭的一湖心事。

容珮扶着她自后湖便沿着九幽廊桥回去,贴心道:``今日之事是叫娘娘为难,可娘娘为什么还是去劝皇上了?''

如懿将被风吹得松散的发丝抿好,正一正发髻边的一支佛手纹镶珊瑚珠栀子钗,轻声道:``你也觉得本宫犯不上?''

容珮想一想,低眉顺目道:``有时候,多一事不如少一事,娘娘现下事事安稳,稳坐后宫,何必去蹚这摊浑水呢。''她有些担心,``万一惹恼了皇上\ldots\ldots{}''

如懿淡然道:``皇上和太后到底是母子,躲得过初一,躲不过十五,总是要见的。''

``可舒妃和庆嫔是太后的人,太后不用她们,而用娘娘您,这件事便不好办\ldots\ldots 自然娘娘是能办好的,只是太冒险了些,何况太后昔年到底对乌拉那拉皇后太狠辣了。''

如懿凝望着红河日下,巨大而无所不在的余晖将圆明园中的一切都笼罩其下,染上一抹金紫色的暗光。

``太阳总会下山,就如花总会凋谢。不为过去的恩怨,也不为眼前的得失,只为来日。''如懿的语中带了一分冷静至极的无奈,``来日,本宫总有花残粉褪,红颜衰老的时刻,彼时若因本宫失宠而连累自己的孩子,那么太后还可以是最后一重依靠。哪怕没有权势,太后终究还是太后,本宫没有母族可以依靠,若连自己都靠不住,那么今日帮太后一把,便是帮来日的自己一把了。''

容珮忙伸手掩住她的口,急急道:``娘娘正当盛宠,又接连有孕,怎会如此呢?''

如懿眼中是一片清明的了然:``有盛,便有盛极而衰的时候,谁也逃不过。''

容珮微微颔首,忽然道:``若是乌拉那拉皇后在世,不知会作何感想?''

如懿笑着戳了戳她:``以姑母的明智,一定不会如本宫这般犹疑,而是立刻便会答应了。''

到了晚膳时分,皇帝便急急进了长春仙馆,皇帝进了殿,见侍奉的宫人们一应退下了,连太后最信任的福珈亦不在身边,便知太后是有要紧的话要说,忙恭恭敬敬请了安,坐在下首。

为怕烟火气息灼热,殿中烛火点得不多,有些沉浊偏暗。初夏傍晚的暑意被殿中银盆里蓄着的积雪冲淡,那凉意缓缓如水,透骨袭来。手边一盏玉色嵌螺钿云龙纹盖碗里泡着上好的碧螺春,第二开滚水冲泡之后的翠绿叶面都已经尽情舒展开来,衬着玉色茶盏色泽更加绿润莹透。

皇帝眼看着太后沉着脸,周身散发着微沉而凛冽的气息,心底便隐隐有些不安。名为母子这么多年,皇帝自十余岁时便养在太后膝下,从未见过太后有这般隐怒沉沉的时候,便是昔年乌拉那拉皇后步步紧逼之时,太后亦是笑容恬淡,不露一毫声色。

这样的女子,也有沉不住气的时候?

皇帝默默想着,在惊诧之余,亦多了一分平和从容,原来再睿智相谋的女子,亦不过逃不脱儿女柔肠。

这样想着,他的神色便松弛了不少,口吻愈加温和孝谨:``皇额娘急召儿子来此,不知为何?若是天气炎热,宫人供奉不周,皇额娘尽管告知儿子就是。''

太后的脸色被耳畔郁蓝的嵌东珠点翠金耳坠掩映得有些肃然发青:``宫人伺候不周,哀家自然可以告诉皇帝,若哀家自己的儿子不孝,哀家又能告诉谁去?''

皇帝闻得此言,遽然起身道:``皇额娘的话,儿子不敢承受。''

太后冷然目视片刻,沉沉道:``皇帝不敢?国事要紧,哀家不敢计较皇帝晨昏定省的礼节,只是有一句话,不得不问问皇帝。''她深深吸一口气,``自达瓦齐求亲以来已有十日,皇帝如何定夺自己亲妹的来日?''

皇帝垂眸片刻,温和地一字一字道:``端淑妹妹自幼为先帝掌上明珠,朕怎肯让妹妹孤老终身,达瓦齐骁勇善战,刚毅有谋,是可以托付终身的男子。''

太后几乎倒吸一口凉气,双唇颤颤良久,方说得出话来:``皇帝的意思是\ldots\ldots{}''

皇帝和缓地笑:``妹妹嫁与准噶尔许久,与多尔札一直不睦,未曾生养。如今天意如此,要妹妹再嫁一位合意郎君。儿子这个做兄长的,岂有不成全的?想来皇额娘得右,也一定为得佳婿而欣慰。''

太后震颤须臾,厉声道:``端淑初嫁不睦,哀家不能怪皇帝。当时先帝病重垂危,端淑虽然年幼,但先帝再无年长的亲女,为保社稷安定,为保皇帝安然顺遂登基,哀家再不舍也只能遂了皇帝的心意,让她下嫁准噶尔。可如今她夫君已死,准噶尔内乱,皇帝身为兄长,身为人君,不接回身处动乱之中的妹妹,还要她再度出嫁,还是嫁与手刃夫君的仇人,这置孔孟之道于何地?置皇家颜面于何地?''

皇帝不惊不恼,含着笃然的笑意,垂眸以示恭顺:``皇额娘放心,皇家的颜面就是公主再嫁嫁得风光体面,保住一方安宁。孔孟之道朕虽然尊崇,但那到底是汉人的礼节,咱们满蒙之人不必事事遵从。否则,当年顺治帝娶弟妇董鄂皇贵妃,岂非要成为千夫所指,让儿臣这个为人子孙的,也要站出来谴责么?''

太后目光坚定,毫无退让之意:``顺治帝娶弟妇董鄂皇贵妃之时,是我大清刚刚入关未顺民俗之时。可如今我大清开国百年,难道还要学关外那些未开化之时的遗俗。让百姓们在背后讥笑咱们还是关外的蛮子,睡在京城的地界上还留着满洲帐篷和地窖子的习气?''

皇帝俊秀的面容上笼上一层薄薄的笑容,带着薄薄若飞霜的肃然:``皇额娘不必动气,儿臣何尝不想迎回妹妹?但如今达瓦齐在噶尔颇得人心,深得亲贵拥戴。朕若强行用兵,一来边境不宁;二来不啻与整个准噶尔为敌,更为艰难;三来,天山一带的大小和卓隐隐有蠢蠢欲动之势,朕若让他们连成一片,必会成为心腹大患。''

太后的面容在烛火的映耀下显得阴暗不定,冷笑道:``皇帝到底是以江山为要,嫡亲妹妹亦可弃之不顾啊!果然是个好皇帝,好皇帝!''

皇帝脸色渐渐不豫,仍极力勉强着口吻上的恭顺:``皇额娘指责儿子,儿子无话可回。但皇额娘可曾想过,即便朕即刻发兵前往准噶尔平息达瓦齐,但端淑妹妹身在准噶尔早已被软禁,若达瓦齐恼羞成怒,一时毁了妹妹名节,或不顾一切杀了妹妹,皇额娘是否又要怪罪儿子不孝?这样的结果,皇额娘可曾想过?与其如此,不如顺水推舟,将妹妹嫁与达瓦齐,便也无事了。也当是妹妹初婚不慎,多尔札对妹妹不甚爱重,如今天意所在,要让妹妹得个一心想娶她的好夫君吧!''

太后像受不住寒冷似的,浑身栗栗发颤,良久,郎然笑道:``好!好!好!皇帝这般思虑周全,倒是哀家这个老婆子多操心了。''她缓缓地站起身,那目光仿佛最锋利的宝剑一样凝固着凌杀之意,直锥到皇帝心底。``其实皇帝最怕的,是达瓦齐要用你妹妹的性命来要挟皇帝付出其他的东西吧。如今可以不费一兵一卒就平息了准噶尔的叛乱,皇帝你自然是肯的。''她仰起脸长笑不已,``宫里的女人啊,哪怕是贵为公主,还是逃不掉受人摆布的命运,真是天可怜见儿?!''

烛火在皇帝眉心跃跃跳动,皇帝十分镇定,慢慢啜了口茶,道:``皇额娘不必过于担心,孝贤皇后是儿子的结发妻子,当年蒙古求娶孝贤皇后的嫡女和敬公主,她亦能深明大义啊。''

``皇帝有此贤妻,真是皇帝的好福气。''她颓然含笑,脸上多了几许无能为力的苍老,``哀家无用,这辈子只得两个公主,帮不了皇帝的千秋江山多久,如今啊,你的皇后又怀了身孕,皇帝你已经有那么多阿哥了,若是得个公主多好,来日一个个替你和亲远嫁,平定江山,可胜过百万雄兵呢。''

皇帝脸上的肌肉微微的搐,有冷冽的怒意划过眼底,旋即含了不动声色的笑意道:``皇额娘说得极是。女子倾城一笑,有时更胜男子孔武之力。当年孝庄皇太后为力保顺治爷的江山,不惜以一身牵制摄政王多尔衮。''她将这一抹笑意化作深深一揖,``自然了,儿子不会那么不孝,舍出自己的亲额娘去,自然会为皇额娘颐养天年,以尽人子孝道。''

太后一怔,跌坐于九凤宝座之内,伸出手颤颤指着皇帝道:``你\ldots\ldots 你\ldots\ldots 皇帝,你好!你好!''

皇帝含笑,恭谨道:``有皇额娘调教多年,儿子自然不敢不好。夜深,皇额娘早些睡吧。不日端淑长公主大婚,一切礼仪,还得皇额娘主持呢。这样,妹妹才好嫁得风风光光啊!''

太后看着皇帝萧然离去,怔怔地落下泪来,向着帘后转出的福珈道:``福珈!福珈!这就是哀家当年选出的好儿子!他\ldots\ldots 他竟是这样任性执妄,听不得旁人半句啊!''

福珈默然落泪,说不出一句安慰的言语,只得紧紧拥住太后,任由她伤心欲绝。

鎏金青兽烛台上的烛火跳跃几下,被从长窗灌入的凉风忽地扑灭,只袅袅升起一缕乳白轻烟,仿似最无奈的一声叹息,幽幽化作深宫里一抹凄微的苍凉。

数日后,如懿与海兰结伴而行,后湖上一湖新荷嫩绿,风凉似玉,曲水回廊悠悠转转,倒有不胜清凉之意。

海兰搀扶着如懿缓缓行走,端详着如懿的身形道:``娘娘的身子更圆润了些。臣妾瞧着上一胎肚子尖尖儿的,这一胎却有些圆,怕是个公主吧。''

如懿见侍女们远远跟着,低声笑道:``生永璂的时候多少谨慎,想吃酸的也不敢露出来,只肯说吃辣的。如今倒真是爱吃辣的了,连小厨房都开玩笑,说给本宫炒菜的锅子都变辣了。''

海兰小心翼翼地抚着如懿的肚子微笑:``是个公主便好。女儿是额娘的贴心小棉袄,臣妾便一直遗憾,膝下只有一个永琪,来日分府出宫,臣妾便连个说贴心话的人都没有了。''

如懿望着湖上碧波盈盈,莲舟荡漾,翠色荷叶接天碧,芙蕖映日别样红,水波荡漾间,折出凌波水华,流光千转。风送荷芰十里香,宫人们采莲的歌声在碧叶红莲间萦绕,依稀唱的是:``荷叶罗裙一色裁,芙蓉向脸两边开。乱入池中看不风,闻歌始觉有人来\ldots\ldots{}''

歌声回环轻旋,隔着水上觳波听来,犹有一唱三叹,敲晶破玉之妙,她知道,那是玉妍承宠的新主意,十分合皇帝的心意。

这样二八年华的妙龄少女,唱起来歌喉如珠,十分动人。如懿有些黯然,谁知道此刻欢欢喜喜唱着歌的少女,来日的命途又是如何呢?

她抚着自己肚子的手便有些迟缓,郁然叹道:``真是公主又如何?你且看太后亲生的公主尚且如此\ldots\ldots{}''

海兰瞧了瞧四周,连忙掩住她口:``娘娘不要说不吉利之言。''

如懿黯然垂眸:``本宫不过是唇亡齿寒,兔死狐悲罢了。''

海兰闻言亦有些伤感,便问:``端淑长公主再嫁之事定下了么?''

如懿颔首道:``已成定局,皇上已经下旨,封准噶尔台吉达瓦齐为亲王,于九月十二日迎娶端淑固伦长公主,如今礼部和内务府都已经忙起来了。''

海兰微微颔首:``再忙也是悄悄儿的。大清至今未出过公主再嫁之事,到底也是要脸面的。公主这次大婚可比不上上回风光了。''

``公主上回远嫁,正逢先帝垂危。一起仓促就事,哪里能多体面呢。这次嫁的更是自己的杀夫仇人。听说皇上已经给了公主密旨,要她一切以国事为重,不许有轻生之念。''

海兰越发压低了声音道:``公主在外是太后的掣肘,太后在内更是公主的顾虑,彼此牵念,最后只能遂了皇上的心意了。''

如懿明艳饱满的神色逐渐失去华彩:``端淑长公主如此,孝贤皇后亲生的和敬公主亦如此,别的公主还能如何呢?不过是生于帝王家,万般皆无奈罢了。''

海兰默然哀伤,亦不知如何接话,只掐了一脉荷叶默默地掰着,看着自己断月形的指甲印将那荷叶掐得凌乱不堪。

正沉吟间,只见三宝匆匆赶上来,打了个千儿道:``皇后娘娘,愉妃娘娘,舒妃那儿\ldots\ldots{}''

如懿遽然转身,问道:``是不是十阿哥\ldots\ldots{}''

三宝垂首道:``是。十阿哥不幸,已经过世了。''

如懿与海兰对视一眼,只觉得心中一阵阵抽痛,那个孩子,尚未来得及取名的孩子,幼小的,柔软的,又是如此苍白的,意这么去了。她不敢想象意欢会有多么伤心,十阿哥病着的这些日子里,意欢的眼睛已经成了两汪泉水,无止境地淌着眼泪,仿佛那些眼泪永远也流淌不完一样。

如懿情不自禁地便往回走,三宝急得拼命爬到她身前磕头道:``皇后娘娘,您不能去,您不能去!''

如懿喝道:``起开!''

海兰忙扶着如懿,手上加紧了力气,扯住如懿道:``娘娘!是不能去!您怀着身孕,快要生产了,丧仪悲伤之地,您是不能踏足的!''

如懿吃力地撑起腰肢,正色道:``本宫是皇后,一切邪妄不至本宫之身,本宫不怕的,本宫的孩子自然不会怕!''

如懿和海兰赶到春雨舒和之时,宫人们都已经退到了庭院之外,开始用白色的布缦来装点这座失去了幼小生命的宫苑。

如懿悄然步入寝殿,只见意欢穿着一袭棠色暗花缎大镶边纱氅,一把青丝以素金镂空扁方高高挽起,疏疏缀以几点青玉珠花,打扮得甚是清爽整齐,并无半点哀伤之色,如懿正自诧异,悄悄走近,却见意欢安静地坐在孩子的摇篮边,双手怀抱胸前,紧紧抱着一个洋红缎打籽彩绣襁褓,口中轻轻地哼着:``风吹号,雷打鼓,松树伴着桦树舞,哈哈带着弓和箭,打猎进山谷,哟哟呼,打猎不怕苦,过雪坎,爬冰湖,藏在老虎必经路,拉满弓来猛射箭,哟哟呼,除掉拦路虎\ldots\ldots{}''

她轻轻地哼唱着,歌声中带了如许温然慈爱之意,一抹如懿从未见过的温柔笑意如涟漪般在她唇边轻轻漾开,一手抚摸着怀中孩子已经苍白没有血色的面孔。

如懿望着她,心中似一块薄瓷,渐渐蔓延上细碎而酸楚的裂纹,她回首看了海兰一眼,海兰走近了,柔声笑着哄道:``好妹妹,你也抱得累了,我来替你抱一抱十阿哥吧。''

意欢警觉地抬起头,紧紧抱着孩子往后一缩,以戒备的目光看着如懿和海兰。

海兰温声道:``你唱得累不累?是不是渴了?''她从桌边倒了一盏热茶,招手道:``快来喝口水,否则嗓子唱哑了,可不好听了,十阿哥不会喜欢呢。''

意欢无限爱怜地看了看怀中的孩子,温柔道:``十阿哥不会喜欢?''

海兰笑意温婉,亲热道:``可不是?十阿哥听了你唱歌可喜欢呢,等下我的五阿哥也来,好么?''

意欢微微松了松手,不知是否该放下怀中的孩子,如懿好声好气地哄着道:``你去喝水吧,孩子的襁褓该换一换啦!本宫知道你不喜欢别人碰十阿哥,本宫来吧。你放心的,是不是?''

意欢迟疑片刻,小心翼翼地将孩子放到如懿怀中,爱怜地摸了摸孩子的脸,浅笑如冬日里最贴身的锦衾一般暖和,她柔声道:``额娘去喝口水,立刻回来,好孩子,你别怕啊!''

意欢双手放开的一瞬,如懿摸到了孩子的脸,那脸是冰冷的,没有一丝活气,甚至有些僵硬了。如懿心中一酸,泪水情不自禁地滑落下来,她如何敢给意欢瞧见,慌忙背转身擦去了。

意欢匆匆喝完水,只盯着如懿怀中的孩子,迫不及待伸手要抱回。她迫切而不舍地道:``我的孩子只肯要我抱的,给我吧。''

如懿见她如此,仿佛还不知道孩子早已死去,只得柔声道:``意欢,你累了,本宫替你抱一会儿吧。''

意欢脸上的慈爱之色顿时消去,如一匹警觉的母狼,狠狠盯着如懿道:``你要做什么?你要抢我的孩子做什么?''

海兰忍不住拭泪道:``舒妃,十阿哥已经过去了,你\ldots\ldots{}''

她话音尚未落,意欢用力搡了如懿一把,扑上前从如懿怀中夺过孩子紧紧抱住,将脸贴在他全然失去温度的小脸上,她的神色旋即温和,温柔甜美的笑容像从花间飞起蹁跹的蝴蝶,游弋在她的青黛眉宇之间。她继续轻轻地哼唱。回首盈然一笑:``小点儿声,十阿哥睡着了,他不喜欢别人吵着他睡觉呢。''

海兰看了看如懿,带了一抹酸楚的不忍,轻声道:``舒妃妹妹怕是伤心得神志不清了。''她转而担忧不已,``这可怎么好?''

暮色以优柔的姿态渐渐拂上宫苑的琉璃碧瓦,流泻下轻瀑般淡金的光芒,穿过重重纱帷的风极轻柔,轻轻地拔弄着如懿鬓边一支九转金枝玲珑步摇,垂下的水晶串珠莹莹晃动,风时有几丝幽幽甜甜的花香,细细嗅去,竟是茶蘼的气味,淡雅得让人觉得全身都融化在这样轻柔的风里似的。

明明是这样温暖的斜阳庭院,如懿不知怎的,忽然想起许多年前的一日,仿佛还是意欢初初承宠的日子。某一日绿琐窗纱明月透的时候,看她独立淡月疏风之下,看她翔鸾妆详、粲花衫绣,轻轻吟唱不知谁的词句。那婉转的诗句此刻却分明在心头,``淡烟疏风冷黄昏,零落茶蘼花片,损春痕''。

如今的余晖斜灿,却何尝不是淡烟疏风冷黄昏,眼看着茶蘼落尽,一场花事了。

\hypertarget{ux7b2cux4e8cux5341ux516dux7ae0-ux9189ux68a6}{%
\chapter{第二十六章
醉梦}\label{ux7b2cux4e8cux5341ux516dux7ae0-ux9189ux68a6}}

海兰与如懿陪在一侧,看着意欢神志迷乱,满心不忍,却又实在劝不得。海兰便问守在一旁的荷惜:``皇上知道了么?可去请过了?''

荷惜揉着发红的眼睛:``去请了。可皇上正和内务府商议端淑长公主再嫁准噶尔达瓦齐之事,一时不得空儿过来。''

海兰看着如懿,忧烦道:``怕不只是为了政事,皇上亦是怕触景伤情吧?''

如懿心底蓦地一动,冷笑道:``触景伤情?''

是呢,可不是要触景伤情?十阿哥生下来便是肾虚体弱,缠绵病中,与药石为伍,焉知不是当年皇帝一碗碗堕胎药赏给意欢喝下的缘故,伤了母体,亦损了孩子。

所以,才不敢,也不愿来吧!

如懿的心肠转瞬刚硬,徐徐抬起手腕,玉镯与雕银臂环铮铮碰撞有声,仿佛是最静柔的召唤。她探手至意欢身边,含了几许柔和的声音,却有着旁观的冷静与清定,道:``孩子已经死了!意欢,去!去给皇上亲眼瞧瞧,瞧瞧他的孩子是怎么先天不足不治而死的!只有让他自己瞧瞧,才能刻骨铭心,永志不忘!''

意欢猛然抬首,死死地盯着如懿,发出一声凄恻悲凉的哀呼:``不!我的孩子没有死!没有死!''她紧紧搂着怀中的孩子,``他会笑,会哭,会动,会喊我额娘了。我打得孩子不会死!不会死!''

她的哭声悲鸣呜咽,如同母兽向月的凄呼,响彻宫阙九霄,久久不散。

海兰扶住她肩膀,落泪道:``舒妃妹妹,十阿哥真的已经过去了。你若有心,就让他皇阿玛见见他最后一面。这个孩子,毕竟是你和皇上唯一的孩子啊。''

许是海兰所言的``唯一''打动了她,意欢隐忍许久的泪终于喷薄而出。如懿牵着她的手出去:``把你的眼泪去掉给皇上看,你的丧子之痛,也应该是他的痛彻心扉。''

意欢抱着孩子疾奔而出,海兰依傍在如懿身边,仿佛一枝婉转的女萝,奇怪道:``娘娘此举,仿佛是深怨皇上?''

如懿的唇角含了一缕苦笑:``或许是本宫在宫中浸淫日深,本宫所能想到的,是这个孩子不能白白死去,意欢不能白白伤心。且孩子的死,难道皇上没有牵涉前因于其中么?''

海兰浅浅一笑,好似一江刚刚融化的春水:``娘娘这样,臣妾很高兴。''她眸中微微一亮,仿佛彩虹的光霓,``这才是深处宫中的存活之道啊!''

十阿哥的丧仪已经过了头七,而意欢,仍旧沉溺于丧子之痛中,无法自拔。

许是十阿哥的死去后凄惨模样刺激了身为人父的皇帝,皇帝特许恩遇早夭的十阿哥随葬端慧皇太子园寝。这样的殊荣,亦可见皇帝对十阿哥之死的伤怀了。

意欢深深谢恩之后,仍是伤心不已,卧床难起。如懿前去探望时,她仅着一层素白如霜的单衣躺在床上,手中死死抓着十阿哥穿过的肚兜贴在面颊上,血色自唇上浅浅隐去,青丝如衰蓬苦草无力地自枕上蜿蜒倾下,锦被下的她脆弱得仿佛若一片即将被暖阳化去的青雪。

如懿倚在门边,想起自己从冷宫出来时初见意欢的那一日,墨瞳淡淡潋滟如浮波,笑意娆柔如临水花颜。那样明亮的容颜,几乎如一道雪紫电光,划破了暗沉天际,让人无法逼视。

如懿自知劝不得,亦不忍观,只得将带来的燕窝汤羹放在她身前喂她喝了半盏,才默默离去。

离开春雨舒和之后,如懿心情郁郁不乐,便扶了容璟往四宜书屋去探望正在读书的永琪。

彼时正在午后,宫中人大多正在酣眠,庭院楼台格外寂静。天光疏疏落落,雨线漫漫如纷白的蚕丝,将这渺渺无极的空远的天与地,就这样缠绵逶迤在一起,再难隔离。如懿穿着半旧的月白色团荷花暗纹薄绸长衣,漫着明珠丝履,扶着腰缓缓走过悠长曲折的回廊。雨滴打在重重垂檐青瓦上,打在中庭芭蕉舒展开的新嫩阔大绿叶上,清越之声如大珠小珠落玉盘。

绕过武陵春色的绾春轩时,如懿尚闷闷不觉。武陵春色四周遍种山桃千百株,参错夹杂林麓间。若待三月时节,落英缤纷,浮漾水面,或朝曦夕阳,光炫绮树,酣雪烘霞,其美莫可名状。

而此时,亦不当桃花时节,再好的武陵人远,也是春色空负。

吸引如懿的,是一串骊珠声声和韵闲。

那分明是一副极不错的嗓音,若得时日调教,自然会更清妙,一声声唱着的,是极端艳袅娜的一首唱词:

没乱里春情难遣,蓦地里怀人幽怨。则为俺生小婵娟,拣名门一倒一倒里神仙眷。甚良缘,把青春抛得远。俺的谁情谁见,则索因循腼腆。想幽梦谁边,和春光暗流转。迁延,这衷怀哪出言。淹煎,泼残生除问天。

静静的午后,沿着雨声绵绵,那声线清亮好似莺莺燕燕春语关关。过了片刻,那女声幽咽婉扬,又唱到:

好景艳阳天。万紫千红开遍。满雕栏宝砌,云簇霞鲜。督春工珍护芳菲,免被那晓风吹颤。使佳人才子少系念,梦儿也十分欢忭。

虽无人应和,但那歌声与雨声相伴,似名泉花低流溪涧,十分动听。

如懿沉下了脸,冷冷道:``十阿哥新丧,皇上与舒妃都陈郁不悦,谁在这里唱这样靡艳的词调?''

三宝上前道:``回娘娘的话,绾春轩是令妃的住处。听闻这些日子皇上都甚少招幸令妃,所以她闲下来在向南府的歌伎学习昆曲唱词呢。''

如懿面无表情:``三宝,去绾春轩查看,不论是谁在十阿哥丧中不知轻重唱这些欢词靡曲,一律掌嘴五十,让她去十阿哥梓宫前跪上一日一夜作罚。''

第二日,如懿便在为十阿哥上香时,看到了双目红肿,两颊高高肿起带着红痕的嬿婉。

嬿婉见了如懿便有些怯怯的,缩着身子伏在地上:``臣妾恭迎皇后娘娘。''

如懿并不顾目于她,只拈香敬上。许久,她才缓缓道:``本宫责罚你,算是轻的。''

嬿婉哀哀垂泪,十分恭谨:``臣妾一时忘情,自知不该在十阿哥丧期唱曲。皇后娘娘无论怎样责罚,臣妾都甘心承受。只是娘娘\ldots\ldots{}''她仰起墨玉色的眸子,含了楚楚的泪,``不知为何,臣妾总觉得娘娘对臣妾不如往日了。是否臣妾莽撞,无意中做了冒犯娘娘之事,还请娘娘明言,臣妾愿意承受一切后果,但求与娘娘相待如往日。''

她楚楚可怜的神色在瞬间激起如懿最心底的不屑与鄙夷,然后,她不认为有必要与之多言,只淡然道:``这两年来你所做的这些事,当本宫都不知道么?''

嬿婉伏下身体,如一只卑躬屈膝的受惊的小兽,俯首低眉,道:``皇后娘娘所言若是指臣妾当日一时糊涂未能劝得皇上饮鹿血之事,臣妾真心知错。若娘娘还不解气,臣妾任凭责罚。''

如懿看着她姣好的与自己有几分相似的面庞,摇首道:``本宫对你所做的责罚只是明面上之事,你私下的所作所为,你自己当一清二楚。若以后你安分度日,本宫可以不与你计较;若再想施什么手段,本宫也容不得你。''她说罢,拂袖离去。

嬿婉在她走后,旋即仰起身体。春婵忙扶住嬿婉起身道:``小主,仔细跪得膝盖疼。''

嬿婉冷笑数声:``好厉害的皇后!好大的口气!''她到底有些许不安。``春婵,你说,皇后到底知道了什么?''

春婵柔顺道:``皇后娘娘此举,大约只是因为与舒妃交好,同情她丧子的缘故。若真知道了什么,以皇后娘娘今日的态度,哪里能容得下小主呢?''

嬿婉的脸色如寒潮即将来临前浓翳的天色,望向如懿背景的目光,含了一丝不驯的阴翳神色,宛如夜寒林间的孤鸮厉鹫,竦寒惊独,在静默中散出怨恨而厉毒的光芒。

比之伤心欲绝,更让如懿担心的是意欢的彻底麻木。意欢仿佛失去了对这个世界的所有知觉,不会哭,不会笑,对任何人的言语都置若罔闻。待到数日后意欢能勉强起身之时,便只把所有的心思和精力都用在了抄录皇帝的御诗之上。

皇帝亦来看望过她几次,甚至不得已硬生生夺去了她手中的笔墨。然而,她只是怔怔地望着皇帝,伸出手道:``还给我,还给我!''

皇帝不禁揽住她落泪:``意欢,你还年轻,会有孩子的。''

她只死死将孩子的衣物抱在怀中,喃喃道:``我只要这个孩子,只要这个!''

然后,在悲痛之余,将自己更疯狂地沉浸在纸张与笔墨之中。

一开始没有人敢去懂意欢辛苦手抄的御诗,直到最后,众人渐渐明白,她是在皇帝早年所作的御诗里,寻找着自己爱过、存活过的痕迹和那些爱情带来的短暂而苦涩的结果。

意欢迅速地憔悴下去,像一脉失去了水分的干枯花朵,只等着彻底萎谢的那一天。

有几次如懿和海兰在她身边陪守着她,亦不能感觉到她抄写之余其他活着的痕迹。连每一次前往十阿哥的梓宫焚烧遗物与经卷,亦是不落一滴眼泪,更不许人陪伴,只她一人守着孩子的棺椁,低低倾诉。

宫人们私下都议论,舒妃因着十阿哥的死形同疯魔,连太后的劝说亦不管不顾,充耳未闻。唯有海兰向如懿凄然低诉,那是一个母亲最大的心死,不可挽回。

这一日,意欢到十阿哥的梓宫前,正见嬿婉穿了一袭银白色素纱点桃氅衣,打扮得十分素净,跪在十阿哥的棺椁前,慢慢地往火盆里烧着一卷经幡,垂泪不已。

意欢静静在她身边跪下,打开一个黑雕漆长抽匣,将里面折好的元宝彩纸一一取出,神色十分冷淡:``不是你的孩子,你来做什么?''

嬿婉的泪落在咝咝窜起火苗内,溅起骤然跳动的火花,哀戚道:``姐姐是来哭十阿哥,我是来哭一哭自己的孩子。''

意欢自永寿宫之后便不大喜欢嬿婉的妩媚惑主,她又是个喜怒形于色不喜欢掩饰之人,所以见了嬿婉便淡淡地不甚搭理。然而,此刻看嬿婉如此伤心欲绝,亦不觉触动了心肠,放缓了声音道:``你有什么孩子?''

嬿婉伸出手,试探地抚上意欢的小腹。意欢下意识地退避了寸许,见嬿婉神色痴痴惘惘,并无任何恶意,亦不知她要做什么,便直直僵在了那里不动。嬿婉的手势十分柔缓,像拂面的春风,轻淡而温暖,带着小心翼翼的珍视,低柔道:``姐姐,我的好姐姐,你是为十阿哥伤心,伤心得连自己都不要了。其实细想想,你总比我好多了。你的孩子好歹在你肚子里,你享了怀胎十月的期待,一朝降生的喜悦,你看过他的笑,陪过他哭,和他一起悲喜。可是,我的孩子呢?''她睁大了凄惶欲绝的眼,盯着意欢,喃喃道:``我的孩子在哪里?''

嬿婉的双手冰凉,隔着衣衫意欢也能感觉到她指尖潮湿的寒意,意欢有些不忍,亦奇怪:``你的孩子?''

蜿蜒似笑非笑,似哭非哭,像是魔怔了一般,``是啊,姐姐,你的孩子好歹还在你的腹中活过,好歹还在这个世间露了个脸,陪了你一遭。可是我的孩子呢?''她紧紧抚住自己空空如也的腹部,惶然落泪,``我的孩子连到我肚子里待上片刻的运气也没有。我盼啊盼,盼得眼睛都直了,我的孩子也来不了!他来不了我的肚子里,更来不了这个世上。''她睁着泪水迷蒙的眼,近乎癫狂般伤心,``你知道是为什么吗?''

意欢怔怔地道:``为什么?''

嬿婉仰天凄苦地笑,抹去眼角的泪,打开手边的乌木镇漆四色菊花捧盒,端出一碗乌墨色的汤药,药汁显然刚熬好没多久,散发着温热的气息。嬿婉端到意欢鼻尖,含泪道:``这碗汤药的味道,姐姐一定觉得很熟悉吧?''

意欢大为诧异,双眸一瞬闪过深深的不解:``你怎么会有我的坐胎药?''

嬿婉的泪如散落的珍珠,滚滚坠落在碗中,晕开乌墨的涟漪:``姐姐,是我蠢,是我贪心。我羡慕皇上赏赐你坐胎药的恩遇,我也想早日怀上一个自己的孩子,所以偷偷捡了你喝过的药渣配了一模一样的坐胎药,偷偷地喝。甚至我喝得比你还勤快,每次侍寝之后就大口大口地喝,连药渣也不剩下!''

意欢震惊不已:``那你\ldots\ldots 还没有孩子?''

嬿婉抹去腮边的泪,痴痴道:``是啊!我喝得比你勤快,却没有孩子。姐姐漏喝了几次,却反而有了孩子。''她逼视着她,目中灼灼有凌厉的光,``所以,姐姐,你不觉得奇怪么?这可是太医圣手齐鲁配的药啊!''

意欢战栗地退后一步,紧紧靠在十阿哥的棺椁边缘:``奇怪?有什么课奇怪的?''

``坐胎药没让咱们快快怀上孩子,这不奇怪么?于是,我去太医院私下找了好些太医询问,他们都是同一张嘴同一条舌头,都说这是上好的坐胎药。我便信了。可是姐姐,是你告诉我的,你漏喝了很多次反而有孕了。所以,我便托人去了宫外,拿药渣子和方子一问,才知道啊\ldots\ldots{}''她拖长了音调,迟迟不肯说下去,只斜飞了清亮而无辜的眼,欲语还休,清泪纵横。

意欢似乎意识到什么,声音都有些发颤:``你知道什么?''

嬿婉的泪汹涌滑落,逼视着她,不留分毫余地:``姐姐啊,难道你真不知道那是什么?否则你为什么不喝?''

意欢稍稍平静:``我不喝,只是因为喝了这些年都未有动静,也灰了心了。连皇后娘娘也说,天意而已,何必苦苦依赖药物,所以我的求子之心也淡了。''

嬿婉蹙眉:``难道皇后娘娘也没告诉你是什么?''

意欢沉静道:``皇后娘娘甚少喝坐胎药,她自然没有告诉过我。''

嬿婉的震惊只是瞬间,转瞬平静道:``那么,我来告诉你。''她的唇角衔了一丝决绝而悲切的笑容,``我和姐姐喝了多年的,从来不是坐胎药。皇上嫌你是叶赫那拉氏的女子,嫌你会生出爱新觉罗氏仇雠的种子,所以给你喝的是避免有孕的药物。''

意欢大为震惊,脸色顿时雪白,舌尖颤颤:``我不相信!''

嬿婉取出袖中的方子,抖到她眼前:``姐姐不信?姐姐且看这方子上的药物有没有错。上面所书此药是避免有孕之物,乃是出自京中几位名医之手,怎么有错?''她看着意欢的目光在接触到方子之时的瞬间如燃烧殆尽的灰烬,死沉沉地发暗,继续道:``皇后娘娘说得对,是药三分毒啊,所以我得知真相后停了药至今也怀不上孩子。所以姐姐怀着十阿哥的时候肾虚且带入了十阿哥的胎里,才使得十阿哥天生虚弱,不治而死啊!''她双膝一软,跪倒在火盆前,手里松松抓了一把纸钱扬起漫天如雪,又哭又笑,``孩子啊,可怜的孩子啊,你死在谁手里不好,偏偏是你的阿玛害死了你啊。什么恩宠,什么疼爱,都是假的啊!我可怜的孩子!''

嬿婉恸哭失声,直到身后剧烈的狂奔之声散去,才缓缓站起身,抚着十阿哥的棺椁,露出了一丝怨毒而快意的笑容。

意欢直闯进芳碧丛的时候,皇帝正握了一卷雪白画轴在手,临窗细观。一缕缕淡金色的日光透进屋子,卷起碎金似的微尘,恍若幽幽一梦。那光线洒落皇帝全身,点染勾勒出清朗的轮廓,衬着皇帝身后一座十二扇镂雕古檀黑木卷草缠枝屏风,繁绮华丽中透着缥缈的仙风意境。

意欢的呼吸有一瞬的凝滞,泪便漫上了眼眶。泪眼朦胧里,恍惚看见十数年前初见时的皇帝,风姿迢迢,玉树琳琅,便这样在她面前,露出初阳般明耀的笑容。

那是她这一生见过的最美好的笑容。

年轻的宫女半蹲半跪侍奉在侧打着羽扇。殿中极静,只有他沉缓的呼吸与八珍兽角镂空小铜炉里香片焚烧时哗剥的微响。那是上好的龙诞香,只需一星,香气便染上衣襟透入肌理,往往数日不散。

这样的气味,是她这么些年的安心所在,而此时此刻,却觉得陌生而森然。

皇帝对她的无礼的突如其来并不十分诧异,笑意如温煦的六月晨曦:``怎么这么急匆匆跑来了?满头都是汗!''他看着跟进来意图阻止的李玉,挥手道:``去取一块温毛巾来替舒妃擦一擦,别拿凉的,一热一凉,容易风寒。''

这般脉脉温情,是意欢数十年来珍惜且安享惯了的,可是此时听得入耳,却似薄薄的利刃刮着耳膜,生生地疼。

李玉安静退了出去,连皇帝身边的宫女亦看出她神情的异样,手中羽扇不知不觉缓下来,生怕有丝毫惊动。

意欢觉得躯体都有些僵硬了,勉强福了一福道:``皇上,臣妾有话对您说。''

\hypertarget{ux7b2cux4e8cux5341ux4e03ux7ae0-ux70c8ux706b}{%
\chapter{第二十七章
烈火}\label{ux7b2cux4e8cux5341ux4e03ux7ae0-ux70c8ux706b}}

皇帝挥了挥手,示意身边的人出去,恰逢李玉端了温毛巾上来,皇帝亲自去了,欲替她拭了汗水。意欢不自觉地避开他的手,皇帝有些微的尴尬,还是伸手替她擦了,温声道:``大热天的,怎么反而是一头冷汗?''

李玉看着情形不对,赶紧退下了。意欢的手有些发颤,欲语,先红了眼眶:``皇上,你这样对臣妾好,是真心的么?''

皇帝眼中有薄薄的雾气,让人看不清底色:``怎么好好儿问起这样的话来?''

他的语气温暖如常,听不出一丝异样,连意欢都疑惑了,难道她所知的,并不真么?于是索性问出:``皇上,这些年来,您给臣妾喝的坐胎药到低是什么?''

皇帝取过桌上一把折扇,缓缓摇着道:``坐胎药当然是让你有孕的药,否则你怎么会和朕有孩子呢?''

意欢心底一软,旋即道:``可是臣妾私下托人去问了,那些药并不是坐胎药,而是让人侍寝后不能有孕的药。''她睁大了疑惑的眼,颤颤道:``皇上,否则臣妾怎么会断断续续停了药之后反而有孕,之前每次服用却一直未能有孕呢?''

皇帝有片刻的失神,方淡淡道:``外头江湖游医的话不足取信,宫中都是太医,难道太医的医术还不及他们么?''

不过是一瞬间的无语凝滞,已经落入意欢眼中。她拼命摇头,泪水已经忍不住潸潸落下:``皇上,臣妾也想知道。宫外的也是名医,为何他们的喉舌不同与太医院的喉舌?其实,自从怀上十阿哥之后,臣妾也一直心存疑惑,为何之前屡屡坐胎药不见效,却是停药之后便有了孩子?而十阿哥为何会肾虚体弱,臣妾有孕的时候也是肾虚体弱?安知不是这坐胎药久服伤身的缘故么?''

仿若一卷冰浪陡然澎湃击下,震惊与激冷之余,皇帝无言以对。半响,他的叹息如扫过落叶的秋风:``舒妃,有些事何必追根究底,寻思太多,只是陡然增加自己的痛苦罢了。''

意欢脚下一个踉跄,似是震惊到了极处,亦不可置信到了极处。``追根究底?原来皇上也怕臣妾追根究底!''她的泪水无声地滚落,夹杂着深深酸楚与难言的恨意,``那么再容许臣妾追根究底一次。皇上多年来对臣妾虚情假意,屡屡不许臣妾有孕,难道是因为臣妾的出身叶赫那拉氏的缘故么?''

皇帝收了折扇,重重落在案几上,神色间多了几分凛冽:``舒妃,你是受了谁的指使在朕的身边,你当朕真的不知么?就算太后当日举荐了你侍奉朕左右,朕可以当你是懵然无知,但为了和敬与柔淑谁下嫁蒙古之事你劝朕的那些话,你和你身后的人,心思便是昭然若揭了。''

意欢眼中的沉痛如随波浮漾的碎冰,未曾刺伤别人,先伤了自己。``皇上认定了臣妾是叶赫那拉氏的女儿,是爱新觉罗氏仇雠,所以会受旁人摆布,谋害皇上?所以防备臣妾忌讳臣妾到如此地步?''

皇帝沉声道:``叶赫那拉氏也罢了,朕不是不知道,你是太后挑给朕的人,一直安在朕身边,是什么居心?''

太阳的光影疏疏地从窗棂里漏进来。皇帝原本便俽长的背影被拉得老长老长,斜斜映在漫地金砖之上。她的心骤然疼痛起来,那种痛更胜于孩子死在她怀中的那一刻。仿佛所有积累的伤口都彻底裂开了,被狠狠洒满了新盐。

意欢紧紧抱住自己的手臂,像是支撑不住似的,凄然厉声道:``臣妾虽然是太后挑选了送与皇上的,又得太后悉心点拨皇上的喜好厌恶。能得以陪伴皇上身侧,臣妾真心感激太后。但即便如此,也不代表臣妾会受太后所指。臣妾对皇上的心是真的!这些年来,难道皇上都不知么?''

皇帝的眼底闪过一丝疑忌,唇边的笑意如一柄刮骨利剑,让人森冷不已。他轻诮笑道:``太后在深宫多年,怎么会调教出一个对朕有真心的女子陪侍在朕身边,这样如何为她做事为她说话?不只是你,庆嫔7也好玫嫔也好,即便是富察氏送来的晋嫔,也不过如此罢了。''

意欢的泪凝在腮边,她狠狠抹去,浑不在意花了妆容,一抹唇脂凝在颔下,仿佛一道凄艳的血痕。她恨声道:``好厉害的皇上,好算计的太后!你们母子彼此较量,扯了我进去做什么?我清清白白一个女儿家,原以为受了太后引荐之恩,可以陪在自己心爱的男子身边,所以有时亦肯为太后进言几句。但我一心一意只在皇上你身边,却白白做了你们母子争执的棋子,毁我一生,连我的孩子亦不能保全!''她死死盯着皇帝,似乎要从他心底探寻出什么,``那么皇上,既然你如此疑忌太后,大可将我们这样的人弃之如敝屣,何必虚与委蛇,非得做出一副恩爱不已的样子,让人恶心!''

``恶心?''皇帝勃然变色,索性坦然道,``你们不也乐在其中安享朕的恩宠么?太后喜欢朕宠爱你们,朕就宠爱给她看!也叫她老人家放心!''他冷冷道,``人生如戏,左右大家不过是逢场作戏的戏子而已。''

意欢静默片刻,终于戚然冷笑,那笑声仿佛霜雪覆于冰湖之上,彻骨生冷:``原来这些年,都是错的!只我还蒙在鼓里,以为一心待皇上,皇上待我也总有几分真心。原来错了啊,都是错了啊!''

她在雪白而模糊的泪光里,望着那座十二扇镂雕古檀黒木卷草缠枝屏风,上头用大团簇拥的牡丹环绕口吐明珠的瑞兽,屏身乃上等墨玉精心雕琢镂空,枝蔓花朵,一花一叶,无不栩栩如生,屏风两端各有一联,是乌沉沉的墨色混了金粉,一书``和合长久'',一书``芳辰如意''。那是多好的祝词,仿佛这人间无不顺心遂意,花好月圆人长久,却原来不过是芳心绮梦,都是一场镜花水月的冰冷空虚而已。

皇帝的目光,如寒潭,如深渊,有深不见底的澈寒:``舒妃,你是错了。你的错便是不该去探寻所谓的真相。很多的美好便在与不知,你又何必要来问朕?既然你问朕,又不欲朕骗你,便是你自寻烦恼了。''

意欢只觉得身体轻飘飘的,皇帝的声音像是在极远处,渺渺飘飘地又近了,浮浮沉沉入了耳。意欢浑身簌簌发抖,仿佛小时贪那雪花洁白,执意久久握在手中。雪融化了,便再抓一把,结果直冷到心尖里。她强撑着福了一福,惨然笑道:``皇上说得是,是臣妾的错,臣妾有罪,是臣妾不该,在那年皇上祭陵归来时,摇摇一见倾心。是臣妾\ldots\ldots 都是臣妾的错。''

她木然转身,脚步虚浮地离开。李玉候在门边,有些担心地望着皇帝,试探着道:``皇上\ldots\ldots{}''

皇帝并不以为意:``罢了,这是舒妃自己想听的话,不必理会。只看着她,不许去旁人那里胡言乱语。''

意欢也不知自己是怎么回到春雨舒和的。仿佛魂魄还留在芳碧丛,躯体却无知无觉地游弋回来了。她遣开随侍的宫女,将自己闭锁殿阁内,一张一张翻出多年来抄录的皇帝的御诗。

在皇帝身边多年,便是一直承恩殊遇。意欢并不是善于邀宠的女子,虽然自知美貌,或许皇帝喜爱的也只是她的貌美。可这么多年的日夜相随,他容忍自己的率性直言,容忍着自己的冷傲不群,总以为是有些真心的,为着这些真心,她亦深深爱慕着他,爱慕他的俊朗,他的才华,他的风姿,那万人之上的男子,对自己的深深眷顾,她能回报的,只是在他身后,将他多年所作的诗文------工整抄录,视若珍宝。

却原来啊,不过是活在谎言与欺骗之中,累了自己,也累了孩子。

她痴痴地笑着,在明朗白昼里点起蜡烛,将那叠细心整理了多年,连稍有一笔不整都要全盘重新抄录的诗文一张一张点到烛火上烧了起来。她点燃一张,便扔一张,亦不管是扔到了纱帐上还是桌帷上。

泪水汹涌地滑落,滴在烧起来的纸张上,滋起更盛的火焰。她全不理会火苗灼烧上了宛若春葱芊芊的手指,只望着满殿飞舞的火蝶黑焰,满面晶莹的泪珠,哀婉吟道:``而今才道当时错,心绪凄迷。红泪偷垂,满眼春风百事非。情知此后无来计,强说欢期,一别如斯,落尽犁花月又西。''她痴痴怔怔地笑着,``而今才道当时错\ldots\ldots 都是错!都是错的啊!''

她一遍一遍地吟唱,仿佛吟唱着自己醉梦迷离的人生,一别当欢。

待如懿得知失火的消息匆匆赶到时,春雨舒和的殿阁已经焚烧成一片火海。宫人们拼命呼喊号叫,端着一切可用的器物往里泼着水,然而,火势实在太大,又值盛夏,连水龙亦显得微不足道。

李玉指挥着一众宫人,满头灰汗,急的连连跺脚不已,见了如懿,忍不住呜咽道:``皇后娘娘,这可怎么好?''

如懿急急问道:``人有没有事?舒妃呢?''

李玉哭丧着脸道:``发现起火耳朵时候已经晚了,舒妃娘娘一早把人都赶到了外头,等赶过来救火的时候,里头一点儿声响都没有了。只怕是\ldots\ldots{}''

如懿心下大怆,一个踉跄,勉强扶住容珮的手站稳了道:``救人!快救人!''

李玉跪下道:``皇后娘娘,怕是不成了。火势太大,没人冲的进去。而且这把火,怕就是淑妃娘娘自己烧起来的。她是一心寻死啊!''

有清泪肆意蜿蜒而下,如懿怆然道:``她为什么突然寻死?为什么?''

李玉期期艾艾道:``舒妃自焚前,曾发了疯一样冲进了芳碧丛寻皇上,奴才守在外头,隐隐约约听得什么坐胎药,什么太后指使,旁的也不知了。''

如懿顿时了然,心中彻痛如数九寒冰。

这样烈性的女子,若然知道那碗坐胎药背后的真相,如何肯苟活,再伴随那个男人身旁。

容珮急道:``不管怎么样,还是要救救舒妃啊。娘娘,您说是不是?''

如懿望着漫天大火熊熊吞灭了殿宇,心下如大雨滂沱抽挞,终如死灰般哀寂,凄然转首道:``不必了。''

意欢,这个剔透如玉髓冰魄的女子,便这样将自己化于一片烈火之中,焚心以火,不留自己与旁人半分余地。

这世上,有哪个少女不曾怀着最绮丽的一颗春心?初初入宫时的意欢,绮年玉貌的意欢,独承恩露的意欢,对未来的深宫生涯一定有着无限美好的憧憬。那站在万人中央的拥有万张荣光的九五之尊,会携过她的手,与她一生情长。以为是满城芳菲,却已经春色和烟老,落花委地凉。

如懿怔怔地想着,一步一伤,心里似有千万东西涌了出来,无穷无尽的悲哀芳菲脱缰的野马齐齐撞向胸口,那种疼痛芳菲是从心头游曳而下,直直坠入腹中,像冰冷的小蛇吐着鲜红的芯子,咝咝地琢咬啃啮着。她痛得弯下腰去,死死按住了小腹,混不觉身后逶迤一地,已经有鲜血淋漓蜿蜒。直到容珮的惊呼声骤然响起,她终于在惊痛之中,失去了最后的知觉。

醒来时已是天色将暮,如懿一直在沉沉的昏睡之中,只觉得四体百骸,无一不在疼痛,似乎有无数人在呼唤着她,除了腹中下坠般的绞痛,她使不出半点儿力气。

最后的最后,是新生儿的啼哭,让她渐渐清醒。醒转时海兰已经伴在了身侧,且喜且忧,抱过粉色的襁褓,露出一张通红的小脸,喜极而泣:``皇后娘娘,是一位公主呢。''

乾隆十八年六月二十三,如懿生下了皇五女。这亦是和敬公主之后皇帝膝下唯一一位嫡出的公主。许是皇帝女儿稀少,许是五公主出生半月前皇十子的夭折,皇帝对五公主格外珍视,特早早定了封号``和宜'',取其``万事皆宜''之意,又取了乳名``璟兕''。

``兕''者,小雌犀牛也。皇帝每每与如懿言起,便希望这位年幼娇嫩的女儿如小犀牛一般健康,能抵挡一切不测和疾病。

如懿虽是笑言,却也隐隐觉得不详,只道:``唐太宗钟爱长孙皇后所生的幼女晋阳公主,公主的乳名也叫兕子,只可惜未能养大。''

皇帝摆手,爽朗笑道:``所以,咱们的女儿是璟兕啊。璟乃玉之光彩,既美丽剔透,又强壮健康。''他说罢又抱起璟兕亲了又亲,璟兕似乎很喜欢这样的亲昵的举动,直朝着皇帝笑。

皇帝十分欣悦:``朕有这么多儿女,唯有璟兕,朕抱着她的时候她会笑得那么甜。''

皇帝这样喜悦,浑然忘了春雨舒和大火中自焚而死的意欢,那样刚烈的女子,连一死也不能在皇帝心上划下深深的印迹。

总在生下女儿的欢喜空隙里感到唇亡齿寒的悲凉。因为十阿哥和舒妃的接连去世,所以连着璟兕出生的喜事,如懿也将应赏给一应伺候宫人和接生嬷嬷们的赏银减半赐下。虽然为首的田嬷嬷也赔着笑脸向如懿提起赏银减半之事,如懿亦只道:``十阿哥与舒妃过世,本该赏赐你们的喜事也不能张扬。这次且自委屈你们了,下回再有嫔妃生产,一定一应补足你们。''

田嬷嬷哪里忍得,一时笑道:``舒妃再怎么也不过是妃妾,如何能与皇后娘娘比尊贵。便是她没了,也不能损了娘娘生下小公主的喜庆啊。''

如懿正痛惜舒妃之死,这话听得十分不耐,便沉下脸不语。

如此,田嬷嬷再要抓乖卖巧分辨些什么,但见如懿神色不豫,也只得掩下了眉间悻悻之色,再也无话。

如懿趁着皇帝高兴,婉转提起:``皇上这么疼爱公主,臣妾自然高兴。只是公主出生那一日,便是舒妃离世那一日,还是请皇上看在公主面上,不要责怪舒妃自戕之罪。''

皇帝只顾着怀中小小的人儿,微微皱眉道:``既然皇后求情,朕便罢了。只是这样张狂的女子,焚火烧宫,实在可恶。''

如懿心中一搐,勉强维持着脸上的笑意:``舒妃之死,大概也是过于绝望吧。''

皇帝的笑意冷凝在嘴角,旋即看她一眼,眸光微冷:``皇后此话何意?''

如懿平静的神色在烈烈日光下显得无可挑剔,道:``舒妃痛失爱子,可不是绝望了么?''

皇帝的笑意便有些萧索:``十阿哥,是可惜了。''他低首,见璟兕可爱的笑容,忍不住伸手逗弄,``只是,既然留不住的,那便是没缘分,也不必多想了。''

如懿望着皇帝对璟兕疼爱的笑容,亦是默然。皇帝还欲多陪陪如懿与璟兕,李玉却在外头相请,道诸臣已在御书房等候,商议洪泽湖水患一事。

如懿隐隐约约知道,洪泽湖水大溢,卲伯运河二闸冲决,高邮、宝应诸县都被水淹严重,当下也不敢阻拦,只得殷殷送了皇帝出去。

皇帝离去后,容珮替如懿披了一袭雪絮纱的虹影披风在身,悄然劝道:``皇上正在兴头上,您瞧皇上多疼爱小公主啊,何必这个时候扫兴,提起舒妃小主呢?''

如懿眸子里掠过一点星火,旋即黯然不已:``本宫若不提,后宫之中便无人再敢提。你瞧着舒妃过身之后,皇上何曾提过她一句,只当没这个人罢了。''她的眉心凝住了一丝疑惑,``只是本宫一直疑惑,李玉说舒妃自焚前曾闯入芳碧丛向皇上提起坐胎药之事,这件事本宫也是偶然得知,显然皇上一直不欲人张扬,那么舒妃又如何得知?''

容珮眸光一转,旋即低眉顺目:``奴婢偶然得知,那日舒妃前往芳碧丛之前,曾到十阿哥梓宫前。所说\ldots\ldots{}''她声音压得愈加低,``令妃也去过。''

如懿描得细细的眉毛拧了起来,仿佛蜷曲的螺子,登时警觉:``她去做什么?''

容珮抿了抿唇道:``娘娘也这样想?奴婢总觉得令妃小主阴晴不定,难以把握。许多事或许捉不住她做的,可总有个疑影儿,让人心里不安。''

如懿舒了一口气道:``原来你和本宫想的一样。这样,晚膳后你便去绾春轩瞧瞧,先不要张扬,找了令妃过来。''

容珮忙应着道:``是。奴婢会做得掩密一些。只是娘娘也不必担心什么,如今娘娘儿女双全,皇上又这样对待您,您的中宫之位稳如磐石,要处置谁便是谁罢了。''

案上的鎏金博山炉中,香气细细,淡薄如天上的浮云。许多事,明明恍如就在眼前,确实捉摸不定,难以把握。如懿的笑仿佛是井底舀起来的水波,不够清澈,带着青苔的幽腻和波影晃动的破碎:``容珮,你也觉得皇上待本宫很好?''

容珮笑道:``可不是?皇上来得最多的就是咱们这儿了。''

如懿浅浅笑道:``这样的念头,曾几何时,孝贤皇后转过,嘉贵妃转过,舒妃也转过。可是后来啊,都成了镜花水月。本宫一直想,本宫以为得到的,美好的,是不是只是一梦无痕。或者只是这样,容珮,本宫便是得到了举案齐眉。心中亦是意难平。''

容珮蹙眉,不解道:``意难平?娘娘有什么不平的?''

如懿欲言,想想便也罢了,只是笑:``你不懂,不过,不懂也好。舒妃便是懂得太多,才容不得自己的心在这污浊的尘世里了。''

\hypertarget{ux7b2cux4e8cux5341ux516bux7ae0-ux81eaux4fdd}{%
\chapter{第二十八章
自保}\label{ux7b2cux4e8cux5341ux516bux7ae0-ux81eaux4fdd}}

太阳虽已落山,天色却还延续着虚弱不堪的亮白,只是有半边天空已经有了山雨欲来的暗沉,仿佛墨汁欲化未化,凝成疏散的云条的形状。桌上铺着的锦帷是古翠银线绣的西番莲花纹,发着暗定定的光,看得久了,眼前也有些发晕。

太后的声音低沉而缓慢,是年老的女子特有的质感,像是焚久了的香料,带着古旧的气息:``怎么?跪不住了?''

嬿婉的膝盖早已失去了知觉,只是顺服地低着头:``臣妾不敢。''她偷眼看着窗外,薄薄的夜色如同涨潮的无声江水,迅猛而沉静地吞没了大片天空,将最后仅剩的亮色逼迫成只有西山落日处还剩余一痕极淡的深红,旋即连那最后的微亮亦沉没殆尽,只剩下大雨将至前的沉闷气息逐渐蔓延。

这样压抑的枯寂里,只听得一脉袅袅如风起涟漪般的笛声,自庭院廊下舒展而来。那笛声极为凄婉,仿佛沾染了秋日院中衰败与西风中的采木枯萎的香干,摇曳婉转,扶摇抑扬。

太后斜倚在软榻上,由着福珈半跪在脚边用玉槌有节奏地敲着小腿,取过一枚玉搔头挠了挠,惬意道:``听得出是什么曲子么?''

嬿婉战战兢兢地道:``是《惊梦》。''

太后微微一笑,将玉搔头随手一撂:``听说你在跟南府的乐师学唱《牡丹亭》,耳力倒是见长。''

嬿婉低垂着头,不安道:``臣妾只是闲来无事,打发时间罢了。''

太后了然道:``怎么?不急着见皇帝邀宠,反而闲下心来了?这倒不太像你的性子啊。''

嬿婉面红耳赤,只得道:``是臣妾无能。''

``你会无能?''太后嗤笑一声,坐起身来,肃然道:``你都惊了旁人的梦了,填进了舒妃和十阿哥的命了,你还无能?''

嬿婉惊了一身冷汗,立刻扬起身子道:``太后恕罪,臣妾不敢!''

``不敢的事情你不也------做了么?''太后缓和的语气,一一道来,``从舒妃突然闯入芳碧丛问起坐胎药一事,哀家就觉得奇怪。那坐胎药里的古怪,皇上知,太医知,他们却都不知道哀家也知。舒妃一直蒙在鼓里,突然知道了,自然不会是从咱们嘴里说出去的。而你偷偷学着舒妃的坐胎药喝,后来却突然不喝了,自然是知道了其中的古怪。而舒妃去见皇帝之前只在十阿哥的梓宫前见过你。除了你,还会有谁来告诉她真相?''

嬿婉听着太后一一道来,恍如五雷轰顶,瑟瑟不已,只喃喃道:``太后,太后\ldots\ldots{}''

太后冷笑一声,拨着小指上的金錾古云纹米珠图案寿护甲,慢条斯理道:``只是光一碗坐胎药,舒妃到底连十阿哥也生了,哪怕是皇帝做过这些事,也是不能作数的了。她也不至于心智迷糊立刻去寻皇帝。除非啊,这碗坐胎药喝她的丧子之痛有关,她才会禁不住刺激发了狂。所以哀家便疑心了,那碗坐胎药若是真的损失肾器,那也不会到了孕中才致使舒妃脱发肾虚,以致伤了十阿哥,坐下了胎里带出来的病痛,该早早儿出现些症状才是。哀家这样疑心,顺藤摸瓜查了下去,终于查出了一些好东西。''她唤道:``福珈,叫令妃瞧瞧。''

福珈答应着起身,从黄杨木屉子里取出一个小纸包来,放到她跟前,太后道:``令妃,舒妃有孕的时候,你给她吃的东西全在这儿了。哀家不说别的,每一日一包,你自己来哀家宫里吃下去,哀家便什么也不说了。''

嬿婉看着那包东西,想要伸手,却在碰到的一刻如触电般缩回了手,柔弱香肩随着她不可控制的啜泣轻轻颤抖,再不敢打开。

太后的神色阴沉不可捉摸,喝道:``怎么?敢给别人吃的东西,自己便不敢吃了么?吃!''

嬿婉仿佛面对强敌的小兽,吓得站站不能自已,拼命叩首道:``太后恕罪,太后恕罪。臣妾再也不敢了!''

``不敢?''太后神情一松,笑道:``那你自己说吧,到底对舒妃和十阿哥做了什么?''

嬿婉瘫软在地上,泪流满面,声音控制不住似的从喉间发出:``太后明鉴,是臣妾一时糊涂油蒙了心,嫉妒舒妃承恩有孕,在她的饮食中加入会慢慢肾虚脱发的药物。臣妾\ldots\ldots 臣妾\ldots\ldots 只是想她容貌稍稍损毁,不再得皇上盛宠,并非有意毒害十阿哥的。''

``那么,江与彬得皇后嘱咐,赶回来为舒妃医治,却中途因病耽搁,也是你做的手脚了?''

嬿婉惶惶道:``是。是臣妾买通了驿丞给他们下了腹泻的药物,又耽搁延医问药的时候,让他们阻在了半路,不能及时赶回。''

``就算没了江与彬,愉妃是个心细的,她受皇后之托照拂舒妃,你要让她分心无暇顾及,必然是要找五阿哥下手了?''

嬿婉只得承认:``也是臣妾收服了五阿哥的乳母,在五阿哥入睡后悄悄掀开衣被让他受凉,使愉妃忙于照顾亲子,无暇顾及舒妃并不十分明显的抱恙。''

太后长叹一口气:``福珈,你听听,这样好的心思谋算,便是当年的乌拉那拉皇后也不能及啊!哀家在深宫里寂寞了这些年,倒真遇上了一个厉害的人物呢!''

福珈轻声道:``太后不寂寞了。只是满宫的嫔妃皇嗣都要折损了。''她说罢,退到一旁,又点亮了几盏描金蟠枝烛。

天色已然全黑,外头欲雨未雨的闷风吹得檐下宫灯簌簌摇曳,漾出不安的昏黄光影。

太后的目光冰冷如寒锥:``你有多少本事,敢谋害皇嗣?谋害皇帝的宠妃?''

嬿婉一气儿说了出来,倒也镇静了许多,索性坦承道:``太后如此在意舒妃,无非舒妃是太后举荐的才貌双全之人。但皇上归根究底还是在意她叶赫那拉氏的出身,到底不是万全之人。恐怕皇上也觉得是太后举荐的枕边人,还不大放心呢。''她扣了首,仰起娇美而年轻的面庞,``左右舒妃怀孕的时候伤了肾气,容貌毁损,补也补不回来了。如今人也死了,太后何必还介意她这颗废子呢?''

太后冷笑道:``舒妃是废子,那你是什么?''

嬿婉思量着道:``臣妾是害舒妃不错,但舒妃身为太后亲手调教的人,居然禁不住臣妾几句言语,也未免无用!且臣妾是害她,却未曾逼迫她自焚,她这般不爱惜性命,自然是因为对皇上用心太过的缘故,既然她侍奉太后,怎可对皇上过于有心呢?''

太后舒展笑道:``哀家自然知道舒妃对皇帝有心的,为着她有心哀家才肯重用她。因为有心有情,才是真作假时假亦真,才会让人难以辨别,也只有舒妃替哀家说话的嘴怀着的是一颗对皇帝的真心,自然也会让人以为她说的是真心实意的话了。''

嬿婉深吸一口气道:``臣妾也对皇上有心,但臣妾是依附之心,邀宠之心。或者说,臣妾对皇上的真心,恰如皇上对臣妾那么多,一点点,指甲盖似的。而非像舒妃一样愚蠢,付出一颗全部真心,不能自拔。''她的笑容意味深长,``若是自己深陷其中,又如何能对太后全心全意呢?''

长久的静默,烛火一跳一跳,摇曳不定,将殿中暗红的流苏锦帐透成沉闷不可言的绛紫色。待得久了,好似人也成了其中一粒,暗淡而无声。

``哀家留心这么多年,舒妃是棵极好的苗子,只可惜用心太深,反而害了自己的一生!''太后喟然摇首,``可见这宫里,你可以有野心,可以有假意,但绝不能有一丝真心,否则就是害人害己,自寻死路了。''

嬿婉深深伏拜:``太后教诲,臣妾铭记于心。''她仰起脸大着胆子道:``臣妾斗胆,舒妃能为太后效力的,从此之后,臣妾也会为太后效犬马之劳。''

太后微眯了双眼,蓄起一丝锐利的光芒:``你的心思倒打量得好,既要哀家饶恕了你,以后还得哀家保全,还要美名其曰为哀家办事。你这样的心有七窍的伶俐人儿,哀家怕还来不及,哪里还敢用你呢?''

嬿婉俯下身体,让自己看起来像一只无路可去的小兽,虽然狡猾,却无力自保:``太后历经三朝,有什么人没见过,有什么事没经历过,臣妾再伶俐,如何及得上太后分毫呢,生死荣辱也在太后一念之间。若得太后成全,臣妾粉身碎骨,也必当涌泉相报。''

嬿婉十分谦恭,几乎如卑微的尘芥俯首与太后足下。太后正欲言,却见小宫女喜珀进来,请了个安道:``太后,令妃小主宫里的人来请,说皇后娘娘打发了容珮姑姑在寻令妃小主呢,看样子像是有点儿着急。''

嬿婉身子一颤,畏惧地缩紧了身子,睁着惊慌无助的眸,膝行到太后跟前,抱着她的双膝道:``太后,太后,皇后不会发现了什么吧?''

``以皇后的聪慧,倒也难说!''太后俯视着她,笑意清冷而透彻,如雪上月光清寒,``怎么?自己做过的事,这便怕了?''

嬿婉谦恭地将自己的身体俯到太后的足边,几乎将额头磕在她雪青色掐金满绣竹蝶纹落珠软底鞋的鞋尖:``太后,臣妾求您庇佑,求您庇佑!往后臣妾一定唯太后之命是从,甘受太后驱使,以报太后今日之恩。''

片刻的沉吟,静寂得能听见窗外风声悠悠穿过廊下的声音。太后抚着护甲,漫不经心道:``好了。哀家既然受了你的心意,自然会庇佑你。皇后能疑心的,不过就是和哀家一样,知道舒妃死前在十阿哥的梓宫前见过你。你便记得告诉皇后,是哀家知道了你在十阿哥死后学唱昆曲犯了忌讳,所以责罚了你,要你去十阿哥梓宫前思过,你才会遇上了舒妃的。''

嬿婉的眼底迸发出闪亮的喜色,心悦诚服地再度拜倒:``臣妾歇过太后。''

天后微微颔首:``那你赶紧去吧。记得,皇后如今正当盛宠,她又是个严性子,你越谦卑越自责便好。没有十足的证据,她也不能把你怎样。''

嬿婉答应着,忙恭恭敬敬整衣而去。

福珈看着她离开,捡起地上的纸包,笑吟吟道:``太后准备的是什么?把令妃吓得什么话都说了。''

太后失笑,拿护甲尖点着那纸包拨弄:``你不信哀家备下了令妃害舒妃的毒药?''

福珈低眉顺目道:``这件事当时去查或许还有蛛丝马迹,如今隔了那么久,哪里还有痕迹可循呢?''她莞尔一笑,``别是太后吓唬令妃的吧?''

太后嗤地一笑:``那你自己喝了吧,也就是寻常的一副泻药,她要真吃了一时腹痛如绞,痛得怕了,也会自己说出来。左右哀家就是试她一试罢了,果然还是年轻,经不得吓。''

``如今是还年轻,但这样的心机深沉,滴水不漏,若再长些年纪,心术只会更坏。''福珈有些鄙薄,亦有些担心,``这样公=工于心计手段狠辣的人,太后真要用她?''

太后沉吟片刻,才下定决心般颔首道:``自然了。要用就得用这样狡猾如狐的人,要只单纯可爱的白兔来做什么?养着好玩儿么?之前哀家所用的舒妃、玫嫔和庆嫔,玫嫔嫉妒,窝里乱起来,害得庆嫔不能生育,也害了自己。舒妃是美艳绝伦,又有才学,但凡是看不破,身陷情字不能自拔,一把火把自己烧死了。这样的人,还不是一个个落了旁人的算计而不自知。所以令妃是个可以用的人。''

福珈沉吟道:``可以令妃刚侍奉皇上的时候倒好得宠,如今却不如从前了。''

太后浑然不以为意,只道:``令妃恩宠淡薄,才知道要来求助于哀家。否则她从不从哀家身上有所求,自然也不有所依附了。哀家看她家世寒微,出身又低,却有万分好强之心。如今她在宫里处境如此尴尬,哀家拉她一把,她自然知道哀家的好处,也落了把柄在哀家手里,以后只能乖乖顺服听话。''

福珈心悦诚服:``太后心胸有万全之策,奴婢远远不及。不过以奴婢愚见,要令妃娘娘得宠只怕也不难,她这张脸,可是与皇后有几分相似的,又比皇后年轻。''

福珈低首道:``那么舒妃小主的身后事\ldots\ldots{}''

太后闲闲地拔着纽子上坠下的玛瑙松石塔坠儿,断然道:``诚如令妃所言,舒妃早已是一颗废子。人都死了,公道于她也无关紧要了,不必理会也罢。左右皇帝是要脸面的人,慧贤皇后和孝贤皇后身前有差错,慎嫔更是不堪,皇帝对外到底不肯声张,给她们留了颜面的。舒妃顶多是惹了皇帝嫌恶,外面的丧仪总是要过过面子的。''

福珈脸上闪过一丝怜悯,依旧恭顺道:``是。''

太后缓了一口气,伸手拔下发髻后的银簪子挑了挑烧得乌黑蜷曲的烛芯,有些郁然道:``福珈,你是不是觉得哀家太过狠心了?''

福珈面色柔婉,一如她身上的浅绛色暗花缎如意坎肩底下的牙色长袍,温和得没有半点属于自己的光彩:``太后的心胸和眼界,奴婢如何敢揣测。''

太后以手支颐,脂粉均和的面庞下有细细如鱼尾纹的衰老蔓延耳上,她的无奈与苍老一般无可回避,哀然道:``哀家能有什么心胸和眼界?所有的心胸和眼界,都大不过皇帝的意思去。哀家的端淑和柔淑\ldots\ldots{}''太后沉静片刻,声音微微哽咽,``不能再有这样的事了。哀家费尽心思,只不过想保护自己两个女儿的周全,却也是不能。端淑像颗棋子似的被摆布一生\ldots\ldots 若再发生些什么\ldots\ldots 哀家实在是不敢想。若是皇帝身边没个咱们的自己的人,若真有点什么动静,咱们就真的是蒙在鼓里,一点儿办法一点儿主意都没有了。''

福珈的声音如温暖厚实的棉絮:``太后别担心。''

太后紧紧攥住福珈的手,像是寻找支撑住自己力气的似的:``哀家也不想怎么样,只是想皇帝身边能有一双自己的耳朵,知道皇帝想什么做什么,别在牵扯了哀家的女儿就好。''她伏在福珈的手臂上,虚弱地喃喃道:``别怪哀家狠心,哀家也没有办法。''

太后低低地啜泣着,素日的刚强褪尽,她也不过是一个母亲,一个无能为力的母亲而已。

福珈伸过手,安抚似的搭着太后的肩,眸中微含着泪光,沉静道:``太后,不会了,再不会了。''

意欢惨烈的自焚,对外亦不过是道她忆子成狂心智损伤,才会不慎之下焚火烧了自己的殿宇,困死在其中。为此,意欢啊阿玛兵部左侍郎永绶尚且来不及为爱女的早亡抹一把伤心泪,先战战兢兢请罪,自承教女无方,失火焚殿之罪。

容珮闻知了,鄙夷不已:``是亲生的女儿要紧还是圆明园的一座偏殿要紧?永绶也太不知好歹了!''

如懿看着摇篮中沉沉睡着的幼女,叹息道:``永绶便是知道好歹轻重,才会先行请罪,女儿和外孙都不在了,总还有别的亲眷在。他这样做,是以免皇上责怪牵连了家人。''

容珮摇头感慨道:``真是可怜!''

如懿披着一件雪色底的浅碧云纹披风,身上是一色的碧湖青色罗衣,衣襟四周刺绣锦纹也是略深一些绿色藤萝缠花样,如泛漪微绿。头上用青玉东珠扁方挽了个松松的发髻,其间缀着几点零星的翡翠珠花。唯一夺目些的,是一对攒珠笄垂落到耳侧的长长珠玉璎珞,和百褶垂花如意裙上绣着的一双金鹧鸪,依偎在密织银线浅红海棠花枝上,嘀呖婉转。

这样清淡的打扮,似一株吐露昙花,虽然不似皇后的尊荣华贵,但也合她刚出月的样子。

如懿俯下身,盯着年幼的女儿熟睡中安详的笑容,别过头道``是可怜!生在这儿是可怜,一个个被送进这里更可怜。皇上没有追封舒妃,只是按着妃位下葬,可知心里是极忌讳焚宫的事的,若传出去,岂不坏了皇上最在意的圣明名声。''

容珮急道:``十阿哥和舒妃都死了,难不成皇上还要追究?''

窗外花盛似海,如锦如绣,端的是一派盛世华景。如懿淡然道:``追究才是真坏了名声,皇上一定会安抚永绶几句,把这事儿含糊过去的。''

容珮松了一口气,手里轻摇着一叶半透明的芙蓉团扇,替如懿驱赶着午后酷热的暑意。殿中风轮轻轻,送来玉簪花甜甜的气息,混合着黄底寿字如意纹大瓮中供着的硕大冰块,殿中颇有几分蕴静的凉意。

庭院中有幼蝉微弱的鸣叫声,一丝递着一丝,把声线拉又细又长,听得人昏昏欲睡。如懿闭目正欲谁去,忽然听得容珮轻声问道:``娘娘方才说人一个个送进来,是指\ldots\ldots{}''

如懿嗤地一笑,睁开眼眸道:``本宫才出了月子,不能伺候皇上,舒妃骤然离世,眼下嘉贵妃虽然得宠,但到底也是年轻了。皇上跟前不能没有人伺候,可不是如今有了合适的人了?''

容珮扇着扇子,道:``皇后娘娘是说戴湄若?''

如懿轻轻瞟她一眼:``封疆大吏,正二品闽浙总督那苏图的女儿,镶黄旗人。可算是出身尊贵了吧?''

容珮掰着指头道:``满朝也不过只设了八个总督。直隶、两江、陕甘、闽浙、湖广、两广、四川、云贵。''她咋舌,``再加上镶黄旗的出身,乖乖,可了不得了。这一来,进宫怕是封个贵人也不够了吧?''

如懿拨着耳垂上翠玉片海棠叶耳坠:``贵人可不委屈了。封嫔封妃,至少是一宫之主。''她听得摇篮中的璟兕在睡梦嘤嘤不安地哭了两声,忙俯身抱起哄了半响,才道,``你可知那苏图是什么来历?他的伯父白海青出使准噶尔时坚贞不屈,极力护得大清的颜面,自此加太子太保赠一品大臣。白海青的长子来文任镇江将军,次子佛伦任领侍卫内大臣,三子戴鹤由副都统征准噶尔,前番阵亡,皇上便赠云骑尉祀昭忠祠。其家可见显赫。''

容珮迟疑道:``事关准噶尔?皇上不是许嫁了端淑长公主以和为贵么?怎么对准噶尔征战不屈的也加赏了?''

``宽严并济,本乃为君之道。皇上岂会落人口实,以为只凭一个公主求得安宁。战许功,和是为了百姓,这才是皇上的君威所在啊。''

容珮托腮凝神道:``这戴氏会什么样的妙人儿呢?总不会丑若无盐吧?那便好玩儿了。''

如懿轻轻排着怀中的女儿,嗤笑道:``便是无盐,皇上也不会冷落。何况以皇上的眼力。怎会要一个无盐的女入宫?左右七月二十日戴氏入宫,便能见到了。''

\hypertarget{ux7b2cux4e8cux5341ux4e5dux7ae0-ux8fdbux9000}{%
\chapter{第二十九章
进退}\label{ux7b2cux4e8cux5341ux4e5dux7ae0-ux8fdbux9000}}

容珮正要说话,却见云枝捧了银盅药盏进来,道:``皇后娘娘,您的汤药好了。''

容珮伸手接过,试了试温度道:``正好热热儿,皇后娘娘可以喝了。这汤药是江太医特意拟的方子,以当归、川芎、桃仁、干姜、甘草灸和黄酒入药,特意加了肉桂,化瘀生新,温经止痛的。娘娘喝了吧。''

如懿伸手接过仰头喝了:``本宫记得这样的药是产后七日内服用的,怎么如今又用上了,还添了一味肉桂?''

容珮不假思索道:``江太医亲拟的方子,必然是好的。前些日子嬷嬷小腹冷痛,想是淤血不下,所以江太医又叮嘱了用这汤药。''她所有所思,不禁有些艳羡,``江太医为人忠心,对蕊心姑姑又这般好,蕊心姑姑好福气。''

如懿偏过头看着她笑叹道:``蕊心半生辛苦,若不是为了本宫,早该嫁与江与彬,不必落得半身残疾。所幸,将与并是个好夫君。这样的福气,便不说你,本宫也难盼得。''

容珮忙看了看四周,见四遭无人,方低声道:``这样的话,默默再说不得。''

容珮跪下道:``娘娘是皇后,又儿女双全,这样的事永远落不到皇后娘娘身上。''

如懿微微出神,看着窗下一蓬石榴开得如火如荼,那灼烈的红色,在红墙围起的圈禁之中,倒映这天光幽蓝,几乎要燃烧起来一般。她缓缓道:``这样的话,当年也有人对孝贤皇后说过,后来还不是红颜枯骨,百计不能免除么。''她见容珮还要劝,勉强笑道:``瞧本宫,好端端地说这个做什么?倒是你,是该给你留心,好好儿寻一个好人家嫁了。''

容珮慌忙磕了个头,正色道:``奴婢不嫁,奴婢要终身追随皇后娘娘。这宫里在哪里都要受人欺负,出了宫又有什么好的,万一嫁的男人只是看中奴婢伺候过娘娘的身份,那下半辈子有什么趣儿。奴婢就只跟着娘娘,一世陪着娘娘。''

如懿心下感动,挽住她的手道:``好容珮,亏得你的性子能在本宫身边辅助。也罢,若有了可心的人,你在高速本宫,本宫替你做主吧。''

二人正说着话,外头三宝便清嗓子道:``皇后娘娘,愉妃小主过来请安了。''

如懿忙道:``快请进来。''

外头湘妃竹帘打起,一个纤瘦的身影盈盈一动,已然进来,福了福身道:``臣妾给皇后娘娘请安,皇后娘娘福寿安康。''

因着天气炎热,海兰只穿了一件藕荷色暗绣玉兰纱氅衣,底下是月色水纹绫波裥裙,连陪着的雪白领子,亦是颜色淡淡的点点暗金桂花纹样。恰如他的装扮一般,脂粉均淡,最寻常的宫样发髻亦不过星星点点烧蓝银翠珠花点缀,并斜簪一枚小巧的银丝曲簪而已。

如懿挽了她的手起来,亲热道:``外头怪热的,怎么这个时候过来?容珮,快去取一盏凉好的冰碗来。''她说罢,将手里的绢子递给她,``走得满头是汗,快擦一擦吧。''

海兰伸手接过,略拭了拭汗,抿嘴一笑:``哪里这么热了,娘娘这儿安静凉快得很,臣妾坐下便舒畅多了。''

如懿打量着她的装束,未免有些嗔怪道:``好歹也是妃位,又是阿哥的生母,怎么打扮得越发清简了。''

海兰接过容珮递上来的冰碗,轻轻啜了一口,浅浅笑得温暖:``左右臣妾也不必在皇上跟前伺候,偶尔被皇上叫去问问永琪的起居,也不过略说说话就回来了,着实不必打扮。''

如懿微微沉吟,想起海兰平生,虽然位居妃位,但君王的恩宠却早早就断了绝,实在也是可怜,便道:``话虽这样说\ldots\ldots{}''

海兰却不以为意,只是含了一抹深浅得宜的笑:``话虽这样说,只要皇上如今心里眼里有永琪,臣妾也便心安了。''

如懿握一握她的手道:``你放心,求仁得仁。对了,这个时辰,永琪在午睡吧?''

海兰白净的面上露出一丝喜色,却又担忧:``永琪性子好强,哪肯歇一歇。皇上前几日偶然提了一句圣祖康熙爷精通天文历算,他便在苦学呢。臣妾怕他热坏了身子,要他休息片刻,他也不肯,只喝了点绿豆百合汤便忙着读书了。''

如懿颔首道:``永琪争气是好事,也让咱们两个做额娘的欣慰。只是用功虽好,也要顾着点儿自己的身子。''

海兰轻轻搅着冰碗里的蜜瓜,银勺触及碗中的碎冰,声音清冽而细碎。她笑嗔道:``娘娘说得是。只是皇上如今更器重嘉贵妃的四阿哥永珹,每隔三日就要召唤到身边问功课的,永琪不过五六日才被叫去一次。臣妾也叮嘱了永琪,虽然用功,但不可露了痕迹,太过点眼。皇后娘娘是知道嘉贵妃的性子的,一向目下无人,如今她的儿子得意,更容不下旁人了。''

如懿听得十分入心,便道:``你的心思和本宫一样。来日方长,咱们不争这一时的长短,且由她得意吧。''

海兰抚摸着手上一颗蜜蜡戒指,颇为犹疑:``这些日子臣妾的耳朵里刮过几阵风,不知可也刮到娘娘耳朵里了?''

如懿取了一枚青杏放在口中,酸得微微闭上了眼镜,道:``每日刮的风多了,你且说说,是哪一阵风让你也留心了。''

海兰欲言又止,然而,还是耐不住,看着摇篮中熟睡的小公主,爱怜地抚摸上她苹果般红润的脸庞,道:``皇后娘娘生下了玉雪可爱的公主,有子有女,便是一个好字,可是落在旁人眼里,却未必见得是好。''

如懿爽然一笑,示意她吃一粒缠丝玛瑙盘中的杏子:``你且尝尝这个,酸酸的很生津止渴额。''她理了理衣襟上鎏金光素圆扣垂下的金金细丝流苏,笑道:``本宫觉得好的,旁人未必觉得是好。在宫里,生个公主算得什么,只有皇子才是依靠。纯贵妃生了两个皇子之后才得以为四公主,皇上虽然喜爱,可纯贵妃自己却不过可可。嘉贵妃更是,每每许愿,只求得子,勿要生女。无非就是因为皇子才是地位荣宠的依靠,而公主却是可有可无的。是么?''

海兰微微颔首,牵动发髻边的银线流苏脉脉晃出一点儿薄薄的微亮:``臣妾只有永琪一个儿子,娘娘亦只有十二阿哥。想当年,孝贤皇后在世,有富察氏的身家深厚,也盼望多多得子。可见皇子多些,地位是可安稳不少。''她盈盈一笑,略略提起精神,``幸好皇后娘娘恩眷正盛,只怕很快就会又有一位皇子了。''

如懿掩唇一笑清妍幽幽:``承你吉言,若真这样生下去,可成什么了?''她拍一拍海兰的手,``但本宫知道,宫中也唯有你,才会这样真心祝愿本宫。''

海兰的眼角闪过一丝凄楚:``若是舒妃还在,一定也会这样真心祝福娘娘。只可惜君情淡薄,可惜了她绮年玉貌了。''她微带了一丝哽咽,``只是也怪舒妃太看不穿了,宫中何来夫妻真心,她看得太重,所以连自己也赔了进去。''她说罢,只是摇头叹息。

如懿神色黯然如秋风黄叶,缓缓坠落:``很早之前,你便有这样的言语提醒本宫。所以本宫万幸,比舒妃多明白一些。''

海兰默默片刻,眼中有清明的懂得:``皇后娘娘久在宫中,看过的也比一叶障目的舒妃多得多。臣妾只求\ldots\ldots{}''

如懿未及她说完,低低道:``你要说的本宫明白。求不得情,便求一条命在,一世安稳。''

海兰露出了然的笑意,与如懿双手交握:``皇后娘娘有嫡子十二阿哥,永琪来日一定会好好儿辅佐十二阿哥,咱们会一世都安安稳稳的。''她轻声道,``这个心愿这样小,臣妾每每礼佛参拜,都许这个愿望。佛祖听见,一定会成全的。''

如懿婉然笑道:``是,一定会成全的。''

圆明园虽然比宫中清凉,但京中的天气向来是秋冬极寒、夏日苦热,如懿午睡醒来,哄了哄璟兕,又陪着永璂玩耍了一会儿,便携了容珮往芳碧丛去。

七月正是京中最为酷热之时,皇帝心性最不耐热,按着以往的规矩,便要去承德的避暑山庄,正好也可行木兰秋狩。这几日不知为何事耽搁了,一直滞留在书房中,夜夜也招幸嫔妃。如懿心中疑惑,也少不得去看看。

如懿才下了辇轿,却见金玉妍携了四阿哥永珹喜滋滋从芳碧丛正殿出来,母子俩俱是一脸欢喜自傲。如懿坐在辇轿中,本已闷热难当,骤然看了玉妍得意扬扬的样子,心中愈加不悦。倒是李玉乖觉,忙扶了如懿的手低声道:``皇后娘娘,这几日皇上不招幸嫔妃,嘉贵妃便借口暑热难行,怕四阿哥中暑,每每都陪着四阿哥来见皇上。''

如懿轻轻一嗤:``她倒聪明!总能想着法子见皇上!''

李玉恭敬道:``那是因为嘉贵妃比不得皇后娘娘,可以任何时候都能见到皇上。身份不同,自然行事也不同了。''

如懿一笑置之,举目望见玉妍的容颜,虽然年过四十,却丝毫不见美人迟暮之色。她纵使不喜玉妍,亦不得不感叹,此女艳妆的面庞丝毫无可挑剔,恍若是初入潜邸的年岁,风华如攀上枝头盛开的凌霄花,明艳不可方物。仿佛连岁月也对她格外厚待,不曾让她失去最美好的容色。

如懿不觉感慨:``难怪皇上这些年都宠爱她,也不是没有道理。''

容珮低笑道:``嘉贵妃最擅养颜,听闻她平时总以红参煮了汤汁沐浴浸泡,又以此物洗面浸手,才会肤白胜雪,容颜长驻。左不过她娘家李朝最盛产这个,难不成娘娘还以为她最喜食家乡泡菜,才会如此曼妙?''

如懿笑道:``当真有此奇效,也是她有耐心了。''

如懿扶了容珮的手缓缓步上台阶。殿前皆是金砖曼地,乌沉沉的如上好的墨玉,被日头一晒,反起一片白茫茫的刺眼,越加觉得烦热难当。

玉妍见是如懿,便牵着永珹的手施礼相见。如懿倒也客气:``天气这么热,永城还来皇上跟前伴驾,可见皇上对永珹的器重。''

玉妍着一身锦茜色八团喜春逢如意襟展衣,裙裾上更是遍刺金枝纹样,头上亦是金宝红翠,摇曳生辉。在艳阳之下,格外刺眼夺目,更显得花枝招展,一团华贵喜气。玉妍见儿子得脸,亦不觉露了几分得意之色,道:``皇后娘娘说得是。皇上说永珹长大了,前头大阿哥和二阿哥不在了,三阿哥又庸碌,许多事只跟永珹商量。只要能为皇上分忧,这天气哪怕是要晒化了咱们母子,也是要来的。''

如懿听得这些话不入耳,当下也不计较,左右人多耳杂,自然有人会把这样的话传去给永璋的生母纯贵妃绿筠听。她只是见永珹长成了英气勃勃的少年,眉眼间却是和他母亲一般的得意,便含笑道:``永珹,皇阿玛如此器重你,你可要格外用心,有什么不懂的,多问问师傅,也可指点你一二。''

永珹少年心性,也不加掩饰,便道:``回皇额娘的话,皇阿玛问儿子的,书房的师傅也指点不了。''

如懿奇道:``哦?本宫也听闻皇上这些天忙于政事,和群臣商议,原来也告诉你了。果然,咱们这些妇道人家,都是耳聋目盲,什么都不知道的。''

少年郎的眼中闪耀着明亮的欢喜:``是。皇阿玛这些日子都在为南河侵亏案烦恼。''

如懿略有耳闻,便道:``京中酷热,但南方淫雨连绵。听闻洪泽湖水位暴涨,漫过坝口,邵伯运河二闸冲决,淹了高邮、宝应诸县。''

永珹一一道来:``皇阿玛如今已经命刑部尚书刘统勋、兵部尚书舒赫德及署河臣策楞赶赴水患工次督工赈灾,查办此事。还拨了江西、湖北米粮各十万石赈江南灾,至于拨米粮之事,都已交给儿臣跟着查办了,也让五弟跟着儿子一起学着。''

他说到末了一句,唇边已颇有趾高气扬之色,仿佛永琪亦不过是他小小的随从。玉妍看着儿子,一脸的喜不自禁,拿了绢子替他擦汗,口中似是嗔怪,唇边却笑意深深:``好了。你皇阿玛交代你去做,你好好儿做便是了,也别忘了提携提携你五弟。听说这河运上的事是高斌管照的,亏他还是慧贤皇贵妃的阿玛呢,原该做事做老成了的,却也这样无用!''

如懿的笑容淡了下来,盯着永珹道:``都是自家兄弟,有什么提携不提携的话。兄友弟恭,皇上自然会欢喜的。''

永珹被她盯得有些不自在,只得垂首答了``是。''

玉妍正在兴头上,哪里听得讲这样的话,却也不便发作,便抚着永珹的肩膀道:``永珹,额娘平生最得意有三件事。一是以李朝宗室王女的身份许嫁上国;二是得幸嫁与你皇阿玛,恩爱多年;三便是生了你们兄弟几个,个个是儿子。''她妩媚的眼波流盼生辉,似笑非笑地嗔了如懿一眼,只看着永珹道,``有时候啊,额娘也想生个女儿,可是细想想,女儿有什么用啊,文不能建基业,武不能上战场,一个不好,便和端淑长公主似的嫁了好远不能回身边,还要喝蛮子们厮混,真是\ldots\ldots{}''她细白滑腻的手指扬了扬手中的洒金水红娟子,像一只招摇飞展的蝴蝶,微微欠了身子娇滴滴道:``哎呀!皇后娘娘,臣妾失言,可不是说皇后娘娘生了公主有什么不好。儿女双全,又是在这个年岁上得的一对儿金童玉女,真真是难得的福气呢。''

容珮听她说得不堪,皱了皱眉便要说话,如懿暗暗按住她的手,淡淡笑道:``岁月不饶人,想来嘉贵妃虚长本宫几岁,一定更有感触呢。''她转而笑得恬淡从容,``出身李朝就是这般好,听闻李朝盛产红参,每年奉与嘉贵妃许多,听闻嘉贵妃常用红参水沐浴洗漱,所以才得这般容颜光滑,可见李朝的妙人妙物真是不少呢。''

玉妍越发得意,笑吟吟道:``其实这些好有什么呢,只要臣妾的几位阿哥争气,有什么好儿是将来没有的呢。''

如懿暗暗失笑,面上却不露分毫:``可不是?只是嘉贵妃和李朝的娘家也未免小气了一些,这么好的红参藏着掖着不给宫里的姐妹用也罢了,怎么连太后也不奉与呢?为媳为妾之道,难道李朝都没有教与嘉贵妃么?''

玉妍蹙了蹙描得秀长的柳叶眉,有些不服气道:``不仅臣妾,李朝每年进奉太后的红参也不少呢。''

容珮轻轻``咦''了一声,恭恭敬敬道:``嘉贵妃小主对太后一片孝心,李朝也恭谨有加。只是这孝心对着太后,还是嘉贵妃小主自己的私心重了点儿啊,否则怎么奉与太后的红参还不够太后沐浴保养呢。啧啧\ldots\ldots 真是\ldots\ldots{}''

玉妍面上一阵红一阵白,正欲辩白,如懿温然笑着,含了不容置疑的口吻道:``容珮,当然不是嘉贵妃和李朝小气,是太后节俭,不喜奢靡罢了。佛家曰人生在世不过一皮囊而已。爱憎嗔痴喜怒哀乐都须节制,更不必为贪嗔喜恶怒着迷陷入其中。''她垂眸望着永珹:``永珹,你皇阿玛喜欢你器重你,把你作为储位皇子的表率,你更不宜轻言喜怒,露了轻狂神色,叫奴才们笑话。''

永珹听如懿郑重教诲,也即刻收了得意之色,垂首答允。

容珮撇了一抹笑道:``四阿哥有什么不知道,尽管请教皇后娘娘,娘娘是您的嫡母,与皇上体通一心,比不得那些下九流上不得台面的,生生教坏了您,让您失了皇上的喜欢。''

玉妍面色铁青,如被眼霜,却也实在挑不出什么,只得拽了永珹的手,施礼退开。

如懿看了看玉妍的神色,不觉低声笑道:``容珮,你的嘴也太坏了。''

容珮有些讪讪,却也直言:``奴婢对着心坏的人嘴才坏。娘娘何曾看奴婢对愉妃小主和舒妃小主她们这么说过话么?''

如懿笑着戳了戳她的面颊,便进殿去了。

芳碧丛书房里极安静。为着皇帝这几日繁忙喜静,连廊下素日挂着的各色鸟笼都摘走了,只怕哪一声嘀咕莺啭吵着了皇帝,惹来弥天大祸,殿中虽供着风轮,仍有两对小宫女站在皇帝身后举着芭蕉翠明扇交相鼓风,却不敢有一点儿呼吸声重了,怕吵着皇帝。

如懿见皇帝只是伏案疾书,便示意跟着的菱枝放下手中的食盒,和容珮一起退下去。如懿行礼如仪,皇帝扶了她一把,道:``天气热,皇后刚出月子,一路过来,仔细中暑。''

如懿听他声音闷闷的,想是为国事烦忧,也不敢多言,便静静守在一旁,替皇帝研墨。皇帝很快在奏折上写了几笔,揉了揉额角,转首见小太监伺候在侧,便扬了扬脸示意他们下去,方道:``你来得正好,朕忙了一日,正想和你说说话。''

如懿笑道:``臣妾还怕吵着皇上,惹皇上烦恼呢。''

皇帝扬了扬嘴角算是笑:``怎会?朕只要一想到咱们的璟兕,心里欢喜,怎么会烦恼呢?''

如懿停下手中的墨,替皇帝斟上茶水,道:``皇上喝几杯茶润润喉吧。''

皇帝饮了口茶,如话家常:``朕偶尔听见后宫几句闲话,说舒妃任性纵火焚宫,是因为与皇后亲近,一向得皇后纵容的缘故?''

如懿见皇帝似是开着一个不经意的玩笑,并无多少认真的神色,可是背后不禁一凉,仿佛风轮吹着冰雕的寒意透过澹澹衣衫,直坠入四肢百骸。皇帝近日并不曾招幸嫔妃,既是因为意欢自焚难免郁郁,另则又忙于政事,若说听到后宫的闲话,无非只是见过金玉妍而已。如懿心中暗恨,不觉咬紧了贝齿,更不敢将皇帝的话当做玩笑来听,即可屈身跪下道:``皇上这样的话,虽是玩笑一句,可臣妾实不敢听。不知后宫有谁这样不把皇上天威放在眼中,敢这样肆意胡言,真是臣妾管教后宫不严之过。''

皇帝笑容微敛,眼底多了几分漆黑的凝重:``哦?这话怎么是不把朕的天威放在眼中了?''

如懿垂首谨慎道:``舒妃宫中失火,后宫上下皆知是她思念十阿哥,伤心过甚,才会一时烛火不慎惹起大火,也折损了自己。谁又敢胡言舒妃自焚?妃嫔自裁本是大罪,何况是烧宫且活生生烧死了自己?这样胡嚼舌根的话传出去,旁人还当皇上的后宫是个什么逼死人的地方呢。''如懿说到此处,不免抬头看了眼皇帝,见他只是以沉默相对,眼中却多了几分薄而透的凛冽,仿佛细碎的冰屑,微微扎着肌肤。她垂下眼睑,一脸自责,``何况臣妾虽喜爱舒妃,但也是因为她侍奉皇上多年,心中唯有皇上一人,又诞育了十阿哥。平时虽然不与宫中姐妹多亲热,但也是个知道分寸、言行不得罪人的。若论臣妾与舒妃亲近,哪比得上舒妃多年来得皇上宠爱关怀,所以皇上听来的这些话,明里指着臣妾纵容舒妃,岂不知是暗指皇上宠爱舒妃才骄纵出焚宫的祸事。这样的大不敬冒犯皇上的话,臣妾如何敢入耳呢?''

皇帝静了片刻,似是在审视如懿,但见她神色坦荡,并无半分矫饰之意,眼中是寒冰亦化作了三月的绿水宁和,伸手笑着扶起如懿道:``皇后的话入情入理。朕不过也是一句听来的闲话而已。''

御座旁边放置了黄底万寿海水纹大氅,上头供着雕刻成玲珑亭台楼阁的冰雕,因着放得久了,那冰雕慢慢融化,再美的雕刻也渐渐成了面目全非,只听得水滴声缓缓一落,一落,如敲打在心间。

如懿屈膝久了,膝盖似被虫蚁咬啮着,一阵阵酸痛发痒,顺势扶着皇帝的手臂站起身来,盈盈一笑,转而正色道:``皇上说得是。只是皇上可以把这样的话当玩笑当闲话,臣妾却不敢。舒妃虽死,到底是后宫姐妹一场。她尸骨未寒,又有皇上和臣妾为平息奴才们的胡乱揣测,反复言说舒妃宫中失火只是意外,为何还有这样昏聩的话说出来。臣妾细细想来,不觉心惊,能说出这样糊涂话来的,不仅没把一同伺候皇上的情分算进去,更是把臣妾与皇上的嘱咐当作耳旁风了。''她抬眼看着皇帝的神色,旋即如常道:``自然了。臣妾想,这样没心智的话,能说出来也只能是底下伺候的糊涂奴才罢了,必不会是嫔妃宫眷。待臣妾回去,一定命人严查,看谁的舌头这么不安分,臣妾必定狠狠惩治。''

如懿素来神色清冷,即便一笑亦有几分月淡霜浓的意味。此刻窗外蓬勃的艳阳透过明媚的花树妍影,无遮无拦照进来,映在她微微苍白的脸上,越显得她肤色如霜华澹澹。

皇帝的脸色微微一沉,很快笑着欣慰地拍拍如懿的手,神色和悦如九月金澄澄的暖阳:``有皇后在,朕自然放心。''

如懿莞尔一笑,似是鱼皇帝亲密无间,但唯有她自己知道,方才皇帝必定是听信了金玉妍的言语来试探与她,却是如何让她汗湿了重衣,仿佛芒刺在背。当真是一步也轻易不得。然而,她亦不能不心惊,永珹日渐得皇帝器重,他毕竟在诸位皇子中年纪颇长,永璂年幼尚不知事,永琪出身不如永珹,暂时只得韬光养晦。母凭子贵,金玉妍的一言一行在皇帝心中分量日重,如懿自己便是由着贵妃、皇贵妃之位一步步登上后位的,如何能不介意。想到此节,如懿暗暗攥紧了手中的绢子,那绢子上的金丝八宝缨子细细地磨着掌心,被冷汗洇湿了,痒痒地发刺。她只得愈加用力攥住了,才能屛住脸上气定神闲的温柔乡笑意。

殿中关闭得久了,有些微微地气闷。如懿伸手推开后窗,但见午后的阳光安静地铺满朱红碧翠宫苑的每一个角落,一树一树红白紫薇簌簌当风开得正盛,衬着日色浓淡相宜。日光洒过窗外宫殿飞翘的棱角投下影来,在室中缓缓移动,风姿绰绰,好似涟漪轻漾,恍然生出了一种无言相对的忧郁和惆怅。偶尔有凉风徐徐贯入,拂来殿中一脉清透。隔着远远的山水泼墨透纱屏风,吹动帏帘下素银镂花香球微击有声,像是夜半雨霖铃。满室都是这样空茫的风声与雨声,倒不像是在酷热的日子里了。

如懿从泥金花瓣匣里取了几片新鲜刮辣的薄荷叶放进青铜顶球麒麟香炉里,那浓郁至甜腻的百合香亦多了几分清醒的气息。她做完这一切,方从带来的红竹食盒里取出一碗莲子百合红豆羹来,柔婉笑道:``一早冰着的甜羹,怕太冰了伤胃。此刻凉凉的,正好喝呢。''

皇帝瞧了一眼,不觉笑着刮了刮如懿的脸颊道:``红豆生南国,最是相思物。皇后有心。''

如懿轻巧侧首一避,笑道:``百年和好,莲子通心,皇上怎的只看见红豆了?''

皇帝舀了一口,闭目品位道:``是用莲花上的露水熬的羹汤,有清甜的气味。一碗甜羹,皇后也用心至此么?''

如懿的笑如同一位痴痴望着夫君额妻子,温婉而满足:``臣妾再用心也不过这些小巧而已,不必永珹和永琪能干,能为皇上分忧。''

皇帝道:``来时碰到永珹与嘉贵妃了?''

如懿替皇帝揉着肩膀,缓声道:``嘉贵妃教子有方,不只永珹,以后永璇和永瑆也能学着哥哥的样子呢。''

皇帝倒是对永珹颇为赞许:``嘉贵妃虽然拔尖儿要强,有些轻浮不大稳重,但永珹却是极好的。上次木兰围场之事后,朕实在对他刮目相看,又比永琪更机灵好胜。男儿家嘛,好胜也不是坏事。''

如懿俨然是一副慈母情怀,接口道:``最难得是兄友弟恭,不骄不矜,还口口声声说要提携五阿哥呢。也是愉妃出身寒微,不能与嘉贵妃相较。难得嘉贵妃有这份心,这般教导孩儿重视手足之情。''

皇帝的脸色登时有几分不豫:``他们是兄弟,即便愉妃出身差些,伺候朕的时候不多,但也不说不上要永珹提携永琪,都是庶子罢了。何况永琪还养在皇后你的膝下,有半个嫡子的名分在。''

``什么嫡子庶子!''如懿蕴了三分笑意,``臣妾心里,能为皇上分忧的,才是好孩子。''她半是叹半是赞,``到底是永珹能干,小小年纪,也能在运河钱娘上为皇上分担了。可见得这些事,还是自己的孩子来办妥当。有句话嘉贵妃说得对,高斌是做事做老成了,却也不济事了。''

皇帝剑眉一扬,已含了几分不满,声线亦提高:``这样的话是嘉贵妃说的?她身为嫔妃,怎可妄言政事!这几日她陪永珹进来,朕但凡与永珹论及南河侵亏案时,也只许她在侧殿候着。可见这样的话,必是永珹说与他额娘听的!''

如懿有些战战兢兢,忙看了一眼皇帝,欠身谢罪道:``皇上恕罪,嘉贵妃是永珹的生母,永珹说些给他额娘听,也不算大罪啊!''她一脸的谨小慎微,``何况皇上偶尔也会和臣妾提起几句政事,臣妾无知应答几句,看来是臣妾悖妄了。''

皇帝含怒叹息道:``如懿,你便不知了。朕是皇帝,你是皇后,有些话朕可以说,你可以听。但永珹刚涉政事,朕愿意听听他的见解,也叮嘱过他,身为皇子,凡事不可轻易对人言,喜恶不可轻易为人知,连对身边至亲之人亦是如此。''他摇头,``不想他一转身,还是忘了朕的叮嘱。''

如懿赔笑道:``永珹年轻,有些不谨慎也是有的。''

皇帝道:``这便是永琪的好处了。说话不多,朕有问才答,也不肯妄言。高斌在南河案上是有不妥,但毕竟是朕的老臣,好与不好,也轮不到嘉贵妃与永珹来置喙。看来是朕太过宠着永珹,让他过于得志了。''

如懿见皇帝动气,忙替他抚了抚心口,婉声道:``皇上所言极是。永珹心直口快,将皇上嘱咐办的事和臣妾或是嘉贵妃说说便算了,若出去也这般胸无城府,轻率直言,可便露了皇上的心思了。本来嘛,天威深远,岂是臣下可以随意揣测的,更何况轻易告诉人知道。''

皇帝眸中的阴沉更深,如懿也不再言,只是又添了甜羹,奉与皇帝。二人正相对,却见李玉进来道:``皇上,后日辰时二刻,总督那苏图之女戴氏湄若便将入宫。请旨,何处安置。''

皇帝徐徐喝完一碗甜羹,道:``皇后在此,问皇后便是。''

如懿想了想道:``且不知皇上打算给戴氏什么位分,臣妾也好安排合她身份的住所。''

皇帝沉吟片刻,便道:``戴氏是总督之女,又是镶黄旗的出生。她尚年轻,便给个嫔位吧。''他的手指笃笃敲在沉香木的桌上,思量着道:``封号便拟为忻字,取欢欣喜悦之情,为六宫添一点儿喜气吧。''

如懿即可道:``那臣妾便将同乐院指给忻嫔吧。''她屈身万福,保持着皇后应有的气度,将一缕酸辛无声地抿下,``恭喜皇上新得佳人。''

皇帝浅浅笑着:``皇后如此安排甚好。李玉,你便去打点着吧。''

此后几日,如懿再未听闻金玉妍陪伴永珹前往芳碧丛觐见皇帝,每每求见,也是李玉客客气气挡在外头,寻个由头回绝。便是永珹,见皇帝的时候也不如往常这般多了。

这一日的午睡刚起,如懿只觉得身上乏力,哄了一会儿永琪和璟兕,便看着容珮捧了花房里新供的大蓬淡红蔷薇来插瓶。

那样娇艳的花朵,带露沁香,仿若芳华正盛的美人,惹人怜爱。

如懿掩唇慵懒打了个呵欠,靠在丝绣玉兰花软枕上,慵懒道:``皇上昨夜又是歇在忻嫔那儿?''

容珮将插着蔷薇花的青金白纹瓶捧到如懿跟前,道:``可不是?自从皇上那日在柳荫深处偶遇了忻嫔,便喜欢得不得了。''

如懿取过一把小银剪子,随手剪去多余的花枝:``那时忻嫔刚进宫,不认识皇上,语言天真,反而让皇上十分中意,可见也是缘分。''

容珮道:``缘分不缘分的奴婢不知。忻嫔年轻貌美,如今这般得宠,宫中几句无人可及。皇后娘娘是否要留心些。''

如懿修剪着瓶中大蓬蔷薇的花枝,淡淡道:``忻嫔出身高贵,性子活泼烂漫,皇上宠爱她也是情理之中。何况自从玫嫔离世,舒妃自焚,嘉贵妃也被皇上冷落,纯贵妃与愉妃、婉嫔都不甚得宠,唯有庆嫔和颖嫔出挑些,再不然就是几个位分低的贵人、常在,皇上跟前是许久没有新人了。''

容珮撇撇嘴道:``年轻貌美是好,可谁不是从年轻貌美过来的?奴婢听闻皇上这些日子夜夜歇在忻嫔的同乐院,又赏赐无数,真真是殊宠呢。''

如懿转过脸,对着妆台上的紫铜鸾花镜,细细端详地看着镜中的女子,纵然是云鬓如雾,风姿宛然依稀如当年,仔细描摹后眉如远山含翠,唇如红缨沁朱,一颦一笑皆是国母的落落大方,气镇御内。只是眉梢眼角悄悄攀援而上的细纹已如春草蔓生,不可阻挡。她的美好,已经如盛放到极致的花朵,有种芳华将衰开到荼蘼的艳致。连自己都明白,这样的好,终将一日不如一日了。

如懿下意识地取出一盒绿梅粉,想要补上眼角的细碎的纹路,才扑了几下,不觉黯然失笑:``最是人间留不住,朱颜辞镜花辞树。有时候看着今日容颜老于昨日,还总是痴心妄想,想多留住一颗青春也是好的,却连自己也不得不承认,终究是老了,也难怪皇上喜欢新人。''

容珮朗声正气道:``中宫便是中宫,正室便是正室,哪怕有些妾侍个个貌美如花,也不能和娘娘比肩的。''

如懿微微颔首,语意沉着:``也是。是人如何会不老,红颜青春与年轻时的爱恋一般恍如朝露,逝去无痕,又何必苦苦执着。拿得住在手心里的,从来不是这些。''

容珮眉目肃然,沉吟着道:``娘娘说得极是。只是皇后娘娘方才说起嫔妃们,还忘了还有一位令妃。''

如懿仔细避开蔷薇花枝上的细刺,冷冷道:``本宫没忘。虽然上回着你去寻令妃,你回禀本宫她正在太后宫中受训斥,又说为了十阿哥死后唱昆曲见罪于本宫,才被与太后罚去十阿哥灵前跪着,偶遇了舒妃,与舒妃的死并无干系。但不知怎的,本宫心里总不舒服。这些日子她都自闭与宫中思过,倒是安静些了。''她的心思微沉,``这几日她日日写了请罪表献于本宫,述及往日情分,言辞倒也可怜。''

容珮轻哼一声道:``狐媚子都是狐媚子,再请罪也脱不了那可怜巴巴样儿!至于她安静不安静,一路看着才知道。''

如懿闻着清甜的花香,心中稍稍愉悦:``好了,那便不必理会她,由着她去吧。皇上过几日要去木兰围场秋狩,本宫才出月子不久,自然不能相陪,皇上可挑了什么人陪去伺候么?''

容珮道:``除了最得宠的忻嫔,便是颖嫔和恪常在。另则,皇上带了四阿哥和五阿哥,自然也带了嘉贵妃和愉妃小主。''

如懿听得``愉妃''二字,心下稍暖:``其实海兰虽然失宠,但皇上总愿意和她说说话,与她解语相伴,又用永琪争气,倒也稳妥,不失为一条求存之道。''

容珮微微凝眉:``娘娘这样说,有句话奴婢倒是僭越了,但不说出来,奴婢到底心中每个着落,还请娘娘宽恕奴婢失言之罪。''

如懿折了一枝浅红蔷薇簪在鬓边,照花前后镜,口中徐徐道:``你说便是。''

容珮道:``如今皇上的储位皇子之中,没了大阿哥和二阿哥不提,三阿哥郁郁不得志。皇子之中,咱们十二阿哥固然是嫡子,但到底年幼,眼下皇上又最喜欢四阿哥。这些日子皇子固然有些疏远嘉贵妃和四阿哥,但是四阿哥极力奔走,为江南筹集钱粮,十分卖力,皇上又喜欢了。奴婢想\ldots\ldots{}''她欲言又止,还是忍不住道,``奴婢想嘉贵妃一心是个不安分的,又有李朝的娘家靠山,怕是想替四阿哥谋夺太子之位也未可知。''

如懿轻轻一嗤:``什么也未可知,这是笃定的心思。嘉贵妃当年盯着后位不放,如今自然是看着太子之位。''

容珮见如懿这样说,越发大了胆子道:``奴婢想着,除了四阿哥,皇上还喜欢五阿哥。若皇上动了立长的心思,咱们看来,自然是选五阿哥比选四阿哥好。可即便是五阿哥养在娘娘下过,恕奴婢说句不知轻重的话,五阿哥到底不是娘娘肚子里出来的,再好再孝顺也是隔了层肚皮的。''

如懿正拨弄着手中一把象牙嵌青玉月牙梳,听得此言,手势也缓了下来。外头暑气正盛,人声寂寂,唯有翠盖深处的蝉不知疲倦地叫着,咝一声又咝一声地枯寂。那声音听得久了,像一条细细的绳索勒在心上,七缠八绕的,烦乱不堪。

如懿长嘘一口气道:``容珮,除了你也不会再有第二人来和本宫说这样的话。便是海兰和本宫如此亲近,这一层上也是有忌讳的。这件事本宫自生了永璂,心里颠来倒去想了许多次,如今也跟你说句掏心窝的话吧。''她镇一镇,声音沉缓入耳,``只要本宫是皇太后,永璂未必要是太子。''

容珮浑身一震,神色大变,旋即跪下道:``娘娘的意思是\ldots\ldots{}''

如懿握紧了手中的梳子,神色沉缓如磐石:``永璂还小,虽然是嫡子,但一切尚未可知。若永琪贤能有担当,他为储君也是好事,何必妄求亲子?永璂来日若做一个富贵王爷,也是好的。''

容珮低头思索片刻,道:``娘娘真这样想。''

如懿看着她,眸中澄静一片:``你与本宫之间,没有虚言。''

容珮定了定神,道:``无论娘娘怎么选怎么做,奴婢都追随娘娘。''

正说着,只见李玉进来道:``皇后娘娘,皇上说了,请您晚膳时分带着五公主往芳碧丛一同用膳。''

如懿颔首道:``知道了。''

李玉躬身退下,如懿吩咐道:``容珮,去准备沐浴更衣,本宫要去见皇上。''

\hypertarget{ux7b2cux4e09ux5341ux7ae0-ux6606ux8273}{%
\chapter{第三十章 昆艳}\label{ux7b2cux4e09ux5341ux7ae0-ux6606ux8273}}

天色将晚,暑气隐隐退却,凉风如玉而至,渐渐清凉,倒也惬意。如懿抱着璟兕与皇帝一同用膳。

皇帝见了如懿,便伸手挽了她一同坐下。皇帝才要侧身,不觉留驻,在她鬓边轻嗅流连,展颜笑道:``今日怎么这样香,可是用了上回西洋送来的香水?''

如懿轻俏一笑:``一路过来荷香满苑,若说衣染荷花清芬,倒是有几分道理。''

容珮在旁笑得抿嘴:``回皇上的话。皇后娘娘总说那西洋香水不易得,皇上除了给太后和几位长公主,满宫里只给娘娘留了两瓶,娘娘倒不大舍得用它呢。倒是皇上上回送来的西洋自鸣钟,娘娘喜欢得紧,只是如今怕吵着五公主,也收起来了。''

皇帝笑道:``如懿如懿,你也真是小气。什么好的不用,都收着做什么?''

如懿笑吟吟睇着他:``知道皇上心疼璟兕,但凡好的,臣妾都留给璟兕做嫁妆吧,到时候皇上便说臣妾大方又舍得了。''

容珮亦笑:``皇后娘娘别的小气,可皇上为娘娘亲制的绿梅粉,皇后娘娘最是舍得,每日必用无疑。''

皇帝旋即明白,抚掌道:``是了。你一向喜爱天然气味,所以连宫中制香也不甚用,何况西洋香水。''他撇嘴,眼底含着一抹深深的笑意,``原来朕赏错了人,反倒错费了。''

如懿摇首长叹:``可不是呢。臣妾心里原是将一番心意看得比千里迢迢来的西洋玩意儿重得多了。''

说罢,二人相视而笑。

皇帝罢手道:``都做额娘的人了,还这般伶牙俐齿。朕便找个与你性子相投的人来。''

李玉忙到:``回皇上皇后的话,忻嫔小主已在外候着了,预备为皇上皇后侍膳。奴才即刻去请。''说罢湘妃竹帘一打,只见一个玲珑娇小的女子盈盈而入,俏生生行了礼道:``皇上万福金安,皇后娘娘万福金安。''说罢又向着如懿行大礼,``臣妾忻嫔戴氏,叩见皇后娘娘。''

如懿见她抬头,果真生得极是妍好,不过十六七岁年纪,眉目间迤逦光耀,肌映晨霞,云鬓翠翘,一颦一笑均是天真明媚,娇丽之色便在艳阳之下也唔半分瑕疵。她活像一枚红儿饱满的石榴子,甜蜜多汁,晶莹得让人忍不住去亲吻细啜。宫中美人虽多,然而,像忻嫔一般澄澈中带着清甜的,却真是少有。

如懿便含笑:``快起来吧。在外头候着本就热,一进来又跪又拜,仔细一个脚滑跌成个不倒翁,皇上可要心疼了。''

忻嫔一双眸子如暗夜里星光璀璨,立即笑道:``原来皇后娘娘也喜欢不倒翁。臣妾再家时收了好些,还有无锡的大阿福。臣妾初初入宫,想着宫里什么都有,所以特备了一些打算送给十二阿哥和五公主呢。''

如懿听她言语俏皮,虽然出身大家,却无一点儿娇矜之气,活泼爽快之余也不失了分寸。又看她侍奉膳食时笑语如珠,并无寻常嫔妃的拘谨约束,心下便有几分欢喜。

一时饭毕,皇帝兴致颇好,便道:``圆明园中荷花正盛,让朕想起那年去杭州,未曾逢上六月荷花别样红,当真是遗憾。''

忻嫔接过侍女递上的茶水漱了口,乖巧道:``臣妾碎阿玛一直住在杭州,如今进了圆明园,觉得园子里兼有北地与南方两样风光,许多地方修得和江南风景一般无二,真正好呢。''

如懿笑道:``忻嫔的阿玛是闽浙总督,一直在南边长大,她说不错,必然是不错的。''

彼时小太监进忠端了水来伺候皇帝洗手,便道:``奴才今儿下午经过福海一带,见那里荷花正开得好呢,十里荷香,奴才都舍不得离开了。''

皇帝拿帕子拭净了手,起身道:``那便去吧。''

福海边凉风徐至,十里风荷如朝云缓缓,轻曳于烟水渺渺间,带着水波茫茫清气,格外凉爽宜人。

皇帝笑道:``不是朕宠坏了忻嫔,是她的确有可宠爱之处。''

如懿含笑道:``若说宫中嫔妃如繁花似锦,殷红粉白,那忻嫔便是开得格外清新俏丽的一朵。''

皇帝笑着握住她的手:``皇后的比方不错,可朕更觉得忻嫔的性子如凉风宜人,拂面清爽。''

如懿逗弄着乳母怀中的璟兕:``皇上这句可是极高的褒奖,真要羡煞宫中的姐妹了。''

皇帝笑叹着揉了揉眉心:``这些日子为江南水灾之事烦恼,也幸得忻嫔言语天真,才让朕高兴了些。朕也想皇后方才的比方来说忻嫔实在不够出挑,可若真论出挑,宫中性子对别致的却是舒妃,如翠竹生生,宁折不弯\ldots\ldots{}''皇帝话未说完,自己的神色也冷了下来,摆手道:``罢了,不说她了。这么傲气本不是什么好事。''

忻嫔转过头,鬓边的碎珠流苏如水波轻漾,有行云流水般的轻俏,她好奇道:``舒妃是谁?怎会有女子如翠竹?''她见皇帝脸色不豫,很快醒神,脆生生笑道:``其实太过傲气有什么好,譬如翠竹,譬如梅花,被积雪一压容易折断,换作臣妾呀,便喜欢做一枝女萝,有乔木可以依托便是了。''

如懿听忻嫔说得无忧无虑,蓦然想起前人的诗句:女萝附松柏,妄谓可始终。大概世间许多女子的梦想,只是希望有乔木松柏般的男子可以依托始终而已吧。

皇帝笑着捏一捏忻嫔红润的脸,笑道:``朕便是喜欢女萝的婉顺。''

朝蕣玉佩迎,高松女萝附。如懿低下头来,看着荔枝红缠枝金丝葡萄纹饰的袖口,繁复的金丝刺绣,缠绕着紫瑛与浅绿莹石密密堆砌三寸来阔的葡萄纹堆绣花边。那样果实累累的葡萄,原来也有着最柔软的藤蔓,才能攀援依附,求得保全。她微微一笑,凝视着十指尖尖,指甲上凤仙花染出的红痕似那一日春雨舒和的火色,红得刺痛眼眸。

她想,或许她和意欢这些年的亲近,也是因为彼此都不是女萝心性的人吧。

如懿知道皇帝心中介怀,也不顺嘴说下去,便指着一丛深红玫瑰向璟兕道:``玫瑰花儿好看,又红又香,只是多刺,璟兕可喜欢么?''

皇帝伸手抚着璟兕的脸庞,疼惜道:``身为公主,可不得像玫瑰一般,没点儿刺儿也太轻易被人折去了。''

忻嫔正折了一枝紫薇比在腮边,笑道:``公主还没长成,皇上就先怕被惜花人采折了呢,可真真是阿玛最疼女儿啊。''

如懿见她言语毫无心机,便也笑道:``你在家时,你阿玛一定也最疼你。''

忻嫔满脸骄傲:``皇后娘娘说得对极了!阿玛有好几个儿子,可是却最疼臣妾,总说臣妾是他的小棉袄,最贴心了。''

如懿故意扑一扑手中的刺绣玉兰叶子青罗扇,扇柄上的杏红流苏垂在她白皙的手背上像流霞迷离。她仰面看天叹道:``难怪了。如今正值盛暑,忻嫔你的阿玛热得受不了小棉袄了,便只好送进宫来了。''

忻嫔脸上红霞飞转,``哎呀''一声,躲到皇帝身后去了,片刻才探头道:``皇后娘娘原来这么爱笑话人。''

正说笑着,只听云间微风过,引来湖上清雅歌声,带着青萍红菱的淡淡香气,零零散散地飘来。

那是一把清婉遏云的女声,曼声唱道:``袅晴丝吹来闲庭院,摇漾春如线。停半晌整花钿,没揣菱花偷人半面,迤逗的彩云偏。我步香闺怎便把全身现。''

这歌声倒是极应景,只闻其声不见其人,极目望去,之间菰叶丛丛,莲叶田田,举出半人高的荷枝殷红如剑,如何看得见歌者是谁。唯有那拖得长长的音调如泣如诉,仿佛初春夜的融雪化开,檐头叮当,亦似朝露清圆,滚落与莲叶,坠于浮萍,更添了入暮时分的缠绵和哀怨。

芙蕖盈芳,成双的白鹭在粼粼波光中起起落落,偶尔有鸳鸯成双成对悠游而过,绵绵的歌声再度在碧波红莲间萦回。

皇帝似乎听得入神,便也停下了脚步,静静侧耳细听。

黄昏的流霞铺散如绮艳的锦,一叶扁舟于潺潺流水中划出,舟上堆满荷花莲叶,沐着清风徐徐,浅浅划近。一个身影纤纤的素衣女子坐在船上,缓缓唱道:``没乱里春情难遣,蓦地里怀人幽怨。则为俺生小婵娟,拣名门一例、一例里神仙眷。甚良缘,把青春抛的远!俺的睡情谁见?则索因循腼腆。想幽梦谁边,和春光暗流转?迁延,这衷怀那处言?''

这一声声女儿心肠既艳且悲,如诉衷肠,且那女声清澈高扬,飞旋而上,如被流云阻住,凄绝缠绵处,连禽鸟无知也难免幽幽咽咽,垂首黯然。

如懿隐隐听得耳熟,已然明白是谁。转首却见皇帝脸庞的棱角因这歌声而清润柔和,露出温煦如初阳般的笑意,不觉退后一步,正对上随侍在皇帝身后的凌云彻懂的眼。

果然,凌云彻亦猜到了那人是谁,只是微微摇头,便垂眸守在一遍,仿佛未曾听见一般。

如懿的嘴角微沉,神色便阴了下去。

所有人都陶醉在她的歌声里,璟兕虽年幼,亦止了笑闹,全神贯注地听着。一曲罢了,忻嫔忍不住拍手道:``唱得真好!臣妾在江南听了那么多昆曲,没有人能唱得这般情韵婉转,臣妾的心肠都被她唱软了。''

皇帝负手长立,温然轻吁道:``歌声柔婉,让朕觉得圆明园高墙无情,棱角生硬,亦少了许多粗粝,生出几许温柔。''

凌云彻眉心灼灼一跳,恭声道:``皇上与忻嫔小主说得是,微臣久听昆曲,也觉得是宫中南府戏班的最好。可见世间好的,都已在宫中了。''

皇帝颔首:``嗯,唱词既艳,情致又深,大约真是南府的歌伎了。''

``涉江玩秋水,爱此红蕖鲜。攀荷弄其珠,荡漾不成圆。佳人彩云里,欲赠隔远天。相思无因见,怅望凉风前。红莲当前,佳人便在眼前,皇上真是好艳福呢。''如懿畅然吟诵,向忻嫔使个眼色,忻嫔虽然心思简单,但也聪明,即刻挽住皇上的手臂道:``这不知是南府哪位歌伎唱昆曲呢,臣妾倒觉得,水面风荷圆,此时唱这首《游园惊梦》不算最合时宜,《采莲曲》才是最佳的。不如请皇上和皇后娘娘移步,往臣妾宫里一同听曲吧。''

如懿见忻嫔这般乖觉,心中愈加欢喜,也乐得顺水推舟:``也好,外头到底还有些热,五公主年幼,怕身子吃不消。如此,便打扰忻嫔妹妹了。''

皇帝似有几分犹豫,举眸往那船上望去,如懿看一眼李玉,李玉忙拍了拍额头道:``哎呀!都怪奴才,往日里皇上少往福海来,怕有婢子不知,在此练曲呢。奴才这便去看看。''

皇帝还要再看,忻嫔已然挽住了皇帝,笑着去了。

如懿微微松了一口气,落后两步:``是令妃?''

凌云彻苦笑道:``是她的嗓音。少年时她便喜爱昆曲,有几分功底,微臣听得出她的声音。''

容珮哼道:``原以为她安静了几日,原来躲在这里呢。''

如懿瞥她一眼:``你既不喜欢,就替本宫去打发了她,不许在有这狐媚样子了。''

容珮即刻答应了``是'',雷厉风行地去了。容珮才绕过双曲桥到了湖边,却见小舟已然停泊在岸,李玉正躬身和一素衣女子说话。容珮心里没好气,却不肯露了鄙薄的神色拉低了自己的身份,便上前恭恭敬敬行了一礼:``令妃娘娘万安。''

嬿婉原见李玉到来,知道皇帝就在近侧,以为是皇帝遣李玉来传自己,正喜滋滋问了一声:``是皇上派公公前来么?''此时乍然见了容珮,不觉花容乍变,勉强镇定道:``容姑姑怎么来了?''

容珮气定神闲道:``奴婢陪皇上、皇后娘娘、忻嫔小主和五公主散步,偶然听到昆曲,皇上和皇后娘娘随口问了一句,便派奴婢和李公公前来查看。''她见嬿婉一身浅柳色的蹙银线丝绣蝴蝶兰素纱衣深浅重叠,点缀着点点粉色桃花落在衣襟袖口,仿佛轻轻一呵就能化去。那粉红浅绿簇拥在一起本是庸俗,奈何她身段如弱柳纤纤,容貌一如夹岸桃花蘸水轻敷,胭色娇秾,只显得她愈加明艳动人。

容珮看着她便有气,脸上去笑着道:``皇上说,是哪家南府的歌伎不知礼数,在此唱曲惊扰圣驾,惹得忻嫔小主说唱这曲子不合时宜,还不如听《采莲曲》呢。''她皮笑肉不笑地努努嘴,``原来是令妃娘娘啊,那奴婢还是去回禀一声吧。''她故作为难道,``可是叫奴婢怎么回呢?难不成说皇上的嫔妃唱曲而跟南府的歌伎似的吧。这可真真是为难了。''

嬿婉听得此节,一腔欢喜期盼如被泼了兜头霜雪,脸色不可控制地灰败下去,只是尚不能完全相信,巴巴儿看着李玉。

李玉见嬿婉的泪光泛了上来,笑眯眯道:``容姑姑来得正好,奴才也正为这如何回话的事烦恼呢。这照实回吧,怕皇上说令妃娘娘不自重,被人以为是南府的歌伎,皇上的面子也过不去。若不回呢,这皇上问起是谁,还不好充数。''

容珮一脸的无奈与为难:``可不是?这曲儿若皇上喜欢,请令妃娘娘在皇上面前私下娱情,那是闺房之乐。可若皇上一时起了兴致,说让令妃娘娘当着皇后娘娘和各宫小主的面再唱一回,那可怎么算呢?''

嬿婉气得几乎要呕出血来,却也不敢露了一分不满,只得拼命压抑着,委委屈屈道:``既然皇上以为是南府的歌伎,那\ldots\ldots 那便还是请李公公这般回了吧。本宫\ldots\ldots{}''她缓一缓气息,露出如常的如花笑靥,``本宫不过是自己唱着玩儿罢了,不曾想会惊动了皇上和皇后。''

容珮微微一笑:``既然令妃娘娘自己也不想惊动,那李公公便好回话了。''

李玉一揖到底:``如此,奴才便可回禀了,多谢令妃娘娘教诲。''

经了这事,嬿婉更加郁郁沉寂,不几日皇帝领了嫔妃们前往热河秋狩,她也便称了病,日日请了太医延医问药。如懿与太后尚留在圆明园中避暑清养,听得容珮回禀,还以为嬿婉做作,打发了太医去看,果然回说是郁闷伤肝,要仔细调养。

皇帝既去了避暑山庄,如懿也不欲嬿婉在眼前,立刻遣人送她回紫禁城静养,得了眼前的清静。

自皇帝携了几个亲近的嫔妃前往热河秋狩,也远了紫禁城中的宫规森严。如懿与余下的嫔妃们住在圆明园中,倒也清闲自在。海兰本是要陪伴永琪一同随皇帝前往避暑山庄伴驾的,只是念着如懿才出月子不久,心力不如以前,一味吃药调理着,便自请留在了圆明园中陪伴,于是素日里往来的便也是绿筠、海兰和婉茵了。

如懿见海兰时时陪在跟前,便道:``皇上许你去热河伴驾是好事,你何必自己推脱了。''

海兰逗弄着九曲回廊下银笼架上的一双黄鹂,道:``有嘉贵妃那趾高气扬的人在,有什么意思?还不如这儿清清静静的。且臣妾不去,也是圆了纯贵妃的面子,她的三阿哥也没得去热河呢。''

如懿斜靠在红木卷牡丹纹美人靠上,笑吟吟道:``你倒是打算得精刮,只是你不去,永琪怕没人照应。''

海兰给架子上的黄鹂添上一斛清水,细长的珐琅点翠护甲闪着幽蓝莹莹的光,侍弄得颇有兴致,口中道:``臣妾不能陪永琪一辈子的,许多事他自己去做反而干净利落。扯上臣妾这样的额娘,本不是什么光彩事。''

如懿婉转看她一眼,嗔道:``你呀,又来了!做人要看以后福气,永珹有嘉贵妃这样的额娘,未必就多光彩了。''

海兰唇边安静的笑色如她耳垂上一对雪色珍珠耳坠一般,再美亦是不夺目的温润光泽:``也是。只是光彩不光彩的,咱们也只能暗中看着防着嘉贵妃罢了。她做的那许多事,终究也没法子处置了她。''她微微沉吟,道,``最近皇上屡屡赞许永珹协办赈济江南的钱粮得力,虽然不太宠幸嘉贵妃,但对她也总还和颜悦色。不过臣妾冷眼看着,皇帝对嘉贵妃到底是不如往日了,有时候想想,嘉贵妃有三个儿子,娘家又得力,又是潜邸伺候上来的老人了,竟也会有这样的时候。再看看自己,也没什么好怨的了。''

如懿的神色淡然宁静,掐下廊边一盆海棠花的嫣红花骨朵儿在手中把玩:``新人像御花园里的鲜花一茬一茬开不败,谁还顾得上流连从前看过的花儿呢。便是芳华正浓都会看腻,何况是花期将过。所以在宫里不要妄图去挽留什么,抓得住眼前能抓的东西才最要紧。''

海兰轻笑着按住如懿的手,拈起一朵海棠在如懿唇边一晃,骤然正色道:``哀音易生悲兆。皇后娘娘儿女双全,这样没福气的话不能出自您的扣。''她抿嘴,有些幸灾乐祸的快活,``听说前几日令妃又不安分,还是娘娘弹压了她。其实令妃已然失宠,又生性狐媚,娘娘何不干净利落处置了,省得在眼前讨嫌。''

如懿见周遭并无旁人,闲闲取过一把青玉螺钿缀胭脂缠丝玛瑙的小扇轻摇:``海兰,令妃固然失宠,皇上却未曾废除她位分,依然留着她妃位的尊位,你知道是为何么?''

海兰冷冷一嗤,自嘲道:``年轻貌美,自然让人存有旧情。若是都如臣妾一般让人见之生厌,倒也清静了。''

如懿伸出手,替她正一正燕尾后一把小巧的金粉莲花紫翡七齿梳,柔声道:``宫中若论绣工,无人可出你右。''

海兰握住她的手,恳切道:``姐姐腹有诗书气自华。''

如懿羽睫微垂,只是浅浅一笑,似乎不以为然:``腹有诗书,温柔婉约,不是慧贤皇贵妃最擅长的么?孝贤皇后克己持家,也算精打细算,有主母之风。嘉贵妃精通李朝器乐,剑舞鼓瑟样样都精绝,所以哪怕屡次不得圣意,也还有如今的尊荣。玫嫔弹得一手好琵琶,庆嫔会得唱元曲。舒妃精通诗词,书法清丽。颖嫔弓马骑射,无一不精。便是忻嫔新贵上位,宠擅一时,也是因为幼承闺训,小儿女情态中不失大家风范。唯有令妃,她是不同的。''

海兰撇了撇嘴,不甚放在心上:``她出身宫女,大字不识几个。便是年幼家中富足,也未得好好儿教养,一味轻薄狐媚,辜负了那张与娘娘有三分相似的面孔。''

如懿喟然轻叹:``你的眼光精到。这固然是令妃的短处,却不知也是她的长处。''

海兰睁大了眼,似是不信:``长处?''

如懿婉声道:``我们所拥有的技艺与学识,涵养与气质,都是在见到皇上前已经所有的。皇上所欣赏的,是一个已然完成的成品。而比之我们,令妃在见到皇上时,更像一张未曾落笔的白纸,无知、简单,却可以由着皇上的性子肆意描绘。纵然她拿着燕窝细粉挥霍暴发,纵然她连甜白釉也不识,可是一旦她所学所知,气度愈加恬美清雅,轻柔妩媚,那都是在见到皇上后所得,或者说,皇上不经意间一手培养的,所以皇上看着今时今日的她,总还会有几分怜惜与容忍。''

海兰凝神片刻,锋锐的护甲划过半透明的轻罗蒙就的扇面,发出轻微的行将破碎的咝咝声:``那就更留不得了。''

如懿轻缓地拍拍她的手背:``不到万不得已,不要做那样的事。''她的神色着烟雨蒙蒙的哀声与愧疚,``海兰,许多话,本宫可以瞒着任何人,却无须瞒你。孝贤皇后的二阿哥\ldots\ldots 本宫总是日夜不安。尤其为人母亲之后,更是念及便心惊不已。海兰,若说本宫毕生有一亏心事,便是这桩了。所以,许多事,未必赶尽杀绝才是好。''

海兰见如懿动了哀情,雪白的面孔在明耀的日光下隐隐发青,不免生了不安之意,忙挽了如懿的手进内殿,道:``不过小小嫔妃,不值得娘娘伤神。''她望了望过于炫目的天光,关切道:``外头热,娘娘仔细中暑才是。''

恰好有小宫女捧上酸梅汤来,如懿勉强和缓了神色,正端起欲饮,海兰见了忙道:``娘娘才出月子没多久,可不能吃酸梅这样收敛的东西,否则气血不畅可便坏了。''她唤来容珮:``如今虽是盛暑,娘娘的东西可碰不得酸凉的,还是换一碗薏仁红枣羹来,去湿补血最好不过的。''

容珮抿嘴笑道:``是奴婢们不当心了,多谢愉妃小主提点,说来江太医也算是个心细的了,竟还是比不过愉妃小主,事事替娘娘留心。''

海兰望着如懿,一脸真诚:``那有什么,娘娘怎么替本宫留心的,本宫也是一样的。''她见容珮退下,便低声道:``永琪跟着永珹一起调度钱粮,永珹事事争先,拔尖卖乖,臣妾已经按着娘娘的嘱咐,要永琪万事唯永珹马首是瞻,不要争先出头。''

如懿拿着一方葡萄紫綾销如意云纹绢子擦了擦额头沁出的细汗,道:``如今永珹得意,且由他得意。少年气盛,容易登高,也必跌重。等哪天永珹落下来了,便也轮到永琪锋芒毕露的时候,不必急于一时。''

正说着,菱枝进来奉上一个锦盒,道:``皇后娘娘,内务府新制了一批镂金红宝的护甲,请娘娘赏玩。''

如懿``嗯''了一声,挥手示意菱枝退下。海兰剥了颗葡萄递到如懿手中:``有皇后娘娘为永琪筹谋,臣妾很安心。''她想起一事,``对了,上回听说令妃抱病,如今送回宫中,也有十来日了吧。''

如懿打开锦盒,随手翻看盒中宝光流离的各色护甲,漫不经心道:``令妃既病着,本宫就由她落个清静。左右宫里的嫔妃都跟着来圆明园避暑了,让她回宫和先帝的老太妃们做伴儿,也静静心。''

海兰一笑,便和如懿抵着头一起炼选护甲比在指上把玩。二人正得趣,只见三宝急急进来打了千儿道:``皇后娘娘,李公公从避暑山庄传来的消息,请您过目。''他说罢,递上一个宫中最寻常的宫样荷包,便是宫女们最常佩戴的普通样式。如懿颔首示意他退下,取过一把银剪子剔开荷包缝合处的绣线,取出一张纸条来。如懿才看了一眼,脸色微白,旋即冷笑一声,手心紧紧蜷起。

海兰见如懿如此,亦知必生了事端,忙接过她手中的纸条一看,矍然变色:``令妃复宠?她不是回紫禁城了么?''

如懿取了一枚翡翠七金绞丝护甲套在指上,微微一笑:``本宫当她回了紫禁城,却不想在避暑山庄唱出这么一出好戏来,不能亲眼看见,真是可惜了!''如懿一笑如春华生露,映着朝阳晨光莹然,然而,她眼中却一分笑意也无,那种清冷的神色,如她指上护甲的尖端金光一闪,让人寒意顿生。

海兰的颓然如秋风中瑟瑟的叶:``令妃的手脚倒是快,一个不留神便复宠了。''她攥紧了手中的纸条,反反复复地揉搓着:``只是已然复宠,咱们想阻止也难了。''她峨眉轻扬,将那颓然即刻扫去,恍若又是一潭静水深沉,``只是啊,能复宠的,也还会再失宠。皇后娘娘,咱们不怕等。''

如懿笃定一笑,并不十分放在心上:``本宫已经和你说过皇上的心思,看来倒真是防不胜防。罢了,潮起潮落见得多了,不在这一时。何况身为皇后,若是时时事事只专注于和嫔妃争宠计较,怕也是真真忙不过来,反倒失了大局。''

如此留了心意,消息接二连三传来,不外是嬿婉如何到了避暑山庄,如何扮成小宫女的样子在清晨时分初秋红叶下素衣微凉,临风吟唱昆曲,引得皇帝心意迟迟,一举复宠。又如何陪着皇帝策马行猎,英姿飒爽。如何与颖嫔、忻嫔平分春色,渐渐更胜一筹。

如懿听在耳中,却也不意外:``令妃在皇上身边多年,自然比新得宠的颖嫔、忻嫔更懂得皇上的心思。何况她大起大落过,比一直顺风顺水的嫔妃们更懂得把握。''

海兰凝眉一笑,落了一子在棋盘上:``所以啊,有时候光是年轻貌美也不是够的,年岁是资历,亦是风情啊。''

如懿凝神片刻,也落了一子。那棋子是象牙雕琢而成的,落在汉白玉的棋盘上玎玲有声:``何必拐着弯把大家都夸进去,倒说得咱们这些半老徐娘都得意。''如懿一笑,``也别总想着咱们这些女儿家的事,后宫的事,顶破了天也只是女人们的是非。对了,永琪如何?''

海兰笑吟吟道:``左右风头都是永珹的。对了,臣妾倒是听说河务布政使富勒赫奏劾外河同知陈克济、海防同知王德宜亏帑贪污,并言及洪泽湖水溢,通判周冕未为准备,致使水漫不能抵挡。''

如懿捻了一枚棋子蹙眉道:``这些名字怎么这么耳熟?''

海兰将雪白一子落在如懿的半局黑子之中:``这些人都是高斌的部下,而高斌这些日子都在何工上奉职,这也是他的分内之事。皇后娘娘忘了么?''

如懿轻嗤道:``皇上年年写悼诗追念慧贤皇贵妃,不知这份恩义会不会随着岁月流逝而淡薄呢?''

海兰的脸庞恬淡若秋水宁和:``永琪递回来的消息,皇上严责高斌徇纵,似有拿高斌革职之意。''

如懿沉吟:``似乎有不代表一定会。''

海兰浅浅笑道:``那臣妾让永琪推把手吧。虽然人已入土,往日恩怨可以一笔勾销,但想到慧贤皇贵妃在世对臣妾的苛待,臣妾真是终身难以忘怀啊!''

如懿会心一笑:``虽然慧贤皇贵妃离世多年,但本宫也不希望看到她的母家在前朝蹦跶了。''她随手翻乱棋局,``就这么着了吧。''

\hypertarget{ux5982ux61ffux4f20-ux7b2cux4e94ux518c}{%
\part{如懿传 第五册}\label{ux5982ux61ffux4f20-ux7b2cux4e94ux518c}}

\hypertarget{ux7b2cux4e00ux7ae0-ux79cbux6247}{%
\chapter{第一章 秋扇}\label{ux7b2cux4e00ux7ae0-ux79cbux6247}}

霜般的凉意伴着浅浅的金色轻烟,染黄了嫩绿的树叶,亦红透了枫树半边。御花园的清秋菊花随着秋虫唧唧渐次开放,金菊、白菊、红菊、紫菊锦绣盛开,晕染出一片胜于春色的旖旎。而其中开得最盛的一枝,便是再度得幸的嬿婉。

如懿再次见到嬿婉时,已是九月十五回銮之后。大约在避暑山庄极为得幸,如懿见到她时,从她丰润微翘的唇瓣,便知晓了她如何得宠的种种传言。

热河行宫木兰秋狝的飒飒英姿,衬着昆曲悠扬的袅娜情韵,刚柔并济,如何不动人情肠呢?

回宫当日的夜晚,嬿婉便赶来拜见如懿。她穿了一身江南织造新贡的浅浅樱花色轻容真珠锦,像四月樱花翩翩飘落时最难挽留的一抹柔丽,撞入眼帘时,娇嫩得令人连呼吸也不自觉地轻微了。那衣裙针线细密,用绒只一二丝,以针如发细者绣成,设色精妙,光彩射目。裙裾上一对并蒂花鸟极尽绰约谗唼之态,风动处色如月华,飘扬绚烂,华丽而不失婉约之气。袖口用米珠并萤石穿以淡银白色的丝线绣了精致的半开梨花,更见清雅别致,与她精心绾就的发髻上数枚云母水晶同心花钿交相辉映,更兼一对金镶玉步摇上镂金蝶翅,镶着精琢玉串珠,长长垂下,并着六对小巧的滚金流珠发簪,格外有一种华贵之美。

此时明月悬空,玉宇清宁,月光无尘无瑕入窗,不觉盈满一室。嬿婉容颜剔透,在烛火下如无瑕美玉,连如懿也不由得注目。原来皇帝的恩幸与荣宠,可以让一个女人绽放得如此娇美。

嬿婉见了如懿,徐徐恭敬拜倒:``皇后娘娘凤体安康,福绥绵长。''

如懿置身九莲凤尾宝座之上,俯视着她道:``有令妃伺候皇上,本宫自然凤体安康,福绥绵长。''

嬿婉的声音柔婉得如春日枝上呖呖婉转的百灵:``臣妾身为嫔妃,伺候皇上是应当的。''

容珮递上茶水,笑吟吟道:``嫔妃伺候皇上自然是应当的,但打扮成宫女尾随皇上去避暑山庄唱着曲儿伺候,奴婢在宫里这些年,也是头一回听闻。''

嬿婉含笑望着容珮道:``本宫怎么伺候皇上,只要皇上高兴,你一个奴婢能置喙什么?''

如懿拨弄着手里的蜜蜡佛珠,那圆润饱满的珠子在她手心缓缓地一下一下滑过。她沉声道:``容珮是不能置喙,只是本宫也在想,你既病着要回紫禁城静养,怎么突然便去了避暑山庄了。你这病啊也太厉害了,能让你精神百倍奔赴千里到皇上身边。这样好的病,只怕是宫里人人都要羡慕了。''

嬿婉似一只在溪边啜饮溪水受到惊吓的小鹿,白皙娇嫩的手按在胸口,惶然欲泣:``臣妾想着自己病重,一心惦念皇上,只怕不见上皇上一面,若是自己撑不住,岂不终身抱憾?所以左右拼着一死,才大胆去了避暑山庄。''

如懿抬头望着殿顶的水彩壁画,金粉灿灿,描摹的神仙故事仿佛是最好的一台戏,演着不真实的喜怒哀乐。她不屑地笑道:``原来令妃的病一到避暑山庄就可以即刻痊愈,还能歌会唱了。''

嬿婉的声音细细柔柔,仿佛能掐出水来:``情不知所起,一往而深,生者可以死,死可以生。相思无因,生死都是一念间,何况臣妾区区之病,一见皇上,自然什么都好了。''她抬头瞥一眼如懿,``或者说,皇上洪福齐天,荫庇臣妾了。''

这样的言语,自然是无可挑剔。落在男人的眼中、耳里,怕更是触动柔肠吧。

如懿垂下眼眸,浅浅划过一丝冷笑:``这样说来,倒是本宫不好,不让你见皇上,才叫你惹出一身的女儿病来。''

嬿婉的微笑如秋水生波,涟漪缓缓,双目中甚至浮升起一层朦胧的水雾。她美丽的容颜温顺而驯服,让人不由得生怜:``臣妾自知冒犯宫规,此刻来见皇后娘娘,便是来谢罪了,更有一份大礼献予皇后娘娘。''

如懿好整以暇,垂眸把玩着指上的双色碧玺戒指,道:``什么大礼?说来听听。''

嬿婉柔声地一字一字吐出:``高斌被革职了。''

如懿心头一跳,面上却平和得波澜不兴:``慧贤皇贵妃死了这么久,皇上即便有几分旧情也淡薄得差不多了,想必你也进言不少,高斌才会被革职得这么快。''

嬿婉谦卑道:``即使臣妾费些口舌功夫也不能让慧贤皇贵妃起死回生来向皇后娘娘谢罪,所以只好拿她阿玛抵过了。若娘娘觉得臣妾此事不够将功抵过,臣妾任凭皇后娘娘责罚。''

片刻的静默后,如懿很快微微一笑,语气和缓道:``你是皇上跟前的宠妃,责罚了你,谁伺候皇上呢?罢了吧。''

嬿婉跪下,膝行到如懿跟前,一脸楚楚:``臣妾从前有所过失,皆因出身卑微,不识大体,但臣妾敬重皇后娘娘之心,从无拂违。臣妾虽然愚笨,但求能趋奉皇后娘娘左右,奉洒扫之责,臣妾就欢喜不尽了。''

容珮满面堆笑,出口却字字犀利:``令妃小主要在皇后娘娘身边奉洒扫之责,那奴婢们该去哪儿了呢。得了,皇后娘娘由奴婢们伺候,小主尽心伺候皇上便是。若能六宫里个个安分,便是皇后娘娘的清闲了。''

如懿看了容珮一眼,笑得从容宁和:``好了。时辰不早了,本宫记得今日皇上是翻了你的牌子,快去养心殿侍寝吧。你的心意,本宫都领了。''

嬿婉俯首三拜,躬身退去。容珮望着她出去了,狠狠地啐了一口道:``做作!矫情!''

如懿按一按容珮的手:``方才你的言语里已经敲打过她了,不必再说什么。''

容珮气咻咻道:``皇后娘娘怎不借此机会责罚令妃擅自离宫之罪?''

如懿取过一个珐琅雕花盒,用食指蘸了一点儿薄荷膏轻轻揉着额角,徐徐道:``你以为令妃真的是来谢罪想要将功抵过的?她告诉本宫她能让高斌革职,是提醒本宫她在皇上心中的分量。''

容珮撇嘴道:``高斌革职,那是五阿哥的本事,她也敢来沾这个功劳。''

如懿摆一摆手,指间的红宝金戒指划出一道流丽的光影,熠熠生辉:``永琪虽然在高斌革职的事上出了力,但不能显山露水太着了痕迹,况他毕竟年少,一直收敛羽翼,不能出头太多。令妃敢说这个话,自然不怕本宫去查。可见高斌革职,的确是令妃出力更多。''如懿凝神片刻,``而且本宫也一直疑惑,令妃当日装病假意要回宫静养,如何能一路妥妥当当去了避暑山庄,一定是有人暗中相助,这个人\ldots{}''

如懿沉吟,捻着一串东珠碧玺十八子手串不语,那手串上垂落的两颗翠质结珠,沙沙地打在她手指上,有微雨颤颤似的凉。

容珮惊异道:``娘娘是怀疑\ldots{}''

如懿手势一滞,缓缓摇头:``要真疑心,人人都有可疑。只是到了这一步,令妃必有贵人相助,又得皇上宠爱,风头正盛,咱们何必去动皇上心尖子上的人,拂了皇上的心意。女人啊,有得宠就有失宠,等她失宠时便简单了。''

容珮担心道:``可如今令妃这样得宠,连忻嫔都被比下去了\ldots{}''

``忻嫔是不会被比下去的。忻嫔虽然性子直爽,但不是蠢笨的人。何况皇上重视准噶尔之事,是不会冷落了忻嫔的。''如懿以指尖佛珠的冰凉,来平静灼热的气息,``不是令妃得宠便是旁人得宠,你方唱罢我登场,风水轮流转罢了。本宫是皇后,是中宫,无论谁得宠都不会改变。何不冷眼旁观,暂取个分明呢。''

容珮稍稍放心,低声道:``只是令妃尚且年轻,迟早会为皇上生下龙胎,那时候她的地位岂不更加稳固?娘娘可要稍作防范?''

月光似皎皎流素,泻入室内。如懿轻匀的妆容柔美平和,浸润在月影中,更添了一丝稳重:``论及子女,难道纯贵妃与嘉贵妃的孩子还不多?若要地位稳固,只在皇上心意,而非其他。皇上已经有那么多皇子、公主,即便令妃生下什么,孩子年幼,也不必怕。''如懿长叹一声,幽幽道,``本宫所担心的,只是令妃的心性。容珮,你可看到她的手指上多了好些红肿处?''

容珮蹙眉疑道:``奴婢看到了。只是令妃恩宠正盛,养尊处优,难道还要自己劳作?''

窗台下一盆绿菊开得那样好,浸在洁净的月光底下,寂寂孤绝。如懿折下一枝把玩,摇头道:``那是被弓弦勒出的痕迹。听闻在避暑山庄时,令妃常常陪伴皇上行猎骑射。本宫记得令妃是汉军旗出身,不比满蒙女子擅于骑射,她一定是暗中下了不少苦工练习才会如此。这个女子,外表柔弱,内心刚强,不可小觑了。''

容珮犹疑道:``那咱们该怎么做?''

如懿轻轻嗅了嗅绿菊清苦的甘馨,静静捻着一串绿玉髓佛珠,缓缓拨动:``知其底细,静观其变。''

嬿婉在养心殿的围房除去衣衫,卸妆披发,被宫女们裹上锦被,交到侍寝太监手中。寝殿内皇帝已然斜倚在榻上等她。明黄的赤绣蟠龙锦缎帷帐铺天盖地落落垂下,嬿婉听着宫人们的脚步渐次退远,便从自己的粉红锦被中钻出,一点一点挪入皇帝怀中,露出一张洗去铅华后素白如芙蕖的脸。

皇帝笑着抚摸她的脸颊:``朕就喜欢你蛾眉不扫,铅华不御,就像那日朕在避暑山庄红叶漫天下见到你一身素淡,让朕惊艳之余念念不忘。''

嬿婉看着烛光莹亮,照得帐上所悬的碧金坠八宝纹饰,华彩夺目,直刺入心,让她心生欢喜。仿佛只有这样华丽的璀璨,才能让她那颗不定的心有了着落。

终于,终于又可以在这里度过一个清漫的长夜。用自己得意而欢愉的笑声,去照亮紫禁城中那些寂寞而妒恨的眼。

嬿婉将半张粉面埋在皇帝怀中,娇滴滴道:``是皇上长情顾念,不厌弃臣妾这张看了多年的脸面而已。''

皇帝笑着吻上她的面颊,手指留恋着她光腻的颈,低语细细:``能让朕不厌弃的,便是你的好处。''

嬿婉躲避着皇帝的胡须拂上面颊,笑声如风中银铃般清脆呖呖。她略一挣扎,牵动耳垂一对垂珠蓝玉珰。她低低痛呼了一声,也不顾耳垂疼痛,先摘下耳珰捧在手心对着烛火细细查看,十分在意。片刻,见耳珰浑然无损,嬿婉复又小心戴上,柔声道:``是臣妾不小心了。''

皇帝见她如此在意,便道:``这耳珰朕见你常常戴着,你很喜欢么?看着倒是有些眼熟。''

嬿婉爱惜地抚着耳珰上垂落的两颗晶莹剔透的明珠,生了几分寥落的怅然:``臣妾说了,皇上不会怪罪臣妾?''

皇帝轻怜密爱道:``自然不会。你说什么,朕都喜欢。''

嬿婉娇怯怯地抬眼:``这副耳珰是舒妃生前喜爱的,也是她遗物之一。臣妾顾念多年姐妹之情,特意寻来做个念想。''

皇帝脸上闪过一丝乌云般的阴翳,淡淡道:``宫里好东西多的是,明日朕赏你十对明珠耳珰,供你佩戴。过世人的东西不吉,便不要再碰了。''

嬿婉怯生生道:``皇上说得是。只是臣妾怜悯舒妃早逝,十阿哥也早早夭折,心里总是放不下。''

皇帝念及十阿哥,也有些不忍,道:``从前朕是见你与舒妃来往,想来也是你心肠软,才这般放不下。舒妃也罢了,十阿哥,也是可怜。''

嬿婉眼角闪落两滴晶莹的泪珠,落在她莹白如玉的面颊上,显得格外楚楚:``若十阿哥不曾早夭,舒妃也不会疯魔了心性。说来当时舒妃骤然有孕,臣妾十分羡慕,连皇后娘娘也时常感叹不及舒妃的福气,谁知到头来竟是舒妃先去了。''

皇帝默然片刻,也生出几许哀叹之意:``朕多有皇子早夭,不仅是十阿哥,还有二阿哥、七阿哥和九阿哥,想来父子缘薄,竟是上苍不悯。''

嬿婉轻拭眼角泪痕:``为父子母女皆是缘分。臣妾自己没有子女,也是缘分太薄的缘故。臣妾记得当时皇后娘娘尚未生育十二阿哥和五公主,听闻舒妃姐姐有孕,也是羡慕感慨,竟至酒醉。臣妾伴随娘娘多年,也从未见娘娘有这样失态的时候。幸而皇后娘娘如今儿女双全,也是福报到了。''

皇帝眉心一动,曲折如川:``皇后一向持重,即便羡慕,何至酒醉?''

嬿婉依偎在皇帝胸前,低柔道:``臣妾若非亲眼所见,也不能相信。不过后来皇后娘娘对舒妃姐姐的身孕关怀备至,时时嘘寒问暖,舒妃姐姐才能顺利产下十阿哥,可见皇后娘娘慈心了。只是唯一不足的是,舒妃姐姐孕中突然脱发,以致损及腹中的十阿哥,想来缘分注定,让我们姐妹不能多相伴几年。''她说到此节,越发伤感,低低啜泣不已。

皇帝安慰地拍着她消瘦的肩头:``朕记得,当年皇后与朕巡幸江南,还特意派了江与彬赶回宫中照料。皇后也算尽心了。''

嬿婉哀哀若梨花春雨:``是啊。连在宫中陪伴舒妃姐姐的,也是皇后娘娘的好姐妹愉妃呢。愉妃生养过五阿哥,到底稳当些,何况当时五阿哥还寄养在皇后娘娘名下,是半个嫡子呢。臣妾也一直羡慕舒妃姐姐,一直得皇上这般宠爱,生下的十阿哥也比五阿哥得皇上喜欢多了。''

皇帝的眼中闪过一丝疑惑,不过一瞬,旋即若无其事地抚上她的下颌,呵气轻绵:``好了,良宵苦短,何必总念着这些。''

嬿婉泪痕未干,低低嘤咛一声,噗嗤一笑,伏在了皇帝怀中,双双卷入红衾软枕之间。

皇帝自回宫之后,多半歇在嬿婉和颖嫔宫中,得闲也往忻嫔、恪常在处去,六宫的其余妃嫔,倒是疏懒了许多。绿筠和海兰不得宠便也罢了,玉妍是头一个不乐意的,庆嫔和晋嫔亦是年轻,嘴上便有些不肯饶人了。

如懿偶尔听见几句,便和言劝道:``莫说年轻貌美的人日子还长,便是嘉贵妃又有什么可说的呢?当日在避暑山庄嘉贵妃是嫔妃中位分最高的,还不是眼睁睁地看着令妃复宠,如今又何必把这些酸话撂到宫里来。''

玉妍气得银牙暗碎,亦只是无可奈何,便笑道:``皇后娘娘原来已经这般好脾气了。臣妾还当娘娘气性一如当年,杀伐决断,眼里容不得沙子呢。''

如懿扬一扬手里的浅杏色绢子,吩咐了芸枝给各位嫔妃添上吃食点心,应答间无一丝停滞,只是如行云流水般从容:``岁月匆匆如流水,如今自己都为人母了,什么火爆性子也都磨砺得和缓了。嘉贵妃不是更该深有体会么?''

幸而永珹风头正盛,玉妍倒也能得些安慰,便道:``臣妾自知年华渐逝,比不得皇后娘娘位高恩深,只能把全副心思寄托在儿子身上了。''她摇一摇手中的金红芍药团花扇,晃得象牙扇柄上的桃红流苏沙沙作响,``臣妾都年过四十了,幸好有个大儿子争气,眼看着要成家开府,也有个指望,若是儿女年幼的,得盼到什么时候才是个头呢。''

婉茵听得这话明里暗里都是在讽刺如懿,她又是个万事和为贵的性子,忙笑着打岔道:``都快到十月里了,这些日子夜里都寒浸浸的,嘉贵妃怎么还拿着扇子呢?''

玉妍盈盈一笑,明眸皓齿:``我诗书上虽不算通,但秋扇见捐的典故还是知道的。''她眼光流转,盈盈浮波,瞟着如懿道,```常恐秋节至,凉飙夺炎热。弃捐箧笥中,恩情中道绝。'婉嫔你早不大得宠也罢了,咱们这些但凡得过皇上宠幸的人,谁不怕有一日成了这秋日的扇子被人随手扔了呢?所以我才越发舍不得,哪怕天冷了,总还是带着啊。''

婉茵是个老实人,口舌上哪里争得过玉妍,只得低头不语了。如懿清浅一笑,转而肃然:``人人都说秋扇见捐是秋扇可怜,换作本宫,倒觉得是秋扇自作自受。所谓团扇,夏日固然可爱,舍不得离手,到了秋冬时节不合时宜,自然会弃之一旁。若是为人聪明,夏日是团扇送凉风,冬日是手炉暖人心,那被人喜爱还来不及,哪里舍得丢弃一旁呢?所以合时宜,知进退是最要紧的。''

海兰望向如懿,会心一笑:``皇后娘娘说得极是。皇上又不是汉成帝这样的昏君,哪里就独宠了赵飞燕姐妹,让旁的姐妹们落个秋扇见捐的下场呢。幸而嘉贵妃是开玩笑,否则还让人以为是在背后诋毁皇上的圣明呢。''

海兰在人前向来寡言少语,却字字绵里藏针,刺得玉妍脸上的肌肉微微一搐,随手撂下了扇子,呵斥身边的丽心道:``茶都凉了,还不添些水来,真没眼色。''

如懿与海兰相视而笑,再不顾玉妍,只转首看着绿筠亲切道:``本宫前日见了皇上,提起永璋是诸位皇子中最年长的,如今永珹和永琪都很出息,也该让永璋这个长子好好做个表率,为宗室朝廷多尽些心力了,且皇上已经答允了。''

玉妍的脸色登时有些不好看,她沉吟片刻,旋即满脸堆笑:``哎呀!原来皇后娘娘是前日才见到皇上的,只是呀,怕前日说定的事昨日或许就变卦了。如今皇上一心在令妃身上,或许昆曲儿听得骨头一酥便忘了呢。''

嬿婉本安静地坐在角落里,听见提及自己,忙对着玉妍赔笑道:``皇上不过得闲在妹妹那里坐坐,听听曲儿罢了,心意还是都在皇后娘娘身上呢。''

玉妍``咯''地冷笑一声:``皇上原本就是在你那儿听听曲儿罢了,和从前南府出身的玫嫔弹琵琶一样,都是个消遣罢了,还能多认真呢。如今玫嫔死了这些日子,皇上可一句都没提起过呢。都是玩意儿罢了!''她长叹一声,迎向如懿的目光,``说来皇后娘娘疼纯贵妃的三阿哥也是应当的,谁叫皇后娘娘与行三的阿哥最有缘呢。''

这话便是蓄意的挑衅了,刻薄到如懿连一贯的矜持都险险维持不住。是啊,多少年前的旧事了,若不是玉妍是潜邸的旧人,怕是连如懿自己的记忆都已经模糊成了二十多年前一抹昏黄而朦胧的月光了。

颖嫔本是出身蒙古,资历又浅,原不知这些底细,忍不住问道:``皇后娘娘生的是十二阿哥,又不是三阿哥,哪来什么和行三的阿哥最有缘呢?''

绿筠听得不安,不觉连连蹙眉。海兰旋即一笑,挡在前头道:``什么有缘不有缘的?嘉贵妃最爱说笑了。''

玉妍正巴不得颖嫔这一句,掩口笑道:``愉妃有什么可心虚要拦着的?当年皇后娘娘不是没嫁成先帝的三阿哥么。哪怕有缘,也是有缘无分哪!皇后娘娘,您说是么?''

如懿淡淡一笑,眼底蓄起冷冽的寒光,缓缓道:``嘉贵妃说话越来越风趣了。容珮,把内务府新制的一对赤金灯笼耳环拿来,赏赐给嘉贵妃。''

玉妍听得``耳环''两字,浑身一颤,不自觉地摸着自己耳垂,便打了个寒噤。

嬿婉看玉妍尴尬,乐得讨如懿的喜欢,便道:``皇上新赏了臣妾好些首饰,臣妾便挑几对上好的耳环,一并送予嘉贵妃。''

忻嫔最不喜看嬿婉这般嘚瑟,撇撇嘴道:``人说锦上添花便好,要是送礼也送成了落井下石,那便是坏了心术了。''

如懿深知二人平分秋色,彼此之间自然少不得明争暗斗,也懒得理会,只说笑了几句,便也散了。

\hypertarget{ux7b2cux4e8cux7ae0-ux7687ux5b50}{%
\chapter{第二章 皇子}\label{ux7b2cux4e8cux7ae0-ux7687ux5b50}}

日子安静了几天,这一日秋风习习,寒意如一层冰冷的羽衣披覆于身。可是外头的阳光却明灿如金,是一个极好的秋日晴好午后,如懿在窗下榻上和衣养神,听着镂花长窗外乳母哄着永璂玩耍,孩子清脆的笑声,总是让人心神放松,生出几分慵怠之意。

这几日皇帝在前朝忙于准噶尔之事。听闻皇帝命令东归而来的杜尔伯特台吉车移居乌里雅苏台,此事引起了新封的准噶尔亲王,端淑长公主额驸达瓦齐的不满,一怒之下便不肯遣使来京参见,扬言必要车移出乌里雅苏台才肯罢休。

准噶尔部与杜尔伯特部的纷争由来已久。尤其乾隆十八年,达瓦齐为夺多尔札权位,举兵征战,洗劫了杜尔伯特部,夺走了大批牲畜、粮草、财物,还大肆掠走儿童妇女,使杜尔伯特部浩劫空前。车凌身为部落之首,忍无可忍,只得率领一万多部众离开了世居的额尔齐斯河牧坞,动迁归附大清到达乌里雅苏台。皇帝对车凌率万余众倾心来归的行为极为满意,不仅亲自接见了车凌,还特封为亲王。以表嘉奖。为显郑重,皇帝特命四阿哥永珹和五阿哥永琪筹备接风的礼仪,以表对车凌来归的喜悦之心。

这一来,永珹自然在前朝备受瞩目,连着金玉妍亦在后宫十分得脸。嫔妃们虽不敢公然当着如懿的面趋奉玉妍,然后私下迎来送往,启祥宫的门槛也险险被踏烂了。甚至连多年不曾侍寝承宠的海兰,因着永琪的面子,也常常有位分低微的嫔妃们陪着奉承说话。

如懿只作不知,亦不需翊坤宫中宫人闲话,只自取了清净度日。

阳光曛暖,连御园芳渚上的闲鹤也伴着沙暖成双成对交颈而眠,寝殿前的拾花垂珠帘帐安静低垂,散出淡白色的熠熠柔光,一晃,又一晃,让人直欲睡去。正睡意朦胧间,却听三宝进来悄悄站在了身边。如懿听得动静,亦懒怠睁眼,只慵倦道:``什么事?''

三宝的身影映在海棠春睡销金帐上,随着风动隐隐摇曳不定,仿佛同他的语气一般,有一丝难掩的焦灼:``愉妃小主急着求见娘娘,听说是五阿哥受了皇上的叱责,不大好呢。''

如懿豁然睁开眼眸,睡意全消,心中却本能地不信:``永琪素来行事妥当,怎会突然受皇上叱责?''

三宝喏喏道:``这个奴才也不知了。''

如懿即刻坐起,沉声唤道:``容珮,伺候本宫梳洗更衣。三宝,请愉妃进来,暖阁稍候。''

如懿见到海兰时不禁吓了一跳,海兰向来是安静如鸢尾的女子,是深海蓝色般的静致,花开自芬芳,花落亦不悲伤。如懿与她相识相伴多年,何曾见过她这般惊慌失措的样子,汹涌的眼泪冲刷了脂粉的痕迹,更显悲苦之色,而素净的装扮,让她更像是一位无助的母亲,而非一个久居深宫的得体妇人。海兰一见如懿便双膝一软跪了下去,凄然道;``皇后娘娘,求您救救永琪!''

如懿见她如此,不免有些不安,忙携了海兰的手起来,问道:``究竟出了什么事?''

不问则已,一问之下海兰的泪水更是如秋洪奔泻:``皇后娘娘,永琪受了皇上的叱责\ldots{}''一话未完,她哭得更厉害了。

如懿见不得她这般哭泣,蹙眉道:``哪有儿子不受父亲叱责的,当时宠坏了的孩子么?''她摘下纽子上的水色绢子,替她擦拭泪水,``好好说便是。''

海兰极力忍了泪道:``皇上命永珹和永琪对杜尔伯特部亲王车凌郑重相待,两个孩子固然是极尽礼数,不肯懈怠,但永琪那孩子就是年轻,说话不知轻重,不好好跟着永珹学事便也罢了,居然私下说了句`皇阿玛这般厚待车凌,是要将端淑姑姑的夫君放在何地呢?达瓦齐尚不足惜,但也要顾及端淑姑母的颜面啊!'''

如懿心中一沉,倒吸了一口凉气:``永琪说者无心,可是居然被有心人听了去,告诉了皇上是么?而且这个有心人还是他的好兄长永珹对不对?''

海兰哭得哽咽,只是一味点头,半响才道:``永珹也是当玩笑话说给皇上听,小孩子能懂什么?可是皇上\ldots{}''她忍不住又要哭,但见如懿盯着她,只好攥着绢子抹去泪水,``皇上听了大为生气,说永琪心中只有家事,而无国事;只有亲眷,没有君臣!永琪哪里听过这样重的训斥,当下就向皇上请罪,皇上罚他在御书房跪了一个时辰,才叫赶了出来,再不许他理杜尔伯特部亲王之事!''好好调教些时日,教会他如何管好自己的舌头,

如懿的面色越来越阴沉,与她温和的声线并不相符:``不许理便不许理吧。把永琪带回来,好好调教些时日,教会他如何管好自己的舌头,不要在人前人后落下把柄。否则,这次受的是训斥,下次便不知道是什么了。''

海兰悲泣不已,如被雨水种种拍打的花朵,低下了细弱的茎叶:``娘娘与臣妾这么多年悉心调教,竟也让永琪落了个不许理事、备受训斥的地步。臣妾想想真是伤心,这些年来,受过皇上训斥的皇子,哪一个是有好下场的?大阿哥抱憾而死,三阿哥郁郁寡欢,如今竟也轮到臣妾的永琪了。''

檐下的秋风贴着地面打着旋儿冰冷地拂上裙角,如懿盯着海兰,以沉静的目光安抚她慌乱失措的神情。她的声线并不高,却有着让人安定的力量,道:``海兰,你觉得咱们悉心教出来的孩子,会不会说这样昏聩悖乱的话?''

海兰愣了愣,含泪摇头:``不会。永琪是个好孩子,臣妾不信他会忤逆君父,他只是无心而已。''

``是啊,勇气是咱们费了心血教出来的好孩子。可是\ldots{}''如懿的目光渐次凉下去,失了原有温和、慈爱的温度,``他若的确说出了这样的话,咱们也没有法子。''

如懿看了一眼跪在地上哭得妆容凌乱的海兰,转过身,语气淡漠如霜雪:``容珮,扶愉妃回宫。她的儿子失了分寸,她可别再失了分寸叫皇上厌弃了。''

海兰望着如懿的背影被一重重掀起又放下的珠帘淹没,无声地张了张嘴,伤心地伏倒在地。

此后,永琪便沉寂了下去,连着海兰的延禧宫也再无人踏足。落在任何人眼中,失去皇帝欢心的永琪都如一枚弃子,无人问津。哪怕宫人们暗地里议论起来,也觉得永琪的未来并不会比苏绿筠郁郁不得志的三阿哥永璋更好。更甚的是,海兰的身份远不及身为贵妃的绿筠高贵,更不及她膝下多子,所以永琪最好的出路,也不过是如早死的大阿哥永璜一般了。

人情如逐渐寒冷的天气,逼迫着海兰母子。永琪不愿见人,海兰便也紧闭了宫门。在人前也愈加不肯多言一句,两人只关起门来安静度日。

偶尔皇帝问起一句:``皇后,永琪到底也是养在你名下的孩子。朕虽然生气,你也不为他求情?''

如懿安安静静地服侍皇帝穿好上朝穿的袍服,以平静如秋水的眉目相对:``皇上叱责永琪,必然有要叱责他的道理。臣妾身为嫡母,不能管教好永琪已然是失责,如何还敢觍着颜面为他求情?''

皇帝满意地颔首:``皇后能如此公正,不偏不倚就好。''他挽过如懿的手,``上朝还早,朕很想再看看永璂。如懿,你陪朕去。''

二人言笑晏晏,再不提及永琪。而与永琪的落寞相比,永珹更显得一枝独秀,占尽了风光。

因着准噶尔亲王达瓦齐未遣使来京,皇帝并不曾顾及这个妹夫的颜面,反而待车凌愈加隆重。永珹更是进言,不必对达瓦齐假以颜色。因而到了十一月,皇帝便下谕暂停与准噶尔的贸易。

而更令永珹蒸蒸日上被皇帝援以为臂膀的,是轰动一时的江西生员刘震宇案。彼时江西生员刘震宇以所著《治平新策》中有``更易衣服制度''等语被人告发,引来皇帝勃然震怒。

那一日,如懿正抱着璟兕陪伴在皇帝身侧,见皇帝勃然大怒,将《治平新策》抛掷于地,便道:``皇上何必这样生气,区区小事,交给孩子们处置便是了,生气只会伤了龙体啊。''

皇帝凝眸道:``你的意思是\ldots{}''

如懿拍着璟兕,笑容轻柔恬静:``永璋和永珹都长大了,足以为皇上分忧。这个时候,不是两位阿哥正候在殿外要向皇上请安么,皇上大可听听两个孩子是什么主张,合不合皇上的心意,再做决断也不迟啊。''

皇帝沉吟片刻,便嘱咐李玉唤了两位阿哥入殿,如懿只道``妇人不得干政'',抱了璟兕便转入内殿。

京城进入了漫长的秋冬季节,连风沙也渐渐强烈。空气里永远浸淫着干燥的风尘气息,失去了潮湿而缱绻的温度,唯有大朵大朵的菊花抱香枝头,极尽怒放,开得欲生欲死。

如懿闲来无事,抱着璟兕轻轻哼唱不已。

那是张养浩的一段双调《庆东原》,南府戏班的歌伎娓娓唱来,甚合她的心意,那词曲记得分明。

``人羡麒麟画,知他谁是谁?想这虚名声到底原无益。用了无穷的气力,使了无穷的见识,费了无限的心机。几个得全身,都不如醉了重还醉。''

如懿轻轻哼唱,引得璟兕咯咯笑个不已。外头风声簌簌,引来书房里的言语一字一字清晰入耳。

是三阿哥永璋唯唯诺诺的声音:``儿臣不知,但凭皇阿玛做主。''

皇帝的声音便有些不悦:``朕问你,难道你自己连主张也没有么?''

如懿想也想得到永璋谨慎的模样,必定被逼出了一头冷汗。那边厢永璋正字斟句酌道:``儿臣以为,刘震宇通篇也只有这几句不敬之语,且江南文人的诗书,自圣祖康熙、世宗雍正以来,都颇受严苛,若皇阿玛能从轻发落,江南士子必定感念皇阿玛厚恩。''

有良久的沉默,却是四阿哥永珹的声音打破了这略显诡异的安静。他的声音朗朗的,比之永璋,中气颇足:``皇阿玛,儿臣以为三哥的主意过于宽纵了。自我大清入关以来,江南士子最不驯服,屡屡以诗书文字冒犯天威,屡教不改。从圣祖到世宗都对此严加惩处,绝不轻纵。皇阿玛与儿子都是列祖列宗的贤孝子孙,必定仰承祖训,绝不宽宥!''

皇帝的声音听不出半分喜怒,甚是宁和:``那么永珹,你作何打算?''

永珹的回答斩钉截铁,没有半分柔和的意度:``刘震宇竟敢言`更易衣服制度',实乃悖逆妄言,非死不能谢罪于大清。''

永璋似乎有怜悯之意,求道:``皇阿玛,今年浙江上虞人丁文彬因衍圣公孔昭焕揭发其制造逆书,刑部审实,皇阿玛已下令行磔刑,将其车裂,还株连甚广,闹得文人们人心惶惶,终日难安,不敢写诗作文。此次的事,皇阿玛何不恩威并济,稍稍宽恕,也好让士子文人们感念皇阿玛的恩德。''

永珹哼了一声道:``三哥这话便错了,越是宽纵,他们越是不知天高地厚,何曾感激皇恩浩荡,反倒越发放肆了!否则这样的事怎么会屡禁不止?昔年我大清入关,第一条便是`留头不留发,留发不留头'。连陈名夏这样为顺治爷所器重的汉臣,因说了一句`若要天下安,复发留衣冠'的大逆之言,就被顺治爷处以绞刑。皇阿玛圣明,自然不会放过了那些大逆不道的贼子!''

皇帝的沉默只有须臾,变化为一字一字的冷冽:``刘震宇自其祖父以来受我大清恩泽已百余年,且身受礼教,不是无知愚民,竟敢如此狂诞,居心实在悖逆。查刘震宇妄议国家定制,即日处斩。告知府县,书版销毁。这件事,永珹,便交予你去办了。''

皇帝的言语没有丝毫容情之处,如懿听在耳中,颇为惊心。然而永珹得意的笑声更是声声入耳。``儿臣一定会极力督办,请皇阿玛放心。''

情歌悠扬,如懿自知嗓音不如嬿婉的悠扬甜美,声声动人,可是此时金波潋滟浮银瓮,翠袖殷勤捧玉钟。对一缕绿杨烟,看一弯梨花月,卧一枕海棠风。手指轻叩,悠扬之曲娓娓溢出,深吸一口清冽的空气,淡淡菊香散尽,幽怀袅袅。

``晁错原无罪,和衣东市中,利和名爱把人般弄。付能元刂刻成些事功,却又早遭逢著祸凶。''

如懿心念微动,含了一抹沉稳笑意,抱紧怀中的孩子。

离去时已是夜深时分,唯有李玉带着十数小太监迎候在外。趁着李玉扶上辇轿的时候,如懿低声道;``多谢你,才有今日的永珹。''

李玉笑得恭敬:``奴才只是讨好主子罢了,四阿哥为皇上所喜,奴才自然会提醒四阿哥怎样讨皇上喜欢。奴才也只是提醒而已,什么舌头说什么话,全在四阿哥自己。来日成也好,败也罢,可不干奴才的事。''

如懿笑道;``他的事,自然与咱们是无碍的。''

二人相视一笑,彼此俱是了然。如懿抬首望月,只见玉蟾空明澹澹,心下更是澄明一片。

京城的四季泾渭分明,春暖秋凉,夏暑冬寒,就好比紫禁城中的跟红顶白,唯有城中人才能冷暖自知。半余年来,如懿固然因为一双子女颇得皇帝恩幸,地位稳固如旧。而金玉妍也甚得宫人奉承,只因四阿哥永珹得到皇帝的重视。而曾经与永珹一般得皇帝青眼的五阿哥永琪,却如昙花一现,归于沉寂。

待到乾隆十九年的夏天缓缓到来时,已然有一种说法甚嚣尘上,那便是嘉贵妃金玉妍的四阿哥永珹有继承宗兆之像,即将登临太子之位。

这样的话自然不会是空穴来风,而皇帝对永珹的种种殊宠,更像是印证了这一虚无缥缈的传言。

四月,和敬公主之夫,额驸色布腾巴勒珠尔腾入觐,皇帝欣喜不已,命大学士傅恒与永珹至张家口迎接,封额驸为贝勒。

五月,准噶尔内乱,皇帝命两路进兵取伊犁,又让三阿哥永璋与四阿哥永珹同在兵部演习军务。然而明眼人都看得出,皇帝只问永珹军事之道,并请尚书房师傅教导兵书,而对永璋,不过尔尔。

到了八月,皇帝驻跸吉林,诣温德亨山望祭长白山、松花江。赈齐齐哈尔三城水灾,阅辉发城。除了带着如懿与嫡子永璂,便是永珹作陪。九月间,又是永珹随皇帝谒永陵、昭陵、福陵。

荣宠之盛,连朝中诸臣也对这位少年皇子十分趋奉,处处礼敬有加,恰如半个太子般看待。

而内宫之中,皇帝虽然宠幸如懿与嬿婉、颖嫔、忻嫔等人居多,对年长的玉妍的召幸日益稀少,却也常去坐坐,或命陪侍用膳,或是赏赐众多。比之绿筠的位高而恩稀,玉妍也算是宠遇不衰了。

绿筠人前虽不言语,到了如懿面前却忍不住愁眉坐叹:``臣妾如今年长,有时候想起当年抚养过永璜,母子一场,眼前总是浮起他英年早逝的样子。如今臣妾也不敢求别的了,只求永璋能安安稳稳地度日,别如他大哥一般便是万幸了。''

如懿捧着一盏江南新贡的龙井细细品味,闻言不由得惊诧:``永璋虽然受皇上的训斥,那也是孝贤皇后过世那年的事了。怎么如今好好的,你又说起那般丧气话来?''

绿筠忍不住叹息道:``臣妾自知年老色衰,自从永璜和永璋被皇上叱责冷待之后,臣妾便落了个教子不善的罪名,不得皇上爱幸。臣妾只求母子平安度日。可是皇后娘娘不知,嘉贵妃每每见了臣妾冷嘲热讽之外,永璋和永珹一起当差,竟也要看永珹脸色,受他言语奚落。我们母子,居然可怜到这个地步了。也怪臣妾当年糊涂,想让永璋争一争太子之位,才落得今日。''她越说越伤心,跪下哭求道,``臣妾知错了,臣妾只希望从此能过得安生些,还求皇后娘娘保全。''

绿筠处境尴尬,如懿不是不知。三阿哥永璋一直不得皇帝青眼,以致庸碌。绿筠所生的四公主璟妍虽然得皇帝喜爱,但到底是庶出之女。而六阿哥永瑢才十一岁,皇帝幼子众多,也不甚放在心上。绿筠虽然与玉妍年岁相差不多,却不及玉妍善于保养,争奇斗妍,又懂得邀宠,自然是过得不尽如人意了。

如懿见绿筠如此,念及当年在潜邸中的情分,且永璜和永璋被牵累的事多少有自己的缘故在,也不免触动心肠,挽起她道:``这话便是言重了,皇上不是不顾念旧情的人,嘉贵妃的性子你又不是不知道,有什么额娘就有什么儿子,一时得意过头也是有的。永璋如今是皇上的长子,以后封爵开府,有你们的安稳荣华呢。''

绿筠闻言稍稍安慰,抹泪道:``有皇后娘娘这句话臣妾便安心了。说来臣妾哪里就到了哭哭啼啼的时候呢,愉妃妹妹和永琪岂不更可怜?''

话音未落,却见李玉进来,见了绿筠便是一个大礼,满脸堆笑:``原来纯贵妃娘娘在这儿,叫奴才好找!''

绿筠颇为诧异,也不知出了何事,便有些慌张:``怎么了?是不是永璋哪里不好,又叫皇上训责了?''

李玉喜滋滋道:``这是哪儿的话呀!恭喜纯贵妃娘娘,今日皇上翻了您的牌子,且会到钟粹宫与您一同进膳,您赶紧准备着伺候吧。''

绿筠吃了一惊,像是久久不能相信。她的笑容僵在了脸上,摸了脸又去摸衣裳,喜得实在不知该怎么办才好,念念道:``本宫多少年没侍寝了,皇上今儿怎么想起本宫来了?''

李玉笑道:``贵妃娘娘忘了,进而使您当年入潜邸的日子呀!皇上可惦记着呢。''

绿筠这一喜可非同小可,呆坐着落下泪来,喃喃自语:``皇上可记得,本宫自己都忘了,皇上居然还记得!''

如懿笑着推了她一把:``这是大喜的事,可见皇上念着您的旧情,怎么还要哭呢?''她心念电转,忽地想起一事,唤过容珮道:``去把嘉贵妃昨日进献给本宫的项圈拿来。''

那原是一方极华美的赤金盘五凤朝阳牡丹项圈,以黄金屈曲成凤凰昂首之行,其上缀以明珠美玉,花式繁丽,并以红宝翡翠伏成牡丹花枝,晶莹辉耀。

如懿亲自将项圈交至绿筠手中,推心置腹道:``这个项圈足够耀眼,衣衫首饰不必再过于华丽,以免喧宾夺主,失了你本真之美。''她特特提了一句,``这样好的东西本宫也没有,还是嘉贵妃孝敬的。也罢,借花献佛,添一添你今夜的喜气吧。''

\hypertarget{ux7b2cux4e09ux7ae0-ux8336ux5fc3}{%
\chapter{第三章 茶心}\label{ux7b2cux4e09ux7ae0-ux8336ux5fc3}}

绿筠喜不自胜,再三谢过,忙忙赶着回去了。

容珮见绿筠走远了,疑惑道;``昨日嘉贵妃送这个项圈来,名为孝敬娘娘,实是炫耀她所有是娘娘没有的。''她鄙夷道:``这样的好东西,凤凰与牡丹的首饰,嘉贵妃也配!''

如懿缓缓抚摸雪白领子上垂下的珍珠璎珞:``凤凰与牡丹,原本该是中宫所有。可是本宫没有的东西,嘉贵妃却能随手拿出,你说是为什么呢?''

如懿不待容珮应答,举眸处见永琪携了一卷书卷入内,不觉便含笑。如懿注目于他,这些日子的萧索并未为他爽朗清举的容止染上半分憔悴,反而添了岩岩若孤松之独立朗然。

如懿心下欣慰,忙招了招手,亲切道:``拿着什么?给皇额娘瞧瞧。''

永琪见了如懿,便收了一脸颓丧怯弱之色,爽朗一笑,将书卷递到如懿跟前,兴奋道;``儿臣自己编的书,叫《蕉桐膡稿》,虽然才编了一点儿,但总想着给皇额娘瞧瞧。''

如懿的手指翻过雪白的书页,笑道:``你自己喜欢,便是最好的。自己找些喜欢的事做,也省得听了旁人的闲言闲语。''

永琪微微有些黯然:``儿臣倒还好,只是不能为额娘争气,让额娘伤心了。今儿早起见额娘又在烦心,儿臣问了两句,才知是额娘母家的几个远亲又变着法子来要钱。额娘虽然身在妃位,但一向无宠,但凡有些赏赐和月银也都用在了儿臣身上,哪里禁得住他们磨盘儿似的要。但若回绝,人家又在背后恶言恶语。好容易搜罗了些首饰送出去,他们又像见了血的苍蝇,纷至沓来。''

如懿听得蹙眉:``谁家没有几个恶亲戚,你叫你额娘不用理会就是。也是的,这些事你额娘都不曾告诉本宫。''

永琪黯然摇头:``家丑不可外扬,额娘也是要脸面的人,所以不曾说起。连儿臣都是反复追问才知道些。额娘提起就要伤心,总说家世寒微帮不上儿臣,才生出这许多烦恼。''

``愉妃只有你一个儿子,操心是难免的。''如懿淡然一笑,温和道:``只要有来日,一时的委屈都不算什么。''

永琪用力点了点头:``皇额娘的教诲,儿臣都记住了。''

如懿颔首道;``外人都说你是闲来无聊丧了心智,才以编书为寄托,还整日闭门不出,出门也不多话。告诉皇额娘,除了编书,平日还做些什么?''

永琪认真道:``写字。皇额娘告诉过儿臣,写字能静心。''

如懿温然一笑,和煦如初阳:``无事时戒一偷字,有事时戒一乱字。你能这样,便是最好。对了,你额娘如何?还这么为你哭哭啼啼么?''

永琪道:``已经好多了。儿臣安静,额娘自然也不会心乱。''

如懿稍稍放心:``你额娘久在深宫,这些分寸总还是有的。''

永琪思忖片刻,有些不忿道:``只是今日儿臣路上过来,见四哥好不威风,去启祥宫向佳贵妃娘娘请安,也带了好些随从,煊煊赫赫,见了儿臣又嘲讽了几句。''

如懿浅浅含笑,以温煦的目光注视着他:``这半年来,永珹见了你,不都爱逞些口舌的功夫么。你忍他了么?''

永琪低头:``是。儿臣都会忍耐。''

如懿笑而不语,闲闲地拨弄着手中的白玉透雕茶盏,浅碧色的茶汤蒸腾着雪白的水汽,将她的容颜掩得润泽而朦胧。如懿倒了一盏清茶,递与永琪手中:``尝一尝这龙井,如何?''

永琪不解其意,喝了一口道;``甚好。''

如懿徐徐道:``龙井好茶,入口固然上佳。但皇额娘喜欢一种茶,不仅要茶香袭人,更要名字清雅贴切,才配得入口。譬如这道龙井,额娘觉得用来比喻你此时此刻的处境最是恰当。''

永琪不解地皱了皱眉,恭敬道:``儿臣不懂,洗耳恭听。''

如懿看着盏中杏绿汤色,映得白玉茶盏绰然生碧,恍若一方凝翠盈盈:``如今的你,好比龙困井中,该当如何?''

永琪眉峰一扬,眼中闪过一道流星般的光彩,旋即低首一脸沉稳:``是龙,便不会长困于井中。一时忍耐,只待时飞。''

如懿为他添上茶水,神色慈爱:``龙井味醇香郁,入口齿颊生香。但好茶不仅于此,更可以清心也,皇额娘希望你可以潜心静气,以图来日。''

盏中茶叶在水中一芽一叶舒展开来,细嫩成朵,香馥若兰,如同永琪舒展的笑容。``皇额娘的苦心,儿臣一定细细品味。''他想了想又说,``儿臣听说四哥结交群臣,场面上的应付极大。每每李朝进献人参,或黄玉、红玉等各色玉石,四哥都分送群臣府中,连各府女眷也得到李朝所产的虹缎为佳礼。''

``李朝的虹缎素以色泽艳丽、织物经密而闻名,常以锦绣江山、秀丽景致映在彩虹上,再将彩虹七色染在缎子上。李朝人力、物力不足,这虹缎极费工夫,实在难得,也难为了永珹这般出手大方。''如懿微微一笑,眸中神色仿若结冰的湖面,丝毫不见波澜,``你的心思本宫都明白。只是这样的话不比你亲自去告诉你皇阿玛,自然会有人去说。你要做的无非是让人多添些口舌便是。口舌多了,是非自然也就多了。''

永琪心领神会:``皇额娘嘱咐的事,儿臣都会尽力做到最好。''

如懿轻轻握住他的手,细心地抚平半旧的青线云纹袖口间稀皱的痕迹:``皇额娘知道你这大半年来过得不好。但,你若忍不得一时,便盼不得一世。会很快了。''

永琪郑重颔首,眸中唯余一片墨色深沉的老练沉稳。

隔了几日便有消息传来,乃是皇帝的一道谕旨,下令朝中官员不得与诸皇子来往。

这道谕旨来得甚是蹊跷,然而明眼人都明白,三阿哥永璋和五阿哥永琪被冷落,其余皇子都还年幼,能与朝中官员往来的,不外是风头正盛的四阿哥永珹。

李玉来时,见如懿兴致颇好,正抱着璟兕赏玩青花大缸中的锦鲤。廊下养着时鲜花卉,檐下养着的红嘴相思鸟啁啾啼啭,交颈缠绵,好不可人。

因天气暑热,如懿又喜莲花,皇帝特意命人在庭院里放置了数个青花大缸,养着金色锦鲤与巴掌大的碗莲。缸中红白二色的碗莲开了两三朵浮在水面,游鱼穿梭摇曳,引逗得如懿和几个宫女倚着栏杆,坐在青绸宝莲绣墩上拿了鱼食抛喂嬉笑。

如懿看璟兕笑得开怀,便将她交到了乳母怀里,因着去逗弄鸟儿,方才道;``皇上怎么突然下了这样的旨意?也不怕伤了永珹的面子。''

``面子是自己给自己的,若要旁人来给,那都是虚的。''李玉一笑,``前几日皇上陪伴纯贵妃,见她戴着的项圈夺目,便问了句来历,纯贵妃便老实说了。这样规制的项圈难得,奴才记得两广总督福臻所进献的礼物里便有这一样,只是不知怎么到了嘉贵妃手里,便如实回禀了。''

``你这般回禀,皇上当然会疑心去查,是不是?''如懿掐了几朵新鲜玉簪在手中,留得一手余香。

李玉道:``皇上要查的,自然会雷厉风行查得明明白白。四阿哥结交群臣之事早已流言如沸,如今不过是在适当的时候让皇上的耳朵听见而已。更何况四阿哥敢从两广总督处收受如此贵重的礼物赠予嘉贵妃,如此内外勾结,皇上哪有不忌讳的。''

``听说封疆大吏们争相结交四阿哥,送礼予他,可是总还是有明白人的吧?本宫听说忻嫔的阿玛那苏图便不是这样随波逐流的人。''

李玉低眉顺目:``可不是么?所以皇上连带着对忻嫔都格外恩宠有加,这两日都是忻嫔侍寝。''

如懿随手将玉簪花簪上丰厚漆黑的新月髻:``虽然有这样的旨意,但皇上还是重视四阿哥的,不是么?''

李玉的目光透着深邃之意:``皇上是重视四阿哥。可五阿哥自被皇上叱责冷落之后,反而得了皇太后的青眼。塞翁失马,焉知非福啊!''

如懿微微垂头,细细理顺胸前的翡翠蝴蝶流苏。一节湖水色绣青白玉兰的罗纱袖子如流水滑落,凝脂皓腕上的紫玉手镯琳琅有声;``不管怎么说,木兰围场救父的功劳,四阿哥可是拔得头筹啊!''

李玉笑得高深:``皇上喜爱四阿哥是不假,木兰围场救父的功劳也是真。可是那日救皇上的,不止四阿哥,还有五阿哥和凌大人,咱们可是有目共睹的。至于是不是头筹\ldots{}''他话锋一转,``奴才当时一在,得问问在场的人才好。''

如懿笑着剜了李玉一眼:``越发一副老狐狸的样子了。人呢?''

李玉躬身笑道:``凌大人早已候在宫外,只等娘娘传见。''

如懿摇了摇手中的团扇,懒懒道:``外头怪热的,请凌大人入殿相见吧。容珮,凌大人喜欢的大红袍备下了么?''

容珮含笑道:``早备下了。''

凌云彻疾步入殿。他立在如懿跟前,被疏密有致的窗格滤得明媚温淡的阳光覆过他的眉眼。一身纱质官服透着光线浮起流水般光泽,整个人亦失了几分平日的英武,多了几分温润之意。

如懿不知怎的,在凝神的一瞬想起的是皇帝的面容。多少年的朝夕相对,红袖相伴,她记忆力骤然能想起的,依然是初见时皇帝月光般清澈皎洁的容颜。时光荏苒,为他添上的是天家的贵胄气度,亦是浮华的浸淫,带上了奢靡的风流气息。如今的皇帝,虽然年过四十,英姿不减,依旧有着夺目的光华,但更像是一块金镶玉,固然放置于锦绣彩盒之内,饰以珠珞华彩,但早已失却了那种摄人心魄的清洁之姿。更让人觉得太过易碎,不可依靠。

而眼前的凌云彻,却有着风下松的青翠之姿,生与草木,却独立丛中,可为人蔽一时风雨。

这样的念头尚未转完,凌云彻已然躬身行礼。他礼敬而不带讨好的意味,凛然有别与众人。

如懿十分客气,示意他起身,看着容珮奉上茶来,又命赐座。

橙滟滟的茶水如朝霞流映,如懿示意他喝一口,柔缓道:``这大红袍是道好茶,红袍加身,本宫在这里先恭喜凌大人升官之喜了。''

茶香还留在口颊之内,凌云彻不觉诧异道;``奴才在皇上身边侍奉,何来突然升官之喜?''

如懿的眉眼清冽如艳阳下的水波澹澹,说得十分坦然:``凌大人能再度回宫,凭的是木兰围场勇救皇上的忠心。只是与其三人分享功劳,不如凌大人独占其功,如此岂非没有升官之喜?''

凌云彻眼中有一片清明的懂得:``微臣如何敢独占其功,那日木兰围场之事,明明是五阿哥冒险救父,挡在皇上身前,功劳最大。微臣不过是偶然经过而已。''

如懿轻叹如风:``冒险救父的是永珹,若不是他放箭射杀受惊的野马,皇上也不能得保万全。说到底,永琪不过是个最痴傻的孩子,只会挡在皇上身前以身犯险罢了。''

凌云彻道:``以身犯险舍出自己才是最大的孝心。背后放箭,说得好是救人,若放的是冷箭,或许也是伤人了。''

如懿忽然目光一凝,冷然道:``凌大人,虽然本宫当日未在木兰围场的林中,但一直有些疑惑。皇上遇险,怎么凌大人和永珹、永琪便会那么巧就出现救了皇上?''

如懿敛声注目于凌云彻,似要从他脸上寻出一丝半痕的破绽,然而承接她目光的,唯有些许讶异与一片坦诚。凌云彻拱手道:``皇上洪福齐天,也是上天垂恩,给微臣与两位阿哥这样救护皇上的机会罢了。''

他的淡定原在如懿意料之中,却不想如此无懈可击。如懿暗笑,她也不过是在疑心之余略作试探而已,时过境迁,许多事已无法再彻查。而凌云彻的表情,给了她的揣测一个阻绝的可能。

如懿盈然一笑,神色瞬间松快,和悦如暖风醺然:``凌大人无须急着辩解。本宫此言,不过是长久以来的一个疑问而已。自然了,永琪当年不过十二岁,能救护皇上也是机缘巧合而已。只是\ldots{}''她略略沉吟,``自从围场之事后,这两年皇上每每去木兰秋狝,都要格外加派人手跟随,总不能畅快狩猎,也颇束手束脚。且当年暗中安置弓弩放冷箭之人一直未曾查明,到底也是一块心病。连本宫也日夜担忧,生怕再有人会对皇上不利。凌大人时时追随皇上身边,有这样的阴狠之人潜伏暗中,只怕大人也要悬心吧?''

凌云彻的目光如同被风扑到的烛火微微一跳,旋即安稳如常:``当日皇上说过一句话,微臣铭记于心。皇上说:`忠于朕的人都来救朕了!害朕的人,此时一定躲得最远!'\,''

如懿的语气隐然有了一丝迫人的意味:``本宫倒是觉得,有时候救人的人,也会是害人的那人。凌大人认为呢?''

凌云彻起身,一揖到底,以一漾温和目色相对:``娘娘说得是。当日微臣细察过,那两支暗箭都不曾喂毒。若皇上在原地不动,应当只是虚惊一场。''

``是么?''如懿目光澄明,如清朗雪光拂过于他,``那么凌大人,那日,你做了什么?''

凌云彻一滞,眸光低回而避,额上已生出薄薄汗珠。片刻,他决然抬首:``皇后娘娘,当日微臣牵颖嫔娘娘的爱驹在外遛马,曾先入林中,发现架于树枝间的弓弩。''

如懿疑惑道:``本宫记得那时查明,那弓弩并非需要有人当场施放冷箭,而是架在树枝间以银丝绷住。只要银丝一受触碰断裂,冷箭自会发出。''

``是。因树林偏僻,少有人来,所以微臣只是好奇,因而掩在树后观望。谁想皇上起兴追马至林间,枝上弓弩便发,骇然眼见变生肘腋。且当日那野马骤然闯入林间,也是因为草木间涂上了发情母马的体液,才引得野马奔来躁动。围场官员也有说是有人备下弓弩只为射杀野马。''

如懿道:``凌大人不觉得这话是推脱之词么?难怪皇上之后震怒,要严惩木兰围场的官员。依本宫看,只怕真是有人费尽心机要暗害皇上,借以自重。''

凌云彻将肺腑之音尽数吐出:``今日皇后娘娘既然疑心,那微臣一定细细查访。只要是皇后娘娘吩咐的,微臣都会尽力去做,尽心去做,以还娘娘一个明白交代。''

如懿涂了胭脂红蔻丹的指甲映在白玉茶盏上,莹然生辉。她轻抿茶水,柔声道:``本宫何曾吩咐过你什么,一切皆在大人自己。''

午后的日光被重重湘妃竹帘滤去酷热的意味,显得格外清凉。凌云彻有一瞬的怔忡,望着眼前的女子,梨花般淡淡的妆容,隐约有兰麝逸香,那双水波潋滟的命眸似乎比从前多出一丝温柔,是那种难得而珍贵的温柔。似乎是对着他,亦像是对着她所期许的未来。她秀长的眉眼总是隐着浅淡的笑意,那笑意却是一种惯常的颜色,像是固有的习惯,只是笑而已,却让人无法捉摸到底是喜是怒。

他在自己怔忡醒来的须臾,有一个念头直逼入心,若她的笑是真心欢喜便好。

凌云彻黯然躬身,徐徐告退,走出重重花影掩映的翊坤宫。有带着暑热的风灌入衣衫的缝隙,他只觉得凉意透背,才知冷汗已湿透了一身。举手抬目,凌云彻望见一片湛蓝如璧的天色,仿佛一块上好的琉璃脆,通透澄明。恰有雪白的群鸟盘旋低鸣,振翅而过。

他的心在此刻分明而了然,若不为她,亦要为了自己。千辛万苦走到这里,岂可便宜了旁人,都得是自己的,是她的才好。

如懿看着凌云彻离去,面上不觉衔了一丝温然笑意:``容珮,着大红袍还有多少?''

容珮答道:``这大红袍是今春福建的贡品,咱们吃了小半年,还有五六斤吧。''

如懿笑道:``那便尽数留着给凌大人,贺他来日升迁之喜。''

容珮取过一把翠绿黄边流苏芭蕉扇,一下一下扇出清凉的风:``娘娘便这般笃定,凌大人一定会有这样的大喜?''

如懿睫毛轻轻扬起,便如蝶翼扑扇,露出深幽如水的眼波:``不是大喜,便是大悲,他没有选择。''如懿牵动湘妃竹帘上的五色丝线流苏,半卷轻帘。一眼望去,庭院中错错落落开着芍药、龙胆、合欢、茑萝、凤仙、石榴、木香、紫薇、惠兰、长春、笑靥、月季、百日红、千叶桃、玉绣球、飞燕草,红红翠翠,缤纷绚烂,如堆出一天一地的繁华金色。彼时荷钱正铸,榴火欲燃,迎着雕梁燕语,绮槛莺啼,静院明轩,溶溶泄泄。谁会想到这般气序清和、昼长人倦的天地里,会有着让人心神难安的来日。

容珮眸光一转,已然猜到几分:``娘娘是说\ldots{}''

``虽然已经过了两年,但皇上并未真正放下木兰围场遇险之事。你只瞧每年再去承德,皇上布下的人手这样多,便知道没有查出放冷箭的真凶,是如何让皇上寝食难安。''

容珮吃惊:``娘娘是怀疑救驾的人中有人自己安排了这一出?''

如懿眼波中并无一丝涟漪:``本宫也只是疑心而已。凌云彻有没有这样的心思和举动本宫无处查知,但是方才试探他几句,他倒沉得住气。能这样沉得住气的人,便不会自己引火烧身。而永珹,本宫实在不能不疑心。''

容珮着实不安,一把芭蕉扇握在手中,不觉停了扇动:``几年来四阿哥母子是有不少举动,那娘娘不告诉皇上?''

``告诉皇上?''如懿凝眸看她,``如果皇上问起,为何本宫不早早说出这疑心,而是等永琪寥落之时再提,是否有庇护永琪攻讦永珹之心,本宫该如何作答?或者皇上又问,本宫若是疑心,为何不早说,让凌云彻这般有嫌疑之人长在皇上身侧,又是何居心,本宫又该如何作答?此时本宫并未眼见,只是耳闻才有疑惑,并无如山铁证啊!''

容珮慨叹道:``如此,娘娘的确是两难了。可是这件事若是凌大人做的,这样一个居心叵测的人在皇上身边,对皇上岂不有害?''

``不会。''如懿看得通透,``他苦心孤诣只是想回到紫禁城中争得属于他的一份荣华富贵。为了这个心愿而布下杀局,他没这个本事,也没这个必要。如今他心愿得偿,更不会有任何不利于皇上的举动,来害了自己辛苦挣来的这份安稳。''她弹了弹水葱似的半透明的指甲,``既然这件事本宫有疑心,那么迟早皇上也有疑心。你不是不知道皇上的性子,最是多疑。等哪日他想起这层缘故来,凌云彻也好,永琪也好,都脱不了嫌疑。与其如此,不如早点儿有个了断。''

容珮轻轻叹息,似有几分不放心。连如懿自己也有些恍惚,为何就这般轻易信了凌云彻,宁可做一个懵懂不知之人。或许,她是真的不喜金玉妍与永珹,宁愿他们落了这个疑影儿;亦或是因为昔年冷宫扶助之情,是他于冰雪中送来一丝春暖。

纱幕微浮,卷帘人去,庭中晴丝袅袅,光影骀荡,远远有昆曲袅娜飞云,穿过宫院高墙,缥缈而来。

那是一本《玉簪记》,也唯有嬿婉缠绵清亮的嗓音唱来,才能这般一曲一折,悠悠入耳,亦入了心肠。

``粉墙花影自重重,帘卷残荷水殿风。抱琴弹向月明中,香袅金猊动。人在蓬莱第几宫?''

午后的阳光有些慵懒,温煦中夹着涩涩而蓬勃的芳香。娜依一夏最后的绚美,连花草亦知秋光将近,带着竭尽全力欲仙欲死的气性,拼力盛放至妖冶。

如懿本与嬿婉心性疏离,此刻听她曲意绵绵,亦不禁和着拍子随声吟唱。

``朱弦声杳恨溶溶,长叹空随几阵风。仙郎何处入帘栊?早是人惊恐。莫不是为听云水声寒一曲中?''

\hypertarget{ux7b2cux56dbux7ae0-ux6728ux5170}{%
\chapter{第四章 木兰}\label{ux7b2cux56dbux7ae0-ux6728ux5170}}

这样阳光曛暖,兰谢竹摇的日子,就在平生浮梦里愈加光影疏疏、春色流转。待到恍然醒神时,已是乳母抱了午睡醒来的永璂来寻她。

儿啼声唤起如懿的人母心肠,才笑觉自己的恍惚来得莫名。如懿伸手抱过扑向她的爱子,听他牙牙学语:``额娘,额娘。''片刻又笑着咧开嘴,``五哥哥,五哥哥。''

永琪一向待这个幼弟十分亲厚,如同胞手足一般,得空儿便会来看他。如懿听永璂呼唤,便唤进三宝问:``五阿哥这两日还不曾来过,去了哪里?''

三宝忙道:``回皇后娘娘的话,五阿哥陪着太后抄录佛经去了。''

如懿哄着怀中的永璂,随口问:``这些日子五阿哥常陪着太后么?''

三宝道:``也不是常常,偶尔而已。太后常常请阿哥们相伴慈宁宫说话,或是抄录佛经。不是五阿哥,便是六阿哥。''

太后喜爱纯贵妃苏绿筠所生之子,众人皆知。不过六阿哥长得虎头虎脑,十分活泼,原也格外招人喜爱。如懿含着欣慰的笑,如今,太后的眼里也看得见别的阿哥了。

如懿问道:``不显眼吧?''

三宝忙压低了声音:``不显眼。愉妃小主和五阿哥都受皇上冷落,没人理会延禧宫的动静。''

容珮怔了怔:``怎么太后如今也看得上五阿哥了?从前因为五阿哥是娘娘名分上的养子,太后可不怎么搭理呢。''

如懿瞟了她一眼:``问话也不动脑子了,你自己琢磨琢磨。''

容珮想了又想,眼神一亮;``哎呀!奴婢懂了。当日五阿哥为端淑长公主思虑,固然是见罪于皇上,却是大大地讨了太后的喜欢!''

如懿轻轻地拍着怀中的永璂,口中道:``端淑长公主是太后的长女,太后虽然不顾及达瓦齐,但端淑长公主的颜面与处境,她总是在意的。皇上善待车凌,达瓦齐大怒,自然也不会给端淑长公主好脸色看了。有永琪这句贴心窝子的话,即便受了皇上的训斥,太后一定也会念着永琪的好的。''

容珮道:``左右这几年在皇上跟前,是哪位阿哥也比不上四阿哥。能另辟蹊径得太后的好,那自然是好。可是太后虽然受皇上孝养,但不理会朝政的事,即便有太后疼爱,便又怎样呢。''

如懿但笑不语,只是看着孩子的笑脸,专注而喜悦。

这便是太后的厉害之处了。她在先帝身边多年,与朝中老臣多是相识,哪里会真的一点儿用处都没有。可她偏偏这般淡然无争,仿佛不理世事。如懿却是清楚的,连皇帝的后宫也少不得有太后的人,而玉妍与永珹只眼看着皇帝,却无视太后,便是目光短浅,大错特错了。

时数日后,木兰围场进献数匹刚驯化的野马养入御苑,供宫中赏玩。皇帝颇为有兴,便携嫔妃皇子前往赏看。金风初起,枫叶初红,烈烈如火。雪白的马匹养在笼中,映着园中红叶,十分好看。都是初到宫中陌生的环境,那些马儿到底野性未驯,并不听驯马师的话,摇头摆尾,不时低嘶几声,用前蹄挠着沙地,似乎很是不安。

马蹄踢铁栏的声音格外刺耳,忻嫔依偎在皇帝身边,脸上带着几分娇怯,一双明眸却闪着无限好奇,笑道:``这些驯马师也真无用!平素驯惯了的畜生也不能让它们安静下来。''她目光清亮,逡巡过皇帝身后数位皇子,笑生两靥,``听说诸位阿哥都善于狩猎,若是野马不受驯,一箭射死便也罢了。是不是?''

永珹虽未受皇帝训斥,然而也感受到皇帝对他的疏远。且这些日子皇帝宠爱忻嫔,并不去玉妍宫里,他难免为额娘抱不平,便朗朗声争强道:``忻娘娘这话便差了,这些马匹驯养不已,若是都一箭射杀了,哪里还有更好玩的供给宫里呢?''

忻嫔本与永珹差不了几岁,也是心性高傲的年纪,有些不服,道:``听四阿哥的意思,是能驯服了这些野马么?''

永珹轻笑一声,也不看她,径自卷起袖子走到笼前,都弄了片刻。谁知那些野马似是十分喜欢永珹,一时也停了烦躁,乖乖低首打了两个响鼻。

玉妍见状,不免得意,扯了扯身边的八阿哥永璇,永璇立刻会意,立刻拍手笑道:``四哥,好厉害!好厉害!''

忻嫔见永珹得意,不屑地撇了撇嘴道;``雕虫小技。哪里及得上皇上驯服四海平定天下的本事!''

皇帝见忻嫔气恼起来一脸小儿女情态,不觉好笑:``永珹,那些野马倒是听你的话!''

此时,凌云彻陪伴皇帝身侧,立刻含笑奉承道:``皇上说得是。每年木兰围场秋狝之时,四阿哥都会亲自喂养围场中所驯养的马匹。正因如此,所以年年秋狝,四阿哥骑术最佳。''

永琪恍然大悟:``难怪四哥去喂围场的马都不带儿臣去,原来竟有这般缘故,怕儿臣夺了四哥的名头呢!''

皇帝悬在嘴角的笑意微微一敛,仿佛不经意道:``凌云彻,你是说四阿哥每年到围场都和这些野马亲近?''

凌云彻的样子极敦厚:``微臣在木兰围场当值两年,都曾眼见。后来随皇上狩猎,也见过几次。''他满眼钦羡之色,``四阿哥天赋异禀,寻常人实难企及。''

皇帝看着铁笼外几位驯马师束手无策,唯独永珹取了甘草喂食马儿,甚是得心应手,眼中不觉多了一位狐疑神色。当下也不多言,只是说笑取乐。

当夜皇帝便不愿召幸别的嫔妃,而是独自来到翊坤宫与如懿相守。红烛摇曳,皇帝睡梦中的神色并不安宁,如懿侧卧他怀中,看他眉心深锁,呓语不断,隐隐心惊,亦不能入梦,只听着夜半小雨淅淅沥沥叩响窗棂。良久,雨声越繁,打在飞檐琉璃瓦上,打在中庭阔大的芭蕉叶上,打在几欲被秋风吹得萎谢的花瓣上,声声清越。

心潮起伏间,又是风露微凉的时节啊。

夜色浓不可破,皇帝从梦中惊坐起,带着满身湿漉漉的冰凉的汗水,疾呼道:``来人!来人!''

即刻有守夜的宫人闻声上前叩门,如懿忙忙坐起身来,按住皇帝的手心,向外道:``没什么事!退下吧!''

九月初的雨夜,已有些微凉,晚风透过霞影绛纱糊的窗微微吹了进来,翡翠银光冷画屏在一双红柱微光下,闪烁着明灭的光。如懿取过床边的氅衣披在皇帝身上,又起身递了一盏热茶在皇帝手中,柔声关切:``皇上又梦魇了么?''

皇帝将盏中的热茶一饮而尽,仿佛攫取了茶水中的温热,才能稍稍安神。``如懿,朕虽然君临天下,可是午夜梦回,每每梦见自己年少时无人问津的孤独与悲苦。朕的生母早逝,皇阿玛又嫌弃朕的出身,少有问津。哪怕朕今日富有四海,一人独处时,也总害怕自己会回到年少时一无所有的日子。''

如懿紧紧握住皇帝的手:``怎么会?皇上有臣妾,有皇额娘,有那么多嫔妃、皇子和公主,怎么会一无所有?''

皇帝的神色无助而惶惑,仿佛被雨露沾湿的秋叶,薄而脆枯。``朕有皇额娘,可她是太后,不是朕的亲额娘。朕有那么多嫔妃,可是她们在朕身边,为了荣宠,为了家族,为了自己,甚至为了太后,有几个人是真心为朕?朕的儿子们一天天长大,朕在他们心里,不仅是父亲,是君王,更是他们虎视眈眈的宝座上碍着他们一步登天的人。至于朕的女人,朕疼她们爱她们,可若有一天朕要为了自己的江山舍出她们的情爱与姻缘时,她们会不会怨恨朕?父女一场,若落得她们的怨怼,朕又于心何安?''

翠竹窗栊下,茜红纱影影绰绰。如懿心下微凉,仿佛斜风细雨也飘到了自己心上。``那么臣妾呢?皇上如何看臣妾?''

皇帝的声音有些疲倦,闭目道:``如懿,你有没有算计过朕?有没有?''

如懿的心跳陡然间漏了一拍。她看着皇帝,庆幸他此刻闭上了双眸。因为连她自己亦不知,自己的神色会是何等难看。这些年来,她如何算计过皇帝,只有她自己明白,可是皇帝也未曾如她所期许一般真心诚意待她。他许她后位荣华,她替他生儿育女,做一个恪尽职守的皇后。到头来,也不过是落得这般彼此算计的疑心而已。

也罢,也罢,不如不看。如懿看着床帏间的鎏金银鸾钩弯如新月,帐钩上垂下细若瓜子的金叶子流苏,一把把细碎地折射着黄粼粼的光,针芒似的戳着她的眼睛。她静了片刻,衔了一丝苦笑:``皇上如何待臣妾的,臣妾也是如何待皇上。彼此同心同意而已。''

有风吹过,三两枝竹枝细瘦,婆娑划过窗纱,风雨萧瑟,夜蛰寂寂。皇帝的气息稍稍平稳,他睁开眼,眼中却有着深不可知的伤感和畏惧:``如懿,朕方才梦见了永璜,朕的第一个儿子。朕梦见他死不瞑目,问朕为何不肯立他为太子?然后是永珹,朕这些年所疼爱、欣赏的儿子,朕梦见自己回到追逐野马独自进入林间的那一日,那两支射向朕的冷箭,到底是谁?是谁想要朕的性命?''

皇帝疑心的答案已经呼之欲出,如懿将惊惶缓缓吐出口:``皇上是疑心永珹?永珹可是皇上的亲子啊!''

皇帝黯然摆首:``亲子又如何?圣祖康熙晚年九子夺嫡是何等惨烈。皇位在上,本没有父子亲情。''他的神情悲伤而疲惫,``今日朕才知原来永珹善于引逗野马,朕从来不知\ldots 而那日,就是一匹野马引了朕入林中的\ldots{}''他长叹一声,``而朕无意间听凌云彻说起,那日他赶来救朕时,明明看见永珹骑马紧在他之后立刻如林,不知为何却没有先来救朕,反而颇有观望之态,直到朕命悬一线,他才出手相救。''

时已入秋,宫苑内有月桂悄然绽放,如细细的蕊芽,此刻和着雨气渗进,香气清绵,缓和了殿中波云诡谲的气氛。

如懿的声音从喉舌底下缥缈而出:``皇上真的疑心永珹么?''

``朕不是不知道自己的儿子。嘉贵妃当初对后位有多热切,永珹对太子之位便有多热切。朕也知道嘉贵妃的用心,只有她身份高贵,她的儿子身份高贵,她的母族才会牢牢依附于大清,地位更加稳固。''皇帝静了精神,``可是凌云彻的话也不能全信,朕虽然知道他当年是被罚在木兰围场做苦役,才机缘巧合救了朕,可真有这么机缘巧合么?所以朕连夜派人赶去承德细细查问那日永珹的行踪,是否真如凌云彻所言。如果永珹真的以朕的安危博取欢心\ldots{}''他眼中闪过一丝狠戾的阴光,``那他就不配做朕的儿子了!''

寝殿中安静极了,眼下绵绵不绝的雨水缀成一面巨大的雨帘,幕天席地,包围了整座深深宫苑。满室都是空茫雨声,如懿的欣慰不过一瞬,忽而心惊。皇帝是这样对永珹,那么来日,会不会也这样对自己的永琪和永璂?

自己这样步步为营筹谋一切,是不是也是把自己的儿子们推向了更危险的境地?她不能去想,亦容不得自己去想。这样的念头只要一转,她便会想起幽禁冷宫的不堪岁月。她也曾对别人留情,结果让自己落得不生不死的境地。她无数次对自己说,只要一旦寻得敌人的空隙,便不会再留半分情面。

若来日永珹登上帝位,金玉妍成为圣母皇太后,自己想要凭母后皇太后的身份安度余年,都只能是妄想了。

像是漂泊在黑夜的雨湖上,唯有一叶扁舟载着自己和身边的男子。对于未来,他们同样深深畏惧,并且觉得不可把握。只能奋力划动船桨,哪怕能划得更远些,也是好的。

这样的深夜里,他们与担忧夜雨会浇破屋顶,担忧明日无粟米充饥的一对贫民夫妇相比,并无半分差别。

窗外冷雨窸窣,绵密的雨水让人心生伤感,想要寻一个依靠。皇帝展臂拥住她:``如懿,有时候朕庆幸自己生在帝王家,才能得到今日的荣耀。可是有时候,朕也会遗憾,遗憾自己为何生在帝王家,连骨肉亲情、夫妻情分都不能保全!''

如懿知道皇帝语中所指,未必是对着自己。许是言及孝贤皇后,也可能是慧贤皇贵妃,更或许是宫中的任一妃嫔。可她还是忍不住打了个寒噤,若有一日,他们彼此间的算计都露了底,所谓的帝后,所谓的夫妻,是否也到了分崩离析、不能保全的境地?

到头来,不过都是孑然一身,孤家寡人罢了!

雨越发大了。竹叶上雨水滴沥,风声呜咽如诉。雨线仿佛是上天洒下的无数凌乱的丝,绵绵碎碎,缠绕于天地之间。如懿突然看见内心巨大的不可弥补的空洞,铺天盖地地充满了恐惧与孤独。

他们穿着同色的明黄寝衣,款长的袖在烛光里薄明如翼,簌簌地透着凉意。

她贵为一国之后,母仪天下。他是一朝之君,威临万方。

可是说到底,她不过是一个女人,她也不过是一个男人。在初秋的雨夜里,褪去了所有的荣耀与光辉,不过是一对心事孤清、不能彼此温暖的夫妻。

夜深,他们复又躺下,像从前一样,头并着头同枕而眠。他的头发抵着她的青丝,彼此交缠,仿佛是结发一般亲密,却背对着背,怀着各自不可言说的心事,不能入眠。

雨水晦暝,长夜幽幽,如懿轻轻为他掖紧衾被,又更紧地裹住自己,紧紧闭上了眼睛。只期望在梦境中,彼此都有一处光明温暖的境地可栖,来安慰现实不可触摸的冰凉。

从承德归来的密使带回来的是模棱两可的答案。当日的确有人见到永珹策马如林,却不知去的是否是皇帝所去的方向。

所有的决断,永珹的未来,皆在皇帝一念之间,或者说,皇帝的疑心是否会大于父子骨血的亲情。

如懿所能做的,凌云彻所能安排的,也仅止于此。若答案太过分明,只会让皇帝往其他的方向与怀疑。这时她所不希望,也不敢的。

如懿甚至皇帝的踌躇与不悦,便备下点心,抱着璟兕来到养心殿探视,希望以女儿天真无邪的笑意,宽慰皇帝难以决断时的暴躁与迷乱。而更要紧的,也只有怀中幼女的不谙世事,才更显得成年的皇子是如何野心勃勃,居心叵测。

步上养心殿的层层玉阶,迎接她的,是李玉堆满笑容的脸。可是拿笑容底下,分明有难以掩饰的焦虑与担忧;``皇后娘娘,皇上不愿见任何人,连令妃小主和忻嫔小主方才来请安,都被挡在了门外呢。''

如懿微微蹙眉:``不只是为四阿哥的事吧?''

李玉道:``娘娘圣明,于内是四阿哥的事烦心,在外是前朝的事,奴才隐隐约约听见,是准噶尔的事。今儿晌午皇上还连着见了两拨儿大臣一起商议呢。这不,人才刚走,又赶着看折子了。''

如懿凝神片刻,温然道:``皇上累了半日,本宫备下了冰糖百合马蹄羹,你送进去给皇上吧。''李玉躬身接过。如懿努努嘴,示意乳母抱着璟兕上前:``五公主想念皇上了,你带公主进去。等下纯贵妃也会派人送四公主过来,一同陪伴皇上。''

李玉拍着额头笑道:``是呢。早起皇上还问起五公主,还是皇后娘娘惦记着,先送了公主来。''

如懿深深地看了李玉一眼,眼神恍若无意掠过站在廊下的凌云彻,摸着璟兕粉雕玉琢的小脸:``等下好好送公主回来就是。''

她携了容珮的手布下台阶,正瞧见绿筠亲自送了四公主前来,见了如懿老远便含笑施礼,恭谨道:``皇后娘娘万福金安。''

如懿忙扶住了,见纯贵妃一袭玫瑰紫二色金银线华衫,系一痕浅玉银泥飞云领子,云髻峨峨,翠华摇摇,戴着碧玉瓒凤钗并一对新折的深紫月季花,显然是着意打扮过。如懿笑吟吟道:``纯贵妃何须这般客气,皇上正等着两位公主呢,快送公主进去吧。''

绿筠示意乳母抱了四公主入殿,极力压低了嗓音,却压不住满脸喜色:``不知怎的,皇上如今倒肯惦记着臣妾了,大发了两拨儿人送了东西来给臣妾和永璋、永瑢,都是今年新贡的贡品呢。多少年皇上没这么厚赏了。听说愉妃那儿也是一样呢。''

有风拂面,微凉。如懿紧了紧身上的玉萝色素锦披风,丝滑的缎面在秋日盛阳下折射出柔软的波纹似的亮光,上面的团绣暗金向日葵花纹亦是低调的华丽。

``皇上疼你们,这是好事。惦记着孩子就是惦记着你,都是一样的。''

绿筠眼角有薄薄的泪光,感慨道:``皇后娘娘,臣妾自知不能与年轻的宠妃们相较。只要皇上疼爱臣妾的孩子,别忘了他们,臣妾就心满意足了。''

她的话,何尝不是一个母亲最深切的盼望。

如懿的手安抚似的划过绿筠的手背,像是某种许诺与安慰:``好好安心,永璋和永瑢有的是机会。''

\hypertarget{ux7b2cux4e94ux7ae0-ux9ec4ux9e44ux6b4c}{%
\chapter{第五章 黄鹄歌}\label{ux7b2cux4e94ux7ae0-ux9ec4ux9e44ux6b4c}}

绿筠喜不自禁,再三谢过,目送了如懿离开。

行至半路时,如懿惦念着永琪仍在尚书房苦读,便转道先去看他。尚书房庭院中桐荫静碧,朗朗读书声声声入耳。

``北路古来难,年光独认寒。朔云侵鬓起,边月向眉残。芦井寻沙到,花门度碛看。薰风一万里,来处是长安。''

如懿含了一抹会心的笑意,走近几步,行至书房窗边,凝神细听着越来越清晰的读书声。

容珮低声问;``皇后娘娘不进去么?''

如懿轻轻摆手,继续伫立,倚窗听着永琪的声音。里头稍稍停顿,以无限唏嘘的口吻,复又诵读另一首诗。

``吾家嫁我兮天一方,远托异国兮乌孙王。穹庐为室兮旃为墙,以肉为食兮酪为浆。居常土思兮心内伤,愿为黄鹄兮归故乡。''

听罢,如懿默思一阵,似是触动,才命容珮道:``去看看吧。''

容珮扶了如懿的手进去,满室书香中,永琪孑然立于西窗梧桐影下。永琪见她来了,忙上前亲热地唤道:``皇额娘。''

如懿环顾四周,唯见书壁磊落,便问:``只有你一人在么?其他阿哥呢?''

永琪娓娓道来:``三哥和六弟回纯娘娘宫中了。四哥这几日心绪不定,无心读书,一直没来尚书房。八弟年幼贪玩,四哥不来,他自然也不肯来了。''

如懿替永琪理一理衣领,含笑道:``旁人怎样你不必管,自己好好读书就是。''

永琪有些兴奋,眼中明亮有光:``皇额娘,昨日皇阿玛召见儿臣了。''

如懿颔首:``你皇阿玛可是问了你关于准噶尔之事?''

永琪连连点头,好奇道:``皇额娘如何得知?是皇阿玛告诉您的吗?''

如懿笑着在窗边坐下:``你读的这些诗虽未直言边塞事,却句句事关边塞事。皇额娘才隐约猜到。''她停了停,``那你皇阿玛是什么意思?你又如何应答?''

永琪眼中的兴奋之色退却,换上一副少年老成的语气:``儿臣年少懵懂,能有什么意思?自然以皇阿玛的训示为上。''

如懿油然而生一股欢喜。皇帝自然是喜欢有主见的儿子,可太有主见了,他也未必喜欢,反生忌惮。永琪善于察言观色,能以皇帝马首是瞻,自然是万全之策。如懿欣慰道:``那你皇阿玛怎么说?''

永琪道:``皇阿玛十分思念远嫁的亲妹,儿臣的姑母端淑长公主。''

只一言,如懿完全了然:``你方才念的第一首诗,是杨巨源的《送太和公主和藩》。唐宪宗女封太和公主,远嫁回鹘崇德可汗。''

永琪微微思忖:``比起终身远嫁不得归国的王昭君与刘细君,太和公主远嫁二十年后,在唐武宗年间归国,也算幸运了。''

``所以你读细君公主的《黄鹄歌》时会这般伤感。''如懿伸手抚摸永琪的额头,``你也在可怜你的端淑姑母,是不是?''

永琪的伤感如漩涡般在面上一瞬而过,旋即坚定道:``但愿公主远嫁在我朝是最后一次。儿臣有生之年,不希望再看到任何一位公主远离京城。儿臣更希望五妹妹嫁得好郎君,与皇额娘朝夕可见,以全孝道。所以儿臣已经向皇阿玛言说,当年端淑姑母远嫁准噶尔多尔札已是为难,为保大清安定再嫁达瓦齐更是不易。如今达瓦齐既然不思姻亲之德,如此不驯,皇阿玛也不必再姑息了。不如请端淑姑母还朝便是。''

永琪的话既是恳请,也是情势所在。皇帝对达瓦齐的姑息,一则是因为达瓦齐在准噶尔颇有人望,他若驯顺,则准噶尔安定,反之他若不驯,准噶尔便更难掌控,更会与蠢蠢欲动的天山寒部沆瀣一气,皇帝势必不能容忍;二则自杜尔伯特部车凌归附,皇帝更是如虎添翼,得了一股深知准噶尔情势的力量;三则太后对端淑长公主再嫁之事耿耿于怀,常以母女不能相见为憾事,皇帝此举,也是缓和与太后的关系。这样一箭三雕的妙事,可见对准噶尔用兵,势在必行。

如懿的心被永琪的这句话深深感动:``好孩子,你的愿望令皇额娘甚是欣慰。''她握住永琪的手,``从前惹你皇阿玛生气的话是为了保全自己,免得成为永珹母子的眼中钉,成了出头椽子。如今永珹眼见是被你皇阿玛厌弃了,是该到你崭露头角的时候了。''

永琪仰着脸,露出深深的依赖与信任:``皇额娘,当初儿臣故意说那句话给四哥听见,惹皇阿玛生气,但得皇阿奶欢心。如今达瓦齐无礼在先,儿臣对准噶尔的态度转变,顺着皇阿玛说,为接端淑姑母成全皇阿奶的母女之情,更为大清安定才对准噶尔佣兵,皇阿玛自然欢喜。''

如懿深深欢悦,永琪自然是她与愉妃悉心调教长大,然而十三岁的永琪,已经展露出她们所未能预期的才具。幼聪慧学,博学多才,习马步射,武技俱精。不仅娴习满、蒙、汉三语,更熟谙天文、地理、历算。尤其精于书法绘画,所书八线法手卷,甚为精密。然而才学事小,更难得的是他心思缜密,善于揣摩人心,真真是一个极难得的能如鱼得水的孩子。

如懿这般想着,不免升起一腔慈母心怀:``有你这般心思,也不枉本宫与你额娘苦心多年了。''她殷殷嘱咐,``好好去陪你额娘,这些日子她可为你担足了心思。''

永琪爽朗笑道:``额娘一开始是担心,但时日久了,又与皇额娘知心多年,多少猜到了几分,如今也好了。''他忽然郑重了神色,一揖到底,``儿臣多承皇额娘关怀,心中感念。额娘出身克里也特使小族,家中人丁凋零,仅有的亲眷也是来讨嫌的多,成事不足败事有余,只会叫额娘烦心的。幸好宫里还有皇额娘庇护,否则儿臣一介庶子,额娘又无宠,真不知会到如何田地去。''

如懿叹口气,爱怜地看着他:``你这孩子什么都好,偏生这样多心。什么庶子不庶子的话,都是旁人在背后的议论,你何苦听进去这般挂心。只要你自己争气,哪怕你额娘无宠,自然也会母以子贵。''

永琪尚显稚嫩的脸上含着感激的神色,郑重其事地点头:``儿臣都听皇额娘的。''

如懿回到宫中,因着心中欢喜,看着秋色撩人,便起了兴致,命宫女们往庭院中采集新开的金桂,预备酿下桂花酒。永璂在旁看着热闹,也伸出胖乎乎的小手,想要参与其中。

容珮看着众人欢欢喜喜地忙碌,一壁哄着永璂,一壁趁人不备低声向如懿道:``娘娘倒是真疼五阿哥,五阿哥有愉妃小主心疼,又有娘娘庇护,真是好福气。看如今这个样子,四阿哥是不成了,不知道太子之位会不会轮到五阿哥呢?''

容珮嘴上这般说,眼睛却直觑着如懿。如懿折了一枝金桂在鼻端轻嗅,道:``永璂年幼,哪怕皇上要立他为太子,也总得等他年长些才是。可要等到永璂年长,那还得多少年数?夜长梦多,比永璂年长的那些阿哥,哪个是好相与的?一个个处心积虑,都盯着太子之位呢。与其如此,被别人争了先,还不如让永琪占住了位子。''

容珮有些把握不定:``占住了位子,还留得住给十二阿哥吗?到底,十二阿哥才是娘娘亲生的啊。从前的大阿哥虽然也得娘娘抚育几年,到底还是变了心性,五阿哥他\ldots{}''

``永璜要为自己争气,一时用力用心过甚,错了主意也是寻常。到底后来本宫没有在他身边时时提点。至于永琪,海兰与本宫一直同心同德,情如姐妹。若是连海兰都不信,这宫里便没有本宫可以相信的人了。''如懿温然一笑,含了沉沉的稳笃,``容珮,眼睛看得见的不要只在眼前方寸之地,而要考虑长远,是不是永璂登基为新帝不要紧,要紧的是本宫是笃定的母后皇太后!''如懿弯下腰,抱起永璂,笑着逗弄道:``天家富贵难得,皇帝之位更是难坐。好孩子,额娘只要你一辈子平安福贵就好。何必一定要做皇上呢?''

如懿正逗着怀中的孩子,看着他天真的笑颜,只觉得一身的疲惫皆烟消云散。凌云彻跟在李玉身后,陪着璟兕和乳母们一同进到翊坤宫庭院。只见丛丛桂色之后,如懿的笑颜清澈如林间泉水,他心中不觉一动,好像耳根后头烧着一把灼灼的火,一直随着血脉蔓延下去。

如懿听得动静,转首见是他们,便淡了笑容道:``有劳李公公了,还特意送了公主回来。''

李玉知道如懿的心意,便道:``公主是千尊万贵的金枝玉叶,奴才能陪伴公主,是奴才的福分。而且奴才怕自己手脚没力气,乳母们也伺候得不当心,所以特意请了凌大人相陪,一路护送。''

如懿只看着怀中的永璂,淡淡道:``凌大人辛苦。''

凌云彻躬身道:``是公主不嫌弃微臣伺候不周。''他再度欠身,``许久没向皇后娘娘请安了。娘娘万福金安。''

李玉忙道:``方才凌大人来之前,皇上刚下了口谕,晋凌大人为御前一等侍卫。凌大人是该来皇后娘娘请安的。''

``恭喜凌大人。凌大人尽心侍奉皇上,是该有升迁之喜。容珮,拿本宫的一对玉瓶赏给凌大人。''如懿将永璂递到乳母怀中,转身入了殿内。

二人跟着如懿一同入了正殿。

容珮一拍额头道:``李公公,那对玉瓶我不知搁在哪儿了,您帮我一起找找。''

李玉何等乖觉,答应着便转到里间和容珮一起去寻。如懿侧身在暖阁内的榻上坐下,慢慢剥着一枚红橘道:``你倒是很能干。承德传来这样的消息,虽然没有实指是永珹做的,但皇上既然封赏了你,便是落定了信的是你,疑心了永珹。''

凌云彻长舒了一口气:``不是微臣能干。蝼蚁尚且偷生,微臣的命虽然卑微,但也不想失了这卑贱性命。''

如懿的手指沾染上清凉而黏腻的汁液,散发出甜蜜的甘香:``木兰围场的事本宫不管你插手了多少,但你既然是皇上的御前侍卫,得皇上器重,就理应护卫皇上周全。若皇上再有了什么差池,那便是你连自己的脑袋也不要了。''

凌云彻深深叩首:``微臣谨记皇后娘娘教诲。''

如懿盯着他,轻声道:``当年木兰围场的事若是有人精心布置,那人便真是心思长远了。''

凌云彻的目光触上她的视线,并不回避,``微臣当日被罚去木兰围场,本是因为心思鲁直,才会受了他人算计。幸蒙皇上不弃,才能再度侍奉皇上身边,微臣一定尽心尽力,为皇上和皇后娘娘办事,肝脑涂地,万死不辞。''

如懿听他再三撇清,又述说忠心,心中稍稍安定:``你有本事保得住自己的完全,本宫就可以用你这个有本事的人。反之,再多的忠心也不顶用。所以你凡事保住自己再说。''

凌云彻心头一热,如浪潮迭起,目光再不能移开。如懿鸦翅般的睫毛微微一垂,落下圆弧般的阴影,只低头专心致志剥着橘子,再不看他。

这样的静默,仿佛连时间也停住了脚步。外头枝叶疏疏,映着一轮秋阳。她的衣袖轻轻起落,摇曳了长窗中漏进的浅金阳光,牵起幽凉的影。

他明知道,见她一面是那样难。虽然如懿也会常常出现在他的视线之中,如同嬿婉一样。但他亦只能远远地看着,偶尔欠首示意而已。如何能这般在她面前,隔着这样近的距离,安安静静地听她说话。

他喉舌发热,好像神志亦远离了自己,脱口道:``皇后娘娘不喜欢的命,微臣可以替皇后娘娘出去。皇后娘娘在意的性命,微臣一定好好替皇后娘娘保全。''

如懿抬首瞥了他一眼,目光清冷如霜雪,并无半分温度:``你自己说什么话自己要知道分寸,好好管着你的舌头,就像爱惜你自己的性命与前程一样。''她顿一顿,``惢心进宫的时候偶然说起,说你与茂倩的夫妻情分不过尔尔?''

凌云彻一怔,仿佛有冰雪扑上面颊,凉了他灼热的心意。他只得坦诚道;``微臣忙于宫中戍卫之事,是有些冷落她,让她有了怨言。''

如懿凝视他片刻:``功名前程固然要紧,但皇上所赐的婚事也不能不谐,你自己有数吧。''说罢,她再不顾他,只是垂首默默,恍若他不在眼前一般。

容珮与李玉捧着一双玉瓶从里头出来,容珮笑吟吟递到凌云彻手里,道:``凌大人,恭喜了。''

凌云彻忙收敛心神,再三谢过,才与李玉一同退了出去。

次日,皇帝下旨以准噶尔内乱之名,命两路进兵取伊犁,征讨达瓦齐。车凌因熟悉准噶尔情形,洞悉军务,被任命为参赞大臣,指挥作战,并征调杜尔伯特不两千士兵参战。同日,皇帝以永珹早已成年之故,出居宫外贝勒府,无事不得入宫,连向生母请安亦不被允准,形同冷落宫外。而玉妍所生的另两子,八阿哥永璇已经六岁,住在阿哥所方便往尚书房读书,而十一阿哥永瑆因为不满三岁,才被允许留在玉妍宫中养育。

这般安排,分明是嫌弃玉妍教子不善了。

永珹的事本是莫须有,只在皇帝心中揣度。皇帝并未直接明说,但也再未见过玉妍,连她在养心殿外苦苦跪求了一夜,也不曾理会,只叫李玉扶了她回去静思安养。

如此,公众顿时安静,再不敢有人轻言太子之事了。

此时的永琪,如冉冉升起的红日,朝夕随奉皇帝左右,十分恭谨谦和,多半以皇帝之意为己意,又常与三阿哥永璋有商有量,处处尊重这位兄长。待到皇帝问及时,才偶尔提一两句,也在点子上。哪怕得到皇帝赞许也不骄矜,处处合黄帝心意。

如此这般,绿筠也格外欢喜。虽然永璋早年就被皇帝绝了太子之念,但永琪尊敬兄长,提携幼弟,连着绿筠的日子也好过许多。宫中无人不交口称赞这位五阿哥贤良有德,比昔日骄横的永珹,不知好了多少。

玉妍与永珹受了如此长大的打击,颜面大伤,一时寂寂无闻。除了必须的合宫陛见,便闭上宫门度日,连晨昏定省也称病不见。然而细细考究,也不是称病,而是真病下了。玉妍生生这般母子分离,一时间心神大损,日夜不安。每每入睡不久,便惊醒大呼,时时觉得有人要加害于她母子。癫狂之时,便直呼是如懿、绿筠、海兰或是嬿婉等人都要害她。如懿连连打发了几拨儿太医去看,都被玉妍赶了出来,皇帝知道后更是生气,亲自派了齐鲁去医治,又开了安神药,却总是效用不大。

因着害怕有人加害,玉妍命人搜罗了各色各犬豢养在启祥宫,才能安静许多,也不再那么害怕了。如此一来,一时间宫中犬吠连连,闹得合宫不安,烦不胜烦。如懿再四命人去启祥宫驱逐那些狗,然而玉妍大哭大闹,不能成事。

如懿如何肯与她计较,便丢开不理。倒是忻嫔的性子第一个耐不住,便去向皇帝哭诉,加之嬿婉软言相劝,皇帝便命人将启祥宫中的狗全番驱走,只说是怕惊着了永瑆。玉妍哭闹不休,连连磕头,只说人不如狗忠心,把狗赶走之后自己成日惊惶,怕也不久于世。皇帝无奈,只得留了两条巴儿狗给她赏玩便罢。

于是宫里的人说起来,都说玉妍和永珹是结交外臣谋夺太子之位被皇帝知晓,才骤然失宠。玉妍也因此发了失心疯。

再见到皇帝时,已是两日后了。如懿往太后处请安,却见太后愁容满面,正为准噶尔之事而忧心忡忡。

如懿想来想去有些不安,便往养心殿里去。秋日的阳光落在养心殿的澄金地砖上有明晃晃的光影,如置身于金灿浮波之内。

皇帝颀长的背影背对着她,面对着一幅巨大的江山万里图,出身不已。如懿缓步走近,柔声道:``皇上恨不能以目光为剑,直刺准噶尔,是不是?''

皇帝的专注里有肃杀的气息:``朕忍得太久了。从端淑远嫁准噶尔那一日起,朕就在想,有朝一日,可以不用再遣嫁皇女了。所以让端淑再次改嫁达瓦齐的时候,太后责怪朕,嫔妃劝朕。但只有朕自己知道有多为难,有多无奈。端淑是长公主,也是朕的妹妹,可是朕不能不暂且忍耐一时,等待更好的时机。如今杜尔伯特部归来,准噶尔人心浮动,朕终于等到这个时候了。''

如懿心中触动,她知道的,她选的这个人,从来不是一味隐忍不图来日的人。

如懿满心喜悦,欠身道:``恭喜皇上,终于等到这一日。臣妾万幸,能与皇上一同等到这一日。''

皇帝盯着江山万里图上准噶尔那一块,以朱笔一掷,勾画出凌厉的锋芒。他不掩踌躇满志之情,长叹入啸,胸怀舒然:``朕隐忍多年,舍出亲妹的一段姻缘,如今终于能扬眉吐气,直取楼兰!''

如懿婉声道:``能有这一日,端淑长公主终于可以归来,她一定也很高兴。母女团聚,太后多年郁结,也可欣慰少许了。只是\ldots{}''她觑着皇帝被日光拂耀的清俊面庞,轻声说出自己的担忧,``可是端淑长公主虽然嫁给达瓦齐,但我朝军马攻向准噶尔,乱军之中本就危险万分,若达瓦齐恼羞成怒意挟持公主,或欲杀了公主泄愤,那么\ldots{}''

她的话语尚未完全说出口,已听得殿外太后含怒的声响。她老迈而微带嘶哑的声音随着龙头拐杖的凿地声怆然入耳:``皇帝,皇帝,哀家召唤你来慈宁宫,你一直迁延不肯前来。好!你既然不肯来,那么哀家来求见你,你为何又避而不见?''

李玉的声音惊惶而焦灼,道;``太后娘娘,皇上正忙于国事,实在无暇见您!''

``无暇见哀家?难道陪着自己的皇后,便是国事了么?''

如懿这才想起,自己前来养心殿,辇轿自然就在养心殿外停着,才受了太后如此言语。如懿顿时大窘,忙跪下道:``皇上,臣妾疏忽,让臣妾出去向太后请罪吧。''

皇帝神色冷肃,伸手扶起她,微微摇了摇头。他的面庞映着长窗上``六合同春''的吉祥如意的花纹,那样好的口彩,填金朱漆的纹样,怎么看都是欢喜。可是一窗相隔,外头却是太后焦痛不已的慈母之心。

皇帝的神色在光影的照拂下明暗不定。如懿见他如此,越发不敢多言,只得屏息静气立在皇帝身旁。

``皇后与皇帝真是同心同德,长公主陷于危难之中而不顾,哀家求见却闭门不见,真是一对好夫妻啊!''

太后说得太急,不觉呛了一口气,连连咳嗽不已。福珈惊呼道:``太后,太后,您怎么了?''

李玉吓得带了哭腔:``太后娘娘!您万圣之尊,可要保重啊!''

``保重?''太后平复了气息,悲愤道,``哀家还保重什么?皇上下令攻打自己的妹婿,达瓦齐是乱臣贼子,哀家无话可说,可是端淑是皇帝亲妹,身在乱军之中,皇帝也不顾及她的性命么?''

李玉的磕头声砰砰作响:``太后娘娘,皇上善于用兵,前线的军士都会以保护长公主为先的!您安心回慈宁宫吧?''

``回慈宁宫?等着收哀家女儿的尸首么?''太后冷笑道,``刀剑无眼,何况准噶尔蛮夷,若是挟持长公主,只怕皇帝也不会顾惜吧?''

皇帝再听不下去,他深吸一口气,豁然打开殿门,跪下身道:``皇额娘,您身为太后之尊,自然明白社稷重于一切。不是儿子舍出了皇妹,是社稷舍出了皇妹。''他郑重地磕了个头,目光沉静如琥珀,一丝不为所动,``但请皇额娘回宫安养,以免动摇军心,让前线将士有所顾虑,不能全心全意平定准噶尔,带回端淑。''

如懿跪在皇帝身后,听得这一句,心头一颤,如坠寒冰之中,不自觉地抬起头去看太后。太后身体微微一晃,踉跄几步,仰面悲怆笑道:``好儿子,果然是哀家教出的好儿子,懂得来逼迫哀家了。''她的伤感与软弱不过一瞬,便狠狠拿龙头拐杖支撑住自己的身体,冷下脸道,``哀家来求你,是要你顾及母子兄妹的情分。既然皇帝撂下这句话来,那好,哀家就回慈宁宫静养,日日诵经念佛,求佛祖保佑皇帝一切遂心,那么皇帝也能怜悯哀家的端淑,保她完全!''

太后说罢,扶住福珈的手缓缓步下台阶。如懿看着太后的背影,华服之下,她的脚步分明有些摇晃,再不是记忆中那泰山崩于眼前而不乱的深宫贵妇了。

\hypertarget{ux7b2cux516dux7ae0-ux4f24ux60c5ux8584}{%
\chapter{第六章 伤情薄}\label{ux7b2cux516dux7ae0-ux4f24ux60c5ux8584}}

如懿的眼角忽然有些湿润,像是风不经意地钻入眼底,吹下了她眼前朦胧的一片。深思恍惚间,有尖锐的恐惧深深地攫住她的心头,会不会来日,她也会如太后一般,连自己的儿女也不能保全?

她不敢,也容不得自己做这样悲观而无望的念想。打断她思绪的是皇帝沙哑而低沉的声音。皇帝神色黯然:``如懿,你会不会觉得朕太过不顾亲情?''

这样的话,她如何答得出。若是说皇帝不顾亲情,固然是冒犯龙颜。若是说皇帝顾念亲情,那么端淑算什么?来日若轮到自己的璟兕,那又算什么?她胸腔内千回百转,终究只能道:``皇上心中,大局重于私情。若在寻常人家,固然是兄妹之情与大局之间选择两难,可是生在天家,人人都有自己的不得已。但愿从此以后,皇上再无这样的不得已。''

皇帝黯然一叹,揽过如懿的肩:``朕知道你在担心什么。当日许端淑再嫁之时,朕就已经想好,这是最后一次,大清的最后一次,再也不会有远嫁的公主了。''

自此,太后果然静守在慈宁宫内,半步都不出,只拈香礼佛,日夜为端淑长公主祝祷。宫中之事悉数在如懿手中,而嫔妃们亦朝夕殷勤请安,翊坤宫内时时笑语盈盈,衣香浮动。

此时,如懿抱了永璂在怀,听着嫔妃们在坐下闲谈,亦不过淡淡含笑。绿筠因着三阿哥永璋不似从前那样在皇帝跟前没脸,也多了几分从前的开朗,奉承着如懿道:``话说回来,还是嘉贵妃和四阿哥太贪心不足了。皇上略略抬举些,便得陇望蜀,盯着她不该想也不配想的东西。''她递过一个黄金柑逗着永璂笑道:``现放着皇后娘娘亲生的十二阿哥呢,她也做起这样的梦来了。''

如懿浅笑道:``本朝并无非要立嫡之说。太祖高皇帝努尔哈赤立过多位大妃,元妃佟佳氏生了褚英和代善,继妃富察氏生了莽古尔泰和德格类,最后一位大妃乌拉那拉氏生了阿济格、多尔衮和多铎。可是最后继位的却是生前为侧妃的叶赫那拉氏所生的太宗皇太极。说来太祖早年也不过是庶子而已。所以本宫看来,只要有才学,能为江山出谋出力,才是皇上的好儿子。咱们不论嫡庶,只论贤能。''

这一席话,听得绿筠心悦诚服。海兰亦柔缓笑道:``论起来除了嘉贵妃,就是纯贵妃皇子最多,三阿哥又是长子,更是其他皇子们的榜样。永琪每每回来都说给我听,三阿哥是如何如何沉稳,有三阿哥在,他做事也有个主心骨了。''

这话是谦逊,亦说得绿筠眉开眼笑,欣喜不已:``永琪这话最懂事,真真他们几个都是好兄弟,不像嘉贵妃教出来的孩子,没个好脸色对人。''她说罢,继而正色,竖起双指,``只是臣妾的阿哥无论好与不好,臣妾都在此发誓,臣妾的孩子只懂效忠大清,效忠皇上,效忠未来的主子,绝无半分夺嫡妄想。''

如懿似是十分意外,便沉静了容色道:``好端端的,说这样的话做什么?''

绿筠无比郑重地摇头,缓缓扫视周遭众人:``臣妾有着三阿哥和六阿哥两位皇子,难免会有人揣测臣妾会倚仗着儿子们不尊皇后。今日,臣妾便索性在这里说个明白。在座的姐妹们或有子嗣,或来日也会诞下皇嗣,不如今日一并分明,以免以后再起争端,叫人以为咱们后宫里都失了上下尊卑,乱了嫡庶规矩了。''

她说罢,海兰亦郑重屈身:``纯贵妃姐姐久在宫中,见事明白。臣妾跟随纯贵妃姐姐,唯皇后娘娘马首是瞻,绝无夺嫡生乱之心,否则神明在上,只管取了臣妾满门去便是。''

她这一说,和人还敢不起身,一一道了明白。

如懿听众人一一起誓,方示意容珮扶了为首的绿筠起来,含了温煦笑意道:``纯贵妃与愉妃教子有方,连本宫看着都羡慕。''她望着坐下一众年轻妃嫔,尤其注目着忻嫔和颖嫔道:``你们都年轻,又得皇上的喜爱,更该好好为皇上添几个皇子。''

忻嫔和颖嫔忙起身谢过。嬿婉坐在海兰之后,听着嫔妃们莺声呖呖地说笑不已,又句句说在孩子上,不免心中酸涩,有些落落寡欢。且她虽得宠,但在如懿跟前一向不太得脸,索性只是黯然。

如懿见嬿婉讪讪地独坐在花枝招展的嫔妃之中,话锋一转:``令妃,今日是你的生辰,皇上昨日便嘱咐了内务府备下银丝面送去你宫里,还另有赏赐。咱们也贺一贺你芳辰之喜。''

嬿婉骤然听见如懿提起自己的生辰,忙撑着一脸笑容:``臣妾多谢皇后娘娘关怀。''

如懿看她一眼,神色淡淡,``今夜皇上大约回去你宫里,你好好伺候着吧。''

嬿婉听如懿对自己说话的语气,十足十是一个当家大妇对卑下侍妾的口吻。想着如懿也不过是由侍妾而及后位的,心口便似被一只手狠狠攥住了揉搓着,酸痛得透不过气来,脸上却无论如何也不能让笑容有稍许褪色。

忻嫔与颖嫔都与嬿婉正当宠,年轻气盛,便也不大肯让着,嘴上贺寿,脸上笑容却淡淡的。如此,大家说笑一晌,便也散了。

到了午后时分,皇帝果然派了小太监进忠过来传旨,让嬿婉准备着夜来接驾。进忠笑眯眯道:``皇上午膳时分就惦记着小主亲手做的旋覆花汤和松黄饼,可见皇上多想念小主。''

春婵故意打趣儿笑道:``旋覆花汤易得,拿旋覆花、新绛和茜草煮成就好,可这松黄饼却不好做。春来松花黄,和蜜做饼状,得用三月的松花调了新蜜做成,现在哪儿得呢?''

进忠的目光黏在嬿婉身上,觍着脸拉着嬿婉的衣袖道:``小主,春婵姐姐惯会哄人玩儿。皇上惦记着令妃小主,就没有小主做不到的。否则皇上怎么会日思夜想着呢?''

春婵哪里不晓得嬿婉的心思,忙扯了进忠的手挥开,道;``小主,您瞧进忠这个猴崽子的油滑样儿,都是小主惯的。''

嬿婉取过一双翡翠嵌珍珠手钏套在玉臂上,笑吟吟道:``本宫肯惯着进忠,那是进忠有值得本宫惯着的地方。进忠,你说是不是?''

进忠忙打了千儿道:``奴才多谢小主赏识之恩。''

嬿婉试了试那手钏,对着窗外明朗日色,手钏上的翡翠沉静通透,如同一汪绿水,那珍珠在日光照耀下,更是光滑流灿,熠熠生辉。嬿婉摇了摇头,顺势将手钏脱出,放在了进忠手上:``皇后当年怎么赏识你师傅李玉,本宫就怎么赏识你,都是一样的。你师傅的今日就是你的来日,别觉得有什么不如人的。''

进忠忙磕了头道:``小主的教诲,奴才没有一日不记在心里的。当初奴才家里缺银子使,奴才的月钱不够,是小主一次次周济奴才家里。小主的大恩,奴才至死不忘。''

嬿婉浅浅一笑,如娇花初绽:``靠人周济能过一时,却过不了一世。想要以后永远不缺银子,也不求人,便要自己争气。去吧,去皇上跟前好好当差,有你的好。''

进忠死死地攥着手钏,千恩万谢地出去了。

春婵瞥了进忠一眼,看他走远了,方才狠狠啐了一口道:``没根的东西,也敢对着小主拉拉扯扯。小主没看他的眼睛,就盯着您不放。也不打量打量自己是什么玩意儿!''

嬿婉目光冷厉,看了看被进忠扯过的袖子:``陪本宫去更衣,这件衣裳剪了它,本宫不想再穿了。''

春婵立刻答应了,扶着嬿婉进去了。

清夜无尘,月色如银。半弯月亮挂在柳树梢头,透着霞影窗纱映照殿内,朦朦胧胧,仿佛笼了一层乳白色的薄雾。寝殿的窗下搁着数盆宝珠山茶,碗口大的花朵吐露芬芳,其中一株千叶大红的尤其艳丽,映着红烛成双,有一股甜醉的芳香。

花梨木五福捧寿桌上搁着几样精致小菜,酒残犹有余香在,醺得相对而坐的两人眉目含春,盈然生情。

嬿婉只穿着家常的乳白撒桃花纹红琵琶襟上杉,金丝串珠滚边,华美中透着轻艳。下面是绛紫细裥褶子海棠缠枝软纱长裙,杨柳色的绵长丝绦飘飘袅袅,缀了鸳鸯双喜玉佩的合欢刺绣香包。她绾着蓬松的云髻,插玉梳,簪银缀珠的蝶恋花步摇,眉心有珍珠珊瑚翠钿,眉眼轻垂,肤白胜雪。

嬿婉的眉眼点了桃花妆,像是粉色的桃花飞斜,嗔了皇帝一眼:``皇上说臣妾腰肢细软,穿窄肩长裙最好看,臣妾才胆敢一试。''她媚眼如飞,低低啐了一口:``皇上说什么汉家满家,还不都是皇上的人罢了。''她说罢,低首拨弦,拂筝起音。

那秦筝的音色本是清凉刚烈,施弦高急,筝筝然也,可是到了嬿婉指间,却平添了几分妩媚柔婉、千回百转之意。

她轻吟慢唱,是一曲《长生殿》。

``那君王看承得似明珠没两,整日里高擎在掌。赛过那汉飞在昭阳。可正是玉楼中巢翡翠,金殿上锁着鸳鸯,宵偎昼傍。直弄得那官家舍不得半刻,心儿上。守住情场,占断柔乡,美甘甘写不了风流帐。行厮并坐一双,端的是欢浓爱长,博得个月夜花朝同受享。''

素来不曾有以秦筝配着昆曲的唱腔低吟浅唱,嬿婉这般不按章法,却也别有心裁。皇帝擎着羊脂白玉盏,那杯盏是白璧莹透的玉,酒是清冽透彻的琥珀色。他似沉醉在歌喉清亮之中,一盏接一盏,痛饮欢畅。

那筝音悠悠扬扬,俨若行云流波,顺畅无滞,时而如云雾绵绵萦绕于雪峰,时而如秋水淙淙幽咽于山间。嬿婉抚挑筝弦,素腕如玉,眼波效益却随着玉颈优雅起伏流转,飞旋与皇帝身侧。须臾,筝音渐渐低柔下来,絮絮舒缓,好似少女在蓬蓬花树下低声细语,那唱词却是数不尽的风流袅娜,伴着嬿婉的一瞥一笑,漫溢幽延。

一曲终了。皇帝闭着双眸,击掌缓缓吟道:``哀筝一弄湘江曲,声声写尽湘波绿。纤指十三弦,细将幽恨传。当筵秋水慢,玉柱斜飞雁。弹到断肠时,春山眉黛低。''他睁开眼,眼底是一朵一朵绽放的笑色,``令妃,你总是这般别出新意,叫朕惊喜。''

嬿婉的眼波如柔软的蚕丝萦绕在皇帝身上,一刻也不肯松开,娇嗔道:``若臣妾都和别人一样,皇上就不会喜欢臣妾了。且皇上喜欢臣妾的,旁人未必就喜欢了。''她似嗔似怨,吐气如兰,``多少人背后多嫌着臣妾呢,说臣妾邪花入室。''

皇帝的呼吸间有浓郁的酒香,仿若夜色下大蓬绽放的红色蔷薇,也唯有这种外邦进贡的名贵洋酒,才会有这样灼烈而冶艳的芬芳。他大笑不止:``邪?怎么邪?''

嬿婉的身段如随风轻荡的柳条,往皇帝身上轻轻一漾,便又蜻蜓点水般闪开。她媚眼如星,盈盈道:``就说臣妾这般邪着招引皇上,邪着留住皇上。''

``还邪着勾引朕是么?''皇帝捏着她的脸,故作寻思,``然后便是那句话,等着看邪不胜正是么?''

嬿婉背过身,娇滴滴道:``皇上都知道,皇上声明。''

皇帝搂过她在膝上,朗声笑道;``朕就是喜欢你邪,如何?邪在里头,对着爱假正经的人却也能正经一番,你这是内邪外正。''皇帝面颊猩红,靠近她时有甜蜜的酒液气息,``所以朕喜欢你,会在准噶尔战事之时还惦记着你的生辰来看你。''他舒展身体,难掩慵倦之意,``金戈铁马之事固然能让一个男人雄心万丈,但对这如花笑靥,百转柔情,才是真正的轻松自在。''

嬿婉笑得花枝乱颤,伏倒在皇帝怀中。皇帝拥抱着她,仰首将酒液灌入喉咙。他的唇色如朱,显然是醉得厉害了,放声吟道:``长爱碧阑干影,芙蓉秋水开时。脸红凝露学娇啼。霞觞熏冷艳,云髻袅纤枝。''

皇帝吟罢,只是凝视着她,似乎要从她脸上寻出一丝映证。

两下无言,有一痕尴尬从眼波底下悄然漫过,嬿婉垂首脉脉道:``皇上说的这些,臣妾不大懂。''她露出几分戚然,几分娇色,``皇上是不是嫌弃臣妾不学无术,只会弹个筝唱个曲儿?''

皇帝笑着捏一捏她的脸颊:``你不必懂,因为这阙词说的就是你这样的美人。你已经是了,何必再懂?''

嬿婉悠悠笑开,唇边梨涡轻漾,笑颜如灼灼桃花,明媚得让人睁不开眼睛,可是心底,分明有一丝春寒般的料峭声声凝住了。她忍了又忍,趁着皇帝浓醉,耳鬓厮磨的间隙,终于忍不住问:``皇上,臣妾伺候您那么多年,您到底喜欢臣妾什么呢?''

皇帝降沉重的额头靠在她肩上,丝绸细软的质地叫人浑身舒畅:``你性子柔婉如丝,善解人意,又善厨艺,更会唱昆曲。朕每次一听你的昆曲,就觉得如置三月花海之中,身心舒畅。''

嬿婉心头微微一松:``可是臣妾也快不年轻了。宫里颖嫔、忻嫔、晋嫔、庆嫔都比臣妾年轻貌美,皇上怎不多去陪陪她们?''

皇帝醉意深沉,口齿含糊而缓慢:``她们是貌美,但是美貌和美貌是不一样的。颖嫔是北方胭脂,忻嫔是南方佳丽,晋嫔是世家闺秀,庆嫔是小家碧玉。而你,令妃你\ldots{}''他伸手爱惜地抚摸嬿婉月光般皎洁的脸,``你跟如懿年轻的时候真是像。有时候朕看着你,会以为是年轻时的如懿就在朕身边,一直未曾离去。''

嬿婉仿佛是挨了一记重重的耳光,这样猝不及防,打得她眼冒金星,头昏脑涨。她只觉得脸颊上一阵阵滚烫,烫得她发痛,几欲流下眼泪来。她死死地咬住了嘴唇。那样痛,仿佛只有这样,才可以抵抗皇帝的话带给她的巨大的羞辱。嬿婉原是知道的,她与如懿长得有些像,但是她从不以为那是她得宠的最大甚至是唯一的原因。她懂得自己的好,她懂得的。可是她却未承想,他会这样毫不顾忌,当着自己的面径直说出。

他,浑然是不在乎的,不在乎真相被戳破那一刻她的尴尬,她的屈辱,她的痛侮。

有夜风轻叩窗棂,她的思绪不可扼制地念及另一个男子。曾经真正将她视若掌中瑰宝的、心心念念只看见她的好的那个男子,终究是被她轻易辜负了。

而眼前这个人,与自己肌肤相亲,要仰望终身的男人,却将她所有的好,都只依附于与另一个人相依的皮相之上。

她看着醉醺醺的皇帝,忍不住心底的冷笑。如懿?他就是那样唤皇后的闺名。他唤颖嫔、忻嫔、庆嫔、晋嫔,还有自己,令妃,都是以封号名位称呼,全然忘记了她们也有名字,那些柔美如带露花瓣般的文字聚成的名字。

原来她们在他心里,不过如此而已。人与人啊,到底是不一样的。

她轻吁了一口气,以此来平复自己激荡如潮的心情。她擎起酒杯,默默地斟了一盏,仰头喝下。酒液虽有辛辣的甜蜜,入口的一瞬却是清凉。她又斟一盏,看着白玉酒盏玲珑如冰,剔透如雪,而那琥珀色的酒液,连得宠的忻嫔和颖嫔也不能一见。唯有她,伴随君侧,可以随意入喉。

她这样想着,胸口便不似方才那般难受。皇帝只醉在酒中,浑然不觉她的异样。嬿婉想,或许在深宫多年沉浮,她已经学会了隐忍,除了笑得发酸的唇角,自己也不觉有任何异样。

皇帝爱怜地望着她:``朕看着你,就像看着如懿当年。可是你的性子,却比如懿柔软多了。如懿,如懿,她即便温柔的时候,也是戴着清刚气的。''

十月二十三的夜,已经有疏疏落落的清寒,殿中的宝珠山茶硕大嫣红的花盘慵慵欲坠,红艳得几乎要滴出血来。每一朵花的花瓣都繁复如绢绡堆叠,映得嬿婉的脸庞失了血色般苍白。

嬿婉眼睁睁看着皇帝骤然离去,拥拥簇簇的一行人散去后,唯有风声寂寞呼啸。她想要呼唤些什么,明知无用,只得生生忍住了。有抽空力气一样的软弱迅疾裹住了她,她在春婵身边,两滴泪无声地滑落:``皇上是嫌弃本宫了,皇上念的诗词,本宫都不懂。''

春婵忙劝道:``小主别在意,宫里有几个小主懂得这些汉人的诗词呢?除了皇后,便是死了的舒妃和慧贤皇贵妃。''

嬿婉默默垂泪:``本宫也想有好一点儿的出身,也想有先生教习诗书。可是本宫的阿玛在时无暇顾及这些,他心里只有儿子,没有女儿。等阿玛过世了,便更没有这样的机会了。本宫每每见皇上和皇后谈论诗书,心里总是羡慕。为什么本宫的前半辈子,就这么潦潦草草过去了。''

春婵的手上加了几分力气,牢牢扶住嬿婉如掌上飞燕般轻盈的身姿:``前半辈子过去了不要紧,小主,咱们要紧的是下半辈子。''

有泪光在嬿婉眼底如星芒一闪,很快便消逝不见。嬿婉站直了身子,声音瞬间清冷如寒冰般坚硬:``是。咱们只看以后!''她顿一顿,``春婵,本宫和皇后的脸像不像?''

春婵仔仔细细看了许久,怯怯道:``只有一点点,实在不算很像。''

嬿婉的笑声在夜风里听来玲玲玎玎,有玉石相击的冷脆:``哪怕脸像,本宫的心也断断不会和她一样!''

嬿婉的话音散落在风中,回应她的唯有远远的几声犬吠。嬿婉的脸上闪过无可掩饰的厌恶,烦憎道:``讨厌的人,养的狗也讨人厌!''

春婵忙忙劝道:``小主讨厌,除了便是了!反正猫儿狗儿的,病死的也有许多。''

心念旋转如疾电,嬿婉沉闷的心头刹那被照亮,微微一笑不言。

\hypertarget{ux7b2cux4e03ux7ae0-ux897fux98ceux51c9}{%
\chapter{第七章 西风凉}\label{ux7b2cux4e03ux7ae0-ux897fux98ceux51c9}}

夜色如轻纱扬起,四散弥漫。倏尔有凉风吹过,不经意扑灭了几盏摇曳的灯火。容珮侧身逐一点亮灯盏,动作轻悄无声。偶尔有烛火照亮她鬓间的烧蓝点珠绢花,幽蓝如星芒的暗光一闪,仿佛落蕊芳郁,沉静熠熠。

如懿拿拨子挑抹琴弦,反反复复弹着一曲晏殊的《蝶恋花》。宋词原本最合红妆浅唱,何况是晏殊的词,是最该十六七岁女郎执红牙板在雨夜轻吟低叹的。如懿一向不擅歌艺,只是爱极了宋词的清婉秀致,口角吟香,便取了七弦琴细细拨弄,反复吟诵。

``碧草池塘春又晚,小叶风娇,尚学娥妆浅。双燕来时还念远,珠帘绣户杨花满。绿柱频移弦易断,细看奏筝,正似人情钥。一曲啼乌心绪乱,红颜暗与流年换。''

这样哀凉的词,念来犹觉心中沁凉。

容珮默默上前添上茶水,轻声问道:``花好月圆之夜,娘娘正当盛时,怎么念这么伤心的词呢?''

如懿轻哂,该如何言说呢?晏殊明明是个男子啊,却这般懂得女儿心肠。若是有这样一个人,在这样苍苔露冷、花径风寒的日子里常相伴随,明白自己种种不可言说的心事,那该有多好啊!这样的心念不过一转,自己也不禁失笑了。她是皇后啊,高高在上的皇后,在这金堆玉砌的锦绣宫苑中,到头来不过是怀着和平凡妇人同样的梦想而己。

正沉吟间,却见一道长长的影子不知何时映在了地上。如懿举眸望去,却见皇帝颀长的身影掩在轻卷的帘后,面色如霞,深深望着她不语。

惊异只在一瞬,如懿连忙起身下拜:''皇上万福金安。''她抬首,闻到一阵醇然的酒气,不觉道,``夜深了,皇上喝了酒怎么还过来?李玉呢?''

皇帝缓步走近,脚下微微有些踉跄,却迎住她,将她紧紧揽入怀中:``朕在永寿宫陪令妃过寿,秦筝那么刚冷的乐器都能被令妃弹得如斯甜腻。如懿,你的月琴却是醒酒的。朕从栩坤宫外经过,听见你的琴音,便忍不住进来了。''

如懿在他突如其来的拥抱里动弹不得,只得低低道:``臣妾琴音粗陋,惊扰皇上了。''她微微侧脸,吩咐退在一旁低首看着脚尖的容珮,``给皇上倒上热茶,再去备醒酒汤来。''

皇帝并不肯放手,只将脸埋在她颈窝里,散出温热潮湿的气息,每一字都带了沉沉的酒气:``如懿,你比朕前两日见你时又清减了些许。你穿截得真好看,天水碧色很衬你,可是你的眉梢眼角略微带了一丝郁郁之气。''

如懿低首,看着自己身上的天水碧色暗绣芙蓉含露寝衣。那样清素的颜色,配着自己逐渐暗转的年华,大概是很相宜的。只是皇帝突兀的亲昵,忽然唤起了她沉睡已久的记忆。初入潜邸的那些年岁里,他也喜欢这样拥着自己,细语呢喃。

皇帝抬起头,盯住她的眼睛,醉意里有一丝漠漠轻寒:``如懿,朕与你几十年夫妻,你陪着朕从皇子成为君王,朕陪着你从娘御而至皇后,朕和你有一双儿女,聪慧可爱。如懿,你还在难过什么?''他靠得更近一些,``不要说你很高兴,朕听你念那首词,朕知道,你心里其实是难过的。''

阁中立着一架玉兰鹦鹉镏金琉璃立屏,十二扇琉璃面上光洁莹透,屏风一侧有三层五足银香炉,镂空间隙中袅袅升起乌沉香。那是异邦进贡的香料,有厚郁的芬芳,仿佛沉沉披拂在身上。如懿侧首看见自己不饰妆容后素白而微微松弛的肌肤,不觉生了几分自惭形秽。她知道的,宫苑之中,她并非最美,彼时有意欢,近处亦有金玉妍。而皇帝的秀目丰眉、姿容闲疏,仿佛并未被年岁带去多少,反而多了一层被岁月浸润后的温和,像年久的墨,被摩擎多年的玉,气质冷峻高远而不失温润。

哪伯有一双儿女,他们之间,终究是会慢慢疏离的吧?这样的念头在如懿心间一跳,竟扯出了生生的疼。她从未想过,自己会有这样不祥的念头。

如懿的声音低微得像蝴蝶扑棱的翅:``臣妾只是伤感红颜易老,并无他念。''

皇帝轻轻一嗤:``红颜未老恩先断,是不是?那种末等殡妃的伤感之念,皇后尊贵之身,何必沾染?且朕自问殡妃虽多,但不算寡恩,便如婉殡之流,每隔一两月也必会去坐坐看望。''

``皇上自然不算寡恩之人。''如懿勉强一笑,``只是臣妾虽得皇上厚爱,但思及平生,总有若干不足之念。譬如,臣妾出身乌拉那拉氏;譬如,臣妾的阿玛早亡,不得看见臣妾封为皇后的荣光;譬如,乌拉那拉氏族中凋零。臣妾总是想,若无皇上赐予臣妾正位中宫的荣光,或许臣妾的日子会一直黯淡下去吧。''

如懿语中的伤感好似蒙蒙细雨,沾染上皇帝的睫毛。他摩擎着光腻的茶盏,静静听着,良久,轻声道:``朕有时候总是做梦,尤其是在百日大典之后,朕会梦到自己的额娘。''皇帝的声音像被露水沾湿的枯叶,瑟瑟有声,``朕从来就没有见过她的样子。真的。朕出生的时候她就难产而死。朕从懂事起就知道这样出身卑微的额娘是朕的耻辱,朕的母亲只有如今的皇额娘,当年的熹贵妃。朕也很想太后就是联的亲额娘。''他苦笑,``如今看来,朕竟也是做梦。哪怕朕以天下之富奉养太后,哪怕平日里可以母慈子孝,可到了要紧时候,不是骨肉血亲便到底也不是的。''他一哂,眉眼间有风露微凉,``母子不似母子\ldots{}''

有半句话如懿咽了下去,夫妻也不似夫妻啊!这不就是宫廷深深里的日子么?

如懿低低道:``太后还是不肯见皇上么?''

乌沉香细细,一丝一缕沁入心腑,耳边只剩下皇帝风一样轻的叹息:``太后心中只有亲生的公主而己,并没有朕这个儿子。''他的叹息戛然而止,``自然,无论太后怎样待朕,准噶尔之战是不会停止的。朕能做的,只有尽量保全端淑的安全。仅此而已。''他的笑有些无奈,``有时候看来,太后真是一个倔强而强势的女子。哪怕近日她在慈宁宫闭门不出,潜心祈愿,前朝仍有言官不断向朕进言,请求先救端淑再攻打准噶尔。''他苦笑,``联对太后,着实敬畏,也敬而远之。''

如懿的手以蝴蝶轻触花蕊的姿势温柔拂上他醺红的面颊:``太后的确威势,也足以让人敬畏,但是皇上不必太过放在心上。太后曾对臣妾说过,一个没有软肋的人,才能真正强大。而两位长公主,正是太后最大的软肋。''

``软肋?''皇帝轻笑,眼中却只是寒星般的微光,并无暖惫,``那么朕的软肋是什么?如懿,朕会是你的软肋么?''

锦帷绣幔低低垂落,夜寒薄薄侵人。清夜漫漫,因着他此身孤寒寥寥,撩起如懿心底的温情。

原来,他们是一样寂寞的。她默然靠近他,伸手与他紧紧拥抱,拥抱彼此的默契。

这一刻,心如灯花并蕊开。

宫中的夜宁静而清长,并非人人都能和如懿与皇帝一般安稳地睡到天亮。

外头风声呜呜,嬿婉一整夜不能安枕,起来气色便不大好。春婵知道嬿婉有起床气,和澜翠使了个眼色,越发连梳头也轻手轻脚的。小宫女捧了一碗花生桂圆莲子羹进来,澜翠接了恭恭敬敬奉在嬿婉跟前。嬿婉横了一眼,不悦道:``每日起来就喝这个,说是讨个好彩头,喝得舌头都腻了,还是没有孩子。什么`莲'生贵子,都是哄本宫的!''

澜翠如何敢接话,这粥原也本是嬿婉求子心切,才嘱咐了每日要喝的。嬿婉抬头见镜子里自己的发髻上簪着一枝金镶珍珠宝石瓶簪,那簪柄是``童子报平安''图案,一颗硕大的玛瑙雕琢成舞蹈状童子,抱着蓝宝石制宝瓶,下镶绿松石并珊瑚珠,枝杈上缠绕金累丝点翠花纹、如意,嵌一``安''字,那本是嬿婉特特嘱咐了内务府做的,平日里甚是心爱,总是戴着。此刻她心里有气,伸手拔下往妆台上一撂,便是``咚''的一声脆响。

澜翠和春婵吓得噤若寒蝉,更不敢说话。嬿婉正欲站起身来,忽然身子一晃,扶住额头道:``头好晕!''

她话未说完,俯身呕出几口清水。澜翠和春婵急急扶住她,脸上却不觉带了喜色:``小主头晕呕吐,莫不是\ldots{}''

二人相视一眼,皆是含笑。嬿婉半信半疑,满面欢喜:``那,是不是该去请太医\ldots 快请太医。''

话音未落,却是太监王蟾在外头回禀道:``小主,齐太医来请平安脉了。''

齐鲁是皇帝身边多年的老太医了,自嬿婉当宠后一直为她调理脉息。嬿婉当下不敢怠慢,喜不自胜道:``来得正好,还不赶紧请进来!''

齐鲁进来便恭恭敬敬行过礼,待澜翠取过一方手帕搭在嬿婉手腕上,他方才伸出手凝神搭脉。片刻,他又细看嬿婉神色,问道:``小主今日有呕吐么?''

``这是第一次。''嬿婉急切道,``齐太医,本宫可是有孕么?''

齐鲁摇头道:``脉象不是喜脉。''他见嬿婉的笑意迅疾陨落,仍继续问道,``微臣开给小主的汤药,小主可按时吃么?''

春蝉忙道:``小主都按时吃的,一次也没落下。''

齐鲁微微点头,又看嬿婉的舌苔,神色似乎有些凝重。

嬿婉着急道:``本宫一直按照齐大人所言调养,更加了好些滋补汤药,就是希望尽快有孕,可为何迟迟没有动静?''

齐鲁神色郑重,亦是叹惋:``微臣伺候令妃小主己经有一段时日,小主一直急着有孕,不听微臣之言,进补过甚,反而闹得气血虚旺,不能立即有孕。''

嬿婉的身体迫向前一些:``那到底有没有快些有孕的法子?''

``这个么\ldots{}''齐鲁沉吟,捋须不语。

嬿婉使一个眼色,春婵转入内室,很快捧出一个锦盒,打开,里头的珍宝闪耀,直直送到齐鲁脸跟前,晃得他睁不开眼睛。

齐鲁一怔,忙起身道:``小主,小主,微臣不敢。''

嬿婉衔了一缕浅浅的笑意:``这么点儿心意,当然让齐大人不为所动。齐大人放心,这只是十分之一的数目,若本宫能快快有孕,为皇上诞育子嗣,来日一定奉上十倍之数,供大人赏玩。''

齐鲁望着锦盒中闪耀的各色宝石,心想他在宫中当差多年,虽得皇帝重用,也不过一介太医,何曾见过这么多珠宝。想来嬿蜿得皇帝宠遇最深,这些珠宝玉器在她眼中不过尔尔。他眼中闪过一丝贪婪之色,双手因为激动微微有些颤抖,目光不觉看向嬿婉。

嬿婉扬着水葱似的手指,轻笑道:``本宫得皇上宠爱,有孕生子是迟早之事,只是希望得齐太医相助,越早有孕越好。这样简单的事,太医也不肯帮本宫一把么?''

齐鲁拿袖子擦了擦脸上沁出的汗水,迟疑着道:``办法不是没有。要想尽快有孕,可用汤药调理。譬如说每年十次月事的,可调理成每年十二次或者更多,这样受孕的机会也多。但是药皆有毒性,哪怕微臣再小心,总会有伤身之虞,何况是这样催孕的药物。小主三思。''

嬿婉秀眉一挑,急急道:``真有这样的法子?灵验么?''她到底有些后怕,``可有什么坏处?''

齐鲁不敢不直言,``这个么\ldots 月事过多,自然伤女子气血,容易见老!''

一丝俱色和犹豫凝在嬿婉眉心,她喃喃迟疑:``很快就会见老么?''

齐鲁忙道:``现下自然不会,但三五年后,便会明显。''

嬿婉情不自禁地伸出手,抚上自己滑若春绸的肌肤。对镜自照的时候,她犹是自信的。因着保养得宜,或许也是未曾生育过,比之更年轻的忻殡、颖殡之流,她并不见老,一点儿也不,依旧是吹弹可破的肌肤,丰颜妙目,顾盼生色。

所有的犹豫只在一瞬,她的话语刚毅而决绝:``那就烦请齐太医用药吧!''

宫中的日子平淡而短浅,乾隆二十年的春日随着水畔千万朵迎春齐齐绽放,香气随着露水被春阳蒸熨得氤氲缭绕,沁人心脾。这一年的春天,就是这般淡淡的鹅黄色,一点一点涂染了深红色的干涸而寂寞的宫墙。

朝廷对准噶尔的战事节节胜利,很大一部分是因为车凌率部归附后,在平定达瓦齐的战争中出尽全力,所以前线的好消息偶尔一字半句从宫墙重重间漏进时,平添了殡妃们的笑语,也隐然加深了慈宁宫中静修祈愿的太后的优惧。

而后宫中也并非没有喜事,去岁入宫初承思泽的忻缤很快就有了身孕,着实让皇帝欣喜万分。

如懿奉皇帝之命照顾有孕的忻嫔,也添了几许忙碌,然而众人说笑起来,皆是孩子们的事,倒也十分有趣。

这一日,如懿和海兰正陪着忻嫔往宝华殿上香归来,转首见风扑落了忻嫔的帷帽,忙叮嘱道:``仔细别着了风,这个时候若是受凉吃药,只怕会伤着孩子呢。''

忻嫔脸上一红:``皇后娘娘说得是,只是哪里就那么娇贵了呢。''

海兰笑着替她掠去鬓边一朵粉色的落花:``哪里就不娇贵了呢?等生下一位小阿哥,只怕指日就要封妃了呢。''

忻嫔自然高兴,也有些担忧:``那若是个小公主呢?皇上会不会不喜欢?''

海兰忙道:``怎么会不喜欢?皇上本就阿哥多,公主才两位。你瞧四公主五公主就知道了,皇上多喜欢呢。''

如懿道:``阿哥和公主自然都是好的。如今妃位上只有令妃和愉妃,是该多些人才热闹。''她的目光里皆是温暖的关切,``且你年轻,阿玛为准噶尔的事出力,皇上又这样疼你,封贵妃也是指日可待的。''

话音尚未被风吹散,只听横刺里一声犬吠,一只雪白的巴儿狗跳了出来。忻嫔吓得退了一步,正要呵斥,却见后头一个宫装女子缓步踱了出来,唤道:``富贵儿,仔细被人碰着,小心些!''

如懿定睛一看,那人却是多日不出门的嘉贵妃金玉妍。她虽不比当初得意,衣饰却不减华贵,一色明绿地织金纱翔凤氅衣,挽着雪白绸地彩绣花鸟纹领子,垂下蓝紫二色水晶璎珞,裙上更是遍刺金枝纹样,行动间华彩流波。她侧首,发髻间密密点缀的红晶珠花簪和并蒂绢花曳翠摇金,熠熠生辉。

忻嫔当下不悦,低声嘀咕道:``都什么年纪了,还打扮得这样娇艳。''

海兰扯了扯忻嫔的衣袖,示意她不要多言。玉妍向着如懿草草肃了一肃,便横眼看着海兰与忻嫔,二人只得屈膝:``嘉贵妃万福。''

玉妍冷眼看着忻嫔,皮笑肉不笑道:``如今身子重了,人也见胖了。女人啊,就是不能怀着身孕,一怀上穿什么都不好看了,肚子跟顶了口锅子似的!''她冷笑一声,``忻嫔妹妹,如今有孕,皇上也不大去看你了吧?''

忻嫔年轻气盛,哪里受得了这样的话,当即道:``妹妹年轻,自然穿什么都是好看的!比不得人老珠黄还在那里妖调做作!且妹妹虽然有孕,皇上却还眷顾,不像有些人,生出了不肖子孙,让皇上讨厌!''

玉妍如何听不出她言语中的讥讽,当下沉了脸道:``本宫生的什么孩子本宫自己知道。''她死死盯着忻嫔隆起的肚腹,``那你怀了什么东西,你自个儿知道么?如今是欢喜,可千万别是空欢喜了!''许是她的语调略高,脚下名唤``富贵儿''的小狗便凶神恶煞地朝着忻嫔连连吼叫。

忻嫔厌恶不己,又有些害怕,往后退了几步,脸上却毫不示弱:``旁人的空欢喜我是看不见,嘉贵妃娘娘欢喜不欢喜,我倒是看得真真儿的。''

玉妍见忻嫔怕狗,眼中闪过一丝暗喜,用脚尖踢了踢``富贵儿'',驱它向前。忻嫔害怕地躲到海兰身后,急急唤道:``愉妃姐姐。''

如懿原本只冷眼看着,但见玉妍仗犬行凶,便道:``嘉贵妃不是身子不爽不能安枕么?今日天气甚好,回去好好眠一眠吧。''

玉妍咬了咬唇道了声``是'',凤眼横飞斜斜看着忻嫔道:``怕嫔妹妹,有着身孕便少出来走动,若是磕着碰着了,别怪旁边人不当心,只怪你这做娘的自己胡乱晃悠罢了。''她说罢,弯下身亲热地抱起``富贵儿'',兀自转身就要走。

如懿见她这般张狂,早含了一丝怒气,道:``跪下!''

玉妍见如懿发话,一时也不敢离开,只得转身道:``臣妾没做错,为什么要跪?''

如懿神色恬然,微冷的语气却与这三春景色格格不人:``你是贵妃,位分尊贵。你又早进宫,替皇上生儿育女,该知道如何体恤姐妹,照拂孩子。如今你的畜生冒失,自然是你管教不当。''

偏偏忻嫔嘴上不肯饶人:``畜生管教不当也罢了,若自己的孩子都管教不当,那便真是可怜了。''

玉妍气咻咻一哼:``本宫的孩子管教不当,你的便好了么?看生出来是什么再议论吧!''

忻嫔拈起绢子轻轻一笑,正要说话,却见后头嬿婉携了春婵走近,人未至,语先笑:``好不好的总有五阿哥和十二阿哥做榜样呢。瞧皇上多喜欢五阿哥呀,真是最最孝顺有出息的呢。''

玉妍素来不喜嬿婉,见了她便蹙眉:``这样的话,没生养的人不配说!''

嬿婉怯弱弱地行了一礼,含了一缕温文笑意:``妹妹是没有生养,所以羡慕皇后和愉妃、忻嫔的福泽呢。至于嘉贵妃姐姐嘛\ldots{}''她眼神一荡,转脸对者海兰道:``孩子多有什么好,个个争气才是要紧的呢。听说五阿哥最近很受皇上器重,愉妃姐姐真是有福呢。''

海兰神色淡淡的:``有福没福,都一样是皇上的孩子罢了。''

有深切的嫉恨从玉妍婉好的面庞上一闪而过,她盯着海生道:``我的孩子没福了,就轮到你的孩子有福?别做梦了,我就眼睁睁看着,你的永琪夺了本宫永珹的福气,便能有福到什么时候去!''她说罢,拂袖离开。

嬿婉掩袖道:``哎呀!嘉贵妃静养了这些时候,火爆脾气竟一点儿没改呢,当着皇后娘娘的面还这般口不择言,真是无礼。''

如懿看也不看她一眼:``嘉贵妃的火爆脾气不改,你的嘴也未曾说出什么好听的话来,惯会调三窝四挑人嫌隙。''

嬿婉忙忙欠身道:``皇后娘娘,臣妾只是看不过眼\ldots 臣妾\ldots{}''她一急,眼中便有泪珠晃了晃。

如懿懒得看她,径自携了海兰的手离开,亦嘱咐忻嫔:``你怀着孩子,肝火不必那么大。等下本宫会让人送《金刚经》到你宫中,你好好念一念,静静心气吧。''

嬿婉看着如懿与海兰离开,久久欠身相送,神色恭谨异常。片刻,她方站起身,任穿过长街的风悠悠拂上自己的面庞,轻声道:``春婵,你从宫外抱来的那只小狗在哪儿?咱们去瞧瞧。''

春婵道:``在烧灰场那儿交给小太监养着呢,那儿太脏,怕那狗惊了小主,而且那狗\ldots{}''她有些害怕,不敢再说下去。

嬿婉含了稳稳的笑意:``远远地看一眼,就远远地看,本宫喜欢那样的小东西。''

\hypertarget{ux7b2cux516bux7ae0-ux8427ux5899ux6068ux4e0a}{%
\chapter{第八章
萧墙恨(上)}\label{ux7b2cux516bux7ae0-ux8427ux5899ux6068ux4e0a}}

三月的时节,天暖气清。

忻嫔自被如懿提点过几句,也安分了不少。她到底是聪慧的女子,识进退,懂分寸。闲来时海兰也说:``其实令妃似乎很想接近娘娘,求得娘娘的庇护。''

如懿望着御苑中开了一天一地的粉色杏花,风拂花落如雨,伸手接在掌心,道:``你也会说是似乎。难不成你怜悯她?''

海兰低首:``不。臣妾只是觉得令妃的恩宠不可依靠。没有孩子,在这个宫里,一切都是假的。''

``有孩子就能好过到哪里么?你看嘉贵妃便知了。''如懿抬首,见一树杏花如粉色雪花堆拥,又似大片被艳阳照过的云锦,芳菲千繁,似轻绡舒卷。枝丫应着和风将明澈如静水的天空分隔成小小的一块一块,其间若金粉般的日光灿灿洒落,漫天飞舞着轻盈洁白的柳絮,像是被风吹开的雪朵,随风翩翩轻弋,摇曳暗香清溢。

二人正闲话,却见三宝匆匆忙忙赶来,脚下一软竟先跪下了,脸色发白道:``皇后娘娘,八阿哥不好了!''

八阿哥正是玉妍所生的皇八子永璇,如今已经九岁,鞠养在阿哥所。玉妍所生的四阿哥永珹已被皇帝疏远冷落,若八阿哥再出事,岂不是要伤极了玉妍之心。

如懿与海兰对视一眼,连忙问:``到底什么事?''

三宝带了哭腔道:``几位阿哥都跟着师傅在马场上练骑射,不知怎么的,八阿哥从马上摔了下来,痛得昏死过去了!''

海兰便问:``奴才们都怎么伺候的?当时谁离八阿哥最近?''

三宝的脸色更难看:``是\ldots 是五阿哥最近,所以是五阿哥伸手想救八阿哥,可是来不及。那马儿跟疯了似的跑,谁也拦不住啊,只能眼睁睁看着八阿哥摔下马来了!''

海兰脸色发白,人更晃了一晃。如懿情知不好,哪怕要避嫌隙,此刻也不能避开了,忙问道:``八阿哥人呢?''

海兰亦急得发昏,连连问:``五阿哥人呢?''

三宝不知该先答谁好,只得道:``五阿哥和侍卫们抱了八阿哥回阿哥所了,此刻太医正在救治呢。''

如懿连忙吩咐:``去请嘉贵妃到阿哥所照拂八阿哥。愉妃,你跟本宫先去看看!''

阿哥所内己经乱得沸反盈天,金玉妍早己赶到,哭得声嘶力竭,成了个泪人儿。见了如懿和海兰进来,对着如懿尚且不敢如何,却一把揪住了海兰的衣襟撕扯不断,口口声声说是永琪害的永璇。

永琪何尝见过这般阵势,一早跪在了滴雨檐下叩头不止。如懿看得心疼,忙叫宫人伸手劝起。不过那么一刻,海兰己经被玉妍揉搓得衣衫凌乱,珠翠斜倒,玉妍自己亦是满脸泪痕,狼狈不堪。

如懿当即喝道:``都闹成这个样子,叫太医怎么医治永璇!''

众人草草安静下来,如懿不容喘息,即刻吩咐道:``今日在马场伺候八阿哥的奴才,一律打发去慎刑司细细审问。还有太医,八阿哥年幼,容不得一点儿闪失,你们务必谨慎医治,不要落下什么毛病。嘉贵妃,你可以留在这里陪着八阿哥,但必须安静,以免吵扰影响太医医治。''

如懿这般雷厉风行地布置下去,玉妍也停了喧哗,只是睁着不甘的眼恨恨道:``臣妾听说,永璇坠马之时是永琪离他最近!你!''她死死剜着海兰,``你的儿子夺了永珹的恩宠还不够,还伤了我的永璇!若是永璇有什么闪失,我一定不会饶过你们!''

如懿不动声色将海兰护在身后,以不容置疑的口吻道:``你我都为人母,难免有私情。若是本宫来处置,你也不会心安,所以永琪是否牵涉其中,这件事本宫与愉妃都不会过问,全权交予皇上处置。你若再要吵闹。本宫也不会再让你陪护永璇!''

玉妍无言以对,只得偃旗息鼓,含泪去看顾榻上半身带血的永璇。

如懿见海兰惊惶,轻声安慰道:``事情尚未分明,只是意外也未可知。你自己先张皇失措,反而叫人怀疑。''

海兰忍住啜泣道:``永琪刚刚得皇上青眼,就扯上这些说不清的事,岂非我们母子福薄?''

``是否福薄,不是你们母子能定的。本宫先去看看永琪。''如懿行至廊下,见永琪连连叩首,额头己经一片乌青,心下一软,忙扶住了他道:``好了!你又没错,忙着磕头做什么?''

海兰欲语,泪水险险先滑落下来,只得忍耐着道:``永琪,这件事是否与你相干?''

永琪脸上的惊惶如浮云暂时停驻,他的语气软弱中仍有一丝坚定:``皇额娘,儿臣在这里磕头,并非自己有错,更非害了八弟,而是希望以此稍稍平息嘉娘娘的怒火,让她可以专心照顾八弟。''

如懿松一口气,微笑道:``皇额娘就知道你不会的。至于今日之事,会让你皇阿玛彻查,还你一个清白。''

里头隐约有孩子疼痛时的呻吟呼号和金玉妍无法停止的悲泣。如懿心头一酸,永琪敏锐地察觉她神情的变化,有些犹疑道:``八弟年幼,又伤得可怜,皇阿玛会不会不信儿臣?''

如懿正色道:``你若未做过,坦然就是。''她低声道,``要跪也去养心殿前跪着。去吧,本宫也要去见你皇阿玛了。''

对于如懿的独善其身,皇帝倒是赞同:``你到底是永琪的养母,这些事掺在里头,于你自己也无益。''

如懿额首:``是。臣妾的本分是照顾后宫,所以会命太医好生医治永璇,也会劝慰嘉贵妃。自然了,还有忻嫔呢,太医说她的胎像极好,一定会为皇上生一个健康的孩子。''

皇帝以手覆额,烦恼道:``前朝的政事再烦琐,也有头绪可寻,哪怕是边界的战事,千军万马,朕也可运筹帷幌。可朕的儿女之事,实在是让人烦恼。''

如懿笑吟吟道:``多子多福。享福之前必受烦忧,如此才觉得这福气来之不易,着实可贵。''

皇帝抚着她的手道:``但愿如此。那么这件事,朕便交给李玉去办。''

如懿思付道:``李玉是御前伺候的内臣,若有些事要出宫查办,恐怕不便。此事也不宜张扬,叫人以为皇家纷争不断,还是请皇上让御前得力的侍卫去一起查办更好些。''

皇帝不假思索,唤进凌云彻道:``那么八阿哥坠马之事,朕便交由你带人和李玉同去查办。''

凌云彻的眼帘恭谨垂下:``是,微臣遵旨。''

凌云彻做事倒是雷厉风行,李玉前往慎刑司查问伺候永璇的宫人,他便赶去了马场细查。遇见如懿时,凌云彻正带着四名侍卫与李玉一同从慎刑司归来。

见了如懿,众人忙跪下行礼。为着看顾永璇和忻嫔,这两日她两处来往,不免有些疲倦,眼下也多了两片淡淡的乌墨色。然而嘉贵妃甚是警觉,也不愿让她过多接近,更多的时候,如懿亦只能遣人照顾,或问问太医如何医治。

众人行礼过后,凌云彻忍不住道:``皇后娘娘辛苦,是为八阿哥操心了。''

长街的风绵绵的,如懿从他眼底探得一点关怀之意,也假作不见,只问:``你们查得如何了?''

李玉忙道:``慎刑司把能用的刑罚都用上了,确实吐不出什么来。但是\ldots{}''

凌云彻眼波微转,浑若无事:``是伺候的宫人们不够用心。至于如何责罚,该请皇上和皇后娘娘示下。''

如懿只觉得疲乏,身上也一阵阵酸软,勉强道:``也好。你们去查问,给皇上一个交代便是。''

凌云彻见如懿脸色不大好,忙欠身道:``娘娘而色无华,是不是近日辛苦?''

容珮忙道:``娘娘方才去太医院看八阿哥的药方,可能药材的气味太重,熏着了娘娘,有些不舒服。奴婢正要陪娘娘回去呢。''

李玉忙忙扶住道:``娘娘玉体操劳,还是赶紧回宫休息吧。''

如懿扶了容珮的手缓步离去。李玉凝神片刻,低声向凌云彻道:``凌大人请借一步说话。''凌云彻示意身后的侍卫退下,与李玉踱至庑房檐下,道:``李公公有话不妨直言。''

李玉袖着手,看了看四周无人,才低声道:``听大人方才审问那些宫人的口气,像是在马场有所发现?''

凌云彻一笑:``瞒不过李公公。''他从袖中取出两枚寸许长的银针,``我听说当日八阿哥所骑的马突然发了性子,将八阿哥颠下马来,事后细查又无所见,结果在那匹马换下来的马鞍上发现了这个。''他眼中有深寒似的凛冽,``银针是藏在皮子底下的,人在马上骑得久了,针会穿出皮子实实扎到马背上。马吃痛所以会发性,却又查不出伤痕,的确做得隐蔽。''

李玉听得事情重大,也郑重了神色:``八阿哥身为皇子,谁敢轻易谋害?凌大人以为是\ldots{}''

凌云彻只是看着李玉:``李公公久在宫闲,您以为是\ldots{}''

李玉脱口道:``八阿哥是嘉贵妃的儿子,自然是对谁有利就是谁做的。''他骤然一惊,``凌大人是在套我的话,这样可不好吧?''

``哪里哪里?''凌云彻摆手笑道,``李公公在皇上身边多年,眼光独到,不比我一个粗人,见识浅薄。''

李玉凑近了,神神秘秘道;``凌大人还来探我的话,只怕是心里也有数了吧?您猜是谁?''

凌云彻脸上的严肃转而化作一个浅笑:``或许是意外也未可知。''他指了指蔚蓝的天空,``或许也是天意。''

李玉何等乖觉,即刻道:``那是。皇上交代给凌大人彻查的,凌大人查到什么,那我查到的也就是什么,绝对和凌大人是一样的。''他拱手,``嘉贵妃摆明了失宠,何必为她得罪一个得宠的人呢?且那人都于咱们俩有恩,这就是该报恩的时候了。''

凌云彻将银针笼进袖中,轻轻一笑:``公公的主意就是我的主意。''二人相视一笑,结伴离去。

这样的主意,或许是在查到银针的一刻就定了的,所以即便是与赵九宵把酒言欢,谈及这件事时,他也是闭口不言。宫闱之中波云诡谲,殡妃之间如何血斗淋漓,诡计百出,他亦有所耳闻,何况,玉妍一向对如懿不驯。

隐隐约约地,他也能知道,八阿哥永璇的坠马,固然是离他最近的五阿哥永琪最有嫌疑,也是五阿哥获益最多,让己经元气大伤的玉妍母子再度重创。但若五阿哥有嫌疑,等同生母愉妃海兰和养母如懿都有嫌疑。他是见过如懿在冷宫中受的苦的,如何肯再让她陷落到那样的嫌疑里去。哪怕仅仅是怀疑,也足以伤及她在宫中来之不易的地位。

所以,他情愿沉默下去,仅仅把这件事视作一次意外。

于是连赵九宵也说:``兄弟,你倒是越来越懂得明哲保身了,难怪步步高升,成了皇上跟前的红人。我呢,就在坤宁宫这儿混着吧,连我喜欢的姑娘都不肯正眼看我一眼。''

凌云彻隐隐约约知道的是,赵九宵喜欢永寿宫的一个宫女,也曾让自己帮着去提亲,他只是摆手:``永寿宫的人呵,还是少沾染的好!''

赵九宵拿了壶酒自斟自饮:``你啊,一朝被蛇咬,十年怕井绳,永寿宫的主位不好,难道她手下的人都不好了?''他颓丧不已,``只可惜,连个宫女都看不上我!''

凌云彻捧着酒壶痛饮,只是一笑。赵九宵喜欢的姑娘看不上赵九宵,他自己喜欢的女子,何曾又能把他看在眼里呢?

幸好,赵九宵不是郁郁的人,很快一扫颓然:``但是,我只要能远远地看着她就好了。偶然看见就可以。''

凌云彻与他击掌,笑叹:``你可真是我的好兄弟!''

怎么不是呢?他也是如此,偶尔能远远地看见她就好。在深宫杨花如雪的回廊转角,在风露沾染、竹叶簌簌的养心殿廊下,或是月色如波之中,她被锦被包裹后露出的青丝一绺。

能看见她的安好,便是心安所在。

他这样想着,任由自己伏案沉醉。有隐约的呜咽声传来,恍惚是阿哥所内金玉妍担心的哭泣声,抑或是哪个失宠的嫔妃在寂静长夜里无助的悲鸣。

他只希望,她永远不要有这样伤心的时候。

八阿哥永璇能起来走动已经是一个月后,无论太医如何精心医治,永璇的一条腿终究是废了。用太医的话说,即便能好,这辈子行走也不能如常人一般了。

金玉妍知道后自然哭得声噎气直,伤心欲死。连皇帝亦来看望了好几次,他看着玉妍哭得可怜,便许她携了十一阿哥永瑆一直住在阿哥所照顾永璇的伤势。

如此一来,玉妍养在宫中的爱犬失了照顾,常日呜呜咽咽,更添了几分凄凉之意。好像这春日的暖阳,即便暖得桃花红、柳叶绿,却再也照不暖嘉贵妃母子的哀凉之心了。

宫里的忧伤总是来得轻浅而短暂。说到底,哀伤到底是别人的,唏嘘几句,陪着落几滴泪,也就完了。谁都有自己新的快乐,期盼着新生的孩子,粉白的脸,红艳的唇,柔软的手脚;期盼着孩子快快长大,会哭会笑会闹,期盼着凤鸾春恩车在黄昏时分准时停驻在自己的宫门口,带着满心欢喜被太监们包裹着送进养心殿的寝殿;期盼着君恩常在啊,好像这个春天,永远也过不完似的。

因着永璇坠马之事,皇帝到底也没迁怒于永琪,如此与海兰也放心些,闲来的时候,如懿便陪着一双儿女在御花园玩耍。

春日的阳光静静的,像一片无声无息拂落的浅金轻纱。御苑中一片寂静,春风掠过数株粉紫浅白的玉兰树,盛开的满树花朵如伶人飞翘的兰花指,纤白柔美,盈盈一盏。那是一种奇特的花卉,千千万蕊,不叶而花,恍如玉树堆雪,绰约生辉。

忻嫔挺着日渐隆起的肚腹坐在一树碧柳下的石凳上,凳上铺着鹅毛软垫,膝上有一卷翻开的书。她低首专注地轻轻诵读,神情恬静,十足一个期待新生命降生的美丽母亲。因着有身孕,忻嫔略略丰腴了一些,此时,半透明的日光自花枝间舒展流溢,无数洁白、深紫的玉兰在她身后开得惊心动魄。她只着了一袭浅粉衣裙,袖口绣着精致的千叶桃花,秀发用碧玉扁方绾起,横簪一枝简净的流珠双股簪。背影染上了金粉霞光的颜色,微红而温煦。

忻嫔对着书卷轻声吟诵古老的字句,因为不熟悉,偶尔有些磕磕绊绊:``朝饮木兰之坠露兮,夕餐秋菊之落英。苟余情其信姱以练要兮,长顑颔亦何伤。''

她读着读着,自己禁不住笑起来,露出雪白的一痕糯米细牙:``皇后娘娘,昨儿臣妾陪伴皇上的时候,一直听皇上在读这几句,说是什么屈原的什么《离骚》。虽然您找来了一字一字教臣妾读了,可臣妾还是读得不论不类。''

如懿含笑转首:``宫里许多嫔妃只认识满蒙文字。你在南边长大,能认得汉字己经很好。何况《离骚》本来就生僻艰难,不是女儿家读的东西。离骚,离骚,本就是遭受忧愁的意思,你又何来忧愁呢?''

``臣妾当然是有忧愁的呀!''忻嫔抚着高高隆起的肚子,掰着手指道,``臣妾担心生孩子的时候会很痛,担心会生不下来,担心像愉妃姐姐一样会受苦,像己故的舒妃一样会掉许多头发,还担心孩子不是全须全尾的\ldots{}''

如懿赶紧捂住她的嘴,呵斥道:``胡说什么,成日想这些乱七八糟的!''她换了柔和的语调,``有太医和嬷嬷在,你会顺顺利利生下孩子的。''

忻嫔虽然口中这样说,脸上却哪里有半丝担心的样子,笑眯眯道:``哎呀,皇后娘娘,臣妾是说着玩儿的。''她指着正在嬉闹的永璂和璟兕道,``臣妾一定会有和十二阿哥与五公主一样可爱的孩子的,他们会慢慢长大,会叫臣妾额娘。真好\ldots{}''她拉着如懿的手晃啊晃,像个年轻不知事的孩子,脸上还残存着一缕最后的天真,``皇后娘娘,您和皇上读的书,臣妾虽然认识那些字,却不知什么意思,您快告诉臣妾吧。''

这样的天真与娇宠,让如懿在时光茬再间依稀窥见自己少女时代的影子,她哪里忍心拒绝,笑嗔道:``你呀,快做额娘的人了,还跟个孩子似的。''

忻嫔笑得简单纯挚:``在臣妾心里,皇后娘娘便是臣妾的姐姐了。姐姐且告诉告诉妹妹吧。''

如懿笑着解释道:``这句话的意思是,早晨我饮木兰上的露滴,晚上我用凋落的菊花花瓣充饥。只要我的情感坚贞不移,形销骨立又有什么关系。''

忻懿忍不住笑道:``臣妾听说屈原是个大男人,原来也爱这样别别扭扭地写诗文。不过皇上读什么,原来皇后娘娘都懂得的。''

皇帝是喜欢么?一开始,是如懿喜欢夜读《离骚》,皇帝听她反复歌咏这几句,只是含笑拨弄她两颐垂落的碎发:``屈原过于孤介,才不容于世。他若稍稍懂得妥协,懂得闭上嘴做一个合时宜的人\ldots{}''

如懿抵着皇帝的额头:``若懂得妥协,那便不是屈原了!''

皇帝轻轻一嗤,拥着她扯过别的话头来说。

忻嫔兀自还在笑:``一个大男人,老扯什么花啊草啊的来吃,真是可爱!''她一说可爱,永璂便拍起手来,连连学语道:``可爱!可爱!''

忻嫔与如懿相视一眼,都忍不住笑了起来。

永璂己经快三岁了,璟兕快两岁,一个穿着绿袍子,一个穿着红裙,都是可爱的年纪。永璂跑得飞快,满地撤欢儿。璟兕才刚刚会走,像扑梭着翅膀学飞小鸟,跟在哥哥身后,笑声如银铃一般。

柳桥花坞,落花飞絮,长与春风作主。大约就是这样的好时光吧

\hypertarget{ux7b2cux4e5dux7ae0-ux8427ux5899ux6068ux4e0b}{%
\chapter{第九章
萧墙恨(下)}\label{ux7b2cux4e5dux7ae0-ux8427ux5899ux6068ux4e0b}}

正笑闹着,远处金玉妍扶着八阿哥永璇拄着拐杖慢慢地走近。听见这里的笑语连连,愈加没有好气,狠狠啐了一口道:``有什么好笑的,今儿且乐,瞧你们能乐到什么时候?''她骂完,眼眶便红了。

永璇拄着拐杖,一步一步艰难地走着,没走几步便呜咽告饶:``额娘,我的腿好疼,我走不动,我走不动了!''

玉妍眼中含泪,死死忍着勉强笑道:``好永璇,好好走,走一走就不疼了!''

永璇听得母亲哄,勉强又走了两步,大概是疼痛难忍,丢了拐杖哭道:``额娘,我不走了!我不走了!''他脚下一滑,一屁股跌坐在地上,放声大哭道,``额娘!我的腿是不是残废了,永远也不会好了!''

玉妍心疼得直哆嗦,紧紧抱住永璇道:``儿子!额娘知道是他们害你,是他们一伙儿害你!他们害了你哥哥还不够,连你也不肯放过!''她生生落下泪来,``额娘没用,不能护着你们。''她使劲推着永璇,用力推,用力推,仿佛这样就能代替他残疾的再也无法伸直的另一条腿,``起来!起来!咱们再走走,额娘扶着你。''

永璇忍不住哭道:``额娘,可是我疼,我好疼!''

玉妍眼里含了一丝狠意,死死顶着永璇不让他倒下来,发狠道:``再疼你也忍一忍。永璇!你的哥哥已经失宠了,永瑆还小,你若撑不住,额娘和李朝母族就真的没指望了!咱们再走走,再走走!''

玉妍推着永璇,一点一点往前走,两个人紧紧依偎着,单薄的身影在春日迟迟里看来格外凄凉。

日色渐渐地黯淡下去,被花影染成浅浅的微红,如懿起身笑道:``天有些凉了,咱们回去吧!''

她的话音未落,横刺里一只灰色的动物猛窜了出来,一时狂吠不己。如懿吃了一惊,忻嫔早已躲到了如懿身后,惊慌道:``哪里来的狗!快来人赶走!快!快!''

宫人们乱作一团,赶紧去驱赶。如懿定睛看去,那是一只脏乎乎的巴儿狗,不知从哪里跑出来的,毛色都失了原本的雪白干净,脏得差点辨不出本来的样子。那狗的眼睛血红血红的,没命价地乱窜,狂躁不己。

如懿只觉得眼熟,却想不起在哪里见过。她只怕伤着孩子,又怕伤着有孕的忻嫔,立时喝道:``赶紧赶走它!''

那狗却像是不怕人似的,窜得更快了,任凭宫人们呼喝,却扑不住它。突然一个跳跃,它便绕道假山石上,向着忻嫔扑来。忻嫔哪里来得及躲闪,腿一软便坐在了石凳上,害怕得尖叫不己。那狗却不理会她,从她肩膀上跳下,直扑向永璂,偏偏永璂没见过狗,大概觉得好玩,站在原地拍着手又跳又笑。

如懿吓得心惊胆战,忙喝道:``永璂!那狗好脏,玩不得的!''

永璂愣了愣,停住了要上前的脚步。更年幼的璟兕看着众人忙乱不己,突然笑着扑了过来,呀呀道:``好玩!好玩!''

那是一身灼灼红色的苏绣衣裙,满满绣着麒麟绣球的花样,连衣角那绣着缠枝宝相花,那花边都用金线细细掠过,在阳光下如细细碎碎的金波荡漾,夺目而娇艳。那是三月三上巳节的时候各宫嫔妃送来的礼物中的一件。庆嫔裁衣,晋嫔做的针线,才捧出这么簇锦似的华衣,特特送给璟兕的。

这样如石榴花般夺目的衣裙,瞬间吸引了那癫狂的狗。那狗像是受到了极大的刺激,几乎是没有犹疑地发疯一样扑向了璟兕。

根本来不及去救,只听见幼儿惊惶凄惨的哭叫声,狗的狂吠声,宫人们的惊呼,还有如懿自己不知如何从喉咙中发出的凄厉的尖叫。只见血花如那艳红的衣衫一般飞溅开来,如懿几乎晕了过去!

也不过是一瞬,就有宫人抢身上去救璟兕。旋即,有更多的宫人涌上去,拿着棍子或石头,或是折下树枝,一切触手可及的工具,手忙脚乱地驱赶那条疯狗!

忻嫔的身体剧烈地摇晃着,凄惶而无助地指着地上喊:``血!好多血!''

是璟兕发疯般的哭喊后又晕厥过去之后身体上沁出的血,还是忻嫔的裙上蜿蜒而下的如红河般的血水。

如懿直冲上去,抱起昏厥过去的璟兕,浑然不觉泪水沾了满面,无助地狂喊:``太医!太医呢?''

璟兕的伤势很严重。

也许是被璟兕的红衣吸引,也许是璟兕皮肉娇嫩,那狗疯狂之下咬了好几口,处处犬牙交错,皮肉翻起,深可见骨。璟兕己不省人事,如懿看着太医惊慌失措的面容,一颗心像是被辘辘碾着,分明己经碎得满是残渣,在冷风里哆嗦着,却又一遍一遍凌迟般被压碾而过。

皇帝赶来时太医己经团团围住了璟兕,止血的止血,上药的上药。

而璟兕的小脸惨白,完全人事不知。

皇帝眼看着嬷嬷们用剪子小心翼翼剪开璟兕凌乱残破的衣衫,眼看着太医们一点一点查验伤口、涂抹药粉,听着璟兕昏迷中痛楚的呻吟,他这样的一个大男人,见惯了战事征杀的男人,他的双手居然也在颤抖,眼里也有止不住的泪。

如懿伏在皇帝怀中,被他紧紧地抱着,仿佛唯有这样,才能止住彼此身体的颤抖。皇帝拍着如懿的肩:``别怕!别怕!皮肉伤而己,没有伤筋动骨,就是不要紧的!''他下手极重,拍得如懿肩头一阵阵痛,嘴里喃喃道:``我们的璟兕这么可爱,一点点皮外伤,哪怕留了点疤,也不会难看的。我们的璟兕\ldots{}''

有温热的泪水落在如懿脸颊上,和她的泪混在一起,潸潸而下。此刻,他们的痛心是一样的。他们的手也紧紧握在一起,支撑着彼此。

这时,三宝进来,打了个千儿,语气里已经隐然含了一丝恨意:``皇上,皇后娘娘,奴才已经带人查明了,那条疯狗\ldots{}''他咬了咬牙,切齿道,``咬伤公主的疯狗是嘉贵妃娘娘豢养的,叫作`富贵儿'!''

皇帝的怒意似火星般迸溅:``那条狗呢?立刻打死!''

``回皇上的话,那狗已经死了,有小太监在假山石头缝里发现了尸体,大约是逃跑的时候自己撞死了!''三宝的语气里含着隐忍克制的恨意,``嘉贵妃娘娘此刻就跪在殿外,要向皇上陈情!''

皇帝怒喝道:``连个畜生都看不住,她还敢来!''

皇帝夺门而出,赶来探视的嫔妃们因不得准许,都在庭院中候着,正议论纷纷,看见皇帝出来,忙鞠身行礼,顷刻间安睁了下来。

金玉妍含了几分怯色跪在廊下,似是受足了委屈,却实在不敢言语。她一见了皇帝,如见了靠山一般,急急膝行到皇帝跟前,抱住了他的双腿放声大哭道:``皇上!皇上!臣妾是冤枉的!臣妾一直在阿哥所照顾永璇,臣妾也不知`富贵儿'怎么会突然发疯跑去咬五公主!皇上!臣妾实在是不知啊!您不能怪罪臣妾,臣妾是无辜的啊!''

玉妍嘴上这般哭喊,到底还是害怕的,眼珠滴溜溜转着,眨落大颗大颗的泪珠。皇帝气得目毗欲裂,伸手便是两个耳光,蹬腿踢开她紧紧抱住的双臂,厉声喝道:``你无辜?那躺在里面的璟兕无辜不无辜?朕的女儿,她还那么小,就要被你养的畜生咬得遍体鳞伤!你在宫里豢养这样的奋生,到底安的是什么心?''

玉妍满脸凄惶,正要辩白,忽见如懿跟了出来,满脸的恨意再克制不住:``皇上,臣妾安的什么心!臣妾倒要问问皇后娘娘,她安的是什么心?''她凄厉呼号,如同夜袅,``皇后娘娘,这是报应!臣妾的永珹和永璇被人算计了,臣妾无能,不能替他们报仇。如今报应来了,恶人自有恶人磨,该轮到她的孩子了!''她呵呵冷笑,如癫如狂,``老天咧,你长着眼睛,你可终于看见了,替我报了仇呀!''

玉妍还要再喊,皇帝早己怒不可遏,一举将她扇倒在地:``你这个毒妇,还敢污蔑皇后!是你驯养的畜生伤人,你还敢攀扯老天爷!''

三宝忙道:``皇上,奴才问清楚了,人人都说这条疯狗平时很得嘉贵妃喜爱,最听嘉贵妃的话了!''

玉妍倒在地上,衣裙沾染了尘灰,满头珠翠散落一地,鬓发蓬乱,狼狈不甘:``臣妾怎敢污蔑皇后娘娘?皇上细想,若臣妾真要害皇后娘娘的孩子,怎不动十二阿哥,不动五阿哥,而要伤了五公主!''

嬿婉站在廊外,一树海棠衬得她身影纤纤。她满脸都是不忍的泪:``很奇怪么?本来嘛,五公主就是皇上和皇后娘娘的心头肉啊!''她声声叹息,抹去腮边几滴泪,``真是可怜,五公主这么小的孩子,伤在儿身,痛在娘心啊!''

颖嫔巴林氏忍不住道:``原来令妃也以为是有人指使的!咱们倒是都想得一样!''她转过脸,望着玉妍幸灾乐祸地一笑。

如懿立在皇帝身后,狠狠剜了玉妍一眼,那眼神如森冷而锋利的剑,恨不能一剑一剑剜出玉妍的肉来,碎成片片。然而她并未动怒,只是将璟兕换下的红衣拎在手中。

海兰扶着如懿,轻声道:``皇上,臣妾听皇后娘娘说起,只是觉得奇怪,听说那条疯狗原本先去招惹的是忻嫔,后又扑向了十二阿哥,可最后为何咬的却是五公主?实在奇怪!''

如懿俯下身,哀婉恳求道:``皇上,臣妾想来想去,那条疯狗本来可能伤害的是忻嫔或者十二阿哥,至于为何突然咬伤了五公主,大约和这件衣衫有关!''她的语气如碰撞的碎冰,生生敲着耳膜,``臣妾记得,这件衣衫是庆嫔裁制,晋嫔绣成的!''

庆嫔陆缨络和晋嫔富察氏本站在人群中,听得此言,吓得慌忙跪了下来,连连摆手道:``皇上,衣衫是臣妾们的心意,但并未想谋害五公主啊!''

皇帝早已气昏了头,如何肯听她们分辩,当下吩咐道:``李玉,拖她们出去各掌嘴三十,罚俸一年,不许再出现在联的跟前!''

李玉答应了一声,正要拖了庆嫔与晋嫔出去,如懿挽住皇帝的手,轻声道:``皇上,事情尚未查清,咱们先别用刑。''她眼圈一红,勉强忍住泪,``璟兕己经这样了,若伤及无辜,只怕也伤了璟兕的福祉。''

庆嫔与晋嫔如逢大赦:``多谢皇后娘娘!''

皇帝极力镇静下来,沉声道:``那就让庆嫔和晋嫔先去宝华殿跪着,替五公主祈求平安。''他揽住如懿,温声安慰;``别怕!别怕!有那么多太医在,璟兕会没事的!''

庭院中寂寂疏落,嫔妃们乌压压跪了一地,鸦雀无声。唯有风簌簌吹过,恍若冰冷的叹息,偶尔有花拂落于地,发出轻微的''扑嗒''``扑嗒''的声响,好像生命凋落时无声的叹惋。

这样的安静让人生了几分害怕。如懿惶惑地依在皇帝身边,脑海中空白一片。直到一个小宫女急急奔近,才打破这惊俱的无声。

却是伺候忻嫔的贴身侍女阿宝,她慌不择路,扑倒在皇帝跟前,哭着求道:``皇上!皇上!不好了!忻嫔小主受了惊吓见了红,陪着的太医说,小主胎气惊动,怕是要早产了!''

皇帝的手明显一搐,额上青筋暴起,瞪着狼狈不堪的玉妍道:``瞧你做的这些好事!''他急忙问阿宝:``忻嫔如何了?接生嬷嬷去了么?''

阿宝哭道:``嬷嬷们已经去了!可是小主的情况很不好,小主一直喊疼,出了好多好多血,一直喊着皇上!''

神思的间隙,如懿想起忻嫔受惊时裙上蜿蜒如注的鲜血,心下也不由得生出一抹担忧。她平静了气息,低声道:``璟兕有臣妾,可是忻嫔只有皇上。''她的手指缓缓离开他温热的掌心,``皇嗣要紧,皇上去看一看吧。''

海兰忙欠身道:``皇上放心,臣妾会在这里陪着皇后娘娘!''

皇帝点头:``李玉,带嘉贵妃回启祥宫,不许任何人探视,也不许她再陪着几位阿哥!''

玉妍还要呼号,李玉使一个眼色,两个小太监上前,死死捂住了她的嘴拉了出去。

皇帝匆匆离去,如懿只觉得疲惫不堪,挥一挥手向嫔妃们道:``你们都退下吧。''

纯贵妃绿筠站在最前头,满脸焦灼:皇后娘娘照料五公主辛苦,臣妾心内不安,愿意随侍。''

绿筠资历颇深,她如此一言,嫔妃们连声道:``臣妾等心内不安,愿愈随侍,照料公主。''

如懿温然道:``你们的心意本宫心领了。''她逐一吩咐,``纯贵妃,你久在宫中,本宫照顾公主,宫中琐事都交由你打理。颖嫔,皇上急着去看顾忻嫔,怕是担心,你去陪着皇上吧。''颖嫔一喜,却不敢笑,忙忙谢恩转身去了。

海兰轻声提醒:``嘉贵妃被禁足,那么水璇和永瑆还在阿哥所。''

可不是,若此时永璇和永瑆再出什么事,旁人必定以为是她报复嘉贵妃,如何还说得清!如懿感念海兰的细心,便向老实人婉茵道:``嘉贵妃禁足,八阿哥足伤未愈,十一阿哥也还年幼。婉嫔,你最稳妥,这些日子便由你在阿哥所照料吧。''

海兰微微颔首,婉茵为人老实忠厚,又胆小怕事,素不和人拉帮结派,只是独善其身,由她去照顾,最无是非了。

如此这般一一安排,如懿方能将一颗动荡不安的心,全数用在照拂璟兕上。只因为,她固然是母亲,更是皇后,再难过,亦不可失了周全。

一灯如豆,残影幢幢。

如懿与海兰陪在璟兕床前,抚摸着她小小的脸蛋。璟兕痛醒过几次,身体也挣扎得厉害,哭声太过凄惶,让人让人耳不忍闻。太医怕她牵动了才包扎好的伤口,只得一点一点灌入安神的药物。于是,璟兕也只在昏睡中呼唤:``额娘!额娘!''

璟兕每一声呼唤,都引下如懿心疼的泪。她仔细查看璟兕的伤处,仿佛那些伤口生了锯齿,也钝钝地磨在自己心上。

海兰拈起绢子,轻柔地为她擦拭:``姐姐,与其哭,不如想想,这到底是为什么。''

如懿的声音静下来:``你也觉得蹊跷?''

海兰的眼里含着锐色,睫毛却如羽轻覆:``狗是不会轻易发疯的,尤其是豢养的狗。''她的声音低柔而犀利,``但是人会发疯。人一疯,狗也跟着疯了。''

如懿遽然惊起:``你是说,有人为了儿子发了疯,所以要赔上本宫的孩子。''

海兰忧心忡忡,眸中有潮湿的雾气:``永璇坠马,永珹失宠,都和臣妾的永琪脱不了干系,她应该冲着臣妾和永琪来。''

如懿神色酸楚:``但永琪是本宫的养子,子凭母贵,何况还有本宫的嫡子永璂。''她眼底的痛楚随着烛火跳跃不定,``永璂应该是首当其冲的。''

有女子凄厉的呼号声交缠着汗水与血水战栗着红墙与碧瓦,旋既又被夜风吹得很远。海兰轻声道:``是忻嫔的声音,听着真惨!''她语中的怜悯如雾轻散,``可惜了,她也逃不脱。只是不知道,金玉妍要对付的,到底是忻嫔还是永璂?''

如懿的手指紧紧攥起,指甲深深嵌入皮肉,恨声道:``金玉妍要对付的,其实是永璂,对不对?忻嫔与她无冤无仇,哪怕生下皇子,也不会危及她和她儿子的地位。而如金玉妍所言,她对本宫有怨,是该对付永璂才对!只是璟兕穿了那件红衫,才会引的那条疯狗扑向她!璟兕真真是无辜!''

``那么庆嫔和晋嫔,总归是有嫌疑的,尤其晋嫔,她可是富察氏的女儿啊!娘娘继位为后,富察氏怎忍得下这口气!''海兰脸上的阴翳越来越重,``无论是谁,这个人都狠毒至极,惊了忻嫔,伤了璟兕,险险也伤了永璂,真是一箭三雕啊!''

如懿看着璟兕在昏睡中依然痛楚的神情,心口一窒,觉得自己就像被火烤着的一尾鱼,慢慢地煎熬着,焦了皮肉,沁出油滴,身心俱焚。

可怜的孩子,真是可怜!如懿咬着牙,霍然起身推窗,对着清风皓月,冷然道:``有本事一个个冲着本宫来!''

海兰依在如懿身侧,摇头道:``她们没本事,动不得姐姐,才只能使这些阴谋诡计!''她的声音清晰且没有温度,``所以姐姐切不可心志软弱,给她们可乘之机!''

如懿缓缓吐出两个字:``自然。''

海兰的声音极轻:``姐姐,您疑心谁?''

如懿闪过一丝凌厉:``谁都疑心!嘉贵妃、庆嫔、晋嫔,谁都不可信!''

海兰靠得她更近些,像是依靠,也是支撑,语中有密密的温情:``姐姐,她们都不可信,我们总在一起!''

如懿用力点头,须臾,``嗒''的一声响,铜漏里滴下了一颗极大的水珠,仿佛滴在如懿的心上,寒冷如九天冰雪,瞬间弥漫全身。她俯下身,紧紧握住璟兕小小的手,贴在自己的面颊上,仿佛唯有这样,才能定下心神来。

容珮悄然走近,唤道:``娘娘。''

如懿头也不回:``什么事?''

容珮的声音里有一丝喜悦:``景阳宫来回话,忻嫔小主生下一位公主。''

如懿微微松了一口气:``知道了。可怜了她,幸好母女平安。''

海兰道:``是早产的孩子。''她掰着指头算了算,``七个月大的孩子,又受了惊吓,得好好养着。''

容珮不敢抬头,只道:``是。报喜的人说,公主的哭声特别弱。''

如懿叹了一声:``你按着规矩,以三倍之数赏赐忻嫔,嘱咐她好好养着,待璟兕好些,本宫便去看她。''

容珮答应着退下了。

夜深幽幽,如懿看着璟兕身上的累累伤口,颤抖着不敢去抚摸,她唯一能做的,只是在太医救治之后,努力祈祷璟兕的伤势想忻嫔平安产下孩子一样,一切都会好起来的,一定会好起来的。

\hypertarget{ux7b2cux5341ux7ae0-ux592dux4ea1}{%
\chapter{第十章 夭亡}\label{ux7b2cux5341ux7ae0-ux592dux4ea1}}

璟兕的高热是在五天后发作的。伤口已经有愈合的趋势,也并未在出血化脓,但是璟兕变得胆小,她拒绝喝水,连看见给她洗漱的清水都会害怕的缩起来。她害怕一切声音,宫人们轻微的脚步声都会让她不安地大哭,甚至连风声都害怕,她一直是恐惧而不安的神色。

起初,如懿以为是那日的事给了她巨大的惊吓,渐渐发觉不对,璟兕有战栗的迹象,恶心呕吐,不愿入睡,并且一反常态地烦躁。

如懿无助地看着江与彬的脸色越来越差,一颗心一点一点地悬了起来。

江与彬惨然道:``娘娘,您得有个准备,五公主怕是得了疯犬病了。那条咬伤五公主的狗\ldots{}''

如懿急急命三宝掘出``富贵儿''的尸体,江与彬查验后回来,连声音都嘶哑了:``皇后娘娘,那条狗的确已经得了疯犬病,所以才会闯入御花园咬伤了五公主。那疯犬病,是会传给人的!''

海兰紧咬下唇,眼中是烈烈恨意:``是金玉妍,是不是?那条狗是她豢养的,一定是她!''

如懿的脸色已经全然失了血色,侧过脸,声音微冷,一字字清去碎冰:``那条狗是金玉妍养的没错,但是它养在启祥宫中,应该很干净才对,为何闯入御花园那天那么脏,而且启祥宫的人也没发现这狗得了病呢?本宫问过三宝,三宝说启祥宫的人提过,那只狗曾经跑丢过几天,一直到出现在御花园咬伤了璟兕。''

容珮恨道:``只有这样,嘉贵妃才撇得清干系啊!''

容珮的话并非没有道理,何况海兰也道:``还有谁比金玉妍更恨咱们呢?''

冤有头债有主,万事皆有因果。眼前,的确是没有人比金玉妍更有做这件事的由头。

但如懿顾不上这个了,她的疾言厉色里透着无比的虚弱:``江与彬,你告诉本宫,你一定会治好五公主!''她的声音像在烈烈秋风里哆嗦,``你能治好的,是不是?''

江与彬汗湿重衣,昂首不已:``微臣无能。''他的话像一把锋利的锯子,狠狠锉在如懿的头顶,自上而下,``这个病,根本无法医治。哪怕是赔上微臣和太医院所有人的性命,都不能了。微臣无用,请皇后娘娘责罚。''

江与彬说这句话的时候,璟兕烧得全身抽搐。她低低痛呼:``额娘!额娘!我难受!''如懿想要伸手去抱她入怀,让她安静下来,可是刚要伸手,已被容珮和海兰死死拉住。江与彬拽住如懿的袍角哀求:``皇后娘娘,使不得!若五公主不小心弄伤了您,连您也会染上这病的!''

高热折磨得小小的孩子说起了胡话,也根本吃不下东西。最后还是海兰想的法子,怕璟兕伤了人,更伤了自己,只得拿被子裹住,再用布条缚住了她

宫人们都不敢轻易碰璟兕,只敢小心翼翼地问:``皇后娘娘,要不要告诉皇上?''

自然是要告诉的,但不是眼下。

也许是天命不佑,也许是皇帝的分心,也许是后宫的灾厄带到了前朝。准噶尔的战事一度陷入僵局,并不顺利,是战是和,尚是未知之数。连忻嫔所生的六公主也好几次险些断了气息。宫人们禀报上去,皇帝亦无暇看顾,只是嘱了太医好生照料。

如此这般,如懿怎么敢随意去打扰。而禀报了太后,太后只有一语,道了声``冤孽!只是可怜了孩子'',重又捻动佛珠,闭门祝祷。

待到精疲力竭时,璟兕的呼吸弱得像游丝一般,细细的,好像随时会断了一样。不过几个时辰,又是发起了高热,继而连便溺也变得困难。

仿佛抓着最后的救命稻草,如懿追问道:``真的不能治了么?''

江与彬道:``如果杀了微臣可以救回公主,微臣愿意!''

如懿掩面:``那么,还能拖几天?''

江与彬不忍:``也就两三天,但是五公主,会活得很痛苦。''

这样的话,也唯有江与彬敢说吧。

如懿双膝一软,瘫倒在窗前。重重罗衣困缚在身上,端丽万方的轻绸软缎,流光溢彩的描金彩线,绣成振翅欲飞的风凰翱翔之姿,凤凰的羽毛皆用细如发丝的金丝垒成,缀以谷粒大的晶石珠,一针一线,千丝万缕,无不华美惊艳,是皇后万千尊荣的象征。

可什么皇后啊,此时此刻,她不过是个无助的母亲,面对命运的捉弄,无能为力。她终于忍不住,倒在海兰怀中放声大哭:``为什么?为什么是璟兕,是我的孩子?!她还不足两岁啊,她会笑,会哭,会叫阿玛和额娘,为什么是她啊?!若是我做错了,要了我的命去便罢了!为什么是我的孩子?!''

如懿从未那么无助过,仿佛自己成了一根细细的弦,只能任由命运的大手弹拨。整个人,无一处不被撕扯拉拨着痛。那痛,谁心刺骨,连绵不绝,哪怕断绝崩裂,她亦只能承受,什么办法也没有。

海兰遣开了众人,紧紧拥住她垂泪,反复道:``姐姐,别哭。别哭。''

话虽这么说,海兰的泪亦如黄梅时节连绵的雨,不断坠落。如懿任由自己哭倒在海兰怀里,声嘶力竭。最后,连如懿自己也恍惚了神志,仿佛是海兰的声音,不断地唤她:``姐姐,别忘了,你还有永璂啊。''

如懿的声音己经哑了,她推着海兰道:``海兰!璟兕是不成了,你去,你去亲自请皇上来,再看一眼璟兕吧。''

海兰连连点头,唤来容珮照应,急急起身往养心殿去。

皇帝匆忙赶来时,璟兕己经气若游丝,高热烧得她面色血红,呵呵地吐着舌头,手指虚弱地挠着自己的脸,烦躁而痛苦。

皇帝骇得脸都白了,食指栗栗发颤,想要伸手去扶抱:``朕的璟兕怎么了?她到底怎么了?''

江与彬忙拦住道:``皇上,不能啊!五公主是得了疯犬病!她,她\ldots{}''

话未说完,江与彬便被皇帝推了个趔趄,差点摔倒在地。皇帝怒喝道:``朕的公主好好的,怎么会得了疯犬病!''

江与彬哪里敢起身,索性伏在地上:``皇上,咬五公主的那条狗是得了疯犬病的,所以五公主也染上了这病。''他惶然,``皇上,这病是治不好的,若是被公主抓伤或咬了,也是会染上这病的呀!''

宫人们虽然想安抚璟兕,但脸上都是急欲躲避的神色。皇帝的手僵在了原地,像寒风初起时冻在冷寒里的枯萎的枝丫。他勉力镇定下来,扶住了如懿,喝道:``来人,快抱住五公主起来,让她别那么难受。''

可是宫人们一脸的避闪不及与畏惧惊怕,只是远远看着璟兕病弱而痛苦的模样,一脸的束手无策,哪里敢更靠近呢!

如懿哭倒在皇帝脚边,心神俱碎:``皇上,我们的孩子,这么乖巧的璟兕,怎么会成了这个样子!''她的哭声撕心裂肺,响彻云霄,``皇上,是谁害了我们的孩子?是谁?!''

如懿几近晕厥,皇帝紧紧地抱住她,支撑着她的身体,心疼地唤道:``璟兕!璟兕!是皇阿玛啊,皇阿玛来看你了!''

璟兕并未露出往日里乖巧甜美的笑容,只是从喉咙里发出含糊的喘息和类似嘶叫的声音。那声音越来越弱,是生命渐渐流逝的征兆。

皇帝再不忍看下去,掩面道:``来人!抱公主起来,快!''

这已经是最严厉的呼喝,可是宫人们面面相觑,还是不敢接近。如懿哭得喘不过气来:``皇上,我们的孩子被人害成了这样!''

凌云彻本守在宫门外,听得如此动静,上前紧紧护住了皇帝和如懿,以防璟兕意外伤人。何止是公主早己不成人样,便是如懿,也憔悴得不成人形。他看着如懿伤心欲绝的神色,又看了看璟兕的模样,咬了咬牙,迅速地脱下外袍,将璟兕紧紧裹住,让她不得动弹,抱到了皇帝跟前。

凌云彻道:``皇上,微臣抱着公主,您瞧瞧她吧。''

容珮在旁边打着下手,帮着凌云彻护住璟兕的身体。璟兕不断地颤抖着,小脸憋得发紫。凌云彻紧紧地抱她在怀里,一刻也不肯放松。如懿感激地望着他,伏在皇帝身边,啜泣不己。皇帝伸出手,轻轻地摸着璟兕的额头,凄然落下泪来。

那是一个父亲最深切的痛楚。

也不知过了多久,璟兕终于安睁了下来。

这是永恒的安静,她又如往日里一般。静静地睡了过去。江与彬凑上前搭了搭脉,又探了探鼻息,落下泪来,拜倒在地,轻声道:``皇上,皇后娘娘,公主己经去了。''

皇帝的手无力地垂落下来,他的双肩微微发颤,脚下踉跄几步,想要从凌云彻怀中抱过璟兕,最终还是有些犹豫地停了手。

头颅里针扎似的作痛,巨大的哀痛如浪潮排山倒海席卷而来,整个人虚脱无力,仿佛就要坠下去。

如懿跌跌撞撞地上前,从凌云彻怀中接过璟兕,将她楼在了自己怀中。她带着痴惘的笑意,轻声道:``璟兕,你累了是不是?你困了,倦了。没关系,额娘抱你睡。来,额娘抱你。你什么都别怕,额娘在呢。''她的笑意温柔如涟漪般在唇边轻轻漾开,她拍着孩子,悠悠地哼唱着,``宝宝睡,乖乖睡\ldots{}''

皇帝的泪在瞬间汹涌而出,他伸出手,抚摸着璟兕的小脸,爱怜地摩挲着,轻声造:``如懿,璟兕的手还是热的,真好\ldots{}''一语未毕,他亦哽咽了。

凌云彻意识到自己的多余,想要多停留片刻,举目见李玉悄悄招手,示意自己离开。他拖着步子走到门外。李玉低声道:``皇上和娘娘伤心,咱们守在这儿就是了。''他叹息,``凌大人,还是您忠心,抱住了五公主。要紧的时候,还是您哪!也是您胆大,五公主那个样子,真是吓人。''

凌云彻僵硬地笑了笑,守在了门外。

璟兕那个样子,他自然也是怕的。他也惜命,也会迟疑,可是如懿,她是那样的伤心。而让璟兕安静下来,不再是那个可怕的样子,是他唯一能替她做的事。

璟兕的丧仪过后,如懿已经憔悴得如一片脆而薄的枯叶,仿佛一触就会彻底破碎了。

皇帝数日不能安枕入眠,伤心不己,破例追封璟兕为和宜固伦公主,按着固伦大长公主的丧仪,随葬端慧皇太子园寝。历来嫡出之女为固伦公主,庶出之女为和硕公主,但那都是在即将下嫁时才可加封。皇帝如此做,亦是出于对璟兕格外的疼爱和怜惜。

然而悲伤之事并未断绝,仅仅隔了一日,忻嫔所生的门公主也因受惊早产而先天不足,随着璟兕去了。皇帝虽然伤心,却也比不上亲眼看着璟兕死去的痛楚,便按着和硕公主的丧仪下葬,连封号也未曾拟定,只叫陪葬在璟兕陵墓之侧。

宫中连丧两位公主,太后又担心端淑的安危,悲泣之声连绵不绝。时入五月,京中进入了难挨的梅雨季节。滴滴答答的愁雨不绝,空气里永远浸淫着潮湿而黏腻的气息,仿佛老天爷也在悲戚落泪。

金玉妍虽未削去贵妃位分,但被剥去了一切贵妃的仪制,只按着常在的份例开销,日子过得苦不堪言。

只是除了咬伤璟兕致死的``富贵儿''是金玉妍曾经豢养的,并无其他可以指证是金玉妍调唆``富贵儿''伤人,且顾及着金玉妍所生的三位皇子,皇帝也未曾再做重责。而庆嫔和晋嫔,也因裁制了那件惹祸的红衣,被皇帝贬斥,降为贵人,日夜在宝华殿抄录经文以作惩罚。

如懿大病了一场,皇帝虽然有心陪护,但前朝战事未宁,有心无力,只得让太医悉心照看。

一时间宫中离丧之像,便至如此哀乱之境了。

深夜孤眠,如懿辗转反侧,一闭上眼便是璟兕的面庞,时而癫狂,时而宁和,交替纷杂,恍若无数的雪片在脑海里纷纷扬扬地下着,冻得发痛。江与彬给她的安神药一碗一碗灌下,却毫无作用,她睁着眼,死死地咬住嘴唇,任由泪水滑落枯瘦的面庞,如同窗外的雨绵绵不绝,洇透了软枕。

心中的痛楚狼奔豕突,没有出口。如懿披了一袭长衣,赤足茫然地走到窗边。萧瑟的风灌满她单薄的寝衣,吹起膨胀的鼓旋。乱发拂过泪眼,仿佛璟兕温软的小小的手又抚上面来,如懿忽然无措地痛哭起来。

哭声惊动了容珮,她推门而入,紧紧扶住了如懿,急切唤道:``娘娘!娘娘!''

如懿哭得硬咽:``容珮!是我不中用,我连自己的孩子都救不了,护不住!''

容珮啜泣着劝道:``娘娘,公主这样活着,也是毫无尊严,只不过是再痛苦挣扎几日罢了。若是早日去了极乐世界,也是一种解脱。''

如懿痛心疾首,额头抵在冰冷坚硬的墙壁上,连连撞击:``璟兕是活得痛苦,可我也不配做她的额娘!我该拼命救她的,可我毫无办法!''

容珮见如懿如此,慌忙挡在墙上:``娘娘,您别这样!您别伤了自己!''容珮含着满眼的泪,仰着脸,沉稳地望着如懿道,``奴婢知道,咱们能做的选择,永远只能是当下能做出的最好的选择。如果有能救公主的办法,娘娘一定会拼上性命的!''

夜雨如注,繁密有声,好像是流不完的眼泪,更像这悲伤死死地烙在人的心上,永远也不能褪去了。

悲伤中的日子静得几乎能生出尘埃。到了五月末,天气渐渐热了起来,往年里嫔妃们都迫不及待地换上轻薄如云霞的彩裙绡衣,秾翠嫩紫、娇青媚红,映着满苑百花盛放,禽鸟翩然,无一不是人比花娇。而今岁,即便是再有心争艳的嫔妃,亦不敢着鲜艳的颜色,化娇丽的妆容,惹来皇帝的不悦。

因着璟兕和六公主的早夭,如懿与忻嫔都神思黯然,四阿哥被冷落,八阿哥足伤,嘉贵妃禁足,庆嫔和晋嫔被罚,太后又忧心端淑长公主的安危,宫中难免是凄凄冷冷,连树上的鸣蝉都弱了声气,有气无力地哼一声,又哼一声,位长了深不见底的哀伤。

任凭时光悠悠一荡,却未能淡了悲伤。

午后的茜纱窗外,大片大片的阳光像团团簇簇的凤凰花般在空中烈烈而绽,散下浅红流金的光影。如懿在素衣无饰了月余后终于有了梳妆打扮的心思,象牙妆台明净依旧,珠钗花簪却蒙了薄薄的尘灰。她并不用容珮和侍女们动手,亲自将蓬松得略有些随意的家常发髻打散,因着悲伤,她几欲逶地的青丝亦有些枯黄,只能蘸了桅子花头油梳理通顺,复又用青玉无纹的扁方绾成高髻。一枝暗金步摇从轻绾的云髻中轻轻斜出,那凌空欲飞的凤凰衔着一串长长的明珠,恰映得前额皎洁明亮,将一个月以来的黯沉略略扫空。几枚简素的镀金莲蓬簪子将发髻密密压实,一朵素白绢菊簪在髻后点缀。

容珮小心翼翼提醒道:``皇后娘娘,公主是晚辈,您已经为她簪了这么久的白花,今日便不必了吧。''

她的提醒是善意的,准噶尔战事未平,一直簪着白花,也并不吉利。如懿轻叹一声,摘下白花,换了白玉雕琢而成的嵌蓝宝石珠花,略略点缀一朵暗蓝色蟹爪菊绢花。

容珮取过玫瑰指膏轻轻送上:``娘娘,您的妆还是太淡了,脸色不好呢。''

如懿对着铜镜细细理妆,不留一丝瑕疵。消瘦的脸颊,上了胭脂;苍白的嘴唇,涂了唇脂;细纹聚集的眉心,点了花钿,一切还如旧时,连耳上的乳白色三联珰玉耳坠子也是璟兕最喜爱看她戴着的。

如懿换上一身月白素织氅衣,点着淡青色落梅瓣的细碎花纹,系了素色暗花领子。这些日子抄录佛经闭门不出,端的是肤白胜雪,而眼神却是惊人的苍冷,如一潭不见底的深渊。

如懿轻声道:``今日是璟兕的五七回魂之日,本宫要送一送她。''

容珮道:``愉妃小主一早来时娘娘还在给公主念经,小主送来了亲手做的十色素斋,说是要供在五公主的灵堂,今夜亥时小主还会陪娘娘一同为公主召唤。''

如懿微微垂了头,云鬓上的蓝宝石玉花的银丝长蕊轻轻颤动:``愉妃有心了。''

容珮赞叹:``这样的心思,满宫里也只有愉妃小主有。''她似想起什么,``皇上派了李公公来传话,今夜也会来陪娘娘为公主召唤。奴掉也把公主生前穿过的衣服和玩过的器具都整理好了,放在公主的小床上。''

如懿额首:``规矩都教过永璂了吧?''

容珮道:``嬷嬷们都教导过了。十二阿哥天资聪颖,断不会出错的。''

悲愁瞬间攫住了她的心,攥得几欲滴下血来:``今日是五七,过世的人会回家最后看看亲人才去投胎。本宫想好好再陪一陪璟兕。''

\hypertarget{ux7b2cux5341ux4e00ux7ae0-ux76f8ux968f}{%
\chapter{第十一章 相随}\label{ux7b2cux5341ux4e00ux7ae0-ux76f8ux968f}}

然而夜色如涨潮的江水,无声无息便拨染了天空。皇帝让李玉传来话,前线六百里加急战报,要与群臣议事,实在脱不开身。

李玉说得仔细:``大军前锋部队进抵伊犁河畔,达瓦齐却仍执迷不悟,负隅顽抗,率部万人,退居伊犁西北方向的格登山,驻营固守,孤注一掷。皇上接到战报便忙到了现下,连晚膳都用得极匆忙。''

如懿明白,亦不勉强,便道:``皇上专心政事,本宫明白,也一定体谅。本宫会替皇上上清香一炷,祭告璟兕。''

与李玉同来的还有凌云彻,他躬身,清癯的面容诚挚而略显悲伤:``微臣向皇上请求,与李公公同来送和宜公主一程。''他的声音轻轻的,带着青苔般的丝缕潮湿,``毕竟,公主是在微臣怀中走的。''

如懿想起璟兕离开前的一幕,眼中浮起隐隐潮气:``那是应该的。凌大人,谢谢你,让璟兕最后走得不那么难堪。''

他躬身,容色轻淡而哀戚:``那是微臣的本分。''

海兰着一色莲青薄绸衣裙,带着永琪在身边,捧着一个白纱绢袋,里头盛着为璟兕魂灵引路的草木灰,徐徐道:``姐姐,时候不早了,我们该去召唤五公主的灵魂归来了。''

夜色如纱微笼,素衣的如懿和海兰由内侍与宫女提起莲形铜灯引路,李玉与凌云彻陪护在后,缓步而去。这一夜并不黑,蓊郁桐荫里款款悬着半弯下弦清月,漫天撤落的星子零零碎碎的,散着微白的光。因为早已吩咐了要行璟兕的``五七''之礼,内务府早预备了下去,将长街两侧的石灯都围上了洁白的布缦。

如懿披着一身素淡至极的石青绸刻玉叶檀心梅披风,系带处坠着两枚银铃档,那是从璟兕的手铃上摘下来的,可以让她循着熟悉的铃声,找到自己。容珮抱了永璂在怀中,让永璂和永琪手里各提着一个小小的羊角琉璃题花灯笼。

如懿轻声道:``这一双灯笼,是璟兕从前最爱玩的。''话未完,她的眼眶又湿润了,只得从海兰手里接过一把草木灰撒出,来掩饰自己无从掩饰的伤感。

永琪很是懂事:``皇额娘,儿臣给妹妹照路,她就可以看见地上的草木灰,跟我们在一块儿了。''

永璂牙牙道:``额娘,儿臣和五哥哥一样。''

如懿的指缝间扬扬撒落一把草木灰:``好孩子,这样妹妹就不会迷路了。她就能找着咱们,和咱们走最后这一程。''

凌云彻陪守在如懿身边,轻声道:``皇后娘娘别难过了,仔细风吹了草木灰,迷了您的眼睛。''

如懿的睫毛上盈着一滴晶莹的泪,她极力忍住,别过头去道:``但愿今夜的风不要太大,不要吹散了这些草木灰,迷了璟兕回家的路。''

凌云彻的声音低沉而温暖:``不会的。和宜公主聪慧过人,知道娘娘在等她,一定会回来的。''

如懿并不看他,只是微微侧首:``多谢你。''

并未以官职相称,也不如常日一般唤他``凌大人'',这样简短的语句,无端地让他觉得亲切。然而,他并不能有多余的表情,只是以略略谦恭的姿态,和李玉一左一右,跟随她身后。

凌云彻看着如懿纤细瘦美的背影,发簪上垂落的碎蓝宝珠珥流苏被风拂动,闪着粼粼的光。他陪在她身后,走过这漫长又漫长的长街,两侧徐徐笔直高陡的红墙,使长街看去越觉纵深,幽幽暗暗,不知前路几何。

他只希望这样的路能长一些,更长一些。

璟兕的灵堂布置在雨花阁内,后头是宝华殿的梵音重重。法师们念着六字箴言,恍如极乐净土。

永璂提着灯笼,学着永琪,将宫人们预备好的灵堂屋顶上的瓦片砸碎在地,极力呼唤:``妹妹,回来!璟兕,你回来!''

永琪极力克制着哽咽声,永璂的声音更稚气,带着浓重的哭音,无限渴盼而伤心。或许在他小小的心里,只要这样高声呼唤,妹妹就会再回到他身旁,和他一起玩闹,一起嬉笑。一如往日。

空气中是瑟瑟的草木香,有白日里阳光曝晒后的勃勃的甘芳气息。如懿跪蹲在灵堂内,将亲手抄录的《往生咒》与纸钱一同焚化在铜盆内。

忽有蛙鸣入耳,如懿有些恍惚,泪水淆然而落,滴在火盆内,引得火苗迅疾跳了一下,腾起幽蓝的火焰:``璟兕最喜欢听蛙鸣声,每次听到都会笑。可是今年,她己经听不到了。''

海兰的笑意温暖如绵,声音亦款款柔丽。她从容引袖,拭去如懿腮边晶莹的一滴泪:``姐姐,璟兕就在我们身边,只是我们看不到罢了,这些蛙声,她都能听到的。自然了,姐姐的伤心她也会知道。''

阁外的松柏投下长而暗的影子,将她的身影遮蔽得越显纤弱。海兰伸手为如懿掸去袖口上纸钱焚烧后扬起又落下的黑蝴蝶似的灰烬,大大的眼眸流露出无限的担心与关切:``姐姐伤心过甚,人也消瘦至此。璟兕那么懂事,看姐姐伤心,也会伤心的。''

如懿努力点头:``你放心。''她将手中的佛经焚烧殆尽,站起身道:``李公公,凌大人,你们也来陪一陪璟兕吧。璟兕喜欢热闹,人多,她就不会寂寞了。''

李玉躬身入内,与凌云彻各自拈起一往香,在璟兕灵前鞠躬行礼。

礼毕已经极晚。月色薄露清辉,那光晕有些模糊,并不怎么明亮,唯有宫人引路的灯盏,如跳动着的跌宕的心,幽光细细。

前头转弯处明黄的辇轿一闪,容珮忽然惊异,回首道:``娘娘,是皇上的御驾。''

如懿怔了一怔,凝神望去,有无限酸楚突然胀满了心的缝隙:``李玉,皇上处理完政事了么?''

李玉看了看皇帝去的方向,有些诺诺:``大概是已经忙完了吧。''

海兰引首前望,低声道:``皇上去的好像是颖嫔宫里,皇上是去看颖嫔了。''

容珮不满,抱紧了怀里的永璂,低声嘟嚷道:``今儿是公主的五七,皇上忙于前朝的事也罢了。怎么到了后宫也不陪娘娘,反而去颖嫔那里?''

永琪忙拉住容珮的手,肃然道:``容姑姑别说了。''

如懿看了看似懂非懂的永璂,抚了抚永琪的额头,苦笑道:``皇上自然有皇上的道理。这些话,别当着孩子的面说。''

李玉低低道:``今日是颖嫔小主的生辰。''

容珮将永璂递到三宝怀里,啐了一口道:``颖嫔的生辰比得上咱们公主的五七要紧么?''

如懿仰望天际遮住月色的乌云,黯然道:``生辰是高兴的事,五七却是伤心,你会愿意记得哪个?''

``可公主是皇上的嫡出女儿\ldots{}''容珮见如懿心如刀绞,亦不敢再说下去。

海兰神色淡然:``皇上的性子,本就是喜欢报喜不报忧的。何况近喜远悲,是人的常性。''

那一刻,如懿是笑着的,可是凌云彻却觉得,那笑意是那样悲切,仿佛再多的眼泪也比不上那一缕微笑带来的伤悲。她的眸子幽怨而深黑,掠过他的眼。

凌云彻的心突然哆嗦了一下,仿佛被利针穿透,那么疼。

如懿独立风露之中,裙角沾染了青石上的夜露。站得久了,经风一拂,只觉肌骨生凉,她不自觉地便打了个寒噤。海兰忙靠紧她的身体,轻声道:``夜凉,姐姐还是回去吧。''

有那么一瞬间,凌云彻突然很想摘下官服外的披风加于如懿瘦削的肩上,替她挡住凉夜的侵袭。

岁月那样长,衣衫那样薄,即便心无可栖处,亦可稍稍温暖。

然而,他并没有那样做,只是扶住了如懿的手臂,亦按住了被涌过的风吹起的扑展如硕大蝶翼的披风:``皇后娘娘这一路伤心,微臣会陪娘娘走下去。''

海兰的目光中隐约浮起一丝疑虑,深深地看向凌云彻。他顿一顿:``愉妃娘娘、李公公,也都会陪皇后娘娘走下去。''

海兰的脸色稍稍和缓,沉声道:``是,我会一直陪着姐姐。这句话,很早前我就说过。如今,以后,也是一样。''

凌云彻不敢再多言,只是随着众人往翊坤宫方向默默行走。

这一夜,原本是嬿婉侍奉皇帝在养心殿用晚膳,按着寻常,她也会顺势留下陪伴皇帝度过宫中寂寞的夜。但皇帝无心顾她,便去了御书房和大臣们商议准噶尔战事。

嬿婉在暖阁里无聊而期盼地等着,绣了一会儿花,发了一会儿呆,慢慢熬着时辰,到了夜深时分,皇帝出了御书房,她极高兴地迎了上去。皇帝还是推开了她,半含着歉疚笑道:``朕得去瞧瞧颖嫔,今日是她的生辰。''

嬿婉当然是知道其中的缘由的。颖嫔的族人为皇帝平定准噶尔战事出力不少,何况满蒙一家,蒙古一直是大清的有力后盾,因而皇帝一直对颖嫔十分眷顾。

嬿婉一直深以家世为憾,这一来自然不悦,却也不敢有丝毫流露,只是以温柔得能滴出水的语调相对:``皇上,今夜是和宜公主的五七之辰。臣妾是怕皇上触目伤情,所以特来养心殿陪伴,皇上何必还要入后宫呢?''

皇帝也笑言相对,只道:``看时辰,只怕皇后已经去雨花阁行过五七的祭礼了。只是今日是颖嫔的生辰,再晚,朕也一定要去看看她的。''

嬿婉情知劝不动,勉强笑道:``皇上要去便早去,何必巴巴儿地到了这个时候才去吵颖嫔妹妹,臣妾也怕皇上明日要早起上朝,格外辛苦。''

皇帝爽然笑道:``这你便不知道了。朕一日没有理会颖嫔,只当不知道她生辰的事,只怕这个时候她都己经生气失落得很了,却又不敢发作。朕此时再去,她才会又惊又喜。''

嬿婉虽然一肚子气,却也只得笑着趋奉道:``皇上就会弄这些心思讨人喜欢。''

皇帝觑着眼看她:``你不喜欢?''

嬿婉只得笑吟吟:``皇上惯会取笑臣妾。那么,臣妾恭送皇上了。''

直到目送皇帝离开,嬿婉才扶了春婵的手离开养心殿。这一路,她有些闷闷的。春婵只道:``小主,皇上去不去看颖嫔,其实也没什么。您怎么倒只提起五公主五七祭礼的事?''

嬿婉``咯''的一声冷笑,清碎如冰:``这些日子皇上有多为五公主伤心,本宫如何不知道?五公主死前是什么模样,如癫如狂,皇上只怕这辈子都忘不了。且这件事,宫里人瞧着都像是谁做的?''

春婵微笑:``那自然是和嘉贵妃脱不了干系了。''

``是了。''嬿婉的唇角浮起得意的笑色,``那皇上为什么不立刻处置了嘉贵妃?依着皇上的性子,伤了他的爱女却还不立即处置,固然是因为嘉贵妃多年得宠的缘故,也是因为她的三个儿子和李朝母族的地位。皇上为难是不知该如何处置,真凶似是非是,皇上处置不了嘉贵妃,便给不了五公主一个交代,当然为难。''她摇着手中的葵纹明绫白团扇,``嘉贵妃的儿子,一个被皇上冷落,一个摔残了腿,真是不济!本宫还以为那几枚针,够送永璇上西天见佛祖了呢!''

``躲得过一时,躲不过一世。如今儿子残了腿,亲额娘又失宠禁足,活着眼睁睁看着才是苦呢。若死了一了百了,岂不没意思了!''春婵一笑,``那日澜翠还和奴婢说嘴,说碰上守坤宁宫的侍卫赵九宵。''

``赵九宵?''嬿婉警觉,``他和澜翠说什么?他们怎么认识的?''

春婵笑道:``有次小主不是召赵九宵来永寿宫,是让澜翠送他的么?怕是那时认识的。那傻小子怕是看上澜翠了,每次初一、十五咱们去坤宁宫,他都想蹭着澜翠说话。可澜翠都不理他,越是这样,他就越缠着澜翠说话。这不,就说起有次他和皇上御前的红人凌大人喝酒,见他袖着几枚银针,那日正是凌大人从马场查八阿哥坠马之事回来的日子。''她见嬿婉的神色逐渐郑重,``这样要紧的事,奴婢特意嘱了澜翠又问了一次。但澜翠说赵九宵什么也不知,进忠也说,凌大人向皇上复命时根本没提过什么银针。奴婢想,凌大人重情重义,怕是查出了什么蛛丝马迹,却什么也不肯说。何况,许多事,根本没有痕迹可查。''

春婵的话,让嬿婉安心。有感动的暖色在嬿婉的脸上漾起,很快,更多的得意覆盖了那抹感动。嬿婉抚摸着手指上凌云彻当年相送的红宝石戒指。暗夜里,它即便是宝石粉做的,亦有珊瑚色的光华流转。嬿婉娇丽一笑:``不管为了什么,也不管本宫怎么对他,这些年他心里有谁,本宫都是知道的。这个人啊,就是嘴硬而已!''

春婵扶住了嬿婉,轻笑道:``那是。小主盛年华光,连皇上都爱不释手,何况是一个小小的侍卫,当然对小主视若天人,捧在掌心了!否则当年为了嘉贵妃的肚兜闹出来的委屈,他怎么平白兜着不说了呢。''她顿一顿,隐秘地笑道,``奴婢还听说,凌大人忙着在宫中当差,很少回宫外的宅子,所以冷落了娇妻,惹得不满呢。''

嬿婉唇角扬得更高,笑容好似兜不住似的,``茂倩只是一个宫女,又是皇上指婚,本来就没什么情意。''

春婵忙道:``凌大人还不是因为心里有小主,看什么人都不能入眼了!''

嬿婉的笑容瞬间凝住:``有的人的心意是难得了,只是皇上么\ldots{}''

春婵恭谨回道:``皇后娘娘这朵花开到了盛时,接下去便只能是盛极而衰。而小主这朵花才开了几瓣儿,有的是无穷无尽的好时候呢。''

嬿婉嗤道:``左右今儿是和宜那短命孩子的五七,咱们便拐去翊坤宫,听听皇后的哭声吧。''

不远的彼端,隐约可见翊坤宫宫门一角。衬在如墨的天色下,盘踞于飞檐之上的兽头朦朦胧胧,却不失庄严之态。

凌云彻陪在如懿身后,心下微凉如晨雾弥漫。

这,便是尽头了。

这一晚,他能陪她走这一段,己是难得的奢望。

翊坤宫一门相隔,她是高高在上的皇后,他依旧是养心殿前小小的御前侍卫。只可遥遥一望,再不能同路而行。

这一段路,已经太难得,太难得了。

李玉先于他躬身施礼:``皇后娘娘,愉妃娘娘,夜已深,两位娘娘早些安置。奴才先告退了。''他的眼神一撩,凌云彻会意,便也照着他的话又说了一遍,还是忍不住道:``皇后娘娘保重,万勿再伤心了。''

海兰挥了挥手:``有劳李公公和凌大人了。''她停一停,``李公公还要赶着去咸福宫伺候皇上和颖嫔,赶紧去吧。''

李玉与凌云彻立在翊坤宫门外,目送如懿与海兰入内,方才躬身离开。凌云彻似有些不舍,脚步微微滞缓,还是赶紧跟上了。

甬道的转角处,嬿婉的脸色己经如数九寒冰,几可冻煞人了。春婵从未见过嬿婉这样的神色,不觉有些害怕,轻声唤道:``小主小主!您怎么了?''

嬿婉迷离的眼波牢牢地注视着前方,她幽幽凝眸处,正是凌云彻渐行渐远的背影。有一抹浓翳的忧伤从眸底流过,伶仃的叹息仿佛划破她的胸腔:``一个男人用这样的眼神看一个女人,是为什么?''

她这样的叹息,似是自问,亦像是在问春婵。

春婵吓得有些懵了,哪里敢接话,只能怯怯低头。

嬿婉亦不需她回答,只是沉浸在自己的伤感之中:``都过去了啊\ldots 都过去了!''她的脸色如湖镜般沉下去,唯有双眸中几点星光水波潋滟,流露出浓不可破的恨意,``可是,哪怕己经是过去,本宫也容不得!喜欢过本宫一时,便要喜欢本宫一世,永远不许变!皇上是这样,他是这样,谁都一样!谁要改变了这个,本宫绝不会放过他!''

乾隆二十年五月,前线捷报频传。达瓦齐自带兵负隅顽抗,军械不整,马力亦疲,各处可调之兵,己收括无遗,使得众心离散,纷纷投降。北路和西路大军分兵两翼各据地势,包围了达瓦齐最后栖身的格登山。清军出其不意,突入敌营,策马横刀,乘夜袭击。达瓦齐及部下措手不及,乱作一团,自相践踏,死者不可胜数,万余敌兵,顷刻瓦解。达瓦齐率两千余人仓皇逃遁,黎明时才被追兵捕到。

皇帝大喜过望,当即下令将达瓦齐及家人解送回京,不许怠慢。

太后于慈宁宫中闭门诵经祝祷多日,听得此消息,情急不己:``端淑如何?''

福珈喜不自禁:``公主无恙,一切平安。''

太后闻言欣慰,长叹一声:``天命庇佑,大清安宁。只是皇帝要如何处置达瓦齐及端淑长公主?''

福珈且笑且流泪,激动道,``皇上恩慈,说于恒有言,曰杀宁育,受俘赦之,不我扩度,又说要宁宥加恩,封达瓦齐为亲王,准许他及子女居住京城,再不北归。''她说得太急,又道,``皇上孝心,以平定准噶尔达瓦齐遣官司祭告天地、社稷、先师孔子,更要为太后您上徽号,以示庆贺。徽号也让内务府似好了,是`裕寿'二字,可见皇上仁孝。''

太后漠然一笑,轻嗤道:``皇帝要真是仁孝,就让端淑与达瓦齐这个逆臣和离,搬入慈宁宫中与哀家同住。''

福珈的笑容一滞,如飘落于湖心上的花瓣,旋即沉没。

太后见她默然,不觉急道:``端淑怎么了?你不是说她一切平安么?''

福珈笑得比哭还难看,踌躇半日,逼不过了才道:``太后万喜,长公主有孕,已经五个月了!''

太后一怔,手中的佛珠滚落在地,咕噜咕噜散了满殿。她踉跄几步,险险跌坐于榻上,不觉泪流满面:``冤孽!冤孽!这么说,哀家的端淑就一辈子要和达瓦齐这个逆贼在一起!为什么?为什么没有人告诉哀家?''

福珈垂泪道:``太后,奴婢也是刚刚知道,听端淑长公主刚有孕时也曾想悄悄除掉孩子,但始终狠不下心,如今也来不及了!''

太后苍老而哀伤的面上闪过一丝戾气,狠道:``怎么来不及?若除了孩子,一了百了,端淑也可以和离了。''

福珈吓了一大跳:``太后,您可别这么说!公主的月份这么大了,若强行堕下孩子,只怕也伤了公主。''

太后一怔,神色旋即软弱而无助,靠在福珈手臂上,热泪淆淆而下:``是啊,哀家可以对任何人狠下心肠,却不能这般对自己的女儿。罢了,罢了,这都是命数啊!''

福珈哭道:``太后,皇上既然决定善待达瓦齐,必定也会善待公主。皇上说了,达瓦齐午门受俘,行献俘礼之后,只要他能痛改前非,输诚投顺,皇帝也会一体封爵,不令他再有所失。这样长公主也能在京城安稳度日了,太后想要见公主还不容易么?''

太后颓然道:``也罢。皇帝行事仁孝,其实心性难以动摇。只要端淑能在哀家膝下朝夕相见,彼此看见平安,哀家也无话可说了。''

\hypertarget{ux7b2cux5341ux4e8cux7ae0-ux4f24ux82b1}{%
\chapter{第十二章 伤花}\label{ux7b2cux5341ux4e8cux7ae0-ux4f24ux82b1}}

如是,达瓦齐被解京师之日,皇帝御午门,封以亲王,赐宝禅寺街居住。端淑宫拜见太后,其时腹部已经隆起,行走不便。母女二人一别二十年,不觉在慈宁宫中抱头痛哭,以诉离情。

达瓦齐从此便在京中与端淑长公主安稳度日,只是他不耐国中风俗,每日只向大池驱鹅逐鸭,沐浴其中以为乐趣。达瓦齐心志颓丧,每日耽于饮食,大吃大喝,日夜不休。他身体极肥,面庞比盘子还大出好许,腰腹阔壮,膻气逼人,不可靠近。公主看不过眼,便请旨常在慈宁宫中居住。皇帝倒也允准,只让太后答允少理后宫之事,方才成全了端淑长公主于太后的母女之情。

如是,宫中也宁和不少,连着太后与如懿也和缓了许多。

偶然在慈宁宫见着端淑,如懿与她性子倒相投。大约见惯了世事颠沛,端淑的性子很平和,也极爽朗通透,与她说话,倒是乐事。

二人说起少年时在宫中相见的情景,端淑不觉掩唇笑道:``那皇后嫂嫂入宫,在一众宫眷中打扮得真是出挑,连衣裙上绣着的牡丹也比别的格格精致不少。我虽是皇家公主,也不免暗暗称奇,原来公卿家的女儿,也不是输阵的。''

真的,年纪小的时候,谁懂隐忍收敛为何物?春花含蕊,哪个不是尽情恣意地盛放着,闹上一春便是一春。

如懿便笑:``公主记性真好。''

端淑微微黯然:``自从远嫁,宫里的日子每一天都在我心里颠倒个过儿,什么都记得清清楚楚的。连额娘袖口上的花样绣的什么颜色,也如在眼前。我还记得,我出嫁那一日,额娘戴着一枚赤金嵌翠凤口镯,那镯子上用红玛瑙碎嵌了一对鸳鸯,我就在想,鸳鸯,鸳鸯怎是这样让人心酸的鸟儿。''

如懿正要出言安慰,端淑先自缓了过来,换了清朗笑意:``如今可好了,我又回来了,一早便向额娘讨了那只镯子,以后便不记挂了。''她又道,``说来那时我可喜欢皇后嫂嫂裙子上的牡丹了,就如今日这件一样。那时我想摸一摸,嫂嫂却似怕我似的,立刻走远了。''

太后盘腿坐在一边,慈爱地听着端淑碎碎言语,仿佛怎么也听不够似的。听到此节,太后便笑:``多少年了,还念着这事儿。那定是你顽皮,皇后不愿理你。''

如懿念及往事,不觉唏嘘:``皇额娘,真不是臣妾矫情莽撞,实在也是怕了。''

端淑咂舌:``皇后的性子,也知什么是怕?''

如懿颔首:``当日皇额娘与臣妾姑母不算和睦,臣妾随着姑母,哪里敢与皇额娘的女儿亲近。且在家时,姨娘所生的女儿绵里藏针,屡屡借着一衣一食生出事端,臣妾虽为嫡出,但不及妹妹得阿玛疼爱,发觉斥责无用,只好避之不及。''

端淑``咦''了一声:``一直以为你出身后族,又是格格,不意家中也这般难相处。''

如懿轻嗤,却也淡然:``天下人家,莫不如是。''她又笑,``当年得罪公主,不想公主如此记仇,看来哪一日必得好好请上一桌筵席,向公主赔罪。''

说着,太后也笑了,道:``你们便时太闲,记着这个论那个。多少旧事了,还来说嘴。''

噫!不意真有今日。

可放下旧日种种恩怨仇隙,笑语一饷。

那,那些曾经放不开的情仇,都是哪里来的呢?莫不真是自寻烦恼。那此刻放不下的,又算什么呢?

她轻轻叹息,坐看天际云起云散,飞鸟四逸。

时近盛夏,京中晴日无云,已经渐渐酷热。因达瓦齐受降之故,李朝等属国也纷纷来贺,派使臣入京,朝中一派喜庆之气。只是因着两位小公主新丧不久,皇帝也无意前往圆明园避暑,只在宫中忙于平定准噶尔之后的种种事宜。

如懿午睡初醒,饮了一碗酸梅汁,便抚着胸口道:``吃得絮了,没什么味道,反而胸闷得很。''

容珮笑道:``这几日天热,娘娘的胃口不好,总是烦闷难受\ldots{}''

容珮的话未完,如懿已经横了她一眼:``不相干的话不要多说,扶本宫起身梳妆,咱们去看看皇上。''

午后的养心殿安静的近乎寂寞,皇帝独立于窗下,长风悠然,拂起他衣炔翩翩,如白鹤舒展的翅,游逸于天际。他的背影肃肃,宛如谛仙。这般无人时,如懿凝望向他,宛若凝望着少年时与他相处的时光,唯有他,唯有自己,再没有别人来打扰他们的宁静。

皇帝的沉醉,在于壁上悬挂的巨幅地图,喃喃道:``准噶尔诸部入版图\ldots 其山川道里应详细相度,载入皇辇全图。自圣祖康熙时至今,三代的梦想与期盼,朕终于实现了。''他兴奋地看向如懿,满眼沉着与喜悦,``如懿,朕已经命人重新绘制新疆地图,将准噶尔之地完整画入。又吩咐在避暑山庄东北面的普宁寺,以满,汉,蒙,藏四种文字刻碑记述我大清平定准噶尔部的历程,定名《平定准噶尔后勒铭伊犁之碑》。你说可好?''

如懿分享着他的快乐,并肩立于他身旁:``皇上完成先祖之愿,理当普天同庆,以告慰列祖列宗。''她微微垂首,靠在他肩上,``臣妾最高兴d是,皇上的山河万里,宏图挥鞭之中,是臣妾何皇上一同经历的。''

皇帝的笑容清湛,抵着她的额头道:``如懿,你这样的话,朕最欢喜。''皇帝指点着万里巨图,挥斥方遒,``平定准噶尔后,便是天山一带的不肯驯服于朕的寒部,还有江南的不服士子,虽然明面上不敢反抗我大清,但暗中诋毁,写诗嘲讽的不在少数,甚至蔚然成风。''

如懿摇一摇手中的轻罗素纱小扇,送上细细清凉:``士子们都是文人,顶多背后牢骚几句,皇上不必在意。''

皇帝冷哼道:``先祖顺治爷宠幸汉臣,他们就敢说出------若要天下安,复发留衣冠,这种大逆不道的话。康熙爷与先帝都极重视民间言论,尤其百姓愚昧,极易受到这些文人士子的蛊惑。

如懿听皇帝说起政事,只得到:``是。''

皇帝侃侃而谈:``不止民间如此,朕的朝廷里难道就清静么?广西巡抚卫哲治告内阁学士胡中藻自负文才,不满朝廷,写诗诽谤。你可知他都写了些什么?''

如懿见皇帝深色不悦,只得顺着说:``臣妾愿意耳闻。''

皇帝冷冷道:``胡中藻姓胡,就惯会胡言乱语,写什么------一世无日月、一把心肠论浊清、斯文欲被蛮、与一世争在丑夷等句,尤其是------一把心肠论浊清之句,加``浊''字于我国号``清''字之上,是何居心?''

如懿听得心有戚戚,只得含笑道:``他一个文人,写诗兴致所致,恐怕没有咬文嚼字那么仔细。''

皇帝眉心一皱,愈加沉肃道:``皇后有所不知,胡中藻不仅如此,他悖逆、抵讪、怨望之处数不胜数,他所出的典试经文题内有'乾三爻不像龙'之句,乾隆乃朕年号,龙与隆同音,显然是诋毁朕,再有'并花已觉单无蒂'句,岂非讥讽孝贤皇后之死。胡中藻鬼魅为心,语言吟诵之间,肆行悖逆抵讪,实非人类之所应有''有凛然的杀气凝在他墨色的眸底,看得如懿心惊胆战,``朕已决定,胡中藻罪不容诛,斩首弃市!''

如懿心头一哆嗦,正欲说话。皇帝看向她的颜色已有几分不满:``皇后难道对这样的不忠之人还心存怜悯么?''

如懿还如何敢多说,只得道:``臣妾不懂政事,只是想,若于文字上如此严苛,天下文人还如何敢读书写字呢?''

``要读就读忠君之书,要写就写忠君之字,如若不然,朕宁可他们个个目不识丁,事事不懂!''

有清风乍起,身上浅紫色棠棣花样的袖口随风展开,飘飘若举,宛如蝴蝶扑扇着阔大的翼,扇得她的思绪更加烦乱。如懿有一瞬的出神,难怪天下男子都喜欢单纯至无知的女子,这样捧在手心,或弃之一旁,她什么都不懂,亦不会怨,不比识文懂字的女子,情丝剔透,心有怨望,才有班婕妤的《团扇歌》,才有卓文君的《白头吟》。

她微笑着,无知无觉的女子,或许叹息几声,哀叹命运不济也便罢了,如何说得出卓文君一般``闻君有两意,故来相决绝''的话呢!这样的才女,固然聪慧玲珑,自然也不够可爱了。

皇帝蹙眉:``皇后,你在笑什么?''

如懿心中一凛,那笑容便僵在了脸上:``臣妾在想,臣妾也喜读诗文,以后更该字字篇篇都小心了。''

皇帝拂袖道:``本就该这样。朕想起胡中藻乃朕先前的首辅鄂尔泰的门生。虽然鄂尔泰已死,但他认人不清,朕已下令将其牌位撤出贤良祠,以儆后人。

如懿口中应着,看着眼前勃然大怒的男子,心思有片刻的恍惚。曾几何时,那个与自己一起谈论《诗经》、一起夜读《纳兰词》的男子呢?他温文尔雅的风姿,怎么此刻就不见了呢?

仿佛记忆中关于他的已越来越模糊,最终也只幻化为一个朦胧而美好的影子,凭自己旖念。

或许,眼前的男子还是和从前一样吧,只是他在意的,再不只是那样美丽如萤火虫般闪烁的文字,而是文字背后的忠诚与稳固吧。

最后,皇帝以一言蔽之:``不管是谁,不管他身在何处。只要悖逆朕的心意的,朕都容不得他们,必定一一征服!''

皇帝的话,自此便开启了平定寒部之战,自然,那也是后话了,然而眼前,如懿只听的皇帝说:``朕平定准噶尔大喜,万国来贺,嘉贵妃金氏的母族李朝也不例外,前朝后宫皆有庆典,这样的场合,嘉贵妃若还禁足不出席,恐怕李朝也会担心,有所异议。''他停一停,有几分为难,看向如懿,``毕竟,璟兕之事并非证据确凿,不能认定了是嘉贵妃所为。''

若是不怪嘉贵妃,又能怪谁呢?如懿满心冷笑,脸上却只能强忍着,露出温婉神色,她太过于明白皇帝的心思,他已经决定的事,又是关乎颜面的事,有何可辩驳的呢?她不屑,亦不欲在这种小事上反对,便以更谦和的笑容相迎:``皇上思虑周全,皇上决定便是,臣妾没有异议。''

皇帝的神色放松了许多,赞许道:``皇后贤惠。''

如懿的笑,柔婉得没有任何生硬与抵触的棱角,怎么能不贤惠呢?在宫中浸淫多年,从姑母而始,有太后点拨,又朝夕见孝贤皇后的模样,她再愚昧冥顽,也该学的些皮毛了吧?于是她索性道:``嘉贵妃禁足后一直是以常在的位分对待,既然黄色要顾着她和李朝的颜面,索性还是恢复贵妃的待遇吧,免得她遇上母族的人抱怨起来,说咱们表里不一委屈了她。''

皇帝不悦地轻嗤:``出了这样的事,嘉贵妃还敢说嘴么?''然而他还是答允了如懿,嘱她细细办妥。

如懿欠身从养心殿告退,三宝便迎上来道:``愉妃小主已经到了翎坤宫,在等着娘娘呢。''

如懿面无表情,只是口中淡淡:``她来得正好,本宫也有事要与她商议。''

三宝见如懿如此神色,知她有不喜之事,更是大气也不敢出,赶紧扶如懿上了辇轿,伺候着回去了。

长街夹道高墙耸立,透不进一缕风来,天上连一丝云彩也无,日头热辣辣地泼洒着热气,连宫女手中擎着的九曲红罗黄凤伞也不能遮蔽分毫,如懿斜在辇轿上,听着抬辇太监们的靴底喋喋地刮着青石板地面,越发觉得窒闷不已。

过了长街的转角,便望得见后宫的重重飞檐,映着金灿灿如火的阳光,像引颈期盼的女人渴望而无奈的眼神。

如懿不知不觉便轻叹了一口气,转首见角门一侧有女子素色的轻纱裙角盈然飞扬,人却痴痴伫立,啜泣不已,在这泼辣辣的红墙金日之下,显得格外清素。

如懿眼神一飞,三宝已经会意,击掌两下,抬轿的太监们脚步便缓了下来。三宝望了一眼,便道:``皇后娘娘,是祈嫔小主。''

如懿有些意外:``祈嫔才出月子不久,怎么站在这儿,也不怕热坏了身子。''

三宝连忙道:``娘娘忘了?前两日祈嫔小主宫里来报,说祈嫔小主没了公主之后一直伤心,所以请了娘家人来说说话。这不,祈嫔小主大概是刚送了娘家人回去吧。''

如懿微微颔首,示意三宝停了辇轿,唤道:``祈嫔。''

祈嫔尚在怔忪之中,一时没有听见,还是伺候她的宫人慌忙推了推她,祈嫔这才回过身来,急急忙忙擦了眼泪,俯身行礼:``皇后娘娘万福金安。''

如懿苦笑:``如今本宫还有什么可安的,还不是与你一样么?''

一句话招落了祈嫔的眼泪,她泪眼朦胧的容颜像被风吹落的白色山茶花的花瓣,再美,亦是带了薄命的哀伤。

如懿步下辇轿,取下纽扣上系着的绢子,亲自替她拭去腮边泪痕:``才出月子,这样哭不怕伤了眼睛么?''

一语未落,祈嫔抬起伤心的眼感激地望着如懿:``皇后娘娘,这样的话,除了臣妾的娘家人,只有您会对臣妾说。''

如懿执着她的手,像是安慰自家小妹。她婉和道:``咱们原本就投缘,如今更是同病相怜,不彼此安慰,还能如何呢?''她停一停,``送了家里人出宫了?''

祈嫔点头:``是。家人进宫也只能陪臣妾一个时辰,说说话就走了。''

如懿温然道:``本宫同意你家人进宫,是为舒散你的伤心,好好宽慰你,而不是更惹你伤心。若叫你难过,不如不见也罢,且你不是足月生产,而是受惊早产了六公主,更要好好养着自己的身子才是。''

祈嫔死死地咬着绢子,忍不住呜咽道:``皇后娘娘,臣妾是没有办法,真的没有办法,臣妾一闭上眼睛,就看见六公主的脸。她一生下来就比小猫儿大不了多少。脸是紫的,人也皱巴巴的,可臣妾看她一眼,就觉得她像足了皇上和臣妾,她是个好看的孩子,臣妾心疼她,可是她不肯心疼臣妾,才活了几天就这么走了。''她的泪大滴大滴地滑落在如懿裸露的手腕上,带着灼热的温度,烫得如懿的心一阵一阵哆嗦,``臣妾就是想着她,睡不着的时候想,睡着了又想,可是臣妾与她的母女情分就这样短,臣妾就是想不明白,她在臣妾肚子里长到这么大,千辛万苦到了人世,难道就只为了或这么几天就丢下臣妾去了么?''

祈嫔哭得伤心欲绝,连如懿身后的三宝也忍不住别过脸去悄悄拭泪,如懿怜悯而同情地抚摸着她的鬓角,随手从她的髻后摘下一朵小小的纯色的白绢花儿在指间,低低道:``这朵花儿,是戴着悼念你的六公主的吧?''

祈嫔有些畏惧地一凛,盯着如懿,嘴唇有些哆嗦,作势就要跪下去:``臣妾,臣妾糊涂。六公主过世月余,臣妾不该再戴这个,宫里头忌讳的,皇后娘娘恕罪。''

如懿的声音凄然而温柔,扶住了她道:``宫里头是忌讳这些白花白朵儿,可本宫不忌讳。''她将鬓边的银器花儿摘下戴在祈嫔髻后,``你伤心,本宫和你一起伤心。你的眼泪,本宫替你一起兜着。只是这朵白绢花,到了本宫这里就是最后了,别再让别人看见,你的六公主才活了这几天,你就伤心成这样,那本宫的璟兕养了这么大,本宫是不是就该伤心得跳进金水河里把自己给淹进去了?本宫跳下去了,也拉上你一同淹着,这样害了咱们孩子的人就越发高兴了。不过,左右咱们都淹没了,那些人的笑声再大,咱们也听不见了,是吧?''

祈嫔猛地一颤,眼里皆是狠戾的光:``皇后娘娘!咱们的孩子是被人害死的!臣妾的六公主不该这么早就出世,更不该这么早就离开了!''她环视着四下,惊惧而狠辣,``是她!是她养着的疯狗害了咱们的孩子!''

祈嫔的身体剧烈地颤抖着,牙齿格格地咬着,仿佛要咬人似的。如懿搂过她,轻声哄着,笑容温柔得能滴出水来:``别这样!别说这样的话!湄若,你的孩子走了,是跟本宫的五公主做伴儿去了。咱姐妹俩在一块儿,到了九泉底下也不会寂寞,她们都在一块儿呢,就跟咱们一样。''如懿一字一字缓声说来,任由心口的烦恶如扑腾的海浪,颠仆起伏。

祈嫔的泪大片大片洇湿了如懿的衣袖,那种腻嗒嗒的感觉,让如懿难过又生厌:``你会哭,本宫也会哭,谁不会伤心呢?可偏偏为什么是咱们伤心?这些眼泪珠子,活该是咱们的人来流,对不对?''她抚摸着祈嫔绾起的青丝,动作轻柔得如在梦中,``你还年轻,应该比本宫更明白。孩子没了,与其伤心的不死不活,还不如想想,加把力气再生下一个,只要能生,就不算完!还有啊,皇上解了嘉贵妃的禁足,她也要出来了。见了面,把你的眼泪收起来,把你的恨也收起来,自己知道便罢,别叫人看见了,也知道该怎么防着你了。知道么?''

祈嫔伏在如懿的臂弯里,只是无声地抽泣着,好像一只受伤的小兽,终于寻到了母兽的庇护,安全的瑟缩成一团。

如懿静静地怕着她的背,仰起脸时,忽而有风至,有大团大团的雪白被吹过宫墙,纷扬如雪。

如懿轻轻地笑了,伸出细薄的手接住,低声叹道:``六月飞雪啊!像不像?''

祈嫔愣愣的抬起脸,低声道:``皇后娘娘,是老天爷觉得我们的孩子死得太怨望了!''她的声音弟弟的,像是从幽门鬼谷传来的女鬼的悲切声,让人心酸之余,又觉不寒而栗。

如懿的神情渐渐淡漠下来,像沾染了飞雪的清寒:``湄若,即便受伤,流血,与其看着它腐烂流脓,溃烂一团,还不如雕上花纹,让它绽放出来,是伤也是花,才不白白痛这一场,明白么?''

\hypertarget{ux7b2cux5341ux4e09ux7ae0-ux51faux55e3}{%
\chapter{第十三章 出嗣}\label{ux7b2cux5341ux4e09ux7ae0-ux51faux55e3}}

金玉妍再次回到众人的视线中时,已经是五月末的天气。比起之前许多年的志得意满、风华正茂,玉妍的美丽如被蚕食的满月,终于有了渐渐月亏之势。

其实,她还是很美的。长白山的冰雪养育出她咄咄逼人的美艳之姿,恍若灼灼的阳光,几乎让人睁不开眼。只是蜡照半笼金翡翠,麝熏微度绣芙蓉,宫中的日子啊,雨是绵绵的,风是瑟瑟的,就这样不知不觉,催得红颜弹指老,刹那芳华,便是``欹红醉浓露,窈窕留馀春''的红药,亦有闲倚晚风生怅望,可怜风雨落朝霞的时节了。

金玉妍倒并无半分颓丧怨望之气,相比因为丧女之痛而变得如木头人一般的忻嫔,携了侍女丽心的手步入翊坤宫的她,依旧丽质浓妆,明艳迫人。

倒是绿筠有些慨叹:``昨日见嘉贵妃陪皇上一同随见李朝的使臣,她的眼妆画得那样浓,还是遮不住眼角的细纹。喷啧,其实都这把年纪了,何必还争这口气呢?''

如懿笑着拿羊脂玉轮细细磨着手背:``何止嘉贵妃,本宫摸着自己的皮肉,也比上一个春天松弛不少。岁月催人老,谁不想多留时光停驻片刻呢。也亏得这几日嘉贵妃陪着皇上见李朝的使者,本宫身子不适,才能偷懒片刻了。''

绿筠自嘲地一笑:``臣妾总归是认了。老就老吧,谁没有老的一天呢。叫臣妾如嘉贵妃一般每日浓妆数个时辰才出门,天不亮就起身对镜梳妆,大半夜了还在用人参熬玫瑰水浸手泡脚的,臣妾想想都觉得累了。''

如懿``扑哧''一笑:``所以呀,活该咱们不如嘉贵妃了。她的细纹是遮不住,可是远远望去时,还是如二八佳人一般。''

玉妍听见这样的话倒是颇为得意,笑吟吟道:``人活一口气,树争一张皮,臣妾出身李朝,学过的谚语并不多,唯有这一句却时时记在心上。若是连自己的脸面也不要了,不肯好好打扮了,那还算什么女人呢?留着鸡皮鹤发惹人笑话么?''

她这样的话,听在忻嫔耳中格外刺心。因着六公主的早夭,忻嫔一直不施浓妆,不饰金玉,往日的活泼在她身上早已不见踪影,只剩下一抹近乎于木纳的优郁。

这样的神情,是极让皇帝心疼的,所以下了旨意,于七月初四之日行册封礼,晋忻嫔为忻妃。

嬿婉在旁含笑道:``皇上七月初四便要封妹妹为忻妃了,妹妹好歹也换件颜色衣裳,笑一笑才好啊。''

忻妃冷冷淡淡道:``我比不得嘉贵妃,自己儿子的腿残废了还能整日笑吟吟对人,便是想学也学不来的。''

金玉妍凤眼斜斜飞转,冷笑道:``忻妃妹妹真是伤心过头了。难道你这般服丧,六公主便能活过来了么?''

六公主的早夭,多多少少与嘉贵妃所养的``富贵儿''有关。虽不能指证为玉妍唆使,但到底是她疑影最重。如此这般放肆言论,连最老实的婉嫔也不觉侧目,悄声道;``嘉贵妃姐姐,这样伤人心的话,还是不要说了吧。''

殿内殿外,皆是寂寂。只有庭前几树石榴开得如火如荼,一阵风过,吹得满树繁花烈烈如焚,几乎烧红了半院空庭。

如懿怔了怔,想起那原是生下璟兕不久后皇帝喜悦,命人移栽到翊坤宫中的石榴,以示多子多福。

嬿蜿闲闲地拨弄着手中的青碧描金茶盏,浅碧色的云雾银峰蒸腾着白蒙蒙的水汽,映出她薄薄的笑意:``人生得意须尽欢。六公主自然不能复生,可八阿哥的腿脚也不能再健步如飞了,四阿哥也不能复宠如前,得皇上欢心。说来啊,还是嘉贵妃姐姐想得开。''

玉妍极重颜面,被嬿婉戳到痛处,脸色瞬间寒了下来,森森道:``虽然本宫的四阿哥一时受小人陷害,连着八阿哥也坠马受伤,可他们是皇家的儿子,哪怕腿不行了,没恩宠了,到底还是凤子龙孙。这个,可由不得本宫想不想得开!''她鄙夷剜着嬿婉,``令妃自己没有孩子,倒惯会管孩子的闲事!''

嬿婉脸上一红,旋即变得紫涨,却也不能辩驳,只得垂下脸,气咻咻地拨着手指上的红宝石戒指。

玉妍见嬿婉气馁,越发盛气凌人。如懿颇为唏嘘:``多子多福,古人的老话,到底是不错的。嫔妃之中,嘉贵妃子嗣最多,这样的福气,咱们是羡慕不来的。''她话锋一转,向着纯贵妃和海兰道:``只是话说回来,三阿哥是皇上的长子,敦厚有礼,五阿哥如今更是在皇上跟前得力,堪为左膀右臂。生子应当如此,才算是祖宗的孝子贤孙,否则只是论一个凤子龙孙的血统,实在算不得什么。想想康熙爷的八阿哥和九阿哥,因争帝位而被先帝削爵圈禁,一个起名阿其那,一个塞思黑,极尽羞辱,哪里还有半点儿凤子龙孙的颜面呢?''

玉妍听得此节,不禁矍然变色:``皇后娘娘是拿康熙爷的八阿哥允禩来比臣妾的八阿哥么?''

如懿也不气恼,只是和颜微笑:``允禩这样的不肖子孙,康熙爷一辈己经出了一个了,怎么嘉贵妃这样多心,以为皇上也会有这样的儿子么?''

玉妍眉心的褶皱稍稍平复,浮起一抹得意的笑,扬了扬手中的水红色滚宝翠蓝珠络的绢子:``皇上的孩子,自然不至于如此。孝贤皇后的丧仪上,大阿哥和三阿哥稍稍失仪,皇上便严厉教训。有了这个做榜样,谁还敢么?再说得远一些,本宫的儿子行八行四本就是占了好运气的。太宗皇太极是皇八子登基,先帝雍正爷是皇四子登基,皇上也是皇四子登基。本宫的孩子再不成器,有祖宗这样的福泽庇佑,也差不到哪儿去的!若是有幸能将这福泽一脉相承下去,也是情理之中啊!''

此言一出,四座皆惊。然而并无人应答,也不屑于应答。如懿亦只是用银签签了一枚樱桃滑入口中,以一丝不易察觉的冷笑默然相对。倒是婉嫔想要说些什么缓和这种诡异的沉默,绿筠忙悄悄按住了她的手,示意她不要多言。

海兰有些怯怯地适时添上一句道:``福泽与否,还真不好说,但是圣祖康熙爷幼年得了一场天花,人人以为是逃不过去的劫难,后来也只是落了几点小小瘢痕,丝毫不影响圣祖的天纵英明。''

玉妍以为众人被震慑住,衔了一缕冷笑道:``所以,别以为本宫的孩子一时不得皇上宠爱,或是有了些许残疾,便轻慢了他们。孩子们的福气,都在后头呢。''

绿筠实在按捺不住:``本宫的三阿哥是不算聪明伶俐,如撇开三阿哥不算,四阿哥也算是皇帝诸子中最年长的。但年长算什么,比谁的胡须长么?现放着皇后娘娘的十二阿哥在呢,哪位皇子的福气也比不上十二阿哥这位嫡子呀!''

如懿看一眼绿筠,谨慎道:``纯贵妃此言差矣!十二阿哥尚且年幼,贤愚如何尚是未知之数。何况嫡子又如何?太祖努尔哈赤的嫡子褚英和圣祖康熙爷的嫡子允礽都因谋逆不孝而被废了太子之位,这便是警戒后人,不要以嫡庶分尊卑贤愚。孩子们自己争气,才是最要紧的。便是眼下还没有孩子的,也不必心急。皇上正当盛年,妹妹们也绮年玉貌,什么福气怕等不到呢。''

一语既出,嫔妃们皆是敬服。绿筠率先起身,领了一众人等行礼:``皇后娘娘教诲,臣妾们谨记在心。''

玉妍伫立其中,未曾躬身,愈加显得格格不入,她只得屈身福了一福:``臣妾明白了。''

如懿拨一拨手边小几上珊瑚釉粉彩花鸟纹瓷瓶里供着的一大把几欲滴露的红色芍药,翠茎红蕊,映叶多情。她温和的笑容中带了一丝沉郁的告诫:``今日阶前红芍药,几花欲老几花新。开时不解比色相,落后始知如幻身。许多事繁华得意只在一时,妹妹们也不必过于执着眼前,还是多求一求后福吧。''她说罢,站起身来,意欲转入内殿。可是才一迈步,脚下一个踉跄,人便斜斜滑了下去。

容珮惊叫一声,忙忙和扑过来的海兰一起牢牢扶住,一迭声唤道:``太医!快请太医!''

如懿的不适晕眩,自然引来了皇帝的关照与陪伴。她闭目和衣躺在床上,听着皇帝的脚步挟着风声而入,不觉含了一丝浅笑。

江与彬跪在床前请脉良久,却是一脸喜色,向急急赶来的皇帝道:``恭喜皇上,恭喜皇后娘娘,皇后娘娘并非凤体不适,而是有喜了!而且已经三个月了。''

窗外的石榴树影映在湖碧窗纱上,风移影动,花枝姗姗,欹然生姿。如懿一脸惊诧与意外,想要笑,却先落了晶莹的泪:``臣妾这几个月晕眩烦闷,原以为是生璟兕的时候落下的病根,没想到竟是有喜了。''她握住皇帝的手,依依道,``皇上,是不是璟兕在天有灵,怕臣妾与皇上膝下寂寞,所以又转世投胎,来做咱们的孩子了。''

因着两位公主早夭,皇帝郁郁多日,如今听闻如懿再度有孕的喜讯,常日阴霾一扫而空,拥着如懿的肩,眼中不觉泛了泪光:``是。璟兕知道咱们想她,所以又回来了。''

海兰与忻妃陪在如懿身边,一脸惊喜,忻妃更是忍不住感泣:``还是皇后娘娘好福气,这么快又有了孩子。这样臣妾也有些盼头了。''她的眼泪还在腮边,继而愤愤不平,``还好刚才愉妃姐姐和容珮扶得快,否则皇后娘娘受了嘉贵妃的闲气,头晕脚滑,伤了腹中皇嗣,可怎么是好?''

海兰亦抚着心口,后怕不已:``还好皇后娘娘没事,否则嘉贵妃万死也难辞其咎了。''

皇帝微笑的眼波倏然转为薄怒:``怎么?嘉贵妃才解了禁足,便又惹是生非了么?''

海兰郁然长叹,却只道:``嘉贵妃的性子,皇上还不知道么?一向是想说什么想做什么便由着自己的!''

此时,绿筠领着众人候在廊下,并不敢进来多问,只预备着随时陪侍。

玉妍不耐烦道:``天气这么热,咱们还要守在这里多久?说来皇后娘娘的凤体也太娇弱了,只是晕眩,又未跌倒!''

绿筠心中愤懑,别过脸不理会她,只向婉嫔道:``也不知娘娘身子如何了?无论多久,咱们都是要等的。''

殿中敞亮,外头的一言半语偶尔落进,像投进湖心的石子,泛起涟漪点点。皇帝起身推窗,转眸向外,庭中绿瘦红肥开得真人,花枝曳曳处落下一蓬蓬水墨似的影子,生出几许清凉。不远处重重花影之后立着金玉妍,一袭宝石蓝片金葡萄花彩宫装衬得窈窕宜人,正握着一柄刺绣洒金牡丹团纱扇,在树下悠然观望花落,毫无关切之意。

皇帝鼻翼微张,冷然道:``中宫凤体违和,嘉贵妃还能如此悠然赏花,真是全无心肝!说!她到底如何冒犯了皇后?''

海兰看着含怒的皇帝,有几分畏惧,藕荷色的衣裙盈然一闪,退后几步道:``事关皇子。臣妾身为人母,不宜多言。''

皇帝略略点头,正要再发问,忻妃``扑通''一声跪倒在地,悲悲切切道:``皇上,臣妾的六公主死得不明不白,臣妾不敢胡乱猜疑是谁暗害。但是嘉贵妃出言不逊,臣妾不敢不言了。''她一字一字,含了蕴蓄多时的恨与怨,一并吐在了字句中,``臣妾以下所言,皆为嘉贵妃今早大放厥词所说,臣妾不敢添加一字半句。请皇上明鉴。''她俯身三拜,模仿着嘉贵妃的口气道,``本宫的儿子行八行四本就是占了好运气的。太宗皇太极是皇八子登基,先帝雍正爷是皇四子登基,皇上也是皇四子登基。本宫的孩子再不成器,有祖宗这样的福泽庇佑,也差不到哪儿去的!若是有幸能将这福泽一脉相承下去,也是情理之中啊!''

有长久的静默,只听得风声簌簌入耳。他的声音极缓极缓:``你们身在后宫,有许多前朝的事,朕不便多说。但是如懿,你是皇后,也该知道一些。''

如懿见皇帝如此郑重,肃然道:``皇上说,臣妾便听着。''

皇帝施施然立于窗下,一身松石蓝刻丝暗金柏纹的长袍,只用明黄带子松松系住,越发长身如岩下松,优雅中不失赫赫之气。然而他的面色却如那松石蓝的缎子,暗沉沉地发闷:``前些日子李朝来贺,提起朕是否有立太子之意。朕也不便多言,便打发了。谁知前几日朕单独召见李朝使者,那人却说\ldots\ldots{}''皇帝深吸一口气,语调更沉,``却说起孝贤皇后生前两位皇子早夭,朕既爱重永珹,何不出继永珹为孝贤皇后嗣子,来日孝贤皇后灵前,也可有人祭祀供奉!''

海兰在皇帝跟前一直讷讷不肯多言,听到此节,亦隐隐失色:``皇后娘娘己有嫡子,永珹若出嗣孝贤皇后为子,岂不宫中有两位嫡子,既是异母所生,长幼有别。哪怕来日无事,只怕也要生出许多是非来!''

忻妃自是年轻,又出身官宦门第,自然晓得其中利害,陡然扬眉厉声:``皇上,若四阿哥出继为孝贤皇后嗣子,那么得逞之后又想要得到什么呢?''

如懿靠在金丝攒海棠芍药厚缎软枕上,微笑如冬日湖上冷冷薄冰,纵然冰上暖阳融融,冰下却依旧水寒刺骨,汹涌流动:``孝贤皇后为嫡后,臣妾为继后,臣妾的孩子自然不能与孝贤皇后之子比肩了。臣妾真的很想知道,皇上盛年,他们这般苦苦不放,到底是为着什么?''

皇帝的面上有着异乎寻常的平静,而眸中却有着凛然拒人于千里的冷漠。他继续道:``自李朝来见,朝廷里也渐渐不大安宁,总有那些不大安分的人窥探朕的心意,说起早立太子之事。''

如懿凝神片刻,掀开覆盖在身的湖蓝华丝锦被,凛然跪下道:``皇上春秋鼎盛,年富力强,何须早立太子!何况自先帝爷起,即便有合意的储君人选,也是放置在`正大光明'的匾额之后,待到龙驭宾天后才能开启,以免再出现康熙爷时九子夺嫡的惨状。说这样话的人,岂非诅咒皇上?实在罪该万死!''

皇帝负手而立,手指的关节因为用力而泛出难看的苍白。他的脸上看不出一丝表情:``说这样话的人,的确罪该万死。朕有嫡子,何须商议立太子之事。来日水到渠成之事,不必再有异议了。''

如懿的脸色白了一白,郑重俯首,恳切道:``永璂才三岁,不比孝贤皇后的两位嫡子,幼年伶俐。哪怕是中宫嫡子,也得好好请师傅教导。能不能有出息,还得成年才看得出来。''

皇帝淡淡叹了一声,扶了如懿起身:``皇后,你有身孕,不许这么跪着,仔细伤了自己。''他扶着如懿在床边坐下,似是无限感怀,``也是。永璂还小,如今朕的儿子里,唯有永琪可堪重用。''

海兰吓了一跳,慌忙跪下,连连叩首道:``皇上!皇上!永琪年长,合该为皇上分忧。但臣妾只有永琪这一个儿子,只盼他早日成家立业,臣妾也可以含饴弄孙,膝下承欢了。''

皇帝微微颔首,静静道:``李玉,传嘉贵妃进来。''

李玉见皇帝脸上一毫神色也不露,有些不解,忙出去传了嘉贵妃进来。

皇帝看着她道:``朕传你进来,是有件喜事要告诉你和愉妃。''

玉妍见海兰与忻妃早已跪着,忙也喜滋滋跪下道:``皇上疼臣妾,臣妾明白,臣妾洗耳恭听。''

皇帝的目光温和些许,徐徐道:``永珹和永琪的年纪也不小了。朕打算在朝中重臣家各选个好女儿,许配给两位皇子为福晋。但你们身为皇子的生母,可有心仪的人家,也可说来给朕听听。''

玉妍见海兰只是沉吟不定,施施然笑道:``先帝在世时最重手足之情,与和怡亲王兄弟清深。和怡亲王的次女嫁与散秩大臣福僧额为妻,福僧额乃和硕额驸。听闻二人生有一位格格,聪慧美丽,大方高贵,配给永珹很是合适。而且格格有皇家血缘,凤子龙孙,这才般配么。''

皇帝的嘴角泛起一缕笑意:``你的思虑倒很周详,凤子龙孙,时时事事想着攀高处去,倒也像你和你儿子的性子。''他瞟一眼海兰:``愉妃,你呢?''

海兰一脸的本分恭谨:``只要女孩儿贤良淑德,能与永琪夫妻和睦,不拘什么门第,都是好的。臣妾心思,还请皇上成全。''

如懿对海兰的应答极为满意,递去一个含笑的眼色,心中暗暗赞许。

皇帝``唔''了一声,脸上的笑容渐渐敛去:``嘉贵妃,看来你比愉妃懂得选儿媳多了。四阿哥若明白你的苦心,倒真能成器了。''

玉妍见被冷落多时的儿子得了皇帝赞许,颇有意外之喜:``皇上说得是。臣妾与永珹母子连心,他都明白的。臣妾总对永珹说,先帝爷为皇子时是四阿哥,皇上也是四阿哥。有这样的榜样珠玉在前,他若能用心做事,必然也能成一点儿气候,不叫皇上生气。''

皇帝听完,眉心骤紧,眼眸暗沉。如懿伴随皇帝多年,知他己是极为愤怒,却见玉妍难得出来后能与皇帝说上这么多话,犹自欢喜不知。

皇帝的暴怒随着一记响亮的耳光落在了玉妍面上,顿时起了五个血红指印,肿得高高。皇帝怒道:``恬不知耻,罔顾人伦!儿子这样,额娘更是不堪!朕还活着呢,你们都打量着四阿哥当皇帝的福泽了!简直昏聩!''

玉妍吓得瞪大了眼睛,连连道:``皇上息怒!臣妾冤枉,臣妾冤枉啊!''

额上几欲迸裂的青筋显示了皇帝愈燃愈烈的怒气:``冤枉!朕冤枉你都觉得腌臜了自己!串通了李朝使者想要自己的儿子去做孝贤皇后的嗣子,也不问问孝贤皇后在九泉之下是否答应!朕且问问你,你的儿子做了孝贤皇后的嗣子,成了嫡出,你们母子还想要谋算些什么?''

玉妍一时不曾悟过来,听到此处,不觉惊声呼道:``出继为嗣子?臣妾全然不知啊!''她满脸泪水,失声唤道,``皇上,便是臣妾母族来使这般说了,也不算全错!到底,到底孝贤皇后在时,也是极喜爱永珹,日日抱在跟前的!''

皇帝怒极,冷道:``你是什么东西,也敢教唆着皇子觊觎皇位了!朕本来对木兰围场之事将信将疑,始终不肯相信朕的儿子会做出悖逆人伦、谋害君父的事情来,如今看来,有你这样的额娘,他不做这样的事倒反而意外了!''

玉妍面色煞白,如同五雷轰顶,紧紧抱住皇帝的双腿辩白道:``皇上说什么木兰围场之事,永珹忠心救父,一心一意只为了皇上,皇上万不可听信小人谗言,诬陷了他呀!''

``朕诬陷他?是他要朕的命!''皇帝气得目毗尽裂,``朕宠爱你多年,倒宠得你们母子不知斤两了!你是为朕生了皇子,可生了皇子又如何?也要看孩子是从谁的肚子里出来!你不过是李朝进献给朕的贡女,也敢仗着几分姿色仗着几个孩子在朕的后宫兴风作浪,谋害皇嗣!''

恍如被利剑戳穿了身体,玉妍像一个被风吹落的稻草人,顿时瘫倒在地:``臣妾谋害皇嗣?明明是她,是她们,害了臣妾的儿子!''嘉贵妃形同疯狗,扑上前来,指着如懿与海兰凄厉地喊道:``皇上!永珹被您冷落,臣妾可以不怨!但是永璇还那么小,他坠马的时候只有愉妃的儿子离得最近。愉妃,皇后!你们敢不敢发誓,不是你们的儿子永琪嫉妒永珹得宠,所以害了永珹被冷落,还想害死永璇!你们这些**!毒妇!''

三宝领着一众宫人手忙脚乱地拉住玉妍,可她像是发疯了一般,力气极大,拼命挣扎着呼喝不己。

海兰似是被玉妍吓坏了,忙忙地躲到一边,啜泣着道:``皇上,臣妾从来没有想过害人,臣妾敢发誓,皇后娘娘也没有!''她举起三指,敬肃发誓:``苍天在上,若我坷里叶特氏海兰与皇后有心加害嘉贵妃之子,便叫我不得好死,死后也永堕阿鼻地狱,不得超生!''

海兰的誓言发得惨绝,玉妍也不觉怔住。只这一瞬间,忻妃己经暴烈而起,厉声号啕:``是你!果然是你害了我的六公主!''她扑向皇帝,声泪俱下:``皇上,您一直不能确信嘉贵妃养的那条疯狗伤人是不是嘉贵妃指使,如今您可听明白了,除了她旁人再无要害咱们的心!一定是她恨极了皇后娘娘的养子五阿哥夺了四阿哥的宠爱,又有八阿哥坠马的嫌疑,所以要报复皇后娘娘,伤及十二阿哥。若不是那日五公主穿了红衣吸引了疯狗被误伤,可能如今便是您的嫡子十二阿哥不在了!而臣妾那日也在场而被误伤,累得六公主早产,先天不足惊惧而死!''她哭得几乎昏死过骈,``皇上啊皇上,都是嘉贵妃这个毒妇算计好了,害此了五公主和六公主啊!''

皇上脸上的肌肉悚然抽搐,暴怒不已。他一把揪住玉妍的头发将她拖倒在地,眼里沁出鲜红的血丝,神色骇人:``**!自己不过是一件贡品,也敢这样谋害朕的孩子!''

玉妍的嘴唇剧烈地哆嗦着,像是不可置信,茫然地睁大了眼,睁得几乎要裂开一般,喃喃道:``贡品?皇上,您说什么贡品,是臣妾听错了,是不是?''

皇帝冷冷地踢开她抱着自己双腿的手,像踢开一块残破的抹布,嫌恶道:``朕明明白白告诉你,你不过是一件贡品而已,你的儿子岂可担社稷重任?若你还不懂,朕就告诉你,当年圣祖康熙拒绝**臣举荐八皇子允禩为太子,理由只有一个,他的生母良妃卫氏是辛者库贱婢,出身低贱,所以她的儿子也不配做太子!今日也是一样,你不过是小国贡女,和一件贡品有什么区别?朕从来没想过让你的儿子做太子!''

须臾的静默,静得如死亡一般。

一声凄厉的呼号最后划破了这静默,如同泣血的杜鹃一般,耗尽心力,悲鸣不已。

皇帝的言语失去了所有的温情与顾念,冰得疹人:``李玉,传旨六宫。四阿哥永珹娶和硕额附福僧额之女为嫡福晋。''他未顾忻妃诧异而不甘的目光,继续道,``朕第四子永珹,出嗣履亲王允祹为后,再不是朕的儿子。''

玉妍身心俱碎,人已痴在了原地,如同丢了魂一般,听得皇帝此言,只是浑身战栗不己。

``朕满足你们母子的心愿,让你们再攀龙附凤一次,娶了想娶的女子,但是朕也绝了你们的狂妄念头。先帝与朕都是四阿哥,这一脉相承的福气,你们便不用痴心妄想了。朕只当再没这个儿子!''皇帝再未看玉妍一眼,以决绝的姿态背身道:``李玉!拖她回启祥宫,朕再不想见她!''

\hypertarget{ux7b2cux5341ux56dbux7ae0-ux4f24ux91d1}{%
\chapter{第十四章 伤金}\label{ux7b2cux5341ux56dbux7ae0-ux4f24ux91d1}}

这一年的夏天,便随着金玉妍的彻底失宠忽忽而过,漫漫沉寂了下去。

如懿的再度有孕,让皇帝几乎将她捧在了手心里,连太后亦感叹:``皇后年岁不小,这几年接连有孕,可见圣眷隆重,真当羡煞宫中殡妃了。''

这话倒是真的。大约是璟兕的早夭,又紧接着怀上了腹中这个孩子,连皇帝都与如懿并头耳语,总觉得是璟兕又回来了。而钦天监更是进言,道:``天上紫微星泛出紫光,乃是祥瑞之兆,皇后娘娘这一胎,必定是上承天心,下安宗兆的祥瑞之胎,贵不可言。''

钦天监素来观察天象,预知祸福,皇帝十分相信。且璟兕与六公主夭折后,皇帝也极盼望如她腹中的孩子能带来更多的欢喜,冲一冲宫中的悲怨之气故而更是大喜过望。这样的爱宠和怜悯,让皇帝待如懿如珠似宝,若非有紧急朝务,必定每日都来陪如懿用膳说话。

如懿虽不十分相信钦天监的喜报,总以为又几分阿谀奉承讨得皇帝欢心的意思,却也不说破,只是一笑而已。

宫中都沉浸在中宫有喜的喜庆之中,浑然忘记还有金玉妍这个人了。

秋风飒飒,红叶落索。寒霜满天,霰雪如织。

乾隆二十年的初冬,十一月,小雪初至。

如懿的月份已经很大了,眼看着临盆之日逐渐近了,人渐渐慵懒,身子也越发笨重。翊坤宫中早已让人挖好了喜坑,如懿的额娘也进宫来陪着。而六宫之人,也是日日前来陪侍。当真是门挺热闹,连门槛都要被踏破了。

这一日,江与彬来请了如懿的脉,如懿斜靠在床上,慵懒的姿势让人想起夏日碧波池中盛绽的莲花。

江与彬道:``孩子在腹中一切都好,娘娘月份渐大,起坐间要小心。尤其这几日下雪了,出门格外仔细脚滑。''

容佩抿嘴笑道:``江大人总把咱们奴啤该当心的事都说了。''

如懿抚着高高隆起的肚子,含笑道:``都生了两回孩子了,自然什么都懂了。倒是难为你们惢心惦记着,如今自己都是两个孩子的娘了,还只为本宫操心。''

江与彬道:``惢心伺候了娘娘小半辈子,哪有不上心的。这些日子下雪,她腿脚不方便,不能来给娘娘请安,就只在家埋头做小衣服呢,希望能进献给娘娘腹中的小阿哥。''

殿中供着一溜盛开的水仙,盆盆花瓣十余片卷成一簇。花冠由轻黄颜色慢慢泛上淡白,映着翠绿修长的数百叶片,便称``玉玲珑''。此时水仙被殿中铜火盆中的银炭一醺,花香四溢,宛如甜酒醉人。

如懿笑吟吟道:``你说是小阿哥,齐太医也说是小阿哥。真就这么准么?''

海兰笑着道:``不止太医这么说,这回连钦天监也开口,说皇后娘娘这一胎是祥瑞至极的福胎呢。''

如懿拂一拂身上盖着的桃紫苏织金棉被,被面上用银线彩织着和合童子嬉戏图,映着樱桃红棉帐上瓜瓞绵绵的花色一天一地都是花团锦簇迎接着新生的欢喜。连素来衣着素雅的海兰,鬓边亦簪了一朵胭脂花色重瓣山茶。如懿看着那金黄纷叠的花蕊,含着笑暗暗寻思:这一枝品种算是``赛洛阳'',还是``醉杨妃''?

都不要紧,左右都是喜悦的红。

忻妃无限羡慕地小心翼冀地抚摸着如懿的肚子,眼里有晶莹的泪光:``还是皇后娘娘的福气最好。臣妾想,这是五公主又回来了。''

如懿看着她,不觉怜悯,温柔道:``你放心,六公主还会回来的。本宫入宫多年,才有如今连连有喜的福分。你还年轻,福报会更深的。''

忻妃闪过一丝喜色,旋即切齿道:``皇后娘娘说得是,臣妾相信福报。更相信报应。''她快意地道,``听说金玉妍病入膏肓,快不成了。''

如懿颇有些意外:``病入膏肓?本宫怎么都不知道?''

海兰忙道:``皇后娘娘有着身孕,谁敢胡说这样不吉利的事儿,吵扰了皇后娘娘的清静。只是嘉贵妃怕是真的不成了,皇后娘娘可知道,李朝又遣了一拨儿年轻女孩子过来,说是打发给宫里伺候的,其实还不是看着嘉贵妃不成了,所以急忙又物色了新人来,生怕失了恩宠靠山。''

忻妃冷笑一声:``愉妃姐姐,这个我隐约听说了,也不是这一回了。自从嘉贵妃失宠,四阿哥出嗣,李朝巴巴儿拨了多少女孩子过来,皇上不是都赐给各府的贝勒亲王们了么?一个都没留在宫里。''

如懿轻轻摇头:``这回却不一样了。李朝如此殷勤,皇上盛情难却,昨夜来用膳时说起,己经留下了一位宋氏为贵人。听说也是两班贵族之女,还是李朝世子亲自挑选的美人,不日就要进宫了。这样,也不算太拂了李朝的面子,也是定了他们的心。''

忻妃鄙夷地撇撇嘴,将绢子塞进手腕的绞丝白玉镯里:``李朝的心也太急了,嘉贵妃还没死呢,就这么赤眉白眼地送新人来了。倒是咱们没盼着她咽气她母族的人先盼上了。''

如懿靠着背后的馥香花团纹软枕,沉吟着道:``嘉贵妃病成这样,皇上去看过么?''

``皇上忙于朝政,并不得空儿。''忻妃含了一缕痛快的笑色,双颊微红,``自从四阿哥出嗣,皇上再未去看过嘉贵妃了。何况永寿宫那位有了身孕,皇上一得空儿,除了陪伴娘娘,也常去看她呢。''

忻妃所指,是永寿宫的令妃嬿婉,多年的殷殷盼子之后,十一月间,太医终于为她诊出了喜脉,如何能不叫她欣喜若狂?连皇帝也格外爱怜。

海兰轻叹一声,如贴着地面旋过的冷风:``自从娘娘有孕,皇上召幸最多的便是令妃,有孕也是意料之中了。''

忻妃道:``令妃微贱时总被盛贵妃欺凌,如今嘉贵妃落寞,她却得意至此,真是风水轮流转了。''

枕边有一柄紫玉琢双鱼莲花如意。那原是皇帝亲手赐了她安枕的,通身的紫玉细腻水润,触手生温。上部玉色洁白,琢成两尾鱼儿栩栩如生,随波灵活游弋。底部玉色却是渐渐泛紫,纹饰成繁绮的缠枝并蒂莲花模样,温润异常。

如懿抚着滑腻的玉柄,浅浅含笑,慵懒道,``嘉贵妃落得今日,也多亏妹妹的阿玛济事。''

忻妃切齿,含了极痛快的笑容:``她既要了臣妾爱女的性命,落得如此地步,也是报应不爽!也怪她和李朝的人都糊涂油蒙了心。臣妾阿玛朝中为官多年,门生故旧总还是有的,只稍稍去那李朝使者跟前提了一句若四阿哥出继为幸贤皇后嗣子,那人便巴不得去了,也不打里着皇上是什么性子!''

``你做得极好。''如懿赞过,若有所思道,``宫里有谁去看过嘉贵妃么?''

海兰见她在意,便道:``嘉贵妃在宫里的人缘,皇后娘娘您是知道的。如今她的处境又那么难堪,四阿哥也打发出去出继给旁人了,更没人搭理她了。''

忻妃恨恨啐了一口:``自作孽,不可活!''

如懿眼波宛转,看一眼江与彬:``嘉贵妃真的不成了?''

江与彬道:``微臣看过嘉贵妃的脉案,只怕去留只在这几日了。''

如懿抚着睡得微微蓬松的鬓发,慵懒道:``虽然宫里的人都不喜欢嘉贵妃,但本宫是皇后,不能不去看看,有些话也不能不问个真切。备辇轿吧。''

启祥宫原在养心数之后,离皇帝的居处只有一步之遥,可见多年爱宠恩眷。然而,如今却是长门一步地,不肯暂回车了。

雪中风冷,吹得那落尽秋叶的梧桐空枝簌簌有声。庭院里花草衰败,连原本该伺候着的宫人们也不知去哪里躲懒了。唯有几株枫树堆落的残红片片,从薄薄的积雪里露出一丝刺目的暗红。

如懿抚着容佩的手小心地走着,明黄缠枝牡丹翟凤朝阳番丝鹤氅被风吹得张扬而起,在冷寂的庭院中如艳色的蝶,展开硕大华丽的双翅,越发显得庭院寂寂,重门深闭。

春来赫赫去匆匆,刺眼繁华转眼空。当年富贵锦绣之地,宠极一时的嘉贵妃,亦落得辘轳金井,满砌落花红冷的境地。

如懿进去的时候,启祥宫里暗腾腾的,好像所有的光都不能照进这个曾经风光无限的宫殿里。如懿微眯了一会儿眼睛,才能渐渐适应从明澈阳光下走昏暗室内的不适。她心里有些诧异,才发觉原来并不是光线的缘故,而是所有的描金家具、珠玉摆设、纱帘罗帐,都像积年的旧物一般,灰扑扑的,没有任何光彩。仿佛这座金碧辉煌的宫殿,也随着它的主人一同黯淡了下去。

如懿虽然恨极了玉妍,但乍见此处凄荒,亦有些心惊。她不可置信地伸出手,手指轻抚之处,无不蓄了一层厚厚的尘灰。如懿忍不住呛了两口,容佩赶紧取过绢子替她擦拭了,喝道:``人都去哪里了?

这才有宫人急惶惶进来,像是在哪里偷懒取暖,脸都醺得红扑扑的。

容佩见有人来,越发生气:``大胆!你们是怎么伺候贵妃的?''

宫人们吓得跪了一地,纷纷磕头道:``皇后娘娘恕罪,容姑姑恕罪。不是奴才们不好好伺候,是贵妃小主自从病了之后,就不许奴才们再打扫这殿中的一事一物了。''

容佩蹙了蹙眉头,严厉道:``放肆!贵妃小主是病着糊涂了,你们也跟着糊涂?分明就是你们欺负贵妃在病中就肆意偷懒了。要我说,一律拖去慎刑司重责五十大棍,看还敢不敢藐视贵妃!''

宫人们哪里禁得起容佩这样的口气,早吓得磕头不己:``容姑姑饶命,容姑姑饶命,奴才们再不敢了。''

如懿听着心烦,便挥手道:``你们都跪在这里求饶命,谁在里头伺候贵妃?''

宫人们面面相觑,唯有丽心是从潜邸便伺候金玉妍的,格外有脸面些,便大着胆子道:``贵妃小主不许奴才们在旁伺候着。都赶了出来。''

如懿拿绢子抵在鼻尖,不耐烦道:``贵妃生着病,不过是一时的胡话,你们也肯听着?''

丽心吓得脸都白了:``皇后娘娘恕罪,不是奴婢大胆不伺候,是小主任谁伺候着,都要大动肝火,说奴才们是来看笑话的,所以奴才们没贵妃召唤,也不敢近前了。''

正在纷乱中,只听得里头微弱一声唤:``谁在外头?''

如懿耳尖,立刻听见了,摆一摆手道:``都出去!''

宫人们立刻散了候在外头,容城扶了如懿缓步进去。寝殿比大殿中愈加昏暗不堪,隔着微弱的雪光,如懿看见瓶里供着的一束金丝爪菊己经彻底枯萎了,乌黑萎靡的一束斜在瓶里,滴落下气味不明的粘稠汁液。

如懿觉得有些恶心,便别过头不再去看。容佩想替她找个锦凳坐一坐,却也找不见一个干净没灰的,只好忍耐着挑了一个还能入眼些的,用绢子擦了擦,又铺上另一块干净的绢子,请了如懿坐下。

玉妍支着身子,仿佛看了许久,才能辨出她来,``咯''地笑了一声:``原来是皇后啊!''那笑声像黑夜里栖在枝头的夜枭似的,冷不丁``嘎''的一声叫,让人毛骨悚然。她见了如懿,并不起身,依旧懒懒地斜在床上,死死地盯着如懿高高的肚子,道:``皇后娘娘的肚子都这么大了,怎么还肯大驾光临,走到启祥宫这么个晦气地方。''

如懿淡淡道:``听说你病着,过来瞧瞧你。可好些了么?''

玉妍只剩了枯瘦一把,神情疏懒,也未梳头,披着一头散发,语气慵倦中含了一丝尖锐的恶毒:``病着起不来身请安,也没什么好茶水招待您的,坐坐就走吧。您是有福有寿的贵人,害了人都损不到自己的福气的,别沾了我这个病人的霉气,沾上了您可赶不走它了!''

容佩听她出言不敬,连该有的称呼也没一句,不觉有些生气,但见如懿安然处之,也只得忍气袖手一旁。

如懿坐得靠近玉妍床头,鼻尖一清二楚地闻到她身上散发出来的气息。那是一个重病的人身上才有的行将糜烂的气味,如同花谢前那种腐烂的芬芳,从底子里便是那种汁液丰盈又饱胀得即将流逝的甘腐。还有一些,是如懿要掩鼻的,那是一股淡淡的腥臭味儿,是久未梳洗还是别的,她也说不清。如懿下意识下拿绢子掩了掩鼻子,忽然瞟见玉妍的寝衣,袖口都己经抽丝了,露着毛毛的边,像是被什么动物咬过似的,参差不齐,而袖口的里边,居然还积着一圈乌黑油腻的垢。

如懿冷眼看着,道:``从前你是最爱干净的,如今怎么成了这个样子?''

玉妍睁大着眼睛看着她,懒懒道:``再怎么干净,等到了地底下一埋,都是一样的。''

如懿道:``哪怕是病了,好好看太医,拾掇拾掇,也能好的。何必这么由着自己作践自己?''

玉妍整个人是干瘦透了,像是薄薄的一张皮附在一把瘦嶙嶙的骨头上,冷不丁看着,还以为是一副骨架。袖口下露出的一节手臂,想一段枯柴似的,露着蚯蚓般突起的青筋。如懿依稀还记得她刚入府的时候,白。圆润,好像一枝洗净了的人参似的。再后来,那种婴儿似的圆润退了一些,也是格外饱满的面孔,嫩得能掐出水来。哪怕是不久之前,玉妍的手臂还是洁白的藕段似的,一串串玲珑七宝金钏子套在手上,和她的笑声一样鲜亮妩媚。

玉妍见如懿望着自己,冷笑连连:``皇后娘娘何必这般虚情假意?是我自己来作践自己么?满宫里谁不知道皇上亲口说的,还是当着你的面说的,我不过是件贡品。一件贡品,扔了也就扔了,碎了也就碎了,有什么可作践自己的!''

玉妍是病得虚透了的人,说不了几句话,便大口大口地喘息着。她的头晃了晃,一把披散的青丝扫过如懿的手背,刺得如懿差点跳起来。玉妍的头发是满宫里最好的,她也极爱惜,每日都要用煮过的红参水浸洗,端的是油光水滑,宛如青云逶迤,连上用的墨缎那般光洁也比不上分毫。可是如今,这把头发扫在手上,竟如毛刺一般扎人,借着一缕微光望着,竟像是秋日里的枯草一般,没有半分生气。

如懿见她如此,虽然满心厌恨,也不免有些伤感,只得道:``皇上是气狠了,一时的气话。你要真放在心上,那就是你的不懂事了。''

``不懂事?''玉妍凄凉地笑了一声,``我这一辈子,自以为是以朝鲜宗女的身份入侍皇家,自以为是家族王室的荣耀。为了这个,我要强了一辈子,争了一辈子,终于争到了贵妃的荣耀,生下了皇子为依靠。结果到头来,不过是人家嘴里一句`一件贡品而已,你的儿子岂可担社稷重任'。''玉妍呵呵冷笑,悲绝地仰起头,``我自己的尊严脸面全都葬送不算,连我的儿子们都成了贡品的孩子,还连累了他们一生一世。''

如懿看她如此凄微神色,不觉从满心愤恨中漾起几分戚戚之意:``皇子们到底是皇上的亲生儿子,虽然也是皇上一时的气话,可皇上还不是照样疼爱。''

``疼爱?''玉妍的眼睛睁得老大,在枯瘦不堪的脸上越发显得狰狞可怖,``皇后,你是加的女人,你应该比我更知道母凭子贵子凭母贵的道理!康熙皇帝在世的时候,八阿哥人称贤王,被满朝大臣推举为太子。结果呢,康熙爷以一句`辛者库贱婢之子'就彻底断送了这个儿子的前程。可不是,八阿哥的娘亲是辛者库的贱婢,低贱到不能再低贱。可是再低贱也好,还不是皇帝自己选的女人。我跟着皇上一辈子,结果临了还害了自己的孩子,给李朝王室蒙羞!我这样活着辜负了王的期待,还有什么意思!''

如懿默然片刻:``是没什么意思了。你自己的心死了,你母族的心也死了。今儿特特来告诉你一件喜事,前些日子,李朝又送了一拨儿年轻的女孩子入宫,想要献给皇上邀宠。这些女孩子该是今年的第几拨儿了?''她倏然一笑,如冰雪艳阳之姿,空中却字字如针,``不过也恭喜你,皇上盛情难却,已经选了一位宋氏为贵人,听说还是李朝世子千挑万选出来的美人,跟选你一样,不几日就要进宫了,有家乡人一起作伴,也不会像如今这般寂寞了。这样千挑万选出来的女子,一定不逊于你当年的容色吧?只是本宫冷眼瞧着,她若是走了你的老路,再花容月貌也没意思。''

玉妍原本静静听着,听到此处,唯见自己胸口剧烈地起伏着,像大海中狂湃的浪涛,骇然起伏:``我知道你们都瞧不起我,瞧不起我四十多了还整日涂脂抹粉,穿红戴绿,不肯服老。瞧不起我拼命献媚,讨好皇上。''玉妍的身体猛地一抖,嗓音愈加凄厉,用力捶着床沿,砰砰道,``可是他们凭什么!凭什么这么厌弃我!我一辈子是为了自己,为了我的儿子,可算起来都是为了李朝。为了我的母族,为了我嫁来这里前世子的殷殷嘱托!从我踏出李朝的疆土那一刻起,我的心从未变过!可我还没死呢,他们倒都当我死了,急吼吼地送了新人来,是怕我连累了他们的荣华富贵么?''

如懿直直地盯着她,一毫也不肯放过,迫近了道:``你的心没变过,你的母族也是!你若有用,自然对你事事上心;一旦无用,就是无人理会的弃子。本宫便再告诉你一句,断了你的痴心妄想。今日皇上那儿己经得了李朝世子的上书,说你并非李朝人氏,而是你金氏家族的正室不知从哪里抱来的野孩儿当自己的女儿甚至说不清你到底是李朝人、汉人还是哪儿来的。所以你根本连李朝人氏都不算,为他们拼上了性命算计旁人做什么?''

\hypertarget{ux7b2cux5341ux4e94ux7ae0-ux60bcux7389}{%
\chapter{第十五章 悼玉}\label{ux7b2cux5341ux4e94ux7ae0-ux60bcux7389}}

有良久的死寂,殿中只闻得涸泽之鱼一般艰难而浑浊的呼吸。有长长的清泪,从玉妍的颊边无声滚落。她痴痴怔怔,似是自问:``世子?世子?不会的,不会是世子!我的世子!''她抓着如懿的手腕,像是害怕极了,轻轻地问:``那,我究竟是什么人?我是哪儿来的?我是不是金玉妍?我是谁?''

如懿撇开她枯枝似的手,淡淡道:``本宫不知。''

玉妍紧紧地搂抱着自己,像是畏冷到了极处,蜷缩着,蜷缩着,只余下灰蒙蒙的床帐上一个孤独的影子。须臾,她仰天怒视,嘶哑的喉咙长啸道:``世子,世子,你为何要这样待我?我尚且未死,你便只当我死了么?''

玉妍低低地吸泣着,那声音却比哭号更撕扯着心肺。如懿抚着自己的肚子,冷笑着摇头道:``世态炎凉,本就如此。本宫不知道临行前你的世子如何对你寄予厚望殷切嘱托,但想来如今也是一样嘱托了宋氏的。你为了这样凉薄的世子和母族赔上了自己的一辈子,真是不值得。说到头,你是为了谁呢?''

玉妍几乎痴癫,眼神疯狂而无力。如懿逼近一些,迫视着她:``本宫今日来告诉你这么多,就是想听你一句实话。本宫的五公主,到底是不是你害死的?''

玉妍乌黑的眼眸如同两丸墨色的石珠,玲玲滚动。她讥笑一声:``你的五公主死了,忻妃的六公主也死了。人人都算到了我头上,我认了。但是皇后娘娘,我活不了多久了,你给我一句实话,我的用璇坠马,是不是你们指使永琪做的?''

如懿的泪一瞬间熨热了眼眶,攥紧了手,硬声道:``没有!这句没有不仅是担保了乌拉那拉如懿,也担保了珂里叶特海兰和爱新觉罗永琪!''

玉妍愣了一愣,倔强地梗着脖子,厉声道:``那么我也没有害你的女儿,害忻妃的女儿!我也发誓,`富贵儿',`富贵儿'咬了你的女儿,惊了忻妃的胎气,绝对不是我指使教唆的!''她的牙齿白森森的,死死咬在暗紫的嘴唇上,咬出一排深深的血印子,目光如锥,一锥子一锥子狠狠扎在如懿身上,``至死我也不明白,我的`富贵儿'怎么会偷偷跑出了启祥宫,又得了咬人的疯犬病,那时我全部心思都在永璇的伤势上,我什么都顾不得了\ldots\ldots{}''

仿佛有巨石投入心湖,巨大而澎湃的波浪激得如懿心口一阵一阵发痛。她的

璟兕,活泼可爱的璟兕,再也不能在她膝下欢笑,一声一声唤她``额娘''了。

良久的静默。喉头的酸涩从心底泛起,通得如懿的声音如同泣血:``不是你?还有谁会恨极了本宫,恨极了本宫的孩子?''

``要害你的人多了去了,谁知道呢?''玉妍的目光胶着在她身上,渐渐失去了灼热的气息,变得冷淡而失落。她疲倦地垂下身子:``可是皇后娘娘,哪怕你起了誓,我还是不相信你,一点儿都不信!不止不信你,我谁都不信。你们都想害我,害我的孩子,如今,我快死了,皇上也不要四阿哥了,总算遂了你们的心愿了''。

如懿的眼眶微微有些泛红,忍耐着性子道:``本宫也不信你。但本宫的心愿,从来不是要害你的孩子!''

玉妍是虚透了的人,脖子上的青筋突兀地梗着,映着枯黄的脸色,恍若一片泥淖中的枯叶:``皇后,你这个人原本和孝贤皇后不一样。孝贤皇后活了一辈子,活得都是虚的。为了一个皇后的虚壳,什么她都藏着掖着忍耐着。难不成做了皇后,一个个都成了供起来的虚菩萨,说的话叫人听着真恶心。''她``嘿''地一笑,瞟着如懿道:``不过呢,原来做了皇后也都是一样的。咱们那皇上的性子,做妃子时个个都无事,嚣张也是直爽的好性儿。可若成了皇后,与他并肩,他却是事事留心,步步猜疑了。所以这个皇后,真是当得好没意思吧?''

如懿静静地注视着她道:``有没有意思,你未曾做皇后一日,就不必替本宫操这份心了。当年你指使着孝贤皇后身边的素心,哄她以为是为孝贤皇后尽心。

借着孝贤皇后的名头做尽了害人的事,是不是?''

玉妍满脸嘲讽地瞟着如懿,拢着自己枯草似的头发,妩媚一笑:``怎么,皇上都疑心素心的死是纯贵妃做的,才连消带打厌弃了她的大阿哥和三阿哥,绝了他们的太子之路,皇后娘娘倒疑心起我来了。''

如懿的面孔阴沉如山雨欲来的天空,``皇上曾经在素心死后查过她家中,可是除了些宫中的银子,实在也看不出什么。既可以是皇后额外赏赐的,也可以疑心是纯贵妃买通的。只是本宫实在不放心,又命人细细去查素心出宫时去过的当铺,才发觉她当过的东西里,有一枚你戴过的镯子。这便无可抵赖了吧?''她凝神须臾,从袖中取出一个小小的纸包,递到玉妍跟前拆开道,``这个东西,你自己总认得清楚吧?''

玉妍眉心剧烈一跳,别过脸道:``你找到这个了?我还当你把什么事都算在了孝贤皇后和慧贤皇贵妃这两个替死鬼头上呢?''

如懿用尖尖的护甲拨弄着纸包里的蛇莓果子和水银朱砂的粉末,随手丢到玉妍身前:``慧贤皇贵妃跟前的双喜会驱蛇,何必还要用蛇莓的汁液在怡嫔宫里引来蝮蛇?连皇上用刑拷打双喜时,他招认的那些事里也真真没有害怡嫔的。本宫也曾以为是孝贤皇后所为,回来想想也有不妥之处。连本宫在冷宫时,孝贤皇后与慧贤皇贵妃指使人用寒凉之物害得本宫与惢心饱受风湿之苦之事,本宫亦察觉,其实孝贤皇后并不懂得食物药性。这么说来,一直传闻的哲悯皇贵妃被孝贤皇后所害之事,便值得商榷了。''如懿眼中的恨意更盛,``直到永璜临死前,本宫才得知,原来告知他哲悯皇贵妃乃是长久服食相克的食物而死,甚至连她素日吃的是什么都说得清清楚楚。那么除了是害死哲悯皇贵妃又嫁祸皇后的那个人,还会有谁?''

玉妍低头思索片刻,苦笑道:``那日是我一时不察失言了,居然被你听出了蛛丝马迹。好,便是这样,那又如何?''

如懿只觉得牙关真真发紧,咬得几乎要碎了一般:``本宫原也想不通你是为了什么,要一个个除去这些人。直到你害得纯贵妃的儿子断了太子之路,本宫便再明白不过了、永璜失了生母,便再也斗不过别的皇子。用璋又被娇生惯养,不得皇上喜欢。而那时你还没有身孕,玫嫔和怡嫔相继失了孩子,所以你的永珹一出生,便是皇上登基后的第一个孩子,得了皇上如此钟爱。''

玉妍不经意地怒了努嘴,拿绢子擦了擦唇边垂落的口涎:``我千里迢迢从李朝而来,虽然得宠,却也不算稳固无虞。孝贤皇后生了嫡子那是没办法,她自己对皇子之事也格外上心,实在无处下手,只得日后再筹谋。何况她虽无意要你性命,但人哪,一旦有了私心,再有在暗处利用的推动,也不难了。你们两虎相争,许多事皇上疑心是她做的,天长日久,总能拉她下来。且她的儿子那么短命,一个个都去了,到省得我的功夫了。这么一来,除去那些想赶在我前头生下孩子的**,永珹便顺理成章得皇上喜欢了。''

``你打的算盘的确是好!慧贤皇贵妃受孝贤皇后的笼络,孝贤皇后却是你的替死鬼,连纯贵妃也是。要不是她们一个个倒下了,你藏了那么久的原形也显不出来。从你布下死局冤枉本宫与安吉波桑大师暖昧之时,本宫便知道,前头的一个个完了,真正害本宫的人就得自己跳出来了。这么说来,孝贤皇后至死不认利用阿箬来害本宫入冷宫之事,想来背后也是你怂恿的了。你自己却明里暗里和阿箬过不去,倒叫人撇清了是你怂恿了素心去找的阿箬吧。你也不必否认,这件事也是后来惢心嫁了人出去,偶尔见到阿箬的阿玛桂铎,才知桂铎竟知道惢心这个人。阿箬发迹与她息息相关,再想起素心与你关系密切,便不难知晓了。''

玉妍安静地听她说着,神色从容而安宁:``你都己经想得那么明白了,还来问我做什么。''她唇边衔着一缕得瑟,``我偏不告诉你,偏不承认。你再疑心,没有我的答案,你心里总是纠缠难受。这样,我最高兴。''

如懿的眼眶被怒火熬得通红,她极力忍耐着道:``你与本宫也算挣了一辈子,彼此也没有过几句好话。或许再说得难听些,本宫厌你恨你也不是一日两日了,但本宫从未想过要你死。''

玉妍瞪着她,讥笑道:``这个自然了。死了多痛快,你自然要看着我不死不活,活着比死了还难受,你才痛快呢!''

如懿含笑:``你倒真聪明,和你说话,痛快!''如懿看着她胸口剧烈地起伏,神色愈加平静,``本宫听太医说,你不肯吃药也不肯医治,今日一来又看见你这副样子,知道你是自暴自弃定了。可是你到底是为人母亲的,不为别人,若叫你的孩子们看见你这个样子,岂不是伤了他们的心?你自然是因为皇上的气话受辱,他们何尝不也承受着同样的屈辱。这个时候,你这个做母亲的不好好宽慰他们,还在这个自己耍性子作践自己,那真当是不自爱了。''

玉妍仰起脸,无神地望着积灰的连珠帐顶,颓然道:``皇后,你也是个母亲。我问问你,如果你和你的孩子都溺到了水里,你是愿意自己沉下去,还是拉了他们一起下去?我现在的处境就是如此。我们李朝王室风雨飘摇,一直依附大清,祈求大清庇佑。我\ldots\ldots{}''玉妍猛然睁大了眼睛,气息急促起来,``我一辈子都是李朝的荣耀,可是到头来,却成了李朝的耻辱!他们想要像甩了破鞋似的甩了我,他们!他们!''她不知想到什么,眼神忽地一跳,``世子一定是对我死心了,才会故意撇清的,一定是!不!我不!我不!世子,不要对我死心!我还活着,我还有我的孩子,我是李朝人,我是!我是\ldots\ldots{}''她话未说完,忽然一口痰涌了上来,两眼发直,双手抓向虚空处,直直向后倒去。

如懿见状,也不觉吃了一惊,忙道:``容佩,赶紧扶贵妃躺下。''

容佩见玉妍被褥油腻发黑,一时有些不敢下手。如懿蛾眉一蹙,也顾不得自己挺着肚子,伸手按了玉妍躺下,又取过一个软枕替她垫着。容佩急忙去倒茶水,结果发现桌上连一应的茶具都脏乎乎的,茶壶里更没有半滴水,不觉含怒道:``在外头能喘气的人,赶紧送水来!''

容佩一声喝,立马有宫人伺候了洁净的茶水进来,又赶紧低眉顺眼退出去了。容佩倒了一盏,发现也是普通的茶水,一时也计较不得什么,赶紧送到玉妍唇边。玉妍连着喝了两杯,才稍稍缓过气来。

玉妍躺在枕上,仰着脸像是瞪着不知名的遥远处,慢慢摇头道:``不中用了,我自己知道自己,要强的心太过,如今竟是不能了。早知道自己不过是个贡品,不过是被人随时可以甩去的一件破衣裳,一双烂鞋子,当年何必要这般和你争皇后之位,这么拼了命生育皇子。这么费尽心机,到头来不过是连累了无辜的孩子,都是一场空罢了。皇上\ldots\ldots 我也算是看透了,虚情假意了一辈子,总以为还有些真心,临了不过是如此\ldots\ldots{}''她长叹一声,忽然挣扎着揪过自己披散的长发。大概久未梳洗,她的一头青丝如干蓬的秋草,她浑然不觉,只是哆嗦着手吃力地编着辫子,慢慢笑出声来,。当年,我的头发那么黑,那么亮,那么好看。我在李朝,虽然是个小小的宗室之女,可是我那么年轻,什么都可以期盼,什么都可以从头来过。我可以嫁入王宫,成为世子的嫔妃,守着他那么温柔的笑容过一辈子。算了,那样的话和这里也都一样,还是得不明不白地争一辈子。可是,可是他们都不要我了,他们连李朝人都不让我做,让我死了都是一个无依无靠的孤魂野鬼。如果可以从头来过,我要选一个心爱的阿里郎,一辈子不用争不用抢,一定是家中地位最尊崇的正妻,得到丈夫的关爱和尊重。我可以生好多好多的孩子,新年的时候,和他们一起打年糕、跳春舞。我\ldots\ldots 我\ldots\ldots''

玉妍忽然说不下去了,喉头如硬住了一般,僵直地喘着气,眼角慢慢淌下两滴浑浊的泪,脸上却带着希冀。憧憬的笑,仿佛有无尽的满足,只沉浸在自己的世界里。

如懿的心一下空落落的,恨了那么久,到了生命的最终,看若她行将死去,居然不是快乐,而是无限心酸。她悄悄地扶起容佩的手,慢慢踱到门外。

外头的雪光太过明亮,亮得如懿几乎睁不开眼睛。有一瞬间的刺痛,不知为何,她竟然感觉眼中有汹涌的泪意即将决堤而出。忍了又忍,睁开眼时,如懿宛如平日一般端庄肃然。她看着满院子伺候的宫人,只留下一句话:``好好伺候嘉贵妃,务必尽心尽力送她终老。''

她的语落轻声,如细雪四散。有幽幽漫漫的昆曲声爬过宫境重苑,仿佛是嬿婉的歌声,清绵而不知疲倦,伴随着纷飞如樱翩落的雪花点点,拉长了庭院深深中梨花锁闭的哀怨。

``寒风料峭透冰绡,香炉懒去烧。血痕一缕在眉梢,胭脂红让娇。孤影怯,弱魂飘,春丝命一条。满楼箱霜月夜迢迢,天明恨不消。''

如懿隐约记得,那是《桃花扇》中李香君的唱词。冻云残雪阻长桥,闭红楼冶游人少。栏杆低雁字,帘幕挂冰条:炭冷香消,人瘦晚风峭。那些曾经花月正春风的人呵,从今都罢却了。

回到宫中,如懿也只是默默地。皇帝照例过来陪她用膳。彼此说了些后宫的事,却没有提起玉妍,好像完全不知道她重病似的。如懿便索性提了一句:``今日上午,臣妾去看过嘉贵妃了。''

皇帝淡淡地``哦''了一声,并无半分在意之色,只是温然叮嘱:``如懿,你临盆之期将近,怀的又是钦天监所言的祥瑞之胎。咱们的永璂己经十分聪明可爱,你这一胎钦天监又极言显贵,这个孩子来日必成大器,所以这些不干净的地方,你便不要再去了。''

如懿低下温婉的侧脸,支着腰身道:``臣妾明白。但嘉贵妃眼看着快不行了,臣妾是皇后,于情于理都该去看一眼。''她的眉梢染上郁郁的墨色,``何况人之将死,许多话,臣妾不去问个明白,也实在难以安心。''

有须臾的静默,只听得皇帝的呼吸变得滞缓而悠长,不过很快,他只是如常道:``她肯说么?''

如懿咬着唇微微摆首:``她有她的恨,她的怨,却至死不肯言明。''她深吸一口气,将胸腔里翻腾的怨恨死死按压下去,``嘉贵妃说,她便知道,也不会说,不会认,由得臣妾夜夜悬心,不得好过!''

他冷笑,微薄的唇角一勾,目光里有灼热得通人的厌弃:``她若说了。岂不是连累了她最牵念的母族李朝?''他将手中银筷重重一搁,上头坠着的细银链子发出抖动的栗栗声,``今儿午后看折子,还有一件更可笑的事呢。李朝上书来说。查知金玉妍确是抱养来的女儿。李朝嫡庶分明更甚于我朝,庶出之子尚且沦为仆婢,何况是不知何处抱来的野种?抱养金玉妍的夫妇二人,已被李朝君主流放。又说金玉妍不知血缘何处,连是否是李朝人氏也难分辨,只得叩请我大清上邦裁决。''

皇帝说得如同玩笑一般,如懿本该是解恨的,更应快意畅然,可字字落在耳中,她只觉得如重锤敲落,心中霎时凛然。明明是暖如三春的内殿,穿着华衣重重,背脊却一阵阵发凉,又通出薄薄的汗。

凉薄如此!原来所谓博弈权术,她,或是拼上整个后宫女子的心术权谋。都不及那些人的万分之一!

金玉妍固然有错,但她拼尽一生,不过是为了母族之荣,却到头来,只是一枚无用的弃子,被人轻易抛弃,抛得那样彻底,再无翻身之机。

原来她们的一生,再姹紫嫣红,占尽春色,却也逃不过落红凋零、碾身尘泥的命数。

还是皇帝的声音唤回如懿的魂灵所在:``这件事,皇后怎么看?''殿中光影幽幽,皇帝缓缓摩挲着大拇指上的绿玉髓赤金扳指,``皇后若觉得金氏之事李朝有脱不清的干系,那朕一定会好好问责,以求还皇后一个明白。''

如懿极力自持,凝眸处,分明是他极为认真的神色,可那认真里,却总有着她难以探及、不能碰触的意味。

若真要给她一个分明,何必要问,自然迫不及待去做。若要来问,本是存了犹疑,存了不愿探知之心。

她目光中有一瞬微冷的光,唇边的笑意越见越深沉:``嘉贵妃落得这般地步,李朝自然恨不得撇得干净,又送来佳丽新人示好。但嘉贵妃一生所为只有李朝,若说没有李朝的悉心调教,也不至于此。''她停下,分明见到皇帝的瞳孔微微紧缩。

她在心底里苦涩地笑,唇间却换了更婉转的语调:``只是嘉贵妃血缘并非李朝,又身在大清,李朝即便想主使,也做不得什么。且李朝自归属大清,一向敬服上邦。若为区区一女子而兴师问罪,也有失我大国气度。且嘉贵妃并非李朝人氏,混淆血统入宫为妃之事若传扬出去,庶民无知,还不知要如何揣测,多生妄语。''

皇帝的眼睛有些眯着,目光在柔丽日色的映照下,含了朦胧而闪烁的笑意。他将她的手合在掌心,动情道:``皇后能放下一己情怀,以朕的江山安稳为重,朕心甚是安慰。''

她低着头,依偎在他身侧,感受着他的掌心握住自己手指的温度。分不清,究竟是他的掌心更凉,还是自己的肌肤更凉。也许只是天气的缘故,他和她的手是一般凉。有那么一瞬,她的心底是难以摒去的绝望,抑也抑不住似的,横冲直撞地漫溢出来。即便是这般肌肤相亲,有着血脉相连的结合,原来也是咫尺天涯,迈不过那一步的距离。

窗外一枝红梅旖旎怒放,皇帝凝眸片刻,眸中如同冰封的湖面,除了彻骨寒意,不见一丝动容之色:``生生死死,花开花落,皆是命数。她心性狠毒,害死了朕的璟兕和六公主。想来老天也不会庇佑!''皇帝停一停,慢慢啜着一碗野鸡崽子酸笋汤,不疾不徐道,``若嘉贵妃真不行了,便叫内务府预备着后事吧。别一时间乱起来,没个着落。''

如懿便也仿若无事一般:``嘉贵妃的后事臣妾可以吩咐内务府去办。左右外头不知道嘉贵妃所作所为,后事必得顾及颜面,还是得给她死后哀荣,别叫旁人生了无谓的揣测。''

皇帝的眉宇间有淡淡的阴翳:``你怀着身孕,别沾染越不相干的悖晦事。等朕有了打算,交给纯贵妃和愉妃料理便是。''

如懿凝神,笑得一脸婉顺,道:``皇上,嘉贵妃这样病着,她的两位阿哥总养在阿哥所也不成事,总得托了人照管才好。尤其永璇,腿上落了伤,嬷嬷们再细心,怕也照顾得不够周全。''

皇帝随口道:``永珹那个不孝子己经出去了,永璇腿脚不便,永瑆年幼,是该有个养母照顾便好。皇后的意思是\ldots\ldots{}''

如懿道:``阿哥所的事一直是婉嫔帮忙料理着,婉嫔年长无子,人也细心温顺,交由她照顾也是好的。再者\ldots\ldots{}''

皇帝点头道:''也好。他们的生母阴毒不训,养母是得格外安分的才好。婉嫔虽好,到底还是在这后宫里。朕的意思,是想交由寿康宫的太妃们抚养,让永瑆每日聆听佛音禅语,也好修个好心性。''

皇帝这般说,自然是不欲在宫中时常见到永瑆,才挪去了素日不必相见的太妃们那里。如懿心知皇帝对金玉妍是厌恶到了极处,也不便反驳,只道了会去安排。零星又说了几句皇子们读书的事,皇帝便回了养心殿处理政务。如懿月份渐大,起坐极不方便,便只送了皇帝到殿门口。因着家常,如懿只披了件雍紫毛边

的银狐琵琶襟**,皇帝含笑替她紧了紧微松的领口,温言道:``今夜是十五月圆之夜,朕会再过来陪你,也陪陪咱们的孩子。''

这顿饭如懿无甚胃口,用完了膳慢慢吸着茶水看着宫女们收拾膳食。

容佩见收拾的宫人们都出去了,方才道:``活该!皇上就早该这么不待嘉贵妃了,也省得她一副狐媚狠毒的心肠。奴牌看了心里真痛快!''

如懿衔了一丝快意:``待见不待见,原本就在皇上一念之间。''她怔了怔,赤金护甲敲在紫铜手炉上叮当作响,``容佩,本宫会不会也有那一天呢?''

说完,连她自己也吓了一大跳。容佩脸都白了,慌忙道:``娘娘,您说什么哪?您是皇后,怎么会和她们一样!''

如懿有些惘然,心下迷迷瞪瞪的,脱口道:``皇后也是女人,也不过是皇上的女人之一。今日待见的,或许也有不待见的一日。''

容佩吓得赶紧捂住她的嘴,急得赤眉白眼道:``皇后娘娘,您是最尊贵的女人,不兴这样胡说的。''

如懿黯然片刻,静静地望着窗外突然乌沉的天空:``天暗下来了呢。''

铅云低垂,暗暗压城,有簌簌的响声扑扑打在檐上。容佩望了几眼,便道:``娘娘,是下小雪了呢。''

如懿这才觉得有些寒意,微微瑟缩着道:``是啊!十一月里了,是该下雪了。''

容佩便道:``奴婢去替娘娘换个新手炉暖暖,再加两个炭盆进来。''

如懿点点头,听着外头的雪声沙沙,心里牵挂不已:``你去阅是楼看看,永琪在读书么?若是在,让人给他添些冬衣和手炉。永琪只顾着读书,不在这些事上留心,伺候的奴才怕是有不周到的。''

容佩答应着去了。如懿坐在那里,只觉得周身寒浸浸的,便着几个小宫女伺候着午睡了。

\hypertarget{ux7b2cux5341ux516dux7ae0-ux6dd1ux5609}{%
\chapter{第十六章 淑嘉}\label{ux7b2cux5341ux516dux7ae0-ux6dd1ux5609}}

金玉妍是在当天傍晚过世的。下着小雪的冬夜,宫人们自然疏懒了许多。到了夜间时分,伺候玉妍的宫人们才发现她早己没有了气息,像一脉薄脆枯叶,被细雪无声掩埋。

似乎是预知到了死神的来临,玉妍难得地穿截整齐了,梳洗得十分清爽干净,还薄薄地施了脂粉,犹如往常般明媚娇艳。她换了一身李朝家乡的衣装,玫红色绣花短上衣,粉红光绸下裙,梳了整整齐齐的一根大辫子,饰以金箔宝珞,一如她数十年前初入王府为侍妾的那一日。

伺候她最久的丽心来如懿宫中报丧,哭泣着道:``晌午过后,贵妃小主就命奴婢替她梳洗。奴婢还以为小主是听了皇后娘娘的劝,终于想开了。谁知道梳洗完了小主说要一个人静一静,到了傍晚咱们送晚膳进去时,才发现小主已经没气了。''

此时,如懿正在卸妆等着皇帝过来,听得这个消息,神色平静,波澜不兴。有快意的痛楚犀利的划过心间,半晌,她才缓缓问道:``嘉贵妃去的时候可安静么?''

丽心伤心道:``很安静,如同睡去了一般,脸上还带着笑。''

如懿静了片刻,轻轻摆手:``去禀告皇上吧。好好说,就说嘉贵妃去得安乐。''

丽心哭着退下了。如懿缓步走到窗前,外头积了一地的雪水,还不如下得大些,白白的,一片干净。如今望去,只觉得湿流滚水汪汪的,很是粘腻汪荡,不尴不尬。就如同玉妍锦绣的一生,最后还是落了这样一个不尴不尬的结局。

次日是十一月十六,老天爷停了雪,却是淅淅沥沥地下起雨来。这样的寒冷天气,下雨更麻烦过下雪,愈加让人心情抑郁。苏绿筠、嬿婉和海兰等几个高位的嫔妃们先赶到了皇后宫中问安。皇帝与如懿并肩坐着,两人都是郁郁不乐的样子。殡妃们自然前晚就得到了金玉妍离世的消息,虽然金玉妍在宫中人缘极差,并无人喜欢她,但嫔妃们见了面总难免唏嘘几句,又着意宽慰了帝后一番,言语间尽是姐妹情深。

嬿婉微微红了眼眶:``一早起来便看见下着雨,怕是老天爷也在痛心嘉贵妃骤然离世,和咱们一样呢。''说着她正要哭出声,如懿淡淡道:``眼下也没什么好哭的,替嘉贵妃守灵的时候,有你们掉眼泪的。''

海兰捻着蜜蜡佛珠念了几句``阿弥陀佛'',只是静静垂首。绿筠便叹道:``嘉贵妃也是错了主意,折腾自己也折腾孩子。若是安安分分的,也不至于折了自己的福气,落得这样的下场。只是如今就这么走了,听说梓宫已停在了静安庄。''

如懿便转头向皇帝道:``嘉贵妃虽然在世的时候不安分,但就骤然这么去了,身后的事,总要办得好看些。不为别的,只为她在宫里的位分和诞育的子嗣。''

皇帝点点头,众人才看清皇帝的眼下乌青了一片,想是昨夜也没睡好。嬿婉柔声劝道:``皇上为嘉贵妃姐姐伤心,昨夜肯定是没有睡好了。臣妾命人炖了一盅参汤带过来,皇上好歹提提神吧。''

海兰默默看她一眼,叹道:``到底是令妃细心,来皇后娘娘宫中,还记得带了参汤给皇上。''

嬿婉温婉道:``昨夜是十五之夜,皇上必定在皇后宫中。也是妹妹的一点儿心意,若是多余,还请愉妃姐姐指教。''

绿筠拿绢子拭了拭眼角,慢条斯理道:``令妃妹妹的心意怎么会是多余?只不过是在皇后宫中,有什么皇后都会照顾周全,哪里缺令妃你一碗参汤了,你还是顾着自己的身孕要紧。''

绿筠积年的资历在,说话自然有分量。嬿婉诚惶诚恐起身道:``皇后娘娘恕

罪,臣妾无心之失,但请娘娘宽恕。''

如懿心下不耐烦,口气淡淡道:``嘉贵妃刚离世,还在天上睁着眼睛看着呢。你们过来若是好好商量嘉贵妃的丧仪,那便还是一场姐妹情分。如果这时候还要拈酸吃醋的,本宫怕嘉贵妃在天之灵不安,皇上的心里也跟着不安。''

这话说得有些重,连绿筠也微微变色,忙领着嬿婉跪下。

皇帝不耐道:``都起来吧。''说罢转头向如懿:``朕的意思,嘉贵妃伺候朕二十多年,又诞育过四位皇子,可谓尽心尽力。朕一早下朝后去见过太后,太后也很是伤怀,下旨追封嘉贵妃为皇贵妃。''

如鼓听闻,领着众人行礼如仪:``臣妾替皇贵妃谢过皇上,谢过太后。''

皇帝点点头:``朕己经命内务府拟了谧号来看,最后选了一个`淑'字,就追封为淑嘉皇贵妃。''

如懿心头冷笑,好一个``淑''字!好讽刺的``淑''字!他竟也是那般嫌弃她,嫌弃到要拿她的身后来做个笑话。如懿这般想着,与海兰目光相接之时,只见她瞬即将眼中的鄙夷之色敛了,换将一副哀戚之色。

嬿婉极力忍着笑意,含泪戚戚,偏要再追一句:``皇贵妃姐姐一生贤淑,皇上选的这个谥号是再贴切不过了。''

如懿心念一动,婉声道:``淑嘉皇贵妃在世的时候,最疼爱几位皇子,但永珹已经成年,又出嗣履亲王,永璇和永瑆虽然年幼,倒也都是懂事的孩子。皇上不若也给他们一些恩典,也叫没娘的孩子自己能顾全自己些。''

皇上眼皮一跳,握一握如懿的手,温然道:``还是皇后想得周到。永珹出继,已经是贝勒。永璇和永瑆,朕也会给他们贝子的爵位,且有太妃们照顾,一切无碍。''

如懿听皇帝言下之意,知道是将几个小阿哥托付给了她,便起身正色道:``太妃,们久在宫中,熟知礼仪,一定会好好教导皇子。臣妾身为嫡母,也一定会从旁看顾。''

皇帝神色微微一松,微露几分倦色:``有皇后这句话,朕也放心了。''

海兰轻声道:``皇上,淑嘉皇贵妃虽然过世,可李朝新送来的贵人宋氏不日就要入宫了,臣妾奉旨协理六宫,多嘴问一句,宋贵人安置何处?''

皇帝随口道:``朕收下宋贵人只为情面,也不想看见她再想起嘉贵妃。送宋贵人去圆明园居住吧。

如懿心头一怔,人才刚走,茶却已经凉透了。然而也好,她心中冷然而快意,抚着肚子寻思,害了她女儿的人,只能是这样的下场!

海兰恭声答应了,皇帝回头看顾嬿婉:``令妃,肤有些累了,去你宫中歇息。''嬿婉连忙答了句``是''。皇帝又道:``皇后和令妃都有着身孕,不必去淑嘉皇贵妃的丧仪了,叫纯贵妃和愉妃帮衬着料理吧。''

二人依依谢过。如懿欠身将要相送,忽然念及金玉妍临死前的话,不觉一凛,若诚如她所言,她并未真心要害璟兕和六公主,那么会是谁?还会有谁?

这样的念头不过一转,全身已经寒透彻骨。她不敢去细想,只得将骤然而生的一缕怜悯之情缓缓吐出:``皇上,淑嘉皇贵妃是李朝王室宗女,如今骤然离世,皇上将追封皇贵妃的恩典和加封皇子的消息传到李朝,也算了了淑嘉皇贵妃一桩心愿,赏她荣耀。''

皇帝本往殿门外走了几步,听如懿这般请求,不觉停住脚步。嬿婉见机赶紧扶住皇帝的手,柔声道:``皇后娘娘这么顾全皇贵妃,皇上也请体念娘娘的一番心意吧。''

皇帝转过头打量了如懿两眼,微微颔首道:``既然皇后这么有心,朕怎能不成全。朕最后能为淑嘉皇贵妃做的事,一定会做,免得旁人落了口舌,说朕是凉薄之人。''

皇帝离去后,如懿打发了绿筠去办玉妍的后事,只留下海兰在身边陪着。两人进了暖阁,容佩送了茶点上来,便领着人退下了。

海兰亲自将茶盏递到如鼓面前,温声道:``皇后娘娘。今日的事,皇上显然原本只是想追封而己,您请了那两个恩典,皇上怕是不高兴了。''她的疑惑更深,``娘娘一向深恶金玉妍,怎的今日还要为她求情,保全她死后最后的一点颜面?''

如懿扶着微痛的额头,喝了一口热茶,才觉得心口暖和了一点儿:``本宫何尝不知这个?金玉妍身死,给得再多也只是身后的虚名,本宫是是怕皇上背了凉薄的恶名啊。何况\ldots\ldots{}''她勾起一抹冷笑,``三宝已经查知,送去静安庄梓宫里的,根本不是金玉妍!''

海兰惊得睁大了眼:``是谁?''

如懿抚着额头,打量尾指上套的金护甲上嵌着冰色缠绿丝的翡翠珠子,闲闲

道:``在圆明园伺候过皇上的一个官女子上个月殁了,本是停了棺椁要送进妃陵里的,如今和金玉妍换了个个儿。''

海兰骇笑:``那倒是个有福气的!从此身受香火,便是皇贵妃的哀荣了。''

如懿衔着一丝快意,然而涌到唇边的叹息如伶仃的雾水:``金玉妍临死也绝不承认蓄意用`富贵儿'害了本宫的璟兕!人之将死,其言也善。若她说的是真的\ldots\ldots{}''

海兰骤然一凛,眼中有锋芒聚起:``若不是她,还能有谁?'她眸中的锋芒仿若锐利的银针,闪着尖锐的寒光,``是令妃,是庆贵人,是晋贵人,还有谁?。''

如懿的唇边含了一丝犹疑:``若是我们错了\ldots\ldots 若是这件事,从永璇坠马开始就是被人算计在内的,连着金玉妍,连着本宫和忻妃,一个也不落下\ldots\ldots{}''她的脸色越来越难看,几欲破裂,``那么这个人的心思,实在是阴毒可怕!''

海兰见如懿呼吸越来越急促,忙劝道:``娘娘别多想,更别怜悯了金玉妍枉死。说句不入耳的,她算不得枉死!争了一辈子,算计了一辈子,处处与娘娘为敌,何况五公主和六公主早夭,到底是和她脱不了干系!所以,死了也不算冤!''

如懿的神思似乎有些飘远:``当日金玉妍发疯一般要本宫与你发誓,有没有害过她的孩子。其实撇开了永璇坠马之事不算,咱们是算计过永珹的。''

海兰定定神,镇静道:``娘娘,臣妾己经发过誓了。哪怕要应誓,也只应在臣妾一人身上,与娘娘无关!''她爱惜地抚着如懿硕大浑圈的肚子,``娘娘快要生了,钦天监都说怀的是个祥瑞的孩子,娘娘不要去想这些不吉利的事了。''

如懿默然片刻,缓缓道:``死了一个金玉妍,的确不算冤!金玉妍年轻时就是这样的性子,争强好胜,什么都要拔尖。这个性子,放在年轻的时候看着还泼辣可爱,如今人到中年,还是这样的性子,难免显得尖酸。还常常为了些许小事和旁人闹不痛快,惹得人人讨嫌。''

海兰取了一块梅花糕片放在嘴里尝了尝,低声道:``这也罢了。说到底,皇上是不痛快前朝立太子的事。金玉妍一辈子想给她儿子争上太子之位,却也死在了这个上头。''

如懿轻嘘一口气,慢慢吸了茶水道:``皇上年富力强,春秋鼎盛,前朝提这样的话,不是自己打嘴么?皇上对后宫一向宽厚,可是这点儿皇位的心思,却用得极重。咱们都有儿子,以后更要加倍小心,别落了口舌。''

海兰望着如懿,信任地点点头。两人看着窗外细雨纷飞,一时两下无言,便也默默了。

皇帝在嬿婉宫中睡了一会儿,醒来已是两个时辰后了。嬿婉早已换过一身家常的湖水蓝绣银线丹桂的锦袍,松松绾杌了一个弯月髻,见皇帝醒了,不由自主便含了几分甜笑,伺候着皇帝在榻上躺着,把新笼的一个暖炉放进锦被里。自己搬了个小杌子坐在近处,慢慢剥了红橘喂到皇帝嘴边。

皇帝握一握她的手,笑道:``手冰凉冰凉的,躺上来肤替你焐一焐。''

嬿婉含羞一笑,恰如春花始绽,盈盈满满:``皇上爱怜,臣妾谢过。''她低首摸着尚且扁平的小腹,笑道:``臣妾也想躺着呢,只是腹中的小阿哥不愿意臣妾躺下来,只愿意臣妾坐着。''

皇帝一笑:``孩子的话是该比肤的话要紧。''皇帝接过她递来的橘子,还送到她唇边,``你有孕之后爱吃酸甜的,多吃些吧。''

嬿婉吃了两瓣,笑吟吟道:``酸酸甜甜的,很是落胃呢。''

皇帝摸一摸她的小腹,笑道:``你喜欢就好。只是才两个月,哪里就知道是阿哥了。''他的笑意顿敛,有些伤感,``自从朕的五公主和六公主夭折,朕一直希望能再添个公主便好了。''

嬿婉微微一怔,旋即盈然笑道:``小阿哥小公主都是好的。只是皇上不是说臣妾爱吃酸甜么。酸儿辣女,怕这一胎若是个阿哥,皇上可得答允臣妾,再给臣妾一个公主。有子有女,才算一个好字。''

皇帝笑着抚一抚的脸,爱怜道:``这有什么难的,朕答允你就是。''

嬿婉伏在皇帝胸前,乖顺得如一只依傍着暖炉的猫咪,蜷缩着身体,柔声道:``皇上何不多歇一歇,可是惦记着淑嘉皇贵妃的身后事么?皇上真是情深义重,所以斯娘娘也和皇上一般顾念淑嘉皇贵妃在世时的情谊呢。''

皇帝眼中微含几分笑意,伸手托住嬿婉小巧的下颌:``你也觉得皇后很好?''

嬿婉的神色柔顺得如一匹软滑的丝缎:``可不是?皇后娘娘恩泽六宫,淑嘉皇贵妃在世的时候虽然对皇后屡有不敬,没想到皇后还是以德报怨,为了淑嘉皇贵妃的丧仪好看些,屡屡求皇上恩典。''

皇帝直视着她,慢慢道:``这样不好么?''

炭盆里的银霜炭``哔啵哔啵''地响着,冒着温暖的火星。嬿婉顺手将橘子皮扔进炭盆里,散出一阵暖暖的甘香。嬿婉看皇帝的神色极为温和,眼中便有了无限的柔情与温顺:``对皇后娘娘,自然是好的。可是臣妾想,夫妻之道,贵在尊重夫君。君臣之道,贵在尊崇君主。其实给淑嘉皇贵妃的阿哥们恩典,把哀荣传到李朝,这些不必皇后娘娘说,皇上看着与淑嘉皇贵妃恩义的分儿上,也会一一赏赐。可是皇后娘娘是思虑周全了,岂不显得皇上恩典寡薄,让人非议。''

皇帝松开握着她手腕的手,眼神瞬间冷了下来,道:``皇后是六宫之主,你的话也不算太错。这样吧,朕带你去皇后跟前,把你这些话亲口跟皇后说说,好叫她有则改之,无则加勉。''

嬿婉眼神一怯,脸色微微有些发白:``皇上\ldots\ldots 臣妾是为了皇上思虑\ldots\ldots{}''

``为了联?为了联就可以肆意刻薄皇后?''皇帝坐起身,冷冷道,``你刚才和朕说夫妻君臣,可见你是懂得尊卑的。既懂尊卑,皇后有什么不是,你大可当着她的面说。在她面前只说贤惠,到了朕跟前就说皇后的不是。那么朕要看看你这条舌头到底是怎么长的?''

嬿婉情知不好,立刻跪下,哀哀求道:``皇上,臣妾不敢非议皇后,只是一切为皇上着想。臣妾自知人微言轻,有所谏言皇后也未必肯听,只当皇上是臣妾枕边夫君,才畅所欲言,无所顾忌。臣妾不是有心诋毁皇后,还请皇上明察。''

嬿婉一张清水芙蓉脸,一向最适合楚楚可怜的神情,如今苍白着脸哀哀相告,皇帝也不免有些心软,便道:``好了,有着身孕别动不动就跪,起来说话吧。''

嬿婉这才敢起身,倚在皇帝腿边,如受了惊的黄鹂,楚楚道:``臣妾有口无心,是想到什么说什么罢了,并不是有心议论谁的,还请皇上宽恕。''

皇帝的神气有些懒懒的:``嬿婉,知道联为什么喜欢你唱昆曲么?''嬿婉怯生生地摇头,一张脸如春花含露,皇帝的口气不觉软了几分,``昆曲柔婉,最适合你不过。而皇后就像戈阳腔,有些刚气,不够婉媚。''

嬿婉抬着娇怯得能滴出水来的眼眸:``那皇上喜欢什么?''

皇帝的笑意淡薄如云岫:``各有千秋,朕都喜欢。所以嬿婉,别丢了你柔婉的好性子。''皇帝说罢,便抬了抬腿,嬿婉即刻会意,替皇帝套上了江牙海纹靴子。皇帝起身道:``你是不是有心,朕心里有数。好了,联去看看淑嘉皇贵妃的丧仪。''

嬿婉一惊,忙含笑扯住皇帝的衣袖道,``皇上,您方才说要在这儿用午膳的。午膳已经备下了,您用了再走吧?''

皇帝朝外扬声唤了一句``李玉'',头也不回地出去,口中道:``皇后即将临盆,腹中所怀乃是祥瑞之子,联得去陪陪她,你自己慢慢吃吧。''

嬿婉无可奈何地屈身福了一福,恭送皇帝出去。

皇帝走得远了,守门的小太监赶紧将团福弹花赤色锦帘放了下来。一阵寒气还是卷了进来,嬿婉仿若受不住冷似的,不自觉便打了一个寒战。伺候她的贴身宫女春蝉从外头进来,一眼瞧见了,赶紧递了一盏热热的红枣燕窝汤到嬿婉手里,又朝外头使了个眼色,让伺候的人都退了下去。

嬿婉捧着红枣燕窝汤,氤氲的热气扑上脸来,又暖又湿。她出神片刻,缓了口气,问道:``春蝉,皇上出去的时候,是什么脸色?''

春掸低着头道:``小主宽心,没什么脸色。''她停一停,``小主可是惹皇上生气了?''

嬿婉轻叹一声,怔忡半日,缓缓道:``若说生气,也算不上。若说不是生气,总之也是不高兴了。左右本宫如今有孕,皇上是不会不来的。''

春婵沉吟道:``小主还是说了皇后的不是?''

嬿婉沉思片刻,搁下手中的红枣燕窝汤,拨着护甲上晶莹璀璨的珍珠粒,慢慢道:``方才皇上睡着的时候,本宫就这么和你商量了。你的意思也是让本宫说。''

春婵含了一缕笑意,端过嬿婉手边的红枣汤,问道:``这红枣燕窝汤是小主喜欢的,小主怎么不喝了?''

嬿婉细细的眉毛微微蹙起:``烫得慌。等等再喝。''

春婵贴心地端着吹了又吹,才递到嬿婉手里:``有些话自然是要说的。就好比您刚喝上嘴的汤总嫌烫,您耐着性子慢慢吹了再喝,一口比一口能下嘴,一口比一口暖您的心窝。等不烫嘴了,就是贴心的好东西了。''

嬿婉看了春婵一眼,慢慢地小口啜着汤水,忽然会心一笑:``是啊,喝着喝着,一个味道喝多了便惯了,不仅不烫嘴,还又香又甜呢。''

春婵目光一闪,笑道:``可不是?皇上现如今宠爱皇后,不过是因为钦天监说皇后所怀之子何等祥瑞有福。''

嬿婉将一碗燕窝汤喝完,掰着指头算了算日子:``皇后快临盆了吧?你去,请田嬷嬷来说说话,本宫有着身孕,迟早也得学一学这些女人生产的经验。''

春婵微微迟疑:``小主,如今田嬷嬷不大肯来咱们这儿呢。她唯一的宝贝儿子田俊又在京中捐了个九品修武校尉的官职,有了前程,如今她也算享清福了。''

嬿婉``咯''的一声轻笑,嫣然百媚:``是么?当了官儿是有了前程。只是啊,官场上的尔虞我诈不比后宫里浅半分,他们母子也得谨慎谨慎才好啊!要不然都跟淑嘉皇贵妃似的,最后也不过落成个输家而己!''

\hypertarget{ux7b2cux5341ux4e03ux7ae0-ux7ed5ux9888}{%
\chapter{第十七章 绕颈}\label{ux7b2cux5341ux4e03ux7ae0-ux7ed5ux9888}}

如懿的生产是在十二月二十一日的丑时一刻开始发作的.与往常不同,除了接生的嬷嬷和太医伴随在侧,连钦天监的监正与监副也守在偏殿,候着星象所昭示的祥瑞之胎的诞临。

冬夜深寒,皇帝坐在偏殿,听着如懿痛楚的呻吟声,连连搓手不己,急道:``朕不便进产房,你去唤个嬷嬷来问问,是什么缘故,怎么还没动静?''

海兰一脸焦灼,一时按捺不住,陪着皇帝道:``皇上,要不臣妾进去瞧瞧?''

皇帝的口吻不安且不耐,道:``这话你方才就问过,接生嬷嬷们说孩子的胎位不大好,不容易生,其他并无大碍.人多反而手杂,朕才不让你进去的。''

李玉看出皇帝的焦急与担心,忙劝道:``皇上安心,皇后娘娘己经生产过两次,这次不会有碍,一定会顺顺利利生下一个小阿哥的。''

钦天监监正忙赔笑道:``李公公所言甚是。皇后娘娘胎气发动的时候也是个上上吉时呢。微臣已经算过,只要在日中前后出生,那么皇后娘娘这一胎无论男女,一定贵不可言。''

皇帝长嘘一口气,稍稍轻松几分:``若是公主便罢,朕便立即封为固伦公主.若是皇子,朕连名字都想好了,便叫永璟,取玉之华彩之意.''

钦天监监正连连道,``璟,玉光彩也。皇子行永字辈,公主行璟字辈,皇上取此名,可见重视。且皇后娘娘怀上此胎之时,紫微星华光闪耀,皇上取此佳名,真是最合适不过了。''

天色将明时分,如懿的呻吟声随着一声痛厉的呼叫戛然而止.皇帝有过几多子女,听到这一声痛呼,便知是要生了。然而期待中的儿啼声并未响起,只是一片难堪的静默。

监正听得声音征了怔:``这是生了么?这么快?可还没到日中时分啊!''

李玉伸长了脖子向外探去,轻声道:``听这声音像是生了呀?怎么还没儿啼声呢?''

他的话音未落,隐约有几声惊惶的低呼响起,海兰心里微微一沉,不知怎的,便觉得周身寒浸浸的,像是外头的寒气透骨通进.可是殿内,分明是红恰箩炭烧得滚热,入置三春啊!

偏殿的门骤然被推开,接生的嬷嬷和太医们跌跌撞撞进来,哭丧着脸道:``皇上恕罪!皇上恕罪啊!''

皇帝的脸色倏然如寒霜冻结,厉声道:``怎么了?是不是皇后不好?''

为首的正是田嬷嬷,她吓得瑟瑟发抖,回禀道:``回皇上的话.皇后娘娘产下了一个小阿哥。''皇帝神色一松,尚来不及迸出一个笑容,田嬷嬷又道:``可是小阿哥才离了娘胎,就没了气息,已经离世了。''

皇帝大惊之下踉跄几步,跌坐在紫檀座椅之中.海兰急得脸色大变,顿足道:``那皇后娘娘呢?皇后娘娘如何?''

江与彬跪在地上道:``皇后娘娘因为生产时用力过度,气竭昏厥。微臣已经给娘娘服下山参汤,静养片刻就会好的。''

皇帝的声音有些发颤,目光在殿中搜寻不断:``小阿哥,朕的小阿哥呢?''

菱枝抱了一个小小的襁褓在怀,含泪上前道:``皇上,小阿哥在此,只是无缘了。''

皇帝的手微微发抖,想要去掀开盖着孩子面容的白绢,却无论如何也拈不住那白绢。到底是海兰忍不住,掀起白绢望了一眼,孩子已经被擦洗干净了,面颊青紫发黑,双眼紧闭,显然是被脐带勒住活活窒息而死.

海兰眼中一热,泪水潸潸滚落。她用力捂着嘴,不让哭声从指缝间溢出,勉力道:``好好抱下去吧。''

皇帝看了孩子一眼,目光如被烈风扑灭了的火苗,颤颤巍巍,已忍不住落下泪来。他的气息像哽在喉头一般,抽搐着道:``小阿哥怎会如此?

一众接生嬷嬷吓得筛糠似的乱抖,如何说得出话来。还是江与彬忍了泪道:``皇上,小阿哥一出生便没了气息。嬷嬷们抱出来时微臣查看过,是脐带绕在了小阿哥的脖子上,足足绕了三圈,才使得小阿哥窒息而死。''

海兰的心口像是被巨石死死压住,压得喘不过气来。她的脑中一片混沌,脸色难看极了,半晌才说得出话来,厉声道:``按着规矩,后妃生产之时太医都是候在外头以备不时之需,只有接生嬷嬷们可以守在身边,当时到底是谁接生的?说!''

海兰一向温和静默,即便协理六宫,也是宽和待下,何曾有过如此声色俱厉的时候。后头跪着的一个接生嬷嬷道:``奴婢等六人为皇后娘娘接生。但从皇后娘娘体内接出小阿哥的,唯有田嬷嬷一人.因为田嬷嬷是奴婢等人中伺候各宫小主生产最多的,资历最深,经验也老到,所以这最难的事,都由田嬷嬷亲力亲为。''

田嬷嬷一脸惊恐不安:``皇上,皇上,奴婢伺候皇上与先帝两朝的后宫嫔妃生产,这样的事也是第一次见到。奴牌实在惶恐。''她汗如雨下,拼命磕头不已,``皇上恕罪,皇上恕罪啊!''

海兰的嘴唇哆嗦着,喝道:``小阿哥在皇后腹中一直安好,胎动如常,只是胎位稍稍不正而已,怎会在离开母体之时才发现脐带绕颈没了气息?''

田嬷嬷的汗水滴落在地上,洇出油腻腻的水光。她惶然道:``回愉妃娘娘的话,妇人生产,本就形同在鬼门关走了一遭.皇后娘娘年近四十,身体自然不如

年轻时适合养育.且,且有五公主夭折之事伤怀,所以影响小阿哥也未可知.''

另一接生嬷嬷亦道:``皇上,愉妃娘娘,孩子在母腹中,本来一切就只凭太医脉象诊断判定是否安好。然而生产之事险之又险,什么事都会发生,小阿哥的胎位又不太正,这样的事在民间也是常见,所以,所以\ldots{}''

她话音未落,皇帝一样瞥见立在一旁的钦天监监正,立刻飞起一脚踹向他身上,那监正如何敢躲避,生生受了这一脚,滚落地上。

皇帝双目通红,既怒且伤心,道:``你们不是说皇后这一胎怀的是祥瑞之子,上承天心,下安宗兆,还说紫微星泛出紫光,是祥瑞之兆!如今看来,全是一派胡言!''

那监正连滚带爬地跪起来,匍匐在地,磕头如捣蒜:``皇上!皇上!微臣夜观星象,不敢胡言啊!且微臣也说了,阿哥在日中前后出生是最吉祥的.至于为何绕颈而死,微臣,微臣也不知为何会如此?''他痛得龇牙咧嘴,却实在不敢痛呼出声,只得咬着牙道,``皇上要责罚,微臣自甘领受。只是微臣也不知为何如此,但求死个明白。''他磕了个头道,``皇上,微臣请问皇后娘娘生辰何时?''

皇帝气得脸色铁青,如何说得出话来,扬了扬下巴。李玉会意,便道:``皇后娘娘的生辰是戊戌年二月初十日酉时三刻.你这样卑贱的奴才,能知道皇后娘娘的生辰,也算死而无憾了。''

监正掰着指头,眉心紧锁,算了片刻道:``皇上,皇后娘娘是戊戌年所生,生肖为狗。而今年是乙亥年,生肖为猪。流年对冲,以生肖大者为胜,生肖小者非死即伤。''他看了看窗外天色,又道,``此刻正是卯时二刻,天色欲明未明,皇后娘娘生辰是酉时三刻,正是日暮时分,二者也是相冲.本来皇子属阳,若能在日中时分出生,便会贵不可言。可从皇后娘娘的生辰来看,命相极阴,才克住了小阿哥在此时出生,结果断了性命啊!''

海兰未等听完,己经勃然大怒.她气得浑身乱颤,发髻间的珠花钗珞玎玲作响:``小阿哥未生之时,你极尽阿谀,言说祥瑞.小阿哥出生夭折,便将一切都推脱到皇后娘娘身上。''她直挺挺跪下:``皇上,臣妾恳请皇上治钦天监监正妄言犯上之罪。''

那监正吓得伏在地上不敢起身:''皇上,皇上,微臣不敢妄言.恕微臣狂狂妄,五公主被疯犬咬伤而死,也正是因为皇后娘娘命相极阴,才招来犬患,从而累及在旁的忻妃娘娘和六公主啊!''

海兰惊怒交加,转首怒叱道:``你胆敢污蔑皇后!简直罪该万死!''

皇帝的面色变了又变,两颊边的肌肉微微抽搐着,仿佛有惊涛骇浪在他的皮肉之下起伏而过。良久的静默,几乎能听到众人面上的冷汗一滴滴滑落于地的声响。火盆里的炭火熊熊地燃着,一芒一芒的火星灼烧了人的眼睛,偶尔``哔啵''一声轻响,几乎能惊了人的心腑。

皇帝的声音极轻,像是疲倦极了,连那一字一句,都是极吃力才能吐出:``十三阿哥赐名永璟,乃朕嫡子,朕心所爱.然天不假年,未能全父子缘分.追赠十三阿哥为悼瑞皇子,随葬端惫太子园寝。''他顿一顿,``一众接生人等,照料皇后生产不力,一律出宫,永不再用。钦天监监正,妄言乱上,污蔑皇后,革职,杖毙.''他说罢,遽然起身离去,衣袍带起的风拂到海兰面上,她无端端一凛,只觉拂面生寒。

海兰膝行两步,跟上皇帝道:``皇上不去看看皇后娘娘么?''

皇帝的脸对着殿外熹微的晨光,唯余身后一片暗影,将海兰团团笼罩:``皇后生产辛苦,愉妃好好陪陪她吧,也叫江与彬好生照料.联累了,且去歇一歇.十三阿哥的事,你缓缓告诉她吧.''

海兰还要再说,一阵冷风卷着雪子飕飕扑上身来。半晌,人都散尽了,连江与彬都赶去了如懿殿中伺候。她木然地站在殿门前,身子无力地倚靠在阔大的殿门上,任由生硬的檀木雕花生生地硌着自己裸露的手腕,浑然不觉痛楚.

叶心赶忙扶住她道:``小主,您别站在风口上,仔细伤了身子。''

海兰吃力地摇摇头:``姐姐又一个孩子没了,这样不明不白地,不知姐姐知道了,会伤心到何种境地。''

叶心将一个画珐琅三阳开泰纹手炉塞到她手里,替她暖上了,道:``小主关心皇后娘娘也得留心自己的身子啊,否则还有谁能陪着皇后娘娘劝慰呢?往后的日子,还靠小主呢。''

海兰望着外头雪子纷扬洒落,那一丁一丁细白冷硬的雪子落在殿外的青石地上,敲打出``咝咝''的响声.那雪白一色看得久了,仿佛是钻到了自己的眼底,一星一星的冷,冷得连满心的酸楚亦不能化作热泪流出。

也不知过了多久,她雪白而模糊的视线里终于有旁人闯入,那是闻讯匆匆赶来的绿筠和忻妃。

忻妃尚未来得及走近,已经满脸是泪,泣道:``为什么保不住?为什么都保不住?''

绿筠连忙按下她的手,劝慰道:``忻妃妹妹,这个时候别只顾着自己伤心了。''她四下张望一转,忙问海兰:``皇上就这么走了?''

海兰默默点头:``只叫我陪着皇后娘娘。''

绿筠本就憔悴见老,一急之下皱纹更深:``皇后娘娘还不知道吧?若是知道了,可这么好呢?''她似乎有些胆怯,然而见周遭并无旁人,还是说道,``皇上不在,可不大好啊!''

忻妃雪白的牙齿咬在薄薄的红唇上,印出一排深深的齿痕:``皇后娘娘痛失小阿哥,还要被钦天监的人低毁,那监正死了也是活该!''

绿筠闻言,呆了片刻,念了句``阿弥陀佛'',轻声道:``皇上杀了钦天监的人,怕是不会信他们的胡言乱语了吧?''

海兰不知该如何应答,只是抬起满是忧惧的眼,深深看着绿筠,道:``十三阿哥一出娘胎就天折了,皇后娘娘伤心疲惫,恐怕无力照管十三阿哥的丧仪.姐姐位分尊贵,乃群妃之首,十三阿哥丧仪之事,就都有劳姐姐了.''

绿筠连连颔首,拭去眼角泪痕:``出了这么大的事,我能做的也唯有这些了,一定会尽心尽力。''

三人正自商议,只见小宫女菱枝过来请道:``三位小主,皇后娘娘醒了。''

菱枝为难地咬一咬唇,海兰会意:``你且下去,咱们去瞧瞧皇后娘娘。''

一踏入寝殿内,四周的火盆都燃得旺旺的,让人如入三春之境.殿中己经收拾了一遍,原本备着的婴儿的摇床衣物都己被挪走了,连产房中本会有的血腥气也被浓浓的苏合香掩了过去。

如懿已经醒转过来,身体尚不能大动弹,眼眸却在四下里搜寻,见得海兰进来,忙急急仰起身来道:``海兰!海兰,我的孩子呢?孩子去了哪里?''

宫人们都静静避在殿外,连江与彬也躲出去熬药了,唯有容佩守在床边,默默垂泪不已。如懿焦急地拍着床沿,苍白的两颊泛着异样的潮红:``皇上呢?皇上怎么也不在?我问容佩,她竞像是疯魔了,什么也不说!''

海兰分明是能看出如懿眼底的惊恐,她汗湿的发梢粘腻在鬓边和额头,一袭暗红的寝衣是残血般的颜色,衬得她的面色越发显出有衰老悄然而至的底色。她的皮肉有些许松弛的痕迹,她的眼角有了细细的纹,当然,不细看是永远看不见的。她的青丝,失去了往日华彩般的墨色,有衰草寒烟的脆与薄。但她还是自己的姐姐,彼此依靠的人。

心意电转的瞬间,滚烫的泪水逆流而至心底。海兰定了定神,缓缓道:``姐姐,小阿哥与你缘分太浅,已经走了。''

绿筠急得连连跺足,在后轻声道:``愉妃,你一向最得体,怎么也不缓缓说。说得怎么急,也不怕皇后娘娘伤心!''

如懿的瞳孔倏然睁大,枯焦而煞白的双唇不自禁地颤抖着:``你说什么?''

忻妃不忍再听下去,掩面低低吸泣.海兰望着如懿,神色平静得如风雨即将到来前的大海,一痕波澜也未兴起:``姐姐,孩子一离开你的身体就没了气息。脐带在脖子上绕了三圈,谁也救不得他!''

如懿一句话也说不出来,只是死死地盯着海兰,目光几欲噬人。那颤抖像是会传染一般,从她的唇蔓延到她的身体,剧烈地、无法控制地颤抖着。她拼尽了全力,才发出含糊不清的几个字节。海兰努力地分辨着,才勉强听清楚,那是如懿在唤:``孩子,我的孩子!''

痛不欲生,真真是痛不欲生!如懿只觉得从五脏六腑中涌出一股撕裂的疼痛,随着每一口活着的喘息,蔓延到四肢百骸,蔓延到整个灵魂,掏肺剜心,排山倒海。

她所呼出的热气,所吸进的微寒的空气,仿佛两把尖锐的锋刃,狠狠剖开她的身体,一刀一刀清晰地划动。

海兰原以为如懿会大哭,会崩溃,会声嘶力竭,然而如懿极力地克制着,连泪也未曾落下,只是以绝望的眼无助地寻找:``让我看他一眼,我的孩子,让我看他一眼。''

绿筠缓步上前,忍着泪道:``皇后娘娘,未免您伤心,皇上己经吩咐送了十三阿哥出去,让您不必见了。您,您节哀吧。''

如懿缓缓地摇着头,一下,又一下,每一下都像是拼尽了全力一般,沙哑着喉咙道:``不!不!他在我腹中十月,每一天我都感知到他的存在,怎么会没了?就这样没了?我不信,我不信我千辛万苦生下的孩子,会就这么弃我而去!我不信!''她死死地抓着海兰的手臂,眸中闪着近乎疯狂的光芒,``钦天监不是说我的孩子是祥瑞之胎,贵不可言么?我的孩子怎么会死?不会的!不会的!''

忻妃触动不已,伏在如懿床边,凄然落泪道:``皇后娘娘,钦天监的舌头反复不定,一会儿说您的孩子贵不可言,一会儿又说是您的生辰八字与十三阿哥相冲,克死了阿哥!他们的话听不得的!''她的泪汹涌而落,勾起痛失爱女的伤心,``皇后娘娘,十三阿哥走了,您不见也好.多看一眼,只是多添一份伤心罢了。臣妾当日眼睁睁看着六公主走了,那种锥心之痛,不如不见。''

绿筠见忻妃如此伤怀,只怕她勾起如懿更深沉的痛,只得扯过了她,对着海兰道:``愉妃妹妹,忻妃如此伤心,不宜在这儿劝解皇后娘娘,我还是先陪她回去。''

海兰微微领首,示意容佩送了出去.

殿中再无他人.如懿颓然仰面倒在榻上,眼中的泪水恣肆流下,却无一点儿哭声.海兰静静坐在她身边,拿着绢子不停地替她擦着眼角潸潸不绝的泪,浑然不觉那是一件徒劳无功的事。

如懿的眼无神地盯着帐顶,樱红的连珠帐上密密缀着米拉大的雪珠,闪着晶莹的微光.底下是``和合童子''的花样,两个活泼可爱、长发披肩的孩童,或手持荷花,或手捧圆盒,盒中飞出五只蝙蝠,憨态可掬,十分惹人喜爱,正是得子的喜兆.连被褥床帐上都是天竺、牡丹、瓜瓞和长春花的图案,一天一地地铺展开来,是瓜瓞绵绵、福泽长远的好意头.那样喧闹热烈的颜色,此刻却衬出如懿的面容如冷寒的碎雪,被尘烟的黯灰覆盖。

如懿的声音像是从邈远的天际传来,幽幽晃晃:``海兰,这是我的报应。''

海兰柔声道:``姐姐,孩子己经没了,您的身子却还是要的.胡思乱想,只会更伤身伤心。''

如懿并不看她,只是痴痴喃喃道:``真的.海兰,这是我的报应.哪怕不是我自己动手,也是我害死了孝贤皇后的二阿哥和七阿哥.我害了旁人的孩子,所以如今也轮到我自己了一命抵一命,我的璟兕和十三阿哥也没有了。''

海兰的眼底闪过一丝锐色,紧紧握住如懿的手臂道:``姐姐,一个孩子没了而已,再生就是了!哪怕不能生了,咱们还有永琪和永璂呢!若论报应,我一点儿也不信!宫中双手染上血腥的人还少么?说句不怕忌讳的话,太后娘娘如今稳居慈宁宫,当年也不知是如何杀伐决断呢?若有他日身为太后来做报应,姐姐有什么可害怕的?''她的神色愈加坚定,仿佛逆风伏倒的劲草,风过又屹屹而立,``若真有下地狱的劫数报应,我总和姐姐一起就是了!''

如懿无声的啜泣,泪一滴滴从腮边滑过,带着滚烫的灼烧过的气息,仿佛皮肤也因此散出焦裂的疼痛:``海兰,钦天监的人说是我克死了我的孩子,是不是?''

海蓝冷冷道:``这样说的那个人,已经被杖毙了。长着这样的舌头,千刀万剐也不足惜。''

如懿的脸带着茫然的痛楚:``孩子没有了,难道怪我么?皇上一向对钦天监的话深信不疑,他一定是听进去了,是不是?''

海兰怔了一怔,旋即道:``姐姐,杀钦天监监正的旨意,正是出自于皇上。皇上不会相信的。''

如懿的神情苦涩得如吞了一枚黄连:``杀了钦天监监正,不代表皇上不信这些话。否则,此刻他怎会撇下我一人在此。''

海兰的眉眼间尽是痛惜之色,紧紧握住她冰凉而潮湿的手心:``姐姐,既然知道只有自己一个人,那就更不能只是一味伤心。''

如懿的软弱只在一瞬,旋即回过神来,用力擦去腮边泪痕,疑道:``海兰,我的孩子日日在腹中胎动如常,太医也说安然无事,怎会突然脐带绕颈而死?''

二人正自说话,江与彬端了一碗汤药走进,恭声道:``皇后娘娘,这是安神补血的汤药,您尽快服下吧。''

如懿仰起身,迫视着他道:``江与彬,本宫怀胎十月,你日日诊脉,孩子是否一直无恙?''

江与彬朗然道:``娘娘有孕之时安稳无碍,微臣一切都可以担保.''他犹疑,``但是生产之事,微臣虽然参与,但只能候在屏风之外,并不能走近,所以\ldots{}''

如懿疑心更重:``所以只在接生嬷嬷身上,是不是?''

江与彬只得道:``是。''

海兰秀眉微蹙:``生产之事生死一线,姐姐是疑心接生嬷嬷对孩子动了手脚?您是中宫皇后,他们可是不要命了?且这件事若真查得出蹊跷也罢,若查不出什么,只怕皇上和太后还要怪姐姐不肯安分。''

``她们不是不要命,只看她们自己。''如懿紧紧捂着胸口,竭力平复气息,``这件事不查问透彻,本宫总是不能甘心!璟兕已经不明不白死了,十三阿哥不能再这般死得不明不白。无论如何,必得细细去查。若真的天意如此,本宫也无话可说了!''

\hypertarget{ux7b2cux5341ux516bux7ae0-ux79bbux6790}{%
\chapter{第十八章 离析}\label{ux7b2cux5341ux516bux7ae0-ux79bbux6790}}

这样的心念不过一动,如懿遣容珮去回禀皇帝之时,皇帝也未曾见她,只是辗转吩咐了李玉道:``这些接生嬷嬷伺候过先帝与朕两朝,没有功劳也有苦劳,皇后要查问也可,只是别用刑太过,以免伤了阴骘。''

此时,冬雪正盛,嬿婉与晋贵人富察氏在暖融融的永寿宫中,只穿着略略单薄的颜色锦衣,越发衬得一张脸娇嫩得能沁出水来。这样好的年纪.只求美艳动人,何惧外头冬寒凛冽呢。二人侍奉在皇帝身侧,听得李玉转述容珮之言,晋贵人扬一扬绢子,娇声道:``皇上所言甚是。依照臣妾看来,还是不要用刑才好。

皇后娘娘的孩子没了,伤心迁怒之余还要用刑,嫌宫里的哭声还不够多么?且不说别的,令妃娘娘还有着身孕呢,听不得这些凄楚声音。''

嬿婉的肚腹还不明显,她惯性地扶着腰肢坐在皇帝身侧,一脸的不忍,柔声道:``臣妾为求福祉,这些日子都在宝华殿参拜,希望能平平安安生下孩子来。''她轻叹一声,``说来这些接生嬷嬷都是积年的老嬷嬷了,要赶出去臣妾已经心中不忍,还指望着能有她们替臣妾接生呢,若是那些手脚不利落,当差不久的,臣妾也不放心。''

皇帝握一握嬿婉微微发凉的手,声音虽然倦哑,却也极力安慰她道:``你放心。这些人出去了,自然挑好的来伺候你。你第一次有孕,难免担心,也是有的。瞧瞧你,手这样凉,可是穿得太单薄了?''

嬿婉勉强支起一缕惨淡的笑容,臻首微垂,甚是楚楚:``臣妾只是想着十三阿哥,又听皇后娘娘要用刑,所以害怕\ldots{}''她话未说完,怯生生看了皇帝一眼,按着心口,似是不堪承受这般忧惧的心绪,``臣妾知道自己胆小,皇后娘娘爱子心切,无论怎样严刑拷问,都是应该的。''

晋贵人冷着一张俏脸,道:``怨不得令妃娘娘听着害怕。十三阿哥才走,这样用刑查问的话,也只有皇后娘娘才说得出来。若是孝贤皇后还在世,以她悲悯和善之心,一定不会这样做了。''晋贵人又呖呖道,``且十三阿哥被脐带绕颈而死,又干接生嬷嬷们何事?孩子在腹中好不好的,难道皇后娘娘自己不知?怕是因为钦天监说天象祥瑞的缘故,皇后娘娘才故意闭口不言的吧。''

皇帝横了叽叽喳喳的晋贵人一眼,也未置可否,只吩咐道:``李玉,那就告诉皇后,她要查便查,只不许用刑就是了,也当为十三阿哥积点儿福气。''

话传到如懿耳中,她只能苦笑。若不用刑,如何撬得开这些在深宫中浸淫己久、油滑老练的嬷嬷们?达般言说,皇帝必也以为是生产意外之故。更甚者,或许也是认定了是自己与孩子相冲的缘故吧。

人言可畏,众口铄金。有时何须众口,只需一人之口,击中软肋,便可积毁销骨了。

容珮无可奈何道:``皇上这么说,只怕咱们想查也查不出什么了。''她愤愤难平,``偏偏晋贵人的口舌那样不安分,一口一个孝贤皇后比着娘娘,生怕显不出她娘家人的贤惠么?''

这样的话语,几乎要激起如懿心底最深处的厌憎与嫌恶。纵然死者已逝,留子世人的是她显于外在的节俭克己之德行,皇帝亦多作深情缅怀之状,只是不曾露于世人的恶相,却偏偏要以一句``悲悯和善之心''来掩饰么?那一瞬,她真的很想冷笑,然而那笑意涌到嘴边,却似有丝丝缕缕的寒意蔓延进骨髓深处,更觉得悲怆难言。她与富察琅嬅斗了半世,莫不是出尽百宝费尽心机,到头来又如何,却是生生折了自己孩子的寿数。这算不算是对于一个母亲最深切而不能救赎的报复?

这样的心念苦苦缠逼于思绪的凌乱沉沦之间,逼得她几近疯狂。许久,如懿才勉力坐起,掠一掠鬓边蓬乱的发丝,咬着牙一字一字道:``皇上不许查.怕是心里认定了钦天监的言说。皇上一向相信天象之言,之前以为本宫所怀之胎贵不可言,才如此欣喜。如今出了这样的事,才会格外失望。所谓登高必跌重,便是如此了。''

容珮垂下脸,谨慎的面容上含了一丝精明:``这件事奴婢思来想去,总觉得不妥。之前娘娘有孕,钦天监突然说娘娘这一胎如何祥瑞,如何贵重,等十三阿哥一过世,又说是娘娘与十三阿哥相冲才克死了阿哥。这一捧一砸,起伏太大,便是要人不信也难,所以,皇上才会冷落了娘娘。''她看着如懿,殷殷道,``奴婢心里有个念想,若钦天监这些言语是一早有人安排了算计娘娘的\ldots{}''

如懿骤然一凛,抓住容珮的手腕道:``你也这么想?''

容珮望着如懿苍白如雪的面颊,唇上嵌着深深的印子,这些日子,如懿的心痛与自责,她无不看在眼里。思前想后,容珮只得微微颔首:``奴婢只是胡思乱想罢了。''

长久的愕然之后,如懿的面容只余下惊痛骇然的沉影,她叹息的尾音带过一缕沉痛至极的悲伤,哀切道:``容珮,原来你与本宫想到一处了。本宫素来与钦天监无甚来往,从前怀永璂与璟兕也并未有这些话传出,怎的突然这一胎便极其祥瑞了。若真是有人背后算计,便真真是可怕至极了。''

容珮道:``只可惜钦天监监正已死,咱们也查不出什么了。但只要娘娘有了防备,咱们便不怕了。''

窗外的寒风簌簌地扑着窗上薄薄的明纸,仿佛有什么猛兽呼啸着想要扑入。

沉默的相对间,如懿只觉得彻骨森寒,冷得她连齿根都在发颤·容珮牢牢地扶着她单薄的身体,温言道;``皇后娘娘,万事都得自己保重·养好了身子,才能替十三阿哥要个明白啊。''

如懿正欲说话,只见刻丝紫天鹿衔芝的厚缎帘子一掀,三宝带着一股冷风急匆匆进来,道:``皇后娘娘,奴才奉您的懿旨往阿哥所的灵堂向十三阿哥致祭,结果碰上了江太医。江太医说皇上不许对接生嬷嬷们用刑,怕是查不出什么,想再看看十三阿哥的遗体。今日本是要将十三阿哥的遗体运往端慧太子的园寝下葬了,奴才和江太医好说歹说,只推说皇后娘娘思念十三阿哥不已,让奴才开棺再看一眼,结果便发现十三阿哥的脸上出现了五个黑色的指印。''

如懿一颗心猛地一颤,连声音都变了:``付么指印?''

这么冷的天气,三宝的额头居然目着汗,蒸出白腾腾的热气。他急切道:

``江太医知道不妥,细细查验了.才发觉那五个指印是包在十三阿哥嘴边的。这样的指印是有人用力过猛留下的痕迹,十三阿哥刚过世的时候是瞧不出来的,只有过了几天才会显现出来。''

如懿的心怦怦地跳着,剧烈地颤抖,仿佛要从嗓子眼中冒了出来:``你的意思是有人曾经捂住过十三阿哥的嘴?''她只觉得是谁的手紧紧捏住了自己的喉咙,那股可怕的念头几乎要吞没了她所有的理智,``若按接生嬷嬷所言,十三阿哥哥真是一出生就死了,何必要捂住他的嘴?难道,难道本宫的十三阿哥出生时明明是活着的?''

三宝急急道:``江太医也是这样以为。江太医疑心十三阿哥明明是平安出生,却在头刚离开娘娘母体之时就被人捂住嘴不许出声,又拿脐带活活绕死的。

因为若十三阿哥一出生便没了气息,那指印根本不会在死后数日显现出来,必得是活着的时候按下去的,才会如此,所以江太医立刻回禀了皇上!''

浑身的气血拼命地涌上头来,像是无数的巨浪澎湃撞击着她残碎如秋叶般的一颗心,抛至浪尖,又狠狠撞在礁石之上。如懿几乎能听见自己的骨血撞在坚硬的磐石之上迸裂碎成齑粉的声音。暗红的血丝如蛛网布上她的眼,浓郁得几乎要滴出血来。她听见自己的牙齿咯咯撞击的声音:``接生嬷嬷们一个都不许放过,尤其是替本富接生的田嬷嬷!查!替本宫枉死的孩子查个水落石出!''

慎刑司的精奇嬷嬷们向来刑比狱官,做事十分精干利落。皇帝闻讯后更是惊怒交加,立刻下旨严查。精奇嬷嬷们得了皇帝的旨意,即刻将已经出宫的接生嬷嬷一一寻回宫中,关入慎刑司细细查问。精奇嬷嬷们见事关皇后与帝裔,如何敢不经心,慎刑司七十二道刑罚流水般用了上去,尤其是对田嬷嬷,刑讯更是严厉,又有皇帝身边的太监进忠亲自督阵审问,不过一日一夜便有了消息。

如懿生产之后本就元气大伤,更满心牵挂着幼子夭折之事,只觉得度日如年,煎熬异常。补身的汤药一碗碗地喝下去,那酸涩而苦辛的气味像是永远地留在了喉舌之中,无论如何也不能洗去。连她自己亦觉得总是恍恍惚惚如在梦中,闭眼时反复还肚腹隆起怀着孩子,唯有在这样的梦中,那种丧子的切肤之痛,才会稍稍消减。二梦醒之时,她挣扎着摸到自己已然平坦的肚腹,而孩子却在即将降临时便已魂归九霄,便是心痛不已。

那明明是日日在她腹中踢着她的鲜活的孩子啊,更应该是睁开眼看得见这个人世的孩子,却连一声啼哭也不能发出,就这样凄惨地去了!

这样日日夜夜地伤神,让如懿迅速地憔悴下去。而皇帝,便实在这样的凄楚里见到了她伤心欲绝的面孔。

这是如懿生产后皇帝第一次踏入翊坤宫。两下的默然里,彼此都有些生疏.唯有侍女们有条不紊地端上茶水与酥点,将往日做惯的。一切又熟稔地再做一遍。

这样的彼此相对,依稀是熟悉的。皇帝的面色并不好看,隐隐透着暗青色的灰败,仿佛外头飞絮扯棉般落着雪的天空。

仇恨与哀痛绞在如懿心口.仿佛比着谁的气力大似的,拼命撕扯绞缠着。如懿的脸色尚且平静无澜,嘴唇却不由得哆嗦,吃力地从榻上撵起身子来,切切地望着皇帝:``皇上此来,可是永璟的死已经分明了?''

皇帝手放在八重莲五铜炭盆上暖了又暖,口中冷冷道:``替你接生的嬷嬷田氏已经招了,而且招得一千二净,一字不落。''

如懿的瞳孔倏然一跳,仿佛双眼被针刺了似的,几乎要沁出血色的红来.她产后伤心,本是虚透了的人,如何禁得起这样的刺激,只觉得一阵晕眩,天地也要颠覆过来,口中犹自念念:``她招了什么?她是为什么?''

皇帝别过脸,怒意与伤心浮溢在眉间:``田氏已经招了,她说是皇后你苛待于她,她心怀怨恨,才会在接生时一时起了歹念,捂住永璟的嘴甩脐带活活绕死了他。而这一切,她手脚既快,又被锦被掩着,旁人根本无从察觉。''

呼吸有一瞬的停滞,她的脑中嗡嗡地响着,那种喧嚣与吵闹像山中暴雨来临前卷起满地残枝枯叶呼啸奔突的烈风,吹打得人也成了薄薄的一片碎叶,卷越又落下,只余惊痛与近乎昏厥的眩目力竭。她的喉咙里翻出暗哑的``咝咝''声:

``臣妾如何苛待于她了?她要如此丧心病狂,害臣妾的孩子?''

过于激动的情绪牵扯着如懿消瘦的身体,她伏在堆起的棉被软帐中,激烈地喘息着。

皇帝的眼角闪着晶亮的一点徽光,那微光里,是无声的悲觉:``璟兕出生之时,正逢舒妃之死,是你下旨说舒妃新丧,璟兕出生的赏赐一应减半,是么?''

容珮忙递了水给如懿喂下,又一下一下抚着她的脊背。如懿好容易平复许,仰起脸静静道:``所以田氏才心怀怨恨么?臣妾自认这样做并无过错。''

皇帝抚着额头,那明黄的袖口绣着艳色的嫣红、宝蓝、碧青,缠成绵延不尽的万字不到头的花样,却衬得他的脸色是那样黯淡,如同烧尽了的余灰,扑腾成死白的静寂。许是天气的缘故,许是内心的燥郁,她的嘴唇有干裂的纹路,深红的底色上泛起雪沫股的白携,让他的言语格外沉缓而吃力:``你自然是以为并无过错。田氏说,彼时她正欲为儿子捐官,正缺一笔银子。她在你宫里伺候你生产辛苦,而你待下严苛,并无优容,也不曾额外赏赐众人。且田氏当日也为赏银之事求过你,你却不肯格外开恩。因你的缘故,她的儿子才耽搁了前程,只捐到了一个修武校尉的官职,否则,会有更好的前程。''

如懿怔怔地靠在容珮臂弯里,片刻才回过神来:``彼时,舒妃新丧不宜大加赏赐,且前线大清的子弟正与准噶尔征伐,粮草军银哪一项不是开销。后宫可以俭省些银子,虽然少,也是绵薄之力。臣妾不肯因自己皇后的身份而格外优容奴婢,正是怕不正之风由臣妾宫中而起,这样也有错么?''她死死地攥着手中的湖蓝色滑丝云丝被,那是上好的苏织云丝,握在手里滑腻如小儿的肌肤,可是此刻,她的手心里全是冷汗,涩涩地团着那块滑丝,皱起稀烂一团,``一个人存心作恶,必定有万千理由。但所有理由叠在一起,也敌不过是她愿意作恶而已。而田氏这样的话近乎搪塞,臣妾不信,愿意与她对质!''

皇帝额头的青筋如隐伏的虬龙,突突地几欲跃出:``已经无用了。田氏受刑不过,招供之后自知必死,已经咬舌自尽了。''他的眼底凛凛如刀锋,``田氏以为一死可以了之,朕怎会如此便宜了她。即便死了,也要施以磔刑。不千刀万剐,不足以泄朕心头之恨。''

无尽的恨意在如懿胸腔里激烈地膨胀,几乎要冲破她的身体。她的牙齿格格地发抖:``的确是干刀万剐死不足惜。因为田氏一死,就是死无对证!合该诛了她的九族,才能让臣妾的永璟在九泉之下瞑目!''她再也忍不住,痛哭失声,那悲鸣声如同孤凄的杜鹃,泣血哀啼,``臣妾的永璟明明应该活着生下来,睁开眼好好地看一看他的阿玛与额娘,谁知才离了臣妾就被人活活勒死,臣妾\ldots 臣妾好恨啊!''

皇帝的泪忍了又忍,终于没有滚落下来,凝成眼底的森然寒气:``朕如何不想诛了田氏的九族?田氏只有一个儿子,要杀了他易如反掌。可是田氏的先祖是从龙入关的包衣,又是镶黄旗出身,祸不延三族,更遑论九族。朕要杀也只能杀她一个。'',

如懿浑身哆嗦得不能自己,像是被抽去了所有的力气一般。她俯倒在轻软的锦被堆叠之中,仍佛自己也成了那绵软的一缕,轻飘飘的,没有着落,只是任由眼泪如肆意的泉水,流过自己的身体与哀伤至碎的心。

良久,有温热的液体,滴一滴洇落她的发间,她原以为是自己的泪,抬起头才见是皇帝站在她身前,无声地落下泪来。他的声音有沉沉的哀伤:``如懿,田氏固然死不足惜,可追根究底,这件事难道与你全然无关么?你是六宫之主,你怎么驾驭后富,朕并不多过问。可永璟的死,若是你御下温厚.何至于如此?''

如懿的眼睛睁得极大,那心碎与震惊的神色如混在一起的瓷器的碎片,闪若寒冽的光,牢牢地粘着皇帝。她沙哑的声音恍若撕裂的绸缎,不可置信一般问道:``皇上是怪臣妾么?臣妾自身为皇后,心知不如孝贤皇后家世高贵,所以更是时时提点自己,要尽到一个皇后的职责。臣妾不是舍不得一点儿银子,而是遵循孝贤皇后节俭之道,也为前线战事思虑,才严格约束后宫嫔妃,奴婢,以免奢靡。''

皇帝缓缓地摇头,极缓却极用力,仿佛巨石沉沉叩在心间:``皇后以为自己没有做错,朕也不能多指摘你什么。奴才们是下贱,可若是你肯多体恤一些,也不至如此。太后闻知永璟惨死,也十分伤心,终日在宝华殿祝祷超度。佛家论因果,难道不是皇后种下的因果?''

皇帝的一字一句,沉闷得像是天际远远的雷声,隐在层层乌云之后,却有雷滚九天之势。如懿像是行走在滚滚雷电下的人,轰然而迷乱.模糊的泪眼里,皇帝缂金彩云蓝龙青白狐皮龙袍上堆出祥云金日的三重深浅缂金线,刺得她双眸发痛。那九条蓝龙各自张开犀利的爪,仿佛要腾云而飞,无孔不入地扑上身来。

一缕苦涩的笑缓缓在她唇边绽开如破碎的花朵,被暴雨拍打之后,从枝头翻飞落下。舌尖像是被咬破了,极痛,极涩:``那么皇上也认定是如钦天监厨言·是臣妾克死了自己的孩子?''

``钦天监的话固然荒谬,但身为人母,有这样的前因后果,皇后也必定于心有愧吧?连朕都不能说服自己,此事完全与你无关。''他仰天长叹,``朕的永璟,朕盼了那么久,本该是比永璂更有出息的孩子。''

他说罢,拖着沉沉的步予踱出殿外。如懿目送他离去,分明感知到他与她之间巨大而深绝的鸿沟在不断扩延。尖锐的痛感从心尖上划过,一刀又一刀,是愧,是悔,还是难以抑制的伤痕欲绝?

宫人们看着如懿的样子,吓得不知所措,慌忙跪了一地。也不知过了多久,还是容珮牵着小小的永璂来到如懿跟前,含泪道:``小阿哥惨死,皇上好似伤心过度才会如此对娘娘说活,皇上一定会明白过来的。''

如懿空洞的眼不知落在何处,虚弱而迷茫地道:``容珮,纵然不是本宫的错,可永璟的死,朕的没有本宫种下的前因么?''

容珮直挺挺地跪着,将永璂推到如懿跟前,道:``娘娘固然伤心小阿哥的死,可是哪怕五公主走了,小阿哥也走了,您还有十二阿哥呢。十二阿哥是翊坤宫仅剩的独苗了,可万万不能再有任何闪失了。''

如懿怔忪间看着窗外白晕晕的雪光迷蒙,纷繁的雪朵如尖而锐的细细砂石,铺天铺地地砸着。她紧紧拥住了同样害怕而伤心的永瑾,仿佛只有这样抱着他,才能攫取一点儿温暖自己的力量。

深深的宫苑回廊,冰雪深寒,唯余这一对母子凄冷而哀绝的哭声。

这一年的冬天仿佛格外寒冷。如同坠落在深寒冻冷的井底,如懿举首望见那样小小一团天空,而自己置身于黑沉局促之中,寸步难行。

太后自端淑长公主归来,早已不再过问六宫之事,只在慈宁宫颐养天年。

偶尔来看如懿,亦不过叮嘱几句,要她保重自己,便也转去看有孕的令妃了。比照着深受恩眷的令妃,如懿的翊坤宫实在可算是门可罗雀。虽然无人敢亏待翊坤宫,但是像避忌着什么不吉利的瘟疫似的,人人不愿靠近半分。如懿索性免了每日嫔妃们的晨昏定省,连海兰、忻妃和绿筠,如懿也不愿让她们来,只道:``你们一个受皇上眷顾,一个有皇子和公主,何必来本宫这里,惹得皇上不痛快。''

绿筠讪讪离去,倒是忻妃极不服气,且怨且叹:``如今皇上的一颗心都在令妃那里,臣妾们算什么?来与不来,皇上都不放在眼里。''

如懿紧一紧身上的石青攒珠银鼠大氅,定定地望着檐下积水冻成的冰柱,尺许长的透明晶体,反射着晶莹的日光。可那日光,仿佛永远也照不进堆绣锁金的翊坤宫.如懿轻叹一声:``何必倔强?不顾着自己,也得顾着孩子和母族.若受本宫的牵连,连你的恩宠也淡了,那你还怎么去盼着你未来的孩子呢?''

忻妃眼底隐隐有泪光闪动,``那\ldots 那臣妾去劝皇上。''她咬着蜃,难过道,``外头的那话传得那么难听,都是说\ldots 臣妾真不想皇上听了这些难堪的话去。''

``难听?''如懿漠然相对,无非是说本宫无福,克死了自己的孩子。世事炎凉,拜高踩低,本不过如此。本宫此番若是平安生下十三阿哥,自然人人奉承,锦上添花.说本宫是积福深重之人,所以折了---个女儿之后便得了一个皇子补偿。如今失子,自然有暗地里称愿的,满嘴可怜说本宫罪孽深重才牵连可孩子了。落井下石,便是宫中之人最擅长的了。''

忻妃到底年轻,哪里受得住这样的话,狠狠啐了一口道:``这么说来,那些贱嘴薄舌的也是这么背后议论臣妾的么?臣妾一定要去告诉皇上,割了他们的舌头!''

如懿淡淡扫她一眼,摆首道:``这个时候,劝也好,哭诉也好,只会让皇上想起不悦之事,连累你自己。忻妃,好好顾着自己吧,你的父祖族人在准噶尔立下的功劳,可不能因为你的任性就淡抹了。''

忻妃无声地张了张嘴,想说什么,终宄还是忍住了。她懊丧道:``皇后娘娘,臣妾一直养在深闺里,有什么说什么,从未有过这样的时候,惩说什么却不得不闭上嘴。娘娘,臣妾知道进了宫说话做事不比在家,须得时时小心,臣妾进宫前阿玛和额娘也是千叮万嘱,可是到了如今,臣妾还是没有办法习惯。''

海兰爱怜地替忻妃掠了掠鬓边蓬松的碎发,婉言道:``忻妃妹妹,你是初来宫中不久,又一直都算得宠,所以不知道其中的厉害。有些事,哪怕没办法习惯,也必得逼着自己习惯。钝刀子割肉还挫着铁锈,谁不是一天天这么熬过来的。''

忻妃沉不住气,气急道:``可是这明明是莫须有的事\ldots{}''

如懿瞥她一眼,斩钉截铁道:``就是因为莫须有才最伤人。你不见宋高宗为何要斩岳飞,也就是`莫须有'三个字啊。人的疑心啊,比什么利器都能杀人!''

忻妃被噎得瞪大了眼睛,愣了半天,无奈叹道:``如今臣妾可算明白了。原先在家时总看阿玛当差战战兢兢,原来咱们在宫里和在前朝没有两样。''

如懿低下头,看着淡淡的日光把自己的身影拖得老长老长,渐渐成了虚晃一抹,低声道;``回去吧,好好伺候皇上.令妃有着身孕,皇上在宠她也不会让她侍寝。听说颖嫔她们一群蒙古妃嫔已经自成了一党,铆着劲儿在争宠呢,你若是有心,就得为自己打算。''

忻妃低头思量了片刻,再抬起脸时已没有了方才那种激动和毛糙,只有着与她年龄不符的一份沉静,她恭敬的了一礼:``多谢皇后娘娘提点。臣妾先告退,只待来日。''

\hypertarget{ux7b2cux5341ux4e5dux7ae0-ux6697ux9999}{%
\chapter{第十九章 暗香}\label{ux7b2cux5341ux4e5dux7ae0-ux6697ux9999}}

如懿轻抚额头,目送忻妃离去。太阳穴突突地跳着,酸痛不已。她静了片刻,轻声道:``海兰,你也走吧。''

海兰坐在如懿身前的紫檀雕番莲卷叶绣墩上,慢条斯理地顺领子上吹落的米珠流苏,轻而坚决地摇了摇头:``臣妾本就无宠,不怕这些。''

如懿望着她,叹息道:``可是永琪\ldots{}''

``永琪大了,皇上不会因为臣妾这个额娘无宠而不器重他,所以无论如何,臣妾都会陪着娘娘。''她顿了顿,眼底有泪光莹然,``就像从前一样。''

眼里有绵绵的感动,一波一波涌上心头。这么些年,从潜邸到宫中,唯有海兰,是未曾变过的,也唯有这份不变,才让人从森冷的壁垒里觅得一丝温暖。

海兰轻声道:``臣妾方才已经让容珮送了十二阿哥去养心殿里请安了。皇上可以不愿意见娘娘,但不能不见自己的亲身儿子。或许见了十二阿哥,皇上心里也能念及娘娘的好。说到底,皇上也是在意十三阿哥的缘故,所以才这般介怀。男人啊,心里究竟是自己的血脉子嗣最要紧。''

如懿轻轻摇首:``皇上素来疑心重,这个节骨眼上,何必\ldots{}''她想再说,然而还是沉默了,只是盯着檐下冰柱闪烁的寒光,长叹道:``这个冬天,怎么这么长啊!''

永璂被容珮拉着手进了养心殿书房,恭恭敬敬请了个安,稚声稚气道:``皇阿玛万福,令娘娘万福。''

嬿婉着了一件家常的春色锦缠枝葡萄纹长衣,领口细细的风毛衬得她孕中的脸如皎洁的月盘。嬿婉云髻半绾,斜着一枝翠玉镂凤长簪,疏疏点着几朵琉璃珠花,正支着腰肢伏在案上翻着一本书卷。她见了永璂,顾不得肚腹已经微微隆起,欠身回礼道:``十二阿哥有礼。''

皇帝忙扶着嬿婉的手臂,眼中有关切之情流转轻溢,道:``你有这身子,朕叮嘱过你,不必那么拘礼。''说罢又含笑看着永璂:``来,起来。道皇阿玛这儿来。''

容珮看着永璂跑到皇帝身边,利索地爬到皇帝的腿上坐着,笑容满面道:``十二阿哥惦记着皇上,一直嚷嚷着要来看皇上。这不,奴婢拗不过阿哥,雪才停就送了阿哥过来。''

皇帝心疼地搓着永璂微冷的小手:``外头那么冷,仔细冻着。你额娘只有你这一个\ldots{}''他下意识地停了嘴。

容珮机警道:``皇上说得是,所以皇后娘娘任谁也不放心,只许奴婢带着照看阿哥。皇上瞧瞧,阿哥是不是又长高了?''

皇帝搂着永璂看了又看,道:``是长高了。可是\ldots 仿佛也瘦了。''

永璂低下脸,一副快哭出来的表情:``皇阿玛不来看儿臣,儿臣也想小弟弟。''

嬿婉面上微微一动,旋即又是谦卑柔和的神色,含笑温柔道:``十二阿哥年幼,就深具孝悌之情,实在难得。说来也是可怜,十三阿哥本该是好好的和十二阿哥一块儿呢。田氏真是死不足惜。''

皇帝的脸色不由自主地沉了一沉,容珮听出嬿婉弦外之音,剜了她一眼,复又一脸恭顺地低着头,看着自己的脚尖。皇帝看着永璂道:``皇阿玛忙于朝政,不能常去看你。你若想皇阿玛,就常来养心殿。''

永璂一脸天真地仰起脸:``那额娘也想皇阿玛呢,她也能来看皇阿玛么?''

皇帝微微语塞,只是笑:``等皇阿玛闲了,就去看你额娘。''他唤过李玉,吩咐道:``天寒路滑,又刚停了雪,你和凌云彻一同送永璂回翊坤宫,仔细着些。''

永璂乖巧地跳下来,行了一礼:``儿臣告退。''他转头看见长几上兽耳羊脂花瓶里供着老大一束红梅,巴巴地望着皇帝道:``皇阿玛,儿臣想去御花园折梅花,额娘喜欢的。''

皇帝怔了怔,旋即笑道:``当然可以。李玉,你们好好护着去吧。''

永璂乖乖离去,嬿婉抚着腰肢,一脸爱怜欢喜:``十二阿哥有皇后娘娘调教,这般懂事会说话,真是难得。只盼臣妾的孩子出生,也能赶得上十二阿哥半分乖巧,臣妾就心满意足了。''她因为有孕而变得圆润的脸庞被领口雪白的风毛簇拥着,如十五饱满莹亮的月,散着格外柔和的朦胧的光。

皇帝嘴角的笑意淡了下来:``孩子天真,孺慕之思做不得假。''

嬿婉的笑意更柔,仿佛细细一弯弧线:``皇上说的是。臣妾只是感慨,也是心有余悸。臣妾不过几月也要生产,真怕宫里接生的嬷嬷中还有如田氏这般心狠手辣的\ldots{}''她按着心口,仿佛不胜柔弱,``臣妾侍奉皇上多年,好不容易才怀上这个孩子,臣妾真是怕。''

皇帝的唇线有清冷的弧度,映着窗外的雪光,更添几分肃然之色:``你嘴上直,性子却软,不会有人这么害你的。''

嬿婉的叹息如悠悠的轻风旋转:``那日听晋贵人闲话,有前因便又后果。皇后娘娘一向把持后宫严厉,不顺己意的便一言不听。若是对下宽厚多恩些,田氏也不至于如此。''她觑着皇帝的神色,``晋贵人一向不喜皇后娘娘,嘴里自然没什么好话,臣妾只当是耳边风刮过了,也请皇上不要过于在意才好。''

皇帝也不作声,径自走回书桌前,牵过嬿婉的手:``来,永璂来之前你和朕说什么来着?你的声音真好听,朕喜欢听你说话。''

嬿婉柔柔道:``是。''她取过那卷书,依依念道:``诸花及诸叶香者,俱可蒸露。''她念了一句,忽而嫣然一笑,道:``那日臣妾嘴馋,恰好内务府的桂花清露没有了,臣妾便叫澜翠折了新鲜桂花用热水冲泡,以为虽比不得桂花清露,但总能得十之二三的清甜,结果便被皇上取笑了。''

皇帝笑吟吟道:``若以热水直接浇到香花上,只会坏了花朵的天然香气。也唯有你这般天真,想出这样的主意。''

嬿婉面上一红,十分忸怩:``臣妾不懂风雅之道,但幸好皇上懂得,臣妾用心揣摩,也总算明白了些许,所以按古方所言制了几款花露放在宫中,以备皇上随时品用。''她掰着手指道:``玫瑰花露柔肝和胃,百合花露滋阴清热,茉莉花露理气安神,碧桃花露养血润颜,梅花\ldots{}''她沉吟片刻,自觉失言,终究没说下去,只是俏生生道,``皇上是不是觉得臣妾进益了。''

嬿婉如清水芙蓉般的面容在明亮的殿中被窗外雪光镀上了更加温婉的轮廓。有时候一个眼错,看到嬿婉,会让人想起年轻时的如懿的脸,只是完全不同于如懿的冰雪之姿。嬿婉的美,更凡俗而亲切,带着烟火气息,像开在庭院里一朵随手可以攀折的粉红蔷薇。

皇帝笑着揉一揉她的头发,眼神中尽是宠溺之情:``是了。你聪慧伶俐,没有什么学不会,也没有什么学不好的。''他转过脸问:``进保,今日备着什么点心?朕有些饿了。''

进保应了一声,便道:``今日御膳房备着的是暗香汤和水仙白玉酥。''

皇帝皱了皱眉,便有些不悦:``水仙白玉酥也罢了,好好地怎么想起做暗香汤了?''

进保见皇帝的气来得莫名其妙,只得答道:``御膳房做的点心都是按着节气来的。暗香汤取腊月早梅所制,入口清甜。水仙白玉酥也是做成水仙花五瓣的模样,绵软松爽。若\ldots 皇上不喜欢,奴才就叫他们去换。''

皇帝不耐烦地摆摆手:``罢了。都是吃絮了的东西,也没什么意思。''他看着嬿婉:``你喜欢吃什么,朕叫御膳房送来,朕陪你一起吃。''

嬿婉含笑谢过,托腮想了几样,皇帝便嘱咐进保去御膳房拿了。嬿婉一脸欢欢喜喜的样子,温柔乖巧得叫人忍不住轻怜密爱。他牵过她的手,抚着她鼓起的肚子,絮絮地有一句没一句地嘱咐着什么。其实他并不知道自己说了什么,思绪跳宕的空隙间,他想起某一年的冬日,其实想不起是哪一年了,或许年年如是,如懿披着深红的斗篷,站在梅枝下仔细挑选着合适的初开的梅朵,以备来日泡成这一盏有暗香浮动的暗香汤。

连那汤方他都一字一句地记得清楚:``腊月早梅,清晨摘半开花朵,连蒂入瓷瓶。每一两,用炒盐一两撒入。勿用手抄坏,箬叶厚纸密封。入夏取开,先置蜜少许于盏内,加花三四朵,滚水注入,花开如生。充茶,香甚可爱。''

这是从《养小录》上得来的方子,如懿一见便喜欢得紧。她那样喜欢梅花,与梅花有关的都爱不释手。为表郑重,也为谢她的玲珑心意,是自己亲手抄录的方子,存在她的妆盒底下。如今这盏甜汤已经成了御膳房向例的点心。那么她呢?她可曾喝到这一碗最爱的暗香汤?

如懿静静靠在花梨边座漆心罗汉长榻的银丝软装上,螺钿小几上的一盏暗香汤已然凉透,不再冒着丝丝缕缕氤氲的乳白热气。如懿的心思有些飘忽,侧耳听着窗外冰柱融化时点滴的淅沥微声,滴落在冰冷坚硬的砖地上。

寒兰坐在长榻的另一侧,取了一管紫毫,低头仔细抄录着一卷佛经,抬头看了如懿一眼,道:``这暗香汤都凉透了,姐姐都没喝上一口,看来真的是没什么胃口。等下我亲自下厨,去做几个姐姐喜欢的小菜吧。''

如懿宁和一笑,那笑容却只是牵动了嘴角的弧度,内里却黯然无色:``那便是我的口福了。''她说罢,将自己手里一个平金素纹手炉塞到了海兰怀里,``抄了半天的佛经了,虽说殿里有火盆,但手总露在外头,仔细冻着。''

海兰叹口气,柔声道:``十三阿哥走得早,我们没能为他做些什么。虽然我平素不信六道轮回,但此刻却真心盼望十三阿哥能早日超脱轮回之苦,登上西方极乐世界。''

如懿眼中微有晶莹之色,颔首道:``这些日子你陪着我抄录了九百九十九卷经文,若是十三阿哥有知,也可稍稍安慰。''她扭一扭酸痛的手腕,苦笑道:``适时歇一歇,别和我一样,伤了手便不得不停下来了。''

二人正坐着说话,庭院中骤然有响亮的脚步声响起。如懿听得动静,不由得抬起头来。

三宝在外头欢欢喜喜道:``十二阿哥回来了。瞧小脸儿红的,别是冻着了吧?来,奴才替您烧个暖炉暖暖。''

却是听得乳母嬷嬷们簇拥着永璂进来,请了安道:``皇后娘娘万福金安。''

永璂亦跟着道:``额娘万安。''他一说罢,扑入如懿怀中,扭股糖似的拧着。

如懿搓着他的小手,笑嗔道:``越发没规矩了。手这么冷,快下去添件衣服。''

永璂点点头:``儿臣折了额娘最喜欢的梅花,额娘记得要看啊。''

如懿含笑看他跟着乳母去了偏殿。寒兰忙起身道:``也不知十二阿哥的话说得圆满不圆满?臣妾去瞧瞧。''如懿见她急急出去,裙摆都闪成了一朵花儿,轻轻摇摇头,复又低首去理那千丝万缕、色彩纷呈的丝线。

如懿见凌云彻站在门边,不觉微笑:``凌大人来了。''她唤过容珮:``给凌大人看座。''

凌云彻手里抱着大束的白梅,一时不便坐下。那些梅枝显然是精心挑选过,傲立的舒枝之上每朵梅花都是欲开未开的姿态,盈然待放,还有脉脉细雪沾染。只是殿中温暖,那细雪很快化作晶莹水珠,显得那朵朵白梅不着尘泥,莹洁剔透。

如懿微微失笑:``瞧本宫糊涂了,你抱着这些梅花,如何能坐下。''她显然被这些清洁莹透的花朵吸引,眸中微有亮色,``如今翊坤宫的人不打出去,虽是冬日,好久不见梅花了。''

容珮接过凌云彻手中的花,抿嘴一笑,欢喜道:``凌大人有心了。我们娘娘最喜欢梅花了。''

凌云彻将花递到容珮手里,看她抱了花朵去偏殿寻合适的花瓶,方才不好意思地笑笑:``是十二阿哥的一片心意,微臣只是帮十二阿哥摘了送来。希望皇后娘娘看在十二阿哥的孝心上,可以稍稍展颜。''

如懿欣慰道:``永璂很是孝顺。''

如懿偏着头,髻边一直鎏金碧玉瓒凤钗上垂落一串白玉,那玉色洁白,与她苍白的面孔殊无二致。她的形容清减了不少,淡妆素容的样子更显出眉目间难掩的一丝忧郁。凌云彻不知怎的,就觉得心口微微哆嗦,徒然酸楚不已。他情不自禁道:``皇后娘娘身子可养得好些了么?一直惦记着,也不能\ldots{}''他觉得自己说得不恰当,赶紧道:``其实皇上也惦记着。''

如懿淡淡一笑,那笑容像是浮在碎冰上的阳光,细细碎碎的,没有丝毫暖意:``境遇再坏,坏得过从前咱们在冷宫的那个时候么?本宫不会不要自己的身子,一定会养好的。''

凌云彻面庞上紧绷的弧度随着这句话而松弛下来:``皇后娘娘喜欢梅花就多看看。微臣也喜欢梅花。''

如懿注目于那些洁白无尘的花朵,口中不经意道:``难得听你说喜欢什么花儿草儿的。''

凌云彻安静片刻,道:``梅花已开,寒冬虽在,但也快是春天了。微臣知道皇后娘娘喜欢梅花,所以新学了一首诗,在娘娘面前班门弄斧了。''

如懿颇有兴致,长长的睫毛扬起,眸中有星子般的亮色:``你也学诗了?''

凌云彻有些难为情:``从前好歹上过几年私塾。皇后娘娘别笑话微臣。''他清一清嗓子,朗然念道:``冰雪林中著此身,不同桃李混芳尘。忽然一夜清香发,散作乾坤万里春。''

翊坤宫的暖阁宽阔良深,几近无声的静谧让空气里有种凝固的感觉,几乎能听清同掐丝珐琅八角炭盆里红箩炭``哔剥''燃烧的轻响。嗯,那种轻响,也是温热的,如同他此刻的心情。他不是不知道她这些日子的清冷幽闭,无数次想要寻个机会来看看她,哪怕只是说上几句话,就如当年在冷宫一般。可是人在跟前,他能想到的,竟是幼年时学过的这首诗。他不知道自己是怎么说出这些话的,或许是这个寒冷的冬日颇为应景,或许是那束白梅正好勾起了他封闭而压抑的情思。他暗暗自嘲,果然自己是不擅长安慰别人的人,连找一首写他喜欢的梅花的诗,也是这样简单而朴素。

如懿的声线清凌凌的,若不细听,几乎难以察觉那一丝即将痊愈的沙哑。她极客气地道:``是王冕的《白梅》,和眼前这束花倒应景,难为你记得。有心了。''

凌云彻一脸诚挚,动容道:``微臣知道自己是个粗人,但冬去春来,只是一瞬之间,还请娘娘暂且忍耐。''他挠了挠额头,苦苦思索片刻,眼中骤然一亮,如熠熠的火苗,``微臣还背过一首,前头不大记得了,但后面几句真是好,微臣看过久久记在心里。`横笛何愁听,斜枝倚病看。逆风如解意,容易莫摧残。'微臣也希望逆风解意,让娘娘能顺心如意。''

如懿的笑意渐渐淡下去,成了幽微一抹,仿若落日时分即将被夜色吞没的最后一缕霞光:``你的好意本宫心领了。但是逆风如何能解意,只盼自己熬得住风势强劲,莫被容易摧残罢了。''

他微微抬首,不敢直视着如懿,只是以眼角的余光瞥见她梅子色缀绣银丝梅朵紫狐大衣,那样暗沉的红的底色像是展不开的一个笑颜,凝在了那里,并无一丝欢喜的气息。连那银丝绣簇的梅花,也像一滴滴斑驳的泪痕,闪着剔透的水光。她长长的裙幅逶迤在紫檀足榻上,文着浅蓝凤尾的图案,一尾一尾的翎羽,是飞不起来的翅膀,在略显幽暗的暖阁内幽幽闪烁着月牙般的光泽。

这样的默然相对,于他是极难得的奢求。森严的宫里,他每每侍奉十二阿哥或五阿哥至翊坤宫,或是极偶然地陪伴她回宫,才能稍稍有较近的距离。这样的距离,已是极大的温暖。他忽然想起冷宫的岁月里,他有他的心无旁骛,如懿亦有如懿的心之所在,而那时,隔了一扇冷宫旧门,青苔深重的距离,他和她吹着同一阵风,看过同一片云彩,反而能随心所欲地说说心底事。

这样的记忆,如今看来,如同天山上的雪莲般弥足珍贵。

如懿的思绪仿佛悬挂在遥远的云端,渺渺不可触摸。许久,她忽然道:``凌云彻,除了当值之外,你还常出宫吧?本宫要托付你一件事。''

凌云彻旋即肃然,端正神色道:``微臣听命。''

如懿的眼眸明明沉静如水,却有着碎冰浮涌的凛冽:``田氏已死,但这件事本宫总是不安心。原本可以托付惢心去查,可她一个妇道人家,又身有残疾,总是不便。若你能在出宫时替本宫彻查此事,那便最好不过了。''

凌云彻心领神会:``微臣知道田氏尚有一子,爱之逾越性命。或许可以从他身上探知一二。''

如懿松了一口气,眸中闪过一点感激之意:``多谢你。这件事很难,或许已经死无对证,或许不小心还会让你牵涉其中,有损你的青云之路。你肯帮本宫,是成全了本宫与十三阿哥一番母子之情。若真的到田氏为止再无任何隐情,那么十三阿哥在九泉之下,也可以稍稍瞑目。''她再度郑重谢过,``在宫中近乎半生,本宫可以信赖的人不多,可以托付的人更不多。幸好还有你和愉妃。凌云彻,多谢。''

凌云彻微微一震,似是被她最后的一声呼唤触动,疏朗的眉目间骤然有了一丝难以言喻的温柔。情思空白的须臾,他忽然闻到了一缕淡淡的梅香,清芬馥郁,幽幽间教人心醉神驰。他分不清那幽醉的暗香来自何方,他只是一心一意地盼望,哪怕能够暗香如故,也不要有零落成泥碾作尘的那一日。

他不知哪里来了这样大的勇气,抬起头望着她,专注地,目光明朗而清澈。他的声音沉郁朗朗:``微臣没有别的办法。从前冷宫岁月,彼此落魄,还可以相互关照。如今云泥之别,微臣能做的,只有守在宫门外不远不近的距离守护娘娘,或是偶尔伴随娘娘身边,踏着娘娘的足印去走娘娘走过的路,读着娘娘爱读的诗词,看着娘娘喜欢的梅花,微臣才觉得,与娘娘之间的距离可以没有那么远。''

心底的冷漠,仿佛被这些话语一一震动,漾起微微地涟漪,闪着零星的银色的光晕,如春日的樱花散落于湖面。那种轻触的温柔,也是震动。

她恍惚地想,是多久以前,曾经有人也对她说过这样绵而暖的话。

二人正坐着说话,庭院中骤然有响亮的脚步声响起。如懿听得动静,不由得抬起头来。

三宝在外头欢欢喜喜道:``十二阿哥回来了。瞧小脸儿红的,别是冻着了吧?来,奴才替您烧个暖炉暖暖。''

却是听得乳母嬷嬷们簇拥着永璂进来,请了安道:``皇后娘娘万福金安。''

永璂亦跟着道:``额娘万安。''他一说罢,扑入如懿怀中,扭股糖似的拧着。

如懿搓着他的小手,笑嗔道:``越发没规矩了。手这么冷,快下去添件衣服。''

永璂点点头:``儿臣折了额娘最喜欢的梅花,额娘记得要看啊。''

如懿含笑看他跟着乳母去了偏殿。寒兰忙起身道:``也不知十二阿哥的话说得圆满不圆满?臣妾去瞧瞧。''如懿见她急急出去,裙摆都闪成了一朵花儿,轻轻摇摇头,复又低首去理那千丝万缕、色彩纷呈的丝线。

如懿见凌云彻站在门边,不觉微笑:``凌大人来了。''她唤过容珮:``给凌大人看座。''

凌云彻手里抱着大束的白梅,一时不便坐下。那些梅枝显然是精心挑选过,傲立的舒枝之上每朵梅花都是欲开未开的姿态,盈然待放,还有脉脉细雪沾染。只是殿中温暖,那细雪很快化作晶莹水珠,显得那朵朵白梅不着尘泥,莹洁剔透。

如懿微微失笑:``瞧本宫糊涂了,你抱着这些梅花,如何能坐下。''她显然被这些清洁莹透的花朵吸引,眸中微有亮色,``如今翊坤宫的人不打出去,虽是冬日,好久不见梅花了。''

容珮接过凌云彻手中的花,抿嘴一笑,欢喜道:``凌大人有心了。我们娘娘最喜欢梅花了。''

凌云彻将花递到容珮手里,看她抱了花朵去偏殿寻合适的花瓶,方才不好意思地笑笑:``是十二阿哥的一片心意,微臣只是帮十二阿哥摘了送来。希望皇后娘娘看在十二阿哥的孝心上,可以稍稍展颜。''

如懿欣慰道:``永璂很是孝顺。''

如懿偏着头,髻边一直鎏金碧玉瓒凤钗上垂落一串白玉,那玉色洁白,与她苍白的面孔殊无二致。她的形容清减了不少,淡妆素容的样子更显出眉目间难掩的一丝忧郁。凌云彻不知怎的,就觉得心口微微哆嗦,徒然酸楚不已。他情不自禁道:``皇后娘娘身子可养得好些了么?一直惦记着,也不能\ldots{}''他觉得自己说得不恰当,赶紧道:``其实皇上也惦记着。''

如懿淡淡一笑,那笑容像是浮在碎冰上的阳光,细细碎碎的,没有丝毫暖意:``境遇再坏,坏得过从前咱们在冷宫的那个时候么?本宫不会不要自己的身子,一定会养好的。''

凌云彻面庞上紧绷的弧度随着这句话而松弛下来:``皇后娘娘喜欢梅花就多看看。微臣也喜欢梅花。''

如懿注目于那些洁白无尘的花朵,口中不经意道:``难得听你说喜欢什么花儿草儿的。''

凌云彻安静片刻,道:``梅花已开,寒冬虽在,但也快是春天了。微臣知道皇后娘娘喜欢梅花,所以新学了一首诗,在娘娘面前班门弄斧了。''

如懿颇有兴致,长长的睫毛扬起,眸中有星子般的亮色:``你也学诗了?''

凌云彻有些难为情:``从前好歹上过几年私塾。皇后娘娘别笑话微臣。''他清一清嗓子,朗然念道:``冰雪林中著此身,不同桃李混芳尘。忽然一夜清香发,散作乾坤万里春。''

翊坤宫的暖阁宽阔良深,几近无声的静谧让空气里有种凝固的感觉,几乎能听清同掐丝珐琅八角炭盆里红箩炭``哔剥''燃烧的轻响。嗯,那种轻响,也是温热的,如同他此刻的心情。他不是不知道她这些日子的清冷幽闭,无数次想要寻个机会来看看她,哪怕只是说上几句话,就如当年在冷宫一般。可是人在跟前,他能想到的,竟是幼年时学过的这首诗。他不知道自己是怎么说出这些话的,或许是这个寒冷的冬日颇为应景,或许是那束白梅正好勾起了他封闭而压抑的情思。他暗暗自嘲,果然自己是不擅长安慰别人的人,连找一首写他喜欢的梅花的诗,也是这样简单而朴素。

如懿的声线清凌凌的,若不细听,几乎难以察觉那一丝即将痊愈的沙哑。她极客气地道:``是王冕的《白梅》,和眼前这束花倒应景,难为你记得。有心了。''

凌云彻一脸诚挚,动容道:``微臣知道自己是个粗人,但冬去春来,只是一瞬之间,还请娘娘暂且忍耐。''他挠了挠额头,苦苦思索片刻,眼中骤然一亮,如熠熠的火苗,``微臣还背过一首,前头不大记得了,但后面几句真是好,微臣看过久久记在心里。`横笛何愁听,斜枝倚病看。逆风如解意,容易莫摧残。'微臣也希望逆风解意,让娘娘能顺心如意。''

如懿的笑意渐渐淡下去,成了幽微一抹,仿若落日时分即将被夜色吞没的最后一缕霞光:``你的好意本宫心领了。但是逆风如何能解意,只盼自己熬得住风势强劲,莫被容易摧残罢了。''

他微微抬首,不敢直视着如懿,只是以眼角的余光瞥见她梅子色缀绣银丝梅朵紫狐大衣,那样暗沉的红的底色像是展不开的一个笑颜,凝在了那里,并无一丝欢喜的气息。连那银丝绣簇的梅花,也像一滴滴斑驳的泪痕,闪着剔透的水光。她长长的裙幅逶迤在紫檀足榻上,文着浅蓝凤尾的图案,一尾一尾的翎羽,是飞不起来的翅膀,在略显幽暗的暖阁内幽幽闪烁着月牙般的光泽。

这样的默然相对,于他是极难得的奢求。森严的宫里,他每每侍奉十二阿哥或五阿哥至翊坤宫,或是极偶然地陪伴她回宫,才能稍稍有较近的距离。这样的距离,已是极大的温暖。他忽然想起冷宫的岁月里,他有他的心无旁骛,如懿亦有如懿的心之所在,而那时,隔了一扇冷宫旧门,青苔深重的距离,他和她吹着同一阵风,看过同一片云彩,反而能随心所欲地说说心底事。

这样的记忆,如今看来,如同天山上的雪莲般弥足珍贵。

如懿的思绪仿佛悬挂在遥远的云端,渺渺不可触摸。许久,她忽然道:``凌云彻,除了当值之外,你还常出宫吧?本宫要托付你一件事。''

凌云彻旋即肃然,端正神色道:``微臣听命。''

如懿的眼眸明明沉静如水,却有着碎冰浮涌的凛冽:``田氏已死,但这件事本宫总是不安心。原本可以托付惢心去查,可她一个妇道人家,又身有残疾,总是不便。若你能在出宫时替本宫彻查此事,那便最好不过了。''

凌云彻心领神会:``微臣知道田氏尚有一子,爱之逾越性命。或许可以从他身上探知一二。''

如懿松了一口气,眸中闪过一点感激之意:``多谢你。这件事很难,或许已经死无对证,或许不小心还会让你牵涉其中,有损你的青云之路。你肯帮本宫,是成全了本宫与十三阿哥一番母子之情。若真的到田氏为止再无任何隐情,那么十三阿哥在九泉之下,也可以稍稍瞑目。''她再度郑重谢过,``在宫中近乎半生,本宫可以信赖的人不多,可以托付的人更不多。幸好还有你和愉妃。凌云彻,多谢。''

凌云彻微微一震,似是被她最后的一声呼唤触动,疏朗的眉目间骤然有了一丝难以言喻的温柔。情思空白的须臾,他忽然闻到了一缕淡淡的梅香,清芬馥郁,幽幽间教人心醉神驰。他分不清那幽醉的暗香来自何方,他只是一心一意地盼望,哪怕能够暗香如故,也不要有零落成泥碾作尘的那一日。

他不知哪里来了这样大的勇气,抬起头望着她,专注地,目光明朗而清澈。他的声音沉郁朗朗:``微臣没有别的办法。从前冷宫岁月,彼此落魄,还可以相互关照。如今云泥之别,微臣能做的,只有守在宫门外不远不近的距离守护娘娘,或是偶尔伴随娘娘身边,踏着娘娘的足印去走娘娘走过的路,读着娘娘爱读的诗词,看着娘娘喜欢的梅花,微臣才觉得,与娘娘之间的距离可以没有那么远。''

心底的冷漠,仿佛被这些话语一一震动,漾起微微地涟漪,闪着零星的银色的光晕,如春日的樱花散落于湖面。那种轻触的温柔,也是震动。

她恍惚地想,是多久以前,曾经有人也对她说过这样绵而暖的话。

夕阳笼罩了整个紫禁城,暮霭宛如潺湲流动的河水,流溢过此起彼伏的殿台楼阁;流溢过飞翘的檐角,盘踞的鸱吻;流溢过每一座寂寞而无声的宫殿。殿内静得恍若一池秋水。如懿的两腮粉得好似蘸水桃花一般,唇角抿出了一丝了然的笑意。旋即,她便觉得那是不应当的,连笑也是不合宜的。她蹙了蹙眉心,静静地挤出疏离而客气的神色,将他显而易见的温情以自己疏冷而高华的母仪姿态隔绝于外。

红尘紫陌,俗世迢迢,他自有他的举案齐眉,她亦有她的难以割舍。他与她之间,是错了季节的花朵,连一丝绽放的可能也无。

\hypertarget{ux7b2cux4e8cux5341ux7ae0-ux5f02ux53d8}{%
\chapter{第二十章 异变}\label{ux7b2cux4e8cux5341ux7ae0-ux5f02ux53d8}}

须臾的死寂似乎并不给殿中的两人少许回旋的余地,反而有重重逼仄的畏惧从如懿的心底溢出。她的理智和直觉提醒着她这些温情背后可能的残酷后果,并且在她目睹凌云彻渐渐变成云霞红的耳根和瞥见帘外不知何时进来袖手而立的海兰时,那股畏惧与警惕更加凛冽地如冰雪覆上发烫的额头,灌入脑缝。

她的身份,是这个帝国所有者的女人。永不能改变,至死也不能!

如懿的神情瞬间庄肃而冷然,含有几分矜持之意:``多谢大人关怀。昔年彼此照顾的情谊,本宫与愉妃都铭记在心。''

海兰听得提到自己名字,不觉款款上前,软声道:``自然了,皇后娘娘念及旧恩,时时事事不忘提携凌大人,凌大人也要知恩图报,不要陷娘娘于危墙之下。''

海兰的容色安宁平和若平湖秋月,却字字句句都落在身份尊卑的天渊之别上。凌云彻眼中的火焰如被泼了凉水,瞬息黯淡不见。他退后一步,依足了规矩道:``愉妃娘娘字字句句,微臣都懂得,不敢逾越忘恩。''

海兰沉着而矜持地颔首,保持着优雅的仪态:``有凌大人这句话,本宫与皇后娘娘也可安心了。''她端然一笑,``对了。凌大人成日忙碌于宫中,难得出宫,既不要忘了皇后娘娘吩咐的差事,也别忘了安慰家中娇妻。毕竟,那是皇上钦赐的姻缘呢。''

凌云彻克制地黯然一笑,衔住眼底的一丝苍凉孤绝,躬身告退。

海兰见他出去,方在如懿身边坐下,屏息静气,凝视不语。

如懿知她心思,便道:``有什么话,但说不妨。''

海兰不自觉地靠近如懿,眼里有沉浮不定的疑惑:``姐姐有的不觉得凌侍卫对您格外亲厚?''

如懿的目光停驻在她身上,伸手掠去她鬓边发丝所沾的一星浮尘,淡淡一哂:``我与他彼此救助扶持,自然格外亲厚。''

海兰斟酌着词句,仿佛极难启齿:``姐姐,我的意思是,凌侍卫对姐姐的亲厚,更多的是\ldots 男女之情。''

如懿蹙眉:``不要胡说,凌云彻已有妻室。''

``但他们夫妻并不和睦。''海兰微微迟疑,见如懿眸中颇有探询之意,索性道:``听说茂倩仗着是满军旗上三旗的出身,并不怎么将凌云彻放在眼里,所以夫妻间屡屡争执不睦。''

如懿不以为意,浅浅一笑漾起几分感慨:``哪有夫妻不争执吵闹的,外头人家也有外头人家的好处,夫妻拌嘴也是当着面儿。不比宫里,夫妻君臣,什么都搁在心里,思量了许多遍也不能只说出来。''

海兰盯着如懿,轻声细语间夹着犀利的锋锐:``我要说的不是这个。姐姐聪慧,难道真的从未察觉凌云彻对姐姐有意。姐姐,难道您一点儿也不知?''

如懿清婉一笑,向着海兰道:``许多事,你若不想知道,便永远也不会知道。有时候视而不见,比事事察觉要自在许多。''

海兰轻嘘一声:``姐姐果然是知道的。''她眼中多了一丝轻快的笑意,``因为姐姐不喜欢,才故作不知,对不对?''

如懿轻叹:``我已暗示过,要他善待妻室。我自有我自己曾经中意之人。''

海兰微微一怔,继而笑:``姐姐是说皇上?多少年夫妻了,眼看着新人蜂至,姐姐还说这样的话。''

如懿敛容,沉静的容色如带雪的梅瓣,莹白中有薄薄的寒透之意:``海兰,我知道你要说什么。在我嫁给皇上为侧福晋为妾室的那一日,我就知道皇上身边永远不会只有我一个女人,他所爱恋怜惜的,也绝不只我一个。自从成为皇后,我便更明白这个道理。所以我可以容忍,容忍自己在年华老去的同时皇上的身边有越来越多的女人,因为我知道我争不了,也争不到,只是枉然而已。不止是皇后的身份束缚着我,更是因为我比谁都明白,愿得一人心,在这个宫里是永世不可得的梦想。''

海兰微微扬眸,凝视着如懿:``所以姐姐就可以这样忍让到底?''

悠长的叹息静默得如同贴着金砖旋过的带着雪子的风,如懿望着朱碧墙上自己削薄的侧影,那暗淡的影色也不免有憔悴零落之意:``皇上身边的人再多,我们毕竟是少年夫妻。哪怕我什么都不求,亦求一点儿信任,一点儿尊严,仅此而已。这,便是我的底线。''

``人传欢负情,我自未尝见。三更开门去,始知子夜变。''海兰鬓边的一朵碎玉银丝珠花随着她臻首轻摇,颤颤若风中细蕊,``皇上对姐姐的信任和尊重,在封后那一日,连我也差点儿相信了。可是如今呢,那些所谓的信任和尊重,能换来多姐姐一句丧子之痛的安慰么?还是姐姐一定要到覆水难收那一日,才能真正死心?''

如懿默然不语,只是看着海兰鬓边那一朵珠花出神。海兰虽然向来无宠,但终究身在妃位,儿子又得皇帝欢心,所以也略略装饰。且皇帝登基多年,性子里喜好奢华的本意渐渐流露,也看不惯嫔妃过于简素,所以海兰饰在燕尾上的一朵翠翘明珠压发,那明珠便也罢了,不过是拇指大的光润浑圆一颗,有目眩迷离的光晕,那翠翘是用上好的翠鸟的羽,且是软翠,细腻纤柔。

那样雍容而精致的翠蓝,映着她白净的容颜,有泠泠的冷光翠华,让人无端便生了清冷涩意。她唇边有酸楚的笑意,如秋风里枝头瑟瑟的叶,轻轻吟道:``弹破庄周梦,两翅驾东风,三百座名园,一采一个空。谁道风流中,唬杀寻芳的蜜蜂。''她的声音脆脆的,落在殿中有空响的回音,``姐姐熟读宋词元曲,自然知道这支曲子。''

如懿的笑意萧疏得如一缕残风,``你是说,我们爱的男人,不过是一只寻芳花间不知疲倦的大蝴蝶?''

海兰的笑容转瞬如初雪消逝:``姐姐,那是您爱的男人,不是我们。''她的花语清晰如薄薄的刀锋,划下不可逾越的冷淡,``我只是皇上的妃妾,与他同眠数载,育有一子,仅此而已。''

在连续失去了爱女和幼子之后,如懿再粗心,亦发现了衰老的不期而至。那是一样无法抗拒的东西,原本她提着一口气,以为可以摒得住失去孩子的伤心,以为可以用佛经偈文来安抚自己的痛心与责备,可是这样日里夜里忍着泪,清晨醒转时,还是能抚摸到泪水浸淫过枕被的痕迹。

红丝穿露珠帘冷,百尺哑哑下纤绠。翊坤宫寂寥冷清的日子里,时光仿佛机杼声声中经穿维度的枯燥与死板。如懿愈加懒于梳妆,只得在逢十日嫔妃不得不拜见的日子里,她才勉强打起精神草草应对。对着妆镜时,哪怕光线再晦暗,她都能敏捷地发现隐蔽在发间的银丝,原本只是一丝,一根,渐渐如秋霜掩映后的枯蓬,一丛一丛密密地长出。当荣配不得不一次次用桑叶乌发膏为她染黑发色的时候,如懿亦颓然:``掩住了白发,眼角的细纹又该如何呢?''

那细细的纹路,仿佛是轻绵的蛛网,幼细无声的蔓延在眼角和面颊。再多的脂粉,也敷不上干涩的肌肤,那是昨夜思子的泪痕划过,无法再吃住脂粉的滑腻与香润。

闲来无事时,太后也会偶尔来看她,亦会温言安慰:``皇后莫要如此伤心了。''

这是如懿与太后之间难得的平静而略显温情的相处。自从端淑长公主归来,太后仿佛一夜之间变回了一个慈爱而温和且无欲无求的妇人,含饴弄孙,与女儿相伴,闲逸度日。她身上再没有往日那种精明犀利的光彩,而是以平和的姿态,与她闲话几句。自然,太后也会带来皇帝的消息。虽然几乎不再见面,皇帝也有慰藉的话语传来。

她并不曾体会到那些话语之后的温度,因为这样的话,客气、疏远、矜持有度,太像是不得不显示皇家礼仪的某种客套。她只是仰视着太后平静的姿容,默默地想,是要行经了多少崎岖远途,跋涉了多少山重水复,才可以得到太后这般光明而宁和的手梢。

虽然有太后这样的安慰,也有皇帝的话语传来,但皇帝终究未曾再踏入翊坤宫中。孩子的死,终究已经成了他们之间难以解开的心结。自然,比之一个中年丧子丧女的哀伤女子,他更乐意见到那些年轻的娇艳的面庞,如盛开的四时花朵,宜喜宜嗔,让他轻易忘却哀愁。而她,只能在苔冷风凉的孤寂里,紧紧抱住唯一的永璂,来支撑自己行将崩溃的心境。

此时的热闹,只在嬿婉的永寿宫。哪怕是冰天雪地时节,那儿也是春繁花事闹得天地。嬿婉正怀着她的第一个孩子,开始她真正踌躇满志的人生。无论腹中是男孩还是女孩,都意味着曾经以为不能生育的梦魇的过去。她终于能抬头挺胸,在这个后宫厮杀,惊雷波动之地争得自己的一席之位。

真的,多少次午夜梦回,嬿婉看着锦绣堆叠的永寿宫,看着数不尽的华美衣裳、绫罗珠宝,寂寞地闪耀着死冷的华泽。她死死地抓着它们,触手冰凉或坚硬,却不得不提醒着自己:这些华丽,只是没有生命的附属,她只有去寻得一个有生命的依靠,才不至于在未来红颜流逝的日子寂寞地芳华老去,成为紫禁城中一朵随时可以被风卷得凌乱而去的柳絮。

哪怕是皇帝在身边的夜里,她同样是不安心的。此时此刻自己唯一的男人在自己身边,下一时下一刻,他又会在哪里。就好像他的心,如同吹拂不定的风一般,此刻拂上这朵花枝流连不已,下一刻又在另一朵上。尤其是年轻的妃嫔们源源不断地入宫,她更是畏惧。总有一日,这个男人会成为一只盲目的蝴蝶,迷乱在花枝招展之中。

所以,当月光清冷而淡漠地一点一点爬过她的皮肤之时,她在伸手不可触摸的黑夜,一次一次闭紧了喉舌,紧抱住自己:``一定,一定要有一个自己的孩子。''

所以,这一次的有孕,足以让嬿婉欣喜若狂。

嬿婉在这欣喜里仔细打量着东西六宫的恩泽如沐。如懿的恩宠早已连同永璟的死一同消亡,即便有皇后的身份依凭容颜和精力到底不如往日了。昔日得宠的舒妃也跟着她的孩子一起香消玉殒,连宿敌嘉贵妃都死了。颖嫔和忻妃虽然得宠,到底位分还越不过她去。因此,嬿婉几乎是毫无后顾之忧地在宫中安享着圣宠的眷顾。

这是她最春风得意的时刻,连宫人们望向她的目光都带着一种深深的艳羡与敬慕。那才是万千宠爱于一身的宠妃啊。

比之于永寿宫的门庭若市,翊坤宫真真是冷寂到了极点。除了海兰还时时过来,绿筠和忻妃也偶有踏足,除此之外,便是年节时应景的点缀了。并且凌云彻并没有再入翊坤宫来,大约是没有合适的时机,或是御前的事务太过繁重,容不得他脱开身来,渐渐地也没有了消息。而这些日子,因着时气所感,永璂的身体也不大好,逢着一阵春潮反复便有些发热咳嗽,如懿一颗心悬在那里,便是一颗也不能放松。

是太知道不能失去了。璟兕、永璟,一个个孩子连着离开了自己。她是一个多么无能为力的母亲,所以,便是违反宫规,她也不得不求了太后,将永璂挪到了自己身边。

太后自然是应允的,只是望着如懿哀哀的神色,生了几分怜悯之意:``皇后,永璂既然不大好,何不求皇帝将孩子挪去你身边照顾?见面三分情,说说孩子的事,夫妻俩的感情多少也能扭转些。你与皇帝只有这一个永璂了,皇帝不会不在乎的。''

心底的酸楚与委屈如何能言说,更兼着积郁的自责,如噬骨的蚁,一点一点细细咬啮。如懿只能淡淡苦笑:``儿臣不是一个好额娘,如何再敢惊动皇上,只求能照顾好永璂,才能稍稍安心。''

太后静静凝视她片刻:``有些事,皇上不肯迈出那一步,难道你就不肯么?哀家看得出来,皇帝对你并非全不在意。''

仿佛是谁尖利的指甲在眼中狠狠一戳,逼得如懿几乎要落下泪来。她只是一味低首,望着身侧黄花梨木花架上的一盆幽幽春兰,那细长青翠的叶片是锋锐的刃,一片一片薄薄地贴着肉刮过去。良久,她亦只是无言。不是不肯倾诉,而是许多事,忍得久了,伤心久了,不知从何说起,也唯有无言而已。

太后无法可劝,也不愿对着她愁肠百结,只得好言嘱咐了退下。还是福珈乖觉,见如懿这般,便向着太后道:``太后娘娘,恕奴婢直言,只怕皇后心里有苦,却是说不出来。''

太后沉着脸看不出喜怒,徐徐道:``皇后是苦,从前一心一意对付着孝贤皇后和慧贤皇贵妃,以为事儿散了,淑嘉皇贵妃又挑着头不安分。如今淑嘉皇贵妃去了,孩子又接二连三地出事。也罢,说来本宫也不大信,从前孝贤皇后什么都有,何必事事跟嫔妃过不去,又说是淑嘉皇贵妃的挑唆。难道哀家真是老了,许多事看不明白了么?''

福珈忙忙赔笑道:``太后是有福之人,哪里有空儿成日去琢磨她们那些刁钻心思。这么多年,怕是看也看烦了。''

太后叹道:``从前哀家是不大理会,由着这趟浑水浑下去,如今看来,皇后自己也福薄。''

福珈道:``宫里是趟浑水,可太后不是还有令妃娘娘这双眼睛么?''

太后默默出了会儿神,缓缓道:``那是从前。如今哀家有女儿在身边安享天伦,理这些做什么。留着令妃,也是怕再生出什么事端,防着一手罢了。但令妃那性子,表面乖顺,内里却自有一套,也不是个好拿捏的。哀家且由着她去,省得说得多了,反而叫她留了旁的心思。''

福珈口中答应着,眼里却是闪烁:``失了儿女是天命,嫔御不谐是常理,这都是说得出来的苦。可皇后她\ldots{}''

太后从细白青瓷芙蓉碟里取了一块什锦柳絮香糕,那碧绿莹莹的糕点上粘着细碎的白屑,真如点点柳絮,雪白可爱。太后就着手吃了小半,睨了福珈一眼:``有话便直说,闪不着你的舌头。''

福珈忙恭谨道:``太后这几日嫌春寒不打出去,岂不知宫里正流传着一首诗呢。''

太后垂首拨弄着檀色嵌明松绿团幅纹样蹙金绣袍的鎏金盘花扣上垂落的紫翡翠鸟明珠流苏,笑容淡淡地问:``什么诗?''

福珈笑了笑,不自然地摸了摸鬓边一枝烧蓝米珠松石福寿花朵,有些僵硬地学者背诵道:``独旦歌来三忌周,心惊岁月信如流。断魂恰值清明节,饮恨难忘齐鲁游。岂必新琴终不及,究输旧剑久相投。圣湖桃柳方明媚,怪底今朝只益愁。''

太后面色一冷,牵扯得眉心也微微一蹙:``这诗像是皇帝的手笔,是怀念孝贤皇后的么?''

福珈恭声道:``太后娘你明鉴,正是皇上怀念孝贤皇后的旧诗。只不过诗中所提的三忌周,是指孝贤皇后皇后崩逝三年的时候。''她悄悄看一眼太后的神色,不动声色道,``所以奴婢说,是旧诗。''

太后静默片刻,扯出矜持的笑容:``孝贤皇后崩逝三年,那个时候,如今的皇后才与皇帝成婚吧。立后的是皇帝的意思,写下`岂必新琴终不及,究输旧剑久相投'的也是皇帝的手笔。旧爱新欢两相顾全,这才真真是个多情的好皇帝呢。''

福珈见太后笑得冷寂,便道:``孝贤皇后如见此诗,想来九泉之下也颇安慰。孝贤皇后生前是得皇上礼遇敬重,但令妃所得的儿女情长,鬓边厮磨怕也不多。有句老话便是了,妻不如妾,妾不如偷。未曾想人去之后,皇上却写了那么多诗文祭悼,可见皇上终究是念着孝贤皇后的。''

``皇帝一生之中,最重嫡子,自然也看重发妻。最不许人说他薄情寡义。''太后薄薄的笑意倒映在手边一盏暗红色的金橘姜蜜水里,幽幽不定。此时,斜阳如血,影影绰绰地照在太后身边身形之后,越发有一种光华万丈之下的孤独与凄暗。``只是写写诗文便可将深情流转天下,得个情深义重的好名声,真是上算!只是哀家虽然对如今的皇后不过可可,可皇帝那诗传扬出来,哀家同为女子,也替皇后觉得难堪。且所谓妻不如妾,妾不如偷,本是说天下男子好色习性,放在咱们皇帝这里,却又是多了一层忌惮皇后与他并肩分了前朝后宫的权位之事了。你便看不出来么,皇后还是贵妃皇贵妃的时候,皇帝待她到底亲厚多了。反而一成皇后,却有些疏冷了。''

福珈亦有些不忍:``是。本来皇后就比不得嫔妃能放下身段争宠,又事事能与皇上商量说得上话,不必那么事事遵从。皇上为了十三阿哥之死疏远了皇后,如今又有这诗传扬出来,也难怪皇后不愿与皇上亲近了。''

殿中点着檀香,乃是异域所贡的白皮老山香,气味尤为沉静袅袅。熏香细细散开雾白清芬,缠绕在暗金色的厚缎帷帐上,一丝一缕无声无息,静静沁入心脾。闻得久了,仿佛远远隔着金沙淘澄过的沉淀与寂静,是另一重世界,安静得仿佛不在人间。太后搁下手里的糕点,淡淡道:``这糕点甜腻腻的,不大像是咱们小厨房的手艺。''

福珈忙转了神色赔笑道:``真是没有太后娘娘不知道的。这柳絮香糕是令妃娘娘宫里进献的。也难为了令妃娘娘,自个儿是北地佳人,却能找到那么好的手艺做出这份江南糕点来。咱们皇上是最爱江南春色的,难怪皇上这么宠着她。''

殿中开阔深远,夕阳斜斜地从檐下如流水蜿蜒而进,散落游蛇般地暗红色光影。太后的面孔在残阳中模糊而不分明:``说来,令妃也算个有心人。哀家调教过那么多嫔妃,她算是一个能无师自通的。从前因着家中教养的缘故略显粗俗些,如今一向要强,也细致得无可挑剔了。做起事来,往往出人意表却更胜一筹。''

福珈不知太后这话是赞许还是贬低,只得含含糊糊道:``那都是太后教导有方。''

``是她自己有心。哀家没有点拨的事儿。令妃都能自己上赶着做在前头了。她日日陪在皇帝身边,皇帝写的诗,她能不知?有意也好,无意也罢,帝后不合,总是她渔翁得利。哀家只是觉得,令妃有些伶俐过头了。''太后轻轻一嗅,似是无比沉醉,``今儿吩咐你点的是白皮老山香,檀香之中最名贵的。福珈,知道哀家为何多年来只喜欢檀香一品么?''

福珈思忖着道:``檀香性收敛,气味醇和,主沉静空灵之味。''

太后的唇角泛起一朵薄薄的笑意:``诸香之中,唯有檀香于心旷神怡之中达于正定,证得自性如来,最具佛性。''她双眸微垂,冷冷道:``只是哀家在后宫中辗转存货一生,看尽世情,这个地方,有人性便算不错,往来都是兽性魔性之人,乃是离佛界最远之地。你岂不知,本在天上之人最不求极乐世界而辛苦求拜者,都是沉沦苦海更甚为身在地狱之人,所以你别瞧着后宫里一个个貌美如花、身披富贵,都是一样的。''

福珈有些不知所措:``好端端的,太后说这些做什么。您是福寿万全之人,和他们不一样。''

``都是一样的。今日的她们,上至皇后,下至嫔妃,在她们眼里,只有到了哀家这个位子才算求得了一辈子最后的安稳,所以她们拼尽全力都会朝着这个位子来。令妃固然是聪明人,懂得在皇帝和皇后如今的冷淡上再雪上加霜一笔。但,哀家的女儿已经都在膝下承欢,哀家只希望借她的耳朵、她的眼睛多知道些皇帝,以求个万全。如今她的手伸得这么快,那么长,倒教哀家觉得此人不甚安分。''

福珈想了想道:``奴婢想着,令妃到底没什么家世,因为这个才得了皇上几分爱怜信任。也因为这个,她翻不过天去,咱们不必防范她什么。太后求了多年的如今都得了,何必多理会后宫这些事。儿孙自有儿孙福,您操心什么,且享自己的清福便是。''

\hypertarget{ux7b2cux4e8cux5341ux4e00ux7ae0-ux6d77ux5170}{%
\chapter{第二十一章
海兰}\label{ux7b2cux4e8cux5341ux4e00ux7ae0-ux6d77ux5170}}

冬日时光便这么一朵朵绽放成了春日林梢的翡绿翠荫。今年御苑春色最是撩人,粉壁花垣,晴光柔暖,春心无处不飞悬。却原来都是旁人的热闹,旁人的锦绣缀在了苍白无声的画卷上,绽出最艳最丽的锦色天地。

容珮长日里见如懿只一心守着永璂,呵护他安好,余事也浑不理会,便也忍不住道:``皇后娘娘,皇上倒是常常唤奴婢去,问起十二阿哥的情形呢。只是奴婢笨嘴拙舌的,回话也回不好。奴婢想着,皇上关怀十二阿哥,许多事娘娘清楚,回得更清楚呢。''

如懿低头仔细看着江与彬新出的一张药方,不以为意道:``本宫不是不知,本宫往太后处请安时,皇上也偶来探望永璂。永璂病情如何,他其实都一清二楚。''

容珮见如懿只是沉着脸默默出神,越发急切道:``皇后娘娘,恕奴婢妄言一句,如今十二阿哥这么病着,娘娘大可借此请皇上过来探视,见面三分情,又顾着孩子,娘娘和皇上也能借机和好了。''

如懿心下一酸,脸上却硬着,并无一丝转圜之意:``永璂这么病着,皇上若是自己不愿意本宫在时来,强求也是无用。''

容珮咬着唇,想要叹,却强忍住了,气道:``这些时日皇上只在令妃小主宫里,只怕也是令妃设计阻拦了。''

日影将庭中的桐树扯下笔直的暗影,这样花香沉郁的融融春色里,也有着寂寞空庭的疏凉。望得久了,那树影是一潭深碧的水,悄然无声地漫上,渐渐迫至头顶。她在那窒息般的脆弱里生了无限感慨:``想要来的谁也拦不住,你有何必这般替皇上掩饰?''

容珮素来沉着,连日的冷遇,也让她生了几分急躁,赤眉白眼着道:``可皇上若不来,岂不是和娘娘越来越疏远了?''

如懿闭上了眼睛,容珮的话是折断了的针,钝痛着刺进了心肺。她极力屏息,将素白无饰的指甲折在手心里,借着皮肉的痛楚定声道:``借孩子生病邀宠,本宫何至于此?''

容珮一时也顾不得了,扬着脸道:``不如此,不得活。这后宫本就是一个泥淖,娘娘何必要做一多出淤泥而不染的白莲?''她觑着如懿的神色,大着胆子道:``娘娘是后宫之主,但也身在后宫之中。许多事,无谓坚持。夫妻之间,低一低头又如何?''

``白莲花?''如懿自嘲地笑笑,在明灿日光下摊开自己素白而单薄的手心,清晰地手纹之中,隐着多少人的鲜血。她愧然:``身在混沌,何来清洁?满宫里干净些的,怕也只有婉嫔。可来日若洪水滔天,谁又避得过?所以本宫低头,又能换来什么?眼前一时安稳,但以后呢?以后的以后呢?''

容珮猛然跪下,恳求道:``不顾眼前,何来以后?皇后娘娘万不能灰了心,丧了意!''

``不灰心,不丧意。夫君乃良人,可以仰望终身!可本宫身为皇后,痛失儿女,家族落寞,又与夫君心生隔阂。本宫又可仰望谁?''一而再,再而三,魅力自持,但深深蹙起的眉心有难以磨灭的悲怆。如懿的眼底漫起不可抑制的泪光,凄然道:``如今满宫里传的什么诗你会不知?皇上拿着本宫与孝贤皇后比,且又有什么可比的。活人哪里争得过死人去!''

容珮从如懿指间抽过娟子,默然替她拭了泪,和声劝道:``皇上这诗听着是搓磨人的心,多少恩爱呢,只在纸头上么?但一时之语作得什么数?且这些年来,皇上想念孝贤皇后,心中有所愧疚,所以写了不少诗文悼念,娘娘不都不甚在意么?说来\ldots{}''她看一眼如懿,直截了当,``说来,这宫里奴婢最敬服的是愉妃小主。她若见了这诗,必定嗤之以鼻,毫不理会。所以论刚强,奴婢及不上愉妃小主半个指头。''

如懿听她赞海兰,不觉忍了酸涩之意,强笑道:``海兰生性洒脱,没有儿女情长的牵挂,这是她一生一世的好处。而本宫从前不在意,是心中有所坚持。经了这三番五次的事,本宫难道不知,自己只占了个皇后的名位,在皇上心里,竟是连立足之地都没有的。本宫还能信什么,坚持什么?不过是强留着夫妻的名分,勉强终老而已。''

``娘娘可勉强不得。您这心思一起,不知要遂了多少人的心愿呢。宫里多少人传着这诗,尽等着瞧咱们翊坤宫的笑话。奴婢已经吩咐了下去,不许底下的人露出败色儿来,也不许与人争执,只当没长耳朵,没听见那些话。''

如懿含了一丝欣慰,拍拍容珮的手,``你在,就是本宫的左臂右膀,让本宫可以全心全意照顾永璂。伺候过本宫的人,阿箬反骨,惢心柔婉,你却是最刚强不过的。有你,本宫放心。''

容珮着实不好意思:``奴婢哪里配得上皇后娘娘这般赞许。奴婢能挡的,是虾兵蟹将。娘娘得自己提着一口气,墙倒众人推。咱们的墙倒不得,只为了冤死的十三阿哥的仇还没报,十二阿哥的前程更辜负不得!''

心似被什么东西撞了一下,隐隐作痛,鼻中也酸楚。日光寂寂,那明亮里也带着落拓。这些日子里,面子上的冷静自持是做给翊坤宫外的冷眼看的,心底的痛楚、委屈和失落,却只能放在人影之后,缩在珠帘重重的孤寂里,一个人默默地吞咽。这样的伤绪说不得,提不得。一提,自己便先溃败如山。所以没有出口,只得由着它熬在心底里,一点点腐蚀着血肉,腐蚀得她蒙然发狂。``本宫知道,这诗突然流传宫中,自然是有古怪。可毕竟白纸黑字是皇上所写,否则谁敢胡乱揣度圣意。本宫自知不是发妻,却也不愿落了这样的口实,叫皇上自己比出高低上来。''

容珮望着如懿倔强而疲惫的容颜,静了半晌,怔怔地说不出话来,良久方叹息不已:``皇后娘娘,奴婢算是看得分明了。在这宫里,有时候若是肯糊涂些浑浑噩噩过去了,便也活得不错。或是什么也不求,什么也不怕,倒也相安无事。可若既要求个两心情长,念着旧日情分,又要维持着尊荣颜面,事事坚持,那么,真当是最最辛苦,又落不得好儿。''

仿佛是暮霭沉沉中,有巨大的钟声自天际轰然传来,直直震落与天灵盖上。曾几何时,也有人这样执意问过:``等你红颜迟暮,机心耗尽,还能凭什么去争宠?姑母问你,宠爱是面子,权势是里子,你要哪一个?''

那是年少青葱的自己,在电转如念间暗暗下定了毕生所愿:``青樱贪心,自然希望两者皆得。但若不能,自然是里子最要紧。''

不不不,如今看来,竟是宠爱可减,权势可消,唯有心底那一份数十载共枕相伴的情意,便是生生明白了不得依靠,却放不下,割不断,更不能信。原来所谓情缘一场,竟是这般抵不得风摧雨销。用尽了所有的力气,终于有了与他并肩共老的可能,才知道,原来所谓皇后,所谓母仪,所谓夫妻,亦不过是高处不胜寒时彼此渐行渐远的冷寂,将往日同行相伴的恩情,如此辗转指间,任流光轻易抛。

这夜下了一晚的沥沥小雨,皇帝宿在永寿宫中,伴着有孕而日渐缠绵的嬿婉。这一夜,皇帝听得雨声潺潺,一早起来精神便不大好。嬿婉听了皇帝大半夜的辗转反侧,生怕他有起床气,便一早悄声起来,嘱咐了小厨房备下了清淡的吃食,才殷勤服侍了皇帝起身。

宫女们端上来的是熬了大半夜的白果松子粥,气味清甘,入口微甜。只要小银吊子绵绵地煮上一瓮,连放了多少糖调味,亦是嬿婉细细斟酌过,有清甜气而不生腻,最适合熨帖不悦的心情。

皇帝尝了两口,果然神色松弛些许,含笑看着嬿婉日渐隆起的肚腹:``你昨夜也睡得不大好,还硬要陪着朕起身。等下朕去前朝,你再好好歇一歇。''

嬿婉半羞半嗔地掩住微微发青的眼圈,娇声道:``臣妾初次有孕,心内总是惶惶不安,生怕一个不小心,便不能有福顺利为皇上诞下麟儿,所以难免缠着皇上些,教皇上不能好好歇息。''

她的笑容细细怯怯的,好似一江刚刚融了冰寒的春水蜿蜒,笑得如此温柔,让人不忍拒绝。这样的温顺驯服教人无从防范,更没有距离,才是世间男子历经千帆后最终的理想。年轻时,固然不喜欢过于循规蹈矩、温顺得没有自我的女子,总将目光停驻于热烈灼艳的美,如火焰般明媚,却是灼人。而这些年繁花过眼,才只聪慧却知掩藏、驯服而温柔风情的女子,才最值得怜惜。恰如眼前的女子,分明有着一张与如懿年轻时有几分肖似的脸,却没有她那般看似圆滑实则冷硬的距离和冷不防便要刺出的无可躲藏的尖锐棱角。有时候他也在后悔,是不是当时的权衡一时砾偏颇,多了几分感性的柔和,才给了如懿可以与自己隐隐抗衡的力量,落得今日这般彼此僵持的局面。

这样的念想,总在不经意间缓缓刺进他几乎要软下的心肠,刺得他浑身一凛,又紧紧裹进身体,以旁人千缕柔情,来换得几宵的沉醉忘怀。皇帝深处臂膀,揽住她纤柔的肩,温柔凝睇:``你什么都好,就是凡事太上心,过于小心谨慎。朕虽然愿意多陪陪你,多陪陪咱们的第一个孩子,可是朕毕竟是国君,不可整日流连后宫。''

嬿婉娇怯怯地缩着身子,她隆起的肚腹显得她身量格外娇小,依在他怀中,一阵风便能吹倒似的。她脸上的笑意快撑不住似的,懂事地道:``皇上说得是,晋贵人也常常这般劝解臣妾,要臣妾以江上社稷为重,不要顾一时的儿女情长。晋贵人出身孝贤皇后母族,大方得体,有她劝着,臣妾心里也舒坦许多。''

皇帝安抚似的拍了拍她圆润明亮的脸庞:``难得晋贵人懂事,倒不糊涂。只是这话说的口气,倒是和当日孝贤皇后一般的正经。''他似有所触动,``为着璟兕之死,晋贵人和庆贵人从嫔位降下,也有许久了吧。朕知道,你是替她们求情。''

嬿婉寒星双眸微微低垂,弱弱道:``皇上痛失五公主和十三阿哥,晋贵人和庆贵人的错也是不能适时安慰君上的伤怀,失了嫔御之道。只是小惩大诫可以整肃后宫,但责罚过久过严怕也伤了后宫祥和。毕竟,晋贵人出自皇上发妻孝贤皇后的母族,庆贵人也是当年太后所选。''

皇帝听她软语相劝,不觉道:``这原该是皇后操心的事,如今却要你有身子的人惦记。罢了,朕会吩咐下去给晋贵人和庆贵人复了嫔位。''

嬿婉笑语相和,见皇帝事事遂愿,提着的一颗心才稍稍放下,又夹了一筷子松花饼,自己吹去细末,才递到皇帝跟前的碟中。那是一个黄底盘龙碟,上写段红``万寿如意''四字,皇帝的目光落在``如意''二字上,眼神便有些飘忽,情不自禁道:``如懿\ldots{}''

嬿婉心口猛地一颤,徒然想起昨夜皇帝辗转半晌,到了三更才朦胧睡去,隐约也有这么一句唤来。夜雨敲窗,她亦困倦,还当是自己听错了,却原来真是唤了那个人的名字。

嬿婉心头暗恨,双手蜷在阔大的滚榴花边云罗袖子底下,恨恨地攥紧,攥得指节都冒着酸意,方才忍住了满心的酸涩痛意,维持着满脸殷切而柔婉的笑容,柔声道:``前几日内务府新制了几柄玉如意,皇上还没赏人吧?臣妾这几夜总睡不大安慰,起来便有些头晕。还请皇上怜惜,赏赐臣妾一柄如意安枕吧。''

皇帝听她这般说,果然见嬿婉脂粉不施,脸色青青的,像一片薄薄的钧窑瓷色,越发可怜见儿的了。他有些怜惜地卧一握她的手腕:``身上不好还只顾着伺候朕?等下朕走了,你再好好歇歇,朕嘱咐齐鲁来替你瞧瞧。再者,若得空儿也少喝别人往来,仔细伤了精神应付。左右这几日你额娘便要入宫来陪你生产,你安心就是。''

嬿婉再四谢过,却见守在殿外的一排小太监里,似是少了个人,便问道:``一向伺候皇上写字的小权儿上哪里去了?这两日竟没见过他。''

皇帝的脸色瞬即一冷,若无其事道:``他伺候朕不当心,把许多不该他看见不该他留心的东西传了出去。这样毛手毛脚,不配在朕身边伺候。''

嬿婉暗暗心惊,脸上却是一丝不露,只道:``也是。在皇上身边伺候,怎能没点儿眼色,倒叫主子还迁就着他!''

皇帝慢慢喝下了一碗红枣银耳,和声道:``你怀着身孕,别想这些。这几日你额娘快进宫了吧?朕叫人备了些金玉首饰,给你额娘妆点吧。''

嬿婉喜不自胜地谢过,眼看着天色不早,方才送了皇帝离去。那明黄的身影在细雨蒙蒙中越来越远,终于成了细微一点,融进了雨丝中再不见踪影。嬿婉倚靠在镂刻繁丽的酸枝红木门边,看着一格一格填金洒朱的``玉堂富贵''花样,玉兰和海棠簇拥着盛开的富丽牡丹,是永生永世开不败的花叶长春。

那么好的意头,看得久了,她心里不自禁地生出了一点儿软弱和惧怕,那样的富贵不败到底的死物,她拼尽了力气抓住了一时,却抓不住一世。

这样的念头才转了一转,嬿婉冷不住打了个寒噤。春婵忙取了云锦累珠披风披在她肩上,道:``小主,仔细雨丝扑着了您受凉。''

嬿婉死死地捏着披风领结上垂下的一粒粒珍珠水晶流苏,那是上好的南珠,因着皇帝的宠爱,亦可轻易取来点缀。那珠子光润,却质地精密,硌得她手心一阵生疼。那疼是再清醒不过的呼唤,她费了那么大的心思才使得如懿和皇帝疏远,如何再能轻纵了过去。

就好比富贵云烟,虽然容易烟消云散,但能握住一时,便也是多一时就好。

也不知过了多久,皇帝早已远去,桌上残冷的膳食也一并收拾了干净。小宫女半跪在阁子里的红木脚榻上,细细铺好软茸茸的锦毯,防着她足下生滑。澜翠端了一碗安胎汤药上来,挥手示意宫人们退下,低声道:``安胎药好了,小主快喝吧。''

那乌沉沉的汤汁,冒着热腾腾的氤氲,泛着苦辛的气味,熏得她眼睛发酸。她银牙暗咬,拿水杏色娟子掩了口鼻,厌道:``一股子药味儿,闻着就叫本宫想起从前那些坐胎药的气味,胃里就犯恶心。''

澜翠笑色生生,道:``从前咱们吃了旁人的暗亏,自然恶心难受,却也只能打落牙和血吞。可如今这安胎药,却是别人求也求不来的,保佑着小主安安稳稳生下龙子,扬眉吐气呢。''

嬿婉被她勾得掌不住一笑,啐道:``胡说些什么?龙子还是丫头,谁知道呢?''

澜翠笑道:``小主福泽深厚,上天必然赐下皇子。哪怕是个公主,先开花后结果,也一定会带来个小阿哥的。''

嬿婉骄傲地抚着肚腹,莞尔道:``你说得是。来日方长,只要会生,还怕没有皇子么。''她微一蹙眉,那笑容便冻在唇角,``只是过两日额娘进宫,怕又要絮叨,要本宫这一胎定得是个皇子。''她说着变更烦心,支着腮不肯言语。

澜翠思忖着道:``小主与其担心这个,不如多留心皇上。方才早膳时,奴婢可瞧着皇上似乎又有些惦记着皇后娘娘了。''

嬿婉轻哼一声,拨弄着凤仙花染过的指甲,滟生生地映着她绯红饱满的脸颊:``有那首诗在,皇上纵然不以为意,但皇后心里会过得去么?是个女人都过不去呢。只可惜了小权儿,才用了他一回,便这么没了。''

澜翠替她吹了吹安胎药的热气,道:``皇上不是好欺瞒的人,有小权儿顶上去也不坏。奴婢会按着先前的约定,替他料理好家人的。''

嬿婉微微颔首,接过安胎药喝下:``那便好。你替本宫多留心着便是。''她想了想,又嘱咐道:``额娘喜欢奢华阔气,她住的偏殿,你仔细打理着吧。''

这一日苍苔露冷,如懿皮了一年半新不旧的棠色春装,隐隐的花纹绣得疏落有致,看不出绣的是什么花,只有风拂过时微见花纹起伏的微澜。她静静坐在窗下,连续数日的阴霾天气已经过去,渐而转蓝的晴空如一方澄净的琉璃,叫人心上略略宽舒,好过疾风骤雨,凄凄折花。

水晶珠帘微动,进来的人却是惢心。她的腿脚不好,走路便格外慢,见了如懿,眼中一热,插烛似的跪了下来,哽咽道:``奴婢恭请皇后娘娘万安,娘娘万福金安。''

如懿一怔,不觉意外而欣喜,忙扶住了她的手道:``惢心,你怎么来了?''

惢心如何肯起来,禁不住泪流满面道:``奴婢自从知道娘娘和十三阿哥的事,日夜焦心不安,偏偏不能进宫来向娘娘请安,只得嘱咐了奴婢的丈夫必得好好伺候娘娘。今日是好容易才通融了内务府进来的。''

如懿忙拉了她起来,容珮见了惢心,亦是十分欢喜,忙张罗着端了茶点进来,又叫三宝搬了小杌子请惢心坐下。惢心反反复复只盯着如懿看个不够,抽泣着道:``奴婢早就有心进宫来看望娘娘,一则生了孩子后身子一直七病八痛的,不敢带了晦气进宫;二则江与彬反复告知奴婢,娘娘身在是非里,只怕奴婢来再添乱。如今时气好些,奴婢也赶紧进宫来给娘娘请安。''

如懿拉着她的手道:``自你嫁人出宫,再要进来也不如从前方便。''她打量着惢心道,``你轻易不进宫来,这趟可是有什么要紧事?''

惢心神色一滞,看了看旁处,掩饰着喝了口茶道:``没什么要紧事,只是惦记着娘娘,总得来看一看才好。''

如懿与惢心相处多年,彼此心性相知,如何不知道她的意思,便指了指四周道:``如今我这里最冷清不过,容珮也不是外人,你有什么话直说便是。''

惢心听得如懿这般,眼看着四下里冷清,便不假思索道:``凌大人得娘娘嘱托,不敢怠慢,竭尽全力彻查了田氏之事,才发觉原来在娘娘怀着十三阿哥时,田氏的独子田俊曾经下狱,罪名便是宵禁后醉酒闹事,被打了四十大板,扔进了牢里。''

如懿疑道:``宵禁后除婚丧疾病,皆不得出行。田俊醉酒闹事,打过也罢了,怎么还关进了牢里?''

惢心道:``若是平日也罢了,凭着田氏在宫里的资历,费点儿银子也能把人捞出来。偏那一日是皇上的万寿节,可不是犯了忌讳。便是大罗神仙,也难救了。''

容珮听着,一时忍不住插嘴道:``既然难救,难不成眼下还在牢里?''

惢心摇头道:``凌大人也是多番打听了才知道,原来田俊被关了几个月,神不知鬼不觉地就被放了出来。''

容珮握紧双拳,焦灼道:``这么蹊跷?''

惢心点头道:``凌大人就是怕中间有什么关节,便找机会与田俊混熟了。两人喝了几次酒后田俊便发牢骚,说自己和他老娘倒霉,便是得罪了人才落到今日这个地步。凌大人故意灌醉了他再问,才知道当日田俊闹事,是和几个狐朋狗友在一块儿人家故意灌醉他。其中灌他最厉害的一个,便有远房亲眷在宫里为妃为嫔。他与他老娘,便是斗不过那个女人,才中了暗算。''

如懿的心像被一只大手紧紧揪住提了起来,冲口问道:``为妃为嫔?是谁?''

惢心的脸上闪过一丝难以置信的苦涩,屏息片刻,重重吐出:``田俊所言,是愉妃!''她顿一顿,咽了口口水,又道:``别说皇后娘娘不信,奴婢也不信。但凌大人细细问过那日与田俊喝酒的人的姓名,其中为首的扎齐,果然是珂里叶特氏的族人,愉妃小主的远房侄子。''

海兰?!

有那么一刻,如懿的脑中全然是一片空白,仿佛下着茫茫的大雪,雪珠夹着冰雹密密匝匝地砸了下来,每一下都那么结实,打得她生生地疼,疼得一阵阵发麻。是谁她都不会震惊,不会有这般刺心之痛!为什么,偏偏是海兰?

如懿不知自己是如何发出的声音,只是一味嘶哑了声音喃喃:``海兰?怎么会是海兰?''

容珮瞪大了眼,一脸不可思议:``旁人便算了,若说是愉妃小主,奴婢也不敢信啊!''

惢心为难地道:``凌大人查出了这些,又去关田俊的牢房打听,才知道扎齐不仅灌醉了田俊,而且在田俊入狱后特意关照过衙门,若是轻纵了田俊这般不尊圣上罔顾君臣的人,他便要找他的姑母愉妃小主好好数落数落罪状,所以衙门里才看管得格外严厉,田俊也吃了不少苦头。但到了后来,通融了官府放出田俊,竟也是扎齐。这一关一放很是古怪,难不成田氏答应了什么,她儿子才能平安无事了?因为连田俊自己也说过,他出狱后他母亲总是惴惴不安,问她也不说,问急了便只会哭,说一切都是为了他才被宫中的人胁迫。田俊再问,田氏却怎么也不肯说了。''

惢心看着如懿逐渐发白的面容,不觉有些后怕:``皇后娘娘,您别这样。凌大人查知了这些,也知道事关重大,不敢轻易告诉娘娘,只得与奴婢商议了,托了奴婢进宫细说。''

如懿只觉得牙齿``咯咯''地发颤,她拼命摇头:``不会!海兰若真这么做,于她有什么好处?''

容珮应声道:``皇后娘娘说得不错,愉妃小主一直和皇后娘娘交好,皇后娘娘又那么疼五阿哥。情分可比不得旁人!''

惢心沉吟片刻,与容珮对视一眼,艰难地道:``熟识扎齐之人曾多次听他扬言,若有皇后娘娘的嫡子在一日,五阿哥便难有登基之望。如果扎齐所言是真,那么愉妃小主也并非没有要害娘娘的理由。''她迟疑片刻,``皇后娘娘看纯贵妃便知道了,她那么胆小没注意的一个人,当日为了三阿哥的前程,不是也对娘娘生了嫌隙么?如今三阿哥、四阿哥不得宠,论年长论得皇上器重,都该是五阿哥了。可若有娘娘的嫡子在\ldots{}''她看了如懿一眼,实在不敢再说下去。

如懿满心满肺的混乱,像是被塞了一把乱丝在她喉舌里,又痒又烦闷。正忧烦忧心,却听听外头的小宫女菱枝忙忙乱乱地进来到:``皇后娘娘,宫里可出大事了呢!''

容珮横了菱枝一眼,呵斥道:``你不是去内务府领夏季的衣料了么?这般沉不住气,想什么样子?''她停了停,威严地问:``出了什么事儿?''

菱枝忙道:``奴婢才从内务府出来,经过延禧宫,谁知延禧宫已经被围了起来,说愉妃小主被皇上禁足了。连伺候愉妃小主的宫人都被带去了慎刑司拷问,说是跟咱们十三阿哥的事有关呢。''

如懿神色一凛,忙定住心神看向惢心:``是不是凌云彻沉不住气,告诉了皇上?''

惢心忙摆手道:``皇后娘娘,凌大人就是不知该如何处置,才托了奴婢进宫细细回禀。若他要告诉皇上,便不是今日了。''

无数个念头在如懿心中纷转如电,她疑惑道:``你才入宫,连我也是刚刚知晓这件事,怎的皇上那儿就知道了?实在是蹊跷!''如懿看一眼容珮:``你且让三宝仔细去打听。''

容珮答应一声便出去了,如懿想了想,又叮嘱道:``惢心,今日你入宫,旁人怎么问都得说是只来给我请安的。旁的一字都不许提,免得麻烦。''

惢心连忙答应了,担心地看着如懿道:``皇后娘娘,奴婢不知道该怎么说,从前日日陪着您倒也不觉得什么,不过是兵来将挡水来土掩罢了。如今在宫外过了几年安稳日子,回头看看,真觉得娘娘辛苦。娘娘憔悴了那么多\ldots 唉,若在寻常人家,孩子没了这种事,哪有夫君不陪着好好安慰的。可在这里离,一扯上天象国运,连娘娘的丧子之痛也成了莫须有的罪名。奴婢实在是\ldots{}''她说不下去,转过头悄悄拭去泪水,又道:``奴婢不能常入宫陪伴娘娘,但求娘娘自己宽心,无论如何,都要自己保重。奴婢会日日在宫外为娘娘祈福的。''

惢心不能再宫中久留,只得忍着泪依依不舍而去。

\hypertarget{ux7b2cux4e8cux5341ux4e8cux7ae0-ux5984ux4e8b}{%
\chapter{第二十二章
妄事}\label{ux7b2cux4e8cux5341ux4e8cux7ae0-ux5984ux4e8b}}

宫中骤然生了这样的变故,如懿也无心留她在这是非之所,便让容珮好好送了出去。这样纷乱着,到了午后,宫中的嫔妃们也陆陆续续来探望,忻妃与淳贵妃固然是半信半疑,然而余者,更多是带了幸灾乐祸的神色,想要窥探这昔日好姐妹之间所生的嫌隙。

如懿倒也不回绝,来了便让坐下,也不与他们多交流,只是静静地坐在暖阁里,捧了一卷诗词闲赏。如此,那些聒噪不休的唇舌也安静了下来,略坐一坐,她们便收起了隐秘而好奇的欲望,无趣地告退出去。

面上若无其事并不能掩去心底的波澜横生。容珮一壁收拾着嫔妃们离去后留下的茶盏,一壁鄙夷道:``凭着这点儿微末道行就想到娘娘面前调三窝四,恨不得看娘娘和愉妃小主立时反目了她们才得意呢。什么人哪!娘娘受委屈这些日子她们避着翊坤宫像避着瘟疫似的,一有风吹草动,便上赶着来看热闹了。''她啐了一口,又奇道:``今儿来了这几拨人,倒不见令妃过来瞧热闹?''

微微发黄的书页有草木清新得质感,触手时微微有些毛躁,想是翻阅得久了,也不复如昔光滑。而自己此刻的心情,何尝也不是如此?像被一双手随意撩拨,由着心思翻来覆去,不能心安。如懿撂下书卷,漫声道:``令妃怀着第一胎,自然格外贵重,轻易不肯走动。''她揉了揉额头,``对了,三宝打听得如何了?''

容珮有些愧色:``御前的嘴都严实得很,三宝什么都打听不到。好容易见着了凌大人,凌大人也不知是何缘故,这事便一下抖了出来。''

如懿沉吟片刻:``那永琪呢?人在哪里?''

容珮道:``听三宝说五阿哥一直把自己关在书房里,什么动静也没有。''她想了想道,``娘娘,您觉得五阿哥是不是太沉得住气了,自己额娘都被禁足了\ldots{}''

如懿垂首思量片刻,不觉唏嘘:``若论心志,皇上这些阿哥里,永琪绝对是翘楚。这个节骨眼上,去求皇上也无济于事,反而牵扯了自己进去,还不如先静下来瞧瞧境况,以不变应万变。''

京城的晚春风沙颇大,今年尤甚,但凡晴好些的日子,总有些灰蒙蒙的影子,遮得明山秀水失了光彩,人亦混混沌沌,活在霾影里。偶尔没有风沙砾砾的日子,便也是细雨萧瑟。春雨是细针,细如牛毫,却扎进肉里般疼。疼,却看不见影子。

细密的雨丝是浅浅的墨色,将百日描摹得如黄昏的月色一般,暗沉沉的。分明是开到荼蘼花事了的时节,听着冷雨无声,倒像是更添了一层秋日里的凉意。那雨幕清绵如同薄软的白纱,被风吹得绵绵渺渺,在紫禁城内外幽幽地游荡,所到之处,都是白茫茫的雾气,将远山近水笼得淡了,远远近近只是苍茫雨色。

慎刑司日日传来的消息却一日坏过一日,不外是今日是谁招了,明日又是谁有了新的旁证,逼得海兰的境况愈加窘迫。终于到了前日午后,皇帝便了旨,将海兰挪去了慎刑司,只说是``从旁协问''。

这话听得轻巧,里头的分量却是人人都掂得出来的。堂堂妃位,皇子生母,进了慎刑司,不死也得脱层皮。何况那样下作的地方,踏进一步便是腌渍了自己,更是逃不得谋害皇嗣的罪名了。

永琪自母妃出事,一直便守在自己书斋中,不闻不问,恍若不知。到了如此地步,终于也急了,抛下了书卷便来求如懿。奈何如懿只是宫门深闭,由着他每日晨起便跪在翊坤宫外哀求。

容珮捧着内务府新送来的夏季衣裳,行了个礼道:``皇后娘娘,五阿哥又跪在外头了呢。真是\ldots{}''如懿头也不抬,只道:``这些经幡绣好了,你便送去宝华殿请大师与初一十五之日悬挂在殿上,诵经祈福。''

容珮一句话噎在了喉头,只得将衣裳整理好,嘟囔着道:``这一季内务府送来的衣裳虽然不迟,但针脚比起来竟不如令妃宫里。''又道:``今日令妃的额娘魏夫人进宫了。真是好大的排场,前簇后拥的,来宫里摆什么谱儿呢。忻妃和舒妃临盆的时候,娘家人也不这样啊。''

如懿短短一句:``要生孩子了,这是喜事!''

``十三阿哥才走,令妃不顾着皇后娘娘伤心,也不顾尊卑上下么?这么点眼!''

``有喜事来冲伤心事,都是好的!''

容珮正要说话,忽然定住了,侧耳听着外头,失色道:``这是五阿哥在磕头呢。他倒是什么也不说,可这磕头就是什么都说了。五阿哥是在求皇后娘娘保全愉妃小主呢,可如今这情势,他开不了这个口。''

``开不了就别开。他就该安分待在书房里,别把自己扯进去。''

``不怪五阿哥,亲额娘出了这个事儿,他年级小,是受不住。''她小心翼翼看着如懿,``皇后娘娘撒手不管,可也是信了慎刑司的证供。也是,一日一份证词,众口一说,奴婢本来不信的,也生了疑影儿。皇后娘娘,您\ldots{}''

``本宫?本宫信与不信有什么要紧?全在皇上。''

任凭外头流言四起,蜚语扰耳,她只安静地守在窗下,挑了金色并玄色丝线,慢慢绣着``卍''字不到头的经幡。那是上好的雪色密缎,一针针拢着紧而密的金线,光线透过薄薄的浅银霞影纱照进来,映在那一纹一纹的花色上,一丝一丝漾起金色的芒,看得久了,灼得人的眼睛也发酸了。

日子这么煎熬着,外头闹腾如沸,她便是沉在水底的静石,任着水波在身边蜿蜒潺湲,她自岿然不动。倒是人却越发见瘦了,一袭霞绉长衣是去年江宁织造进贡的,淡淡地雨后烟霞颜色,春日里穿着略显轻软,如今更显得大了,虚虚地笼在身上,便又搭了一件木兰青素色锦缎外裳,只在袖口和衣襟上碧色夹阴线绣了几枝曼陀罗花,暗香疏影,倒也合她的心境。

容珮看她这般冷淡,全然事不关己似的,也不知该如何说起了。容珮听着外头的叩求声,满目焦灼:``五阿哥孝心,听着怪可怜的。皇后娘娘,这个事,怕只能您能求一求情。好歹,别让她们苦着愉妃小主。''

如懿瞥了她一眼,冷冷淡淡道:``你的意思,是也觉得这事不干愉妃的事了?原本皇上只是禁足了她,如今人都带进了慎刑司去了,你叫本宫还有什么颜面求情,岂不怕对不住本宫枉死的孩儿?''

容珮素知她疼爱永琪不逊于亲子,从未见过她如此冷硬面孔,一时也不知该如何应对,只得道:``奴婢不敢。''

``不敢,便安分守己吧。多少官非,便从那不肯安分上来的。''

二人正说话,却听外头遥遥有击掌声传来,守在外头的小宫女芸枝喜不自胜地进来,欢喜得手脚都不知道往哪儿放了:``启禀皇后娘娘,皇上、皇上过来了呢。娘娘赶紧预备着接驾吧。''

容珮一怔,忽然啐了一口,呵斥道:``皇上来看皇后娘娘,这不是极寻常的事么?瞧你这眼皮子浅的样子,叫外人看见了,还真当娘娘受尽了冷落,皇上来一次都高兴成这样。别人怎么议论那也是别人的事,自个儿先没了一点儿骨气,才叫人笑话呢!''

芸枝被劈头盖脸地说了一通,也自知失了分寸,脸上一阵红一阵青,忙赔笑道:``姑姑教训得是。奴婢们也是为娘娘高兴,一时欢喜过头了。奴婢立刻出去吩咐,叫好生迎驾便是。''

容珮这才赞许地看她一眼,又恭恭敬敬对如懿道:``皇上来了,奴婢此后娘娘更衣接驾吧。''

如懿微微沉吟,见身上衣衫着实太寒素了,便换了一袭浅杏色澹澹薄罗衣衫,才出来,便见皇帝已经进了正殿。数月里寥寥几次的相见,都是在不得不以帝后身份一起出席的场合。彼此隔着重重的距离,维持着应有的礼仪,她的眼角能瞥见的,不过是明黄色的一团朦胧的光晕。此刻骤然间皇帝再度出现在眼前,是触手不及的距离,她只觉得陌生,一股在春暖世界亦不能泯去的冰凉的陌生。

皇帝倒是极客气,对着她的笑容也格外亲切,只是那亲切和客气都是画在天顶壁画上的油彩花朵,再美,再嫣,也是不鲜活的,死气沉沉地悬在本空里,端然妩媚着。

如懿依足了礼仪见过皇帝,皇帝亲自扶了她起来,小心翼翼地关切着:``皇后可还好么?''

同床共枕那么多年,一并生活在这偌大的紫禁城中,从养心殿道翊坤宫并不算遥远,可是到头来,却是他来问一句:``可还好么?''

若是有心,他想知晓关于她的一切,是何等简单之事,却原来,这么简单,也要问一问。鼻尖的酸楚随着她游荡的思绪蔓延无尽,她只得绷着笑脸按着规矩给出不出错的答案:``皇上关怀,臣妾心领了。臣妾一切安好。''

皇帝穿着一身天青色江绸长袍,因是日常的衣衫,只用略深一色的松青色丝线绣了最寻常不过的团福花样,最是简净不过。可细细留意,却音乐倒映着帘外黄昏时分的日影春光,愈加显得他身量欣欣。

皇帝迟疑着伸出手,想要抚摸她的脸颊,那分明是带了几许温情的意味。在他指尖即将触上肌肤的一刻,如懿不知怎的,下意识地侧了侧脸,仿佛他的指尖带着几许灼人的温度。

皇帝便有些尴尬,恰好容珮端了茶来,见两人都是默默坐着,便机警道:``昨儿半夜里皇后娘娘便有几声咳嗽,想是时气不大好的缘故,所以奴婢给娘娘备的茶也是下火的金线菊茶。''她端过一盏甜汤放在皇帝跟前,恭谨道:``御膳房别的都好,可论这一盏暗香汤,想来是比不过翊坤宫的。''她悄悄看一眼皇帝,``到底,是皇后娘娘的一点儿慧心。且如今春燥,喝这个也是润肺生津的。只皇上别怪奴婢准备得不合时宜便好。''

容珮说这便要告罪,皇帝往苏瓷汤盏轻轻一嗅,慨叹道:``果然清甜馥郁,便是御膳房也比不上的。''他抿了一口,看了眼容珮,道:``既是心意,又哪来什么不合时宜。你这丫头一向快人快语,如今怎么也瞻前顾后起来了?''

``奴婢能不瞻前顾后么?''容珮轻叹一声,仿佛一言难尽似的,便垂手退了下去。因着这一声叹息,连着整个翊坤宫都蕴着满满的委屈似的。皇帝看着宫人们都退了下去,才道:``朕原以为是你苛待了田氏才惹出后来种种事端,那么固然田氏该死,朕心里却总也又道过不去的坎儿,所以哪怕记挂着你,总迈不出那一步来看看你。''他的嗓音沙沙的,像风吹过树叶的沙沙声响,又好似春夜里的细雨敲打着竹枝的声音一般,``可若朕与你的孩子是被你身边最亲近的人假借田氏之手暗算,那么如懿\ldots 朕不只是委屈了你,更是委屈了自己。委屈着自己不来看你,不来和你说说话,不来和你一起惦记咱们的孩子。''

他的语气那样伤感,浑然是一个经历着丧子之痛的父亲。可是如懿明白,他的伤感也不会多久的,很快就会有新的孩子落地,粉白的小脸,红润的唇,呱呱地哭泣或是笑着。那时,便有了更多新生的喜悦。

檐下昏黄的日影,静静希翼无声。庭院中有无数海棠齐齐绽放,香气随光影氤氲缭绕,沁人心脾。花枝的影子透过轻薄如烟的霞影绛罗窗纱映在螺钿案几上,斜阳穿过花瓣的间隙落下来,仿佛在二人间落下了一道无形的高墙。

若在青葱年少时,听到他这样的话,一定会感动落泪吧?然而此刻,如懿还是落泪了。不为别的,只为她的思子之情。她悄然引袖,掩去于这短短一瞬滑落的泪水,问道:``皇上所说的亲近之人,是指愉妃么?臣妾很想知道个中原委。''

皇帝蹙了蹙眉,道:``朕一早得到刑部的上疏,说田氏之子田俊于前日突然横死家中,是被人用刀刃所杀。找到他的尸身时,在他身边发现一枚女子所用的金丝镯,像是打斗时落下的。因田俊身份特殊,他母亲田氏牵涉宫中之事,当地官府为求慎重,便上报了刑部。刑部派人去看时发觉这金丝镯像是内务府的手工,便不敢怠慢,找到了内务府的记档,才发现那是愉妃的东西。而杀人者也很快被找到,正是愉妃的远房侄子扎齐。扎齐一用刑便招了,说是愉妃如何指使他杀了田俊灭口,又说愉妃曾指使他让田俊下狱,以此要挟田氏在宫中残杀皇后幼子,便是咱们的永璟。''

那一字一句的惊心动魄,难以从字里行间去寻找它的疏漏。如懿仔细倾听,忽然问:``杀了田俊灭口?为何从前不杀,要到此时才杀?''

皇帝静默片刻,凝视着如懿道:``那便要问皇后了。皇后可曾让朕跟前的凌云彻出宫查访此事?''

他的目光有难掩的疑虑,如懿一怔,便也坦然:``是。臣妾生怕田氏之事背后有人指使,更不欲打草惊蛇,想起皇上每每提及凌侍卫干练,所以曾托他出宫方便时探知一二。''

皇帝这才有些释然,颔首道:``据扎齐所言,他按照愉妃的吩咐,一直暗中留意田俊的行迹。凌云彻与田俊接触之事,他也眼见过一二,便向宫里传递过消息,得了愉妃的叮嘱,才动了杀机的。谁知事出慌乱,便把愉妃赏赐的一个金丝镯落下了。而朕也命人细细搜过田俊家中,他与他姐姐的家书中,甚是愤愤不平,道自己与田氏都是为愉妃所害。朕来翊坤宫前,又问了凌云彻,果然无二。只是凌云彻说,他查得这些后一直未能深信,所以并未来得及将此事禀报于你。''

如懿目光一凛,当即道:``是。凌侍卫一向谨慎,若不得万全并不会告知臣妾。今日臣妾听皇上所言,即便扎齐所说的这些还对付得过去,那么愉妃又为何要害臣妾的孩子?''

皇帝头痛不已,扶着额头唏嘘道:``如懿,朕的儿子中,永琪的确算是出类拔萃,哪怕朕不宠爱愉妃,也不得不偏疼永琪。可是如懿,难道就因为朕偏疼了永琪,才让愉妃有觊觎之心,想要除掉朕的嫡子来给永琪铺路么?看了这些证词,朕也会疑惑,愉妃虽然不得宠,但的确温柔静默,安分守己,从来不争宠。可就是因为她从来不争宠,朕才想,她心里想要的到底是什么?不是荣华,不是富贵,还是朕看不透她,她真正要的,是太子之位。''

有风吹过,庭前落花飞坠,碎红片片,落地绵绵无声。在红墙围成的局促的四方天地里,孩子是她的骨血相依,海兰是她的并肩扶持,而皇帝,是她曾经爱过的枕边人。这些都是她极不愿意失去的人,若是可以,可以再多得到些,她也想得到家族的荣光,夫君的爱怜,还有稳如磐石的皇后地位。

有一瞬间,连如懿自己也有了动摇。人情的凉薄反复,她并非没有看过,甚至很多时候,她已经习以为常。做人,如何会没有一点点私心呢?只是她的孩子只剩了永琪和永璂,她的夫君能给予的爱护实在微薄得可怜。若海兰都一直在暗处虎视眈眈\ldots 她情不自禁打了个寒战,若真是如此,那往后的漫长岁月,她还有什么可以信赖?

如懿静静地坐在那里,只觉得指尖微微发颤,良久,她终于抬起脸,望着皇帝道:``这件事说谁臣妾都会信,但若说是海兰,臣妾至死不信。因为臣妾若是连海兰都不信,这宫里便再没有一个可信之人。''

皇帝的唇角衔着一丝苦涩:``是么?如懿,曾经真年少时,也很相信身边的人。相信皇阿玛真心疼爱朕,只是忙于政务无暇顾及朕;相信朕身为皇子,永远不会有人轻视朕。朕曾经相信的也有很多,但到后来,不过是镜花水月而已。''

如懿的神色异常平静,宛如日光下一掬静水,没有意思波纹:``刑部做事缜密,又人证物证俱在,臣妾也会动了疑心。只是臣妾更疑心的是此事太过凑巧。田氏母子已经死无对证,扎齐的确是海兰的远房侄子,可也未必就真的忠于海兰。若是真正忠心,咬死了不说也罢了,他倒是一用刑就招了,还招得一干二净。这样的人,一点点刑罚可以吐口,那就有的是办法让他说出违心的话。''

皇帝沉吟着道:``你便这样相信愉妃?''

如懿郁郁颔首,却有着无比的郑重:``海兰在臣妾身边多年,若说要害臣妾的孩子,她比谁都有机会。当时十三阿哥尚在腹中,未知男女,哪怕有钦天监的话,到底也是未知之数。若是她忌惮臣妾的嫡子,永璂岂不是更现成,何必要单单对永璟下手?臣妾身为人母,若没有确实的答案,臣妾自己也不能相信!''她郑重下跪,``皇上,这件事已然牵涉太多人,既然已经到了如此地步,但求可以彻查,不要使一人含冤了。''

伶仃的叹息如黄昏时弥漫的烟色,黄寺沉声道:``这件事,朕必定给咱们的孩子一个交代。''他靠近一些,握住她的手道:``到用晚膳的时候了,朕今日留在翊坤宫陪你用膳,可好?''

他的掌心有些潮湿,像有雾的天气,黏腻,湿漉,让人有窒闷的触感。如懿强抑着这种陌生而不悦的触感,尽力笑得和婉得体:``臣妾日进见到纯贵妃,听她说起永瑢十分思念皇上,皇上若得空儿,不如去看看永瑢。小儿孺慕之思,臣妾身为人母,看着也于心不忍。''她顿一顿,``再者六公主离世后,忻妃一直很想再有一个孩子,皇上若得空儿\ldots{}''

皇帝面容上的笑意仿佛窗外的天光,越来越暗,最后凝成一缕虚浮的笑色:``皇后垂爱六宫,果然贤德,那朕便去看看忻妃吧。''他说罢便起身,再未有任何停留,身影如云飘去。唯有天青色袍角一旋,划过黄杨足榻上铺着的黄地蓝花锦毡,牵动空气重一卷卷旋涡般的隔膜。

如懿屈膝依礼相送,口中道:``恭送皇上。''

她一直屈膝保持着恭敬婉顺的姿态,懒得动弹。直到容珮匆匆进来,心疼又不安地扶着她坐下,道:``娘娘这是何苦?皇上愿意留下来陪娘娘用膳,这又不是什么坏事。您也知道皇上的性子,一向最爱惜颜面。您这样拒人于千里,岂不也伤了皇上?''

容珮絮絮间尽是关切心意,如懿乏倦无比,道:``皇上留下的确不是坏事,可于本宫而言,是太累的事。不止人累,心也累。若彼此间终有隔阂,心怀怨怼,何苦虚与委蛇,假笑迎人。若真这样勉强,以皇上的心性,到头来,只怕更伤了颜面。''

容珮半跪在如懿身边,替她抚平衣上的折痕:``为了十三阿哥的死,皇上与娘娘便隔膜至此吗?有时候夫妻间,不是你退一步我退一步的事,马马虎虎也就过了。''

忧色如夜雾无声无息地笼上如懿的面颊,她慨叹道:``只是永璟离世后,本宫才发觉,纵有骨肉情深,有夫妇之义,在皇上心里,也终究在意虚无缥缈的天象之言。''

容珮犹疑着道:``皇家历来重视钦天监之言,也怪不得皇上。而且那时候十三阿哥刚离世,皇上心里不好受,又听了田氏的诬陷之词,难免心里过不去,才疏远了娘娘。''她叹口气,无可奈何道:``可皇上就是皇上,除了娘娘让步,难道还有别的法子么?''

如懿怔了半晌,恍惚道:``这样的天家夫妇,还不如民间贫寒之家,做对寻常夫妻来得容易。''

容珮吓了一大跳,赶紧捂住如懿的嘴,失色道:``娘娘说什么呢!这话若被人听见,可轻可重。何苦贫贱夫妻就好么?奴婢只要一想起自己的额娘\ldots 唉,咱们女人就是这么个命!''

如懿自知失言,忙掩饰着道:``本宫也是一时失言。''

她望着窗外,天色暗沉下来,宫人们在庭院里忙着掌起影罗牛角宫灯。那红色的灯火一盏一盏次第亮起来,虚弱地照亮芒远的黑暗。

\hypertarget{ux7b2cux4e8cux5341ux4e09ux7ae0-ux5debux86caux4e0a}{%
\chapter{第二十三章
巫蛊(上)}\label{ux7b2cux4e8cux5341ux4e09ux7ae0-ux5debux86caux4e0a}}

海兰的事一审便审了许久,自海兰入了慎刑司,事情便一日日拖延下来,渐渐泥牛入海,无甚消息。

慎刑司瞒得上下不透风,根本漏不出一点儿消息来,连海兰是生是死,是否受刑也无从可知。如此一来,永琪更是急得如热锅上的蚂蚁,只是无计可施罢了。

偶尔嫔妃们有一句没一句地在太后跟前提起,便是慈和避事如太后也沉下了连呵斥:``这是什么体面的事么?皇上尚未有任何处置,你们便闲话连篇,当真讨嫌!''

如此,明面上无人再敢言语,暗地里却愈加私语窃窃。

这一日,众人正聚在如懿宫中请安,忽而容珮急急转进,焦灼了声音道:``皇后娘娘,慎刑司里传来消息,愉妃\ldots{}''她稍一沉吟,换了口气道:``珂里叶特氏求见皇后娘娘!''

颖妃是蒙古人,性子最直,当下就问道:``求见?怎么求见?难道请皇后娘娘玉步踏入慎刑司么?这算什么道理!''

忻妃自女儿夭折后,也失了往日的活泼,近日里总是沉默。她徒然听了这一句,闷了片刻,眸中不觉一黯:``珂里叶特氏?难道皇上已经褫夺了海兰姐姐的妃位?''

嬿婉绞着娟子,细细柔柔道:``珂里叶特氏做出这般伤天害理的事,便是没有褫夺妃位,忻妃姐姐,咱们哪里还能与她姐妹相称?''

忻妃旋即红了脸,待要争辩,只见一旁数着蜜蜡佛珠的绿筠悄悄摆了摆手,便只得按捺了性子,再不多言。

末了,还是如懿以淡漠的语气,隔断了一切希望的可能:``珂里叶特氏有谋害本宫孩子之嫌,一切交由慎刑司处置,本宫见她也是枉然!''

一时间,嫔妃们皆知端底,怀揣着关于海兰命运的揣测都散了,唯忻妃与如懿交好,陪着闲话一二。嬿婉待要扶着笨重的身子起身,如懿独独唤了她留下。

嬿婉见了如懿便有几分不自在,但她素来在皇帝跟前软语温存做小伏低惯了,对着如懿也是温温软软一笑,娇不胜力一般。如懿温言道:``听得你额娘入宫来陪你待产。也好,你是头胎,有额娘陪着也安心些。''她唤过菱枝,``这儿有几匹江宁织造进贡来的缎子,本宫瞧着颜色不错,便赐予你额娘裁两身新衣。''

嬿婉扶着腰肢娇怯怯谢过,面色微红:``多谢皇后娘娘关怀。前些日子臣妾额娘刚刚进宫,皇后娘娘便赐了两支老山参,臣妾额娘欢喜得不知怎么才好。偏皇后娘娘身子不适,额娘不敢打扰,不能亲自来谢恩。为着这事,额娘一直挂心呢。''

如懿取过茶盏轻抿一口,漫不经心道:``这两支老山参极好,魏夫人年纪大了,补身很是相宜。''如懿深深地望她一眼,忽然一笑,``希望魏夫人服了山参,可以长命百岁,享享儿女福分!''

嬿婉不知怎的,只觉满心里不舒服,脸上却不肯露出分毫,掬了满盈盈的笑意正行礼谢过,容珮一把用力扶住了她,笑得壁垒分明:``令妃娘娘心中顾着尊卑善恶就好,礼数不在一时。可得仔细着,这是您的头胎,荣华富贵都在上头呢。''

嬿婉哪里敢分辨,容珮又是那样肃杀的性子。待要向如懿软语几句,见她只是悠悠地饮着一盏茶,与忻妃闲话一二,不知怎的,就觉得自己的气焰矮了几分。

待回到自己宫里,嬿婉满腹无从诉说的委屈便平复了好些。嬿婉的额娘魏夫人已然入宫陪产,暂居于永寿宫偏殿。比之上回的挑剔,这回入宫的魏夫人慈祥又大方,对着嬿婉更是有扯也扯不下的殷殷笑容,恨不得鞍前马后事事都替她伺候了周全。此时魏夫人正坐在窗下饮着一盏冰糖金丝燕窝,喜滋滋地看着金海棠花福寿大圆桌上堆着小山似的物件,金灿灿地眩了眼眸。嬿婉懒懒问:``是内务府送来的么?''

魏夫人扬扬得意地起身,小心翼翼地扶过嬿婉往榻边坐下:``这么晚没回来,还当皇后留你说话用夜宵了。''

嬿婉扬一扬娟子,不耐烦道:``晨昏定省,这是规矩。女儿再有着身孕,皇后不要我站就站,坐就坐,一味地立规矩么。''

魏夫人不屑地笑笑,狡黠道:``皇后可不敢为难你!如今你的肚子多金贵呢,她还能不分轻重?如今皇上待她好些,也是可怜她罢了。''她挽住嬿婉的胳膊,亲亲热热道:``你瞧皇上多疼你,这些都是晚膳后送来的赏赐呢。''

嬿婉一眼扫去,料子有上用金寿字缎二匹,江南的绿地五色锦八匹,轻容方孔纱八匹,各色彩绣的云锦蜀缎共十八匹。另有金镶珊瑚项圈一对,金松灵祝寿簪一对,榴开百子镶嵌石翠花六对,赤金点翠镶嵌抱头莲四对,一匣子白净浑圆的南珠,半尺高的紫檀座羊脂白玉观音并一对以玛瑙、珊瑚、玉石和金银打造的和合二盆景,模样活泼,几可乱真\ldots{}

魏夫人``哎呦''一声,捧着一对晶光琉璃的水晶玻璃瓶闻了又闻,奇道:``这是什么东西,摸着冰凉,闻着怪香的。''

澜翠看着魏夫人高兴,便也越发助兴道:``这是西洋来的香水,从前便有,也是只给皇后娘娘宫里的。如今咱们宫里可是独一份呢。''

魏夫人喜得看个不住,满口道:``西洋来的东西,可金贵了吧?额娘听说皇后宫里有个西洋来的自鸣钟,可会叫唤了,只是皇后怕吵给收起来了。这个没福气的,有好东西也不知道稀罕,哪里比得上你讨皇上喜欢。''

嬿婉瞧着欢喜,口中却慵慵道:``额娘的眼皮子也太浅了,皇上三五日便有赏赐,额娘来了几日,还不知道么?有什么值得高兴成这样子的!''

``你不高兴,额娘高兴!额娘八辈子都没见过这样的富贵。''魏夫人拉着她的手细细摩挲着,无限疼惜的样子,``女儿啊,你进了宫,不就为了这泼天的富贵么?终于有了这一天啊!可别忘了额娘和你兄弟,都倚仗着你呢。''

嬿婉瞥了她一眼,索性道:``额娘看中了什么,直说吧!''

``你兄弟到了说亲事的年纪了,自然得挑门富贵的好亲家,咱们也不能太逊色了!''她见嬿婉不大搭理的样子,赔笑道:``自然了,最要紧的是你肚子里的那位,有了他,咱们就什么都不怕了!''

暖阁里一盏盏红烛次第点起。宫人们轻轻取下云影纱描花灯罩,点上一支支臂粗的花烛,又将灯罩笼起,殿内顿时明亮。那是河阳所产的花烛,因皇帝喜好宣和风雅,遂仿宋制,用龙涎、沉香灌烛,焰明而香郁,素来也只在宠妃阁中用。魏夫人深吸两口气,连道``好香!好香''!遂仔细端看嬿婉的肚子。她的笑容藏也藏不住似的,全堆在脸上,真实越看越爱:``哎呀!这肚子尖尖的,准是个阿哥!''

嬿婉抚着高高隆起的腹部,吃力地斜靠在檀香木雕花滴水横榻上,手边支着几个杏子红绫洒金花蔓软枕,上头花叶缠绵的花纹重重叠叠扭成曼妙的图样,如烟似雾般热热闹闹地簇拥着越见圆润的嬿婉。嬿婉有些烦心,赌气似的道:``额娘,你喜欢儿子喜欢得疯了,眼里只瞧得见儿子么?在家时对弟弟是这般,如今盼着我也是这般。''

魏夫人收了笑容,讪讪道:``额娘也是为你好。难道你不盼着是个阿哥么?''

嬿婉瞥了魏夫人一眼,掌不住笑道:``我在宫里,自然是盼望有位皇子,才能立稳脚跟。可若是个公主,却也不错。我瞧着皇上也很是喜爱公主的呢。''

魏夫人念了几句佛,连连叹道:``哎呀,若只是一个公主,有什么用啊?若是个阿哥,那该有多好!''

嬿婉不耐烦地看了魏夫人一眼,恨声道:``我何尝不知道公主无用?可是额娘担心什么,这一胎哪怕是个公主,我也能再生皇子。额娘没听戏文上说么,汉武帝的皇后卫子夫,便是先生了三个公主才生的太子。只要我能生,就不怕没有生出皇子的那一日。''也不知是不是说得急了,她呻吟一声,吃力地扭了扭腰肢,嗔道:``这孩子,只顾在我腹中顽皮了。''

魏夫人爱怜地看着女儿,爱不释手地捧着她的肚子道:``我的好娘娘,你可千万小心些,数不尽的荣华富贵都在他身上呢。你又是头胎,万万仔细着。''

她欲言又止,想了想还是道:``这几日额娘在宫里,旁的没什么,生儿育女的艰难倒是听了一肚子。''她皱着眉头,拔下一枚镶金莲蓬簪子挖了挖耳朵,叹道:``从玫嫔、怡嫔没了的孩儿,道愉妃生子的艰难,那可算是九死一生。忻妃的公主生下来不多久没了,前头淑嘉皇贵妃的九阿哥也是养不大。还有皇后,别看她高高在上,那十三阿哥不是一出娘胎就死了么?''

嬿婉目光一烁,有些不自在地撑了撑腰,啐道:``额娘说这些不吉利的做什么?''

魏夫人忙赔笑道:``额娘是担心你。''

嬿婉从绣籽盘花锦囊中掏出一把金锞子捏在手中把玩,那冰凉的圆润硌在手心里,却沉甸甸地叫人踏实。她梨涡微旋,漫不经心笑道:``额娘,人家没福是人家的事。你且看看咱们,虽说嫔妃有孕至八月时家母可入宫陪伴,可到底也要看皇上心疼谁。忻妃纵然是贵家女,可父母不在身边,到底也是独个儿生产的。愉妃更不必说,早没至亲了。哪里像您,能进宫享享福。''她说罢,微微蹙起眉,娇声道:``额娘,你到底是心疼我,还是心疼我腹中的孩子?''

``疼你和疼他不都一样!''魏夫人弓着腰身,``哎呦!我的小祖宗,可盼着你赶紧出来伸伸胳膊腿儿,好跟着你舅舅耍耍,赶上喝你舅舅一口喜酒呢。''

嬿婉沉吟片刻,凑近了魏夫人道:``上回说弟弟的亲事,可如何了?''

魏夫人不提则罢,一提便懊恼满怀:``不是额娘惦记着你生个阿哥,实在是如今的人势利。你只得宠却没个可以依靠的阿哥,那起子眼皮子浅的人都犹豫着不肯给你兄弟许个好亲事呢。所以,一切都在你肚子上。''

嬿婉哪里肯当真:``说了什么?哄了您不少银子吧?''

魏夫人欢喜道:``算命的仙师说了,你是有运无命,皇后是有命无运!她的皇后能不能当到底,还两说呢。''

嬿婉直皱眉头,嫌弃道:``额娘,这不是好话!您真是糊涂了!''

``糊涂什么?''魏夫人昂起头:``只要你能做皇后,命啊运啊都不怕!对了,额娘拿些东西回去,也好显赫些,知道咱们宫里是有人的,才不敢叫人欺负了咱们去!否则你费尽心思算计着愉\ldots{}''

嬿婉勃然变色,白着面孔立起身来,喝道:``额娘,你满嘴胡咀什么!''

魏夫人见她疾言厉色,身形又隆重,一时被压倒了气势,慌不迭拢了一把金银珠宝在手,讷讷道:``额娘浑说的,你别在意!''

嬿婉见母亲神情委顿,举止猥琐,纵然穿金戴银,却掩不住一股市侩气,只觉得一阵心酸,纵有万丈雄心,此刻也消了一半了。嬿婉见她如此,忙向春婵使了个眼色。春婵会意,笑吟吟引了魏夫人道:``夫人,库房正在点存东西,新送来一批上号的瓷器,奴婢陪您去瞧瞧,有什么好的咱们挑些给公子娶亲时用。''

魏夫人听得高兴,立刻一阵风去了。春婵忙扶了嬿婉坐稳,轻轻巧巧替她捏着肩膀道:``小主别伤心。奴婢冷眼瞧着,夫人偏爱公子爷也不是一日两日了,您心里明白就成。犯不着为这个伤心,仔细动了胎气可是伤自己的身子。''

嬿婉伸手取过一个描金珐琅叠翠骨瓷小圆钵,蘸了些许茉心薄荷露揉着额头,叹息道:``本宫何尝不知?你打量着额娘是来瞧本宫的么?不过是把银子看得重罢了。便是疼本宫肚子里这个,也只瞧着他能带来富贵罢了。''她说着便又恼又是伤心,丢下手中的圆钵,狠狠道,``额娘从小便嫌本宫是女儿家,如今还不是要靠在本宫身上!''

春婵赔笑道:``话说回来,您原也不指望他们,万事都在您自己的筹谋。您既想明白了,更不必伤神。给足了银子不论骨血亲缘便是。''

``人人都有个好娘家,只我是这些不成器的!成日里只想着打秋风拢银子,为了外头那件事,三番五次地向我伸手,也不知多少花在打点上,多少入了自己的私囊。瞧他们这般,我便是要寻个依靠也难!''嬿婉万般烦难,揉着心口气急道:``有些亲缘是血肉上,可不是骨子里的。骨子里的打不断,血肉\ldots{}''她咬着牙,含泪道,``岂不知哪天就被割舍了呢?''

春婵好声好气劝慰道:``小主急什么,您的依靠在肚子里呢。与您血肉相连,骨血难分。您顺顺当当生下来,便是比皇后娘娘都有福了。您瞧她,费尽心思,十三阿哥到底没睁开眼来。''

嬿婉的面色渐渐阴沉,长长的丹蔻指甲敲在冷硬的金珠玉器上发出叮当的清音:``也是。本想着要她胎死腹中,可胎死腹中有什么好玩的?毕竟才在腹中几个月大,也不算个人。要是费尽千辛万苦生下了,睁眼一看是个死胎,那才有意思呢。一想到她这些年挫磨本宫的样子,本宫心里便跟油煎似的,熬得生疼。''

嬿婉的声线像是被利器磋磨着,带着嘶哑的狠意:``只可惜,只死了她的一个女儿一个儿子,还留着一个好好儿的呢。''

春婵低声道:``皇后娘娘年华渐衰,总有力不从心的时候,咱们有的是机会,不怕等!如今这个节骨眼上,已经出了那么多事,可不能再轻举妄动了。''

``本宫已经算不得年轻,新人一个个入宫,本宫还真能以为自己花开不败,恩宠常在么?没有孩子,什么恩宠都是空的!''嬿婉``咯咯''笑一声,``本宫如今什么都不动,什么都不想,只等着孩儿落地,万事再做计较。''二人信手翻着内务府送来的赏赐,挑了好的往库房里存折,余者都留着赏人用。

正计较间,却见皇帝跟前的毓瑚姑姑入内,打了个千儿道:``请令妃娘娘安,娘娘万福金安。''

因着常日里皇帝遣人过来,若非李玉,便是笑眉笑眼的进忠。毓瑚姑姑是积年的老嬷嬷,又不爱说笑,难得出养心殿外的差事。嬿婉乍然见了,颇有些意外,当下站起身笑道:``今儿难得,怎么是姑姑您来了?''

毓瑚淡淡一笑,中规中矩道:``皇后娘娘知道魏夫人进宫来陪伴小主,所以召夫人一见,也可叙叙话。''

嬿婉颇为意外,扬了扬春柳细眉,轻笑道:``姑姑难得来,先坐下喝口水吧。本宫即刻去请额娘出来。但不知皇后娘娘急着传召,所为何事?''

毓瑚含了淡淡的笑,躬身道:``皇后娘娘说小主是第一胎,难得魏夫人亲自入宫陪产,皇后娘娘特意请几位生育过的小主们与魏夫人说叨,以便小主顺利诞下皇嗣。''她一顿,``其实皇后娘娘也不急,小主让夫人慢慢来也可。''

毓瑚是皇帝身边积年的老姑姑,轻易难使唤。嬿婉知道轻重,一向又敬畏,忙不迭嘱咐道:``快请额娘出来!''

魏夫人甫到宫中,因着女儿有孕得宠,受尽了奉承追捧,最是飘在云尖上的时候,一路上又见毓瑚虽然年老体面,举止尊贵,但对着自己和颜悦色,便越是受用,倚了软胶慢悠悠地打量着周遭琉璃金碧。连绵宫殿的轮廓是重重叠叠的山峦的影,一层层倾覆下来,她也挥洒自如,丝毫不惧。

\hypertarget{ux7b2cux4e8cux5341ux56dbux7ae0-ux5debux86caux4e0b}{%
\chapter{第二十四章
巫蛊(下)}\label{ux7b2cux4e8cux5341ux56dbux7ae0-ux5debux86caux4e0b}}

待到了翊坤宫外,魏夫人下了轿,捶了捶腿脚道;''坐惯了轿子.难得站一站,真是腿酸脚乏.''说罢伸出手来,极自然地往毓瑚臂上一搭,昂然立稳了.

毓瑚倒也笑得和缓:''那必是令妃小主孝顺夫人,事事让您享福了.'

后头抬轿的小太监们早已吓得面面相觑,但见两人言笑晏晏,赶紧吐着舌头候在了外头。

才入了透雕垂花仪门,只见迎面赫赫朗朗五间正殿,檐角梁枋皆饰以金琢墨苏画,沥粉贴金,如云蒸霞蔚,烟云叠晕。此时,圆月如银盘悬挂于蓝紫色的夜空,清冷幽光倾泻而下,流在黄琉璃瓦歇上,泼刺刺跃出,掠过一扇扇万字团寿纹步步锦支摘窗,落在玉阶下陈设的铜凤、铜鹤之上,泛出大片如针毡般刺目而锐利的锋芒。

魏夫人愣了片刻,像是睁不开眼一般,拿绢子揉了揉眼角,道:``阿弥陀佛!原以为老身女儿的宫里算是龙宫一般了,没想到皇后娘娘宫里才是王母娘娘的瑶池哪!怪道人人都要进宫,人人都念着做皇后了。''

毓瑚见她说话这般着三不着两,也懒得与她多言,径直道:``皇后娘娘在候着了,咱们别晚了才是。''

魏夫人贪看景致,探头晃脑着,忽地被吓了一跳,捂着心口道:``哎呦!怎么站了一溜的阉人,连个笑影儿也没有,跟活死尸似的!还不如老身女儿宫里,笑眉笑眼的看着喜庆,该叫皇后娘娘好好调教调教,吓着皇上可怎么好!''

毓瑚转首见不过是侍立的两溜宫人,按着本分如木胎泥偶般立着,听得她越说越不成样子,急忙扯了她进殿去了。

魏夫人进了暖阁,犹自絮絮叨叨,陡然间闻得莲香幽幽然然,静弥一室。阁中静谧得恍若无人一般。她不知怎的便生了几分惧意,抬起头来但见暖榻上坐着一对璧人,座下分列着数位衣香翩影的丽人。毓瑚骤然松脱了她的手,自顾自屈膝道:``奴婢见过皇上皇后,两位主子万福。''

魏夫人这才意识到暖榻上着湖水蓝销金长衣、轻袍缓带的男子,正是自己入宫后久未谋面的皇帝贵婿。而他身侧并坐的女子,高耸云髻用随金镶青桃花白玉扁方起绾起,髻上簪着一对垂银丝流苏翡翠七金簪,余者只用大片翡翠与东珠点缀。她着一袭表蓝里紫的蹙银线古梅向蝶纹衣,其实魏夫人并不大分得清那是什么花,影影绰绰是一枝孤瘦的绯色梅花,却也像杏花,抑或桃花。可是月光隔着窗棂落在那女子身上,留下一痕一痕波縠似的水光曳影,无端让人觉得,那隐隐的清寒气息,应该不是姿容亲昵的花朵。

因是在盛夏,殿中并未用香,景泰蓝的大瓮里供着新起出的冰块,取其清凉解暑之意。袅袅腾起的白色氤氲里,那女子侧着脸端坐,唯见雪白耳垂上嵌珍珠花瓣金耳饰纹丝不动,明净的容颜仿如美玉莹光,熠熠生辉。

魏夫人从她服色上推知她的身份,不禁暗暗腹诽,比之女儿的春华正茂、风姿秀媚,眼前这位皇后显然带上了岁月不肯长久恩顾的痕迹。

这般一想,魏夫人只觉得心头畅快。她头一次面见着皇帝,情不自禁笑出来,拍着腿高喊了一句:``贵婿呦------万福万福------''

阁中众人惊得目瞪口呆,一时齐齐怔住。还是李玉反应得快,一把拉住魏夫人跪下道:''夫人快快行礼,这是宫中,并非民间,万万错不得礼数。''

魏夫人这才想起毓瑚叮嘱的礼数,忙扯直了身上酱红色滚六色指宽彩绒边得万福裳,用手指拈起深青色缠枝菊花马面裙,扭着身子道:``妾身魏杨氏拜见皇上皇后,皇上万福金安,皇后万福金安。''

皇帝笑了笑,伸手示意李玉扶起魏夫人,双手徐徐捧着一盏描金青瓮盏轻啜甘茗,留出一个镌刻般深沉的剪影。

皇帝左手边的花梨木青鸾海棠椅上坐着一位着牙黄对襟蕊红如意边绣缠枝杏榴花绫罗旗装的年轻女子,一张俏生生团团笑脸,拈了丝绢笑吟吟道:``夫人果然与皇上是一家人,见面就这般亲热,仿佛咱们与皇上倒生疏了,不比与令妃姐姐一家子亲热!''

另一年长女子穿了一袭浅碧色锦纱起花对襟展衣,裙身上绣着碧绿烟柳。虽然年长些许,但神色极是柔和,观之可亲。她笑着道:``什么一家子不一家子,皇后娘娘与太后的娘家才是和皇上正经的一家子呢。咱们都是皇上的嫔妃罢了,家人也是奴才辈的,要生了自狂之心,算什么呢!''

魏夫人听得不悦,但哪里敢发作,少不得忍气听李玉一一指了引见:``这是纯贵妃小主,这是祈妃小主。''魏夫人一一见过,却听得上首端坐的如懿轻声道:``皇上,难得魏夫人入宫来,听闻魏夫人府上与珂里叶特氏府上同住城东,想必也常常来往吧?''

魏夫人不意如懿问出这句来,连忙道:``妾身与愉妃小主家中并无来往。''

如懿似也不在意,只道:``哦。魏夫人博文广知,定有许多新鲜玩意儿说给咱们听。想必令妃也一直耳闻目染,听得有趣!''

魏夫人喜滋滋张开欲言,却见祈妃扬一扬头。撇嘴道:``皇上,皇后娘娘,这般磨牙做什么,咱们问了她便知。''

魏夫人以为皇帝要问嬿婉生产之事,正备了一肚子话要说,也好为自己先讨些辛苦功劳。却见皇帝微微侧首,一旁的李玉会意,从袖中取出一枚小小的布偶,扎得五颜六彩,一张脸也红红绿绿,肚子滚圆突出,显得格外古怪。

魏夫人见李玉递到自己跟前,伸了头看了几分道:``什么娃娃,做得这般难看,难不成是留着给令妃的小阿哥的?这可不成!''

如懿坐在上首,一张清水脸容并无妆饰,幽幽道:``这样的东西,留给令妃的小阿哥自然不成,给本宫的十三阿哥倒也正好!''

魏夫人愣了愣,讪笑着道:``哪儿能呢!''

李玉从袖中摸出三枚粗亮银针,一针针扎在那布偶的肚腹上,又一拇指粗的布条,上头写着生辰八字,正是戊戌年二月初十日酉时三刻。

魏夫人眼珠一眨,忙低下头道:``这个东西\ldots 妾身不知是什么?''

皇帝慢慢饮了茶水,平视着她,不疾不徐道:``这是皇后的生辰八字,这个布偶肚腹隆起,又刺银针于腹上,乃是在皇宫有孕之时对她饰以巫蛊之术。朕已经使人问过钦天监监副,乃知这是民间巫术,一可害人,二可伤子,三求断子绝孙之效。''

皇帝并不问她是否知晓,只是轻描淡写说过,仿佛只是一桩小事一般。倒是绿筠一脸不忍道:``皇上,这害子伤子已是罪大恶极,可断子绝孙,岂不也绝的是皇上的子孙!其心之毒,闻所未闻。''

魏夫人越听越是害怕,想要抬头却不敢看旁人的脸色,只得结结巴巴道:``皇上,皇后娘娘,这个怎么会有皇后娘娘的生辰八字?妾身不知,妾身\ldots{}''

祈妃鄙夷地横她一眼,冷冷道:``魏夫人的确不知,这个布偶一共有四,分别埋在魏府东南西北四角,在你进宫之后,皇后娘娘派人搜查你宅中,才见着这个。你倒不知?难道魏府私宅,不是你做主么?''

魏夫人越听到后头,越是心惊肉跳。阁中的清凉逼进皮肉里,一阵阵打摆子般森寒,和着自己失措的心跳,``噔噔''地似要蹦出嘴来。

她终于惊慌失措地抬起头来,才发觉四周之人虽然个个含着宁谧笑意,可那笑容却是催魂索命一般厉厉逼来,逼得她目眩神迷,心胆俱裂。

如懿的神色冰冷至极,如同数九寒霜,散着凛凛雪色冰气。她端坐于榻,魏夫人瞧着她容色分明,眉目濯濯,唯有尺步距离,却有冷冽星河的遥遥之感。只听她语色分明:``本宫不知如何得罪了魏夫人,竟得夫人如此诅咒?可是本宫与当日腹中的十三阿哥,何处得罪了夫人么?便是如此,稚子尚未见得天日,又有何辜?方才夫人一入门便唤贤婿,难道也要害到皇上子孙,夫人才欢喜?''

如懿语气和缓,却字字如钢刀,逼得魏夫人难以言对。

祈妃微微侧首,朝着魏夫人粲然一笑。那笑意分明是极甜蜜乖巧的,她的口吻却紧追而来:``夫人莫说不知皇后娘娘生辰。今岁皇后生辰,您托令妃送来的礼物还在库房中呢。''

容不得她有片刻的思量,祈妃又挑眉``咯咯''笑道:``莫不然当日为皇后娘娘生辰送礼为虚,蓄意诅咒谋害才是真?夫人倒真有心思啊!''

魏夫人突遭重责,一时冷汗夹着油腻嗒嗒而下,晕在水晕金砖地上,像雨天时汪着泥泞污浊的小水泡。她团着发福的身子,在地上揉成滚圆一团,讷讷声辩,虚弱地唤道:``妾身没有!妾身没有!皇上明鉴啊!''

``皇上明鉴?''绿筠声线轻绵,充满了无奈的怜悯,``证据确凿,愉妃的亲戚扎齐受不过刑撞墙自尽了。他曾去你府上,与你密谋陷害愉妃之事,也曾亲眼见你做了布偶扎银针施法,埋于府中四角诅咒皇后与皇子。莫不成他还会冤了你么?''

魏夫人尖声惊叫起来:``天杀的扎齐那浑小子,来我府上混吃混喝也罢了,还要满口胡嚼咀!我什么时候扎针做布偶了,给我天大的胆子我都不敢啊!''她又哭又喊,``皇上啊,一定是扎齐那小子羡慕咱们府上有宠,替她姑母愉妃不平,所以埋了布偶陷害妾身啊!''

如懿幽幽一叹,一弧浅浅笑涡旋于面上,衬着满殿烛光,隐有讥色:``是么?方才魏夫人不是说与珂里叶特氏府上素无来往么,怎么扎齐又去贵府混吃混喝了?''

魏夫人大怔,尚未回过神来,祈妃又犀利道:``皇后娘娘方才只问你是否与珂里叶特氏府上有来往,你却想也不想便说与愉妃小主府中并未往来,可见你所知的珂里叶特氏唯有愉妃母家而已。如此前言不搭后语,还敢抵赖说不识扎齐么?''

魏夫人张口结舌,慌不迭伏拜:``皇上,皇上,扎起已经死了!他可都是死前胡言乱语冤枉妾身的啊!什么巫蛊,什么密谋陷害愉妃,妾身全都不知!''

``不知?''祈妃满脸不信之色,``扎齐替她姑母愉妃杀人灭口,还串通接生嬷嬷田氏杀害皇后娘娘的十三阿哥!扎齐死前可是招了,他是与你商议过此事的,不是么?''

魏夫人纵是慌乱,眼下也明白一二,呼天抢地赌咒道:``扎齐那混账货色,每天只吃酒赌钱,他说的话怎么能信?皇上,攀诬皇亲这是大罪啊!妾身敢向神明起咒,绝不曾谋害过皇后娘娘、愉妃娘娘和十三阿哥!''

魏夫人声高气直,晃着胖大的身躯,一时气势不减。绿筠胸前佩一串明珠颈链,底下缀着拇指大的碎紫晶镶水绿翡翠观音像。她自年长失宠,又屡屡受挫,一心只寄望神佛,每天虔心叩拜,此时听得魏夫人对着神明赌咒,一时气不过,摘下颈链重重撂在暗紫锦莲毡上,端然正色道:``你既要对着神明起咒,不如对着它发下毒誓。若是心存良善,未曾伤生便罢,否则便坠入十八层地狱,永受轮回之苦。''

魏夫人眼神一闪,拧着脖子犟声道:``起誓便起誓,妾身不怕!''她说罢,便要举起两指起誓。祈妃``咯''的一声轻笑,冷绵绵道:``夫人要起誓,也不必那身后之事来说嘴。若是真心,不如拿儿女做赌咒。左右您是没做过的,否则呢,您的儿子佐禄沦为贱奴,受刀斩斫身死于非命之苦,您的女儿便废为辛者库贱婢,生生世世成为紫禁城的冤魂。如何?''

祈妃的笑意促狭而刻毒,与她恬美娇俏的容颜并不相符。皇帝闻言微有不悦:``祈妃,你是大家子出身,何必与她一般见识?''

魏夫人原也镇定,待听到拿她儿子做咒,不禁气得满脸涨红,眼中闪烁不定,又听皇帝出言,一时壮了胆子道:``祈妃小主纵然不喜妾身,但到底也是一宫主位,与令妃姐妹相称,怎的如此恶毒,拿人儿女做咒,难不成祈妃小主便没有儿女么?''

这话不说便罢,祈妃幼女夭折怀中,乃是毕生大痛。登时跪下道:``皇上宅厚,所以细细查问,但臣妾深觉此事不审也罢。巫蛊之事出于魏氏宅中,何人可以冤屈?且扎齐出入魏府,也有下人眼见。另则李公公带人搜了魏府,府中所有金银珠宝,大多出自宫中,可见令妃虽然身在宫中,但与家中密切,保不齐此事也有参与!''

绿筠不禁恻然,取了绢子拭泪道:``皇上,可怜天下父母心。魏夫人与皇后娘娘、愉妃有何冤仇,不过是为了女儿的缘故。这件事若说令妃能撇清,臣妾也不大信。''

皇帝略略沉吟,安抚地搭上如懿的手,轻声道:``令妃有着身孕,凡事格外小心,平时连蚂蚁也不敢去踩一只。且她一直未有身孕,好容易怀着第一胎,日日拜佛,她便要作恶,也不敢在这时候。''

如懿忍着心头隐怒,含了一缕凄恻之意,勉力笑道:''皇上安心。臣妾敬重魏夫人年长,令妃有孕,也不敢过于责问,免得惊着她们,所以已让凌云彻带了佐禄入宫盘问,想来也快有结果了。''

皇帝听得说起佐禄,细想了片刻,方道:``是令妃的弟弟?朕见过他一回,不是大家子弟的风度,便也不曾与他说话。''

如懿心中微微平定,淡淡瞟了祈妃一眼,将她唇边将溢未溢的一丝喜色弹压下去,欠身道:``人谁无过、只在罪孽大小。臣妾的孩子固然死得不明,但也不可让旁人受屈。请佐禄来问一问,一则免得惊吓女流,二来听闻佐禄在外一直依仗国舅身份,给他几分教训也好。''

绿筠颇有惊诧之意,摆首道:``什么国舅?正经皇后娘娘的兄弟还未称国舅呢,他倒先端起架子来了。''她横一眼底下跪着的魏夫人,撇嘴道:``总没有谋害皇子与皇后之事,巫蛊之事你总是脱不得的。且又教子无方,纵着儿子横行霸市,算得什么额娘!''

魏夫人本还充着气壮,待闻得佐禄已然入宫别置,神色大变,只得硬着头皮求道:``皇上,佐禄年幼无知,受不得惊吓,只怕胡言乱语,有伤圣听。''

皇帝捧了茶盅在手,心不在焉道:``胡话也是话,朕倒要听听,他能说出什么来!''

魏夫人自知无法,只逼得满头沁出细密冷汗,又不敢伸手去擦,窘迫不已。

不过半柱香时间,凌云彻恭身入内,将一张鬼画符般的布帛交到皇帝手中,肃然立于一旁。皇帝展开布帛,凝神望去,越看脸色越青。那佐禄大字不识几个,字迹歪七扭八,看着本就吃力,又兼文理不通。皇帝只读了个大意,见他语中颠三倒四,虽不说事涉嬿婉,总不离七八,又说起与扎齐喝酒赌局之事,倒也看出个大概。

凌云彻见皇帝恼怒,恭恭敬敬道:``微臣还未来得及问佐禄,他只看见扎齐尸身,便吓得尿了裤子,说话前言不搭后语。微臣问了几句,巫蛊之事大约是女流之辈所为,他并不清楚。但说起与扎齐在哪里喝花酒赌蛐蛐儿,倒是有地方也有人物,想来不假。问起他家财物,也尽说是令妃小主给了魏夫人的。''

魏夫人听得佐禄供词,又气又恼,更兼仓皇神色,满面油汗滴落,正要强辩,只听得一声锐呼:``额娘!你怎会背着女儿做出如此不堪之事?''

那声音甚是尖锐,带了悲切而惊异的哭腔,将殿中的紧张锋利划破。进忠在后头扶着嬿婉,急得赤眉白眼道:``令妃小主,您小心玉体啊!''

嬿婉跌跌撞撞进来,顾不得行礼,扑倒在魏夫人身侧,满面是泪;''女儿不知,您竟然做下这种伤天害理之事,诬陷愉妃,害死皇后娘娘的孩子!额娘,女儿真不能相信,您为何如此?''

魏夫人本就惊慌,听得嬿婉如此说,更是吓得面无人色,颤颤失声:``令妃\ldots 嬿婉\ldots 你这样说额娘!不是我\ldots 不是\ldots{}''

嬿婉扑在魏夫人跟前,紧紧握住她的手:``额娘,这件事是不是你做的?你万万想明白,一步行差踏错,连累女儿不算,别人也会说你教子无方啊!''

魏夫人面上一阵红一阵青,慌不迭摆手:``嬿婉\ldots 你别\ldots{}''她咬着牙,急欲撇开嬿婉的手,``你别冤枉额娘!''

嬿婉死死掐着魏夫人的手,泣道:``额娘!女儿知道,没做过的事您不能乱认!可这件事到底真相如何,您可别害了女儿和弟弟啊!''嬿婉将``弟弟''二字咬得极重,拉扯着魏夫人的衣袖,一双澄清眼眸瞪得通红,似要将她苍白浮肿的面孔看得透彻,``额娘,弟弟还小,他什么都不知道。他只是一时糊涂,才会和扎齐有所牵连。额娘,您别害了弟弟,他还有得救,只要女儿好好管束,不像您一味宠溺,弟弟他会好的。''

嬿婉的情绪过于激动,满面血红欲滴。春婵紧紧扶牢了她,含泪劝道:``小主,小主您别急!这些日子虽说是夫人来看你,可为了舅老爷,您与夫人争了几回,都是自己忍着,家丑不可外扬啊!''

魏夫人梗着嗓子大口大口喘着气,似乎不如此便要历史魂断当场。只见她满脸泪水止不住地潸潸而落,惊惶地大力摇着头,一任泪水湿透衣襟,却说不出半句话来。

\hypertarget{ux7b2cux4e8cux5341ux4e94ux7ae0-ux65adux8155}{%
\chapter{第二十五章
断腕}\label{ux7b2cux4e8cux5341ux4e94ux7ae0-ux65adux8155}}

祈妃一径蹙眉:``令妃妹妹,皇上面前,你这般拉拉扯扯算什么样子,难不成你还要逼迫你额娘吗?''

也不知过了多久,魏夫人的神色终于渐渐平静,只是那平静如同死亡般枯槁幽寂。她无声地抽泣着,忽地甩开嬿婉紧紧攥着的手,匍匐着膝行到皇帝跟前,抱住皇帝的腿,用尽全力呼道:``皇上!都是妾身糊涂,是妾身的罪过!''

皇帝目光微凉,淡淡道:``罪过?你有什么罪过?''

魏夫人的唇被白森森的牙齿咬破,沁出暗红腥涩的血液:``一切罪孽都是妾身做的!皇上明察秋毫,妾身无可抵赖。但这件事\ldots{}''她狠一狠心,``这件事与佐禄和令妃都无关系。令妃身怀六甲,根本不知道妾身做的这些事,佐禄也是蒙在鼓里,受妾身驱使而已。他\ldots 他就是个糊涂人,年纪又小,只知道听妾身的话,什么都不明白!''她说着,不由得痛哭失声。

嬿婉跪伏在地,吃力地托着腰身,嘤嘤而泣:``额娘,你怎么会变得这样!佐禄是好吃懒做,是不识大体,可他孝敬您,听您的话,您却让他蒙在鼓里,用他去做这些丧尽天良之事!''

魏夫人红着双眼,推开嬿婉即将触到自己身体的手,恨声道:``事到如今,还说这些做什么!你怀着身孕不便知道这些事,额娘替你料理了,也是成全你的前程。这样的事,你从前不知道,现在也不必知道!''

绿筠犹自愤愤,且又惊疑:``你与皇后娘娘无冤无仇,何必做下这些孽事?''她瞥一眼嬿婉:``若说是令妃,倒有争宠作孽的嫌疑!''

``令妃争宠?她有什么本事争宠,老实又无用的坯子,我怎会生出她这样一个东西来,凡事还要我替她操心!''她喘息着,拧着嬿婉地胳膊道:``你出身微贱,又不懂争宠!皇后的孩子一个个生下来,你的算什么东西!不过是和纯贵妃的儿子一般,一个不当心便被皇上瞧不起。且你这些年受的苦,哪件又和皇后脱得了干系,被淑嘉皇贵妃欺凌,又几度失宠,都是皇后使的手段!要不是你蠢钝愚笨,怎会落得这番田地!但是额娘不甘心,额娘咽不下这口气,不能眼睁睁看着你糊涂无能,被人欺凌!''

祈妃禁不住倒吸一口冷气:``这话说得实在诛心!令妃得宠失宠,自是她自己的事,与皇后何干?与皇后腹中皇子何干?自己生性狠毒,却要扯上旁人,算得什么!''

魏夫人双拳紧握,看也不看掩面痛哭的嬿婉,扬着脸道:``皇上,一人做事一人当。扎齐是妾身所害,愉妃是妾身所冤,皇后和她腹中皇子也是妾身买通了田嬷嬷所害!妾身无话可说,愿意伏诛!''她眼中流出浑浊的泪,凄厉道:``可是皇上,这件事与妾身的儿子佐禄无干!他只是个不成器的孩子,什么也不知道!也\ldots{}''她瞥一眼娇弱欲坠的嬿婉,极力忍着道:``也与令妃娘娘无干!''

嬿婉嘤嘤啜泣不止:``额娘\ldots 额娘\ldots{}''

如懿望向嬿婉的目光毫无温度,语意冰冷:``用自己和弟弟的前程来要挟你额娘,本宫倒是没想到,你有这般胆气!''

嬿婉素日红润的面庞泛着苍苍微青,她伏在地上,仰起脸看着如懿,似一缕卑微到极处的尘芥,盈盈含泪,无限委屈道:``额娘罪有应得,便是伏法当诛,臣妾也不敢有二言。但皇后娘娘此言,莫不是一开始便要借额娘之错来索臣妾之命。若是如此,臣妾便将腹中孩儿与臣妾之命一并送给了皇后娘娘吧!''

她的眸中尽是苍茫的委屈与哀伤,如白茫茫的洪水,汹涌而来。可是那眼底分明有一丝深深的怨毒,锥心刺骨,向如懿迫来。

绿筠性子再温和也忍不住打了个寒噤,讥诮道:``你腹中孩儿是皇家血脉,不过借你肚腹十月,你有什么资格断他生死,还要送给皇后娘娘!你倒拿着皇上孩儿的性命予取予求么?''

祈妃亦嫌恶道:``怀胎十月的辛苦谁不知晓,拿着孩子说嘴,是要以此要挟皇上和皇后么?''

皇帝断然喝道:``放肆!''

这一语,也不知是怪祈妃还是嬿婉。如懿以温然目光相承,悲悯而淡然:``你真的要以腹中孩儿轻言生死么?''

嬿婉亦知自己出言轻率了,然而如懿的目光看似温润,却如利剑逼得她无所遁形。她心下更急,更觉得腹中抽痛,她一咬牙,猛地抬起腰肢,一个不稳又踉跄斜倒于地上。剧烈的起伏扯动她腹中隐隐的疼痛,心头闪过一丝暗喜,这个孩子,是来救她的,居然此时此刻动了胎气。她死死地抵着疼痛蔓延而上的脱力感,拼着全身的力气厉声唤道:``皇上,臣妾出身寒微,便是谋害皇后娘娘与愉妃,于自己在宫中又有什么好处?蒙此冤屈,臣妾不甘啊!''

她的哭喊撕心裂肺,更兼着满脸痛楚,实是凄绝!

如懿深吸一口气:``皇上,臣妾不相信巫蛊,但臣妾相信人心之毒,可以无所不用其极。今日下的手可以黑到臣妾生产时的接生嬷嬷,可以让臣妾的皇子死得如此冤屈,那么来日,宫中皇嗣的生死,都要落于令妃母女手中么?''

有片刻的寂静,所有人的眼光都聚焦于皇帝,殿中只闻得嬿婉极度压抑、痛楚的呻吟。那呻吟声渐渐难以忍耐,还是进忠发觉异样,惊呼道:``皇上!血!血!''

众人凝眸望去,只见嬿婉裙脚隐约有血色蜿蜒。她捧腹蹙眉,冷汗淋漓,凄楚道:``皇上!皇上!''

绿筠不由得有些着慌:``皇上,看样子,令妃怕是动了胎气,要生了!''

祈妃纵然气盛,可看着嬿婉临产痛楚,不免也软了神色。

嬿婉的目光缠绵而悲切,迟疑地看着皇帝,唤道:``皇上\ldots 皇上\ldots 咱们的孩子\ldots{}''

皇帝略一迟疑,深深望一眼忍痛不已的嬿婉,斑驳的血色似未能打动他的冷峻:``祸乱宫闱者,不可不严惩!魏杨氏狂悖,谋害皇嗣,即刻拖出去,赐毒酒!''皇帝缓和口气,``但魏杨氏难得进宫,令妃到底身在宫中,并不深知底细。何况令妃到底有身孕,即将临盆\ldots{}''他的眼底有无法掩饰的为难,投映于如懿眸中,``那也是朕的孩子。''

嬿婉听得皇帝之令,几欲昏厥,却在惊痛中极力撑住了自己,压抑着哭泣:``臣妾谢皇上留额娘全尸。''

魏夫人面如死灰,被小太监拉扯着往外拖去。在经过嬿婉时,她骤然暴起,死死抓住嬿婉裸露的手腕,想是用劲太大,嬿婉腴白的肌肤被抓出深深的印痕。魏夫人目眦欲裂,凄厉道:``你说的!是你说的!佐禄\ldots 你会好好管束佐禄!''

嬿婉哽咽着连连顿首,急欲脱开魏夫人的牵扯:``额娘,皇上留您死后的体面,不让您身首异处,您要谢恩。''她的眼底蓄满了泪,叩首连连:``皇上,臣妾会拿一辈子谢您的恩情和体面!''

魏夫人再无言语,直挺挺倒在地上,被进忠拖了出去。嬿婉掩袖欲哭,禁不住腹中刀绞般疼痛,终于呜咽着痛呼出声。

如懿微微定住,到底无法说出口。她是怕的,是真的。曾经无法生育的年岁里,她真是恨,恨得牙齿都咬碎了,硌着满口的碎棱坚角,一口口往下吞。她是恨的,所以在冷宫绝望的岁月里,明明知道那些棉絮和芦花会害死孱弱的永琏,她还是告诉了海兰,由着海兰和绿筠用共同的仇恨,将那个小小孩子送上死路。

可是那时没有想过,有一日,她会活着出了冷宫,可以呼吸着冷宫之外不曾腐败的空气,她会一步步走到后位之上,会有自己的孩子。

那种隐藏着的罪悔,是日日夜夜的折磨。

海兰不害怕,因为她是海兰,无所畏惧的强大的海兰。她害怕,她愧疚,她忏悔,因为她有那么多的牵挂,因为她不曾想过,许多年后,她也会饱尝丧子之痛。

这样的静寂,还是绿筠率先打破。她捻着手腕上十八子蜜蜡珊瑚珠手串,面色微白:``去母留子,也是可行之道。''

如懿瞬间睁眸,意识到皇帝是不会这般做的,不为别的,只为皇帝亦是失母之人。她深深呼吸,压制住功亏一篑的颓败感,轻缓道:``找个妥当的接生嬷嬷,照顾令妃生产。''她欠身:``皇上,那么臣妾,亲自去接愉妃出慎刑司。''

皇帝颔首,微觉歉然:``愉妃无端受此冤屈,是该皇后亲自迎接,才可平息流言。''

嬿婉被王蟾扶着扶着上了软轿,浑身被巨大而陌生的疼痛绞缠着,忍不住哭出声来。春蝉两手发颤,抓着嬿婉的手道:``小主放心,即刻就到永寿宫了。太医和接生嬷嬷很快就会到!''

嬿婉扭着脖子看着身后渐行渐远的翊坤宫,泣道:``皇上,皇上\ldots{}''

春蝉难过而不安:``小主,皇上是不会来的。您安心,安心生下一个皇子,事情便会有转机的。''她说罢,又急急催促抬轿的太监:``快些!快些!没看小主受不住了么!''

太监奔走时衣袍带起的风显得杂乱而灼热,而另一种绝望的哭泣声,唤醒了嬿婉疼痛的神经。她慌慌张张直起身子,寻觅着那哭声的来源,戚戚唤道:``额娘!额娘!''

甬道的转角处,嬿婉骤然看到魏夫人被拖曳的身体,她再也忍耐不住,放声痛哭。春蝉见机,忙上前几步,拉住为首的进忠,切切道:``进忠公公,看在往日的情分上,您让小主和夫人再说两句话吧。就当送夫人最后一程。''

进忠为难地搓着手,看见软轿上的嬿婉又是疼又是哭,跺了跺脚,退到一旁道:``好吧!可得快点儿,否则连我的脑袋也得丢了。''

春蝉忙忙答应,示意小太监们轻稳放下软轿。嬿婉忍痛扑向魏夫人的身体,哭道:``额娘,额娘,对不住!女儿保全不了你!''

过于沉重的绝望让魏夫人保有了难得的平静,她目光凌厉;''我不只为了你,更为了佐禄!''

嬿婉热切的悲哀倏然一凉:``原来到了这个时候,额娘最放心不下的还是佐禄!''

魏夫人狠狠盯住她:``你为了自己连额娘都可以要挟!哼哼!我和你阿玛早知道,女儿是靠不住的!''她迫视着嬿婉,``佐禄,他是魏家唯一的男丁,唯一的血脉。你给额娘发誓,无论如何,都会保全他,护着他,就像护着你肚子里的这块肉,护着魏氏满门未来的希望!''

一语催落了嬿婉无尽的热泪,她咬着唇,极力道:``额娘,女儿听您的话,您不会白死!''她伤心欲绝,忍不住低低呼痛。

魏夫人强打起精神,喘着粗气道:``嬿婉!是你蠢!是额娘蠢!咱们一直费尽心机,想要铲除一个个障碍,殊不知却舍大取小,走了无数弯路!''

嬿婉咬得唇色发紫,急切道:``额娘,您说什么?''

魏夫人照着自己的面孔狠狠抽了一个耳光,抽得嘴角淌血。她嘶哑着声音道:``嬿婉,额娘算是看清楚了!除去谁都没有用,绞尽脑汁,用尽手腕,还不如专心对付一个!''

嬿婉惊呼:``皇后!''

魏夫人切齿道:``是!除去她的孩子算什么,她照旧是皇后!还不如一了百了,将她扳倒。算命的仙师说了,你是有运无命,那贱人是有命无运!就凭着这句话,你一定要夺了她的皇后之位,让她生不如死!''她还欲再说,进忠忍不住催促:``小主,拖不得了!您也得留着奴才的脑袋好给您效力啊!''

魏夫人灰心到底,泫然含悲,被进忠拖着,一壁低呼:``嬿婉,额娘能帮你的,只有到这里了。你自己\ldots 你自己\ldots 好好护着佐禄,别负了额娘用命换的\ldots{}''

带着暑气的风潮湿而黏腻,将她悲切的尾音拖得无比凄厉。嬿婉想要追上去,却被身体的剧痛扯住,险险跌倒。春蝉与澜碧慌得相对哭泣,拼命扶住了嬿婉,茫然四顾,忽然叫起来:``小主,齐太医来了!小主,齐太医来了!''

海兰扶着宫女缓缓走出,有些跌跌撞撞,不大稳当。她精神倒还好,瘦了一圈,也憔悴了不少,好像一夜之间便苍老了五六岁,但眉目间那种濯濯如碧水春柳的淡然却未曾淡去,还是那样谦和,却透着一股什么也不在意的气韵。

她的脚步有些滞缓,慢慢地,一步又一步,好似许久不下床的人终于踏到了坚实的地面,脚步却是那样绵软。叶心与春熙一边一个扶着她,也甚是吃力。

如懿领着永琪候在慎刑司门外,见了她出来,忙伸手稳稳扶住她的手肘。永琪早已泪流满面,跪下叩首道:``额娘!额娘!''

海兰深深地看他一眼,伸手拉他起来:``还好,尚不算过于毛躁。''

如懿握着她薄如寸纸的手腕,不觉深皱了眉心:``瘦了好些,都能摸着骨头了。''

海兰见了如懿,想要展颜笑,却先是落下泪来:``姐姐。''她见如懿一脸担忧,忙道,``这些日子你也不好过吧?''

如懿爽然一笑,眸中闪过一点流星般微蓝幽光:``撒网收鱼,总比浑浑噩噩任人鱼肉好得多。''

海兰半靠在如懿身上,低声道:``我听叶心学舌,似乎是为了巫蛊之事?''

如懿不以为然,面上笑涡一闪:``药引子而已,否则怎见药力?''

``真有其事?''

``去搜魏府的人是李玉带去的,做些手脚也不算委屈了他们。若无巫蛊事,哪里勾得清皇上心底余毒,既然他总以为是本宫妨害自己的十三阿哥,相信天象祸福之说,那么巫蛊毒害,他也更会相信。''

海兰颔首,含了安定之意:``是。我们已经忍得太久。只是折损了姐姐的一个阿哥,才换了他额娘的一条命,实在太不上算!''

``不管什么命,都是人命!本宫所要的,不过是一命抵一命。如今她失宠于皇上,她兄弟佐禄也没了依靠,如同丧家之犬却还成日惹是生非,也够叫她伤神的了。''

海兰不肯放心:``姐姐真觉得令妃会安分守已?''她侧耳倾听,``是谁在叫喊?是令妃要生了,是不是?''

``管她生什么。她已是无依无靠,唯残命而已。若是赶尽杀绝,反而叫皇上疑心。''如懿无端生了几分疲累,``本宫与皇上之间,彼此疑心至此,若不再留三分余地恐怕伤人一千,自损八百,反而不好!''

海兰嗤嗤一笑,眼中尽是不屑:``姐姐还是在意皇上?''

如懿的忧郁凝于眉心:``不是在意皇上,是在意`夫妻'二字。本宫与皇上少年相伴,悠悠数十载,难不成要为了旁人走到分崩离析之地么?''

海兰浑不在意,拍去衣上尘灰:``此事之后,皇上可曾好生安慰姐姐么?''

``事过境迁,安慰有何用?本宫与皇上都已过了半生,即便年华渐去,又连遭创痛,容色朽顿不如年轻的嫔妃了。但偶尔见见,闲话儿女,便也全得过情面了。''

海兰一笑,大大方方道:``姐姐这话,说的倒是我了。'

``所以皇上喜欢谁,由着他去便是。本宫只瞧着你,别再吃这样的暗亏就好。''她怜爱地看着海兰,伸出手为她细细理顺凌乱鬓发,柔缓道:``在慎刑司受苦了,本宫让容珮炖了你最喜欢的山药莲子炖水鸭,此时估计烂烂的了,正好入口。''

海兰轻笑,神色亦活泛许多:``有姐姐的嘱咐,虽然所住牢笼窄小,不便伸开手足,但心里安宁,倒也不算受苦。''她看着永琪,一双明眸似要看得他成了个水晶人:``听说你到底沉不住气,去求了皇额娘救我,是么?''

小小的少年面上尽是赭色,忸怩不堪。

海兰凝视着他,笑影渐渐收敛:``你这般做,便是不信你皇额娘会真心救助于我,才做出这般丑态,是么?''

如懿按住她的手,微微摇头:``到底是小孩子,咱们什么都瞒着他,他是你亲生子,难道无动于衷?也幸好他急得日日来叩首,旁人才信本宫真厌恨了你,才能被咱们找到蛛丝马迹。''

海兰盯着羞愧的永琪,见他越发低下头去,摇首不已:``你皇额娘疼你,才为你说话。今日额娘告诉你明白,你的错,一是轻信人言,二是疑心嫡母,三则救助亦无方向。你知道额娘是因十三阿哥缘故而进慎刑司,皇后为十三阿哥生母,若无额娘与你皇额娘情分,你求之何用?''

永琪满眼是泪,强忍着不敢去擦,只得生生忍住道:``可是求皇阿玛和太后娘娘也是无用的。''

``当然无用!''海兰断言道,``乱花渐欲迷人眼,此时你更要留心你皇额娘与皇阿玛的举动,看看是否有可以助益之处。再不然,李玉和凌云彻处都可旁敲侧击一二,何至于做出这般慌乱无用之举。要知道,为人处世,一旦过于急切,便会乱了分寸,败相尽现。''

永琪被训得面红耳赤,嗫嚅分辨道:``儿子当然是信皇额娘的\ldots{}''

海兰深深剜他一眼,含了沉沉的失望,道:``虽然信任,却不能一信到底,不能贯彻始终,便是你最大的错处!''

永琪喃喃着想要辩白,如懿温和地目视他,抚着他的肩膀:``皇额娘知道,你虽年幼,却饱经世态炎凉,知道一切要靠自己,要信自己。但,本宫虽是皇后,是永璂额娘,也是从小教育着你的额娘。''

永琪俊逸的面庞涨得通红,深深叩首,默然不言。

七公主的平安诞落,已经是一夜之后。

此时的永寿宫已经人仰马翻,人人自危。只春蝉与澜翠两个大宫女还在旁殷勤服侍,底下的人全不知避到何处去了。放眼阁中,唯有几个接生嬷嬷,有一搭没一搭地忙着。

嬿婉从阵痛中苏醒过来,眼底干涸得没有一滴眼泪,凄惶地望着阁顶销金菱花图样,那点点碎金成了落进眼底的刺,深深扎进软肉里。她的咽喉因为长时间生产时的疼痛呼喊而沙哑,却依旧喃喃:``怎么会是公主?怎么会?''

春蝉怯怯宽慰:``小主别这么着,月子里伤心是要落下病根儿的。公主,公主也好。公主贴心呢。''她极力转着脑子,``小主您忘了,比起皇子,皇上也更喜欢公主呢。''

嬿婉听得``皇上''二字,微微挣出几分力气:``皇上,皇上知道了吗?''

春蝉与正端进热水的澜翠对视一眼,还是道:``皇上已经打发毓瑚姑姑来看过一眼,回去复命了。''

嬿婉眼底的热切被浇灭殆尽:``皇上和本宫一样,都盼着是位皇子!为什么偏偏是个没用的公主?若是皇子,本宫便有办法脱出困境!为什么?''

春蝉吓得赶紧捂住她的嘴:``小主!小主!公主也好,皇子也好,您总算母子平安,也不枉夫人\ldots{}''她有些畏惧,``方才进忠来回话,夫人已经上路。小主,您可别忘了夫人临终嘱托,一定得善待自己啊!''

正说着,七公主嘤嘤哭了起来,她的哭声极其微弱,也怕吵着伤心烦恼的嬿婉似的。不知怎的,这小儿的哭声便触动了嬿婉的心肠,终于叹口气道:``抱来给本宫瞧瞧。''

澜翠见嬿婉有兴致,忙抱了七公主上去,喜滋滋道:``小主快看,七公主长得多好看!''

嬿婉恹恹地瞥一眼红锦襁褓中的婴孩,皱眉道:``脸皱巴巴的,没有本宫好看,也不大像皇上。''

澜翠吐了吐舌头:``孩子小时候都这样,长大就好看了,女大十八变哪!''

嬿婉随意抚了抚七公主的小脸,疑道:``怎么哭声这么弱?是不是饿了?''

乳母是早已挑好的韩娘,她上前福了一福,抱过公主哄着道:``回小主的话,公主喝过奶了,就是身子弱。小主是头胎,生得缓慢,公主也遭罪些。''她掰着指头,``哎呦!今儿已经是七月十六了。公主是昨夜生下的,正好是七月十五的中元节!''

另一个乳母``哎呦''一声,嘴快道:``中元节,可不就是鬼节嘛!''

春蝉凶凶地横了乳母一眼,怒道:``嘴里胡嚼什么!公主也是你们能议论的?还不赶紧抱下去喂公主!''

乳母们抱着公主讪讪退下,外头隐约还有谁嘟囔:``神气什么!生了公主皇上也不来看一眼,早就失宠了的,还威风八面的!''

``七公主出生的日子可不好,和前头淑嘉皇贵妃的八阿哥一样,都是鬼节生的。''

``你们瞧八阿哥,那条腿好了也是一瘸一拐的。咱们七公主也可怜,令妃娘娘又是这个境地,可见是被她额娘连累透了。''

``一辈子就只能得这么一个公主了,公主能算什么依靠呢?连愉妃都不如,只怕这辈子都完了。''

所谓的绝望,大概就是这样毫无希望。原本意料中的锦绣人生,会因为突如其来的失算,全盘崩溃。

她望着窗外凄寒如雪的月光,揉了揉干涩的眼,哑然哭泣。

生下公主后的数日里,嬿婉抱着小小的,瘦弱的婴孩,听着她哀哀的像病弱小猫般得哭声,仿佛也在替自己申诉着无尽的委屈、失望、惶恐与愤恨。

人人都以为她完了,是么?恍惚的一瞬间,连她自己也这么觉得,却又很快安慰自己,还年轻,一切还可以重头来过。

嬿婉无声落泪。仿佛只有这温热咸涩的泪水,才能抵御四面八方汹涌而来的惶惑。正默默念想间,却见李玉带着两个小宫女进来,恭恭敬敬向她请了安道:``令妃娘娘万福。''

嬿婉几乎是欣喜若狂,慌慌张张擦了泪,忙不迭起身道:``李公公来了,可是皇上想念公主,要公公抱去么?''

李玉的笑容淡淡的,维持着疏离的客气,像冬日里的毛太阳,明亮,却没有热度。``回小主的话,皇上是惦记着七公主了。但想着小主还在月子里,亲自照拂不便,所以特命奴才带了去。''

嬿婉一怔,大为意外:``公主还那么小,便要抱去阿哥所了么?''她慌里慌张,``公主还小,离不得额娘。''

``小主此言差矣。宫中规矩,若非皇上特许可由亲娘养育,皇子和公主都会交由乳母在阿哥所带着,或是交给身份更尊贵的嫔妃为养母。''李玉道,``皇上的意思,颖嫔小主膝下无子却出身高贵,可以替小主抚养七公主。''

澜翠失声唤道:``怎么会?颖嫔小主只是嫔位,我们小主可是妃位啊!''

李玉沉下脸道:``颖嫔小主虽然是嫔位,但却出身蒙古贵戚。颖嫔小主又是诸位蒙古嫔妃之首,其贵重受宠,岂能只按位分序列。''

澜翠深知嬿婉对七公主身为女儿身颇为失望,但也知道这个孩子的要紧,欲再分辨,但见李玉神色冷淡,也只得噤声了。

\hypertarget{ux7b2cux4e8cux5341ux516dux7ae0-ux5973ux5fc3}{%
\chapter{第二十六章
女心}\label{ux7b2cux4e8cux5341ux516dux7ae0-ux5973ux5fc3}}

嬿婉惨白着脸,紧紧拥住怀中的孩子,一脸不舍。她是再清楚不过了,从此之后,皇帝若想起这孩子,自会去颖嫔处探望。便是养在阿哥所还好些,他可以买通了乳母多多美言,引得皇帝来看自己。若是去了颖嫔处,又有哪个乳母敢多言。自己的血脉,到最后竟成了为他人作嫁衣裳了。她凄声喊起来:``不成的!李公公,求您告诉皇上,颖嫔年轻没生养过,又要常伴圣驾,哪里得空儿抚养孩子,还是留在本宫这儿吧。''

李玉公瑾垂首,不疾不徐道:``皇上倒是想把七公主送去位分高的娘娘们那儿,只是怕小主没脸面罢了。皇后娘娘虽是嫡母,但魏夫人做出那些事儿,皇上怎还肯为难娘娘抚养小主的孩子。便是纯贵妃和祈妃、愉妃三位小主,一听也是摆手,说是实在不敢!得,皇上千挑万选,顾虑着公主的前程,好歹选了颖嫔。您要还觉得不成,那奴才只好去回皇上的话,您静听皇上的处置吧。''

嬿婉久在皇帝身边,自然明白李玉话中的利害,忍了又忍,只得哀哀道:``李公公,没有旁的法子了么?''

李玉摇头道:``皇上还肯费心为七公主找位养母,便算是尽心了。''他一抬下巴,两位小宫女晓得厉害,动作利索地请了个安,径自从嬿婉怀中抱过了孩子,便去招呼乳娘们跟上。

嬿婉见状便要哭。李玉笑吟吟道:``小主别急,祖宗定下这样的规矩,也是希望嫔妃们能更好地伺候皇上,别被孩子拉扯了恩宠。您呀,别哭,哭坏了眼睛,还怎么伺候皇上呢。''说罢,便抱着公主,自行告退。

嬿婉直直噎住,欲哭无泪。恩宠,她哪里还能指望恩宠呢,连最后一道博得垂怜的法子都被收去,还要生生承受这般锥心之语。她低低啜泣,无语望天:``额娘,我没有办法了,我真的没有办法\ldots{}''

澜翠见她伤心,忙递了绢子为她擦拭,手忙脚乱劝道:``小主,嬷嬷交代了,月子里不能哭,伤眼睛呢。''她说着,便急着看一旁的春蝉:``素日你最会劝小主了,今日怎么都不作声!''

春蝉立在门边,暗红朱漆门勾勒得她穿着暗青素衣的身量格外醒目而高挑。她袖手旁观:``小主如今成壮士了。壮士断腕固然痛,可只有痛才能提醒自己还活着。小主忘记当年和奴婢在花房受苦的日子了么?皮肉之苦已然熬过,再受得住这离丧之苦,小主便再无畏惧了。''

嬿婉泪眼婆娑:``壮士断腕?''

春蝉定定道:``是。小主舍得夫人,舍得在宫外的荣耀,从花房的奴婢到启祥宫的宫女,从官女子的位分上开始熬起,都是为了什么?不为别的,只为自己。''她斩钉截铁,``都为了自己的尊荣,这也是奴婢跟着您死心塌地的原因,咱们都盼着自己好。您的娘家,您的额娘和弟弟,其实说白了帮不上小主分毫,甚至夫人还偏心,拿着小主的体己一味宠着舅少爷。''

嬿婉喃喃嗫嚅:``是。皇上最不喜欢嫔妃娘家显赫,即使张扬些也不喜欢。与其如此,还不如断得干净。''她的目光逐渐清明,``孤身一人,无可依靠,才能紧紧靠着皇上。''

春蝉取过象牙妆台上一瓶青玉香膏递到嬿婉手中,柔声道:``听嬷嬷说,月子里的女子气血两虚,面浮眼肿,必得好好调养,才能美艳如昔。''她看一眼澜翠,``澜翠,还不恭喜小主?''

澜翠浑然不知,奇道:``恭喜?''

春蝉笃定笑着道:``小主一直希望有所生养,为此费心多年。如今得偿所愿,生下公主,可知小主体健,以后生养无碍。且民间说,先开花后结果,小主能生公主,就能生皇子。''

嬿婉的容色渐渐坚定:``是了。只要本宫还能得到皇上的恩宠,便总有一日能生出皇子来。''她忽而泄气,``可是虽有额娘担着罪名,可皇上也不会再宠爱本宫了。''

春蝉取过一面铜鎏金芭蕉小靶镜为嬿婉照着,笑盈盈道:``小主对镜瞧瞧,虽然生下公主才三天,又经丧母之痛,但容颜未减,反增楚楚可怜。皇上最爱的,便是这种柔弱美人。只要熊阿朱沉下心气悉心调理,一定会容颜更胜往昔。至于公主嘛\ldots{}''她微微一笑,``送去颖嫔那儿也好,颖嫔自己没有孩子,不会不疼公主,她又是个急脾气,只怕有的忙活呢。''

嬿婉用手指拨开凌乱垂落的发丝,心神渐定:``人之将死,其言也善。额娘说得对,皇后她断了本宫的荣耀、家族的指望。额娘死了,家也没了,但只要本宫剩着,就不算完!''

盛夏漫过,天气渐凉。皇帝来翊坤宫的时日渐渐多了,日子,仿佛又回到了从前不咸不浅的时光,就如那些惊涛骇浪的起伏,从来没有发生过。

抬头望去,红粉盛年,流淌于红墙碧苑。

海兰还是常常来与如懿闲话,二人并肩立于廊庑之下,远眺着殿脊飞檐,重叠如淡墨色的远山,看黄叶落索,飘零坠坠。

海兰见到皇帝还是那么落落大方,谦和自持,仿佛从未有过慎刑司的困辱与窘迫。她如此淡然,皇帝反而有些不好意思,屡屡赏赐,又对她和永琪关怀备至。然而海兰却对琳琅满目的赏赐付诸一笑:``臣妾侍奉皇上多年,牙齿也有磕着舌头的时候,何况长久相处呢。皇上不提,臣妾都忘记了。''

如此,皇帝讪讪之余,对海兰也越发敬重。

无人时,如懿便笑她:``真能心无芥蒂,忘却蒙冤不白之苦?''

海兰横眉:``自然不能,我从未忘记,我所有的辛苦颠沛、荣华寂寞,都是拜他所赐。必得感恩戴德,铭记于心,终生不忘。''她看如懿,颇有问询之意,``自十三阿哥离世,历经风波,姐姐对皇上似乎也有所不同?''

``能有如何不同?不过是明白你多年劝道终究成真。许多夫妻无情无爱,也可以平淡一生。省得爱恋纠葛,在乎越多,伤得越深。''如懿伸手接住一片坠落于枝头的黄叶,脆薄的行将碎裂的触感让她感伤不已,``多年夫妻,有时候皇上如此疑心,真叫人心寒。''

``多年夫妻?''海兰瞠目,``便是猫儿狗儿,养了几十年,也有些情分。''她出言犀利如锋,``有些事,姐姐难道未曾发觉么?我早已失宠,多年不曾侍寝,又与世无争,为何皇上会轻信他人?只因永琪一日日长大成才,皇上虽然器重,只怕也因当年永璜之事,对年长的皇子颇为忌惮了。''

如懿念及永璜的英年早逝,不觉泪眼潸然:``木秀于林,风必摧之。生于皇家,太过庸懦自然不好,可若格外出挑,也是一桩心病。''

海兰颔首,挽住如懿的手臂:``姐姐,我原想着自己出身小姓,没什么家世,想替永琪娶一位才德双全又出身世家的福晋,也好有所助益,现在看来,怕是不成。''

如懿触动心思,连忙道:``你说得极是。家世过于显赫,难免依仗母家权势,但若太寒门小户,也委屈了永琪。你的心思本宫明白,无非是向皇上示弱,表明永琪安分守己。''

海兰长叹一声:``我与皇上,虽不敢称夫妻,但也是妾侍。非得以前朝君臣之道来维系保全,实在也累得慌。''她望着如懿的眼,``可我知道,姐姐比我更难。我的委屈,不过是蒙冤,而姐姐,却实实在在饱尝丧子之痛,还被皇上冷落疑忌。姐姐真的可以释然么?否则每天强颜欢笑,也是辛苦。''

会辛苦么?如懿不答,却辗转自问。朝夕相对时,他与她客气,温和,越来越像一对经年长久的夫妻,懂得对方的底线所在,不去轻易触碰。那是因为实在太知道了,许多溃疡烂在那里,救不得,治不好,一碰则伤筋动骨,痛彻心扉。只好假装看不见,假装不存在。

所以,也算不得强颜欢笑,而是明知只能如此,才能抵御伤痛之后渐行渐远的疏离与不能信任。

永璂逐渐长大,皇帝对他也越发督促得紧。凡是晚膳之后,必要亲自过问功课,每逢旬日,便亲自教习马术武艺,端的是一位慈父。

如此一来,人心反倒安定了。

自从端慧太子与七阿哥早夭,皇帝爱重四阿哥,连着他生母淑嘉皇贵妃也炙手可热,颠倒于后宫。而后四阿哥失宠,五阿哥永琪深得皇帝信任倚重,又是如懿养在膝下,引得人心浮动,难免将他视作储君。如今如懿自己的儿子得皇帝这般用心照拂,落在外人眼里,毕竟是中宫所出,名正言顺,又可遂了皇帝一向欲立嫡子之心。可是身为亲母,如懿是知道的,永璂年少体弱,经历了丧弟风波、人情冷暖之后,小小的孩童愈加沉默寡言,学起文韬武艺,自不如永璜与永琪年幼时那般聪慧敏捷。

待到无人时分,夫妻二人枕畔私语,如懿亦不觉叹惋:``说道文武之才,虽然永璂得皇上悉心调教,可比之永琪当年,却显得资质平平了。''

皇帝笑着抚了抚她的脸,温和道:``哪有你这样的额娘的,旁人都偏心自己的儿子也来不及,你却尽夸别人好,永璂才多大,永琪多大,你便这般比了!''

如懿轻轻啐了一口,倚在皇帝臂弯里,任由一把青丝逶迤拖曳:``什么别人不别人的,永琪、永珹他们,哪个不是臣妾的儿子了?''

皇帝揽她入怀,笑声朗朗:``有皇后如此,是朕的福气。''

如懿见他正在兴头上,是最好说话的时候,便道:``父母之爱子,则为之计深远。皇上爱重永璂,臣妾心里固然高兴,可臣妾是他额娘,也比旁人更清楚不过。永璂,他的天资不如永琪,甚至,连永璜当年也比不上。''

皇帝颇为惊异:``朕疼自己的儿子,你怎的好好地生出这般念想来?''

如懿感慨道:``皇上疼他,臣妾欢喜不已,可就怕是太疼爱了,过犹不及。臣妾瞧皇上这些日子给永璂读的书,大半是君王治国之道。永璂年纪尚小不说,落在旁人眼里,还当皇上动了立储之意,反倒生出许多无谓的是非来。''

皇帝闻言亦是唏嘘:``朕年轻时时念着嫡子的好处,想着若是弟兄众多,嫡子是最名正言顺的。如今自己为人父,年纪渐长,却也发觉,国赖长君也是正理。可到底如何\ldots{}''

如懿轻声道:``老祖宗的教训最好,国赖长君。若长中立贤,更是不错。''她谦和道:``皇上,妇人不得干政,臣妾无心的。''

皇帝笑着拥住她:``如懿,你没有干政。你是朕选的皇后,懂得在最合适的时候说最合适的话,做最合适的事。朕希望你,一直如此。''

如懿婉然一笑:``所以有件事,臣妾不得不提了。''

皇帝轻吻她的额头,懒懒道:``什么要紧事,连枕畔低语温存都抵不得了。''

如懿半仰着肩,躲避着他追寻而来的青青的胡渣:``皇上,永璜与永琏早逝,永璋与永珹一个出宫建府,一个出嗣,但都已成家。如今永琪已然成年,也到了成家立业的时候。皇上可曾考虑过,要为他选一个什么样的福晋?''

皇帝眉眼弯弯,笑看着她:``愉妃倒是向朕提过一次,说自己出身寒微,不敢娶一个高门华第的媳妇儿,只消人品佳即可。你既是嫡母,又疼永琪,你是如何打算的?''

如懿一笑:``皇上是慈父,岂有思虑不全的,非要来考较臣妾。''她略一沉吟,``愉妃的话臣妾不爱听,动辄牵扯家世,连累永琪也自觉卑微。依臣妾看,福晋的德容言功须得出众,才配得上永琪。至于门第,不高不低,可堪般配便好。''

皇帝不觉失笑:``咱们已是皇家,还要般配,哪儿有这么好的门第?你呀,心里还是偏疼永琪。''

如懿偏着脸,青丝软软垂落:``皇上的话臣妾不爱听,永璋的福晋难道不是臣妾与皇上商量着细细挑的,便是他的侧福晋也出身完颜氏大族,纯贵妃一见几个媳妇儿就高兴。''

皇帝绞着她一缕青丝于指上,凝神道:``永琪的婚事朕细想过了,已有了极好的人选,便是鄂尔泰的孙女,四川总督鄂弼之女,西林觉罗氏。''

如懿闻言,不觉一怔,强笑道:``鄂尔泰是先皇留给皇上的辅政大臣,本配享太庙,入贤良祠。若不是被胡中藻牵连,也不会被撤出贤良祠,还赔上了侄子鄂昌的性命,累得全族惴惴。''她悄悄望着皇帝:``娶这样人家的女儿\ldots{}''

皇帝慨然含笑:``正是合适。永琪娶鄂尔泰的孙女,一则以示天家宽宏,不计旧事;二则宽慰鄂尔泰全族,也算勉励他在朝为官的子侄;再则,这样的人家家训甚严,教出来的女儿必定不错,又不会煊赫嚣张,目中无人。''

如懿深以为然,亦不得不赞叹皇帝的心思缜密。若非这样的老臣之后,如何配得上永琪。且又是曾打压过的老臣,即对指婚感激涕零,又不会附为羽翼,结党营私。

他望着他闭目静思的容颜,有那么一瞬,感到熟悉的陌生。还是那张脸,她亲眼见证着他逐渐成熟,逐渐老去的每一分细节。可是却那么陌生,或许她还是爱着这个人,这副皮囊,但他的心早已不复从前模样。曾经的爱逐次凋零,就像她越来越明白,或许他真的是一代天骄,只是,也真的不算一个钟情的丈夫吧。

或许,这样的明白也是一种警醒,她会与他这样平淡老去,日渐疏离,再无年轻时痴痴的爱恋与信任。

年岁摧毁的,不仅是饱满丰沛的青春,也是他与她曾经最可珍惜的一切。

宫中的日子平静无澜,若过得惯,一日一日,白驹过隙,是极容易过的。可是曾经得过宠却又失去的人,最是难熬。

长门一步地,不肯暂回车。连带着池馆寂寥,兰菊凋零。至此,宫车过处,再无一回恩幸。

嬿婉,便是如此。

她的失宠,随着七公主养于颖嫔膝下,变成了水落后突兀而出的峭石,人人显而易见。她不是没有想过法子,但都被进忠委婉拒绝:``小主何苦碰这个钉子,上回奴才不小心提了一句,皇上就横了奴才一眼,幸好师傅没听见,皇后娘娘也不在旁,否则奴才的性命早没了。''

也不是没有去求过太后,太后索性闭门不见,出来的却是福珈,叹道:``太后留着小主,只是为了在皇上身边留一个温婉进言之人,本不欲小主做出这样的事来。结果小主自作主张,不仅下手,还下这么黑的手,伙同您那糊涂额娘在宫里作耗。太后如今潜心修佛,听不得这样的腌臜事,小主还是不必再来请安了。''

嬿婉也想过再唱起袅袅的昆曲,引来昔日的恩遇与怜惜。却才歌喉一展,颖嫔那儿依然打发人来:``令妃要唱也别这个时候,您的亲女儿七公主听不得这些动静。等下哭起来,皇上怪罪,可叫咱们颖嫔小主怎么回呢?小主替您受着累,您却快活,皇上知道了,可要怎么怪你?''

嬿婉听着嬷嬷义正词严的话,只得讪讪闭了口笑道:``颖嫔妹妹甫带孩子,怕有不惯,本宫亲手做了些小儿衣裳,还请嬷嬷送去给公主。''

偏嬷嬷满脸是笑,却半分不肯通融:``皇上虽未明说,但内务府都得了消息,小主虽是妃位,但宫里一些开销按着官女子来。小主自己都紧巴巴的,何必还替公主操心,一切都要颖嫔呢。''

一忍再忍,总有机会可觅。

过了中秋便是重阳,是合宫陛见为太后庆贺的正日子,皇帝自然也会来。她依稀是记得的,曾经的舒妃,叶赫那拉意欢,便是重阳菊开之时,一曲清歌,凌云而上。

嬿婉早两日便准备了起来,取出尚未穿过的新衣,比着鎏银铜镜搅衣自观。才试了两件,春蝉便婉劝:``小主,这两件新衣是去年制裁了尚未来得及穿的,今岁新的,内务府一直迁延着不曾送来。''

她听得出春蝉的难处,因着她的失宠,内务府早停了送每季的衣裳首饰。唯剩的两件新衣,其实早就是旧衫了。宫中所用的绫罗是天边溜转的云朵,风吹云散,每一日都是新的针脚,艳的花纹,迷了人的眼睛,看也看不过来。

孝贤皇后过世后,后宫女眷早不肯那么简素。便是皇帝,也是穷奢极欲之人,爱她们如花朵招摇地绽放,每一朵都晕彩迷离,每一日又胜过昨日的样子。如懿亦是,她是锦绣堆叠里长大的闺秀,什么稀罕物儿没见过,什么也不放在心上,也甚少在衣饰、首饰、器皿上约束嫔妃,所以素日相见,无不穷尽奇巧。

去岁的衣衫啊,若是被人瞧出,必是要惹笑话的。

女人的争奇斗艳,便是这一针一线上的锱铢必较。长一寸,短一分,细碎,琐屑,却无比认真,付尽心力。

所以嬿婉愈加精心,衣衫虽是旧样,但花钿翡翠是不怕的,只要水头足,色儿透,一样叫人不敢小觑。且她如今的身份,虽还是妃位,却是官女子的份例,外头的体面不可失,又不可张扬。好容易择定了浅浅橘瓣红含苞菊蕊挑银纹锦袍,一色水嫩绿翠的翡翠绞丝鸾凤花钿,点缀零星的翠榴石米花珠簪,倒也美得收放自如,含蓄温婉。

等嬿婉打扮得恰如其分,引颈盼着辇轿来候,等来的却是一脸为难的进忠。他的靴子蹭在殿门口不肯再走近。嬿婉欢喜道:``进忠,皇上让你来接本宫么?''

进忠苦涩地摇头,看着嬿婉的清丽妆容,道:``小主别费这个心了,今晚的重阳夜宴小主不必去了。''

嬿婉登时急了,那红晕浮过胭脂的娇艳,直直逼了出来:``怎么会?今日是合宫陛见得日子。本宫要给太后敬酒磕头,皇上也会来。''

进忠的脸越发黄了,期期艾艾道:``小主,今儿夜宴,根本没安排您的座次。您\ldots{}''

似腊月冰水兜头浇下,彻骨寒凉。他足下的水粉色柳荫黄鹂花盆一个不稳,险险跌倒于地,还是进忠眼疾手快扶住了:``小主,下回吧,总有下回。''

嬿婉犹不肯死心,攥着进忠的袖子,痴痴问:``是皇上特意要你来告诉本宫的么?''

进忠摇头:``不是。是奴才怕您不知,冒冒失失去了,反叫人笑话。''

嬿婉死死扯着进忠不放,两眼都直了:``进忠,有没有法子,有没有?见面三分情,皇上见了本宫,会原谅本宫的。你想个法子,让本宫可以去重阳夜宴,好不好?''

进忠赤眉白眼,又急又无奈:``小主,奴才不过是个伺候人的家伙,能有什么法子?重阳夜宴的座次是皇后娘娘排定了给皇上过目的,皇上当时就无异议,您去了可不是驳了皇上的意思。''他说罢,急急道:``奴才还有差事,先走了。若被皇上知道奴才来通报消息,那可吃罪不起。''

春蝉赶紧上来扶着,嬿婉坐在九枝西番莲花紫绒贵妃榻上,满眼的泪争先恐后地出来,一口气却不上不下,涌到了喉头,哽得她晕厥了过去。

\hypertarget{ux7b2cux4e8cux5341ux4e03ux7ae0-ux6c89ux6d6e}{%
\chapter{第二十七章
沉浮}\label{ux7b2cux4e8cux5341ux4e03ux7ae0-ux6c89ux6d6e}}

宫中的日子平静无澜,若过得惯,一日一日,白驹过隙,是极容易过的。可是曾经得过宠却又失去的人,最是难熬。

长门一步地,不肯暂回车。连带着池馆寂寥,兰菊凋零。至此,宫车过处,再无一回恩幸。

嬿婉,便是如此。

她的失宠,随着七公主养于颖嫔膝下,变成了水落后突兀而出的峭石,人人显而易见。她不是没有想过法子,但都被进忠委婉拒绝:``小主何苦碰这个钉子,上回奴才不小心提了一句,皇上就横了奴才一眼,幸好师傅没听见,皇后娘娘也不在旁,否则奴才的性命早没了。''

也不是没有去求过太后,太后索性闭门不见,出来的却是福珈,叹道:``太后留着小主,只是为了在皇上身边留一个温婉进言之人,本不欲小主做出这样的事来。结果小主自作主张,不仅下手,还下这么黑的手,伙同您那糊涂额娘在宫里作耗。太后如今潜心修佛,听不得这样的腌臜事,小主还是不必再来请安了。''

嬿婉也想过再唱起袅袅的昆曲,引来昔日的恩遇与怜惜。却才歌喉一展,颖嫔那儿依然打发人来:``令妃要唱也别这个时候,您的亲女儿七公主听不得这些动静。等下哭起来,皇上怪罪,可叫咱们颖嫔小主怎么回呢?小主替您受着累,您却快活,皇上知道了,可要怎么怪你?''

嬿婉听着嬷嬷义正词严的话,只得讪讪闭了口笑道:``颖嫔妹妹甫带孩子,怕有不惯,本宫亲手做了些小儿衣裳,还请嬷嬷送去给公主。''

偏嬷嬷满脸是笑,却半分不肯通融:``皇上虽未明说,但内务府都得了消息,小主虽是妃位,但宫里一些开销按着官女子来。小主自己都紧巴巴的,何必还替公主操心,一切都要颖嫔呢。''

一忍再忍,总有机会可觅。

过了中秋便是重阳,是合宫陛见为太后庆贺的正日子,皇帝自然也会来。她依稀是记得的,曾经的舒妃,叶赫那拉意欢,便是重阳菊开之时,一曲清歌,凌云而上。

嬿婉早两日便准备了起来,取出尚未穿过的新衣,比着鎏银铜镜搅衣自观。才试了两件,春蝉便婉劝:``小主,这两件新衣是去年制裁了尚未来得及穿的,今岁新的,内务府一直迁延着不曾送来。''

她听得出春蝉的难处,因着她的失宠,内务府早停了送每季的衣裳首饰。唯剩的两件新衣,其实早就是旧衫了。宫中所用的绫罗是天边溜转的云朵,风吹云散,每一日都是新的针脚,艳的花纹,迷了人的眼睛,看也看不过来。

孝贤皇后过世后,后宫女眷早不肯那么简素。便是皇帝,也是穷奢极欲之人,爱她们如花朵招摇地绽放,每一朵都晕彩迷离,每一日又胜过昨日的样子。如懿亦是,她是锦绣堆叠里长大的闺秀,什么稀罕物儿没见过,什么也不放在心上,也甚少在衣饰、首饰、器皿上约束嫔妃,所以素日相见,无不穷尽奇巧。

去岁的衣衫啊,若是被人瞧出,必是要惹笑话的。

女人的争奇斗艳,便是这一针一线上的锱铢必较。长一寸,短一分,细碎,琐屑,却无比认真,付尽心力。

所以嬿婉愈加精心,衣衫虽是旧样,但花钿翡翠是不怕的,只要水头足,色儿透,一样叫人不敢小觑。且她如今的身份,虽还是妃位,却是官女子的份例,外头的体面不可失,又不可张扬。好容易择定了浅浅橘瓣红含苞菊蕊挑银纹锦袍,一色水嫩绿翠的翡翠绞丝鸾凤花钿,点缀零星的翠榴石米花珠簪,倒也美得收放自如,含蓄温婉。

等嬿婉打扮得恰如其分,引颈盼着辇轿来候,等来的却是一脸为难的进忠。他的靴子蹭在殿门口不肯再走近。嬿婉欢喜道:``进忠,皇上让你来接本宫么?''

进忠苦涩地摇头,看着嬿婉的清丽妆容,道:``小主别费这个心了,今晚的重阳夜宴小主不必去了。''

嬿婉登时急了,那红晕浮过胭脂的娇艳,直直逼了出来:``怎么会?今日是合宫陛见得日子。本宫要给太后敬酒磕头,皇上也会来。''

进忠的脸越发黄了,期期艾艾道:``小主,今儿夜宴,根本没安排您的座次。您\ldots{}''

似腊月冰水兜头浇下,彻骨寒凉。他足下的水粉色柳荫黄鹂花盆一个不稳,险险跌倒于地,还是进忠眼疾手快扶住了:``小主,下回吧,总有下回。''

嬿婉犹不肯死心,攥着进忠的袖子,痴痴问:``是皇上特意要你来告诉本宫的么?''

进忠摇头:``不是。是奴才怕您不知,冒冒失失去了,反叫人笑话。''

嬿婉死死扯着进忠不放,两眼都直了:``进忠,有没有法子,有没有?见面三分情,皇上见了本宫,会原谅本宫的。你想个法子,让本宫可以去重阳夜宴,好不好?''

进忠赤眉白眼,又急又无奈:``小主,奴才不过是个伺候人的家伙,能有什么法子?重阳夜宴的座次是皇后娘娘排定了给皇上过目的,皇上当时就无异议,您去了可不是驳了皇上的意思。''他说罢,急急道:``奴才还有差事,先走了。若被皇上知道奴才来通报消息,那可吃罪不起。''

春蝉赶紧上来扶着,嬿婉坐在九枝西番莲花紫绒贵妃榻上,满眼的泪争先恐后地出来,一口气却不上不下,涌到了喉头,哽得她晕厥了过去。

二人正说着话,只听``咚''的一声,湖中溅起尺高的水花,落到嬿婉衣上。太湖石后传来男童快活的笑声,嬿婉登时有些恼,正欲喝问,想起如今失势,先气短了三分,低低怨道:``谁这般胡闹,今冬寒冷,本宫只有这一件厚衣裳了,弄湿了可怎么好?''

春蝉忙不迭拿绢子替嬿婉擦拭着,愁道:``宫里连炭火都没了,本就冷得很,这可怎么给小主烘干呢?''说着,她便探头过去,只见一个三岁大的孩子,一个人爬在湖边横出的太湖石上掷石子玩。那孩子长得壮实,衣着华贵,揪着小小的辫儿,憨态可掬。

春蝉蹙眉道:``不是宫里的阿哥,怕是哪家的福晋带进来的不懂事的孩子。''她看了看,又道:``真是不懂事的孩子!那石头上积满了青苔,又高又滑,仔细摔下来才是。''

嬿婉气恼而不甘:``这么顽皮的孩子,摔下来才好呢。''

正说着,又有几颗石子儿落入湖中,溅起雪白的水花,赢来那孩子欢快的鼓掌声。嬿婉连连皱眉,扶着春蝉的手便走。才行几步,只听得远远有数人唤道:``世子!世子!别躲啦!快出来吧!'

嬿婉一怔,问道:``世子?''

春蝉``哎呀''一声,压低了声音道:``小主,听说和敬公主带着世子庆佑入宫,别就是这个孩子吧?瞧着年纪也差不多。''

二人凝神远眺,只见翠叶落尽的柳枝懒洋洋地斜垂着,那孩子爬在太湖石的青苔上,手舞足蹈地乐着,浑不顾足下青苔滑腻。春蝉不大放心:``唉!那石头滑腻,别掉下来,那怎么好?小主,若真是世子,奴婢赶紧去抱下来,别出了什么事儿。''

嬿婉细白的牙齿死死咬在暗红的唇瓣上,一下按住她的手臂,轻轻嘘了一声。她腰肢轻折,捡起一枚石子,瞅准那孩子足下,用力一掷,那孩子显然被突如其来的异物吓到,足下一跌。

只听得有重物落水之声,扑腾的哗啦声,夹杂着断续的哭喊呼叫。春蝉吓得脸都白了,还来不及反应,只觉得按着自己手臂的重压倏然抽去,又一声重响,水光扑溅。她定睛之时,嬿婉已然落到了水中,死死拉住了那孩子的手。

春蝉吓得两腿发软,她拼命逼迫自己镇静下来,尖声呼道:``救命!救命啊!''

宫人们是怎么赶来的,怎么捞起了嬿婉和那孩子,春蝉依然不大记得了。她只记得,湖里溅起的水夹杂着冬日的碎冰迸到了她的面孔上,擦得她脸皮生疼生疼的。她抢过去抱着嬿婉,嬿婉力竭倒在她怀里,浑身都在滴水。嬿婉的全身都在发抖,抖得不可遏制。并无太多人理会她们,他们都簇拥着那个孩子,慌乱地叫唤着,夹着哭腔,``世子!世子!'',或是``庆佑''!

嬿婉的眼睛在听到``庆佑''二字时倏然亮起,像被点亮的烛光,明媚地闪着神采。嬿婉低低道:``幸好!赌赢了!''

春蝉看着嬿婉冻得惨白的面孔,想起她曾经柔润的面庞,含春的眼角,只觉得无限心酸。她自小是宫女出身,受过万般委屈,只想凭着嬿婉的恩宠可以如人头地,却不想,身为宫妃,嬿婉也是那样难。那样难,反叫她生起相依为命的依赖。已经走上了这条路,除了争宠,毫无退路。

春蝉努力想笑,手触碰到嬿婉冰冷的面孔,只觉得那股寒意顺着指尖渗到她的心里。她凄惶地哭着:``太医呢?太医!谁来救救小主!''

皇帝见到嬿婉时,已经是两个时辰后了。宫人们簇拥着庆佑去了,幸好还有人记得嬿婉,找来棉被裹了她抬回永寿宫中。

嬿婉裹着厚厚的棉被,牙齿都在打战。纵然殿阁中点了十数火盆,那暖气仍然驱不走她落水后的寒意。那寒意是长着牙齿的,细细地,一点点地啃着她,无处不在似的。嬿婉坐在那里,看着烧得红彤彤的炭盆围着自己,那种熟悉的红箩炭的气味,让她觉得踏实。

真的,她从来不知道,这些曾经拥有却不曾在意的东西,有着如此现实而强大的力量。譬如,皇帝衣上沾染的龙涎香,红箩炭轻声的``哔剥'',织锦云罗的绵软,羽缎鹅绒的轻暖,这些能让她愉快的东西,也让她心生贪婪。

皇帝从门外进来时,带着蒙蒙的阳光的颜色,沐着金色的光辉。她眷恋地看着,蓦地俯身下去。她知道自己的卑微和脆弱,哪怕身居妃位,没有他的眷念与宠爱,她便是枝头摇曳的黄叶,只有坠落一途。

皇帝显然已去看过了庆佑,所以神色并不焦灼。他的口气极温和:``庆佑顽皮,趁璟瑟午睡,乳母打盹,偷偷溜出来玩耍。幸得你瞧见救了她。方才璟瑟哭得死去活来,朕也看着心疼。''

皇帝的话颇有劝慰之意,只见他身后红影摇曳,一个女子爽朗笑道:``皇上为了这个外孙好是揪心,看着庆佑无恙,就过来看令妃姐姐了。''

嬿婉如何听不出她话里的意思,不过是指她在皇帝心中无足轻重而已。她却不能反驳,因为实在太清楚地知道,自从七公主养在颖嫔身边,颖嫔更得宠爱。嬿婉觉得喉咙里一阵阵发紧,那原该是属于她的宠爱。

嬿婉笑得欣慰,打着战道:``孩子无恙就好。''

颖嫔挑着眉眼,似笑非笑地看着她:``也真是巧。庆佑偷溜出来,偏姐姐瞧见了,偏姐姐跳下水去救。当真无巧不成书,好像天意是要成全姐姐似的。''

春蝉眼珠一转,抱了个汤婆子递给尚未完全缓过气的嬿婉,难过道:``可不是!小主从未见过世子,却能不顾自己不懂水性就往下跳。唉,小主真是喜欢孩子的人。''

皇帝的面色柔缓了几分:``是了,朕记得嬿婉是不懂水性的,唉,你也不当心自己,亏得近旁的宫人们发觉得早,否则连你也填了进去。''皇帝说着,凝视着她,徐徐问:``这个时辰,你怎么在那儿?''

嬿婉一滞,未语,泪却潸然而落,楚楚可怜。

春蝉何等机警,眼角亦湿了几分:``皇上有所不知。自从七公主养在颖嫔宫中,小主日夜思念,总盼着见一见公主才好。御花园离颖嫔宫里不远,小主就盼着颖嫔能抱公主去御花园玩耍,小主能远远看上一眼也好。''

颖嫔轻嗤一声,媚眼如丝:``皇上,那个时辰正是午睡的时候,冬日里风大,臣妾再不懂事,也不会抱着公主往风口上去呀。''

皇帝眼睫一闪,微有疑色。嬿婉凄然开口:``皇上,如今是冬日吗?风很大吗?臣妾都不觉得。臣妾甚至分不清白天黑夜的区别。臣妾只想自己的孩子,臣妾的孩子\ldots{}''

春蝉含泪道:``皇上,自从七公主抱养在颖嫔宫中,小主日夜思念,神思恍惚\ldots{}''她犹豫着看了一眼嬿婉,难过道:``小主的神志与往常不同\ldots{}''

皇帝的眼底闪过一丝不忍:``儿女养在别的嫔妃处是常有的事。颖嫔出身高贵,性格大方\ldots{}''他叹口气,``别称呼七公主了,颖嫔给她起了名字,叫璟妧。''

``璟妧,璟妧\ldots{}''嬿婉喃喃呼唤,眼泪肆意而出,紧紧地裹着被子,颤抖着声音道:``臣妾知道,臣妾不是一个好额娘。出身微贱,学识浅薄。但是皇上,臣妾的爱女之心是一样的,并非因为臣妾罪过有所缺失,反而让臣妾觉得更对不起她。''

颖嫔听出她话中之意,急急道:``皇上,臣妾侍奉皇上多年,唯一的遗憾便是未有生育。幸得皇上垂爱,将璟妧养在膝下。臣妾每日亲自照顾,如同己出,臣妾实在舍不得。''

皇帝安抚地握住颖嫔的手,柔声道:``上次你阿玛入宫觐见,特特提起你为膝下虚空苦恼,所以朕特意将璟妧养在你身边,也好略作宽慰。''

颖嫔粲然一笑,反牵住皇帝的手,颇为安心。

颖嫔望着嬿婉浑身湿腻腻的样子,满脸关切之意:``令妃落水,得好好养一阵子才好。皇上,您答应了臣妾一起用晚膳,时辰不早,咱们早些回去吧。''

皇帝朝着颖嫔温柔一笑,转身意欲离去:``虽然你也是孩子的长辈,但朕还是要谢你,谢你救了庆佑。朕只有这一个外孙,璟瑟只有这一个儿子,幸好他没事,幸好\ldots{}''

``皇上,和敬公主只有一个儿子,臣妾也只有一个女儿璟妧。皇上,璟妧有颖嫔悉心养育,臣妾不敢奢求能将璟妧接回身边,让颖嫔备受分离之苦。但求皇上垂怜,让臣妾能再有一个自己的孩子吧!''

皇帝脚步一缓,却未出声。龙袍的一角拂过深红色的门槛,旋起浅金色的尘灰,将他身影送得更远。嬿婉失望的泪坠落在飞蓬般的烟灰里,落成晶亮的不完满的水滴。

是夜,皇帝本欲独自歇在养心殿中。或许是颖嫔处婴儿的啼哭让他有所念及,或许是白日的落水之事让他仍有余悸,在合上奏折之后,他唤来了李玉。

李玉的毕恭毕敬似乎惹来皇帝的不甚耐烦,他问:``敬事房是否送绿头牌来?''

李玉道:``敬事房的人正候在外头呢。''他击掌两下,徐安捧着绿头牌进来。灯火明耀之下,红木盘中牌子泛着绿幽幽的华彩,仿佛是招人的手,引着皇帝的目光凝住。

皇帝的手如行云流水般划过,在``令妃''的牌子上略略一停,复又逡巡,末了停在``婉嫔''的绿头牌上。

徐安愕然,还是李玉赔笑:``皇上真是长情之人,您是有些日子未见婉嫔了。''

皇帝看他一眼:``去吧。''

徐安哈着腰道:``奴才这就去接婉嫔小主。''他迈开步子,才走到殿门口,只听身后郁然一声长叹:``换令妃来吧。''

徐安不知皇帝为何心意忽变,却也不敢多问,赶紧答应着去了。

这一夜翻牌子的风波很快湮灭在日常生活的琐碎里,似乎谁也没有放在心上,那是因为,实在也不值得放在心上。而下一个月,皇帝又召幸了她一次。此后,皇帝对嬿婉仍是不加理会,连官女子的开销也未改变。一切,彷如旧日。

而嬿婉,却因着这两次宠幸,实实有了身孕。

江与彬传来这消息的时候,茜纱窗下滤来浅橘淡金的骀荡春光,安静地落在螺钿小几上新折的一捧尺多高的绚烂海棠枝上。花开如流波碎锦。却是无香,极是雅静。

熏风微来,曳动珍珠垂帘的波縠越发缱绻而温媚。春衫薄媚,软缎衣袖悄然退至皓腕之上,如懿只是静静落下一枚白玉棋子,淡淡含笑。

海兰坐在如懿对面,拈了一枚黑子浅浅蹙眉:``令妃倒真是个有福气的,才生下七公主多久呢,便又有了孩子。''

江与彬沉声道:``是,已经五个月了。令妃有孕后并不敢请太医院请脉安胎,所以一直到显怀,太医院才知情。''

如懿挑眉:``她胆子倒大。''

海兰轻嗤:``不是胆大,是胆子太小!生怕咱们害了她这辛苦怀上的孩子。''她颇有些埋怨:``从她跳下水救了和敬公主的心肝宝贝,姐姐就该万分防着她东山再起。到底,皇上还是宠幸了她两回。''

如懿轻轻摇头:``宠幸又如何?哪怕知道令妃又有了身孕,皇上也不过吩咐内务府按着贵妃份例伺候,赏了东西,却也不曾去看过她。不像祈妃,才有了两个月身孕,皇上便金尊玉贵地捧着。''

海兰不以为然:``令妃的出身怎能与祈妃比?祈妃这回好容易有了身孕,且祈妃的六公主是跟着姐姐的五公主一同去的,皇上自然格外心疼些。''

如懿明眸微凝:``令妃的身子,江与彬你是知道底细的。''

海兰眼中微有疑惑之色,江与彬神色不动:``令妃小主生育七公主时颇受折磨,加之产后不调,屡受气郁,身子一直虚弱,是不宜有孕的。''

如懿抬起手,整理燕尾簪子,上面簪了新鲜芍药花,衬着裳色胭云缎长衣上大蓬素色的暗纹,越显得容色清淡:``他自己的身子自己知道,还要这般强求。''

海兰的眸色趋于平静:``还有和敬公主,也是姐姐不得不在意的吧。毕竟,她是皇上最钟爱的固伦公主,孝贤皇后嫡出的女儿。为着令妃救了爱子,她也会有所援引的吧。''

白玉子落在碧玉棋盘上余音微凉,恰如如懿此刻的感慨:``有时候死亡或许真的算一件好事,可以弥补曾经的不完美。孝贤皇后离世日久,皇上的愧疚越深,便越是怀念。这些年皇上为孝贤皇后所作的挽诗还少么?连几近济南都不肯进城,只因是孝贤皇后薨逝之地。''

海兰静默不语,只是以懂得的沉默来安慰彼此的孤凉。半晌,她才轻语:``经了十三阿哥之事,姐姐的心似乎淡了,许多事也不再在意。''

殿内美人对坐珠帘卷,殿外是绵绵袅袅的晴光万缕。宝鼎香暖,花竹葱茏,也不过是寸断了的时光里荒芜的影子。翊坤宫琼楼玉宇,琪花芝草,与废弃千年的伽蓝寺又有何异?心落了灰,如经卷蒙尘,再难翻动。

如懿苦笑:``本宫想得到的终究难求,还不如暂守自己所能有的。''

许多事其实再明白不过,即便有着皇后之尊,即便有着彼此原谅后的再度信任,可唯有经历过此间的骇浪惊涛,才知自己所有的一切是如何脆弱,甚至不堪一击。如懿再不能也没有力量去施行何等的决绝。

如懿的话说完不过三月,嬿婉便于七月十七日早产了一位皇子。此子序列十四,取名永璐。皇帝依言将永璐留在嬿婉身边抚养,也在洗三之日按照寻常皇子诞生的规矩赏赐,并无半分另待。可是嬿婉的喜悦并没有维持多久,这个过早降临于人世的孩子便因先天不足,发起了高烧。

出生的孩子甚是娇嫩,嬿婉衣不解带,日夜不眠,守在永璐身旁。比之七公主璟妧,永璐更似她的命根,值得她穷尽所有力量守护。然而孩子持续的高烧与抽搐让嬿婉数度惊厥,在求医问药之余,也请来萨满法师于永寿宫中作法。

萨满的世界里,病痛的一切来源都是妖邪作祟,便也直言,让嬿婉将孩子挪于宫中阳气最重之地暂养。

春蝉闻言便明白,一味搓手为难:``阳气最重,莫过于养心殿。只是\ldots{}''

嬿婉看着怀中气息微弱的永璐,睁着哭得如红桃的眼,鼓足了勇气便往外冲:``本宫去求皇上!''

\hypertarget{ux7b2cux4e8cux5341ux516bux7ae0-ux65b0ux79c0}{%
\chapter{第二十八章
新秀}\label{ux7b2cux4e8cux5341ux516bux7ae0-ux65b0ux79c0}}

京中夏日炎炎,夜来也有不退的热息。微风不起,水晶帘止。唯有殿中供着的满捧蔷薇,缀着艳红莹透的花瓣,被冰雕的凉意凝住郁郁花香。

皇帝在暖阁翻阅书卷,如懿相伴在侧,往青玉狮螭耳炉中添入一小块压成莲花状的香印,又加以银叶和云母片,使香气均匀。那袅袅淡烟,溢出雨后梧桐脉脉翠色的清逸,衬得四周越发安宁。

嬿婉跪伏在外已有一刻,她的哭声哀哀欲绝:``皇上阳气甚足,可以抵御一切妖邪。臣妾恳请您将永璐暂养在养心殿,求您龙气庇佑,让永璐渡过这一劫。''

她的哭求声撕心裂肺,足以让任何一个路人动容。如懿伴在皇帝身侧,轻声询问:``皇上,令妃如此哭求,您不答应么?''

殿外的哭求带着寒绝的气息:``皇上!皇上!臣妾父母俱亡,兄弟戴罪。除了您的怜悯,除了永璐,臣妾便无依无靠。若是永璐不保,臣妾宁可跪死在宫门前!''

皇帝的眼底有着罕见的哀伤与迷茫:``如懿,朕很难去断定永璟之死是否一定与令妃有关,但朕真真切切地知道,若非朕这般宠爱,她的额娘也不会生了妄心来谋害你的孩子。''

如懿定定望着皇帝:``臣妾不敢多言,但求皇上明白。''

皇帝的面上闪过一丝软弱:``可在门外的,也是朕的儿子,真不能完全置之不理。''

如懿颔首,侧身坐在他身边:``令妃的请求不算是过分,可若说永璟之死她完全无辜,臣妾也不敢全信。''

皇帝握住她的手,他的手心是潮湿的,在夜风依旧醺热之下,触觉微凉。她轻轻叹息:``皇上固然应该救永璐,不为别的,只为他是您的血脉。但\ldots{}''

皇帝点头,打开殿门,居高临下地望着怀抱永璐哭得妆容凌乱的嬿婉:``你与永璐留下,朕许你在此照料。''

接下来的十数日,嬿婉与永璐暂居于偏殿臻祥馆内,留太医数名一同照顾。皇帝每日必探视永璐,却甚少与嬿婉说话。嬿婉亦不多求,之事衣不解带悉心相守,夜来目不交睫,白日便跪在佛像前祝祷,人也消瘦不少。

不过半月,嬿婉便添了下红之症,接连的生产对她的身体损伤颇大,又兼两次都未曾好好坐月,气恼忧烦。她起初还不敢明言,只是忍着照顾永璐,直到不能起身,才不得不于永璐床榻之侧再添一床,方便就近医治照顾。

这一来,便是和敬公主也添了怜悯之心,入宫时瞧见一二,便嘱人送了人参燕窝过去。偶然没有宫人伺候在前时,和敬抱着小儿,引袖哀哀求道:``令娘娘再有不是,皇阿玛也该看在认得分儿上。再者永璐早产,令娘娘卧病,不都是为了救庆佑而起的。''

皇帝只疼爱地摸着庆佑绯红滚圆的小脸,仿佛未曾听到与令妃相关之语:``庆佑只是小名儿。''他沉吟,``得起个压得住的大名。嗯,像他父亲一般是个英雄。就叫鄂勒哲特穆尔额尔克巴拜!''

和敬含笑:``是钢铁的意思,真是个好名字。''

皇帝笑语:``大难不死,必有后福。上次落到水里都能无恙,是个后福无穷的孩子。''

和敬眼中泛起一层泪光,婉声劝道:``皇阿玛,女儿的儿子固然后福无穷,可永璐还躺在偏殿呢。令娘娘纵有千错万错,爱子之情是不错的。且内务府只按贵人份例给永寿宫开销,令娘娘还养着永璐,母子俩以后的委屈大着呢。''

皇帝脸色微沉,侧身坐下端过茶水抿了一口:``你替令妃求情?''

和敬颇有恻然之色:``一个女人没有夫君的恩宠,想要安然度日是何等艰难。当然皇阿玛忙于政事,陪伴额娘的时候不多,额娘贵为皇后,又是也不得不防着嫔妃僭越,何况令娘娘只是出身汉军旗的小小妃子。''

皇帝微有不豫之色,对着和敬仍是语气温然:``璟瑟,后宫中许多事,你并不明白。''

和敬低头,拂弄着衣角垂落的银丝串碎玛瑙络子:``女儿不明白,皇阿玛也未必明白。额娘薨逝之后,皇阿玛才知许多事原是误会。可是与额娘生死两隔,许多事终究也来不及了。若令娘娘之事真有误会在其中,却牵连母子三人,皇阿玛是否也觉得无辜?''

和敬所言,字字锥心,几乎勾起皇帝心底的隐痛。他拍一拍和敬的手,温和道:``璟瑟,皇阿玛年纪大了,只有你会这么对皇阿玛说话。''

和敬嫣然一笑,却不失端庄风范:``女儿是皇阿玛的长女,也是唯一的嫡女,是皇阿玛抱着长大的。''她凝神片刻,``而且,女儿也是心疼皇阿玛。十三弟夭折,皇阿玛一定很希望十四弟可以康健成长。''

皇帝拧一拧她的鼻子:``果然再怎么长大,终究是朕的小女儿。朕会吩咐下去,复令妃素日待遇,也会常去看她们母子。''

和敬神色安娴,静静施礼。她胸前鎏金莲苞扣上垂落的流苏是琉璃蓝色,长长地拂落在她云蓝暗纹闪金片樱花衣袖上。她行动间腰肢轻曲,流苏却纹丝不动。

皇帝看着她姣好容颜,气质玉曜,不觉黯然:``璟瑟,你与你额娘长得真像。她嫁与朕的时候,也很喜欢这样笑。''

和敬如樱红唇抿起一抹温娆笑意:``额娘在天有灵,一定明白皇阿玛对她的记挂。''二人言罢,皇帝便去祈妃宫中。此时祈妃已然有孕,皇帝甚为关怀。而祈妃也因为六公主的早夭,格外地小心翼翼,几乎闭门不出,安心养胎。

和敬转曲廊,入偏殿,见了正在督促乳母喝药化给永璐的嬿婉,嬿婉见了和敬,忙忙迎上来,笑中却带了泪:``公主,你来了。''

和敬细黑的眉微微蹙起:``不必这样哭,我知道永璐快好了。''

嬿婉殷勤劝坐,又从春蝉手中亲自接了茶盅奉上,颇为赧然:``臣妾身边没有什么好茶,这是去岁的毛尖,还请公主将就着喝。''

和敬接住茶盏,却也不喝,只是随手撩于一边。嬿婉会意,示意春蝉带了众人退下。乳色的水汽将和敬端正的脸模糊出一点儿柔和的神色,她淡淡笑道:``恭喜,很快就能复了妃子之位,皇阿玛也会常来看你们母子。''

嬿婉泪盈于睫,却怕和敬不喜,只得忍住了,伏身就要叩谢:``多谢公主大恩。''

和敬也不看她,捻着绢子端坐着:``行礼便大可不必了,你毕竟是我的庶母。要皇阿玛知道,还以为我不懂得尊敬长辈。''嬿婉答应着便要起身,和敬又道:``若是额娘还在,你们都是侍奉她的妾侍,我也不会对你另眼相看。要知道,能救庆佑,虽是我要谢你的,但也是你的本分。''

嬿婉连连诺诺:``我也不过是巧合。能救了世子,是积善积福之事,是成全了我。''

``积善积福?额娘生前倒是驭下和善,温柔勤俭。''和敬轻轻地叹息一声,无限怅惘,``可惜,额娘这么早便不在了。''

嬿婉谦卑而恭敬:``我曾经侍奉过孝贤皇后,孝贤皇后温和端庄,气度高华。我心里,只有她一人才是垂范天下的皇后。''

和敬瞟着她:``我成全你,并非因为你这些话。我只是不喜欢看那个人霸占了额娘的后位。那个位子,不是她的,也不必叫她安稳坐着。''

嬿婉低首敛眉,不敢应答,只是谦卑地道:``皇后终究是皇后\ldots{}''

和敬冷冷打断:``我相信你不是无用之人。你可以凭着孩子的病况住进养心殿得到皇阿玛的宠爱,就不会辜负我的期望。恰如你知我知,永璐的病,其实并没那么要紧。''

嬿婉扬起惨白的素颜,望着和敬笃定的笑意,将它深深记在了心里。

到了十二月间,北风正劲,祈妃便生下了一个女儿,序第八,取名璟婳。祈妃自得此女,以为六公主再度而来,欣喜若狂,将玉团似的女儿疼得不知该如何才好,将其余事都撇在一边,专心养育公主。

而此时,嬿婉已然再度有孕,并于次年生下皇九女璟妘。虽然自此皇帝对她的宠幸不比往日,但接连三年生下子女,如二十一年七月十五日所生的皇七女璟妧,二十二年七月十七日生皇十四子永璐,二十三年七月十五日生皇九女璟妘。连续的生育到底巩固了嬿婉的地位,让她成为与纯贵妃一般生育最多的嫔妃。

嬿婉立于长廊之下,逗着乳母怀中的永璐和璟妘,柔柔微笑。她的眼底是深深的渴望与期盼:``本宫可不能成了第二个纯贵妃,只有尊位而无宠爱。''

乾隆二十三年秋,因着宫中嫔妃渐长,皇帝少有可心之人。嬿婉连续生育,难免损了身体,不得不暂停了侍寝,卧床养息。而向来得宠的祈妃也因生下八公主产后惊风,便缠绵病榻,亦不便再侍奉君上。内务府便提议要广选秀女充斥后宫,也好为皇家绵延子嗣。

这一年九月,便由如懿和太后陪着皇帝主持了殿选。这次入选的,除了太后母家的远亲钮钴禄氏为诚贵人,礼部尚书德保之女索绰伦氏为瑞贵人,最为出挑的,应当是蒙古霍硕特部亲王送来的女儿蓝曦格格。另有几位位分偏低的常在,都是江南织造特意送入宫中的汉军旗包衣,虽然身份低微,但个个都是容貌昳丽的江南佳丽。霍硕特氏蓝曦一入宫便被封为恂嫔,格外受皇帝恩宠。大约也是如前朝所言,霍硕特部不如大清的姻亲博尔济吉特氏一般显赫出众,并且因为曾经暗地里自主准噶尔部作乱而被皇帝侧目,为求一席保全之地,也不得不与其他部族一般献出自己的女儿与大清共结姻亲之好来寻得庇护。

恂嫔的一枝独秀,连着十六年选秀入宫的颖嫔巴林氏、恭贵人林氏、禧贵人西林觉罗氏、恪贵人拜尔果斯氏,成为妃位以下的嫔妃中恩眷最盛的女子。亦因为她们的年轻的美与活力,格外受到皇帝的垂怜。再加之更早入宫的令妃,帝王的垂爱,便常常流连在她们这些娇然盛放的花朵之上。

宫中的选秀,向来不过是循例而已。把这天下的美人都收罗一遍,才是尽了皇家的权势了。其实皇帝宫中妃嫔的来源,选秀不过是一小拨儿,有宫女承恩侍上的,有外头大臣亲贵进献的,有蒙古各部选的,林林总总,总是有新的美人一朵一朵地开在御花园里头,谢了一朵再开数十朵,永远没有凋零的时候。

这一日是选秀后的第三日,一切新人的封号住所都已安排妥当,如懿便携了容珮去养心殿书房看望皇帝。

这一年入冬早,十月间便下了几场大雪,倚梅园的梅花早已绽了好些花苞,盈盈欲放。如懿看了欢喜,便命人折了几枝最好的白梅,一并带了过来。

书房里静悄悄的。皇帝坐在堆积如山的折子后头,李玉带了两个机灵的小太监随侍在旁。金鼎香炉里悠然扬起一缕白烟,如懿轻轻一嗅,便知是皇帝常用的沉水香,旋即请了一安道:``沉水香辛、苦、温,暖腰膝,去邪气,有温中清神之效,这个时节用是再好不过了。''

皇帝见她来了,搁下笔含笑道:``好是好,但是沉水香是暖香,问多了难免昏昏欲睡,若是开窗,也不合宜。''

如懿只是一笑,折下几朵白梅的花苞放进香炉里,再盖上鹤嘴赤金香炉盖,将其余的白梅供养在清水瓶中,安静道:``梅花有清冽之气,犹以白梅为甚。暖香中有清气,皇上可喜欢么?''

皇帝含了欣悦之意,起手携过她的手道:``外头刚下过雪,怎么还过来,也不怕着了寒气?''

如懿扬一扬脸,容珮端出一盘焦香四溢的烤羊肉和一壶白酒来。如懿道:``想起从前在潜邸中,和皇上偷偷烤了羊肉喝酒,今日就特意烤了这个,以慰当日豪情。''

皇帝惊喜道:``正好外头下过雪,咱们移到窗下来,边看雪边吃这个。''说罢又笑,``折了白梅来这般清雅,原来也是个酒肉之徒。''

如懿俏然一笑:``喝酒吃肉,原来就是人生雅事,皇上何必把它说俗了呢。难不成还不许臣妾`老夫聊发少年狂'么?''

李玉和容珮立刻布置,二人挪到暖阁的窗下,将酒肉搁在小几上,将长窗支了起来。如懿冷得一哆嗦,笑道:``可受不了,这么大的风,好冷!''

皇帝倒了一杯酒送到她嘴边:``来,赶紧喝一口暖暖。喝下就不冷了。''

如懿一仰脖子喝下,见皇帝只顾着吃那烤羊肉,不觉得意:``皇上是不是吃着觉得不大一样?''

皇帝连连下筷,笑道:``没有腥膻味,是口外的肥羊。肉质细嫩,应该还是小羊。''皇帝闭上眼细细品了片刻,``有松枝的清香,还有菊花的甘冽\ldots{}''

``全中了!''如懿抚掌大乐,``就是用松枝烤的,考的时候羊肚子里撒了经霜的菊花瓣。皇上是个吃客!''

皇帝扬扬自得:``每日处理着天下的朝政,也该享用这天下的美食、美景与美人。''

如懿连连摇头,鬓边一支赤金凤东珠发簪的红宝琉璃流苏沙沙地打在鬓边,仿若迎风的红梅点点,越发衬得人面桃花:``皇上刚选了秀女,还嫌这美人不足么?''

皇帝笑吟吟道:``你以为朕选进来的一定是年轻貌美的女子?''他扬声唤道:``李玉,把朕案上的第三份折子拿来。''

如懿喝了一盅酒,抱着手炉取暖,只见李玉递了一份折子上来。皇帝吩咐道:``李玉,给皇后瞧瞧。''

如懿却不伸手去接,只盈盈看着皇帝,笑得慧黠:``不算干政?''

皇帝失笑:``后宫之事,不算干政。''

如懿呵了呵手,打开一看,不觉失笑:``博尔济吉特部地赛桑王爷是疯了么?三十岁的女儿还要送进宫为嫔妃,还说不求名分高贵,只求以贵人身份侍奉在侧,奉洒扫之职。赛桑王爷的格格,草原上的明珠,哪里找不到好人家了?''

皇帝亦是摇头:``据说赛桑的女儿厄音珠格格曾经许配过三次人家,都是未过门男方就暴毙了。草原上的喇嘛替她算过,要嫁世间最尊贵之人才能降得住她的克夫之命,所以赛桑一拖再拖,就拖出了一个三十岁还云英未嫁的女儿。''

如懿沉吟片刻,夹着一筷子羊肉却不吃,倒被冷风吹了一阵,直吹的银筷子的细链子簌簌作响,却只瞧着皇帝不作声。

皇帝道:``你想到什么?直说便是。''

如懿抿了抿唇道:``喇嘛的传说只是一种说法,为何从前不提,如今却突然提起来?厄音珠格格未嫁先丧夫的确是可怜,不过若不是霍硕特部的蓝曦格格被皇上册为嫔御,恐怕博尔济吉特部也不会如此焦灼吧?''

皇帝饮了一口酒,恋上微微泛起晕红光彩:``你再说便是。''

``臣妾听闻草原各部一直不睦,虽然都臣服于大清,但私下里争夺烧杀之事也时有耳闻。霍硕特部与博尔济吉特部不睦已久,博尔济吉特部是爱新觉罗氏的姻亲,若要选妃,本就该博尔济吉特部为先,估计霍硕特部亲王也是看准了博尔济吉特部无适龄的少女可选,所以才会送了女儿蓝曦格格,以求来日若有纷争,可得皇上庇护。且自成准噶尔之事后,霍硕特部自知见罪于大清,也是示好之举。这样一来,博尔济吉特部王爷可不是要着急了?选来选去,只有一个三十岁的亲生女儿,也只好忙不迭地送来了。''

皇帝朗声笑道:``皇后见微知著。那么皇后以为,朕该如何?''

如懿起身行礼道:``皇上胸怀天下,是蒙古各部若掌中之物,区区女子之事,怎会要问臣妾,自然是早有定夺了。''

皇帝执过她的手笑道:``你是皇后,朕自然要知道你与朕是不是一心?''

这话却是问得险了。她是皇后,自然不能心胸狭窄,落了个妒忌的罪名。何况\ldots 她有六宫之主的位子,宫里多一个人,只好比御苑里多开了一朵花,便有什么可怕的。她悄悄打量着皇帝的神色,她还是悠然自得的样子,仿佛是毫不在意。可是如懿却知道,他这样的神情,便是什么都拿准了的,偏偏,他又是那样多情的性子。

如懿沉思片刻,思量着慢慢道:``其实只要是博尔济吉特部王爷的女儿,不管是三十老女还是丑若无盐,皇上都不会在意。因为皇上的心胸里,选秀进来的,不只是一个女人,而是蒙古各部的平衡之势。''

皇帝的眼幽深若潭水,一点一点地绽出笑的涟漪:``不愧是朕的皇后。''

如懿含笑道:``那么,皇上如何定夺?''

``朕取的不是一个女子,一个嫔妃,而是蒙古的博尔济吉特部。''他咬重了口音,拿手指蘸了白酒在小几上写了个``取''字,``是取,而不是娶,取一女子在宫中,多一个不多,少一个不少。''

如懿浅浅失笑:``皇上如今正宠着恂嫔,倒不怕她吃味?''

皇帝轻哼一声:``朕便是要所有人都知道,即便是朕宠着谁,也不是高枕无忧。既然都是朕的奴才,权衡一些,也叫他们好自为之。''他停下,夹了一筷羊肉慢慢嚼了,``有了蓝曦和厄音珠在宫中,便是平衡了霍硕特部和博尔济吉特部在宫中的势力。而朕,未必要给恩宠,只要是礼遇即可,就如一个摆设一般。''

如懿心中微寒,仿佛是殿外的风不经意吹入了心中,吹起了一层冰瑟之意。

容不得她多想几分,皇帝的声音已经在耳边:``朕已想好,给博尔济吉特氏厄音珠嫔位,与霍硕特氏位分相同。''他微微沉吟,``便封为豫嫔。皇后看看还有什么宫殿可以安置?''

如懿旋即回过神来,笑容如常平和:``这次的新人里,恂嫔和诚贵人住在景仁宫,便是恂嫔为主位。瑞贵人、白常在、陆常在跟着祈妃住在景阳宫。承乾宫暂时无人住着。''她小心翼翼地觑着皇帝的脸色,暗示着可能会到来的让他不悦的记忆,``倒是舒妃死后,储秀宫一直空着,尚无人居住。不如\ldots{}''

皇帝的神色瞬间冷了下来,仿佛为窗外冰雪所浸,冻得寒冷而坚硬。他摆一摆手,很快打断了如懿话语的尾音:``不必了。''

他的话简短而有深重的力度,如懿立即明白,却还是试探:``可是皇上,舒妃死后,储秀宫一直空着,也不大好。''

皇帝的脸色似乎是厌弃,不愿多谈及:``舒妃自焚,乃不祥之人,她的居处也不必让旁人先住着。至于承乾宫,与你的翊坤宫相对,没有合适的人,朕也宁可空着。''他略略缓和,提高了唇角扬起的弧度,``豫嫔么,不拘哪个宫里,先让她住着,当个主位就是。''

如懿思忖着道:``永和宫自玫嫔死后尚无主位,只有几个位分低的贵人、常在住着,倒也合适。''

皇帝拨着盘中的羊肉,漫不经心道:``那就是永和宫吧。''

\hypertarget{ux7b2cux4e8cux5341ux4e5dux7ae0-ux8c6bux5ad4}{%
\chapter{第二十九章
豫嫔}\label{ux7b2cux4e8cux5341ux4e5dux7ae0-ux8c6bux5ad4}}

春日迟迟之时,新入宫的恂嫔霍硕特蓝曦和豫嫔博尔济吉特厄音珠恰如红花白蔷,平分了这一春的胜景韶光。

对于皇帝的宠爱灼热,已经三十岁的豫嫔厄音珠自然是喜不自胜,恨不能日日欢愉相伴,不舍皇帝左右。厄音珠虽然不算年轻,但相貌甚美,既有着蒙古女子奔放丰硕的健美,也有着痴痴切切地缠着皇帝的娇痴。不同于豫嫔对雨露之恩的眷恋,恂嫔的容色浅静得近乎淡漠,仿佛岩壁上重重的青苔,面朝阳光的照拂,来也承受,去也淡淡,并不如何热切与在意。而她的美,只在这冷淡的光晕里如昙花---般在幽夜里悄然绽放。

自然地,以皇帝如今的心肠,一个浑身绽放着热情的、无须他多动心思去讨好的女子比一个对他的示好亦淡淡的女子更讨他喜欢。

丽出身博尔济吉特后族的豫嫔,也因着皇帝的宠爱而很快骄横且目空一切。

所以当如懿对着敬事房记档上屡屡出现的``豫嫔''的载录而心生疑惑时,海兰悄声在旁告知:``皇后娘娘有所不知吧?豫嫔太会拔尖卖乖,有几次明明是恂嫔在养心殿伺候,可是豫嫔也敢求见皇上痴缠,惹得恂嫔待不下去,自己走了。''

如懿蹙眉:``有这样的事?本宫怎么不知?''

海兰摇首道:``恂嫔那个人,倒真像是个不争宠的。出了这样的事也伤脸面,大约是不好意思说吧。臣妾也是听与恂嫔同住的诚贵人说起,才隐隐约约知道一些。''

外头春色如海,一阵阵的花香如海浪层层荡迭,将人浸淫其间,闻得香气绵绵,几欲骨酥。如懿点点头,撩拨身旁-丛牡丹上滴下的晶莹露珠,凝神道:

``其实本宫一直也觉得奇怪,霍硕特部与博尔济吉特部积怨己久,各自送女儿入宫也是为了宫中平衡,怎的恂嫔倒像不把这恩宠放在心上似的,全不似豫嫔这般热切,也不愿与宫中嫔妃多来往,倒与她阿玛的初衷不一了?''

海兰笑言:``或许是每个人的性子不一样吧。可臣妾冷眼瞧着,恂嫔倒真不是做作。也许她出身蒙古,心思爽朗,不喜这般献媚讨好也是有的。''

``心思爽朗?''如懿一笑,撂下手中的记档,``本宫看恂嫔总爱在无人处出神,怕是有什么不能见人的心思,倒真未见爽朗。至于不能相争,霍硕特部自从暗中相助准噶尔之后,皇上冷眼,他们部落一日不如一日,恂嫔不能与博尔济吉特氏相比倒是真的。''

海兰抿嘴一笑,将切好的雪梨递到如懿面前:``娘娘你这个人呀,眼晴比旁人毒就罢了,看出来便看出来了,何必要说出来呢。皇上收了恂嫔,已经是安了霍硕特部的心了,还要如何?''

如懿细细的眉尖拧了一拧,仿佛蜷曲的墨珠。``恂嫔也罢,看来是豫嫔不大安分。''

海兰拨着指尖上凤仙花新染的颜色,那水红一瓣,开得娇弱而妩媚:``博尔济吉特氏的出身,当然不肯安分了。赛桑王爷留羞这个宝贝女儿到了三十岁,可是有大用处的呢!''海兰忽而一笑,凑到如懿耳边,低语道。``听说豫嫔第一回侍寝,居然挠了皇上的鼻子。''

如懿听得面上绯红,半是讶异半是不信,嗔道:``你又胡说!这些事怎能知道?''

海兰面色微红,低低啐了一口:``臣安也不过是听令妃身边的澜翠拖怨.娘娘知道她这个人,嘴快又爱抱不平,定是她哪里打听出来。只为这个,令妃都抱怨她狐媚子呢.虽然颖嫔也是蒙古的,为着这个也不搭理她。不过臣妾也觉得此话有七八分真,否则豫嫔怎如此得宠。赛桑王爷养了她三十年,自然是个和咱们不一样的大宝贝。''说着二人也笑了。

然而,接下来的日子也颇蹊跷。

皇帝人到中年,自然比不得年轻时候,虽然照常临幸嫔妃,侍寝如轮转,但到底日渐稀落了下来。

这一日午后,如懿陪着皇帝在养心殿里,斜阳依依,照出一室静谧。外头的辛夷花开得正盛,深紫色的花蕾如一朵朵火焰燃烧一般,恣肆地张扬着短暂的美丽。那真是花期短暂的美好,艳阳滋暖,它便当春发生,可若一夕风雨,便会零落黄损,委地尘泥。

但,那是顾不得的。花开正好,盛年芳华,都只恣意享用便好。

如懿与皇帝对坐,握一卷《诗经》在手,彼此猜谜。不过是猜到哪一页,便要对方背诵,若是有错,便要受罚。皇帝与如懿都习读汉文,《诗经》并难不倒他们,一页一页猜下来,皆是流利,到把永璂惹得急了。每每猜一页,便抢着背诵下来。稚子幼纯,将那一页诗文朗朗诵来,当真是有趣。也难为他,自《桃天》至《硕鼠》或《邶风》,无不流利。

皇帝连连颔首:``永璂很好。这都是谁教你的?''

永璂仰着脸,伏在皇帝膝上:``皇额娘教,五哥也教。''

皇帝越发高兴:``永琪不错,有了妻室,也不忘教导兄弟。''他抚着永璂额头,谆谆叮嘱:``你五哥自小学问好,许多文章一读即能背诵,你能么?''

永璂倒是老实:``不能,大多要八九遍才会。若是长,十来遍也有。''

皇帝微微摇头,又点头,笑道:``你比你五哥是不如。但,这么小年纪,也算难得了。''说罢又赞永琪,``此子甚好,成家立室后敬重福晋,又不沉溺女色,很是用功。''他说罢,仿佛有些累,便支了支腰,换了个姿势。

如懿打心底里欣慰,不觉笑道:``永琪年长,自是应该的。要不骄不躁才好。''

正说话间,齐鲁向例来请平安脉。他越见老迈,精神却好,向皇帝和如懿请了安,搭了脉,欲言又止道:``皇上脉息康健,一向都好。''

如懿知他老练,不动声色:``本宫瞧皇上面色,最近总是萎黄,可是时气之故?''

皇帝轻咳一声,如懿便默然,牵了永璂告退:``等会儿永璂的福晋还要进宫请安,臣妾先行回去。''

皇帝应准了,如懿牵过永璂的手盈盈告退。到了殿外,她将水璂变到容珮手中,扬一扬脸,容珮即刻会意,带了永璂往阶下候着。

齐鲁年迈,耳力日弱,说话的声音也有些大。如懿临风脚下,只作看着殿前辛夷花出神。荡漾的风拂起她花萼青双绣梅花锦缎外裳,髻上一支红纹缠丝玛瑙响铃簪缀着玉珠子,玲玲地响着细碎的点子,里头的话语却隐隐入耳。

皇帝道:``朕腰间日渐酸乏,前日那些药吃着并不大用。可有别的法子?''

齐鲁的声音干巴巴的:``皇上肾气略弱,合该补养。微臣会调些益气补肾的药物来\ldots{}''

里头的声音渐次低下去。

如懿眉心皱起来,看了候在外头的李玉一眼,缓步走下台阶。李玉乖觉跟上,如懿轻声道:``皇上近日在吃什么药?''

李玉为难,搓着手道:``这些日子的记档,豫嫔小主不如往日多了。可\ldots 皇上还是喜欢她。别的小主,多半早早送了出来。''

这话说得含蓄,但足以让如懿明白。她面上腾地一红,便不再言语。

到了是日夜间,皇帝翻的是恪贵人的牌子。这本也无奇,皇帝这些日子,尽顾着临幸年轻的嫔妃。如懿向来困倦晚,因着白日里永琪的福晋来过,便留了海兰在宫里,二人一壁插花样子,一壁闲话家常。

那本不是接嫔妃侍寝的凤鸾春恩车经过的时辰,外头却隐隐有哭声,夹杂在辘辘车声里,在静寂的春夜,听来格外幽凄。

容珮何等精明,已然来回报:``是凤鸾春恩车,送了恪贵人回来。''

时辰不对。

如懿抬起头,正对上海兰同样狐疑的双眸,海兰失笑:``难不成有人和臣妾当年一样,侍寝不成被抬了出来。那是该哭的。''

年岁滔滔流过,也不算什么坏事。说起曾经的窘事,也可全然当作笑谈。

如懿睇她一眼,微微蹙眉;``什么了不得的大事,哭哭啼啼的,明日便成了宫里的笑话。''

容珮会意:``那奴婢即刻去请恪贵人回来。''

不过片刻,恪贵人便进来了.越本是温顺的女子,如今一双眼哭得和桃子似的,满面涨得虾子红,窘迫地搓着衣襟,却忍不住不哭。

如懿赐了她坐下,又命菱枝端了热茶来看她喝下,方才和颜悦色道:``有什么事,尽管告诉本宫。一个人哭哭啼啼,却成了说不出的委屈。''

恪贵人张了张舌头,又把话头咽下,只是向隅嘤嘤而泣。海兰抚了抚她肩头,``哎呀''一声:``春夜里凉,你若冻着了,岂不是叫家里人也牵挂。在宫里举目不见亲,有什么话只管在翊坤宫说,都不怕。''

恪贵人双目浮肿,垂着脸盯着鞋尖上绣着的并蒂桃花朵儿,那一色一色的粉红,开得娇俏明媚,浑然映出她的失意与委屈。她的声音低低的,像蚊子咬着耳朵:``臣妾也不知自己怎么了?伺候了皇上多年,如今倒不懂得伺候了。''

这话有些糊涂.如懿与海兰面面相觑,都有些不安。如懿索性劝她:``话不说穿,除了自个儿难受,也叫旁人糊涂。''

恪贵人盯了如懿一眼,扑通跪下,抱着如懿的裙裾哭道:``皇后娘娘,臣妾也不知哪里伺候得不好。皇上处理政务想是累了,精神气儿不好,臣妾也不敢狐媚皇上,便劝皇上歇息。谁知皇上推了臣妾一把,怪臣妾不懂伺候。''

暖阁里的都是侍过寝的嫔妃,自然懂得``精神气儿不好''是什么意思。海兰怕恪贵人不自在,索性看着别处的影子装聋作哑。

如懿听了这话头,便知不好劝说.只得拉了她起身:``好了,这事儿也不怪你,皇上的心自该在前朝,如今西陲的战事揪着皇上的心昵。''

她不劝尚好,一劝,恪贵人哭得越发厉害:``臣妾向来不是很得皇上喜欢,不过每月侍奉皇上一两回。可这些日子,不止臣妾,许多姐妹都瞧了皇上的脸色。是不是豫嫔一入宫,臣妾等都没有立足之地了昵?''

如懿听得话中有话,便问:``除了你,还有谁?''

恪贵人掰着指头道:``恭贵人、瑞贵人、禧贵人,连颖嫔姐姐都吃了挂落儿,只不过都咬着被角偷偷儿哭罢了,唯有恂嫔,她也被送了出来,只她不在意,''

她说起的,多是蒙古嫔妃,一向又要好,闺房里自然可能说起。如懿听得心惊肉跳,只维持面上平和;``那又干豫嫔什么事?''

恪贵人眼神一跳,有些胆怯,旋即咬着手里的水红绢子恨恨道:``皇上只说豫嫔会伺候人,唯她没有被早早送出来。''

呵,是如懿疏忽了,只看着是记档上侍寝的日子,缺未注意时辰。如懿安慰了恪贵人,便叫好好送回去。海兰睨她一眼,摇了摇头,只道:``恪贵人一说,臣妾可越发好奇豫嫔了,可是什么来头呢?''

这一日逢着李玉不当班,如懿便唤来了他细细追问。李玉忸怩得很,浑身不自在,吞吞吐吐才说了个明白。原来这些日子侍寝,唯有豫嫔最得眷宠,皇帝一时也离不开,而若换了旁人,次日皇帝便有些焦躁,要去唤齐鲁来。

事已至此,如懿亦不能再问,又细细问了皇帝饮食睡眠,倒也如常,也只得打发李玉走了。

如懿心事重重,海兰知她忧心,论起御花园春色繁盛,特意便带了她一同往园子里去。

如懿与海兰挽着手,漫步园中看着春光如斯,夭桃娇杏,色色芳菲,不负春光,怡然而开,便道:``好好的闷坐在宫里说旁人的闲事,还不如来这里走一走呢。春色如许,可莫辜负了。''

海兰笑吟吟道:``皇上不肯辜负六宫春色,雨露均沾,咱们也且乐咱们的便罢。''

花木扶疏,荫荫滴翠,掩映着一座湖石假山。山前一对狮子石座上各有一石刻龙头,潺潺清水从中涌出,溅出一片蒸腾如沸的雪白水汽。假山上薛荔藤萝,杜若白芷,点缀得宣。一座小小飞翼似的亭子立在假山顶上,一个着茜桃红华锦宫装的女子正坐亭中,偶有笑语落下。

``本宫的母家博尔济吉特氏历来只出皇后,本宫仅为嫔位,自然是委屈了。''

似乎是宫女的声音:``皇上不是答应了小主会即刻封妃么?咱们赶在恂嫔前头成了妃子,可不是打了霍硕特部的脸?小主可是为老王爷争气了!''

豫嫔的声音趾高气扬:``不仅是妃位,贵妃,皇贵妃,本富都会一一得到·左右皇上宠爱本宫,不喜旁人,本富有什么可怕的。''

那宫女道:``皇上如此宠爱小主,旁人都成了东施丑妇,看也不看一眼。即便哪日废了皇后由您顶上也是有的,谁叫咱们博尔济吉特氏专出皇后呢!''

豫嫉笑得欢喜而骄傲:``可不是?从太宗的孝端皇后、孝庄皇后,世祖的孝惠皇后,咱们博尔济吉特氏可是出了不少皇后的,如今的皇后也不过是皇上的续弦继妻,那中宫的宝座能不能坐稳,还是两说呢。''

二人笑语得趣。海兰驻足听了半晌,冷笑一声:``皇上要封豫嫔为妃?怎的娘娘与臣妾都不知晓。''

如懿低头拨弄着护甲上缀着的红宝石粒,不咸不淡道:``这样的话,自然是枕畔私语了。且只是封妃,有什么可张扬的。本宫瞧她恨不得坐上后位才高兴昵!''

海兰蹙眉,嫌恶道:``小小妃妾,也敢凌辱中宫!姐姐也该让她知道天多高地多厚。''

如懿蕴起一抹笑色,清恰如天际杏花淡淡的柔粉:``此刻豫嫔是皇上心尖子上的人,本宫何必去惹这个不痛快。且一次传杖就能灭得了一个人的野心么?笑话!''她神色淡然,转脸道,``听说这阵子纯贵妃身上一直不大好,咱们去瞧瞧她。她也可怜,日夜为了儿子熬心血,也是撑不住了。''

海兰虽然着恼,但如懿这般说,也只得随着她去了。

二人看过绿筠,已是傍晚时分。陪着皇帝用膳的是媾婉。如懿行经永寿宫,看着传菜的太监陆陆续续鱼贯出入,十分齐整安静。皇帝用膳,想来满、蒙、汉菜色齐全,一时流水价往来。海兰眼尖,忽然努了努嘴,见对面长街的转角根下,一个小宫女伸着半个脑袋盯着永寿宫门口。那宫女本掩着身子,若非偶尔被风卷起浅绿裙角,暮色四合之际,倒也不易察觉。

容珮撇了撇嘴,不屑道:``如今底下人越发没规矩了,争风吃醋都派人盯到别人宫门口了,也不管教管教。''

如懿便问:``你认得她?''

容珮点头:``鬼鬼祟祟的主子便有鬼鬼祟祟的奴才,上不得台面,是豫嫔带来的宫女朵云。''

如懿也不多留,只作没瞧见,对三宝道:``留神着点儿。''三宝应承着,众人照旧回宫不提。过了两日,三宝便有了消息:``朵云什么都没做,只看着皇上用膳完毕,便走了。''

如懿思忖片刻:``皇上近日用了什么菜色,你都查了么?''

三宝抹着额上的汗:``都问了.御膳房的规矩,皇上每顿所用菜色大多不同,十日之内绝不重样。倒是皇上喜欢御田米煮的白米饭,每日都用。''他靠近,低声道,``奴才还查了,为皇蠢上做御田米饭的,是与豫嫔小主沾亲带故的。''

如懿眼神一跳,旋即淡然,挥了挥手:``下上吧。''

次日,皇帝下朝,来翊坤宫看过了永璂,便与如懿说起豫嫔封妃之事:``恂嫔虽然年轻,但总是冷冷淡淡的,不如豫嫔温柔热情,又出身高贵。''

如懿脸上瞧不出分毫不悦之色:``说来博尔济吉特氏本是比霍硕特部尊贵些。''

皇帝以为她赞成,便也中下怀:``朕给豫嫔妃位,也是给她母家脸面。所以皇后,豫嫔封妃的礼仪,一定要格外隆重。''

如懿答应着,一脸欢愉得体:``豫嫔既得皇上心意,臣妾一定会好好办妥封妃之事,务求体面风光。''

皇帝走后,如懿便唤来豫嫔密密商量封妃之事。如懿的谦和之色,让豫嫔愈加得意,连容珮奉上的一对金风双头珊瑚珠钗亦不客气地笑纳:``皇后娘娘如此厚爱,臣妾也不敢推辞了。''

如懿含笑:``本宫年纪渐长,看你们几个年轻的伺候皇上如此妥帖,本宫自然高兴。''

外头有乐声传进,如丝如缕,悠扬清逸,反反复复只唱着同一首曲子。

``宝髻偏宜宫样,莲脸嫩,体红香。眉黛不须张敞画,天教入鬟长。莫倚倾国貌,嫁取个,有情郎。彼此当年少,莫负好时光。''

``\ldots 莫倚倾国貌,嫁取个,有情郎。彼此当年少,莫负好时光。如懿闻声侧耳倾听,不禁轻吟浅唱。

豫嫔听了数遍,也生了好奇之心:``怎么皇后娘娘根喜欢这首歌么?外头的歌姬一直在唱这首呢。''

如懿温柔的面庞泛起无限怅惘:``这酋曲子是唐玄宗的《好时光》。本宫与皇上多年相处,皇上最爱在晨起时分这首曲予。如今本宫年长,不比你们时时能见到皇上,所以唤来歌姬解闷罢了.''

豫嫔``哎哟''一声,眸中晶亮一转,侧耳听了片刻,掩唇笑道:``娘娘是中宫皇后,怎么会见不到皇上?可是怪臣妾陪着皇上太多么?''

如懿抚着云鬓青丝,苦笑道:``色衰而爱弛,每曰晨起看见新生的白发,就提醒着本宫青春不再。而太年轻的女子,娇纵任性,皇上也未必喜欢。如你这般解风情,又有大家名门的尊贵,最合皇上心意。所以新人里头,皇上也只属意你封妃。''

容珮忍不住插嘴:``是呢。令妃娘娘入宫多年,儿女成群,也不过是妃位。

小主真是前途无量。''

如懿越发器重,扶住豫嫔的双手:``册封礼的事本宫会为你安排好,一定让你风风光光,享受博尔济吉特氏该享受的荣耀。''

豫嫔饱满如银月盘的脸上洋溢着无可掩饰的喜悦,欠身告退:``那便多谢皇后娘娘了。''

她说罢,便扶了侍女的手大剌刺离去。容珮见她这般,忧色忡忡道:``皇后娘娘近日爱听这首曲子也罢了,怎么好好的让豫嫔听去,窥知了皇上和娘娘的喜好。好没意思。''

``有没有意思,不在这一时!''如懿轻轻一笑,``如今本宫算是知道豫嫔的好处了,待字闺中久了,竟是个妇人的体貌,稚童的脑子。难怪是男人都会喜欢。''她侧首取过一把小银剪子,看着镂雕云龙碧玉瓶中供着一捧捧碧桃花,挑了数段有致之枝,------利落剪下,轻轻哼唱:``莫倚倾国貌,嫁取个,有情郎。

镀此当年少,莫负好时光\ldots''

\hypertarget{ux7b2cux4e09ux5341ux7ae0-ux9999ux89c1ux6b22}{%
\chapter{第三十章
香见欢}\label{ux7b2cux4e09ux5341ux7ae0-ux9999ux89c1ux6b22}}

豫嫔的封妃之日是在三月初一。内务府早就将妃位的袍服衣冠送入永和宫中。

``宝髻偏宜宫样,莲脸嫩,体红香。眉黛不须张敞画,天教人鬓长。奠倚倾国貌,嫁取个,有情郎。彼此当年少,莫负好时光。''

豫妃轻轻哼唱,歌声悠悠荡荡,情意脉脉,回荡在永和宫的朱墙红壁之下,袅袅回旋无尽。

那歌声,直直挑起了皇帝心底的隐痛。几乎是在同一瞬间,豫妃听到了皇帝的怒吼:``你在胡唱些什么?''

豫妃惊得手中的象牙玉梳也落在了地上,慌忙伏身跪拜:``皇上恕罪!皇上恕罪!''

皇帝喝道:``哪儿学来这些东西?好好一个蒙古女子,学什么唱词?''

豫妃慌慌张张道:``皇上恕罪。臣妾只是见皇上喜欢听令妃唱昆曲,又雅好词曲,所以向南府学了这首曲子。臣妾,臣妾\ldots{}''

她讷讷分辩,正在精心修饰中的面庞带着茫然无知的惊惶露露在皇帝眼前,也露出她真实年纪带来的眼角细细的纹路和微微松弛的肌肤。

再如何用心遮掩,初老的痕迹,如何敌得过宫中众多风华正艳的脸。何况是这样新妆正半的脸容,本就是半成的俏丽。

皇帝厉声喝道:``什么彼此当年少,莫负好时光!朕是年近五十,但你也是三十老女。难道嫁与朕,便是委屈了你了么?''豫妃惶惶然,正仰起面来要申辩,皇帝狠狠啐了一口在她面上,``别人想着要年少郎君也罢了,凭你都三十岁了,朕是看在大清数位皇后都出身博尔济吉特氏的分儿上才格外优容与你,却纵得你这般不知廉耻,痴心妄想!''

李玉在旁跪劝道:``皇上息怒,皇上息怒。''

皇帝气得喉中发喘,提足便走,只留豫妃软瘫在地,嘤嘤哭泣。

皇帝气冲冲走出永和宫,正遇见宫外的如懿,不觉微微一怔:``皇后怎么来了?''

如懿的眼里半含着感慨与情动:``臣妾方从茶库过来,选了些六安进贡的瓜片,是皇上喜欢喝的。谁知经过永和宫,听见里头有人唱《好时光》,不觉便停住了。''

记忆牵扯的瞬间,皇帝脸庞的线条慢慢柔和下来,缓声道:``这首歌,是你当年最爱唱的。''

如懿微微颔首,隐隐有泪光盈然:``是臣妾初嫁与皇上时,皇上教给臣妾的。眉黛不须张敞画,天教入鬓长。所以臣妾画眉的时候,总记得当年皇上为臣妾描眉的光景。''有春风轻缓拂面,记忆里的画面总带着浅粉的杏桃色,迷迷蒙蒙,是最好的时光。她黯然道:``原来如今,豫妃也会唱了。''

皇帝的脸色沉了又沉,冷冷道:``她不配!''他伸出手引她并肩向前,``这首歌朕只教过你,除了你,谁也不配唱。''

如懿轻轻一笑:``彼此当年少,那样的好时光,臣妾与皇上都没有辜负。''

皇帝眼底有温然的颜色,郁郁青青,那样润泽而温和。她知道,只这一刻,这份温情是只对着她,没有别人,哪怕日渐年老色衰,他与她,终究还有一份回忆在。不容侵袭。

身后隐隐有悲绝的哭声传来,那股哀伤,几欲冲破红墙,却被牢牢困住。

如懿并不在意,只是温婉问道:``皇上,臣妾在宫里备下了午膳,可否请皇上同去?''

皇帝自然允准,如懿与他并肩而行,唇边有一丝笃定的笑意。

这一顿饭吃得清爽简单,时令蔬菜新鲜碧碧绿,配着入口不腻的野鸭汤,几盘面食点缀。

皇帝便笑话如懿:``春江水暖鸭先知,菜色正合春令,最宜养生之道。只足以汤配米饭最佳,怎用花卷、糜子同食?皇后是连一碗米饭都小气么?''

如懿有些尴尬,屏退众人,方才低声道:``臣妾正是觉得皇上所食米饭无益,才自作主张。''她轻叹,屈膝道,``皇上,都是臣妾无能,若非永琪,只怕臣妾与皇上都懵然不知。''

她说着,击掌两下,永琪进来道:``皇阿玛,皇额娘万安。''

皇帝看他:``有话便说。''

永琪跪下道:``皇阿玛,去岁东南干旱无雨,影响收成,朝廷曾派人赈灾送米。如今春日正短粮,儿臣特意让人从东南取了些朝廷发放的米粮来,想送进宫请御膳房烹煮,与皇阿玛同食,也是了解民间疾苦。谁知御膳房做米饭的厨子支支吾吾,儿臣起疑,便叫人尝了皇阿玛素日所食的御田米饭,却是无恙。''

皇帝瞠目:``既然无恙,你想说什么?''

永琪叩首道:``为皇阿玛试饭菜的皆是太监,所以这米饭他们吃下去无恙。

儿臣想着皇阿玛一饮一食皆当万分小心,又特意请了太医来看,才知皇阿玛所用的御田米饭,都被人买通了厨子下了一味凉药。''

皇帝大惊:``什么凉药?''

永琪面红耳赤:``此中缘故,儿臣已然请了齐鲁齐太医来。'他说罢,便叩首离开。

齐鲁候在外头,早已战战兢兢,进来便一股脑儿道得清楚:``所谓凉药,是专供女子排除异己讨夫君欢心所用的,与咱们中原的暖情药不同,那凉药必得是夫君与旁的女子同寝前所用,若不知不觉服下,总觉酸软倦意,四肢乏力,不能畅意,过了三五个时辰,药性过去,男子便能精神如常,而下药的女子则以此固宠。''

皇帝的面上一层层泛起红浪,是心头的血,挟着一股子暗红直冲上来.掩也掩不住.这样难堪的后宫纷争,却是被心爱的儿子无意中一手揭开,揭开荣华金粉下的龌龊与不堪。如何不叫他赧然,平添恼意。

皇帝额头的青筋根根跳动,一下,又一下,极是强劲:``是谁做下的?''

如懿静静道:``豫妃。永琪说,那厨子已然招了。''

皇帝十分着意:``有毒无毒?''

``无毒。''齐鲁急急忙忙道,``皇上前些日子龙体不快,便是这凉药的缘故。掺在米饭里,无色无味,尽够了。''他慌忙跪下,``微臣无用,不能早些察觉,以致皇上多用药石,都是微臣无能。''

皇帝眉心突突地跳着,咬着牙道:``此事不是你能知道的。若非永琪纯孝,只怕也不能知。''

如懿愀然不乐:``也是臣妾无用,料理六宫不周,才使恪贵人等人平白受了委屈!''

齐鲁似是要撇清前些时日施药无用的干系,又追上一句:``皇上龙体本来无恙,只是被人刻意用药,才精神委顿,不能安心处理朝政。若停了此药,微臣再以温补药物徐徐增进,便可大安了。''

皇帝遣了齐鲁下去,面红耳赤:``贱妇蠢钝,如此争宠,真是不堪。''

如懿婉然含泪:``是药三分毒。豫妃纵然只为争宠,但手段下作,不惜以皇上龙体为轻,实在不堪。''

皇帝紧握双掌,冷哼一声:``豫妃?''

如懿徐徐劝道:``今日是豫妃的封妃之日,皇上的口谕早已传遍六宫,可不要因为一时的怒气伤了龙体。且此事传出,也实在有损皇上圣誉!''

皇帝肃然片刻,只听他呼吸声越来越沉:``朕的旨意已下,断难回转!但博尔济吉特氏狂妄轻浮,心机险恶,怎配为妃侍奉朕左右?李玉,传朕的旨意,封妃照旧,但朕,再不愿见这贱婢。告诉敬事房,将她绿头牌摘下,再不许侍寝,将她禁足于自己殿阁内,无旨不得出来!她便只是这个紫禁城的豫妃,而非朕的豫妃!''

豫妃的骤然失宠,固然引起端侧纷纭。但,谁肯去追究真相,也无从得知真相。流言永远比真相更花样迭出,荒唐下作,从这个人的舌头流到那个人的舌头,永远得着不确定的乐趣,添油加醋,热辣香艳。此中秘闻,厨子已然招供,豫妃也早无抵赖。只是豫妃禁足宫内,再不见天日。

这样的一时之秀,出身望族的宠妃,也可轻描淡写回收拂去,皇后做得久了,真正有一番甘苦在心头.亦懂得如何借力打力,不费吹灰之劲。

真正担忧的,后宫也唯有一个接连有孕的嬿婉。然,为皇帝诞育子嗣的嫔妃不少,也算不得心头大患。有亲生子,有后位在手,如懿并不慌张.只要自己活着,都不算太难。

而让她心弦弹动的,反而是天山的寒部节节败退之后,兆惠所要带回来处置的一个女子。

寒氏香见。

而皇帝,听闻之后亦不过一哂:``区区女子而已.也值得这般郑重!荒谬!''

许多年后,如懿回想起初见香见的那一日,是三月刚过的时候,天气是隐隐躁动的春意荡漾。按着节令的二十四番花信,如懿掰着指头守过惊蛰,一候桃花,二候棣棠,三候蔷薇。海兰傍在她身边,笑语盈盈数着春光花事,再便是春分,一候海棠,二候梨花,三候木兰。

那也不过是个再平常不过的日子。所谓的庆功宴,和每一次宫廷欢宴并无差别。歌依旧那么情绵绵,舞依旧那么意缠缠。每一个日子都是金色的尘埃,飞舞在阳光下,将灰暗染成耀目的金绚,空洞而忙乱。日复一日,便也习惯了这种一成不变,就像抚摸着长长的红色高墙,一路摸索,稍有停顿之后,还是这样无止境的红色的压抑。

直到,直到,香见入宫。

紫禁城所有的寡淡与重复,都因为她,戛然而止。

那一日的歌舞欢饮,依旧媚俗不堪.连舞姬的每一个动作,都似木偶一般一丝不苟地僵硬而死板.上至太后,下至王公福晋。笑容都是那么恰到好处,合乎标准。连年轻的嫔妃们,亦沾染了宫墙殿阙沉闷的气息,显得中规中矩,也死气沉沉。

是意气风发的兆惠,打破了殿中欢饮的滞闷。自然,他是有这个资格的。

作为平定寒部的功臣,他举杯贺道:``皇上,平定边疆之乱,乃出自皇上天纵之谋,徽臣不过是奉旨而行.亦步亦趋。寒歧夜郎自大,终究不堪一击,微臣亦不敢居功。只是此次回京。微臣自汗布得到一件至宝,特地献与皇上。''

嬿婉轻轻一哂,不以为意:``区区女子而已,哪怕是征服寒部的象征,也不必这般郑正其事吧!''

绿筠素不喜嬿婉,也不禁附和:``令妃所言极是。此夫之女,多不吉利!

带入宫中,哪怕只为献俘,也太晦气!''

如懿与海兰对视一眼,深知能让兆惠这般大张其事的,必不会是简单女子,所以在想象里,早已勾勒出一个凌厉、倔强的形象。

而香见,便在那一刻,徐徐步入眼帘。她雪色的裙抉翩然如烟,像一株雪莲,清澈纯然,绽放在冰雪山巅。那种眩目夺神的风仪,让她在一瞬问忘记了呼吸该如何进行。后来如懿才知道,她这样装扮,并非刻意引起他人注意,而是在为她未嫁的夫君服丧。如懿很想在回忆里唤起一点儿那日对于她惊心动魄的美丽的细节,可是她已经不记得了。印象里,是一道灼灼日光横绝殿内,而香见,就自那目眩神迷的光影里静静走出,旁若无人。

她近乎苍白的面庞不着一点儿粉黛,由于过度的伤心和颠沛的旅途,她有些憔悴。长发轻绾,那种随意而不经装点的租糙并未能抹去她分毫的美丽,而更显出她真实的却让人不敢直视的丰采。

在那一瞬间,她清晰无误地听到整个紫禁城发出了一丝沉重的叹息。她再明白不过,那是所有后宫女子的自知之明和对未卜前程的哀叹。

而所有男人们的叹息,是在心底的。因为谁都明白,这样的女子一旦入了皇帝的眼,便再无任何人可染指的机会了。

如懿的心念这样迟钝地转动,可是她的视线根本移不开分毫,直到近身的嬿婉紧紧握住了她的手。

这种突如其来的亲近让如懿深感不适,她尽可能地敛容端坐,却听见嬿婉近乎哀鸣般的悲绝:``皇后娘娘,这种亡族败家的妖孽荡妇,绝不可入宫。''

嬿婉的话,咬牙切齿,带着牙根死死砥磨的戒备。如懿不动声色地推开她的手,想要说话,却情不自禁地望向了皇帝。

瞠目结舌,是他唯一的神态。唯有喉结的鼓动,暗示着他狂热而绝对的欲望,如懿,几乎是默不可知地叹息了一声。

那是没有办法的事。

兆惠得意扬扬,道:``皇上,这便是寒岐的未婚妻---一香见。''

太后蹙眉道:``香见?她已为人妻么?''

兆惠忙道:``太后容微臣禀告。香见之父为寒部台吉阿提,与寒歧为同姓。香见白幼与寒歧许有婚约,但因其父一直不喜寒歧蠢蠢野心,所以一直未曾许嫁,拖延至今,而寒歧也曾扬言.功成之日,便是娶香见之时。''

香见似有不忍,切齿道:``我阿爹虽然不喜寒歧,但我与他自幼有婚约幼。部落之事我不懂,寒歧待我一片情真我却比谁都明白。虽然未嫁,但有婚约,我也是未亡人之身。如今寒歧身死,我与他的情分怎可一笔了上?!''

兆惠想是听多了她这般冷淡的言语,倒也不以为忤,依旧笑眯眯道:``香见乃寒部第一美人,名动天山。又因她名香见,爱佩沙枣花,玉容未近,芳香袭人,所以人称`香妃',深得天山备部敬重,几乎奉若神明。''

太后微微颁首,数着手中拇指大的十八子粉翠碧玺念珠,邪念珠上垂落的赤金小佛牌不安地晃动着。太后闭上眼,轻声道:``原以为笑得好看才是美人,不承想真美人动怒亦是国色。我见犹怜,何况年轻子!''

海兰的目光极淡泊,是波澜不兴的古井,平静地映出香见的绝世姿容,她轻挥着手中一柄象牙镂花苏绣扇,牵动杏色流苏徐徐摇曳,有一下没一下地打在她湖水色刻丝梨花双蝶的袖口:``臣妾活了这一辈子,从未见过这样的美人。先前淑嘉皇贵妃与舒妃在时,真是一双丽姝,可比得眼前人,也成了足下尘泥了。''

绿筠微有妒色,自惭形秽:``哀哉!哀哉!幸好那两位去得早,舒妃还罢了,若淑嘉皇贵妃还在,她最爱惜最得意的便是自己的容颜,可不得活活气死过去!''

绿筠的话并非虚言。皇帝最懂得赏识世间女子的美好,宫中嫔妃,一肌一容,无不尽态极妍,尤以金玉妍和意欢最为出挑。玉妍的艳,是盛夏的阳光,咄咄逼人,不留余地;意欢的素,是朱阁绮户里映进的一轮上弦丹色,清明而洁净。但,在出尘而来的香见面前,她们毕生的美好鲜妍,都威了珠玑影下蒙垢的鱼目。

兆惠颇有嘚瑟:``皇上!寒歧身死,香见自请入宫,以身抵罪!''

颖嫔最沉不住气,怒目对上兆惠谄媚而得意的笑容,她极力克制着自己的声音:``既为降奴,怎可侍奉君上!''

香见既不跪拜,也不行礼。盈然伫立,飘飘欲仙,不带一丝笑意;``我从未说过自请入宫,以身抵罪时你们强加给我的命运!今日我肯来这里,不过是你们拿我族人的性命要挟,要我以俘虏之身,接受你们的种种摆布。''

皇帝充耳未闻,只是定定的望着她,痴痴怔怔道:``你冷不冷?''

众人一惊,哪里敢接语。香见不屑地了皇帝一眼\_冷然不语,兆惠笑道:

``皇上,香见既承父命,有与我大清修好之意。阿提愿代表寒部.请求皇上宽恕,望不要迁怒于那些渴盼和平的寒部民众。然则阿提深爱此女.因此送女入富,望以此女一舞,平息干戈。一切安排.请皇上定夺。''

皇帝惊喜不已,喃喃道:``你会跳舞?''

香见的容颜是十五月圆下的空明静水,从容自若,道:``是。寒歧最爱我的舞姿,所以遍请各部舞师教习。为了不辜负他一片爱惜,我的舞自然不差。''

皇帝注目于容色和蔼的太后,恭谨道:``兆惠平定寒部,得一佳人。皇额娘可愿意观她一舞?''

太后以宁和微笑相对:``曾闻汉武帝时事夫人一顾倾人城,再顾倾人国。哀家愿意观舞.''

``我这一舞是为我父亲,为了我部族活着的你所谓的俘虏。但求你放过他们,许他们回乡,不要受离乡背井之苦.''

兆惠嗤笑道:``你说得头头是道。若是一舞不能让皇上惊艳,什么口舌都是白费!''

香见咬着下唇,凄苦气恼中不失倔强之色。她霍然旋身,裙袂如硕大的蝶翅飞扬,凌波微步摇曳香影,抽手夺过凌云彻佩戴的宝剑,笔直而出。

这一惊非同小可,已有胆小的嫔妃惊叫出声,侍卫们慌作一团拦在皇帝身前。皇帝遽然喝道:``不要伤着她!不要!''

香见凛然一笑,举剑而舞,影动处,恍如银练游走。舞剑之人却身轻似燕,白衣翩然扬起,如一团雪影飞旋。她舞姿游弋处,不似江南烟柳随风依依,而是大漠里的胡杨,柔而不折。一时间,珠贯锦绣的靡靡之曲也失尽颜色,不自觉地停下,唯有她素手迤逦轻扬处,不细看,还以为满月清亮的光晕转过朱阁绮户,陡然照避。

有风从殿门间悠悠贯入,拂起她的捃袂,飘舞旖旎,翩翩若春云,叫人神为之夺。

如懿目光轻扫处,所有在座的男子,目眩神移,色为之迷。而女人们,若无经年的气量屏住脸上妒忌、艳羡与自惭的复杂神情,那么在香见面前,也就成了一粒渺小而黯淡的灰芥。

所有的春光乍泄,如何比得上香见倾城一舞。

正当心神摇曳之际.忽然听得``铛''的一声响,仿佛是金属碰撞时发出的尖锐而刺耳的叫嚣。如懿情急之下,握住了皇帝的手臂,失声唤道:``皇上!''

凌云彻己然挺身护在如懿与皇帝身前,镇静道:``香见姑媳舞得入神,忘了御前三尺不可见兵刃。''

如懿的心跳失了节奏,低首看去,原来凌云彻一手以空剑挑开了香见手中的长剑,唯余香见一脸未能得逞的孤愤恼恨,死死盯着皇帝,懊丧地丢开手。

\hypertarget{ux5982ux61ffux4f20-ux7b2cux516dux518c}{%
\part{如懿传 第六册}\label{ux5982ux61ffux4f20-ux7b2cux516dux518c}}

\hypertarget{ux7b2cux4e00ux7ae0-ux9999ux4e8b}{%
\chapter{第一章 香事}\label{ux7b2cux4e00ux7ae0-ux9999ux4e8b}}

其实香见的眼睛很美,似一眸春水,照得人生出碧凉寒意。而那寒意深处,尽是凛凛杀机。

皇帝的嘴唇微微泛白,面孔却是少年人才有的桃花泛水时的桃红艳灼,他极和蔼地劝下凌云彻,``寒氏不懂御前规矩,你仔细伤着她。''

话音未落,如懿已然觉得太过露骨,却又不便劝什么,只向凌云彻道:``把刀剑利器收起,免得误伤他人。''

凌云彻答应着退到一旁。香见泫然欲泣,却死死忍住了眼泪,仰天长叹,``寒歧,对不起,我报不了你的仇了!''

太后笑意淡淡,仿佛是看着一场闹剧,慵懒道:``寒氏,你可不是真的想要行刺皇帝吧?容你挥剑起舞,是我大清的勇士并不将这些雕虫小技放在眼里。你还真以为到了御前,就能任你为所欲为?''

嬿婉满脸鄙夷之色,``夜郎自大,还真把自己瞧得多了不起了!拼上整个部族的力量,也不过是蚂蚁撼树,还想行刺皇上?''她转了隐隐笑意,软语道:``皇上,此等逆贼,不必姑息。若皇上心慈,也须得即刻赶出宫去!''

皇帝不为所动,只是望着香见温煦如春风,``下次再不许动兵刃了。化干戈为玉帛,朕让你们不远万里来京,就为如是。你可千万别糊涂了。''

香见见皇帝如此殷切,愈加不豫,冷冷道:``挥以钢刀,再给蜜糖。皇帝就是这样将我寒部落玩弄于股掌,来满足自己平定疆域的野心么?''

皇帝原本善于辞令,可眼见香见动怒,亦是皓月清辉、花树凝雪之貌,口中讷讷,一时不能应对。

``愚蠢!''如懿的声音似晴空春雷,骤然划过私语切切的殿中,她双眸微垂,覆落如乌云般的阴翳,语气凌厉,脸上神情却如常清淡,``寒歧以一己私欲,不惜动摇边地安宁,平地起干戈,引来杀生大祸,只能说是咎由自取。你既口口声声自称为寒歧的未亡人,就该赎他往昔罪孽,化干戈为玉帛,保全族人安稳。岂可血溅当场,为这样妄动生杀之事的人殉情?''

香见悲愤不已,双眸血红,指着皇帝道:``可他杀死了我心爱之人,又连累我族人不能保全,成为阶下囚虏,我怎能不恨!我自知杀不得他,但我要以我的鲜血,来写下对皇帝、对你的王朝最深的诅咒!''

``本宫听你念及族人,以为你总算深明大义。可如今看来,也是感情用事、无知鲁莽之徒!皇上为何兴兵寒部?你族人为何成为阶下囚虏?皆因寒歧战起不义。所谓武道,乃指止戈为武!皇上为保家国才不得不出兵平叛。归根究底,大小寒才是使你们家园不保之人。因战伤命,不仁!因战亡族,不义!为这样的不仁不义之徒伤害自己,埋下仇恨,你便罔顾了你父亲与族人的心意,成为不智不孝之人。这样看来,你倒与寒歧是一双绝配!''

香见激怒不已,满脸涨得血红,死死盯着如懿。如懿也不惧,只将纤纤十指垂落于十二朵西番莲沉香紫广袖之外,似霞光萦旋,自云端拂过。

半晌,香见似觉对不上如懿的气定神闲,气息稍馁,怔怔垂下泪来,凄然道:``我怎会不知寒歧起兵,只为满足自己私欲,并非真正为族人争取利益。可我没有办法,他是我心爱的男子,他勇猛,他有智谋,他是草原上的骏马,天空翱翔的雄鹰。我劝他,求他,想要改变他,可他不听我的。在他的心里,只有他的雄图大业。可那样的雄图大业,会毁了整个寒部。''她颓然坐倒于地,痛哭失声,``我只是一个女子,我知道他的错,他的罪,可我对他的情感,是无法改变的。''

如懿望向太后,见她颇为慨然,心下自是怜惜。太后温然轻语,``寒部损毁大半,你与族人千里迢迢入京不易,皇上要见你们,自然不会严加责备,一定会体谅你们身不由己的苦楚。''

皇帝深深颔首,容色清明,``皇额娘所言极是,皇后的话也是朕的心声。''他的目光如柔软的春绸,紧紧包裹着凄苦无依的香见,``你放心。朕会设伊犁将军统辖边地各部,再设参赞大臣管理寒部,一定会为你们重建家园,重归富庶安定的日子。''他见香见只是落泪不语,沉浸在巨大的哀恸之中,浑然未将他的话放在心上,也不觉有些尴尬。

太后见此情形,便好言解围道:``你一路风尘辛苦,又兼饱受惊吓。哀家让人替你在京中整理一个宅子,你与族人且安心住下。过些时日,皇帝会给你一个恰如其分的名位,让你以尊荣之身,回到\ldots{}''

太后话音未落,皇帝急急打断,心急火燎道:``皇额娘思虑极是,儿子也是如此认为。''他唤道:``毓瑚,你带寒香见入承乾宫沐浴更衣,暂住歇息!''他寻思片刻,似下了极大的决心,深吸一口气,``寒部事宜,朕有许多不明之处。将寒香见带入承乾宫,朕会细细问明。''

如懿听得太后之意,大约是想给香见一个固山格格或多罗格格的名位,或是给个诰封,加以厚待安抚之后再送回本部,如此两下安然,也有些神意松弛。岂料皇帝之语突兀而起,惊得四座震动,一时不知该如何应对!

绿筠惊得失色,又不敢看皇帝,只得低着头绞着绢子,压抑喉头即将涌出的咳嗽。忻妃求助似的望着如懿。嬿婉又惊又怒,只不敢露了神色,少不得死死按捺住。太后想要说什么,嘴唇微张,但还是忍住了,默默数着念珠不语。而其余嫔妃,无不色变,默叹。

如懿眉心一动,正欲出言,只觉得手背上多了温暖的沉重。她回首,但见海兰目视前方,平和无澜,只是微微摇首,暗示她不要多言。

如懿胸口一闷,已然抽出了自己的手,稳稳站起,屈身道:``皇上,臣妾忝居皇后之位,不敢不多说一句,承乾宫乃六宫之地,不宜外命妇擅居,还请皇上思量。''

她的话,再明白不过。寒香见怎么封诰安抚都无妨,只要于大局安定有益,她都只会赞成,不会有一丝反对。可若将此女引入后宫,皇帝初见便已神魂无措,若真成为嫔妃,只怕凭空要惹出无端大祸。

皇帝哪里能细细分辨她语中深意,急不可耐道:``奉皇太后懿旨,寒香见移居承乾宫,为承乾宫主位。''

如懿只觉得胸口大震,恍若巨石从天坠落,她却毫无防备,眼见得正中心口,脑中一片白雪纷坠的空茫。而眼前的香见,一味沉浸在哀哭追思之中,全然不懂这道旨意是何意思。如懿极力镇定心神,正色唤道:``皇上,寒氏方才指剑于皇上,此刻就纳入宫中,只怕她心性未驯\ldots{}''

皇帝一摆手,收起眼底汪洋般的迷恋,口角决断如锋,将众人的疑虑与震惊生生割裂,``不必多言,朕自有分寸。''他起身,欲走出殿外,嬿婉忍不住上前几步,将笑意漫上酸楚而焦虑的容颜,``皇上,您方才说过,要去臣妾宫里看永璐。''

皇帝转首看她,那笑容显得有些敷衍,``朕若得空,就会去看永璐。''他的目光空洞而并无留恋的意味,只有逡巡过茫然失神的香见时,才满溢着温软而缠绵的情味。他郑重嘱咐李玉,``将承乾宫好好打理出来。否则,朕就摘了你的脑袋。''李玉诺诺答应,悄然抹去额头冷汗。皇帝再不多言,阔步离去,将一众目瞪口呆尚未回过神来的人丢在身后。

嬿婉见皇帝三魂不见七魄,手心一阵阵冷汗直冒,滑腻得几乎抓不住绢子。如懿轻叹一声,向着身边的海兰低低道:``皇上他,已经不知自己在说什么了。''

海兰轻蔑一笑,``皇上也算见惯天下美女。皇后娘娘且看座下内宠这般多,皇上什么没有见过。今日初见,皇上就这般忘乎所以,皇后娘娘不劝劝么?''

如懿心下微凉,仿佛秋日寒蝉冷露逼仄浸入,``海兰,本宫从未见过皇上这般模样。本宫\ldots{}''她欲言,却有无力感深深攫住了四肢百骸,``你看皇上这个样子,本宫说什么,他还听得见么?''

嬿婉从未见如懿这般灰心丧气,想要说什么,却又颓然坐下了。

嬿婉无可奈何,求助似的望向太后。太后并不看她,含了一丝苦笑,``奉皇太后懿旨。你们都在这里,可曾听见哀家下什么旨意?''

如懿满心不安,立刻屈膝向太后道:``儿臣无能,请皇额娘降罪。''

太后缓缓拨动手中的念珠,``你的确无能。''她将视线扫向一旁心急如焚的嬿婉,``枉你连连生育,也算得皇帝欢心。皇帝如此,你不是也一言不发无能为力么!所以谁也怪不得谁!真要追究,那就是咱们的皇帝心气太过坚硬,无人可以动摇。牢牢记着这句,有你们的好儿!''

嬿婉悄然望向颖嫔处,见她一脸气恨难耐,也不稍加掩饰,只得默然垂首,勉强笑道:``太后莫往心里去。皇上\ldots 皇上一时纵情,说不定一时半会儿心劲过了,也就丢开手了。''

太后并不作声,只是将忧疑的目光投向如懿,沉声道:``皇后,你相信么?''

如懿沉默着低首,太后长叹一声,忧然起身,``哀家本想给寒氏一个固山格格或多罗格格的名位,让她在外安然度日,也好安抚寒部其余人等。却不想皇帝陡然生了招纳后宫的心志。此女入宫,只怕后宫从此永无宁日。皇后,你好自为之吧!''

太后的忧惧是永夜来临前的蒙昧,将惶惑不安的情绪传递到每颗心的底处。如懿身形微微一晃,复又稳稳站住,``有皇额娘在,儿臣等有所依靠,必无忧虑。''

话虽如此,可走到殿外时,如懿还是觉得心头的窒闷如殿外阴翳的铅云,低垂着重重逼迫而下。山雨欲来呵!

她扶着容珮的手,听着心浮气躁的颖嫔在耳边聒噪:

``皇后娘娘,这种亡族克夫的妖女,怎配入宫侍候皇上?

``皇后娘娘,这种祸水,虽然没有嫁人,但到底也是许过人家的,怎么可以为嫔为妃呢?

``皇后娘娘,您得拿个主意啊!''

如懿只觉得脑仁隐隐作痛,终于忍耐不得,以沉默的姿态定定望向她,``那么,你觉得本宫该拿什么主意呢?''

颖嫔登时哑然,却按捺不住气性,急道:``皇后娘娘,皇上即便娶遍蒙古各部,臣妾也不敢有丝毫异议,只为满蒙联姻乃是国俗。可是这种边地小部,又是逆臣贼子的亲眷,野心昭昭,皇上怎能娶她在侧?''

长街的风霍霍穿行,将颖嫔最后的质问扯出尖厉的余音。这话勾得绿筠原本带着病色的面孔愈加颤颤,``皇后娘娘,颖嫔妹妹这话倒说得是。那寒氏今日敢挥剑直指皇上,明日保不齐要做出什么谋逆之事。和这样的女子在一起,只怕会危害皇上龙体啊!''

如懿立在长街正中,任凭啸行的风吹起轻飘的云丝袍角,飞起如扑腾的蝶。她面色阴沉,如坠寒冰,``这样的话,本宫难道没有劝皇上么?''她看向默默跟在身后的忻妃,温然道:``忻妃,你如何打算?''

忻妃垂着脸,静静道:``回皇后娘娘的话,臣妾什么打算也没有。臣妾好容易才有了八公主,一心一意只以公主为念,不作他想。''

如懿微微颔首,``你本是甘于满足之人,如今有了公主,更加恬淡随和。''

忻妃牵动唇角柔和笑意,低头捻着衣角,``臣妾进宫时,阿玛就说过,得不高不低之位,争不荣不辱之地,才得长久平安。''

如懿眼中闪过欣慰之色,牵过她的手道:``春来风燥,于小儿不宜。你先回去看顾八公主吧,免得她惦念。''

忻妃闻言,如逢大赦,急急请安告退。如懿徐徐环视周遭之人,缓声道:``都像忻妃这般有个记挂多好。人有记挂,才会心安,少了那么多心思心眼在旁人身上。''

绿筠有些讪讪,默默退了两步,掩身人后。如懿向着她绽出温和笑颜,``纯贵妃,听说永璋的侧福晋又替他生了个女儿。真好,含饴弄孙,这是旁人羡慕不来的福气。''

绿筠如何不懂,又露出那副怯怯的神气,垂首恭谨,``皇后娘娘说得是。孩子的寄名符还没换,臣妾心中记挂,先告退了。''

如懿关切,唇角绽出一片明净的愉悦,``昨儿皇上赐了本宫两支极好的山参,等会儿本宫便着人给你送去。这两个月来你的咳疾一直未愈,太医说怕是伤着肺腑了,必得好好养着。你切莫操心太过了,你的福气,还长着呢。''

绿筠一壁答应,忍不住又侧首咳了几声,勉强笑道:``皇后娘娘的教诲臣妾都懂了,也请娘娘宽心,皇上只说让她移居承乾宫,终究还没定位分,只怕一切还来得及。''

如此,颖嫔也有些尴尬,不自在地摸着衣袖上繁复的缀珠花纹,眼睛望着不知名的地方,鼻子轻哼一声,``什么位分不位分,都给了主位了,到时候不是妃位便是嫔位,都要和臣妾平起平坐了。''

如懿笑吟吟望着她,口气却肃然,``颖嫔,蒙古诸妃中,你资历最深,也最得皇上宠爱。可是你入宫多年都未有生育,只能抚养令妃之女。若能有一儿半女稳固地位,说话也会更有分量了。''

颖嫔的面孔是典型的蒙古女子的圆脸。可她长得那样好看,是圆月,是玉盘。若是面上那种心高气傲的神气可以稍稍减弱些,她的美会有更摄人的意味。这一刻,她终于被如懿的话击中,不安地低下了高昂的头颅,退到路边,恭送如懿离开。

待回到翊坤宫中,容珮奉上了凉到正好的百合酿金桂露,小心翼翼道:``春来风沙大,易生了燥火,娘娘先喝碗甜露吧。''

如懿就着她的手喝了几口,温润的甜意顺着喉舌流入身体,才觉得浑身的烦闷减去了些许。外头的风更大了,吹得窗扇扑棱作响。菱枝带着小宫女忙不迭地将窗扇密密关上,生怕吵着郁郁沉闷的如懿。

容珮低低道:``看样子是要下大雨了呢。这个时候,开窗风大吹着人,关上又闷得很,真是左右两难。''

如懿眸色沉郁,瞟她一眼,``说话不要这样语带双关。这样的话本宫听得还少么?''

容珮慌忙跪下道:``娘娘心里烦,奴婢知道。可如今这个局势,娘娘不也是两难么。''

如懿伸手蘸了点薄荷膏,轻轻揉着额头,任由清凉的气息渗透肌理,抚平焦躁,``山雨欲来,谁能阻挡?熬得过去的就好好活下来,熬不过去的就成了吹落的残枝败叶。''她郁然长叹,``唉,听着一堆人聒噪,听得本宫脑仁发麻。''

容珮两眼一扫,道:``愉妃小主倒没来说什么。出了殿就没见她人影。''

如懿浅浅一笑,稍有安慰之色,``海兰轻易不开口,要是开口,必定是要紧的话。不像旁人闲扯八道,却无章法。''

两人正说着,却听外头三宝道:``皇后娘娘,愉妃小主来向娘娘请安。''

如懿看一眼容珮,由着她扶正身子,理云鬓,正衣衫,方才道:``请。''

外头湘妃竹帘轻轻一打,海兰已然转了进来,福了一福道:``外头要落雨了,天气怪闷的,便去花房选了些燕草来,清芬满室,又可宁神,最适宜姐姐了。''

如懿淡淡一笑,将手边盛着荔枝蓼花的银罗碟推向海兰,``这荔枝蓼花是你最爱吃的,尝一些吧。''说罢,又向容珮道:``愉妃身子弱,吃不惯百合这样性凉的东西,你去端一碗梨肉枇杷饮来吧。''

海兰取了一片荔枝蓼花慢慢吃了,方道:``姐姐还有闲情逸致想着我爱吃什么,我也谢姐姐一番心意吧。''她起身,牵过如懿的手步至廊下,盈然一笑,``姐姐瞧,我把这些燕草都放在庭中,风吹草动,是不是很好看?''

如懿看着庭下风吹草仰,起伏无状,深深望向她,``疾风知劲草,你想告诉本宫这个么?''

风频频刮起,庭中十数盆燕草修长的草叶狂舞若碧蛇。海兰穿着浅绿的衣衫,盈盈身姿在卷席着微尘的狂风中显得格外怯弱。她的衣裙上绣着大朵大朵盛放的玉色菡萏,被风鼓动得如波縠荡迭的涟漪。她倚在朱漆红柱下,定定道:``人说劲草才能在疾风后留存,我却不太相信。因为只有柔弱的草,懂得随风变化,才不会被摧折。姐姐有没有见过,狂风之后,首先倒下的都是平时看似枝粗叶壮的大树,而细弱的草叶,风来则倒,风去则仰,最后才能安然无事。我很希望,姐姐不要做一棵树,而要如燕草一般,虽然细弱,但能审时度势,俯仰自如,才能清芬满天下。''

仿若有雨水从天空坠落,跌入水面,漾起涟漪微澜。如懿的眸光有了些微变化,她的声音极低,``你觉得,本宫说了不该说的话?''

海兰扶住如懿的手臂,郑重道:``恕我说句大不敬的话,姐姐以为皇后和嫔妃有什么区别么?在我来看,虽然名分有别,但都是仰皇上鼻息,看他喜怒做人。姐姐今日驳斥了寒氏那些昏话,于大礼义正词严,于小节得皇上欢心,最好不过了。我虽在旁不能置喙,但心里也为姐姐击节赞叹。''

如懿纵然为香见之事恼怒,提起皇帝平定边地的韬略,亦不禁欢喜,``皇上心怀大略,平定边地,有不世之功,岂能被寒氏的儿女情长诋毁?本宫虽然身在后宫,不能出去见识扫平叛乱的沙场之战,也能感知皇上运筹帷幄的天纵之才。''

海兰轻轻叹息,``所以姐姐这般忍耐不住?''

\hypertarget{ux7b2cux4e8cux7ae0-ux597dux9011}{%
\chapter{第二章 好逑}\label{ux7b2cux4e8cux7ae0-ux597dux9011}}

这一语,是锋利的刃,割破如懿强忍的抑郁伤怀,``皇上喜新不厌旧,这般性情从本宫嫁与他便知晓。可皇上从不为小儿女情怀所动,当年对慧贤皇贵妃、淑嘉皇贵妃都不曾蒙蔽心志。可今日你也是亲眼所见,皇上看见寒香见时那种迷乱的神情!海兰,本宫陪了皇上大半辈子,他有过太多太多的女人,可是本宫从未见过他用这样的眼神去看一个人。''

``皇上善饮,所以极少喝醉。可是皇上看寒氏的眼神,连最好的酒都不能那样醉人。''海兰低低自嘲,``枉我也曾得过皇上恩宠,原来人与人,就是这般不同。''她的软弱只在瞬间,很快淡泊如常,``不过,我并不会像姐姐那般伤心,像令妃那般失落。早就知道是自己不会得到的东西,就放弃对他的渴望。可惜,姐姐不会懂得。''

如懿黯然失神,``是。本宫就是不懂得,所以才会在大庭广众下劝阻皇上。本宫很傻,对不对?''

海兰安慰地抚过如懿的手,``说对也罢,说错也罢。姐姐是皇后,冠冕堂皇的劝阻总要有一声。但,一言半句也就够了。姐姐知道,承乾宫是什么地方,顺承乾坤,乃是非宠妃不得住的地方。没想到啊,承乾宫空置了数十年,最后竟是让一个逆臣的未亡人住了进去。''

如懿伤感不已,她引袖,以避绝尘埃的姿态,掩去于这短短一瞬间难以抑制的痛苦,``本宫最不明白的是,皇上一生胸怀大略,为何人到中年,才会老夫聊发少年狂,对一个初见的女子这般狂热痴爱?也不顾臣民议论了么?皇上最爱惜声名,竟然为了她,连声名也不要了!''

``皇上固执己见,少有被人动摇。姐姐要牢牢记住这一点,切莫以卵击石,损害自己。另则,人呢,一生总要发一回狂。从前皇上喜欢舒妃的冷冽,如今碰到一个更野性难驯的,岂不平生意趣?所以,姐姐别在这风口浪尖上做什么。旁人再不满,也不会真作声的。''狂风卷起飞扬的尘土,在殿阁的上空肆意飞舞。海兰伸出手,替她遮住眼前纷飞的杂尘,低柔道:``姐姐,眼前的景象混乱不堪,只会脏了你的眼睛。闭上眼,我们不去看。''

如懿强迫自己安静下来,``不看,不听,就可以不存在吗?''

海兰沉静道:``顾着眼前,顾着自己,才最要紧。''她忽而一嗤,带了几分轻藐意味,``不过,姐姐也不必那么在意,事情或许也未坏到那一步。你说,皇上娶淑嘉皇贵妃、慧贤皇贵妃,娶颖嫔、恂嫔、忻妃,都是为了什么?''

如懿瞬间读懂了海兰眼底的蔑视,``本宫固然明白,联姻是最好的笼络和安抚。或许皇上真有此意,可寒氏如此刚烈,怕勉强反而不好!''

海兰的笑意味深长,``对于猎人,不温驯的猎物才是最有逐猎之趣的。''

静默的瞬间,有雨水倾盆而下,哗哗有声,澂起满地尘泥飞溅。如懿与海兰,站在檐下,望着暴烈肆虐的雨水沿着屋檐激流而下,将朱红艳润的重重宫墙染成血色的深红,整个皇宫,便被笼罩在一团巨大的水雾之中,朦胧不见去路。

很久以后,如懿回想起香见初入宫闱的日子,都觉得那段时光是那么朦胧一团。人便像走在大雾中,不知身在何处。大约是每一日都会有让人震撼的新消息传来,让她觉得,平静是一件再难企求的事。

而春日忽冷忽热的时气,夹杂着春雨的潮闷,适时地为如懿的卧病找到了最好的借口。而她的病弱闭门,与太后紧闭宫中一心求佛的举动如出一辙,为后宫的纷乱做下了最好的沉默而尴尬的注脚。

自然,嫔妃们的怨苦声最重,但这一点也不妨碍皇帝频频出入承乾宫的热情与执着。因为哀怨归哀怨,诅咒归诅咒,乖觉顺时是生存的最好法则,谁也不会真的一头碰到皇帝跟前向他大吐苦水。

于是,紫禁城后宫的日子,便在这样的诡异而热切的气氛中踟蹰而前。

只是,所有人的目光,都无一例外地投向了风口浪尖上的承乾宫。其实哪怕假借着时气之由避卧翊坤宫,外头的风吹草动何尝不会一一扫入耳际?

譬如,当香见真正意识到何为移居承乾宫为主位后,她发疯般号啕大哭,举起宝剑数度想要冲出承乾宫,却被凌云彻领着侍卫重重围住。直到皇帝送来她父亲手书,要她安住宫内承奉君上,她才在崩溃后如死寂般平静下来。

譬如,皇帝将历年所藏的珍品悉数送入承乾宫,只为博香见一笑。而她却连眼皮也不肯抬,一味视若尘芥。若是她性起,恸哭之余便将赏赐能碎则碎,如绸锻布帛,则拿过剪子一一剪裂,一壁冷笑连连。每每皇帝到来,她也漠然相向,不发一言。即便皇帝为她带来族人的消息,她也冷言冷语,从不肯启唇一笑。

譬如,她不肯换下素白衣饰,每日只在宫中祈祷她的真神,保佑寒歧死后得以安宁,也借以表示自己乃寒歧的未亡人。对此,皇帝也从不勉强,只吩咐内务府日夜赶制她部族衣衫,或描金刺绣,或镶饰串珠,无不极尽奢丽,供她赏玩。而香见,只是置于一旁,只以自己带来的旧衫更换。

譬如,她每日祈祷之后,只将目光专注地投向家乡的方向,全然不顾望穿秋水,也穿不透重重宫墙。而皇帝,就在她的身后,痴痴望着她的身影,哪怕静坐整日,也不腻烦。

譬如,皇帝怜惜她思乡寂寞,吩咐御膳房每日送上她家乡饭菜,力求精致可口。她却郁郁寡欢。皇帝派人遣她从前的侍女入宫服侍,又嫌人手不足,请她族人中擅歌舞者入宫相娱,却惹来香见睹人思乡,流泪更甚。

皇帝从未有过这样的耐心和热情,自从香见入承乾宫,皇帝每日必有三五次去看她。余者皆过宫门而不入,惹得三宫六院,怨声载道。而那怨声,皇帝自然是听不见的。也幸得香见如此冷待皇帝,皇帝失望之余,才会去嬿婉与忻妃、颖嫔那里稍坐片刻,以得安慰。

但六宫冷待之象,已然初见端倪。

这足以让每一个曾经身承雨露的女子惴揣不安。连婉嫔亦慨叹自怜,``自潜邸起,臣妾也算陪伴皇上日久,可若说皇上对哪位女子钟情至此,臣妾可真未见过。''

海兰伴在身侧,替如懿端过补身的汤药,轻轻吹着道:``皇后娘娘别听这些话,对凤体无益。还是快喝了汤药吧,凉了越发苦。''

如懿接过汤药喝了一口,不觉蹙了蹙眉心。左右那都是些平肝理气、补血养肾的汤药,喝不坏人的。婉嫔大约是意识到这些话会引起女人天性里的妒忌,有些不大好意思地抿了抿唇,取过切好的雪梨嚼了一片,轻叹道:``皇后娘娘这些日子没出去,听说三阿哥又挨了皇上的训斥呢。''

如懿迅速抬眼看了看海兰,取过系在玉镯上的絹子细细拭了唇角,``是啊,镇日这么待着,都快成井底之蛙了。婉嫔,到底是为什么事?''

婉嫔不忍道:``自三阿哥娶了福晋移居宫外,皇上见他性子平和许多,父子间也能闲谈几句。听说\ldots 听说三阿哥言语不慎,得罪了皇上。''她的话语焉不详,叫人听着着急。

海兰会意,拿清水给如懿漱了嘴,方才道:``也是前两天的事,那日三阿哥进宫请安。皇上兴致正好便与他多说了几句,又问起宫外风物人情。三阿哥也是个老实人不知道忌讳,便说外头流言纷纷,都说新入宫的寒氏是妖姬,克夫、亡族,现在又要入宫动摇大清江山来了。''

婉嫔摇头道:``三阿哥也是糊涂,这些话怎可以说给皇上听,岂不知皇上最不喜听这些报忧不报喜的话么?''

如懿忧惧长叹,倚在枕边咳嗽了几声,勉强道:``皇上的性子三阿哥总不留心,难免吃亏。''

婉嫔的眼角含着一缕愁苦,``皇上见话不投机,便问起纯贵妃的身子。娘娘也知道的,自从三阿哥受了皇上训斥绝了太子之念,就成了纯贵妃的一桩心病。总怕父子不合,日夜悬心,如今即便潜心修佛,但身子的泰半不安,都是从这桩事情上起的。''

如懿如何不知,当年皇帝如何在灵前怒斥大阿哥与三阿哥,那种怒发冲冠的景象,多年后仍是历历在目。

海兰温然感触道:``婉嫔妹妹说得是。皇上从来就不喜欢三阿哥娇生惯养,经了这件事,父子越发生分了。如今稍稍好些,三阿哥也太心无城府,张口就来了。大约也是心疼纯贵妃姐姐身子不爽,又受冷落,所以替额娘不平。''

如懿立时警觉,忍不住支起身子来,急切道:``永璋说了什么?''

海兰与婉嫔对视一眼,都有几分欲言又止,到底还是海兰先道:``三阿哥自然是说了纯贵妃的病情,唉,到底也是可怜。除了宫中宴饮,纯贵妃已经每顿茹素,为子女祈求平安。可三阿哥还是自个儿撞了上去,说纯贵妃的病本不重,却是寒氏入宫,才被克的!皇上当时就怒了,说外头愚民昏话,三阿哥也值得记在心里拿到御前来嚼咀,说他越来越不长进。足足骂了大半个时辰,才叫轰出宫去。唉,寒氏心性倔强,皇上求之不得,竟把一腔怒气都撒在了三阿哥身上。吓得三阿哥回去之后便高热烧身,昏迷不醒。''

如懿听得心头乱跳,急道:``三阿哥胆子小,内心又没什么成算,见了皇上本就跟老鼠见了猫儿似的,这下可不吓破了胆!''

婉嫔捂着心口,慌兮兮道:``可不就是吓破了胆!太医已经去看过了,说惊惧交加,直冲心脉,怕是\ldots{}''

如懿听着不祥,呵斥道:``不许胡说!永璋才多大,福气还在后头呢。''她顿一顿,理了理蓬乱的鬓发,轻声道:``你们得空便替本宫去瞧瞧纯贵妃,她只怕是担心坏了!也劝劝她,皇上过了气头就好了,不要往心里去。''婉嫔最心软不过,携着海兰一同答应了。

如懿还是不放心,``永琪\ldots{}''

海兰淡然自若,``皇后娘娘放心。臣妾已经叮嘱永琪,他不会犯下与他哥哥一样的错误。''如懿听着海兰的话语,莫名觉得安心。眼前这个女子,经历过恩宠荣辱的打磨,经历过时光的手残酷地雕琢,仿佛一枚采摘后被遗落的青梅,即便肉身腐毁,却有余留的清新与梗硬。长久处之,让人安心。

但那安心,只是外在赋予的力量。一时间,三人俱是沉默了。内心的起伏里,不知是在感伤绿筠的命运,还是为永璋的前途担忧。殿中静静的,唯听得四面水声,顺着琉璃瓦当急速飞溅而下。

春日里难得的倾盆大雨带着缠绵黏着的水汽弥漫四溢,将殿阁里焚烧的檀香冲得气味寡淡。正沉默间,却见外头湿淋淋冲进一个人来,却是跟着李玉的徒弟小夏。他像个水人儿似的滚进来,唬得婉嫔避之不及。如懿慌了一拍,定睛看去,肃然道:``这个时候,你怎么慌慌张张过来?''

小夏想是急坏了,脸上分不清是水还是泪,哭丧着脸道:``师傅走不开,叫奴才赶紧来知会娘娘一声,纯贵妃小主惹得皇上大怒,挨了一记窝心脚,都呕血了。皇上叫她回宫养着,她也不听,正在养心殿外大雨里头跪着呢。''

如懿只觉得心口一阵阵发紧,她是知道绿筠的身子的,咳疾伤了肺腑,已是重症,哪里经得起这般受罪。她听见自己的声调变了旋律,``到底怎么回事?好端端的皇上怎会这般动怒?''

小夏``嗐''了一声道:``还不是纯贵妃放心不下三阿哥,挣扎着过来向皇上求情,结果言语不慎惹得皇上恨起,就\ldots 就一时没忍住。''

婉嫔胆子小,当下吓得眼泪就下来了。小夏道:``娘娘知道,太后如今是不管事了。再这样下去,怕是要出人命。师傅没个主意,还请皇后娘娘去瞧瞧。''

如懿听得心头火烧火燎,一壁撑着起身,一壁唤了容珮来更衣梳洗,又道:``婉嫔,这事怕有得忙乱。你先去钟粹宫里候着,叫人烧好热水,备下姜汤,请了太医预备着。''

婉嫔忙忙拭了眼泪去了。海兰悄悄扯住如懿衣袖,忧心道:``这件事牵涉着寒氏在内,姐姐真要去蹚这浑水?''

如懿行色匆匆,将宽大的衣袍系于单薄的肉身之上,拢起绿雾云鬟,``绿筠与我们相伴多年,纵有误会,但恩义不浅。本宫不想看她就此殒命。''

海兰见容珮为如懿整理妆容,取过一把十二折竹骨伞,语意清朗坚定,``那么,臣妾为姐姐打伞,风雨同行。''

待如懿与海兰赶到养心殿外时,分辨良久,才看到那伏在汉白玉阶前叩首不已的渺小身影,竟是病弱不堪的绿筠。纵有小太监打伞在侧,她浑身也尽被雨水浇得湿透,衣衫薄薄地贴附在身上,寒气顿生。

如懿急忙解下霞影紫绣栀子散花茜纱披风,兜头兜脸将绿筠裹住,沉声道:``有什么话回宫再说,不许在这儿作践自己身子。''

绿筠哭得俯仰不定,死死擭住如懿的袖子,放声悲泣,``皇后娘娘,臣妾的永璋高热烧得昏迷不醒,实在快不成了!臣妾来求皇上宽恕永璋的罪,这孩子是无心的,他不是故意要顶撞皇上的!皇后娘娘,您别管臣妾,您替臣妾求求皇上,宽恕了永璋吧!''

海兰连忙扶住了绿筠,死命拖她起身,不让她跪在汹涌的急雨与水洼之中,``贵妃姐姐,你快起来,自己的身子要紧。永璋病着,一切都指望着你呢。你何苦在皇上气头上再重提此事!''

绿筠闻得此声,愈加悲切,``皇后娘娘,您不知道永璋病成那样糊涂,还心心念念唤着他皇阿玛,不停地说`皇阿玛息怒'。臣妾身为他的额娘,真是不忍心啊!''

如懿示意宫女上前扶住,安慰道:``你别着急,过了这几日,皇上定会明白过来的。''绿筠被拖扯着半倚在侍女身上,泪眼婆娑,一张脸青白得可怕。如懿定神望去,更是心惊。纵然有雨水冲洗,绿筠的衣襟上仍有斑斑点点暗紫的血迹,触目惊心。

如懿连忙道:``怎么呕血了,可是伤在哪儿了?''

可心带着哭腔道:``皇后娘娘,皇上方才生气,一脚踢在了小主的心窝上。小主不防,所以呕了血了。''

雨水猝不及防地扑上身来,春日的雨水尚有寒气,立得久了,雨水如鞭挥落,抽得脸上、身上一阵阵发痛。她犹自如此,何况绿筠是病久了的人。奈何绿筠无论如何也不肯离开,挣扎着往地上跪去,``皇后娘娘,求您开恩,让臣妾跪在这儿直到皇上息怒!''她仰起脸,痛声哭喊:``皇上,若有什么责罚,都让臣妾受着吧。臣妾教子不善,都是臣妾的过错。''她每说一句,便往前膝行一步,重重叩首。如此反复数次,直到行至殿前廊下,复又退回瓢泼大雨中,再度开始。皮肉碰击砖地的声音在雨中显得格外沉闷而悠长,仿佛重锤落于心间,恻然疼痛。

数次之后,如懿再忍不住,匆匆步上玉阶立于养心殿门外。哀求道:``皇上开恩,请顾怜纯贵妃有病在身,实在不宜如此劳动。皇上息怒开恩啊!''

她的恳求在雨水茫茫中听来格外微弱。连她自己也不知道,这样的恳求是否会得到皇帝的回应。她忽然觉得,自己是如此渺小,如同阶下茫然叩首哀痛不已的绿筠一般,微如尘芥。

也不知过了多久,养心殿的朱漆填金门霍然打开,门扇开合间沉重的余音,为她唤起一缕希望。

皇帝颀长的身形投下巨大如剑削的影子,将她被水汽氳得潮湿的身体覆盖而下。他的声音如同从遥远的天际传来,冷漠而渺远,``皇后不好好待在自己宫里,陪着疯妇一起糊涂做什么?''

如懿心头阵阵发紧,连忙道:``皇上,纯贵妃有病在身,一时糊涂冲撞了皇上,还请皇上恕罪,容她回宫吧!''

皇帝冷然道:``朕从未要她留在养心殿前现眼。她自己执意如此,朕有什么办法?''

绿筠见皇帝出来,手忙脚乱匍匐上前,抓住皇帝的袍角,泣不成声,``皇上!是臣妾的错,臣妾不该向永璋说起后宫之事,不该让他对承乾宫心生怨怼。但臣妾真的不是有心的,永璋也是说者无心,他只是心直口快。皇上,您知道的,他就是这么个孩子,您别与他计较啊!''

皇帝一脚踢开她的手,厌恶道:``这样的话你已经说了很多遍,朕听着也厌烦了。你从没什么好主意教你的孩子。永璋庸懦,永瑢无能,幸好璟妍是个女儿家,否则又被你耽误了一个。''他指着廊下打着伞默默候立的海兰,越发气不打一处来,``你不能学孝贤皇后当年怎么管教皇子,也大可学一学愉妃。同样生了儿子,永琪还比你的儿子出息,但她就不会钻营,懂得安分守己,懂得如何做一个好额娘。而不是像你这般,惹是生非,心术不正!''

绿筠惊得面色惨然,呼吸急促如潮,一仰身险险倒在如懿怀中。如懿听皇帝的话说得狠戾,知道是动了真怒,忙拉过绿筠在身后,劝道:``皇上息怒。纯贵妃为了永璋已经伤心坏了,她担不起皇上这般重贵。''``她担不起?''皇帝从袖中取出一物,掷于绿筠面前,``朕刚才踹你那一脚不是朕气糊涂了,那是你该受的!当年你自己做下的好事,还敢说自己不是心术不正!你和淑嘉皇贵妃一样,便是有你们这样的额娘,才有这般不肖之子!''

如懿见绿筠脸色苍白,几欲昏厥,忙扶住了她。目光扫视之处,却见皇帝抛下的是一枚烧蓝鎏金蜂点翠绣球珠花,那式样极是眼熟。如懿细细辨认,讶异道:``皇上,这枚珠花是您当年赏赐纯贵妃的,一共六对。这一枚怎会在您手中?''

皇帝激怒不堪,``她自己做的好事,自己知道!当日素心死得蹊跷,死时手中紧紧捏着这枚珠花,能说与她毫无干系!''

\hypertarget{ux7b2cux4e09ux7ae0-ux503eux96e8}{%
\chapter{第三章 倾雨}\label{ux7b2cux4e09ux7ae0-ux503eux96e8}}

仿佛有巨浪汹涌澎湃而下,那是多少年前的旧事了。或与金玉妍有关,或许也有绿筠的嫌隙。但,那毕竟是许久以前的事了。岁月荒芜了烟草,谁还分得清真假呢?要紧的是,这些年来,绿筠的确不是本性恶毒之人。

绿筠激动得说不出话来,拼命摇头,喉中发出荷荷怪声,一张脸紫涨不堪,几乎要喘不过气来。

海兰静静跪下,看着几欲晕厥的绿筠,柔声道:``皇上,皇后娘娘不说话,是与臣妾想的一样。多年前的事了,谁还说得清到底是谁害了谁,还是偶然巧合,或是被人设局陷害?孝贤皇后与素心都闭目于九泉,咱们又何必苦苦追宄?臣妾恳请皇上一句,息事宁人,也当为寒氏求个安宁吧。''

她的话,让皇帝的怒气稍稍平息,如懿将绿筠扶到海兰怀中,使个眼色示意她们退下,温然劝慰道:``皇上,寒氏初入宫闱,已然惹来无数非议。纯贵妃资历既深,又有儿女,便是说了什么不中听的话,您听过也罢了,何必与女子计较?''说罢,盈然起身,挽住皇帝手臂,缓缓踏入暧阁,将一室喧闹留于殿外。

如懿与皇帝一并坐下,捧过皇帝吃残的茶,挥手倒去,盈盈一笑,``所有烦恼事,如这残茶,泼去可好?''

皇帝犹有余怒,别过头道:``朕也想不恼。可气的是贱妇久在宫闱,还这般不识大体,引起纷扰。''

如懿思忖片刻,用清水缓缓冲洗杯盏,投入陈皮与甘菊,以滚水冲泡,看着甘菊一瓣瓣绽开于水中,盛放出宁神甘和的怡然香气,方才递与皇帝,``纯贵妃的性子算是好相与,都有些微怨言,何况旁人?皇上纵然爱惜寒氏,也不能引起六宫怨言。雨露均沾,才是六宫和睦之道。''

皇帝怔了片刻,颇为苦恼,握住她的手道:``如懿,你一定觉得朕昏了头是不是?朕宠爱寒氏,自己也觉得是在发疯。可朕一点办法也没有,完全不受控制,做任何事,就想换她真心一笑。''如懿听着他字字句句,直如剜心一般,抛开皇帝的手道:``皇上对着臣妾说这样的话,是当臣妾为无欲无求无心无肝的女子么?可以任由夫君向自己诉说对别的女子的衷肠痴心!''

皇帝懊丧不已,牵住她的手丝毫不肯放松,``如懿,除了你,这样的话朕还能对谁说?朕对着寒氏已经有无限烦恼,可后宫还是不让朕有片刻安宁!朕能征服最凶蛮的部族,却征服不了一个女人的心,你叫朕如何不恼不恨?''

如懿满心气不过,愈加掺了酸涩之意,道:``皇上纵然满心要征服寒氏,又与纯贵妃母子何干!再不然,永璋还年轻没历练过,何苦这样唬着他?''

皇帝一提永璋,便生不豫,``永璋是朕的亲生子,朕怎么会不疼他?可是朕每每见他,都是这般懦弱无能的样子。朕真是恨铁不成钢!''

如懿切切劝慰,殷殷道:``皇上待永璋,每每呵斥多于教导。也难为皇上,有那么多阿哥,难免不能一一细心。可于纯贵妃而言,三阿哥是她爱子,她如何不焦心爱惜?皇上所言所行,不仅伤了父子之情,也伤了纯贵妃的心。''

皇帝将手中杯盏重重一顿,``慈母多败儿。若无她宠溺,永璋不会被纵得这般不成样子。若非她挑唆,永璋怎会擅言宫闱之事,议论长辈妃妾?若她肯严加管教,当年也不会生出那般夺嫡之心\ldots{}''

``严加管教并非镇日耳提面命,呵斥责骂,而是告诉孩子们,什么该做,什么不该。便是做得不好,到底孩子们还年轻,慢慢改过便是。皇上何至于动辄打骂,寒了子女心意?''

皇帝甚为不满,睨着她道:``如懿,朕知道你口舌伶俐。但令妃也有她的好处,温言软语,是朵解语良花。她可从不敢对朕这般说话。''

如懿一滞,不意皇帝会说出这番话来。然而顶撞亦是不宜的,且看绿筠便知道。她将心口的滞郁压了又压,缓一缓急促的气息,极力柔婉道:``皇上的话,臣妾记着了。臣妾只是想,永璋再不好,到底还是个淳厚的孩子。当年便是有过夺嫡之心,这么多年的挫磨,惶惶不可终日,也尽够他学乖了。皇上教导阿哥们严格些自然是好,可若伤了孩子的心,怕要挽回也难了。皇上难道忘了永璜英年早逝么?如今又要赔进一个永璋,天家父子,何至于薄情如此!''

皇帝听如懿说得伤怀,也不禁软了心肠,慨然道:``朕是对永璜和永璋多有不满,深觉二子野心勃勃,不肯安分。可他们到底是朕的儿子,这些年,怕也不好过\ldots{}''

如懿黯然道:``皇上说得是。早年阿哥们不懂事,总是因为孝贤皇后是嫡后,是皇上心爱尊重之人。可如今为了一个名分未定的嫔妃,就连对纯贵妃多年侍奉之苦也不怜悯,对永璋的拳拳孝心也视而不见。那么,恕臣妾直言,这便是皇上的过错了。''

皇帝横眉冷对,``皇后,连你也要逆朕的心意?''

如懿伤感而气恼,``臣妾不是要逆皇上心意,而是觉得皇上一向仁和御下,前几日申斥了永璋,今日又对他额娘大发雷霆,难免伤了宫中祥和。纵然纯贵妃有什么错处,皇上念在她生儿育女,多年劳苦,也宽恕了吧。''

皇帝沉默良久,有几分愧意,``今日是朕急躁了,勾起当年孝贤皇后的旧恨,又想起素心死时,手里握着的珠花便是纯贵妃的。想着他们母子这般勾结蒙蔽违逆朕,朕真是一时恼恨过了头。''

如懿凄声求道:``这么多年了,皇上虽然对素心的死有所疑虑,但毕竟一枚珠花做不得数,皇上都没有提起。而臣妾敢拿自己性命发誓,这件事,确是当年金玉研栽赃所致!''

皇帝连连冷笑,凄惶不已,``金玉妍?人都死了,许多事未必都能水落石出!也不必什么事都扯到死人身上!当年孝贤皇后仙逝,宫里多少见不得人的事,你以为纯贵妃就事事干净了!朕的身边,可不知都是些什么人呢!''

如懿心头颤颤,凄然中带了一抹难以抑制的凌厉,``皇上今日这般怨怼,不过是因寒氏而起。臣妾不敢劝皇上不要宠爱寒氏,但若为了一个新人,惹得六宫不宁,父子失和,实在太因小失大了。''

皇帝断然挥手,将如懿的劝诫生生截断,``寒氏之事朕自有分寸,后宫不许妄议。种种是非,都是因为后宫女子妒心甚重,饶舌起的是非,没的带坏了朕的阿哥!诸位阿哥之中,永璋最是年长,他若起了这个头,叫朕还怎么教导其余阿哥!''

如懿万般放心不下,``自从永璜死后,永璋就是皇上的长子。皇上要严格教导孩子,臣妾无话可说,可过严吓着了孩子,又有什么意思。永璋自己也是有儿子的人了,还被皇上吓成这样,您叫他以后怎么做人阿玛?''

皇帝长叹一声,脸色稍解,``罢了。你叫江与彬亲自去瞧瞧,就说是朕放心不下。''他说罢又气,``说来还是纯贵妃自小宠坏了他,一点风浪也经不得,这便吓着了,日后如何能成器?''

如懿郁郁不安,``皇上还要怪罪纯贵妃母子么?一个两个都病成了这样,人在病中心志弱,别落下病根才好。皇上得好好安慰纯贵妃才是。''

皇帝终归也过意不去,缓了缓道:``朕伤了自己儿子的颜面也不好过。但永璜庸懦,不堪王爵。念在纯贵妃侍奉朕多年,也算小心谨慎。朕今日又伤得她重了,便给纯贵妃恩典,晋封她为皇贵妃吧。''

如懿心中闷闷地难受,以母子颜面身体之损,换来一个皇贵妃的虚名,到底值得不值得?容不得她心思念转,皇帝已然道:``既然纯贵妃病着,封皇贵妃的仪式能简则简,不必过于张扬了。''

于是,皇帝气恼归气恼,事情终究是圆过去了。

绿筠受了这番折辱,心气大损,身体也急剧地败坏下去。如懿最放心婉嫔稳妥,叫她时常打点着钟粹宫的事宜,其余人等一概不许去吵扰绿筠静养,才算把各色目光拦在了钟粹宫外。

然而绿筠的境况很是不好,虽则有晋封皇贵妃的喜事,但她的病情却毫无好转。反而像被蛀透了的腐木,摧枯拉朽般倒塌下去。

如懿与海兰一日三次去看绿筠,她却只是面壁相向,嶙峋的肩胛骨凸显于湖色生绢寝衣之下,骸突可怖。她无力起身,只是对着床壁一味哭泣,背身不肯相见。唯有侍女含泪相告,绿筠每日呕血不止,怕是实在不成了。

无人时,如懿独自守在绿筠床边,为她梳理披散逶迤的青丝,说起宫外永璋府中的点滴。更多的时候,绿筠像一潭死水,平静得让人害怕。

良久,她才涩然应答:``皇后娘娘,臣妾罪孽太深,连累了自己的孩子。您就让臣妾安静等死,换回皇上对永璋的疼爱吧。永璋,他实在是太苦了。''

如懿握着一把象牙梳,低低道:``皇上已经遣太医去看永璋了。为了表示对你的歉疚,皇上也下旨封了你为皇贵妃。绿筠,高兴点儿,想开些,好好活着。''

绿筠枯瘦的肩轻轻一动,像是骷髅的骨嘎嘎有声,她似乎是在笑,笑声里带了哭腔,``中年呕血,命也不得久了。也好,臣妾这一辈子的心血,都给了孩子,若能以臣妾一死,换来皇上对永璋的谅解,那臣妾心甘情愿。至于这个皇贵妃,皇上也知道臣妾快死了吧?当年慧贤皇贵妃死前,皇上也封了她为皇贵妃,金玉姘更不用说。看来皇上厌弃了谁,盼着谁快死了,就许她一个皇贵妃。皇上,他好仁慈啊!''

如懿酸楚不已,手指轻颤,只得忍住了道:``本宫知道,这回你是伤透了心。你为皇上生儿育女一辈子,最后还落得皇上如此猜忌。本宫看着,也倍觉唇亡齿寒。''

绿筠的声音在颤抖,``臣妾做梦也没想到,皇上会为了一枚连臣妾自己都不知什么时候掉的珠花,便如此猜忌。臣妾失宠这么久,自己也不知所为何事。难怪,难怪,活该臣妾死得糊涂!''她说罢,向隅无声,也拒绝服药,只默默等死。

这样的日子并没有维持多久。

乾隆二十五年四月十九日,皇贵妃苏绿筠,薨。谥号纯惠。

她在一个春雨沥沥的夜晚寂然死去,死得无声无息。宫女们为她送来早晨需要服用的汤药时,才发现她的身体已然凉透,头却依然向着宫外永璋府邸的方向。这个性格软弱的女子,就这样默默逝去。好像暴雨里枝头残弱摇曳的花朵,冥然凋零。

很快,她的儿子,三阿哥永璋也追随他的母亲而去。母子相伴地下,也算有所依靠。

这对母子的遽然离世,并没有惹起宫中过多的关注。因为连同皇帝,所有人的目光都聚集在如寒冰困城的承乾宫。一对失宠而死的母子,实在不能让人有任何谈兴。

这一个闷热的夏季,就是这般让人室息而无力。皇帝的热情愈高,征服欲愈强烈。所有女人的心,便一分、一分地冷下去。

这一年的秋天,皇帝也没有去木兰秋狝。所有的追逐狩猎,如何比得上收获一个绝世佳人冷傲的心?他一直忙碌着,除了朝政之外,就是出入依旧冷漠的承乾宫。

这一日,秋色初起,皇帝于秋色茫茫中踏入静谧的承乾宫内殿,面上有不胜欢喜之态。偌大的承乾宫中,其实寂静得如荒漠戈壁,毫无生气。只因香见并不喜欢宫人服侍,素日只让自己从前的侍女在侧,除了向真神祈祷,只是呆坐终日,不言不语。而承乾宫外,宫禁格外森严,虽然皇帝从不禁止她出行,可是在那次失败的奔逃之外,她再无行走宫闱的欲望。

皇帝转入内殿时,香见正倚在暖阁窗下,寂然望着天边日暮,愈坠愈浓。皇帝见她侧影如剪,绝美容颜中满溢刚烈清绝之色,不觉心旌动摇,缓下了脚步,凝望她翩然的身姿。

暮霞沉沉,天际细月如钩。寂寞空庭,黄叶醉染,宫人逐一点亮檐下琉璃宫灯,一任晕黄灯光,幽幽洒落。微黄的暖色下,香见的肤色仍是见惯的苍白,和着身上层层银线绢罗纱衣,神色始终淡漠如在无人之境。这样的她,有一种近乎支离破碎的脆弱感,像是秋夜白露,却不知会在何时,倏然被阳光蒸发,消逝不见。

这样的感觉让皇帝深深不安,他迫近两步,静静含笑向她,低声下气道:``香见,朕来瞧你。''

她并不理会,甚至连身形也未挪动一分,只是望着天际扑梭展翅的乌鸦,露出一丝神往之色。皇帝对她这样的冷漠已然习以为常,便示意李玉捧过手中满插枫叶的玉瓶,讨好地笑道:``这才入秋,御花园的枫叶红了。眹知道你不喜欢出去,特意折来给你细赏。''

那一捧枫叶烈烈如血,殷红欲滴,给满殿的冷落平添一痕融融之温。香见充耳未闻,李玉乖巧地上前,将玉瓶捧至她面前,却招来她低低惊惧的呼喊和一脸的厌恶痛恨,``拿走!拿走!''

幽居承乾宫数月之后,她已然失去了刚入宫时的激烈。更多的时候,是如死水般的沉寂。所以,这一刻她突如其来的情绪波动,惊得皇帝伸手就要揽住她,急急安慰道:``别急!别急!你若不喜欢,朕便叫人撤走!''

李玉见状迅疾退下,将枫叶丢到外头小太监手中,又垂手侍立一旁。香见像是怕碰到什么污秽一般,剧烈地挥动双手,避免皇帝的手触及自己,一壁恨道:``你们就喜欢这样恶心的树叶?像血一样,像大军攻进我们的部族时一样,都是血,到处都是血!太可怕了!''

皇帝见她如此激动,生怕她伤着自己,忙忙退开两步道:``香见!战争的确会有流血牺牲,但一切都过去了!你不要去想,不要去记得,好好留在宫里。朕会好好待你的!''

``好好待我?''香见倏然怔住,惘然凄笑不已,``乌鸦都可以在天空自由地飞,我为什么不能再骑着骏马回到我的故乡?你放我走,我要回我的家乡,和我的父亲、族人在一起。''她的话语里带着深深的哀求与凄凉,``让我回去吧!我要去找我们的家园,我要去给寒歧守他的坟墓!''

皇帝的目光遽然一跳,像是被疾风闪过的火焰。他温和地笑,如要融化的甜沙。``香见,半个月前你已看过你父亲的亲笔书信,他希望你为了自己的族人,留在朕身边。''他悄悄走近一些,眼神越发温柔,``香见,你知道朕的那些妃子吗?颖嫔和恪贵人出身蒙古,豫妃是博尔济吉特部送入宫的,恂嫔是霍硕特部的格格,淑嘉皇贵妃是李朝贵女。每一个部族想要与大清永远和平安定,都会与朕结为姻亲。寒部也不例外。因为只有至为稳固的婚姻,才能确保朕会将恩泽永世施于对方。''

香见悲绝而愤怒,沉沉低吼:``我知道,父亲一定是受了你的逼迫。''

``不是逼迫。''皇帝负手而立,闲闲而沉笃,``是你父亲懂得世易时移,要保全部族的长久安稳。你在朕身边,是最好的办法。''他看一眼李玉,李玉即刻会意,从进保手中捧过香色嫉位袍服,恭恭敬敬端到香见面前。

香见一见便移开目光,大有抗拒之色。皇帝凝望她的眼满是温柔,``你入宫多时,一直未肯更换满服。朕想着你身份未明,一时也不勉强。只是你的身份若一直悬而未决,宫中流言蜚语也不甚好听,连皇额娘也颇有微词。''他一顿,语意中透出一丝坚决,``朕意已决,决定册封你为嫔位,封号容。即日易服,好好做朕的妃子!''

香见大为惊恐,如避瘟疫,``不!我不!我不要做你的妃子。我有我的心上人,寒歧虽然死了,虽然有过错,可我不能改变我的心意!''

皇帝微微蹙眉,仍是笑意温煦,傲然道:``做朕的女人,不比做一个逆臣妻子好么?何况你与他只是定了婚约,并非真正嫁与他,何必在意这些?''

香见的目光如冷剑一般,缓缓打量着他,带了几分不屑,``我在意的除了我与寒歧的情感,更是你的品行。这几天这儿有丧仪,我知道的。你的儿子刚死,你的皇贵妃也死了,是因为我。他们尸骨未寒,你怎么能立刻和我在一起!''

皇帝骤然听她提起永璋母子之死,面色大为尴尬,他微微咳嗽一声,勉强道:``妃妾之死,庶子之死,都是他们自己怀罪而死。朕已经不追究了,也许了他们死后哀荣。而且人虽死,日子却要过。''

``人虽死,日子却要过?死的人是陪伴了你多年的女人,是你的儿子!''香见的脸上是难以置信而带来的怒意与鄙夷,``不!你这么对待他们,也会这样对待我!我不要和你这样的人在一起!''

皇帝抢身上前,紧紧捉住香见素白柔荑,急切道:``不会的!香见,朕的心意你已经看见了。朕从来没有为一个人等待那么久!朕会宠爱你,疼惜你。让你成为朕最宠爱的女人!朕一定会!''他扬起下颌,示意李玉捧过嫔位袍服,柔情万千,``穿上它,香见,成为朕的女人,好不好?''

香见极是抵触,仿佛被皇帝捉住双手是极不堪的事。她的脸因此而显得扭曲,极力挣扎着想要摆脱皇帝的触碰。她身形本就小巧,兼着裙袂蹁跹,挣扎间若素雪飞扬,皇帝一时情动,使了一个眼色,李玉当即乖觉退下,将殿门紧紧掩住。

皇帝眼见无人在侧,伸手便欲将她捕入怀中。香见如何肯依,拼死往后退开,以期避得越远越好。

皇帝的口吻热切而混乱,眼底有燃烧的火色轰然绽开,``香见!他不过是一个部落的首领,而朕是一国之君,万里江山的主人。你的美色,只有在朕身边才最适合。香见!香见!到朕这里来!''

香见欲哭无泪,左右躲闪,却是那样无力而单弱。皇帝手上一用劲,她向后一退,手臂上素衣飞裂,露出大半截欺霜赛雪的晶莹娇肌。

\hypertarget{ux7b2cux56dbux7ae0-ux7ea2ux989cux54c0ux4e0a}{%
\chapter{第四章
红颜哀(上)}\label{ux7b2cux56dbux7ae0-ux7ea2ux989cux54c0ux4e0a}}

有瞬息的恍惚,仿佛惊见冰山雪莲自万丈冰雪间骤然绽放,目眩神迷,口中油讷。香见又羞又气,趁着这一瞬的松脱,身形轻旋,自他掌心逃出。象牙缕碎金妆台上正搁着一把刮眉的小银刀,那薄薄一片,原不在皇帝为防她自戕所收走的利器之内。她伸出右手,将那闪着银光的小刀横在颈前,厉声喝道:``你别过来!''

皇帝大惊,却也极快地镇定下来,``香见!你别糊涂!那把刀根本不足以割开你的喉咙,顶多只会让你留下一道疤痕。你也不用妄想用这个东西来行刺朕。你冷静些,别做伤害自己也伤害朕的事!''

香见死死抓着小银刀,泫然欲泣,却被深重的绝望与愤怒湮没,``我不会再行刺你。因为这样,会给我的族人带来弥天大祸。而且,我心里也明白,虽然你打败了寒歧,但你是对的。寒歧妄图以战争来获得更多的权力,使我的族人们陷于战火之中,不得安宁。可是我没有办法,我明知道寒歧是错的,我还是爱他,就像爱我的天神一样。''

皇帝的喉间有``咝咝''的喘息声,是极力压制的羞辱与怒火。他克制着道:``难道这些日子,你还看不出朕对你有多好?香见,你不要挑战朕对你的爱惜与忍耐。''

她满目悲枪,好像在大雪中迷茫失去方向的孤狼,哀伤深入骨髓,``我是寒歧未婚的妻子,我不能成为你的妃子,让自己成为他死后仍然不能消失的屈辱!''她一步步踉跄后退,摇首道,``我知道你是皇帝,你坐拥天下,你拥有让我的族人存亡的力量。所以我不能毁灭你,但我可以毁灭我自己!''

她话音未落,右手高高举起银刀,挥手便往自己如花似玉的面孔上用力割去!皇帝大惊失色,只觉得浑身的血液一下子涌到了头顶,四肢百骸酸软而冰冷,抽去了所有力气。他来不及想,也来不及反应,揉身扑了上去,以身体挡开那雪亮的锋刃。

有滚烫的猩红喷薄而出,溅出一道血色的弧。

皇帝整个人扑倒在她身上,那把银刀飞得老远,``铮''的一声落在绵软的地毯上,嚣张地滴落暗红色的鲜血。皇帝眉头也不皱一下,只死死盯着那血迹的出处,怔然落下泪来。

香见吹弹可破的侧脸上,一道小指长的伤口横过鬂边。那把银刀虽小,锋刃却薄,虽然只是轻轻刮过,但香见脸上已划出一道深深血痕,翻出皮肉的色泽。皇帝又是心疼又是焦急,生怕她又伤着自己,紧紧将她圈入臂弯牢牢箍住,不许挣扎,一壁低声喝道:``李玉,凌云彻,进来!''

李玉慌忙入内,一见此景,吓得腿也软了,情不自禁跪在了地上,呜咽着哭起来。

凌云彻暗暗踢了他一脚,皱着眉将地上的银刀捡起,用布帛裹住收入怀中。皇帝不耐烦道:``叫你进来就是看你哭么?''

李玉抽噎着道:``皇上恕罪,奴才看见香见小主受伤,就好像什么稀罕爱物儿受损,心里难过得什么似的!''

皇帝横他一眼,正要说话,骤见香见脸颊犹有新鲜血液汩汩渗出。他面色煞白,正要仔细察看,凌云彻眼疾手快,立刻抢到跟前扯过香见手边的绢子将皇帝的手腕紧紧裹住。他的脸色变得极难看,低低道:``皇上的左手也伤着了,可要请太医来?''

李玉一听皇帝受伤,吓得魂飞魄散,立刻膝行上前,翻开绢子一看,皇帝手腕外侧的伤几可见骨,幸好只是伤在外侧,否则动了筋脉,只怕要生出弥天大祸。香见本自挣扎,但见皇帝伤口即便有绢子扎住,仍不断渗出血液,可见伤口之深,她亦不敢随意动弹。

凌云彻使个眼色,李玉忙上前扶了香见往榻边坐下,这边厢凌云彻已牢牢扶住了皇帝,悄声道:``皇上和小主的伤势,都是非请太医不可的。只是这件事干系重大,微臣必得请皇上示下。''

皇帝犹豫良久,显是不欲让人知道此事端底,然而见香见面上渗出细红血滴,心头阵阵绞痛,浑然不觉自己伤口之痛。

香见神色痴惘,恍恍惚饱地垂下泪来,哽咽道:``对不住!是我自己不想活了,并不是有心要伤着你!''

皇帝何曾听过她如此低言软语,只觉得魂销骨酥,游荡天外,心下更是垂怜不已。半晌,他只得咬了咬牙,低声嘱咐,``李玉,去请齐鲁来。记得,切莫声张!''

李玉连滚带爬去了。凌云彻取过地上撕裂的布帛,将就着将地上血迹擦干净,垂手恭声道:``皇上,微臣什么也不曾看见,什么也不曾听见。''

皇帝长嘘一口气,用不曾受伤的左手拍了拍他的肩膀,含着痛楚的笑意微微颔首。

待到齐鲁来时,又是一通忙乱。皇帝见了齐鲁,顾不得自己伤口尚在滴血,执意让他先去看香见。

李玉急得砰砰磕头,``小主的血已经自己止住,可见还是皇上伤得厉害。您若不让瞧,小主心里也不安哪!''皇帝的伤势不浅,寻医问药虽难,更难的是太医院取药煎熬都得经过人手,还得用金疮药,实在难以隐瞒,不禁急得老汗纵横。还是凌云彻警觉,取出银刀在手腕划了一道,又示意齐鲁取过纱帛将自己手腕缠上,道:``一切有劳齐太医。''

齐鲁顿时松了口气,又去瞧香见。他细细瞧了伤口,便摇头道:``小主的伤在脸上,要愈合不难,可要不留症痕,请恕微臣实在无能。''

香见斜靠在榻上,怔怔望着九色描绘的洒金嵌朱彩顶,惘然落泪,``我连这条命都不想要了,还要保全这容颜作甚,毁便毁了!''

皇帝满腹心疼气恼发作不得,重重挥落手边一个青花瓷盏,溅开无数雪片似的碎瓷。李玉慌得抖衣乱颤,哭丧着脸道:``皇上,事情已经这样了,求您的动静别太大!这不还有太后娘娘呢么,如果她老人家知道了,指不定小主得多可怜呢。''

皇帝闻言一怔,只得敛气道:``罢了!今晚的事不许外传,否则朕摘了你们的脑袋!''

齐鲁畏惧不已,却又不敢不禀告,连声音都发颤了,``皇上,微臣实在是没有办法。好在小主的伤口浅,又伤在鬓边。若是鬓发梳得好,可以掩盖。再不然,涂脂抹粉之后也不大看得出。微臣也一定尽力,找到最好的药材为小主消去伤痕。''

凌云彻忍着痛在旁道:``皇上,此事若有人问起,只能说小主自己不慎,划伤了脸颊。而皇上的手这几日怕也不能轻动,必得养好伤势才行。''

李玉苦恼不已,``皇上只记挂着小主,可不想您的手上也是要留疤的,万一被谁看见传出去,这可怎么好?便是皇上不摘奴才的脑袋,奴才的脑袋也铁定保不住了!''

皇帝气怒不堪,闻言更是心烦,狠狠照着他肩膀踹了一脚道:``你少多嘴!朕自有分寸!大不了朕再不宣那些饶舌婆子侍寝便是!''

李玉抱着肩膀,痛得不敢哼哼,只得涕泪满面,缩着身子连连点头。

如懿得知消息时,已是夜来时分。并非李玉与凌云彻多嘴,而是皇帝手腕的伤势,实是吃重,皇帝又不欲惊动他人,不得已之下,只得唤来如懿。

彼时如懿正在窗下陪着永瑾习字。小小的孩子,握笔甚是用力。他写完一幅字,交与如懿手中,极认真地问:``额娘,我写的字好么?''

如懿看得仔细,笑着抚他额头,``比上回写得好。皇阿玛指点你了,是么?''

永瑾稚声稚气道:``不是啊。从前都是皇阿玛教我习字,皇阿玛许久不得空了,便是五哥教我。''

如懿骤然想起,皇帝为了香见顾不上六宫中人,哪里又得空过问皇子们的功课呢。她默然片刻,微笑道:``不错,你五哥的字极好,有他教你,自然不错。''

永瑾一笑,甚是高兴。话虽这样说,如懿却是知道的,比之永琪小时的聪颖,永瑾已是不如。等到开蒙读书,无论习文写字,都是比永琪当年差了一截。才知天赋等事,真是比不来。可是,那有什么要紧,永瑾终究是她最可爱的孩子。

母子俩相伴言笑,窗台上羊脂玉瓶内供着数脉枫叶,色泽完美而艳丽,将空气中浅霜般微凉的天气点得暖意融融。

是李玉的骤然而至惊破这一室的宁谧,如懿乍然闻得,只觉得一阵阵透骨寒意沁入背心,指尖腻得发滑,支撑不住似的。她极力扶着紫檀螵钿小桌的一角,撑着身体,压低了嗓音问:``太后知道了么?''

李玉慌忙摇头,旋即气馁,``皇后娘娘,这件事怕不好隐瞒,您先去瞧瞧再说吧。''

如懿扶了李玉的手,只带了容珮便匆匆赶去。她从未这样慌乱过,哪怕是那年受冤即将被掷入冷宫,她也知道,如果有皇帝的一隙信任,有自己的一念求生,便不会沦落于万劫死地。可是这些日子,她当真是恍惚了。所有的一切因为香见的到来全然打破,进入光怪陆离之境。每一天会发生什么事,她完全不能预计,亦不能掌控。因为是他,那个立于世间权势之巅的男子,神魂颠倒,不知所以。

到头来,果真是他先出了事端。

如懿这样想着,足下一阵阵酸软,仿佛是双脚落在了棉花上,半点也不得力,若非李玉与容珮大力扶着,她都不知道自己是如何走到养心殿来的。直至进了暖阁,看见皇帝手腕上犹有鲜血斑斑渗出,只觉骨上长出根根利刺,由内向外剌入肌肉,顶到肤层,剌得她不知该如何抵御。

幸好,她内心的担忧与惶惑并未让她在见到皇帝的那一刻泪如雨下失声痛哭。她犹存几分镇定,屈膝问安,与往常无异。

皇帝见她不哭,想要说什么,嘴唇微微一张,却含了几分愧怍。他唤她,``如懿。''

或许这一刻,一个呼唤了数十年的名字,会比一个名位更叫人安心。

皇帝面色萎黄,形容委顿,素日那种轻云出岫的倜傥之姿与无所不能的唯我独尊之气全数消弭。她看着他,不知怎的生出了一股怜悯,和着积郁多日的怨与怒,一并涌了出来。怔了片刻,她静静道:``臣妾赶来养心殿前往承乾宫看了一眼,寒氏无恙。''

皇帝登时松了口气,脸色复了少许红润,``朕让李玉去传你,也更无放心之人可以去探承乾宫的消息。''他唏嘘,有急不可待的关切,``香见如何?''

如懿极力克制着满心里横冲直撞的怨意,``身体已然无恙,只是脸上的伤,定是要留下疮痕了。''

皇帝喜出望外,``真的?只要身体无恙就好。容颜之事,并不要紧。''

有无限的酸楚,却不知从何说起,原来他待香见,是这般情深。任她与他相随多年,这样情深,她亦从未见过。

真的,她一直觉得皇帝待自己甚好,便是彼此疑心之后,平日细节照拂,他亦无一不悉心。自然,这样的好并不是只对着她一人。宫中上下,无一不得,便是连不甚承宠的海兰与婉茵,也不少得他嘘寒问暖。所以论``雨露均沾''四字,皇帝是当之无愧的。

正因着如此,便也不知情深几许是如何样子。总看着戏台上水袖飞扬,听着唱词婉转,因着从未在身边见过,便总以为不过是人世的绮想,天上落入人间的传说。唯见他这般喜爱女子颜色之人,真心关切,甚至不惜她容颜是否毁损。她才觉得孤凉。

真是孤凉。原来这一生,一路颠沛走来,得到后位,得到荣光。真正的情爱,她却是生生在他与旁人身上才得见。而自己,不过是枉自欺骗了自己,哄着自己,以为年少渴盼的真心相许,已然得到,却是镜花水月,明明成空,仍懵然不知。

她终于忍耐不住情绪的奔突,走近他身侧坐下,抚着他受伤的手腕,轻声细语,``皇上不是从来没有受过伤,可是这是唯一一次,因为一个女人而受伤。皇上,不知这可算是一个满洲勇士的荣光?''

皇帝讪讪,情不自禁地抚过伤处,``你不要担忧,皮肉伤而已。有齐鲁在,朕没事。''

``皇上没事?皇上乃天子之尊,不可任情妄为。何况您一举一动关系天下臣民。臣妾虽不知皇上与寒氏发生何事才会同时受伤,但皇上可知,臣妾方才虽只在寝殿外看了寒氏一眼,但她的生无可恋之心,便是臣妾这个外人也看得明白。''

皇帝避开她的目光,默然片刻,哑声道:``香见倔强,一时不能转園。今日她亦是失手,才会划伤自己,也误伤了朕。好了,你放心,过了这一阵,伤势痊愈,此事便过去了。''

如懿口舌涩然,``既然皇上无恙,那为何还要唤来臣妾?''

皇帝亦有几分着恼,苍白面色上隐隐有铁青,``你是朕的皇后,合该为朕分忧。朕亦不想有人发觉朕的伤势,再起风波。''

如懿听出他语气中的不满,看着他手腕殷红的血珠犹自从层层白布下洇出,亦是心软,``那皇上打算如何隐瞒此事?若被太后与王公知晓,只怕会掀起轩然大波,除了严惩寒氏,更会让臣民指贵皇上因宠失度,损害皇上的威严。''

皇帝气色稍和,握住她的手,``如懿,你懂得分寸。不愧是朕亲自选的皇后。''他眸中隐有忧意,``如懿,若此事传开,知道朕的手是为香见所伤,平地起谣言,逼迫香见离宫。朕也觉得麻烦不堪。''

``是啊。赔上了纯惠皇贵妃和永璋的性命,宫里才无人敢再提此事。太后对此颇为不满,虽然臣妾再三言说是纯惠皇贵妃侍奉不周又宠溺永璋,永璋亦有失言之错,才受了皇上斥责。可终究事情如何,皇上与臣妾心知肚明。''

他听出如懿的不满,语气便有几分软弱,``如懿,绿筠与永璋之死,朕也难过,所以他们母子一个追封为纯惠皇贵妃,一个追封为循郡王。''

是利刃在心上沙沙地刮着,刮去薄薄的皮肉,沁出细密的鲜血。她已觉不出刀刃的锋利,只是痛,密密麻麻,无处不在。她的声线茫然而软弱,``追封也不过是死后哀荣。皇上在意的,终究只是为了寒氏!只是皇上的真心,寒氏并不肯接受,才逼出今日的险事。何况寒氏容颜已毁,皇上还是这般执着么?''

皇帝坐在暖阁榻上,殿中红烛灼艳,勾勒出他微微佝偻的背影。如懿的鼻尖微微发酸,他一直是意气风发之人,想要的都能得到,从未有任何挫磨将他推于如此软弱之境。``如懿,你想问的,朕也思量过。身为帝王,万人之上,是不可以动心的。因为心一动,便万劫生。所以朕一直理智,哪怕是明知舒妃对朕情深万千,联也只能懂得,只能怜惜。如此而已。''

她明知是不能问的。皇帝的话已经到了明处,再问,亦不过是自取其辱。可是她还是忍不住,忍不住,只为自己身为女子,只为曾经那样热烈地与他相知相许,``那么臣妾呢?''

皇帝深深地望着她,闪过一丝愧色,歉疚地道:``如懿,朕待你好,你懂得朕,咱们彼此相知相惜。若论情爱,朕自然是喜欢你的,否则你又怎能成为朕的皇后?''

``喜欢?''惊痛之绪如沸油烈煎,滴滴逼熬,``皇上,您自然是喜欢臣妾的,只是喜欢得不够。或者,这`喜欢'二字,于您而言,是不太重要的。就如愤怒、忧郁、欢喜一般,只是一种情绪而已。''如懿牢牢地盯着皇帝,她挪不开自己的视线,也停不下自己的口舌,仿佛这样,便能逼迫那个不想听到的答案出现在耳边,``而且这喜欢,怕是对谁都一样的吧?对孝贤皇后是,慧贤皇贵妃是,舒妃是,令妃是,析妃也是。那么臣妾只是空占了个名位,与她们有何不同?也是,臣妾本来也不过是妃妾出身,忝居后位。真正能让皇上情深意动,不顾一切的,唯有寒香见一个!''

皇帝的沉默是无言的承认,叫她心生焦躁。那焦躁是野火,烧得尽春风劲草,也烧得尽她极力维持的理智。``皇上这般神魂颠倒,罔顾一切。恕臣妾不敢放肆,却不得不放肆!臣妾身为皇后,不能眼看着皇上罔顾身后名望,逼迫一区区女子,且是一个愿意为有婚约之人守贞的女子。''

皇帝的眉高高挑起,满蓄了轻蔑之意,``守贞?我满族男子,不以礼教为念。''

如懿如何肯退让,``皇上难道是想效法顺治爷娶弟媳董鄂氏为妃?且不说顺治爷与董鄂妃两情相悦,可百年之后论起顺治爷生平,便是连后人也不能不以此为憾事!何况顺治爷为娶董鄂妃,上逆母后之意,下伤后妃祥和,惹得怨声载道,六宫生变。皇上难道能不引以为鉴?''

皇帝冷笑一声,``男子钟情也是错么?皇后竟也如无知妇人,说出这般醋妒昏话!''

到底是哪一个字,挑痛了他最后那根不能触碰的神经。如懿定定地望着皇帝,不能动弹,唯有以激烈的言语宣泄此刻难以言喻的难过。``钟情一人固然无错。若今日皇上下旨,为迎寒香见入宫,废了六宫嫔御,只专心对着她一人一生一世。臣妾便无话可说,立刻铰了头发,青灯古佛了此残生。''她满目痛惜,``我大清开国以来,不乏钟情专一的男子。太宗皇太极钟爱宸妃,因宸妃早逝以致痛心而死;顺治爷独宠董鄂妃,生出无数事端。是!钟情一人固然不错,臣妾身为女子,毕生所愿也不过如此。但要为一人之情而伤无数人的心怀,又是何必!''她极力缓和了口气,``皇上向来提倡儒家礼学,每每经过山东,都要祭拜孔子,又教导皇子们都要研习儒家经学。怎么到了今日,却为一己狂热,将这些都抛诸脑后,惹得天下文人士子都寒了心么?''

皇帝张口结舌,气得发怔。半晌,他才缓缓伸出手,抓住如懿的手臂,``如懿,朕这一生都没有纵情任性过,你就当朕任性,就这么任性一回,没有礼教,没有规矩,让朕一心一意喜爱一个女子,可不可以?''

\hypertarget{ux7b2cux4e94ux7ae0-ux7ea2ux989cux54c0ux4e0b}{%
\chapter{第五章
红颜哀(下)}\label{ux7b2cux4e94ux7ae0-ux7ea2ux989cux54c0ux4e0b}}

如懿惊得倒退一步,几乎要跌坐于地,幸好被容珮扶住了。如懿立时变色,喝道:``出去!''容珮吓得急忙转身,如懿厉声道,``方才本宫与皇上说了什么,你都没有听见。出了这个门,你没长嘴,也没有耳朵,一个字都不许漏出去!''

她见周围打发得干净,终于禁不住软弱了下来,``皇上说出这样的话来,是要锥臣妾的心么?方才那些话臣妾不许人知道,是怕落下话柄叫人讥刺皇上!''

皇帝大约也是气昏了头,恼道:``有什么可讥刺的?朕只是真心喜爱一个女子而已。''

如懿戚然相对,``既是真心,自该叫人欢喜,何来勉强与难过,逼得寒氏一心求死!''

皇帝微微语塞,旋即道:``朕在准备一份礼物,只要假以时日完成,朕一定会让香见回心转意,侍奉朕身侧!''

如懿睁大了眼眸,眼底的伤心渐渐蔓延出一丝鄙夷的意味,``是么?但皇上大可扪心自问,是真心爱怜寒氏,还是为了一己私欲与好胜之心?''

他喃喃:``在今日之前,连朕自己也一直以为喜欢的是香见的容貌。直到她自毁容颜,朕才明白,朕喜欢的,是她坚持自己的倔强,是她对寒歧的坚贞。这些,都是朕没有的。''

她的嗓子一阵阵发涩,仿佛难以启齿,却依旧忍不住问:``就因为皇上自己没有,所以一定要从寒氏身上得到?''

皇帝低着头,斜倚着身体,似乎无奈疲倦到了极处,可他的眼底仍有渴求闪烁,``如懿,朕从来就没有得不到的人,得不到的事,香见是唯一一个。你别叫朕留下遗憾,好不好?如懿,香见她不想活了,可朕不能失去她,真的。如懿,让她活下来,让她愿意活下来,在朕身边,好不好?''

她答允不了,嗓子眼张不开,嘴唇紧紧地抿着。她不过是一个女人,一个有着夫君的女人。可偏偏,自己的夫君却这般来要求自己。

如懿苦笑不已,``皇上对臣妾说出这样的要求,是浑然不觉得臣妾是你的妻子,你的女人,而只是一个皇后的身份么?''

皇帝诧然片刻,旋即释然,``如懿,你既是皇后,就该承担中宫的职贵,而非一意儿女情长。''

``皇上要臣妾做的事,臣妾真的觉得很难。臣妾自登后位,才渐渐觉出当年孝贤皇后的难处。若是一个对夫君全无眷慕之心的女子,如何能让皇上放心处理六宫之事?但若对夫君有眷慕之情,又该如何违背自己的心意放下儿女情长来不偏不倚地处置?皇上虽将臣妾捧于皇后之地,却也不啻将臣妾置于两难之地。''

``两难么?''皇帝的目光虚浮在远处,``如懿,若是孝贤皇后还在,她会做到的。她是一个贤德的皇后,她会恪尽皇后的本分,来为朕处置妥当。''

仿佛数九寒月有冰水夹杂着无数尖锐的冰凌兜头而下,连血液都冻住了,却还能辨出那种面对疼痛却无可抵御的软弱。如懿打了个寒噤,仿佛看着一个不认识的人,渐渐浮出一个虚茫的笑靥。从前他对孝贤皇后的种种不合心意,终于因了她身后误会的解开,因多年的追忆,因了自己与他的种种磨砺,化为了时光里不肯老去的温柔,化为了自己在他心中的不合心意。

她神色凄楚,面带冷冽,``皇上这样重的话,臣妾承受不起。''

皇帝将手落在她手背上,似乎要将她的不甘与抗拒压下,``既然承受不起,便好好去做。别辜负了朕对你的用心。''

如懿抬首,遇上他凜冽的目光,心思却被他搭着自己的手腕的力度所吸引。那是他受伤的手,无意拂落于她手上,却并无往日的亲密,更是一种无言的压制。可是,她却未能感觉到他的手带来的力度。

他受伤的左手,浑然使不上力气。

悲切之意油然而生。有泪,凄然坠落,洇入沾着他鲜血的白纱。

她终于妥协,``皇上所托,臣妾不敢辜负。可以尽力劝服寒氏萌发求生之意,但不能令她一定肯在皇帝身边。''她凝视皇帝的伤口,``皇上伤在手腕,可暂以衣袖遮掩。这几日请皇上勿见嫔妃,也勿召人侍寝,以免有更多人知道皇上的伤势。''

皇帝喟然,稍有欣慰,``朕也这样想,只是苦无理由。''

如懿凝神片刻,``有。战事大局已定,但死伤将士无数。皇上要斋戒数日,以慰亡魂。''

皇帝旋即会意,``战事有伤天和,朕会举行法事,更会独居养心殿斋戒。''他一顿,``君者为人伦之极,五伦无不系于君。臣奉君,子遵父,妻从夫,不可倒置也。皇后深明事理,婉順谦恭,朕很欣慰。那么香见之事,朕也一并交予你了。''

如懿以从未有过的郑重容色凜然相对,``皇上所托,臣妾身为皇后,不敢不允。但臣妾所允,只以皇后身份,而非皇上妻室。从今以后,皇上所言所托,臣妾都不敢失皇后分寸,却也仅以皇后分寸而已。但请皇上明白。''

皇帝憔悴的面孔上满是愕然与震惊,``如懿,你说什么?''

她的眼底蓄满了泪水,那种滚烫的热度,仿佛要烫得她看不清眼前的一切。如若可以,她真的愿意自己是盲的,看不清所有蒙昧的温情挑破后残忍而冷酷的真相,可是她秉持了最后的礼仪与气度,``臣妾蒙皇上厚爱,忝居后位。所能做的,也仅是皇后应该做的。''

她俯身三拜,以极其尊崇的态度,谦卑己身,缓缓退离。

如懿见到香见,已经是两日后的事情。

不是未曾想过该以何种姿态面对寒香见的一心求死,而是太多的混乱与冲击,在那一日养心殿对谈之后,将她极力维持的理智冲打得近如齑粉。

她全然是以麻木的状态将皇帝所希望见到的一切一一布置下去。幸好中宫的威仪尚在,而之前皇帝极力弥补的密切与热络让后宫诸人不敢对她的言行有分毫质疑。

如懿看着这一切缓缓进行,只是不能克制地想要冷笑。何谓狐假虎威,便是如此。她便是那一只倚仗老虎威势的狐狸,以为自己得到想要得到的所有,亦不过是凭借好风飞上青天的风筝,唯有游丝一线。一旦风去,便只余重重坠落粉身碎骨的命运。

可时日稍久,便会有另一种意味。她所从未察觉过的意味渐渐萌生。如果,没有一丝属于自己的情愫,而是克尽己责地做好一个皇后应有的职贵,那也不算是一件太难的事。甚至,会因为只需恪守已然成熟的条条框框,便能不功不过,安然度日,也算一个不错的皇后。

香见受伤之事并非不能外传,所以很快让嫔妃们更添了好奇与幸灾乐祸的心情,更是茶余饭后最好的谈资。而皇帝不再踏足承乾宫,仿佛对她容颜毁损而失望至极,亦让嫔妃们多了一丝希望与愉悦的寄托,盼望着皇帝将她弃如敝屣,再不理会。

但凡一个寻常人,都会这般想。

因为对于一个男子而言,秉窈窕之姿,具冰雪之貌,是最大的吸引,而一个失去了美貌的女子,便是连一个寻常妇人都不如了。

所以无人不这般揣测,这场疯狂的迷恋,最后了结于寒香见与皇帝争执时的失手自毁。

每每传来消息的是进保,皇帝身边这个素来不苟言笑面目死板的中年太监。

这些并不算是好消息,亦是意料之中的消息。

香见绝食。

这是很自然的事。如果毁去自己的美貌并不能断绝一个人的狂热,那么断绝生命,是最后的,也是最无奈的举措。

如果让香见死去,那会满足很多人的愿望,让人大大松一口气。

可她若真死去\ldots 如懿忽然想起了皇帝按住自己的那只手,那只受伤的左手,勉力压着自己的手,却偏偏使不上力气。如懿鼻尖一酸,她从未觉得这个男人如此软弱而让她心生怜悯。而在昼夜扰乱她心绪的震动与伤心之后,怜悯居然成了占据她心房最多的情绪。

而且,让皇帝愉悦,不正是一个皇后应当的职责么?

如懿自嘲地笑笑,拣过一袭杏子黄盘金彩绣翔凤穿芍药团花紫绫袍,脚上凤纹朱锦罗鞋,簪上九转连珠赤金琉璃飞鸾步摇,烂漫明丽的翠华钿并朱红宝树珊瑚花饰点缀。

华光明艳的色泽撞得眼帘微微生疼,才知绫罗衣衫是勇气,贴肉予以温度,撑住她灰败的内心,予以表面的光鲜。日复一日,行走下去。

着实,也比朝夕相对数十年的男子可靠。

如懿扶着容珮的手踏入承乾宫寝殿时,已然微微倒吸了一口冷气。皇帝性喜奢丽,自孝贤皇后丧期满三年后,除了长春宫一应如旧,其余殿阁连着太后的慈宁宫一应装饰一新,绮靡繁丽。而承乾宫长久无人居住,乃香见入宫后草草打扫出来,其规制陈设,华丽更胜于她的中宫。连最爱繁华的金玉妍在世,也不得不居于下风。随便一个眼风扫去,搁着的藏青花玉凤莲转心瓶乃宋徽宗所珍藏,一对龙香握鱼是汉成帝皇后赵飞燕所有。殿角随意搁着的一丛三尺高的珊瑚树,通体莹红润泽,鲜妍欲滴,隐隐有宝光流溢。妆台上一大捧盒东海进贡的珍珠,颗颗浑圆如拇指大小,饱满明净,就那般开了盒子随手摒着,也无人在意。林林色色,错落有致,光华迷离,纵使她贵为皇后,有些也不曾见过。

而平静卧于斑彩鸳鸯万金锦上的香见,却与这金摇玉耀的华丽人间格格不入。她是一捧春雪,冰凉如霜,却美得短暂,瞬间就能化去一般。

彼时午后轻暖的秋阳透进豆绿罗影纱,照得寝殿内微尘轻扬,碎金似的迷漫。因着如懿的到来,宫人们都退了下去。殿中梨花木矮架上供着一盆香山子,香气幽幽若若,又不见烟火气,甘宁清甜的香气让人通体舒泰,宛在梦中。那香山子原是取百斤左右的紫油伽蓝香精心镂雕而成。那伽蓝香难得,宫人们取一星两星制成金累丝香包已算得趣,何况是这样大件。如懿未曾细想,只一意凝睇。

她从没有见过这样的女子,即使在濒死的一刻,还能美得如此不沾风尘,宛若谪仙。

有一个大不敬的念头从脑海中疾闪而过。虽然岁月对皇帝格外厚爱,使他仍有英姿枫枫、玉山嫌峨之态,但比之香见,亦不过是紫芝之畔的青苔和油腻的朽木,不堪佳配。

她有一瞬的好奇,那个让香见心心念念的男人,会是个怎样的人?

这样的念头,挑破彼此视线并无交集的尴尬。

她侧身,顺着容琢搬来的桃花木竹节番草纹绣墩坐下,示意众人退下,方才缓缓开口:``听闻一个人濒死的时候,可以看见他最想见的人,你是否在等这一刻?''

香见神色呆滞,死死地盯着蓝田玉轻羽尾帐钩挽起梨花青冰绡缠枝宝罗帐顶。宫人们强行替她换过了天水绿白点梅枝纱衫,也是她部族的制式,长长的雪色长珠缕络逶逸横逸,如她一般毫无生气。

如懿不在意她的沉默,只是出神,``其实本宫也很好奇,寒歧到底是怎样人物。你若不与本宫说说,怕是知道他记得他的人也会越来越少了。''

香见的眼珠是定在白水银里的两丸琥珀,清透却僵死,没有一丝活气,唯有在听到寒歧的名字时稍稍一颤,旋即又复死寂。她喃喃,那低语声沙哑近乎干裂,是两日未曾进水的缘故,``寒歧?很久没人和我提他了。''

``你身边的侍女固然是你的族人,却也不愿意提这个为你们部族引来战火的男子了吧。''如懿仰着头,拨着罗帐上垂落的南红坠崧蓝流苏,那南红红艳如锦,质地糯润,捏在手里华润而沉静。``可是,本宫真的很好奇,他为何会让你念念不忘?说来好笑,本宫自出闺阁,见过的男子也不过这么几个,每日起坐便是太监服侍。本宫真的很难想象,你们曾经经历过什么,可以有这般似海深情?''

香见吃力地扬起唇角,露出一丝讥诮,嘶哑着道:``你和那个皇帝,都不会懂的。''她欲再说,便咳嗽起来,可见言语艰难。如懿见她入瓮,暗觉她单纯执拗,便取过桌上容珮留下的汤盏,徐徐引至她唇边,``是么?本宫是不懂,因为外头传言,他杀人如麻?''

香见亦不在意那盏中汤汁是什么,起初还呛了两口,渐渐饮下一二,急着辩解道:``不是!不是的!''她眼里流下一滴泪来,``他只是太想做一个英雄,太想可以脱离别人的控制和束缚,随心所欲。他\ldots 真的不是一个坏人。''

``不自量力、以卵击石这些词已经用得太多。寒歧只是想得到,却忘记了可能会付出的代价。本宫真的很担心,若是你死了,这世间记得他的好的人,便再也没有了。''

``没有了?''她的泪晶莹一滴,洇入盘螭朝阳葵纹枕。那攒金线秋阳葵花的图案明艳如生,益发显出她不堪的绝望,``是啊。我喜欢寒歧的时候才十三岁,那时他十六岁。他的眼睛那么明亮,天上的星星都比不上他。我在野外被狼群追逐,是他赶来救我,和狼群搏斗。他带着我骑马,放牧,带我去看冰山上的雪莲花。他说雪莲花是不能摘的,因为在他心里,雪莲花和我一样美丽。他知道我喜欢沙枣花的香气,便在我的屋子外种满了沙枣树。他答应我,只要我们的部族可以挣脱大清的束缚,他就可以带着荣光迎娶我。''

如懿轻轻唏嘘,``结果,世事于你,于他,都不过是一场幻想。''

``是。他的骄傲,烧死了自己,也烧毁了整个部族的安宁。那场仗打了几天几夜,我和部族里的女人、孩子们都躲了起来,直到廝杀声全部消失。我在夜色里寻找他,直到天明才在成堆的尸体下找到他。他浑身都是血,失去了一条臂膀,身上全是刀伤。他再也不会对我笑,对我说话,带我去摘雪莲花了。''

如懿替她抹去唇边流下的汤汁,徐徐道:``一个人过于渴望强大,只是因为他的渺小,寒歧有千错万错,对你总算不错。本宫不想多去议论一个已死之人的是非,只是要你明白,寒部已经失去了一个寒歧,不能再失去一个你。''

香见的眼是漫天星子坠落后的沉寂永夜,``我不过是一个礼物,已经在这里留了这些日子,也总有毁损的时候。我死在这个污秽地方,也是尽了我这个礼物的本分?''

``你方才喝的是红参汤,不是白水,一时死不了。既死不了,便好好听本宫说几句话再死。''如懿拨着凤仙花染就的半透明的指甲,这些日子她本无心妆饰,连指甲上的浅红残褪了也未曾发觉。她神色恬淡,一意浅谈,``你的寒歧死在了大清的将领手中,你的部族险险灭于铁蹄之下。可是你想想,为什么你的父亲还要把你这个将死之人送到京城来,而且你的族人也欣然同意?因为他们都知道,你是一个希望,是让你的族人好好活下去的希望。''

``希望?''香见满脸是泪,悲绝摆首,``不。从我的部族被刀刃血洗的时候,从寒歧的身体在我怀里变得冰冷的时候,我就没有希望了?我怎么还能去做一个别人的希望!''

如懿凝视着她,平静而从容,``当然。你也可以不做这个希望。拿刀抹脖子,挂上长巾把自己悬到梁上,服毒或者拿你漂亮的头撞到墙上去,一了百了的法子多了,随你选一个。但是你死了,哪天皇上听了谁的劝要再灭了你的部族,要对你的族人斩草除根,还有谁会来劝一句,保全下他们的性命和家园?''

香见震惊而愤怒,无以复加,``皇上\ldots 你们的大军\ldots 都是魔鬼,都是魔鬼!神灵会惩罚你们的!''

``成王败寇,连神灵也不外如是。否则孙悟空怎会被如来压在五指山下?如果今日是你们寒部灭了大清,我们也一定呼号不已,喊着你们是魔鬼!''她伸出手,示意香见坐起身,``我们都是女人,管不了男人的野心,也管不了男人的天下。我们能管着的,是凭一个女人的本事,将她想守护的人和事,都一点不漏地守下来,''

香见的面孔上挂满了莹然泪水。若不是亲眼所见,如懿几乎不能相信,这个世上居然有人连哭泣,甚至以带着疤痕的容颜哭泣,也可以这般宛若凌波仙子。她终于有一点明白,她的丈夫人到中年,还有那股像秋水一样发了狂满涨的热情的原因。

香见的手搭在如懿的手上,吃力地斜签起身子,悲伤哭泣:``万千勇士都守不住我们的家园,凭我,能守住什么呢?''

如懿深吸一口气,望着外头秋高气爽的碧蓝广天,沉声道:``男人们守不住的东西,往往女人就能做到。因为一个女人的韧性和忍耐,是任何人都不能比拟的。人人都说越王勾践卧薪尝胆,忍辱负重,本宫倒觉得越王夫人才是真正的英雄豪杰。越国战败于吴国,勾践所受的苦不过是他应当承受的那份。越王夫人身处深宫,也被丈夫牵连受辱,还要安慰失意的丈夫忍耐奋发,她的毅力与韧劲才是最值得钦佩的。''

香见睁着满是泪水的眼,``可是我不是越王夫人,我\ldots{}''

如懿的目光无比锐利,逼视着她,``你方才说过,你不过是一件礼物。一个人能了解自己的处境,明白自己的身份,倒也不是坏事。本宫就问你,既被作为礼物送来,你可愿尽一个礼物所有的责任和义务,好好地安分守己做好你的礼物?''

香见美丽的大眼睛里布满了迷惘与不解。

如懿春山微蹙,耐着性子娓娓道来,``如果于你而言,死去的情人比活着的族人要紧,那么本宫也不必再费事和你多说什么。可是你要觉得逝者不可追,活着的那些人更值得你牵挂,就像你父亲把你送来的本意一样,好好地做一个礼物。美丽、夺目,并且让送你来的人得到益处。这就是一件礼物的本分。''

香见唇色干枯,眼底的血丝如罗布的蛛网,却拢不住她的悲愤,``难道我就不能有其他的选择?像普通人一样做自己的选择?''

如懿俯下身,看着美丽而哀伤的容颜,似一朵开在冰凌上的无瑕而剔透的雪花。可是即便天寒地冻,雪花亦不会留存长久,只能被冻得僵冷,萎谢于地。香见的美似乎传递着她无法言语的悲楚,让看到的人也心生悲凉。如懿挽着她的手起身,``本宫和你一样,最大的悲哀就是没有选择。所以这个宫里,上至皇后,下至宫女,每个人活着,挣扎着,都是为了可以多一点选择。就譬如你,有了恩宠,有了凭仗,就可以选择为不为你的族人说话,选择说出怎样有用的话。如果你没有恩宠,那就是没有任何选择。''

香见嘤嘤含泣,``那你,你是皇后,你有没有过自己可以选择的事?''

``皇后只是一个身份,甚至是一个比你束缚更多的身份。所以本宫从来无从选择,只是逼迫自己顺天应时,如此而已。''如懿起身,将方才喝剩的半盏参汤置于她身前,红澄澄的汤汁倒映着她绝美的容颜,``你要知道,盼着你死的人很多,但都是你的敌人和无关紧要的人。希望你活着的人也不少,那都是你的至亲你的族人。选择成全哪一边,都由你。''

她转身离去,不欲多停留。仿佛香见的哀绝,亦是她的无奈。

万千人之上的皇后与一个战败送来的礼物,原也没什么不同。她忽然想起豫妃将要入宫那一日,皇帝的笑语,``不过是摆设而已''。

当日笑语,如今忆起只觉得惊心动魄。

如懿扶着容珮的手走了老远,神色依旧怔忡不宁,半晌,低语道:``容珮,你有没有觉得,我们都很像一件摆设?''

容珮惶惑地看了身后跟随的十数宫人,不解道:``摆设?''

``是啊。恂嫔是霍硕特部的摆设,豫妃是博尔济吉特氏的摆设,舒妃是叶赫那拉氏的摆设,淑嘉皇贵妃是李朝王室的摆设。她们每个人摆在宫里,都是家族的象征,族人的荣光。皇子和公主们,是子嗣繁衍、皇室兴旺的摆设。太后呢,是母慈子孝的需要,是向世人展示皇家恩义的摆设。除了面上那层需要,里头的滋味儿,如人饮水,冷暖自知。''

容珮听得满心怅惘,忙堆了笑劝道:``娘娘,您想太多了。外头寒凉,咱们回宫吧。''

如懿抬起头,眯着眼看着晴好日光,像是洒落满天金色的碎屑,叫人觉得温暖。她其实羡慕的,是连尘埃这样无根轻飘的事物,来一阵风,想去哪儿,就能去哪儿。可这一辈子,她的身,她的心,都是注定要禁锢在这紫禁城里了。怎么飘也飘不出这高墙去。不,她哪里有飘的资格!

依稀是小时候跟着乳母嬷嬷们去寺庙里参拜。高大庄严的佛像,被装饰得宝光金灿,叫人不敢逼视。仿佛他们生来,就是这样高高在上,受万人景仰膜拜,受世间万千香火供奉。没有喜怒哀乐,从来没有,他们所有的职责,便是在那个位子上,只消在那里就好。

如懿耸了耸肩,像是禁不住秋日里的几许寒意似的。眼前便是秋意如醉,可是那浓醉的枫红菊灿,与她也是不相干的。如懿像是被隔绝在了自己的世界里,任凭外头秋色正浓,她兀自冷露寒霜,残叶萧萧。

容珮有些不安心,又唤了一句:``娘娘\ldots{}''

如懿微微笑出声来,``你觉不觉得,本宫就像是庙里的塑像,宫里头的摆设?''

容珮知她经历了这些事,难免颓丧,只得好言劝道:``娘娘\ldots 您别多想了。''

``是了。摆设是连自己的念想都没有的。没有思想,才能安于做一个摆设啊!''她浮起一个虚弱的笑,``如果寒氏听了本宫的劝,本宫就是完成了皇上的嘱托,尽到了皇后的职责。''她轻嗤,眼底隐有泪光浮动,``多好的一个摆设!''

\hypertarget{ux7b2cux516dux7ae0-ux5b9dux6708ux660e}{%
\chapter{第六章 宝月明}\label{ux7b2cux516dux7ae0-ux5b9dux6708ux660e}}

皇帝按着斋戒之名,静了数日。一切安排就绪,倒也不曾走漏风声。香见逐渐复了饮食,虽不大与人言语,却也叫人松了一口气。

皇帝见了如懿,益发和颜悦色,``这次的事,皇后做得极好,朕心甚慰。以后,皇后只需这般恪守本分就好。''

恪守本分?她在心底里冷笑出来。她与他之间,原也不过如此。

追随数十年,根本无须情悦意好,不过各司其职便了。

是她痴心妄想,原就是她痴心妄想。

接下来的日子,秋霖潸潸,阴晴不定,忻妃为时气所感,病势愈见缠绵,便将八公主托在海兰身边照拂。如懿得闲时便听永琪说说成亲后的琐事,看着小儿女童音稚语,倒也勉强度日。只是,她不能静下来,亦不敢。一静,听着那雨滴竹梢,深打芭蕉,心中忧闷,更觉泣血。

时在深秋,寒意瑟瑟。这一日皇帝斋戒已毕,兴致甚佳,便传旨合宫往宝月楼去赏京中景致。太后是第一个辞了的,她久不理宫中事,对宝月楼登高之事自然意兴阑珊。如懿倒是以忻妃之病辞了不去,皇帝却道:``皇后不在,亦无趣味。''

如懿知与皇帝龃龉已种,亦不愿深拂他意,只得应承了,严妆华服携合宫嫔妃而往。因着皇帝兴致颇高,便是卧病的忻妃也挣扎着来了。忻妃见了如懿便笑,悄声道:``皇上如今的性子喜怒不定,臣妾可不敢扫皇上的兴。''

如懿近她耳边,悄声道:``若是十分支撑不住,便告诉本宫。''

忻妃虚白面容上泛起一抹樱红。如懿暗暗叹气,她原是那样活泼的人,如今也熬得枯瘦如柴。这日子,当真是煎熬得紧。

正说话间,已然到了宝月楼下。那宝月楼在南海一带,那儿原无宫室,从瀛台上望去过于空旷无景。皇帝便决意要建一座楼宇,做临水赏月之处。那殿阁去岁动工,秋日已成,建得如月中广寒宫一般,故名宝月楼。皇帝亦曾笑语,不知哪位女子登高,才比得上月中青女素娥的婵娟风姿。

忻妃笑吟吟道:``皇上总说宝月楼建得精致,便是连嫦娥都住得。今日唤了咱们这么多人来赏秋,可不是一群嫦娥挤破了头。''

她素来风趣活泼,便是颖嫔这样不苟言笑之人,也掌不住笑了,伸手去拧她的嘴,``这般病着,还要饶舌。哄得太医一日三趟去瞧你,就是矫情。''

忻妃俏生生立在那里,``我再矫情,也盼不得皇上来看一眼。只能哄几口吃喝,饱口腹之欲罢了。''

笑语罢,却是李玉先迎了上来,``皇后娘娘,皇上与小主已经到了。''

众人一时未解小主指哪位,但合宫嫔妃皆至,却是如懿先明白过来,挑眉道:``寒氏?''

李玉点头,众人登时寂然。如懿却也不意外,携了嫔妃上楼。宝月楼楼高两层,飞甍重檐,琉璃瓦顶,意趣雅致,气象高洁。还未等留神细观,皇帝已然携了香见从里头出来。

香见的精神仍不大好,但换了浅紫白双绣雪莲花轻罗长裙,烟霞紫绫裙素淡无纹。长发曼鬋,鬓黑如漆,其光可鉴,只以浅一色的紫羽并雪色珍珠点缀,简约的衣衫无心中显出惊世之美。

只是这美,亦有残缺。但香见浑不在意,更不掩饰,任那粉红伤口横亘于众目睽睽之下,兀自淡漠,目视自己的足尖。

有窃窃私语之声,她亦淡然处之。仿佛这世上一切,甚少有经她心者。皇帝看着她,目光眷眷,舍不得挪开半分。

还是嬿婉先婉然含笑,``皇上命臣妾等赏秋,不知景致美在何处,还请皇上告诉才好。''

皇帝缓过神来,笑道:``还是令妃敏慧。宝月楼新成,北可眺三海,南可观街市,东可看紫禁,西可望远山。''

他一一指点,挥斥间颇为自得,将红尘阡陌、万户人家行云流水般划过。每有所指,嫔妃们皆惊叹、欢悦、喜笑、媚语,唯有香见如冷月照澄江一般遗世独立,不闻世事。却是颖嫔先``咦''了一声,指着不远处一显是新建的祈福堂道:``这不是寒部的祈福堂么?''

此言一出,连香见亦惊动,急急看向颖嫔所指处。果然那祈福堂金顶火檐,高起云涌,极尽辉煌之能事。

香见死死盯着那间祈福堂,不觉热泪盈然。熟悉的亲切果然熨帖了她孤独的乡情,亦适时地柔和了她一直如冰山雪岩的孤绝。那一刻,如懿才觉得,她并非九天谪落的仙子,遗世于尘外。她也有世间女子的一颦一笑、热泪与愁眉。

皇帝定定地望着她,眼中尽是痴慕之色,``香见,这祈福堂是朕按照你家乡规制所建,你还喜欢么?若是还有哪里不好尽管告诉朕便是。''

香见无语凝噎,片刻才缓过神来,恢复了往日的淡漠,``极尽华丽,无一不像,只是空落落一座祈福堂,落在这里有什么意思?''

皇帝眸中情意更盛,恨不能缠绕于她身上,他有些小心翼翼,带点讨好的意味,``有寺无人,谁来尊敬神明呢?寒部偏僻,朕已令你部中族人老幼妇孺者移住京中,与祈福堂相对。这样你即便不出宫,也可看到家乡风貌,不会再独自愁闷了。''

香见每听一句,眼中震动之色愈深。那些话是勒紧的铁弦,惊得她不知如何言语,茫然地望向如懿。如懿看着皇帝,他的眼睛,是寒潭深渊,分明柔情似水,却存着志在必得之意。她辨不出心底是何滋味,酸楚且陌生,她从未见过他用这样的眼神去看过任何一个人,从来没有。还是海兰悄然上前,在衣袖下牵住她冰凉而潮湿的指尖,笑靥蕴暖,``皇上胸怀天下,还能顾及臣妾等心思,果真心细如发。香见妹妹家中遥远,定是思乡情切,若是能见一见族人宽慰心思,身子也必好了。皇后娘娘每与臣妾说起此事,都是忧心香见妹妹的身子呢?''

皇帝听得入耳,笑意更浓,``此刻你的族人都已来了,你愿意见一见么?''

嫔妃们眼见如此,隐隐有骚动之意,窃窃之声,不绝于耳。嬿婉唇边冷光陡盛,旋又隐入春波笑意之中,上前亲切地挽住香见的臂膀,柔声道:``从前我家乡在盛京,初至京城多觉不惯。妹妹远道而来,必定也是。''她温婉劝道:``皇上,快请妹妹的族人来吧。妹妹一定很想见呢。''

香见不惯于这样的热络,急急抽出手,垂眸不语。皇帝击掌两下,便有小太监引了数十位寒部打扮的人来,来者多是老幼妇孺,一个个互相搀扶着,畏畏缩缩立在楼下。进忠刚要唤他们行礼,皇帝摆摆手,挽过香见行至楼前,向下道:``看看你的族人,他们也在瞧你呢。''

香见迫不及待地引身向前,浑不觉皇帝仍挽着她的手。她热泪潸潸,``这是阿里娅婶婶和她的小儿子。这是拜玲耶婆婆,她年纪大了,耳朵不好。还有穆妮尔,她才六岁,在战争中失去了一条腿。''迎着楼下欢呼雀跃之声,她情不自禁地笑着喃喃,``为什么?为什么他们会来?''

皇帝诚挚地看着她,捧住她的脸,正色道:``你以为联只是安慰你的思乡之情么?朕接来的这些人里,没有一个壮丁,那是因为年轻力壮的人该留在寒部修复疮痍,再建家园。而这些老弱妇孺,无家可依,也禁不起边陲风沙。所以朕将他们接来京城,可以安然度日。你,欢喜么?''

如何能不欢喜?可香见只觉得彻骨寒冷,一动也不能动,任由他扯着。她望着楼下熟悉的族人,恍如自己成了一尊冻实了的冰雕,从里到外冷透了。

再也不能妄想离开了,连死,也不能。困在宫里那么多日子,从来没有一刻如此的绝望。她是走不脱了。他或许真是爱她,可也在要挟她。她完全没有办法,因为爱与压制,或者是他最惯用的最轻而易举的办法。

如懿看着香见,她的绝望如此了然。她只觉得怜悯。所谓身不由己,原来人人如是。

金风十里,丽人玉颜,花压鬂云偏。红叶白露,远山流岚,京中的美人与秋色让人目眩神醉,如懿却醉不了。她看着远远的黛色山峦绵延起伏,正是千山叶落,孤雁低旋之景。唯见万里屋云间老翅掠空,哀哀悲鸣,曳下苍凉悲怆之音。绮丽明媚,深情相许都落了繁华盛世的注脚,谁还见忍泪自吞的无声凄楚。

皇帝轻拥着她,像是轻拥着一团正融的春雪,在她耳边低声絮絮:``香见,朕知道你心里在笑话朕,整个紫禁城也都在笑话朕。朕娶了一个败军亡族的人的女人,娶了一个有过婚约的女人,一个异族部落的女人。更要笑话的是,这个女人的心不在朕的身上,她甚至还恨着朕,厌恶朕,恨不得逃离朕。''

皇帝说着,气息温热地拂上香见的面颊。香见下意识地偏过头,缩着手,回避他任何可能的接近。

皇帝苦笑道:``可是朕从来没有这么喜欢过一个女人。朕有过那么多女人,宠过那么多女人。曾经喜欢的一个,朕扶着她坐上了皇后之位。可是朕直到见到你,才发觉原来男人对女人的喜爱不只是可以细水长流的,它可以像地底的火山一样,埋了上千年,轰然全喷了出来。朕对你,就是这样的。''

嫔妃们站得稍远,未曾听得皇帝的一字一句。如懿就在近旁,清晰入耳。她有轻微的晕眩,眼前的世界是粉碎的雪片,冷冷地打在心上,她感觉自己鼻息的迟缓,钝钝地,每一呼吸,都有挫磨的痛。

不是不知道他会对着旁的女人甜言蜜语,只是未曾亲耳听过,所以也不过是模糊的揣想,偶尔来扰乱自己平静的心绪。她是第一次,听着他对旁人说自己。原来她的存在,不过是一个已然不要紧的旧爱,像发黄的流云缎,纵使矜贵,那也是不体面的陈旧。她,不过是来陪衬皇帝天荒地老荡气回肠的新爱的点缀。

真是可笑!曾经履冰雪,践荆棘,这样千辛万苦走到他身边,蒙他所爱获得与他并肩而立的资格,也不过是陪衬来日的新人笑罢了。

香见残存的笑意渐渐褪去,只余下白雪覆野似的冷戚,有滚烫的泪水从她的眼中潸潸而落,最后成了无声蜿蜒的溪流。

皇帝听着香见族人们的欢呼声,揽过香见柔弱的肩,好声好气地哄道:``别哭!别哭!你看你的族人们多高兴,你可也是高兴坏了?''

香见如何说得出话来,更不敢叫楼下的族民们看见她的泪容,少不得侧了身子,避侧在皇帝身畔。皇帝便伸出手,宠溺地轻轻拍着她的背。如此一来,落在旁人眼中,更像是皇帝与她格外亲近似的。

随行的妃嫔们多半已铁青了脸,或是含了讥讽的笑,晋嫔冷笑连连,向着嬿婉小声说:``什么贞洁烈妇,都是做给旁人看的。不过是矫情引逗皇上罢了,这般欲拒还迎的。''

忻妃蹙了蹙眉,喟叹道:``费了好大的功夫还是要随着皇上,那之前那些都算什么了?''

也不知是谁暗暗嘀咕了一句:``狐媚子就是狐媚子,最会这些勾引人的下作手段!''这一句话引得嫔妃们连连额首,只避着前头陶陶然的皇帝而已。

如懿听得不像样子,转首深深瞧了她们一眼,嫔妃们立时噤声,不敢再言语半句,一个个眼观鼻、鼻观心地安分了下来。

恰好皇帝扬首,吩咐李玉赏赐楼下族民,好好送他们回长安街居住,便喜滋滋道:``香见,承乾宫虽然富丽,但你住得不喜欢。朕打算把宝月楼赏赐给你,你便住在这里,日夜可以看到家乡景致,也好安心。''

嬿婉见香见并不作声,便知道她已无抗拒之意。她将一口酸气活生生吞下,脆脆笑道:``皇上这般安排,妹妹必定喜欢。''她上前一步,凑趣道,``皇上当初一直说要给妹妹一个名分,却因国事繁忙耽误了。今儿臣妾就替妹妹讨个喜。皇上定了名分,臣妾姐妹间也好称呼相处啊。''

皇帝甚是赞许,忙里偷闲瞟了嬿婉一眼,将那笑容蜻蜓点水似的恩赐于她,``令妃所言甚是。朕已想好,就封寒香见为容贵人。虽然你容颜有损,在朕眼里还是如初见一般清妩极妍。还有\ldots{}''他提高了声线,``你从寒部而来,宫中规矩未必样样周到。朕希望在这宫中人人可以容得下你,与你和睦相处。''

这话分明是提醒了。

倒是嬿婉淡然含笑,``皇上说得是。臣妾等身为妃妾,自然得和睦一心才是。说来容贵人册封真是喜事呢。倒叫臣妾想起来,南边移来的荔枝树一直未曾结果,今年不知怎的却结了两百多颗果子。可见容贵人入宫带来祥瑞,又让皇上事事得了好结果。''

这话说得皇帝喜笑顔开。

如懿遥遥听着,微蕴了一丝讥讽,目色悲悯。皇帝忽然唤她:``皇后不为朕高兴么?怎么一个笑容也没有?''

如懿举眸,静静道:``臣妾与皇上夫妻一体,一喜俱喜,一悲俱悲。如今皇上接了容贵人族人来,容贵人自然感激皇上恩德。皇上心愿得偿,真是恭喜!''

嬿婉的笑意几乎要浮到眉毛上,她低下头将那缕不合时宜的笑尽力按捺,方俯身相拜,以谦恭而诚恳的姿势,稽首道贺:``容贵人正需皇上安慰陪伴,臣妾理当告退。愿容贵人自此后与皇上两心相许,珍重到老。''

她的话,再及时不过,将皇帝与如懿僵持后的尴尬与冷淡旋即化去,也解了嫔妃们的局促。一刹那的冷寂,有三三两两的嫔妃笑语相贺。然后,更多。

在一片喜悦与热闹中,皇帝望向嬿婉的目光带着赞许与些许温情,``朕明白你的用心。秋日寒凉,你怀着身孕行如此大礼,仔细伤了身子。''

嬿婉的笑顔全然发自内心,无半分破绽,``只要皇上欢欣喜悦,臣妾也安心了。''

皇帝凝视她,笑意更深。不知谁说了一句:``眼看又要起风,咱们快些回去吧。''

真的是起风了。方才还是晴蓝天色,转瞬暗了半边,有风旋着满地落叶疾疾打转。

嫔妃们巴不得这一句,跟着请安告退。皇帝见香见面有倦色,忙示意侍女扶了她下楼歇息,方才沉下脸道:``皇后口中说恭喜,面上却无喜色,算不算口不应心?''

蛾眉若能带着九秋清霜,大约便是如懿此刻的模样,``臣妾倒想陪皇上笑一笑,只是若容贵人能真心一笑,臣妾倒也愿意。''

皇帝愈发不豫,``醋妒!''

如懿却也不恼,一双眼眸秋水寒澄,有泠泠清光,``臣妾是女子,不是圣人,固然有七情六欲。所以既要看得六宫的醋妒,也要看得容贵人的伤怀。''

``伤怀?''皇帝冷冷一嗤,略带嘲讽地看着她,``皇后位高权重,谁知眼力却不如往日了。容贵人落泪,是感念朕保全族人之恩,知晓朕的情意。''

``哦,皇上真的这般相信么?''风猎猎地吹,拂过鬓边的点翠玫瑰金花钿,细细的烧蓝流苏打着脸颊,凉一阵,又凉一阵。她心下有严霜覆落,较轻吟道:``千古艰难惟一死,伤心岂独息夫人。''{[}1{]}

皇帝作色,``你讽刺朕是楚文王?''

如懿见他隐然动了真怒,原想着低一低头,然而见他这般疾言厉色,显是心虚,便也迎着他道:``皇上是不是楚文王臣妾不知,但容贵人真心可惜,为着保全族人,少不得也要对着皇上强颜欢笑!''她见皇帝额上青筋突起,依旧道,``皇上若要寒部真心归顺,自可以德服人。何必用容贵人与她的族人互相挟制,灰着心侍奉皇上左右!这般做固然是得了美人臣服,但若只得了人得不到心,又失了六宫的祥和,又有什么意思!''

皇帝断然喝道:``听听你这些话,哪里有国母的气度!六宫不睦,自然是你御下无方。语涉国政,便是你这个皇后的无知不慎!后宫不得干政是老祖宗的训示,你若敢犯雷池一步,纵然你是朕的皇后,朕也绝不宽宥!''

``后宫不得干政,臣妾牢记于心。皇上就当臣妾醋妒也好,无知也好,臣妾求皇上一个明白!皇上为了容贵人,不惜拿制衡前朝的法子来对付她,这岂是明君所为?''她屈膝在地,抱着皇帝凄然道,``皇上百年之后,难道也要被人议论如楚文王一般迫人委身于己么?''

皇帝的鼻翼微微张着,不由分说便扬起手来。如懿吃了一惊,只直直地看着他的手掌落下,竟是避无可避,只得闭上眼睛,打算生生受了这一掌!

良久,却是无声。只有一只手,冰凉地拂过自己的鬓发,牵扯起她心底钝痛。有温热的水珠缓缓滴落在面上,她有些不可相信,睁眼看去,却见皇帝以手覆额,无限痛苦道:``如懿,你说的朕如何不懂。一开始,朕真的只是想挫磨掉寒氏余部的锐气,才同意他们送香见入宫做一个礼物,想着哪怕她入宫,朕冷着她就是。可直到朕看到她的第一眼,她那么美,那么沉静。朕根本移不开自己的目光,那一刻,朕知道自己没有办法了。朕一生的教养,一生的骄傲,都抵不过她看朕一眼。如懿,朕真的是没有办法,才会动出那样的法子,用她的族人来留她在身边。朕知道,朕是得不到她的心了,可是有她这个人也是好的。朕是真的想让她髙兴些,让她愿意留在朕身边。''

她满心凄楚,``皇上又来跟臣妾说这样的话\ldots{}''

皇帝沉浸在自己的思绪里,抽丝剥茧娓娓低诉,``六宫里的人那么多,朕只想安安静静守着她。若她肯对朕笑一笑,朕比得到什么都高兴。如懿,己经几十年了,从朕登基,从朕得到皇位开始,朕的一心便给了前朝。朕要守着祖宗的江山基业,要亲手建立一个盛世王朝!朕为此费尽心血,却忘记了,自己也是一个普通人,有着普通人的渴望!如懿,朕长到这般年岁,渴望过皇权,渴望过皇阿玛的关爱,可这都过去了。朕如今最渴望的,只有她一个。''

如懿起初还静静听着,听到最后,禁不住浑身乱颤,``偌大的后宫,皇上只想要她一个!那也好,从臣妾起,一个个剪了头发离宫清静,何必听皇上说这些锥心之语!身为皇上枕边人,皇上这些话自然是伤透臣妾的心,但皇上不在乎,皇上愿意说,臣妾便听着,只当自己是死的罢了!可列祖列宗在上,皇上这些混乱之语,做个情圣倒也罢了,若身为君王,如何对得起大清江山!''

皇帝软弱地垂着泪,仰首轻轻道:``如懿,朕对你说这些话,原以为你是懂朕的。却原来,也不过如此。那么这些话,只当朕白说了吧!''

如懿的胸腔剧烈地起伏着,强自按下心神,定定道:``臣妾方才那些话,是身为皇后理应说的。''她不知怎的,满心满肺里都是难言的委顿之情,逼得她站也站不住,几乎要跌坐下来,``臣妾陪伴皇上数十年,不敢自称与皇上心有灵犀,但也自以为和皇上略有心意相通之处。如今看来,多少年夫妻相伴,竟也全是白费了。臣妾,无话可说,也不能再说,臣妾告退。''

天色铁灰,阴阴欲雨。如懿步下阶梯的脚步有些紊乱,皇帝一阵心紧,急急跟上。李玉与凌云彻见帝后如此,不觉也慌了神。

才出宝月楼,已然有急雨打落。皇帝唤道:``皇后,下雨了。''

如懿并不回头,但觉头顶红云一亮,原来是一把胭红绸伞开在了头顶。是皇帝的声音,``别淋着雨。明日嫔妃还要拜见你。''

碎雨纷飞中,容珮手执红伞,扶着披着暗金西番莲纹雪锻大氅的如懿缓步向前。

她终究还是忍不住,迎着银丝万缕,回首望去。映入眼帘的,却是皇帝朝着宝月楼疾步而去的身影。寒雨纷纷,她的心终至绝望。

凌云彻本跟着皇帝,不知怎的慢下步子,撑着暗黄油纸伞,朝着她。一步一步,缓缓而来。

{[}1{]}出自清代诗人邓汉仪的《题息夫人庙》。全诗为:``楚宫傭扫眉黛新,只自无言对暮春。千古艰难惟一死,伤心岂独息夫人。''邓汉仪,字孝威,号旧山,别号旧山梅农、钵叟。明末吴县诸生,邓旭之弟。息夫人,春秋时期息国国君的夫人,出生于陈国的妫姓世家,因嫁于息国国君,又称息妫,后楚文王以武力灭息国而得之。因容颜绝代,目如秋水,脸似桃花又称为``桃花夫人''。

\hypertarget{ux7b2cux4e03ux7ae0-ux73afux654c}{%
\chapter{第七章 环敌}\label{ux7b2cux4e03ux7ae0-ux73afux654c}}

天下事往往莫不如此。之前有多么不愿意接受的,万般抵触的,待到既成事实,便会劝着自己接受,慢慢习惯。譬如宫娥嫔妃,眼见着香见名分已定,送入养心殿侍寝,连如懿与太后亦不作声,背地里嘀咕几句,便也忍下了。

香见侍寝后的第一日,她便随嫔妃们同来翊坤宫拜见如懿,并不特立独行,只是随众择了自己的位次坐下,孤坐少言。香见再不执着于着自己部落的衣衫,换过了宫装打扮。虽是同样的服制装束,香见的美却是琉璃上游弋过的月色清清,美得凛然出尘。

香见的面色照例是白得发青,是玉,对着阳光便能透明的乳青色的玉,极名贵的那种,且透而薄,让人不敢轻易去碰触。仿佛轻轻一呵气,便能散成尘屑碎去。因着瘦突,她的下颌尖尖的,是青桃的尖,有日光蒙昧地照着她的侧脸,都能看清细细的、水蜜桃似的绒。年轻在她身上显得特别美好,连那一道疤痕都成了粉色的亲吻的痕。她梳着最寻常不过的两把头,点缀着几朵青色镶风毛旗装,连一丝花纹也无,也是近乎朴素的低调。对着阳光,才能留意到衣上浮着的青花凹纹。除此之外,只在衣襟纽子上别了一朵她最爱的沙枣花。如此清简,比着旁人的精雕细琢,她生生成了简简几笔画就的淡墨写意美人,有一种漫不经心的意犹未尽。

那是一种安守规制下的潦草。一个女子,必定是对生活无望,对身边的男子无望,才会待自己这般潦草而不经意。

待到人都散了,如懿只留下了香见,由海兰一同陪着。香见倒也安宁,定定坐了,想要喝茶,却不太喝得惯。容珮眼见,便换过了牛乳茶,香见直饮了两碗才罢。这等痛快,让如懿从心底安定了。

如此,怕是真的不会再寻死了。如懿唇角便有了一星笑意,``活着比死了艰难。你肯如此,便是什么都不怕了。''

香见的神色淡淡的,垂着脸,``已经过了最想弃世的那一刻。''她停一停,抠着小指上的鎏金掐丝云母嵌东菱玉护甲,她戴不惯那东西,却也不摘下,一直别扭地拨弄着,``站在树底下看着蝼蚁,想着也不过如蝼蚁一般活着,便也不算是太坏的事了。''

如懿想起方才嫔妃们对着她那种艳羡而妒忌的神色,轻轻叹了口气,``既然你己经侍寝,少不得也要和宫里人来往。那些人,你不必理会就好。''

她淡淡一笑,那笑意朦胧得如初冬晨起的白雾,湿漉漉的,``我会恪守对您的规矩,是因为您教明白了我许多。''

如懿有一丝歉然,``其实你知道,本宫劝你,一半为了皇上,一半为了你。''

香见用指尖抹去嘴唇上乳白一滴,``不管你为了什么,至少只有你会对我说那样的话。''

海兰盈盈一笑,``为了劝你的缘故,多半人都恨死了皇后娘娘。劝活了你便是留下了六宫不宁。幸好你还能体谅皇后娘娘的一片心,也不枉了。''

香见眉头挑起柳叶横逸,``只是我很不明白,你为什么会去劝一个被你丈夫痴缠的女子,你不觉得你盼我死了或是出宫会更好么?''这样直接的话,大概只有香见这般心地纯净的女子才会了当问出。有时候真觉得,这个女子真是独特,就如她衣襟上别着的沙枣花,清香盈盈,是她所从未见过的。

海兰欲言又止,只是默然叹息。如懿拨着手里的镂空松竹梅珐琅赤金手炉,淡淡道:``作为一个妻子,本宫何尝不这样想。但作为一个皇后,更多的是职责,顺服地去服从,而非让自己的情感舒服。''

海兰温言道:``皇后娘娘也曾想让你出宫,但那更多是为了皇上的清誉。为了你,皇上承受的指责不少。''

香见眉心皱起,显然是嫌恶,``那是他自己该承受的。''言毕,她轻轻一叹,似是无限愁烦,亦像自语,``己经侍寝了,我没法子不打算,怎样才可以没有身孕呢?''

如懿只觉得心头急剧一跳,隐隐骇然,眼看海兰也是颇为惊诧,静静一想,反倒对香见生了无限怜悯。

人到绝境,原来所求的,只是这个。

当然有许多的法子,也有一劳永逸的法子,海兰嘴唇微张,但还是紧紧抿住了。也是,谁敢告诉她这个。

香见倒也不再问,仿佛只是不经心的闲话罢了。她只是木木地坐着,半晌无话。天光将她的身影拉得老长老长,如懿看着那细细长长的黑影,心底一阵酸,一阵凉,寂然无言了。

过了黄昏,便是皇帝往慈宁宫请安的时辰。自从端淑长公主归来,又产下麟儿,太后含怡弄孙,往日的凌厉消散不见,与皇帝也彼此相处安然了。这是极好的事,皇上本重孝名,面子上一向顾得周全,逢太后寿辰,也必以奇珍异宝相贺。加上太后再少理后宫事,两宫之间,愈见和睦,倒真有几分母慈子孝的样子了。

皇帝守着斋戒,本为养伤。幸好伤口不深,皇帝素日的底子也在,很快口子便愈合了。只是一时还碰不得重物使不得力,拿袖口小心掩着,不欲人知。

如懿避着皇帝,皇帝也避着如懿,这些日子便是去慈宁宫请安,也是各自错开了时辰。这日,皇帝去得略早,进殿便见容珮候在外头,心知如懿在内。但再要退出也不合宜,足下一定,还是照旧入内。

太后见了皇帝,便是欢喜,招了手唤他近前,托着一副西洋鎏金水晶老花镜道:``皇后送来的什么稀罕物儿。哀家前几日说了一句眼神不好,皇后便弄了来。果真有心。''

如懿见了皇帝进来,早早施了礼,立在一旁。皇帝笑吟吟道:``皇额娘还记得么?去年有个西洋自鸣钟,也有趣得紧。儿子也送了您一个。''

太后笑着连连摆手,``每半个时辰便跳出一只珐琅彩雀叫几声,哀家嫌它吵闹,又实在喜欢它精致,便叫福珈收起来了。说起来,还是咱们的更漏好,又准又静。''

太后得趣,皇帝与如懿自然也陪着。正巧福珈捧了海棠花饰雕漆填金云龙红木盘来,上头置着三柄硕大的如意,每柄都有两尺来长,沉甸甸的华贵,分别是莲花锦地纹嵌镶青玉如意、玛瑙巧雕冰梅枝喜鹊双彩如意,另有一把和田白玉如意,通体纯白,浑如凝脂,只以大红夹金线流苏为坠。

太后指着三把如意道:``下个月初九是你五弟弘昼的孙子百日的好日子,皇帝你也瞧瞧,这三把如意送哪一柄去最好?''

皇帝随口道:``皇额娘的眼力,挑的东西自然是最好的。''

太后含笑道:``人老了眼力也不行,叫皇后帮着瞧瞧,她也只说哪个都好。还是你来选。''

皇帝这才仔细去看,一一道:``这白玉如意乃和田出产,玉质极佳,只是百日之喜,用纯白似乎不合。青玉如意亦好,是西洋的工匠做的,样式新巧些。''

太后看了皇帝一眼,只不作声。果然皇帝道:``只是西洋的玩意儿固然精巧,却不登大雅之堂,平日赏玩便好,送正日子的礼便不宜了。唯有这把喜鹊双彩的,虽然俗些,但热闹喜庆,用的是红白双色玛瑙作底,十分难得。''

太后微微颔首,``便是这把吧。''她说着,捧起那双彩如意细细抚摸,``质地细润,纹理瑰丽,的确是好\ldots{}''她手上陡然一松,``哎哟''一声,那如意便沉沉脱了手,直直往地下坠落。

如懿本能地伸手去拦。不意皇帝靠得更近,一双手早伸了出去,挡在了她的臂上。她心底一紧,想起那如意入手发沉,又兼下坠,力道甚重,而皇帝的左手,是有伤的。

正想着,皇帝己然接住了那把如意。他眉心一皱,显然是触到了痛处,只强忍着笑得如常,``幸好不曾跌落,否则伤了,哪儿来如意呢?''

太后笑逐颜开,``还是皇帝手稳。福珈,既然皇帝已然选好了,快收起来吧。''

如此,三人闲话了片刻,皇帝便匆匆告辞了。如懿惦记着永璂的功课,亦不多留,也请安告退。待得二人都走了,太后面上温沉的笑意逐渐敛去,看着一旁的福珈,定定道:``果然传言不虚。皇帝的手,的确有伤。寒氏\ldots{}''她眸光一敛,复又沉静,``可惜了。''

如是七八日,皇帝都歇在宝月楼。如巨石坠落湖心,惊得众人闲语纷纷,恨不得问到如懿跟前。但看如懿波澜不惊,只得含了笑生生忍住了。

如懿倒不甚在意,皇帝的沉迷和对旁人的冷落,倒是给了她一个喘气的时候,经了那次,她与他,是相见也漠然了。她早过了对男欢女爱肉身缠绵沉溺的时候,且宫里的女子,若非最得宠的那会儿,都是惯了孤枕,并头而眠皮肉相贴倒成了难得的事,盛大得让人累得慌。有次婉嫔说笑起来,说皇帝骤然不知哪天忽然想起她,便翻了她的牌子侍寝,她慌得什么似的,像锯了嘴的葫芦不知该说什么,手脚都没处放了,才想起原来己经十二年零三个月四天未曾侍寝过了。

说罢,如懿与海兰都笑了,连病卧着的忻妃都笑得前仰后合。笑罢,眼角都有泪光隐隐。多少凄楚,都在这笑语中了。

这一日皇帝下了朝,眼见起了北风,嘱咐人多往宝月楼中送了红萝炭,又闻新折的沙枣花到了,便喜道:``容贵人最爱沙枣花的香气,一日也离不得的。''

李玉笑道:``皇上在宝月楼周围多种沙枣树,便是为了容贵人喜欢。只可惜容贵人思念家乡,寒部送来的沙枣花,她看了最高兴。''

皇帝一壁嘱咐人送去,一壁道:``朕去看看容贵人。''他起步要走,想想还是停住,``朕有些日子没见到永璐了,也记挂着璟婳。''

秋末冬岁,白昼日短,嬿婉正闷坐着,斜倚暖阁,看着乳母们哄了两个娇嫩的孩子爬着玩兔儿爷。澜翠便骂:``兔儿爷是中秋玩的,都什么时候了,还让阿哥和公主玩着过了时的东西。''

嬿婉便有些懒懒的,``兔儿爷是过了时的,本宫不也一样不叫人惦记。''

澜翠听了这口气便有些慌,心知皇帝不来是如何也劝不得的。可满宫里谁不一样,要见皇帝,得望穿了重重宫墙望穿了宝月楼才见得到。

嬿婉推开窗,深秋的风己经有刮骨的凉,吹起她衣领上出好的风毛,柔腻腻地拂着。她喃喃道:``瞧这风吹的,整个紫禁城的炕都冷了,只有宝月楼是暖和的,热乎乎的。''

春婵悄声劝道:``小主,您别这么说。''

嬿婉缓缓合上描金镂``福寿长春''的窗扇,看着华丽的洒金藕合珠帘寂寞地垂着,没有半分有人进来的吉样,百无聊赖地耷拉着,不觉生了几分凄凉之意,``从前,这宫里的炕也是暖的,可是容贵人一进宫,怕是再也暖不起来了。''

春婵忙低声道:``小主别伤心,好歹小主还有阿哥和公主呢。不信您瞧瞧皇后宫里,也一样是冷清清的。''

嬿婉扬了扬手,``皇后怕什么,她是中宫,谁也挤不了她的地儿。可本宫不一样,嫔妃们的地儿就那么大,她躺下了,本宫就连站着的地儿都没有了。''

正闷着,忽听外头太监敞亮的嗓门喜气洋洋喊道:``皇上驾到------''那响亮的脆声跟鞭炮似的,嬿婉喜不自胜地站起来,脚下带着风迎到了门外。直到手臂挽住了皇帝的手臂,那龙袍柔软的绣纹摩擎着她的手心,才觉得真切。

皇帝真是来了。

嬿婉本来穿了一件石榴子红的锦袍,上头漫漫地绣着菘蓝绿的叶与樱草黄的花。那花本是半开的,无精打采的。可是皇帝一来,每一叶与瓣都染上饱满欲滴的彩色,每一朵都是欲说还休的情意,在新鲜跳跃的红底子上闪闪欲动。

皇帝着了她一眼,便去逗璟婳和永璐。两个孩子有些日子没见到皇帝,有些生疏。皇帝兴味索然,便打量着道,``这衣裳你穿了好看。可惜香见不爱穿这样艳的颜色。也是,她那样的人儿,穿得艳便俗了。''

嬿婉堆在脸上的笑顿时就酸了,她忍着鼻尖的酸涩,亲手接过春婵斟上来的茶,娇声道:``皇上好在意容贵人,容贵人真是有福。可皇上别只宠她一个,忘了臣妾和永璐呀!''

皇帝心不在焉,出神片刻才醒过来,含含糊糊笑道:``你说朕宠什么?''

嬿婉心中一紧,旋即笑容满面道:``臣妾说,容贵人初入宫中,皇上别一味宠着她便算好了,要多多关心,知她想些什么要些什么才是!''

皇帝一怔,豁然开朗,起身向外疾走道:``是呢,朕怎么没想到,她最想要的该是这个才是!有个孩子,便有个依傍了。''

嬿婉正捧过金线青莲茶盅,冷不防皇帝冲出,吓得茶水险险泼出。澜翠急切道:``皇上,您饮一口茶再走,小主为等您,出了三遍茶色才好的呢。''

话未说完,皇帝己经走得远了。嬿婉切齿道:``还喊什么?哪里的好茶都比不上宝月楼的茶叶末子香呢!''

澜翠吓得哪里敢说话,嬿婉气冲冲的,璟婳和永璐一吓,此起彼伏地哭起来。嬿婉便有些不耐,``我的好祖宗,你们皇阿玛来了生疏什么,难不成几日不来就不认得了么?''

乳母们依依地哄着,嬿婉揉搓着衣裳,想起皇帝的话,更是烦郁。她定了定神,起身道:``换件衣裳。带了永璐和璟婳去慈宁宫,本宫要好好向太后请安。''

这一日晨起,如懿便按着规矩往慈宁宫请安去。过了那么多年岁,时光温柔了眉眼的凌厉,磨平了心智的棱角,她与太后,倒有了几分寻常人家婆媳相处的恬然。

自然,有多么亲近是不必的。恩怨太久,自己都计算不清了。但是坐下来一杯清茶一柱檀香,倒是能撩起许多往日的细碎。

真的,连如懿自己也未曾想到,能与太后相处成这般模样。

所以当如懿惯常般走进慈宁宫的暖阁时,见太后正背对着她,阁子里清晰地有小银剪子一张一合的清脆声,她便笑:``皇额娘万安。''

太后无声,如懿走近几步,``皇顺娘可是在修剪御花园里的金桂,花香甘馥,闻着便觉得甜。''

剪子的声音戛然而止,太后放下银剪,端然侧身坐下,抿了口甘冽茶水。

如懿乍见了宝蓝月影瓶中供着的那束花枝,险险惊得没立稳,那是几折沙枣花枝,己然被太后剪去所有零碎,只剩光秃秀的枝干。

如懿瞬间便定下心来,笑道:``皇额娘不喜欢这沙枣花,慈宁宫里不用就是。皇额娘何必都剪了,仔细伤着自己的手。''

太后淡淡一笑,那笑意却是碎冰上泛起的亮儿,叫人发寒,``从前只听闻唐玄宗为杨贵妃千里送荔枝,跑死了许多马儿。到了皇帝这里,倒也来了这一出一骑红尘妃子笑,无人知是枣花来。真真是一段奇闻了。''

如懿慌忙便跪下了。这不是她该说的,也做不得什么。跪下是最好的姿态。

太后道:``哀家明白你的意思。这件事你固然是不知的,皇帝又喜欢气派,便是靡费些也没什么。到底不是孝贤皇后在的时候了,还能劝劝皇帝节俭为上。''

如懿的面上就红了,``儿臣无能。''后宫如懿传全文,

太后客客气气地笑了,``你哪里无能,哀家瞧你也实在能干。寒氏的脸怎么伤的?皇帝的手是怎么伤的?这次是伤了皇帝的手,下回呢?再举起刀子来能要了皇帝的命。便没动刀子,色字乃刮骨钢刀,多少英雄好汉都受不住。何况皇帝在兴头上。你还替他左右瞒着,打着斋戒之名保全他的颜面,也真够难的。''

如懿额头上冷汗直迸,原来太后早就都知道了。哪怕她困坐深宫吃斋念佛,不过问宫中事。但她只以儿女为念,故洞若观火。

如懿磕了个头,心悦诚服地拜倒下去,``皇额娘既然都知道,儿臣也不敢隐瞒。但儿臣这么做,只一心为了皇上。若是张扬出去,实在有损皇家圣明。''

天光悠长,扯得珠帘的影子晃晃悠悠,有了生命。这样墨漆漆的生命突兀地耸立在四周,诡异地瞄着她。太后凝视如懿片刻,长长地嘘了口气,``我的儿,你是一番苦心。是皇帝昏了头,一颗心都被寒氏迷住了,险些连祖宗规矩都不要了。哀家不能阻止寒氏入宫,也不能阻止她侍寝。但你可曾想过,按她这么个侍寝法儿,若是生下了孩子来,该如何呢?''

如懿赔着笑,却如何敢说香见也抗拒着孩子的到来,只得道:``也未必这么快\ldots{}''

太后截然打断:``身孕这回事,一股子运气一来,就住在肚子里了。哀家知道,寒氏肯活下来,是皇帝要你去劝的。可你也明白,那是勉强的。一个女子怀着怨气侍奉着男人,那是什么事儿都做得出来的,便是把她族人都拉来了住着也一样。皇帝若再脑子一热,非得立了寒氏的孩子,就如当日顺治爷定要立董鄂皇贵妃之子一般,哀家这个太后也阻止不得。那也好,倒叫咱们辛苦打下的寒部,不费吹灰之力便占了大清江山。

如懿的心鼓鼓地跳着,每一跳,都胀得生疼,``那皇额娘如何打算?''

太后眼帘微垂,轻轻一嗽,福珈端着一壶青瓷汤盏进来。太后道:``一应都准备好了。喝下去,要她一了百了。''

如懿的面色瞬间苍白了,膝行上前,恳切道:``皇额娘这么做固然是为江山万年思虑,但皇上正在爱宠容贵人的兴头上,若贸然处置,恐怕伤了皇上的心。''

太后嘴角一弯,``哀家知道,皇帝心疼寒氏。可这碗药下去,她侍寝依旧,便也生不出孩子来了。这并未违背皇帝的意思,哀家也并不要寒氏的性命,只要她来日孩子的性命。''

如懿垂脸半晌,终于仰起头,对上太后静若寒潭的目光,``皇额娘,您明知这样做,皇上会恨臣妾。''

殿中点着幽幽的檀香,南红串玻拍珠帘悠然轻卷,袅娜的烟雾在重重的锦帐间凝成一抹,又絮絮飘散,弥漫于华殿之中。

太后的声音沉沉的,像是钻着耳膜,``哀家知道你不愿意去,一是下不得手,二则还是太在乎皇帝的心意。可你是否想过,你当日替皇帝劝服寒氏留下性命,是皇帝拿着皇后应尽的职责迫着你去。但哀家

今日迫你,也是一样。只为你是六宫之主,安定后宫是你的职责。所以,这件事是哀家的意思,却也只能让你亲手端去看她喝下。''

如懿的手撑在地上,寸厚的锦毯按在手心绵绵的软,却也发痒。那痒是夏日里的小虫子,一点一点咬着皮肉钻进去,百折不挠。她听见自己的声音:``六宫之主的职责,就是听从他人没有自己么?儿臣既得听皇上,又得听太后,除了两难,别无他法。''

太后笑意温和,``你可知道当年皇帝为何会选你继位为后,只因你家道中落,再非显赫。母家也无人在朝为官。比不得孝贤皇后满门富贵,除了依附皇帝,你并无其他法子。如今,你便尝到这里头的好处了。所以哀家劝你一句,想要坐稳后位,该听的听,该做的做便是了。''

如懿跪在阳光底下,十月的日色透过翡色烟罗纱似晕开的桃花蘸水,雾气蒙蒙,可她的背脊上却一阵一阵发着寒。

容下香见的命,是顺皇帝的意,亦开罪了六宫嫔妃。迫使香见喝下这碗汤药,是顺了太后的意,安了嫔妃的心,却是大大逆了皇帝的欢意。她在焦灼里,忽而想起香见那日的话,她打了个激灵,若是有了孩子,香见会如何?

太后并未容她细想,抚着怀中一把金丝檀琢碧玺如意,徐徐道:``非我族类,其心必异。皇帝要寒氏,哀家容她。可要再有子用上的事,那便不能了。其中利容,你自己掂量吧。''

\hypertarget{ux7b2cux516bux7ae0-ux7a7aux6708ux5e7d}{%
\chapter{第八章 空月幽}\label{ux7b2cux516bux7ae0-ux7a7aux6708ux5e7d}}

如懿不知道自己是如何出的慈宁宫,飘飘忽忽的,足下无力。待走到宝月楼外,她的魂总算回来了,一颗心亦沉沉定了下去。

举眸望去,见到的人竟是婉嫔。

西风渐起,呜咽着穿过红影碧栏的宫阙,婉嫔着一身深竹月色缂丝并蒂莲纹锦衫,披着一斗珠莎青绉绸皮袄,越发显得怯弱无比,如寒潭瘦鹤。她见了如懿,怯怯行过礼,大是不好意思。

如懿见她戴着一色全新的猫儿眼赤金吴翠花钿,不由得停下步笑道:``皇上新赏的?昨儿内务府才送来的。''

婉嫔面色微红,垂着脸道:``皇上惦念,臣妾铭感于心。''她说着,下巴几乎低到了胸上,嘤嘤道,``只是臣妾也快有半年没见着皇上了。''

如懿打量她,``你来这儿,是想见皇上?''

婉嫔窘得满脸通红,越发支支吾吾,``不是,臣妾只是好奇\ldots{}''她低低叹息,``臣妾只是好奇,皇上那么宠爱的女子,平日起居坐立,会是何等模样?''

如懿一怔,蓦地想起宫中曾有传闻,说婉嫔有一股子痴病,总爱在最得宠的嫔妃宫门外窥伺,而平素往来者,多是得皇帝欢心的女子。

这般想来,倒是真有些影儿。

从前得宠时的海兰、意欢与自己,后来一阵的嬿婉。便是和嬿婉疏远后,她也只是静静看着,保持着刻意的距离。

并非趋炎附势,婉嫔也不算那样的人。她,一直是六宫莺燕里最沉默安静的影子。

如懿便道:``容贵人是很美。''

婉嫔脸涨得血红,``不,皇后娘娘。''她的神气有些肃然,``臣妾喜欢看容贵人,只是因为臣妾好奇,好奇能否从她的一言一行中,看到自己得皇上多看一眼的可能。''她赧然,眼底的火光黯淡下去,那淡然的语气底下,伤感自怜是一根根细细的银针,戳进肉里也不见血,``可是,臣妾从她们身上看到的,永远是不可能。皇后娘娘,您知道么?臣妾见得最多的,记得最深的,便是皇上的背影。很多次皇上从臣妾的宫门前进宫,臣妾都盼着,皇上,他或许可以走错一次,走到臣妾宫里。可是,从来没有过,一次也没有。他脸上的欢喜臣妾记不清了,因为那从不是对着臣妾的。可他的背影,一直在臣妾心里,见不着皇上的时候,想一会儿,心口便暖一会儿。''

并不是不知道婉嫔的过往与宠遇。只是哪怕亲近如自己,原来也不知,素来默默无闻的她,竟也存了这样一段旖旎而纯粹的期盼。

如懿温言道:``婉嫔,你多虑了。''

婉嫔的眼底蓄满了泪水,静静道:``臣妾不过是一个最普通的女子,相貌平平,才德平平。在潜邸里是最不起眼的格格,在宫里是无人记得的嫔御。皇上玉树之姿,臣妾蒲柳之质,能得到皇上的一夕照拂,己经是臣妾毕生最值得荣耀的事。''她的痴念焚烧着眼底薄薄的水光,``臣妾不敢去妄想得到多少宠爱,只是想皇上偶然经过人群时,可以多看臣妾一眼。于是,臣妾想尽一切办法希望自己可以起眼些不那么普通些,才发现能想到的法子,也不过是最普通的法子。''

那些普通的字眼,在婉嫔平淡的口吻里,是刮着心口的锈刃,嚓嚓地磨着,未曾见血,也是生疼。如懿听着,没有一句可以安慰的话语。她能如何呢?她不也是那万千身影中的一个?

片刻,如懿听见自己干涩的声音:``你一向安分守己,皇上待你也不算不好。''

婉嫔浅浅地笑,凄凉而寂寥,``安分守己是因为臣妾实在没有一点可以引得皇上多一瞬注目的能力。而皇上,四季恩赏不少,也未曾亏待了臣妾。但是皇后娘娘,臣妾便是想多在皇上心上停留一刻,也那么难么?''

不是难,不是。情意之事,从来不是你期待多少,便可以得到多少。或许长久的守望,不过是将你的身影凝成望夫石恒定的姿势,而盼不一缕真心的目光。真是凄凉。

婉嫔遥望着楼上倚栏凝眸的香见,螓首轻摆,无比渴慕又无尽惋借,``臣妾若能得容贵人万分之一的宠爱,此生无憾。只可惜,容贵人太不惜福了。''

或许宫中之人,无不是这样想的吧。如懿目送婉嫔茕茕离开。才知宝月楼楼外,一样的痴心情长,却注定一双人,一段心,终究不得圆满。

香见独自坐在二楼,倚栏望着远处的祈福堂,神色痴惘,浑不觉如懿的到来。香见的侍女见了如懿,便得了凤凰似的迎进来,道:``皇后娘娘来了。我们小主正闷坐着呢,整日看着长安街和祈福堂,也不是个事儿呀。''

如懿淡淡笑,``难得有她喜欢的东西,随她去吧。''

那侍女扶住了香见,香见见了如懿,起身福了一福,``娘娘万安。''

如懿便笑,``京城十月风沙大,进去坐吧。''

宝月楼的布置浑然是第二个承乾宫,只是涂彩上多了好些寒部的样式。原本许多养心殿的起坐之物和摆设都挪来了这里,显见皇帝是常来的。

如懿亦不多观,便问:``方才过来瞧见婉嫔,也不知在宝月楼下仰望你多久了。''

香见漠然,``见过一两次。她很奇怪,总不上楼。''她嗤地一笑,``旁人眼里,我也很奇怪吧。这个宫里的人,都奇怪得很。原本不奇怪的,进了这里也都成了怪物。''

她笑语自若,浑然不介意用这样锋利的语气来戏谑自己。就如她的妆容,明明可以将两翼增阔,微卷,如薄薄的蝉翼,便可遮住脸上的疤痕。可她偏不,大刺刺朝天露着,全然不在乎。

不过终究年轻,香见也好奇,``她到底瞧我做什么?''

如懿答得平静,``羡慕你的恩宠,是她毕生盼不来的福气。''

``啊!''香见恍然大悟,``皇上不爱她,对么?她对皇上,就如皇上对我。一厢情愿,真是没有意思。''她旋即笑得冷漠,``不过,也是咎由自取。我待他便如他待旁人。因果轮回,都是自己作下的自己受。''

香见说话间神色便不大好看,恹恹的,如懿便撇了话头,``楼下挪了好些沙枣树来,等到开花的季节,必定好看。''

香见冷笑一声,``皇上以为娜来这些沙枣花,便是我想要的了?所谓物离乡则变,沙枣树到了这儿,怎么腾挪也长不了。''她手边铺金酸枝木圆桌上供着一盆碧玺珊瑚玉雕花,她随手扯下几片玩儿,又撂下了,``方才才好笑呢。皇上好端端地派了个太医来说要为我调理身子,可以早日有孕。''

她说着,厉声冷笑,如泣血的杜鹃,神色凄楚欲泣。

那笑声让如懿心底发酸,``可是你侍寝多日,有孕也是常事。''

香见笑得前仰后合,``所以我问太医,我不要有孕,有没有不孕的法子,那个胆小鬼,居然吓跑了。''

那侍女听她这般口无遮拦,忙端了酸奶疙瘩和奶油馓子来奉上,赔着笑道:''皇后娘娘莫见怪,小主是与您亲近才这样直言不讳,当着皇上的面,小主并不这样,只是不大爱说话。''说罢,又频频向香见使眼色。

懂得护主,便是忠仆。

香见叹口气,只好忍下了,向如懿道:``我们寒部人爱吃这个,皇后娘娘喜欢么?''

如懿留意着皇帝极尊重香见的饮食,另辟了小厨房为香见单做,便取了一枚酸奶疙瘩吃了,``是极好的。皇上也顾念你。''

香见扬了扬嘴角,算是挤出一个笑。如懿抬了抬手,容珮便将手里的小棉托子打开,小心翼翼捧出那盏汤药来。

``你有你想要的,本宫也有不得不做到的。这碗东西,本宫是奉皇太后之命送来的。喝与不喝,在你。''

香见咬着指头,哧哧地笑起来,像是碰到一件极有趣的事,``怎么?我自己没死,太后也盼着我死了。这倒好,皇上总不会怪太后吧?''

如懿见她如此痛快,反倒难以启齿。她不得不深吸一口气,朗朗道:``这药要不了你的命,只是成全了你的念想。一口喝下去,再不能有所生育。''

香见在胸腔里长长地笑了一声,二话不说,端起汤盏便朝喉咙里灌下去。

她的动作过于激烈,汤药溅出几点落在她明蓝绣暗紫羽纹的衣襟上,像是溅出的几点鲜血,暗红地凝固着。她一饮而尽,尺阔的衣袖被漾起水面般纹纹波澜,有着一种决绝的洒脱与哀凉。

香见唇角一勾,目光灼灼注视着如懿,``我的肚子,只生我喜欢的男人的孩子,而他,不必了!''她漫不经心地嘱咐侍女,``那个太医走了没多久,去叫回来吧。''

那的确是一碗好药,见效极快。半个时辰后,香见便开始腹痛,血崩。如懿守在寝殿外,听着太医与嬷嬷们忙碌的声音,久久不闻香见一声痛楚的呻吟。

如懿坐在暖阳下,近乎透明的阳光落在秋香色的霞影纱上,那一旋一旋的波纹兜着圈儿,似乎要把整个人都卷到海底去。

她的整个脑袋都是空茫茫的。有宫女们跑进跑出的杂乱声,连服侍香见的侍女,看着她的眼光都带着怨恨。是,谁都看见的,是她光明正大带粉这碗汤药进来的。

沉默相伴的,唯有容珮。她握一握如懿的手,``皇后娘娘,事已至此,没有办法的。''

这话说的,不知是自己还是香见。如懿极力想笑一笑,才发觉舌底都是苦的。

皇帝来得很快,几乎带着风声。他并未注意到如懿亦在,只是急急冲进寝殿。很快,那阵风声便转到她跟前,她习惯性地起身屈膝行礼,面而来的却是一记响亮的掌捆。

他厉声喝道:``毒妇!你给她喝了什么?''他的话音在战栗,破碎得不成样子。

她的脸上一阵烫,一阵寒,到了末了,除了痛,便再也没有旁的感觉。

他从没有骂过她,也不曾弹过她一个指头。哪怕是最难堪的冷宫岁月里,哪怕是永璟死后,彼此疏远到了极处,都从未有过。他一直是眉目多情、温和从容的男子。

却原来,也有今日!也有今日!

如懿全身都在发抖,止不住似的,凭她几乎要咬碎了银牙,捏断了手指,用力得四肢百骸都发酸僵住了,都止不住。战栗得久了,她竟奇异似的安静下来。

日色是一块晶莹剔透的凝冻,也冻住了她。半晌,她涩哑的喉舌才说得出话来,``皇上,原来你我之间,已然到了这般地步?''她忍着痛,行礼如仪,``这碗汤药是臣妾拿来的,臣妾无话可说。''

皇帝满眼通红,几乎要沁出血来,``太医说香见再不能生了。你听听,她都痛得哭不出来了!''

如懿的嗓子眼里冒着火,烧得她快要干涸了,``太医说得没错。那碗药就是绝了生育的。''她顿一顿,呼吸艰难,``喝与不喝,是容贵人自己的主意。皇上为了她固然可以神魂颠倒,不顾一切。哪怕杀了臣妾,若能泄恨,臣妾自甘承受!''

皇帝指着寝殿方向,痛心得呼吸都滞缓下来,胸腔急剧地起伏着,``你知道她躺在里面,全是血!朕有多难过么?你明知道朕那么喜欢香见,若香见有了孩子,她会更懂得朕,跟随朕\ldots{}''

她的声音细细地发尖,刺痛皇帝不安分的神经,``可是许多事,是改变不得的!容贵人愿意留在宫里,愿意伺候皇上!可她的心,皇上终究是得不到!只是皇上自己不能接受,一厢情愿罢了!''

她脸上已然挨了一掌,不过是再挨第二掌,还能如何呢?他不过是这样,目光刀子似地割着她的皮肤,钝钝地磨进肉里,血汩汩地流。

她总是戳痛了他心底最不能碰的东西。可这话,大约天底下也唯有她敢产。这皇后的身份如此堂皇,肉身冠冕,可底子里痛着的,却是她如懿这颗心。真是可笑!

打破这死一般沉寂的,是太后威严的声音,仿佛是从云端传来,渺渺不可知,却是镇定了所有人的惊惶与错乱。太后捻着佛珠,扶着海兰稳步而来,缓缓扫视众人。海兰一进来便看见了如懿,但见她脸颊高起,红肿不堪,眼中一红,迅速低下头,立到了如懿身后。

太后苍老的身形显得威严而不可抗拒,``皇帝要的是寒氏,谁也没拦着你,你也如愿以偿。既然你从前就没提过要寒氏有孩子,那么哀家让皇后除去寒氏将来的孩子,也是无可厚非!''

皇帝不敢抗拒,嘴唇微微张合,如涸辙之鲋。太后徐徐坐下,``皇帝,你想说的哀家都知道。你有多痛心哀家也看见了。但是非我族类其心必异,与其来日寒氏生下孩子频起风波,不如让她清清静静一个人,得了你的宠爱,也绝了满宫殡妃的怨怼。''

太后的话无懈可击,皇帝只得低头,双眸浑浊,答应着``是''。他努力挤出笑,眼睛却觑着如懿,``皇额娘久不理宫中事了,怎么也在乎起香见的事了。''

太后何等精明,如何不知皇帝所指,``倒真不是皇后来告诉哀家的。哀家只有皇帝一个儿子,自然是皇帝在乎什么,哀家也在乎什么罢了。只是哀家有句话不得不说,有时候爱之适足以害之。皇帝,若无你的过分沉溺,本无人在意寒氏的生死荣辱。你的宠爱太过煊赫,才把她逼到了绝处。''

皇帝的脸上蔓生出一种近乎颓废的惘然,他缓缓摇头,``纵然皇额娘心意如此,但这碗药到底是皇后端来的。她是中宫,是六宫之主,母仪天下,如何可以做出这种绝朕后嗣之事?''

太后朗然自若,``药是哀家给皇后的,喝下去是寒氏自己的主意。皇帝要怪,只能怪自己拢不住寒氏心甘情愿为你生下孩儿。''她说着,霍然捏住皇帝的手腕。皇帝一时不防,骤然吃痛,痛得眉毛都拧作了一块儿。太后松开手,轻轻替皇帝吹了吹伤处,和颜悦色道:``你是哀家的儿子,若不是心疼你,心疼你的名声,也不致如此。''

皇帝矍然变色,目光狐疑,但见如懿只定定对视着他的目光,毫无退俱之色,他忽然添了几分心虚的委顿。看向身后小太监们的神色多了一丝凌厉。海兰见皇帝僵持不豫,捧过一盏茶水奉上,``皇上别急,有什么话慢慢说。太后也是关心您呀。''

皇帝略略缓和,接过茶盅润了润起皮的嘴唇,轻咳一声,``皇额娘所言极是。宫中所有是非,皆因妒忌争宠而起。儿子深觉嫔御之流,得空得多学学愉妃。愉纪安分守己,从不争宠,也不妄生是非。''

这话便是打如懿的脸了。他看她,也不过如此,将她视作妒妇一流。

海兰听得皇帝隐隐之怒中对她犹有褒赞之语,也不过谦柔一笑,宁和如常,``皇上夸奖,臣妾不敢承受。臣妾谨遵嫔妃之德,不敢逾越。''她恭谨行礼,柔和中不失肃然神态,``不过皇上,皇后娘娘心系皇上,才会出旁人不出之语。这不是皇上一直赞许皇后的长处么?''

这话柔中带刚,皇帝一时也无言,倒是寝殿里喊了出来,``容贵人醒了!醒了!''

皇帝所有的怨与怒在这一刻被浑然丢下,他急匆匆入内,浑不见太后暗自摇首的黯然。底下的太医、奴才们跪了一地,看着苏醒过来的香见,如逢大赦一般。

皇帝搂住她的肩膀,又不敢箍着怕弄疼了她,只得抽了手由侍女替她擦着脸。香见的眼是空茫的黑,望着帐子顶儿,轻轻抚着肚子,``我是不能生了,是么?''

皇帝落下泪来,紧紧摇着她的手,想将手心的温热缓过她的虚弱与冰凉,``香见,你别怕,只是没了孩子而己\ldots 朕会好好待你\ldots 朕\ldots{}''语未毕,他已泪流潜然。

香见的脸容逐渐安详,她仰起身子来,像一片抽尽了水分的枯叶,轻飘飘地捧在侍女们手上。她的声音飘忽无力,仿佛随时就会断绝,``那碗药,是我自己要喝的。生与不生,我自己定。''

皇帝的脸迅速白了下去,那种白,是冬日的残雪,带着积久的尘埃的浊气,隐隐发黑。他的嘴唇都在哆嗦,不知是愤怒还是伤心。海兰快意地撇了撇嘴,着意去看如懿的伤处。

香见望着他,神色柔和了几许,``皇上,我本不该来这个宫里,更不该得你的宠爱。你就当我无福,承受不起。我来日的孩子,更承受不起。你要我伺候你,我便清清净净伺候你一辈子便是了。''

寥寥几语,是无限的伤感与灰心。

皇帝错愕地看着她,渐渐委顿下来,``你的意思,皇额娘的意思,朕都明白了。朕会克制对你的爱意,尽量不去伤害你。''他霍然起身,在那一瞬迅速恢复了往日的从容与决断。

``李玉,传旨下去。着容贵人晋容嫔,令妃晋令贵妃,颖嫔晋颖妃,庆嫔为庆妃。皇后倦乏,力有不逮。后宫诸事,交由令贵妃权宜协理。''

如懿定定地站在那里,任由热泪在眼眶里一点一点咬啮着,终究不肯,不肯落下一滴。

\hypertarget{ux7b2cux4e5dux7ae0-ux6885ux8fb9ux5f71ux8fb9}{%
\chapter{第九章
梅边影边}\label{ux7b2cux4e5dux7ae0-ux6885ux8fb9ux5f71ux8fb9}}

冬天是什么时候来临的,如懿根本没有察觉。举目望天时,见整个紫禁城都己是冰雪琉璃世界,才知心境的悲寒,已与这白雪冬寒没有半分区别。

因着嬿婉素性爱热闹鲜艳,自协理六宫,连红墙飞檐都不寂寞。各色水晶琉璃风灯点得如银花雪浪,连落尽黄叶的枝干上都悬满了小儿手掌大小的橘灯,配着绿绸剪的叶子,红红翠翠,上下争辉,真是琉璃堆簇世界,锦绣风流。

冻云飞雪,唯有翊坤宫红门深掩,独遗世外。寒风料峭透冰绡,香炉亦懒去烧。拥着白腋紫貂毳衣,独倚榻上,捧了一卷《清静经》翻阅。

已然到了下学时分,永璂还未回来。容珮进来挑了挑火盆里的炭,看它又迸起几星红光,方搓着手道:``这个时辰还未回来,伺候的人也没来回禀一声,十二阿哥今儿怕是又在皇上那儿用晚膳了。''

如懿``嗯''了一声,便也不答。

容珮自己给自己找话儿:``皇上虽然冷落了娘娘,对十二阿哥却越来越热络,也常带在身边,也是好事。''

殿中静极了,只听到指尖与书页相触的微声,嗒一下,又一下,是委地的落花,坠进心里一阵阵发颤。容珮叹了口气,道:``娘娘素来不爱看这些书,这几日倒不肯放手。''

``这书不好么?''如懿的平静让人发寒,仿佛是落入寒潭的人,不挣扎,不呼喊,只是静静,静静,沉溺下去。

容珮不作声,只是叹了口气。如懿笑影清浅,``你跟在本宫身边,旁的没学会,倒学会了叹气。''

容珮红了眼圈,伏在如懿身边,``娘娘苦了自己了。''

如懿讶异,定定看着她,``一本书而已,你何来这种喟叹。《清静经》甚好,讲求的是老子的`清静无为',认为人若能清静,即可得道,住世长年。而获得清静之法,唯有观空。本宫如今的际遇,看看这样的书不是很好么?''

容珮无言,只得立起身来,``等下愉妃小主还会来陪娘娘用膳,奴婢先去预备着。''

如懿颔首,``小厨房还照应得过来么?内务府有无克扣?''

容珮正要答,只见福寿弹花锦帘一掀,海兰领着忻妃进来,笑吟吟道:``怎么会克扣?令贵妃协理六宫,施恩上下,无不妥帖。''

忻妃病色不减,一袭茜色罗遍绣锦袍穿在身上,又虚虚地空了一圈,精心刺绣的缠枝海棠云纹更有种繁漪涟动的华美。她摘下藕荷色遍地洒金碧纹湘江大毛斗篷交在宫女手里,抱着一个珐琅花鸟紫铜手炉在如懿身畔坐下。她笼着发髻,额上一抹水莲色滴珠水獭抹额烁着星子曳金的微光,正中一块拇指大的金丝猫儿眼,幽蓝深海之夜的浑圆一颗,晃出一隙碧水波澜微漾的光芒,添了她面上一丝甜柔之色。

如懿道:``这抹额的样子好俏皮,又暖和,最合你如今用。''

忻妃衔了一丝冷笑,``半个月前令贵妃着人送来的。说是内务府新出的样子,又暖和又精致,特特来送了臣妾。臣妾起先还不肯戴,不知皇上怎的知道了,还问了臣妾一句。所以今日特意戴着来四处招摇,也好成全令贵妃的贤名。''

海兰温然笑道:``可不是,那么大一颗猫儿眼,令贵妃说是波斯的贡品,病人戴着相宜,便特意缀上了给忻妃妹妹。''她说着卷起紫棠色遍地锦的袖子,露出一对金丝镶粉红芙蓉玉镯子,手镯三节,以嵌翠环并粉红玉制成芙蓉花瓣式,色色俏丽,中嵌东珠一颗,如芙蓉花蕊,明耀华灿。海兰轻嗤一声:``永琪在皇上跟前得脸,令贵妃便也送了臣妾这样大的礼。''

如懿合上书卷,轻笑,``她如今越发圆滑,可算历练出来了。''说着又看忻妃,``你身上一直不好,怎么还出来?外头风雪大呢。''

忻妃俏脸一板,曳得鬓上双耳同心玉芍药花钿映着烛火一闪一闪,花瓣下坠着长长一串金累丝攒珠宝石流苏,在耳侧晃悠悠。她哼道:``臣妾偏要来,省得叫那起子小人看笑话,以为翊坤宫怎样了呢。''

如懿本自郁郁,听得她这样说,也掌不脾气道:``都是做额娘的人了,还这么个脾气,真真是宠坏了你。''

忻妃眉心一黯,垂下脸来,``从前是刚入宫不谙世事,才什么都不怕,如今左右是明白了,只要臣妾的阿玛在,无论臣妾病成什么样子,皇上都是眷顾着臣妾和的。既然如此,臣妾又何必对小人嬖妾假以辞色?''她唤来宫女,喜盈盈道:``臣妾宫里新制了几道小菜,是暖身补气的,冬日里用最好。''

说着三人便坐下来,由着宫人们侍奉着用了晚膳。

如懿不是不明白,自己的落寞,难免要被人轻鄙,若不是忻妃和海兰常常往来,顾着她皇后的颜面,还不知要被人轻贱到什么地步。到底,忻妃有着家世,有着军功,海兰有着永琪,无人敢轻看了她们去。

可是她的永璂是越来越远了。

起初,不过是常留在皇帝身边用午膳,渐渐连晚膳也留着。往来相送,是熟捻的凌云彻并几个小太监。

凌云彻请了安,便道:``皇上待十二阿哥极好,娘娘安心。''

她听得出凌云彻话中的安慰,永璂,是她的指望。

于是便在无人时问永璂:``皇阿玛除了问你的学业,还问什么呢?''

永璂天真地望着她,``皇阿玛问五哥好不好?因为五哥常给我讲书,也教我射箭。皇阿玛还经常考我学问,可是\ldots 可是\ldots{}''小小的人儿有些不好意思,``皇阿玛说,五哥在我这个年岁,己经可以写很成文理的文章,还可以连射三箭中靶心了。''

他有些气馁,如懿捧着他的小脸,爱怜道:``永璂,在你出生前,皇额娘只盼望你身体康健,品行端正。至于能否成为不世之奇才,从不是皇额娘的指望。所以你也无须自怨自艾。''

永璂瞪着黑白分明的眼,欣喜道:``皇额娘,您真的不觉得儿子蠢笨?''

``你不是蠢笨,是你五哥天资聪颖,但也无须人人都像他一样。永琪有永琪的好,你也有你的好。比如皇阿玛赏你的白玉霜方糕,你便记得皇额娘喜欢,留给皇额娘吃。''

永璂连连颔首,``是啊,我记得皇额娘不喜欢吃青梅丝的,可不知怎的,以前御膳房的白玉霜方糕都是不放青梅丝的,现下都放了。所以我给皇额娘的,都是把青梅丝剔了的。''

如懿微微一怔,容珮已然反应过来,咳嗽了一声。如懿抚着他的脸道:``好孩子,皇额娘有时候真的很怕,很怕自己对你怀有越来越高的期待,而忘记了刚行到你时的愿望。皇额娘只希望你一生平安顺遂。所以你不必事事都和永琪比较。''

永璂道:``那皇额娘也是很喜欢五哥的,皇阿玛也喜欢。''

如懿轻笑,``是。你五哥小时候一直养在皇额娘身边,与你的同胞兄弟无异。''

永璂重重点头,``嗯。可是五哥如今来得少了呢。''

容珮听他这般说,忙道:``十二阿哥,您快睡吧,时候不早了呢。''说罢,便唤了乳母嬷嬷进来,抱着永璂走了。

烛芯爆起一朵亮烈的花,骤然明焰,旋即黯然失色。殿中暗了下来,容珮见如懿静坐着不语,轻叹一息,拔下发髻上的银如意簪子剔了一剔,那火焰又亮了起来。容珮道:``皇后娘娘,五阿哥是有许久不大来了,虽然东西照常送来\ldots{}''

``明哲保身是宫中的处事之道。永琪的前景还不明朗,无谓为了本宫惹上是非,且愉妃不是常来么?''

容珮静了一刻,指着荔枝纹素蓝碟中的白玉霜方糕道:``难为十二阿哥的孝心,只是皇后娘娘最爱吃白玉霜方糕,御膳房又何必为了讨好令贵妃撒上这许多青梅丝,故作矫情?''

如懿静静道:``跟红顶白乃是宫中风气,连本宫喜欢的东西都要讨令贵妃喜欢,可见令贵妃得宠。好了,只要永璂孝顺,本宫还有何求呢?

容珮掠了掠鬓边碎发,叹道:``如今令贵妃显赫,本以为皇上会格外疼爱容嫔呢,原来到手了也不过如是。''

如懿不言不语,只是想着那日海兰来时,所说的话语。``皇上赞我贤惠不醋妒,姐姐也实在不必往心里去。皇上这么说,不过是拿着我激姐姐罢了。''她黯然神伤,``其实宫中谁人不知,我的身子,便是想争宠也不能的。皇上也是,拿我们姐妹之间的情分做筏子,又有什么意思?''

如懿向来与海兰不分彼此,便道:``你见事从来明白,所以在宫中多年,平稳无碍。不比我,起起伏伏,终究无定。''

海兰端详着她,心疼道:``姐姐,我和你不一样。我从来不喜欢不太稳定的东西,比如男人的感情,比如荣宠。我在意的,信任的,都是确定的不会轻易变化的,就像我和姐姐长久以来的彼此依靠,就像我和永琪之间不会变更的血缘。''

情意固然会变化,便如从前深爱之人,也可渐成陌路。而永琪的疏远,虽然微不可察,可她毕竟抚养了永琪十数年,又如何全然不知。毕竟,她与永琪,从无那般深刻的血缘。而逐渐长大的永璂,虽然不够聪颖敏慧,但也是个乖巧的孩子,又占着嫡子的名分。永琪,怕也是介怀的吧。

怔松间,人情的冷暖如冰雪沁冷,逼入心间,她看着格花六棱窗外一钩新月,白霜霜的,月头尖利如银钩玉划,生生划进眼底,却勾不出半点泪意。

于是,她镇日只是坐在这里,看天光东起西坠,无声流转。日色也好,雪光也好,都是与她最亲密不过的。不会因为际遇的改变,更改一分亲近。而白日过去,夜色照旧而来。大约紫禁城中不分高低贵贱,肯一视同仁的,也唯有它们了。

人言嘈杂,无不是是非之处。如懿渐渐不大出去,也免了嫔妃们的请安之礼。便是太后,亦觉着雪天路难行,免了她的晨昏定省。

倒是那一日,京中最早的一场春雪停止,如懿忧心着雪后难行,放心不下永璂,便远远出去迎着。过了翊坤宫便是永寿宫,再往前便是皇帝的养心殿。行经时听得永寿宫内按歌之声,门前轿辇齐集,便知是嫔妃们都在永寿宫相聚取乐。

容珮轻轻啐了一声:``正经皇后娘娘还在呢,却把令贵妃当成了主子,刚下完雪也赶来凑热闹。''她的声音略低,``听闻,令贵妃刚有了一个多月的身子。''

这么快又有了身孕,真是圣眷正隆。难怪这般鲜花着锦。

如懿不愿多停留,只道:``咱们去螽斯门外等候永璂便是。''

才行至螽斯门,便有扫雪的小太监请安,道:``启察皇后娘娘,十二阿哥听凌大人说御花园的迎春花开了,说要折雪中迎春送给娘娘,己经往御花园去了。''

如懿又是心疼又是感动,嗔道:``这孩子,也不怕雪地里滑。''说着,便往御花园去。

雪野茫茫,天地间静无一人,只听得足下珠履踏着积雪之声。白雪素光之中,果有迎春点点鹅黄,似疏落的金黄的星子。有欢快的童声响起,唤道:``皇额娘。''

她心底一软,似要化去。循声望去,果见凌云彻抱着永璂,缓步过来。永璂的小脸冻得微红,一手抱着一束尚带雪珠的迎春,一手挥着。贴身的小太监们跟在后头。

凌云彻放他下来,向着如懿行礼。永璂笑呵呵道:``皇额娘,儿子知道您喜欢梅花,可是冬梅快谢了。凌云彻说迎春金黄,与腊梅肖似,儿子便想折来送您。''他有些怯怯的,``虽然雪后寒冷,但凌云彻照顾得儿子很好。皇额娘,我真的不怕冷。''

如懿虎着脸,本想吓吓永璂但听得小儿娇声软语,哪里还狠得起心肠,便道;``那你要多谢凌大人,肯陪你做这些小儿把戏。''

三宝见得永璂的猞猁皮袍下沾了大块春雪,那春雪比不得冬雪坚冷,一触便化,不经意便沾湿了衣衫。他忙抱过永璂,道:``好阿哥,奴才带您去养性斋理一理衣裳。还有这迎春,都是雪珠子了,等下化了冷着您。''他说着,便领了小太监去,只留容珮远远陪着侍候。

天地间是如此深深寂静,可以听见雪落枯枝的声音,清泠泠的,细碎的,绵延不断,此起彼伏。

如懿先自笑了:``没想到时隔数年,本宫又落得如此惨境。是不是似曾相识?''

凌云彻默然片刻,``可惜冬日过去,微臣已经没有梅花可送。''

如懿轻轻一笑,那笑意薄得像天际淡淡的浮云,很快便会被风吹散,``梅花再能傲霜雪,也有零落成泥碾作尘的时候。即便你送来一冬梅花,本宫也会在下一个春夏秋冬过着无宠萧索的日子。''

凌云彻的目光仿若无意扫过她的面孔,很快低首垂眸,``梅花易谢,终难长久。微臣不会再送这个了。''

``也对。你如今侍奉皇上劳碌,又要替本宫接送永璂,实在辛苦。''如懿拨弄着指间初开的迎春,那星星点点的鹅黄,柔嫩动人,``何况本宫从来就不是高洁的梅花,是你误会了。''

凌云彻眸中澄澈清定,坦然而望,``或许皇后娘娘不是风霜高洁,但微臣看见的是你求存的冰雪寒霜之地。''

眼底有温热一溢,她居然会为了他的话,湿润了枯涸的眼。

他停一停,从袖中抽出一卷小小短轴,交于容珮手中,``微臣从未学过画画,勉力学了一冬,才会这个。还请皇后娘娘莫要见笑。''

她将他眼底的渴盼清晰映入心间,沉吟片刻,还是伸手从容珮处接过,徐徐展开。她的手极美,与卷轴的雪白之色不相上下,融若清霜。她纤长的指以一种清艳姿态停驻在紫檀轴上,像一朵盛放的杜若。

那是一卷墨梅图,临幕的是宋人画梅的意境,用浓淡相间的水墨晕染,疏枝浅朵,珠蕊隐现,倍觉孤条遒劲,风神绰约。那笔触似是练习了无数遍,但仍有稚拙的痕迹,显然是新学不久。便是永璂,也可画得更好些。

她想笑,心底却无限酸楚。他端庄的眉目间,衔着的一丝温默的柔软,轻染了坚毅的从容。他唇际的笑容是时雪后初霁的天空,碧澈澄清,那份关切,一览无余。

不知怎的,她忽然想起闺中时光。晨风细凉,庭院中赤红芍药盛放,饱满的花盘慵慵欲坠,每一朵都是重绡叠绢,盛开得不知天地何处。金色的阳光从朱红色的阁子边流过,她抬起手,遮住肆无忌惮漫入眼帘的几束阳光。绣楼下,额娘在赞许花开当时,唤她折来簪鬓。她笑着答允,回眸去,云朵洁白,天色湛蓝。

她在冰雪之中,忽而有那样安闲的心境。仿佛少年之际,身边的关切来得自然而真心。

是有多久,没有过这样的体会?步步为营,步步惊心,如履薄冰的日子,已经太久太久。

思绪的流转,莫名地牵动着心肠。她看着他暗红色的斗篷,寻常的御前侍卫的样色,深蓝色的袍角微露一痕,在下蕴蕴漾漾,闪着幽微的光,细细迷离。世事原是如此,不过咫尺的即离,你也明知他的好,但他同你永远没有半分干系,就如隔着银汉迢迢,牵不到,挂不仁。所有的相知,都在滔滔流年的浊浪里,缱绻着流过去,流过去,永无交集。

她转过身,避开他的目光,走远两步。在侧身时举起袖袂,以不经意的姿态掩去一星溢出的泪光。

她恍然惊觉,他对自己的情意,恰如青翠竹叶上脉脉延伸的纹理,细微,却清晰可见。

如懿收起卷轴,交至容珮手中,轻声道:``多谢。''她觅一话头,来疏散此刻的心绪繁复,``皇上常往宝月楼去么?天寒路远,皇上须得小心才是。''

礼数是最刻意的距离。凌云彻退开两步,回复往日的恭谨节制,``皇后娘娘心念皇上,微臣回去自当回察。不过娘娘放心,皇上己不似从前,两三日才去宝月楼看容嫔小主一次,三五日才翻一次牌子。''

心底的讶异突兀而出。这些日子来,她未曾过问皇帝行踪,也无人来告知,唯有容珮的只言片语,才知皇帝少去。原来再狂热的爱慕,也有自然熄止的一日。

凌云彻看清她眼底的疑惑,又道:``皇上还是很宠爱容嫔小主,便是说宠冠六宫也不为过。只是皇上偶然说起,怕再如从前这般情不能已,是害了容嫔小主。所以如今也常往各宫走动,也算雨露均沾。''

``过分之爱,亦是过分之害。''她一语轻漠。若是皇帝明白,他与她也不至今日。

凌云彻拱手道:``娘娘安心,皇上已然明白。想来娘娘雨过天晴之日,亦不远了。''

如懿恍然明白过来,``所以你让永璂送本宫迎春,是迎来春禧之意么?''她见凌云彻颔不觉惘然失笑,``不会的。凌云彻,一个男人,是不喜欢身边的女子见过他最失态的模样的。何况他己然清醒,会更厌恶本宫的亲眼所见、亲耳所闻。''

她旋身,不忍将他的失望尽收眼底,``不过还是多谢凌大人照颐好永璂。对了,永琪也常去养心殿,对永璂可还好么?''

``兄弟情深,叫人羡慕。''他一顿,还是道,``可是比之往日,总有不如。也不知是否是皇上常将十二阿哥带在身边的缘故。''

如懿涩然,亦不便再言,眼见三宝带了永璂回来,便也离去。

那一厢天寒雪冻,殿中却和暖入春,嬿婉见缤妃们一壁取乐罢,都尽兴走了,方才困倦地蜷在酸枝木九节樱花杨妃榻上,拥了一袭紫貂暖裘。天云晦暗,暮色沉沉,仿佛又有一场大雪要落。暖阁里摆着两盆大红的宝珠山茶,浓绿欲滴的叶片间镶嵌着一朵朵殷红如醉的花,如正春风得意的美人面。嬿婉套着藕荷镶赤红、宝蓝、赭金三色宽边的锦袍,袖口露着春葱似的指尖,她百无聊赖,道:``都说来看给本宫道喜,闹了一晌才肯去,真是乏人。''

澜翠甩了甩辫子,抿嘴笑道:``小主新封贵妃,又生下十五阿哥。这是双喜临门的大喜事。''

春婵抱了十香烷花软枕上来,``小主拿软枕垫着,舒服些呢。''

嬿婉娇滴滴地嗔着,一张白皙娇艳的面庞妩媚地侧了侧,道:``哪里就这么娇贵了,生完都三个月了。''

澜翠嗓门敞亮,``哪里能不娇贵呢?皇后形同虚设,宫里最尊贵的便是小主。如今您正炙手可热,皇上多宠着您哪,连容嫔那么得意,也冷了下来。''

呵,这真是一生里最畅意的一段日子。旧爱已然落下,新宠也未能威胁她,初尝权力滋味,甜蜜如醉。孩子一个接一个地出生,都是依傍。她从未这般痛快过,不必畏首畏尾,随着自己的心意摆布一切,自有人山呼簇拥。难怪,一个个顶着花般面孔,竭尽全力,不管姿势是否好看,都要爬上这山巅来。

果然顶上风光,是难以细述的美好。

但,总还是有点阻碍,譬如,翊坤宫那人,终究是这个紫禁城的女主人。她还是侍妾,战战兢兢,守着礼仪尊卑,要对她俯首屈膝。

春婵见她神色不大好,便来打趣:``小主可知道,婉嫔真是痴心。这么冷的天,只要皇上经过她宫门外,她必定仰首企盼。唉,年岁大了还一股子痴情,真真可怜。''

看,这便是宫里,痴情的身段摆出来,也得顶着一张如花似玉的面孔,否则便落了笑话。也真是唇亡齿寒,兔死狐悲,年华逝去,若无一点依傍,便生生成了他人的谈资,徒增笑料。

澜翠替嬿婉掖好貂裘,那紫红滟滟的皮子好似盛开的一簇绮丽繁花,映得她面庞亦带了一抹沉郁的华贵气息。她的手指上缠着髻后散落的一束柔娆青丝,抿唇轻笑。一个女子,当真是要男人的疼爱,才养得出温柔华贵气来,否则,总是苦相,显得鄙薄。但,她心底到底生了一丝鄙夷,轻轻咬着牙道:``到底是没本事留住皇上的心。''

澜翠``咦''了一声,``小主是说皇后娘娘么?''

春婵横她一眼,满面堆笑,``婉嫔是,皇后也是。小主,如今皇后势单力薄,皇上又誉顾小主。有些枕头风,您多吹上一吹,皇后要爬起来也难了''

嬿婉的笑容和缓而温柔,仿佛晨曦中一朵初绽的浅浅粉红的花,让人见之不由得生亲近之情,却与她此时口中的冷漠并不相符,``敢于直言,懂得进言,是皇后一直以来的优点,也是皇上引以为信任的由来。只是一个人的优点,放在外头,自然是一辈子的好处。可是进了宫里,再好的优点,也会成为弱点。''

春婵蹙着眉头,拢一拢手腕上的虾须点珠银鎏金镯子,``可是若要皇后娘娘离开六宫之主的位置,小主却不能不向皇上进言。都是刮耳朵的风,只看小主怎么吹了。''

嬿婉的笑容倏然收住,僵在唇边,凛然有杀气,``本宫年轻的时候也犯过这样的错,以为自己的话能打动皇上。后来发现,并非本宫说的话有多好,而是正合时宜而己。但一时说得不合宜,却给自己带来无限的辛苦与麻烦。所以本宫学了个乖,以后再不多言了。不说,才不会说错。''

春蝉与澜翠对视一眼,讪讪低首,``可是所谓杀敌制胜,若不出手,机会便过了。''

嬿婉慵慵地侧身,发髻上一串双尾攒珠凤钗,凤口上垂落的红珊瑚珠子坠着薄薄的赤金云头,柔柔地散在青丝之上,温柔旖旎。她倦得很,``本宫乏了,这些日子也不便侍寝,便成全了婉嫔吧\ldots{}''她的声音渐次低下去,忽然嗅到什么气味,凤眸倏然睁开,呵斥道,``谁摘了腊梅来,一股酒气,好生难闻!''

澜翠悚然一惊,忙回头去寻,春婵好生劝慰道:``小主最不喜梅花,无人会摘来。''澜翠忙碌片刻,终于在供着的清水瓮里寻到几朵风干泡着的腊梅,苦笑道:``定是底下奴才疏忽,想添水中清气,才不小心加的,奴掉立刻撤换掉。''

嬿婉这才平伏了气息,道:``冬日少花,可养水仙与茶花,记得不许梅花入我永寿宫。''

\hypertarget{ux7b2cux5341ux7ae0-ux6545ux5251}{%
\chapter{第十章 故剑}\label{ux7b2cux5341ux7ae0-ux6545ux5251}}

日子还是这般缓缓过着,冬去春又来,时光的循环往复,无声无息。不经意间海棠深红,是风不鸣枝、云色轻润的初春。呵,又一年好景。这一次的冷淡不同于往日,如懿渐渐发觉,永璂留在翊坤宫的时间越来越短。除了上书房,除了学骑射,剩余的时间,他多半留在了养心殿,随在皇帝身边,习文修武。

这原是好事,如今却让她觉得惶恐。

永璂的默默远离似乎是无意,却又按部就班。

偶尔永璂回来,看到玉净瓶中已然枯萎的迎春花枝,便哧哧笑:``皇额娘,御花园中的牡丹、丁香、玉兰都己经开了,儿子再折了新的来。这些枯萎的花枝,便不要留了。''

如懿捏一捏他滚圆的小脸,笑道:``迎春虽然枯萎,但皇额娘想留住的是你的心意。对了,最近皇阿玛留你在养心殿做什么?''

永璂打了个呵欠,忙忍住,``皇阿玛请了新的师傅和谙达,给儿子教习骑射和满汉文字。可是皇额娘,我好累呀。我每日都睡不够。''

如懿心疼,却又劝不得,只好道:``好孩子,尽力而为吧。实在不能,便告诉皇阿玛。''

永璂怯怯地摇头,``皇额娘,儿子不敢。儿子怕皇阿玛会失望。''他握一握拳,``儿子会努力学好的。''

如懿搂着他,默然无言。

很快,凌云彻与小太监们又过来,领着永璂回养心殿。如懿无可奈何,倚门目送永璂走远。

容珮进来道:``皇后娘娘,再过十来天便是孝贤皇后的死忌,宫中主持祭祀,您可去么?''

如懿缓声道:``自然去。不去,便又是一条醋妒的罪状。''

容珮颔首:``也好。方才奴婢去内务府取春日要换的帐帷,见婉嫔与令贵妃出入长春宫,倒是难得。''

如懿微蹙春山眉,``婉嫔是个老好人,但也不大和令贵妃来往,怎么一起去了长春宫?''

容珮道:``或许令贵妃协理六宫,今年祭祀孝贤皇后之事,会做得格外好看些。''

这份疑惑,数日后海兰来探望她时,便得以解了。海兰也颇诧异,道:``姐姐知道么?这几日侍寝,居然不是令贵妃也不是容嫔,而是婉嫔呢。入宫数十年,倒从未这般得宠过。人人都说,她与令贵妃往来数次,便得了皇上的意,定是令贵妃在皇上面前多多提了婉嫔的缘故。''

如懿见她笑意清湛,有戏谑之意,便道:``你也不信,是么?''

海兰掩袖道:``还是永琪细心才在养心殿留意到,原来孝贤皇后忌日将至,婉嫔将皇上多年来悼怀孝贤皇后之诗整理抄录,集录成册,在养心殿和长春宫各奉了一本。''

``那么如今,该是宫中追怀孝贤皇后成风,以期得到婉嫔一般的重视了吧。只是婉嫔,不似会动这般脑筋之人?''

海兰叹道:``娘娘何苦这般聪敏,的确是令贵妃指点的。只是您以为令贵妃这般苦心孤诣,只是为了捧婉嫔得到几夕恩宠么?''

``婉嫔温顺软弱,一心渴望得到皇上爱怜。她这一生,也算孤苦。令贵妃自然明白她想要得到什么。宫中思怀孝贤皇后恩德,自然事事拿本宫与孝贤皇后相较,本宫这个皇后已然失宠,便更无立锥之地了。''她顿一顿,``看来经历世事挫磨,令贵妃老辣了许多。''

海兰轻哼一声,不以为然,``皇后终究是皇后,哪怕前头有许多个,人死不能复生。只要姐姐活着,谁也夺不走您的后位。''

如懿微微怅然,``是么?死亡固然能夺走后位,但皇上的庆弃也会。你可忘了,顺治爷的博尔济吉特皇后,不就是被降为静妃了么?''

海兰的眼底闪过深深的惊痛,急忙捂住她的嘴,``姐姐不许胡说。''

不说又如何,事实在眼前,总不能装作眼瞎耳聋,糊里糊涂过日子。

婉嫔誊写的诗稿,适时地勾起了皇帝对孝贤皇后的思念,连带着宫中嫔妃,都对故世的琅嬅称颂不已。因着如懿的不足,她的不知勤俭,她的不解人意,她的醋妒嫉恨,孝贤皇后不出一言违逆的温柔成了皇帝莫大的追思与缅怀之德。除了对富察氏家族一贯的厚待,傅恒的青云直上,孝贤皇后子侄的青眼有加,同为富察氏的晋贵人亦晋位为晋嫔。而闲来无事,皇帝也常往长春宫中,睹物思人。

这仿佛已经是一种习惯。连和敬公主归宁,亦哨叹不己,``这般情深,若额娘在世时便享到,可谓此生无憾。''

话虽这样说,如懿到底还是皇后。失去了权柄与宠爱,名位尚在。

亲蚕日的前一日,按着往年的例子,如懿自然是要领着六宫嫔妃前往亲蚕,以示天下重农桑之意。所以她必得来皇帝宫中,向他讲述明日亲蚕礼上要做的事宜。这是惯例,她也只是循例言说,并不需与他相对许久。

可是步上养心殿的台阶时,才知皇帝并不在。候着的小太监很是恭谨,告诉她皇帝会很快归来,请皇后耐心略等。

似乎没有一定要离开的理由,她也并未打算过于去拂皇帝的面子,便安然推开殿门,静坐于暖阁中等待。

春阳和暖,是薄薄的融化的蜜糖颜色。望得久了,会有沉醉之意。她坐在暖阁里,看着曾经熟悉的每日必见的一切,只觉得恍如隔世。黄杨木花架子向南挪了一寸之地,紫檀书架上的书又换了好些,白玉和田花槽换成了紫翡双月垂珠花瓶。

还有一沓新誊写的纸稿。

如懿随手一翻,眼神便定在了上头,挪不开半分。她认得,那是婉嫔的字迹,誊的是皇帝的诗。可那上面的每一首,每一行,每一字,都是关于另一个女人的情意。

日光一寸寸西斜下去。如懿坐在暖阁里,一页一页静静翻阅,身上寒浸浸地冷。指尖上流过的,是皇帝如斯的情意。

她一直知道他的愧疚,他的思念,他的结发之情。却不想,那人在时薄薄的情,历经时间温柔地发酵,竟成了浓浓的追忆,再不可化去。

``谒陵之便来临酹,设不来临太矫情。我亦百年过半百,君知生界本无生。''

她轻轻地笑了出来。想起从前的新琴旧剑之诗。

``岂必新琴终不及,究输旧剑久相投。''

连她自己也想不到,看到这一卷卷深情厚谊一刻,心中的难过如百丈坚冰,只能由着自己落下去,落下去,眼睁睁落到不见底的深渊去。她却居然还笑得出来。

原来最难过的一刻,竟然已不是此刻。是永璟死后他的冷淡与疏远,是香见再不能生育后他的厌恶与抗拒,让她居然习惯了这种浩浩愁、茫茫悲,任凭心底绞肉似的搓着,亦能沉缓了呼吸,一字不漏地看完。

舍不得不看,忍不住不看。

字字分明,哪怕从前也有耳闻,但一直不肯去听,不肯去看,到如今到底是成了落在眼底的灰烬,烫得疼。其实,一直到金玉妍死后,如懿才觉得愧悔,觉得自己可笑,原来与富察琅嬅缠斗半世,到后来连自己也不分明,到底是落在谁的彀中。

待到明白时,己然半生都过去了。

于是,琅嬅便成了皇帝心底的一朵伤花,带着血色,盛绽怒放。她的一生,她活着的时候,都未如她死去之后,这般深深地铭刻于心。

琅嬅,她终究是如愿以偿的。

要她看见这些的那个人,一定也很失望吧。那个人,是多么希望看到自己的愤怒与眼泪。

而她居然能笑,笑得凄然欲泣,却无半滴眼泪。

原来一个人难过到了极处,是可以没有眼泪的。而这样的难过,一而再,再而三。若真泗泪傍沱,呼天抢地,只怕连一双眼化作流泪泉都是不够的。

如懿终于看完了最后一个字,从天下皆知的《述悲赋》,到许多连她都从不知晓的只言片语,绿衣悼亡。她听得见自己的呼吸,细弱、悠长、绵软,续续断断

她抬起头,才惊见那一袭天青色玄线蝠纹长袍,生生撞疼了她的眼。

她竟未察觉,他是何时进来的。她也不敢去想,他是以何种神色,端详着她看着自己的夫君对另一个女子的情深意切。

多年礼数的教养,比她的心思更顺从而自然。如懿起身,行礼如仪。

皇帝的语气听不出任何端倪,神色冷冽如冰。不过这一向日子,他偶然见到她,便是这般面孔,倒也寻常。

李玉的脸早吓白了。大约从方才进来,皇帝便不许他出声。皇帝坐下,拐了口李玉奉上的茶水,蹙眉道:``今儿怎么想起用枫露茶了。令贵妃给朕挑的金线春芽甚好,换那个。''

她听得懂皇帝的意思,枫露茶是她从前挑了放在养心殿的。李玉斟上此茶,不过是让皇帝念着她从前的心意。

这意思再明白不过了。李玉尴尬,忙退了下去。她却不尴尬,又福一福,``臣妾告退。''

皇帝觑着她,``你的规矩是孝贤皇后在世时调教的。如今孝贤皇后去了,你也这般不知进退了么?''

如懿欠身,面目温顺得无可挑剔,``臣妾知道皇上往长春宫追念孝贤皇后,睹物思人。正巧见暖阁里有新誊的皇上的御制诗,篇篇情深,字字血泪。臣妾细观,念着孝贤皇后昔日为何得皇上这般爱重,也可加勉。''

皇帝看着她,那眼神是寒雨夜里的电光,是明亮的锋刃,``孝贤皇后在时,温和驯顺,从不敢拂逆联,也不会争风吃醋,更不会作此冷嘲热讽之语。终究是你出身教养,不如富察大族多矣。''

她扬起眉,精心描过的青黛色是高悬的新月,冷冷挂在高寒深蓝的天际,``臣妾这般不如,皇上垂爱,属意臣妾为继后,当真是错爱了。''

皇帝也不言语,冷冷看着她,随手去翻阅那些诗词,徐徐道:``婉嫔从来不声不响,难得有这样的心思,能将朕对孝贤皇后追念的只字片语集拢。朕自己看着,也是愧悔又感动。''

如懿凝眸,将细纹般碎裂的痛楚掩于平淡的口吻之下,``是。不止皇上,臣妾看了也很感动。这些年来,皇上只要经过济南,都会绕城而过,不肯进城,只为孝贤皇后病逝于济南。孝贤皇后的遗物都留在长春宫中,这么多年一桌一椅都未曾动过,是旧日面貌。睹物思人,岂不伤怀?连孝贤皇后曾亲手做的燧囊,也供在宫中。而对和敬公主,也疼爱逾常,惠及额附。若非婉嫔有心,臣妾虽知皇上常有悼亡之作,却不意有如此之多。''

皇帝听她娓娓道来,眸中连半点涟漪也无,不觉眼角飞起,谑道:``皇后真是贤惠,半点妒意也无。''

如懿的唇是晚春谢了的残红,浅浅的绯色,沉静不己,``皇上曾经指责臣妾嫉妒容缤,臣妾受教。至于孝贤皇后,乃是皇上发妻,皇上情深几许,都是人之常情,臣妾难道会与离世之人苦争高下么?''

皇帝的口气温和了几许,``如懿,这些诗,朕并非是说你不好。''

``臣妾的不足臣妾自知。''她笑色颇黯,``皇上,臣妾看了您对孝贤皇后的深情,真是欣慰。哪一日臣妾弃世而去,昨日种种,皇上或许也不与臣妾计较了吧。''

皇帝的脸色有些难看,是阴阴欲雨的混沌,``你的意思,是朕不曾好好爱惜孝贤皇后,待她身死之后才万般追忆,空自错付了?''

她的笑是淡淡的稀薄的云影,``皇上误会了。臣妾说过,只是欣慰而己。人死万事空,真好,一切烦恼皆消。''

清日无尘,日丽风柔。日色如金,柳荫浅碧。园中早樱开得正好,折三两枝以清水养在古莲纹青釉瓶内,一束一束娇艳的轻粉,如蓬蓬的云霞,撩动人心。那樱花是刚折的,沾染了草间薄露,静奉殿内,只觉那粉色的云揉进了眼帘里,望着肌骨生相对之时,唯有他与她是冷的。笑也冷,静也冷;言语是冷,无言也是冷。竟然觅不到一丝温沉的暖。

那些记忆中深入骨髓的爱意与依靠、期盼与渴求呢?她这一生所有,无一不与眼前的男子息息相关,却不想,到了此时此刻,看着他,也是寒意顿生。

皇帝听着她的淡然,她的冷漠,微微摇首,``如懿,朕冷落你的这些日子,你倒是通透了许多。可是你对朕,连一个女人该有的情绪都没有了么?朕倒想起来,当日在宝月楼,对着朕与容嫔,你是何等措辞激烈。''

如懿简直不相信自己的耳朵,骇然失笑。她一双眸子深深盯着他,``那么臣妾该如何?撒娇、吃醋、嫉妒,还是吵闹?臣妾不知道何种作为是对,何种作为是错。如果皇上盼着臣妾嫉妒伤心,那当日为何责骂臣妾醋妒害了容嫔。若是皇上希望臣妾保持皇后应有的气度与容忍,那您希望在臣妾的脸上看到何种情绪?无论臣妾如何做,都不能成全您的心意。既然都是错,臣妾受着就是了。''

皇帝一字一字缓缓地道:``如懿,朕己经老了,年岁越大,越怀念当年孝贤皇后的温和隐忍。如懿,你的锋芒太利。为何不能如孝贤皇后一般?朕不悦时发怒时,孝贤皇后都格外温顺宁和,你却一定要说出伤朕的话么?''

``有的话,许多人不能说,不敢说。臣妾也想忍住不言,却一生也未学会。臣妾听闻皇上常去长春宫睹物思人,悼念孝贤皇后。臣妾只是觉得,生前未能好好待她,信任她,身后百般思念追悔,有何意义?''她俯身三拜,郑重道,``皇上,臣妾知道您的不满。臣妾也自知无能,有负于皇上,更不知如何顺应才是对。''

她穿着瘦瘦的浅青丝绵旗装,镶着玉萝色的边,窄窄地裹着身体。因是来见皇帝,绣纹也格外郑重些,绣千枝千叶排紫平金海棠,每一花,每一瓣,缠金绕紫。她在胸前如意双花纽子上坠了一枚刺绣香囊,沉甸甸的,缀着白玉蝴蝶的坠子。每一起伏,重重敲在胸上,沉闷无声。

皇帝听她的话,只觉早春寒气缓缓浸衣,胸中一股窒闷,无从宣泄,他忍了忍气,沉声道:``朕鞠育永璂多日,也觉得这孩子该悉心管教。你的性子素来别扭,不如将永璂挪去愉妃处教养,也可学得永琪七八分样子。你便好好静心,守己思过吧。''

那是迟早要来的命数。

然而如懿还是悚然大震,``皇上,永璂是臣妾的亲生子!''

``那又如何?''皇帝的口吻淡漠如烟,``令贵妃尚有公主养在颖妃膝下,你既然要静心思过,带着孩子亦不方便。''他眼波流漾,似有几分居高临下的鄙夷,``怎么?你会求朕?''

他是看死了她,不过是一介女子,毕生所得,不过是依附于他。她的心底在抽痛,可是跟着这样不识抬举的额娘,又有什么益处。她屈膝,温柔有礼,``多谢皇上,愉妃与臣妾情同姐妹,永璂送到愉妃身边教养,来日也可学得永琪的好处,为皇上分忧。''

她言毕,再不停顿,急急退却。

她走得极快,足下带着风,以决绝的姿态压抑着心底渐渐迫出的疼痛。

永璂不能在身边,固然是大恸,可与其让孩子的眼睛过早地看清自己身为皇后却备受冷落的尴尬,看清世态炎凉的碾磨,不如送去海兰那里,得一分清静自在。

盘旋在脑海中的,分明是皇帝多年来写下的深情之语,故剑情深,她不过是一把新琴。噫!这么多年的相随相伴,情感被岁月渐渐熬煎,己逝的人被风霜剥蚀了所有不悦的记忆,成为崭新完美的一个人儿。而自己,却因为活着,因为呼吸着,却熬成了不堪入目的焦蝴,烙在他眼底心上,叫人嫌恶。那么,又为何要苦苦痴缠,分崩离析,走到连活着都是一种错误的境地。

这般念头,似一把锋锐的青霜剑,狠狠刺入她心口。因着太锋利,来得太突兀,竟连半分血渍都不见。她只能任它这般刺着,一拔出来只会鲜血飞溅。她知道的,从她看到那句话的时候,那柄剑便终身再难拔去。容珮见她这般跌跌撞撞出来,吓得面色青白,急急扶住了,也不敢多问。

她倦得很,低声道:``回宫。''

没有可以觅得温暖的地方,这样的痛楚与耻辱也无人可诉,只得回到冰冷的宫苑,哪怕自己蜷缩起来舔舐伤口,也好过在这里再多留片刻。台阶怎的那样长,总走不到尽头。迎面而来的,竟是一身华衣的婉嫔,身姿楚楚,下得辇轿来。

婉嫔瞧见如懿,便有愧色,也不敢避,只得行了莫大的礼数,当着冷风迎头跪下,凄凄道:``皇后娘娘万安。''

一股子鲜血涌到喉头,逼得嗓子眼发甜。就是眼前这个女子,这个一往情深的女子,将这些悼亡之作,齐齐凑到她眼前,叫她看见。

深深吸一口气,定定站住,依旧绷出素来端和的皇后之范,沉着道:``起来!''

虽然正是当行得令的时候,有难得的宠眷,她也不过是一身烟霞色华云缎穿珠绣双抱兰萱袍子。那样精工绣致的衣裳,落在她身上总有不胜之态,仿佛撑不起料子的骨架似的,怯怯地叫人怜惜。那领口与袖口滚着水青色的边,点着一朵一朵暗红的千叶石榴,是初夏将至的欢喜与茂盛,一簇簇漫漫开着,是点燃的火焰,直直焚进她的心底,焚得都快成了灰烬。

如懿沉沉打量着她,``很好。听闻孝贤皇后死忌将至,你倒是想了极好的法子,略表皇上与孝贤皇后恩深义重。''

婉嫔听她这般说,早没了主心骨,更怯了三分,哪里还敢抬头。她见如懿气息深长,像是忍着一口怨气不发,更兼容珮神色慌乱,早猜到了几分,慌忙道:``皇后娘娘恕罪。''

``恕罪?你何罪之有?''她的声息微微一抖,很快恢复肃然的平静,``你不过是告诉了本宫一些本宫一直充耳不闻假装不曾看见的东西。''她郁然松一口气,``不是你,也有别人,迟早有人要逼着本宫看清事实,看清自己不如别人。''

婉嫔牵着她的袖子,满脸的惶惑与不安,依依道:``皇后娘娘,臣妾知道不该拿孝贤皇后去邀宠。可是,可是\ldots{}''她咬着唇,想是用力,咬出了深深的印子,``可是皇上从来没好好看过臣妾一眼,臣妾只是想让皇上记得,还有臣妾这么一个人。''

不能不怜悯她的一腔情意,但若被人利用,又是多么可惜。如懿便问:``是谁教你的?''

``是令贵妃,她可怜臣妾,所以教了臣妾这个法子,也果然有用,连和敬公主亦赞不绝口。''婉嫔怯生生看着如懿,不胜卑弱,一双手不知该放置何处,泪如雨下,``皇后娘娘,对不住。对不住。''

非得被人利用,才得以在所爱之人的眼中有立锥之地,却又能站多久?婉嫔已然拔得头筹,可后来人何等聪明,早有晋嫔之流,将皇帝悼亡孝贤皇后的诗词,刊印出来,流传天下。到头来,也不过是为他人作嫁衣裳。

如懿凝视着她,长叹一声,抽袖而去。

婉嫔不是一个坏人。甚至,她是一个难得的好人。隐忍、温婉,连爱意亦深沉低调,从不轻易伤害人。但,有时好人也会不讨人喜欢,坏人也不一定让人讨厌。

在婉嫔处,她照见的是沉默隐忍的爱意,是无言的企盼与守望,而香见处是盛大的欢悦与渴爱之下令人战栗避拒的惶恐与挣扎。那么她呢,她的爱,她曾经一往情深执念不肯放低的爱,都给了谁呢?

是那个眉目清澈的少年,永远在她的记忆深处,轻轻唤她一声:``青樱。''

那是一生里最好的年岁了,丢不开,舍不得,忘不掉,却再也回不去了。

如懿这般沉寂,便是连容珮也看不过眼了。她思虑再三,还是出言:``皇后娘娘娘,令贵妃如此操纵婉嫔,讨了皇上与和敬公主欢心,您便什么也不做么?''

如懿望着窗外阴阴欲坠的天气,沉声道:``本宫如今的处境,若凭一己之力,那是什么也做不了,你去请毓瑚来一趟吧。''

毓瑚来得倒是很快,恭恭敬敬向如懿请了安,便道:``奴婢来之前常听福珈说起,太后娘娘虽然己经不管事了,可眼瞧着令贵妃坐大,也是不喜。唉,说来也是昔年太后过于宽纵,小觑了她,才致如今的地步。太后娘娘偶尔提及,也很是懊悔。''

如懿颔首,这些年皇帝与太后的关系和缓不少,加之太后几乎不理前朝后宫事宜,只安心颐养天年,皇帝更是有心弥合昔日母子情分的嫌隙,不由拿出少年时对太后的敬慕之心,尽天下之力极尽奉养。晨昏定省,节庆问安。每逢生辰重阳,更是搜罗天下奇珍,以博太后一笑。太后了尽世事,如何不知,于是越发沉静,专心于佛道,享儿孙之乐。这般平衡下来,母子之间更见诚笃。所以太后纵使不喜嬿婉,也绝对不会主动出言。

如懿便道:``诸多子女之中,皇上最疼惜的和敬公主。盖因孝贤皇后早逝,皇上心中总是痛惜。但公主何等尊贵的身份,总与嫔御亲近,也不是正理呀。其中的缘故,还请毓瑚姑姑分晓。毕竟,您是皇上跟前的老人啊。''

毓瑚忙忙叩首,起身离去。

和敬公主因是嫡出,素来自恃身份,矜持高贵,但对毓瑚这样侍奉皇帝多年的老人,却很是和颜悦色。和敬一壁吩咐了侍女给毓瑚上茶,一壁让了坐下,十分客气。二人倾谈良久,和敬渐渐少了言语,只是轻啜茶水。

半晌,和敬方问:``毓瑚姑姑,您方才说的可都当真?''

毓瑚了然微笑:``公主若不信,大可去查。当日令贵妃还是花房宫女,因在长春宫失手砸了盆花,才被孝贤皇后拨去淑嘉皇贵妃那儿教导,谁知淑嘉皇贵妃心狠手辣,那些年令贵妃备受折磨,您说她恨不恨淑嘉皇贵妃?''

和敬哂笑,不屑道:``淑嘉皇贵妃的性子,向来是得罪的多,结缘的少。她这般厉害,令贵妃自然怨恨无比。可令贵妃也会恨额娘么?''

毓瑚一脸恭谨,欠身道:``公主深通人情世故,个中情由,您细想就能明白。''和敬低首沉思,拨弄着小指上寸许长的鎏金缠花护甲,默然片刻,方才含了冷峻之色,``是了。哪怕令贵妃不敢明着怨恨额娘,可也必定不是她所说的对额娘满怀敬重。她当日就是花言巧语蒙骗我,借额娘的情分接近我。毓瑚姑姑,你说是不是?只是姑姑为何到今日才告诉我这些?倒由得令贵妃巧言令色。''毓瑚叹口气,遥遥望着长春宫方向,神色恭敬至极,``孝贤皇后节俭自持,是女中表率,深得皇上与后宫诸人敬重。原本令贵妃只是与公主亲近,奴婢也不明就里。可如今令贵妃协理六宫,还借着皇上写给孝贤皇后的悼诗兴风作浪,借机打压皇后,奴婢实在是觉得太过了。''和敬唇边的笑意淡漠下来,她望着别处,冷然出声:``你是不满皇后委屈?''

毓瑚一脸恳切,推心置腹,``不。奴婢伺候皇上多年,是不喜欢有人在背后翻云覆雨,借亡故之人邀宠献媚,排除异己。孝贤皇后是公主的亲额娘,想来公主也不忍心看孝贤皇后死后被人当作争宠夺利的由头,不得安宁。''

和敬挑了挑眉头,抿了一口茶水,似笑非笑道:``那姑姑为何不告诉皇阿玛?说与我又有何益?''

毓瑚倒也不含糊,迎着和敬的疑惑道:``这些事,只怕在无知的人眼中,还以为是公主不满皇后才做的。令贵妃唆使婉嫔借孝贤皇后争宠,以此坐收渔翁之利,却让人以为是公主行事离间帝后,奴婢实在替公主不值。公主您是皇上唯一的嫡女,尊贵无匹啊,万不可沾染污名,受人连累。''

和敬长舒一口气:``你的意思,我都明白了。''

毓瑚方才款款起身告辞。和敬望着她的身影,眉头的阴翳益发浓重。

京城的春天,干燥得发脆,兼着漫天柳絮轻舞飞扬,是粉白色的琐碎。偶尔,有零星的雨水,让她想起童年江南连绵的雨季。

天气好的时候,永琪为皇帝处理了一些简单的政务,便往延禧宫来请安。院落里静悄悄的,空旷得很。深紫色的玉兰花相继开放,饱满的花萼满盛春光,散发出沁人的幽香,从清静庭院悠扬起落入了雅静内殿。

东侧殿里有朗朗的读书声传来,是永璂的声音。永琪也不多停留,抬足便往里走。

海兰独自坐在窗下,就着清朗天光绣着一件什么物事。她拈针走线,长长睫毛在脸上留下两片羽翼似的阴影,脖颈弯成一个好看的弧度。

永琪心底一软,这就是他的额娘,永远娴静温和的额娘。

海兰穿着一件家常的玉兰色印银错金竹叶纹织锦裙,外头罩着暗紫色团花比甲。做工虽不难,但质地、剪裁俱上乘。头上绾着累金丝嵌蓝宝石花钿,手腕上一副羊脂白班雕梅花云鹤如意镯玲珑有致。

永琪很是安慰,因着自己在皇帝跟前得意,额娘的境遇也越来越好,虽然依旧不得宠,却无人敢怠慢,吃穿所用,俱是上等。这般想着,素日的劳心劳力,都成了理所应当。他,只盼着额娘好过。

于是走过去行礼请安,海兰见了儿子来,喜不自胜地扶住道:``瞧你这孩子,定是急忙忙赶来,头发都乱了。''

永琪见她方才仔细绣着什么物事,走近一看,是一件冬日里穿的石青缎绣八团莲花白狐慊皮褂,每一朵捧出,都是重重瓣瓣的金线绣莲花。他便道:``额娘在做什么绣活?这些细致活计伤眼睛,交给下人去做吧。''海兰道:``是你皇额娘的东西。''

永琪笑道:``儿子知道。若不是皇额娘的东西,额娘怎会如此上心?''海兰郁郁难安,``如今内务府懒怠,这件衣裳领口破了也不肯补上。容珮的绣活儿不行,你皇额娘\ldots 近来眼睛不大好,要自己动手也不能。''永琪犹豫片刻,``儿子听说了,宫中追奉孝贤皇后成风,皇额娘处境难堪。连永璂也不能留在身边。''

海兰摆摆手,不欲再言,向他道:``来。头发乱了,额娘给你梳梳。''永琪乖顺坐下,由着海兰打散了头发,细细梳理。

永琪闭着眼,极享受似的。他轻声地,像是不能确信,又不敢触碰似的,低低道:``额娘,皇阿玛真的是疼爱我么?''

海兰的手势极温柔,替他细细蓖着头发,``怎么这么问?''永琪眼皮低垂,底下的眸子却不安地转动,``额娘,皇阿玛并不宠爱您,为什么他会疼爱我?是真的因为我做得无可挑剔,还是我,不过是皇阿玛寄托的希望,让他看到永琏和永琮长大成人后成为他理想的模样。''

海兰抚着他的额头,温沉道:``你皇阿玛疼爱嫡子,是众所周知之事。他一心渴盼的,是孝贤皇后所生之子可以长大成人继承帝祚。只可惜,永琏和永琮都福薄。但永琪,不必理会旁的,你自己争气便是。''

永琪搓着手,``皇阿玛也很疼爱永璂,还把他送来延禧宫给额娘抚养。儿子明白,皇额娘失势,额娘与世无争,反而能给永璂些许安定时日。''

``那是当然,鸾胶再续,弦断再接,你皇额娘身为继后,生下的永璂自然是嫡子。只可惜,哪怕都是妻子,续弦总不如结发。你皇额娘的为难之处,便在这里。况她家世不比孝贤皇后满门富贵荣耀,身后无人,孤苦无依。''海兰的托付温婉而沉重,``永琪,你已经长大,得多扶持你皇额娘才是。''永琪双目微睁,沉吟片刻,``额娘所言甚是。皇额娘虽然得罪了皇阿玛,但地位无忧。且皇额娘还有永璂,永璂才是皇额娘唯一的儿子。''``你难道不算你皇额娘的儿子么?''海兰长叹一声,``自你出生,额娘便再无恩宠。多少年寒夜孤灯,唯有自己知道罢了。若无你皇额娘将你养在膝下,视若己出。阿哥所里有多少养不大的孩子,你或许也成了一个。所以永琪,你一定要和永璂一样孝顺你皇额娘,待她要如待我一样。''永琪抓住海兰的手,语意沉沉,``我是额娘的儿子,当然孝顺额娘。对皇额娘,我心里也明白她的恩德,知道该怎么做。永璂\ldots{}''他顿一顿,``儿子也会好好照顾永璂。''海兰很是欣慰,温言道:``永琪,永璂天资平平,不如你幼时聪颖。但先天不足后天可补,你做兄长的,要好好督促他才是。''

永琪眸中微微一黯,点头称是。

海兰将手中的鎏金珊瑚绿松坠角缠上收好的辫梢,柔声道:``好了。''永琪翻于一看,笑道:``还是额娘梳的辫子最好。芸角最会梳头发,也不及额娘手巧。''

海兰挑着眼角含笑看着他,``芸角?便是你新纳的那个侍妾胡氏?''

永琪大是赧然,``福晋告诉额娘的?是外头饮酒时三姐姐的额附送的丫头,盛情难却,儿子只好收了。不承想倒是个玲珑剔透的女孩子,儿子便将她收了房封了格格了。''海兰微笑,看着儿子的目光尽是疼惜,``你常和外头的人来往,赠妾之事也是常有。额娘倒想看看是怎么个出挑人物,就成了你心尖上的人儿了。只是规矩在这儿,额娘能见的媳妇儿,只有你的福晋和侧福晋,格格是不入流的,入不得宫。''永琪颇为怜惜,``是。若不是身份上不能够,便是一个侧福晋也委屈了她。''

海兰听得微微皱眉,道:``一个侍妾而己,你便再喜欢,也别过于偏宠,伤了你福晋的心。更要记着,这样的轻薄的话可不许再说出口。''

永琪面皮薄,脸上微红,诺诺称是。海兰见儿子如此,哪里还忍心说他,笑靥温然,``难得有一个你可心的人儿,若能为你绵延子嗣,自然也少不得她的前程。''

母子俩说着话,己然是暮色四合时分,永琪赶着出宫回去。他迎着最后一缕霞色步出延禧宫外,四下温柔的风夹杂着后宫女子特有的脂粉香气盈盈裹缠上来。永琪静静屏息,想念着指尖划过芸角面孔的滑腻。芸角的话犹自留在耳边,``五爷,您的前程是您自己的,谁都别想,谁都别管,顾着您自己才是对的。''

\hypertarget{ux7b2cux5341ux4e00ux7ae0-ux6731ux8272ux70c8ux4e0a}{%
\chapter{第十一章
朱色烈(上)}\label{ux7b2cux5341ux4e00ux7ae0-ux6731ux8272ux70c8ux4e0a}}

自从豫妃失宠,香见与嬿婉平分春色,宫里渐渐也安静些。只是茶余饭后总有嫔妃爱拿豫妃当笑话,既是封妃,也是失宠,惹得永和宫门庭冷落,寂寂长久。不觉叫人想起曾经永和宫的主位玫嫔,也不过盛极一时,便随风凋落。其实也无他,恰如汹捅的波涛之后总会坠入深沉的平静,而潺的静涴水深流之中,也会有偶尔落下的碎石,激起涟漪荡漾。曾与她争锋一时的恂嫔,却未因豫妃的失宠而迎风争上。仿佛随着当日被豫妃夺宠,她也无喜无优,沉寂了下来。由着香见与嬿婉擅宠一时,花开各表。乾隆二十六年的夏日与往年并无不同,其时天方入夏,暖阁内的六棱花长扇窗格上蒙着薄薄的浅银色翠影纱,因着午后熏风暖暖,淡青色的湘妃竹帘也高高卷着。庭院里的栀子花洁白芬芳,被风一扑,迎面拂来阵阵沾染着阳光气息的蓬勃花香。初夏的暑气尚且不重,是一种热闹的融融的甜味,与乳色的阳光绞在一起,连宫殿的瓦釜飞甍都带着流光错彩的印迹,连庭下梧桐都染上含翠沐金的华彩。如此,花气与初夏甘冽的暑味重叠纵横,一室内皆是清通敞亮。如懿虽已不大理事,但偶尔也会翻阅敬事房的记档。长日无事,她便只穿了家常的玉色碧罗点栀子花绣袍,一头乌丝松松绾着,斜插了一支通透琉璃簪,垂着碎红宝流苏,叫日光一映,连带燕尾后的翠钿都跟着微微一粲。这般打扮,简丽而不落俗,也不算全消磨了心气。她看了数页便疑惑,``皇上曾经也算宠爱恂嫔,如今怎么倒不理会了?''

忻妃落了产后失调的症候,终日病恹恹的。她坐在如懿下首,八公主被海兰抱在怀中逗弄,忻妃吃力地笑了笑,``再宠爱也不过如此,新鲜劲儿过了就丢开手了。''

手边的翠眉镶金华小胆瓶中斜斜插着一束大红的石榴花。那样明艳的深绿嫣红金彩,逗得八公主看个不止。海兰拔下发髻上一枚青金蝴蝶米珠花引着八公主,一壁笑道:``旁人说这个话也罢了,你千盼万盼终于盼到了自己的孩子,也说这样的丧气话?''

忻妃定定地坐着,产后的病痛虚弱缠得她瘦骨伶仃,一件浅玫瑰红绣嫩黄折枝玉兰绮霞缎长衣虚虚地笼在身上,宽大得不着边际。越发衬得她面色无华,唇白目滞。因着瘦,她的颧骨高高地耸起,原本一双点漆明眸空落落地张大在面孔上,无神而空洞。

如懿小指上的纯金镂空织花锻雕护甲轻轻划过暗红的档本面,安慰道:``你拼尽辛苦生下八公主,产后失调皇上也是心疼。你还年轻,本宫会叫江与彬细细为你调理,待好起来了,再生一个阿哥与八公主做伴。''

忻妃勉力一笑,``从前年轻不懂事,总以为仗着年纪小得皇上的宠爱。如今,也不过是挣命罢了。唉,臣妾的身子自己知道,只是可怜八公主年幼,为她熬一日是一日吧。''

海兰亲昵地吻了吻八公主粉嫩的额头,怜惜地看着忻妃,``你为了生八公主大出血失调,但好歹还有你阿玛,八公主有你和这位外祖在,必不会吃亏。等你身子好了又能侍寝,皇上必会格外疼借你的。''

话虽如此,忻妃也只是苦笑,``话是这般说,皇上也疼爱公主,可能不能侍寝,到底差了一层。八公主这么大了,皇上尚未给个封号,可见未曾上心,只顾着令贵妃的几个儿女罢了。说到底,所谓恩宠,不过是夜夜相亲,否则皇上眼里臣妾也是可有可无。其间厉害,愉妃姐姐不也清楚?''

海兰垂着脸,静静不语。如懿托腮凝神,``你的辛苦委屈咱们都知道。可恂嫔难道不知?她原比豫妃年轻,只是不大会得狐媚,随遇而安得很。如今豫妃失宠,本该她东山再起,却这般默默。本宫方才瞧她侍寝的记档,初入宫最盛时十日有三次,如今小时年了才一次。便是有容嫔这般擅宠,也不该如此啊。''

海兰的话不无道理。自从容嫔绝了生育,皇帝对她的狂热便渐渐淡了几分,虽然还是这般轻怜蜜爱,宠遇隆重,可到底克制了许多。对于六宫嫔妃,也是雨露均施,颇为眷顾。所以除却或病或失宠的几位,恂嫔的冷遇,不可谓不引人注目。

只是话虽如此,如懿失宠,忻妃抱病,能与皇帝见上的,也唯有子凭母贵的海兰了。因着永琪得力,皇帝对着海兰也越来越肯假以辞色。所以宫中嫔妃,除了对着协理六宫甫又生了十五阿哥永琰的嬿婉毕恭毕敬,其次便是最尊重海兰了。

也因为海兰的位分持重,如懿便是失宠,还能维持着温水一样平淡的生活,无人惊扰。为解如懿的忧闷,海兰便常过来,有时也携着同样寂寞的忻妃,一同理线、绣花、作诗、煎茶,逗着八公主,或是说说永璂的日常琐事。秋日的午后听风吹落叶声,暑天的黄昏一起吃冰水湃过的新鲜果子,还有容嫔处送来的哈密瓜,倒也安闲。

因着起了疑虑,偶尔海兰独自与皇帝相对时,也会问一句,``近日姐妹们在一处,臣妾倒见恂嫔仿佛瘦了些。''

皇帝将海兰新绣的一枚翡翠色绣袋流苏坠系在身上,不以为意道:``是么?朕倒有些日子不曾见她了。''

海兰替他理顺了明黄米珠流苏,小心翼翼拣了话道:``恂嫔独自在宫中,家乡亲人也离得远,格外孤苦。臣妾偶然看见她孤身一人,也觉得可怜。''

皇帝原低头看着绣袋上的花纹,闻言不觉冷笑,``怎么?她也给你脸子瞧?朕一向自诩不曾薄待身边人,唯她气性大。朕刚宠她时却还好,后来豫妃得宠,朕冷落她些,后来再去,却对着朕连个笑脸也没有了。既如此,朕去瞧她脸色么?''

海兰蕴了含蓄的笑,``是。恂嫔的性子是内向些,也不大与人说话,却没有冒犯臣妾。听人说她无事便在自己宫里拉马头琴,臣妾怕她存了什么心事\ldots{}''

皇帝摆手不耐道:``她拉着马头琴便能自得其乐,朕又何必过分宠她,若是宠得多了,难保不是第二个豫妃!也别叫她以为博尔济吉特氏失宠,她霍硕特部就能给朕颜色看了。''他缓一缓口气,``再者,她是霍硕特部的女儿,朕当年纳她,是为了安霍硕特部的心,要他们真心驯服。所以朕会给她颜面,不会薄待。但进了宫,宠是自己争的,难不成还要朕迁就她?''

海兰见皇帝不豫,忙扯了话头说起永璂与永琪读书之事,皇帝便也撇过不提了。

这一夜细雨微凉,六月初的时节,细雨蒙蒙,染湿流光,紫禁城底下的万物便坐转作了凌然的昏黄。皇帝本欲留海兰在养心殿用膳,奈何海兰记挂着永璂早起咳嗽了两声,放心不下,便辞了离去。

入夏后皇帝兴致颇好,又思念和敬公主,常叫她携子入宫,祖孙三代同乐。和敬早年长居深宫,一草一木皆是旧情,更喜陪着皇帝在长春宫中坐坐,有时傅恒也作陪,一同说及孝贤皇后在时的往事,睹物思人,常常一陪就是一整日。这般圣宠,便是几个皇子也不及,人人都道是孝贤皇后的缘故,恩及公主,更惠泽富察氏全族。,于是宫中人等对和敬公主奉承更甚,恨不得亲身巴结,可和敬的性子是目下无尘,也甚少将人放在眼中,只是我行我素。

这一日从长春宫出来,侍奉和敬多年的崔嬷嬷便殷勤打着伞上来,又取了香帕递给和敬,道:``天儿热,公主仔细中了暑气。奴婢在阁中备好了消暑的莲心汤,您回去就能喝了。''

和敬颔首,又问了几句闲话。崔嬷嬷见和敬神色不错,方才道:``公主,听说您进宫了,令贵妃巴巴儿地派人请您去喝茶呢。这不令贵妃身边的澜翠一直在长春宫外候着请您,后来险险中暑了,才叫奴婢打发回去了。''

和敬听完,倒也直截了当,``不去。''

崔嬷嬷赔笑道:``人家如今好歹是贵妃了,又有协理六宫之权\ldots{}''

和敬鼻息微重,轻轻一哼,取过袖中一把小巧玲珑的绢扇打开扇了几下,道:``婢妾就是婢妾,哪怕给她个皇贵妃也不配给额娘提鞋。我堂堂一个嫡出公主,敷衍她是给她脸面,不理会她也是情理之中。一想到她那小家子气讨好我的样子,就觉得恶心。若非毓瑚提醒,我竟不防,被她算计了。''

崔嬷嬷忙忙点头称是,一手接过和敬手中的扇子,用力扇出凉风:``公主着奴婢打听了,当日令贵妃被送到淑嘉皇贵妃那儿教导,的确是由孝贤皇后而起。可到底是从前的事了。''

暑光雪白,照得紫禁城碧瓦红墙热气腾腾,连琉璃瓦也晶光荡漾,似大泼热火流溢。和敬心底越发不耐烦,用鼻音道:``那更可见这个人心术不正了。''

崔嬷嬷想了想,还是说道:``公主不看僧面看佛面吧,毕竟令贵妃舍身忘我,救过咱们庆佑小主子呢。''

和敬冷淡,``若非如此,我还能与她说话?就是看在庆佑的分儿上罢了。''

崔嬷嬷心知和敬的脾气,哪敢再多言。一行人正要转过长街,却见嬿婉扶着春婵的手过来,老远就笑盈盈的,直朝和敬看过来。

崔嬷嬷情知避不过,只得低声道:``公主,说曹操曹操就到。''

和敬正皱眉间,嬿婉己经亲亲热热地迎上来,挽住了和敬的手道:``本叫澜翠来,请公主到我宫里坐坐,谁知这丫头的身子不中用,候了一个时辰便中暑了。这不我就亲自来了,我宫里备了好茶,还有进贡的蜜瓜,甜脆多汁,请公主去尝尝吧。''

和敬哪里肯与她假以辞色,抽出手便道:``这天儿热烘烘的,身上便懒惰。我今日没心情,哪里也不想去。''

嬿婉笑意不减:``那改日也好\ldots{}''

和敬扶着崔嬷嬷的手径自往前走:``多谢好意,再说吧,崔嬷嬷,我们走。''``花,霏,雪,整,理''

嬿婉被冷在原地,一时反应不过来。直到和敬公主去了好远,她才苦笑出来,``这位公主,可真难伺候。也不知我哪里得罪了她。''

春婵顺着嬿婉的话头道:``和敬公主脾气好大,便是皇上也不与她计较,毕竟是嫡出的公主啊\ldots{}''

嬿婉倒也不以为忤:``她就是这样,少不得多哄着些。我纵使身居贵妃之位,也开罪不起啊。''

和敬见过嬿婉,气色便不大好。崔嬷嬷少不得劝道:``公主啊,伸手不打笑脸人。何况令贵妃又得宠,如今的气势,连皇后也莫能奈何呢。''

和敬毫不理会,只由着崔嬷嬷扶着她,足下步伐更快。才过栩坤宫,却见如懿携了容珮出来。和敬虽然与如懿不睦,但礼数倒也不差,立刻站住了脚行礼,``给皇额娘请安。''

如懿温言道:``璟瑟,起来吧。''

和敬得了如懿许可,方才直起身来,往檐下阴凉处避了避。如懿打量和敬片刻,笑道:``有一点本宫很佩服公主,你与本宫有母女之名,却无母女之情,但公主对着本宫礼数周全,再不是本宫与皇上成婚时言辞犀利的公主了。''

和敬挺直了背脊,恭敬中不失威仪,``礼数之道是额娘亲自教导,儿臣不敢违背。且如今你是嫡母,儿臣是公主中最长的一个,更要成为弟妹们的表率。不能让乌拉那拉氏说富察氏的女儿无礼。''

和敬本就是嫡出公主的气势,加之烈日之下一袭红衣,更觉凛然不可冒犯。如懿微微颔首,``公主这般有心气,真是好事。对了,今日怎么不见公主带庆佑入宫?''

和敬听提到爱子,脸色温柔不少,``小儿家顽皮,带进宫不太方便。怕吵着皇阿玛呢。''

如懿便道:``也是。若再有不小心落水,成全了旁人的事,本宫这个皇祖母听着也不忍心。''

这语中的机锋,和敬如何听不明白,她旋即挑眉,面色不豫,``皇额娘的意思是\ldots{}''

如懿说得云淡风轻,``毕竟当日庆佑如何落水谁也没看见,万一是有心人拿庆佑的安危做文章呢?自然了,本宫素来是多心之人,也是多嘴一句罢了。''

和敬迟疑片刻,正要说什么,硬生生闭住了嘴唇,施礼离开。

待回到阁中,已是汗湿罗衣。崔嬷嬷伺候着和敬更衣完毕,又奉上莲心汤,才打发了众人出去,亲自取扇给和敬扇着。那檀香木扇不比绢罗轻盈,动静间香风阵阵,颇有宁神之效。和敬面上愠怒的红潮渐渐褪去,崔嬷嬷才敢开口:``今儿皇后娘娘的话,公主可听进心里去了?''

和敬犹疑片刻,``我素来是不喜欢乌拉那拉氏的。无他,只为我额娘的缘故。可令贵妃其心可疑,也不足信。''

``那您是怀疑庆佑小主子落水的事的确是被令贵妃暗算了?''

和敬静了片刻,方下定了决心一般,``当日之事无人见证,令贵妃自己也不会承认。再多纠缠,也无用。''

``那公主的意思是\ldots{}''

``我是孝贤皇后的嫡女,与嫔御何干?从今往后,令贵妃莫来纠缠我,我也远着她,彼此再不相干。她若对庆佑有恩,这些年我对她的提携也够了。若真是她害了庆佑受惊落水,哼,反正我也不会再帮她。她想借着我打压皇后往上爬也算够了,若真是觊觎皇后之位,她也配!至于皇后么,想借着我两虎相斗,谁都别做梦!''

崔嬷嬷忙道:``是。咱们只管自己。您是最尊贵的嫡出公主,谁都只有巴结您的。''

过了两日,正是要过六月六晾经节的日子。若逢晴天,宫内的全部銮驾都要陈列出来暴晒,皇史、宫内的档案、实录、御制文集等,也要摆在庭院中通风晾晒,连宝华殿与雨花阁所贮的经文也不例外。

偏从这两日起,一直阴雨绵绵。晾经节之事自然是不能了。嬿婉虽然协理六宫,但规矩极严,事事做小伏低,必来禀告如懿的。便由如懿来回禀皇帝,将晾经节之事简略处之。

这一年间,如懿与皇帝的来往,多是这般公事模样。无多少话语好讲,简明扼要地说过,便匆匆离开,不肯多逗留。

这日如懿扶了容珮的手步上玉阶,李玉便迎上来道:``皇后娘娘,皇上往永寿宫去看十五阿哥了,怕一时半会儿回不来呢。''

如懿倒也不讶异,嬿婉新生的十五阿哥永琰,雪白可爱,如个小小的福娃娃一般讨人喜欢,难怪皇帝去永寿宫的次数更多。

如懿只是关切地问李玉,``你怎的没陪皇上去?''

李玉脸色一黯,有些讪讪,``奴才老了,进忠去了。''

寥寥一语,如懿便了然。嬿婉得宠,进忠在皇帝面前也格外得脸,加之年轻娇健,比李玉自然称心许多。

如懿好言安慰,``你是伺候皇上的老人儿了,自然有你的好处。''说着,她便瞧见了守卫在廊下的凌云彻,脖颈裸露处带了两抹血痕,拿雪白的衣领遮掩着,却也不能全遮住。如懿细心,驻足问:``怎么伤了?''

凌云彻皱了皱眉,正欲搪塞,跟在身后送出来的李玉捂嘴笑道:``茂倩厉害得很,抓的!''

凌云彻听李玉插嘴,颇有些怪他多舌,便横了一眼。如懿见伤处皮肉翻起,显是指甲用力抓出的。她微有骇然,``怎的下手这般狠?''

他忙掩饰着道:``不要紧,皮肉伤而己。''

李玉甩了甩拂尘,摇头道:``皇后娘娘有所不知,虽是赐婚,却是怨侣。早动上手了,凌大人是男人,不能回手,躲不过就成这样了。''

凌云彻别过脸,很是不好意思,他克制着低喝一句,``李公公!''

李玉乖觉地住口。如懿不大好受,也不便多言,便叮嘱容珮:``咱们宫里有极好的白药,等下取些来。''容珮答应着,如懿看向凌云彻,温然道:``夫妻之间彼此难以相处最苦。若能缓和,便各退一步吧。''

凌云彻似乎有些出神,如懿不知他是否听进去,也不便久留,只得去了。过了咸和右门便往翊坤宫去,容珮有一搭没一搭地说着:``十二阿哥午睡醒了想去御花园看荷花,可外头下着雨,怕再着了风寒,愉妃小主和奴婢们便拦下了。''

如懿含笑,``这孩子,读书不怎样,倒与他皇阿玛一般,雅爱花草。''她喟然叹息,伸手轻拂清凉雨丝,``可惜,他不在本宫身边,本宫要知道他的消息,也只能是听说。''她停一停,``永璂既看不到荷花,本宫便去折些,送去海兰宫里插瓶,永璂也不必冒雨去看了。''这般商议着,如懿便扶了容珮的手往御花园去。

六月荷花起自碧池。风荷轻曳于蒙蒙水雾间,隔着烟雨缥缈,夜色茫茫,杳无人影。却有隐约的铮铮声从烟雨深处低回而来。

如懿立在伞下,侧耳倾听,``仿佛是马头琴的声音。''她听了片刻,``弹奏的是《朱色烈》。''

马头琴声呜咽,隔着雨打荷叶的淙淙声愈加低转幽咽,仿佛雨水清寒逼仄入骨,生出凉意。容珮疑道:``夜雨无人,谁在弹这情情爱爱的曲子?''

她转首,见荷叶底下有几点微弱的莹亮火光,仔细辨去,竟是几盏彩纸折就的荷花灯。

如懿道:``今儿不是什么正日子,怎么有人在这儿点荷花灯祈福?''

她见前头正是浮碧亭,便道:``雨有些大,去亭中避一避吧。''

灯火移动,众人前行。才近亭子,却听得马头琴声戛然而止,一个袅袅婷婷的身影从亭中站起,匆匆迈出。如懿却看清了,唤道:``恂嫔。''

那女子站住脚,有些不安,``皇后娘娘。''

如懿按捺下心底的疑感,气定神闲,``喜欢在夜雨中拉马头琴,倒颇有情致。只是怎么一个人,伺候的人呢?''

恂嫔有些不好意思,``她们听腻了臣妾拉马头琴,臣旁也不爱她们吵扰,便打发去御花园外守着了。''

如懿笑着打量她,``大约你来来去去只爱拉一首曲子。''她停一停,``可是想家了?''

恂嫔忍耐着拨了拨鬓边的碎红宝串珠流苏,``臣妾不喜欢流苏簪子珠宝花儿的,累赘!也不喜欢宽袍大袖和花盆底鞋。穿戴着它们,臣妾得慢慢走路,细声细气说话,连转头都得怕耳坠甩在脸上。''她的脸上洋溢起满满的神往,``臣妾想家了,想家人,想草原,想草原上的牛羊。''

``所以在水里放了莲花灯祈求家人平安?''

恂嫔重重点头,满脸诚挚,``每天骑着马拿着刀,多危险!臣妾希望,希望一切平安。''

如懿含笑,``你喜欢骑马么?颖妃也是蒙古人,她喜欢骑马,多烈的马她都不怕。''

恂嫔眼睛一亮,露了几分笑涡,``臣妾也喜欢,在草原的时候,臣妾最爱跑马,能跑上一个白天,累了便躺下来。天是蓝的,望不到尽头,不像这儿,天是一块一块的,四四方方小小的,看着难受。''她黯然,很快又笑,``草原上开满了花儿,那些花儿真香,开遍了整个草原。不像御花园的花,美是极美,可却没有那种热烈的香味儿。''

如懿有些震惊,望向她的目光愈加柔和,``人人都想进紫禁城,羡墓紫禁城的富贵。你却不是。你一定也不喜欢自称臣妾,记着那么多称呼规矩。''

她怀抱着马头琴,低垂着脸,``那一年,臣妾不能不进宫。臣妾的父亲一时糊涂,帮助过准噶尔部,才让我们部族受了皇上的冷落。父亲没有办法,才一定要送臣妾进宫向皇上表示悔过与忠心。可臣妾不会争宠,不会讨好皇上,不会像豫妃那样\ldots{}''

如懿看着她的黯然与失落,``不会也不必勉强,皇上不会薄待你。''

恂嫔抚弄着马头琴,笑意酸涩,``是啊。吃的穿的用的都是这世间最好的,要付出的代价就是乖乖地坐在宫里,像井底之蛙。乖顺、听话,安静,没有棱角,没有怨言。''她秀一声,颇有英气,``当然,皇上不会薄待臣妾。因为臣妾在宫里,就是一个让霍硕特部安心的最好摆设。所以哪怕当日豫妃与臣妾争宠,臣妾也不在意。因为她不明白,她和臣妾并没有两样。''她轻蔑一笑,``即便她今日失宠,皇上不也好好待她了么?''

如懿面色沉静下来,``你是个明白人,可是你活得并不甘心。''

恂嫔细长的眸子飞扬起一抹凛冽,``是。哪怕是个摆设,也会有个念想。''她的情绪有些激动,昂首间露出脖子上一条松石链子,下面坠着的并非珠玉,而是一颗白森森的狼牙。

如懿心底一动,伸手拈起那枚狼牙,``一直听闻蒙古部落喜欢以狼牙护身,且须得是用部落英雄亲手打死的狼王之牙。百闻不如一见,你这枚可是吗?''

恂嫔的脸上闪过一丝羞涩和慌乱,伸手扯过那枚狼牙,旋即如常道:``臣妾也不知道,旁人给的,随便戴着罢了。''匆促间,如懿看见她的手,清瘦嶙峋,一把峭骨,隐隐凸起浑圆青色的筋脉,与她轻盈秀丽的身段面容并不相符。就好似,她柔顺驯服之下,深深隐藏的执拗且执着的性格。

恂嫔福一福身,``天色不早,臣妾先告退了。''

如懿见她匆忙离去,伸手接住落下的雨水,似是自语,``你方才拉的《朱色烈》,是讲述男女坚贞之情的曲子。曲传心声,你若思念皇上,自能够见到。''

恂嫔脚下一滞,回头静静看着她,眸中尽是幽沉的哀伤。

亭外雨水,落得越发大了。落在阔大碧绿的荷叶上,滴溜一转,迅疾滑落。好像,一滴巨大而悲伤的泪。

时光悠悠一宕,乾隆二十六年的夏日便这般到了深处。

到了八月,皇帝照例是要巡幸木兰,带着朝臣、诸皇子与后宫嫔妃。皇帝虽与如懿到了见面无言的地步,但外面的颜面到底是顾着的,又有皇子在。木兰秋狝也没有如懿不去的理由。且此番秋狝,蒙古各部王公都列位其间,几位嫁往蒙古的公主也会携额附前来,端的盛大。因而皇帝也不无烦恼地对如懿说:``既然蒙古王公皆在,豫妃与颖妃都是蒙古亲贵出身,不可不去了。''

如懿明白他语底深意,``颖妃当时得令,又抚养着七公主,自然无不去之理。只是豫妃,自封妃那日禁足,也有两年了吧。除了合宫陛见之日,都不曾出来过。''

皇帝显是嫌恶,``也罢,这次会与豫妃父亲博尔济吉特部王爷赛桑相见,她若不怕也不便。''

如懿颔首赞许,``博尔济吉特部世代与我大清联姻,若因豫妃之过而怠慢博尔济吉特部,也不相宜。''她目光轻轻一扫,旋即恭谨垂眸,``且皇上对外,一直顾及豫妃颜面,不曾言她失宠之事,所以赛桑王爷也还不知。''

皇帝不耐烦道:``且这次会面众人皆在,他们父女俩也说不上什么,见过便罢。''

如懿也不多言,微含一缕讽意,低头饮茶。片刻,她方道:``那么恂嫔,也去么?''

皇帝的神色在听到恂嫔时骤然不豫,蹙眉道:``自然是去的。''他顿一顿,若有所思,``只是有件事,朕尚未来得及告诉她。恂嫔的父亲和族人协助我大军扫平寒部余孽时出了意外,死伤大半,恂嫔的父亲也不在了。''

早起的和风徐徐鼓入袖中,隔开了肌肤和光滑的丝缎,生起幽幽凉意。那风经了花木葱郁,回廊九曲,折折荡荡,再旋过乌黑的水磨金砖地面,已经变得柔和了些许。窗外渐盛的阳光带了温热的劲力一格格投进殿中,如浮漾的碎金漫漫腾腾,连皇帝清俊的面容上都浮着一层金灿灿的粉光。

如懿瞧不清他的模样,也不愿去瞧。她眉尖大蹙,愁云频起,惊讶道:``是何时的事?''

皇帝默然须臾,``快一年了。''

如懿惊得差点跳起,到底是多年的涵养教她忍耐了下来。思忖间,那么就是容嫔入宫后不久的事,到底也折在了那场战事的余波里。她打量着皇帝,他居然瞒了那么久,那么不动声色,还能对着恂嫔,一切如常。

如懿想到此节,微微地笑了。皇帝甚是不悦,``皇后笑什么?''

如懿明眸微瞬,容色淡然,``皇上动心忍性,泰山崩于眼前而不乱。此等事情,自然不必悬于心。''

皇帝凝视她片刻,似乎在分辨她的语气里有多少真心的意味。片刻,他道:``恂嫔不去也不是。如今霍硕特部是她的异母兄长主持,还是那句话,人堆里见上一眼,不知道也罢了。''他顿一顿,``去木兰之事内务府会打点,后宫女眷事宜由令贵妃打点,你再过目便是。''他潦潦说罢,起身道,``朕还有些奏折处理,你先跪安吧。''

如懿答应着出去了,彼时晨阳高升,阶下草木无声,暑气渐渐迫人。偶尔有风经过,木叶相触之声萧萧漱漱,混作一片,恍如乱雨。如懿想,到底是要挨过夏末,到初秋去了。

\hypertarget{ux7b2cux5341ux4e8cux7ae0-ux6731ux8272ux70c8ux4e0b}{%
\chapter{第十二章
朱色烈(下)}\label{ux7b2cux5341ux4e8cux7ae0-ux6731ux8272ux70c8ux4e0b}}

虽是木兰秋狝,搭帐在外,皇帝的住处亦是精靡到了极处。空间既宏大,布置亦精巧,虽说精简再精简,到底也是皇家格局。帐篷的顶部举头可见绚烂夺目的贴金箔莲花纹天花蔓重重叠叠,累成天花乱坠模样,四壁皆是青蓝色蒙古样式的吉祥纹理,环环相扣,每行走一步,似乎就有迷乱不知所终之意。而嫔妃们的住处,也按着位分序列一一如是安排。

木兰秋狝是皇家旧规,皇帝素来遵从``习武木兰''之举,又性喜骑射,所以几乎年年都带王公大臣、八旗精兵与后妃子女至此。围猎二十余日后,皇帝必得举行盛大宴会,饮酒歌舞,摔跤比武,并宴请蒙古王公等,同享盛事。

木兰围场草原广亵,绿茵坦荡无际,天与云、与草原相融相连。每至晴空万里,天高云淡之际,茫茫林海捧出清晨红日,喷薄四射,霞光万道。或是日暮西山,残阳如血,亦生红河日下之感。

到了此处,皇帝骑马射猎,最喜携颖妃、豫妃、恂嫔、恪贵人等蒙古嫔妃,她们既青春少艾,又有飒爽英姿,一一换了鲜艳紧俏的袍服,艳美无俦。身边又有成年的皇子相随,除了已经出嗣的六阿哥永珞,便是永琪。彼时八阿哥永璇足上有疾,十一阿哥永理与十二阿哥永瑾同岁,都还年幼,只能拿着小弓骑着小马游戏。十四阿哥永璐与十五阿哥永琰更不足提,尚是怀抱小儿。如此一来,永琪更是风头大盛。

而如懿唯一的好处,便是宫规不那么严谨,可以常常见得永璂了。因着此回蒙古王公颇多,皇帝为示亲厚,多在颖妃、恪贵人处歇息,豫妃固然不得亲近天颜,恂嫔却是淡淡的不甚邀宠,皇帝也不愿多与她亲近了。只是无人时,恂嫔却也向李玉和永琪打听,
``为何此次狩猎'不见本宫父亲,却是异母哥哥来昵?''

永琪慧根早发,含笑谦恭道:
``恂娘娘安心,或者秋狝繁累,老王爷不来也是情理之中。''这般应付了,回头永琪便细细叮嘱海兰,顺带着告知如懿,``车马劳顿,除了皇阿玛召宴,这些日子额娘闭门不要见人,只安心休息便好,免得是非。''

如此,林海探幽,千骑飞驰,静则听百鸟啼鸣,动则射狍鹿奔突。皇帝收获颇多,众人溢美不绝,兴致更高。

这一日皇帝领着诸位皇子出去,皇帝独得了一只黑熊。永璂年幼,也射了一头狍子,皇帝神色淡淡的,也不肯多赞许一句。

恰恰和敬公主在旁,便郁郁不乐,``皇阿玛,儿臣记得端慧太子在世时,六岁便可行猎射得一只小鹿了。''

永瑾闻言越发颓丧,手足无措地望着如懿,垂首不语。皇帝未置可否,只道:``前些时日朕拘着你在养心殿读书,骑射上未免生疏了。罢了,回头叫你谙达多教你些。''永瑾诺诺答应了,想往如懿身边靠,眼见皇帝并不理会,只得垂头丧气立到海兰身前去了。

而永琪归来,只得老弱之物,皇帝便更不悦。永琪施礼,谦谦道:``我朝以马上得天下,儿臣不敢忘记祖训,所以有所射猎。但儿臣见母鹿幼兽颇为可怜,而壮年猛兽猎得虽可增荣光,但幼兽抚育皆赖壮者。想及野兽也有母子之情,儿臣不忍,一律放生,留其繁衍。''

这番话说得皇帝龙颜大悦,抚着永琪肩头道:``能文能武固然好,但有悲悯怜下的仁爱之心,朕更感欣慰。''说罢,便解下自己身上的双龙抢珠赤红缎披风披于永琪身上,``郊野风露,你小心身子。''

永琪欣然应允,恭谨谢过。如懿与海兰相视一笑,更是欣慰。然而永琪起身的一瞬,足下微微一僵,海兰正与皇帝说话,一时未曾察觉,如懿心念一动,趁着人不留意,便低声道:``永琪,你的腿怎么了?''

永琪面色微沉,不欲在人前多言,便道:``起初觉得寒热,仿佛感冒风邪.这两日一直奔波马上,有些筋骨疼痛,但不热不红,无甚症状。皇额娘放心,想必无大碍。''

如懿知他要强,在皇帝面前更不肯示弱呼痛,还是不大放心``本宫记得先帝时怡亲王允祥也曾有过这般病痛,你要格外仔细些。等晚膳过后,本宫着江与彬去瞧你。''

永琪见皇帝满面春风,如何肯扫这个兴,便恳求道:
``皇阿玛正在兴头上,若此刻传御医,当着各部王公的面,若有什么传言便不好了。''说罢又笑,``儿臣府里也有御医,回去瞧了便是。''

如懿回首,见皇帝正拉着永璂的手嘱咐着什么,也不敢多言,便答应着去了。

这一晚便在大帐外环坐饮宴。出宫在外,饮食不比宫内精细,反多了各色野味,将白日所猎获的禽物烹得鲜香可口,诸人更是饮酒助兴。清夜无尘,月色如银。更兼燃了无数篝火,有蒙古女子挥起五色长袖跳起歌舞,比之宫中的纤腰袅娜更有奔放热烈之意,引来喝彩声无数。如懿陪伴皇帝身侧,海兰与嬿婉分坐了左右两首。因着女眷们矜持,除了颖妃与嬿婉口齿伶俐说笑,其余人都懒懒的。恂嫔更是告了假,连晚宴都不曾出来。

酒过三巡,众人都有了薄薄的醉意,如懿不胜酒力,目光更眷着永瑾。海兰会意,便道:``皇上,十二阿哥累了,不如先随皇后娘娘回去。''

皇帝与王公们饮酒正酣,便挥了挥手。如懿欣喜若狂,忙牵着永璂退下了。

八月中旬的夜风已有了飒飒的凉意。如懿面红耳热,被风一扑,不觉已浸凉了衣襟。容珮便道:``皇后娘娘和十二阿哥走小路吧,回去近些,避避风也好。''

草原上无遮无拦,夜风吹拂,散落草木互相触碰后如海浪般晃迭的轻音。一轮圆月排云而出,月色熠熠洒落,照亮不远处的河岸上开着的轻盈的粉紫野花。

永瑾大大地松一口气,跳跃着像只小麻雀,``额娘额娘,今天儿子不用背书,师傅也不会查功课。真好!''他闭着眼睛深吸一口气,
``额娘,这里的花好香,甜甜的。我骑在马背上的时候只想着要猎点什么回去皇阿玛才高兴,都没闻到花有香味儿。''

如懿爱怜不已。永瑾也不过是个孩子,贪玩是孩子的本性,却要被牢牢拘着每日如个小大人般刻苦成熟,真真是难为了他。如懿牵着永瑾的手紧紧不肯放,依依道:``永璂,额娘很久没闻到宫外的气息了。你闻到没有,河水的气味是甘洌的,夹杂着花香。宫里的花儿朵儿都是精心培育的,带着匠气。这里的花,都是活泼泼的,无拘无束。''

永瑾嗯嗯啊啊地点着头,欢欢喜喜地好奇张望。容珮笑吟吟道:``宫外的人都艳羡宫里的富贵,宫里的人都盼着外头的自由。人都一样,得了这个,盼着那个。''

母子二人说笑着,便往帐篷深处走去。后头三五宫人引着灯追随,脚步声都漫在万叶千枝的风声里。

这一带都是宫女们所住的青帷帐篷,夜来都在御前服侍,一座座帐篷都空着,一星烛火也无,又靠近河边,格外昏暗。容珮低声道:``这儿不比外头好走,但绕过去离娘娘住的地方近。''

正说着,忽然见一个硕大的影子立在帐篷后,如懿骇了一跳,已有宫人失声唤起来,``莫不是撞上熊了?''

永瑾一吓,挡在如懿身前,粗声壮气道:``额娘,儿子在这里。''那影子似乎也受惊不小,立刻分开,便可辨出是两个人影,一高一矮,高者健硕,似乎是个壮年男子,穿着侍卫袍服。那矮的苗条纤秀,居然是宫装打扮。先前,他们竟是紧紧抱在一起。

这一惊可非同小可。想是哪个宫女与侍卫相好,躲在此处亲热。如懿将永璂护到身后,容珮扬起灯笼,厉声喝道:``是谁?''

便是想跑也来不及了,灯火明灭处,那女子分明是早先告假的恂嫔霍硕特蓝曦。四目相对处,她面上犹有泪痕,凄然沉痛,不似往日。那男子形容陌生,脸上亦有哀容。

永瑾探着头,先喊了一声,``恂娘娘。''

如懿深觉不妥,便按了按容佩的手,沉声道:``恂嫔,你在这里私会男子,你可不要命了么?''

那男子低声问:``这个女人是谁?''

恂嫔冷冷一笑,艳光四射,``咱们仇人的妻子。''她扬一扬头,并无惧色,``皇后,是你自己撞上来的。''

周遭唯闻草叶萧萧之声,泠泠似幽然泣声。如懿听得她语中狠辣之意,想要呼喊,才想起侍卫离这里都远。她缓和了惊惧之下僵硬的面颊,低声道:``你若要性命,速速离开,不要在此枉费唇舌。否则你是皇家嫔妃,你身边这个人便只有五马分尸之路!''

恂嫔与那男子对视一眼,似有犹豫之意,相望之间,无限爱怜珍重。

恂嫔迟疑,``你肯放过他?''

如懿压抑着心底的慌乱,沉静道:``要他离了这里,本宫未曾见过,你也未曾见过,各自相安。''这是最好的法子,也保全眼下的自己。

恂嫔正沉吟间,只听身后一声亮烈女声划破静谧夜空,将草木温润之声骤然撕裂,``有刺客!有刺客!''

如懿仓促转首,只见豫妃携着两名侍女惊惶大呼,奔得略远?如懿心下一凉,还来不及反应,一把雪亮长刀已然架在了永璂喉下,将永璂扯了过去.永璂吓得怔了,一张小脸雪白,张着嘴发不出声音。容珮不知被谁踢翻在地,一脸痛处,挣扎着要向永璂爬来。

恂嫔怒目而视,``是你带着豫妃来的?''

如懿连连摇头,
``本宫不知她为何跟在身后\ldots{}''她的一颗心剧烈地蹦着,沉沉地撕扯着痛,``你先放了永璂!他还小,什么都不懂!''

说话间,有不少侍卫提足奔跑之声传近,隐隐有兵刃出鞘。恂嫔咬着唇,气若无状,``阿诺达,来不及了!''

阿诺达持刀在后胁迫着永璂,沉着道:``蓝曦,你别怕!我既然敢来见你,便料到有这一日!当日我不能留你在部族,又不能在战场护你父亲周全,今日无论如何,一定要带你逃离这里,免得深受其苦。''

如懿听他碎言片语,便知是霍硕特部征战中活下来的人,又是霍硕特老王爷的亲信,心底陡然更寒了几分。恂嫔望着他,眸中情意沉沉,便有知心长相重。

她心急如焚,喃喃安抚着永瑾,生怕他一时大哭起来恼了阿诺达,一壁连声道:``永璂,你别怕!不要哭!不要哭!''

永璂怔怔地瞪着一双乌沉沉的眼睛,眼泪滴溜溜汪了满眼,死死忍着泪点点头,轻轻唤道:``额娘。''

如懿的心都快要绞碎了。她戚然求道:``永瑾只是个孩子,你挟持我,挟持我啊!放他过来,我是皇后,你挟持我他们或许能放了你。''

阿诺达迟疑片刻,恂嫔冷哼一声,``你虽然是皇后,可在皇帝眼里,咱们这些女子都如草芥一般。你这个皇后形同失宠,带着有什么用?''

阿诺达颔首,闷声道:``不错!你们的皇帝出了名的薄情寡性,他是怎么待蓝曦的,我都知道!你这个皇后也不过是个可怜虫!''

如懿仿佛被人当面狠狠掴了一掌,面皮火烧火燎着,这么多年,她也明白自己的可怜。至少还留着皇后虚尊的面,却从未有人敢当着她的面,这样清楚无误地挑明了出来,她不过也只是个可怜虫。

谁比谁低贱,谁又比谁高贵,都是一样的。

她顾不得这些,按捺着情急道:``纵使如此,一个孩子能抵什么?你伤了他皇上

更会要你的命!''

灯火越逼越近,几乎照清了阿诺达与恂嫔阴郁的面孔。兵刃声铮然作响,却谁也不敢上前,生怕误伤了皇子。阿诺达有恃无恐,挟持着永璂向恂嫔使了个眼色,恂嫔紧紧攥着他的衣角,二入慢慢向后退去。

彼时盛宴方才散去,蒙古王公们稀稀落落留着几个。皇帝虽然醉眼迷蒙,很快也被惊动,立时赶了过来。

如懿见着永璂小小的面孔早已无人色,犹自倔强着不肯哭出来,一颗心早揉得稀碎。远远见得暗沉夜里灯火挑明之中皇帝的明黄一色急急赶来,不知怎的,心下便安稳了许多。

因着事态紧急,皇帝先自赶来,后头跟着几个胆大的嫔妃。

皇帝扫了阿诺达一眼,根本不看恂嫔,气定神闲,``你也逃不出这里,不如放了朕的十二阿哥,你与恂嫔也自有个好下场。''

阿诺达鄙夷道:``你们爱新觉罗的人最会扯谎欺瞒。当年你有心让我们霍硕特部的族人清扫寒部残军,却不告知寒部余孽手中尚有火器,只让老王爷带精锐前往,也不派兵增援。否则我们霍硕特部的精锐怎会都折在了那场战事里?''

``兵器无眼,征战自有伤亡。我大清将士平定西陲无不如是。怎么你们霍硕特部便格外矜贵些?''

阿诺达双眼血红,愤怒不已,``明明是你不满老王爷曾同情你的敌人准噶尔都,才趁机剪除异己,捧了对你唯命是从的小王爷上位。可惜了我们霍硕特部的壮年,都为了你的阴谋私心枉死!''

皇帝斥道:``为朝廷尽心,怎算枉死!凭你这句话,便可诛心!''他肃然喝道:``来人!围住他们!''

恂嫔闻言,连忙护在阿诺达身前,喝道:``谁敢动我们!''她扬起细长的眉毛,神色凛冽,指着永璂道:``除非皇上肯背上杀子之名,那咱们便是一同死了也不枉!''

她说罢,咯咯地笑着。那清脆的声音落在风里像某种野兽的嘶鸣。

如懿的瞳孔紧缩着,面庞惨白。海兰紧紧扶住她的手,想要安慰,分明也失却了往日的沉定。

前头皇帝的面色愈加难看,他紧紧抿着唇,手指的关节因为用力而微微泛白.他看向恂嫔的目色带了肃杀之意,``婢子淫贱,脏了朕的后宫。''

恂嫔冷淡至极,``淫贱,还是宫里的人淫贱?我与阿诺达本是青梅竹马,为了保全霍硕特部我才不得不与他分离入宫。因为我们都知道,部族的利益永远高过自己。所以哪怕我一点儿都不喜欢你,我都会逼着自己面对你,侍奉你,对你恭顺,可是你是怎么对我们霍硕特部的?你害得我家破人亡,还蓄意隐瞒。那么我要离开这个地方,也是情理之中!''

``离开?''皇上略含讽刺,``生是紫禁城的人,死是紫禁城的鬼。你入宫前你的父亲没有教过你吗?''

``我为什么不走?''她言辞激烈,有太多压抑让她不快乐,终于在此刻释放,``我活在宫里'和容嫔一样,没有一刻是快乐的。我都觉得喘不过气来。如今我失去了我的父亲,我的部族,还要和你这个虚伪的男人在一起,让我觉得恶心!''她看着被阿诺达挟持的永璂,``用你儿子的性命,换我们的自由!''

皇帝缓和的语调中渗出丝丝阴郁,``你永远要记得,你是朕的人。放了永璂,朕会给你留条生路。''

恂嫔连连冷笑,``我是蒙古出身,好歹也是一族的公主。不比有些人,日日宣称是雍和宫出生,谁知是生在热河行宫里的。难怪年年秋狝,必得来这儿垂吊,略表孝心。这样表里不一的虚伪之人,我不愿与他相伴至死。''

众人听到此节,知她是暗指皇帝乃是热河行宫宫女李金桂所生,当年先帝误饮鹿血,一时情动临幸了卑贱宫女,才得了此子,为此还被康熙爷大为申斥。这一直是先帝生前羞事,更是皇帝最不能提的奇耻隐痛。宫中虽然人人暗知,却无人敢提,乃是禁中最大的忌讳。

嬿婉矍然变色,喝道:``贱婢无知,岂敢拿皇上身世胡言乱语?''

皇帝眼底闪过一抹感激与动容,面色的肌肉却隐隐抽搐。

恂嫔仰天笑道:``皇上,你还真当自己是与太后母慈子孝呢?这般天家母子,只为名分好看,底下的龌龊事还当旁人都是瞎子不知道么?皇帝若真要为天下仁孝的表率,那便追封李氏为圣母皇太后又如何?只不过怕天下人都耻笑自己是个宫女生的罢了。''

分明是猎猎秋风,拂上面却有彻骨的寒意。那一瞬间,如懿居然忘记了刀锋抵触在永瑾喉头的冷厉锋锐,只觉得一颗心突突地狂跳着,噔一下,又噔一下,用力地牵扯着,每一下,都那么痛。她死死地盯着皇帝的面孔,看着他雪白中泛着铁青的面色,看着他脸颊的肌肉剧烈地搐动,她没来由地觉得害怕,比自己命悬一线更加害怕。

这样隐秘的事,陡然公之于众,皇帝该要如何自处?

她太知道了,许多事,不能碰,不能说。哪怕是高高在上的帝王,亦有他的底线与痛处。

皇帝脸色铁青,如懿从未见过他如此骇人的模样。一时不知该如何反应。然而,更怕的是,皇帝若一时暴怒,那永瑾该如何是好?

她禁不住低唤:``皇上息怒!不是该生气的时候。''

皇帝眼神一扫,永琪已然会意,悄悄退后两步。

恂嫔满腔激愤,未曾稍有消减,``皇上不是一向自诩风流多情么?实则世间最无情之人,便是皇上你!豫妃年届三十,她父亲还一心希望她入宫,皇上嘴上说垂伶她,不计年纪纳她入宫,其实宠幸过后就把她扔在宫中自生自灭,只是需要时才装点门面!皇上若是多情,就不会把那么多的女人困在宫中名为雨露均沾实则作棋子利用!皇上若真的多情,就不会利用我母族剿灭寒部残军,趁机灭我部族精锐!我看不惯你们满口仁义双手染血!今日你要多情,你就拿你自己的命来换你儿子的命吧!''。

恂嫔激昂陈词,不知何时,永琪悄然掩身上前,以迅雷不及掩耳之势,将恂嫔挟持在手,以同样的姿势,举刀相向。

\hypertarget{ux7b2cux5341ux4e09ux7ae0-ux7ea2ux7c89ux610f}{%
\chapter{第十三章
红粉意}\label{ux7b2cux5341ux4e09ux7ae0-ux7ea2ux7c89ux610f}}

事出突然,根本无人反应过来。

永琪无比镇定,``一个换一个,别说你犯险来见恂嫔,会连她的命也不顾。''

阿诺达矍然变色,厉声喝道:``把蓝曦还给我!''

永琪气定神闲,``我要我的兄弟,你要这个女人,很公平。''

阿诺达的脸色变了又变,阴沉不定。恂嫔抵在永琪刀下,恋恋望向阿诺达,蚀骨

相思如丝如缕,眉间心上,早已无计回避。

那电光石火的一瞬,如懿终于懂得了恂嫔的心,她从未这般看过皇帝,从来没

有。难怪她一定要跟他走,便如那一曲苍凉缠绵的《朱色烈》,总要向着心爱的人奔

去。

永琪不疾不徐,``你冒险前来就是为了带恂嫔走,定然不舍得她死在我刀下。你

细想想,只要你不肯,皇阿玛只是失去其中一个皇子,你却失去了唯一的爱侣,值不

值得?''

恂嫔凄惶摇头,叫道:``阿诺达!别相信他们!你放了手中的人质,你我都不能

活。''

永琪笑而不语,只是挥手示意侍卫们退得更远,而自己挟着恂嫔跟随上前,手中

的银刀却勒紧了些许,嵌入恂嫔雪白皮肉之中。阿诺达神色悲痛,挟着永璂缓缓向草

原边缘退去。

夜色茫茫,如能吞噬一切。阿诺达眼见离得众人远些,喝道:``我跟你换!''

永琪颔首,稍稍松开手。阿诺达见他如此,手臂一松,将永璂狠狠推开,便要伸

手去拉永琪怀中的恂嫔。

永璂如逢大赦,才刚迈出两步,想是惊惶,吓得膝盖一软,扑倒在地。说时迟那

时快,皇帝已然搭弓在手,拉了满弦,霍然射出一箭。阿诺达离永璂不过两步远,立

时中箭,手臂尚能动。他双目瞪得通红,发出凄厉一声,举起匕首猱身便要扑向摔倒

的永璂。

永璂吓得人都傻了,眼见得寒光扑来,哪里还能反应。海兰惊呼一声,如懿唯觉

脑中一片白茫茫,像是下着纷纷扬扬的厉雪,将她整个人裹了进去,泪便滚滚落了下

来。她几乎是本能一般,朝着永瑾扑去,将他护在身下。

这是她唯一的孩子,哪怕拿了她的命去,也不能伤着永瑾半分。

电光石火间,她已然看见,那匕首落下的银锐的尖,离自己不过数寸远。听着此

起彼伏的惊呼声,她等待着不能逃脱的锋刃的刺入。却是有一股巨大的劲力盖在自己

身后,以及,利器刺穿皮肉的闷响。

居然,没有一丝疼痛。

那么,那声音,从何而来?

转过身去,才发现阿诺达已然横倒于地。如懿从惊悸里抬起头,先去看怀中的永

瑾。永瑾紧紧地拥着她的手臂,眼泪流了下来,``额娘。''

她细细察看,一切无恙,除了受惊的模样,一点伤痕都没有。她飘落云外的心回

来了一半,把永璂抱个不够。须臾,她终于回过神来,有高大的身影挡在她身前,让

她看不见任何危险的痕迹。那暗沉的蓝色.是御前侍卫的服色。

她的心思定了又定,是凌云彻。她定神看去,才见他肩头血流汩汩,染红了半边

袖子,自然而然沾到她身上。显然方才阿诺达那一刀,是他替他们母子挡了下来.

海兰与容珮急急赶上前来,侍卫们架着倒在地上的阿诺达将其拖开,海兰看着她轻轻啜泣,容珮护着永瑾。如懿的心一下一下重重地抽搐着,她的声调都在颤抖,``要不要紧?''

凌云彻抿着嘴唇,沉默地摇摇头。他并无痛楚之色,从容而坦然,是天边皎洁的明月光。他低声道:``你们平安就好。''

那一刻,永瑾、如懿、凌云彻,他们三人彼此相依。心与心的距离,由天涯至彼端,如此遥远,又如此贴近。

天地孤清,生命亦渺小。但奋不顾身可以来相救的,唯有这个人,而那个名正言顺可以来救自己的,本该伴在自己身边的男子,仍是这般丰神俊朗,却是立在一群花容失色的嫔妃中间,遥遥望着自己,目光中有沉沉的急切。

飞身相救与一个急切的眼神,哪个更值得依靠?

她在清醒中,混沌地流下泪来。

可以真正在身边的,原来一直都不在。

就如冷宫那一段煎熬的岁月,倚墙想靠的,也唯有一个凌云彻而已。

然而她未及多想,永琪已然上前,恭敬地请她,``皇额娘与十二弟是否安好?赶紧请太医瞧瞧才是。''

如懿见他沉稳走来,转眸看去,却见恂嫔亦倒在地上。永琪见如懿注目,轻轻一笑,轻松道:
``解决了。儿臣不会容这般逆贼伤害皇额娘与十二弟。''

果然,恂嫔胸腔上有血液喷薄而出,溅了满地,如盛开的野芳。她尚有一口气在,芳钿委地,落红残碎。

永琪沉定如山,口吻却轻松,``这种损害皇阿玛清誉的人,留不得。只是污了皇额娘的眼,可见她连死也有罪过。''

这样的淡然决绝,大抵是皇帝所欣赏的,也是她与海兰多年教导的期望。可是这一刻,她却觉得眼前的永琪如此陌生。

所有人都是陌生的,在素日的熟悉与了解之外。大概人在险境,才看得清另一面。

海兰有些警觉,不动声色地扶着如懿距离凌云彻远些,再远些,口中温婉而客气,``凌大人护主有功,皇上自当奖赏。''

这一语,是泾渭分明的尊卑。

凌云彻拱手,转身向皇帝屈膝``皇上,微臣护主不力,以致皇后娘娘与十二阿哥饱受惊吓,还请皇上恕罪。''

皇帝徐徐舒一口气,``皇后母子无碍便好。''

凌云彻躬身退至一边。皇帝伸出手臂,温和道:``皇后饱受惊吓,快过来吧。''

凉风习习,几能透骨。她站在那里,居然一步也迈不开,似是牢牢定在了原地。

她真希望自己只是长在这茫茫草原的一株细草,无知无觉到老。

海兰轻轻推了推她的手臂,她还是没法动弹一下,直到有挣扎爬行的声音,挑动她已然麻木的神经。

目光落定处,只见恂嫔的胸前汩汩流出鲜红的血液,如一眼红色的泉,流溢不断,将胸口锦衣重重染透。血腥气逐渐弥散。她气息微弱,身体一颤一颤抽动着,犹自睁大了双眼,死死盯着阿诺达的尸身,不肯移开半分。

她回眸轻轻一笑,将皇帝隐隐的怒意满意地收入眼底,瞟一眼凌云彻,缓缓道:``皇上,你看你,在自己妻儿面前,还不如---个侍卫抵用。所以我哪怕死,也要离你远远的。''

她说着,吃力地挪动着身体,每动一寸,鲜血涌出更多,在浓绿的草叶上染下触目的痕迹。她艰难地挪到阿诺达身边,伸出手合上他望向自己的僵冷的眼皮。她的手势温柔极了,像爱护着毕生的珍宝。她的气息愈加无力,几近力竭。她微笑着,像一朵烈烈绽放的木棉,将自己的躯体依偎到阿诺达怀中,长长地舒出一口气,含笑逝去,再无牵挂。

皇帝默默看着眼前一切,额上青筋粗烈暴起,喝道:``五马分尸!将此贱奴二人五马分尸!''

侍卫们响亮地答应着,伸手便去拖开二人,豫妃微翘着嘴唇,含了冰尖似的笑意,嘶嘶然冷笑,``奸夫淫妇,死不足惜。''

皇帝也不看她,``的确死不足惜。便是死上千遍,也难以泄恨。''他一顿,``吩咐下去,恂嫔霍硕特氏突发急病,薨于行在。''

他的语底是森森的杀意,嬿婉纵然得宠,也不觉打了个寒噤,悄然退开了半分,一双烟波妙目,只定在凌云彻身上,眼见他面色白了又白,心中酸涩更浓,须臾间,皇帝的目光如冷箭一般幽幽扫着凌云彻,``御前侍卫凌云彻救护皇后与皇子有功,赏黄马褂一件。''他轻声垂问:``皇后,你和永璂还好吧?''

她的心底冷如万丈寒冰,彻头彻尾弥漫至四肢百骸的每一缝隙,偏偏还要维持着最得体端和的笑容,双眸低垂,轻声道:``都好。''金步摇在鬓角上摇曳起粼粼的珠光,更显得一张脸剔透得仿佛在发着幽幽的光泽。可惜,那光泽是幽暗的阴沉,一如她此时的心境。

皇帝走近两步,摸了摸永瑾的头,示意容珮带着离开,便挽过如懿的手,``起风了,别站在这儿。回朕的大帐去。''

这是许久未曾有的亲近。

嬿婉领着众人立在后头,知趣道:``臣妾等恭送皇上皇后。''

如懿的手被他握在掌心,是腻湿的冰凉。那是她手心的汗水,在惊惧无助的那一刻所留的印迹,浑不如他的手心,温暖而干燥。她忍了又忍,轻轻地抽出自己的手,抑起脸低低道``皇上便要射杀阿诺达,何必急在一时,如此沉不住气,拿永璂性命犯险!臣妾死不足惜,可永璂是您的嫡亲儿子!''

皇帝错愕地转首望着她,目光由温热转凉.他携着她,继续目视前方,``朕的嫡亲儿子,没有那么无用的。若是永琏在,便会机敏自保,便是永琪年幼时,也不会这般无用。''他仰天长叹,骤然声如洪钟,``龙生龙凤生凤,为何朕与你所生的儿子这般平庸!''

不过简短一语,身后所有人都被惊住。人人色变,望着帝后不知所措。

如懿如遭雷击,她居然没有听到自己心跳的声音。连那种牵扯般的疼痛,都感受不到了。她回首看着数步之遥处,一脸委屈的永璂,只觉得荒谬而酸楚,``纵然永璂资质不如永琪,但孩儿家幼小敏感,无不将父母之言视若天命,如何能这般当着人诋毁!叫永璂来日如何做人!''

如懿心头一阵恶浪翻涌,冷然道:``皇上天纵英明,永瑾如何能比!''

豫妃听到此节,仗着这几日皇帝顾她颜面,疾走几步,腰肢一摆,扭上前来,扬着绢子道:``哎呀!皇上说得是,虽说是龙生龙,可若配的不是凤凰而是山鸡,那哪里还能生出好的来!''

皇帝也不理她,只是负手在后,郁然叹息,``若永琏与永琮在此,有孝贤皇后的温淑品性悉心教导,也不致朕今日膝下荒芜。''

只这一语,便是将诸子都撂下了。

还是永琪机警,立刻跪下道:``今日之祸,都是儿臣不察。但请皇阿玛息怒,儿

臣一定严加防范,再不许有此等惊扰圣驾之事。''

皇帝轻轻``唔''了一声,温和道``你是朕的好儿子。今日料理霍硕特氏,也是你当机立断。''

永琪谢恩起身,揽过满脸惊愕与委屈的永璂,道:``十二弟年幼,未曾见过如此场面,难免受惊吓,儿臣会带十二弟回去加以劝慰。往后也会多带十二弟骑马射箭,

忘祖宗马上得天下。''

皇帝微微颔首。如懿见豫妃在侧,愈发厌恶。她未曾察觉自己语气的青锋锐气,蓦然盯着一壁快意的豫妃,呵斥道:``有功该赏,有罪当罚!豫妃,你可知罪?''

豫妃一怔,扬一扬骄傲的头颅,娇声呖呖道:``皇后娘娘,臣妾发现刺客,事先鸣警,护着皇上,有何罪过?''

如懿面色冷峻,一头乌黑的长发高髻绾起,横簪的一支凌空欲飞的九风金步摇震颤不已,曳出迷离碎光,``若不是你贸然出声,永璂怎会被挟持,险险丧命!你以皇家子嗣为赌注,不能沉住气定住神,若是刺客因你贸然疾呼暴起,伤了皇上,又该当何论?''

豫妃哪里肯服气,强辩道:``皇上有天神护佑,万事平安!''

如懿冷然道
:``是么?天子安危,子嗣安危,岂可以你区区之身而犯险!恂嫔与阿诺达犯事在先,可一场泼天风波,终究由你而起。来人,给本宫狠狠掌她的嘴,务必要她记住今日教训。''

豫妃见皇帝漠然无视,也生了怕意,登时跪下,呜咽着道:``皇上,皇后娘娘曲解臣妾\ldots{}''

皇帝哪里容她说完,右手微伸,己然扶住了颖妃手臂,道:``朕倦得很,去你那儿。''他头也不回,``令贵妃,罚完了豫妃,照旧送回宫里去。''

嬿婉曲折纤腰,柔柔道:``是。是否照旧禁足?''

皇帝道:``要行责罚是皇后的职责,至于禁足,不必了。''

颖妃欢喜着,忙拥着皇帝去了。只余呆若木鸡的豫妃留在当地,不知是悲是喜。

草原上风声猎猎,如懿紧紧抱着永璂,沉声道:``动手。''

所谓的掌嘴有两种,一种是批颊打脸,是寻常责罚,另一种是用三寸长乌木板击打嘴唇。那乌木板质地坚实,打下去便会肿胀,再者皮肉破裂,牙齿脱落。容珮从未见如懿动过如此大怒,立即从三宝手中接过乌木板,卷起衣袖便开始动手。豫妃吓得魂飞魄散,挣扎着要求饶,两个小太监立时上去死死架住了她,又防她痛呼乱骂,便拿白绸子勒住了嘴,容珮举手便打。

皇帝虽然离去,嫔妃们皆在,眼见乌木板与娇嫩的皮肉相触,溅起点点的血珠子。嬿婉不知含了哪门子怒气,亦僵着脸不肯求情。众人见皇后与贵妃都没好气色,又不喜豫妃从前的乔张做致,更无人肯求情。豫妃扭动着躲避,可哪里避得过,容珮下手既狠又准,毫不留情,直打得血沫飞溅,一声闷响,竟是豫妃的门牙和着鲜血落了下来,嘣地坠在地上,又跳了两跳,血糊糊白碌碌地滚了开去。

恪贵人胆小,吓得惊呼一声,躲到海兰身后。海兰温和地拍拍她的手,回首柔声道:``规矩已经做了。皇后娘娘莫再动气,明早请贵妃做主将豫妃妹妹送回去吧。''

嬿婉面无表情,``愉妃姐姐说得是。''她目视豫妃,如视尘芥般轻渺,``牙齿倒易补上。不过豫妃也当记得,什么话该说,什么话不该说了。''

说罢,如懿先起身,众人径自离去,只丢下豫妃一人,又怒又怕,哀哀哭倒在地。

嬿婉回到帐中,一张芙蓉秀面冷冷沉下,气息深长而压抑。春婵见得她神色不好,忙遣了众人出去,殷殷端上一碗樱桃酥酪来。那牛乳凝膏如雪,樱桃是今岁的末茬时鲜制成了干果,一粒粒便如鲜红珊瑚珠一般,仍不失甜美醇厚之味,惹人垂涎。

春婵小心觑着她脸色道:``小主,喝碗酥酪润润喉咙吧。方才受了那场惊吓\ldots。''

嬿婉厉声道:``是惊吓!本宫还没想到他不要命到这种地步!''她的声音尖厉,虽然极力压低,却像碎瓷片锋利地划过,拖起尖长的尾音,``都怪豫妃这个贱婢,生出这些事端!真是贱人是非多!''

嬿婉抄起春婵手上的酥酪盏,手高高举起,便欲向地下掼去。春婵吓得跪下,急道:``小主,今夜风波太多,您别再惊了圣驾。''

这话极是有理。嬿婉已是数子之母,又有协理六宫之责,位高权重。一时惊动起来,便又是一场风波。嬿婉面上一搐,极力克制着慢慢放下来,若无其事地道:``这酥酪凉了,撤了吧。''

她说罢,气犹未解,``皇上如何这般心软了。贱婢轻狂,合该送回去禁足,关她个不见天日才好!''

春婵微露喜色,``小主不觉得,皇上宽纵豫妃,是因着皇后娘娘在皇上心里的分量又轻了么?''

嬿婉一怔,旋即明白过来,轻嘘道:``也许吧。可怜了凌云彻,拼命救了一个皇上不看重的女人,他又值什么?难道眼里、心里,对她就这般放不下了么?''

嬿婉别过脸去,眼角闪烁一点晶亮,春婵正以为是今日敷面施妆所用的迎蝶粉里所研磨的珍珠过多,才这般妍亮。待定睛瞧去,才发觉是一滴晶莹的泪珠,薄薄垂在靥边,绵延坠落。

春婵吓得心惊肉跳,半晌不敢抬头去看。也不知过了多久,嬿婉沉声道:``本宫的妆匣呢?''

春婵利索去取来了,那是一个檀香木的双层小妆匣,贴着薄薄的合欢同喜的金箔花样,镶点着色色雪白的小米珠,极是精致华丽。因是夜深,帐中只秉着数盏小小的油灯,昏暗暗照得双眼发涩。嬿婉纤手一扬,匣子开启,春婵只觉得满目珠光,哪里睁得开眼。那匣子里累累堆着数粒拇指大的祖母绿,玻璃莹翠。翡翠兼冰种与翠种二色,如静水沉沉,汪在匣中。珍珠之物更是散落其间,难计其数,只粒粒浑圆,金黄润泽,是海中所产的金珠。另有红、蓝宝石与双色西瓜碧玺散在那里,都是难得之物。

春婵知道嬿婉素来爱惜此等珍物,兼着她复宠之后连连生育,皇帝欣悦,又赏赐不少,加之她历年邀宠所有,实在不少。然而嬿婉的目光稍一留恋,打开最底下一个屉子,摸出一个暗格,取出一枚银戒指。

春婵眼尖,一眼瞧出上面的红宝石不过是用残碎的红宝石屑磨粉制成,虽然也是鲜艳的红色,但光华凋谢,毫无华彩,着实不值几个钱。便是放在这个匣中,也是玷污了那些名贵珠翠。哪里比得上那几块鸽子蛋大小的血红宝石,华彩熠熠,光色流转。

但是春婵是认得的,偶尔,极其难得的时候,嬿婉会取出这枚戒指,戴在指上。譬如,她刚侍候嬿婉侍寝的前一日;譬如,那一年凌云彻被唤进永寿宫的时候;譬如,嬿婉发觉凌云彻对皇后的眼神有异的时候。她不敢去想,也不愿去想,那些隐秘而诡异的陈年秘事。那些匪夷所思的过往,恰如这枚戒指此刻被嬿婉戴在保养得如春葱般的纤纤手指上。

春婵终于忍不住道:``小主,您看那块鸽血红的宝石,若是叫内务府制成戒指,衬着您肤色白皙,最能显出红宝石的光艳剔透来。''

嬿婉低着头,若有所思,轻轻抚着指上的宝石粉戒指,``有些东西起于微时,虽

然粗鄙,戴一戴也无妨。也好提醒本宫别忘了旧时来路。''

春婵素来知道这位主子最忌讳旁人提她的宫人出身,罪臣之女。如今自己提起来,她也讪讪不好接口,只得委婉劝道:``小主与凌大人有往日旧谊,小主心慈,自然怜悯凌大人今日险境。只是凌大人救皇后有功,自然平步青云,小主无须担心。''

嬿婉眼底一红,旋即别过头,攥着手里的绢子道:``他是平步青云还是自毁前程,本宫怕他自己都分不清楚。在皇上面前这般逞强,不顾一切去救皇后和十二阿哥,岂不是显得皇上凉薄\ldots{}''

春婵机敏道:``是啊!凌大人都不顾一切了,小主还顾什么呢?''嬿婉一怔,泪汪汪望着春婵,春婵低低柔声,``损了凌云彻一个,便可以彻底扳倒皇后.再不济,总也动摇了皇后的根本。小主可千万别忘了魏夫人临终前的叮咛啊。''

嬿婉静一静,冷然道:``奸夫淫妇也真是无用,挟持了永瑾,也不能一了百了。一块儿死了才好呢。''

春婵沉静道:``虽然是失宠的皇后的儿子,到底也有嫡子的名分,一块儿了了,咱们的小阿哥才有指望啊。真是可惜了。所以,来日的事,咱们还是指望自己,指望不上别人呢。''

喧嚣已去,夜静到了深处,草原上虫声密密唧唧,清晰入耳.风拂幽凉,吹得帐幕微微鼓起,如起伏的浪潮。那灯光便又忽闪了几下.嬿婉沉默不言.一张清水面孔郁阴沉了下去。

\hypertarget{ux7b2cux5341ux56dbux7ae0-ux6728ux5170ux60c5}{%
\chapter{第十四章
木兰情}\label{ux7b2cux5341ux56dbux7ae0-ux6728ux5170ux60c5}}

永璂受了这般委屈惊吓,当晚便发起了高热,嘟囔着胡话,神志模糊。小小的人儿,烧得满脸通红,只是含糊不清地道:``额娘!我不怕!不怕!''说着又胡乱挥手,``额娘!您别怪儿子!儿子没有给您争气!''

如懿眼看着璟兕与永璂夭折在怀中,如何还受得起这般折磨,一副柔肠都要搓磨碎了。好在海兰还镇定,一壁唤来太医,一壁命三宝去请皇帝。已是更深露重,如懿黯然道:``皇上歇在颖妃那里,此时去请,只怕皇上不悦。''

海兰跺了跺脚,恼道:``这个时候难道还顾着皇上春宵风流?永瑾是嫡子,若是伤着什么,可如何是好?''她看一眼立在一旁的永琪,咬了咬牙道:``三宝只是个奴才,只怕见不到皇上。若是碰上进忠那起子小人作祟,又是一场气受。永琪,便是你去!''

永琪有些不知所措,搓着手迟疑道:``额娘!儿子是臣下,又是晚辈,去皇阿玛嫔妃帐外,似是不妥。''

海兰急道:``再不妥,躺在这儿的是你亲弟弟,也是你皇阿玛唯一的嫡子。你不疼他护他,还能有谁?''

永琪的脸色微微一沉,但见生母与嫡母都慌了神,只得道:``那儿子立刻就去。''

永琪才出去,江与彬已经掀了大帐的帘子进来,利索地请了安,道:``皇后娘娘万福,愉妃小主万福。''

如懿焦灼不安,``不必拘礼,先去看永瑾!''她低首,见江与彬指尖犹有来洗净的血痕,旋即明白他从何处而来,便问:``凌云彻如何了?''

江与彬和缓道:``皇后娘娘送去的金疮药已然用了。但凌大人伤在肩胛,伤重透骨,只怕伤愈以后,逢到寒湿天气,都会有隐痛。''

如懿鼻尖一酸,那酸楚的隐痛轻绵得没有着落处,纠缠到心腑五脏间去,牵绊出一缕难以言喻的柔软,柔软至无力。

她一直辗转于尘埃浑浊里,唯有他一心扑来,心地明净纯挚,许她一缕洁白干净的照耀。思绪起伏间,眼底隐然有泪光。海兰温然笑劝,``姐姐这是担心皇上了,方才姐姐还在说,若是身受这一刀的是皇上,那该如何是好?可怜姐姐身为皇后,又要为十二阿哥担忧,又为皇上忧心,还系着后宫的安宁,实在是为难。''

江与彬略一沉吟,``如今是令贵妃协理后宫,门禁不严才惹来大祸。皇后娘娘一直静心避世,当然不干皇后娘娘的事。''

海兰投去一个赞许的目光,如懿颔首道:``江太医的话发人深省,与医术一般高明。快请移步去瞧瞧永璂吧。''

江与彬拎着药箱疾步走进,搭了脉,看了舌苔,一番望闻问切,方才缓了眉心沉重的曲折,道:``十二阿哥是惊风了。''

如懿未闻此名,急得攥紧了绢子,``是什么症候?''

江与彬道:``惊风乃外感时邪,暴受惊恐所致。小儿神气怯弱,元气未充,不耐意外刺激,若暴受惊恐,使神明受扰,肝风内动,便会有此症。微臣立即开药方为阿哥延治。''

如懿喉头一松,语调终复如常,``有你这句话,本宫放心许多。''

正说着,永琪进来,束手立在一旁。如懿见他颇有懊恼之色,已然猜到几分,心下更凉。海兰便问:``你皇阿玛呢?''

永琪踌躇片刻,道:``颖娘娘听闻十二弟抱病,也不敢阻拦。是皇阿玛,皇阿玛说夜来困乏,先不过来了。''

深掩的帐帷挡住了幽咽风声,任它游走于月色如霜的荒野中。皇帝的面容在如懿的脑海里瞬间变得遥远而陌生,心底有绝望的哀凉恣意生长。

如懿领首,庄重之色无可挑剔,``龙体为重,是本宫疏忽了。夜深你劳碌一日,先去歇息吧。''

见永琪退出,江与彬又道:``行在里应备着琥珀抱龙丸,有镇惊安神之效,可先用温水化了服下。微臣还会开些人参、甘草益气扶正;菖蒲、石决明熄风开窍,不过此病可大可小,阿哥身边一定要有妥当之人细心照拂。''

如懿连连答应了,江与彬便叫跟着的小太监取了药丸来化了,亲眼见永瑾服下。如懿才叫容珮跟着下去取药方,自己则守在永瑾身边,握着他的手,细细为他擦拭额上汗水,潸然落下泪来,``海兰,终究是我无用,护不住自己的孩子。''

海兰怜惜地在她身边,温柔道:``姐姐别这样说。做阿玛的都没有担当,叫一小女子该当如何?''

心底轰然一声,一种无可依靠、临危被弃的怨与恨,再次沉沉袭来。如懿撑着目眶,泪意逼得眼底通红,挑起不堪言的沉痛,``海兰,为什么我们的夫君,在危难之时,连一双可以依靠躲避的臂膀也无。我们苦苦依傍着这个男人,争夺那一点点恩宠,到底是为了什么?只是为了大难来时,他的袖手旁观么?''

海兰眸底乌沉,冷峭道:``刘邦与项羽夺天下时,可以嫌自己与吕后所生的一双儿女累赘,数次踢下车去。这般薄情,最后还不是君临天下?谁会计较这些。姐姐,我们能依靠的,唯有自己。''

如懿含泪,反问道:``可是身在这里,不得不仰人鼻息。你我早年入宫,所有学会的一切都只是怎么在宫里活下去,活得好。我知道你也许怪我,今日初发现阿诺达与恂嫔时,我曾有一念姑息,希望他们可以逃出去。恂嫔的确胆大妄为,可她留在宫里又有什么意义?舍弃自己,舍弃青梅竹马的恋人想要求得族人的平安都不能。留在宫里,等待她的除了无宠的孤独和悲凉,还有什么?皇上逼得她家破人亡,却连一丝惭愧也无,对着这样的人,如何能安然活下去?''

似有若无的叹息,在一盏盏跳跃不定的烛火明灭中沉沉拂落。海兰压低了声音不无担忧,``姐姐,难道你是羡慕恂嫔有阿诺达?''

如懿恻然摆首,``怎会?我从陪在皇上身边那一刻起,便知道,我这一世可以有的男子,可以依靠的男子,只有他一人。我所有的荣辱悲喜,都只在他一念之间。曾几何时,生儿育女也罢,争权夺利也罢,到头来只是希望在他身边可以长久些,更长久些。可是如今,我只羡慕,恂嫔有离开这个地方的机会。''

海兰眸光一凉,神色黯淡了下来,``姐姐想去哪里?''

幽静的烛光一芯芯暗红地浮漫在帐幕上,像是映在灰白的江水涟漪里,冷清出奇。灯笼的暖红化开了暗夜的沉寂与阴森,将一双身影长长曳在地上,愈加凄清。

如懿郁郁道:``自进紫禁城,我早已无处可去。所以总是忍不住遥想,离开了重重的守卫,外面的天是否是纯净的蓝色?不像我们在宫苑里所见的四四方方一块。外面的日子是怎么过的?油盐酱醋虽然琐碎,是否也日曰平凡而温馨?''

言语间总是寂寥。若是这一生过得平安顺遂,何来这些小小的期盼,可以脱出自由身,得一息安乐。如此想着,海兰也沉默了。

不知过了多久,海兰仰起面来,忽然挣出两朵灿烂的笑靥,起身道:``皇上。''

如懿转首看去,不知何时皇帝已然到来,立在帐边,无声地凝视着榻上的永璂。

如懿亦起身,与海兰一同请了安。皇帝挥了挥手,``愉妃,你也累了,退下吧。''

海兰知道皇帝有意独自与如懿说话,递了个惴惴的眼神,忙离开了。

侍奉的人早被打发了下去,如懿便自己倒了热茶递上,``夜来风寒,皇上还是来了。''

皇帝简短道:``本不想来,但总还有些挂心。''皇帝径自走到永璂身边坐下,抚着永璂的额头仔细端详道,``这孩子,睡着了也皱着眉头,总不安乐的样子。''

不是不心酸的。永璂的年纪正是半懂不懂的时候,这些日子被送在海兰身边抚养,眼看着自己受了皇帝的冷落,他如何不明白些许冷暖之情?小小年纪便要承受这些,却隐忍不能对人言,也是他享着泼天富贵之余不能负担的重荷吧。

皇帝的手指缓缓地抚摸着,循序至嘴角,忧声道:``朕记得永琏小时候很爱笑,可是孝贤皇后重规矩,日日训导,永琏也不太活泼了。虽然稳重,但总有点老气横秋。永琮一生下来就多病痛,一半儿奶一半儿药喂养的,笑得更少。朕真的很希望,自己的孩子可以高兴些,再高兴些。''

他的语气很少这样柔和,是一种颓丧的柔和,让人酸楚,他继续说着:``朕有过很多个皇子。去了的永琏和永琮,是朕最期盼的嫡子。可惜他们都天寿无延。永璜的野心太重,永璋懦弱无能,永碱被他额娘金氏引到了邪路上,和永瑢一样只能出嗣。永璇已经伤了脚,永瑆一味贪玩。永璐和永琰尚是黄口小儿。朕将至知天命之年,膝下唯有永琪一个成器,还有永璂这个嫡子。''

如懿接口道:``永琪文武双全,行事妥帖周全,是个难得的人才。''

皇帝感慨不已:``是。永琪是很好,唯一所缺的只是一个嫡出的身份.因此朕更对永璂寄予厚望,希望他可以有永琪的天分与勤学,哪怕有一半也好。''

如懿哽咽难言,一口气抵在喉间,上不得,下不来。永琪固然是她的骄傲与心血,永瑾也是她十月怀胎一朝痛楚所得的瑰宝。她极力平复着心绪,道:``皇上所言,自然是对永起有无限指望。臣妾想着,哪怕他不能担负皇上心中的重托,若是能
以一已之力成为朝廷的栋梁,尽辅佐之力,也是好的。''

正说话间,容珮端了药进来,一见皇帝在此,忙行礼问安,皇帝道:``汤药搁下,出去吧。''

容珮急忙退出,如懿端起汤药,轻轻吹着,细心喂到永璂唇边。药汁顺着他的口落至咽喉,并无呕吐的迹象。如懿稍稍心安,拿绢子擦拭了永瑾唇边药迹,复又一点一点喂进。

皇帝看她无微不至,也不觉有几分心软,然而见永璂这般病弱,不觉又蹙眉:``朕对你的儿子也算是悉心教导,这些日子来都亲自带在身边。可惜这孩子天资有限,永琏和永琮在时\ldots{}''

如懿硬生生忍着气喂着汤药,听得心头如刀铰一般,实在忍无可忍,``臣妾的儿子?皇上,天资有别,永瑾或许不如旁人,臣妾也无话可说,总之是辜负了您的心意。来日他若好,自然是爱新觉罗的子孙,便是不好,又能只把他归于乌拉那拉氏么?''

皇帝听她口气冷硬,丝毫不肯服软婉转,也不觉有气,``永琏和永琮的好,自然是有孝贤皇后谆谆教导,费尽心力。''

如懿见一碗汤药喂到了底,那乌沉沉的药汁,搅起了底下的残渣,泛着辛苦的气息。她的口舌里全是这种辛辣苦涩,便跪下道:``永璂不好,皇上大可看作是臣妾无德无能,既非大家出身,也无德容言功的修养。可永璂到底是您的儿子,纵有不是,何必人前贬低,又是在他饱受惊吓的时候。若您能好好安慰他几句,全了父子之情,孩子也不致惊吓委屈到如此地步。''

皇帝默然片刻,``永璂被挟持,朕何尝不心疼?可当着人前,他这般无用,朕如何不寒心?''

如懿绷在面上的笑意渺漫如烟云,带着蒙蒙的雨气,``臣妾才真真是寒心!永璂不过九岁,还是懵懂稚子。于您心中,到底是孩子的平安康健要紧,还是人前的颜面要紧?是舐犊情深要紧,还是君臣颜面要紧?''她戚然落泪,逼视着他,并无退却之意,``皇上,臣妾有时候真的不懂,您心中真正在意的,到底是什么?''

皇帝目光如剑,朗朗然掷地有声,``朕要的不仅是一个皇子,更是帝国的继承者。''他的面上闪过一丝痛心与焦灼,``有能者非嫡出,嫡出者力不及,朕如何能不忧心忡忡!''他静了片刻,冷冷道,``皇后,朕让你静心思过,看来你还是未曾改了自己这等疾言厉色的过错。''

一颗狂跳至错乱的心静静定了下来,如懿叩首,
``皇上,臣妾知错。但臣妾一直以为,臣妾的直言是皇上所在意的。夫妻君臣,无不可直言。''

皇帝无声垂下眼险,投出两弯深青色的阴影,``皇后,朕是皇帝!''

如懿沉静相对,``皇上,您是人父,也是人夫!''

``放肆!''他的呵斥声是累累的磐石,滚滚坠下,``别以为你是皇后!
皇后也是奴才,你们都是朕的奴才!别妄想干涉朕,动摇朕!''

是什么东西,被无声地碾得粉碎。心中纠结的爱怨痴嗔,伴着一声复一声的刻漏。从心上残忍地镇压,再无重圆的可能。

她唇角挑起一丝冷笑,干涸的眼底有冷焰跳跃,``皇上说得真好!金玉良言,臣妾受教了!''

皇帝盯着她,似乎要迫到她的眼底心内,``有两句话,朕好好教了你。你牢牢记住。一句是凡事三思。你今日在这个位置,就是朕的皇后。皇后是朕的女人,也不过是后宫一个品衔官位,和前朝的文臣武将没什么区别。孔夫子云`吾日三省吾身',说的就是要常思己过,知道自己的分寸。朕再教你一句话,这句话只有两个字,`顺服',你是皇后,你顺服则是嫔妃顺服。朕立你为皇后,便是要你做后宫的表率,天下女子的表率。''

他说罢,再不顾如懿,拂袖离去。唯余她跪在坚冷的地上,寒意浸浸,蚀骨灭身。

直至木兰秋狝回宫,直至永璂病愈,复被送至海兰身边养育,直至如懿再度避世于翊坤宫中,她没有再与皇帝有一言的交集。心里反反复复念着的,是从前读过的一句诗,``与我偕老,老使我怨''。年少时未曾期许过的,连失望时也未曾想过,原来他是这样自负,自负至凉薄的人。

恂嫔的死也无人再提起,迅速湮没于秋狝后盛宴举杯的欢浪里。左右她的生与死都逃不开紫禁城重重红墙的禁锢,依旧按着恂嫔的名位,草草下葬。

那仿佛也是她日后的收梢,永远看不见光明的尾巴。

偶尔的安慰是,在秋狝回銮的途中,遥遥望见凌云彻的背影,如远山巍峨,心里便定了又定。还好,还有他在。

并无说话的机会,也不欲在此点眼。凌云彻虽然救了他们母子,可皇帝并不那么喜欢,赏赐归赏赐,却连一句安慰褒奖的话也没有。可不是,谁喜欢用旁人的英勇气概来彰显自己的自私凉薄呢?

海兰亦常常陪在她身边,她更不喜凌云彻靠近。保持着刻意的距离,维持着尊卑的高低,除了眼神流转的交集,知道彼此都是无恙,便是最好的安慰了。

过了初秋便是深秋,连着初冬,京城的冷意总是来得迅疾且不动声色。画堂深锁,肌骨暗销,因着这料峭的寒意而显得合宜了许多。左右皇帝的恩宠,都只留在了宝月楼和永寿宫。

御花园中的枫树叶缘已全然泛红,万叶干声,迎风作响。她岑寂独立,一袭寻常深浅二紫色缎袍,舒袖临风,卷起衣袂翩翩,湛然如谪仙。看得久了,那紫便融进了漫天的血红之中,浑然不见踪影。她就会想起那一夜的恂嫔,她胸前的血,阿诺达的血,似乎添了御苑枫色的一笔浓墨重彩。

这般想着,回首才见有人来,竟是香见。

她穿一身月白衣裙,披风也是浅浅的莲紫色,滚了一圈薄薄雪狐风毛。她的头发松松拿鎏金扁方绾成横髻,珠钿疏疏却精致,缀着新鲜胭脂花,簪着一枚绞串珍珠银流苏长簪。恰如宫人所言,哪怕皇帝不如从前那般痴狂,待她到底是宠爱无俦的。虽然她无心装扮,可素日所用无一不贵,哪怕随手用上一二,都是倾城之物。只那一支长簪,那流苏勾勒精心,丝丝如女子青丝纤细,绕成花鸟纹样,再纤纤坠下,非工匠耗目半岁不可得。明珠颗颗比拇指还大,泛着柔和的粉红色,乃是采珠女潜入深海所得,便是奉上万金也难求得。连身上衣衫裁成,必是织造府倾心制成,最先供她挑选。

香见却不甚在意,她解下风帽,露出秋水空蒙的双眼。蛾眉照例是淡淡扫,朱唇也只是随意点就,是慵懒梳妆的模样。御苑中有四季不凋的常青树,亦有满天冉烈的红叶,她静静地立于其下,清艳不可移目。

香见不复从前倨傲,也学会了宫中礼仪,只是显得生疏,``皇后万安。''

容珮惊诧得合不拢嘴,但见如懿目光扫来,立刻低眉敛容。

如懿颔首为礼,道:``你难得出来。''

香见轻嗤,``就算要被困死在这里一辈子,也得看看自己的牢笼是什么样子。皇后娘娘不也是这样么?''她抚着手臂,``你应该见过天上的鸟儿吧?被剪断了翅膀,哪里还能飞呢。到头来,我的勇气还不如恂嫔。''

如懿道:``你也知道了?说来恂嫔的父亲惨死,族人凋零,无所牵挂才冒险犯大不韪。你终究不同,牵绊太多。''

``平日里看恂嫔闷声不响,倒做出这样惊天动地的事来。''香见满是钦慕,``不承想是她,做了我最想做的事。''

如懿看她一身宫装打扮,花盆底的鞋履款款走来也无不妥,便道:``你仿佛适应了许多。''

初寒的风掠过,如秋水般泠泠爽爽,身上的衣裙被风鼓起,窸窸窣窣如悄声细语,是静夜里涌动的细浪。

``适应容嫔这个身份么?''她一笑,嫣然无双,``据说按着皇上如今的宠爱,我迟早会登临妃位,或者贵妃位,是么?''她笑色骤冷,``我不怕告诉你,穿着这身衣裳,行着这些礼仪,我心里想着的,只有我愿意想的人。''

红叶的光泽浸染上如懿所穿的浅紫云纹大襟外衫,交织的艳色迸出华丽的质感,并且装点出一种温暖的假象。

如懿看着她,``这样的话,你肯对本宫说?''

``有何不可?''她目光清澈,``因为这个地方,只有你真心劝我活下来,顾着我身后的族人。算来,你当年也是为了皇上才这般劝我,可到头来,这宫里唯一的一点真心,竟也是你给我的。''

日色正好,映得屋角脊兽流光错彩,风里泛起了阵阵素菊香,红叶纷纷璀璨着含朱流金的光芒,又是太平年景里的晴好时光。谁理会,她们各自心事凋落。

驻足间,却见李玉陪着永璂自慈宁宫一带过来,永瑾见了如懿,面露喜色,连忙唤道:``额娘!''

如懿一把抱住他,喜得泪盈于睫,``永璂,你胖了些。''

永璂点头,很是高兴,``愉娘娘对我很好,额娘放心。''

如懿心头暖洋,``有你愉娘娘在,额娘当然放心。''

李玉上前道:
``皇后娘娘,十二阿哥刚去向太后请安。太后听闻十二阿哥在木兰围场身受惊吓,也很是挂怀呢。''

年华滔滔而去,太后也成了垂垂老矣的白发妇人,守着膝下温婉孝顺的女儿平和度日,也越来越有一副老人家才有的软心肠,疼爱稚子晚辈,更怜永瑾不得在如懿身边教养,所以格外照拂。

容嫔向来不喜人多,转身去了。如懿见只有李玉带着乳母嬷嬷陪侍,并有两名御前侍卫,不见素日常陪着的凌云彻,便道:``仿佛许久不见凌大人了。''

李玉面色一沉,复又笑道:
``自从木兰秋猕凌大人救护有功,皇上便格外器重,总留在御前。''

永璂朗朗道:
``儿子也久不见凌侍卫了。皇阿玛说不必他再照顾我往来。''他想一想,迟疑着道,``其实儿子觉得凌侍卫性子温和,又能救儿子,实在是很好的。''

李玉嘴角微微垂落,似有苦衷,然而很快笑道:``阿哥快别这么说了。凌侍卫是侍奉皇上的,若无皇上关切,凌侍卫怎能救您?到底还是皇上恩泽庇佑,您与皇后娘娘才能安然无恙啊。''

越是机巧地掩饰,越是有什么不可言说的秘密。有狐疑的阴翳蔽上心间,如懿温然道:``永璂,额娘为你缝制了一件冬衣,你和容珮回翊坤宫试试。''永璂乖顺地答应,跟着容珮走了。

如懿定定望着李玉,沉声道:``你也不大好过吧?否则陪着永璂住慈宁宫请安这等小事怎都是你一个御前大总管来做?''

李玉恭顺垂眸,``做人有高有低,进忠年轻力健,嘴乖舌滑,又有令贵妃在身后,自然得意些。但十二阿哥是嫡子,奴才有幸侍奉,是奴才的福气。''

如懿郁郁不乐,``永璂虽是嫡子,但与永琏和永琮在时相比,大为不如,木兰围场一事,皇上几度看轻永璂,要你侍奉,也是不尴不尬。''她目光陡然锐利,``你且如此,凌云彻更是不好吧?''

``山高水低总是常有。凌大人救主有功是好事,但太过显眼,只怕皇上心里也未必乐意。''他连连摇头,``说来自从豫妃不必被禁足,每日在宫中闲荡,也是点眼。只怕皇上看凌大人,也是这个样子吧。''

心底的微凉如这个季节不期而至的清霜,她低低道:``若是见到凌大人,请叮嘱他好好保重,韬光养晦。待得冬去春来,自然可以一切无恙。这句话,本宫也说与你听。''

李玉郑重颔首,拱手辞去。

\hypertarget{ux7b2cux5341ux4e94ux7ae0-ux6d41ux8a00}{%
\chapter{第十五章 流言}\label{ux7b2cux5341ux4e94ux7ae0-ux6d41ux8a00}}

而关于如懿和凌云彻的流言,是在乾隆二十六年的初冬开始甚嚣尘上。人人都在传言,中宫皇后是如何和一个比她小一岁的侍卫眉目传情,私相授受了二十年。如懿一开始只装作不闻不问,也不愿理会这些无稽之谈。可是流言的传播,永远比最厉害的瘟疫传播得更快。很快,她就发觉,无论自己走到哪里,恭敬温顺的脸孔一背转过去,就是窥探、好奇、讥讽与笑话。

乌拉那拉氏高傲的血液流淌在四肢百骸里。如懿情愿被人狠狠地扇耳刮子,也受不了背后的阴毒流言。但很快,另一种新的流言便覆盖了这种旧闻。新的流言便是。令贵妃魏嬿婉与御前侍卫凌云彻曾是私订终身的青梅竹马的恋人。这个传闻似乎比如懿的传闻更容易让人相信,毕竟,相对年轻貌美的宠妃比高高在上不苟言笑的皇后更适合香艳而扑朔迷离的故事。而这个故事,似乎证人更多,曾经冷宫的侍卫、四执库的嬷嬷,似乎都能说上一点有鼻子有眼的段子。

这一点让嬿婉很是气结,却又无可奈何。连她自己都不曾想到,那段尘封在紫禁城犄角旮旯里的未曾绽放完全的感情,会突然有眉有眼地跳到跟前来。

而当如懿在看到海兰教诲着四执库的嬷嬷怎样把关于嬿婉和凌云彻的故事讲得绘声绘色而又不把自己牵扯入内的时候,她终于难以抑制心头的怒火,传了海兰入了翊坤宫道:``你是疯了么?这样做,虽然撇清了我,但是对凌云彻而言,还不是一样要下地狱!''

海兰的目光意味深长地在如懿身上探询,``凌云彻成为磨心又怎样?他要下地狱又怎样。只要那个人不是姐姐,我就敢去做:何况魏嬿婉要害姐姐,我又怎么会容许她得逞?以其人之道还施其人之身,是最好的办法!''

如懿心痛,``那会害死凌云彻的!''

海兰快意地笑着,``那又怎样?如果一个凌云彻能赔进一个令贵妃,我觉得划算极了。''她的目光中浮起深深的忧虑,
``可是姐姐,怎么你舍不得一个凌云彻么?,,

如懿断然以拒,``凌云彻多次救助于我,他不该成为我和魏嬿婉之间彼此争斗的牺牲品。''她逼视着海兰,``海兰,你以前并不这样。''

``姐姐以前也不这样,我们都曾经温良恭俭让,柔弱无依等待保护,后来才发觉一切成空。''海兰满不在乎,``姐姐,每个人在这里都会发疯。我们若不跟着一起疯,迟早也逃不掉!''海兰忧心道,``姐姐,我说句僭越的话,不要有自己在乎的人。不要!否则您在乎的人一定会成为您的软肋的。''

如懿不言,只是紧紧抿住了双唇。

寒衣一重重添上,暖炉也一个个生起。来不及叹``天凉好个秋'',便到了``晚来天欲雪''的时节。有时候闲来无事,听着窗外风涌叶落声,恍然间觉得自己是坐在江心一叶孤舟上,眼见江水东流,飘摇不定。

如懿与皇帝倒也常见到,只是典仪时分不必说话,他与她只需保持着庄重肃穆的模样,如供在殿上的神尊,宝相庄严,供人瞩目便可。私下间独自相见的机会略同于无,因为即便是言说内宫事宜,嬿婉也多是在的。于是,说的话也越发冠冕堂皇。所以,有时候连她自己也恍惚,在当年的当年,在遥不可及的日子里,那些动人的情话是怎样从同一张嘴里甜润地说出的呢?

这般想着,这一日皇帝的召见,便有些意料之外。

因着新雪初降,殿中已经通了地龙,一室暖洋如春。阁中铺了新色猩猩毡,花梨罗汉床上设着明黄彩绣云龙吐珠并八寿联春的靠背引枕,一应的黄缎金龙缂丝垫上展着赤红火狐皮坐褥,陈设中华贵而不失新意。

如懿低首垂眉,以恭敬婉顺的姿态保持着刻意的距离,清凌凌道:``皇上久不见臣妾,今日一召,不知所为何事?''

她的态度不卑不亢,虽是含了婉仪之态,却如皮肤下触手可摸的瘦嶙嶙的骨骼,有坚硬的棱角。

皇帝郁然一叹,``皇后是怪朕么?''

如懿笑意清幽,
``不是怪,而是臣妾久不见皇上,独自一人惯了。今日乍见,怕礼仪久疏,叫皇上怪罪。''

皇帝神色和缓,牵过她的手坐下。温言道:``皇后这话,便是怨怼了。''

皇帝还是如常的温柔笑靥,声音却干脆得没有一缕尾音,``窗外微雪夹着雨声入耳动人,皇后可否为朕抚琴一曲,以衬这初冬雨雪。''

其实琴艺并非为如懿最擅长的,若论抚琴,除了昔日的高唏月,如今宫中最擅长的,却是忻妃。且皇帝一向对女子的才艺颇为挑剔,若非最能合他心意的,情愿不听不品。她旋即漾起谦逊的笑,``皇上知道的,臣妾一向不擅抚琴,算不得个中翘楚,忻妃抚琴堪称国手,还是请忻妃过来为皇上清音悦耳吧。''

皇帝扬一扬手,``并非国手才能琴声动人,偶尔听一听皇后的琴音,或许也别有情韵。''

如懿浅浅垂眸,终究觉得不必过于拒绝,只得道:``皇上想听什么,臣妾弹奏一曲便是。''

皇帝幽然远望天际,``天寒雨冻,便弹一曲寒雨之词吧。却也不要让人觉得冬日深长无望,有新春之意才好。''

如懿淡淡道:``恭敬不如从命,只是皇上别怪臣妾才疏学浅才好。''

皇帝的笑容薄薄的,像穿不透雾气的阳光,``抚琴之妙在于得之心而应之手,心中所思,便是手中之韵。皇后随心便可。''

如懿随手拨动七弦琴,泠泠有声。那幽幽之声如寒冰下缓缓流动的溪水,与碎冰相触,清泠颤颤,这样的曲调,最适合弹奏清婉练达的词曲。她抚弦起声,清朗吟诵:``怅卧新春白袷衣,白门寥落意多违。红楼隔雨相望冷,珠箔飘灯独自归。远路应悲春畹晚,残宵犹得梦依稀。玉珰缄札何由达,万里云罗一雁飞。''

皇帝斜倚在暖阁的软榻上,银盆中的红箩炭蕴着融融的暖意,和着炭盆中新折松枝的气味,让人酥沉中又有甘洌清新之意。皇帝穿得轻暖,一袭狐裘搭在膝上,脸上有醺暖的珊瑚色,慵懒道``这首李商隐`2'的《春雨》倒很是切合意境。果然冬日才至,皇后便渴盼三春时节了。''

如懿盈盈道:``京中寒日长久,难免期盼春暖花开之时。''

皇帝轻轻一嗤,``春日迟迟,眼下雨雪霏霏。皇后是否触景伤情,觉得朕这些日子在令贵妃处颇多,而陪伴皇后少了些,以致皇后有`红楼隔雨相望冷,珠箔飘灯独自自归'之叹?''

如懿见皇帝半是玩笑的神色,心中稍稍有些紧张,仍是笑语盈盈,``皇上忙于国事,在后宫的时候本就不多。且皇上心性温柔,颇多眷顾,来了也不能冷落各宫,总要多走走,何况令贵妃儿女众多,皇上多去陪伴也是应当的。''

皇帝神色愈加和悦,``皇后宽仁体恤,果然是中宫风范。只是\ldots,他稍稍靠近,颇有戏谑之意,``皇后丝毫也无嫉妒之心么?''

皇帝靠得那样近,呼吸间温热的气息潮湿地拂在她的耳后。可是分明,那样的气息里和着脂粉旖旎的清甜,仿佛是芬芳的花朵,凝在他的口唇鼻息之间。如懿下意识地微微侧首,避过那香甜的侵袭,指上琴音袅袅,端然道:``嫉妒乃嫔妃大罪,臣妾虽然居于后位,也不敢有此心念。这是皇上教导的,臣妾铭记于心。''

皇帝微凉的指尖拂过她耳垂上碧玉桐叶垂珠坠,那碧玉有沁凉的触感,摇曳着轻轻触上脖间裸露的肌肤。她在心底默然叹息,叹息自己此刻不易轻信的心。皇帝的笑声有湿润的亲昵,``如懿,若是还在从前潜邸里,你可一定不会说这样冠冕堂皇的话!''

``今时不同往日,皇上给了臣妾什么,臣妾就得遵循什么。''

皇帝停了停,有些感叹,``唯一不变的,你还是那样喜欢李商隐的诗。''

如懿淡然低首,和着琴弦的余韵道:``李商隐词曲裱丽,缠绵悱恻,臣妾小女子之心,难免偏爱。不似皇上所爱,多有金戈铁马,气吞万里如虎之势。''

``李商隐诗虽好,但早年爱慕侍奉大唐公主的宫人,多有绯丽语句,难免损了品格。''他停一停,漫不经心道,``皇后以为,若在如今,若有这般爱慕宫中女子之人,该如何处置?''

如懿侧首沉吟片刻,温然笑道:``若真是一双有情人,男未娶女未嫁,姻缘合当,也可成全一段佳话。''

皇帝轻哼一声,面上忽然凄寒迫人,``皇后也知道男未娶女未嫁,才能姻缘合当。可是在朕看来,私心觊觎宫中之人,哪怕只是地位卑下的宫女,也罪该万死!''皇帝冷声道:``李玉,传旨下去,御前侍卫凌云彻无礼犯上,即刻杖毙!''

李玉见皇帝陡然色变,尚不知出了何事,只得忙忙答应了,脚下却故意缓了两步。

如懿脸色一变,勉强笑道:``凌侍卫一向得皇上器重,又蒙皇上赐婚,今日不知犯了什么错事,惹得皇上龙颜大怒?''

皇帝唇角有冰冷的弧度,``皇后不明白?''

如懿隐隐觉得不好,只得强笑道:``臣妾愚昧。''

皇帝的声线陡然严厉,``皇后不知,那还有谁更清楚个中滋味?皇后连念诗都不忘有`万里云罗一雁飞'之句,岂不是也在记挂凌云彻这个名中有`云'字的大逆之徒?''

有些微的怔忡,仿佛是不敢相信自己的耳朵。那些话明明已经余音散去,却砸在了耳边,嗡嗡地用力刮着耳膜。有冷风灌入口中,掀起舌底的惊讶难耐,如懿在突如其来的惊惧中难忍诧异之色,道:``大逆之徒?凌云彻救臣妾母子有功,怎成大逆?且臣妾相伴皇上日久,皇上怎会有此疑心?''

皇帝低首拨着拇指上浅浅寒绿色的翡翠扳指,那扳指是极难得的龙石种,唯岩洞中所生,有冬暖夏凉之效。那色泽更如丝绸般光滑细腻,温润之致,荧光四射,望之便生寒意,更映得皇帝神色淡淡的。他道:``日久能见人心,亦能生情,不是么?''

她默然片刻,忽而明白了什么,嘴角泛出一丝幽寂笑容,``原来皇上这般疑心臣妾。那么今日邀约臣妾前来奏琴,无论臣妾弹奏什么,皇上都准备了这番话说与臣妾听,是么?''

皇帝倨傲地看着她,眸色有一丝伤怀,更灼灼燃烧起暗红的愤怒,``琴为心声,皇后念念难忘,连词曲亦不肯稍稍忘怀。''

如懿胸中翳闷难平,失声笑道:``那么如皇上所言,哪怕臣妾某日悠然望云,也是情之所至,不能克己。所以从此之后,臣妾若要显得心怀坦荡,便不可抬首了?''

皇帝的眉心重重皱起,
``你遇事一向不屑辩驳,如今一说他,你便怒不可遏,可见心虚。''

``臣妾心虚?''如懿挑眉凝视,毫不避让,迎着他的怒气冲天道,``到底是皇上心虚,还是臣妾心虚?一切情由,不过是因为恂嫔与阿诺达行刺之时是凌云彻舍身救臣妾母子,而皇上一心泄愤,重伤阿诺达,不惜以永璂安危为赌注。所以事后回想,为给自己几分台阶,却先扯了臣妾的不贞,来掩饰皇上不恤!''

皇帝闻言,额头青筋暴跳而起,反手一记耳光重重打下,``你放肆!''

有良久的寂静,仿佛所有尚有东西都死透了,静静的没有半点声响。连那一声耳光的余音都成了幻觉。他立在离她一步的距离,右手疲软地垂下。而她,竟忘却了面孔上热辣辣的痛灼。有猩红的血滴热热的,黏稠的,从唇角滴落,像是皑皑白雪里绽开的红梅。她顾不得去擦,只是由着那血红缓缓落下,洇入春荣秋茂图的沉香红锦毯。毯沿两列打着万字不到头的金沙线,中间缀着浑圆的米珠,毯绒细软密实,便是落足亦无声。何况那小小血珠,不过是浸淫其中捧出更娇艳的一抹红灿。

她伸手蘸了蘸那抹血红的热,苍白的面上支起摇摇欲坠的笑容,郑重行了大礼,``皇上恩赏责罚,都是雨露之恩。臣妾斗胆,请皇上给个明白。皇上今日这一掌,到底是臣妾真有不赦之罪,还是只为皇上一时疑心?''

冷然相对而立。檐下吹来阵阵寒风,闪着零星的惨白雪子,疏疏散入殿内,把他赤色蟠龙夹银线坠玉珠雪狐长袍打得瑟瑟作响。雪光惨然,把阁中二人扫落的身影扯得悠悠长长,交叠在一起。数十年无所不谈,身形交融,到如今竟是相顾无言,唯有冷漠与隔阂。恰如地上的影,似是亲密不可分隔,却已经是愈行愈远,心已荒芜。正无言处,忽听得外头喧闹声大作,似是李玉阻挡不住,豫妃急切的声音直传入内,``皇上,臣妾有要事相见,皇上!''

皇帝久久不见她,无心理会。正要出言打发,只见两扇朱漆填金殿门轰然而开。豫妃直冲了进来。

想是太过心急,豫妃云鬓微微蓬松,几缕鬓发黏在面颊上,越发显得脂粉光腻。她狠狠叩了个头道:``皇上,臣妾叩见皇上!''

她语中所言,浑然无视一旁的如懿。只是在偶然目见她唇边血痕时,微含了一丝诧异与幸灾乐祸。

皇帝连看亦懒得看她,不耐烦道: ``养心殿你也敢擅闯么?当真是糊涂透了!''

豫妃带了哭腔,狠狠磕了个头道:``臣妾已久不能得见天颜,今日擅闯养心殿,自知是寻死,也实在是有一事关系宫闱清平,所以臣妾不得不冒死一见。''

话音未落,只听得嬿婉一声娇啼,在后头急急赶进,一把拉扯了豫妃手臂,喝道:``你在本宫那儿疯还不够,还寻来这里,真是疯魔了么?''她见帝后皆在,虽然急赤白脸,却也忙中不乱,行礼如仪,``皇上万福金安,皇后娘娘福寿康泰。''

豫妃讥笑一声,``宫里出了这般丑事,你还只顾着行礼跪拜,还不许我告诉,真要手臂断了往袖子里折么?我虽出身蒙古,但礼义廉耻、忠贞孝义还是知道的!''豫妃用力挥开嫌婉的手,斥道,``你拉扯我做什么?身为贵妃,协理六宫,却胆小如鼠,无德无能!''

如懿虽然与皇帝冷眼相对,闻得此言也不禁皱眉道:``什么丑事?皇家清誉,容得你这般放肆胡言么?''

皇帝转过头来,喝道:``你前次僭越,藐视君上,朕看在博尔济吉特氏世代功勋的分儿上宽宥了你。你要再敢任意妄为,欺辱贵妃,朕便废了你的位分送你回蒙古去!''

嬿婉见皇帝着恼,忙跪下哀哀道:``皇上恕罪!豫妃也是心急火燎才口不择言,可豫妃所说,真当是胡言乱语失心疯了!您可千万别信她。''说罢,她悄悄看了如懿一眼,只是苦笑。

豫妃登时大怒,两眼竖起盯着嬿婉,如要吞人一般,``什么失心疯?若不是铁证如山'我怎敢舍出这条性命来说!''她转过脸,膝行到皇帝跟前,紧紧扯着他的袍角,厉声喊道:``皇上,皇后娘娘与人有私,臣妾不敢隐瞒啊!,,

她的哭腔才拖了一半,只听``啪啪''两声脆响,脸颊已经高高肿起。原是嬿婉冲到她身前,狠狠给了两掌,怒道:``你在本宫面前肆意便也罢了,可皇上皇后在上,你也敢把你那些见不得人的蠢话抬到面上来!''她说罢便含泪,``皇上,臣妾枉然协理六富,实则御下无方,全不能为皇上皇后分忧!''

如懿乍然闻得豫妃说出这番话来,不觉望着皇帝惨然而笑,``难怪皇上今日这般质问臣妾,原来风言风语,自豫妃便有了!''她说着看向面色惨白的嬿婉,衔了一缕讽意,``看这样子,豫妃必然是先去了你那儿闹腾。自然了,你身娇体弱,哪里拦得住,只好由着她闹到皇上跟前来了。''

嬿婉面色涨得通红,嘤嘤道:``臣妾人微言轻,素来被宫中姐妹小觑,空担了协理六宫之名,实则难以服众。且豫妃所言,兹事体大,臣妾也不敢由着她胡来!''

豫妃恼恨地看着如懿道:``你纵然贵为皇后,然而德行有亏,也有脸申斥旁人么?''

如懿怒极反笑,目光逡巡在皇帝与豫妃面上,冷然笑道:``今日你却不是第一个面斥本富德行有亏的了。本宫倒想听听,除了侍卫,你们还能想出谁来?太医?亲王?再不成连太监也算上。是个男人都往本宫身上扯便罢!''

豫妃冷着脸,毫不畏惧,目光灼灼直视如懿,``倒也攀扯不上旁人!行不正自然为人诟病,便是凌云彻一个了!''

如懿气急攻心,哑然失笑,拊掌道:``好!好!难怪豫妃曾得皇上数月欢心,果然还是会揣摩上意。难道在你们眼中,救命之恩便是阴私之情么?狭隘至此,真是闻所未闻!''

她的话虽指着豫妃,皇帝又如何不知她深意,一张面孔愈见冷峻。

嬿婉乍闻此名,陡然乱了气息,一时且惊且疑。片刻,她忽而生了微凉如雨的笑意,朗声道:``若说是旁人,本宫还能信一二分。只是凌云彻,哪怕铁证如山,本宫也不相信!''

豫妃冷眼睨着嬿婉,气哼哼道:``你倒知他?别以为他是皇上身边近侍,便如此奉承偏帮!我便瞧不上你们这些滑头!''

嬿婉扶着皇帝手臂,切切道:``皇上,臣妾出身寒微,与凌云彻原是同乡,自幼相识。若说一句青梅竹马,臣妾也不敢驳回。''

皇帝目色陡然凌厉,似笑非笑道:``好!好!原来朕的皇后和贵妃,都与朕的近臣相熟,朕倒浑然不知,做了个糊涂人!''

这话颇为森厉,嬿婉粉面涨得血红,顺着皇帝手臂上丝滑锦袍倏地跪下,仰面含泪泣道:``皇上明鉴!臣妾今日敢言,便是问心无愧。凌云彻比臣妾早几年入宫,臣妾为宫女时,因着同乡颇多照应。此事若是旧年间的侍卫宫女,怕还有几个知道的。臣妾也不怕皇上细查。只因偶然照拂几次,反惹了闲言闲语。臣妾为着彼此名誉,便疏远了。直到凌云彻救驾有功,侍奉皇上身边,大约是怨怪臣妾早年疏远,他也不大理会臣妾。可怜同乡之谊,便成了陌路了。''

这略略一席话,有多少前尘往事夹杂在风烟间扑面而来,迷得如懿隐隐生痛。她听嬿婉哀婉道来,中间无数曲折缘故略去不提,倒成了一个无辜之人,心底不免暗暗冷笑。

果然皇帝静了片刻,伸手扶她起来,语气己然缓和了不少,``你敢不畏人言告诉朕昔日之事,可见心底坦荡。何况谁无幼年一同长大之人,便是青梅竹马之谊,如今疏远了便也罢了。起来吧。''皇帝略一沉吟,扶住她侧身坐了,温声道,``你曾夜雨长跪殿外,伤了膝盖。不要动辄跪着,仔细身子要紧。''

这般话,显然是说与如懿听了。如懿只觉得字字都是尖锐的银针,针针戳心,绵绵密密无止无尽,心中翳闷压得透不过气来。索性她也不理皇帝是否在意,扶着朱漆泥金雕心炕桌坐下。天气尚寒,花梨罗汉床上铺着厚厚的赤红火狐皮坐褥,人在其上,总有落入云端的绵与厚。可此时此刻,荆棘丛中步步艰辛,她才体会何为如坐针毡。

可是,她不会怕。因为她是如懿,自幼浸淫深宫的如懿。多少惊涛骇浪,她都看过,都颠沛过,才一路艰难行来。

如懿倏然含笑,颜色却冷,``令贵妃倒是先行把自己撇得干净!''

豫妃默默听了半日,早已不耐之甚,``皇上!臣妾不理令贵妃与凌云彻如何,左右也是微末小事。可臣妾今番胆敢告诉,的确是有人证物证的!''她狠狠咬着唇,闪耀着满脸得色,``那人证便是凌云彻的枕边人,宫女乌雅茂倩!''

皇帝目中一瞬,口气却疏懒了些许,``是么?茂倩是朕赐婚于凌云彻的。她,偶尔进宫向朕请安,虽然言语间也有些责怪夫君忙碌不顾家中之意,但如你所说,却是从来没有。''

豫妃立时急道:``皇上,那日木兰围场恂嫔谋刺,凌云彻不顾皇上先救皇后,臣妾已生疑惑。但念及茂倩乃凌云彻妻室,便派人将他奋不顾身之事告知茂倩,也安慰茂倩一切平安。谁知茂倩听闻之后不曾为凌云彻救皇后而喜,反而大哭大闹,语出怨怼。臣妾听闻后更加疑惑,回京后立刻召茂倩入富细问原委,才知他夫妻二人不睦吕久,只为凌云彻心有所属。''

皇帝越听眉头越紧,问道:``茂倩何在?''

豫妃扬眉含笑,急急道:``皇上莫急,臣妾为求万全,已带了茂情入富,在外候着了!''

皇帝默然片刻,那沉吟分明有山雨欲来之势,迫得殿内诸人大气亦不敢喘一声。还是嬿婉穸着胆子婉言劝道:``皇上,茂倩固然是御前宫女,但凌云彻也屡屡救驾有功。着要对质,不可光听茂倩一面之词。''

皇帝瞟了立在一旁的李玉一眼,漠然道:``凌云彻何在?''

李玉正听得抓心挠肺,愁肠百结,忽听得这一句,忙不迭道:``皇上,凌云彻今日当值,只还未到时辰,尚在庑房歇息!''

皇帝扬一扬脸,唤道:``庑房近在咫尺,叫进忠去!你先唤茂倩进来。''

李玉心知皇帝如此,是知他与凌云彻私下交好,防他泄露,心底越发不安,只得先至殿门前唤了茂倩进来。

\hypertarget{ux7b2cux5341ux516dux7ae0-ux8302ux5029}{%
\chapter{第十六章 茂倩}\label{ux7b2cux5341ux516dux7ae0-ux8302ux5029}}

茂倩因是旧日皇帝御前的宫女,又是满洲女儿,打扮得格外体面。只见她一身荣蓝色新缎描银掐花缂丝出灰鼠毛褙子,蜜荷色缠枝团花马面裙,头梳一个端端正正的小两把头,簪着红绒绒花朵,绾了一枚玳瑁镶珠石扁方,也不用流苏簪饰,倒显得落落大方。她显然刻意打扮过,一身颜色衣裳显得温和可亲,唯有一双吊梢眉,才有几分凌厉之气。

她虽出宫多年,但对御前规矩极为熟稔,行云流水般行叩了大安,也不起身,楚楚道:``奴婢蒙皇上赐婚,不能日日侍奉跟前,今日未曾奉诏便擅自入宫、无论皇上等下如何责罚,都请受了奴婢一片孝心。''说罢,又重重磕了三个头。

皇上打量着她的气色,虽然妆容精心描穆,细看之下仍可见她眼角眉梢的憔悴之色,当下便有些不豫,``怎么?朕赐婚与你和凌云彻,你们夫妻却过的这般不好此,豫妃何必巴巴儿找着你来呢?想吐出来的话别噎着,自个儿给自个儿添堵。''

皇帝横她一眼,``你倒是半点颜面也不想留?''

如懿缓缓抚着手中的销金菱花手炉,金器装了小块的红箩炭本就烫手,所以得护着里外发烧的银鼠皮手笼。可是那烫却成了现下唯一的取暖之物。眼前的这些人,这些话,无一不是冷的,是冻住了的污水,一口口逼着人吞下去,冷得叫人恶心。

她淡淡瞟皇帝一眼。似笑非笑道:``皇上没有给臣妾留半分颜面,旁人自然爱更不会留了。臣妾便是自己想留着,也是枉然。''

茂倩倒也不惧,对着如懿恭恭敬敬行了一礼,徐徐道:``奴婢伺候皇上多年,由人至心是皇上无不知的。今日对着主子,也不敢有所欺瞒。凌云彻对外是一个极好的夫君,无人不赞。可到了屋里,虽然起初也对奴婢装模作样嘘寒问暖,可他对奴婢从不放在心上。''她面上微红,垂首道,
``不瞒皇上,奴婢与凌云彻成婚多年,做夫妻的日子不过十来日。他连奴婢手心是否有疤痕亦不知。''

皇帝微微颔首,``你右手手心有一疤痕,是刚进宫伺候朕时不防被火烛烧伤的。''

茂倩满眼泪光,连连俯拜道:``皇上怜悯,奴婢铭记于心。''

媾婉微吸一口冷气,极力缓和着道:``你也糊涂,凌云彻侍奉皇上身边,是多少要紧的大事得记着,微末小事忘了也是有的。他为着忠君而少陪你些,你也该多体谅。''

茂倩忍着羞涩,面色涨红道:``起初奴婢也极力开解自己,可渐渐久了,才看出些端倪。''她说到此节,又恨又恼,``他倒不是忠君\ldots{}''''她骤然盯住如懿,眼中进出一丝冷光,``他所有心耳意神,倒是全记挂在了皇后娘娘身上。''

如懿迎着她的目光,慵倦地掸了掸手中的杏色水绫绢子,``好了,终于说到这句了,也不枉豫妃一番辛苦找了你来。只是这话便和戏文似的,唱了开头就让人猜得到下头,真真也是无趣至极。''

茂倩面容阴冷,恻恻道:``皇后娘娘倒真是成竹在胸。奴婢也不怕做个小人,到底与他夫妻多年,或是醉酒,或是梦呓,他心心念念的唯有皇后娘娘一人哪!''

她话未说完,只见凌云彻大步跨进,躬身一礼,朝着茂倩气得目呲尽裂,``我只知隔墙有耳须得防贼,却不想你我共枕多年连梦呓也字字当真。''

茂倩与凌云彻一照面,气不打一处来,再不复方才极为克制的仪态,冷笑一声道:``俗话说酒后吐真言,梦中话心声。若不是同枕共眠,怎知你心底龌龊隐事,竟这般日思夜想,梦里也不能忘''她红了双眼啐道,``你也敢道我是贼,采花淫贼才恬不知耻!''

凌云彻勃然大怒,``这是御前,你当是家里,任你疯癫胡言?''

茂倩泪光一闪,死命咬了牙,伸出长长的指甲戳着他面颊道:``你还记得家里?不知多早晚才回来一趟,早忘光了吧?''

凌云彻气得脸色铁青,碍着在御前,索性别过头不理她。

茂倩见此,越发生了天大的委屈,抱屈道:``那日豫妃小主遣人来报你平安,说道你奋不顾身去救皇后娘娘。人人道你忠勇,唯有我知道你那见不得人的心事。救驾一事,不过是你与皇后有私,才奸情流露而已。什么忠勇,呸!''

凌云彻本自隐忍不言,听她说得不堪,终究忍不住道:``什么村话浑语,也敢污蔑皇后娘娘清誉!''

茂倩凑到他跟前,团团追着他,一双眼却斜斜飞着横向如懿,愈显得凶悍泼辣,道:``清誉?我倒要瞧瞧是什么清誉,勾得别人的男人神魂颠倒!连在梦中也口里心里放不下,一味唤着皇后娘娘闺名。''茂倩本就眉梢吊起,一恼恨起来那眉毛更是根根竖起,凌厉狰狞,恶狠狠道,``如懿,如懿,倒真是个吉祥如意一昕难忘的好名字!''

凌云彻怒极,也顾不得在御前,反手便是一掌,方肃然叩首道:``皇上,微臣不懂管束妻房,乃敢在御前无礼,惊了圣驾,微臣自甘领罪!''

皇帝冷哼一声,嬿婉厉声责道:``打得好!是该好好管束!在御前这般忘了规矩,胡乱争执,打死也不为过。''

茂倩又气又恼,拼命砰砰磕头如山响,流着泪道:``皇上,奴婢今日一来,自知死罪,不过是拼个鱼死网破,好叫自己活个明白罢了。''她目中几欲喷火,捂着半边高高肿起的脸向着如懿笑道:``今儿是什么好日子,皇后娘娘领了皇上的责打,奴婢也领了自己夫君的责打!真真都是妻室失德的日子了!''

嬿婉愈看愈是皱眉,喝止道:``什么妻室失德,皇后娘娘何等尊贵!只凭你妄议主子,就该立时杖毙。''

豫妃护住茂倩在身后,委屈不已,``贵妃娘娘协理六富,见不得这些腌臜事儿。但火烧眉毛,也别只顾着胳膊断了往袖子里藏,一味掩饰。多少脏的臭的,都污到中宫了!若是贵妃自认汉军旗出身,管不得咱们后宫满蒙的事儿,我也怨不得什么。''

嬿婉协理六宫,最恨旁人拿汉军旗出身说嘴,登时气得花容失色,连连抚胸喘息,一手指着她一味落泪,直说不出话来。

皇帝的目光是悬崖上的冰,高处不胜寒,他缓缓扫了豫妃一眼,``你倒是嘴上半分不肯积德,连着把令贵妃也指桑骂槐进去。便是你真告了皇后之错,朕也治污蔑贵妃之罪。''

如懿听他口口声声只顾着嬿婉,一腔心血都化作了丝丝酸气,蚀着心房,不觉道:``皇上当真是好夫君。''

皇帝并不接话,只瞧着茂倩满腹辛酸地说下去,``我身为满人,嫁与你汉军旗已然委屈。我恪守妻房本分,见你冷淡,我便心知有异。却不想你这般大胆,出入宫闱这般不检点!''

凌云彻抱拳膝行至皇帝跟前,凛然正色道:``皇上,梦呓之事,茂倩一入口说而已,根本无法对质,如何当真?''

``不当真?''茂倩含了无限讽色,从怀中贴身处取出一枚小小荷包摸出一张纸笺展开,念道:``二十年四月二十,一次。二十年十二月二十二日,又一。二十五年九月十三,再一。一次还算偶然,五年间梦呓三次,我却不信了,到底是为了什么?你且别急。你在家中与我同床,虽不理我,要听你这些话也不难。你也无须怪我用尽心机,你对我这般冷落,我夜夜难眠,也是情理之中。为人妻子,被分宠不算什么,但夫君心中半分也无自己,你要我不怨不恨也难。''

凌云彻骇然变色,静了片刻,方决然摇头,向着皇帝正色道:``皇上,微臣夫妇虽是指婚,之前未曾相熟。微臣孤苦一身,得皇上垂爱才成家立室,所以一直怀有敬爱妻子之心。成婚后微臣让茂倩主理家事,一应所求无有不允,也无半分不尊重。''但神色略显戚然,``茂倩久在御前,规矩自然周到,但难免有拿大之意。且她总瞧不起微臣乃是汉军旗人,言语间对微臣先人也有轻鄙,微臣才对她生了疏远,以致她心怀怨怼,所以惹出这般泼天是非。微臣管束无方,自甘领罪。''

嬿婉低声啜泣,叹道:``皇上,凌大人所言也有道理。且看豫妃比臣妾低了一阶,也能出口便讥刺臣妾出身,一家子屋檐下的夫妇,难免牙齿碰了舌头,生了龃龉。''

如懿见嬿婉替凌云彻辩白,不觉暗暗诧异,却也不露声色,只冷冷瞧着她不作声。

皇帝缓缓坐下,足上的金线暗纹五福捧寿靴在红毡毯上一下一下用力蹭着,笑着向嬿婉道:
``你倒风起就知叶落,很会推己及人。''

嬿婉素日陪着皇帝时日不少,也知他七八分性子,听得如此说,唬得忙要起身告罪。皇帝依旧笑了笑道:``得了,朕随口一说罢了。你闹得这般坐立不安做什么?'

如此嫌婉更不敢答话了。皇帝觑着如懿,掰了指头道:``凌云彻梦呓,朕本也觉得是无稽之谈,姑且听一耳朵罢了。谁知这日子倒是颇有趣味,皇后,你说昵?''

如懿若有所思,很快镇定心神,徐徐道:``二十年四月二十,是皇上与臣妾璟兕天亡之日。二十年十二月二十二日,是永璟夭折的次日。二十五年九月十三,是皇上发觉容嫔不能生育深责臣妾之时。''

皇帝眸色如剑,锋锐几可见血, ``如此看来,凌云彻与皇后真是悲喜与共。''

如懿淡淡``哦''了一声,端然立起,福了福道:``与其说这些日子是与臣妾悲喜与共,还不如说是与皇上休戚相关。唤臣妾闺名真假尚未可知,便真是唤了,大约也是因为皇上的缘故。''

皇帝恼怒而又警觉,为如懿这一副身在其中却又袖手旁观的姿态。他正待开口.如懿扬眸,声音微冷,轻轻道:``如意。''

嬿婉微微失色,颤颤道:``皇后娘娘说什么?''

如懿心中一定,从容道:
``本宫说的是如意,如意吉祥的如意。如何?难道你是以为本富在唤自己闺名么?''她恻然望着皇帝,有破冰涌泉般的委屈,却硬生生忍了哽咽,``凌云彻若真有梦呓,臣妾私心以为他是为皇上祝祷顺心如意,而说,如意,二字。倒是茂倩心意难以揣测,为何倒认定了是说臣妾闺名呢?''

皇帝的面孔有须臾的松弛,旋即有天沉沉欲雨之色,看着茂倩道:``怎的,你倒这般有心了?''

茂倩气苦不已,拿绢子拭泪道:``皇上,奴婢实不敢冤枉攀附,此事一而再再而三,奴婢也心存疑虑,不敢确实。直到奴婢发现了一样东西。''

豫妃会意,啪啪击掌两下,只见她的贴身宫女捧了一个锦袱大盒上来,利索打开。只见里头是一双极旧的乌布靴子,大约年头久了,布料褪了一层颜色,隐隐有些发白,料子也极酥,怕是一个不小心便会碎成片片。而那穿靴人想是也格外小心,东西虽旧,却没穿过几次,针脚犹新,显然只是遭岁月安静洗褪。如懿只觉得心头突突乱跳,她怎会不认识,这双靴子,便是她出冷富前为凌云彻所制。不想恁些年过去,他却这般爱惜。

凌云彻的面孔白了又白,终于泛出一层死灰般的锈青,``这双靴子,你怎翻了出来?''

茂倩也不废话,径自道:``你素日的东西都爱如珍宝,收在自己的桐木箱子里锁着,一针一线一件破布衣衫都不许我妄动。我便奇怪,你家中本就贫寒孤苦,哪来什么值钱东西,便爱得跟眼珠子似的了!我几经小心,才趁你不防寻人配了钥匙,在箱子底下翻腾出这么个稀罕物儿。今日索性带进宫瞧瞧,也请主子们教我一个明白!''

她说罢,见嬿婉亦停了啜泣好奇打量,越发生了勇气,捧出靴子一翻,各露出一枚如意云纹图案,冷笑道:``奴婢久在宫中,也知道皇后娘娘闺名尊贵。今日既舍了脸面、性命上来,便舍着脸说一句,这如意云纹因含了娘娘闺名谐音,乃皇后娘娘素日最爱的绣样。巧不巧的,倒也暗合了奴婢愚夫的名字。''

豫妃笑一声,似墨色夜间栖在枝头的老鸹,``如意云纹?茂倩,你若不说个明白,咱们都成了蒙在鼓里的糊涂人儿了!''

有一瞬的怔忡,记忆的尘灰拂面而来,带着昏黄的色调,陈旧而温暖,如懿骤然想起在冷富的岁月,那种凄寒之苦,那种绝望之苦,如同阴冷潮湿的青苔,死死长在了骨子里。

她克制着情绪,摘下长而锐的镂银缀碎玉护甲,伸出素白的指尖,用微凉的皮肤细细感知着岁月重重轧过后的碾痕。

嬿婉的眼珠死死盯着如懿的动作,狐疑之色越来越浓,渐渐转成惶然之态,颤声道:``皇后娘娘,您\ldots{}''

豫妃抢在嬿婉身前,描得乌黑的眉高高挑起,``皇后娘娘真是心软易动情,看见个靴子都这般忍耐不得,见了活生生的人岂不是自个儿都要酥倒了。''

豫妃的话太过不堪,听得茂倩眼内出火,恨声道:``皇上,怨不得奴婢背弃夫君,原来,原来他们------''她一手撑在地上,一手指着如懿,却又不十分敢,转而指向凌云彻,气得浑身战栗如打摆子一般。

如懿的伤怀凝成凄楚的郁叹,``臣妾乍见此物,如何能不喟然伤感。当年蕊心亲手缝制这双靴子,以报答凌大人火海相救的恩德。如今岁月流逝,蕊心已然跛了一足,不复当年之态。''她静静道,``这针脚分明是蕊心的绣功,皇上若不信,只管比对。''

嬿婉失声道:
``是蕊心?''她似乎不是很信,转头只觑着皇帝面色,不敢再出声。

豫妃吃了一惊,却很快嗤笑道:
``皇后娘娘拿这种话唬什么人呢?一有事儿就拿自己的心腹出来顶包,谁不知蕊心曾是您的贴身侍婢,宁可被打废了腿也不会说您半句不是的,您就妥妥儿叫她认了吧!''

如懿根本不屑与她分辩,只定定望着皇帝,眸中秋水静寒,若一池深潭,``臣妾的绣功虽比不得海兰,但日夜相处,耳濡目染,也总有八九分功力,是而皇帝一应衣衫上凡有用如意纹的,几乎都出自臣妾之手,以示贴心相伴。皇上若不信,大可取过来看,一比就知。''

嬿婉十分为难,``皇后娘娘,这靴子是十几年的东西了。您知道绣功这个东西日益精进,总会有所变化,只怕难以断定。''

如懿轻轻一笑,``皇上穿过的衣物,便是数十年前的,都有存档。虽然费些工夫,但也好找。''

皇帝微微颔首,``若问毓瑚,一问便知。''

如懿听他语中颇有安慰缓和之意,但见凌云彻在旁,不觉含了忿郁,朗朗道:``臣妾不怕对质,只怕疑心生暗鬼,不明不白。''她说罢,转首微微侧目豫妃,顺手从鎏金花苞纽子上解下杏色水绫绢子掷于地上,沉声道:``皇上所用如意纹图样都是臣妾手绣,而臣妾所用的绢子自己顾不过来,又不耐烦内务府的绣工花哨繁,一贯都是蕊心绣的,后来便是容珮学着。如今哪怕蕊心出嫁宫中,有时惦记臣妾,在家时绣了令江与彬送进来的。其针脚纹理疏密大小不同,皇上一比可知。''便又吩咐,``茂倩,你拿起来给皇上细瞧瞧,自己也瞧清楚,也好叫本宫落个分明。''

皇帝细细看过,脸色微霁,``二者有细微之差,但的确不同。''

如懿笑色幽幽,``还请皇上取了旧日衣裳来,比个分明。''

皇帝摆手,呷了一口茶,淡笑道:``不必。朕亲眼看过,自然明白。''

如懿向着凌云彻稍稍欠身,``凌大人,你对本宫和蕊心有相救之恩,本宫和蕊心
一直铭记于心。本宫不怕直说,这双靴子,合该本宫自己也做一双谢你。不过本宫虽然喜好刺绣,但纯属雅玩,自己人瞧个玩意儿也罢了,入不得外人之目。''

凌云彻眉心一沉,旋即明白她言下之意,已将自己与皇帝亲疏分得再明不过。
他如何不会意,只得按下舌底一丝酸涩,应声道:``皇后娘娘仁厚悯下,微臣感激不尽。''

茂倩显然也是意外之极,一时呆若木鸡,不知该如何反应,却是豫妃先尖声喊了起来。她的声线本就尖细,现下声嘶力竭,更是如裂帛一般,``皇上,您信她?这种说辞留着哄自己吧!''

皇帝再无法忍耐,喝道:``谁在外头?将豫妃拉出去清静!''

李玉慌忙垂手进来,身后跟着两个身强力壮的小太监,恭恭敬敬道:``奴才请旨,如何处置?''

皇帝冷然,断声喝道:``将豫妃关入慎刑司,由着她自生自灭,非死不得出来!''豫妃瞪大了双眼,如何肯服,扯直了脖子呼道:``皇上!皇上!臣妾对您一片赤诚,不忍心您被淫妇蒙蔽呀!皇上!您为何要凉了臣妾一腔忠心啊?''

李玉哪里容得她喊,使个眼色叫小太监们架住了,忙扯了布条塞住她的嘴。豫妃拼命挣扎着,嘴里呜呜有声,凄厉无比。

皇帝轻哼一声,冷冷淡淡道:``你得多谢皇后,若无朕许诺皇后,宫中再无冷宫
之地,只怕你要去皇后曾经待过的地方了此残生了。''

豫妃犹自挣扎,呜呜哀求,一壁含了阴毒目光,恨不得一口吞了如懿。如懿轻轻摇头,不屑道:``蠢材,岂不知你去慎刑司,并非冒犯本宫,而是冒犯了皇上。你想污蔑污本宫,却不知也是侵辱皇上,无论本宫罪名坐实与否,你都损了皇上圣誉,谁能容你!''她瞥一眼皇帝,似笑非笑,``皇上肯听你说那么多,不是因为皇上喜欢听,而是圣心宽容。只是你也把皇上的大度看得太过了。难道不知你本宫真的如你所愿被废,你也落不得好儿么?究竟是谁给了你这心机自寻死路来?''

豫妃本还挣扎,听得此处,身子渐渐瘫在一边,眼神失了锐气渐渐涣散。皇帝道一声,``去吧!朕是瞧在蒙古面上,一直留了你妃位安养至今,你既去了慎刑司,不管生前如何,死后哀荣朕也会一并给你,算是给蒙古一个交代。''言毕,小太监们像拖着死狗一般将她拖出去了。

茂倩眼见事变如此,浑身栗栗发颤,匍匐于地,早没了方才的刚猛泼辣。

皇帝的靴尖有一下没一下地蹭着,闲闲道:``茂倩,朕当日将你赐婚于凌云彻,后来你数次入宫谢恩,都不曾说起他待你疏忽。今日却撕破脸面反口,倒像是朕不恩恤体下,错了你的姻缘了。''

茂倩如何禁得起皇帝这样的话,不禁泪流满面,伏地哭道:``皇上恩泽深厚,本想为奴婢寻一个好依靠。却不想汉军旗卑贱不通人事。奴婢本想嫁鸡随鸡.委曲求全,却不想还是守着顽石一般。''

皇帝尚未出言,如懿已然听不下去,嬿婉听她提及汉军旗身份,念及自己虽然位及贵妃,掌协理六宫之权,但为着这身份总不大叫人敬服越发觉得面上烧热,暗暗咬了牙不语。茂倩犹自不觉,喋喋不休,如懿沉下面孔道:``茂倩,你虽然说自己严守妻子规矩,委曲求全,但言语间大有藐视夫君之意,本宫虽是第一次耳闻,也觉得难耐。何况凌云彻与你相守多年,男儿自要颜面,怎容得你日夜诋毁,实在太伤夫妻情分。而皇上自登基以来,一直讲求满汉一家,何况凌云彻也是八旗子弟,不过分属汉军旗,与你又有何分别,你怎就生了一双势利眼,高看自己!''

嬿婉听如懿出言斥责,心下大快,亦为凌云彻多年之苦生了怜意,亦道:``本宫今日听你说话,真是牙尖嘴利。说起汉军旗,本宫是汉军旗,去了的纯惠皇贵妃和慧贤皇贵妃,哪个不是汉军旗?皇上恩待咱们,到了你却生了凌蔑之心,真真枉费你在御前伺候多年,说出去平白叫人笑话!''

凌云彻怒目圆睁,连连以拳捶地,顿首道:``蠢妇!蠢妇!这些我都可容忍,但你跟豫妃同流合污,污蔑皇后,你还要命不要?''

茂倩本已软了,听得此节,咬着牙昂起身体,落泪冷笑道:``凌云彻!我是拼着不要这条命了!我岂不知妻子悖逆丈夫是大罪,只不想一辈子做个糊涂鬼罢了。碰上豫妃是机缘巧合,若无她,我迟早也要闹个明白。''

凌云彻怆然摇头,且悲且怒,``如今你可闹明白了?为着你的明白却要闹得宫中不宁,家中不安,自己夫君颜面不顾,连皇上和皇后的清誉都险险毁在毁在你手中。茂倩,你是皇上赐婚,我如何会不敬你?奈何你事事要强争先,一味要从身份地位上压倒我,试问我如何能爱你惜你?冰冻三尺非一日之寒,事到如今,我自然也有错,罢了,罢了。''

茂倩听得泪如雨下,硬生生忍着道:``你自然以为自己待我不差,天下薄情人哪个不也这样以为?我纵然在家中掌权,但为人妻子,什么最最重要?难道只为钱财在手,夫君尊重么?岂不知尊重亦是疏远,轻怜蜜爱,真心体贴才是最难得。你嫌弃我言语轻蔑,何不努力上进挣个前程功名,又或者可以如旁人夫君一般,哄我让我,爱我容我?可你偏偏油盐不进,对我不理不睬,我如何能受你这般气?我若忍了你,也枉费自己在御前伺候那么多年了。''

如懿双耳再不忍听她聒噪,喟然叹道:``因你在御前伺候资历颇深,所以凌云彻哪怕身为御前侍卫,也赶不上你。你是满军旗,他是汉军旗,更不能与你比肩。须知夫妇之间,彼此厚待尊重,才有真心怜爱。你们这般做夫妻,也真难为了他。''

皇帝静静听她言毕,取了一枚腌渍梅子吃了,又缓缓饮一口清茶,方摇首道:``茂倩,你在朕跟前的时候,百伶百俐,要强顾颜面而事事做得极好。所以朕放心将你嫁与凌云彻,可谁知却是弄巧成拙,将佳偶做了怨偶了。''他双目微斜,在如懿面上轻轻一旋,恍若无意般叹道:``须知臣奉君,子遵父,妻从夫,不可倒置也。妻子再强,也得以夫为天,何来自己的想法由头,你可是大错特错了。''

原本如懿说话,茂倩只是梗着脖颈不肯言语,虽是默默听了,却不甚敬服。待到皇帝出言,她才有些害怕,叩首道:``皇上,奴婢不敢,可奴婢真是委屈\ldots{}''

皇帝摆摆手,``好了。今日之事朕也不耐烦,发落了一个豫妃,当是求个清静。既然你与凌云彻不睦,既是朕赐婚,少不得也是朕来做个恶人。''他横一眼凌云彻,``夫妻不睦,但由头多在你身上。你的罪过,朕一一替你记着。''

凌云彻一凛,想看一眼如懿,却少不得生生收住了目光,低首道:``是。''

皇帝的面色稍稍温和些许,``也罢,覆水难收,今日回去,你们也再做不得夫妻。便由朕做主,你写一封放妻书与茂倩,二人就此别过吧。''

茂倩大惊失色,险险哭出声来,只得用力捂住了嘴,别过脸任由泪水潸潸而落。

凌云彻深深叩首,俯仰三次,只是默然无言,静静退了出去。

皇帝看了看身侧哀哀弱弱的嬿婉,颇有几分怜惜意味,``你担着协理六宫之责,却不能为皇上皇后分忧,连一个豫妃都弹压不得。''

皇帝见她娇弱不胜之态,愈加怜惜,``你虽是贵妃,但资历终究浅些,昔日愉妃也掌过协理六宫的权责,不过如今孙子都有了,年纪渐长,难以分身罢了,你有事多问问她便好。''他微抬下颌,嬿婉明白,便道:``多谢皇上指点,那臣妾先带茂倩回宫梳洗,再着人送出宫去。''

如懿见二人喁喁细语,浑不理自己所在,便索性起身,福了一福道:``既然事了,臣妾先行告退。''

皇帝微微一笑,竟是无限怜惜之意,密密凝成唇角温厚的笑色,``方才皇后面上不小心伤了,朕叫人取些清凉祛瘀的膏药来,替你抹一点儿便也好了。''

如懿心中一凛,不知他何意,即刻道:``些微小事,臣妾自己会做,不劳皇上费心。''
皇帝轻叹道:``你也是,自己这般不当心,少不得朕替你留心便是了。''

如懿听他意中所指,似乎有话要说,便也无可无不可,斜签着坐下,取了一瓤剖好的橙子,蘸了如雪新盐,吃了一片。

\hypertarget{ux7b2cux5341ux4e03ux7ae0-ux540cux6797ux9e1f}{%
\chapter{第十七章
同林鸟}\label{ux7b2cux5341ux4e03ux7ae0-ux540cux6797ux9e1f}}

须臾,人都退尽了。殿中静得若沉在深潭之底,想着方才的喧闹,竟像是遥遥望着另一重天际般可笑。外头的雪点子有些大了,落在琉璃瓦上有细微的沙沙声。如懿抬起眼望了望那窗格间的一隙,却是铅云低垂,要落大雪了。

如懿不言,将剥下的新橙皮随手丢进象鼻三足夔沿鎏金珐琅大火盆里,又顺手拿赤铜火夹子夹了几根松枝进去。那橙皮与松枝被火气一蒸,殿中浊气也变得清爽而甘甜。只是那清爽是湃了雪的冷冽,直冲头顶,冲得她心底一阵阵发酸,像是小时候一气吃多了未腌透的梅子,那酸气从口腔里直冲顶心,复又坠落五脏六腑,连一口气也透不过来。

皇帝缓缓行至她身边,伸手将她拉起,柔声道:``地上冷,总蹲着不好。听太医说你这两年咳疾重了,自己也要好生保养。''

如懿不说话,也不看他,取过一枚小银剪子,慢慢铰着手指上水葱似的指甲。皇帝笑了笑,``对着朕这般没话说么,宁可铰指甲。''

如懿木然地扬了扬唇角,算是对着皇帝笑了,``相见无好言,臣妾无话可说。''

皇帝轻嘘一声,从李玉手里接过一个杏子大的描金合欢青玉镂花钵打开,示意他下去,自己拿无名指蘸了点浅青色的膏体,手势极轻极轻地落在她的面颊上。那药膏极是清凉,触手却绵若无物,仿佛瞬即便融进了肌理之中。她忽而笑意寂寥,``皇上的手势真好。''

皇帝自负一笑,``比之太医算是绰绰有余了吧。''

如懿笑着摇摇头,却不置可否。皇帝笑着阻止,气息暖暖拂在她面上,``别动,仔细朕涂歪了。''

他细心替她涂好膏药,仔细端详片刻,``方才朕手重了,你可不许怪朕。''

如懿的唇角勾起一抹冷冽笑容,含着遥遥不可亲近的淡漠,语气却是说不出的恭顺温婉,``雷霆雨露,均是皇恩。臣妾自甘承受。''

皇帝手指上的寒龙石扳指闪着幽绿一线,悠悠晃晃,恍若皇帝略显失望的口气,``这话便是和朕赌气了。''

如懿浅浅一笑,似含了一丝通透,``有气可以赌么?一切都由臣妾自己受着,皇上潇洒来去,才不必赌这份气。''她停一停,``皇上特意留下臣妾,大约不只是为了说这些无关痛痒之事吧?''

皇帝的手指用力一搓,微微凝神,``无关痛痒?那什么才值得你费神痛心?''他一顿,无味地摆摆手,撩开手中的镂花钵,任由它骨碌碌滚得远了,瑟缩在团锦华枕中。他的神色有种难以名状的邈远,像是有雾气氤氲,难以探知底下的情味,``有件事,豫妃的草包脑袋不太够用,便由朕来问你吧。''

那话虽说得简单寻常,却隐隐有种山雨欲来的逼仄。如懿不急不缓道:``皇上既然知道豫妃草包,也值得把她的话放在心上?还是其实即便无豫妃与茂倩之事,皇上心中疑根深种,早难以拔去。臣妾真的很想知道,到底是因为什么,皇上会自认比不过小小侍卫在臣妾心中的地位?''

皇帝好看的眉毛深深蹙起,厌倦不已,``那么,你觉得朕在你心目还有地位可言么?自朕立你为后,你事事自专。朕有所宠幸,你便蓄意阻挠。朕有所爱,你也百般为难。容嫔与你固然还算和睦,但朕一想起她不能生育的汤药便是你亲手端去,朕便忍无可忍。''

如懿听他勾起旧事,仍是耿耿不能释怀,不禁气结,``皇上知道,若是容嫔待皇上之心如皇上待她一般,她断断不会喝那碗汤药。皇上这般出离愤怒,不过也是情知一片痴心相待,容嫔却无可无不可罢了。''

皇帝恼羞成怒,高高举起手来,如懿分毫不退,只是冷笑,``臣妾左脸已经受了皇上一掌,也请皇上雨露均施,赏臣妾右脸一掌吧!''

皇帝气急,荷荷而笑,``好!好!容嫔之事就算朕痴心所付,但她到底是朕的人了,前尘往事,朕也不与你计较。''

每一字入耳,都是将已经锥在心上的刺又逼进些许。如懿径直望着皇帝道:``皇上不计较前尘往事?那么皇上就是要计较今日之事了。''

皇帝面有怫然之色,``豫妃腹内草莽,昔日朕怜悯她年长入宫,又念她是蒙古格格,所以格外垂爱,谁知助长她骄横轻浮的个性。这些朕都不说了,今日她找到茂倩,也算是对你积怨已深,寻隙报复。朕可以不理会她,处置了她,让她与卑贱奴才混迹一处,老死在慎刑司。''他眉心曲折愈深,如同如懿起伏悬坠的心思,``但朕来问你,惢心一向是你手足心腹,你是她的主心骨。许多事你只需一想,甚至不必出声,她都会一一为你做好。是不是?''

心头如同针刺,刺得愈深,却不见血,唯知血肉间隔实实被冷硬利器分离剥开,痛得钻心刺骨。她明知那样难堪的话,她是不愿听到的,可是与其他说,却宁可食自己说出来。她扬起脸,硬声道:``所以皇上以为,那双靴子,那朵如意云纹,即便是惢心所绣,也是臣妾授意。只因臣妾与惢心主仆连心,是么?''

皇帝神色复杂,颇为忌讳,``有些话难听,何必说出来?''

如懿毫不避讳,直直道:``话再难听,也比藏在心里好。藏在心里便是一根刺,刺得久了便会流脓腐烂,也伤了自己的心。''

皇帝拂袖离她远些,``你不怕做出伤朕之心的事,朕还顾全你的颜面,你也该知足。''

有一瞬的恍惚,她不知对着他,该说怎样的话才算是得体。仿佛每一句、每一字,都是将彼此推得更远,推到万劫不复的境地,再无转圜,``今日茂倩虽然对臣妾颇有指摘,但臣妾不怪她,也不怨她。因为比之豫妃寻机报复,茂倩实是太不甘心!她的怨怼,臣妾如何不懂。为人妻子,最重要的便是夫君。凌云彻与她并非两情相悦,难免有所疏忽,才惹来今日是非。可臣妾与皇上多年相随,无话不可说,无事不可言。皇上有刺在心,不肯明言,可嫌臣妾不顾颜面说了出来。这般言行,彼此生分至如此境地,臣妾如何知足!''

皇帝的脸色愈来愈难看,如绷得死死的弦,禁不住哪句话就要断裂。他神色如寒霜被雪,冷冽不可直视,``朕以为冷淡你这些日子,你能静心思过,有所了悟。谁知皇后你真是越来越大胆了。''

``大胆么?''数年的冷漠相待,遥远的距离之后,却是难言的孤寂和孤寂里不肯退让的倔强、酸楚、粗涩,一点点磨砺着属于她的时光。那一瞬间,匆匆数载的幽寂与哀怨,凝成眼角一点冰雪般寒光,``还是皇上身为人君,心胸却如芥子一末,容不下半点与己不合之事。皇上介意凌云彻舍身救护臣妾,无非是因为自己身为人君,更为夫君,妻子有难不能以身相护。凌云彻救护有功,何错之有?他的错,无非是救了别人的妻子,让她夫君毫无作为,还为恂嫔置妻儿安危于无物,在人前露了凉薄之相。皇上深觉愧怍,自然容不得他!''

静默间,她听得皇帝沉重而粗剌剌的呼吸声。她再知道不过,他是动了真怒。曾几何时,他这样愤怒的时候,是自己伴随身边软语相劝。曾几何时,他的喜与怒她都紧紧系在心上,宁可自己百般委屈,也不肯添他一丝烦忧。而时至今日,她明知这些话会让他不快,让他激怒,却也不吐不快,忍不得,受不得。原来所谓夫妻,也不过如此,不过如此。

可是她已不是当年的她,他亦不复从前。自己固然是他的妻子,他是自己的夫君,可除了夫妻名分尚在,除了那依稀可寻的皮相,那个人,却脱胎换骨,早成了一具陌生的躯体。

皇帝并不喝止,只是摆首,冷淡若十二月的霜雪,``你说的这些话,可见心魔深重,难以自拔。''

如懿神色凄然,楚楚道:``臣妾固然心魔难去,皇上又何尝不是任凭心魔猖獗?若不是皇上将凌云彻舍命救臣妾母子的忠心视作男女之私,耿耿于怀,今日茂倩也好,豫妃也罢,哪里惹得出这番风波是非?一切一切,不过是因为皇上自己已然认定,才由得污浊之言,肆虐宫中!''

皇帝并无言语,只是手掌翻覆间,重重落在紫檀木几上,那紫檀本就沉若磐石,这一掌用力极重,只闻得碎石飞溅之声,如懿下意识地用手去挡,只觉得手心一刺,有硬物刺入皮肉之感。她垂首望去,锦红色绒毯之上,纷裂的绿玉碎碎零落。她心里一紧,下意识地先去看皇帝的手。他发白的拇指上,有暗红色血珠缓缓滴落。她本能地伸出手想去抚摸那伤口,却在手指触到他微凉皮肤的一瞬,被他森冷的语调生生拦住,``仔细你自个儿的手。''

她很难去探知,他话中的意味是否是显然的嫌弃,只是木然翻过自己的手,瞧见一粒绿玉碎飞过,擦破了掌心肌肤,留下一道渗血红痕。心底一片幽凉,手上的刺痛不过微笑一息,浑然未曾注意。才知苍茫痛楚之下,早忘却了皮肉之痛。

她看着殷红之上点点绿碎触目惊心,不觉茫然悲戚,轻轻道:``所谓玉碎,原来如此。''

皇帝显然吃痛,眉心不适地扭曲着,眉梢挑起,俯视于她,``理会这些小事做什么?''

她恍然醒悟,``臣妾去唤太医。''

皇帝霍然摁住她的手腕,``不必。这样急急招了太医来,若是传到外人耳中,成什么样子!''

如懿满心苦涩,如吞了一枚黄连在口中,连唇角的笑也勾起了那般苦冷意味,``今日茂倩这般胡闹,皇上倒不怕有流言蜚语传出去么?''

皇帝的手抓得她太紧,压得伤口血液滴滴渗出,在苍白的皮肤上,显得触目惊心。皇帝怔了怔,显是发觉了她的痛楚,随手扯过她纽子上系的杏色水绫绢子抹了几把,随手撂下道:``回去悄悄叫江与彬替你悄悄,无须声张。至于茂倩,朕自会处置,令她不许妄言。令贵妃懂得分寸,也不会外传半字。''

如懿有恍惚的失神,``是了,皇上回宫,自有令贵妃曲意照料,是臣妾多虑了。''

皇帝正要出言呵斥,那一缕怒气却泯然成一声悠长叹息,``如懿,为何你说话竟这般尖酸了?''

如懿恍然失笑,``皇上,臣妾不是尖酸,只是心酸。臣妾与皇上自少年相伴,几经风雨,如今却彼此猜疑,事事疑忌。令贵妃与容嫔相伴皇上之数自然不能与臣妾相较,一个得皇上信任,一个得皇上万千爱惜。臣妾看在眼中,五味杂陈,实不忍言。''

皇帝目中闪过一丝惊诧与不满,``你是皇后,任凭朕怎么宠爱她们,予她们权重宠幸,你都是皇后,谁也越不过你去。''他顿一顿,``你还记得孝贤皇后么?若不是过于在意,她又怎会心力交瘁,盛年早逝?朕劝你一句,宽心为上。''

这些话,险险逼落她的泪来,``臣妾前半生与孝贤皇后纠缠不休,近年来静极,才渐渐明白孝贤皇后之心。孝贤皇后家世显赫,儿女双全,又是嫡妻,尚且求不得夫君之心,才生危惴之感。臣妾如何能与孝贤皇后比肩?能跻身后位,不过缘于与皇上彼此相知之情,如今几乎不能保全,更觉如履薄冰。''

皇帝不语,只以静默姿态,凝神望着窗外碎雪零丁。如懿亦不作声,只是俯身拾起那块绢子,以极轻极柔的动作,敷上他拇指的伤口。皇帝定了定神,肃然道:``令贵妃理事之才远不如你,无非温柔妥帖些,才能上下照应。等你好些,六宫之事还是交由你来打理吧。也少些闲言闲语,以为帝后离心,平生揣测。''

如懿愣了片刻,不想皇帝说出这番话来。不知怎的,她只觉得哀凉,却搜觅不出一丝温热的暖意。像是沉溺在水底湖藻中的人,看着远方结冰的湖水之上摇曳破碎的影,那些陈年旧事,如暴雪纷纷下坠,砸在冰面之上,晃动着她的世界。她缓缓起身,保持着行礼谢恩的姿态,以逐渐干涸的双目相望,静静道:``皇上此意,若是对臣妾毫无疑心而起,臣妾自当感激于心。可若皇上只为平息六宫流言而施恩泽,人前授予臣妾权柄,人后却怀疑臣妾清白,那臣妾实不能坦然接受。''

皇帝的唇线越抿越紧,仿佛生怕决堤的情绪会一涌而出,他极力克制道:``皇后,你便这般不识抬举么?''

``或许臣妾不识抬举,但比之表面文章、虚与委蛇,真心相待不会那么累。''她起身再拜,``皇上,臣妾年长身倦,怕是不能将六宫之事料理周全。您属意于谁,便是谁吧。臣妾倦得很,先告退了。''

她扶着酸软的膝,缓缓前行几步,听得他的声音自后沉沉传来,无限怆然,``皇后,你与朕一定要这样么?''

脚下一滞,如坠铅般沉重。她却不肯回头,怕去看他的面孔,那逐渐老去的却依旧棱角坚硬的面孔,``从皇上疑心臣妾的那一刻,从臣妾认定皇上疑心的那一刻,好像我们,就再也走不到一块儿了。皇上,或许您有不是,臣妾也有不是。但这不是,想要消弭,似乎很难了。在臣妾被凌云彻所救的那一刻,皇上看着臣妾的眼神,不是为臣妾得救而欣喜,反而疑云丛生,臣妾的心便凉了。这些日子,臣妾一直在想,皇上会不会说出这些伤人之语,却原来还是逃不过。''

皇帝的沉郁中隐隐有激愤如雷霆逼近,``从容嫔进宫之后,从你被凌云彻所救之后,你每每与朕言及你的倦怠,难道与朕一起,真的让你如此厌倦么?''

有滚烫的泪无声而落,烫得她一颗心骤然缩起,不是不觉哀伤,只是哀伤之后,更多的是了然的绝望,``臣妾所在意的从不是容嫔是否进宫,而是皇上不惜一切的执着,伤人伤己。甚至臣妾,其实是很喜欢容嫔的性子的,可皇上,却生生逼迫着她,也伤及后宫诸人。至于凌云彻,臣妾浑然不知皇上有何可介意,还是连自己也觉得,对于一个女子的爱护,尚不如一个侍卫的忠义。心既疏远,身何能从?皇上,臣妾无话可说了。''

她说罢,再不肯停留,唯有裙裾拂过金殿的转角,那沙沙的摩擦的微声,仿佛岁月无情的手,磨砺着他与她之间仅剩的脆薄如碎纸的情感。她明明知道的,那样脆弱的一点温情,是黄昏残留的夕照,眼睁睁看着它被黑夜的暗色一点点吞噬,却无能为力,只余满心悲怆!

永寿宫偏殿里烘着极暖的地龙,春婵脱去了大毛的衣裳,只一袭暗紫色宫女装束,手脚轻便地伺候着茂倩。茂倩换过了一身衣裳,重又梳好发髻,坐在暖炕上哭得声噎气直,险险昏死过去。春婵蹲下身用沉甸甸的火筷子拨了拨大铜脚炉里的炭,让它烧得更烈些,在旁劝道:``姑姑不要这样,既然婚事不谐,早早了断了便好。姑姑有这般身家,又有御前伺候的身份,还愁什么好人儿不得。''

茂倩才匀了脸,又哭得满脸涕泪,恨声道:``你知道什么?我拼着一口气,只为他不让我好过,我也不让他好过罢了。离了他,旁人不知道拿多少难听的话说我呢。''

春婵犯愁道:``那也是。男人啊,在一块儿过日子都有那许多抱怨呢,如今写了放妻书,能给姑姑你多少好过,也不知怎么嚼舌根呢。他倒落了个自在。''

茂倩掩面哭道:``我原也想忍忍过下去便罢,奈何吞不下这口气罢了。干脆闹到御前,落实了他和皇后的罪名也好,省得我看着日夜心烦。谁知皇上不信,姓凌的也浑然无事,倒成了我小人之心诬告了。''

春婵掩唇诡秘一笑,``皇上不信?那也未必。''

茂倩拿绢子拭了泪,好奇道:``你怎知道?''

``豫妃嚼舌根犯是非,那是皇上一早便多嫌了她,如今正好有个由头而已。可姑姑是举证的,豫妃不过领了你来。为何你平安无事,还脱了这遭罪的姻缘?你以为皇上真的半分没有信你?''

茂倩转念一想,破涕为笑,``是啊。我在皇上跟前多年,素知皇上许多心事是不肯说出来的,并非面上看着这般好相与。当年要我嫁与凌云彻那个混账,一是赐婚荣耀笼络着他,二也是因为凌云彻在御前伺候,不能有二心。才叫我嫁与他之后从旁看着。如今御赐的姻缘平白断了,难保皇上心里不恼恨那混账。''

春婵叹口气,拨了拨鬓边的点翠玛瑙珠绒花,道:``皇上恼恨凌云彻也罢了,终究不干咱们的事。可若恼了皇后,不知又要生出多少风浪。这些年皇后渐渐离心,便是咱们下人也看得明白。从前总不知为了什么缘故,姑姑你来了,咱们都明白了,左不过是皇后心里有了别人了。''

茂倩复又哭道:``春婵,你也是明眼人。今儿那个样子,凌云彻那混账虽一句话不偏帮,可他的心耳意神,哪一会儿不在皇后身上了?人该是母仪天下,偏她得不着皇上的宠爱,来寻思旁人的男人。说那如意云纹是惢心绣的,说凌云彻梦里唤的不是她,打死我也不信。''

春婵听得连连摇头,感慨不已,伸手端了热茶给她,又亲手拧了热帕子给她抹脸,温言劝道:``别说你不信,这样牵强的话,我也不信,只怕皇上心里更不信。可没有办法啊,姑姑你一番心血,拿出来的却都不是铁证,谁能信服啊!''

二人正说话,却听门外小太监恭恭敬敬唤道:``茂倩姑姑在里头么?奴才给您送东西来。''

茂倩因听人来,便端端正正坐了,春婵也退到一旁忙活着替茂倩整理换下来的衣裳,彼此隔得远远的。茂倩见那小太监进来,手里捧了一封银票并一雪白纸张,道:``姑姑,这是凌大人着奴才送来的。''

茂倩别过头,哼了一声道:``这会子急吼吼地送银票来做什么?打量着拿银子哄我高兴么?''

那小太监苦笑着道:``茂倩姑姑,这银票是凌大人的。他说他多年积蓄,大半给了姑姑,想着姑姑以后要一人度日,难免辛苦,念在夫妻一场,他所余的,都给姑姑罢了,也当好聚好散。另一封是凌大人的放妻书。凌大人托奴才交付与你,还有一句话,`夫妻缘尽,各落清静'。''

茂倩身子一凛,双手剧烈地颤抖着,``好!好!皇上一句吩咐而已,他就这么迫不及待要休了我!我偏不成全他!''

那小太监原是养心殿伺候的,有些身份,见她这般拿乔,也按捺不住道:``姑姑您不成全,皇上也已经发话了。姑姑,您在御前多年,难道看不出真是得罪了皇上?皇上没说要凌大人休了你,只说是放妻书,您知足吧!''说罢,径自搁下,打了个千儿出去了。

茂倩气得浑身乱颤,想要起身,一下子又跌坐了下去。春婵忙不迭去扶,口中道:``姑姑这是何苦来着。自己该说的话没说到点子上,该吐的东西没吐干净!这会儿谁来可怜你呢。倒是成全了凌云彻,往后待在宫里,一心一意看着他日夜思念之人。你做了他十来年妻房,还不是被他甩脚底泥般甩了,还落个不贤的罪名!''

茂倩两眼直欲喷出火来,倚在春婵身上,发狠道:``既说我不贤,又将我弃如敝屣,我何必还替他藏着掖着,有桩事儿,我疑心久了,少不得一并告诉了贵妃娘娘,请贵妃娘娘替我做主。''

春婵吓得连连摆手,向四处看了看道:``我的好姑姑,您还瞧不出来,我们贵妃小主便是个菩萨性子,连豫妃也降伏不住的,哪里替你做得了主?便是如今皇后娘娘这般失宠,我们贵妃这般老实,见了她气也不敢喘的。''

茂倩严重直直淌下来泪来,``我命苦,这般受人欺侮,再没人做主。''

春婵想了想道:``皇后娘娘素来脸酸心硬,不能容人的。我们小主也可怜姑姑,只碍着皇后娘娘厉害罢了。但若姑姑说的真有其事,铁证如山,那我们小主为着宫规严谨,少不得也要替你主持公道。''她说着,忽又灰了信,``只是你疑心的事儿,还没个影儿呢。再被驳回来,你连命都没了!还是凡事想个万全才好。''

茂倩细细寻思了片刻,道:``这件事细说起来,关系着前头淑嘉皇贵妃的八阿哥永璇坠马之事。''

春婵心下一紧,禁不住打了个哆嗦。茂倩不满地横她一眼,``你胆子也忒小了,这话听着那么怕么?''

春婵忙赔笑道:``这件事可大可小,说小了是八阿哥伤了腿成了跛子,往大了说,后来淑嘉皇贵妃报复皇后,放狗咬伤了五公主,又惊吓了有孕的忻妃,牵连着六公主病弱而死,后来淑嘉皇贵妃又活活气死了,干系着多少性命呢?''

茂倩抿着唇道:``我何尝不知道个中厉害?那件事当年便是凌云彻亲自去查的。我嫁给他多年后,有次听他与赵九宵喝酒,两人都有些醉了,赵九宵嘴快,说他为了皇后娘娘这般犯险,却什么也不肯说。我那时端了酒去,在窗外听见便留了心,知道那事和两枚银针、一个马鞍有关。而那些东西,我见凌云彻在家中柴房的杂物里翻动过,如今若去翻一翻,怕是还在。''

春婵听得心口突突乱跳,险险跪下,道:``我的好姑姑,你这话里有多少文章,我可不敢听。您今夜别出宫了,赶紧着下人把这些东西找来,再找人证,给您做主吧。''

茂倩双手紧握,想了想唤进自己的贴身丫鬟,低声嘱咐了几句,道:``你赶紧出去,找了这些东西来。''

春婵见那小丫鬟出去了,往窗外看了一眼,笑道:``姑姑先歇息,小主身边怕离不了我伺候,我先过去得了。''她说罢,便急急往嬿婉身边去了。

\hypertarget{ux7b2cux5341ux516bux7ae0-ux5206ux98de}{%
\chapter{第十八章 分飞}\label{ux7b2cux5341ux516bux7ae0-ux5206ux98de}}

是夜,皇帝便往永寿宫中来,不过略看了看嬿婉,便要往宝月楼去。

嬿婉少不得笑语嫣然,``晚膳时臣妾见有几样膳食精巧,想要送去宝月楼,才想起今儿是斋戒,容嫔妹妹断不肯吃这些东西,这才罢了。''

皇帝恍然醒觉,``也是。既是斋戒之日,容嫔会彻夜诵读经文,不见外人,朕也不必去瞧她了。''

嬿婉抿唇一笑,温温软软道:``皇上一向最将容嫔妹妹的事放在心上,今儿怎么浑忘了。臣妾可要为容嫔抱不平了。''

皇帝不置可否地一笑,牵过她的手一并坐下,摩挲着道:``你待容嫔却好。''

嬿婉低着曲线优美的颈,柔顺道:``容嫔妹妹远离家乡,孤身一人,承恩已久却膝下孤凉,臣妾也曾多年未育,很明白她的心境。由己及人,总忍不住对她好些。只是容嫔妹妹性子孤介,不太喜欢臣妾。所以臣妾有时想对她更好些,也不知该从何做起。''

皇帝脸色僵冷,直到听嬿婉说完,才怜惜地抚着她的手,温言道:``她的性子素来如此,待朕也是一样。你心意到了就好。''

二人正说着话,澜翠端了茶水上来,笑吟吟道:``这是今岁新贡的松阳银猴,小主吃着觉得很好,所以特意等皇上来了一起尝尝。''

皇帝笑道:``你也喜欢这个?''

嬿婉笑容甘芳,让人有亲切的松弛,``虽然不算名贵茶种,但臣妾喜欢它入口回甘,平实亲和,没有高高在上的疏远之感。仿佛邻家女儿,品之可亲。''她见皇帝只是沉思不语,又笑道:``臣妾掌管六宫之事,但见茶叶一项,每年便支用颇大。宫中素来以饮名茶为习,若是愿意多尝尝松阳银猴之类,所费不多,亦有新味,也是不错。''

皇帝沉吟片刻,伸手接过青玉金线茶盏抿了一口,淡淡笑道:``皇后为皇贵妃主理六宫时,一度也引松阳银猴入宫,想是有旧例可循。你若愿意多看看典册掌故,想来可以安排。''

嬿婉闻言不禁有些讪讪,皇帝言下之意,便是觉她不熟悉宫中掌故了。她不觉羞赧,``臣妾愚钝,还望皇上恕罪。''

皇帝拢过她的肩,安慰道:``你虽身为贵妃,但到底资历尚浅,便是婉嫔与愉妃也比你久经世故,你难免有些稚嫩。但是你性子温婉,凡事上下融洽,不严苛冷峻,这是你的好处。''他停一停,``自然也是皇后的缘故,她身子不好,你得多担待些。''

嬿婉秀眉紧蹙,这才稍稍和缓些,含笑示意澜翠递过茶盏来。澜翠正捧过茶盏,手中陡得一滑,一盏滚烫茶水瞬时浇在了嬿婉手上,烫起一大片绯红颜色。

嬿婉雪雪呼痛,澜翠吓得傻了,跪跌在地上拼命磕头不已。皇帝捧着嬿婉的手连连呼气,宫人们忙乱着又是端冷水来给嬿婉浸手,又是取了清凉消肿的膏药涂抹,一壁又急急去召太医。嬿婉痛得满眼含泪,只咬着唇不说话。皇帝一时怒极,狠狠踹了澜翠一脚,喝道:``这等刁钻惫懒的奴才,还不拉去慎刑司!''

王蟾忙答应着拉了浑身哆嗦的澜翠下去。皇帝又安慰了嬿婉许久,本欲留下,耐不住嬿婉苦苦劝道:``皇上今夜便是留在臣妾这儿,也怕是担心臣妾的伤势,不能好好歇息,还不如回养心殿安寝。''

皇帝如何肯允,嬿婉又道:``皇上若实在不放心,大可留了李玉在这儿伺候。李玉本就细心周到,若有不妥,可及时禀告皇上。''

皇帝亦怕留在这儿,嬿婉事事亲力亲为服侍,反倒不得养息,叮嘱了几句,留下李玉便起身去了。

这一夜养心殿中,皇帝便睡得不大安稳。本唤了婉嫔来侍寝,才一见面,见婉嫔打扮停当,却讷讷寡言,不觉又是恼又是笑,``怎么?见了朕便这般怕么?话也不肯说了。''

婉嫔手足无措,``臣妾\ldots 臣妾已经多年未曾侍寝,生怕自己不够妥当\ldots{}''

皇帝苦笑道:``罢了。朕召你来,不过是因为你乃潜邸旧人,可以夜话闲聊,你既这般局促,罢了,朕叫人送你回宫吧。''

婉嫔面皮赤红,只得无言告退。皇帝索然寡味,进忠在旁赔笑道:``皇上,婉嫔本就年岁渐长,不宜侍寝。不若唤了别的小主来侍奉可好?''

皇帝摆手,不耐烦道:``朕何愁谁来侍寝?不过是想找个人说说话罢了。''进忠欲言又止,皇帝横他一眼道,``平日里你鬼主意最多,有话便直说。''

进忠忙躬身道:``皇上,其实有个人在外候着许久了,也有话要对皇上说。''

榻前一盏紫铜鹤形烛台孤然耸立,曳下瘦长的影子,越发显得凄惶难言。皇帝慵懒道:``谁?''

进忠悄悄觑着皇帝脸色道:``茂倩。''

皇帝陡然坐起,厌烦道:``叫她早些出宫安分些,今日之事朕便不与她计较了。''

进忠赶紧趴下磕了个头道:``皇上,茂倩说,此事她若不说与皇上知道,宁可一头碰死在养心殿前的石阶上。奴才见她情愿一死也要上禀天听,才不得不来禀告。''

皇帝静了片刻,缓缓道:``唤她进来吧。''

海兰回到延禧宫中,已是中夜了。叶心服侍着她脱下半新石青色绣白玉兰花缎面狐毛大氅,接过她手中的珐琅透雕手炉,心疼道:``小主今儿在皇后娘娘那儿留得晚,赶紧歇息吧。这手炉都凉了,奴婢去换上炭,给您再暖个汤婆子睡下。''

海兰叹道:``姐姐受了这么大的委屈,只有我陪着她说说话罢了。你自己也瞧见了,姐姐挨了那一掌,脸上肿成那样,也不知什么时候能消得去。''

二人正说着话,却见永琪从里头暖阁转了出来,迎上来请了安道:``额娘总算回来了,叫儿子好等。''

海兰见他满脸关切,甚有孝心,一时欢喜,也有些诧异,``你这孩子,这么晚了也不回自己府里,在这儿做什么?成家立室的人了,也不怕你福晋惦记。''

永琪忙笑道:``今儿原是见外头送了好些紫貂皮子和人参来,所以儿子特意挑了好的,送来给额娘和皇额娘。''

海兰听他提及如懿,不觉喟然忧惧,``如今你要见你皇额娘,也不大方便。这些东西,额娘自会转交。''她看着长身玉立的儿子,不觉生了几分疼惜之意,``看你这么孝敬你皇额娘,也算姐姐没白疼你一场。''

永琪有些愧疚,道:``儿子本该亲自去向皇额娘问安。只是皇额娘如今的情形,儿子也得明哲保身些。''他扶了海兰坐下,``额娘也累了,暖阁里儿子刚叫人添了热炭,您快坐下歇歇。红枣银耳羹也刚煨好,热热的正好用呢。''

海兰见他这般细心,愈加安慰,拉了他一并坐下,道:``你素来孝顺,额娘都知道。''

永琪见无人在旁,踌躇片刻,低声道:``额娘与皇额娘亲厚,那也是应当的。只是也得小心些,免得惹皇阿玛不悦。''

海兰摆摆手,接过叶心添好的手炉捧着,温言道:``自你出生,额娘便是无宠之人,何必在意这些。''她面色微微一沉,有些不豫之色,``你素性谨慎,又文武双全,你皇阿玛便视你为第一得意之人。你明哲保身是不错,对你皇额娘的孝心也不必尽在明面上。可内里,你皇额娘疼你可不亚于她亲生的永璂,你心里可得明白。''

一席话说得永琪冷汗涟涟,忙敛衽跪下道:``额娘的话儿子怎会不知?只是自三哥离世,儿子便是长子身份,不得不万事斟酌,便有对皇额娘十二分孝敬之心,也只敢露了三分。毕竟皇额娘与皇阿玛不睦,儿子也不敢在明面上过亲近了翊坤宫。''

海兰瞥他一眼,语意清冷,``你这个想头固然不错。若不是你天资聪颖,又谨小慎微,也无今日气候。''她见永琪一味低头,亦是不忍,``地上湿寒,别尽跪着了。入秋腿上的附骨疽更易发作,总是隐隐作痛,益发得小心些。''

永琪下意识地摸了摸腿侧,也不以为意,``太医总是那些套话,什么三阴不足,外邪过盛。左不过黄豆大小一颗,不痛不痒的,也没什么。''

海兰叹道:``你离宫开府,自成一家,虽然有福晋替你操持,自己也得事事留心。''她一顿,似想起什么,``我听跟着你的诚贵说,你身为兄长,在书房读书勤勉依旧,可堪榜样,而且下了学\ldots 待令贵妃的几个阿哥也极好。''

永琪嘴唇微微嗫嚅,还是坦然道:``令娘娘协理六宫,深得皇阿玛宠幸。儿子疼爱几位年幼的弟弟,也是尽兄长的职责。''他略一犹豫,一双澄澈眼眸望着海兰道:``额娘在宫里资历虽深,但恩眷不隆,儿子这般做,也是希望额娘与令娘娘面上过得去,别损了额娘的尊荣清宁。''

海兰爱惜地抚一抚他的额头,叹息道:``你要强周全是好,但也别为求万全,什么事儿都自己忍着。年纪轻轻的,绸缪太过,也损心神。再说你素性要强,有什么头痛脑热也忍着不说,可自己身子总要当心。''她话锋一转,婉转道:``上回听你说起长了附骨疽,额娘急得什么似的,问了太医。说是先头的怡亲王父子都得过,确是不大要紧。你精于骑射,风餐露宿、骑马射猎所致也未可知。''她说着,语调一沉,有些不大好意思,``不过,太医也说,冷浴后贪凉寒湿侵袭,或房欲之后盖覆单薄,寒邪乘虚入里,也会成此疾。终究,你得当心你自己身子。''

永琪面上一红,旋即含笑道:``这个额娘大可放心。儿子的嫡福晋西林觉罗氏和侧福晋索绰罗氏都是皇阿玛、皇额娘和您亲自替儿子选的,她俩温良恭俭,实是贤妻。''

海兰扑哧一笑,轻轻点了点他的额头,笑骂道:``当着额娘的面心虚什么。额娘岂不知你对嫡福晋和侧福晋不过面上的情分,而索绰罗氏擅生养,你的几个儿子多是她所出,可你最心疼的还是格格胡氏。别的也就罢了,额娘只担心一个\ldots{}''

永琪见海兰颇有责怪之意,忙不迭解释道:``额娘所担心的,不过是胡氏出身寒微,是府里买来的丫头做了通房封了格格,但她性子也算乖巧,安分守己,从不逾矩。''

海兰不禁摇头,``额娘才说这一句,你便有这许多话替她分辩,可见偏心。虽说王公贵戚都三妻四妾,你别有宠妾灭妻的逆行便好。''

永琪笑意温和谨顺,``额娘说得是。儿子的福晋都温顺贤良,胡氏虽然娇艳些,但也不大出格,服侍得儿子极好,对福晋们也恭谨。额娘可曾听过福晋抱怨?''

海兰温然生笑,``你的福晋都是老实的,额娘也希望你有贤内助。你若争气,你皇额娘的日子也好过些。''

永琪正要答应,忽然笑意一滞,颇为犹疑,``额娘,儿子也的确想为皇额娘争气。可有句话,关起门来只能咱们母子间说得。''

海兰知他素性缜密,便也着紧,道:``怎么?''

永琪踌躇片刻,似是十分为难,``额娘,儿子说句不当说的话。额娘与皇额娘情同姐妹,皇额娘也待儿子如亲生。可十二弟一日日大了,儿子虽与他亲厚,但也不能不多思虑几分。十二弟才是皇阿玛的嫡子,中宫所出。''他苦笑,``有他在,儿子终究是名不正言不顺。便是他日封得亲王,也不过是为他人作嫁衣裳罢了。''

海兰唇角的笑意逐渐冷却,如寒天里冻住的雪花,闪着苍冷的雪白微光。永琪看着她的笑容,不自觉地后退两步,畏惧地低下头不敢言语。

海兰的声音没有丝毫温度,``跪下!''

永琪哪里敢违逆,双膝一软便跪倒在地。海兰将指上的镂金丝嵌珊瑚珠护甲一枚枚摘下,一记耳光清脆地响在永琪左脸,很快又落在右脸。她的手并不停歇,一下下用力打着,眼中泪水涟涟。``如果没有你皇额娘,我们母子当年便死在了延禧宫里,你的眼睛哪里睁得开见见这人世?如果没有你皇额娘,你就是个失宠嫔妃的庶子,谁会来理你分毫?你能上书房读书,能文习武,你能博你皇阿玛欢心,你能在那么多兄弟中脱颖而出,是谁为你筹谋?不为别的,只为你养在你皇额娘膝下,才有今日的荣华!便是你能写得一手好书法,都是你皇额娘亲手教你。她为你尽心挑选贤妻,为你成家立业。她为你费的心思,连对她亲生的十二阿哥都比不上。如今你却糊涂油蒙了心,说出这般忤逆的话来,额娘听着,真真是寒心!''

永琪哪里还敢接话,俯下颀长的身子连连叩头,扇着自己耳光道:``额娘息怒!额娘息怒!儿子不孝,一时昏了头说胡话,额娘切莫气伤了身子!''

``身子?''海兰指着他,满脸是泪,冷笑道:``你还知道额娘的身子!额娘不过是个废人,早就失了你皇阿玛的宠爱,不过是熬一天是一天罢了。若无你皇额娘对你悉心照拂,只怕要养大你都难。你别今日得了尊贵,便忘了自己的来历!''

永琪难过道:``儿子也是糊涂,总觉得自己再讨皇阿玛喜欢,总比不得十二弟天之骄子,生来尊贵。皇额娘疼儿子,也不过是为自己的儿子来日有个臂膀而已。''

``十二阿哥尊贵,那是他额娘贵为皇后,没什么可争的!你这般话,便是戳额娘的心了,也是打你自己的脸。要怪便只怪你没投生个好肚皮罢了。额娘失宠多年,从来不以为侮。因为让人轻贱的,从不是出身,而是自己的品格行事。你若这样想,和当年的大阿哥又有什么分别?你大哥得了你皇额娘多年抚育,却不思感激不念养育之恩,才落得如此下场。而你如今身为长子,已是你皇阿玛的左膀右臂。你若真有那个福气,定要尊你皇额娘为母后皇太后,额娘便是做太妃也不要紧。若你没那个福气,安心做个亲王享尽富贵,辅佐你十二弟,也是情理之中。你可仔细!别还没到那个位子,便先动了不该有的心思。你大哥、三哥和四哥,都是前车之鉴!''

永琪冷汗淋漓,抖衣而颤,``额娘息怒,儿子明白。''

``明白?''海兰一把托起他下颌,肃然道,``你不明白!从你托生到我肚子里那一日,你便在受着旁人算计!要不是你皇额娘与我彼此扶持,我怀着你时冒险服了些许有毒的药物才从冷宫解了你皇额娘的冤屈,她又在我生你时陪伴在侧,事必躬亲,这世间早没你这个人了!所以,少生事端,安分守己!额娘和你的福气才能长远!''

永琪如同五雷轰顶,望着海兰,颤声道:``额娘,你为了皇额娘,竟然服毒,那时还怀着儿子,额娘你\ldots{}''

海兰松开手,静静地凝视着他,拈过绢子,温柔地为他拭去额边冷汗,神色温柔而坚定得不可抗拒,``永琪,人要活下去,总是不得不用些法子。额娘一直觉得对不住你。但是你也不能为着今日的荣华而妄生猜疑之心。你便是要猜疑额娘,也断不能去猜疑你的皇额娘!这句话,你牢牢地记住!''

永琪泣不成声。在他成长的记忆力,他很少哭,真的很少。这样无声地哽咽,肩膀用力地颤抖着。他伏在自己的臂弯里,背脊如黑夜里起伏的山脉。海兰的手沉稳地搁在他肩上,任由泪水静静滑落,``永琪,额娘知道,你在宫里长大,兄弟不似兄弟,父子更似君臣。你疑心多些便可防范多些。但人生而不易,你若是再疑心曾对你有养育之恩的人,便是天诛地灭。额娘谁都不信,只信你皇额娘。你也一样,记得!''

永琪沉重而用力地点着头,仿佛只有这样,才能将海兰的教诲沉沉刻画在心中。他的脸色寂寥而凄楚,``额娘,难道你最心疼的人,不是儿子?''

海兰半蹲着身子,伸手抚着他年轻而饱满的面庞,依稀分辨出皇帝隽逸倜傥的模样,``你和你皇阿玛年轻时长得真是像。只可惜,他心里从来没有我,我心里也从来没有他。额娘最心疼的人,是乌拉那拉如懿,是爱新觉罗永琪。可额娘不得不明白告诉你,我与你皇额娘在一起的时日更长更久更贴近。我们之间的信任,无人可以动摇。额娘希望你明白,对你好的人,别去辜负她、背叛她。''她站起身,倦倦道,``永琪,宫门已经下钥,你便留在这儿睡下,好好想想明白吧。''

她缓缓站起身,唯留永琪半靠在暖榻的踏脚上,疲倦而凄凉。他悲戚地紧紧拢住自己的身体,将喉底的哽咽死死压住,``额娘,额娘,你为什么这样待我?''寒夜冻雨,凄瑟敲窗,落在花梨木透雕藤萝松缠枝窗格上发出生硬单调的声音。天地寂寞,唯有以此簌簌相应。

天地寂寞,静夜无声。皇帝双眸微红,可见已困倦到了极处。他看着跪在眼前匍匐屈身的身影,沉肃的口吻中隐含着一丝不易察觉的沙哑,``茂倩,你的话已经说完了,可朕还是不信。''

茂倩面色铁青,两颊泛着决绝的晕红,恭顺地匍匐在地,``皇上,若说凌云彻梦呓之事不算铁证,可这两枚银针与这个马鞍,却真真是铁证如山。若不是为了包庇皇后意图杀害八阿哥之事,这两枚银针凌云彻为何要藏着掖着不能见人?奴婢思虑良久,事涉皇裔,不能不冒死相禀。''

皇帝颇有玩味之色,眸中阴沉不定,举起那两枚银针在眼前,沉吟道:``银针已有积垢,是积年旧物。针孔与马鞍底下的孔痕也相吻合,的确不是造假之物。但茂倩,你与凌云彻早是怨侣,如今积怨更深。哪怕是物证笃然,朕也不能全信。''

茂倩垂首片刻,眼里闪过一丝怨毒恨色,举首道:``物证已在,皇上所不能信的,不过是奴婢这个人证。奴婢已说过,当日之事赵九宵也知情。眼下他人在宫中,皇上一问便知。''

皇帝并不看她,只专注于银针之上,冷冷道:``还须你说?朕已经吩咐进保将他带了来。''他击掌两声,外头进保已经听得,领了赵九宵入内跪下。

皇帝道:``李玉呢?''

进保回禀道:``皇上知道李公公与凌大人私交甚厚,怕有消息泄露。所以奴才传皇上的旨意,请李公公今夜往孝贤皇后陵上送祭品去了。至于其他人,有奴才在,他们近不了养心殿三尺。''

皇帝扬一扬首,示意他出去,只冷眼瞧着瑟瑟缩缩的赵九宵道:``唤你来所为何事,你自己也知道吧?''

赵九宵初次面圣,早已头昏脑涨如在梦中。及至了明彩辉煌的殿阁里,浑身软绵绵如同酒醉,吓得一跌倒地,连连叩首不已,大着舌头道:``奴才愚昧,奴才不知。''

皇帝视他如目下尘芥,哪肯轻易费一词一句。还是茂倩乖觉,指着地上的东西道:``赵九宵,这个马鞍你总认得吧?''

九宵一见那马鞍,心底一凛,猛然清醒了不少,连连摇头不已。

茂倩料得他不会轻易认了,不觉抱臂冷笑道:``你与凌云彻那点勾当,皇上还会不知吗?八阿哥马场坠伤之事皇上已经了然于胸,不过白问你一句,瞧你对大清忠不忠心罢了,你还敢蒙蔽圣上吗?''

九宵吓得冷汗如浆,但见皇帝成竹在胸,以为皇帝早已知晓,慌不迭道:``皇上,这个马鞍奴才知道,当年八阿哥坠马,凌云彻奉命去查,才知八阿哥坠马乃是因为马匹受惊。''

皇帝也不听他絮叨,不耐烦道:``马匹受惊乃是两枚银针穿透马鞍底下的皮子,这些朕都知道。但凌云彻当初奉朕旨意追查,却未曾向朕回禀,这是为何?''

九宵瞠目结舌,呆呆道:``皇上都知道了?那\ldots 那其他事,奴才不知。''

茂倩尖着嗓子,像生锈的刀片沙沙刮着耳膜,``你会不知?你是他的手足兄弟,我不过是一件破衣烂衫。他什么事情你不知道?这些事他是替谁瞒下的?为了谁凌云彻那混账才敢连皇上都蒙蔽!你便招了吧!''

九宵骤然色变,却也不屑,``鸡鸣狗盗之辈。以为偷了马鞍和银针出来,就能诬陷自己的夫君了吗?也难怪这些年凌云彻看不上你,换了我也看不上!''他奓着胆子向皇帝道:``皇上一片好意赐婚,可这悍妇刁蛮不驯,但凡夫君有一点不合意,就横鼻子瞪眼睛,更别说凌云彻若当值晚些回去,或与邻家妇人招呼一声,她必要吵骂。微臣与凌云彻知交多年,虽也屡屡劝他要夫妻和睦,可也着实看不下去。''他见皇帝面色不变,只闲闲听着,越发壮胆,``皇上,这女人醋妒,又小心眼儿,她说的话实在不能相信。''

皇帝也不看他,只伸手细细抚触那马鞍,细看上头的针孔,``这马鞍是马场用的样子,也有些年头了,上头的针孔也与这两枚银针一般无二。茂倩,你便这么有心,一早便存下心思陷害你的枕边人了么?''

这话虽是质问,但语中之意直逼赵九宵。九宵再不经事,也不免畏惧不已。

茂倩自以为得意,昂首道:``皇上,奴婢之所以到今日才向皇上告知此事。一则因为前事不明,怕有误会。今日见凌云彻百般维护皇后娘娘,倒落实了心头疑虑。奴婢想,当年八阿哥坠马致残一事,宫中曾纷传是五阿哥所害。凌云彻奉旨彻查,却诸多隐瞒。想来他与愉妃小主并无来往,也不会为她隐瞒。能让他做出这般欺君犯上之事的,唯有是皇后娘娘了。''她仰着脖子,眼底闪着恶毒的冷光,``奴婢私心揣测,会否这件事连五阿哥也被蒙蔽,乃是皇后娘娘的一箭双雕之计。''

皇帝神色冷凝,映着窗外呼啸凛冽的风声,格外瘆人。他沉沉道:``你说什么?''

茂倩膝行两步上前,声线诡异而隐秘,像一条绷直的铁弦,死死缠绕上柔软的颈,``皇后娘娘有自己的亲生子,从前疼五阿哥也是为了有个依靠。如今自己有了儿子,五阿哥又天资聪颖,能文能武,皇后娘娘怎能不为自己的儿子打算!八阿哥坠马这件事,若是扯上了五阿哥的罪过,自然断绝了他的皇位之路。若是不然,八阿哥落下残疾,一是不能继承大业,二也报了皇后娘娘对淑嘉皇贵妃的旧仇!''

殿外,是伸手不见五指的黑夜,养心殿、翊坤宫、永寿宫,成百上千座殿宇楼阁,都冻成了阴霾里巍峨不动的影。明明殿内,生着数十个火盆,和煦如春。可是皇帝立在那里,只觉得血液从脚底开始冰冷,缓缓凝滞,慢慢逼上胸腔,冷凝了喉舌。连手心逼出的汗意,也是寒冻的雨珠,冰冷地硌着。高处不胜寒,终究是高处不胜寒。

他的声音已经嘶哑了,眼底纵横着暗红的血丝,``所以,你觉得,朕的璟兕死于非命,完全是因为她有这么一个心肠歹毒的额娘,是不是?''

茂倩的歇斯底里撕破了暗夜最后的宁谧,也撕破了皇帝心底最脆弱的伤口,``是!五公主玉雪可爱,要不是有这样的额娘,皇上,您会看着五公主长大,长得亭亭玉立,成为大清最美丽的公主。您可以亲眼看着她出嫁,有一个好夫君,有一个美满的人生,而不是早早夭折,沦为后宫争宠的牺牲品。''

皇帝的泪汹涌而出,他跌跌撞撞几步,颓然坐倒在罗汉榻上,泣不成声地还道:``璟兕!朕的璟兕\ldots{}''

赵九宵从未见过皇帝这般模样,吓得魂飞天外,半晌才回过神来,对着茂倩怒目而视,``你这女人,血口喷人!''赵九宵急得满面通红,恨不得上前扯住她,``你别胡说!别胡说!皇后娘娘心存恩泽,必有福报!她不是这样的人!''

皇帝闻言凝神,须臾,骤然冷笑,``是了!朕想起来,当年出冷宫之后,是皇后请求朕让凌云彻离开冷宫往坤宁宫守卫,之后凌云彻才有平步青云之机,来朕身边伺候。''他面色微白,颇有余悸,``想来真是后怕。朕的肱骨之侧,居然是旁人心腹!''

赵九宵又急又慌,拼命磕头道:``皇上别多心!皇后娘娘与您多年夫妻,她信得过的人才敢送到皇上身边陪伴左右!你别误会了皇后娘娘一片真心呀!''

``真心?''皇帝的笑意酸楚而悲切,``从前朕真的觉得皇后对朕一片真心,如今看来,竟是连朕自己也不懂得了。若这真心之后藏着利刃,那朕真是避无可避了。''他挥一挥手,``茂倩,今日你说的话够多了。比你伺候朕那么多年说的话都多。朕听够了,你先下去吧。朕有些话,还想再问问赵九宵。''

茂倩诺诺答应着,躬身告退。她起身离去,殿门的开合间牵动冷风如利剑般直刺过来,九宵浑身战栗着,跪伏一边。他正不知该如何应对,只见一个女子闪身进来,款步行至自己身边,跪下道:``皇上万安,贵妃小主遣奴婢来向皇上请罪。''她磕了个头,战战兢兢道,``贵妃小主敷了药睡了几个时辰,醒来叫人去给茂倩姑姑加些火盆,怕她冻着,才知茂倩姑姑一早跑来了养心殿见皇上。''

皇帝淡淡道:``不妨。令贵妃烫伤了本就不大好,茂倩趁乱跑出来找朕,她哪里顾得上。''

春婵满面惧色,愁眉苦脸道:``皇上,小主本要亲自前来向皇上请罪,奈何太医说小主伤势可轻可重,还是不动为妙。好歹算是劝住了。''

皇帝的脸色稍稍缓和,关切道:``太医瞧了,说贵妃伤得要不要紧?''

春婵忙回禀道:``皇上放心,太医说只要勤于上药,仔细照拂,也不打紧。说来也怪澜翠。''她的眼神往九宵身上一瞟,抱怨道,``澜翠也算伺候了小主多年,竟还这么不当心。奴婢出来时还见她吓得哭,这么伤着了小主,还不知该怎么罚她呢。''

皇帝嘴角一沉,没好气道:``烫了身上可大可小,是得交给慎刑司好好惩治。''

皇帝的话仿佛一阵寒气,直逼九宵身上,九宵打了个寒战,忽然想起方才宫门外候着时,进忠向着他皮笑肉不笑道:``仔细点说话,你心上人的性命,还在令贵妃手里呢。''

他本还有些糊涂,听得此节,也再明白不过了。

春婵听皇帝动怒,连忙赔笑道:``请皇上恕罪,澜翠一向手脚还勤快,怕也是一时有误,小主说看在澜翠多年伺候的分儿上,还请皇上将澜翠留给小主自己处置,别送去了慎刑司受那些零碎苦楚,也免得家丑外扬。''她恻然不忍,``到底,澜翠已经挨了三十大棍呢。''

皇帝还欲说话,想了想道:``也好。贵妃素来心慈,凡事肯留余地,不似\ldots{}''他想了想,``你去告诉贵妃,澜翠如何处置,都交由她自己决定。''

春婵恭谨领命,看了跪在地上瑟瑟发抖的赵九宵一眼,默默退下了。

殿中安静得如在无人之境,九宵一心记挂着澜翠,抬首才见皇帝静默无声,逼视着他。片刻,皇帝的声音铮然响起,``你也不必留心扯谎,这里只有朕,外头只有进忠守着。不吐出真话来,离了养心殿,你便进慎刑司吧。到时候,谁也救不得你了。''

九宵惶惑地听着,不知怎的,他挺直的脊梁骨渐渐发软,终于像被抽去了全身的骨骼,流着泪趴倒在了地上。

\hypertarget{ux7b2cux5341ux4e5dux7ae0-ux8fb1ux8eab}{%
\chapter{第十九章 辱身}\label{ux7b2cux5341ux4e5dux7ae0-ux8fb1ux8eab}}

夜已深沉,雪花敲在瓦檐上的声音扑棱扑棱的,像是谁撒着坚硬的小石子儿,一下一下惊着心肠。嬿婉并没睡好,睁着双眼拥着锦衾,静静听着风发出怪兽般阴沉的呼号,低声唤道:``春婵。''

春婵抱着膝盖靠在床边打盹,听得嬿婉召唤,忙睁开蒙昧的眼,答应道:``小主?''

嬿婉的声音在发飘,她极轻声地问:``事情真的都过去了吗?''

春婵低柔道:``进忠亲自来递过消息,赵九宵招了。虽然招得含糊其辞,可也隐隐约约透露了皇后与凌云彻有私。他除了养心殿就求进忠救澜翠,说他为了澜翠连最违心的话都说了。真是一片痴情!''春婵虽然这么说,口中却满是讥讽,``他哪里知道,小主只是拿澜翠与他做戏。进忠敷衍着答应了,说他答得模棱两可,是最好不过的,小主一定会留着澜翠不死。然后赵九宵与茂倩都被连夜带出宫外。听说茂倩出了永定门就被扔进了河沟里,不淹死也冻死了。赵九宵是流放之刑,罪名便是在坤宁宫有大不敬之举。''

嬿婉抓着枕上一把金线流苏,一双眼在漆黑的夜里闪着幽幽暗光,``皇上是不会放过茂倩的。''

春婵急道:``皇上难道不信茂倩的话才这么做?''

那金线本就生硬,硌在手心里一阵阵发凉,``皇上就是信了,才要灭口。茂倩恨毒了凌云彻,保不齐哪天就嚷嚷开来,皇上当然不能留着这个后患再生波澜。至于赵九宵,皇上还留着他,只怕哪一日还想挖出什么话来。''

春婵大松一口气,抚着心口道:``皇上疑心重,奴婢还怕皇上不信呢。''

嬿婉凝神思忖,``依着皇上的性子,想必不会全信。但人的疑心就像是无底幽洞,只消勾起一点,便会叫人如坠泥潭,越陷越深,哪怕是贮海积山也休想再填平分毫!''她缓着气息,慢慢道,``春婵,一个人但凡要布下局来,就得要多多的人来显得周全,万无一失。众口铄金自然容易积毁销骨,一旦撕开了口子,便什么都拦不住了。''

春婵担忧,``能万无一失么?''

嬿婉伸着手指,在松软的棉被上一道一道慢慢划着,指甲划过娇嫩的蚕丝有轻微的沙沙声,她在乌定定的夜里睁着眼,发出骇人的光芒,``世间事未必都周全到万无一失,但有三个字便够了。那三个字,便是`莫须有'。''

``莫须有?''

``对!莫须有,或许可能有。因为人的疑心胜过一切铁证如山。因为只要他坚信,便一切坚不可摧。但如有了疑心,疑心生暗鬼,哪怕无事也成了是非。历代以来,死在`莫须有'三字上的,还少么?''

春婵不解,``小主这么说,只消那双如意云纹的靴子便可让皇后和凌云彻说不清道不明了,何必还扯出八阿哥的事!''

``皇上最恨有人在太子之事上作祟。这些年皇上最看重永琪,眼看着一定会封为太子,若知道皇后这么多年对永琪都只是虚与委蛇,以求依傍,又为了永璂连永琪也不放过,那么皇上会作何感想?这件事便传了出去,叫永琪和皇后生分了母子之情,那本宫也净赚了!''

春婵会意,立即道:``小主放心。这件事奴婢会想办法传到五阿哥府中,再叫胡格格使劲吹吹枕头风,她会尽力的。''

嬿婉倚靠在金线攒枝花枕上,含着轻快的笑意低低道:``田嬷嬷和田俊虽然死了,但叫本宫找到了田嬷嬷与前夫生下的女儿,按着永琪的喜好悉心调教,不枉她得了永琪那么多的宠爱。''她正得意,忽地想到一事,不觉神色恻然,``对了,皇上如何处置凌云彻?''

春婵一愣,不知如何反应,只得如实回禀,``这件事皇上只交给了进忠去办,想是干系厉害,进忠一个字也不敢吐,也叫奴婢别问,怕八成是没好下场了!''

嬿婉怔住,张口欲言。一瞬间,只有一种欲落泪的心疼,催得她怆然含悲,``这件事本宫原也不想那么快闹出来,或者换个旁的法子也好。谁知豫妃深恨皇后害她失宠,硬生生忍了这么多年,只等闹出这回事来!凌云彻一旦有事,她便寻到茂倩,可见二人私下相与已深!''

春婵婉言劝道:``小主就是心软,顾惜与凌大人自幼相识之情。可是凌大人糊涂油蒙了心,不顾小主一心只为皇后。这便是自作自受了!如今豫妃既然闹了出来,良机难逢。小主少不得顺水推舟!''

嬿婉侧首哀然,``多年了为了得皇上欢心扫除异己,本宫没少利用凌云彻。可归根结底,要损他一条性命来扳倒皇后,也实在\ldots{}''

春婵见她伤怀不已,机敏接口道:``实在是天赐良机,千载难逢!小主不为别的,难道忘了夫人临死前的嘱咐么?小主无母无弟,落得孤苦地步,是谁害的!别说奴婢心狠,为了小主和阿哥的前程荣光,便是折了澜翠在宫里的安稳也没什么!''

嬿婉听她口气决断,少不得振作心气道:``也罢!难为你瞧出了赵九宵对澜翠的情意,逼迫他供出凌云彻,否则咱们再难压倒皇后。赵九宵死罪可免,活罪难逃。只是留着这个活口,再要翻供叫皇后东山再起,便不好了!''

``奴婢省得,一定会叫人在赵九宵流放途中料理干净!不留后患。''春婵稍一思索,连忙求情道,``澜翠年纪也大了,小主答应过,此事一了便会借口不用她了送她出宫。奴婢会着人送她还乡。''

嬿婉正犹豫,忽地咬了咬唇,冷道,``既然要不留后患,那么澜翠也别留着了,一并干净。本宫已经让王蟾去办了。''

春婵与澜翠一同服侍嬿婉多年,心知澜翠虽不比自己与嬿婉亲近,却也一贯得力。竟不防嬿婉说出这番话来,当真是惊心动魄。她深知嬿婉心性坚定,劝无可劝,也少不得忍泪答允了。

直到出了殿阁,春婵才觉得一阵阵后怕,天寒难忍,怎及心头寒冰。她正镇定心神,眼见王蟾进来,忙一把拉过他往角落里去,这才敢问:``澜翠到底如何了?''

王蟾袖着手,一脸惧色:``奉小主之命,送了澜翠上路了。''

春婵急道:``怎么走的?''

王蟾连连摇头,很是伤感,``一顿饭菜,都是有毒的,也算留了全尸。唉,我跟内务府报了澜翠得了绞肠痧,送去火场化了。''

春婵不禁含悲:``我与澜翠一同服侍小主多年,澜翠一贯得力。小主的心怎么这么狠了?连自己人也不放过。澜翠可是一直忠心耿耿的呀。''

王蟾紧张地抓住春婵的袖子,四周张望了无人,才放下心来:``我的好姐姐,甭管别人了。哪天一不留神,我和你就踏了澜翠的老路了。咱们呀,自求多福吧。''

春婵一想到嬿婉方才脸色,也是后怕,只得掩了口,将哭声咽了下去。

人在兴头上的时候,日子是一条光滑的绮丽的绸,顺着它滑溜溜地游荡,荡得无边无际,如在云端之上。可不如意的时候,日子就成了发霉的蒜瓣,过一天就是一瓣儿,像是被硬塞进了喉咙,辛辣、发涩、萎靡、霉烂,吞不下,吐不出,说不尽的酸涩苦辛。

这样的日子,过了三十六天。

如懿记得再清楚不过,整整三十六天。这三十六天里,皇帝没有再见过她,生活仿佛又回到了往常那种近乎决断的隔绝。隔着一条长街的两端,她与皇帝各自过着自己或绚烂或寂寞的岁月。

也没人知道凌云彻的消息。他仿佛在人间彻底蒸发,无声无息。有人说,他与茂倩和离,触怒天威,被赶出宫外。有人说,他盗取宫中宝物,与他的兄弟赵九宵一同被流放边塞。还有人说,他气不过茂倩无礼无德,一怒之下出家做了和尚。

但任凭流言纷纷,不过是一个小小侍卫的故事,闲言两句,就如抛入湖心的小石子,晕开两圈涟漪也便无声无息了。只是任凭李玉与如懿用尽法子,也得不到凌云彻半点消息。

有时候,没有消息,比最坏的消息,更让人觉得可怕。

直到,直到那一日。大雪初停,满庭冰雪映着宫墙的暗红辉泽,折出一地惨然的银白。室内虽然燃着数个炭盆,但殿内不足以因此和暖,冷津津的。窗外刮着巨风,击打着窗棂,如野马奔腾嘶鸣,驰于浩浩原野。如懿伏在案边,用浅红的笔墨画上一瓣梅花,凑成``九九消寒图'',便又算熬过了一日。自从凌云彻消失后,她的心没有一刻得到安宁。而沉寂的翊坤宫,就如大雪冰封后的紫禁城,晶莹、璀璨,却是一座华美的没有生气的死地。

所以,当太监们的靴底桀桀踏破积雪的沉硬时,栖落在廊檐下啄食的乌鸦也被惊得飞起。映着这萧然落索的天气,散落一层层破碎的哀鸣。

进忠进了暖阁,向如懿恭恭敬敬施礼问安,笑吟吟道:``皇上说,有一礼物要赐予皇后,请皇后欢喜笑纳。''

如懿连眼皮也不抬,淡淡道:``是么?''

进忠皮笑肉不笑道:``皇上口谕,赐凌云彻为翊坤宫太监。即日入侍皇后。''

没有人回应,只有幽长而乱了节拍的呼吸,在死寂的殿中闷闷响起。进忠略略定神,看见如懿平静的脸庞,宛如大雪过后的旷野,透露出死一般的震惊与痛惜。

她能清晰地听见自己的心跳,狠狠漏了一拍。几乎是喘不口气来,她真的忘记了,呼吸是何物。

直到,直到进忠唤了凌云彻进来。

许是大伤初愈,他整张面孔苍白得近乎透明,人瘦成了一杆枯竹,被两个小太监半扶半拉扯着。进忠含了谦恭的笑意,``凌云彻,还不给主子请安。''

凌云彻望着她,艰难地弯下腰去,``奴才六品太监凌云彻,给皇后娘娘请安。''

进忠浑然是教训的口吻,面上却是那种似笑非笑的神情,``从前你是伺候皇上的,如今伺候皇后娘娘。皇上与皇后体同一心,你可别生了轻慢之心,一定要好好伺候,做好奴才的本分。''

这话本无错,可如懿听着耳中,浑身如被针刺,胃中翻江倒海地恶心。

从未这般恶心过。

偏偏进忠还道:``除了凌公公,皇上还赐皇后娘娘真珠龙华十二领,甜白瓷葫芦瓶两对,玛瑙灵芝如意件一对,同心结一对,都是成双成对的好东西呢。''他又笑,``皇上还说,有些日子没见娘娘了,今晚会来与娘娘同进晚膳,请娘娘预备着。''说罢,便领了人将东西搁下,出去了。

容珮熟门熟路地将东西接下,便领了宫人退下收入库房,一并也掩上殿门,只余凌云彻与如懿二人。

相对间,唯有黯然。

她的喉间像是吞了一枚黄连,吐不出,咽不下,唯有她自己明白,那种苦涩的汁液是如何无可遏制地逼入心间,恣肆流溢。

她的舌头都在颤抖,字不成语,``我没有想到,会到这种地步。''她恍惚,``凌云彻,我们怎么会到了这地步?''

如懿蹲下身来,以一种同等的姿态,凝望着他的眼睛。她分明从他漆黑的眼底,看到了自己的哀伤与歉意,还有那种无可言说的屈辱与痛心。

``皇上的疑心,已经毁了微臣\ldots{}''他很快觉出自称上的不合宜,笨拙地改口,隐忍着巨大的屈辱,``毁了奴才,不能再毁了娘娘。''他想笑,那笑意却是惨然,``其实皇上,不算疑心错了。奴才是自作自受,若再牵连娘娘,是奴才万古难赦之罪。''

她穿着高高的花盆底,蹲在地上本就有些艰难。她双手撑在石青洒金晕锦毯上,因为过度的用力,指甲泛起暗朱色。那分明是鲜血的颜色,可是她觉得冷,无来由的彻骨的冷。殿内烧着地龙,燃着火盆,可是她感觉不到一丝暖意。仿佛有风,吹起她裙角的涟漪。可是窗门紧闭,并无漏进一丝风的可能。

凌云彻的指尖抵着她的指尖,是寒冰与寒冰的相触。他轻声说:``娘娘,你在发抖。''

呵,她居然感觉不出自己在颤抖,就像自己满心的痛,眼底却干涸得发涩,没有一滴泪。

连眼泪,都不知从何流起。

她可以听见自己的生意,枯哑、艰涩,像发锈的铁皮,``对不住。凌云彻,对不住。''

他的声音极轻,唯有她靠得这般近,才能听清那声音里的一丝战栗,``娘娘没有对不住我。这样也好,我终于可以名正言顺地陪伴在你身边,也可以结束一段痛苦的姻缘。于我,于茂倩,都是好事。''他忽然扬首,叩拜,``多谢皇后娘娘成全奴才。''

如懿沉重地摆首,``不,你不是奴才。你明明可以有更好的前程,却因为我而成为低贱的奴才。''

云彻苦笑,那笑容底下隐隐有几分平静的痛楚,``一等侍卫也好,太监也好,其实都不过是宫里的奴才,并无区别。如果皇上此举可以平息怒火,保全娘娘,那奴才甘之如饴的。''

天地间宛然有雷声震震,风卷残云疾聚疾散,悲悯与哀伤翻涌而上,不可遏止,泪水潸潸而下。她背着他,不愿让他瞧见自己的眼泪,连哽咽也沉没着吞入喉底。

可是她遏制不住,自己颤抖的双肩。

凌云彻仰起身,静静凝视如懿的身影。殿中声息全无,珠帘重重掩映,空余雪色残照。她的侧影与一枝瘦梅相似,有不胜之态。他黯然不已,``皇后娘娘是为奴才难过么?奴才低贱,不值得娘娘难过。''

``不是的,不是。''她的悲怆因为懂得而更显脆弱,``凌云彻,我在这个地方,我站在万千人中央,哪怕我笑着的,也只有你看见我眼底的一点泪光。这半生里,我的荣耀或许未曾与你同享,但每一次落魄,都是你默默扶持。''

他轻轻笑,仿佛十五月夜流泻的月光,清澈而温暖,``能如此,是奴才的福气。也多谢皇后娘娘终于肯告知,原来你只是假作不知。''

如懿的视线回避着,盯着不知名的某处,怆然道:``可是凌云彻,如今你近在身旁,我却根本不知该如何与你相处。''

``皇后娘娘不必在意。你只当奴才是你宫里的一根柱子,一个摆设,无关痛痒,不加理会,这就是最好的相处。也唯有如此,皇上才会满意。''他顿一顿,语意幽沉,``皇上要奴才入翊坤宫侍奉,不就为了如此么?夜里皇上来用晚膳,娘娘万万要记得这个。''

皇帝来得很快,日已将暮,烟霭沉沉,飞起的檐角在深红浅金的暮霞的底上渐渐变成暗色的剪影。寒冬斜阳深,星子挂在远远的天角,绽着冷冷的光,像冷峭的眉眼。

皇帝缓步进来,许多日子没来,他半点也不生疏,拣了旧日的位子坐下,便翻如懿抛在小几上常看的书。

皇帝拉过如懿的手顺势将她依在身侧,道:``怎么看起老子的书,你并不喜欢黄老之说的。过两日朕择几本好书给你瞧。''

他的话有蜜的滋味,是惯常的熟与甜,亲昵在动静间自然流泻。

如懿索性靠着他坐下,睇一眼道:``正等着皇上拣好的书来呢。对了,听说画苑送来几幅宋代王冕的梅花图,什么时候皇上带臣妾细赏?''

他温柔极了,``你若想去,什么时候都可以。''他眼睛一扫,``对了,小凌子过来,伺候得好么?''

如懿觉得自己的牙齿一阵阵发寒战冷,她的舌头抵着牙齿,逼出温声细语,``多谢皇上。小凌子是伺候过皇上的人,在皇上身边久了,再怎么不好也会好。''

皇帝的笑意无可挑剔,看她的眼神似乎很满意。他抚着她的手背,``那就好,朕今日特意让御膳房做了你素日爱吃的菜,朕陪你一起。''

言毕,李玉低眉顺眼击掌两下,外头送菜的太监便流水价上来。

荔枝腰子、持炉珍珠鸡、芝鹿双寿、菇鹤齐福、奶房玉蕊羹、蛤蜊鲫鱼、五珍脍、虾鱼汤齑、酿冬菇盒、醋浸百合,还有一个热气腾腾的猴头蘑扒鱼翅锅子。

如懿扫了一眼,便已看清。那并不是她喜欢的菜色,尤其是腰子与蛤蜊,她从不肯吃。但他的意思,再明白不过。

不喜欢的,必得喜欢。不能接受的,也一定要接受。

她的笑是烟水照花颜,雾色蒙蒙,``多谢皇上,果然是臣妾喜欢的。''

容珮命宫人们多多儿挑亮了烛火,二人对坐着,皇帝岛:``叫小凌子来伺候。''

凌云彻打了个千儿,恭恭敬敬道:``奴才给皇上请安,皇上万福金安。''

他说得字正腔圆,如流水般自然。皇帝颔首,``打发你来翊坤宫伺候,倒是合适。''他顿一顿,眼睛一瞟,``皇后爱吃荔枝腰子,你给添上。''

如懿本能地想要抗拒,可凌云彻浑然不知情,已经送到了如懿手边,她觉得乌银筷子握在手里发沉,屏息片刻,还是咬了下去。

软、滑、嫩,像咬着另一片舌头,可还是有腥气,那种令人不悦的腥臊。她极力克制着,还是忍不住蹙起了眉头。

皇帝冷然道:``皇后一向爱吃这菜,可是伺候的人不好,败了你的兴致?''

凌云彻何等乖觉,立刻俯下身叩首,``奴才有罪,奴才不懂伺候。还请皇上降罪。''

他这般配合,皇帝反倒无法发作。如懿忍着心底的酸涩,冷眼看着,徐徐道:``自己出去领罚吧。''

凌云彻步行道廊下,举起手噼噼啪啪打起耳光。他下手极重,如懿与皇帝细细嚼着,听着那耳光声脆脆的一下,又一下,重重地打着。殿中宫女太监们个个垂下了头去。

一顿晚膳,吃得索然无味,如同嚼蜡。皇帝也匆匆停箸,道:``罢了。''

凌云彻便又进来谢恩,他对自己下手极重,脸高高地肿起,``奴才多谢皇上皇后恩典。''

如懿看着他高大的身形卑躬屈膝下去,眼中不可抑制地漫上酸涩的微痛。辛辣之味亦哽上了喉头,沙沙地刺痒着。

她说不出一句话,也无话可说。

诸般喜忧,冷暖错杂,扰攘乱心。

皇帝的眼是一泊温和柔漾的水,分明又有些刺沉的意味,``皇后不必为这等下人生气。今夜朕会留在这里陪你。''

如懿得体地表现出应有的欢喜,``夜露风寒,皇上不宜出行。留在这儿,臣妾喜不自胜。''

远黛空蒙,月华流盈,自深蓝高空漫无边际地铺洒下来,勾勒出翊坤宫柔和朦胧的轮廓。

烛火幽曳不定,皇帝平卧于如懿身侧,二人并肩躺着,双目紧闭,以此来抵触见到彼此的模样。

原来真会这样厌恶,厌恶到近在身旁也不愿一见。

如懿闭着眼睛,听着沉沉的心跳声,``皇上,臣妾真是要谢凌云彻,没有他,您已经一年三个月二十四天没有走进翊坤宫了。''

皇帝说得悠而缓,轻飘得若一朵浮荡的云,``朕来看你,不好么?''

如懿一字一字道:``感激不尽,欢欣无尽。''

皇帝的声音幽幽响起,``你猜,凌云彻在听什么?''

如懿明白他想说什么,依旧闭着眼,冷然道:``他是上夜的太监,得听着寝殿里的动静。自然皇上做什么,他便听到什么。''

皇帝轻轻一嗤,像是在偷笑得意的鼠,牵得七珍锦心流苏轻轻颤着。

如懿眼珠轻轻一转,触到眼皮,有微微的疼。她问:``皇上希望凌云彻听到什么?''

``如今他听到的,也是他不能的。''

如懿的唇角泛起冷篾的笑意,``是吗?那也是皇上的恩典。且凌云彻戍守养心殿的时候,许多事他也未必不曾听见过。都是奴才,皇上如今倒肯在意了。''

皇帝的声音极平静,像暴风雨来临前平静的海面,汪蓝深沉,``从前他有七情六欲,听着或许难受。如今朕替他了了六根尘缘,他也该停了痴心妄想,得个安分。''

他以迅雷之势翻起身,伏在她身上。他的身体是热的,滚烫,像焚着一把野火,轰轰地烧,碰到的人都跟着燃烧起来,焦躁的,愤怒的,不能自已。她触到他的皮肤,凝霜似的白,这具身体,曾沉溺于各式女子的身体和肌肤,娇嫩的,柔软的,雪白的,粉腻的,如今又在她的身上。他明绸寝衣的结子不知何时已经散了,露出一痕肉,松松软软的,像一幅澄心堂纸那么软,让人生出一种欲望,若是泼墨淋漓一场,该有多痛快。

団云花纹蝉翼素帐蓬蓬地兜出一方天地,那是极好的冰纨,绣着浅紫的兰花与团团的小巧的蝶,那绣功精巧细致,非三十年功力不可得。那只淡黄与粉青二色的蝶似欲振翅飞入浅白流云间,一双双腻着蝶翅,不离不散。里头满是丝线般滑腻而交织的纠缠,丝丝缕缕,难以分隔。他不说话,也不动,一双幽黑的眼睛直直地看着如懿,锋利得好像玻璃碎片,割着肌肤生疼。她睁开眼,定定地回视他,并无退缩之意。

皇帝嗤地笑了,``你很久没有这样看着朕了。''

如懿亦轻嗤,微凉的指尖上浅粉色的凤仙花汁像少女明媚的唇,一点一点轻吻着他的脸庞,``皇上,你猜臣妾在你的眼睛里看到了什么?''

``当然是你。朕现在就看着你。''

``那臣妾在你眼里是什么样子呢?''她似乎是在梦呓,轻柔而含糊,``臣妾在你的眼里,有松弛的眼尾,微垂的嘴角。嗯,臣妾的额头不复明亮,有细细的纹。''

皇帝的手停在她的脖颈处,停得略久,有点点潮湿,是沾了晚露的花叶。他倦怠下来,慵慵道:``你一定要这样扫兴么?''他的唇角扬起来,轻轻地拍一拍她的脸,发出一点清脆的声响,``不过确实,比起新人,皇后自然是老了。''

笑影幽幽暗暗地开在她的眼角与眉梢,``是啊。臣妾多谢皇上恩宠眷顾,长日不衰。''

她忽然想起来,这灯有个名字,叫暖雪灯,簇簇火焰在温热的空气里虚弱地跳跃着,是雪后灯光映照的晕黄。她别过头,看得久了,那灯成了模糊的一团,像是烧颓了的香灰末子。

皇帝扬声道:``谁在外头?''

如懿一凛,扬起身子,``皇上要什么?''

皇帝丝毫不理会她。须臾,便有宫人答应着爬到了殿门口的窸窣声。是容珮,恭敬道:``皇上,奴婢在。''

皇帝施施然,眼底甚至有一抹晶亮笑意,``里头的水冷了,换一壶来,朕口干。''

容珮呵着手正要答应,皇帝又道:``叫小凌子。朕喝的水要几分热,小凌子清楚。''

容珮面色为难,很快响亮地答应了一声。凌云彻便在她身后四五步远,皇帝刻意大声,他自然听得清楚。肩膀有难以察觉的一丝微颤,很快平和下来,转身去拿水。冬日的水凉得快,凌云彻手脚也快,不过片刻便抱了一个白铜仙鹤嘴莲瓣茶壶进来,低眉顺眼,十足一个中年太监的温顺模样。

皇帝呵一声笑,``怎么?胡子掉完了,眉眼也温顺多了,是个当奴才的样子。''

凌云彻不卑不亢,弯下腰去,``侍卫是奴才,太监也是奴才,都是伺候皇上的。''

``是么?那朕与皇后体同一心,你就好好伺候皇后便是。''他睨一眼如懿,笑得温柔而暧昧,``今夜,皇后累了。''

凌云彻不动如山,嘴里答允着,侧身去倒茶。如懿低着头,掩在帘帐之后,拨着郁金色敷彩飞银轻容寝衣上的菡萏花苞纽子。一下,一下,洇着手汗滑腻腻的,把握不住。

凌云彻奉上茶水,皇帝泰然自若地饮了半杯,留了半杯送到如懿嘴边,叫如懿就着他的手喝了。凌云彻一直恭敬地半屈着身体,无声无息若木偶泥胎。

终于,凌云彻退下了,如懿半仰着身子,静静地望着皇帝,眼底有幽冷的光,``皇上的面子全上了么?臣妾可否做得足够?''

皇帝斜着眼睨她,``你越来越放肆了。''

如懿眸中澄定,``皇上要凌云彻净身入宫,岂不是因为心中疑根深种,认定臣妾与他有私么?如今看他非男非女,受尽折磨,皇上一定很高兴吧?''

皇帝漫不经心地抚着帐上的琉璃银鱼帐钩,``他既忠心于你\ldots{}''他瞟一眼如懿,缓缓道,``和朕,也无心于妻房家事,那么做个宦官,日夜侍奉于内,不是更好?''

如懿如何听不出他语中之意,手上一双碧玉翠色环颤得泠泠有声。但很快,这轻微的声响被如懿的笑声所湮没。

她轻轻地笑着,笑声越来越响亮,在深寂的夜里听来有悚然之意。她便这样沉醉地笑着,笑着,笑到眼泪流出来,似乎快乐得不知所以。

\hypertarget{ux7b2cux4e8cux5341ux7ae0-ux7a83ux5fc3}{%
\chapter{第二十章 窃心}\label{ux7b2cux4e8cux5341ux7ae0-ux7a83ux5fc3}}

次日清晨起来,皇帝的沉默如山,压得人喘不过气。如懿起身要替他掩上龙袍的扣,他的手轻轻一推,将她推出千山万水的远。如懿便索性收了手,温温柔柔立在一旁。皇帝一言不发,由着李玉和容珮伺候了上朝去。

如懿松了一口气,浑身都松懈了下来,靠在床栏上。容珮低低道:``娘娘昨夜没睡好吧?''

如懿只道:``拿些消炎去肿的药酒给凌云彻,再拿煮熟了的鸡蛋替他揉。''

容珮难过道:``奴婢都问过了,凌\ldots 小凌子不肯,他说只有自己肿着脸带着伤,皇上看了才能消气些。''

如懿无声地叹息,``难为他了。''

她抬着眼,凝视着帐顶一只只欲飞未飞的蝴蝶,那么美,却是死的,永远也飞不起来,只是寻一个合适的位置,被钉在那里,供人瞻仰。

这样的日子,永远也没有尽头。

皇帝坐在养心殿内,批了一沓折子,下笔渐渐狂乱无章。他气馁地丢下笔,仰面无言。

十二扇青玉罗汉屏风后群裾一闪,却是穿着缠枝银丝杏子红缎袍的嬿婉捧着一盏银耳白果羹迤逦而出,盈盈唤道:``皇上。''

她和婉的语调,配着如江南杏花烟雨的颜色,恰到好处地安抚着皇帝枯涸毛躁的心思。他抬一抬手,勉强一笑,``嬿婉,你来了。''

嬿婉袅袅婷婷立住,道:``臣妾念着天寒,叫人给各宫的常在答应们都选了鹅羽斗篷并一件狐皮锦袍。虽说是位分低,到底也是伺候皇上的人,若太寒素冻着了,叫臣妾心里怎么过得去。''

皇帝握一握她的手,``有你协理六宫,朕很放心。只是你这般厚待她们,宫里的银子怎么够?''

嬿婉抿唇一笑,嫣然百媚,``臣妾儿女众多,分例也跟着多,加之太后疼爱孩子,难免有些赏赐。其实孩儿家的用什么呢,臣妾从哪里省一抿银子,也够原上姐妹间的面子了。''

皇帝微微一笑,``你温柔贤惠,朕心甚慰。''

嬿婉退后两步,如杨柳依依,轻盈拜倒,``皇上,臣妾初掌宫中事,许多事权衡不定,怕有错漏。毕竟皇后娘娘正位中宫,一向处事果敢决断,臣妾不敢妄行。''

``果敢决断,直爽无忌?那固然是皇后的好处。''皇帝笑容忽敛,神色间甚是冷峭,``皇后并非没有她的好处,只是那好处是她本就有的,朕初见之下觉得惊艳,长久相处,那惊艳却成了棱角,划破皮肉,鲜血淋漓,实不能忍耐。''

这样美的一个女子,说起话来更让人如沐春风,``臣妾自知出身微寒,见识俗陋,不堪与皇后娘娘相较。''

皇帝仔细端详,``是。一开始的你,的确不够风雅美好,但正因如此,你今日所有的好,都是因为朕而得到。看你盛放于朕掌心,朕很欣慰。''他的笑意骤然一冷,``对了,有件事朕须得告诉你一声。凌云彻,朕打发去翊坤宫当宫监了。''

心跳骤然漏跳了一拍。那瞬间的空白里,是谁在她心上狠狠捅了一刀,刀锋全没,却全然不见血色。

明明,她是听进忠说起过这件事。当时的自己,已然觉得浑身血液逆流。可是此时此刻,再度得知,却不想仍是这般痛。

嬿婉的脑海里疾转过一个念头,情愿他死,情愿是死了,也远胜于这般活着,屈辱,低贱,受着一刀一刀的凌迟。可话到嘴边,她居然听见自己的声音纹丝不乱,``皇上容他一条性命,已是圣恩浩荡。凌云彻有生之年,必当肝脑涂地,才能报皇上的宽仁恩德。''

皇帝浓墨色的眉轩然一挑,``凌云彻到底是你同乡,与你一同长大。你毫不在意?''

嬿婉低眉顺眼,雪肤花貌在浅浅的樱色胭脂的晕染下,依然是贞静的模样。哪怕春事烂漫到难收难管,她依然是傍在身边的一株桃花,简单而温柔,临水花开。她深深败倒,谦卑而渺小的身形,却迸发出斩钉截铁的力量,``臣妾毕生唯一所挂怀之男子,天地间唯有皇上一人。便是臣妾的儿子,长大后自有自己的路要走,而臣妾是要一生一世侍奉皇上左右的。''

皇帝伸出手,紧握她细细一截皓腕,亲自扶她起身,``好了。你的心思,朕都知晓。''他的声音像被蛀了一个洞,空茫茫的,``那么嬿婉,你相信凌云彻和皇后有私么?''

嬿婉怯怯道:``臣妾不知。但臣妾想,皇上为何要将凌云彻送往翊坤宫为宫监,身体虽非男儿,心却未必改变。将凌云彻置于翊坤宫内,太过\ldots{}''她怯怯地抬眼望着皇帝,不敢再说下去
。

皇帝怔住,一瞬间眸底五味纷繁,他挥一挥手道:``朕懂了。''外头李玉道:
``皇上,容嫔小主到。''

这是宫里不成文的规矩,容嫔面前,谁都是要退避三舍的。不为别的,只为皇帝昔日对她的轰烈的爱意。

嬿婉自然识趣,连忙告退。

香见缓步进来,恍若未见嬿婉。皇帝早早站起身来,声调软了七分,``香见。''

只这一声轻柔的唤,嬿婉便知道,哪怕自己有贵妃之尊,但比起香见这个小小的嫔位,在皇帝心里的分量,不知轻到何处去了。

嬿婉掩门而出脸颊一阵发酸,心硬如铁。幸好,幸好香见不能生育,否则,自己的一辈子,是再无出头之日了。

香见打扮得素净,不饰珠翠,只以一枚无纹的青玉扁方绾起一头青丝。她静立在那里,便是铅云低垂之下一朵素白的雪花,从天空飘落,轻轻落在眼睫上,便是昏暗天空里最透亮的晶莹。

皇帝一扫倦乏之色,欣喜道:``你难得肯来养心殿。''

这么多年,香见一直未曾学会拐弯抹角的说话方式,她直截了当,``皇上不该如此对皇后娘娘。''

皇帝讶然,``你为皇后才来养心殿?''

香见淡淡笑,那笑容芳香洁净,恬然自若,``有何不可?''她敛容正色,``皇上不该疑心皇后,不该疑心皇后之余还如此不问皂白严厉处置凌侍卫,更不该将处置过的凌侍卫送进皇后宫中服侍。''

皇帝听她直言不讳,脸下的肌肤一层层烫起来,烫得他着恼,``这不是你核过问之事。皇后害你不能生养,你还为她说话,你\ldots{}''

香见盈然欠身,面无表情,``那是臣妾愿意的,皇上不肯恼臣妾,所以恼皇后罢了。''

皇帝轻声呵斥,对着她却实在凶不起来,``不要由着性子胡言乱语。皇后对你是大失分寸不辨进退。对着凌云彻却是情难自抑浑然忘我。她若明白自己的身份,就该亲自下令处死凌云彻,断了流言蜚语,也还了自己清白。''

``然后呢?''香见讥讽,``皇后的清白就该建立在牺牲一个无辜的男人身上,然后心安理得地伴随皇上身边,浑然忘却一条人命?''她春山黛眉飞扬立起,``皇上早知臣妾心中一直思念寒歧,为何从来不怒不责?皇后之罪尚不能有定论,皇上就这般怒火中烧,失了理智么?''

皇帝拂袖,``你牵挂与自己曾有婚约之人,乃是情理之中。皇后早年就嫁与朕,半道心意游荡,实不可恕!皇后乃是国母,如此行止有失,简直大伤体统!''

香见紧紧抿着唇,若有所思地细细打量着皇帝,不觉生出一缕温静的哀色与怜悯,``皇上这般恼怒,到底是为了`体统'二字,还是颜面,更抑或是因为在意皇后,视皇后为亲近,才不容他人有敬慕之心?''

皇帝背转身去,冷然决绝,``胡说!''

香见呵地轻笑,长长地叹气,``臣妾陪伴皇上之时颇多,冷眼看了良久,自为臣妾而使皇上皇后生分,难道不是因为皇上在乎皇后违背了自己的心意么?若是无关之人,严惩即可,何必两相生疏呢?皇上便是在意,所以才会介意,介意一个无关紧要之人。''

皇帝伸展手臂,将香见揽入怀中,低低道:``不要说了,香见,不要说。''

她的鬓发柔软地拂在他的面颊上,像绵绵的春草,却萧瑟到无言。他不是不知晓,怀中的女子,哪怕依偎在他怀中,她的心一直是冰雪巅的一朵雪莲,盛放或枯萎,从来与他遥遥隔绝,毫不相干。

他如此痴绝地仰望,不过是明白,无论他何等纵情,何等放任,那些立在身后的人,永远是不会离开的。

世间哀苦离散如秋草寒烟迷离,年年岁岁荣枯在他遥远的少年时代。可他一直愿意相信,哪怕世事无常,他到底有过一个忠心的琅,一个诚摯的如懿,他的妻们。

可是如今,琅已然尸骨萧寒。如懿,如懿的心,竟也会慢慢走向一个微不起眼的低贱卑微的男子么?

他沉吟良久,任凭思绪苦缠,拉扯不断。

能够确定的,唯有当年,他们风华正盛的葱茏岁月。她于漫天夭浓的粉色樱花下转过头来,朝他拈花一笑。那无边无际的粉色烂漫不知春光短纵,开得肆无忌惮,拼却一生醉颜。却经不得一夕风拂,便落英如雨,轻红委地。那时的他们,哪里懂得这个。他所有的心思,都落在初见的她身上,轻拢的发丝间,犹有一瓣粉红轻悄停留。他忍不住走近,轻声唤她,``青樱。''

往昔的温柔无声撼动,让他有一袭难以言喻的酸楚。也不过一瞬的停留,他忽然想起凌云彻的脸,那张被他狠狠挫砺过的脸,居然还有那般克制的从容。他到底是把凌云彻送到了翊坤宫的檐下。连他自己的心也模糊了,究竟是为了什么?究竟想看到些什么?

皇帝无端地腻烦起来,这个把戏,实在糟透了,无趣极了。他的心在寂寂沉坠,他不能任由他与如懿的关系走入庞大而不见天日的暗淡中去。不能。

他心意沉沉,转至坚决。他低低呢喃,似是自语,``香见,朕知道该怎么做。''

这是一场数十年都未曾见过的大雪,纷纷扬扬,碎玉片绫。连活了半辈子的老宫人都搓着手道,从未见过这样大的雪。

视野里全是白茫茫一片,无数白雪如割碎了的白锦无休无止地往下撒着,仿佛谁的热泪,落到一半就被冻住,却淌也淌不完似的。

一个白日下来,地上早积了尺厚的雪,整座紫禁城早已是银装素裹,为了驱散这令人室息的死白,一个个火红宫灯早早点燃,顺风摇曳于廊下与庭院,在漫地银白中投下一个个硕大的橘红的影,跳脱的,渺小的,带来暂时的一点温暖和安心。

凌云彻很安分,一应殿内的功夫都交予三宝照应。他只守在殿外,与如懿保持着刻意的距离,谨守着尊卑的尺度,无可挑剔。唯一要紧的功夫,是哪怕天再寒,雪再大,他都会去御花园中折来新鲜的腊梅花插在碎纹白瓷花觚中,莹黄的花瓣薄而晶透,散着一缕若有若无的清幽香气。凌云彻全然把这当作一件大事来做,一丝不苟,亦不许旁人插手。

连容珮私下里亦喟然,``凌云彻受辱之后仍能如此严谨,实在是护着娘娘。''

如懿坐在那里,打量无名指上套的镂金护甲上嵌着梅花五瓣珊瑚珠子,那是密宗所贡的红珊瑚,饱满油润,殷红如血。呵,真是如血,看得久了,那血就像是沁到了眼底,叫人心生不安。她抚摸着半旧的里外发烧的银貂手笼,迟疑着道:``容珮,你觉得这件事到这儿便完结了么?''

容珮深吸口气,瞪着眼道:``凌云彻都成了\ldots 公公,还不算完么?''

如懿摇一摇头,``本宫也不知道。''她听着硬硌硌的雪密密敲打着瓦檐的簌簌声,``对了,下那么大的雪,你记得给宫里人多添些衣裳。另外,永璂房里\ldots{}''她叹口气,``幸而永璂这几日都留在养心殿。若是他回来,见到凌云彻成了公公,本宫要如何解释呢?''

但,永璂并未再见到凌云彻。

大雪两日后终于放晴。皇帝如常往翊坤宫来,他品茗片刻,忽而目光一扫,瞥到立在正殿外的凌云彻,便向如懿道:``有件事朕得告诉你,你宫里有人手脚不大干净,得仔细査査。''

他说得慢条斯理,仿佛是一件不大要紧的事。如懿目光一烁,``皇上指谁?''

皇帝轻嗅茶香,道:``凌云彻。''

果然是他。

预料之中的祸事来得更早,如懿一颗心已然坠了下去,口气却淡,依旧低头绣着给海兰的一枚郁金色盘花籽香荷包,海蓝色的丝线绵绵不断地绣着兰萱忘忧的图纹,``什么了不得的东西,竟要皇上亲自过问?''

皇帝闲闲放下手中的脂玉夔龙茶盅,``凌云彻盜走了朕在翊坤宫中的一件至宝,即时押入慎刑司,拷问不出,不得轻饶。''他托起如懿的下巴,``这么镇定,不向朕求情?''

如懿冷冷瞥他一眼,``皇上认定他有错,旁人求情又有何用?只是臣妾不明自,皇上心怀壮思,怎会连芥子之事都不肯放过?''

``人走千里坦途都无妨,只是鞋履中的石子,若不铲除,便会伤了自己。这样的人,留在你宫里,朕也不放心。''他唤道:``来人!''

进忠响亮地答应了一声进来,``皇上,奴才在。''

皇帝淡淡道:``将翊坤宫太监凌云彻关入慎刑司细细拷问,务必说出真相为止。

如懿端坐于位上,看着众人将毫不反抗的凌云彻拖了出去。她看见他最后的眼神,那样平静,如一潭死水,平静得彻骨凄寒。

如懿缓缓道:``皇上不在乎冤枉了人么,还是觉得真与假,其实全然不重要?''

皇帝的眸子定定地看着如懿,那水波柔和的双眸里隐着刺冷的光,好似殿外素色的雪。半响,他才幽幽地轻叹一口气,``真与假,朕也很想知道。皇后,你呢?''

这个世间本没有真相。所有的真相,只在乎皇帝一念之间,连生死祸福亦是。

没有人可以由着自己,没有人可以主宰自己。

真是疯狂,所有的人都这样活着,营营役役,浑浑噩噩。真是疯狂。整个紫禁城,都是一群疯子的狂欢与哭号。

她这样想着,忽而笑出了声,清脆的,冷冽的,是冰珠落在坚石上的冷脆。

皇帝古怪地看着她,``你真是疯了。''

如懿笑了片刻,拈着银针对着光,慢慢地继续着手中的绣纹,连皇帝离开,也未起身相送。

殿中,唯有一缕梅香,幽幽动人。如懿浑然不觉,那银针何时戳进了肉里,沁出暗红的血。

殿外天寒地冻,殿内串着地龙,供着火盆。宫苑里人都不知跑哪里去了,暖阁里只有容珮蹲在地上,拿火筷子拨着火盆里烧得将熄的炭。她手势轻巧,眼看着炭火一芒一芒的红星渐渐褪成暗银色的灰烬,又翻出几点猩红的火星。

京城严寒,但从未有哪一日如今日这般冷过。雪化了又下,反反复复,一层冷意覆了另一层,将紫禁城内外冻了个透透的。窗外雪子飘得有些急。敲在冻住的瓦檐上,打出``咝咝''的微响。那声音虽轻,却乱,且汪样一片,沙沙地烦心。如懿眉目间有几分神伤,听着那纷纷落落的声音出神。

容珮拨了炭净了手,端过一碗煨好的粟子薯蓉羹奉上,``虽说天暖心冷,但娘娘也别自己泄了气。''如懿接过来尝了一口,温热的甜食让人在孤寂悲苦中稍稍有松弛的力量。可惜,她并没有胃口。

容珮也不多劝,只道:``这些日子内务府拨了不少宫里的人走,说是伺候娘娘不周,却也不说什么时候再拨人来。''她看一眼如懿,``内务府不敢这样做,多半是皇上的意思。''

如懿缓缓道:``皇上原要本宫静心,人少些也好。皇上想怎么做,由得他去。''她口气虽闲,但到底幽怨太深。容珮知道此事于如懿伤得太深,想要释然也是不能。且那日之后,凌云彻便再无消息,慎刑司里瞒得滴水不漏,谁也打听不出什么。

如懿烦乱地摆弄着窗前长几上的蜜蜡琥珀攒花盆景,如一般的嫩黄,润泽鲜妍。那还是海兰送来的,告诉她蜜蜡可以宁神静气,定痛压惊.

她的惊与痛,还算少么?再好的蜜蜡,亦不过是外物,聊作安慰。

隐隐听得软帘掀动窸窣有声,她不必猜,也知道是谁来了。

自从那日皇帝离开,嫔妃中唯一肯来看望的,也唯有海兰了。
然而对着海兰问询而关切的目光,她亦不知从何答起。

幸好,海兰亦不多问。

如懿闻声抬首,果然是海兰进来。叶心帮海兰解下杏子绿羽锻大毛斗篷,海兰便含笑迎上来,``永琪和他福晋送了好些府里制的点心来,倒比宫里的新巧些,也不那么甜,便拿来与姐姐尝尝。''

如懿心神不定,``永琪有心,时时送东西来。''

海兰欣慰,``咱们悉心教导出来的孩子,知晓进退之道,必定青出于蓝。''

如懿看她一眼,``你是觉得我这个长辈,不如晩辈懂得进退?''

海兰捡过如懿手边的那只荷包,自从凌云彻离开,如懿也无心再绣。如何继续呢?兰萱忘忧,她根本深陷忧愁,不知如何脱离。海兰低首道:
``皇上执意要处置凌云彻,姐姐若只是不闻不问,或许还不能解去皇上疑心。''

``不该是他的错,不该由他来承担。而且,皇上不会到此为止,他一定会让凌云彻死的。一定会。''

海兰的口气发沉,带着寒霜气,``死便死,与姐姐有什么相干?不过姐姐光袖手旁观还不够,要解出困局,保住无虞,最好的法子,便是由姐姐要凌云彻死。''

如懿的目光一跳,几乎控制不住自己的情绪,``我做不到。你也知道,哪怕我这样做了,也只是暂保无虞。不知道什么时候,为了什么事,皇上又要疑心!狂潮迭起,我快受不住了。''

海兰盯着她,死死抓着她的手,决绝道:``姐姐,受不住也得受。就像走不动了,爬也要继续爬下去。姐姐,咱们已经熬了这么多年,不能半途废弃,更不能为了一个不相干的男人来影响你的未来。''

如懿狂热地喊起来,她极力克制着自己的声音,仿佛如此,才能克制住满心的伤痛,``己经够了!够了!凌云彻犯了什么弥天大错,皇上要对他施以宫刑让他受奇耻大辱,还非要他的性命不可?''

``凌云彻没有错,姐姐也没有错。可只要皇上觉得你们有错,错也是错,无错也是错。但话说回来,皇上的心思其实很好猜。凌云彻对姐姐照拂,比照出他这个夫君的冷漠。凌云彻对姐姐的安慰,比照出他这个夫君的无情。无人可比,无情无义也不算明显,可有人比照,上下立见,皇上如何能忍?''海兰摇头,惋惜不已,``凌云彻,真是可怜。''

``可怜?''如懿失意地笑,``海兰,这些日子,我总梦到那些死去了的人,富察琅,高晞月,金玉妍,白蕊姫。那些和我们斗了一辈子,斗得命都没了的,也不过是些可怜人。但是,谁来可怜可怜她们,谁来可怜可怜我们呢?''

海兰分明有一丝神伤,却丝毫不肯示弱,``若说可怜,谁不可怜?谁叫我们是生在这里的人。姐姐,你若是可怜他,那么你只会比他更可怜。所以,由姐姐下令杀了凌云彻,是最好不过的。''

身体的深处,有某种不知名的痛,剧烈地磨扯着她。如懿的手一颤,推开海兰的手,冷然道:``这件事,我不会做。''她深吸一口气,``凌云彻,是一个好人。''

海兰的声音陡地尖锐,像划破苍穹的亮蓝色的电,``凌云彻是很好。姐姐若不进宫,若不是皇后,嫁得这样一个夫君,门楣虽然低些,但这一生也不枉了!但世事不可扭转,姐姐既是皇后,就得保得住自己,也牺牲得了别人!''

如懿看着她难抑的激动,忽而明白了什么,她渐渐软弱下来,低低喃喃,``海兰,什么时候我们才可以像宫外的人一样,平凡,普通,但是正常。不会在这个地方,日复一日地疯狂。''

海兰无声地哽咽,走近如懿,抚摸着她的头发。如懿的发髻上缀着碧玡瑶累珠花钿。那浓淡相宜的碧色上,雕琢着一对小巧精致的鸳鸯,交颈相缠,亲昵无俦,连那一尾尾羽毛,都清晰可见。她半拥着如懿,忽然想起哪里听来的一句诗。

合昏尚知时,鸳鸯不独宿。

她悲悯地看着怀中的如懿,心意更是定如磐石。

\hypertarget{ux7b2cux4e8cux5341ux4e00ux7ae0-ux4e91ux53bbux4e91ux65e0ux8e2a}{%
\chapter{第二十一章
云去云无踪}\label{ux7b2cux4e8cux5341ux4e00ux7ae0-ux4e91ux53bbux4e91ux65e0ux8e2a}}

莲步轻移,小心避过满地的污秽霉烂之物,强忍着恶心,避忌着狱内阴腐霉臭的气味。是多久了,没有踏足过这样阴森冷寒的下贱地儿。而每一步,都会勾起她从前并不愉悦的记忆。

好容易站定,解下宫女所披的暗紫色碎花斗篷,将宫女腰牌收入怀里,向外朗声道:``我奉小主之命前来探望,你们外头伺候就是。''

有人声远远诺诺在后,答应着殷勤道:``姑姑您自己仔细着。''

凌云彻闻声,只是斜倒在草垫上纹丝不动。那女子步履盈盈,那绢子在鼻尖轻轻扬了扬,放下手中厚棉包袱打开,露出一个红漆食盒,一屉屉卸了下来,取出一壶温好的黄酒,一碗热气腾腾的鸡丝汤面并口蘑肉片和一盘炒酸白菜。

她忍耐着不悦的气味,柔声道:``云彻哥哥,是我。''

旧日里熟悉的称呼唤起蒙昧而温柔的记忆。他心头微微一颤,很快被深切的酸楚与恨意浸染,强撑着痛楚的身体,一点一点缓缓直起身子来。

往日简单的动作对于伤后的云彻而言,无比艰难。他费了好大的力气,挣扎着坐正,望着来人,定神道:``是你?''他冷然相望,``慎刑司苦地,令贵妃娘娘尊贵,怎可踏足?''

嬿婉的颈微微曲着,在灰暗的壁上投下柔美的弧度,轻柔道:``云彻哥哥,我知道你受苦了。''她勉强微笑,``这地儿虽脏,可阿玛死后家道艰难,我又不是没见过这种境地。''

云彻的目光极淡,像是落在她面上蔼蔼薄薄的云影,无端就看得她低下了头。

嬿婉从袖中取出一个小小瓷瓶,递到他身旁,又迅疾缩回手,避免触碰到他衣下污浊的草垫,关切道:``我知道你受了重刑,这是我托王蟾去要来的。听说他们做太监的\ldots 挨了那一刀,都\ldots 都用这个药,才好得快\ldots{}''

她语气发涩,极力避免着语中对他痛处的触碰。她见云彻并不答话,也不看那瓶药,只得无话找话,``你还是这么爱干净,都到这个境地了,还换了干净衣裳。''

云彻掸了掸身上的月蓝长衫,淡漠道:``我本清洁,却被人泼了污水弄脏。你也知道的,是不是?''

嬿婉保持着温柔而恰到好处的笑容,``你的难处,谁不知道呢?只恨皇上深信不疑,才叫你受了种种罪过。''她双手捧起面条,殷切道,``我亲自下厨做的小菜,都是你从前最喜欢的,快尝一尝吧。''

云彻大量了她几眼,神色疏远,``从前喜欢的,如今未必喜欢了。只是令贵妃娘娘深夜换了宫女装束,夜行而来,不会只为我送些菜肴来吧。还是断头菜肴,临终一别,你是送我来了?''

嬿婉闻言一怔,泪盈于睫,``你倒是快人快语,不怕忌讳。''她倒了一盅黄酒,递到他唇边,云彻别过头不理,她也不在乎,一仰头自己喝了,红着眼睛道,``我探了皇上的口风,你是犯了男人最不能犯的忌讳,是必死无疑了。今儿我便冒死来送一送你。当年进的紫禁城,开头是你陪着我的。如今你走到了末路,我便来送送你,也算圆了一场情谊。''

``情谊?''他轻轻一嗤,乜斜着她道,``贵妃娘娘高高在上,我已经沦为奴才里的奴才。怎敢攀附娘娘旧日情谊,岂不玷污娘娘一世清名?''

嬿婉望着他,一滴泪在美眸里滚来滚去,险险要落下来,``云彻哥哥,临了,你还这么恨我?''

云彻笑得极恬淡,目光温煦得如四月的阳光,``我为什么要恨你?难不成是你害得我人不人鬼不鬼?''

嬿婉喉中一滞,心头一阵绞痛,愧得几乎抬不起头来。

云彻的咳嗽声在狭小潮闷的室内,听来尤为惊心。那种咳嗽,是重刑之后无力的喘动,扯出胸腔沙沙的空响与难以为继的痛楚。他强自忍痛道:``你等一等。''

嬿婉足下一滞,不知怎的便缓住了脚步,却不忍回头,去看她带伤憔悴的面庞。她有些心虚,连声线也虚浮,极力自持,``还有什么话么?''

云彻咳中有笑,``你我至此,本该无话可说。可是嬿婉,在我心里,总还记得你从前的模样。可惜,那个嬿婉,早已不在了。''

嬿婉眼中一酸,望出来的景物已蒙了一层泛白的莹光,``既知不在,何必再挽留?或者本宫便告诉你,嬿婉便是嬿婉,从来不曾变过,只是你看不明白罢了。''

云彻惋然长叹,``是啊!从前的嬿婉和如今并无二致。我所珍惜的,只是我心里的嬿婉。''一手按着胸口,一手扶着木栅,沉缓道,``有一样东西,是我送给心里的嬿婉的,你已不是她了,可否将那样东西还我?''

嬿婉心上紧紧一抽,不觉攥紧了手指,涩然道:``什么?''

一晌无言,昏暗幽闷的室内,苟延残喘的烛火下,嬿婉保养得宜的雪嫩指上,一枚红宝石粉的戒指,闪着幽暗枯涩的微光。连它也自惭形秽,仿佛配不上那水葱似的手指的柔嫩尊贵。

云彻无言,只是慢慢地摊开双手,``我此生所有,唯有此物。我当年虽然微薄,却倾尽全力相赠予我曾心爱的女子。如今物是人非,这枚戒指与她已不匹配,不如由我带走,相随黄土之下,也让我不致寂寞。''

嬿婉的泪,险险从眼眶里逼落。她仰着脸,望着霉湿的天花板,逼迫着自己,忍一忍,再忍一忍,将眼泪逼了回去。那戒指像是长在了她指上,一味发涩难以滑落。

她使劲地拔着,忍着气,忍着痛,忍着不舍,哑声道:``这枚戒指,对你那么重要么?''

他眼底有深情相许,``数十年沧桑,唯有此物不变,怎能不珍重再珍重!''

有那么一丝温情,在心底最柔软的地方轻轻蔓延。两小无猜的青涩,青梅竹马的甜蜜,都成了时光磨砺下不堪回首的过往,每一次想起,都是模糊的触痛。可只有她知道,那是怎样欢悦着滑过的日子,温柔地弹跳在她的心房。(花.霏.雪.整.理)

她不肯回头,叫他看见自己神伤的不舍,只是拼命攥着戒指,哪怕弄痛了手指,仍是狠狠地,狠狠地,像对自己撒着气一般扯落了下来,重重抛到地上,沉声道:``本宫不在乎!皇上自有好的赏给本宫!本宫要什么宝石戒指没有,便成全你了!''

凌云彻吃力地弯下腰,从霉烂的稻草堆里拾起那枚暗红戒指,含了一缕淡薄至诡的笑意,郑重行礼,``令贵妃成全,我可以无怨而死。凌云彻,在此谢过令贵妃大恩。''

他的话,终究成了一根根细碎而锐利的芒刺,生生扎进她偶尔柔软得会疼痛的心上。连她自己也不知道,在明知凌云彻会走向死亡的一刻,在她亲手推他坠落地狱万劫不复的一刻,她会这般心痛,痛得整颗心都像被放在刀锋上一寸一寸铰过。

她扶着灰颓的墙壁,仿佛再度被扯回晦涩无光的少女时代。那样窘迫的家境,家徒四壁,偏偏还有对自己可有可无的额娘。她便那样瑟缩在墙角,看着阿玛冷青色的僵硬的尸身,茫然不知前路何处。

可这一刻,她是高高在上的贵妃,获尽君王眷宠的目光,却对自己周身侵袭而来的伤心无可抵御。

甬道的风呼啦出来,透骨彻寒,她蜷缩在墙壁,回望慎刑司内一灯如豆,残焰摇曳,忍了又忍的泪,终于无声无息地汹涌而出。

嬿婉泪色潸潸,狭长的甬道内月色如霜,清冷冷地透骨刺入。她受不住似的打了个寒噤,紧了紧身上的暗紫色碎花斗篷,无声离去。

海兰携了三宝,静静望着嬿婉离去的背影,眼底闪过一丝阴鸷,冷冷道:``你可得牢牢记着,凌云彻死前,令贵妃还来看过他。''

三宝满脸愤色,用力点了点头。海兰身姿微扬,望着瓦檐积着的雪色寒霜,淡漠得没有一丝表情,``走吧。''

方行至慎刑司门前,那犯困的两个守卫见了海兰却又不识,只见她这般华贵清丽,也唬了一跳,忙强打精神点头哈腰,``您是\ldots{}''

三宝朗声道:``这是愉妃娘娘。''

那俩侍卫忙不迭请安道:``愉妃娘娘万安。您贵步怎么到这腌臜地方?''

海兰垂着眼皮,捧着手里的鎏金垂花手炉,淡淡道:``凌云彻在么?''

一侍卫赔笑道:``在!在!只今儿什么日子,刚永寿宫的宫女来瞧过他,愉妃娘娘也劳动尊驾了。''

一语未落,那侍卫脸上已经挨了一掌,三宝啐道:``你什么身份,也敢过问愉妃娘娘的事儿!''

那侍卫挨了打,拼命哈着腰,苦着脸道:``奴才不敢!奴才不敢!''

海兰眼皮微抬,金丝点翡翠甲落在手炉上玎然有声,她的声音虽轻,却字字清晰入耳,``本宫是奉皇后娘娘之命前来。牢牢记住了,不许多言。''

那侍卫哪里还敢作声,忙让着海兰进去了。

狱中潮湿,海兰扶着三宝的手步步稳当,浑不在意地上秽物。凌云彻经了方才一番,已然牵动浑身伤处,正坐在草垛上歇息。

他的呼吸微长浊重,带着濒死的气息,让人心头发酸。须臾,他觉得眼前一亮,一个翠玉紫衫的女子满头珠光华耀,立在栏外静静不语。

他微微一怔,瞬目辨了片刻,似有些不敢相信,``愉妃娘娘?''他很快淡然含笑,``愉妃娘娘甚少这般严妆丽服,夜行而来,只怕就为点眼些要人记得。''

海兰浅浅一笑,``临死还不糊涂,也不枉我为你走这一遭。''她环视四周,``令贵妃肯为了你来这污秽之地,也算是纡尊降贵,也是她对你的一份心。''

云彻支着身躯,``愉妃娘娘所言,是为皇后娘娘抱不平。明明当年与我有私的是令贵妃,到头来却污了皇后娘娘清誉。''

海兰银牙微咬,``清誉既污,哪怕不能洗去全部污言秽语,也要尽力一试,扫去大半。''她凝眸,望着凌云彻,``你懂么?''

云彻定定回望,坦然无惊,``微臣懂得。宫刑不过是皇上最初的愤怒而已,并未能宣泄殆尽。我知道的,唯有我一死,皇后娘娘才能无恙。''

海兰轻轻吐出几个字,``算你聪明。原来我关切姐姐的心,你也是一样的。''

云彻苦笑,``愉妃娘娘在皇上身边多年,深知皇上性情。这点,我与您一样。''

海兰的手轻柔一拂,怜悯道:``所以了。你也知道的,你虽然必须死,却也不能自裁。鸩酒和匕首,我都给不了你。''

云彻嘴唇微微一颤,旋即淡然,``我若自裁,便坐实了畏罪自杀的罪名。我若是畏罪,那么皇后娘娘的是非便洗脱不去了。''

海兰嘴角的笑意越来越浅,``你很聪明。所以我此番来,是奉了姐姐的旨意,要赐你加官进爵,一路好走。''

云彻的神情有一瞬的凝滞,拂袖起身,掸落月蓝长袍上的灰尘,保持着清洁而端正的面容,``凌云彻卑微之身,为皇后娘娘一死,义不容辞。只是云彻之死,并非有罪,只为洗清自身孽障,报答娘娘知遇之情。''

海兰颔首,如秋日的蜻蜓点落于水面的涟漪,``这番话,我会明明白白转告皇上。你已经受尽尊严之辱,若能一死,皇上心头的气结散去,自然不会再迁怒姐姐了。''

云彻含笑淡然,``那我死有所值。多谢愉妃娘娘成全。''

海兰的口吻极认真肃然,``你要记得,是皇后娘娘成全你。''

云彻跪拜如仪,``奴才多谢皇后娘娘恩典,甘愿受死。''

海兰扬一扬脸,示意三宝上前,``动手吧,利落些,让凌云彻走得顺顺当当。''

三宝往前走了一步,手却不肯动,有些迟疑地望着海兰,``愉妃娘娘,咱们这么做,皇后娘娘若知道了,怕是\ldots{}''

云彻原本平静的面容微微一搐,像是冻结千年的寒冰,忽然被阳光拂至,有了碎裂的痕迹,``皇后娘娘她不知道\ldots{}''

海兰上前一步,以平静得近乎死寂的目光抑制住他神色的细微变化,轻缓道:``无关紧要。你死,姐姐才会好。''

云彻垂下眼睑,微长的睫毛覆在憔悴而苍白的面颊上落下深重的阴影,他轻嘘一口气,``其实真是很惋惜,我也很害怕结束自己的性命。因为一旦死去,多年来所记得的一切便会全然化为乌有。''他仰面,仿佛承接露水的荷叶,从污浊中扬起清怡的意态,``这些日子,在身体的伤痛之中,我一直想起皇后娘娘在冷宫时落魄而绝望的容颜。所以,我再也不想娘娘回到那样困顿的境地中去。''

海兰的眼底闪过一抹不忍,温然道:``世事凄寒,你多次救助姐姐,姐姐都是记得的。''

云彻的笑颜明亮得几能照见慎刑司破落昏暗的囚房,``那真好。我在想,我没有子嗣,父母早亡,兄弟为我弃义自尽,妻室又与我离绝,不过也万幸,因此而不会牵连更多的人。这世间能记得我最多的,唯有皇后娘娘了。''

三宝愈加不忍心,几乎要落下泪来,踌躇着道:``愉妃娘娘,要不咱们想想还有没有别的法子了?''

海兰深吸一口气,有罕见的断然和决绝,没有一丝犹豫,道:``事已至此,早已没有回头路可走,更无半分回旋之地。''她抬起下颌,有冷然如冰雪的神情,不怒自威,``姐姐早就说过,我与她体同一心,姐姐的意思就是我的意思,都是一样的。''她横了三宝一眼,目光没有丝毫温度,冷冷道:``三宝,你要记着,谁是你的主子,你要为谁尽心尽力。''

三宝凝神须臾,咬了咬牙,伸手扶住凌云彻的臂膀,含了一抹泪光,恭敬道:``您请吧。''

云彻吃力地扬起唇角,``愉妃娘娘,我方才说的话,并非是想避死,而是觉得死有所值。''他无比郑重,鞠身道,``愉妃娘娘,烦请将我临死之言,告知皇后娘娘。请皇后娘娘善自珍重,否则,这世间连唯一能记得我的人都没有了。这样,我才死得其所。''

海兰的嘴唇微微发颤,她死死咬住,许久,终于咬出一个深深的血红的印子,正色道:``你这样的话若是落到皇上耳中,真是比真与姐姐有染更严重百倍。中宫的清誉怎能容你如此毁损?中宫的威仪尊贵,又如何会记得你这样的草芥之人?''她的话说得肃然,视线不自觉地避开云彻恳切而坦然的目光。她的指尖簌簌地颤动,凤仙花染就的纤纤素指泛起暗红的血滴似的摇曳。末了,她还是长叹一声,``罢了,你的话我会一字不遗地传到。毕竟,我也和你一样,只希望姐姐安好无恙。''

云彻含着感激的笑意,``多谢愉妃娘娘美意。''他慨然叹道,``云彻一生孤苦,几度离难受屈。若非皇后娘娘将我起于污泥之地,我何曾能有一日畅意?唯今一死,一偿多年相知之意。''

他闲闲道来,谈笑之间,仿佛生死亦是轻于鸿毛之事。那种脉脉的温暖与他此刻清癯衰败的面容并不相符,然而海兰心底像被什么动物的细爪子一下一下地挠着,不重,却咝咝地痛。

积蓄多年的疑惑如阴翳出岫,喷薄涌出,她知道他快死了,且必死无疑,这句话不问,只怕再也得不到答案,只会腐烂成为心底永远洗拔不清的淤积。她示意三宝等人退到门外,迫近于他,缓声道:``其实我一直想问,你对姐姐,到底是何等情意?是真心思慕姐姐\ldots{}''她犹豫片刻,``还是只把她当做魏嬿婉之后的第二人?''

他的目光清澈得能见到自己惶惑而不安的面容,``嬿婉于我,是少年时的情意,如今已不堪回首。而皇后\ldots{}''他忽然笑,``愉妃娘娘,你相信么?有些感情会自男女相悦而起,却最终超越男女之情。''

海兰的脸上有不能掩饰的畏惧与回避,``那是不是更可怕?''

云彻笑意淡淡,``我不知道,但多年以来,我深觉我所得到的欢喜,比忧惧更多。所以,此生无憾。''

海兰素来心思沉敏,此刻亦有糊涂神色,甚是不解。片刻,她沉沉摇头,``我不相信。''

云彻宽和一笑,``我知道许多人都不信,但皇后娘娘懂得,便已足够。我只盼两相安好,哪怕隔得再远,哪怕只能偶然一见,也能见她真心笑颜,我亦心安。若不能如此,哪怕失我之欢,只她安好便罢。''

海兰怔在原地,仿佛震动已极,久久痴痴不能语,似乎有万千思量,须得细细分辨。许久,她终于缓缓道:``你说的我虽不是很懂,也不是很信,我总以为,男女之间并无这样的情感,但,或许,你是真心的,也是对的。只为你这句话,还有什么未了的心事,我都会尽全力为你去办。''

云彻微微摇头,摸索着从袖中摸出一枚红宝石粉戒指摊在手心,定定道:``这是我很多年前送给嬿婉的。''

海兰颇为意外,却很快镇定,``见她戴过几次,还以为她怎么稀罕这么不值钱的东西,原来有这么一段故事。''

云彻微微颔首,难过道:``总算她还有心。''他深深望住海兰,``这个东西,算是我和嬿婉的定情信物。至于有没有用,都交于你了。''

他微微一笑,甚是恬和,``我快死了,你还活着。以后皇后娘娘的一切,便只能烦着你了。''他凝神片刻,艰难启齿,``我知道,这次的事,少不了嬿婉的嫌疑。但,请你看在这枚戒指的分儿上,且恕她一次。''他咬一咬牙,``若她往后还是心术不正,那么,我也帮不得她了。这枚戒指,还是有用处的。''

海兰的眼死死盯着墙角某处,似要钻透了墙洞。良久,她终于重重地点头,别过脸,不愿再面对凌云彻云淡风轻的脸,``我听你这一回!''说着又吩咐,``三宝!快些!别夜长梦多!''

云彻十分配合,步履艰难地走到行刑的阔长凳上。那条长凳宽四尺,长七尺,正好躺下一个人。因是用了多年,留着不少污秽的痕迹,宫中不知多少宫人便死在这长凳上。海兰瞥了一眼,无端地便有些恶心,上面那些痕迹分明是一个个垂死的人留下的挣扎,汗液,尿迹,或是被绳子勒出的血痕。云彻并不在意,他平躺其上,如同卧于高榻,从容而闲和,仿佛告别了人世间所有的繁杂痛苦,终于能得一息歇息。

三宝吩咐跟随的小太监拿拇指粗的绳索连着长凳绑住云彻的身体,愧歉地在他耳边悄声道:``对不住您了。往后奴才年年给您烧香叩头。''

云彻淡淡含笑,``动手吧。我能为皇后娘娘做的事,唯此一件,往后便要你多尽心了。''

三宝答应一声,别过头去拿袖子擦了擦眼泪,回转脸来叮嘱小太监们道:``动手吧,让凌大人走得痛快些!别磨磨蹭蹭地难受。''

小太监们利索地将黄纸盖在云彻面上,三宝含了一口清水正要往他脸上喷,恍惚有含糊的声音从云彻口中溢出,三宝忙掀开纸道:``您还有什么未了的心愿,奴才一定替您办到。''

云彻的神色极为安然,轻嗅片刻,闭目凝神,含着一缕向往的醺然笑意,轻声道:``好香!是外头的梅花开了吧?''

三宝点点头,``头先进来时,是瞧见外头的腊梅开了几朵。''

``只可惜,天寒风雪时,我不能再为皇后娘娘折下一枝梅花相送了。''云彻满足地点头,``来年若莱祭拜,只带一枝梅花就好。''他再无别言,任凭黄纸和着水黏腻地吸附上面颊。

有温热的泪凝在眼角,再忍不住,缓缓落下。再没有人比海兰更明白,那枝梅花,是谁的孤鸿之影握在指间,暗香浮动,中意了一生。

急促的呼吸声如同拍案的狂潮涌动,良久,终于没有了声息。海兰转过头去,湿透的七重黄纸,死死地覆在凌云彻的面庞上,勾勒出他五官的轮廓。只是那轮廓,如暗夜无星的天光下远处山影沉伏的姿态,再无任何回应。

他终究,如她所愿,死了。

如懿听到这个消息时,并无太多情绪的起伏,一任海兰跪在她身前,缓缓述说来龙去脉。

海兰业已说完,极尽细致,一字不漏。她跪在地下,仰头看着如懿,意料之外的平静让她有些不安,只得轻声唤:``姐姐,''她的声音大了些,``臣妾自问一心为了姐姐,没有做错。''

如懿只觉得嗓子眼里冲上一股腥甜的气味,她屏息,死死忍住那股气味的冲涌,眼神落在海兰的裙角上,她银蓝色的裙角上盛放着一朵一朵荼蘼花,那样雪白的香花,用银灰和淡白二色丝线细细绣成,开得那样簇拥,密密匝匝的,好像堆积着的燃尽了的烟灰。只是那热与烫还是在的,哪怕不见火星,仍是滚烫地抵在她的眉心眸底,让她清晰而分明地听见,自己皮肉焦糊时发出的细微的声音。

那种声音,只有她自己听得见。

她缓过一口气来,每吐出一个字,嗓子里都像是被锋利的细刃毛刺刺地割着,那样难受,居然也没有变了声调,还是那样雍容和婉,``海兰,我早说过,你做的事,和我自己做,是一样的。''

她这样静和从容,海兰反倒生出怕来。她是想好了的,什么都想到了,她的叱责,她的限泪,她的愤怒。那是应该的,是自己先自作主张,处死了一个一直对她那么好的人。可面对着如懿的平和,她居然害怕得无所适从。

海兰捧着她的手道:``姐姐,你是不是觉得我做错了?''

如懿黯然坐着,她发现自己的身体困住了一个不安分的兽。那兽在撕咬她,让她痛不可当。可是她不能动,不能哭,不能挣扎。如懿只是凄然苦笑,``你是为我好,怎会有错?凌云彻更是无错。''

海兰恍然,切切唤道:``姐姐\ldots{}''

如懿不为所动,只是沉浸在自己的思绪里,幽幽道:``一个并不重要的人,你做了,便做了吧。''

海兰脸上的忧色越来越重,惶然唤:``姐姐,你若不高兴,大可骂我,打我\ldots{}''她神色楚楚,怕到了极点,``姐姐\ldots 你别笑\ldots 你别\ldots{}''她骇到了极处,惶惑地望着如懿,急切道,``姐姐,他都死了,你便实实在在告诉我一句话,你对他,到底是怎样的情分?''

如懿抚了抚自己的脸,她的手指僵硬得仿佛不是自己的了,缓缓地触碰到肌肤时,才觉得脸上的肉是软和的,她似是自言自语,``我在笑么?我怎么不觉得?''她木然地转过脸,看着一脸急迫快要哭出来的海兰,唇边的笑意仿佛一朵风刀霜剑后凋残零落的暗红泛白的花,``海兰,这辈子,让我觉得热,觉得冷的,唯有皇上。可是在我寒冷彻骨的时候,让我觉得暖和的,是你,还有凌云彻。``

海兰的头无力地低垂下去,``姐姐,我与你多年的情分。原来在你心里,我不过和他一般。姐姐,我不知道我该高兴还是难过。他害得你清誉受损,几乎不能翻身。姐姐,他\ldots{}''

海兰看着如懿苍白如雪的容色,不敢再说下去。如懿的眸底有近似于冰封般的平静,然而海兰却如见到了惊涛骇浪一般,惶惶失色。如懿的声音极轻,``海兰,你我多年依靠,凌云彻亦是彼此扶持。无关情爱,本是相知。海兰,我原以为你会懂得。却不想,你也会这样问。''

海兰的嘴唇颤颤地抖索,仿佛深秋枝头最后一片挣扎的枯叶,她泪光潋滟的眸睁得大大的,几乎落泪潸潸,``姐姐,你要真难过,这里只有我和你,你哭出来,也没人知道。''她膝行两步上前,抱住如懿的腿,``姐姐,你别这样笑,我害怕的紧。''

如懿仿佛是在梦呓,带着迷蒙的笑色,轻轻道:``我没事,有什么可哭的。我只是倦得很。''她摆摆手,强撑着无知无觉的身体站起来,``我去歇一歇,你先回去吧。''

她起身,足下一跌,险险被地上寸许厚的锦绒密毯绊倒。她的手肘重重撞在花梨木鹤啸流云长桌上,那花梨木质地坚实,一撞之下痛不可言,却哪里抵得上海兰说的云彻的死,这般刮骨至深。

海兰尚来不及扶,如懿已然站起。她走得极缓,极缓,她湖色的裙角拂在地上,仿佛寒烟薄雾,迷蒙浮转,身后的重重珠影纱帘被她撞落,惊落重重涟漪,她完全不曾察觉,只觉得那样倦,那样倦,真要躺下来好好歇一歇。

海兰见她如此,本能地想起身追上去,然而足下一软,不免瘫倒在地。

如懿缓步走入内殿,怆然坐于床榻之上,瞥见象牙妆台的铜镜里,自已失色的容颜映在天青色散珠梅花的锦帐之上,恍若堆雪。真的很想哭,因为身体深处的隐痛,依稀是身体某处的血肉被人生生剜下,可是她看不见,分明没有任何破损,可是她却能感觉,血液汩汩流出后四肢百骸逐渐变冷的僵硬。

可是她不能哭,亦没有泪。眼底如此干涸,干涸得几乎要裂开,却没有一滴泪溢出。只能将发颤的牙关死死咬紧,咬成一如既往的平静与漠然。

也不知过了多久,她才发觉自己的指尖有温热厚腻的触感,一点一滴,渐渐蔓延。她木然垂首,才见自己的衣襟指尖之上,已有鲜红的血滴点点散落。她分辨良久,才发觉原来那鲜血来自自己的嘴唇,却不知是何时被咬破。

是,她没有泪,也不能流泪。只能流血。

没有人知道,也未必有人明白,凌云彻之于她,并非年少时炙热的爱恋。他是生长于她身侧的一棵树,枝叶茂繁,翠色苍苍。为她遮风挡雨,停靠一时。然而,如今已经没有了,只余她曝露于茫茫天地之间,一任烈日焦烤,风雪欺身,冷雨飘零。

\hypertarget{ux7b2cux4e8cux5341ux4e8cux7ae0-ux4f5bux97f3ux60caux7f20ux5fc3}{%
\chapter{第二十二章
佛音惊缠心}\label{ux7b2cux4e8cux5341ux4e8cux7ae0-ux4f5bux97f3ux60caux7f20ux5fc3}}

没有凌云彻的日子,也一样飞驰而去,不做丝毫停滞。日子静寂得与死亡没有半分区别。如懿一直试图去怀想,曾经没有凌云彻的日子,她是如何度过的。

那是许久许久以前了,久得就像一个古远的梦,让人辨不清它是否真实地存在过。潜邸的岁月里,她还年轻,和每一个青春少艾的女子并无不同,鲜红的唇,大大的眼睹,皮肤洁白得像新磨出的米浆,幼腻动人。她身边的男子,有和田美玉般的面容,寒夜星辰般的眼睛,和蓬勃清朗的五陵少年的贵质风雅。

当然,他偶尔也有郁郁,譬如朝政上的不得意,譬如诸瑛的弃世,那种阴郁是欲雨的天气,让人想拥住他,心疼他,与他甘苦与共。

她一直是这样以为的,这个男子,是她的未来,她的终身,她的生死相依。却原来,甘美时他一直都在,凄苦时浑不见踪影。

所有的艰难苦辛,只有凌云彻在身后,默然相随。

那是她的半生,半生的姻缘里,她一直在皇帝身边,却未曾注目,身后,只有凌云彻,为了她,可以不顾一切。

他的情意,如懿早知道,却无法有一点点回应。哪怕她明明,已把他的好,刻于骨,铭于心。

孤寂的日子里,她开始害怕下雨。

晴日里的紫禁城并不那么阴森,甚至还有几分富丽辉煌的格局。可是一落雨,那是另一个世界。浩浩茫茫的雨水像是永远在冲刷着墙头如血的颜色。而细雨纷纷时,整个紫禁城都像一个哀哀的鬼魂,在雨水里戚戚地茕茕而立。

真的,年轻时无知无觉,什么都不怕。如今年华渐渐衰折了,反倒生出怕来。

她没有权势煊赫的母族,没有贴心的女儿,儿子也唯独只剩了一个,已然送去了海兰那里。夫君,早已是形同没有。其实她何尝真正拥有过。曾经有的,不过是他的---点儿情意,这儿一点儿,那儿一点儿,从来没周全过。因着这样,皇后的名分也不过成了虚空,她倒成了孑然一身,孤零零一个儿。

有时想想,真是虚妄。一段执着数十年的情感,一朝跌宕断裂,竞是因着另一段情感。是他,亲自引着自己到热闹繁华锦绣族拥里来,却也是他,亲手丢开了她,遗她在孤清里。

到头来,伴随手边的,唯有那一卷墨梅,不会随时气的变化,盛开依然。

二十九年四月二十八日,久病的忻妃弃世而去。如懿与海兰守在灵床前,看着年幼的八公主穿着雪白的孝服哭得惊天动地,心下凄怆,相顾无言。那一夜,除了风声,万籁俱寂。她想起刚入宫时的忻妃,那样爱笑,如山花烂漫。最后离世的一轧,枯瘦一把,不盈一握。

不过十年,紫禁城中又添了一把红颜枯骨。她临去时没有一言,只是盯着幼小的八公主久久不肯闭上双眼。

还是如懿先明白过来,道:``你放心,本宫与愉妃会照顾好璟婳。''

忻妃艰难地点头,一缕芳魂终肯消散。

而彼时,皇帝又新纳了福常在、柏常在、武常在与宁常在,四人都是正当嘉年的少女,各擅其美,如四季开不败的花朵。总是花落花开,旧人去,新人来,从未寂寞过。而二十七年的十一月,一向擅宠的嬿婉,又生下了皇十六子。

比起后宫,前朝的气象更为明朗。二十八年五月初五,九州清晏因雷暴失火,因是深夜,殿中唯有皇帝与和亲王下棋做伴,弘昼骤见火起,吓得夺路而逃。幸得住在侧殿的永琪发觉得早,立刻背起皇帝逃出生天。

自此,储位之事,便有分晓。

乾隆三十年正月,皇帝决意再度南巡。说起此事时,是皇帝的爱女和敬公主最先知晓。彼时父女二人立于孝贤皇后画像前,哀思难绝。

画像上的孝贤皇后仍是盛年绮貌,而皇帝却是半百之人,渐渐有了老态。自与皇后疏远之后,嫔御之间皇帝亦少流连,倒是在长春宫中枯坐更久。

皇帝轻抚画像,哀叹不已:``城上斜阳画角哀,沈园非复旧池台。伤心桥下春波绿,曾是惊鸿照影来。朕前些日子读到陆游哀悼唐婉的诗,就很想念你。琅嬅,从前朕对不住你的地方不少,如今想要和你说说话,竟也不能了。''

和敬公主依偎在皇帝身边,露出几分少有的小女儿情态,依依道:``皇阿玛,您想念额娘,额娘都是知道的。''

皇帝拍拍和敬的手,``朕想着过了新年就再南巡。可每次想到你额娘在济南过世,朕便觉得济南是一座伤心之城,不肯一入。''

和敬看着皇帝的哀色,也是不忍,便劝慰道:``这两年来宫里的动静闹得这么大,京城里虽还瞒得严实,儿臣却也知道了些许,只是不好开口。皇阿玛如此怀念额娘,一半是因为再无人可与额娘比肩,另一半,也是皇额娘处事有些太不像话了。如此,皇阿玛想去南巡散散心,也是好的。''

皇帝走了两步,到榻边坐下,``皇后不大理宫中事,令贵妃也算是个能干的,容嫔固然也好\ldots 但都不能与腻额娘相比。朕环顾六宫,竟也觉得空虚得很。''

这样的话,真是伤心之语了。皇帝自尊要强,最重颜面。此刻说出这般话语,连和敬也不免伤怀。这样的繁花锦绣,热闹簇拥。每至后宫,那些娇艳如花的容颜无不笑颜奉承,皇帝心里,最眷念的却还是旧时人,旧时情。

和敬不觉湿润了眼眶,``儿臣知道,所以这些年哪怕令贵妃协理六宫得体,又连连生育,您到底也还没松了口给她皇贵妃的尊荣。''

皇帝淡淡道:``前几位皇贵妃的尊荣,都是病重了才给的。皇后位居中宫,贸然给了魏氏皇贵妃之位,也损了她的体面。且朕瞧着,这几年你和魏氏也疏远了,不复从前亲密。''

``都是皇阿玛的后妃,儿臣身为公主,本不该过从太密。从前与令娘娘来往,也是因为她对庆佑有恩。可纵使如此,也有皇阿玛嘉奖令娘娘,儿臣与她太亲近也不合规矩呀。''

皇帝微露赞许之色,``到底是孝贤皇后的女儿,处事公正,更是明理。''

和敬谦逊道:``不管皇额娘如何,皇阿玛还是顾及她的。说来令贵妃出身小家子,到底也不配做主六宫事宜。对了皇阿玛,这回南巡,皇额娘可要去?''

皇帝倒也未曾迟疑,``皇后自然要去的,留她在京中显得帝后不谐,徒惹人话柄。且皇后,年少时在江南住过,也喜欢苏杭一带。''

这话到了末尾,连和敬都听出了皇帝语底的伤感。帝后不睦已是宫中尽人皆知之事,可皇帝到底还是顾念着与皇后的少年情分。或许人到垂老,当一切行将崩散之时,才更体味出年少情怀的美好吧。

定下出巡的那日,正是凌云彻三年的祭日。不便张扬,如懿便在清晨时分,前往宝华殿悄悄上一炷香。

宝华殿乃是宫中僧人祈福之所,一应洒扫杂役皆由宫人打理。这一日新雪初霁,晨光清冷如白露。如懿也不曾知会宝华殿众法师,只携了容珮前往,静静陈香礼佛,寄托哀思。

容珮备齐了一应物事,婉声道:``皇后娘娘从前并不这般殷勤往宝华殿去。''

如懿一脸温静,``从前总以为无所畏惧,如今才知自己样样不能。人既微弱,便只能仰赖神佛。''

彼时天色微亮,半钩弯月凄凄隐没于云翳。一众僧人未曾奉诏,便也不曾预备迎接。这般无拘无束,反倒落了清闲,由着如懿独自坐于佛台之下,仰之弥高。

宝华殿中的陈设看似简朴无华,却隐隐有着考究到了极致的堂皇。殿中分列着十数盏青玉香灯,引着大卷的白檀木香,香气温润沉静,不动声色地按住了浮逸的心神。

待念过数遍经文,起身踏出殿门时,已是天色明净如一方光华玉璧。庭中积雪不盈寸,唯余一片空明。唯有来时足印清晰落于雪上,明白无误地告知她来时路是如何步步走过。

心中不免郁郁,如果这一世为人,跌跌撞撞而过,都能这般步步稳当,知道前路如何,去往何处,该有多好。

她仰起头,静静立于檐下。因是独自前来礼佛,她也打扮得格外素净,一身莲青色衣衫,用金银二色丝线挑着落梅花朵。发髻梳得简净,只用青玉莲瓣扁方绾起,零星点缀数枚点翠嵌蓝珠花,横簪一支白玉长簪而已。

彼时朝霞初露,映照着雪光灿灿,空气中隐约有腊梅的气味遥遥传来,寒雪清浅,暗香浮动。天际有深蓝色的云霭,与流火般的霞色交叠如层层薄纱,似清非清,似见非见,朦胧迤逦如硕大的凤凰的翅。

仿佛是许多年前,他们都还年轻的时候,皇帝站在葱郁的花树之下,晚霞的辽阔绮丽是无澜的波影,与他璀璨的笑容融为人世间最美好的向往。那粉色的一天一地衬得他眉眼恋恋,在那里笑着看她。他的笑容是初霁后明媚的雪光,纵使天寒地冻,亦有温暖人的力量。

可,那真的是很久很久的以往了。

久得连她亦迷惘,那是不是纯粹是年少时模糊的影像,只能凭此慰藉逐渐老去的年华。

她这样想着,轻轻叹了口气。微闻身后有窸窣之声,她很快掩饰了黯然之色,如常般雍容清冷,转身目视后方,只见一垂垂老矣的青衣僧人手执半旧的竹帚,徐缓清扫阶下落雪。如懿凝眸片刻,轻声道:``你是谁?''

那僧人微微抬眸,辨别她服色,不卑不亢行礼,``皇后娘娘。''

如懿见他须发皆白,神色安宁,便也生了几分亲近,微微颔首。

那僧人舒袖敛容,``皇后娘娘今日怎有兴驾临宝华殿,僧人不曾远迎,实在失礼。''

如懿清浅一笑,掩不住眼角悒悒的细纹与疲倦的暗青,``本无心惊扰众人,只是昨夜梦见早夭的一双儿女,清晨想到很快就要随皇上出行,便来祈求心安,也来求得一路平安。''

那僧人道:``皇上出行是不久后来日之事,但前事已过多年,皇后娘娘还是放不下亡人么?''

不知怎的,便有了倾诉的欲望。仿佛身染佛香的人,与之言语也能叫人心生平静。她徐徐道:``幼女夭折于怀中,幼子尚不得见天日便弃父母而去,日夜思之,悬于心头。''

其实,她甚少对人说及璟兕与永璟之事。一任时光潺潺流去,只将哀思静埋于心头,郁积成破碎的碎石棱角,在不经意间剌穿柔软的心肺。

那是一个母亲的永殇。

如懿见那僧人面貌苍老,不觉好奇,``从前未曾见过师父?''

那扫地僧人停了手中沙沙声,合十含笑,``皇后娘娘每一次来我都记得。第一次,仿佛是先帝雍正年间,皇后娘娘随姑母前来。那时,皇后娘娘还是闺中格格。''

如懿想了想,前尘依稀如是。只是不知不觉,自己的半生,从莽莽撞撞的青涩少女,从步步警醒的嫔御岁月,而至今日的高处不胜寒,竟也点缀了旁人半世的眼眸。她这般想着,不觉松了心弦,徐徐道:``那是数十年前的事了呢。''

那扫地僧人微笑淡淡,``我在此修习半生,记得刚入宝华殿侍奉时,乃是康熙五十年。多年来我不过是宝华殿数百诵经僧人之一,皇后娘娘自然不曾留意。''

如懿鬓边的一支羊脂白玉如意点翠长簪被冷风摇曳起细碎的海棠明珠坠,纵是金玉华贵,凌风亦不过瑟瑟不能自已。她轻声感叹道:``三朝繁华,师父尽收取底。''她停一停,含了几分犹豫,``曾读佛经,有一句读来惊心动魄。言说`爱欲于人,犹如执炬,逆风而行,必有烧手之患'。敢问师父一句,何为人世恩爱?''

那僧人含笑,``心念前因,彼此不相欺瞒,得温存相待,乃是恩爱。''

如懿听了动容,却蓄意存了挑剔之心,道:``师父是佛门中人,
也懂得人世情爱?''

那僧人颇从容,``佛祖怜悯苍生,人世情爱尽在眼中心底。不能涉入其中,却可以懂得。''他凝眉须臾,``我在宝华殿精心修习逾五十年,不过是在渺乱中求一方清净。有时冷眼旁观,只觉哪怕读通佛法万卷,亦难解心底疑惑。''

如懿扬眉轻笑,``师父也有疑惑?''

``红尘与清净不过一墙之隔,修为不足,自然有疑惑。''

``本宫愿闻其详。''

``世间事,争其能争,不争其不能争。但何谓能争?何谓不能争?而施主所问,是否也是欲争之所,那么得到恩爱,又要凭借恩爱争夺何物?纠纠缠缠,何处才是止境?''如懿一时被诘住,僧人轻敛袍袖,悠然道,``如果争来争去,争的却是虚无之象。拼上生死祸福,折尽一生欢悦,不过是镜花水月,那又是所为何来?''

宛如有九重惊雷滚滚,直贯入脑海,天地间汹涌云滚电翻,骤聚骤散。无数积郁的辛酸悲苦夹杂着重重的悲与喜翻腾而上,不可遏止。

多年来苦苦支撑,宄竟是为了什么?她的家人已经有足够的安稳,凭着孝敬宪皇后的余恩,也足以平安一世。乌拉那拉氏并无太过出色的族人,皇帝亦无心格外提拔,许以要职。她这个皇后,其实无后顾之忧,亦是无可以依凭的母族靠山。她的永璂,唯一的几子,并无永琪一般出色,来日若是可以做个富贵亲王,倒也清贵安闲。

可若她依旧挣扎在后位上,永璂年弱,资质不算出类拔萃,不过中人而已。自幼娇养,性子又偏柔弱。上有诸位成年兄长,下有得宠的幼弟,来日若真在位上,当日圣祖康熙九王夺嫡的景象,她却也是听过的,如何不叫人心惊胆寒?她是个母亲,她再了解不过的,凭着她没有母族可以倚仗的境况,永璂要站稳脚跟,实在也是千难万。

她可以保护他到什么时候?从一开始的打算,她便只希望他是富贵闲人,一生波澜无惊。

她不觉痴怔,喃喃轻语,``本宫一直以为自己可以坚持什么,可以明白自己要得到什么。可是细想想,其实本宫并不十分清醒。从前被先帝的三阿哥拒婚无路可去,是皇上暂许了本宫一个安稳。可那安稳之后,本宫真正想要的,却一直得不到。本宫想要夫妻恩情,那纵然是痴心妄想。便是想要一份不相欺不相负的信任,迁延退却,多年来亦苦苦支撑却难以得到。期盼得久了,连自己也会动摇。是否本宫想要得到的东西,在这红墙之内却根本不曾存在。既然如此,那宄竟是不是本宫错了?是本宫想在镜花水月之地求无根无存之物?''

那扫地僧手执竹帚,轻缓划过积雪的青石砖地,缓缓吟道:``一切有为法,如梦幻泡影,如露亦如电,应作如是观。''他悠悠漾漾轻叹一声,在空旷的规间徘徊无己。他半旧的袍裾静拂残雪而过,口中的念诵声渐行渐远,``不在此岸,不在彼岸,不在中流,问君身在何处?无过去心,无将来心,无现在心,还汝本来面目!''

皑皑雪中,那僧人人影渺渺,去到他该去之地。

有温热的泪水终至潸潸而落,她的本来面目,如被尘埃玷污的雪迹,早已不知清明何处。

不知过了多久,容珮携了一袭天青色竹叶纹镶金线凤尾的大毛斗蓬,那暗沉沉青色,是雨后的一丝明亮,却也不是那般灼艳,幸而容珮缠了一圈紫狐毛在领口,才增了几许华艳。只是那华艳亦是死气沉沉的,是生灵的血肉,点缀了她的清贵。容珮将斗篷披在她肩头,轻声关切:``天寒,皇后娘娘要保重自身。''

如懿痴立几许。

容珮低声道:``这几夜娘娘睡得并不好。夜来幽梦辗转,含糊提起旧事。''

不必容珮说,如懿也记得那些梦境。梦里都是小儿女情态,她胭脂初嫁时,初入宫闱如履薄冰时,甫离冷宫缓步走向他时,还有,还有,他要她站到自己身旁之时。那些话,她都清晰地记得。

他总是说:``你放心。''

可是这一生,她何曾放心过?不过是放掉了自己的心,再也回不来了。

梦里旧事如烟绮,醒来才更觉现实的坚冷,避无可避。

容珮迟疑着道:``娘娘还惦着皇上当时说的话么?为什么人说过的话总是那么容易改变?九五之尊不应该是一言九鼎么?''

那是容珮的困惑,或许也是天下女心的困惑吧?

如懿惘然地想,冰雪琉璃让她的心境无比清明,``不。或许每个人,当时所说的话都是真心的。但是却忘了,心意本来就是很容易改变的。彼时的话只是彼时的心境,若念念不忘信到往后,原是我轻信的过错。''

时光迁延二月余,御驾于三十年闰二月抵杭州。艳羡江南,乘兴南游,于一位帝国的国君而言,并非难事。何况天下和靖,百业兴盛,是最富烧风流的年代。从辽阔的白山黑水、塞北风烟,到晴雨江南、明好云贵,他可蠲赋恩赏,观民察吏,亦可眺览山川之佳秀,民物之丰美,一览煌煌天朝下他所拥有的万里江山。

初到杭州的那一日,下着丝丝寒雨。江南二月已见薄薄春色,只是雨气湿冷胶着,远不如京中的风物干燥。可是立于龙舟之首,望着两岸冒雨跪伏的官员肃然无声,迎面是湿润的清风,足下是蜿蜒的碧水,天地间那样的温柔,仿佛回到第一次来杭州的时光。

杭州于嬿婉是福地,于庆妃亦是。而皇帝此次除了陪伴太后,更携上了至爱的容嫔香见,一定要与她同来领略山水烟柔之美。

待得住行宫驻跸,皇帝便迫不及待往山水间去。行宫一带本近西湖与孤山,又因多梅花,孤山又名梅屿,乃是宋代林和靖隐居之所。皇帝见如懿一贯冷清,恰逢着那日她生辰,便道:``孤山赏梅甚好,有湘英、绿萼等,花色不一,是你所喜欢的。''

如懿颌首,正要应承,皇帝又摇头,``可惜了,叫孤山,名字听着不祥。''

皇帝最爱风雅,如懿便道:``不若皇上改个名儿也罢。''

皇帝仔细思忖,却又不喜,``康熙爷来此也未改名,朕也不便改了。''

皇帝最爱风雅,如懿便道:``不若皇上改个名儿也罢。''

皇帝仔细思忖,却又不喜,``康熙爷来此也未改名,朕也不便改了。''

于是敛衣而行,往``西湖十八景''去。雍正年间李卫修缮西湖一带,景致尤美,湖山春社、功德崇坊有沙堤平坦,垂杨披拂,湖波荡樣,晓雾迷离。万绿丛中,丹宫碧殿掩映林表。玉带晴虹、海霞西爽则回廊绕水,朱栏倒影,金碧澄鲜。桥畔花柳夹映,晴光照灼。梅林归鹤、鱼沼秋蓉则环池植木芙蓉,花时烂若锦绣。莲池松舍、宝石凤亭、亭湾骑射、玉泉鱼跃、凤岭松涛、湖心平眺、韬光观海、西溪探梅各有趣致。吴山大观、天竺香市可见民间欢愉,云栖梵径便闻朝鱼暮鼓,与天籁相应答,至
此豁然心开,万虑顿释。

而如懿最爱的,便是蕉石鸣琴一带,黛色波光,湖渌远映,恍然若乘槎于迢迢天汉。舫前奇石林立,状类阔叶芭蕉,题曰``蕉石山房''。石根处又有天然一池,泉从石罅出,泠泠作声,演清漾碧。临池复置小轩,古雅静洁。若以焦尾琴作《梅花三弄》曲,古音疏越,响入秋云,高山流水,得天然意蕴。

皇帝也颇属意,便向如懿道:``朕住的地方原离这儿近,你若来此月夜弹琴,倒是甚好。''然而,他不过一语,但见如懿沉吟未应,眼底闪过一丝阴翳,冷冷道,``不弹也罢,免得弹起李商隐的《春雨》,无端惹翻旧情。''
烟柳画桥、风帘翠幕的风流,市列珠玑,户盈罗绮的繁华,都未能让他忘却那一段旧事。

嬿婉见皇帝陡生不悦,便婉转劝道:``素来也只是流言,皇上实在不必往心里去。何况,人都不在了,皇后娘娘听了,心里也不好受啊。''

皇帝心意惘然,盯着如懿,目光如锥,``是么?朕还以为人没了,情总还在。''

宫人们举着罗伞,捧着栉巾、痰盂立在远处,虽然只有嬿婉和香见在侧,如懿也受不了这无端而来的羞辱。人已逝去,有时她亦想忘怀,却禁不得皇帝这般三言两语地计较,更生凉薄。

天日正中,暖暖晴光洒落在人周身,犹带一丝温暖余情。香见难得地穿了一袭粉黛色长衫,密密绣了连绵不尽的枣花图样。那是杭绸中新制的一种皎月编,一共才得了两匹,皇帝一匹奉与太后,一匹独赏了香见,供她裁制新衣。那皎月绸不啻寸缕寸金,清雅柔软,若新生儿肌理幼滑。一抹帛光盈然于举手投足间,便已觉清贵宠妃气咄咄逼人。

她站在二月漫天的花事盛开下,轻飘飘道:``前日陪皇上往上天竺焚香顶礼以祝丰年,心里念着当日寒部亡者可得安息,寒歧一缕战魂,也可长眠沙场了吧。''她举眸,若寒星熠熠,``臣妾这般心思,皇上可会责怪?''

皇帝微怔,旋即含笑,无限宠溺怜惜,``只要你高兴,什么都好。''

香见抿嘴一笑,轻诮道:``是么?皇上连臣妾为寒歧祝祷都可原谅,一个莫须有的凌云彻,皇上这几年眉间心上,就这般小气么?''

皇帝无言,如懿不动声色,只是唇角微挑,以表对香见解围的谢意。

嬿婉不胜惶惑,低柔道:``容嫔妹妹,话可不是这般说。你与寒歧毕竟有婚约在前,可皇后娘娘和凌云彻不过是尊卑之分。难道妹妹心里,觉得皇后娘娘与凌云彻便如你与寒歧这般么?''她修长玉指按在心口,连连摇头,``这话姐姐我可不敢听。''

有不敢听,亦有不忍言。明明事关自己,她却无可分辩。才知疑心深种如情根深种,一般难以移除。

她亦没有力气,拔去他心底那根刺。因为那刺,是一条活生生的性命铸成,早已成了她心底不可磨灭的烙印。

初春的风如同绵软的女儿家的手掌,轻轻拂过她的面颊。她听见香见鄙夷的声音,``令贵妃这般善于曲解,也算奇才。''她不必看,也猜得到嬿婉一定是一副娇柔怯弱不敢与之相争的模样。她也懒得去看,免得污了自己的眼睛。

如懿眉目清冷,淡淡道:``原来皇上这般在意臣妾,真是臣妾无上福泽。''

皇帝便横目去瞧嬿婉,``不该你开口之事,无须多言。''

香见便引了她的手,自顾自道:``前面花开得好,皇后娘娘,咱们去瞧。''

步子尚未迈开,已有太监来请,``请皇上旨意,晚膳摆在何处?奴才得预备起来。''

皇帝兴味索然,``晚膳在偏殿便是。扬州府送来的歌伎在何处?朕需佐以歌舞娱情。''

这般吩咐,便是不欲嫔妃侍奉在侧了。如懿便与嬿婉、香见告辞退却。

\hypertarget{ux7b2cux4e8cux5341ux4e09ux7ae0-ux82b1ux4e8bux8273}{%
\chapter{第二十三章
花事艳}\label{ux7b2cux4e8cux5341ux4e09ux7ae0-ux82b1ux4e8bux8273}}

虽然同行的嫔妃不少,又有香见这般得宠的,可皇帝的眼映入了江南的春意如许,亦觉新鲜,所以长夜歌舞,偶尔才宿于嫔妃阁中。

皇帝早先曾在淮扬的清江浦得到一双绝艳女伶,原是评弹的女先儿,名叫昭柔。昭柔弹亦佳,唱亦佳,一口软绵绵的吴侬软语。与她师姐上手持三弦,下手抱琵琶,用吴音评得一口好《隋唐》,抑扬顿挫,轻清柔缓,弦琶琮铮,十分悦耳。尤其昭柔才二十出头的好年华,身段风骚,双眸妩媚,端的是一个尤物,与苏州的甜糯点心一般黏住了白牙哪里肯松口。两日评书下来,皇帝如何还舍得她离开,得空回行宫便带在身边,说完了《陏唐》,还有《描金凤》《白蛇传》《玉靖艇》和《珍珠塔》,一本又一本,唱得山光水影,如痴如醉。

或许皇帝,的确需要新鲜的活泼的安慰。

南巡时过济南城,城池依旧,惊鸿不再。皇帝触景生情,难免想起昔日孝贤皇后仙逝于济南,不觉挥泪黯然,写下一诗,``济南四度不入城,恐防一入百悲生。春三月昔分偏剧,十七年过恨未平。''

随行南巡的和敬公主见到此诗,亦不觉动情,哭泣良久。倒是太后来安慰了几句,``皇帝是个多情的性子。但一个人的情分就那么多,都分了点子去,难免就薄了。和敬,你额娘样样都好,如今的皇后就难免难堪。你是皇帝的长女,自然也盼望圣心和睦,是么?''

太后为和睦,已然这样劝慰。可也挡不住此诗流传,人人回忆皇帝与孝贤皇后的恩情。

当如懿看到这首诗时,已经没有太多的痛楚。因为当日的疑心和疏远,孝贤皇后抱屈而死。所以皇帝用他的后半生来追忆和悼念,寄托他的哀思与悔恨。

有时候想想,如懿竟会心生羡慕。原来天人永隔也是善事,可以泯去所有仇怨,得一息宽厚温存。反正也无非是如此,人人跟随皇帝的心意称颂孝贤皇后的德行,她这个失宠的皇后,更显鄙薄而已。

然而香见好奇不已,``皇上为孝贤皇后写了那么多哀悼诗文,他或许真的很喜欢孝贤皇后吧。''

如懿不知从何答起,便道:``皇上更喜欢你。''

香见绞着手里的绢子,百无聊赖道:``我算是看得通透。皇上的喜欢便宜得很,今日来了明日去,给了这个给那个。人人都喜欢,个个都不心疼,不过如此而己。说来我更是好奇,既然皇上这么喜爱孝贤皇后,怎么做到一壁追思,一壁又唤了歌女舞姬,寻欢作乐呢?''

香见所言,乃是地方官员有伺机取巧者,沿途至一行宫,便献上当地歌女舞姬奉与艳姿。皇帝神色本淡淡的,但见送来女子皆是纤丽翘楚,个个娇小玲珑,姿态柔弱,我见犹怜,远别于北地胭脂的修长身段。而那种柔弱却又熟媚之致,一颦一笑,皆是风情,也不免心动。及至杭州,官员们又想了新奇之术,命人驾御舟泛于西湖之上,歌伎舞姬齐集舟上,既清僻无人惊扰,更可自由无拘。

皇帝醉后不免笑言,``个个如白玉扇坠儿一般,叫人爱不释手。''

这话旁人听见尚作笑言,李玉身为大总管,却不得不存了心思,``若是皇上真有恩幸,遗珠民间,这可如何是好?到底是汉女,又出身低下,若真有此事,只怕皇上的圣誉\ldots{}''他捶胸顿足,``都怪那些官员不知廉耻,为博皇上欢心,连礼义廉耻都不要了。''

如懿亦有耳闻,山外青山楼外楼,西湖歌舞几时休,暖风熏得游人醉,却不知游人心寄何处,是聪明换糊涂。

这样的事,若传出行宫,只怕为臣下百姓所耻笑,她能做的,只是将余怒狠狠压下,再竭尽全力,为他的名声遮掩。

那边厢进忠亦悄悄告知了嬿婉,嬿婉倚在窗下绣榻上,看着架上织造府新贡的各色杭绸绫罗,那些光艳的锦缎如春日濯濯下泛着缠绵亮烈的鲜彩波澜。她慵慵笑道:``繁花似锦,才不会有专宠之虞。皇上既然喜欢,本宫又何必去碰这子?''

进忠担忧道:``小主不怕那些低贱女子夺宠,说来您协理六宫,这些话小主不劝皇上,怕旁人劝了也是无用。''

嬿婉轻轻一嗤,取了一枚蜜渍樱桃放在口中,雪白贝齿一咬,一点鲜红的汁子溅在进忠脸上。进忠涎着脸笑,也舍不得擦。嬿婉啐了他一口,正了正发髻上一枚九转碧玉赤金瓒凤步摇,精巧繁复,金翠灿烂,凤口里衔出几缕细小的流苏穗,红缨珠络缀着嫣红珊瑚细细垂在耳边,沙沙地摩挲着她保养嫩腻的脸颊。她坐起身,莞尔笑道:``进忠,不在其位,不谋其政。本宫只是协理六宫,你也只是御前的副总管。有些事,何必咱们操心,自有人顶着,咱们安享清闲就好。''

进忠眨巴着眼睛听着,犹有不放心之处,``小主说得是。只是太后娘娘如今实在是不理事儿,皇后娘娘也不过是个木呆儿,立在那里好看罢了。能说得上话做得了主的也只有您一个。''

嬿婉将绢子丢到进忠手里,示意他擦去面上的樱桃汁子,那指甲染成粉红色的春葱玉指戳在他额上,``你在皇上跟前多年,这般得宠,是因为比你师父李玉能干么?不过是嘴甜心思活络,懂得讨皇上喜欢。本宫也是如此,侍奉皇上多年,仅仅膝下儿女成群便是了{[}花-霏-
雪-
整-理{]}么?当日的金玉妍何尝不是连生四子。要紧的是讨皇上喜欢。这几年皇上和皇后娘娘怄气,本宫事事顺着皇上的心意,才能到了如今。便是皇上真要收了这些歌舞美姬,本宫也只有赞成没有反对的。''她低眉见进忠只为自己担心,略含了几分矜持的得意,``你不必担心本宫斗不过这起子贱人,本宫也不屑和她们斗。即便没有她们,皇上也常有新宠,哪一个不比那些蹄子出身高贵。若是她们真进了宫,宫里乌泱泱的嫔妃不一个个乌眼鸡似的盯着她们,哪里还需要本宫动手?''

进忠这才落定了心意,满脸堆笑应承着。嬿婉又问:``上回跟着过来的女先儿昭柔,这几日怎不曾见?''

进忠舔着舌头低笑道:``就是会唱评弹,还会什么新鲜招儿?皇上听得腻味了,叫人好生送回了扬州。''

嬿婉似信非信,``真的丢到九宵云外去了?''

进忠不敢隐瞒,``是命人用金宝嵌饰的锦幰钿车送回扬州,还赐予她一对玉如意、金瓶和绿玉簪,甚为厚待。''

嬿婉长舒一口气,``只要皇上最近腻味了,便是赏赐丰厚些,也当是这些日上取乐的花销了。''

进忠踌躇着道:``是,是。昭柔虽然去了,可知府新荐了一位姑娘来,叫作水沐萍的,皇上喜欢得紧。''

嬿婉春山暗蹙,轻鄙道:``这个又是什么来历?不会又是评弹的女先儿吧?''进忠搓着手,不知该怎么说,嬿婉蹙眉,``有什么不可说的,左右离了宫里,皇上是没什么忌讳的了。''

进忠只得道:``是个歌伎,秦楼楚馆里第一把好嗓子,最会唱俗语俚曲。知府说皇上要了解民情,最合宜听这些,所以两日前送了来。''

嬿婉一惊,死死按捺住了,问:``皇后可知道了?''

进忠思付着道:``师父和我、进保都知道了。想必皇后娘娘也会知道。在行宫里出入,哪里瞒得住。为了前头昭柔的事,皇后娘娘已经严禁底下的奴才多口了。''

嬿婉愁肠百结,道:``你先回去,仔细留意着。''进忠答允着,恭谨退下了。

次日起来,依旧是在``蕉石鸣琴''用早膳。待到众妃齐坐,皇帝却久久未来。皇帝一向重视规矩,少有这般晚起的。

如懿缓缓目视在座的嬿婉、庆妃、颖妃与香见,众人皆是面面相觑,其余诸位贵人、常在更是茫然无措。

颖妃最快人快语,``皇后娘娘别瞧臣妾,这些日子臣妾若不是随着姐妹们一块儿,怕也见不到皇上。''

香见冷冷不言,嬿婉赔笑道:``皇后娘娘'臣妾也不知。''

如懿思忖片刻,安之若素,``那就再等。''

―直等到宝鼎香烟冷,皇帝才到了。众人饿得金星四起,少不得松了一口气起身请安。才一抬头如懿便怔住了,皇帝双目微红,眼下发青,面色无华,神色倦怠,显是一夜不得好眠。

皇帝许了众人落座,如懿已然猜到几分,奉上一碗新煨好的九丝汤,道:``这是皇上喜欢的扬州九丝汤。这边的厨子学着用干丝外加火腿丝、笋丝、银鱼丝、木耳丝、口蘑丝、鸡丝烹调而成,又加了竹蛏调味,以增鲜香。皇上先尝尝,以解饥冷疲倦。''

皇帝呷了几口,颇有滋味,脸色缓和许多,众妃才依次动筷。

这一膳用得沉闷。皇帝的疲倦写在脸上,众人也不敢多问,唯如懿不动声色道:``行宫临近西湖,水声带着丝竹弦乐,怕是扰了皇上清梦吧。臣妾今日便请令贵妃一同细査,何处乐声惊扰皇上,一并去了才好。''

嬿婉---惊,忙向如懿使眼色。如懿浑然不觉,只转头对香见道:``上回你跳得胡旋舞极好,回宫后也指点下含中舞姬,可好?''

皇帝有几分尴尬,打了个呵欠,掩饰道:``朕久不来杭州,夜游西湖倦了。御舟上难免有歌舞雅兴,皇后不必计较。''

如懿取了银匙,缓缓搅着盏中的杏仁牛乳,``皇上说得是。旣是这般好歌乐,臣妾与诸位妹妹愿一同观赏,还请皇上不吝恩赐。''

皇帝咳嗽几声,笑道:``皇后的建议不错。若是有月明风清之日,一定邀人同赏。''皇帝说着,草草用了些东西,便回自己殿阁去。
如此,众人也便散了。

如懿向太后请安后,便回到自己的青梧阁中。太后年迈,不耐久游,一直在自己的绛华馆中歇息,也不大出来与众人一同用膳,自享清静。

如懿回到殿中,便有悒悒之色。容珮笑着奉上龙井来,道:``地道的龙井,在杭州喝才最得宜,皇后娘娘细尝尝。''她见如懿眉目怏怏,便道,``娘娘是怎么了?''

如懿勉强振作心绪,道:``我们出来那一日正是凌云彻死祭,他离世三年,唯有本宫与江与彬、惢心、李玉才敢偷偷祭祀。今年本宫与你出宫仓促,只得提前一晚为他焚香祭告。希望他在天有灵,可以原谅本宫的粗率。''

容珮黯然悲伤,``凌大人是有担当的人,可我们能为他做的,也只有这些。''她努力笑了笑,``若是凌大人有知,明白娘娘对他的哀思,也会欣慰。''

二人正言语,却是李玉带着人来,手中各捧了一个食盒。如懿一一瞧去,都是江南名点:千层油糕、双麻酥饼、翡翠烧卖、野鸭菜包、蟹黄蒸饺、鸡丝卷、四喜汤团。

容珮诧异,摸着鬓边的烧蓝串玛瑙珠花,道:``这个时候既非午膳也非晚膳,怎么送了点心来?''

李玉道:``皇上说了,这几日皇后娘娘出游辛苦,便找些地方点心来请娘娘品尝,以慰辛劳。''

说罢,一行人放下东西,便出去了。

容珮细细看了一遍,为难道:``不是甜的就是咸的,都是好吃又黏牙的东西。这么多可怎么吃得完呢?''

如懿苦笑道:``你还不明白么?皇上在原不在吃东西的时候送来这些,只是为了提点本宫,紧紧堵着自己的嘴,不必多言。''

容珮心头一紧,试探着道:``皇后娘娘问了昨夜笙歌之事?''

``你也听见了,那些隐隐传来的词调唱的是什么淫词艳曲?令贵妃昔日以昆曲博得宠幸,好歹那是雅乐。可皇上如今取乐的,都是什么?也太不知保重了。''

容珮只得婉转劝道:``只要皇上不是过分,皇后娘娘就睁一眼闭一眼吧。日子难熬,可不都是这样熬。''

如懿睨她一眼,酸湿道:``容珮,你从不说这样的话。''

容珮想了想,认真点头,``是。说这样的话,于奴婢是违心,也是真心。奴婢真心希望娘娘好,不愿娘娘再受苦。''

如懿握一握她的手,``海兰留在宫里主持事宜,容珮,也唯有你真心待本宫。''

容珮笑道:``奴婢这条命是皇后娘娘捡回来的,自当一切为了娘娘。''

如此一日,也到了夜间时分。皇帝依旧没有翻牌子召嫔妃侍寝。这便意味着,泛舟上的艳事,会照旧而起。

彼时如懿正卸晚妆,容珮取过白玉梳掠鬓,一一替她卸去发上沉甸甸的金嵌宝插梳、点翠云纹簪、金蔓枝攒心紫莹玉珠花、掐金象牙骨扇钗,最后是一支温腻厚润的白玉凤凰,尾羽上垂落一串串青玉碎和红宝石粒子。然后将她绾好的一头青丝放下,用梳子蘸了茉莉花和桑叶煮的花水蓖得清清爽爽。

派去打探的三宝悄悄进来,立在帘下。如懿一眼瞥见,问:``还是昨夜的水沐萍么?''

三宝的影子晃悠悠的,显然是有些慌乱。如懿起疑,平静道:``你说就是。''

三宝素知如懿心性,只得道:``是。水沐萍在御舟上,还有,还有她的六个姐妹。''

如懿的声线因着惊怒而战栗,``姐妹?''

``是。''三宝擦着额头汗水,``水沐萍出身秦楼楚馆,虽说是卖艺不卖身,但到底是烟花女子。她的姐妹,自然也是烟花之地来的。''

如懿回眸,见到容珮错愕得难以置信的神情,想来自己也是如此。心口沉沉地跳跃着,她听见自己的声音在寂寂夜里格外清晰而分明,``备船。本宫要上御舟。''

南地吹来的夜风凜凜,夹着湖上水汽,清冽而洁净,扶起了如懿的裙裾。傅恒带着侍卫过来,目送着如懿上了小舟,竟也不发一语,只是遥遥观望。到底是他身边的侍卫沉不住气,问道:``大人,前头仿佛是皇后娘娘上了船,不会要找皇上吧。这御舟上有\ldots 这可要坏事了。''

傅恒沉思片刻,断然道:``咱们要防备的是刺客,又不是皇后娘娘。再说了,皇后娘娘找皇上,也是天经地义的。不必咱们理会,往后更不许提及这些秘事。''

侍卫们唯唯诺诺,只得缄口不言。

三宝与容珮一脸惴惴相随,并不敢相劝。如懿抬起头,望着十八的月瓣。偶有轻风吹皱水上月华的倒影,涟漪澜澜。远处山如眉峰聚,在舟行的荡漾中拖曳开一道道触目惊心的墨色长影。

湖上静悄悄的,凉风习习拂面,隐约传来初开的花香。那是不知名的花气,浓郁而芬芳,几欲醉去。湖上传来的女子的歌声柔婉清亮,越来越清晰,引着她遂渐靠近御舟。近舟旁是一大株粉色的蘸水桃花,一半开在水上,一半开在水里,在夜风中袅袅摇动,偶有落花曳下,一点两点,随流水飘零。

如懿的猝然到来,让守御舟的侍卫碎不及防,却也不敢阻栏,眼挣睁看她下了小船上了御舟,连李玉与进保也不敢劝阻。李玉担忧地望了如懿一眼,轻轻摇头。

如懿知道,李玉是在劝她。可是,来不及了。从她成为他妻子的那一刻,他的荣辱便与她紧紧相共。

方行至船阁中,浓郁的脂粉香气便扑面袭来。如懿从外面进来,觉得那和暖浓腻的香风如拳头一般兜头兜脸砸在脸上,击得她头晕眼花,半晌才定睛看清了眼前的景象。朱颜绿鬓,粉面含春,二八丽姝,窈窕绰约,宛如一片片彩云依在皇帝身边,不,彩云都露出了雪白轻绵的香肌,盈满御舟。其中一个偎着皇帝,指着肩头衣衫上一蓝云团龙纹,调笑道:``皇上是天子,经您圣手触摸,妾身铭感五内,特意在上衣肩头绣上一条小团龙,以志皇上恩宠。''

还有歌女咿咿呀呀地唱着香艳曲调,惹得众人前仰后合,咬着丝绢哧哧地笑。如懿静静地掀起帘子观望,脑中翻腾着嘈杂的音调,宛如针刺一般。想着那最美的一个,大概便是水沐萍。的确是很美的女子,不似宫中女子的矜持,一个个可远观可亵玩,世俗得无比亲切。像章台绿柳,可以随意攀折。

不知是哪把娇媚女声``呀''地唤起,引着众人发觉了如懿的到来,齐齐望向了她。

如懿的声音如船檐下悬着的小小金铃,是凌冽的清脆,``夜已深,皇上倦了。你们先行退下吧。''

众女燕燕莺莺之声戛然而止,毫无顾忌地打量着她,欲从服色妆容揣测她的身份。

最初的尴尬已然消散,皇帝并无中止兴致的意味,坐直了身体笑吟吟道:``皇后夜来雅兴,陪朕同乐吧。''

如懿觉得肌肤上起了一粒粒的小粟子,恶心不已。她保持面容的平静,``臣妾深觉夜来劳碌,想起皇上还为民间之事烦忧,所以特来请皇上回寝殿安置。''

船阁中灯火皎胶耀耀,将这舱内的一人一物都映得清白分明,无处可躲。有女子敞着肩头,目色轻佻,望着她似笑非笑,似乎等着看一场好戏,未有一人肯动身。便有一小巧艳妩的女子衔着艳红丝绢一角,偏着头,晃得雪白耳垂上两枚翠玉嵌红宝石叶子耳坠滴答晃悠,``皇后娘娘这样子,像不像咱们阁子里来捉拿官人的大妇。除了凶悍,别无用处!''

另---个搭在她肩上,柔柔道:``可别这么说,人家是皇后娘娘呢。''

艳女们咬着耳垂笑得暖昧,皇帝饶有趣味地听着,并无阻止之意。心头便有怒气,如翻腾若奔,如懿强忍着烦恶,徐徐环视,侧身让出门口,冷淡道:``请吧。''

皇帝大为扫兴,又发作不得,只得挥手道:``皇后命你们回去,便回去吧。''

为首的靓丽女子福身告退,``那妾身等明日再来。''说罢,一个妩媚眼神抛去,便是如懿也心旌动摇,险险不能自持。

有女子擦肩而过,随手折下湖色冰纹瓶中一朵晕紫含笑簪在发间。那花朵只在野外开放,芳香幽幽,也不知是谁寻了来插瓶。花的颜色衬得面容娇艳欲滴,有种湿漉漉的滑柔。晕紫含笑浓郁的香气萦绕鼻端,一丝一缕,浸染五脏六腑,一副皮囊都似香气渗得麻了。

如懿瞟了一眼,正是那肩头绣了团龙的女子。她低低唤一声:``容珮。''

为首的靓丽女子福身告退,``那妾身等明日再来。''说罢,一个妩媚眼神抛去,便是如懿也心旌动摇,险险不能自持。

有女子擦肩而过,随手折下湖色冰纹瓶中一朵晕紫含笑簪在发间。那花朵只在野外开放,芳香幽幽,也不知是谁寻了来插瓶。花的颜色衬得面容娇艳欲滴,有种湿漉漉的滑柔。晕紫含笑浓郁的香气萦绕鼻端,一丝一缕,浸染五脏六腑,一副皮囊都似香气渗得麻了。

如懿瞟了一眼,正是那肩头绣了团龙的女子。她低低唤一声:``容珮。''

容珮即刻会意,取过瓶侧一把修剪花枝的剪子,二话不说便揪住那女子,死命压在身下,取起剪子就铰那团龙绣纹。

众人生来未见过容珮这般厉害角色,惊得目瞪口呆,连叫唤也不会了。容珮绷着一张脸,手劲极大,那女子也反抗不得,等到肩头冷飕飕,那团龙纹样已经被铰得干净。容珮闷哼一声道:``天家龙纹,你也配用在肩上?''

那女子这才反应过来,朝着皇帝惊呼一声,嘤嘤啜泣。

皇帝有些进退两难,举首见如懿阴沉面孔,一时也发作不得,便道:``上来便动手动脚做什么?''

如懿温和谦雅,``皇上安心,臣妾不屑与她们动手。自有容珮料理。''她看一眼那号泣女子,连眉头也不肯为她而皱,``好好出去吧。难不成还想留着这团龙纹样向你那些恩客炫耀么?''

为首的水沐萍伸手冉冉扶起那吓哭的女子,清冷道:``我们虽然卖艺,却不是烟花女子,皇后娘娘何必咄咄相逼?''

如懿和婉道:``即使不是自甘风尘,但已在风尘里,尘灰所到之处,难免污及清明。记得切勿得意忘形或自视过高,来日寻个好人家,也是安稳。牵连皇家事,只会自陷是非中,烦恼无尽。''

那女子停了哭泣,躲在水沐萍身后,畏惧地看着如懿。她俯视足下轻媚女子,神态如常庄静。她露出了一缕恬淡笑容,``好好回去,再不提这几日御舟之事,必可一生安然无虞。''

众人散得干净,那脂粉滑腻的气息尚滞留其间。如懿也不作声,亲自推开船枪窗扇,任由凉风悠悠灌入。

唯余了二人相对,比人多时分更窘迫尴尬,因是上了晚妆,不宜太浓艳,只是薄薄施朱,以粉罩之。如懿面上染了淡淡绯红的飞霞妆,晕浓化开,如桃花始芳。她的脸上没有一丝笑意,沿着额边青丝,以水晶、碧玺和金箔做成的五瓣绿梅花钿幽幽一明,愈显得冷艳逼人,竟隐隐生出凌霜傲意。

皇帝轻轻咳一声,``皇后,朕只想唤她们来唱些民间俚曲,了解风物。''

如懿``哦''了一声,``臣妾以为皇上只喜欢听评弹唱《隋唐》。''

皇帝笑道:``上次那个女先儿昭柔\ldots 朕喜听《陏唐》,不过是爱那一段唐太宗与长孙皇后的情深意重,感慰自已的寂寥之意罢了。''

如懿一双妙目澄澈通透,``是么?怎么臣妾记得《隋唐》说的最多的便是`穷土木炀帝逞豪华,选秀女、建洛宫,惹得各府州县邑如同鼎沸'呢?''

皇帝矍然色变,厉声道:``皇后明白自己在说什么吗?此夜何时,皇后胡言乱语,意将图谋不轨么?''

有轻鄙之意从心底蔓然延长,她反唇相问:``皇上以为臣妾独自前来,会行如何不轨之事?''她微微笑,那眼珠却冷冷的,如两丸墨玉,``皇上的日子颇有致趣,每日赏女若赏花,春色无边,不止开在江南岸上。皇上却不怕这些邪花靡草来路不明,会行不轨么?''

皇帝睨着眼瞧她,轻轻笑道:``说到致趣,朕瞧皇后这数年来悒悒不乐,便把皇后的这-份情致---起享了。''

夜色渐深渐浓,轻描着水色桃花的白纱灯罩下透出橘红的烛光,像是一抹水光,泠泠反射着淡淡的华晕。

如懿徐徐道:``皇上一直尊崇孝贤皇后,百般思念。今年是闰二月,否则已是孝贤皇后薨逝之日。臣妾很想知道,若是今日孝贤皇后尚在,皇上是否肯听一言相劝,保全清誉。''

皇帝凝视着她,缓缓摇头,``若是孝贤皇后在,---定不会如你一般顶撞冒犯朕。''

如懿长长地舒了一口气,``是啊。若臣妾对皇上宠幸伶人之事不闻不问,皇上一定以为臣妾不在意皇上,无情才无心,便如当日质问臣妾见到您悼亡孝贤皇后之诗时的感触。可若臣妾为着皇家的颜面考量,为着皇上的龙体思虑,皇上又觉得臣妾倚仗皇后身份横加干涉,不如孝贤皇后恭顺和婉。如此两难,请皇上告知臣妾,臣妾该如何做才对。''

皇帝唇角微微挑起,颇有玩味,``朕曾属意你做皇后,是觉得你是聪明女子,亦有才干。若在两难之地不能做到两全其美,朕要你做皇后做什么?''

她的心思从未这般软弱过,摇着头,绵绵诉说心曲,``皇上,臣妾来不及去想,若是一个皇后该如何两下周全。臣妾只是一个妻子,不希望自己的夫君纵情一时,留下青楼薄幸之名。所以臣妾不去回禀太后,不敢惊动他人,只敢独自漏夜赶来,为皇上驱散这些会污及您圣明的艳女。您数次南巡,是要留下与圣祖康熙爷一般的英名,垂范人世。不能因为一时的兴之所至,而抱憾来日。''她俯下身,重重叩拜,``臣妾无状,但请皇上三思。''

皇帝长叹一声,``如懿,朕这大半生都是在宫里度过,与你并无不同。甚至你都逼朕幸运些,在未嫁时,在闺阁中,无拘无束地享受过。可朕从做皇子起,每一日无不是战战兢兢,如履薄冰。朕见到的女子也都是宫里规行矩步的死板的女子,朕只是好奇,想看看宫外的女子是怎么样的,她们的日子是不是鲜活泼辣,活色生香,所以朕才会留了她们在身边。''

瞧,这便是男人,永远也停不下猎艳的好奇与追逐。

如懿只觉得齿冷,然而亦深深叹息,``皇上很想知道宫外的世界,便巡幸江南,觅香逐艳。可是作为臣妾,也很羡慕民间恬淡自足、喜悦平和的日子。夫妻间虽然过得寒薄,但可以称心如意。''

\hypertarget{ux7b2cux4e8cux5341ux56dbux7ae0-ux4e24ux76f8ux522b}{%
\chapter{第二十四章
两相别}\label{ux7b2cux4e8cux5341ux56dbux7ae0-ux4e24ux76f8ux522b}}

如懿不知道为何,会在这一刻与皇帝说起自己一直以来的念想与盼望。然而她尚念着,脸颊上已重重挨了一掌,被掀在地上。这掌掴实在是突如其来,她被掌风掀开,重重撞在红木镂雕长桌上。那红木质地坚实,一撞之下肋下痛得要裂开---样。脑海里嗡嗡地响着,像下着嘈嘈切切的瓢泼大雨,眼前白点子乱飞。半晌如懿才看得清眼前的景象,她实在不知自己犯了何错,愕然抬头。只见皇帝呼吸粗重,怒视着自己,喉间发出低沉的如兽的闷响,``朕便一直知道,你在朕的身边,却念着与旁人去
过民间生活,享你们的欢欣喜乐。''

皇帝下手颇重,她的发鬌散了大半,凌乱地垂落耳边。泪眼蒙昽里,望出一片雪色清寒,``皇上为何如此多疑揣测?''

皇帝舌底沙哑,粗戾道:``朕多疑?你自嫁与朕,便知朕不会落到民间去守着一个女子终老。那么你所揣想的不是旁人么!''

如懿喟然长叹,``皇帝渴望见到宫外的女人是怎么样的,就可以寻来这么多莺莺燕燕,敢舞喧扰。臣妾不过叹一句羡慕民间夫妻静和,皇上便要掌掴臣妾,是何道理?''

``没有道理,朕即是道理!朕这一生,少年丧母,中年丧妻失子,内有太后,外有朝政,朕有几日过得平安喜乐?如今朕稍稍畅快适意,你便诸多阻挠。这两掌便是告诉你,哪怕今日你是朕的妻子,朕的皇后,你也是朕的奴才,不可违逆朕,反抗朕!''

她望着他,像望着一个全然陌生的人,一颗心反而定了下来,有着落处。

她曾经那样思念他,思念她的弘历,在过往青葱狂热的岁月里。潜邸庭院深深几许,她自清晨他离开便独坐西窗苦苦守候,直至黄昏。外头一直落着绵绵的春雨,不曾稍停。她知道的,那是天地间的思念,如她一般。等她终于听见了黄铜门环轻轻叩动,一颗心随着那扇门的开启,如那个进来的颀长的身影一般,盼来了天光明媚。

那是朝朝暮暮的平静与安乐,于风雨中,盼得君回。

可眼前人,早不是彼时人了。两两相望,唯余失望。

曾经深深眷恋,是因为心里会快乐;而今爱恋弥散,是因为这样才不那么痛苦。

皇帝弯下身来,俯视着她,似要从她面上探寻分辨出什么。他的气息温热地拂在脸上,是夏日雨后的潮腻,``如懿,这几年来你一直不高兴,一直违逆朕。这次若非肤执意要你随行,只怕你也不肯随朕南巡。朕一直在思量,你对朕这般冷淡,是从你心里有了别人开始,还是那人死后?若是为着那人的死,他的死可是你命愉妃去的,朕可没有想他死。''

如懿黯然,灰败了神色,道:``人已作古,连当年所谓的情事也是流言揣测,莫须有之事。皇上却认定了臣妾做过,耿耿于怀,一直不肯放过。''

皇帝凝视着她,伸出手轻轻抚着她的眼皮,轻声道:``如懿,你看着朕的眼睛里全是寒气,冷冷的。朕这样被你看着,冷得受不住。''

他的手抚上她被岁月无声侵烛的肌肤,他的眼底是疏星朗月般的微光,``如懿,你多久没对着朕笑了?''

如懿无声地扯了扯嘴角,牵出一个看似圆满的笑涡,``臣妾会笑。''

皇帝端详,不宽失望,``你不是真心高兴,朕看得出来。你从前笑起来,不是这个样子。''

如懿仰着脸,看着他的眼睛。她曾最爱他的眼睛,黑白分明,仿佛会把她永远深深藏在眼底,``皇上,已经没有从前了。岁月如大江东水,哪怕贵为天子,也不能追回。''

``那么往后呢?往后你还会不会像从前那么笑?''

``已经没有从前了,如何还能那般笑?皇上,那是我们人生里最美好的时候,可惜,永远都不会再有了。臣妾所有的,不过是守着永璂长大,看他娶妻生子,安乐终老。''

烛火一点点暗下去,累累垂落如红珊瑚色的烛泪。夜色迷茫,一双眼里燃着两簇幽暗火苗,在暗夜里溅起幽幽火光。皇帝长嘘一声,无限哀清,``你终究为了他而怨恨朕。朕也实在不明白,他不过一个小小侍卫,为何会得你注目。他那般低贱,你若看向他,连着你自己也低贱了。''

``皇上,您错了。''如懿揽衣起身,端然自立,平视着他。他一直是一个俊美的男子,清癯的面庞、疏秀的双眉、温沉的眼眸和挺直的鼻梁,还有红润的嘴唇。她温柔地呢喃,是情意缠绵的低诉,``臣妾这一生,只一心一意对过一个男子,从来都是。只可惜呵\ldots{}''她幽幽叹息,``臣妾这一生,已经寻不回他了。''她沉浸在自己的想念里,幽幽诉说,``臣妾最美好的年岁里,都是和他一起度过。可惜,每每臣妾危难之时,质疑之时,孤弱之时,他从未在臣妾身边,连愿意拉臣妾一把对臣妾温善的人,他都一心怀疑。那是因为,其实他也很少相信臣妾,也在怀疑臣妾。所以,臣妾开始失望,渐渐也习惯这种失望。失望得久了,便也对他彻底绝望。''

皇帝伤感不已,``不会再有希望么?''

她忽然转眸,静静道:``皇上没有发觉,臣妾已经很久没有用绿梅粉了么?''

那是她刚出冷宫的时候,皇帝细心研磨,用尽心意,制了送与她独用的。

皇帝语气一滞,歉然道:``是朕浑忘了,忘记再送与你。等这次回宫,朕一定让内务府再制了送你。''

``没有必要了。绿梅粉长久不用,便也惯了。''她疏懒地笑,退开两步,保持着与他的距离,``即便臣妾接受了皇上的好意,来日漫长,臣妾等来的,会不会依旧是---次次怀疑,一次次无助,一次次失望后的绝望?''

他天生拥有着微微上翘的嘴角,白皙的肤色,好像对着谁都是那般温和多情。可是他的眼底里其实并无笑意。她曾经爱过的,就是这样一个人。

真是惘然。

皇帝的呼吸声是渐近的潮水,他似乎极力克制着什么,``皇后,朕就是你从前的那个人,只要你想明白,朕会谅解你今日的无状。''

她轻轻一笑,拢住散乱的青丝,引袖取过一把小小银剪,那凛冽的寒光在她指尖闪烁,她剪下三寸胄丝,看它们纷纷垂落于地,``皇上,咱们满人一向爱惜头发,以剪发表示爱侣亡去守身坚贞之意。臣妾待心里的那人,便是如此。从前看不明白,以为他千般万般都可原谅,如今看得明白,才知他痴恋的是旁人,敬慕的是旁人,疼惜的也是旁人,守着他日日夜夜都是煎熬。''

皇帝震惊到无以复加,``你知不知道自己在说些什么?''

如懿迷茫地摇头,却有清醒无比的坚定的眼神,``臣妾知道。皇上,您容许臣妾疯一会儿,听听臣妾这些疯话吧。左右臣妾与您都神志清明的时候,总是无言以对,总是彼此猜忌的。今夜您能把秦楼楚馆的歌伎召上御舟,您不也疯了么?''她笑意迟迟,酸楚至极,``皇上,臣妾出身贵家,自幼看愤妻妾争宠的闹剧,便是臣妾的姑母为皇后之时,臣妾耳濡目染的还少么?及至嫁与您为侧福晋,臣妾哪怕爱慕着您,也不敢求您的一心一意,只希望您的心中有臣妾的分毫之地,臣妾可以凭着这一丝情意,与您偕老。可是伴随您长久,臣妾越来越明白,其实您谁都不信,您缺父子之恩,母子之情,自幼孤立无援,所以对自己的儿子也是一般。所以且不论孝贤皇后,便是臣妾等人,您又真正信了几分?不过是一有风吹草动,便猜疑难平。''

``朕疑心?''皇帝冷笑,脆弱而惶然,``朕如何能不疑心?朕自幼所见是皇额娘与你姑母争宠,彼此无所不用其极。等朕开府封王,登基为帝,你们这些人一个个又做过些什么?为了子嗣,为了宠爱,为了名位,你们也何尝不是无所不用其极?肤对着你们温柔婉顺的笑靥,常常在想,你们到底在想什么?图谋朕的什么?你便以为联从来没有害怕过,朕的孩子一个个死去,你的手便完全干净了?''

她从未想到,他的口中转说出如此言语,头顶似有一道烈雷轰然炸开,心口一阵阵抽疼,疼得她喘不过气来。瞬息之间,震惊、伤心、苦涩、悔恨、愧疚、惊畏,齐齐涌了上来,翻涌五内。她整个人蒙在当场,口干舌燥,无言相对。泪水滚烫地烧灼成一片,她的心灰到了极处,做下的事,终究是要还回去的。

``你居然流泪?''皇帝伸出手,他的指尖很干燥,抚过她的面颊有微刺的疼,
``朕猜疑你与凌云彻,你不曾哭。朕与你疏离多年,你也不曾哭。朕只是问问你的手干不干净,你却哭了。''他倦得很,轻轻摇首,``你们做过的事,朕不想知道,也不想去猜。左不过都是见不得人的恶心事,真叫朕恶心。''

如懿微微颔首,任由泪水滑落,``是。就和皇上赏给舒妃的坐胎药那么恶心,都是---样的。''

他冷冷地俯视她,哀伤如重重迷雾,弥漫渐深,``如懿,你还是从前的青樱么?为何朕觉得你形同疯妇,神志不清?''

``青樱,早已不在了。她和臣妾心里所盼望的那个人,大约会永远在一块儿,却再也寻不见了。但臣妾和皇上,终究是长久相处,彼此暴露得体无完肤,相看生厌。''她睁着眼眸,恬淡至空明,``皇上,是真的。臣妾在宫里的每一日,都在发疯,都在做着自己都觉得不可思议的疯狂的事。高晞月是,金玉妍是,苏绿筠是,白蕊姬是,厄音珠是,蓝曦是,您也是。我们每个人都在发疯,可臣妾分明记得,我们的起初,都不是这样的!''

她手起剪刀落,再度剪下一缕发丝,凄楚哽咽,泣不成声,``这一缕头发,给去了的乌拉那拉青樱。''

皇帝震惊到无可言语,忽然外头一阵响动,竟是嬿婉与和敬公主闯了进来。二人见此情景,不觉惊呆了。还是和敬先回转神来大声道:``皇额娘,您在做什么?''

嬿婉这才如梦方醒,跪下哀泣道:``皇后娘娘,请您住手!''

皇帝气得连连冷笑:``你们来做什么?还觉得不够难堪么?''

和敬忙上前扶住了皇帝,连连抚胸道:``皇阿玛,儿臣怕皇额娘冲撞了您,所以特意赶来。皇额娘,满人不可轻易断发,您这是大不敬!''她说着,便欲上前去抢如懿手中的剪刀,``皇额娘,您再如此,别怪儿臣不认您!''

如懿如何会让和敬抢到,她举起剪子在喉头,冷然道:``和敬公主,你的额娘,唯有孝贤皇后而已,又何必在意我呢?''

嬿婉连连叩首,拉住如懿裙角,``皇后娘娘三思呀。您这一剪子下去,可是剪断了与皇上的情分了。''

如懿厌弃地踢开嬿婉,只是不语。

皇帝唇色雪白,咬牙道:``疯了!皇后已经疯了。''

如懿凄楚不已,郁然长叹,``皇上,您不必再疑心臣妾做了什么错事。臣妾的错事太多太多,您疑心的,您的女人的,您的子嗣的,一股脑儿,全是臣妾的错事。恕臣妾说一句,做您的皇后,在您身边,实在是太累,太倦了。若有来生,臣妾一定要离开这里,离得远远的,越远越好。''

皇帝眸中的郁火渐渐燃烧殆尽,成了冷寂的死灰。他决然摇首,``朕的皇后,可以死,可以废,但绝不可出厌弃之语,藐视君上,失去做臣妇的本分!''他一顿,语气更例,``乌拉那拉氏,你真的是疯了。必有大丧,才可断发。你居然当着朕的面亲手断发,狂悖迷乱!与其你如此疯癱,还不如朕废了你,许彼此一个清静!''

``废了臣妾?''如懿淡然平静,``臣妾一直在想,被皇上所追念的女子,难道一定是皇上所爱么?孝贤皇后也好,慧贤皇贵妃、哲悯皇贵妃也好,还有容嫔,皇上真的爱惜她们么?不过是以此彰显自己情深而已。从头到尾,您都如您最爱的水仙花,临水自照,只爱惜您自己罢了。''

皇帝断然大喝,忿郁难平,``当着儿女与嫔御的面,你都在胡说些什么?来人!''

嬿婉像是受了极大的惊吓,哀求道:``皇上息怒,皇上息怒啊''

和敬只护着皇帝,``皇阿玛保重!皇额娘是疯了,您可不能再气着了呀。''

皇帝喘着粗气,又喝一声,``来人!''

外头的宫人们听得五内焦灼,只不敢进来,闻得这一声唤,忙不迭滚了进来。

皇帝冷若寒冰,``皇后乌拉那拉氏形迹疯迷,不堪承受皇后重责,命福灵安漏夜急送回宫中医治。无朕旨意,不得出翊坤宫半步。今日之事,更不许任何人知晓,否则你们的脑袋,朕都不想留了。''

李玉哪敢多问,正要伸手去扶如懿。皇帝似想起什么,道:``李玉,你身为御前总管,不知劝阻皇后,惊扰圣驾。日后不必在朕跟前伺候,去圆明园当差吧。''

李玉身形一晃,面色惨白,只得诺诺答允了,撤开了手。进保上前,扶住如懿手臂,缓步往外走去。

如懿轻轻一挣,``皇上,这半世里,你对臣妾说过无数次要放心,可臣妾的心从未放下过。今日俗事已了,臣妾倒真可以放心了。''她俯身深拜,淡然自若,``今日一别,相见无期,皇上珍重,``

她被半扶半持着带上小舟。月已西斜。

湖中寂静,只有花开声与飞鸟声,远远近近传过来。那是晚归的夜鹭,在青芦深处发出聒聒深沉的叫声。皓月如霜,落下惨淡白光。

她在恍惚中有一丝错觉,她嫁与弘历的那夜,也是这般月色。他笑盈盈唤她:青樱妹妹。

她回首望去,来时之路与前面去路都茫然不见,天地间终是那片叫人绝望的茫茫水月之色。而唯一沉定的心意,是她明白,哪怕决绝至此,她的一生都会与他牵绊,忘不得他。

次日便有两道旨意下来。一是皇后急病,送回宫中。二是贵妃魏嬿婉晋位皇贵妃,摄六宫事。

这变故来得太大太突如其来,行在里登时慌乱起来,便想去御前探听。谁知总管大太监已在一夜之间由李玉换成了进忠,更显诡谲。嬿婉虽然欢喜得不知所以,也知道即刻镇定下来,加以安抚。外有大臣傅恒主持,内有和敬公主与皇贵妃魏氏,将一切流言死死压住,众人纵然揣测,也不敢多言。这日和敬陪了皇帝半日,劝得皇帝用了晚膳,这才出来。

江南的傍晚,炎夏亦有湿润气息。只是这行宫内外,因为突如其来的变故,才显阴沉莫名。连那署气隐隐亦有黏稠的意味,缠得人透不过气来。

是该早些回京了吧。江南风物再好,又怎及京城呢?

和敬这样想着,举目正见傅恒走过来,便问安道:``舅舅大安。''

舅甥俩亲近,傅恒便问:``公主可否有空,一同走走。''

``和敬回首看看殿内,颔首道:``好。我也正有话对舅舅说。''

夜风习习,有栀子花和夜来香的气味幽幽传来。那雪白的香花气味太过甜郁,和敬素来不喜,不觉皱了皱眉头。

傅恒也未留意,只关切道:``皇上还在生气?''

和敬叹道:``被乌拉那拉氏气得狠了,---时转不过来,一直扬言要废后。舅舅,乌拉那拉氏如何了?''

``福灵安派人来回话,一路上安静得很,也没出什么大事。我只盼着平安回京,若在路上出了岔子\ldots``

和敬看着傅恒担忧的面孔,断然道:``那事情就闹大了。安静回了宫,出再大的事,紫禁城的墙那么高,什么也都捂住了。这事儿在杭州已经闹得够不堪了,可不能再传出什么有损圣誉的话来。''

傅恒沉着道:``一切有我呢。只是公主,这几日令皇贵妃在皇上跟前很得脸吧。''

和敬听得提及嬿婉,便有些不屑,``皇贵妃位同副后,便宜她了。''

傅恒遥望嬿婉住处方向,不觉摇头:``那位的心气高着呢。一个皇贵妃之位,只怕犹不满足。''

和敬的面色阴沉得如黑云压城,``让乌拉那拉氏继位皇后,已经不配。若她还想成为皇后与额娘比肩,那更是痴心妄想。这回的事少不得借了她的力,可若还想往上爬,我也容不得她。''

傅恒闻言便笑了:``魏氏抵位皇贵妃,自然野心勃勃。只是她根基不足,少不得还想借公主之力。自然,公主与我都是不愿意的。''

和敬用力点头,握紧了手指,``舅舅和我想的一样。令皇贵妃心性狡诡,借她的手做事可以,可若要借我们之力成为皇后,我万万不肯。我额娘才是皇阿玛身边最德行出众的皇后,谁也不配和额娘比肩。''

傅恒眼底微有晶莹之色,``公主说得是。乌拉那拉氏登位皇后之日,我曾请公主忍耐。不为别的,只为她正得意,我们却力有不逮,所以只能眼睁睁看她继位皇后,身膺荣光。''

和敬姣好的面孔闪过一丝狠意,``可我从来没有忘记乌拉那拉氏带给额娘的伤心与痛苦。舅舅,我身上也流着富察氏的血,我怎能让富察氏的仇人永踞高位。不,她们永远都不能和额娘比。额娘才是皇阿玛最爱的女人,最贤德的皇后。没有任何人可以取代她,绝对没有。''

傅恒轻轻拍着和敬的肩膀,平抚着她的情绪,二人默然相对,心意了然,这才各自散去。

绛华馆里,太后的神色有些焦灼不安,手里光洁的白铜水烟杆显得一双手也有了岁月摩挲后苍老的痕迹。

皇帝将要说的话已然说完,``皇后自册立以来尚无失德,儿子此次奉皇额娘巡幸江浙,正承欢洽庆之时,皇后性忽改常,于皇额娘前不能恪守孝道。昨夜举动尤乖正埋,迹类疯迷。儿子只能先令其回京,在宫调摄。皇后行事乖违,无端顶撞,儿子哪怕予以废黜,亦理所当然。''

有一瞬间的感怀,有风清凉拂上了眼角,带了湿润的气息。他蓦然想起孤绝的少年时代,人人冷落他忽视他的时节,眼前这个女人曾经给予过他的关怀与照拂。那时节,他们是真心相待的母子,哪怕没有血缘的关系,亦彼此扶持着走了许多年。只是后来,他终于成了皇帝,她亦成了太后,彼此之间反而多了算计。

算计着,算计着,这么多年了呵,这么精明而美貌的女人,原来也会老,也会着急,也会失了分寸与笃定。

皇帝将要说的话已然说完,``皇后自册立以来尚无失德,儿子此次奉皇额娘巡幸江浙,正承欢洽庆之时,皇后性忽改常,于皇额娘前不能恪守孝道。昨夜举动尤乖正埋,迹类疯迷。儿子只能先令其回京,在宫调摄。皇后行事乖违,无端顶撞,儿子哪怕予以废黜,亦理所当然。''

有一瞬间的感怀,有风清凉拂上了眼角,带了湿润的气息。他蓦然想起孤绝的少年时代,人人冷落他忽视他的时节,眼前这个女人曾经给予过他的关怀与照拂。那时节,他们是真心相待的母子,哪怕没有血缘的关系,亦彼此扶持着走了许多年。只是后来,他终于成了皇帝,她亦成了太后,彼此之间反而多了算计。

算计着,算计着,这么多年了呵,这么精明而美貌的女人,原来也会老,也会着急,也会失了分寸与笃定。

这样的念头如春藤缠绕上他的心间,他不自觉地走近了两步,如年少时般依恋,跪俯在了太后跟前,一腔子暖意和软弱填满了心上的缝隙,唤了一声,``额娘。''

太后许久未曾听得皇帝这般动情呼唤,握着烟杆的手颤了一颤,凝神伤感道:``皇额娘你倒是天天叫,但这么个叫法儿,哀家真是许久没听过了。''太后有些出神,仿佛沉浸在对往事遥远而无法停止的追忆中,``你小时候,每日下了学,就急匆匆往哀家宫里赶,一见了哀家就这么唤一声`额娘',然后跟在哀家身边,总舍不得离开。那时候哀家真觉得,你就是哀家的亲生儿子。''

皇帝声音低低的,带着雾水般的潮湿,``在儿子心里,您就是儿子的额娘。''

太后的叹息带了悠长的尾音,有无限唏嘘,``有皇帝这句话,哀家就敢说话了。''她顿一顿,沉声道,``皇帝,你真的想废后?''

皇帝无言,闭目叹息,手中毫无意识地蜷缩着。他沉默片刻,轻轻颔首。

太后久久郁然,``废后乃是失德之举,于国祚更是不祥。想先祖顺治爷一生,最为人诟病的并非独宠董鄂妃,而是废了第一位博尔济吉特皇后。大清开国百年,废后的唯有这一次,皇上可不能步厢治爷的后尘啊!''

皇帝的口气有些强硬,别过脸道:``失德的是皇后,不是朕!皇后生性不驯,屡屡冒犯于朕。还敢不顾国之大忌,亲手断发,朕实在忍无可忍。''

太后懊丧地摆首,重重地敲了敲水烟杆。那水烟杆本是白铜铸成,极有分量,
此刻敲在紫檀桌上,发出闷闷的声响,像远处云后有闷雷盘旋。``满人断发,一为国丧,二为夫丧。皇后出身大家,这件事的确是做得太没有分寸了!''

皇帝隐忍的怒意骤然爆发,手里捧着的茶盏一个不稳,茶水险险拨了出来,``皇后如此狂悖,朕如何还能容忍!''

福珈伺候多年,何曾见过皇帝这副模样,不觉骇得脸色都白了,忙伏到皇帝身边,为他拂衣敛袖,手势轻巧,示意他安静下来。

殿中静得只听得衣衫簌簌的声音。太后沉默片刻,静静道:``皇后失德,自然不能一味容忍。可若要废后,皇帝你自己的声名也会受损。夫妻本为一体,皇后又曾诞育子女。皇帝亲自废立皇后,天下臣民亦会不安。民间休妻尚要有七出之条,皇帝你要如何昭告天下,为何废后?''

皇帝的神色阴郁难定,``妇人七去:不顺父母,为其逆德也;无子,为其绝世也;淫,为其乱族也;妒,为其乱家也;有恶疾,为其不可与共粢盛也;口多言,为其离亲也;窃盗,为其反义也。皇后言行狂悖,直指朕有过,冒犯君上,亦是言太后教子无方,等同不顺父母,也是口多言。皇后正位中宫,多年来驯御嫔下过于严苟,便是忌妒。七出之条皇后犯了三条,朕还不能废后么?而且皇阿玛在世时,乌拉那拉皇后无德,皇阿玛不也曾动了废后之念?这个,皇额娘也是知道的。''

太后念及旧事,不觉深吸一口凉气,``你皇阿玛动了废后之念,但到底也没有废后啊!天下臣民言之凿凿,为君上者,如何能不忌讳?''

``皇额娘从前深受乌拉那拉皇后之苦,从不喜如懿,亦不赞同儿子立如懿为后。如今儿子要废后,应该合了皇额娘心意,皇额娘怎倒不允许了?''

太后的神气渐渐平和,似是极力克制着自己,目光却如明镜,深照着皇帝哀颓愤懑的面孔,``哀家深受乌拉那拉皇后之苦,的确不喜欢乌拉那拉如懿,总觉得她性格过于刚毅,不够柔顺。但当年坚持立后的是皇帝,自然是知道如懿的性格的,从前很喜欢,如今怎倒不喜了?等闲变却故人心,皇帝就不怕人议论你对皇后是色衰爱弛的缘故么?''

皇帝额头的青筋跳了一跳,鼻翼微微张合,``变的是皇后,不是儿子。''

太后合目不语,左手缓缓捻着一串十八子凤眼缀千叶莲华佛珠。那凤眼菩提本在酥油中浸润,温润油亮,在太后苍老温暖的手中辗转轮回,摩挲成这沉沉殿宇内唯一一痕温和的枣红亮色。``是啊。人心都是会变的。当年哀家不赞同立如懿为后是为了皇帝,但今日哀家不赞同废后,为的也是皇帝。如懿继位中宫之后,御下虽然严苟,但皇帝之前并无指责,那么就不能作为今时想要废后的理由。如懿自在潜邸就侍奉,又为皇帝生下二子一女,其姑母又是先帝的孝敬宪皇后,皇帝不能不顾念啊!再者,哀家与如懿的姑母恩怨已久,人老了有什么不可以放下。皇帝人到中年,何必苦苦执着?''

皇帝静静地听着,心思缓缓游逸。思绪盘结无定,他只觉得倦意深重,再也无法负担与她的过往。---度,他也以为,凌云彻死了,一切事端都会成为紫禁城红墙深埋下不值一提的尘埃。可是每―次见她,见到日复一日深重的沉默,和眼底哀伤的阴翳,都会在心里不自觉地衡量与她之间的距离,像在茫茫大雪中渐行渐远的人,他不知道她要去的方向。连那曾经无比接近的仿佛触手可及的距离,也禁不起轻轻地触碰,如水中幻影流离,一探即碎。

何况,何况他才知道,她背着自己,做过那样多的事。

水烟杆上以翡翠镶嵌九只雄狮模样,那深沉的翠色嵌在白铜之上,华光灼目,更兼雕工细腻,栖栩如生,九狮扬爪怒目,几欲跳下身来。皇帝一眼落在那翡翠狮子上,心底便有些厌恶,``内务府的奴才越来越不懂事了,奉送皇额娘的东西该用鸾凤摸样,或是雕些温驯的猫儿图样也罢了,怎么用这么耀武扬威的狮子,戾气太重,不宜皇额娘所用。''

太后瞟了一眼,随口道:``这不是内务府进奉的,是柔淑在外头看了好玩,说花样新奇,才给哀家的。''她话音刚落,旋即明白皇帝心底的不悦,无奈地笑了笑,``怎么?皇帝看了这狮子,想起皇后的言行跟这狮子的爪子利齿一样让你不舒坦了?''

皇帝垂下眼眸,躲避着太后洞察一切的目光,``皇额娘说笑。''他想一想,语中带了不满的怒意,``但有句话皇额娘没说错,皇后的言行不像一个国母,甚至连一个温顺的女人都不是。一味纵情任性,有失国母之尊。更何况她背着朕做的那些事,朕也不忍提。''

``一个不够温顺、不肯装糊涂的女人,自然是不讨男人喜欢的。皇帝坚持废后,大概也是这个缘故吧。至于皇帝所言,皇后背后所做的那些事,自然是见不得人的。''她轻轻一嗤,笑意渺然,摊开自己的手,``可是皇帝自己也知道,论哀家,论你,便是令皇贵妃和宫中任意一人,只怕他们的手都不够干净。活在宫里的人,有几个是清清白白的,逼疯了自己也得装着清醒。这样的日子,皇帝还不清楚么?''

皇帝硬着声气道:``旁人可以是,乌拉那拉如懿不可以。不为别的,只为她是朕亲自选的皇后。''

太后微微一笑,,``皇帝你若不在意皇后,自然也能装糊涂下去,顶多一辈子不闻不问罢I。你们彼此都活得这么清醒,分分寸寸都不肯让步,无非还是彼此太在意的缘故了。因为在意而废后,皇帝你自己觉得值当不值当?且皇帝觉得,废了乌拉那拉氏,谁可以继位为皇后?''

皇帝别过头,``朕在意的是一个皇后该有的言行举止,而非乌拉那拉如懿这个人!若无可以继位皇后的人选,那便空留着后位也罢。免得不合适的人站到不合适的地方去。看若有合适的人,取而代之又何妨?''

太后微眯了双眼,轻轻笑道:``皇帝的意思,是令皇贵妃?''她的唇抿得意蕴深深,``令皇贵妃足够婉顺清媚,但皇帝难道忘记了,她是宫女出身。''

皇帝双眉挑起,赫然冷笑,``怎么宫女便做不得皇后么?若是令皇贵妃识趣,儿子抬举她也是应该的。''

太后一震,蓦然想起,原来他的生母便是一个卑贱的宫女。这样想来,怕也无可无不可吧。

``皇帝如此说,是真的要废弃皇后了?但愿皇帝你能明白自己的心意,每一步都不会有让来日后悔之举。''太后望着他,意味深长,``若要废后,伤的不止是皇帝你的圣明,也是你自己的心。哀家的意思己经说明白了,言尽于此,你自己慢慢思量吧。''太后斜倚着身子,望着皇帝起身欲去的背影,声音沙哑低沉,缓缓地道,``皇帝,当日来面见哀家执意要立如懿为后的人,是你。今时今日执意要废弃她的人也是你,其实哀家身为女子,也真的很想知道,怎么从前喜欢的,如今却那么不喜欢了
呢?''

皇帝眼光有一瞬的迷离,仿佛透过了庭院中烂漫盛放的春桃,看到了遥远的地方,``皇额娘,儿子也不知道。就如儿子不明白,曾经如懿可以对儿子一往情深,为儿子承受种种委屈,如今却这般暴烈狂悖了呢?''他自嘲地摇摇头,身影在花事繁盛里显得单薄清瘦,``大约,人都会变的吧。''

太后目中微澜,泛着淡淡温情,``既然你与如懿都是,那又何必执着废弃她呢?你与她的龃龉疏离,都是彼此在意的缘故。皇帝,彼此留一线,不是为了别的,只为真正废弃她之后,你会后悔,会发现自己对她的在意,那时便真的追悔莫及了。''

``不!''皇帝断然决绝,``儿子不在意。这个女人,皇后不像皇后,妻子不像妻子,奴才也不像奴才。她搁在哪里都不合宜。儿子厌恶这样不合宜的女子。''

太后目光如水,澄澈通透,``若说像皇后,像妻子,莫过于孝贤皇后。若说像奴才,你宫里多的是。可是那时,你又未必喜欢了。当年孝贤皇后在世,你也曾不喜她恪守规矩、古板无情趣。待她死后,才觉出她种种好处。也许来日,如懿死了,你才会想起,她曾有过的好处。''

晴光落在他面上,有照不亮的阴翳。皇帝不复一言,缓身退去。

\hypertarget{ux7b2cux4e8cux5341ux4e94ux7ae0-ux6625ux5f2d}{%
\chapter{第二十五章
春弭}\label{ux7b2cux4e8cux5341ux4e94ux7ae0-ux6625ux5f2d}}

如懿是在一个漆黑的深夜回到翊坤宫的。宫里安静得近乎诡异,空气里顿然失去了江南杏雨烟柳的暖与润,触鼻是清冷的寒意。

她打了个寒噤,身上的素青色云纹折枝莲花大氅显得格外单薄,在夜风里颤颤地抖动。如懿望着熟悉的甬道上一盏一盏亮着的昏黄灯火,仿佛照着自己早已看不清的昏昧前路。一路送她回来的人是福灵安,那是孝贤皇后亲弟傅恒的长子。她与孝贤皇后的恩怨宫中皆知,又当落魄之际,福灵安这一路陪伴,自然没有什么好脸色,照顾也不周全,不过是容珮细心陪伴,才熬了回来。

那又如何?她的未来已然全部断送,何来祈求别人的好颜色?

海兰本没有跟着南巡,她一早得了消息,急得嘴角都上了火,便领着人候在了翊坤宫外。

因着帝后离宫,宫中的烛火都停了一半,黑沉沉的夜里,月色惨淡。青釉色的月光下只见重重金色兽脊安静伏定,冷冷仰天瞪着,呐喊无言。四下里寂然无声,唯听见一乘青帷辂车的车轮轧过古旧的雕花石板路,惊起擔上的宿鸟呱一声扑棱着翅膀飞远了。翊坤宫似一只沉馱怪异的兽,潜伏在暗色之中,唯有宫门口两个斗大的水红色薄绸灯笼,被风曳得晃晃悠悠,如两只不能合上的眼。

宫车辘辘而定,容珮扶了如懿下车,海兰已然带着叶心候在了门外。她陡然见了如懿,看她身着碧水色无绣锻服,桓字髻上簪着几支素净的犀玉扁簪,脸色是病态的苍白。她哪里还按捺得住满腹的凄惶,喊道:``皇后娘娘------''

话到唇边戛然而止,进忠小跑着上来,皮笑肉不笑地道:``愉妃娘娘,这一句皇后娘娘还不知叫得叫不得。您,还是跟奴才一样,先叫一声主子吧,也不算得罪了。''

名分未定,总是落在尷尬地里。

海兰也未看进忠,走到如懿身前,依足规矩施了一礼,轻轻唤:``姐姐。''她仰起清定的眸子,温声道,``你和皇上,终究还是到了这个地步。不过,姐姐终于回来了。外头不安宁,只要回来就好。''

如懿眼底一热,握住她的手,念念道:``海兰。''

海兰的掌心明明是湿的。不知这一路候着自己的消息,海兰是何等焦急失措。她原是静惯了的人,无欲无求,波澜不惊,却为了自己,这般心惊。

如懿生了歉意,静静道:``别慌。''

如何能不慌呢?这话原是安慰罢了。海兰笑意温沉,定定道:``是。咱们还有永琪和永璂。''

进忠道:``愉妃娘娘,主子得赶紧进翊坤宫去。春寒料蛸的,总得进了里头才好歇息,隔了外头不该有的东西。主子也好静心思过啊。''

海兰知道进忠正得势,也不便顶撞,便道:``皇上的旨意本宫已经知道。皇上远巡在外,宫中一切都由本宫打点,翊坤宫事宜,本宫也会照料好。''

进忠笑道:``那是自然的。皇上身边有令皇贵妃照顾,宫里一切还得仰仗愉妃娘娘。''

他刻意咬重了``令皇贵妃''四字,海兰如何不恼,面上却笑得安然,``是。''

进忠又道:``皇上说了,主子一回宫就得进翔坤宫,一应服侍的人都得撤去。只留容珮、菱枝和芸枝三人,免得闲杂人等扰了主子静思己过。''

他话语中未有一丝尊敬之意,如懿哪里肯与他计较,海兰也忍下不言,只是扶住了如懿手臂,``里头连夜已经打点好,臣妾送姐姐进去。''

进忠伸手一栏,``愉妃娘娘,皇上说了,进了翊坤宫就不必出来了。您玉足矜贵,这一步迈不迈,您可得思量清楚了。''

海兰银牙微咬,正要发作。如懿已在袖子上按住了她的手,微微摇头,``你还要替我照顾永璂,更有永琪。''

冷风涌动,在甬道间呼啸穿梭,打得鬓边一支白玉莲首压发缀着的一绺红缨珠流苏,沙沙地打着耳际,是冰冷的疼。海兰眼底泪光一闪,解下自己身上的织金南荑曲字贡缎大氅披在如懿肩上,那大氅的领口袖口皆围有白狐腋子毛,十分和暖。

海兰忍着泪道:``臣妾已经极力安排,但内务府已得皇上旨意,里头\ldots 里头不比往日,姐姐保重。''

如懿合上掌心,从她手背滑过,``海兰,保重。''

如懿不忍再回首,步下匆匆,转入宫中。身后两扇宫门相合,发出沉闷悠长的声音,似将一副绵软心肠,狠狠夹断。

海兰看着她的背影,目送她踏着宫灯倾流而下的一泊光亮缓步走进,泪水潸然而落。

进忠劝道:``时辰不早,愉妃娘娘既已接了主子,也可早点安歇了。''

海兰颔首,``公公一路辛苦。''她正要挪步,只觉得足下唯有窸窣之声,正是如懿素日间不离的一枚金累丝嵌珍珠绿松石蝶舞梅花香囊。那香囊以细金丝累累缀起梅花十二朵,花蕊处均嵌白色珍珠一颗,以绿松石琢成蝴蝶模样,内侧镶金,阴刻梅花十九朵,朵朵如生。囊内存着如懿最爱的沉水香,香气幽然,犹自沾染她衣袂之间。

海兰心底一酸,弯身拾起,紧紧攥在手心,吩咐叶心道:``夜深了,咱们回去吧。''

如懿行至殿内,才知海兰的不得已是为何。连菱枝也禁不住发出惊呼,来感慨殿内天翻地覆的变化。

灯烛被减至两盏,昏黄暗淡。她渐也适应了昏暗,熟悉了周遭物事的轮廓与错落。容珮端起莲形铜灯,小心护着灯芯,替她照亮察看。

自如懿出冷宫,翊坤宫便是她的居所,多年来精心布置,无一不典雅华贵,早已融进一桌一椅之中。可是乍然见到,宫中略微值钱的东西一应都被撤去,连床帷帐帘所用,都换成了宫人所用的青灰布幔。

容珮双唇哆嗦着道:``内务府的人怎可以如此待娘娘?皇上尚未废后,他们便迫不及待了么?''

如懿摆摆手,示意她不必多言。

废后之意昭然若揭,内务府最通上意,如何不知。如懿步进佛堂,见青灯依旧,佛尊含笑,一如从前。菱枝再开柜子,四季衣衫还算周全,连暖阁里如懿的一副绣花架子,各色丝线都还不缺。便知海兰所能极力打点的,便是如此了。

如懿安然盘坐于青绒布蒲团上,拈起一串佛珠,对着拈花慈悲的佛像,念出佛语三千。

她的唇角,绽开郁郁笑色,也好,这便是往后所有的日子了。

春日迟迟,卉木萋萋。翊坤宫外是艳阳如织花事锦簇,而翊坤宫内是青灯古佛寂然终日。

皇帝回宫后不久,便下令收回如懿手中的四份册宝,皇后一份,皇贵妃一份,娴贵妃一份,娴妃一份,并将后宫所有事宜交予新晋的皇贵妃魏嬿婉处置。册宝交出的那一刻,她心底没有一分戚然。只是看着那些曾经属于她的东西,又失去了一分。不要紧,这一路与他风雨同来,不过是得到一些,失去一些,

那是他与她来时的路。从娴妃起,以皇后终,还是走不到天长地久的尽头。

因着册宝收回,嬿婉宫中气焰更盛,众人日日奉承簇拥,将永寿宫捧到了高处。连偶尔出入的和敬闻得喧闹的笑声,也不觉蹙眉,``新封了皇贵妃,摄六宫事,这全然是当年乌拉那拉皇后的做派。只差一步,就是皇后之位了。难怪人人都奉承永寿
宫。''

话固然是气话,但当和敬看到皇帝御桌上本属于如懿的四份册宝,亦是黯然垂叹。

皇帝讶异:``你叹什么气,别告诉朕,你要为乌拉那拉氏求情。''

和敬连称``不敢'',可还是忍不住抱怨,``儿臣只是想着皇阿玛这般生气,令娘娘也该多来陪陪皇阿玛。毕竞她所得所有,都来自皇阿玛。若是得闲,也得教养好几位阿哥和公主,别和翊坤宫娘娘似的,一味和皇阿玛怄气,连孩子都不顾着。''

皇帝原以为她刚摄六宫事,怕也千头万绪,不肯计较,便随口垂问。和敬索性都说了,``宫里多的是趋炎附势,令娘娘怕也身不由己。儿臣过来时,听见永寿宫的笑声,能传遍西六宫了。''

皇帝微微蹙眉,也不指责。和敬觑着皇帝神色,漫不经心地说:``儿臣前几日遇见舅舅,倒听舅舅说起一件行宫里的旧事。''

皇帝这才在意,便问:``什么事?''

和敬坐到皇帝身边,一副女儿家亲昵之色,毫不讳言,``舅舅说起翊坤宫娘娘触怒皇阿玛那日,本是从西湖边上船要去御船上的。那夜本是舅舅戍守在西湖边,他若看到翊坤宫娘娘,原该阻止,也少了一桩纠缠。那时令娘娘还不是皇贵妃呢,也一样忧心皇阿玛,怕御船上守卫不周,所以特意问了舅舅御船上有哪些人。''

皇帝``哦''了一声,随手拨了拨如懿的册宝,``皇贵妃倒是用心,可朕御船上的事,可不干她的事。''

和敬额首道:``舅舅自然是不肯多口的。后来知道翊坤宫娘娘和皇阿玛闹起来,令娘娘急急来扯儿臣同去劝说,这才撞见了翊坤宫娘娘断发这一幂。唉,其实皇阿玛与翊坤宫娘娘也是夫妻,争执也是常情。可这样难堪的事落在儿臣与嫔妃面前,又有奴才们在,这才难以挽回了。''

皇帝眸中漫起阴郁的焰火,``你是说,朕周围的一切,皇贵妃都知道得紧?''

和敬的讶异恰到好处:``不是皇阿玛与令娘娘亲近,令娘娘才知道的么?难道她还有意窥探,才时机如此之巧,正好拉了儿臣撞到翊坤宫娘娘断发的情景?令娘娘素来温柔恭谨,总不至于吧?''

皇帝的脸色渐渐难看,``她既然向傅恒打听过,自然也会向旁人打听。哼,皇贵妃心眼儿挺多。''

和敬微笑:``令娘娘能得皇阿玛多年宠爱,自然心思过人了。哎,皇阿玛,咱们说这些不悦之事做什么?儿臣许久没向皇祖母请安了,儿臣与您同去慈宁宫吧。''

皇帝笑意凝固在唇角,却也不提此事了。

没过多久,又有人带走了三宝和芸枝,只剩了容珮和菱枝在身边。美其名曰,娘娘静心思过,不必太多人打扰,

菱枝气得直哭,拉着容珮的手道:``这算什么?皇上到底没有废黜娘娘,为何只剩了咱们两人伺候。宫里的常在小主才只有两个宫女呢。不,常在还有太监伺候,娘娘却连这点体面也没了。''

容珮只得安慰道:``别哭,别哭。三宝去伺候十二阿哥了,芸枝去了婉嫔小主那里当差,也不算坏。''

如懿只作听不见。她独自留在佛堂内,擦净铜灯上的乌迹,添油点亮,置于佛尊前。天色一分分暗下去,烛光中的佛尊眉目慈蔼,浑不知人间疾苦。她只是奇怪,与其如此麻烦,他为何不直接废黜了自己,也省得这些零碎折磨。想不通,不愿想,她便孤坐于蒲团之上,翻阅着那些艰难晦涩的梵文。

春夜幽凉,冷冽如秋。宫烛焰火摇曳,牵得她身影幽长,漫成孤请一道。冬日的火盆早已撤去,凉意渐渐迫近,逼入骨髄。她穿着青素衬衣,不觉生寒,伸开双臂,紧紧箍住的,难有自己。

有脚步声走近,她以为是容珮,也未抬头。那双足停在自己身前,分明是一双梅紫色松叶长青缕金鞋。

那人弯下身,轻轻拥住她,温柔道:``姐姐,地上凉,着了寒气便不好了。''

这样的声音,入耳安心。除了海兰,再无旁人。

如懿握住她手起身,二人对坐,如懿方问:``你怎么进得来?''

海兰道:``永琪进宫请安,绊住了皇上。你这里又忙忙乱乱的,我趁机打通了关系,进来瞧瞧姐姐。''

如懿用一枚素银镶珍珠扁方绾着髻,梳燕尾后横贯一枚银箔珠花,雨过天青色衬衣,深绿镶边,暗紫如意襟,显得格外清瘦,简静。\^{}

海兰的泪便滚滚而落。如懿笑:``你真是不大哭的人,却每每都为了我哭。看来我是不祥人。''

海兰忙忙去捂她的嘴,``姐姐说话这般不当心。''她用绢子抹了泪,``我让叶心带了些西季穿戴的衣裳和几床被褥,都交予容珮了。姐姐放心,你的贴身衣衫都是我亲手做的,一应无碍。''她又道:``永璂也好。除了去书房便跟着臣妾,或是在太后眼前,太后也对永璂好。''

如懿念了句佛,``可怜我的永璂,太后若能怜悯,我也安心些。''

海兰忍泪道:``姐姐,我进来一趟不易,皇上南巡回来,把李玉打发了去圆明园当差,跟前的差事一应给了进忠,进忠与魏嬿婉沆瀣一气,更是了不得。我以后便要进来看你,怕也难了。''

如懿知她用意,``你费尽心思进来,必有要事说与我听。''

海兰从袖中取出一枚红宝石粉的戒指,无比郑重地放在如懿跟前,``这是凌云彻死前交给我的,我虽不知他真意如何,但是他曾经告诉我,这是他与魏嬿婉的定情之物。''海兰将戒指对着熠熠烛光,那镀金戒面的里侧,分明刻着燕舞云间的图样。

如懿眼神一跳,``你打算如何?''

海兰急切道:``云是凌云彻,燕子是魏嬿婉,其中深意,不言而喻。魏嬿婉如日中天,一旦登上后位,姐姐就万劫不复。若要东山再起,扳倒魏嬿婉,这是最好的法子了。''

``凌云彻是已死之人,我还要拿他做赌注,搏一个未知么?''如懿轻嗤,目光微凉,``我与皇上积重难返,并非只用一枚戒指就能东山再起。''

海兰盯着她,殷殷切切,``姐姐,我知道你有许多的不甘心。你说得对,嫁了这样一个男人,身膺荣华,可是又能得到些什么呢?但是你想想,你还有我,有永璂,有永琪。姐姐,我看得出来,凌云彻是真心为你,不惜自己的性命。既然如此,再用他一回又如何?他如果看你过得好,九泉之下也会含笑的。''

海兰说得太急,几乎被自己呛到。她伸手取过如懿常用的茶盏正要喝,才发现里头连一片茶叶也无,只是冰凉的白水而已。连盛着水的茶盏,亦缺了---角,露出粉白的底子。她愈加凄然,执着如懿的手,不肯放开。

大约是寒气侵体,如懿咳了几声,缓缓沉声,``凌云彻身受污名而死,我不愿他死后不得安宁,再受一重侮辱。且光凭一枚戒指,未必能动摇魏嬿婉的地位。海兰,罢了吧。''

她眸中晶亮,有不可更改的执拗,让海兰有些怕,然而一想到如懿所受的苦楚,海兰如何能依,``不能罢休!我只要想到姐姐所受的痛苦和侮辱,我便闭不上眼睛不能入睡。姐姐,你被关在翊坤宫里,我在延禧宫又何尝好受?姐姐,我们搏一次,好不好?''

已无太多悲伤,如懿的眉间凝着几许温默与疲倦,``蠃了,我依旧是皇后,依旧陪着这个屡屡伤害我的男人。输了,却要搭上你,搭上永琪的大好前程。海兰,我真的倦了。有生之年,我离不开这个地方,死也要死在这里,那就容我安安静静地过下去吧。''

如懿的话铮铮然,如锋刃直中海兰心闻。海兰分明震了一下,眸中惊痛不已。她嘴唇微张,却什么也说不出来,颓然低首。她喃喃,``姐姐,我不知你竟灰心到这种地步。今日的话,便当我没有说过吧。''

她拂袖起身,将那枚戒指笼于怀中,放入衣襟坠子上所佩的金累丝嵌珍珠绿松石蝶舞梅花香囊,珍重安置。``姐姐若是不喜,便由我暂时保存。这枚香囊是姐姐归来时所落,我一并收着,当作念想吧。''

珍珠,本是如懿喜爱之物,所以每有首饰,大多点缀。她正欲答应,忽而掩袖咳嗽两声,面上泛起几许虚弱的红,似为不施粉黛的她添了一痕新润的蔷薇色胭脂。海兰关切道:``怎么好好地咳嗽起来?宫中阴冷,不如请江与彬来看看。''

如懿连连摆手,``春潮反复,咳嗽也是有的。我要说的便是这个,不必再叫江与彬与惢心为我担忧,未免连累,不许再让他们探知我的事。知道么?''

海兰忧心忡忡,嘴上答应了,却还放心不下。如懿道:``不用管我,好好顾着永琪和永璂。永琪腿上的附骨疽如何了?虽是小病痛,也要上心,江与彬治这个颇有见效,得叫他去看看。''

海兰应承着,心疼道:``姐姐还不知道永琪的脾气?讳疾忌医,也总不当回事。总怕自己弱些,别人就拿住了话柄。如今帮着皇上处理政务,也没日没夜的。叫他换个太医,也总说瞧着原来那个就好,不必费事。''

海兰殷殷叮嘱几句,也不敢多留,微有环佩相撞之声,玎玲而去。

如懿静静坐着,任由天光昏暗,逐渐坠落。

那一晚,深碧暗红的帐幕低垂,如懿居然梦见她的姑母------先帝的乌拉那拉皇后。

梦中的姑母未再老去,或者说,她的心已老,相貌也不再重要。她的青丝中夹杂白发,一身皇后凤妆,气势旗然,不减当年。

身畔已无至亲,与姑母梦中相见,也足以让如懿热泪盈眶。她刚唤了一声``姑母'',乌拉那拉皇后却殊无笑意,肃然凝望着她,``如懿,你的皇后凤冠呢?''

她无言,只能沉默以对。

那一晚,深碧暗红的帐幕低垂,如懿居然梦见她的姑母------先帝的乌拉那拉皇后。

梦中的姑母未再老去,或者说,她的心已老,相貌也不再重要。她的青丝中夹杂白发,一身皇后凤妆,气势旗然,不减当年。

身畔已无至亲,与姑母梦中相见,也足以让如懿热泪盈眶。她刚唤了一声``姑母'',乌拉那拉皇后却殊无笑意,肃然凝望着她,``如懿,你的皇后凤冠呢?''

她无言,只能沉默以对。

姑母却冷笑连连,``无用!当真是无用!戴在头上的凤冠,也会被人生生夺取。你我姑侄,便是这般无用么?连自己的男人都守不住,生生看着人为刀俎,我为鱼肉!生生地成了一个个弃妇!''

如懿跪在乌拉那拉皇后跟前,慘然笑道:``姑母,这个世上有没有抓不住的姻缘?我想我就是吧,哪怕是他的女人,是他的妻子,他却总是带给我一重又一重的失望。我们的姻缘,只是有姻无缘。我曾经很爱这个男人,如今却觉得陪伴他身侧,耗
尽我所有的尊严与心力。姑母,我真的很累。''

乌拉那拉皇后厉声呵斥,``累?一个失败的人,有什么资格说自己累,无非就是做得还不够好!你曾深陷情爱之中不能自拔,优柔寡断不能决绝,所以你才落得这般地步!''

``昔日犯下的种种错处,是我咎由自取!如今困锁深宫,我也坦然。''她仰头望着声色俱厉的姑母,``姑母!情爱和权欲固然是魔障,但清醒更让人寒冷,让我们百死不能超脱的,难道只是皇上么?儿女离散,夫妻背心,皇上也未必好到哪里去!''

姑母的嗓音凄厉划过,是恨铁不成钢的无奈,``便是皇帝让你失望又如何?终究只有一个皇帝,抓住了他,便抓住了一辈子的指望。''

``曾经我也这样想,我曾把一生托付于他,渴望得到安稳的人生,可是等待我的,是一次又一次的失望。''如懿渐渐平静,从容道来,``姑母,我以为只有这个男人会让我失望,后来我才知道,真正让我失望的,是我过了几十年的这样的日子。我
不想再这样了。姑母,我想问问您,您活着的日子,有哪一日是真正的平安喜乐,顺遂无忧?''

乌拉那拉皇后看着如懿,眼底有复杂难辨的情绪,终于默然离去,归于鸿冥大荒。

如懿自惊悸中醒来,抹去额上冷汗,一颗提着的心却放了下来。自此,对谁再无愧欠了。因为她,终究成了乌拉那拉氏又一个弃妇。

\hypertarget{ux7b2cux4e8cux5341ux516dux7ae0-ux9501ux91cdux95e8}{%
\chapter{第二十六章
锁重门}\label{ux7b2cux4e8cux5341ux516dux7ae0-ux9501ux91cdux95e8}}

日子渐渐过成了一口井,抬头望得见庭院上空四方的透蓝的天,却再也走不出去。翊坤宫外总是静得出奇,任谁走过都会不自觉地缓下脚步,怕沾染上什么不祥的东西。大凡的人与事都改变了方向,唯有游荡于宫巷的风不会,它依旧会在某个静
夜,忠诚地传来宫苑里丝竹笑语之声。朝喧弦管,暮列笙琶,那是另一重醉生梦死的繁华,与她无关。

永夜里,她很少能安然入睡,亦不太流泪。大约这一生,已经为了不值得的人不值得的事伤怀太多,以致晚来伤心,却不知该如何泪流。

她只是一径思念着,思念着永璂、海兰、永琪与惢心。家中已无他人,乌拉那拉氏的亲族都是远亲,而额娘与兄弟都已相继谢世。她真正成了一个无家可归之人。而这让自己存活了一世的寂寂宫苑,又哪里算得是自己的家呢?

不知不觉间,她便添了一种症候,起初只是声嗄咽痒,烦梦不宁,时常梦见亡故之人,渐渐惊悸咳逆,偶见血痕。好容易延请了太医进来,江与彬一搭脉,已不觉惊愕当地。

她见他如此,已然知道不好,平静道:``你说便是。''

江与彬红了眼睛,``是痨症,症候已深。怕是\ldots{}''

如懿含笑,``不必对人说,拖得一日是一日。''她转而担忧,``永琪有旧疾,是你所善医治的,也不知他如何了。''

江与彬欲言又止,``五阿哥吉人天相,身边不缺名医圣手。娘娘还是顾及自己要紧。''

如何顾及呢?内务府的供应早已是断断续续,四季衣裳的周全都是凭旧衣度日,或者是太后惦记,遣人传递些东西进来。幸得容珮生性坚强,一切都尽力平服。而有两样东西,却是一直未曾断过的。

大约知道如懿每日素衣简髻,于佛龛前静心念经,也当作忏悔之道。每隔三日必有新鲜花卉送进礼佛,春日的玉兰,夏日的白荷,秋日的素菊,冬日的梅花,四季相续,不曾断绝,也将死气沉沉的殿阁略略添置几分鲜活生气。另一则是楂香,虽不是最名贵那种,但也洁净无烟,每月月中,必定送进。于是佛龛前紫檀雕西番莲流云纹平头案正中摆着一只青瓷香炉,左右设了一对天青玉净瓶,供了四时鲜花。

这样的眷顾,不过是因为永琪的惦念。他深得皇帝爱重,到了三十年十一月,已被封为荣亲王。皇帝诸子之中,唯有永琪最先封亲王,皇帝又对其深寄重望。如此形势,便是登临太子之位,也是指日可待。

这般荣宠恩深,便是关在翊坤宫内,亦能从喜乐声中探知一二。菱枝喜极而泣,``若是五阿哥继承大统,娘娘离开此处也有望了。''她掰着指头,``五阿哥颇具孝心,若是肯尊重娘娘,等来日,娘娘还可以是母后皇太后呢。''

容珮却摇头,``菱枝,你不可胡言乱语,为娘娘招来祸患。''她换好清水,仔细供好新送来的白菊。那菊花香气甘洌,隐有清苦气息。她隐然有忧色,``娘娘,若是五阿哥对您关切如初,那么可以送来日常所用的定会是五阿哥,而不是如今不太理宫中事的太后。''

如懿对着日光翻过一页经文,停下来道:``你想说什么,便说吧。''

容珮道:``娘娘,五阿哥送来花卉与檀香,可见他足有能力照顾您日常。可他避而取其轻,大约是因为送花卉、檀香,既可让娘娘潜心礼佛,又向皇上表明态度。''

如懿道:``如此折中,也算两全其美。''

容珮道:``是两全其美,既全了些微孝心,也让皇上知道,他是力赞娘娘静心思过的。''

如懿清眸扬起,``容珮,不许再言永琪之事。他自小争气,费尽多少辛苦才得皇上器重,荣膺亲王之位。''如懿笑得欣慰,``我这个做皇额娘的,想起来便觉得高兴。若是因为我而牵连他,那万万不可。''

容珮不敢再言,其实她的抱怨并非无谓。十二月天寒地冻,太后送来的炭火并不多,前后不继,每日仅能点一个小小的火盆度日,便是将大毛衣裳都裹在身上,也根本不能驱走严寒。只得容珮和菱枝辛劳,烧了热水灌汤婆子,三人围坐着,冻得瑟瑟发抖。比起夏日,这又还不算差了。因为京中的酷热,殿阁中没有冰供,也无艾草熏房,热得痱子四起,蚊虫嗡嗡。那痱子本易冒尖,隔着衣衫磨破,又加之汗液,实在痛疼难当。这样想来,冬日尚能加衣,夏日却不可剥皮了。

倒是菱枝笑着上来凑趣,``皇上封了五阿哥为荣亲王,荣耀显赫,真是个好封号呢。''

如懿正欲笑,心中咯噔一声,莫名觉得不祥,那笑便僵在了脸上。

荣亲王,荣亲王,这个称谓怎的这般耳熟。她蓦然心惊,曾经顺治爷的董鄂皇贵妃,所生的四阿哥深备荣宠,顺治爷一意欲立他为太子,先封荣亲王。啊,那个孩子,便是在受封亲王之后,夭折于襁褓之中了。

纷杂的记忆纷至沓来,逼得她心惊肉跳,手中一松,佛珠便从指间跳脱,散了满地。她急忙遏制住满心杂念,伏在地上一颗一颗捡起散落的佛珠,道:``容珮,去点上檀香,我要为永琪祈福。''

到了三十一年正月,香花与檀香,都停了供奉。如懿深觉不安,还是容珮向守门的侍卫打听了,才知荣亲王永琪旧疾发作,顾不上这些了。

如懿霍然站起,向着门外急切道:``告诉愉妃,告诉荣亲王,请太医江与彬去看,快去!江与彬精通此道,他可以医好荣亲王。''

此去再无消息,时隔两月,翊坤宫的门却开了。菱枝惊惶不定,以为厄运再度来到翊坤宫。而她们,真的再经不起什么了。进来的却是进保和海兰身边的叶心,叶心泣不成声,``娘娘,小主伤心得晕厥过去了。荣亲王\ldots 荣亲王快不成了。''

进保在旁道:``荣亲王沉疴已重,愉妃小主哭求了皇上很久,皇上才允许娘娘去见荣亲王最后一面。''

如懿只觉得足下发软,险险跌倒,她失声呼道:``怎么会?怎么会?永琪还这般年轻\ldots{}''

她的心底像是被钢刀铰刮,舌头一阵阵打结,连句完整的话都说不出来。

幸好软轿己经备下了,进保与叶心半扶半搀将她挪了上去,急急奔往重华宫中。如懿心急如焚,轿外热悉的红墙绿芜,琼林玉殿,都成了流水里的倒影,匆匆掠过。

因着永琪病重,正月里便挪进了重华宫居住。皇帝为皇子时,曾在毓庆宫居住,婚后移居在此。自从皇帝登基,作为肇祥之地升为宫,定名重华。皇帝将永琪安置此处养病,一来方便生母愉妃看顾,二来亦可见皇帝对永琪的重视。

如懿凄凄惶惶踏进西殿,永琪销在床上,已然枯瘦如柴,昏昧不醒。殿中有浓烈的肌肉腐烂的气味,夹杂着脓血的腥气和草药气味,熏人欲倒。还是侍奉的妾室乖觉,焚起薰香细细,一丝---缕,沁入心腑。帘幔低垂,春寒侵人。泪意蒙胧间,恍然还是风姿秀致、英挺如松的少年郎,唤她``皇额娘''。

如懿的泪便落了下来,抓住永琪的手。―年不见,不想他已然瘦弱至此。太医们已然退下了,唯有一个一直侍奉永琪的侍妾还留在身边照拂。如懿见她长得清丽动人,我见犹怜,不免多看了一眼,问道:``永琪何至于此?''

那侍妾跪下身道:``娘娘有所不知,五爷一向好强,不肯落于人后,为了替皇上分忧操持国事,常常是夜以继日,不得安枕。自从得了附骨疽,他怕耽误国事,一直忍痛不肯言,或是找太医开些方子潦草对付,以致毒气深沉,结聚于骨,肉腐骨败,溃烂淋滴,终致气血耗尽。''

如懿斥道:``你既此时还留在永琪身边,必是素日得宠的。既然王爷病得厉害,为何不告知福晋,上报愉妃,请太医好好救治。我也曾叮嘱偷妃,太医院的江与彬素擅此道,为何不请?''

那女子掩袖惊惶,``江太医?什么江太医?妾身从未听过。''她凄然惨笑,神色古怪,``这是命!娘娘,这都是命!做下的孽在这里,报不到自己便是报在儿女身上,真是可怜。''她痴痴笑着,状若癫狂,旁边的侍女忙拉住了她,``芸格格,您可
别伤心坏了说胡话,''说罢,半拉半扯地将她带了出去。

如懿看着永琪,顴骨凸出,面色赤黄,瘦脱不成人形。她内心大恸,也不知永琪何时会醒来,不禁悲从中来,泪水潸然而落。

永琪在昏昧中含糊地抓住她的手,呼道:``额娘!额娘!我对不住皇额娘\ldots{}''

如懿痛至锥心,惨声道:``永琪!皇额娘在这里,永琪!''

永琪额上青筋暴出,拼命摇着头,吃力地睁开眼来。他定睛看是如懿,先是惊惶,继而羞愧,掩面道:``皇额娘,是您来看我。''

如懿惊痛满怀,哭道:``傻孩子,为什么这般要强,讳疾忌医!若是早些请江太医来看,也不会如此。''

永琪目中一旋焰火骤然亮起,他沉痛难耐,``皇额娘,是我没有听您的话。''他的眼角沁出一滴浑独的泪,``皇额娘,我知错了,我真的知错了。''

如懿握住他的手,柔声道:``好孩子,你是皇额娘一手抚养长大,你我母子,何来错不错这样的话?''

永琪的泪汹涌而出,``我落到今日,全是因为太过要强,不肯听从皇额娘所言,用江与彬医治,以致回天无力。不信皇额娘,是我最大的错处。''
那侍妾临去时添的大约是苏和香,那香气浓郁经久,有芳香除秽之效。香烟袅袅,自芙蓉翠叶白玉炉里飙出。那香气太过沉郁,夹杂着满股药气,熏得人满眼晕眩。

她逐渐忆起,自从永璂长大,自从永璂得皇帝亲自教导,永琪望着自己的眼神,便再无幼时那般清澈。是她疏忽了,还是过于相信曾经的母子之情。她一直回避着,回避着和永琪之间某种暗涌的可能。

永琪满面是泪,``皇额娘,我知道额娘伤了您的心。她借着您的名义杀了凌云彻,所以您对她不如从前亲密。凌云彻是您的心结。儿子也知道,若不是額娘与皇额娘一直交好,儿子也不能养在您的膝下,视同嫡出。''他喃喃,望着湛青蓝帐顶上绣
的百蝠晖春图,最吉利的花样,讨着好口彩。富丽热闹的团花用密密实实的彩线绣成,比着永琪的枯黄委顿,越发眼花缭乱。如懿只觉得太阳穴突突地跳着,有些晕眩,永琪还在说着,``皇额娘,我自己最明白不过,我只是庶子,若不是大哥二哥早逝,三哥四哥平庸,皇阿玛的眼睛根本看不到我。另一层,我还是占了永璂的便宜,
他虽是嫡子,但比不得永琏和永琮尊贵,年纪也小。若他大些,皇阿玛便会顺理成章立了他为太子,我哪里还有一丝希望?''

如懿的舌尖一层层发木,``所以,你是为着太子之位,忌惮了永璂,也疏远了我?''

``皇额娘,我不能不怕,我只是一个庶子,哪怕养在您膝下,也比不得永璂。我也知道,永璂不如我幼时聪慧,可他毕竞是嫡子,皇额娘\ldots{}''他眼中的火焰逐渐冷却,悲伤中含着无尽的怔忡与茫然,仿佛是迷路的孩童,``我知道自己做得不对,
皇额娘困在翔坤宫衣食不周,我也未曾尽力照拂,只敢送去香花与檀香,略表关怀,也向皇阿玛表示并无异议,支持皇额娘闭门思过。皇额娘,儿子是不孝,可儿子也知道,因为您的失宠落寞,永璂才不会和儿子有争锋之地。直到皇阿玛封儿子为亲王,儿子的心才放下,可是儿子无福\ldots{}''

她的泪,滚烫地灼烧着脸庞,``永琪,你便为了这一时的忌惮,认为江与彬是皇额娘的人,所以宁可用别人也不用他,是么?''

他死死地盯着帐顶,重重地喘着气,``皇额娘,我并不是有心疏远您和永璂,我只是不敢完全相信,所以只好远着您。永璂是您的亲生子,您要扶持他为太子,要我辅佐也是人之常情。儿子也是不得已\ldots{}''他的面上闪过一这惊惧,``儿子自小在宫里长大,许多事便是没有亲眼见过,也多少有些明白,孝贤皇后的永琏与永琮死得不明不白,三哥永璋无缘无故便不得皇阿玛宠爱,四哥的野心,九弟十弟的英名早夭,还有五妹璟兕,皇额娘,为了储位,为了宝鼎龙座,儿子不能不防\ldots{}''

他的手渐渐凉下去,像冬雪触尽后的冰凉,即将消弭在初春的黄昏。榻前供着十数火盆,三月初的天气,还是寒浸浸的。盆中小小的火苗,一簇簇跳跃着,如幽蓝阴魅的舌,舔蚀不定,晃出一团团暗红的光晕,却没有丝毫的暖意。

那种冷,从骨缝里咝咝冒着,难以抵御。

如懿捧着他的脸,轻轻抵住他的额头,``永琪,你思虑得太多了。你是皇上的长子,又文武双全。本朝有立贤不立嫡之说,永璂更是年幼,如何能与你相较?你若能安安心心,何至于今日\ldots{}''
永琪攀着如懿的手臂,如幼时一般依偎着她,``皇额娘,儿子错了,儿子不该疑忌您要扶十二弟为太子,疏远了您。儿子这段日子病着,总想起昔日在皇额娘膝下的日子,过得安心,踏实。''

他的气息渐渐微弱下去,微弱下去,死水一般毫无波澜,终至令人惶恐的平静

窗外,满眼新绿,染遍林梢。而怀中年轻的生命,已然停止了呼吸。

她静静地抱着永琪,浑然不觉得室中浑浊难忍的气息在遂渐淡去,就如怀中的身体,在逐渐变轻。

那是生命,在缓缓剥离。

也不知过了多久,黄昏的夕阳如溶了的血水,肆意布满了整个天空。余晖斜斜地照进内室,勾勒着花梨木床架上一痕一痕缨络的影子,床棱与顶架上的雕花都是用金粉一笔笔描成的,是花正好月正圆和合长久的故事,燕是双飞燕,人是照花人。一
花一叶,---蝶一莺,花香脉脉,碧枝如丝,在微光里像浮涌的金浪,迷得人睁不开眼睛。

她别过头,才见皇帝站在琉璃帘内,不知何时进来的。他的身后是廊下一排轻红纸灯,不过很快,都要被换成素白了。

皇帝眉头紧蹙,脸上全然是萧瑟的哀恸,双手轻轻顫抖。

如懿乍见他,还来不及起身,泪已落下,``皇上,永琪没了。''

皇帝的身形是僵死的,一点一点挪进来,他的声音没有一丝温度,``永琪临终的话,朕听见了。''他忽然盯住她,扬起手中一柄打开的湘妃竹洒金折扇,狠狠从她的耳畔直劈到了顴上,``这是朕最后一次打你。''

那折扇原是消暑用的东西,玲瑰小巧一把,皇帝常自携在身边,自取清凉。此刻他落手极重,来得又急又狠,居然连洒金扇面都刮破了几折。如懿倒伏在地上,听得有无数细虫在她头颅里死命扎着,耳边嗡嗡乱响,颊上只是发木。她没有反应过来,只是盯着他微白的双鬓,呵,那颜色,像极了除夕夜中纷碎的落雪,像未亡人眼睛,淡白,死沉。她老了,他也老了,都经不得这样沉重的伤痛,而且,是最优秀的孩子。

足有一年不见了呵。

这样慌促的相遇,脸颊上剧烈的肿痛,他却连用手打她亦不肯。她却在依稀的茫然中辨别着他的样子。她清楚地记得,脑海里的,那最后一次相见时,他的模样。他有一点点老,虽然才一年,衰老却如黄昏的阴翳,不可抗拒地到来。

她一直以为,那样的僬悴支离,是她一个人的事。却不想,他也在经历。

真的,真的很想忘记。可在佛音的静谧里,才发觉刻意地忘记是一件很困难的事。那些藏在波澜不惊的浮沉往事之下的,一阕诗词,一种声音。清晨的白露,红樱的绽放,细枝末节,零碎琐屑,都会在对着他的时候汹涌而出。

迎来的,却是迎面两掌。

她的错处,大概是数不胜数。所以并不辩白,只是定定望住他,一双眼眸格外地黑。

皇帝颤声道:``你做了什么?逼得永琪连你遣来的太医都不敢用。你说,你为了永璂,可是暗地谋害了什么?''

她静静道:``皇上,您知道的,臣妾从未向您求取过永璂的前程,从来没有。''

``你嘴上保举永琪,暗地里却阴谋诡害!''他骇然惊痛,热泪纵横,``永琪是朕最出色的儿子啊!''

皇帝正说着话,外头福晋们的哭声嘤嘤响起。方才的妾侍不知从何处冲出来,跪倒在皇帝身前连连叩首不已,厉声道:``皇上!荣亲王生前郁郁难安,不敢接近翊坤宫娘娘。若非如此,荣亲王得翊坤宫娘娘多年养育,怎会这般回避?定是在翊坤宫娘娘处,王爷见了不该见的,听了不该听的。''

有侍卫上前拉她,她哭号难抑,如何肯去?皇帝问:``你是谁?''

还是永琪的福晋答道:``回皇阿玛的话,她是荣亲王府的格格,王爷生前最宠爱的侍妾胡芸角。自从王爷卧病,也是胡氏侍奉最勤。''

芸角呜咽道:``皇上,妾身本不该说这样的话。可王爷即使在病中,也念叨着数位兄弟早夭的惨况,对此郁郁难安,生怕自己有朝一日也不能安稳。妾身是妇道人家,本不明白王爷是什么意思,直到额娘来探望,提到翊坤宫娘娘举荐江与彬江太
医,王爷口中答应,却一直不肯让江太医医治,妾身疑惑追问,才知王爷心思。''她瞪着如懿,哭得声嘶力竭,``王爷,您别丢下妾身,妾身这便跟着您去了!''

她说罢,一头撞在墙上,飞血四溅,似开了一树艳艳桃花,香消玉殒。

皇帝连连冷笑,``好!好!好一个皇额娘,好一个翊坤宫娘娘,连自己的养子都对你心怀畏惧,你自己做下的事情自己明白!''他喝道,``格格胡氏殉主,以侧福晋之礼,好好葬了。''他又向着永琪福晋道,``愉妃伤心不能起身,荣亲王的丧事,便由你和内务府好好主理,皇贵妃也会来照应。''

他没有再理会如懿,任由她孤零零站着。没有人驱赶她,也没有人理会,只是远远地避开她,哭天抢地着开始忙碌起来。她是一个孤清的影子,那有什么要紧?可是她曾经引以为傲的孩子,居然死在了对她的疑忌上。连那个胡芸角,莫名其妙冲出来的胡芸角,都指着那一丝疑惑,可以如此咬定她。

多少年的心血煎熬,只落得如此下场。天家深苑,母子情分,原来是如此呵。

她欲哭无泪。

永琪这般心思,怕是连海兰也不知晓吧。她立在那里,看着红色的宫灯被粗暴地扯落,换上白纸灯笼。素白的雪色一点一点蔓延开来,渐渐成了堆雪天地。

她迟钝地被挪上了软轿,叶心一壁哭一壁陪在身侧。如懿听见自己的牙齿在发抖,``这个胡芸角,査査她的底细。还有,査査为永琪侍疾的太医。''

叶心忙乱地点着头,来不及说什么,软轿便已将如懿送了出去。

如懿是在长街上挣扎着下来的。

她的手心全是潮湿的冷汗,涔涔地洇湿了掌心的每一条细纹。她的膝盖酸软如绵,她半倚着危危红墙,那种虚脱的无力感排山倒海吞袭而来。

不,她一点也不想靠着这堵临渊般的红墙。她泪流满面,说不出一句话,一掌,
又一掌,重重地拍在墙上。以掌心的刺痛,软弱的力量,来撼动这一切。她想出去,想出去。她这一生,从未如此刻,发疯般地想要出去。

她心爱的孩子,心爱的男子,她的青春,她的来日,全部折堕在了这里,成了红墙之下的暗沉的余灰,琉璃瓦上点缀的浮光。

那是她的半生呵!

她精疲力竭地倒下,无声地哽咽。末了,还是叶心强扶了她进了翊坤宫,再度重门深闭,不见来路。

\hypertarget{ux7b2cux4e8cux5341ux4e03ux7ae0-ux65e0ux5904ux8bddux51c4ux51c9ux4e0a}{%
\chapter{第二十七章
无处话凄凉(上)}\label{ux7b2cux4e8cux5341ux4e03ux7ae0-ux65e0ux5904ux8bddux51c4ux51c9ux4e0a}}

后来的事,如懿便不能知了。她总在寂寂的光阴里想起永琪曾经天真无邪的笑靥,他在她的膝下长成的每一件细微琐事。那是她未能保全的他的纯真,毕生的大憾。而永璂,不知他的来日,又是如何。庭院深锁,再无人轻易打扰,连乌雀亦知趣,不来打搅这沉寂深宫。佛堂外的日影每一日朝升暮落,循环往复。虽然单调,却也让人觉得安稳,这般日复一日,光阴迅疾,飞曳无声,走得清冷、寂静。

天气渐渐热起来,到了七月里,紫禁城的暑气一浪接着一浪。太阳一出来,过不了一个时辰地皮儿都烫了。这时节连御花园的花花草草都晒得蔫蔫的,唯有永寿宫里的石榴开得如火如荼,仿佛碧绿的湖水上燃着殷红的云彩,几乎要迷了人的眼睛。

一溜儿的廊檐底下,碧水琉璃瓦映着金砖墁地,纤尘不染,唯觉金灿灿的日光晒下,连永寿宫的每一条砖缝透着金迷绚丽的气息。

嬿婉坐在西暖阁的榻上,一屋子莺莺燕燕围着,极是热闹。虽是刚产下十七阿哥不久,嬿婉倒丝毫不见胖,反而神光明艳,更甚于一班新入宫的年轻嫔妃。她见众人只是围着自己,略略咳了一声,轻笑道:``天气这么热,难为了妹妹们还晨昏过来请安,倒叫本宫生受不起。''

她一说话,众人都静了下来。为首的庆妃资历最长,便先笑道:``皇贵妃主理六宫,位同副后,咱们来请安本是应该的。何况皇贵妃刚涎育了十七阿哥,咱们姐妹怎么说也要来给皇贵妃道喜的。''

晋嫔亦道:``天气热怕什么,规矩总是要守的。再说,咱们也想看看十七阿哥呢。''

庆妃满脸艳羡,``听说皇上隆恩,准许皇贵妃亲自养育十七阿哥不说,还定是每日都要来看十七阿哥的。''

晋嫔笑着抚了抚鬓边的珠翠,斜睨了庆妃一眼,``皇贵妃荣宠,自然是旁人不能比的。''

嬿婉恬然微笑:``晋嫔妹妹说笑了。皇上许本宫亲自抚养十七阿哥,不过是因为本宫除了料理后宫琐事之外也是闲着,所以让本宫带着孩子打发时间罢了。''

嫔妃忙笑道:``皇贵妃执掌六宫每日辛苦,哪里会闲着,到底是皇上体恤娘娘和十七阿哥母子情深,不忍叫娘娘母子分离罢了。''

几位贵人亦笑:``可不是?听说十七阿哥十分可爱,皇上都喜欢得不得了呢,口里心里都是念着。''

嬿婉微笑;``乳娘,既然各位小主都来了,把十七阿哥抱出来,见见各位吧。''

一时乳母抱了十七阿哥出来,十七阿哥犹自睡着,大红夹银丝薄被裹着小小白胖的身子,一身小衣裳上用金钱绣着富贵长命连身纹案,蹬了双虎头鞋。小阿哥胎发间凑出两个可爱的旋涡,粉嘟嘟的小脸泛着娇红,睡得正香。

庆妃将一枚金镶玉锁放在婴儿胸前,笑道:``这块金镶玉锁还是妹妹入宫的时候最贵重的陪嫁,妹妹想着,这样的爱物儿总是要给最有福气的孩子才好。妹妹看十七阿哥天庭饱满,地阁方圆,最是有福气的,若皇贵妃不嫌弃,就收下妹妹一点心意。''

嬿婉满脸含笑,``既是妹妹的心意,本宫却之不恭了。''

庆妃见嬿婉收下,笑得如花朵儿一般。香见坐在一旁,冷冷道:``皇贵妃的孩子自然是最有福气的。只是皇上的嫡子十二阿哥在,谁的福气都是比不上的。''

嬿婉正得意间,一瓢冷水兜头浇下,微微不豫。只碍着容嫔深沐恩宠,连皇帝也格外厚待,却也含笑不语。

晋嫔却不服气,冷笑了一声道:``皇上建了宝月楼给容嫔住着,一应都是按着寒部的规矩来,难怪容嫔你到了今日还分不清咱们的礼数。乌拉那拉氏既然断发被囚,被皇上褫夺了一切封号、册书,形同废后,她的儿子怎么还能算嫡子?放着从前已故的两位太子爷不说,自然是皇贵妃的阿哥最贵重最有福气了。''

香见神色清冷,看也不看她一眼,只缓缓道:``你也知道是形同被废,那就是还没有废后了。皇上一日没下废后的诏书,翊坤宫主子就一日还是皇后,十二阿哥也是名正言顺的嫡子。''

晋嫔笑道:``皇上既然把乌拉那拉氏关在了翊坤宫再不相见,废后也是迟早的事了。''她一脸恭维看着嬿婉,喜滋滋道,``皇贵妃儿女双全,个个都得皇上的欢心,可见皇贵妃的福气在后头呢。嫔妾听说翊坤宫那位病了,怕再熬下去也不长了。''

香见一震,仿佛是不可置信一般,盯着晋嫔道:``你说什么?''

晋嫔看见她眼神幽冷如锥,不觉也有些害怕,嘴上却不肯服输:``我说翊坤宫的福薄命短,也不过这几日了。''

嬿婉温言道:``好了,空口白舌说这些话,本宫可受不起,也不敢听。若是传到了皇上耳中,还以为后宫妄议,只怕要怪罪,妹妹们还是别说了。''

香见霍地站起,蹲了一蹲便算是告退,径自走了。

庆妃皱眉道:``瞧容嫔的样子,这样嚣张,真是半点规矩都不要了。''

嬿婉虽然不悦,面上去依旧微笑温婉,``皇上一向都不与容嫔妹妹讲规矩,也怪不得她。''

晋嫔轻哼一声?:``她以为有皇上的宠爱就为所欲为了么?膝下无子便是没福,那怕是有了子息,也不过和死了的淑嘉皇贵妃一般,上不得台面。''

嬿婉不觉莞尔,忽然瞥见人群中并未有颖妃的身影,口气便有些冷:``怎么?颖妃还没来?''

座中有一二蒙古嫔妃,便解围道:``颖妃娘娘身子不适,所以不来。''

春婵明白自己主子心中的不快,便道:``颖妃小主不来,也总该送七公主来,到底十七阿哥是七公主一母同胞的弟弟,也该来看看。''

那蒙古嫔妃似笑非笑:``七公主孝顺,听闻颖妃娘娘不适,便要亲自陪伴,不肯前来。想来十七阿哥与七公主一母所生,必定能姐弟连心,一切明白。''

嬿婉胸口一闷,想要说什么,到底忍耐了下去,换作温柔笑意:``那也是。颖妃替本宫养育七公主,着实辛苦。的确得保养好身子才是。''

众人言笑晏晏,再也不提起此事。嬿婉看着雪白粉糯的孩子,那样天真的笑脸,也抹不去心中的不快。与自己言语对答的也不过是蒙古嫔妃中的小小贵人,亦无多少谦卑神色。她们所仰仗的,无非是颖妃。而颖妃为蒙古嫔妃之首,多年来不与自己亲近,对翊坤宫也不过礼数而已,所仗的,不过是蒙古诸部的势力,才能隐隐与自己分庭抗礼。她才能以无子之身居妃位,养公主。

而这家世,正是嬿婉所最缺憾的。

嬿婉轻轻握住了拳头,乌拉那拉氏早已落寞,她这个皇贵妃,必得牢牢握住这后宫权柄,压制诸人,才得安生。她轻轻吐一口气,千辛万苦得来的,怎可再被轻易动摇呢?那怕是垂死之人,都有东山再起的可能。唯有生息断绝之人,才是最让人放心的。

看见坐在轿辇上,心急如焚,一味催促着抬轿的太监:``快些!快些!''她素来性子冷淡,又不屑与宫中嫔妃来往,今日如此急促,连伺候她多年的阿吉都暗暗纳罕。

阿吉赔笑道:``小主好歹说句话,您急着要去哪里?''

香见直视前方,``翊坤宫。''

阿吉吓了一跳,连忙跪下拦在轿辇前,``小主三思,翊坤宫去不得。''

香见简短道:``去得。''

阿吉仰脸看着她,``皇上说了,去不得。谁去了就陪皇后在里面待着,再出不来了。''

香见看也不看她,示意小太监们放下轿辇,自己走了下来便往前去。阿吉登时吓得呆了,愣了一愣才醒过神来追上去。

香见足下极快,匆匆到了翊坤宫门口,便见门庭紧闭,灰尘满地,心中不由一酸,便伸手去推门。阿吉忙劝道;``小主,没用的。您忘了,这翊坤宫的门是从里头锁住的。''

香见意外之余也顾不得那么多,径自推门而入。阿吉犹豫片刻,忙闪身跟进去,慌慌张张关了大门。香见走进翊坤宫,只见院子里草木茂盛,倒依稀还是旧日的样子。只是四下里寂静异常,在这夏日底下,倒显得格外冷僻。香见心里担忧,便直直往里走,到了殿前,却突然怔住了。原来殿前的石阶下,却是海兰直挺挺跪在那里,身边还跟着一个太医和一个宫女。

香见入宫五六载,见到海兰的时候并不多,只是重大的年节时才在人群里远远地望见一眼,所以也不熟络。海兰也不知跪了多久,身上都被和湿透了,整个人摇摇欲坠,却只是咬着唇硬挺着。

香见有些不忍,屈膝请了一安道;``愉妃,天气这么热,你这样跪着,当心中暑。''

海兰略略点了点头,眼睛却只望着门口,半分也不肯挪开。她哀哀泣道:``姐姐,你已抱病,为何不让江与彬好好诊治?哪怕病得重了,只要你肯治,也能久些。也省得惢心日日为姐姐病情悬心。''

香见俯下身来,不肯置信,``真的病得那么重么?''她扬声,``皇后,只要你愿意治,我去告诉皇上,皇上再狠心,总会听我的。''

海兰闻声抬首,感泣不已,``是,是,姐姐,皇上会听容嫔的。''她说罢,哀恸不已,``姊姊,你见一见我好不好?永琪已经死了,只剩下我和永璂。姐姐,你若不好好活着,我与永璂还有什么可以寄托?''

里头久久寂寂无声,终于,有女声响起,``海兰,你来看我,是自陷险境之中。真的,不必了。''她的声线温婉而脆薄,``海兰,见与不见,只要你善自保重,彼此就是心安。''

果然,再过了许久,终究还是无人出来。

香见抬头,一小方碧澄的蓝天,被四围宫墙隔出。天上的白云大片大片被朗风吹着,消散得无影无踪,单空余一片孤零零的天空,蓝得空旷而孤独。日颖在暗红色的檐下转移,庭院内寂静无声。

香见黯然地想,这个宫里唯一肯对她好些的人,也终究快要离开了吧。

这般自生自灭,与世隔绝。眼见窗外四壁,薛梦凌霄自由无拘地爬了满墙,荫荫含翠。庭院中松桧盆景因着无人修剪,越发茂盛恣意。夹杂着十数建兰,翠紫芸草,青葱郁然。僻冷之地,也有天机活泼。也好,人已无生气,草木生机也是好的。

苍苔深浓,踏足的却是皇贵妃魏嬿婉。她并未带许多人,只有贴身的春婵并几个小宫女,手里捧着各色衣料首饰和日常所用的物品,并一支儿臂粗的雪参,以红锦裹住,供在红纹木盒中。

嬿婉很是客气,像是常来翊坤宫中,极为熟稔。她全然不理会容佩的扬眉怒意,径自在暖阁榻上坐下,软声细语,``听说姐姐病了,我叫人找了支上好的人参来,给姐姐补身。''

嬿婉说话间,一展春水罗翠色的百子缂丝对襟云锦袍。浅金桃红二色流云纹滚边,每一滚都夹了玫瑰金丝线,行动间闪闪熠熠,如艳阳高照下灼烈艳艳的金色葵花,炫目动人。她盈盈坐着,鞋尖点着地面,晃着鞋面上拇指大的琥珀,以细细米珠围成日月山川之形。比之足上的华丽,嬿婉严妆而来,云鬓高鬟以碧玺、碎玉累金丝缠成连绵不断的点翠牡丹花钿,映着日光耀目生辉,两侧横一支心攒翡翠七尾风流苏,凤嘴里衔下长长一串珍珠红宝流苏,更显得无比尊贵艳丽。

如此清艳华贵,嬿婉的唇角却蕴着一丝浅笑,温和有礼,可见这位宠冠六宫的皇贵妃是如何平易近人。

如懿抱病已久,懒惰说话,那痨症又是极耗人的,磨得她身形消瘦,不施脂粉的容颜平淡至憔悴。但她还是未失仪容,云髻低绾,一丝不乱,佩素金扁方,五瓣梅花银步摇,发髻上缀以明珠数颗,着玉版白暗纹熟罗袍,绣着一色莲青菱花镶边。她有着沉沉的大眼睛,唇色微紫,眉眼轻扬,目光平和。

她并不介怀嬿婉入内以来并未施礼,也的确,她如今的尴尬身分,用什么礼数都不太妥。如懿淡淡道:``不是很要紧,难为皇贵妃来一趟。''

嬿婉看着她并不因名分的差落,而轻慢自己,心底微涩,无端气馁了三分。她振作神气,不知怎的,嘴上便尖刻了三分,``是么?症后既轻,想来也不碍了。那便要恭喜姐姐,皇上定当愿意见到姐姐康健宁和,如春松茂兰。''她顿一顿,似想起什么,轻轻按着自己的胸,不胜柔弱,``哎呀!姐姐莫怪。如今我怎么称呼您呢?您没有皇后册宝,这句娘娘是唤不得了。您年长为尊,我便唤一声姐姐了。''

如懿定定看她一眼,忽而浅浅笑道:``你喜欢唤什么便是什么。''

嬿婉见她不怒不恼,一股暗火腾地跃上心间,娇滴滴举袖掩着红唇道:``也是。姐姐原本贵为皇后,如今皇上收回皇后宝册宝印,也不曾真正废后,这妻不妻妾不妾的,真真是尴尬呢。''

如懿淡淡``呵''一声,``是啊,妻不妻妾不妾的总不成体统,何时皇上会再立皇后呢?''

嬿婉被诘住,见如懿不动声色,嘴上愈加犀利,``姐姐,或许皇上是故意历练,想让您低个头,或许皇上一高兴,又赏了您皇后的尊荣呢。说来我与姐姐都是妾侍出身,姐姐爬得高点儿,我站得低点儿,都是一样的人,姐妹一场,我替皇上说句体己话,指不定还有来日呢。''

如懿目不微瞬,道:``皇贵妃笑言了,我与皇上,此生都不会再相见。''

``是么?虽然五阿哥盛年早逝,让皇上恼了姐姐,可听进忠说起,七月七日之夜,皇上从长春宫归来,行经翊坤宫,居然驻足片刻,可是姐姐重见天日有望了。''

呵,如懿笑意轻浅,``原来皇贵妃贵步挪动,是为此事。''她轻轻``咦''一声,``皇贵妃身膺无上荣宠,居万人之上,为何此等小事,也要挂怀?''

嬿婉语塞,旋即笑得温和,``皇上旧情难忘,姐姐难道不知?对着孝贤皇后语慧贤皇贵妃,也是如此。''

``皇贵妃所言,是皇上对死去之人恩深义重,对活着的人却不加怜惜么?那么冷落如我,皇贵妃也这般着意么?''如懿抬了抬眼皮,懒懒道,``我所失去的,你都一一得到。我所未曾拥有的,你也全然不失。皇贵妃乃是幸运之人,若还是要对我锱铢必较,实在无谓。''

``不是无谓,是凡事应该周全。这也是当日在姐姐身边,妹妹学得的一点皮毛。''

如懿舒一口气,抬起头静静凝视着嬿婉。她端坐着,嘴边衔一丝似是而非的笑意,好整以暇地打量着自己。真是看不出,眼前高贵得毫无破绽的女子,竟会是当年小小的宫女,含悲忍辱,一意飞上枝头。

嬿婉大概是不习惯如懿这种看人的目光,便道:``姐姐怎么这么看我?''

如懿和缓微笑,目色澄澈,``看你的神气,想来过得很好。据说你又生了新的孩子,可见宠眷不衰。这个皇贵妃,想是做得顺遂。''

不过两个月前,嬿婉又生下了皇帝的第十七位皇子,取名永璘。那是皇第五十六岁上又得的儿子,疼爱得不知怎么才好。而彼时,嬿婉也逾四十,可见皇帝的宠爱不衰。作为生母,嬿婉自然备受荣宠。

什么都不缺了。宠爱、位份、儿女、荣华和众人艳羡而恭顺的目光。唯一所缺的,只是一个皇后的名位。却偏偏,还落在眼前这个生气全无的女子身上。她如何能不怨,不急?

然而面上,嬿婉却气定神闲,``瞧姐姐说的,能有什么好不好的?皇上历来新宠不断,旧爱不忘。妹妹我也惯了。对着一个多情的人,最好的办法是什么?我也曾想过斗尽一个又一个女人,消除一个又一个新宠。可以后来我发觉,我耗尽了力气,费尽了心血,斗倒一个女人,只是让另一个女人更快地成为她的新宠。我才明白,对于一个多情的人,要诀便在一个`多'字。宫里的女人越多,他才会越顾不过来。人人争宠,便没有了专宠。没有了专宠,我的日子便安稳了。所以,我由着宫里的嫔妃们多起来,由着她们争奇斗艳。百花齐放,奼紫嫣红,便没有一支独秀了。若是为了这些女人跟皇上怄气,那可真真是犯不上了。姐姐说,是不是?''

夏光蓬盛,正当凌霄花季,庭院台阶下的角落不知何时长出了如斯多嫣红浅橘的花朵,婉转攀缘,生出大片大片凝红深翠,如深沉花海,点缀着楼台的寂寞。热烘烘的风熏然而过,长长的花之轻轻摇曳,那细微的声音,像是春日檐下缠绵的雨。如懿看向窗外,花影密密幢幢,明媚相欢,唯有自己的一颗心,虚了。到底是无情之人,看得通透。

于是如懿便道:``妹妹想明白这些,那就不止是皇贵妃的境地了。''

嬿婉笑语凌厉,``如今我也算看透了。孝贤皇后对着皇上事事谦和忍让,从不顶撞,结果皇上却觉得她过于端方而失情趣,偏就喜欢姐姐你直率敢言。可是等你成了皇后,直率敢言的好处便成了皇上的不知恭敬,事事冒犯。所以皇上便喜欢我的温柔妩媚、恭顺婉约。连您的闺阁气度、知书通文都比不上我得皇上点拨后才一知半解的温顺机慧。果然妻不如妾妾不如偷了。当然了,我也明白,再怎么得皇上宠爱,都是比不过容嫔的。我心服口服。可容嫔再怎么得宠,也无一儿半女。女人呢,年岁渐长,孩子越多,到底也是依傍。''她一顿,越发亲切温婉,``对了,姐姐的永璂,可一直由着愉妃照顾着呢。可惜了愉妃,没了五阿哥,日子就难过了,人也伤心得病歪歪的,不知能否照顾好永璂呢。''

如懿的眼皮轻轻一跳,示意众人下去,方才道:``你终于忍不住,要说你的得意事了,那么?我虽然只见过永琪的侍妾胡氏一次,可那一次她就能咬死了我不放,指我害了永琪。''她鼻尖酸楚,无限叹惋,``真是可惜,宫中的规矩皇子的福晋侧福晋须得进见后妃,而侍妾格格之类地位低微,都无须相见。否则我与愉妃,怎容得此挑拨母子情谊的狐媚女子在侧,日夜蛊惑永琪?''

嬿婉咯咯地笑起来,笑得欢悦而清脆,``永琪这么待姊姊,姐姐还记得挂着那个不肖子呢。说来姐姐也真可怜,抚养过的永璜和永琪,一个利用你,一个疏远你。儿女情分淡薄至此,也真是少见。''她十分得意,``姐姐,我和你不一样。我一直以来就十分纯粹,只是想要得到最好的生活。我知道我出身寒微,能有这样的机会来之不易,我不奢求情爱,不渴望家族荣宠,我十分简单地只想做皇上的宠妃,过越来越好的日子。而你呢,有了荣宠想要尊位,有了情爱还奢求尊严和底线。你要知道,身为皇上的女人,身子发肤荣辱生死都是皇上的,你求得越多,想要守护得越多,便越是告诉旁人,你的软肋有多少。我又何尝不知道,永琪也是你的软肋。左右你的儿子是失去皇上欢心,做不成太子了。若永琪在,万一他顾念情分,来日登基带你出去为母后皇太后,那我这个太妃可如何自处?''

``所以,格格胡氏,到底是你的人?''

嬿婉笑意款款,眉目濯濯,``姐姐很想知道胡芸角的来历么?可惜了,那个女孩子的来历已经被我抹得一干二净。她是良家子出身,清白无可挑剔。若不是做得这般干净,凭愉妃的心思,早就疑心了。可是对于姐姐,芸角也算是故人之后了。她本姓田啊。''

``她姓田。''如懿极力思索,``是田嬷嬷,是不是?可她只有一个儿子啊。''

``姐姐真聪明,芸角是田嬷嬷与前夫的女儿,一直在乡间长大。田嬷嬷惨死,与姐姐有脱不了的干系,我便给芸角指了条捷径。断送了永琪和姐姐的母子之情,断送了姐姐的指望。芸角也真是个懂事的孩子,说完了该说的就一头碰死了,死无对证。既全了孝心,也全了忠义。''

恨到极处,身体内的病痛被牵动。如懿剧烈地咳嗽起来,拿绢子掩住,也掩住那咳出的点滴红色的血沫。她喘息着,渐渐定下心神,``那么永琪的附骨疽也脱不了胡芸角的干系吧?''

嬿婉笑吟吟凑近,一张面孔凝脂般白滑,晃悠在眼前,嘴角衔着诡秘而治艳的笑意,``附骨疽多因风寒湿阻于筋骨,气血凝滞而成。体虚之人露卧风中,或是冷水洗浴后寒湿侵袭,或是房欲知道盖覆单薄,都容易造成此疾。永琪要强,有点病痛也不肯说。他能文能武,更擅骑射,风餐露宿骑马射猎,本就容易得这个病,何况有爱妾在侧,房事之后故意贪凉,病症便会加重。''

如懿怒极,转瞬颜色清淡沉静,一字字清如碎冰,``你做事很周全,越来越缜密。''

嬿婉托着粉杏的腮,轻裁漫拢的云鬓下,远山含黛的长眉,秋水为盈的漆眸,唇红齿白间缓缓吐出,``姐姐,你和愉妃一向精刮,对永琪的福晋和侧福晋都精挑细选,却不想毁在一个小小侍妾身上。永琪的福晋多是父母之命,未必诚心。我便让芸角到他身边,指点她永琪所爱,自然得宠。有她枕边风吹着,永琪又心存疑忌。姐姐啊姐姐,如今永琪已死,我看你再走不出这翊坤宫了。''

嬿婉说着,环视萧索冷落的翊坤宫,不觉畅快。曾经六宫之主的宫苑,如经冷清衰败至此。哪怕是晴明天气,也充斥着从墙皮和廊柱底下发出的陈腐气息,上好的紫檀、花梨和桃花芯目搁置久了,都有那种尘灰寥寥的朽木气味。还有门环上兽首的铜气,若无人首厮磨,铜器得气味会近乎于血腥气,令人窒闷。

可她是欢喜的,欢喜里有疑惧。自己千辛万苦所得的一切,若不能再失败者前炫耀,岂不是衣锦夜行,无人衬托她的快乐。

如懿轻笑,``既然你如此笃定,何必再假惺惺来探视我?分明,心底还是怕的吧?''

嬿婉倒也坦然,``是会怕。怕得来太辛苦,失去却太轻易。怕皇上哪日心念一动,又想起你来。''

如懿瞠目,这样荒谬的念头,也只有富贵闲逸中的人才想得出吧。她摇首,``首得住这个位子一辈子的,固然是尊贵无上的皇后。可若守不住,便也是个下堂弃妇!但是你难道不知,如今的我,那怕是守着皇后这个尊贵无上的名分,也不过就是个下堂弃妇。皇上暂且留了这个各位给我,是顾全他自己的名声罢了。''光阴凝在檐角,迟迟不肯流去。嬿婉有几分难解,如懿却通透,``怎么?你是急着想要拿到这个后位,所以盼着我早些去了吧。我也不妨直言,我已身染痨症,你如愿之日,也不远了。''

嬿婉轻轻``啊哟''一声,捂着心口娇声道;``姐姐,你可千万别死。人活一世,才能看着那些污糟恶心的事儿一件一件应在自己身上,饱受痛心折磨,永远也没个完。活着才好呢,妹妹我盼着您寿比南山哪!''

如懿微微一笑,``活得长久就是福气么?生不如死更是难受。可是皇贵妃,你可从来没赢过我。''

嬿婉得意,``这个妹妹明白。这个世上唯一能赢过你的,不是我,不是香见,也不是孝贤皇后。我们都不是,唯有皇上。要你生,要你死,全在于他。''

如懿明了,亦承认,``是。辗转于一人手心,生死悲喜全由他。当然,你也一样。我倦了,真的倦了。''

嬿婉唇角笑意不减,``是呀,都是皇上定了算的。我赢不了姐姐,可我能借着皇上活得比你久,比你好就成了。我呀,就满足了。''

她说着,笑的花枝轻颤,牵动鬓上花钿,金翠明灭。

也不知笑了多久,嬿婉终于累了。如懿还是那般波澜不惊,如古井深水,沉沉深定。她颇为无趣,拂衣起身,撂下一句话,``若得空,我再来看姐姐。''

待出得宫门,嬿婉扶着春婵的手,才觉出自己两颊酸痛,是刻意笑得久了。她颇有几分惴惴,``乌拉那哪是依旧活着,只怕皇上对她犹有余情,本宫得想个法子才好。''

春婵奉承道:``有小主在,不怕皇上对她余情未了。''

``本宫已经不够年轻了。''嬿婉低低嗤笑一声,``谁能红颜常驻,恩宠不衰?唯有更年轻的新鲜人儿在眼前,皇上在想起那个女人,只能想到她的年华不再,恶形恶状。''她依依嘱咐,``又要到选秀之期,春婵,你好好替本宫留意。''

春蝉连声答应,嬿婉得意地挥手瞟一眼翊坤宫,却未见长街转角处,颖妃与七公主牵手而立,深深蹙眉,厌恶不已。

七公主轻轻晃了晃颖妃的手,``额娘,您这几日身子不适,为何还要来看皇额娘?''

颖妃弯下身,低柔道:``她毕竟还是你的皇额娘,紫禁城的皇后,额娘只是觉得她可怜,才想来看看。''

七公主信任地点点头,依偎在她身边。颖妃揽着她,心底却闪过一丝疑惑。乌拉那拉氏辗转让人托话,请她今日至翊坤宫外,难道只是为了目睹魏嬿婉的得意?

\hypertarget{ux7b2cux4e8cux5341ux516bux7ae0-ux65e0ux5904ux8bddux51c4ux51c9ux4e0b}{%
\chapter{第二十八章
无处话凄凉(下)}\label{ux7b2cux4e8cux5341ux516bux7ae0-ux65e0ux5904ux8bddux51c4ux51c9ux4e0b}}

嬿婉的身后,又是一重又一重宫门深锁之声。雨打梨花深闭门,她合该长长久久,如一株寂寞青苔,苟延残喘与这不见天日的地方,老死其中。

她太知道自己的身体,日复一日的咳喘,几乎已经耗尽了她所有的健康与精气。仿佛一张薄而脆的蛛网,再经不起一点点的风吹雨淋。

如懿立起身,走到古旧的樟木箱子边,张开沁手生凉的铜锁,取出一张小小的帕子,湖蓝色绫绢上,绣着一朵小小的四合如意纹。她并无犹豫,在白昼点亮了蜡烛,将绢子焚上。火舌卷得很快,一下一下蹿上来,舔着绵软的绢子,很快化作灰烬。

如懿的面色平静如澄蓝湖水,``凌云彻,我这一生,能谢谢你的,也唯有如此。愿你来生相知,去一处平安喜乐的境地,福泽一世。''

容珮淡然看她烧完,将灰烬用紫铜屉子拢起,走到庭院中,扬手撒去。

如懿听见自己的声音,清晰而决绝,催促容珮,``快!''

容珮没有哭,将一把小小的匕首从怀袖中抽出,交予如懿手中。她举起匕首对着窗外的日光一照,锋刃上闪着幽蓝光芒,的确是一把利刃。

她无言,轻轻微笑,恬然自若。她望着容珮,低声道:``我一死,你便可以离开。容珮,若是能出去,定要好好活着。''

容珮重重点头,``奴婢伺候您上路。''

如懿眸光轻转,落在绣架上只绣了一半的花样上,那是开了一半的青色樱花,在雪白轻纱上无忧无虑地盛放。还有,还有翻了一半的《墙头马上》,一出唱不完的悲欢离合。

如懿轻叹,忧思重重,``也不知这些,能不能保全我的永璂?''

容珮点头,神色坚定而安宁。

如懿微微一笑,再无留恋。她举刀向胸,刃没至柄。动作很快,手气刀落,只觉得胸口深凉,并无太多鲜血溅出。

如懿仰起脸,窗外日光正盛,一朵,一朵,如盛开的大片木棉,灼热甜香。她在痛楚的蔓延滋生里,忽然忆起一点从前。

晴朗的日光下,满是浓荫翠翠,新开的桐花绛紫雪白,散落清甜滋味。他置身于花叶下,清隽容颜上有笑容明耀,等着她,缓缓走近。

她浑然不记得,那是什么时候的事,是真切的往事,还是缥缈的虚幻?

但,那一定,是他和她的最初。曾经的思念如漫天清寒的冰雪,深入骨髓,可天明日光照耀,只能看着它混同尘埃,污浊地化去,一无所有。

如懿轻轻笑着,在碎裂般的痛楚中,停止了呼吸。

容珮一直跪在如懿身边,面上无一丝悲伤之情。她见如懿微微仰首,向着殿外风生帘动之处,笑意柔和。她半眯着眼睛,不知是在回避七月流金的日光,还是在享受它热情的不会因人而异的照拂。

容珮想,这样半眯着眼,大概是死不瞑目。

一定怨恨许久,也曾企盼许久。但,求不得,却也只能逼着自己放下。

容珮想了想,取过绣架上如懿常用的一把银剪子,她没有丝毫犹豫,将它的利口横过自己的脖颈。

有鲜红的血液喷溅出来,飞溅在发黄陈旧额帷帐上,像一朵朵红梅凄然绽放。她低声道:``奴婢来陪您\ldots{}''

脑海中所有的记忆,停留在她遇见如懿的那一日,她是低贱的奴婢,在圆明园被差役了许多年,忍受了太多的责打与凌辱。是如懿,于辇轿之上俯视她,将她从尘埃泥泞里捞起。

她不过是一介奴婢,能回报的,唯有生死相随。

那一刻,翊坤宫内真是安静,所有生命的气息都静止了,自然也无人听见海兰匆匆推门而来,切切呼唤着:``姐姐,等等我。''

如懿的死讯传到养心殿内,皇帝午睡乍醒。新晋的嫔妃笑靥如花,温顺妥帖地伺候着他起身。他摸了摸那个女人的脸,却想不起她的名字。

不要紧,只要是年轻的、新鲜的、柔嫩的身体,都能抚慰他对于衰老将至的恐惧。何况这些女子,都有这丰盛的笑意,永远只对他绽放,任他轻易采撷。

是进忠进来回禀的,他的口吻,和死了一只蚂蚁并无二致,他说:``翊坤宫娘娘自裁了。''

不知怎的,皇帝一直记得进忠那时的语调,尖尖的,细细的,像划破光滑锦缎的旧剪子,一划,又一划,钝钝的,带着锈迹。皇帝莫名就觉得厌烦。

身边的女子依偎着他,娇声惊呼,``啊呀!死也不好好选个日子,偏在中元节的前一日,真是死了也不让人安宁。''

因是皇帝跟前的新宠,进忠赔笑到:``小主说得是,得请宝华殿好好做场法事才好呢。''

皇帝无言,脑海里,心尖上有一阵深邃的痛楚,只盘旋着无数个念头:她死了?她真的死了?就这样,走在他的前头,没有半分留恋,还是,宁死,她都不愿与他再生活在同一座紫禁城里?

这样的念头刺着他,又锐又痛。他心烦意燥,却难掩心底一重重失望,和那根本无从躲避的痛楚。

那女子还在嘤嘤抱怨,进忠道:``皇上,请旨,该如何处置?''

他答非所问,``翊坤宫之人,为何自裁?唤容珮来,朕要问一问。''

进忠微微迟疑,还是道:``翊坤宫娘娘得肺痨已久,久病缠身,大概生无可望。至于容珮,业已殉主。''

皇帝微微张了张嘴,叹息道:``她走得不算孤单。''

身边的女子语气轻诮,鄙薄之意昭然若揭:``乌拉那拉氏举动疯迷,病势日剧,骤然离世,实在福分浅薄,皇上切勿为她伤心。''

伤心么?当然是,可他不惯在面上表现出来。

进忠走近一步,恭敬请示:``皇上,翊坤宫娘娘身份尴尬,丧仪不知如何处置?''

那女子还在喋喋不休,大约是仗着皇帝宠幸,愈加放肆,``皇上,嫔妃自裁可是大罪,这是乌拉那拉氏公然羞辱您啊。''

皇帝再也忍耐不住,低喝道:``滚出去。''

那女子怔了怔,还未反应过来,眉眼触及皇帝的冷然,才生了惧意,也不敢哭出声,赶紧缩着身子出去了。

这一番倒是意外,连进忠也不曾想到,他只能更低眉顺眼,听皇帝吩咐。

皇帝凝神片刻,再睁开眼时,眼底已经发红,``朕本意予以废黜,终存其位号,已格外优容。可是她宁愿自裁,宁愿这样离弃朕,决绝如此\ldots{}''

进忠小心翼翼:``皇上,翊坤宫娘娘生前公然断发,顶撞皇上,是否还要按皇后丧仪来办?''

皇帝的声线有太多不甘与伤神,竟有几分嘶哑了:``乌拉那拉氏\ldots 她一定很不愿意做朕的皇后。''

进忠立即接口:``那就按庶人礼仪来办?''

皇帝的眼神不知停在何处,``罢了,丧仪就按皇贵妃之例办吧。丧葬事宜,一切从简。永璂呢?让永璂回去视丧,陪她最后一程。''他想一想,``她生前与纯惠皇贵妃交好,也不必麻烦,置于一处便好。''

进忠答应着,正要离开。皇帝忽然唤住他,``翊坤宫之人自裁前,见过什么人?''

进忠踌躇片刻,赔笑道:``皇上,皇贵妃去看过翊坤宫娘娘,送去一些补身之物。其余再没别的了。''

皇帝不作声,却分明看清了进忠眼底的那丝犹豫,``朕知道了。愉妃与乌拉那拉氏亲厚,丧仪的一切事宜由她安排就是。''

进忠一震,立刻道:``是。只是愉妃娘娘刚刚丧子不久,立刻管事怕是力不从心。宫里一直是皇贵妃主事\ldots{}''

皇帝似乎不耐烦:``愉妃若是不成,还有颖妃呢,也可以帮衬。再去传旨,容嫔晋为容妃,享贵妃礼,与愉妃一同照顾永璂。''

进忠连连答应这退出去办差事了。皇帝一言不发,只是看着进忠的背影,手指轻叩在紫檀桌上。

不过须臾,他便吩咐身边的太监金保,``去唤李玉回来,朕要他伺候。''

灵堂就设在翊坤宫里,要不是宫门口的一溜白纱灯笼,真看不出里头正在办丧仪。皇帝吩咐了一切从简,如懿生前又极尽失势,再加之十七阿哥出生,嬿婉反复叮嘱不可有哀乐吓着了他。如此,就算有颖妃和刚晋位为容妃的香见帮衬,海兰能在丧仪上所做的主,也实在不多。

不过,人少也好。于海兰而言,更能清清静静地陪着如懿多一些时候。

海兰这般沉默跪守在灵前,烧着纸钱元宝等物。火舌贪婪得吞着那金纸银纸的元宝,也照亮着海兰苍白至极的面孔。丧子之痛已经夺去了她半条性命,相伴数十年的姐妹离世,更是将她折磨成了行尸走肉。

海兰烧完手里最后一把元宝,凄惶道:``姐姐,说好了要等我回来的,你怎么说了不算话。明明答应了的,一句话,一个字都要当真。你却食言了。''

没有人回应她,可以回应的那个人,早已躺在了棺木中,生气全无。巨大的悲痛将她击打得无法起身,匍匐在地,发出呜咽的悲泣。

良久,有人缓步进来,伸手扶住了她,``愉妃姐姐,你要节哀。''

是婉嫔的声音,海兰缓了片刻,才能说话,``哀莫大于心死,还如何节哀?''

婉嫔素来心善,环顾四周,轻轻叹气,``你瞧这宫里的人情冷暖,翊坤宫娘娘到底还没被废后呢,居然只有我和你来。''

海兰淡漠道:``颖妃在外头主持大局,容妃去陪着十二阿哥了。庆妃胆子小,来转了转就走了。其他人都碍着皇贵妃的面子和皇上的震怒不敢来。''

婉嫔点点头,跪下将地上元宝和纸钱的灰屑拢了拢,柔声安慰:``能来的都是对娘娘真心的。''

海兰颇有几分奇怪,``婉嫔你素日最胆小,怎么也来了?''

婉嫔低首像是被触动了不堪回首的往事,含着羞辱与不安,膝行上前,磕头三下:``我欠了娘娘的,只怕这辈子都还不了了。''

窗外风声呜咽如泣,海兰出神片刻,自言自语道:``要还,总是能还的。''

窗外风声呜咽如泣,皇帝失神地坐着,也不知过了多久。天光明亮得很,可皇帝还是觉得身上寒浸浸,明明是夏日炎炎啊。七月盛暑,怎会有凉意袭人呢?大约,大约真是殿内的冰供得多了些。皇帝伸出手,摸着眼前一支玫瑰簪子。

那是一件旧物了,戴着它的人一定很是爱惜,常在青丝间廝磨,才会有这般光润。

进保递上一盏清茶,``皇上,您看了这簪子很久了。''

皇帝点点头,``她走的时候,唯一的佩饰就是这支簪子。这,是朕很久以前送她的。''

进保轻声唤,``皇上。''

皇帝似乎没有听见,仍是摸着簪子把玩,``她这是什么意思呢?对朕怨恨己极,
却还戴着这支簪子。''

皇帝的眉心曲折渐深,那疑惑盘旋在他心头,甚是难解。进保不知该如何去劝。
翊坤宫丧仪,皇帝没有踏足一步,颖妃主持宝华殿超度之事,皇帝也不过问。按理说,他该是厌弃极了乌拉那拉如懿。可为何,却偏偏拿着这支簪子,不言不语,不饮不食?

进保自知劝不得,只能兀自焦急,直到外头小太监通报皇贵妃到来,他才轻轻舒一口气。或许皇帝,愿意听一听皇贵妃的劝说。

嬿婉进来时,己不见皇帝手中把玩的簪子。她的脚步轻快,全然不像一个刚生育的女子,反而像是一只游荡花丛的蝴蝶,以最美的姿态翩跹。

嬿婉轻盈请安,皇帝微笑着吩咐她起身,早已没了方才的愁云慘淡。

嬿婉侍驾多年,与皇帝也是亲近,便在榻边坐下,傍着皇帝的手背絮絮诉说。不过是宫里的一些琐事,皇帝兴致不大,有一耳朵没一耳朵地听着,嘴上应付:``你是皇贵妃,后宫的事你自可做主。''

嬿婉得了这一句,心思稍定,这才露出几分关心情切之意,``刚去姐姐的宝华殿看过了,颖妃头一回主持这样的大事,实在有些紧张。''

皇帝何等精明,只等着她说下头的话,便也淡淡的:``那你可教导她些。''

嬿婉伸手在皇帝肩上轻轻捶着,甚是体贴。等皇帝舒坦些许,方才柔声细语道:
``臣妾也是心疼颖妃妹妹,既要主持丧仪,还要回去照顾璟妧,实在辛苦。''

皇帝倒是心疼嬿婉,闭目养神,口中应着:``那也没有你辛苦。这几年接连产子,又要亲自照顾。''

这一语倒惹起了嬿婉的伤心事。她手中动作一缓,顺势伏在了皇帝膝上,哀叹不已:``唉,臣妾想着,虽然璟妧是臣妾的长女,但自幼不曾和弟妹一块儿相处。如今璟妧也大了,未免手足情谊淡漠\ldots{}''

若不提,这些都是旧事了。可个中缘由,皇帝是再清楚不过的。嬿婉生育七公主璟妧之时,正是生母惨死、自己地位不保之际,所以这个女儿一直养在颖妃膝下。而颖妃虽然是养母,但一直不曾生养,对这个养女爱得跟眼珠子似的,照顾得无微不至。且颖妃的性子素来不与如懿、嬿婉两派来往,只与自已一般出身蒙古的嫔妃亲近,自成一派,将七公主护得极紧,连生母都甚少见到,更无半分母女之情。

今日嬿婉的话说得如此明白,皇帝也知道了,``你想接璟妧回去?''

嬿婉也不掩饰心迹,倒是一副慈母的关切情怀,``璟妧那孩子自小只和颖妃亲近,对臣妾一直淡淡的。臣妾想,不如让璟妧在臣妾那儿住一段,也好彼此亲近些。''

这话她没有再多说,因为皇帝也知道,接走七公主,等于剜了颖妃的心头肉,她是断断不肯的。然而嬿婉的泪已经涌了出来,啜泣不己,``皇上,璟妧到底是臣妾亲生的,臣妾实在挂念。每每午夜梦回,想到她不在身边,真是心痛\ldots{}''

或许解铃还须系铃人吧。皇帝也不多言,只道:``那就让璟妧去你那儿住一段日子。若是她住得惯,就留在你身边吧。''

嬿婉大喜过望,忙忙周全了礼数便退出了养心殿。她一壁吩咐了王蟾去咸福宫接七公主,一壁打发宫女回去将永寿宫的侧殿整理出来,供七公主居住。

春婵笑吟吟道:``等七公主一回来,几位阿哥公主都养在小主膝下,那可真是团圆了。''

嬿婉微微得意,``为了璟妧的事本宫求皇上多年,难得皇上今日竟然痛快答允了。''

春婵奉承道:``乌拉那拉氏一死,您就是后宮第一人,皇上自然尊重您的意思了
。如今七公主就要回到小主身边,小主事事圆满,再没有不顺心的了。''

嬿婉面上的得意一闪而过,却未肯说出来。斗了那么多年,最后乌拉那拉如懿竟是自栽死了,真是无趣。这般无用的敌手,为她枉费多年,真是冤哉冤哉。不过她一死,这后宫便真是自己的了吧。

数十年光阴流转,谁能想到曾经全无家世的小小宮女,竟会成为宫中位同副后的皇贵妃呢。自然,没有正后,副后亦是等同于皇后了。等三年丧期满,安知坐于凤座的人不是她呢。

心思懵懂间,仿佛已是身着凤袍的自己立于万人中央,接受如山朝拜。然而眼前几个人走过,却只是草草行礼,毫无尊敬之意。

这种冷漠,让嬿婉无法承受,即刻变了容色,``站住!见到本宫怎不行礼?''

为首的正是集万千宠爱于一身的香见,她冷然道:``我是我行我素惯了,向来没规矩的。''

嬿婉气结,看着香见身后两个蒙古嫔妃,恪贵人与恭贵人,喝道:``那你们呢?''

二人互相看了一眼,大约觉得的确失礼了,才道:``咱们跟着容妃娘娘走得快,所以\ldots{}''

嬿婉冷笑:``所以行礼草草,果真眼里没有本宫了。''

恪贵人与恭贵人有些尴尬,香见拦在前头道:``咱们赶着去翊坤宫给主子娘娘磕头,顾不上对皇贵妃的礼仪,也不必见怪。''

嬿婉似乎不相信地重复了一句:``主子娘娘?''

香见正色道:``皇上讲不曾废后。翊坤宫娘娘,自然就是咱们嫔妃们的主子娘娘。''

这下连春婵都忍不住了,忙为主子出头,回嘴道:``荒唐!她不过以里贵妃礼下葬,算得什么主子娘娘?''

香见见主仆这般色变,反而气定神闲地笑了。她的目光如清冷碎冰,划过脸庞时。嬿婉都能察觉那种森森寒意。香见一字一句道:``就算如此,那也是我们心里的主子娘娘。皇贵妃,你可不是。''

香见话音己落,两位蒙古贵人也无半分劝阻之意,显然在她们心底,是认同这句话的。嬿婉心底的怒火己经嗞嗞烧了上来。她知道香见的性子执拗,皇帝都少悖她意思,便挑两个贵人说话,``容妃无礼,你们也要效仿么?''

恭贵人重施了一礼,不卑不亢,``颖妃娘娘主持主子娘娘丧仪,我等蒙古嫔妃,自然追随。告退了,''

众人再不言语,低首告退。

嬿婉气得发怔。她几乎不敢相信,这是她人生最得意的时候,多年劲敌己死,生子揽权,居然被一个有宠无子的嫔妃顶撞不算,连主位都算不上的贵人都敢不将她尊若神明。真是要反了!

春婵见她转瞬间脸色数变,知道是气恼到了极点,忙忙劝说道:``小主,小主,
您别生气。看来这些蒙古嫔妃都追随颖妃,您夺回七公主是对的,正好挫挫颖妃的锐气。叫她们知道谁才是真正的后宫之主。''

是了,这才是症结所在。嬿婉沉住气,一言不发,径自往永寿宫去。

算着时辰,颖妃忙碌于宝华殿和翊坤宫两头,自然无暇顾及七公主,而区区宫人,拦不住王蟾势必为她接回女儿的气势。待得颖妃知道,早就木己成舟了。

嬿婉这么盘算着,己到了永寿宫外,一进宫门,便听到了七公主的吵嚷声。到底是亲生女儿,这么多年分离,嬿婉心疼不己,上前就搂住了七公主,唤道:``璟妧,璟妧。''

璟妧乍见她来了,吓了一跳,勉强叫了一声``令娘娘'',便又挣扎着道:``我要回去,我要回去!我住在咸福宫,不是永寿宫。''

小小---个人儿己经半大,力气不小。嬿婉珠翠满头,绫罗丝滑,一时有些抱不住她。

嬿婉满口价哄着:``好孩子,我是你额娘,听额娘的话,额娘疼你。''

璟妧怔了片刻,细细打量着她,深吸了一口气。嬿婉以为孩子心思转动,正要再柔声劝说,不想璟妧肃然朗声:``不,我要回去。我额娘是颖妃,不是你。''

春婵在一旁忙不迭地劝着哄着:``七公主,小主才是您的亲生额娘啊。''

璟妧的面色渐渐冷下来,略带稚气的白嫩脸庞上露出与年龄不符的沉着与冷静,她的口吻是决断的,不容置疑的,``不是,不是,我是颖妃的女儿。''

若是璟妧撒气撤泼,嬿婉都不会在意,小孩儿嘛,哄哄吓唬几回便好了。可是偏偏,这孩子的神情明白无误地告诉了她,她都知道,都明白。

有寒意从骨血里沁了出来,这个孩子,己经在截断她试图联系起来的母女血脉之情。

真的是来不及了么?后宫尚未完全驯服,连亲生女儿都要远离自己,背叛自己。

这个念头瞬间点燃了她的血液,那燃起的火焰几乎烧噬着她身体的每一寸,让她焦灼、痛苦,以致怒不可遏。

嬿婉的手离开了怀中的女儿,居高临下一般,冷然道:``这孩子,这般不服管教。''

春婵被她的神色吓到,赶紧道:``七公主还小,又一直没在小主身边,慢慢就好了。''

嬿婉不耐烦在宫人们面前露出下风,便顺水推舟道:``也罢,先安顿她住下,和弟妹们亲近亲近,也好让她知道,她是从谁的肚子里出来的。''

当下,玉蟾赶紧拉过了璟妧,殷勤道:``对对,七公主的屋子收拾好了,奴才带您去瞧瞧。''

七月中旬的风,带着酷热的暑气扫上了面庞。轻飘的裙角被傍晚的风轻浮地拂起,嬿婉深深吸了口气,将那如血残阳,留在了身后。

颖妃得知消息时,已是掌灯时分。她从翊坤宫回到咸福宫,正要梳洗更衣来抵去一日的辛苦,却立刻被心急如焚的宫人们围住,告知她七公主被接去永寿宫的消息。

颖妃心底最软弱处被人一刀刺中,几乎是瞬间失了方寸,喝道:``为什么不早来禀告?''

宫人们吓得跪了满地,抖衣瑟瑟。颖妃看着众人畏惧不己,才稍稍恢复了几分理智。是啊,一有皇帝的准许,二有皇贵妃之尊,三则也是最重要的,自己在翊坤宫主持丧仪,一旦如此刻般乱了方寸,要承受失礼之罪的也只有她自己了。

可是璟妧,她怎能夺走璟妧?

没有人知道这个孩子对于颖妃是多么重要。从她抱回婴孩开始,从璟妧软软的小身体,红通通的面孔在她怀里那一刻开始,她就把这个孩子视作了自己的亲生骨肉。

大约是天意不许,虽然得宠多年,颖妃从未有过自己的亲生孩儿。便是一同出身蒙古的妃子,也无人有生育之能。对于一个有宠无子的女子而言,自小养大的孩子,是多么重要。一句心头肉,也不为过。

真的,不是为了权势依靠,而是她真心爱着那个孩子,那个在空落落的紫禁城与她相依相伴的孩子。

是了!就算嬿婉是璟妧的生母又如何?嬿婉素来看重儿子,璟妧的出生又未能为她挽回彼时颓势,她又怎会如自己这般爱惜。璟妧的第一次笑,第一次牙牙学语,第一次学步,第一次风寒发热,都是她陪伴在侧,一一照顾。那个亲娘,又在做什么呢?谋算?毒害?媚宠?不,这些都叫她看不起。

她亲手养大的孩子,怎可回到那样的生母身边去?

\hypertarget{ux7b2cux4e8cux5341ux4e5dux7ae0-ux5e7dux68a6ux4e0a}{%
\chapter{第二十九章
幽梦(上)}\label{ux7b2cux4e8cux5341ux4e5dux7ae0-ux5e7dux68a6ux4e0a}}

颖妃的思绪疯狂地旋转着,脚下己经跌跌撞撞奔了出去。花盆底碍事,被她一脚踢开,只着白袜奔跑。此时一众蒙古嫔妃都得到了消息,赶来慰问。见她这般失态奔出,为首的恪贵人、恭贵人吓得不知所措,只好本能地拦住了颖妃。

颖妃眼里哪有她们,径自喊着``我的璟妧,璟妧啊''。宫女们苦苦哀求,恪贵人先劝道:``有皇上允准,娘娘哪里能带回公主?''

恭贵人见事倒明白,立刻指出症结所在,``定是皇贵妃忌恨娘娘为翊坤宫娘娘主持丧仪,才要夺走七公主。''

颖妃发狠道:``那又如何?就是本宫与咱们这些蒙古姐妹在翊坤宫娘娘与皇贵妃之间从不偏私结党,皇上才格外器重,又怎会因此怪罪?''

恪贵人怯怯道:``总不是因为翊坤宫娘娘自裁,皇上气昏头了吧?''

颖妃气得连连顿足,忽而心念了转,厉声喝道:``皇上是生气还是伤心,谁知道呢?再说翊坤宫娘娘是不是自裁还是两说呢。谁知道是不是被那位所杀,翊坤宫娘娘死前可是见过那位的!''

---众蒙古嫔妃都惊呆了,不觉面面相舰。不知谁轻声嘀咕,``啊!这话可不敢胡说啊。''

怎么会是胡说?

当日的情形再度浮现于眼前。

颖妃执着璟妧小小的手,看着嬿婉得意而出,而那不久,便得到了翊坤宫乌拉那拉氏自裁的消息。

模糊的念头随着心痛越来越清晰。是了,一定是魏嬿婉杀了乌拉那拉氏。便不是亲手所为,也一定是她所逼杀的。一定是!

到底是恭贵人心思细些,低声道:``这话也未必是胡说,我已听到不少风言风语。''

颖妃被夺女之痛烧得容颜扭曲,厉声道:``我带着璟妧进的翊坤宫,翊坤宫娘娘刚气绝不久,而皇贵妃前脚刚离开!''

恪贵人一张俏脸雪白,``娘娘,就算我们有蒙古诸部作靠山,您这样公然诋毁皇贵妃,也是不成的呀!''

颖妃满脸是泪,挣扎着道:``本宫不管!本宫只要自己的女儿!''

这一声哭,众人都静了下来。蒙古诸嫔妃只有颖妃养了一个女儿,这位公主对她们干系极大,嬿婉这般夺女而去,不止昭显她在宫中的权势如日中天,更是不将蒙古放在眼里。而这一切倚仗,不过是皇帝的宠爱,儿女的依靠罢了。

正值持间,一个纤瘦的身影缓步踱进。她的语调低沉而柔微,却掷地有声,``诋毁?这些话宫里好多人都在传呢。''

众人忙行礼道:``愉妃娘娘。''

海兰柔声道:``都起来吧。''她走近颖妃,贴近她耳边低语呢哺,``知道你的孩子被抢走了,我是来帮你的。''

恪贵人面上闪过一丝不信,海兰失了曾经皇后的依傍,失子,无宠她还有什么?

海兰似乎是猜到了诸人的心思,轻声道:``在这个节骨眼上带走七公主,是打击颖妃的良机,也是将你们一众蒙古嫔妃压倒,让她称雄后宫的良机。''

她的话语极轻,却足以让在场所有人震动。

恭贵人旋即明白过来,``有了七公主在手,颖妃娘娘顾及多年母女情谊,势必要向她低头。''她轻哼一声,``咱们蒙古女子,不会欺人,但也不会由着她人欺辱。''

暑气夹杂在晚风里,裹得人浑身每一个毛孔都窒息不堪。那种感觉,像极了睬进泥淖深潭。不可自救,只能眼睁睁看着自己陷入绝望,无可奈何。

颖妃在泪眼迷蒙里仰起头,软弱和伤心并未将这个蒙古女子血液里的坚韧打碎。她紧紧握住了海兰的手,低声道:``我看见了,璟妩也看见了。''

数日来皇帝都是心绪不佳,饮食上多是被退了出来,只说皇帝胃口不佳,绿头牌更是彻底被闲置了。御膳房和敬事房便是着急,也是无可奈何。御前是进忠、进保守着,这二人口风极紧,谁也不知养心殿中的那位至尊,到底是怎么了。

太后虽然挂心,倒也沉得住气。趁着皇帝来请安,便也与他闲话片刻。

皇帝照例是对太后恭敬有加,一壁又道:``皇额娘气色极好。''

太后斜坐在榻上,微微而笑,``有什么好不好的,人老了,懒得费心思。心一宽,气色自然不会差。''

太后语中之意,皇帝如何不明。他似乎不愿继续这个话题,一手拨着黄花梨案上的白玉莲花炉,那氤氳散开的香烟混着殿内冰座上散开的沁凉微润的水汽,那香气仿似也变得雾沉沉的,丝丝缕缕黏在身上,缠绵着不肯离去。

太后见皇帝不开口,便径自说:``乌拉那拉氏的丧仪哀家亲自去了。唉,她到底没有被废后,这丧仪,未免也太简薄了些。''

皇帝似乎怨怼颇深,语调平静得毫无起伏波澜:``她不喜欢做儿子的皇后,丧仪是按照皇贵妃礼仪来办的。也算遂了她的心愿。''

太后轻轻一嗤:``这话就是赌气了。你不让她享有皇后身份,与你合葬,自然是因为心里有气。可按旧例,凡葬在妃园寝内的,无论地位有多低,都各自为券,而乌拉那拉氏却被塞进了纯惠皇贵妃的地宫,堂堂皇后反成了皇贵妃的下属。这也说不过话去呀!''

皇帝眉心一动,有无限心事被挑动。他嘴唇微微张合,犹豫良久,方才低声道:``乌拉那拉氏怨恨儿子,自然不会愿意将来与儿子合葬。且她在世时,几个皇贵妃里也只与纯惠皇贵妃合得来,在一块儿也好。免得地下寂寞,连个说话的人也没有。''

太后晓得皇帝的难堪,然而并不停止追问:``那不设神牌,也无祭享,这连民间的葬礼也不如了吧。''

熏香燃得有些快,重重渺渺地散在二人中间,好似一道纱雾屏风,朦朦胧胧。太后年纪大了,眼目不如从前清亮,竞有几分看不出皇帝的神色微动。

心上柔软处似被什么东西狠狠撞了一下,那种抽痛牵起鼻中的酸楚。皇帝很有些委顿,露出几分难得的软弱,``乌拉那拉氏,她向往的是民间夫妻的生活。做儿子的妻子,让她痛苦。''

太后幽幽一叹:``你这么说,可见把她说过的话放在心里,那又何必如此决绝?''

皇帝极力硬着心肠,冷然道:``皇额娘,是她自裁,与儿子决绝。她做过对不住儿子的事,禁足思过,是朕对她的惩罚。''

太后默不作声,只是定定望着皇帝。那目中的了然与惋惜,皇帝如何不懂只得道:``自然,儿子也有对不住她的地方。''

``到底乌拉那拉氏是与你潜邸便在一起的情分。难道她死了,你还恨她?''

``儿子爱惜的是当年的青樱。对乌拉那拉如懿,她与儿子,彼此失望。''皇帝黯然不己,``说到底,儿子与她是彼此辜负了。她也一定对朕怨到了极处。当年,她还是青樱的时候,直爽,单纯,对朕一心一意。可惜,这些时光,再也回不去了。''

这句话,似乎抽去了皇帝所有撑持着的力气。他还想说什么,然后眼底微沁的泪光己经阻止了他的言语。再开口,必定是哽咽,何必在此露了心防。

是啊,无数的时光匆匆奔涌而去,谁也不复少年时光,他所留恋的青樱,何尝不也是自己放不低的弘历时代?

翩翩少年郎己然垂暮,心头牵念不己的少女,也情绝意断。谁还记得当年,墙头马上遥相顾,一见知君即断肠。或许便是曾经那么在乎,如今就有多么心痛吧。而不想心痛,能做的,便是不在乎,便是厌弃,才能麻木。

末了,还是太后道:``乌拉那拉氏过世,最伤心的还是永璂。皇帝切不可迁怒于孩子身上。''

皇帝道:``几子知道。永璂也是儿子的孩子。只是这孩子畏畏缩缩的,没有些意气风发的样子。永琪从前可不这样,永琪\ldots{}''他轻轻摇头,``永琪己经不在了。''

太后轻嘘道:``哀家何尝不知道永琪是你最得意的儿子。可永琪这般出色,也是乌拉那拉氏多年教养的缘故。''

谈到子嗣,皇帝稍稍缓和神色,``若是永琪还在,儿子怎会伤心至此?这些皇子里头,出嗣的出嗣,早夭的早夭,剩下的几个虽然伶俐,都尚是孩童,不能为朕分忧。皇嗣之事,干系国本。''

太后连连摆手,``承继宗室之事,不需这么早提。你春秋正盛,再为国事辛苦三十年也无妨。只是你的阿哥,多是纯惠、淑嘉二位皇贵妃所生,他们自然是不成器的。余者便是令皇贵妃所出,哀家倒觉得,孩子都养在她膝下,也不是个事儿。''

皇帝并未把此事放在心上,犹自记挂着永璂,``乌拉那拉氏没了,永琪也没了。永璂由愉妃养着,也算彼此安慰。皇额娘,那孩子还得你费心关照些。''

太后微微颔首,父母不合,决绝至此,永璂如何不知?素来父母未能情好的,最吃苦的便是孩子。永璂性格沉闷软弱,多半也是因为如此。里帝大约也是知道此节,怕永璂心中有怨,所以才请托了太后照顾。也唯有太后照顾,才镇得住与如懿不睦的嬿婉吧。

太后轻轻叹息,天家尊荣,享得泼天富贵,却亲情不保,又有何趣味呢?或许真要活到了自己这斑白年纪,才能僅得个中滋味吧。

皇帝这般不乐,嬿婉照例是要领着嫔妃们去请安的。然而这几日她也实在是无心他顾,璟妧到了永寿宫里,不肯吃饭,竟是断了饮食。起初嬿婉也不着急,永寿宮的小厨房手艺远胜于御膳房,什么苏杭点心珍馐美食,但凡小孩子爱吃的,一溜儿流水样供到璟妧面前,便不信她一个孩子扛得住这般诱惑。

然而奇怪的是,璟妧那孩子是出奇的镇静与倔强,死咬着不开口。若是给水便喝,食物一点也不碰,铁了心地要回咸福宫。

嬿婉原打算着颖妃要来闹一闹,便可趁势炫耀自己皇贵圮的威仪,好好训斥她一番,打压气焰。偏偏颖妃不来,她满腔气焰无处可发,想着颖妃是骨子里怕了她,一早酥倒,便转怒为喜了。可谁知一个孩子便闹腾得她头痛不堪,再好的气性也忍耐不住。只为璟妧来来去去就是几句,``我要回咸福宫,我要回额娘身边。''

嬿婉气结:``我才是你的额娘。''

璟妧慢吞吞道:``不是。你不是。不回咸福宫,我宁可不吃饭。''

嬿婉气急了便道:``好,你就算饿死,也是我的女儿。''

璟妧不哭也不闹,稚嫩的脸庞上竟是冷笑,``你真的很喜欢看别死,是不是?''

那目光中的寒意,逼迫得嬿婉忍不住要发抖。她怕什么?风里浪里,刀剑相逼,熬不过这些,如何做得上皇贵妃的位子?可那目光居然是来自亲生女儿,竟让她毫无抵抗之力。就算是输,也不知输在了哪里。

嬿婉恨恨地想,是了,一定是颖妃教坏了孩子,一定是。

她想一想,几乎是带着奔逃的姿态,想去看一看永璘、永琰和九公主璟婳。这些她一手带大的孩子,绝不会如璟妧待她,绝对不会。至少她还拥有那些孩子的依恋与笑脸,她什么都不用怕,不用怕。

李玉到底是宫里的老人儿了,听闻皇帝召唤,一声也不言语,也不问缘由,便打点好了一切,奉茶上前。进忠见到李玉时来不及收住满脸的惊愕,道:``师父回来了。''

李玉不咸不淡道:``圆明园里的差事虽然清闲,但还得回来孝敬皇上。''

他进到养心殿暖阁,恭敬端上茶水。皇帝抿了一口,回味悠长,``三月的龙井茶,七分烫,茶香满口。也唯有你彻得出这一碗恰到好处的茶来。''

李玉跪下道:``皇上不嫌弃奴才年老眼花,奴才感恩不尽。''

皇帝徐徐道:``你回来,要孝敬的必定不止一盏茶。''

李玉恭声道:``奴才已去翊坤宫给娘娘上了香,也打点了容珮的后事。''

皇帝的语声远远的,似从天际缥缈而来,沉沉砸入他耳里,``如懿,到底是如何死的?''

李玉心下一坠,果然,果然皇帝是疑心的。他微微压低声线,``翊坤宫娘娘自裁前,令皇贵妃刚刚离开。随后进去的,还有愉妃、颖妃和七公主。''

李玉几乎以为自己耳朵不清了,他居然清楚地听见皇帝的嗓音微微一颤,``真是自裁?''

李玉如何不知皇帝的疑惑,忙道:``奴才査验过,自裁倒确是自裁。只是奴才不解,翊坤宫娘娘抱病己久是真,但为何早不自裁晚不自裁,偏在令皇贵妃走后自裁。若说是病中绝望,也不大通啊。''

皇帝深吸一口气,将心底呼之欲出的质问按捺下去,只以淡然之色相询,``你的意思,是令皇贵妃说了什么,抑或做了什么?''

李玉缓缓摇首,老成持重,``奴才能査问到的,是显而易见的东西。至于底下是什么,因由是什么,奴才不过是奴才,不懂得査看人心,也不知情由所在。''他一顿,``奴才适才前往翊坤宫,看到了一些东西,特意拿来给皇上细看。''

皇帝默然颔首,李玉击掌两下,有两个小宫女捧了东西进来,那是曾经侍奉过如懿的菱枝和芸枝,她们捧了大幅雪白的锦锻在手,款步走进。

李玉沉声道:``翊坤宫娘娘废居一年余来,无事时只着意于刺绣与诵经。所绣之物无他,只有一二花色。请皇上一顾。''

芸枝和菱枝捧着洁白如霜雪的皎云轻纱,徐徐铺开。皇帝注目片刻,不觉微湿了眼眶。

真的只有二色图样。

青色樱花盛开如蓬云,红荔鲜艳。绮丽之外,其余素白一片,上头的针功细致沉腻,每一朵花瓣不知刺了多少万针,才费尽一瞬一瞬之时,挪万象情感于绢布之上。

眼底的热意越来越烫,几乎有刺痛。他转眸,扬起脸,再扬一扬,生生把泪水逼落下去。他听得自己无波无澜的平静音调,``她身边还留着什么?''

李玉恭谨道:``一幅未曾绣完的绣样,与这些并无二致。另则,娘娘身边还留着一本看了一半的书,是白朴的《墙头马上》。''

他刻意维持着平稳的心跳陡然失去了韵律。那是他与她同听的第一出戏。记忆里的人呵,还是华章子弟,豆蔻梢头的好年岁。

她还是念着的,念着的。念着他们的初初相遇。遥遥相顾,一见倾心。

偏偏,那诗里是这样说的,墙头马上遥相顾,一见知君即断肠。

她与他的最末,终宄只是天人永隔,---世断肠。

皇帝似是自语,``绣样留了一半,书也看了一半,便这般弃世了?''

皇帝的沉默是压在坚冷雪山之巅的寒云,压迫得人透不过气。也不知过了多久,端起茶水轻抿,``进忠虽然得你真传,很会服侍。但他到底是你的徒弟,不比你稳重练达。譬如这一盏茶,也不如你端来温热适口,就让进忠去热河行宮,你留在朕身边好好伺候。''

李玉答应着,垂手立于一旁。皇帝复又提起饱蘸了墨汁的笔,不疾不徐,批阅奏折。

也不知过了多久,更漏泠泠,墁地金砖上投着一帘一帘幽篁细影,令人昏昏欲睡。京中想来暑热,七月更是流火欲燃。殿中供着金盘,上头奉着硕大的冰块,雕刻成花好月圆蝶鸟成双的图案,将殿中洇得蕴静清凉。皇帝跟前的奏折渐渐薄下去,冰块亦渐渐融化,那鸟儿失去了翅膀,蝴蝶亦飞不起来,花己残,月己缺,小水珠滴落在盘中。再美再好,也不过浮华一瞬,再也寻不回来。

外头起风了,蓦然间水育底绣浅粉楼花纹影色帘翻飞,如一色青粉的裙流连而过。恍惚里,是皇帝的声音,轻轻唤了一声,含糊得一如风中掠过的蝴蝶,带起一缕花叶的涟漪。

李玉分明听见,皇帝唤了一声,``青樱。''

呵,李玉恍然想起,从前的从前,他们都还年轻的时候?青樱最爱穿的,便是这一色花叶生生的衣裙。只是,这世间的青樱,早己不在了。连如懿,也魂魄归去。

皇帝眉心微曲,郁然长叹,``她去得好么?''

李玉如何敢说,想了半日,还是道:``翔坤宫带笑意,去得安和。''

``她情愿死,也不愿留在这里。李玉,她不该来这宫里,若是去了外头,海阔天空,她的一生,不致如此。''

李玉喉头一阵阵发酸,``皇上,她苦,您也苦。若是翊坤宫娘娘还活着,哪怕您与她不再相见,奴才知道,您心里便不会那么苦。''

皇帝并不答他的话,只是负手起身,从寝殿榻上的屉子里,取出一方丝绢,青櫻,红荔。岁月更长,人已渐老,但那丝绢,却簇新如旧。他握着那方丝绢在手,久久无言,静静问:``你猜,令皇贵妃对如懿说了什么?''

李玉紧紧地闭着双唇。不必说了,已经什么都不必说了。疑根深种,只等长枝蔓叶,开花结果。他眼中隐隐含泪,难抑心底一丝激动。只凭这一棵疑根,嬿婉即便成为皇后,也不会那么安稳了。

李玉回来的消息一阵风似的传遍了后宫,连带着进忠被远远打发去了热河行宫。
这瞬间的地位翻覆,不得不让有心人去揣测圣意之变背后的玄机。

嬿婉反复追问,得到的答案不过就是皇上嫌进忠伺候得不好,让李玉回来了。这也算情理之中,进忠就算再伶俐,手脚再便捷,李玉到底是打皇帝登基就伺候在身边的人,最熟悉皇帝的习惯与性情。那么再被召回,也是理所当然了。可嬿婉却是害怕的,李玉与如懿交往颇密。如今如懿新死,李玉又回来,莫不是皇帝动了对如懿的怜悯之情,那便不好办了。

春婵不知嬿婉心思,仍在絮絮,``进忠知道去热河行宫当差是逃不得了。但是求娘娘垂怜,让他早日出了行宫,回来伺候。''

嬿婉玉齿轻咬,不动声色道:``既然出去了,热河行宫那么远,路上一个不小心风寒不治死了,或者在行宫里失足淹了,都是有的。进忠,不必再回来了。''

春婵一顿,见嬿婉已然有不满之色,赶紧答应着退出去了。

嬿婉见她出去,又召了敬事房太监过问选秀之事,一时忙碌起来,也顾不上别的了。

春婵一直快步走到了宫门外,王蟾才迎上来,关切道:``脸儿煞白的,中了暑气了?''

春禅像是找到了依靠,压低了声音,急促告诉他,``进忠不能留了。''

王蟾也不意外,只道:``既然小主吩咐了,我会处置。一个进忠,你心疼个什么劲儿。''

春婵满脸后怕,看了看四周无人,方敢道:``我哪里是心疼进忠,不过是想起了澜翠,也这么没了。''

王蟾打了个激灵,一把按住她的口,``小主的脾气你还不知道?惜命吧。''

春婵一口气闷住,差点呛着,连连点头道:``我懂,我懂。''

午后的紫禁城,静得少有人声。日光无遮无拦地洒落,逼起红墙金瓦之上一阵阵白腾腾的暑热。虽说八月了,京城早晚渐凉,但午后酷热,却是半点也未减。这般昏昏欲睡的时节,凝神细听去,才能听到戏乐之声悠悠传来。春婵有些奇怪,``这个时候,谁在传戏呢?''

王蟾苦笑,``是漱芳斋那儿的声音,这不,一定是皇上在听戏呢。''

春婵摇摇头,``翊坤宫娘娘才过世不久,皇上就听戏,也太无情了些。''她想想又笑,``不过话说回来,皇上对翊坤宫娘娘无情,我们小主的地位才稳固无忧啊。''

戏台上的戏子们水袖轻扬,七情六欲都在面上格外浓重。曲调伴着丝竹悠扬起落,是谁在诉说着柔肠衷情:``你道是情词寄与谁,我道来新诗权做媒。我映丽日墙头望,他怎肯袖春风马上归。''

皇帝坐在漱芳斋里,日常所余的爱好,仿佛便只剩了听这一出《垴头马上》。宮人们垂手而立,静若泥胎木偶,无人敢打扰皇帝这份静逸。唯有李玉轻手轻脚侍奉在
侧,斟茶递水,打扇轻摇,间或轻声低语一句,``皇上,快到选秀的时候了,各地待选秀女的名字都报了上来,您可要看看?''

皇帝双目微闭,随着曲调双指轻叩,淡淡道:``罢了。后宫有丧,选秀的事先停一停吧。''

李玉不敢多言,只挑了要紧的说:``选秀的事,皇贵妃费了大心思的。''

皇帝嗤笑:``她肯费心,朕却没这个心思。怎么?她照顾着那么多孩子,又接回了璟妧,还顾得上那么多么?''

李玉欲言又止,外头却传来一声不合时宜的哭声,扰了乐曲里的情意宛然。``皇上,皇上,您救救璟妧吧。''

李玉侧耳,``是颖妃的声音。''

皇帝听得是颖妃,即将要升起的怒意压了下去,吩咐了宫人们让了颖妃进来。颖妃一路梨花带雨进来,哭得几乎噎住:``皇上,皇上,听说璟妧倔强,回到永寿宫一直不肯进食,这可怎么好?''

皇帝虽是训斥,口气却柔缓得很,足见素日对颖妃的客气,``胡说!皇贵妃是璟妧的亲娘,怎会饿着她?''

颖妃性子刚强,极少在皇帝面前哭,撤娇落泪更是罕见。皇帝见她情状,已然纳罕,偏颖妃不接受他的劝说,哭得更凶,``璟妧自小在臣妾身边长大,与皇贵妃的母女情分一时转園不过来,彼此倔着。这璟妧饿坏了身子可怎么好啊?皇上,求您让臣妾接璟妧回来用顿饭吧。''

皇帝一怔,无可奈何,``唉。都是倔性子,哪里像你,更不像她亲额娘。''

颖妃嘴快,``璟妧喜欢她皇额娘,这刚强脾气像足了翊坤宫娘娘。''

话一说完,李玉都变了神色,不知该如何接口。颖妃自知失言,慌得一颗心怦怦乱跳,几乎要跳出腔子来,心中暗怪海兰乱出主意,非要她提这一句。

皇帝面色如常,浑然没有听见这句犯忌讳的话,只是温和道:``朕也饿了。你去带璟妧来养心殿,陪朕用饭吧。''

颖妃欣喜,如一只欢跃的鸟儿,立刻飞了出去。

那边厢嬿婉吩咐着选秀的事宜,让乳母带了九公主璟婳、十五阿哥永琰去陪着璟妧,想着孩子们在一起,总是好说话好玩闹,也便能哄得璟妧吃饭了。璟妧对着弟妹们倒不像对嬿婉那般排斥,也肯说几句话,乳母们便退远了,由着他们在一块儿。

璟婳只比璟妧小一些,已经很明理了。因为和弟弟们一起长大,所受重视不多,所以比起璟妧独受宠爱长大的性子,璟婳要温柔许多,很有几分嬿婉还是宫女时的模样,她劝道:``七姐姐,你快吃饭吧,别惹额娘生气了。''

璟妧冷淡道:``她不是我额娘。''

永琰年纪虽小,却一下明白了其中的关节,只说:``额娘是我们的亲额娘,七姐姐是我们的亲姐姐。''

虽然不说是亲母女,却强调了彼此的血亲和自己不可分割,这下纵然是璟妧也辩驳不得。

璟妧别过头,露出傲然不屑之色,``皇贵妃才不是我额娘,她是坏女人,她害死了皇额娘!''

璟婳一下子急了 :``姐姐胡说!额娘不是坏女人!''

当然翊坤宫外的情景历历在目,确是嬿婉出来之后,便得到了翊坤宫皇后的死讯。璟妧记得清清楚楚,此刻道来也是理直气壮:``她就是坏女人!皇贵妃见了皇额娘,皇额娘才死的。就是皇贵妃害死了皇额娘,我和额娘都看见的。''

嬿婉听说孩子们在一起相处不错,正为自己的妙计得意,赶来享受这绕膝之乐。哪知才到门边,就听得这句锥心之语,霎时变了脸色,连声呵斥:``你说什么?你这孩子,胡说八道什么?''

璟妧被这突如其来的怒喝吓了一跳。待回头见是嬿婉,又露出素日的冷淡鄙薄的神气,转头看着别处。嬿婉气不打一处来,喝道:``果然是颖妃教坏了你,我自会去找她算账。''

璟妧听得她要为难颖妃,果然慌了神色,嘴上却尖利:``你就是坏女人,你害死了皇额娘。你一定还做过许多坏事,所以十四弟、十六弟死了,这是报应!''

嫌婉的心彻底凉了。这就是自己的女儿,心心念念要夺回来打击颖妃的女儿,她的心完全不向着自己。嬿婉心口一阵疼痛,太阳穴突突地跳着,激起锐利的刺痛,挑起青筋根根暴出。嬿婉顺手抓起桌上一把戒尺,拉过璟妧的手心狠狠打下去,``我不是坏女人!这话是谁说的?是颖妃是不是?''

璟妧想躲开,却被嬿婉死死抓住,不得逃离半分。璟妧手心被打得通红,死死忍着不肯求饶,咬着牙道:``你就是坏女人,谁都不喜欢你!我不喜欢你,我讨厌你!
额娘,额娘,快来救我啊。''

璟婳和永琰何曾见过嬿婉这番暴怒模样,早就吓得呆了。璟婳缩在墙角,紧紧捂着嘴什么也不敢说,永琰连反应的能力都没有了,只是喃喃:``别打姐姐,别打姐姐。''

嬿婉盛怒之中,哪里会理会永琰的话,见璟妧不肯求饶,一味嘴硬,下手又凶又快,一下接着一下,``我才是你的额娘,我要好好管教你。''

这般乱糟糟的,乳母们吓得昏头,只晓得赶紧上前抱走璟婳和永琰,不让他们多看。璟妧何等机灵,趁着乳母们一窝蜂上来,立刻挣脱了嬿婉的手,向外跑去。

嬿婉哭得伏倒在地,连起身的力气也无,``我不是坏女人,我不是啊。我都是为了你们,我不是坏女人!啊,我的女儿,为什么要这么待我!''

还是春婵警醒,和王蟾架起了嬿婉,慌不迭道:``小主,咱们快追七公主回来啊。这么跑出去太危险了。''

嬿婉立刻醒过神来,吩咐着去追,自己也跟了出去。

璟妧好容易逃脱出来,奈何饿了几日,腿脚着实不快,而且永寿宫一带她着实少来,也实在辨不清方向,只知道沿着红墙根跑离永寿宫,离得越远越好。

眼看着乳母、宫人们追了出来,嬿婉气急败坏地跟着,璟妧再也忍不住,哭喊道:``额娘,救我啊!额娘!''

这一喊太过凄厉,颖妃本快步往永寿宫来,听得声音,几乎人都站不住了,
一转角循声过来,抱住了璟妧,母女俩抱头痛哭。璟妧受了多日的委屈,见了颖妃才宣泄出来,紧紧抱住她手臂不放,``额娘,你终于来了。璟妧好想你啊。''

颖妃仔仔细细看着璟妧,立即发现她手心的红肿。这个女儿虽非亲生,但一直爱如珍宝,哪里受过这般委屈。颖妃心痛得直落泪,连声追问:``怎么了?你的手怎么了?''

说话间嬿婉赶到了眼前。见了颖妃,嬿婉的慌张伤心旋即被掩饰不见,恢复了皇贵妃的尊荣高傲,清冷道:``本宫的女儿,不用旁人管教。''

颖妃不肯示弱,一把将璟妧拦在身后护住,``我是璟妧的养母,怎么不能护着她?''

嬿婉的唇角含着讥诮之意,居髙临下看着颖妃,``不过是养母,皇上己经将璟妧交回本宫抚养。''

璟妧躲在颖妃身后,咸福宫的宫人将她团团护住,不让永寿宫的人接触。璟妧声色更壮:``不,我是额娘的女儿,不是皇贵妃的女儿!''

颖妃微微一笑,打心底里觉得欣慰,面对嬿婉,也更不畏惧,``看来,璟妧并不认你。''

嬿婉一腔怒火无处可泄,便也不顾及颖妃的身份,作色道:``都是你教坏了璟妧!''

颖妃也不生气,眸中清冷之色愈加浓烈,``我并无教坏孩子,孩子懂得是非,她不喜欢你的为人。其实何止是孩子,即便你位同副后,权倾后宫,至少咱们蒙古这些嫔妃就不服你,不服你这种用龌龊手段上位的女人!''

自从嬿婉封皇贵妃,宫中奉承无数,她哪里受得住这样的气?一时间心血翻涌,气得几乎要呕出血来。春婵在后,轻轻扯了下嬿婉的袖子,低声道:``您是皇贵妃,您教训谁都是应该的。''

是呢。皇贵妃之尊,与这般寻常嫔妃闲言什么,教训便是。且不说这宫里大了一级就足以压死人,嬿婉有子,颖妃无子,就是尊卑之分。

嬿婉的怒色冷却少许,肃然道:``早知道你不服!本宫就教你个乖,教你什么是心服口服!来人,颖妃犯上不敬,给本宫带下去杖责。''

杖责是重刑,何况嬿婉未说杖责多少,便是要挫颖妃的锐气。咸福宫的宫女们,几个胆小的早就冒了冷汗,颖妃根本无所畏惧,只是打量着嬿婉,``我虽然是妃位,但我的背后是蒙古各部。你是皇贵妃,却毫无根基,风雨飘摇。''她含笑逼近,``许多事,不在位分,不在儿女多少,而在前朝后宫,势力交错。这一点,你比不上我。''

嬿婉气得发颤。她们就这般肆无忌禅么?仗着家世,仗着母族,不将她这宠妃放在眼里,还要任意击打她的弱点。

是可忍,孰不可忍。事到如今,撕破脸都不够了。

嬿婉索性下令:``还干看着做什么?给本宫打这个不知天高地厚的人!''

宫人们面面相觑,一时无人敢对颖妃下手。

立刻有宫人跪下求情:``皇贵妃娘娘息怒,皇贵妃娘娘息怒。''

这是真真儿忌惮颖妃的母族势力了!嬿婉眼前一阵晕眩,立刻鼓足了气势再要喝令。却听得一个沉稳女声道:``吵吵嚷嚷做什么?哀家去看了永璂回来,都不得清静。''

太后积威多年,无人不服,当下所有人都跪下了: ``太后娘娘万福金安。''

太后一身青金色锦袍,一头花白头发以翡翠扁方馆住,略略点缀几件金器凤簪,不怒自威。

太后目光扫过嬿婉,将她看得如水晶玻璃人一般,``当了皇贵妃日子也不短了,
还不能令嫔妃信服,看来哀家是得好好教导你。颖妃,你到底位分低些,也该懂得尊卑上下。有什么事不许当着奴才丢份儿,你们到慈宁宫来吧。''

嬿婉哪敢吭气,只得诺诺答允了。颖妃正要揽住璟妧起身,太后伸出手,和颜悦色地拉住了璟妧,笑吟吟走到前头去了。

进了慈宁宫,众人一时无话。嬿婉纵然声气再高,不知怎的,在慈宁宮里,一盆火焰被冰水泼倒一般,就不敢言语了。

太后将璟妧拉在身边,吩咐了福珈为伤口上药。璟妧也争气,一口也不言痛,即便药粉刺痛伤处,也只是一缩手,很快咬牙忍耐。

太后不急不缓地开了口,声音是珠帘深锁下的一抹轻烟徐徐,``再动气也得顾着体面,当众争执,不怕奴才们笑话?往后还怎么服众?嫔妃和睦,才是后宫祥瑞之兆。''

二人规规矩矩答了``是''。

太后便温然看着嬿婉,``尤其是你,皇贵妃。你身负皇帝重望,主理六宫事宜,
更当稳重。''

嬿婉哪敢回嘴,立刻认错。

太后又看颖妃,你出身蒙古,又但也得自重身份,不可当众顶撞。''

颖妃何等乖觉,立刻俯首认错,然后道:``原是臣妾见了璟妧大哭,心疼不己,
所以情急犯上,顶撞了皇贵妃。''

璟妧适时站出,为养母辩白:``皇祖母,皇贵妃打孙女,孙女手痛。''

太后听得璟妧的称呼,便有些许不满:``皇贵妃到底是你额娘,你即便是在穎妃膝下长大,不叫皇贵妃额娘,也得称呼一声令娘娘。''

璟妧顾不得福珈阻拦,上前拉住颖妃的手,情真意切,``皇祖母,这才是儿臣额娘。''

太后怜惜璟妧,也不肯为难她,慈爱道:``你这孩子,虽然没规矩,但也足见颖妃一直疼你。罢了,既然如此,七公主还是交由颖妃抚养吧。''

嬿婉见太后这般轻描淡写就将璟妧交给颖妃,这一番心思岂非付诸东流,忙含泪道:``太后,颖妃年轻,难免对孩子骄纵宠溺,璟妧脾气野性子大,断不能再由旁人教养,臣妾自己的孩子,自己来养吧。''

太后见她情急,也不斥责,只温和道:``你身边己有几个孩子,再带七公主怕也顾不过来。有颖妃为你分忧也是好事。''

颖妃听嬿婉说璟妧的不是,哪里按捺得住,``璟妧好好的,并非皇贵妃所言那么不堪,否则怎会那么得皇上疼惜?''

嬿婉一双妙目圆睁,瞪住了颖妃,气势凜然,``颖妃说得轻巧。璟妧到底不是你亲生,养娘怎如生娘亲?''

猝不及防的一言,慈宁宫中旋即陷入了死一般的寂静。福珈波澜不惊,太后的唇角依然笑意温然,可双眸中尖锐的忧惧一闪,己将嬿婉钉死在了原地。太后蔼然微笑,但那眸子里的星火,分明灼得嬿婉双膝发软,匍匐跪倒在地。

太后轻轻道:``是么?''

这两个字,几乎压得嬿婉粉身碎骨。她己经匍匐在地,不知该如何再显示自己的卑微与无措。巨大的惊惶让她冷汗淋淋,拼命称罪:``臣妾失言,臣妾知错。是,是生娘不如养娘亲,养育之恩大过天。''

太后身坐重重玉绣锦茵之中,背脊挺直,凝神端详着嬿婉,``什么生娘养娘的,
皇贵妃的心思可真多。哀家没你想得繁复,孩子是谁养大的,愿意跟谁走,那就是谁的孩子。璟妧,你要跟着谁,你自己说。''

璟妧紧紧攥着颖妃的手不放,依恋而郑重:``皇祖母,孙女自小到大都是额娘照顾,生病是额娘喂药,天寒是额娘添衣。额娘最疼孙女。''

颖妃激动不己,一把搂住了璟妧,连声道``好孩子,好孩子''。话语未落,已然满面泪痕。

太后冷眼看着嬿婉,``孩子什么都懂。这是她自己选的,你也细想想,自己的言行配不配当孩子的额娘!她病了冷了的时候,你正忙着争宠吧,可有照顾分毫?''

这话己经是极厉害的了,嬿婉除了瑟瑟发抖,只能请罪不己。太后浑不理会,只叮嘱颖妃:``好好照顾璟妧,她明白是非恩怨。记着,孩子和谁亲,谁就是她的亲额娘。''

颖妃感激涕零,哪里还能说什么,只拉住了璟妧一同重重叩首谢恩。

太后道:``你不用谢哀家,要谢就谢皇贵妃自己做下的好事,翊坤宫皇后之死。''她呵一声轻笑,``皇贵妃,你也不用让哀家相信什么。要是连一个孩子都认为是你害死了如懿,你可怎么分说呢?''

嬿婉不知道自己是怎么出的慈宁宫,她深知方才的情急之语戳痛了太后的心。什么养母生母,最为太后所忌讳。她也明白,从此,她再不会得到太后的任何偏帮与支持了。更刺心的是,仿佛谁都认定了如懿是她所杀。连辩白,她都无从辩白起。然而更坏的消息很快传来,皇帝得知了嬿婉对太后的冒犯,索性下旨将永寿宫中嬿婉养育的子女都挪去了擷芳殿由乳母照顾,且只许嬿婉一月见一回。

这其实是不合规矩的,擷芳殿探视,素来是半月一回。皇帝此举,无疑是不喜嬿婉与孩子们多亲近。

永琰被进保带走前,只有一句话,``额娘,你今日的样子好可怕。''

嬿婉不知道他所说的可怕是什么,几乎是脱口而出,``不是我害死乌拉那拉如懿的!不是我!我不是坏女人,是她自己作死,与我无关!永琰,你要相信额娘。乌拉那拉如懿才是坏女人!''

嬿婉的印象里,永琰很少违逆自己,但他还是用很小很小的声音说:``您别这样说皇额娘!''

嬿婉紧紧搂着永琰,``你是我的亲儿子,你怎么帮着外人说话!记着,你只能帮额娘!''

永琰害怕地看着嬿婉,还来不及说什么,就被进保一把抱走了。

嬿婉已经是欲哭无泪,想要追出去再说什么,进保伸手恭敬地拦住,``皇贵妃娘娘,您知道皇上的脾气,最不喜欢旁人违逆圣意。您想想去了的翊坤宫娘娘吧。''

死了的乌拉那拉如懿,想起那个女人,她不该快活大笑么?怎么如懿反而成了她头顶的金箍儿,拘束着她往后的每一步了。

永璘还小,乍然被抱离生母身边,哭得撕心裂肺。嬿婉揪心痛楚,低声啜泣:``孩子,还我的孩子。''

一行人早就去得远了。嬿婉哭得不能自已,``你为什么要这样待我?为什么要带走我的孩子?为什么啊?''

可是她连去求皇帝也不敢,千辛万苦求来的皇贵妃的尊荣,不能不要。除了忍耐,似乎已经没有别的办法。左右是自己亲生的孩子,以后会亲近自己的吧。可是自己,宄竟算什么呢?嬿婉扬起脸,望着灰蒙蒙的天空,尘沙从远处卷来,不见天日。她悲楚地想,于这个庞大的皇室而言,她不过是个生孩子的工具吧?

嬿婉这样想着,眼角的泪也干涸了。无泪可流,是更深的苦涩吧。

然而当着皇帝,嬿婉到底什么也没说。皇帝心情稍稍平复之后,照常翻她的牌子,她也照常侍寝。

有时候皇帝半是调笑:``孩子不在身边,清静许多吧?''

嬿婉一怔,赶紧露出惯常的温顺笑意,``是清静。臣妾可以专心为皇上打理后宫事宜。''

皇帝对她的回答很是满意,捏捏她的下巴,头也不回地走了。

嬿婉轻轻地笑:``皇上的心思本宫越发看不透了,在皇上眼里,本宫是不是就是一个料理后宫事务的工具,一个生孩子的工具?''

春婵连忙劝慰:``您老这么揣摩皇上的心思,太累了。''

\hypertarget{ux7b2cux4e8cux5341ux4e5dux7ae0-ux5e7dux68a6ux4e2d}{%
\chapter{第二十九章
幽梦(中)}\label{ux7b2cux4e8cux5341ux4e5dux7ae0-ux5e7dux68a6ux4e2d}}

春婵连忙劝慰:``您老这么揣摩皇上的心思,太累了。''

嬿婉不言,她真是害怕皇帝,多年承恩,她其实并不知他心里怎么想。一度承恩承宠,看着乌拉那拉氏落败,她几乎舒了一口气,以为胜券在握,可是眼下,却连皇帝有没有为乌拉那拉氏之死疑心自己都不知道。每日活在这样的揣测里,能不如履薄冰,战战兢兢?可是有什么办法,路是她自己选的,己然到了这一步,除了硬着头皮走下去,哪里还有退路?

京城的秋来得很快,转眼就是落叶萧索之际。西风叹息着穿过红墙深影的重重宫阙,掠过残花衰草,凝成霜冷气韵,将这宫苑覆上薄寒。如懿去世己经数月,无人再提起她,宫闱内苑,在嬿婉的操持下,也并未有差错。偶尔得闲,皇帝便与嬿婉在御花园闲步,若是哪日香见肯作陪,皇帝的心情便又好些。

那一日天青云淡,天际是碧淸瓦蓝的颜色,远远眺望,更见万物清明,御花园内一列高大枫木己经泛红,万叶千声,迎风作响,似无数火焰瑟瑟跳动。皇帝着一袭家常暗青团纹长袍,明黄带子一系,衣挟当风,风骨闲适。香见容颜无瑕,如芝兰玉树,令人难以移目。嬿婉素知香见在皇帝心中的地位,又是不能生育之身,所以从来宽忍之至。当着皇帝的面,更是妹妹长妹妹短,无比客气。香见对谁都淡淡的,有一句没一句地应着。

远处几个小宫女踢着绣球,笑声郎朗传来,如银铃铛般清脆。香见好奇地瞥一眼,皇帝便察觉,示意她一同上前观赏。

那是三个十六七岁的宫女,五彩的绣球在她们纤细的足尖似有了生命一般,轻巧地飞来飞去。为首的紫衣宫女最是灵巧,踢起绣球时发髻上的粉色花朵娇柔颤动,衬得她清秀的容颜也似云霞一般绚丽动人。

皇帝一时看住了,颇有几分神往之情。嬿婉微微沉下脸,王蟾知趣,立刻道:``哪儿的宫女那么没眼色,没见皇上和娘娘来了么?''

宫女们吓得停住,慌不迭跪下请安:``奴婢给皇上、皇贵妃娘娘、容妃娘娘请安。''

嬿婉吩咐了众人起身,香见便撇嘴:``狐假虎威,她们踢得好好的,非要打断!''

皇帝看香见很喜欢那绣球游戏,便温言道:``你喜欢,等下朕叫她们踢给你看。''

香见笑意冷清,``人家本是自己玩儿,等要踢给我们看,多少胆战心惊的,哪里还踢得好看呢。''

嬿婉笑吟吟打趣:``容妃这话说的,好像咱们多么吓人似的。''

香见美眸微转,似笑非笑地看着嬿婉,``有的是蛇蝎心肠的人。哎,那小宫女不就被吓着了么?畏畏缩缩的。''

皇帝指着那紫衣宫女,笑言道:``容妃说你呢,别吓着了。''

那紫衣宫女立即上前,语意玲珑:``多谢皇上关怀。奴婢等自己踢绣球玩儿,不想打扰了皇上和娘娘,但请恕罪。''

她这一番话既撇清了香见和嬿婉的言辞交锋,又谢了皇帝的好意,最是圆滑不过,连皇帝也瞩目于她,``口齿好伶俐,抬起头给朕瞧瞧。''

这一瞧不打紧,一双水波潋滟的星眸盈盈望向皇帝,分外清定,仿佛两丸乌墨水晶微微折射出摄人的光芒,让人心神摇曳,不可宁定。皇帝怔了怔,便看向了嬿婉。嬿婉迎着皇帝的目光,再去看那小宫女,笑容有些勉强,``这丫头倒有几分像臣妾年
轻的时候。''

那宫女无比乖觉:``能有几分像皇贵妃,那可真是奴婢的福气了。''

皇帝再问她姓名差事,她也答得流利:``奴婢汪氏,名芙芷,在御花园当差,照料花草。皇上瞧,那几株老梅树,就是奴婢专司照料的。可惜,现下不是开花的时候。''

长得有几分肖似,又是侍弄梅花的宫女,嬿婉猜到了几分,一颗心便直直地往下坠去。

皇帝凝神看着那几株尚未开花的老梅,颇为感慨:``一朵花,未必要到开的时候才最美。早早移个适合它的地儿,等着含苞待放才好。''

嬿婉觉得脸颊都笑得僵住了,``皇上,一个小宫女,在御花园照顾花草挺好的。''

香见的话便不肯饶人了,``哦,皇贵妃不喜欢有人长得像你?那翊坤宫娘娘那时候别也不喜欢你的容貌与之相似吧?''

皇帝也明白嬿婉之意,便道:``香见,好好儿地提她做什么?''说罢,又笑着看嬿婉,``皇贵妃,朕记得当年你也是宫人出身啊。''

嬿婉只觉得足下生刺,站也站不安稳了。谁不知道她是宫女出身,一路艰辛才走到这皇贵妃之位。这份身世来历,素来为嬿婉所忌惮。只为宫里的妃嫔,几乎每一个都在家世上胜她许多,不是官宦之女,便是豪族之后。而她,若是出身再好些,何至于如此辛苦,失去那么多,才踩到这万人之上的地位。

于是嬿婉便低了头,温言婉顺:``皇上好记性。臣妾记得永和宫还有屋子空着。''

皇帝并不接她的话茬儿,只是望着西六宫方向道:``翊坤宮的庭院空着有些曰子了吧。''

嬿婉的心口剧烈一跳,正要说什么,皇帝己经吩咐道:``汪氏封为惇常在,挪去承乾宫吧。''

香见似笑非笑,``除了宝月楼,承乾宮我也偶尔去住。你若住下也好,省得那儿常空着地儿。''

芙芷忙忙谢恩,``容妃娘娘不嫌弃嫔妾,嫔妾谢过大恩,必不敢给容妃娘娘添堵。''

嬿婉连忙答应:``臣妾明白,会将承乾宫打扫一新,再让惇常在住进去。''

皇帝点点头,知道嬿婉立刻要去忙汪氏入住承乾宫之事,便携了香见的手往前走。那汪芙芷何等聪慧,不消皇帝嘱咐,便跟在了身后。

皇帝走了几步,回首见芙芷跟随,有些好笑,``你怎么跟着朕来?''

芙芷脆生生道:``皇上既然封了臣妾为常在,臣妾自然要常常在您身边伴随,才算遵从了圣旨呀。''

皇帝忍俊不禁,笑着伸手点了点芙芷的额头,``不错,不错。''

如此这般,连香见也忍不住笑了。皇帝难得见香见高兴,益发开怀,如此,芙芷的青云之路,便更顺畅了。

待得芙芷从惇常在晋封为惇贵人时,已然是深寒天气。宫中的日子过得轻忽,春夏秋冬的流转也格外迅疾。海兰久驻深宫,除了必不可少的节庆宴饮,从来都是足不出户。这一日大雪将至,香见送了些日常物用,也不急着回去。

延禧宫本就偏僻,除了香见和婉茵,极少有人来往。那种雨打梨花深闭门的幽静,几可将人沉溺其中。海兰闲来无事,仔细擦拭着如懿生前喜欢的一个摆设,香见陪在一旁看了半日,便道:``惇贵人很得皇上喜欢。你看中的人,果然不错。''

海兰笑笑:``有她在,我便知道皇上有没有放下姐姐。而如今最难受的便是魏嬿婉了吧。''

香见不假思索,``有了惇贵人,皇上连到宝月楼看我也少了,我正好落得清静。''

海兰颔首:``容貌肖似姐姐,那股子天不怕地不怕的劲儿,也很像姐姐年轻的时候。而且一得宠就住进承乾宫,可见前途无量。''

``我不知道翊坤宫娘娘年轻时是什么样子,我只知道,她后来的样子,皇上己经不喜欢了。''

``无论姐姐犯下什么大错,她年轻时的样子,是皇上最留恋最喜欢的。''她注目于香见,``你知道么?贤良淑德、循规蹈矩的女人固然适合这宫闱生活,可皇上最喜欢的,是跳脱于规矩之外自由自在的天性。这是你得宠的原因,也是姐姐让皇上念念不忘的原因。''

香见沉默片刻,看着海兰的动作,``你把翊坤宫娘娘的遗物都挪来延禧宫了?翊坤宫还空着呢。''

海兰轻轻摇头,``我看翊坤宫很快就会有新人居住,姐姐曾在延禧宫与我同住,我这儿一直保持着姐姐还在时的样子。就好像,她还活着。''

心底难过汹涌而至,香见湿了眼眶,``她真的己经死了。''

海兰微微一笑,恬静如一枝静静绽放的白梅,``不,姐姐只是去御花园赏花了。她很快就会回来。''

香见喉头哽咽,什么话也说不出来。良久,才微微点头。

海兰看着她,似乎想起什么事,便问:``这个时辰是去给皇贵妃请安的时候了,你自然是不会去的吧。''

香见颇有倨傲之色,``我自然不会去。不过惇贵人,也不会去吧。''

合宫嫔妃请安是宫中对女眷至尊的敬意。如懿死后,享受这份尊荣的自然只有一人之下的皇贵妃嬿婉。然而此时此刻,她的心绪颇不宁静。一众嫔妃行礼之后便默然无言,令得气氛馗尬而无趣,而更馗尬的,是长久以来空着的两个座位,那是属于惇贵人汪芙芷和容妃香见的。

晋嫔是嬿婉的亲信,最是不满:``都这个时辰了,惇贵人还没来。咱们合宫向皇贵妃请安,容妃是得了皇上准许不用致礼的,怎么惇贵人也得了旨意吗?''

颖妃笑道:``惇贵人起初还是迟来,如今索性不来了。这个脾气,定是皇上纵出来的。''

颖妃嘴上似是责怪惇贵人的恃宠生骄,可那背后的意思,嬿婉如何不知,无非是取笑嬿婉不敢去动皇恩深厚的惇贵人罢了。

果然跟着颖妃的禧贵人便道:``惇贵人最得皇上宠爱,就算不来皇贵妃也不会说什么吧。''

嬿婉只得息事宁人,免得她们说出更难听的话来:``悴贵人得宠未久,难免不懂规矩,以后慢慢教导吧。''

恭贵人便笑:``那也要惇贵人受皇贵妃的教才好啊。只怕她不听劝呢。''

嬿婉不想继续这个话题,便另起了话头,``眼下就快腊八了,宫中自然是要过腊八节的,不知诸位姐妹觉得如何办好?本宫虽然受命掌六宫事,也要听听姐妹们的意思。''

众人默不作声,都各自看着别处。或是拨弄手绢,或是看花出神。蒙古嫔妃们倒是一致,都看着颖妃以她马首是瞻。

既然无人答话,嬿婉便按着自己的意思往下说:``既然诸位姐妹都无想头,那本宫以为\ldots{}''

话未说完,倒是香见的声音朗朗泼进来,她自顾自道:``我倒以为,一切节庆都有先头翊坤宫娘娘掌管后宫时的成例可以遵循,何必再出主意?''

嬿婉被截断话头,心中大为不喜,但定睹看是香见,少不得忍耐。她低头抿了抿茶,不动声色地抿去了唇角的愤慨之意,听着春婵替她发作,``容妃娘娘真是稀客。''

香见冷笑:``你主子若不喜欢我来,大可去吿诉皇上。''

香见的唇角微微一扬,笑意明媚,却也有那么一丝显而易见的轻蔑。

嬿婉忍耐着微笑:``盼容妃来还来不及呢。容妃方才说要援引翊坤宫娘娘昔日旧例,只怕皇上会介怀。''

香见满不在乎地往自己座位上一坐,``是皇贵妃自己满心主意,只想施展吧?只是皇贵又有一定把握,你的意思皇上就很喜欢么?''

庆妃的性子谨慎,想了想便道:``因循守旧也并非不好,至少当年翊坤宫娘娘主持节庆,皇上和太后都很满意。''

婉嫔便点头:``庆妃所言极是。''

颖妃也是推波助澜,不肯有一刻消停,``皇贵妃大可推陈出新,只是万一太后不喜,皇上不喜,那可怎么说?''

嬿婉深吸一口气,将那笑容撑得更加饱满,``年节下的安排,正月里的赏赐,本宫都想添一倍\ldots{}''

香见照旧打断她,``翊坤宫娘娘从前怎么做,皇贵妃最好也怎么做。''

那语气里毫无尊重之意,晋嫔实在气不过:``怎么皇贵妃娘娘还拿不得自己的主意么?乌拉那拉氏早已为皇上厌弃,为何要遵循她留下的旧例?''

颖妃不喜嬿婉,更看不上晋嫔,讽刺道:``晋嫔你大概是忘了,翊坤宫娘娘的旧例多是遵循从前孝贤皇后所留下的规矩。孝贤皇后与你都是出身富察氏,你如今要改,岂不是驳了同族的颜面?''

这一来庆妃更是忧心忡忡,``是啊,皇上最尊重孝贤皇后,这些规矩改不得。还是翊坤宫娘娘那时候怎么办,咱么也怎么办吧。''

庆妃虽然无宠无子,但是太后一手提拔,皇帝对她也十分客气。她这般言语,众人更不会有异议。嬿婉一肚子气发作不得,只得看着其余人等,再三追问意见。

颖妃见众人沉默不言,笑吟吟道:``若是皇贵妃此刻得太后万分钟爱,顺太后心意略作更改也无妨。但若失了太后欢心,一做即错,那就不好了。''

谁不知自从七公主被送回颖妃身边,嬿婉便彻底失了太后的欢心。慈宁宫请安觐见,甚少有她的份。便是每回去了,太后也总有理由推说不见,或是与命妇福晋们聊天,将她撂在外头,一候就是一两个时辰。想到此节,蒙古嫔妃们都低头暗笑。

嬿婉满腹气苦,只得道:``既然大家都这么看,那就一切遵循旧例吧。''

这一仗锋羽而归,嫔妃们得意的得意,怕招惹是非的也不愿多留,也便散了。

嬿婉于人后更是气不过,``你瞧瞧这些人,变着法子给本宫添堵,从未真心顺从本宫!''

春婵替她捶着肩,好言劝慰道:``小主别急,凭她们怎样,您都是六宫第一人,
地位最尊的皇贵妃。''

嬿婉抚着心口,将一阵抽痛忍下,缓过一口气道:``就因为本宫只是皇贵妃,也是嫔妃,颖妃、容妃她们眼里才没有本宫,就连小小一个惇贵人都敢藐视本宫。若本宫是皇后\ldots{}''

这念头不过一转,想想也无十分把握,便住了口。春婵想着要哄她高兴,便絮叨着该去擷芳殿看几个孩子,嬿婉才稍稍平和,起身更衣打扮了,便往擷芳殿去。

半年不见,永琰看嫵婉的眼神己经有些拘谨了。嬿婉嗔怪了一番乳母们教导不善,让母子之间失了亲热,便哄着抱着永琰。

因着皇十四子、皇十六子早夭,这个懵懂年纪的十五阿哥永琰,便更为珍贵。且十七阿哥虽好,到底还在襁褓之中,而永琰生性乖巧懂事,很得皇帝的喜爱。这一来,更让嬿婉看到了未来光明的希冀。

嬿婉将爱子抱在膝上,左右端详。永琰有些不好意思,``额娘,我都读书开蒙了,不可这般亲昵,师傅教诲过的。''

嬿婉笑着轻斥,吻着儿子光洁的额头,``胡说!你是额娘的孩子,额娘身上掉下的肉。''

永谈一脸天真:``可皇阿玛说,我得听师傅的。''

童言无忌,而幼小的孩子,最容易在心中记下亲近之人的教诲。嬿婉顺势屏退了仆妇宫人,一一叮嘱:``你在尚书房可以听师傅的,但你心里得明白,你什么都得听额娘的。''嬿婉郑重了神色,紧握住儿子的双手,``永谈,额娘不在你和永璘身边,但你要记着,我们是母子,血浓于水,你们的心只可以向着额娘。将来无论什么时候,你都得向着额娘。知道么?''

嬿婉声声逼迫,永玻乖乖地点头。嬿婉这才放心,将儿子搂在怀里亲个不够。浑然未察觉窗外墙根下,一个瘦小的身影悄悄挪了出去。

皇帝听完来自擷芳殿的禀报,目光冲和,面色平静,眉头眼角皆沉静如水,不着喜怒之态。他只专注在一幅施工草图上,研宄半日,又慎重添上一笔。李玉伺候皇帝曰久,知道越是如此,皇帝越是动了真怒。他暗暗咋舌,天家最忌讳母子过分亲近,来曰外戚专权。皇贵妃这般教导皇子,实在是其心可诛了。

充当耳目的小太监回禀完毕,又垂手退了下去。皇帝头也不抬,吩咐李玉,``去告诉皇贵妃,她要料理后宫的事,以后半年去擷芳殿见一回儿女们就可以了。''

李玉应承了。皇帝又吩咐:``朕要在养心殿里设一座梅坞,里头所用必得都是梅花图案,周遭还要遍植梅花,你将这草图送去内务府,看看何处还需改动。''

皇帝这些日子心思全在建梅坞上头,李玉不敢怠慢,忙接过草图去了。

殿中静到了极处,皇帝揉一揉疲倦的双眼,坐于锦绣软枕之中,听着窗外风声簌簌,如泣如诉。无边的孤寂如水浸满,将他沉溺到了底处。偌大一个深宫,竟然无人能解他心底事。这样的寂寞,几可噬骨。半晌,他才听见外头进保的叫叩门声。

他忽然想起,半个时辰前,他曾派进保去承乾宫接了惇贵人来,那个不知天高地厚任情恣意的女子,自然是比不上昔日如懿的慧心玲珑。可那样天真无拘无束的女子,才比那些背负着野心与规矩束缚的女子,可爱许多。

皇帝想了想,还是愿意见见她,哪怕她浑然未知自己为何驟然得宠。这样.的无知,让他觉得安全。

嬿婉才出擷芳殿,暖轿便被李玉恭敬地拦住了。他三言两语将皇帝的旨意说得分明,浑然不顾那位尊贵的皇贵妃己然面色慘然。她根本连自己错在哪儿都不知道,就要接受着母子分离愈深的后果。

李玉连唤了几声,嬿婉才回过神来,李玉躬身退下,``奴才赶若去内务府交代梅坞建造之事,先告退了。''

嬿婉喃喃:``梅坞?什么梅坞?''

李玉含笑道:``没什么,不过是皇上喜欢梅花,所以打算在养心殿建一小憩之所,遍用梅花图案而己。''

说罢,他匆匆告退。嬿婉呆呆地望着那冬日灰白的天色,含馄暧昧的天际,一丸落阳慘淡,带着昏黄的毛边,白晕晕一团。风声凄冷,那风是越刮越大了,吹得她几乎站不住脚。有泪滾烫地落下,灼得她措手不及。落日渐坠,心也一分分沉寂下去,周遭的一切陷入庞大而无边际的暗淡与昏沉中,无声无息将她沒没于阴影成下。

嬿婉似哭似笑,十分惶感:``皇上果然还念著她,一个惇贵人还不够,皇上还要建一个梅坞!''

存婢待要劝慰,嬿婉却是认死了,
``皇上什么都不说,什么都不过问,可是他心里明明就是放不下。乌拉那拉氏,她好狠,她拼着一死,就是让皇上忘不了、放不下她。还让所有人都以为是我杀了她。她\ldots 她算计得我好苦啊!''

春婵明知嬿婉所言是真,然而人死不能复生,活人又怎么和己逝之人争去。万般苦楚在心头,只得劝了嬿婉回宮才是。然而嬿婉最伤心的是不能与亲生儿女亲近,这一悲非同小可,一时间谁也劝不住,便往养心殿去。

养心殿里正在上灯,烛火通明如流水傾泻,照亮美人的明眸星灿。

芙芷抹着皇帝喜爱的海棠色胭脂,微垂螓首,一弯累丝凤的金珠颤颤垂到髻旁。她依偎在皇帝身边,软语低声:``皇上不是刚画了一幅梅坞的单图送去内务府了么?怎的又画了?''

皇帝左看右看还是不满意,继续专注于此。

芙芷略感无趣.还是尽量寻了话头来说::``皇上很喜欢梅花么?所以要建梅坞?臣妾曾在御花园种植梅花,来日梅坞的梅花,可否由臣妾照料?''

皇帝颔首道:``你若愿意,自然是好。''

皇帝笑笑,挽住她的纤细柔荑,``等联改好这个再说,咱们先去漱芳斋听戏。''

二人正说笑着出了养心殿,却见嬿婉扑上台阶,满面是泪。皇帝笑吟吟关怀备至,``咦?京城风沙这么大么?皇贵妃眯了眼睛?''

嬿婉落泪凄楚,正要哀求。皇帝笑意愈深,``听闻里皇贵妃料理后宮事务十分妥当,处处循照旧例,未曾妄改。朕很欣慰。''

这分明是要她遵循如懿留下来的规矩!

原来,后官的一切,他部了如指掌。他知遒她的难堪,她的委屈,她的劳心劳力却无人尊重。而他,全然不在乎。

嬿婉凑厉地喊道:``皇上!''

皇帝并没有给她开口的机会,径自说道:``你既为联的皇贵妃,一切要以后宮诸事为要,旁事切勿挂怀,免得分心劳神,如慧贤皇贵纪、淑嘉皇贵妃那般憔悴伤身。''

语气是关切的,仿佛他在意着绝她。可强烈的恐惧紧紧撰住了她的心声声。慧贤皇贵妃、淑嘉皇贵妃是怎么死的,她再清楚不过。

芙芷还在那儿火上浇油,``慧贤皇贵妃、淑嘉皇贵妃都頗有家世,还有亲人厢顾探望,送来名货药材,令皇贵妃仿佛不是吧。''

皇帝温和地扶住嬿婉,
``所以皇贵妃,你更得善自保养,无须为儿女事劳心了。好了,别跪着了,起来吧。''

嬿婉的手臂被皇帝触碰着,无端起了密密的---展栗子。她在颤抖,可始没有办法,再恐惧,她也不得逃离。末了,她狠狠地咬着牙关,才能使出最后的力气,强撑着道:``臣妾闻得皇上口谕,特来\ldots 特来谢恩。''

皇帝微笑,眼里闪过一丝冷意,携着惇贵人离去了。嬿婉身子一软,坐在玉阶上,听着风声呜咽如泣,再无半分挣扎的力气。

再见到皇帝的时候.己是过了二月。身为皇贵妃,年下自然有无数要事要忙碌,而手下的奴才们办享并不利索,乎日频出,几乎让她焦头烂额。好容易应付了过去缓过神来,人却憔悴了许多。白日里辛苦操劳,夜里思子情切,连心口的疼痛也日复一日加剧了。

春来得晚,二月二撤了地龙,宫里还是森寒料峭,少不得又添了火盆。夜来无聊,嬿婉正无趣地闷坐着,想着红颜未老恩先断的哀伤,却是敬事房的徐安来传旨宣她侍寝。

嬿婉颇有些意外,自从汪氏得宠,皇帝几乎只召幸她与香见,偶尔想起旁人,也不过是颖妃、诚贵人之流。细算着她也有小半年不曾承宠了。

春禅喜不自胜,一壁替她上妆更衣,一壁嘟嚷:``里上传召总是好事,小主若是能得皇上欢心,说不定阿哥和公主就可以回到您身边了。''

是啊,她的指望,不就是这个么?

于是强打了精神,打算在床笫间百般迎合讨好,可皇帝并无那样的心思,只是嘱咐她睡下,便侧身熟睡了过去。嬿婉莫名其妙,心中惴惴,这一夜自然睡不安稳。到了三更时分,窗外风声更重,犹如在耳畔呜咽。嬿婉心念一突,想着这心痛症该传太医来瞧瞧了。这样蒙昧间睁开眼来,正对上乌沉沉一对眼珠,吓得她``呀''
一声惊呼,倏然缩到了床角。

那人一言不发,只是盯着她。嬿婉慌乱了半晌,才发觉那是皇帝冷漠的眼,她惶恐地缩起身体,``皇上怎么这样看着臣妾?''

烛火燃了半夜,垂下累累珊瑚般的烛泪,火焰子跳了一跳,照得皇帝的面庞阴晴不定。皇帝淡淡道:``没什么。只是想起了旧事睡不着。''他定一定,``皇贵妃,今儿是二月十八。''

嫌婉只觉得脑子都僵住了,含含糊糊道:``是,是什么日子?''

皇帝沉浸在某种思绪中难以自拔,``那一年朕巡幸杭州,也是二月十八,如懿上了龙舟与朕争执,一气之下断发。''

恐惧的情绪狼奔豕突,占据了她的心与身。嬿婉口干舌燥,言语连自己听了都觉乏力,``这么久的事了,皇上别再为此生气了。''

皇帝微笑:``朕不是生气,朕只是好奇。那一晚,皇贵妃,你在做什么呢?''

嬿婉张口结舌:``臣妾\ldots 臣妾不记得了。''

那声音比哭还难听。皇帝根本毫无兴趣,他翻身躺下,恍若无事人一般,``哦,不记得了,那睡吧。''

嬿婉怎么敢睡,她害怕地睁大了眼睛,强自镇定着。四下阒然,有腊梅的花味入夜弥香。她痛恨这种气味,深入骨髄。她知道,他是故意将这花供在殿内。他的心底有森然寒韵,那是怀疑、冷漠和疏离。

而她,无计可施,只能活在他的这种情绪之中。因为她太过明白,只要他疑心起,任何人都逃脱不得,翻转不得。任谁都是。

皇帝闭着眼睛,却知晓她的木然与慌张,慢悠悠道:``怎么?睡不着了?要是睡不着,让李玉早些送你回去。''

她简直如逢大赦,迅速地起身穿衣,逃也似的离开了这牢笼般的养心殿。

窗外风雪蒙蒙,那雪朵夹着檐下吹落的冰喳儿,沙沙地飞舞,天空和大地是融为一体的昏黑与茫然,只有远远近近几盏昏黄的灯笼,像是鬼魅的眼睛。有几点冰喳儿飞落在嬿婉脸上,粗粝的冰冷让刚从温暧中出来的她凜然一颤,刚想将那冰冷掸去时,那冰碴儿迅速化得只剩下一抹凉意。

嬿婉再淸楚不过,此生此世,她都要活在这冰凉凄冷之中。

是啊,她贏到了什么?璟妧的厌恶,永琰、永璘和璟婳的离开。那个汪氏,简直就是乌拉那拉如懿的阴魂,颖妃、容妃、愉妃,她们个个恨不得吃了自己!太后,太后也不是善碴儿!还有皇帝,他的疑心永远不会散去。而她所余的,居然只有一个皇贵妃的头衔,虚空的名位。

嬿婉虚弱到了极处,一口气上不来,那种绞痛再度袭上心头。她昏昏沉沉跌在春婵怀中,仓皇离开。

皇帝闭着眼,却无法沉睡。殿内火烛燃到了尽处,摇摇晃晃,终于熄灭。.外头风雪渐歇,檐下灯笼晃动的声音清晰可闻,只让人愈觉清冷。皇帝轻轻叹息,想起白日里尚书房师傅禀报永琰素日的功课,那可算是一个争气的孩子。暂且留着嬿婉,也不过是看在她还是永琰和永璘的生母。一旦嬿婉被废弃,若再想看重永琰,这孩子只怕终身都要背负着生母带来的屈辱,没有任何登上大宝的机会了。

细想来,他似乎也没有比永琰更出色的儿子了。

皇帝忍耐片刻,终于平伏下气息,摸出了枕下一方绢子,轻轻擓在了手中。

是年春日,嬿婉便被诊出有心悸之症。皇帝顺理成章地晋封了颖紀为颖贵妃,庆妃为庆贵妃,为嬿婉协理六宫事。而容妃虽然名位未升,却是车着皇贵妃的分例,超然于众人。这般相安无事,便到了乾隆三十五年。

这年五月十一,皇十七子永璘满三岁,合宫大庆。此时距嬿婉晋令皇贵妃,摄六宫事己然五年。而永璘,在三年前出生,实足是皇帝的老来幼子,疼爱逾常。按理说,皇帝这般疼爱幼子,自然也是爱屋及乌,宠爱皇贵妃魏氏。

然而这些年,皇帝只与她维持着面子上的客气。私底下的冷淡,她比谁都清楚。皇帝专宠的,唯有容妃寒香见与惇嫔汪芙芷。而芙芷在得宠之后的第二年,皇帝的万寿节后,她很快搬出了与容妃同住的承乾宫,成为翔坤宮新主人,独掌一宫事务。

用皇帝的话说,便是``汪氏细心,由她照顾翔坤宫花草也好''。

当然在后宫诸人看来,这也是理所当然之亊。乌拉那拉如懿己死,荒落的翊坤宫总会有新的主人。而不快的,也唯有卧病的皇贵妃而己。

再者甚得六宫尊重与皇帝爱宠的,便是颖贵妃。除了养育七公主,联姻蒙古,颖贵妃所得的尊荣,早己不下于皇贵妃所有,隐隐有夺其锋芒之意。而于嬿婉,孩子一个个生下,也只能养在擷芳殿,由嬷嬷们悉心照顾。而她,一年中能见孩子的,不过寥寥两三面。

这般主理六宮的权柄宠眷,反而不能将孩儿留在身边养育。宮里自然有颇多闲言闲语。但皇帝与太后的说法却是冠冕,``既然要主理六宫事务,那自然是要专心专意,不可为旁事分心了去''。

据说那日芙芷在翊坤宫赏花时闻言,对着宫女们便是一声冷笑:``如此说来,皇贵妃不过是个紫禁城后宫的管家罢了。''

芙芷那时己是惇嫔,这般不将皇贵妃放在眼里,自然是恩宠深厚的缘故。然而言辞锋芒锐利,也是看出了嬿婉对后宫之事的力不从心,便是位同副后又如何?颖贵妃所领的蒙古妃嫔自然是不屑于嬿婉,自成一派,事事以颖贵妃马首是瞻,公然与她冷然相对。容妃独领盛宠多年,我行我素惯了,便是庆贵妃、愉妃、婉嫔等少伴君侧的妃嫔,也是安静度日,几乎不去应酬她。

后宮这般四分五裂,嬿婉要维持着面子已经极为辛苦。芙芷更是数度叫嬿婉下不来颜面。几次按捺不住去皇帝面前分说,她含泪絮絮半曰,皇帝停笔只是茫然问:``什么?''嬿婉便再也说不下去。

偶然太后听闻,还要含笑奚落:``说来你当皇贵妃日子也不短,怎还是这般不得人心?倒叫哀家疑惑,这皇贵妃的权位你还不拿得稳?''

嬿婉低着头,听着刺心之语,只能低眉顺眼地诺诺,含恨吞下屈辱。怎么能不要权位呢?拼了一切得回来的,就算拿不稳,也不可轻易弃了。

好歹,好歹还有皇十五子永琰呢,那孩子,是最得圣心的。

一开始,总还是有盼头的。便是圣宠大不如前,到底也是唯一的皇贵妃,摄六宫事。这五年来顺应帝心,绝无错漏。而离那个名分尴尬的皇后如懿去世,已然满了三年。三年丧期己过,再度立后也順理成章。这几乎就是封后的前兆,当年的乌拉那拉如懿,何尝不是如此一步步登上后位。

然而她心底知道,那是不会了。除非,除非有一曰母凭子贵,她才可以立于不败之地。

皇家少年知事早,十岁的永琰什么都懂,在来请安的间隙轻声问:``额娘就这么盼着封后么?''

嬿婉抚一抚鬓发上累垂的九凤金丝转珠步摇,柔声道:``额娘苦心保全了自己半世,若真有那一天,也算能松一口气了,''

永琰不置可否,只轻轻摇了摇头,``额娘这些年人前风光,可人后的酸楚,儿子也知道些许。譬如七姐姐一直养在颖贵妃膝下,连她婚事您都不能做主,皇阿玛只和颖贵妃商议,将七姐姐嫁到蒙古。至于九姐姐,在擷芳殿这些年,也不能与您亲近。''

嬿婉被儿子说中刺心事,心底酸涩。这些年,纵然有宠,可皇帝偶尔看向她的目光,却让她情不自禁地打个寒噤。自己真的算是宠遇有加么?可皇帝的心思,她也从未真正明白过。

这样想着,她的语调不觉冷然,``不过是女儿罢了,不在身边也无妨。她们的婚姻,只要对你有助益就好。永琰,只要你争气,你皇阿玛喜欢你.額娘就有问鼎后位的指望。''

永琰轻声道:``那皇额娘\ldots{}''

嬿婉怔了怔,旋即正色,``她己经不是你皇額娘了,你这一声若被外人听见,不知又要多几多麻烦。''嬿婉忽然有些伤感,低声说,``额娘明白你的意思,你是怕身处后位,难免有一日要步乌拉那拉氏的后尘,可是如果额娘真有那一日,或许她的处境也会好过些。''

永琰凝神片刻,``皇阿玛不是那样可以轻易转圜的人,尤其是皇\ldots 乌拉那拉娘娘\ldots{}''

他并未再说下去,因为进保己经过来,匆匆告诉她皇帝风寒发热的消息。

皇帝素来最重养生,很少风寒,至于发热难受,更是难得了。嬿婉担着皇贵妃的职责,不能不去看望。

进了养心殿,转过暖阁,皇帝却不在寝殿,而是在殿后的梅坞,那是一个小小阁子,一色的冰裂纹棂格窗,房内一切所用,皆是梅花纹饰。夏日纳凉,倒也是个不错的所在。只是,嬿婉并不喜欢去。每到此处,她便会想起,想起那个喜爱梅花的女子。

是。哪怕那人己然身死魂消,哪怕胜利的是自己。想起她,嬿婉还是恨意横生。

当下她便对李玉道:``既然皇上得了风寒,怎还在梅坞歇着,不挪去寝殿?''

李玉诺诺,只道皇上乏累不愿挪动,嬿婉也不好发作,立対般勤上前去。

皇帝身子不适,侧卧在榻上,睡得酣熟。房中药物的苦涩中有一缕淸香溢出,那是一种难得的汤饮,几近失传,唯宫中仍有秘藏,名叫桑落青梅饮。每至桑落时,取存着的青梅和泉水酿制而成,香醑淸甜,又有微酸,别调氛氲,真是淸香四溢,闻之心悦。

嬿婉知道多半是皇帝饮药后嘴里发苦,喝了这个,于是问道:``太医来过了?''

果然李玉道: ``是。己经喝了药,皇上才睡下了。''

嬿婉问:``何不早来禀告本宫?''

李玉倒也会说话,``皇上连容妃和惇嫔那儿也未知会,只打算睡会儿就好。但皇贵妃不一样,您位分尊贵,底下人必要来禀吿。''

这番话听着舒心,嬿婉也不敢与李玉这个皇帝跟前的红人多计较。恰见桌子上放了一盏紫铜飞鸾烛台,雪融纱灯罩上面画着笔挺一枝蘸水桃花,光晕朦胧,泛着流水漾春的暖意。

嬿婉随手拨了拨,调转了话头道:``是暖雪灯,放在这儿倒也别致。''

李玉忙道:``是。皇上前些曰子吩咐的,以后都用这个灯。''

皇帝本就生得白净,加之风寒体热,双颊上泛起酡红,轩眉漆黑,让光影映着面颊,越发显得轮廓有致。

殿中有汤饮的甜香,中人欲醉。

她记得《诗经》里的句子,皇帝曾经教过她,还是听翊坤宫中的人念过:桑之未落,其叶沃若。于嗟女兮,无与士耽。桑之落矣,其黄而陨。士也罔极,二三其德。

有些句子记得模糊,她还记得最末的诗句:及尔偕老,老使我怨。淇则有岸,隔则有泮。总角之宴,言笑晏晏。信誓旦旦,不思其反。

那仿佛,是一个女子错付了终身的诗。

嬿婉来不及喟叹,那是故事里的事,与她并不相干。人世花开花落,她顾着自己还来不及。

她想着皇帝这回风寒突如其来,若能悉心照顾左右,说不得会勾起皇帝旧情,缓和她与他实则脆弱无比的关系。于是她上前细看皇帝,轻轻唤了皇帝几声,见皇帝只是熟睡,也不敢再唤。

嬿婉松一口气,``皇上忙于国事,偶感风寒也是有的,只是下回你得提点着,别让皇上伤身。''

李玉苦笑:``是,只是奴才劝不住。''

这些年皇帝的性子益发孤行,嬿婉当然知道。当下也就吩咐了李玉出去,自己一人伺候。

李玉忙道了是,含着一抹笑跪安出去。

嬿婉殷殷挪过一个十香花团锦软枕,轻轻抱住皇帝的脖子意欲放柔了伺候。皇帝忽然一动,挪了挪头,眼角忽而有一滴晶莹滑落。嬿婉暗暗吃惊,更加纳罕,只觉得心里无数个念头突转,目光忽然落在榻上一只音玉匣子上。

她知道的,那是皇帝的爱物。心底的曲意温婉忽然凝成了一抹冷笑,她目光冷冷注视,见匣中竞是空的,并无他物。

哦,这么些年了,皇帝病中决绝,终于肯撂下她了么?

嬿婉心头一松,正要扬起唇角。忽然瞧见皇帝家常穿的赭色团福袍的胸前,露出一色娇艳。她的心思微微一颤,伸手一扯,才见皇帝虚拢胸前的是一方丝絹,大约是经年的旧物了,还是乾隆初年的花样,绣着几朵淡青色的樱花,散落在几颗殷红落枝之侧。

那一年,她还是叫青樱,他也只是弘历。

嬿婉怔在那里,仿佛那丝绢的无数细丝一根根剌进心里,千头万绪,茫然受痛。迷茫间,有琐碎的记忆纷繁沓至,他最喜欢的那出戏,是《墙头马上》。櫻花开时,他最流连。还有最得宠的惇嫔,也是与那人有着几分相似的容颜与性情。

她忽然想起来,今天是什么日子。数年前,便是数年前的七月十四,有一个人,用一把匕首,了断了自己的一生。

原来如此。原来,如此。

这场风寒发热,全是由此而起。

嬿婉心头大恼,双手颤顫,只欲撕碎了这绢子才能泄了大恨。然后这念头不过一瞬,她瞥见皇帝侧颜,便生了害怕。她犹豫片刻,终究放下绢子,慢慢地移到他身边躺下,轻轻抱住了他的臂膀,将头埋于他胸前。这样斜着的姿势并不舒服,足下的麻意慢慢攀到手臂,攀到肩膀。良久,仿佛连心也麻木了。她明明抱着他,他的手臂在怀中发烫,却并未有半分实在的暖意。她一点儿都不想靠近他,拥住他,可是没有办
法,她实在需要一个依靠。因为她此生所有,皆是源于这个男人,

她低首去寻,寻自己的手指,她恍惚觉得若是此刻指间有着那枚红宝石粉的戒
指,或许,或许会好受一些。

可是,早已寻不见了。或许那枚戒指,早随着凌云彻,一起堕入无边黑沉之地。

巨大的震恸之后,唯剩了永息般的麻木,她却觉得自己这一生从未像此时此刻一般清楚明白过。她慢慢地笑出来,这半辈子的恩遇荣宠,荣膺皇贵妃,执掌六宫,位同副后,不过是一场虚空。这一生一世,她与皇后的宝座那么近,却那么远,再无接近的可能了。

因为她知道,她明明以为击败了的,却永远在那里,不曾离开。

可是,早已寻不见了。或许那枚戒指,早随着凌云彻,一起堕入无边黑沉之地。

巨大的震恸之后,唯剩了永息般的麻木,她却觉得自己这一生从未像此时此刻一般清楚明白过。她慢慢地笑出来,这半辈子的恩遇荣宠,荣膺皇贵妃,执掌六宫,位同副后,不过是一场虚空。这一生一世,她与皇后的宝座那么近,却那么远,再无接近的可能了。

因为她知道,她明明以为击败了的,却永远在那里,不曾离开。

从此,那日子便跟落了灰似的,风尘仆仆落下,再也抬不起眉眼。不为别的,只为一颗心就这般灰了。日子跟熬油一般,也熬到了九年之期。勉强振作精神处理后宮的大事,是己然晋为惇妃的芙芷生下了一个女儿,序列为十,人称十公主。

皇帝听得喜讯时,正在梅坞听着戏子们唱《墙头马上》。音韵袅袅,挑动前尘往事里的桃红心事,倒叫这日渐老去的天子动了温柔心肠。

真的,声音是不会老去的,就像曲子里的情事,少年的眉梢眼角,都是藏不住的情意。不像壁上挂着的那幅《湖心亭看雪》的绣样,就算爱护己极,都有了微微泛黄的痕迹。更别说绣这幅画的女子,早己过世许多年了。

自永璘出生,紫禁城九年间未曾闻儿啼,皇帝六十五岁上又得了这个公主,且是盛宠不衰的翊坤宫惇妃所生,真是爱得不知该如何是好。几日几夜逗留在翊坤宫内,
抱着不肯放手,一切封赏都按皇后所生的固伦公主之例安排,倒是惹得颖贵妃感叹不已,这情状倒是像极了当年翊坤宫皇后生五公主时的盛况。

嬿婉是且喜且忧。喜的是惇妃这一胎是女儿,绝不会危及亲生子永琰的地位。忧的是皇帝爱宠幼女,总让她想起昔年五公主慘死之状,梦魇心悸之症又重了几分。

自从恩宠渐薄,嬿婉便添上了这个心悸的症候,常年延医问药。好好的人,几年的汤药伺候着,没病也成了大症候。皇帝倒是来看了她几次,总叮嘱她好好保养,日常宫中琐事,交给庆贵妃、颖贵妃都好。偏偏嬿婉要强,太医说她有病,她也不肯承认,更不肯分权于颖贵妃,死命挣扎着,越发疲惫不堪。于是再有宫务,皇帝也少与她说了,就是七公主的婚事,更是一言不与嬿婉商议,径自与颖贵妃定了,将七公主许配蒙古,定下了终身之约。

这一喜于颖贵妃是非同小可。她本出身蒙古,膝下并未有亲生儿女。得以养育七公主,乃是皇帝深恩,如今皇帝将七公主许嫁蒙古颖贵妃母家,从此满蒙联姻更深,颖贵妃在宫中的地位更是稳若泰山。

宫中闻此喜事,都向颖贵妃道喜,似乎忘却了嬿婉才是七公主生母。七公主眼里从未有这个亲娘,自然不来问候,便是擷芳殿养大的九公主,也不过循例来道喜了一回,稍稍问候便起身走了。

母女情分,不过如此。嬿婉添了一重伤心,终日辗转反侧,更是夜不能寐,虚弱憔悴得不成样子了。

春婵竭力安慰:``小主一切只看着几位阿哥吧。他们才是您的指望呢。''

嬿婉也想安慰自己,可心里酸得言语不得,只得一壁咳嗽,一壁叮嘱春婵:``贺礼再添上三倍。这几年来惇妃得宠,一路从常在升到了妃位,又让皇上老来添女,皇上一定很高兴。''

生个公主而己,也能算福分!春婵心里嘀咕着,却不敢说出口。若是数年前的她,一定会毫不留情地吐出这句讥讽之语。然而这些年,她所侍奉的皇贵妃不过维持着一个空架子,圣眷,早就不在永寿宮停驻了。皇贵妃一言一行战战兢兢、如履薄冰
不说,还要受着底下嫔妃们的冷眼闲气,长久的夜不能寐之后,心悸之症更重。所谓荣华富贵,不过是熬油般度曰罢了。可皇帝好像还是不满意,七公主的婚事只和颖贵妃商议,九公主和永琰的婚事,那是圣意裁定,一句也未问过生母的意思。情势如此,便是她这个心腹,也得学着低头安分。

但是说来,皇帝对嬿婉的儿女们还是很不错的。七公主成婚前封为和硕和静公主,嫁了蒙古亲王拉旺多尔济。然而这份体面,足足是给了颖贵妃的,既是全了她养育七公主多年的情分,又全了蒙古的面子。满蒙联姻,是颖贵妃圣宠十数年不衰的维系,皇帝这番安排,是要将七公主与养母的恩情更重几分,也是对蒙古诸部的看重。

为了这份恩典,听闻颖贵妃私下数度垂泪,感激皇恩深重。便是七公主,也因为嫁的是蒙古亲王,皇帝特意恩许七公主可以随时进宫看望养母颖贵妃。

自然,这些恩典里,皇帝对生母魏嬿婉,是只字未提。然而七公主嫁得好,嬿婉怎敢去添这份不痛快。转眼九公主和恪出嫁,嫁的是兆惠将军的儿子札兰泰。兆惠是朝廷里举足轻重的臣子,武功昭昭。虽然是圣心独定,嬿婉也是满心欢喜。而这位少年皇子,如同冉冉而生的朝阳,贏得了皇帝的注目与关爱。两位姐姐的好姻缘,是给十五阿哥铺好了太子之路。也足见皇帝对永琰的看重与疼爱。

是呢,前头的皇子们死的死,出嗣的出嗣。十五岁的永琰,怎么看都是皇子里最出色的选择。去岁永琰也有了许婚的指望,未来的福晋喜塔腊氏也是皇帝亲定,只不过并非名门大族,嬿婉便有几分不悦,深觉配不上足以令自己骄做的儿子。但无论如何,成婚后便有加封亲王的指望,那么他朝成为太子,也更有希望了吧。

嬿婉这么想着,连入口的汤药也不觉得难以下咽了。何况今日,又有另一重期盼。自从病后,皇帝对她见子女的次数也没那么限制了。至少永琰,可以在告知皇帝后过来永寿宫问安。

嬿婉念着儿子,更是强打了几分精神,笑道:``今儿永琰来,可得好好跟他说说话。''

永琰从养心殿请安出来,并不急着去永寿宫,难得见到九姐和恪,便多说几句话。自从姐弟二人被送到擷芳殿居住,不许生母常常探视,便多了几分相依为命之感,况且他们又是自小一起长大,不比七公主那般疏远。九公主和恪自从出嫁,见到弟弟的机会便少,这一日同来为父皇请安,倒能闲谈几句。提起刚走的七公主,九公主便有些埋怨,``晌午我去看了额娘,略坐了坐就出来了,总比七姐姐好,每回进宮都不去拜见额娘,只当自己是颖贵妃生的。''

永琰很能体谅七公主的难处,温言分辩道:``也难怪七姐姐,自幼不在额娘身边。便是我们,后来在擷芳殿长大,见得额娘少了,也是生疏。''

和恪略略点头,算是能接受这一说法。当日七公主大闹永寿宮,她是记得淸楚分明的。甚至许多年后,她都记得七公主对生母的评价------她是个坏女人,她与皇额娘的死有扯不清的干系。

幼年的她,并未将这话放在心里,甚至深为抵触。可是这些年,生母在宫里左右为难,父皇对生母的冷淡疏离,使她不得不去揣想,那背后真正的原因。那些晦暗的念头如蛛网蒙上心头,叫她烦恼,只得换了话头,挑些喜事来说:``等你有了福晋,让你的福晋多陪陪额娘。喜塔腊氏也算大族,会是个明理贤惠的福晋。''

永谈却苦笑:``额娘未必喜欢这门婚事。''

和恪有些吃惊,永琰会意,解释道:``你还不知道额娘的脾气?什么都想要最好。喜塔腊氏并非如富察氏、钮祜禄氏一般乃名门望族。额娘终究抱憾。''

和恪这般韶龄女子的心境,并不如嫔妃一般辗转求存,一心博宠,何况她天性温和,自以为天之骄女,自然不喜那些阴暗心思。听得生母的心事,她也只是摇头,
``难怪嫔妃不服,内外命妇笑话,额娘确是贪心不足了些,还背着杀害皇额娘的嫌疑。这些年,也不怪七姐姐厌恶额娘。''

儿女不言父母是非,和恪这番话,其实有些重了。永琰很明了她的处境,和恪以和硕公主身份嫁入兆惠府中,自然要风得风要雨得雨,尊贵无匹。可这些年,谁不在私下说一句,这样好的女孩儿,若是出自颖贵妃或是庆贵妃的肚子,前途更是不可限量了。

和恪说完,也有些黯然。她一身浅紫云纹折枝桃花笑春风的锦袍,衬得面容如晨间凝露的青莲,明媚恬静,不可方物。永琰暗暗想,其实他们的生母很少有这般恬和的容颜。太多的欲望,自然让母亲的面庞明艳无匹。可那样多的欲望,任何人都不会喜欢的吧。{[}花。霏。雪。整。理{]}

永琰抬头望着宫苑冬日暗沉沉的天空,默然叹了口气,便往永寿宫去。

永琰来时,嬿婉己经打扮停当,看不出常年卧病后那种消沉的气色。永琰循例问了嬿婉安好,又关心太医用什么药,便道:``额娘若是夜里能睡得安稳,这病就先好了五分了。''

嬿婉怎能安睡,一闭眼,就想起那年深夜,皇帝疑云深重地看着她的眼。那是噩梦的初始。

嬿婉笑笑,敷衍了过去,但见儿子只低着头,便道:``你七姐姐和九姐姐是女孩儿,婚事额娘不能置喙也就罢了,可你是额娘的儿子,怎么不能由额娘说了算?想想真是心酸。''

她难得见儿子,私下相处,难免吐露心事。

永琰还是低着头,好声好气地分说:``额娘,喜塔腊氏门楣不低。''

嬿婉一提起这桩婚事,就颇有怨言:``那也不是出身富察氏、钮祜禄氏这般八大姓氏的家族。她阿玛不过是个副都统,实在对你无所助益。''

永琰赔着笑:``姐夫们都是好家世,圣旨已下,任谁也不能变更了。额娘宽心,想想您已经是皇贵妃,还有什么不足的?''

嬿婉想说什么,忽然气息急促,春婵熟练地替嬿婉抚着背心,递上一粒药丸,嬿婉才有继续说话的力气,``都说母凭子贵。额娘已经是皇贵妃,还能贵到哪个地步?苦心保全了自己半世,没有一日能睡得安稳。若真有登上后位那一天,也算能松一口气了。''

原来病到如此,还有这般念想。永琰垂目望地,益发不肯抬头。是了,他不肯抬头,是有几分害怕,害怕抬头看见生母脂粉过于浓重的面孔。为了掩饰病容,云鬓高髻点满了珠翠琳琅,精心修饰的容颜用浓腻厚重的脂粉紧紧绷住,不见一丝细纹,却也让人看不出本来面目。嬿婉喜用百合香,房中大把大把地燃着,以掩盖常年药草充斥的气味。那药气裹着香气,直冲得他睁不开眼睛。

还是不看的好。

嬿婉未曾察觉儿子的心思,絮絮道:``旁人都喜欢额娘己经贵到了极处,这些年外人看来,我顺风顺水,没有一样不如意的。可额娘觉得自己不如意的事太多了。''

语中心酸,永琰如何不知,可他能劝慰什么,许诺什么,只得道:``额娘素日保重,心思轻些便好了。儿子,儿子改日再来看您。''

嬿婉也知道,儿子不能在永寿宫逗留太久,免得皇帝生疑。可这般急促离开,她又怨尤无比。眼看着儿子出去,一颗心空落落的,更没了依靠。想了半日,恍饱记得今日是什么日子,偏是记得不清不楚,还是春婵吞吞吐吐提起,是嬿婉母亲的生辰。多少年了,她也早是没有父母垂爱之人,便是亲兄弟佐禄,也早不来往了。佐禄并非不清楚母亲是为谁而亡,对这个亲姐姐,恨之入骨。

心沉沉地跳跃着,每一下都带着抽搐的悸痛。这种痛,这些年,她也熟悉了,习惯了。心痛之下是最深的失意,兄弟不成兄弟,儿女不像儿女。便是母亲在时,对她又有几分真心关爱?她这般想着,瑟缩着身体往墨狐大裘里钻去,希冀得到一点温暖。殿内虽然燃着数个炭盆,地龙也传来融融暖意,或许久病孱弱,她还是觉得冷。窗外己经刮起了朔风,击打着暗红的窗格,嘶鸣于幽长复幽长的宫墙。那风声,和数十年前并未两样。那时候,哪怕自己再卑微,也有人真心怜惜,只是这辈子唯一对自己真心的那个人,己经死了。被自己亲手害死了。

嬿婉怔怔地想着,两行淸泪,无声婉蜓而下。

\hypertarget{ux7b2cux4e8cux5341ux4e5dux7ae0-ux5e7dux68a6ux4e0b}{%
\chapter{第二十九章
幽梦(下)}\label{ux7b2cux4e8cux5341ux4e5dux7ae0-ux5e7dux68a6ux4e0b}}

海兰跪坐在佛像跟前,久久地,一下,又一下,缓缓拨动着手中的碧玺佛珠。若不是这样滞缓的动作,提示着她还有一丝活人的气息,那么一身暗蓝半就宫装的她,与一株枯朽的草木全无分别。

婉嫔示意宫女退下,缓缓步至海兰身边,轻声道:``愉妃姐姐,我的日子过得和你没有两样,叫我来瞧瞧你,跟瞧我自己有什么不同呢?''

海兰慢慢地睁开眼,逆着光吃力地分辨婉嫔昏暗而模糊的容颜,莞尔轻笑:``宫里的老姐妹没几个了,大潜邸里一起出来的,也唯有我和婉嫔妹妹你了吧?''

这一句,便勾起了婉嫔积郁的伤心,叹息如秋风,``这么多年,也就姐姐还肯惦记着我。旁人眼里,咱们俩喘着气和不喘气了是一个样儿的吧?''

海兰蓄得长长的指甲剥剥地触在古旧的青石砖地上,发出枯哑的涩涩声。那声音在静得可怖的殿里,有着茫远和细微的回声,听得久了,便也没那么寂寞了。她淡淡道:``这么多年,是多少年了?离皇后姐姐杭州断发之日,已经快十年了吧。''

婉嫔默然垂下花白的首,掰着枯瘦的手指,暗金色的戒指在暗寂的殿内闪着昏而淡的光芒,``是啊。翊坤宫娘娘断发之日是乾隆三十年闰二月十八,是要十年了呢。''她艰难而苦涩地笑了笑,``翊坤宫娘娘离世多年,如今宫里敢提起她的,也就只有咱们老姐妹俩了吧。''

海兰瞥她一眼,笑容幽淡如幽夜的昙花,``你倒不怕?''

婉嫔不自然地笑笑,摸着斑白的鬓发,``一辈子无子无宠,有什么可怕的?我便是在宫里说上一日的翊坤宫娘娘,怕也无人会来理会吧?''她侧耳,凝视听着窗外热闹的鞭炮声,已经是正月二十五了,宫里的热闹还没退呢。那鞭炮声好听是好听,就是听着闹心。``咦?谁宫里唱着昆曲呢,真是好听。''

海兰伸出手,缓缓抖落暗蓝色绣银线折枝五瓣梅衣襟上薄薄的尘埃,``是令皇贵妃传了戏班子,只是除了晋嫔爱应酬,没去几个人。''

婉嫔掰着手指头算日子,``九月初九是她的生辰,今年五十大寿,皇上总会给她热闹下。这点面子,还是有的。到底儿女争气,都有了好出路。''

海兰懒懒道:``九九重阳,她也真会挑出生的时辰,难怪这么有福。''

婉嫔有些感伤,``说来愉妃姐姐的生辰是五月初四,我的生辰是十二月二十,除了内务府还记得送一卷银丝面来,怕是谁都记不得了。有一日皇子起了性子,不知怎么派人送了十卷湖州进贡的丝绸来,喜得我不知怎么才好。谁知送绸的太监却说皇上是贺我的生辰。那一日明明才十月十四,与我的生辰风马牛不相及啊。''她自嘲地拍了拍手,``不过话说回来,我这一辈子都这么过了,倒也算了。''

海兰之着地上的软垫蒲团起身,点燃一束香高举于额头前,淡淡道:``自从姐姐过世,我便再没有过过自己的生辰。乌拉那拉如懿既死,活着的珂里叶特海兰也不过是一具行尸走肉。要不是念着翊坤宫曾嘱咐我不得轻生,要不是为了永琪留下的遗孤绵亿,要不是为了照拂姐姐的永璂,我这把老骨头活着,还有什么意思?''

婉嫔羡慕地看着海兰,扶过她一起在长窗的锦榻边坐下。那锦榻虽说是锦绣堆砌而成,却也不知是用了多少年了,边角都起了毛毛的絮儿,映着昏黄的天光,露出白惨惨的模样。海兰浑不在意,亲自取过一把用旧了的白玉青梅五瓣茶壶斟了一盏清茶递与婉嫔手中,和声道:``尝尝,是皇上年下新赏的茶,说是给我和绵亿尝尝新的。''

婉嫔啜了一口,打量着殿中的器具,叹道:``茶是上好的,可见皇上还是记挂着姐姐和绵亿,年下的赏赐也是不少。说起来,皇孙辈里,皇上最疼的也是绵亿了。''她柔缓道,``既然如此,姐姐何必这么苦了自己?这些东西用着,也太寒碜。''

海兰爱惜地抚摸着那白玉青梅五瓣茶壶,``我宫里所有的这些东西,都是姐姐在时赏赐下来的。人啊,用着用着生了感情,怎么也舍不得丢了。左右都是老婆子了,还讲究什么。''

婉嫔懂得地摇头,``满宫里,也唯有姐姐还念着翊坤宫娘娘的好儿,初三那一惇妃生下了十公主,皇上可欢喜得不得了呢。我去瞧过,十公主长得真是可爱,和多年前的五公主,像是一个模子里刻出来的。''她言毕,似乎意识自己说错了什么,惯性地受惊似的低下头,戚戚地拿绢子抵在鼻首,道:``如今,翊坤宫可是一点儿连皇后活过的影子也没有了。新的宠妃,新的孩子,全落在了那里。人人都高高兴兴的。令皇贵妃也会高兴,最儿女双全的可不就是她了么?这个五十大寿,她可真有福。''

海兰把玩着手中得茶盏,指间枯深得纹理如同她的声音一般沉而暗,``婉嫔妹妹,你可说错了。惇妃的性子是像足了年轻时潜邸里的翊坤宫娘娘,十公主更是长得如五公主再生。有她们在,翊坤宫少不了姐姐的影子。从惇妃一进宫,那便是定了的事儿。那都是皇上的意思。可令皇贵妃能不能庆她的五十大寿,那可都是你的意思。''

婉嫔下了一跳,睁大眼睛盯着海兰,诧异道:``愉妃姐姐,你说什么呢?这样的话可不吉利,若是落在皇贵妃耳中,得生出多大的风波来。''

海兰笑得温婉而贤淑,却看得婉嫔浑身发毛,情不自禁地向里缩了缩身子。海兰柔柔地道:``我说什么?婉嫔妹妹若是不明白,又躲什么呢?''她气定神闲地抿了一口茶,``今日与妹妹一席话,才知妹妹多年在宫中不言不语,却也装了满腔心事的。''她摸着花白的鬓角,轻声道,``赏赐归赏赐,供养归供养。皇上顾着颜面,咱们哪一日也没有被慢待。可是,生了皱纹,白了青丝,有谁正眼看过一眼呢?活在这儿的每一日,又有哪一刻是为自己活的?生辰可以被记错,容颜可以被忘记,但是这口气,这条命呢?都是白白来这世间走了一遭么?''

婉嫔似乎有些害怕,发出嘤嘤的细小声音,像是墙角苟且偷生的蝼蚁一般,``愉妃姐姐,我活着唯唯诺诺了一辈子,那怕慧贤皇贵妃在的时候,孝贤皇后皇着的时候,还有翊坤宫娘娘,我什么人也不得罪,什么话也没乱说,我已经平平安安活了半辈子了。我什么也不求了。''

``人活着没有一点儿声响,人死了更没半分动静。这样活着,和蝼蚁有什么区别?做了几十年的婉嫔,最后一次待寝还是乾隆二十五年吧。那时候,若不是魏嬿婉利用你集齐皇上悼亡孝贤皇后的诗文,利用你动摇姐姐的地位,你又如何能有那几日的恩宠?可是呢,到头来也是徒劳。''海兰慢悠悠道,``将来死后,你会怎么被记下来。婉嫔陈氏,事乾隆潜邸。乾隆间,自答应累进婉嫔。这几个字,费不了史官多少事儿,连哪年死的都未必会写下来。嗯,来日葬在哪里呢?咱们倒是能就一辈子的伴儿,皇上在乾隆十七就为自己建好了裕陵,二十七年妃园寝也已建成,总有咱们的一席之地,冷冰冰地就个伴儿。''

婉嫔畏惧地打量着笑容平静的海兰,怯生生地伸长了脖子,有些按捺不住了好奇,``你想我说些什么话?''

海兰从袖中慢慢抖出一卷薄薄布帛,扔在她跟前,``这些年令皇妃做过的事,都在这儿了。你照着说就是。''

那布帛仿似断了翅的鸟儿,轻悄悄扑在婉嫔身前,溅起蓬勃的浅金色的尘灰,旋在低低的空中,自由地扬起。海兰盯着她,徐徐地带着蛊惑的意味,``看一眼吧,很多事你一定也很想知道。那就看看,看一眼也不会出什么大事。''

婉嫔像是被无形的绳索牢牢缚着,僵直地缩着身体,一动也不敢动,一双眼珠子瞪着老大,仿佛要将那布帛给瞪化了似的。海兰浑不理会,只是拣了串碧玺佛珠在手,一下一下缓慢地拨动着,以指尖与佛珠冰凉的相触声,来抵御此时此刻呼吸的绵远悠长。

也不知过了多久,婉嫔终于忍不住伸出手,抖索地抖开了布帛,一字一字看下去。她的鼻息越来越重,嘴唇无声地张开,如同濒死的苟延残喘的涸辙之鲋。她陡然扬起手中的布帛,压抑着尖声道:``皇贵妃做的下作事再多,干我什么事呢!我才不去!''

海兰薄薄的唇勾起一抹娆柔笑意,伸手亲昵地抚了抚婉嫔身上的藕荷色茧绸绣米珠团福绣球的锦袍,那领口出着细细风毛,如它的主人一般经不得半点惊吓似的,``就算你活腻了,我还没有呢。皇后姐姐死了,永琪死了,我还活着。不只为了永琪留下的这一点骨血绵亿。还有一件更重要紧的事。那便是只有我自己明白。我要是死了,谁还能记得皇后姐姐活在这尘世上的一点一滴呢。皇后姐姐人不在了,可我们一起度过的日子,一天天都在我脑子过一遍,我什么都记得。''

婉嫔一脸震惊与不可置信,一只手将那布帛团抓在手心,双眼怔怔地盯着海兰灰败而憔悴的面容,痴痴道:``你便这样,这样惦记着翊坤宫娘娘?''

海兰凝视着佛像前冰纹青瓷瓶里供着一束绿梅,那雪白如茧丝般的冰裂细纹,如同敲碎在她心上,清晰地蔓延。她甚至能听到那纹裂时刺耳的声音,绵延不断、痛彻心扉。无数的往事夹着如懿清澈德笑容纷纷扬扬如雪花落下,晶莹而冷彻骨髓。

眼底有温热的湿润,阴影里佛祖宽悯慈悲的脸容晦暗得毫不分明。她只觉得荒唐,荒唐得不可理喻。世间的混沌翻覆里,唯有如懿记得她,可是偏偏连如懿,也再不能在身边。她嘶哑着喉咙,任凭泪水潸潸而落,``我不惦记着皇后,我怎能不惦记着皇后?这一生一世,除了我的孩子,唯一惦记着我念着我的人只有皇后姐姐。婉嫔,你是最清楚的,人活一世,不过是图一个记得。有人记得你,牵挂你,念着你,才不是孤零零地来世间走了一遭,不是么?''

婉嫔的眼底闪着晶莹的泪水,那泪光里燃着阴阴的火。她身子扭曲着,几乎要夺门出去,可她的脚却定定地长在地上,跟生了根似的,她低低地压抑地叫着,``你要记得,就自己说去便是!扯上我做什么!''

海兰不疾不徐地迫近她,任由泪水肆意,口气温柔得几乎要化了,``我去?我去皇上会信么?这辈子,我就是和姐姐最要好了,任谁都知道。皇上不会信我的话,他不会信任何一个与人结党交好的人的话。前朝是这样,后宫也是。''

``可那是不成的!''婉嫔几欲泫然,紧紧地攥着海兰的袖子,靠近着她,``令皇贵妃有儿有女,每次失宠都有本事翻身。翊坤宫娘娘死后她更独揽六宫大权!我算什么,我就是一个小小的嫔位,连大声说话都没有听见的小小嫔位。''

``旁人听不见不要紧,只要皇上听见。''海兰意味深长地凝视着她,眼底有深海玄冰般的冷光,``这样的事,只有你能试一试。''她轻轻一嗤,伸手抹去腮边的泪痕,端然收回身体坐直,``旁人听不见不要紧,只要皇上听见。别以为皇贵妃有多么大的万千荣宠,这些年熬下来,她早已不堪一击。只要,出拳的那个人,是皇上。那便是谁也抗不过的。''

婉嫔仍是抗拒,``不!为什么不让惇妃去?她那么得宠,皇上会听她的!''

海兰微笑,那笑意轻飘飘的,``惇妃?她不过就是姐姐的一个影子。她的存在,是时时刻刻提醒着皇贵妃,姐姐并无离开这里,她依旧在皇上心上。''

婉嫔将信将疑地盯着她,呆了片刻,沉声道:``可是,我会死的。''

海兰屏声静气,端端正正地坐在榻上角落的阴影里,酸枝木榻上铺着一色半旧的灰绿茵绒褥子,越发映得她像长在潮湿墙角里的青苔,阴绵绵的没有生气。看得久了,仿佛人也成了木头,呆滞而僵硬。外头想着连绵的爆竹声,噼啪,噼啪,是火药气息的热烈与绽放。那热闹是属于别人的,与她们并不相干。海兰冷笑了一声,``你这样活着,或者死了,在旁人眼里有区别么?明明你还在喘气,多少人眼里,你就是死的!行尸走肉!和我一样!你听外头的鞭炮,那么短促还得响一声,落个动静呢。你呢,谁记得你?''

婉嫔怔怔地听着,也不知过了多久,爆竹喧嚣的气味散得尽了,她软弱地伏下身体,倚在海兰膝边,一下一下,死死绞着手里素绢巾子。``已经几十年了,我伺候皇上已经几十年了。这几十年里,我受过恩宠,掰着手指也数得出来。皇上给了我位分,给了我恩养,他算不得辜负我。可是这一辈子,他有那么多女人,那么多宠妃,他从来都不会记得我吧。''她低低呻吟一声,像是自嘲的笑,又像是悲戚的哭,``于皇上而言,我和寝殿里的一个枕头、一床被子有什么两样?用过便也用过了,抛之脑后。海兰姐姐,我只想要皇上记得我,我不想成为妃陵小小的墓穴里一个无声无息的亡魂。人人都有过恩宠,只有我是捡来的运气。我只是潜邸里小小的侍女,偶而被皇上宠幸了,我才能活到这宫里来,我知道自己卑微,我知道自己受了不该受的福分。可我也是女人,我也会发梦,也会痴想,我活得能被人记住一次,一次就好。''

海兰静静地坐着,听着她呜咽的哭声,缓缓落下泪来。

那一夜,无人知道青衣简装的婉嫔,随着李玉悄然步入养心殿,对皇上说了什么。

红蠋长照,明彻一夜。

婉嫔只是在天明时分疲倦地坐上小轿,见到等候在自己宫中的海兰,轻轻道:``我这一辈子都没对皇上说过那么多话。可是皇上,他居然愿意听说了那么久。''

海兰揽过她,轻声笑道:``那是因为妳说的话都很好听,皇上喜欢听。''

婉嫔倦倦地将头底在海兰肩头,``这些话都是你逼我说的。可是这样被你逼迫一次,真是痛快。我从来没有那么痛快过,我喜欢谁,讨厌谁,我都说完了。那怕立刻被皇上拖出去砍了脑袋,我也不后悔!''

海兰沉静地抚摸着她的脸庞,神色从容,``你说话的声音真好听。满宫里只有你能对皇上说出那样好听的话来。皇上喜欢听你说。''

婉嫔闭着眼睛,眼皮有轻微的颤抖,扇起睫毛如将欲飞翔的翅膀。她的妆容在晨光里有些许模糊地融化了,她的容颜却异常宁和,``我知道,因为我无争无斗活了半辈子,我谁也不依附,谁也不得罪,我活得连一粒尘芥都不如。可是,我说了那么久,连我自己都不记得自己说了什么。''

海兰温柔地微笑着,``嗯。人活一口气,那话便是随着气儿就散了的。你不记得也好。只是皇上呢,皇上记得什么?''

婉嫔的眼皮倏地一跳,``你教的我说过便都忘记了,自己的那句,却记得牢牢的。''

海兰苍老的眉心有不安的褶皱,``你自己?你自己说了什么?''

婉嫔郁郁叹息,``话再多,皇上难免信。他问我,他看着我的眼睛问我。这些事,我如何知道得这般清楚?我便说,皇上,您不在意我,旁人也小瞧我,却不知越是如此,越多是我便悄悄地看得更清楚。皇上半信半疑,便问我,那你为什么偏要到了这时候才来告诉朕?''

海兰的语气温柔得如三月檐下细软夹着花雨的风,眼神却死死地盯着婉嫔的颈,如锐利的针,几乎要穿透她疲倦的身驱,``你说什么了呢?你的委屈别藏在心里,都丢给皇上去。叫他好好看看,他冷落了数十年的女人,留的都是血泪。''

暂时的静默,几乎逼仄得人透不过气来。她觉察到那液体的灼热,心底蓦然勾起了几丝震颤。许多年前,她也是这样依靠着另一个人,以为这样彼此扶持着,便能度完这喧嚣而无趣的一生。却原来,她们连一生的收梢都不知零落何处,望也望不见。

婉嫔闭着眼,像是怕到了极处,蜷缩在她怀里,蓦地睁开眼睛,直直地看着海兰,硬声道:``是。我告诉皇上,可是我晓得,我的委屈不重要。皇上听了一时怜悯,过去便过去了。我知道皇上最怕什么,我知道。''她压低了嗓子,如吐着芯子的蛇,嘶嘶地道,``我看着皇上,我说,皇上,臣妾从前不敢说,可如今十五阿哥大了,出落得俊秀勇毅,是咱们大清未来的栋梁。臣妾拼死,也不敢不说了。''她咬了咬牙,下了死劲一般,``我说,皇上,若来日十五阿哥成了大器,有皇贵妃这样得额娘在,来日我们大清江山,便要落入谁家了?''

海兰震惊到了极处,``你说了这样的话?''

婉嫔重重地点了点头,有着难掩得惶惑,牵着她的衣袖依依道:``我知道的,今日我既开口说了这些,若不能将皇贵妃置于死地,来日还有我的活路么?与江山相比,数十年载恩情算得什么?虽然这些年我从未赢过,但事已至此,我也绝不能输了。''

海兰极力安定下自己有些紊乱的鼻息,骤然松了口气,轻轻抚着婉嫔花白蓬松的的鬓发,了然笑道:``怎么?你也恨毒了皇贵妃么?''

``我原本,只是为了争一口气,才说出你教我的那些话,也当是为我,为你,为仙逝了的翊坤宫娘娘出一口恶气。因为这么多年,我做什么像什么样子,做底下的侍女有侍女的样子,做格格有格格的样子,做嫔妃有嫔妃的样子,可浑不像个人的样子,不敢说,不敢做,不敢动。如今我说得越多,才越知道,这数十年来,我心里的恨原来那么多,因为我最寂寞的年岁里,是她在皇上的温柔与缠绵里绽放得如火如荼。''

海兰的声线柔和得几欲叫人沉醉,``皇上最忌讳的,哪里是她害了多少人,而是如何专权恣肆,目无君上。当年她害皇后姐姐的,不也是如此么?''

婉嫔微微出神,眯了双眼,``可是哪怕我这般说了,皇上也未必会信。''

海兰轻轻一笑,``不要紧。我从来不是要皇上深信不疑,我只要皇上疑心。疑心生暗鬼,皇上性子最多疑不过。多少人便死在了`疑心'二字上,我便不信她能逃脱得了。''

婉嫔攥着海兰的青筋凸起的枯瘦的手``海兰姐姐,如今我知道翊坤宫娘娘为什么喜欢和你一块儿了。你的手真暖和,你的话让人听着舒服。你别走,你在这儿陪陪我,咱们姐妹,就个伴儿。''

海兰看着窗外渐渐明亮的天色,好像一张女人涂得粉白的绝望的面孔,流下赤红色的眼泪。这样一日日孤独地看着日出日落,真是寂寞。

寂寞彻骨。

可是身边的半老女子,何尝不是如此?自己,至少曾经有过如懿,有过永琪,有过永琪的血脉而延续的子孙代代,有过皇帝短暂却远比婉嫔长久得多的恩宠。所以她有念想,有回忆,支撑着度过每一个相似又乏味的日子。所以,她懂得婉嫔的寂寞,那种无声的寂寞,会把人慢慢地腐蚀,腐蚀成一个个蛀洞,然后风化成幽幽深宫里一缕被风吹过的尘沙。

皇帝再度见到海兰的时候,是在梅坞。这些年皇帝虽然关心永琪遗子绵亿的起居,也对海兰颇为厚待,但二人这般面对面说话,已经十数年都不曾有了。梅坞建成多年,海兰还是头一回来,她细细打量着梅坞的每一样布置,已然泪盈双睫。

皇帝拍拍她的肩,很是看重她的意见,``看看,喜欢这儿么?''

海兰舍不得移开目光,``梅坞,都是梅花。臣妾很喜欢。''

皇上听完这一句,很是心满意足,然而他谈论更多的,是甫出生的皇十女和孝公主。这位皇十女自在翊坤宫中出生,便得到了皇帝的无上钟爱。这样深切的慈父之情,让人恍然想起许多年前,那位同样在翊坤宫中出生,却早夭的五公主和宜。

皇帝又提起永琪遗子绵亿的近况,唏嘘不已。末了,皇帝忽来兴致,取出一斛南洋明珠赐予海兰,那明珠颗颗有鸽子蛋大小,华泽莹然。纵然海兰曾经跟着如懿见过色色真奇,亦是暗暗惊叹。

皇帝示意李玉将拿一斛明珠捧至海兰跟前,海兰只淡淡扫了一眼,含笑谢恩,不惊不喜。

皇帝道:``听说你成日吃斋念佛,闭门不出。延禧宫原本寒湿,不宜幽居,不如常来与朕闲话。算来潜邸里过来的人,也唯有你和婉嫔了。''

海兰笑着辞过,``臣妾年老迟钝,怕答不上皇上的话。这一斛明珠\ldots{}''她若有所思,``姐姐在时,喜爱珍珠。可惜在名贵的珍珠也珠黄之时。''

皇帝了然,``你想说长门自是无梳洗,何必珍珠慰寂寥?''

海兰浅浅微笑,``不,皇上恩泽六宫,臣妾感激不尽。听闻皇上新赐了皇贵妃一方西瓜碧玺,大若手掌。''

皇帝笑笑:``朕已命人雕琢成皇贵妃喜欢的水莲,让她拿在手中把玩。''

海兰想笑,还是矜持地抿住了嘴唇,皇帝久不曾有如此厚赏,那位皇贵妃一定很感动吧。

然而皇帝并无兴致继续关于皇贵妃的话题,这个时节御花园的梅花更得他的好感,海兰会意,便陪着皇帝出去。

皇帝温和的眼眸扫却了正月寒朔的冷意,将一袭紫貂大氅亲手披在她肩上。海兰并未有任何受宠若惊的表示。皇帝对她的平静在意料之中,轻轻挽过她的手,``愉妃,陪朕往御花园走一走。''李玉明白,忙带着宫人们退后十步,远远跟着。

冬日晴寒,天色湛蓝一碧。皇帝微微叹息,``已经有数十年了吧,你没有和朕一起走一走了。''

海兰浅浅笑,简短道:``是。''

冬日晴寒,天色湛蓝一碧。皇帝微微叹息,``已经有数十年了吧,你没有和朕一起走一走了。''

海兰浅浅笑,简短道:``是。''

皇帝略有歉意,``永琪英年早逝,你膝下寂寞,朕没有能多陪陪你。''

海兰恭敬而自然,``皇上为天下人操心,不必挂怀臣妾区区之身。''

皇帝驻足,静静凝视,``你仿佛从不为得宠失宠而在意。''海兰的眼睛望着地下,那连理并蒂的青石板镂刻沟壑处,积着意痕痕寒冰。天长地久,花开并蒂,也不过是僵死的冻痕,没有活气的期许。

皇帝见她只是无言,不自在地咳嗽一声,``朕知道,你不喜欢珍珠。喜欢珍珠的人,是如懿。''

他这般猝然提起这个名字,让海兰有些意外。她陡然抬起脸,牵动鬓边烧蓝晶石珠沥沥颤动。她很快镇定下来,``因为所以的珠宝之中,唯有珍珠和生命有关,让人觉得软弱。所以,皇上也不喜欢珍珠。''

皇帝颌首,``人老珠黄,有生命的东西,总是容易消逝萎败。朕也会老,所以海兰,朕喜欢长久的光耀的东西。可以提醒着,至少有不变的东西。''他停一停,``朕赏赐珍珠给你,是觉得,如懿喜欢的东西,你总该会喜欢。''

海兰无所谓地笑了笑,``也不一定。比如姐姐喜欢皇上,臣妾却不是。''

这样大胆而无谓的言语,连皇帝也不觉变了变色,颇不自在。海兰温然欠身,眸色澄净,``臣妾敬慕皇上,姐姐喜欢皇上。这是最大的不同。''

皇帝凝神须臾,轻轻一嗤,叹然道:``是。如懿如果懂得自下而上的敬慕,而不只是喜欢,或许她与朕也不致如此。''

长街的风吹得海兰半边脸发僵,她紧了紧身上软糯温实的大氅,紫貂的毛尖上出着银毫,软软地拂在面上,像曾经,她温柔地扶持着自己的手。

那一刻,她几乎要落下泪来,却惊诧地发现,她原来并不惯于在这男人面前落泪。她微微哽咽,``臣妾以为皇上永远不会想起姐姐,永远那么憎恶她。可皇上却没想过,当年您喜欢姐姐,也是因为姐姐喜欢您。''

``朕,并不憎恶如懿。''他的声音极轻,在自由穿越的风声里些模糊难辨,``朕只是不能接受,到了最末,朕与如懿,都改变了最初的模样。''他抚一抚她的肩膀,``海兰,谢谢你一直为她。所以那斛珍珠,你便留着,就当为她。''

海兰轻声谢恩,从怀中取出一枚红宝石粉的戒指,低柔道:``这枚戒指是姐姐当年命臣妾去赐死凌云彻时,凌云彻握在手里不肯放的。姐姐从没有这样不精致名贵的东西,臣妾很想知道,当年皇帝认定姐姐与凌云彻有私,是否是因为这枚戒指?臣妾不敢问姐姐,只得自己藏了。如今,只当还给皇上吧。''

``是有些眼熟。''皇帝接过,托在掌心。他盯了片刻,似乎在极力思索着什么。有眸中片段的记忆加深了他已有的疑心。这枚戒指,曾经长久地出现在一个女中手上。而似乎凌云彻死后,那双手上再没有了这枚戒指。

呵,他深切地记得,昨夜婉嫔的期期艾艾里,有那么一句,皇贵妃与凌云彻有私,却嫁祸乌拉那拉氏。而之后到来的那人,也并未否认。

那么这枚戒指,算不算一个铁证。

皇帝翻过来,看见戒指背面的痕迹,心下一阵冷然,口角却是微笑:``呵,是嬿婉。嬿舞云间。愉妃,你早就知道了,所以给朕看这么个铁证,是么?''

海兰静静道:``皇上认定姐姐与凌云彻有私,误会了多年。''

海兰看了看越色清寒。``正月二十八,还有二十日,就是姐姐与皇上彻底生分的日子了。''

皇帝的眉间有些黯然微微摇首:``是啊。一晃十年了。朕记得如懿去是之时,是四十九岁。''

海兰走近两步,轻轻微笑:``皇贵妃过了生辰,也是四十九岁了呢。今年他的五十大寿,不知会如何操办?''

皇帝微笑,眼底却有一抹凛冽闪过:``是吗?皇贵妃的寿数,未必就及得过如懿呢。''他一语如玩笑,倒是展臂替她兜上大氅得风帽,柔和地笑了笑,``回去吧。朕也走了,这儿过去,还能顺道看看婉嫔,朕也许久没见她了。''

这是难得得温柔,也算某种难以言喻的释然,她恭谨地目送皇帝离去,左手蜷在袖中,死死抓着一枚金累丝嵌珍珠绿松石蝶舞梅花香囊。许久,她才骤然想起,皇帝忘记从她身上取走那件大氅。

海兰这般想着,忽而念及婉茵,她最想见的人,已经来了呢。

钟粹宫自纯惠贵妃过身,唯有婉嫔寄身其中。数十载光阴匆匆,她安静而寂寞地活着,活得长久而不被打扰,如同这里的一草一木,都沾染上了尘埃苍旧的安息。

皇帝缓步走进来时,婉茵正在专心致志地伏案画画。直到同样好迈的侍女顺心转身去添水,才看见了在门边含笑而立的帝王。顺心久未见皇帝来此,一时未曾反应过来,不觉惊惶行礼,``皇上\ldots 怎么是皇上\ldots{}''

婉茵心无旁骛,细细描摹着笔下男子的侧颜,连眉角也未曾抬起,只是轻声细语,``顺心不要胡说,皇上很多年没来钟粹宫了。''

顺心连忙道:``小主,小主,真是皇上。皇上来看您了。''

婉茵吃惊地抬起头,手中的画笔一落,墨汁染花了柔软的宣纸。婉茵喜极而泣:``皇上,怎么会是您?''

皇帝含笑踱步而进,温言道:``朕说了,得空会来瞧你。婉嫔,这么些年,你就躲在这儿画画?''

婉茵大为不好意思,想要伸手去掩那画像,可那厚厚一沓纸张,哪里掩得去?倒是皇帝手快,已经细细翻阅起来,越是翻看,越是触动:``画的都是朕,年轻的,年老的。婉嫔,你画得真像。''

这一句话,几乎勾落了婉茵的眼泪。她眼底泪花如雪,轻声到:``画了一辈子了,熟能生巧。''

皇帝放下手中画像,不觉长叹:``婉嫔啊婉嫔,这么多年,朕没有顾及你,实在是有负于你。从今往后,朕会好好待你的。''

婉茵身子一震,不觉热泪长流,一时竟说不出一句话来。

皇帝笑着抚过她的脸颊,``怎么?朕吓着你了?''

婉茵自知失礼,连连摇头,脸上笑意渐浓,泪却止不住落下,显得狼狈不已。好容易安静下来,婉茵才小心翼翼道:``皇上,臣妾有一个请求,您能不能坐在臣妾跟前,让臣妾画一画您?''

皇帝诧异:``朕都来了。你还要画么?''

婉茵痴痴地望着皇帝:``皇上,臣妾第一回,离您那么近地画您。不是凭自己的印象和记忆来画\ldots{}''

一语未完,皇帝亦动容,眼见殿阁内一应朴素,便往那榻上端坐,牵过婉茵的手,沉沉道:``好,朕让你好好画。以后都让你好好画吧。''

婉茵心头激动,想要说什么,却不自觉地深拜下去,倚靠在皇帝膝上,再不肯放手。

皇帝摸了摸她妆点素净的发髻,轻声道:``婉嫔,你最远离是非,朕一直没想到,会是你如此留心,告诉朕这一切。''

婉茵的眼底有热泪涌动,她歉然道:``昔年臣妾曾被皇贵妃怂恿,使得翊坤宫娘娘伤心。这是臣妾欠了她的,臣妾要还。''

皇帝笑意酸涩,``欠了如懿?呵,欠她最多的人是\ldots{}''

万茵仰起头,不再年轻的脸庞满是泪水,``皇上,皇上,臣妾自知卑微,能得您一幸是一生最大的幸事。臣妾一直盼望着,您能回头看见臣妾,只要一眼,一眼就好。''

皇帝心底蓦地一软,柔声道:``会的。婉嫔,你与朕都已老去,咱们会相携到老的。''

婉茵想说什么,喉头一热,化作一声低低的呜咽,轻散在风中。

天色已然明朗,皇帝坐在太后跟前,亲热地递上一盏参茶,``皇额娘,天寒难耐,您得格外保重身子。''

太后年纪很大,越发慈祥,看着皇帝笑意吟吟。太后早已不管后宫中事,前朝之事更是听也不肯多听一句,只是赏花养鸟,游园听戏,每日逍遥度日,十分安闲。这一来,皇帝也更放心,二人逐渐亲近,母子情分到渐渐浓厚起来。再加之皇帝有补报之心,对太后极尽恩养,每逢大寿更加尊号、奉厚礼,操办隆重,天下同喜。这些功夫下来,彼此更见和睦。

此刻太后眯着眼听皇帝说完,便问:``你一问,她倒都说了?这么看倒也不是忠仆,怎么肯对你竹筒倒豆子一并都说了?''

皇帝眉间有阴沉之色,``澜翠身死,她就吓怕了。总觉得自己知道太多,命不久矣。便将这几十年的龌龊事,一并说了。''

太后默然片刻,叹道:``午后倒是永璂来给哀家请安,这孩子,总是闷闷的。''

皇帝也是感伤:``没有额娘,性子越发内向了。''他想一想,还是问,``皇额娘,儿子正好想问您,若是做额娘的实在卑劣,而儿女辈却出色,该如何处置?''

太后打量皇帝一眼:``当初汉武帝欲立刘弗陵为帝,弗陵之母钩弋夫人年少多媚。汉武帝怕子少而母壮,再现吕氏之祸,下令去母留子。汉武帝的举措虽然决绝,但不失为一个好法子。''

皇帝这才微现松弛之色:``皇额娘说得是。儿子也是这个意思。''

太后眼底有多沉重的复杂,``哀佳话到这个岁数,什么都看淡了。人活一世,想过想不尽的荣华,受过咬碎牙根的委屈。还有什么放不下的。皇帝,咱们母子都是高寿的命相,积德养福,早日放下介怀之事才好。''

皇帝缓一口气,沉声道:``等事儿一并了了,才是真正放下。有些人的心太大了。儿子还在呢,就借着儿女婚事几度弄权。儿子想着她出身寒微,急欲找些依傍,也不说什么。可如今有些龌龊事她自己做了,还把脏水泼了别人。儿子倒觉得,这样的额娘,如何教出汉昭帝这样的明君呢?''

太后微微点头,伸手拨弄着瓶中一支晚梅,似叹非叹:``这么多年,是该收拾收拾了。''

皇帝唇角一抹若有若无的笑意,伸手抚摸着那枝条遒劲的花朵,神色却犀冷如锋。

\hypertarget{ux7b2cux4e09ux5341ux7ae0-ux4ee4ux61ff}{%
\chapter{第三十章 令懿}\label{ux7b2cux4e09ux5341ux7ae0-ux4ee4ux61ff}}

时欺深寒,冬云冥冥。

皇帝审完春婵,已是天色昏暗。春婵不禁不得几问,便将所知之事,说了个分明。数十年的恩怨生死,夹杂着一个女人的宠遇与野心,在唇齿和唾沫一一吐出。

皇帝听到最后,全然面无表情,``你倒肯说得那么清楚,难道跪皇妃一直看重你。''

春婵浑身多在哆嗦,但口齿还清晰,``澜翠死了,进忠也死了。说不定哪日皇贵妃就要奴婢得性命了。''

皇帝颔首,``懂得惜命的人,才能活得长久。朕会饶恕你的性命。记得闭上你的嘴。''

春婵不意还有性命可以留下,喜得拼命磕头,才被李玉拖下去了。

幽深旷寂的宫室内,一幛白象牙嵌玻璃画描金花鸟大屏风隔开了方才的审问,屏风一侧鎏金花鸟香炉的镂空间隙中袅袅升起辛夷香,木香特异,略带辛味,香似乎已经燃了大半,满室都是袅袅的香,带着肃杀的气息,叫人心生绝望。

皇帝很是平静,唤道:``出来吧。''

嬿婉踟蹰而出,不敢看端坐着的那个目如深潭得沉默的男子。她的双足如同踩于荆棘之上,每一步都在滴血。前行几步之后,她终于瘫软在地。

皇帝静静看着她,``春婵所言,有没有冤枉你?''

深切的恐惧像釉面上细细的冰裂一样,在一瞬间浅淡地布满了全身。

嬿婉眼睛发直,喉咙干涩到了极处,还是忍着痛发出破碎的音节,``皇上,臣妾冤\ldots{}''

``冤枉?''皇帝嗤笑,``你若觉得冤枉,朕就细审你身边每一个人。佐禄、王蟾,有段时候你与和敬公主也有来往,朕不妨也问一问自己的爱女,或许可以听到比春婵所说更多的东西。''

嬿婉畏惧到了极点,忽然满心舒展开来,她冷冷抬眼,索性豁了出去,``自从乌拉那拉氏离世,皇上疑心臣妾多年,终于肯问出满心疑惑了么?''

皇帝满眼戏谑:``那么你打算怎么为朕解惑?''

``臣妾没有杀她。''这句话,嬿婉说得坦然而气足。是如懿自裁,她可没有动手。

皇帝对她的说法毫不意外,``哦,你只否认这件事,也就是说春婵所招认的你害人之事,都是真的了?''

嬿婉见这逼问如山倾倒,浑身一阵颤抖,忽然勇敢起来,``是!都是臣妾所为,那又如何?臣妾若不为了自己,谁还能为臣妾?臣妾都是被逼的。''

那是她椎心泣血的申诉,皇帝浑然不在意,只是语调凉薄:``你们都说自己是被逼迫,淑嘉皇贵妃是,你也是。好像你们有了这个理由,做任何伤天害理的事都情有可原了是不是?''

嬿婉晓得自己在皇帝眼里不过是一只被戏弄的小鼠,这数年的拨弄戏谑,齿爪间的苟延残喘,把她拖得求生不得,求死不能。既然如此,也不过是一死。``不过是一条命,皇上要拿去便是。''

皇帝笑了:``这时候还能如此决绝,到底胜过一般人,难怪能爬到这个地位。好好,你来。你来。''

皇上向她招手,如往日一般亲近,嬿婉冷汗涔涔,挣扎着退后。皇帝也不作声,缓缓起身,走近嬿婉。他的指尖冰冷,全无一点暖意,抬起嬿婉眼的脸,凝望片刻。他荷荷一笑,骤然发作,连扇了数十下耳光。嬿婉眼前一片金星闪烁,脑中又酸又涨,好像口鼻都浸泡在一缸陈醋里。耳朵里做着水陆道场,嗡嗡地铙声锣鼓声喇叭声,远远近近地喧腾着。

皇帝的声音隆隆的,像雷声在响。``你害死了璟兕,你害死了十三阿哥,你害死了朕与如懿的孩子。''她的脑袋有千百斤重,根本抬不起来,唯有温热的液体滚落在手背上、衣袖上。她眯着眼睛看了半日,才看清楚那是自己的血。

那么多的血,从鼻腔、口角滴落而下。嬿婉呜咽着,像一只受伤的兽,垂死挣扎,``臣妾还害死了乌拉那拉如懿。皇上,你是不是很痛心?看你这么痛心,臣妾忽然觉得好痛快!数年如履薄冰,夜不能寐,这会子真正可以痛快了。''

皇帝被她的话激得失了仅剩的平和。他目光如剑,恨不得在她身体上剜出几个洞来。他深恶痛绝,``你这个毒妇!''

嬿婉森然一笑,雪白的牙齿染红色的血液,如要噬人,``臣妾再毒,也受您半生宠爱,臣妾觉得很上算哪。哈哈,皇上,别怪是臣妾害死了乌拉那拉如懿,害死她的人是您。要不是您,谁伤得了乌拉那拉如懿的心,谁能与她生死长离,再不能回头呢?''

皇帝颓然坐倒,他已是六十五岁的老人,哪里受得住这般刺心之语。狂热的恼恨之后,悔意冰凉袭上心头,他喃喃凄楚:``如懿,是朕对不住如懿\ldots{}''

嬿婉击掌而笑:``痛快,真痛快。''

皇帝迫视着她,``这数十年,你对朕半分真心也无,所以到此地步,还能痛快。''

``真心?''嬿婉嗤之以鼻,``您对臣妾有半分真心么?臣妾不过是您的一件玩意儿,您高兴了就捧着臣妾,不高兴了就踩在地上而已。''

夜间北风大作,红肿着双眼的嬿婉跪在金砖地上,任朔风寒气将她脸上的泪水敛聚成冰,她的身躯炒已经麻木,膝盖上的痛楚浑然不觉,只是以眼中的嘲讽,仰望着烛火红焰侧的垂暮天子。

皇帝默然片刻,从袖中取出一枚戒指丢下,``你的真心,都是对他吧?''

那是一枚红宝石戒指,实在是不值钱的东西,一看便知是出自民间寻常银铺,那戒指在锦绒毯上滚了几圈,停在嬿婉脚边,散出幽暗光芒。嬿婉乍见了多年前的爱物,不觉匍匐上前,将它紧紧攥在手心,颤声道:``这枚戒指怎么在你这儿?怎么会在你这儿?''

``怎么?你很在意么?''皇帝弯下腰,将她的神情尽收眼底,``凌云彻,不也是你害死的么?''

那小小的指环硌在手心里,冰凉,坚硬。她像是找到了永生永世的寄慰,在不肯放开。

泪水潸然而落,是欣慰,是失而复得的喜悦。赠予戒指的人早已不在了,而这份情意,足以让她在辛苦恣睢的日子里以安慰平生所失。

皇帝厌恶不已,``你的眼泪,会弄脏朕这里。''他扬声向外,``来人。''

李玉早就准备在外,端着要恭恭敬敬进来。

皇帝连多说一个字都觉得恶心,只道:``给她!''

那一碗汤药如墨汁般浓黑,热气氤氲,散发着魅惑般的甜香。这种突兀的香气不像是寻常药材所有,她惊惧地别过脸,不想去面对。

李玉轻声道:``这一碗牵机药是皇上为小主您准备的,服下后剧痛不已,头足相就,如牵机状,乃是毒中之王。''

求生的意志剥夺了她方才的勇气,嬿婉本能地抗拒:``不!''

李玉端着药凑近,``奴才案皇上吩咐,取来此物。是因为所有毒物之中,牵机药服下最为痛苦,合皇贵妃娘娘所用。''嬿婉还要躲避挣扎,她膝行皇帝身边,拉着他袍角哭泣,``不!不!皇上,臣妾知错了,臣妾知错了。''

皇帝一脚将她踢开,就像踢开足尖的污秽。李玉半是搀扶半是挟制,``皇贵妃切莫挣扎,想想您的诸位阿哥和公主,您可不想您一去,还连累了他们吧。你顺顺利利走了,来日皇上想起您,也少些厌憎之情啊。''

一了百了,这样自己的孩子才能好好活着!是么?嬿婉筋骨酥软,不敢再做抵抗,由着李玉按住了她的下巴,一口一口喂她喝下汤药,一滴不漏。

汤药入口,如利剑直剖肠腹。她知道,是很烈的毒药,药性很快就会发作。

皇帝冷冷道:``带她走,别让她死在这里,污了朕的梅坞。''

嬿婉惨然微笑,紧握着手心,被李玉和进保搀扶着塞进了轿子。

梅坞又恢复了那种恍若深潭静水寂寂无声。从无人敢进来这打扰年迈的皇帝。满殿纷碎的梅花原样装点,催落了皇帝的泪,``如懿,如懿,朕曾经得到你的真心,也给过你真心,可是天人永隔,朕还是失去了你。朕还误会了你和凌云彻,一定很伤你的心\ldots 如懿\ldots 朕还能去哪里找一个真心对朕的人呢?''

四下里无声,前尘就影恍至心头。

轻拈纨扇的少女,身边有三五蝴蝶施施然展翅,围着她翩翩翻飞,她唇角一痕笑意相映,一双清水般的眸子含情相望。一握杏子红绫裙拢住了一袅一袅晴丝,韶光缓燃垂下,无数浅粉色樱花在她身后得纷纷烈烈。

那是荳蔻初成的青樱,盈盈等待着,少年皇子弘历,在她身边并肩相依。

夜幕笼罩了整个帝京,女子的胭脂香,宫阙的沉寂,昔日的温柔,一如皇帝对于往事的记忆,一同沉了下去。

药性发作得很厉害,嬿婉孤身一人卧在永寿宫的寝殿里。人人只道她去过了养心殿像皇帝问安,又悄然而回。因着心悸病,夜来伺候的唯有春婵,宫人们被远远打发到外头伺候,所以无人知晓寝殿内的情况。地上悉铺织金厚毯,其软如绵。燕婉如僵死之虫,全身抽蓄,头和足几乎接触,喉间发出不似人声的呻吟。五脏六腑被毒药腐蚀了一层又一层,从每一寸骨节,到每一个毛孔,都痛得不可遏制。

她只是急切地盼望着,怎么还不死?怎么还不死?

李玉并不肯走,想看着她的惨状,恭谨为首而立。他的眼底有幽深的恨意,``皇贵妃,奴才私心,想看着你药性发作,受尽苦楚。''他缓缓道来,``皇上选了牵机药,而非鹤顶红,就是不想你死得太痛快。奴才呢,就特意和江太医商议,调整了药性,你要受尽痛苦三个时辰后,待到天明时分,才会断了气息。''

嬿婉痛得卷缩成一团,看着身体机械班抽蓄,哑声道:``你好狠\ldots{}''

明纸糊厚厚的,将窗外凛冽的北风隔绝得无声无息,庭院的树影不停摇动,在李玉身后头下斑驳摇移的阴影,应得他唇角的笑容森然可怖,``比起你对翊坤宫娘娘的手段,这实在不算什么。''他转头看看滴漏,``天快亮了,你的大限要到了。奴才先告辞。''

他退下,烛光涂红了窗纸,帷帘上簇簇艳红的花团,开得热烈至极。终其一生,那都是她喜欢的繁荣与热闹。

滴漏单调的响声慢慢蚕食着她最后的生命。嬿婉大口大口地吐出腔子里的血,眼见它们飞溅得老高,像是一颗不肯认命的心,死也要死在高枝上。架子上明黄的皇贵妃袍服笔挺地悬着,五彩的凤凰,丰艳的牡丹,盘旋成吉祥如意的口彩,那原本该是她完满的人生。

可这一刻,她什么也不求了。

嬿婉松开紧握的手心,露出一枚好宝石戒指。她忍着撕裂般的痛楚,颤巍巍将那枚戒指往手指上套。这个小小的动作耗尽了她最后的力气,却也和来她生命最末的一息恬静,``云彻哥哥,我这一辈子唯一对不住的只有你。你等我,我来了,我来找你了。''

视线因着发作的毒性变得模糊不堪。嬿婉恍惚看见年轻的自己,穿着一身恭女装束,欢快地奔向长街那一头等候的凌云彻。

嬿婉心头微甜,那也许是她一生中,最值得纪念的时光。可惜那以后的自己,再未懂得珍惜。

那枚戒指在指尖轻轻发颤,被滑落的汗水滑下,骨碌碌滚了老远。嬿婉睁大了眼睛,却再无半分力气,去寻回那枚戒指。

她带着无限遗憾,停止了气息。

正月二十九的清晨时分,侍奉了嬿婉多年的春婵按照李玉留下的吩咐进去料理,然后发觉这位在翊坤宫后离世多年后纵横六宫的皇贵妃,全身僵成怪异可怖的姿势,断了气息。七窍间流下的乌黑血迹是意料之中。她在惊慌之余,强迫自己冷静下来,用颤抖的手迅即抹去那些类似破绽的血痕。然后以悲伤的哭因告知众人,皇贵妃因为心悸之症遽然离世。

皇帝自然是悲伤逾常。令皇贵妃自宫女始,荣至皇贵妃,位同副后。更为皇帝生下四子二女,宠遇一生,足见恩幸之隆。皇帝伤心不已,丧仪格外隆重,又钦定追溢嬿婉``令懿''二字为封号,以皇贵妃之仪风光下葬,更将新成的水莲碧玺奉与她身侧,以托哀思。

在众人的悲声号泣里,唯有一点疑云难以抹去,为何隆宠一声的皇贵妃,却偏以皇帝最不喜的女子知名追溢。终于有一日,年幼的十七阿哥永璘冲口而出,连一旁连连使眼色的永琰也阻止不住。

皇帝闻言,不觉勾起满腔悲怀,更抚额痛哭,对膝下皇子连称``懿''字乃嘉言懿行,德行美好之称,永璘只得诺诺退下,只余永琰伴随身侧,安慰老父伤怀。而在宫人们私下纷言里,不过是因为逝世令皇贵妃,实在是有三分肖似当年的翊坤宫皇后的缘故吧。那,也是令懿皇贵妃在世时最忌讳不过的了。只是前尘往事,二人俱已芳魂离散,喧嚣一阵后便也无人再提了。只是为着皇帝对令懿皇贵妃的爱宠情深,令懿皇贵妃离世后,伺奉她多年的贴身奴婢春婵无处可去,皇帝也格外抚慰,赐了她一所三进的宅子,又拨了两个婢女伺候,准她出宫安居。说起来这也是做了一辈子的奴才难以企盼来的福泽,懿时间人人皆赞皇帝后待嫔御,恩泽宫人,情深意重。

而唯有李玉知道,被一抬小轿抬着离开的春婵,除了惊恐地发出啊啊之声,再不能言。一边看首她的嬷嬷便道:``春婵,皇上宽厚,看在你供出那人多年的罪行的分儿上,留了一条命给你,还要我守你终老。否则你以为只是一碗哑药这么简单么?好好惜福吧。''

春婵无力地摇头,忽然想起那年澜翠身死的模样,打了个寒战,畏惧地卷缩起了身子,唯余心底一声悲苦,``澜翠,澜翠,从小主不肯护你的那日,我便知道迟早会走你的后路。我没有办法啊,只能听皇上的。谁,谁能拗得过皇上呢?''

春婵的泪倏然落下,好死不如赖活,无伦她做了什么,到底嬿婉死了,澜翠死了,唯有她活着,哪怕是永远缄默地活着。

彼时皇十五子永琰尚是十五岁的少年,骤然失母,底下又有更年幼的弟弟永璘,哥儿俩字是孤苦。皇帝便只了婉嫔陈氏亲与照拂。这在宫中也算是件不大不小的事,因为婉嫔陈氏虽然久在宫中,资历既深,但到底无宠了许久,又是极默默无闻之人。而之前曾经受命抚养永琰的,也是位分既高、资历也不浅的庆贵妃。想来婉嫔乍然受此重托,大约也实在因为她是个勤谨安分之人吧。皇帝便也格外青眼相看,虽然仍无召幸,但素日里便按着贵妃的分例供养,也算怜她照拂两位皇子的辛苦。

但到底,皇帝给了婉嫔如此恩遇,却也未晋她位分。直到乾隆五十九年,才晋了婉妃之分,算是与皇帝一同安居共老了。

自然,这也是后话了。

后来那些年,皇帝的闲暇时光,多半是在长春宫思念孝贤皇后中度过。偶尔在梅坞,他也会听着细子们唱着《墙头马上》,握着一方绢子出神。

戏子们悠然唱着情词婉转,``帘卷虾须,冷清清绿窗朱户,闷杀我独自离居。落可便想金枷,思玉锁,风流的牢狱。''

孤清长又长,在这禁城中悠悠荡荡。

在这孤清里,皇帝也是倦了。他已是须发皆白的老人,怆然独坐,颓颓无语,只在浑浊的眼中漾满疲惫与伤感。他右腕微微使力,一顿一转,笔锋强健有力,于黄笺之上郑重写下``传位于皇十五子永琰''。

他的手指上凛冽的细纹,是被风霜与孤寒重重侵蚀后无声的痕迹。他的手势沉重却无迟疑,将手中黄笺细细迭好,存于锦匣之中,以蜡密封。

李玉远远站在苏绫蟠龙帷帘之外,见皇帝一应完成,才敢捧着茶走近,恭声道:``皇上饮茶,润润喉吧。''

那锦匣似有千斤重,皇帝略略一掂,苦笑道:``朕从未做过这般事,不想,却做得如此流畅而熟稔,仿佛已经做过许多次一般。''

李玉哪敢抬头,弯着腰身愈发显得佝偻而恭谨,``储位之事关系江山命脉,皇上日夜悬心,没有仪刻放松,自然熟稔。''

皇帝轻嘘一声,缓缓抚摸着锦盒上缂丝双龙出云的纹理,沉声道:``不知道皇阿玛当年,是否也如朕今日一般,如释重负,又惴惴不安。''

李玉俯身郑重叩首,``先帝乃千古明君,才选定皇上承掌天下。皇上青出于蓝,一定会为天下苍生定一位仁君。''

皇帝望着他,眸光里闪过一丝模糊的软弱与伤痛,``朕属意的皇子不能留存于世间,以至朕行将老迈,却不得不定下幼主。朕斟酌思量,考究再三,也唯有如此了。''他淡淡嘱咐,``入夜之后,你陪朕往干清宫,朕要亲自放于正大光明匾额之后。''

李玉垂首咬着牙,抿出一丝最诚恳恭顺的笑容,``奴才遵旨。奴才明白,皇上一切,都是为了大清江山。如汉武唐宗,明垂千古。''

皇帝微微出神,笑意如为凉秋霜,``汉武帝晚年思念戾太子,亿及卫氏皇后与戾太子死得不明,更为防主母壮,杀了钩弋夫人赵氏,才利幼子。朕所作所为,倒是真有几分像汉武帝。''

``奴才虽然愚钝,却也听过戏文。武帝雄才大略,为求江山安稳,且将私情搁置一边。唐太宗若无玄武门惊魂,何来太平盛世?且有皇上悉心调教,何愁幼主不成明君?大清江山万年,一切有赖皇上。''李玉说得恳切,眼中隐有老泪闪动,似是十分动情。他忽然一惊,似是知道自己说得不当,立刻反手抽了一巴掌,惶恐道:``皇上恕罪,奴才妄议朝政,合该立即打死!''

皇上摆摆手,``算了。你只是论戏文,也不是旁的。''他长叹无声,``李玉,朕年将迟暮,身边能说说话的老人也唯有你一个了,您有那么多皇子公主,有三宫六院无数,您十全武功,福泽滔天,连老天爷也眼红呢!''

皇帝唇角的苦涩笑意越隐越淡,终于化为一抹悲怆的无助,``不是苍天嫉妒,是朕自己,把自己逼成了孤家寡人。''

李玉唬个不住,连忙道:``皇上坐拥四海,皇上\ldots{}''

皇帝愀然不乐,打断他到:``朕让你往乌拉那拉\ldots 如懿灵前祭酒,你去了么?''

李玉垂着手,动容道:``回皇上,奴才已经去了。也将令懿贵皇妃之事与乌拉那拉娘娘知道,希望她在天之灵有所安慰。''他微微迟疑,还是含了畏惧道:``皇上,请恕奴才死罪。其实乌拉那拉娘娘弃世后,奴才与江太医夫妇,并不曾停了四时宫奉祭祀。''

皇帝身子微微一栗,面上却无一丝喜悲,只是缓缓道:``若在从前,朕会怪你隐瞒之罪。但从婉嫔夜见那回后,朕会谢你,李玉。''他眸底如骤雨初歇后霭沉沉,``如懿一直怪朕,觉得朕没有视她为妻,不似民间夫妇,彼此珍爱关照,才渐行渐远,再不复昔年。朕也一直负气,所以只以皇贵妃礼仪位她治丧,甚至与纯惠皇贵妃安于同一地宫。''

李玉界面道:``皇上,您是顾念诸位皇贵妃之中,唯有纯惠皇贵妃与乌拉那

拉娘娘上算交好,您\ldots''

``如懿是外柔内刚之人,若得纯惠皇贵妃三分庸懦顺服,朕与她也不致如此。生前个性不驯,死后希望她也能沾染一点纯惠皇贵妃的气性。不要再与朕相形陌路。''

李玉满脸哀戚,``皇上,乌拉那拉娘娘总有千般不是,可您一直为许她附葬裕陵,也未单建陵寝,只葬在了妃园寝内,甚至没有自己的宝券。不设神牌,死后也无祭享。如今皇上知道许多是乌拉那拉娘娘也属冤屈,何不许她死后颜面,略加厚待。''

皇帝目光如刀,逡巡在他面上,半日才仰天弥叹,``李玉,朕与如懿屡起争端,可朕最恨的一句,是她竟然羡慕宫外平民夫妻,且将朕九五之尊置于何地?将朕与她多年情意至于何地?或许做朕的妻子,她并不快活。她要做一个庶子,朕就让她勉为其难做一个紫禁皇城中的庶人!''

李玉小心翼翼道:``皇上终究是愿意成全了乌拉那拉皇后的一点愿心。''

皇帝的叹息是潮湿的哀凉,``或许朕也是在很久很久之后,才发觉,当年自以为正确的决定,都是后来追悔莫及的源泉。可是过去的,终究已经过去了。''他叹抚不已,语意微凉,``朕能做的,无非也是如此。若是设了神牌,追封溢号,留下后妃画像,史书载下她只字片噢。那么她生生世世只能是紫禁城的一缕孤魂,魂魄为红墙所拘,不得游荡去她想去的地方。朕用名分留了她一生,却给不了她要的情感与尊重。弃她,或许也是放了她。''

李玉顿了顿,还是奢着胆子道:``可最终皇上明了真相,还是为乌拉那拉娘娘报仇了。''

皇帝哀然道:``可是朕与如懿误会良多,此生无法解开,也无人能解了。''他沉默片刻,``李玉,传旨下去,自朕以后,后妃之选,再不必有乌拉那拉氏族女,且让她们后人,都得一个平凡夫妻的终老吧。''

李玉颔首答应,俯身三次跪拜,``皇上的心意,奴才都明白了。乌拉那拉娘娘有知,也会明白的。''

长久的沉默里,唯有夜风游荡,吹开苏绫如水的波漾,在烛光摇映之下,恍若蘸水桃花点点红晕。

那样的暗红,望得久了,仿佛雪地里孤清冷傲的红梅,晃得刺疼了眼。皇帝看着周遭粉碧涂彩,金灼玉辉,仿佛自己成了博古架上那只描金珐琅粉彩梅花瓶,孤零零地架在高处,虚弱得没有着落。他凄然不已,``夫妻恩情,嫔御恭顺,儿女之福,父母之恩,朕已失却大半。朕,终究,不过是天地间一寡人。''

没有人答应,也无人敢应答,一个帝王最后的寂寞。

夜风缓缓拂来,帘影姗姗。唯余两人垂垂老矣之人,身影幽长,复幽长。

\hypertarget{ux756aux5916-ux4e07ux5bffux957fux591cux5c81ux5c81ux51c9}{%
\chapter{番外
万寿长夜岁岁凉}\label{ux756aux5916-ux4e07ux5bffux957fux591cux5c81ux5c81ux51c9}}

夜风沉缓地吹拂,空气中绵密的花香软软地缠上身来,与酒意一撞,皇帝更觉得心中沉突,整个人醺醺欲睡去。

总管太监李玉的步子迈得又快又稳,一壁轻声督促着抬轿的小太监们,``稳着点儿,别摔着了皇上。''

皇帝朦胧中扶着头,含糊地问:``到哪儿了?''

李玉含笑答道:``皇上,到西六宫的长街了。''

皇帝轻轻``哦''了一声,``是西六宫。李玉,朕仿佛有点儿醉了。''

李玉忙恭谨道:``皇上安心,您一早翻了惇贵人的牌子。奴才已经去通传了,这个时候惇贵人已经备下了醒酒的汤药在承干宫等着您了呢!''

皇帝``唔''了一声,缓缓道:``停下!''

李玉满心托异,却不敢多言,忙着甩手中拂尘,示意抬轿的太监们放落了桥辇。李玉凑上前,``皇上,这儿离承干宫还有一段路,还是让奴才们抬着您走吧。''

皇帝伸出手,李玉忙伸手扶住,皇帝道:``朕觉得酒劲儿上来了。李玉,你扶着朕走一会儿。''

李玉忙躬身道的声``是'',悄悄儿朝后脸一扬。后头跟着的四个小太监会意,变隔了十步之遥,轻悄跟在二人后头。李玉稳稳扶助皇帝的手臂缓缓往前。

皇帝不说话,李玉更不敢说话,也知道皇帝想去哪里,只好默然跟着。月色澄明如清波,温柔浮溢四周,连长街两侧的朱红高墙,也失了往日的沉严肃穆,显出几分娇柔。

皇帝抬头望着月亮,似乎是自言自语:``今儿的月亮真好。''

李玉忙笑:``皇帝是天子,今儿是您的万寿生辰,当然连月亮也要来助兴,格外高亮些。''

皇帝微微一笑:``是啊,今儿是朕的生辰,再过两天就是八月十五中秋,人月团圆,都是好日子。''

李玉见皇帝凝神望月,嘴角仍带着笑意,不知怎的,心里一突,便有些不自在起来,于是赶紧劝道:``皇上,时辰不早,您今儿高兴多喝了点酒,仔细被风扑着,伤了龙体。''

皇帝摇摇头:``酒酣耳热,朕不会凉着。''

李玉悄悄看了皇帝一眼,奓着胆子劝道:``皇上,颖妃娘娘在养心殿等着您哪!''

皇帝冷淡道:``让她等着。''

李玉暗暗吶罕,颖妃巴林氏蒙古贵女,入宫数载,颇得皇帝恩幸。便连皇贵妃魏氏所生的女儿七公主,也交由她扶养。尤其是乌拉那拉皇后过世之后,寻常嫔妃难得见皇帝一面,这位颖妃却常能陪皇帝说话,宠遇可见一斑。而今日皇帝这样抛弃下她不顾,却是从来未有之事。

李玉见皇帝信步往前,环视周遭一眼,忽地想起一事,心中没来由地一慌,脚下都有些踉跄了。

皇帝漫不经心地道:``叫跟着的人都退下,朕见了心烦。''

李玉不敢怠慢,忙回头扬了扬拂尘,四个小太监便躬身后退下去。

李玉上前扶住皇帝的手,皇帝慢悠悠走着,兀自说:``今儿是朕的生辰,朕真高兴。''

李玉忙接口:``高兴高兴。''

皇帝含着笑意,``朕有那么多阿哥、公主,一个个活泼泼的,又聪明又伶俐。''

李玉道:``更难得的是阿哥和公主们都有孝心,尤其是几位阿哥,特别出息。十一阿哥文采风流,写得一本好书法,今日为皇上献上《百寿图》,可真是十一阿哥的一片孝心;就是十五阿哥,虽然年纪小,可当真是志气,能把皇上的御诗一字不差地背下来,啧啧\ldots 真是能干。''

皇帝微瞇了眼,``那不是十五阿哥能干,是他的额娘太能干。''

魏氏身为贵皇妃,位同副后,主理六宫,又子女双全,若不是得了这心悸病身子虚弱很多,只怕也有封后的指望了。

李玉眼珠一转,只装作不懂,笑吟吟道:``可不是,皇上的万寿节都是皇贵妃撑着身子一首操持,不可谓不能干。''

皇帝轻嗤一声,带了几分嘲讽之意,``是啊。朕有那么多嫔妃,个个貌美如花,聪明能干。''

李玉不知皇帝何意,只赔笑说:``皇上的嫔妃们不仅貌美贤慧,而且今日万寿节都为皇上进歌献舞,当真才貌双全。''

皇帝闭上双眼,``可不是?个个都顺从着朕,体贴着朕。只有颖妃还直爽些。''

皇帝晃一晃头,脚步有些不稳,李玉急道:``皇上,皇上您当心着。''

皇帝摆了摆手,``顺从体贴自然是好,可朕怕啊,怕这顺从体贴下面是说不吃口的腌臜心思,污秽手段。朕想一想,就觉得恶心。''

李玉忙笑道:``皇上多虑了,后宫的小主们怎么会是这样的呢?哪怕真有一两个心术不正的,皇上圣明,也一早处置了。''

皇帝低头看着自己的影子,``所以,朕喜欢年轻的女人,心眼儿干净,清透,都说什么自然会说。哪怕有点儿小心思,也藏不住。''

李玉忙忙点头,``皇上说得是。''

皇帝缓步走着,李玉赔笑道:``皇上,再往前就是翊坤宫,旁边的宫里庆妃娘娘和诚贵人都在,再不然,还有几位贵人常在,也都住着。皇上要不要去坐一坐\ldots{}''

皇帝瞥了他一眼,淡淡道:``李玉,你跟了朕几十年,如今倒越发会当差了。''

李玉膝盖一软,连忙跪下,``皇上恕罪,皇上恕罪。''

皇帝轻哼一声,也不理会,径自向前去。李玉跪也不是,站也不是,眼见皇帝越走越远,他咬了咬牙,奢着胆子小跑着跟了上去。

四下里的甬道太过熟悉,连每一块引他向重华宫的青石板上的花纹,他都烂熟于心。皇帝怔忡地走着,越走越快。等到``翊坤宫''三个金漆大字清晰地出现在眼前的时候,皇帝才猛然剎住了脚步。酒意沉突涌上脑门,皇帝只觉得心口一阵一阵激烈地跳着,脚步却凝在了那里。

恍惚还是帝后情睦的岁月,如懿初为皇后。过了那么长的时光,越过了那么多人,她终于走到和自己并肩的地方,成为自己的妻子,而非面容鲜妍而模糊的妾室中的一个。这是他许她的。在自己还是阿哥的时候,他太知道自己虽为帝裔,却出身寒微,连亲生父亲都隐隐看不起自己,对他避而不见。所以他有了熹贵妃这位养母,所以他拼命孝顺这位他带来荣耀家世的养母。他费尽心力用功读书,只为争得属于自己的荣耀。

那个时候,他有出身名门贵族的嫡福晋富察氏,也有了大学士之女、温柔婉妹的高氏。那修高贵而美里的女子,那些深受家中宠爱的女子,他在欢好之后只觉得疏离。她们跟自己的心,到底是不一样的。只有如懿,那时她还叫青樱,是被自己的哥哥瞧不上,才被熹贵妃要来送给他做棋子的女人。连熹贵妃都说:``你可以不宠爱她,却一定要娶了她,善待她。''不错,青樱是可以作为她和三哥弘时争夺地位的一步好棋的。青樱,她也有显赫的出身,她是先帝乌拉那拉皇后的侄女。这重身分,却在后来的日子,成了她的最大的尴尬。

因着先帝乌拉那拉皇后的晚年凄凉,因为乌拉那拉皇后败在当今太后手中,所以青樱入宫后的日子,很不好过。她被冷落了些年,直到那时,他才真正爱上青樱。因为那样的青樱,伴随他多年,深知彼此心性,又真正和自己一样,是富贵锦绣林中心底却依然孤寒之人。

所以他加倍地给予她荣耀,给予她皇后之位。

曾经,也有过琴瑟相谐。而最美好的最初的时光,都停留在了翊坤宫的岁月。

那时,她与他是多么年轻。人生还有无数明灿的可能,他们都真诚地相信,可以一起走到岁月苍老的那一日。

皇帝伸出手,爱惜地抚摸在翊坤宫的大门上。

触手扬起的轻灰令皇帝忍不住咳嗽。他仔细看去,才发觉门上红漆斑驳,连铜钉都长出了暗绿的铜锈。墙头恣意生长的野草,檐角细密的蛛网,都是那样陌生而寥落。

曾经恩深情重的翊坤宫,曾经住着天下之母的翊坤宫,竟也破败如此了么?

其实,也不过这些日子而已。

可以宫廷的冷落,他最清楚不过了。万人之上的他,坐拥天下的他,何尝不也是在年幼时受尽白眼,若不是乳母庇护,又有了熹贵妃的抚育,他何曾能有今日?

所以,他太清楚如懿的骄傲,太清楚该如何挫磨她的骄傲。

哪怕是皇后,也要屈膝在皇帝之下,俯首恭谨。

可是如懿,她有那样锐利的眼神。恰如她断发那一日,如此决绝而凄厉。

万事,终于不可再回转。

皇帝静静地伫立在门前,良久,只是默然。

月亮渐渐西斜,连月光也被夜露染上几分清寒之意。

李玉跪在皇帝身后不远处,连膝盖都麻得没有感觉了,只依稀觉得冷汗流了一层又一层,仿佛永远也流不完一样。

他是不该看见的,就好像,皇帝也不该过来这里。

翊坤宫,应该是皇帝最厌弃的地方;翊坤宫里的人,应该是皇帝最厌弃的人。可是偏偏,在这样普天同庆的万寿节里,在即将花好月圆的中秋夜前,皇帝却在翊坤宫的门前,迟迟徘徊,不愿离去。

也不知过了多久,夜露浸染了云鬓,李玉才觉得有些凉意。他犹豫了半日,终于咬着牙膝行到皇帝跟前。李玉拼命磕了两个头,方敢极低声地说话:``皇上,已经二更了。''

皇帝只是默默不动,仿佛整个人都定在了那里。

李玉眼见皇帝的袍角已被露水浸湿,心中更是惊惧,立刻俯首在地,``皇上,宫中人多口杂,万一\ldots 快中秋了,您要伤了龙体,太后和皇贵妃问起来,奴才胆当不起。''

他不敢再说下去,只是叩首不已。

片刻,皇地叹了口气。那叹息极轻微,像一阵轻风贴着墙根卷过,连李玉自己都疑心是否听错了。皇帝轻声呢喃:``人月两团员?呵,团圆?''

李玉吓得不敢抬头,终于听轻皇帝说了两个字,``回去。''他顿了顿,``将翊坤宫拾出来,给惇贵人住吧。''

他挣扎着站起来,也不顾膝头酸痛,忙扶着皇帝的手去了。

墙头的野草清幽幽地晃着,好像只有风来过。

全书完,至此本系列大结局。

\backmatter
\end{document}
